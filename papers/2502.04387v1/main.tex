%%%%%%%% ICML 2025 EXAMPLE LATEX SUBMISSION FILE %%%%%%%%%%%%%%%%%
%%% TO REMOVE ARXIV VERSION, ADD BACK \newcommand{\ICML@appearing}in icml2025.sty and impact statement
\documentclass{article}

% Recommended, but optional, packages for figures and better typesetting:
\usepackage{microtype}
\usepackage{tikz}
\usepackage{graphicx}
\usepackage{subfigure}
\usepackage{booktabs} % for professional tables
\usepackage{multirow} 
\usepackage{makecell}
% hyperref makes hyperlinks in the resulting PDF.
% If your build breaks (sometimes temporarily if a hyperlink spans a page)
% please comment out the following usepackage line and replace
% \usepackage{icml2025} with \usepackage[nohyperref]{icml2025} above.
\usepackage{hyperref}
\usepackage[justification=centering]{caption}
\usepackage[flushleft]{threeparttable}

% Attempt to make hyperref and algorithmic work together better:
\newcommand{\theHalgorithm}{\arabic{algorithm}}

% Use the following line for the initial blind version submitted for review:
\usepackage[accepted]{icml2025}
\usepackage{numprint} 
% If accepted, instead use the following line for the camera-ready submission:
%\usepackage[accepted]{icml2025}

% For theorems and such
\usepackage{amsmath}
\usepackage{amssymb}
\usepackage{mathtools}
\usepackage{bm}
\usepackage{amsthm}
\usepackage{xspace}
\usepackage{xcolor}
\usepackage{enumitem} 
\usepackage{listings}

% if you use cleveref..
\usepackage[capitalize,noabbrev]{cleveref}

%%%%%%%%%%%%%%%%%%%%%%%%%%%%%%%%
% THEOREMS
%%%%%%%%%%%%%%%%%%%%%%%%%%%%%%%%
\theoremstyle{plain}
\newtheorem{theorem}{Theorem}[section]
\newtheorem{proposition}[theorem]{Proposition}
\newtheorem{lemma}[theorem]{Lemma}
\newtheorem{corollary}[theorem]{Corollary}
\theoremstyle{definition}
\newtheorem{definition}[theorem]{Definition}
\newtheorem{assumption}[theorem]{Assumption}
\theoremstyle{remark}
\newtheorem{remark}[theorem]{Remark}

% Todonotes is useful during development; simply uncomment the next line
%    and comment out the line below the next line to turn off comments
%\usepackage[disable,textsize=tiny]{todonotes}
\usepackage[textsize=tiny]{todonotes}
% \usepackage{siunitx}
\DeclareMathOperator*{\argmin}{arg\,min}
% The \icmltitle you define below is probably too long as a header.
% Therefore, a short form for the running title is supplied here:
\icmltitlerunning{Federated Learning to Personalize PEFT for Multilingual LLMs}
\begin{document}

\twocolumn[
\icmltitle{FedP$^2$EFT: Federated Learning to Personalize \\ Parameter Efficient Fine-Tuning for Multilingual LLMs}

% It is OKAY to include author information, even for blind
% submissions: the style file will automatically remove it for you
% unless you've provided the [accepted] option to the icml2025
% package.

% List of affiliations: The first argument should be a (short)
% identifier you will use later to specify author affiliations
% Academic affiliations should list Department, University, City, Region, Country
% Industry affiliations should list Company, City, Region, Country

% You can specify symbols, otherwise they are numbered in order.
% Ideally, you should not use this facility. Affiliations will be numbered
% in order of appearance and this is the preferred way.
\icmlsetsymbol{equal}{*}

\begin{icmlauthorlist}
\icmlauthor{Royson Lee}{comp}
\icmlauthor{Minyoung Kim}{comp}
\icmlauthor{Fady Rezk}{sch}
\icmlauthor{Rui Li}{comp}
\icmlauthor{Stylianos I. Venieris}{comp}
\icmlauthor{Timothy Hospedales}{comp,sch}
\end{icmlauthorlist}

% \icmlaffiliation{yyy}{Department of XXX, University of YYY, Location, Country}
\icmlaffiliation{comp}{Samsung AI Center, Cambridge, UK}
\icmlaffiliation{sch}{University of Edinburgh, UK}

\icmlcorrespondingauthor{Royson Lee}{royson.lee@samsung.com}
% \icmlcorrespondingauthor{Firstname2 Lastname2}{first2.last2@www.uk}

% You may provide any keywords that you
% find helpful for describing your paper; these are used to populate
% the "keywords" metadata in the PDF but will not be shown in the document
\icmlkeywords{Machine Learning, ICML}

\vskip 0.3in
]

%%% SOME KEYWORDS WE CAN ALL USE :D
\newcommand{\method}{FedP\textsuperscript{2}EFT\xspace}
\newcommand{\seen}{{\em seen}\xspace}
\newcommand{\unseen}{{\em unseen}\xspace}
\newcommand{\basemodel}{{\em base model}\xspace}


\newcommand*\circled[1]{\tikz[baseline=(char.base)]{
            \node[shape=circle,fill,inner sep=1.0pt,scale=0.8] (char) {\textcolor{white}{#1}};}}
\interfootnotelinepenalty=10000
\setlength{\belowdisplayskip}{4.0pt} \setlength{\belowdisplayshortskip}{4.0pt}
\setlength{\abovedisplayskip}{4.0pt} \setlength{\abovedisplayshortskip}{4.0pt}

\renewcommand\theadfont{}
% this must go after the closing bracket ] following \twocolumn[ ...

% This command actually creates the footnote in the first column
% listing the affiliations and the copyright notice.
% The command takes one argument, which is text to display at the start of the footnote.
% The \icmlEqualContribution command is standard text for equal contribution.
% Remove it (just {}) if you do not need this facility.

\printAffiliationsAndNotice{}  % leave blank if no need to mention equal contribution
% \printAffiliationsAndNotice{\icmlEqualContribution} % otherwise use the standard text.

\begin{abstract}
Federated learning (FL) has enabled the training of multilingual large language models (LLMs) on diverse and decentralized multilingual data, especially on low-resource languages. To improve client-specific performance, personalization via the use of parameter-efficient fine-tuning (PEFT) modules such as LoRA is common. 
This involves a {\em personalization strategy} (PS), such as the design of the PEFT adapter structures (\textit{e.g.}, in which layers to add LoRAs and what ranks) and choice of hyperparameters (\textit{e.g.}, learning rates) for fine-tuning. Instead of manual PS configuration, we propose \method{}, a federated {\em learning-to-personalize} method for multilingual LLMs in cross-device FL settings. Unlike most existing PEFT structure selection methods, which are prone to overfitting low-data regimes, \method{} collaboratively learns the optimal personalized PEFT structure for each client via Bayesian sparse rank selection. Evaluations on both simulated and real-world multilingual FL benchmarks demonstrate that \method{} largely outperforms existing personalized fine-tuning methods, while complementing a range of existing FL methods.


%%%% VERSION WITHOUT LATEX FOR OPENREVIEW (30 Jan 2025 20:56)
% Federated learning (FL) has enabled the training of multilingual large language models (LLMs) on diverse and decentralized multilingual data, especially on low-resource languages. To improve client-specific performance, personalization via the use of parameter-efficient fine-tuning (PEFT) modules such as LoRA is common. This involves a personalization strategy (PS), such as the design of the PEFT adapter structures (e.g., in which layers to add LoRAs and what ranks) and choice of hyperparameters (e.g., learning rates) for fine-tuning. Instead of manual PS configuration, we propose FedP$^2$EFT, a federated learning-to-personalize method for multilingual LLMs in cross-device FL settings. Unlike most existing PEFT structure selection methods, which are prone to overfitting low-data regimes, FedP$^2$EFT collaboratively learns the optimal personalized PEFT structure for each client via Bayesian sparse rank selection. Evaluations on both simulated and real-world multilingual FL benchmarks demonstrate that FedP$^2$EFT largely outperforms existing personalized fine-tuning methods, while complementing a range of existing FL methods.

% OLD
% We consider the problem of {\em learning to personalize} a large-scale deep model driven by Federated Learning (FL). Personalization of deep networks typically amounts to adding parameter-efficient fine-tuning (PEFT) modules such as LoRAs and fine-tuning them on private client data. Hence, personalization involves a {\em personalization strategy} (PS) such as the choice of the PEFT adapter structures (\textit{e.g.}, in which layers to add LoRAs and what ranks) and hyperparameter choices for fine-tuning (\textit{e.g.}, learning rates). Conventional practice is to define the PS manually, but in this paper we aim to solve it in an FL manner. 
% We specifically focus on PEFT adapter structure learning through FL, more concretely FL-based optimal LoRA rank selection. While most existing rank selection methods are prone to overfitting, especially in low-data regimes, we propose a more reliable and robust approach that collaboratively learns the optimal personalized LoRA ranks for each client from a Bayesian sparse model selection perspective.
% We empirically demonstrate our method complements a range of popular FL methods, largely outperforming existing non-FL and FL personalization baselines on both simulated and real-world multilingual FL benchmarks. 
\end{abstract}



\section{Introduction}
\label{sec:intro}


\ps{Challenges of technology scaling}

The growing demand for computing performance has always been met by increasing the number of transistors per chip, which is only possible due to CMOS technology scaling.
However, as we keep pushing the boundaries of technology scaling, we encounter multiple challenges.
Firstly, whenever we transition to a more advanced technology node, the non-recurring cost due to physical design, verification, software, mask sets, and prototyping almost doubles \cite{cost-tech-node}.
As a result, designing a chip in an advanced technology node is only economically viable if the chip is manufactured in vast quantities.
Secondly, many chip components such as I/O drivers, analog circuits, or \gls{srams} have reached their scaling limit.
This means that we cannot shrink these components further, even if we use a more advanced technology with a smaller feature size.
Thirdly, advanced technology nodes suffer from high defect rates, diminishing the yield and inflating the recurring cost.
To tackle these challenges, new chip-design paradigms have been developed.

\ps{Why 2.5D integration?}

One of these new paradigms is 2.5D integration, where multiple silicon dies called chiplets are integrated into the same package.
Once designed, a single chiplet can be reused in multiple 2.5D stacked chips, which increases the ratio of production volume to non-recurring cost.
Another advantage is that multiple chiplets - fabricated in different technologies - can be integrated into the same package.
This means that only components that can take full advantage of technology scaling are built in bleeding-edge technologies.
Components that have reached their scaling limit are fabricated in more mature and hence less costly technology nodes.
Furthermore, chiplets are smaller than monolithic chips.
Therefore, manufacturing chiplets results in less silicon area loss due to fabrication defects and hence a higher yield.
Due to these economic advantages, chip vendors such as AMD \cite{amd-chiplet} and NVIDIA \cite{chiplet-book} have adopted the 2.5D integration paradigm.  

\ps{Challenges of 2.5D integration}

An important challenge when designing 2.5D stacked chips is the construction of a low-latency and high-throughput \gls{ici}. 
To build an \gls{ici}, we connect different chiplets using \gls{d2d} links.
These links are fabricated in an organic package substrate, silicon bridge, or silicon interposer, and they are connected to the chiplets using \gls{c4} bumps or microbumps.
The number of bumps per chiplet is limited, and so is the bandwidth of \gls{d2d} links.
In addition to having lower bandwidth than links in monolithic chips, \gls{d2d} links also have higher latency.
This latency is caused by wire delay and by \gls{phys} that are necessary in both the sending and the receiving chiplet.
\gls{phys} are needed to convert between protocols, voltage levels, and frequencies, which are usually different between on-chiplet links and \gls{d2d} links.
Due to these limitations, the \gls{ici} can quickly become a bottleneck.

\ps{How we solve these challenges differently than the related work does.}

Existing approaches to maximize the performance of the \gls{ici} either optimize the placement of chiplets (with potentially heterogeneous shapes) for a predetermined \gls{ici} topology 
\cite{ho,liu,seemuth,eris,osmolovskyi,tap25d,chiou}, select one topology out of a set of candidates \cite{coskun-1, coskun-2}, or they optimize the \gls{ici} topology for a 2D grid of homogeneously shaped chiplets on an active interposer \cite{butterdonut, cluscross, kite}.
To the best of our knowledge, there is no prior work on \gls{ici} topologies for chips with heterogeneously shaped chiplets or with passive silicon interposers or silicon bridges.
To fill this gap, we propose \name, a novel optimization methodology to jointly optimize the chiplet placement and \gls{ici} topology of such architectures.
\ifnb
\else
\newpage
\fi

\ps{Details on \name~and the key idea}

The key idea is as follows: 
We optimize the chiplet placement without a predetermined topology.
For each placement generated by an optimization algorithm, we infer a placement-based \gls{ici} topology by connecting chiplets that are in close proximity in that specific placement.
We then compute the latency and throughput of this combination of placement and topology for different traffic types.
These latencies and throughputs together with the total chip area are used to compute a user-defined quality-score of the placement, which is returned to the optimization algorithm.
Based on this quality score, the algorithm can further optimize the placement.
By following this iterative process, we jointly optimize the chiplet placement and the \gls{ici} topology.

\ps{Short evaluation-summary}

We provide our open-source framework implementing the proposed placement and topology co-optimization methodology, which we evaluate using both synthetic traffic and traffic traces.
A 2D grid of chiplets with a mesh topology is used as a baseline since many proposals for 2.5D stacked chips \cite{dataflow_accel_dnn, cifher, simba, hecaton, dojo} use such an architecture.
We reduce the latency of synthetic L1-to-L2 and L2-to-memory traffic, the two most important traffic types for cache coherency traffic, by up to 28\% and 62\% respectively.
For real traffic traces, we reduce the average packet latency for almost all traces and architectures considered (reduced by an 8\% or 18\% on average depending on the configuration of \gls{phys} within a chiplet).

\section{Related Work}
\label{sec:related}
\textbf{Open-Set Understanding:}
The open-set task, first introduced by Scheirer et al. \cite{scheirer2012toward}, challenges the conventional closed-set paradigm commonly assumed in image recognition. In closed-set models, the testing phase only includes samples from a predefined set of classes known to the model during training. Conversely, open-set recognition addresses scenarios where samples can belong to previously unknown classes that were not present during training. This requires models to both recognize and reject instances from unfamiliar classes, ensuring robustness against unknown inputs. The open-set framework has seen extensive study across multiple areas in computer vision, such as image classification~\cite{bendale2016towards, vaze2022open, yoshihashi2019classification, oza2019c2ae, perera2020generative, chen2021adversarial,zhang2020hybrid}, %
object detection~\cite{han2022expanding, miller2020uncertainty, zhou2023open}, %
and image segmentation~\cite{hwang2021exemplar, pham2018bayesian, cen2021deep}.%


\textbf{Open-vocabulary Semantic Segmentation (OVSS):}
Recent advances in vision-language models (VLMs) such as CLIP~\cite{radford2021learning} have demonstrated that robust, transferable visual representations can be effectively learned from large-scale datasets using only weakly structured natural language descriptions.
Initially adopted in image-level tasks like classification, VLMs leverage both visual and textual embeddings to recognize a diverse set of classes. %
By aligning image features with semantic concepts in a shared space, VLMs achieve a form of zero-shot learning that allows them to identify new classes at test time, offering a flexible framework for generalization \cite{liu2024open,xie2024sed,cho2024cat}. 
Although open-vocabulary learning presents an appealing solution for handling arbitrary classes, scaling this approach to accommodate an ever-growing set of classes poses significant challenges. In theory, a VLM could achieve perfect generalization if its query set contained every conceivable class label. However, as demonstrated by Miller et al. \cite{miller2025open}, adding more classes to the query set does not lead to better performance. In fact, increasing the number of class labels introduces a greater likelihood of misclassifications, leading to degraded model accuracy. This degradation occurs because, as more classes are added, the semantic space becomes increasingly crowded, causing overlaps that make accurate distinctions between classes harder to achieve. 
To tackle scalability, one solution is training class-free models \cite{shin2024towards}, while distinguishability can be improved by enhancing the textual descriptions of the classes \cite{ma2024open,jiao2024collaborative}. However, all these approaches assume that the inference label set is predefined and available at inference time.%




\textbf{Vocabulary-Free Semantic Segmentation (VSS):} Recent research in VSS has focused on developing end-to-end solutions while reducing bias from  ground truth data annotations. The majority of current methods decompose the task into a class-agnostic mask generation and a class association (Mask2Tag). Zero-Seg \cite{rewatbowornwong2023zero} and TAG \cite{kawano2024tag} leverage DINO \cite{caron2021emerging} to generate the masks, followed by CLIP-based \cite{radford2021learning} embedding generation for class association. Zero-Seg processes these embeddings %
following ZeroCap \cite{tewel2021zero} to obtain textual classes, while TAG matches them against an external database. 
Conversely, CaSED \cite{conti2024vocabulary} identifies potential classes by querying an external caption database using a pre-trained VLM. %
Similarly, Auto-Seg \cite{ulger2024autovocabularysemanticsegmentation} %
fine-tunes a captioning model to extract class names at multiple scales, followed by a second stage where an open-vocabulary model generates segmentation masks, with predictions remapped to ground truth classes using LLMs. While these approaches demonstrate promising results, they do not fully explore how this pipeline decomposition impacts model performance, nor do they investigate methods to enrich the textual information in the CLIP encoder. We address these limitations by providing a comprehensive analysis of the text encoder's role and exploring techniques to enhance visual-language understanding through richer textual representations. Moreover, we rigorously test the sensitivity of CLIP to tagger errors, evaluating how inaccuracies in image tagging propagate and impact the final segmentation performance.

\section{Approach}\label{sec:approach} 

In this section, we first formally define the OOD code detection problem for (NL, PL) models (Sec. IV-A), then introduce the overall proposed framework (Sec. IV-B), and finally present details of unsupervised COOD and weakly-supervised COOD+ in Sec. IV-C and Sec. IV-D, respectively.
\subsection{Problem Statement}

Since current state-of-the-art code-related models~\cite{guo2020graphcodebert, guo2022unixcoder} typically extract code semantics by capturing the semantic connection between NL (\ie comment) and PL (\ie code) modalities, we formally defined OOD samples involving these two modalities in the SE context by following the convention in ML~\cite{yang2021generalized, zhou2021contrastive}. Consider a dataset comprising training samples $((t_1, c_1), y_1), ((t_2, c_2), y_2), ...$ from the joint distribution $P((T, C),Y)$ over the space $\mathcal{(T, C)}\times \mathcal{Y}$, and a neural-based code model is trained to learn this distribution. Here, $((t_1, c_1), y_1)$ represents the first input pair of (comment, code) along with its ground-truth prediction in the training corpus. $T$, $C$ and $Y$ are random variables on an input (comment, code) space $\mathcal{(T, C)}$ and a output (semantic) space $\mathcal{Y}$, respectively. OOD code samples refer to instances that typically deviate from the overall training distribution due to distribution shifts. The concept of distribution shift is very \textit{broad}~\cite{yang2021generalized, salehiunified} and can occur in either the marginal distribution $P(T, C)$, or both $P(Y)$ and $P(T, C)$.


We then formally define the OOD code detection task following~\cite{hsu2020generalized, liu2020energy, tian2020few, kim-etal-2023-pseudo} as follows. Given a main code-related task (\eg clone detection, code search, \etc), the objective here is to develop an \textit{auxiliary} scoring function $g: \mathcal{(T, C)} \rightarrow \mathcal{R}$ that assigns higher scores to normal instances where $ ((t, c), y) \in P((T, C),Y)$, and lower scores to OOD instances where $((t, c), y) \notin P((T, C),Y)$. Based on whether to use OOD instances during the main-task training of pre-trained NL-PL models, we define OOD for code in two settings, namely unsupervised and weakly-supervised learning. For the unsupervised setting, only normal data is used in the main-task training. Conversely, weakly-supervised approaches utilize ID and a tiny collection of OOD data (\eg 1\% of ID data)~\cite{tian2020few} in training. In this context, the output space $\mathcal{Y}$ is typically a binary set, indicating normal or abnormal, which is probably unknown during inference. Due to the small number of training OOD data, the OOD samples required by our COOD+ and other existing weakly-supervised approaches~\cite{ruff2020deep, tian2020few} in ML can be generated at minimal cost and feasibly verified by human experts when necessary.

% \textcolor{blue}{In line with prior work~\cite{ming2022delving, DBLP:journals/corr/abs-2104-08812}, a threshold should be set during inference to filter out OOD code instances and retain most ID code instances (e.g., 95\%), considering that real-world deployment usually involves a small number of OODs. Ensuring a high proportion of ID data is crucial to prevent the OOD auxiliary scoring from adversely affecting the performance of the models on their main code-related tasks.}

%By systematically generating OOD data representative of realistic SE scenarios, we aim to show the merits of OOD detection in SE deployment settings. Moreover, weak supervision provides guarantees that the performance of the main downstream task is not adversely affected by the OOD-aware finetuning process~\cite{majhi2021weakly}.

% When OOD samples are drawn from a distribution resulted from distribution shifts on $P(X,Y)$, they do not conform to the patterns, characteristics, or statistical properties of the original $P(X,Y)$ previously used to develop, train, and validate the model, leading to performance and reliability concerns. Hence, it is important to detect when SE models encounter OOD samples for appropriate counter measures, safeguarding their deployment in practical settings.

\subsection{Overview}
\begin{figure*}[!thb]
\begin{center}
 \includegraphics[width=0.98\textwidth]{figs/COOD.pdf}
    \caption{The Overview of Our Proposed COOD and COOD+ Approaches for OOD Detection %The COOD+ approach consists of two main components: a contrastive learning module and a binary OOD rejection network. (1) For unsupervised COOD, only the contrastive learning module is utilized, focusing on maximizing the cosine similarity difference between ID and OOD data. (2)The weakly-supervised COOD includes both the contrastive learning module and the binary OOD rejection network which identifies OOD samples with low ID probabilities using a threshold.
    }
    \label{fig:cad}
\end{center}

\end{figure*}

Overall, there are two versions of our COOD approach: unsupervised COOD and weakly-supervised COOD+. Given a multi-modal (NL, PL) input, the unsupervised COOD learns distinct representations based on a contrastive learning module by utilizing a pre-trained Transformer-based code representation model (\ie GraphCodeBERT~\cite{guo2020graphcodebert}). Then, these representations are mapped to distance-based OOD detection scores in order to indicate whether the test samples are OODs during inference. The weakly-supervised COOD+ further integrates a improved contrastive learning module with a binary OOD rejection module to enhance the detection performance by using a very tiny number of OOD data during model training. The OOD samples are then identified by the detection scores produced by the contrastive learning module as well as the prediction probabilities of the binary OOD rejection module. 

\subsection{Unsupervised COOD}

Our unsupervised COOD approach consists of a contrastive learning (CL) module trained only on ID samples. Specifically, given (comment, code) pairs as input, we fine-tune a comment encoder and a code encoder through a contrastive objective to learn discriminative features, which are expected to help identify OOD samples based on a scoring function. 

The (comment, code) pairs are first converted into the comment and code representations, which are processed by the comment and code encoder, respectively. We use the pre-trained GraphCodeBERT model~\cite{guo2020graphcodebert} as the encoder architecture (\ie backbone). GraphCodeBERT is a Transformer-based model pre-trained on six PLs by taking the (comment, code) pairs as well as the data flow graph of the code as input, which has shown superior performance on code understanding and generation tasks. All the representations of the last hidden states of the GraphCodeBERT encoder are averaged to obtain the sequence-level features of comment and code.


\textbf{Contrastive Learning Module.} To achieve the contrastive learning objective, we fine-tune the base (GraphCodeBERT) encoders with the InfoNCE loss~\cite{oord2018representation}. The comment and code encoders follow the Siamese architecture~\cite{guo2022unixcoder} since they are designed to be identical subnetworks with the same GraphCodeBERT backbones, in which their parameters (\ie weights and biases) are shared during fine-tuning. Parameter sharing can reduce the model size and has shown state-of-the-art performance for the code search task~\cite{shi2023cocosoda}. To extract discriminative features for (comment, code) pairs, we organize them into functionally-similar positive pairs and dissimilar negative (unpaired) pairs. Through a contrastive objective, positive pairs are drawn together, while unpaired comment and code are pulled apart. Specifically, for each positive (comment, code) pair $(t_i, c_i)$ in the batch, the code in each of other pairs and $t_i$ are constructed as in-batch negatives, similarly for the comment side. The loss function then formulates the contrastive learning as a classification task, which maximizes the probability of selecting positives along the diagonal of the similarity matrix (as shown in Fig. \ref{fig:cad}) by taking the \textit{softmax} of projected embedding similarities across the batch.  The loss function can be summarized as follows: 

  
\begin{align}\small
\label{eq:infonce}
\begin{split}
    \mathcal{L}^{CL} &= -\frac{1}{2N}(\sum_{n=1}^{N}\log \frac{e^{sim(v_{t_i},v_{c_i})/\tau}}{\sum_{j=1}^{N}e^{sim(v_{t_i},v_{c_j})/\tau}}\\ &+ \sum_{n=1}^{N}\log \frac{e^{sim(v_{t_i},v_{c_i})/\tau}}{\sum_{j=1}^{N}e^{sim(v_{t_j},v_{c_i})/\tau}})
\end{split}
\end{align}
where $v_{t_i}$ and $v_{c_i}$ represent the extracted features of the comment $t_i$ and the code ${c_i}$. $\tau$ is the temperature hyperparameter, which is set to 0.07 following previous work~\cite{shi2023cocosoda}.  $sim(v_{c_i},v_{t_i})$ and $sim(v_{t_i},v_{c_j})/sim(v_{t_j},v_{c_i})$ represent the cosine similarities between comment and code features for positive and negative pairs, respectively. $N$ is the number of input pairs in the batch. InfoNCE loss is designed for self-supervised learning and learns to distinguish positive pairs from in-batch negatives. Compared to other contrastive losses~\cite{khosla2020supervised, zhou2021contrastive}, it can take advantage of large batch size to automatically construct many diverse in-batch negatives for robustness representation learning, which is more effective to capture the alignment information between comment and code. % Despite~\cite{khosla2020supervised, zhou2021contrastive} being able to learn class-aware representations, they are designed for supervised learning which requires labeled data, and specifically for the code search task, a large amount of negative pairs. 


\textbf{Scoring Function.} Existing OOD detection techniques in ML derive scoring functions based on model's output, which typically map the learned class-probabilistic distributions to OOD detection scores for testing samples. Maximum Softmax Probability (MSP)~\cite{hendrycks2016baseline} is commonly used for OOD scoring. This method uses the maximum classification probability $\max_{l\in L}\textit{softmax}(f(v_{t}, v_{c}))$, where $f(v_{t}, v_{c})$ is the output of the classification model, with low scores indicating low likelihoods of being OOD. However, NL-PL code search models typically utilize the similarity retrieval scores of NL-PL output representations to make predictions. Therefore, to enable simultaneous similarity and OOD inference, we alternatively extract cosine similarity scores of testing NL-PL pairs as OOD detection scores, denoted as $P^{CL}=sim(v_{c},v_{t})$. The underlying intuition behind this scoring metric is that OOD testing samples should receive low retrieval confidence from the model fine-tuned on ID data, which establishes a closer relationship between ID (comment, code) pairs. Hence, this scoring function also assigns higher scores to ID data and lower scores to OOD data similar to previous scoring methods.

\subsection{Weakly-Supervised COOD+}

To further enhance the performance of unsupervised COOD, we extend it to a weakly-supervised detection model, called COOD+, which takes advantage of a few OOD examples. Inspired by~\cite{duong2024general}, our COOD+ combines an improved contrastive learning (CL) and a binary OOD rejection classifier (BC). The improved CL module adopts a margin-based loss~\cite{xue2009svm} which enforces a margin of difference between the cosine similarities of aligned and unaligned (comment, code) pairs, and constrains the cosine similarities of OOD pairs below another margin. The BC module integrates features from both comments and code to calculate the probabilities of OOD pairs. The OOD scoring function is then designed by combining the cosine similarity scores from the CL module and the prediction probabilities from the BC module. Below, we detail each component of our weakly-supervised COOD+.

\textbf{Improved Contrastive Learning (CL) Module.} Given a batch of $N$ input pairs (comprising $N-K$ ID pairs and $K$ OOD pairs), the latent representations are first obtained from the comment and code encoders. Then the margin-based loss is leveraged in the CL module to distinguish representations of ID and OOD data by constraining the cosine similarity. Specifically, the margin-based contrastive loss is first applied to $N-K$ ID code to maximize the difference between aligned (comment, code) pairs and incorrect pairs for each batch: 


%%%
\begin{equation}\scriptsize
\label{eq:contrast_id}
    \hspace{-0.2cm} \mathcal{L}^{ID} = \sum_{i=1}^{N-K} \left(\frac{1}{N}\sum_{j=1, j \neq i}^N \max \left(0,m - s(v_{t_i^+}, v_{c_i^+}) + s(v_{t_j^-}, v_{c_i^+})\right)\right)
\end{equation}


\noindent $s(v_{t_i^+}, v_{c_i^+})$ represents the cosine similarity of representations between each aligned ID pair from all the $N-K$ aligned pairs, and $s(v_{t_j^-},v_{c_i^+})$ represents the cosine similarity of representations between each ID code and all the other $N-1$ comments (\ie the comment is either not aligned with the ID code or from OOD comments). Thus, this margin-based loss encourages the difference between the aligned pairs and the incorrect pairs greater than margin $m$. 

Regarding the $K$ OOD code, we enforce a constraint on the cosine similarity between each OOD code and all the comments, ensuring that the similarity remains below a margin $m$. This constraint is necessary because each OOD code should not align with its corresponding comment, nor with any of the other $K-1$ OOD comments and the $N-K$ ID comments. The loss function is denoted as follows:
\begin{equation}\small
\label{eq:contrast_ood}
\mathcal{L}^{OOD} = \sum_{k=1}^{K}\left(\frac{1}{N}\sum_{i=1}^N \max \left(0,-m + sim(t_j^-, c_k^-)\right)\right),
\end{equation}

\noindent where $sim(t_j^-, c_k^-)$ represents the cosine similarity between each of the $K$ OOD code and all $N$ comments.
Finally, the overall loss for the contrastive module can be expressed as: 


\begin{equation}\small
\label{eq:contrast}
    \mathcal{L}^{CL} = \frac{1}{N}\left(\mathcal{L}^{ID}+ \mathcal{L}^{OOD}\right).
\end{equation}


\textbf{Binary OOD Rejection (BC) Module.}
Besides the CL module, we also introduce a classification module under weakly-supervision for identifying OOD samples. %The key challenge lies in effectively integrating features from both NL and PL to improve classification accuracy.
Inspired by the Replaced Token Detection (RTD) objective utilized in~\cite{feng2020codebert}, we bypass the generation phase since our OOD data are generated prior to training. Therefore, we directly train a rejection network responsible for determining whether (comment, code) pairs are OOD or not, which can be framed as a binary classification problem. Our binary OOD rejection network comprises a 3-layer fully-connected neural network with \textit{Tanh} activation, and the input is based on the concatenation of features from the comment and code encoders: 
$v_i = (v_{t_i}, v_{c_i}, v_{t_i} - v_{c_i}, v_{t_i} + v_{c_i}).$ Apart from utilizing the comment and code features, we also incorporate feature subtraction $v_{t_i} - v_{c_i}$ and aggregation $v_{t_i} + v_{c_i}$.  Additionally, we apply the sigmoid function to the output layer, producing a prediction probability that indicates whether the sample is OOD. We then use binary cross entropy loss for this module:

\begin{equation}\small
    \mathcal{L}^{BC} = \frac{1}{N}\sum_{i=1}^{N-K}(y_i\log p(v_i)+(1-y_i)\log(1-p(v_i))),
\end{equation}


\noindent where $p(v_i)$ is the output probability of the BC module, and $y_i \in [0, 1]$ is the ground-truth label. $y_i = 1$ indicates the input sample is an inlier, while $y_i = 0$ signifies it is an outlier. 

Hence, for weakly-supervised COOD+, we combine the objectives of the CL and the BC modules to jointly train our model, where $\lambda$ is a weight used to balance the loss functions:

\begin{equation}\small
\label{eq:loss}
    \mathcal{L} = \mathcal{L}^{CL} + \lambda \mathcal{L}^{BC}.
\end{equation}

\noindent\textbf{Combined Scoring Function.}
Similar to the unsupervised COOD approach, we utilize the diagonals of the similarity matrix as the OOD detection scores obtained from the CL module. To further improve the detection performance of the weakly-supervised version, we combine these $P^{CL}$ scores with the output probabilities of the BC module, denoted as $P^{BC}$. Here, we convert cosine similarity scores into probabilities using the sigmoid function $P^{CL*}=\sigma(sim(v_{c},v_{t}))$, then use multiplication to create the overall scoring function, yielding $P^{ID}=P^{CL*}\times P^{BC}$. We anticipate that higher scores will be assigned to ID pairs, while lower scores will be assigned to OOD pairs. This combined scoring function aims to enhance the discrimination between inliers and outliers, leading to more effective OOD detection.
% \newpage
\section{Evaluation}\label{sec:expmts}

\subsection{Experimental Setup}

We conduct experiments on multilingual scenarios, where clients with diverse high- and low-resource languages can collaboratively learn how to personalize a given base model to better cater to their language preferences. In all experiments, we divide clients in two pools, \seen{} and \unseen{}, where only the clients in the \seen{} pool actively participate in federated training. We set the maximum number of communication rounds for training the PSG to $150$, randomly sampling $10\%$ of participating clients every round. We use Adam~\cite{Kingma_2014} as the default optimizer for all our experiments. We evaluate on resource budgets $r=2,4,8,16$ where the total rank budget is $r \cdot L$. We summarize the FL scenarios that we consider in our experiments, leaving comprehensive details in Appendix~\ref{appendix:experiments}.

\subsubsection{Tasks, Models, and Datasets}

\noindent\textbf{Text Classification.}~We adopt the pretrained multilingual BERT~\cite{BERT} (mBERT) for all text classification experiments. For datasets, we introduce additional data heterogeneity to the simulated FL setups, XNLI~\cite{XNLI} and MasakhaNEWS~\cite{MasakhaNEWS}, proposed in PE\_FL~\cite{zhao2023breaking}. 

For our XNLI setup, we sample 2k instances for train and $500$ for test in each pool. In contrast to PE\_FL, which had $15$ clients ($1$ language per client), we divide the data equally among $20$ clients for each language. We then adopt the latent Dirichlet allocation (LDA) partition method~\cite{hsu2019measuring, yurochkin2019bayesian}, $y \sim Dir(\alpha)$, to simulate non-IID label shifts among these clients, with $\alpha=0.5$. Hence, there is a total of $600$ clients ($15$ languages $\cdot$ $20$ clients per language $\cdot$ $2$ pools), consisting of both label and feature heterogeneity.

For MasakhaNEWS, we first split the data in each of the $16$ languages by half for each pool. Similar to our XNLI setup, we divide each language's data equally among 10 clients and adopt LDA with $\alpha=0.5$, resulting in $320$ clients in total. Differing from our XNLI setup, each language varies in the amount of samples, adding another layer of data heterogeneity to the setup: quantity skew.

\noindent\textbf{Instruction-Tuning Generation.}~We adopt pretrained MobileLLaMA-1.4B~\cite{mobilellama} and Llama-3.2-3B~\cite{llama3}, which are representative of commonly supported model sizes on recent high-end edge devices~\cite{openelm2024icmlw,2024_mobilequant,edgellm2024tmc}. For each model, we run experiments on the recent Fed-Aya dataset. Fed-Aya is a real-world FL dataset naturally partitioned by annotator ID and each client has data with up to $4$ languages. Out of a total of $38$ clients, we select $8$ clients for our \unseen{} pool. We also split each client's data $80\%/20\%$ for train and test, respectively. Fig.~\ref{fig:fed-aya} shows the distribution of predominant languages, where predominant refers to the client's most commonly used language, in our setup.

\begin{figure}
    \small
    \centering
    \includegraphics[width=0.9\columnwidth]{figures/fedaya_pool.png}
    % \captionsetup{font=small,labelfont=bf}
    \vspace{-1.5em}
    \caption{The number of clients in each predominant language in our Fed-Aya setup.}
    \label{fig:fed-aya}
    \vspace{-2em}
\end{figure}

\subsubsection{Complementary Approaches}\label{sec:complementary}

We show \method{}'s compatibility with both off-the-shelf models and models trained using existing FL methods. Concretely, given a pretrained model, we obtain a \basemodel{} using one of the following approaches: 

\noindent\textbf{Standard FL.} We further train the pretrained model federatedly on the \seen{} pool, either using existing PEFT methods or full fine-tuning~\cite{fedllm-bench, fedpeft},  

\noindent\textbf{Personalized FL.} We adopt two recent personalized FL works: \textit{i)} FedDPA-T~\cite{FedDPA}, which learns per-client personalized LoRA modules in addition to global LoRA modules, and \textit{ii)} DEPT (SPEC)~\cite{DEPT}, which learns per-client personalized token and positional embeddings while keeping the rest of the model shared. The \basemodel{} hence differs for each client.

\noindent\textbf{Off-the-shelf.} We use the pretrained model as the \basemodel{} without additional training.

\begin{table*}[t]
\centering
% \captionsetup{justification=centering}
\caption{Mean±SD Accuracy of each language across 3 different seeds for \seen{} clients of our MasakhaNEWS setup. The pretrained model is trained using Standard FL with full fine-tuning and the resulting \basemodel{} is personalized to each client given a baseline approach.}
\label{tab:masakha_seen}
\begin{scriptsize}\resizebox{0.98\textwidth}{!}{
\begin{tabular}{c|l|l|l|l|l|l|l|l|l|l|l|l|l|l|l|l|l|c}
\toprule
% \textbf{Lora Rank}  
\textbf{$\mathbf{r}$} & \multicolumn{1}{c|}{\textbf{Approach}} & \multicolumn{1}{c|}{\textbf{eng}} & \multicolumn{1}{c|}{\textbf{som}} & \multicolumn{1}{c|}{\textbf{run}} & \multicolumn{1}{c|}{\textbf{fra}} & \multicolumn{1}{c|}{\textbf{lin}} & \multicolumn{1}{c|}{\textbf{ibo}} & \multicolumn{1}{c|}{\textbf{amh}} & \multicolumn{1}{c|}{\textbf{hau}} & \multicolumn{1}{c|}{\textbf{pcm}} & \multicolumn{1}{c|}{\textbf{swa}} & \multicolumn{1}{c|}{\textbf{orm}} & \multicolumn{1}{c|}{\textbf{xho}} & \multicolumn{1}{c|}{\textbf{yor}} & \multicolumn{1}{c|}{\textbf{sna}} & \multicolumn{1}{c|}{\textbf{lug}} & \multicolumn{1}{c|}{\textbf{tir}} & \textbf{Wins} 
\\ \midrule
% \multirow{5}{*}{1}  & LoRA                                   & 90.01±0.10                        & 59.41±0.32                        & 81.16±0.29                        & 88.63±0.00                        & 83.53±0.54                        & 78.97±0.00                        & 45.74±0.00                        & 75.58±0.15                        & 96.05±0.00                        & 78.99±0.34                        & 64.20±0.00                         & 69.37±0.32                        & 79.02±0.00                        & 78.80±0.00                         & 67.57±0.00                        & 44.85±0.00                        & 0             \\ % \cline{2-19} 
%                     & AdaLoRA                              & 89.87±0.00                        & 59.63±0.32                        & 80.96±0.29                        & 88.63±0.00                        & 83.14±0.54                        & 78.97±0.00                        & 45.21±0.00                        & 75.16±0.00                        & 96.05±0.00                        & 78.99±0.34                        & 64.20±0.00                         & 69.59±0.00                        & 79.02±0.00                        & 78.80±0.00                         & 67.87±0.42                        & 44.85±0.00                        & 0             \\ % \cline{2-19} 
%                     & BayesTune-LoRA                            & 89.87±0.00                        & 59.63±0.32                        & 81.16±0.29                        & 88.63±0.00                        & 83.91±0.00                        & 78.97±0.00                        & 45.21±0.00                        & 75.05±0.15                        & 96.05±0.00                        & 78.71±0.20                        & 64.20±0.00                         & 69.37±0.32                        & 78.86±0.23                        & 78.80±0.00                         & 67.57±0.00                        & 44.85±0.00                        & 0             \\ % \cline{2-19} 
%                     & FedL2P                               & 89.87±0.00                        & 59.86±0.00                        & 81.37±0.00                        & 88.63±0.00                        & 82.76±0.00                        & 79.49±0.00                        & 45.21±0.00                        & 75.68±0.15                        & 96.05±0.00                        & 78.85±0.20                        & 64.20±0.00                         & 69.37±0.32                        & 79.02±0.00                        & 78.80±0.00                         & 67.57±0.00                        & 44.85±0.00                        & \textbf{0}    \\ % \cline{2-19} 
%                     & \method{}                                 & \textbf{92.05±0.10}                & \textbf{64.63±0.56}               & \textbf{85.71±0.00}                & \textbf{91.31±0.22}               & \textbf{87.36±0.00}                & \textbf{80.69±0.49}               & \textbf{52.48±0.50}                & \textbf{79.35±0.30}                & 96.05±0.00                        & \textbf{84.87±0.69}               & \textbf{68.72±0.77}               & \textbf{73.42±0.64}               & \textbf{82.93±0.00}                & 78.08±0.25                        & \textbf{69.07±0.42}               & \textbf{52.45±0.34}               & 14            \\ \hline
\multirow{5}{*}{2}  & LoRA                                   & 90.44±0.10                        & 60.09±0.32                        & 81.37±0.51                        & 88.63±0.00                        & 83.53±0.54                        & 79.83±0.24                        & 45.74±0.00                        & 75.79±0.00                        & 96.05±0.00                        & 78.99±0.00                        & 64.20±0.00                         & 69.14±0.32                        & 79.18±0.23                        & 78.80±0.00                         & 67.57±0.00                        & 44.85±0.00                        & 0             \\ % \cline{2-19} 
                    & AdaLoRA                              & 89.87±0.00                        & 59.41±0.32                        & 81.16±0.29                        & 88.63±0.00                        & 82.76±0.00                        & 78.97±0.00                        & 45.21±0.00                        & 75.05±0.15                        & 96.05±0.00                        & 78.57±0.00                        & 63.99±0.29                        & 69.59±0.00                        & 78.86±0.23                        & 78.80±0.00                         & 67.57±0.00                        & 44.85±0.00                        & 0             \\ % \cline{2-19} 
                    & BayesTune-LoRA                            & 89.87±0.00                        & 59.63±0.32                        & 81.37±0.00                        & 88.63±0.00                        & 83.14±0.54                        & 78.97±0.00                        & 45.21±0.00                        & 75.16±0.00                        & 96.05±0.00                        & 78.57±0.00                        & 63.99±0.29                        & 69.59±0.00                        & 79.02±0.00                        & 78.80±0.00                         & 67.57±0.00                        & 44.85±0.00                        & 0             \\ % \cline{2-19} 
                    & FedL2P                               & 90.72±0.59                        & 61.00±1.16                         & 81.99±0.88                        & 89.10±0.67                         & 83.91±0.00                        & 79.66±0.24                        & 45.74±0.00                        & 76.73±1.12                        & 96.05±0.00                        & 79.69±0.40                        & 64.40±0.29                         & 69.14±0.32                        & 79.67±0.61                        & 78.80±0.00                         & 67.87±0.42                        & 45.34±0.69                        & 0             \\ % \cline{2-19} 
                    & \method{}                                 & \textbf{91.98±0.00}                & \textbf{65.54±1.16}               & \textbf{87.79±0.29}               & \textbf{93.52±0.23}               & \textbf{88.51±0.94}               & \textbf{82.56±0.00}                & \textbf{51.95±0.25}               & \textbf{79.66±0.29}               & \textbf{97.59±0.31}               & \textbf{84.73±0.52}               & \textbf{72.22±1.01}               & \textbf{76.35±0.55}               & \textbf{82.11±0.23}               & \textbf{81.34±0.68}               & \textbf{69.07±0.42}               & \textbf{63.48±0.34}               & \textbf{16}   \\ \hline
\multirow{5}{*}{4}  & LoRA                                   & 91.35±0.00                        & 61.90±0.00                         & 83.23±0.00                        & 89.10±0.00                         & 84.68±0.54                        & 80.17±0.24                        & 47.34±0.00                        & 77.57±0.15                        & 96.05±0.00                        & 79.83±0.34                        & 64.61±0.29                        & 69.37±0.32                        & 80.49±0.00                        & 78.80±0.00                         & 67.87±0.42                        & 45.83±0.34                        & 0             \\ % \cline{2-19} 
                    & AdaLoRA                              & 89.87±0.00                        & 59.63±0.32                        & 80.96±0.29                        & 88.63±0.00                        & 83.53±0.54                        & 78.97±0.00                        & 45.21±0.00                        & 75.47±0.00                        & 96.05±0.00                        & 78.57±0.00                        & 63.79±0.29                        & 69.59±0.00                        & 79.02±0.00                        & 78.80±0.00                         & 67.57±0.00                        & 44.85±0.00                        & 0             \\ % \cline{2-19} 
                    & BayesTune-LoRA                            & 89.87±0.00                        & 59.41±0.32                        & 80.96±0.29                        & 88.63±0.00                        & 82.76±0.00                        & 78.97±0.00                        & 45.21±0.00                        & 75.47±0.26                        & 96.05±0.00                        & 78.71±0.20                        & 64.20±0.00                         & 69.59±0.00                        & 79.02±0.00                        & 78.80±0.00                         & 67.57±0.00                        & 44.85±0.00                        & 0             \\ % \cline{2-19} 
                    & FedL2P                               & 91.70±0.40                         & 63.49±1.16                        & 83.23±0.51                        & 90.84±0.97                        & 83.91±0.94                        & 80.00±0.42                         & 47.69±1.40                        & 79.45±1.07                        & 96.05±0.00                        & 81.51±1.03                        & 64.81±0.50                        & 71.62±1.99                        & 81.46±0.80                        & 78.80±0.00                         & 68.17±0.85                        & 46.32±0.60                        & 0             \\ % \cline{2-19} 
                    & \method{}                                 & \textbf{92.05±0.44}               & \textbf{67.12±0.64}               & \textbf{86.34±0.00}                & \textbf{93.36±0.39}               & \textbf{91.19±1.43}               & \textbf{82.56±0.84}               & \textbf{53.90±0.25}                & \textbf{81.66±0.65}               & \textbf{98.46±0.31}               & \textbf{83.89±1.10}                & \textbf{72.84±0.51}               & \textbf{77.70±0.00}                 & \textbf{82.11±0.23}               & \textbf{83.70±0.00}                 & \textbf{74.47±0.42}               & \textbf{67.65±0.00}                & \textbf{16}   \\ \hline
\multirow{5}{*}{8}  & LoRA                                   & 91.56±0.00                        & 63.95±0.00                        & 82.82±0.29                        & 90.84±0.23                        & 86.21±0.00                        & 79.83±0.24                        & 49.82±0.25                        & 78.83±0.15                        & 96.05±0.00                        & 81.79±0.52                        & 65.64±0.29                        & 72.30±0.55                         & 81.79±0.23                        & 79.89±0.00                        & 67.87±0.42                        & 44.85±0.00                        & 0             \\ % \cline{2-19} 
                    & AdaLoRA                              & 89.87±0.00                        & 59.86±0.00                        & 80.96±0.29                        & 88.63±0.00                        & 83.14±0.54                        & 78.97±0.00                        & 45.21±0.00                        & 75.37±0.15                        & 96.05±0.00                        & 78.85±0.20                        & 64.20±0.00                         & 69.59±0.00                        & 78.86±0.23                        & 78.80±0.00                         & 67.57±0.00                        & 44.85±0.00                        & 0             \\ % \cline{2-19} 
                    & BayesTune-LoRA                            & 89.94±0.10                        & 59.41±0.32                        & 80.75±0.00                        & 88.63±0.00                        & 83.14±0.54                        & 78.97±0.00                        & 45.21±0.00                        & 75.47±0.26                        & 96.05±0.00                        & 78.71±0.20                        & 64.20±0.00                         & 69.59±0.00                        & 79.02±0.00                        & 78.80±0.00                         & 67.57±0.00                        & 44.85±0.00                        & 0             \\ % \cline{2-19} 
                    & FedL2P                               & \textbf{91.91±0.26}               & 64.18±0.32                        & 83.02±0.58                        & 91.15±0.59                        & 84.29±0.54                        & 80.51±0.00                        & 48.58±0.67                        & 79.98±0.82                        & 96.27±0.31                        & 82.49±0.79                        & 66.25±1.62                        & 72.07±1.39                        & 81.95±1.05                        & 79.16±0.51                        & 68.47±1.27                        & 47.79±3.65                        & 1             \\ % \cline{2-19} 
                    & \method{}                                 & 91.42±0.20                        & \textbf{68.03±0.96}               & \textbf{85.09±0.00}                & \textbf{92.42±0.39}               & \textbf{91.95±0.00}                & \textbf{81.88±0.24}               & \textbf{55.32±0.43}               & \textbf{82.70±0.26}                & \textbf{98.68±0.00}                & \textbf{84.31±0.20}                & \textbf{73.25±0.58}               & \textbf{78.15±0.32}               & \textbf{84.72±0.46}               & \textbf{85.15±0.26}               & \textbf{74.17±0.42}               & \textbf{68.38±0.60}                & \textbf{15}   \\ \hline
\multirow{5}{*}{16} & LoRA                                   & \textbf{91.91±0.10}                & 64.63±0.00                        & 84.47±0.00                        & 91.47±0.00                        & 86.21±0.00                        & \textbf{81.54±0.00}                         & \textbf{55.85±0.43}               & 81.87±0.15                        & 96.27±0.31                        & \textbf{84.45±0.00}                & 73.66±0.77                        & 73.20±0.32                         & 82.60±0.61                         & 80.80±0.26                         & 70.87±0.42                        & 52.94±0.00                        & 3             \\ % \cline{2-19} 
                    & AdaLoRA                              & 89.87±0.00                        & 59.86±0.00                        & 80.96±0.29                        & 88.63±0.00                        & 82.76±0.00                        & 78.97±0.00                        & 45.21±0.00                        & 75.05±0.39                        & 96.05±0.00                        & 78.85±0.20                        & 63.79±0.29                        & 69.59±0.00                        & 79.02±0.00                        & 78.80±0.00                         & 67.57±0.00                        & 44.85±0.00                        & 0             \\ % \cline{2-19} 
                    & BayesTune-LoRA                            & 90.08±0.18                        & 59.63±0.32                        & 81.16±0.29                        & 88.63±0.00                        & 83.53±0.54                        & 78.97±0.00                        & 45.21±0.00                        & 75.58±0.15                        & 96.05±0.00                        & 78.57±0.00                        & 64.20±0.00                         & 69.37±0.32                        & 79.02±0.00                        & 78.80±0.00                         & 67.57±0.00                        & 44.85±0.00                        & 0             \\ % \cline{2-19} 
                    & FedL2P                               & 91.70±0.10                         & 66.22±0.32                        & \textbf{85.09±0.51}               & \textbf{92.10±0.23}                & 86.21±0.00                        & 80.34±0.24                        & 54.79±1.15                        & 81.13±0.26                        & 96.71±0.00                        & 83.89±0.40                        & 71.81±2.04                        & 75.68±0.00                        & 84.06±0.23                        & 80.61±0.68                        & 70.87±0.85                        & 57.84±4.86                        & 2             \\ % \cline{2-19} 
                    & \method{}                                 & 91.14±0.30                        & \textbf{67.80±0.85}                & 84.47±0.00                        & 91.31±0.22                        & \textbf{91.95±0.00}                & \textbf{81.54±0.42}                        & 51.95±0.25                        & \textbf{83.44±0.39}               & \textbf{98.25±0.31}               & 82.49±0.71                        & \textbf{76.54±0.50}                & \textbf{79.50±0.32}                & \textbf{84.39±0.00}                & \textbf{85.87±0.00}                & \textbf{75.38±0.43}               & \textbf{68.38±0.00}                & \textbf{10}   \\ \bottomrule
\end{tabular}
}
\end{scriptsize}
\vspace{-1.5em}
\end{table*}


\begin{table*}[t]
\centering
\caption{Mean±SD Accuracy of each language for \unseen{} clients of our MasakhaNEWS setup. The pretrained model is trained using Standard FL with full fine-tuning and the resulting \basemodel{} is personalized to each client given a baseline approach.}
\label{tab:masakha_unseen}
\resizebox{0.99\textwidth}{!}{
\begin{tabular}{c|l|l|l|l|l|l|l|l|l|l|l|l|l|l|l|l|l|c}
\toprule
% \textbf{Lora Rank} 
\textbf{$\mathbf{r}$} & \multicolumn{1}{c|}{\textbf{Approach}} & \multicolumn{1}{c|}{\textbf{eng}} & \multicolumn{1}{c|}{\textbf{som}} & \multicolumn{1}{c|}{\textbf{run}} & \multicolumn{1}{c|}{\textbf{fra}} & \multicolumn{1}{c|}{\textbf{lin}} & \multicolumn{1}{c|}{\textbf{ibo}} & \multicolumn{1}{c|}{\textbf{amh}} & \multicolumn{1}{c|}{\textbf{hau}} & \multicolumn{1}{c|}{\textbf{pcm}} & \multicolumn{1}{c|}{\textbf{swa}} & \multicolumn{1}{c|}{\textbf{orm}} & \multicolumn{1}{c|}{\textbf{xho}} & \multicolumn{1}{c|}{\textbf{yor}} & \multicolumn{1}{c|}{\textbf{sna}} & \multicolumn{1}{c|}{\textbf{lug}} & \multicolumn{1}{c|}{\textbf{tir}} & \textbf{Wins}
\\ \midrule
% \multirow{5}{*}{1}  & LoRA                                   & \textbf{90.58±0.20}                & 67.80±0.32                         & 81.37±0.00                        & 84.36±0.00                        & 79.55±0.00                        & 76.92±0.00                        & 46.28±0.00                        & 74.82±0.15                        & 90.20±0.00                         & 75.63±0.00                        & 61.96±0.00                        & 63.76±0.00                        & 79.61±0.00                        & 74.59±0.00                        & 65.18±0.00                        & 41.91±0.00                        & 1             \\ % \cline{2-19} 
%                     & AdaLoRA                              & 90.44±0.27                        & 67.12±0.32                        & 80.96±0.29                        & 84.36±0.00                        & 79.55±0.00                        & 76.92±0.00                        & 46.28±0.00                        & 74.82±0.15                        & 90.20±0.00                         & 75.77±0.20                        & 61.96±0.00                        & 63.98±0.32                        & 79.61±0.00                        & 74.59±0.00                        & 65.18±0.00                        & 41.91±0.00                        & 0             \\ % \cline{2-19} 
%                     & BayesTune-LoRA                            & 90.30±0.00                         & 67.35±0.00                        & 80.75±0.00                        & 84.36±0.00                        & 79.55±0.00                        & 76.92±0.00                        & 46.10±0.25                         & 74.82±0.15                        & 90.20±0.00                         & 75.63±0.00                        & 61.55±0.29                        & 63.76±0.00                        & 79.61±0.00                        & 74.59±0.00                        & 65.18±0.00                        & 41.91±0.00                        & 0             \\ % \cline{2-19} 
%                     & FedL2P                               & 90.15±0.27                        & 67.35±0.56                        & 81.58±0.29                        & 84.36±0.00                        & 79.55±0.00                        & 76.92±0.00                        & 46.28±0.00                        & 75.24±0.00                        & 90.20±0.00                         & 75.49±0.20                        & 61.96±0.00                        & 63.76±0.00                        & 79.94±0.23                        & 74.59±0.00                        & 65.48±0.42                        & 42.16±0.35                        & 0             \\ % \cline{2-19} 
%                     & \method{}                                 & 90.01±0.10                        & \textbf{69.84±0.32}               & \textbf{84.47±0.00}                & \textbf{86.73±0.00}                & \textbf{84.85±0.54}               & \textbf{80.52±0.73}               & \textbf{50.53±1.15}               & \textbf{77.53±0.15}               & \textbf{91.50±0.00}                 & \textbf{76.05±0.34}               & \textbf{69.94±0.87}               & \textbf{71.14±0.55}               & \textbf{82.69±0.61}               & \textbf{75.50±0.25}                & \textbf{66.66±0.42}               & \textbf{48.77±0.34}               & \textbf{15}   \\ \hline
\multirow{5}{*}{2}  & LoRA                                   & \textbf{90.72±0.00}                & 68.48±0.32                        & 81.99±0.51                        & 84.36±0.00                        & 79.55±0.00                        & 76.75±0.24                        & 46.28±0.00                        & 75.45±0.15                        & 90.20±0.00                         & 75.63±0.34                        & 61.96±0.00                        & 64.21±0.63                        & 80.10±0.00                         & 74.59±0.00                        & 65.18±0.00                        & 41.91±0.00                        & 1             \\ % \cline{2-19} 
                    & AdaLoRA                              & 90.30±0.00                         & 67.35±0.00                        & 80.75±0.00                        & 84.36±0.00                        & 79.55±0.00                        & 76.92±0.00                        & 46.28±0.00                        & 74.82±0.15                        & 90.20±0.00                         & 75.91±0.20                        & 61.55±0.29                        & 64.21±0.32                        & 79.61±0.00                        & 74.59±0.00                        & 65.18±0.00                        & 41.91±0.00                        & 0             \\ % \cline{2-19} 
                    & BayesTune-LoRA                            & 90.30±0.00                         & 67.35±0.00                        & 80.75±0.00                        & 84.36±0.00                        & 79.55±0.00                        & 76.92±0.00                        & 45.92±0.25                        & 74.61±0.00                        & 90.20±0.00                         & 75.49±0.20                        & 61.55±0.29                        & 63.76±0.00                        & 79.45±0.23                        & 74.41±0.25                        & 65.18±0.00                        & 42.65±0.00                        & 0             \\ % \cline{2-19} 
                    & FedL2P                               & 90.58±0.10                        & 68.48±0.32                        & 82.61±0.00                        & 84.36±0.00                        & 79.55±0.00                        & 76.41±0.42                        & 46.81±0.75                        & 75.97±0.59                        & 90.20±0.00                         & 75.91±0.40                        & 62.17±0.29                        & 64.43±0.95                        & 80.26±0.23                        & 74.77±0.26                        & 66.07±0.00                        & 41.91±0.00                        & 0             \\ % \cline{2-19} 
                    & \method{}                                 & 90.65±0.10                        & \textbf{72.11±0.56}               & \textbf{88.41±0.29}               & \textbf{88.15±0.39}               & \textbf{86.36±0.00}                & \textbf{81.54±0.42}               & \textbf{51.06±0.00}                & \textbf{79.52±0.30}                & \textbf{93.68±0.31}               & \textbf{78.01±0.40}                & \textbf{73.01±0.50}                & \textbf{78.52±1.10}                & \textbf{83.66±0.23}               & \textbf{80.00±0.44}                & \textbf{69.64±0.00}                & \textbf{58.33±0.34}               & \textbf{15}   \\ \hline
\multirow{5}{*}{4}  & LoRA                                   & 90.79±0.10                        & 68.71±0.00                        & 82.61±0.00                        & 84.36±0.00                        & 78.41±0.00                        & 76.41±0.00                        & 47.87±0.00                        & 75.76±0.30                        & 90.20±0.00                         & 76.05±0.00                        & 63.60±0.29                         & 65.55±0.32                        & 80.58±0.00                        & 74.77±0.26                        & 65.48±0.42                        & 41.91±0.00                        & 0             \\ % \cline{2-19} 
                    & AdaLoRA                              & 90.15±0.10                        & 67.35±0.00                        & 81.58±0.29                        & 84.20±0.22                         & 79.55±0.00                        & 76.92±0.00                        & 46.46±0.25                        & 74.71±0.15                        & 90.20±0.00                         & 75.77±0.20                        & 61.76±0.29                        & 63.76±0.00                        & 79.61±0.00                        & 74.59±0.00                        & 65.18±0.00                        & 42.16±0.35                        & 0             \\ % \cline{2-19} 
                    & BayesTune-LoRA                            & 90.23±0.10                        & 67.12±0.32                        & 80.75±0.00                        & 84.36±0.00                        & 79.55±0.00                        & 76.92±0.00                        & 46.28±0.00                        & 74.71±0.15                        & 90.20±0.00                         & 75.21±0.00                        & 61.35±0.00                        & 63.76±0.00                        & 79.61±0.00                        & 74.59±0.00                        & 65.18±0.00                        & 42.40±0.35                         & 0             \\ % \cline{2-19} 
                    & FedL2P                               & 90.82±0.11                        & 68.37±0.34                        & 83.54±0.93                        & 85.54±1.19                        & 80.12±0.57                        & 76.41±0.51                        & 48.67±0.80                        & 77.43±0.63                        & 90.20±0.00                         & 76.68±0.21                        & 63.80±1.23                         & 67.78±2.02                        & 81.31±0.73                        & 75.95±0.81                        & 66.51±0.45                        & 43.75±1.84                        & 0             \\ % \cline{2-19} 
                    & \method{}                                 & \textbf{90.86±0.36}               & \textbf{72.56±0.32}               & \textbf{86.96±0.00}                & \textbf{91.00±0.39}                & \textbf{87.88±0.54}               & \textbf{81.37±0.48}               & \textbf{52.13±0.00}                & \textbf{81.61±0.15}               & \textbf{94.77±0.00}                & \textbf{79.13±0.20}                & \textbf{74.03±0.29}               & \textbf{80.32±0.32}               & \textbf{83.50±0.00}                 & \textbf{84.86±0.00}                & \textbf{77.08±0.84}               & \textbf{68.87±0.35}               & \textbf{16}   \\ \hline
\multirow{5}{*}{8}  & LoRA                                   & 90.65±0.10                        & 68.71±0.00                        & 83.02±0.29                        & 84.83±0.00                        & 80.68±0.00                        & 77.95±0.00                        & 50.00±0.00                         & 77.43±0.00                        & 90.20±0.00                         & 76.89±0.00                        & 64.42±0.00                        & 66.89±0.32                        & 82.04±0.00                        & 76.58±0.25                        & 66.07±0.00                        & 43.38±0.00                        & 0             \\ % \cline{2-19} 
                    & AdaLoRA                              & 90.23±0.10                        & 67.35±0.00                        & 80.96±0.29                        & 84.36±0.00                        & 79.55±0.00                        & 76.92±0.00                        & 46.28±0.00                        & 74.71±0.15                        & 90.20±0.00                         & 75.63±0.00                        & 61.76±0.29                        & 63.54±0.32                        & 79.45±0.23                        & 74.59±0.00                        & 65.18±0.00                        & 42.16±0.35                        & 0             \\ % \cline{2-19} 
                    & BayesTune-LoRA                            & 90.23±0.27                        & 67.35±0.00                        & 81.37±0.51                        & 84.20±0.22                         & 79.55±0.00                        & 76.92±0.00                        & 46.28±0.00                        & 74.61±0.00                        & 90.20±0.00                         & 75.35±0.20                        & 61.55±0.29                        & 63.98±0.32                        & 79.29±0.23                        & 74.59±0.00                        & 65.18±0.00                        & 42.40±0.35                         & 0             \\ % \cline{2-19} 
                    & FedL2P                               & \textbf{90.79±0.10}                & 68.94±0.32                        & 83.85±0.51                        & 85.62±0.81                        & 80.31±1.07                        & 77.43±0.73                        & 49.29±1.25                        & 77.95±0.97                        & 90.42±0.31                        & 76.89±0.00                        & 65.03±1.80                        & 68.46±1.45                        & 81.72±0.46                        & 76.94±1.02                        & 66.67±0.84                        & 45.10±1.93                         & 1             \\ % \cline{2-19} 
                    & \method{}                                 & 90.37±0.36                        & \textbf{73.70±0.32}                & \textbf{89.23±0.29}               & \textbf{93.36±0.39}               & \textbf{89.39±0.53}               & \textbf{79.66±0.64}               & \textbf{51.42±0.67}               & \textbf{81.09±0.29}               & \textbf{96.08±0.00}                & \textbf{78.99±0.34}               & \textbf{77.10±0.29}                & \textbf{78.52±0.55}               & \textbf{82.85±0.23}               & \textbf{86.49±0.00}                & \textbf{77.38±0.42}               & \textbf{67.65±0.00}                & \textbf{15}   \\ \hline
\multirow{5}{*}{16} & LoRA                                   & 90.93±0.00                        & 70.75±0.00                        & 85.09±0.00                        & 87.36±0.23                        & 82.95±0.00                        & \textbf{80.51±0.00}                         & 50.35±0.25                        & 79.83±0.15                        & 90.85±0.00                        & 76.89±0.00                        & 71.78±0.00                        & 71.81±0.95                        & \textbf{83.17±0.23}               & 80.18±0.67                        & 67.86±0.00                        & 51.96±0.69                        & 1             \\ % \cline{2-19} 
                    & AdaLoRA                              & 90.15±0.10                        & 67.58±0.32                        & 80.75±0.00                        & 84.36±0.00                        & 79.55±0.00                        & 76.92±0.00                        & 46.10±0.25                         & 74.71±0.15                        & 90.20±0.00                         & 75.63±0.34                        & 61.55±0.29                        & 63.98±0.32                        & 79.61±0.00                        & 74.59±0.00                        & 65.18±0.00                        & 42.40±0.35                         & 0             \\ % \cline{2-19} 
                    & BayesTune-LoRA                            & 90.58±0.10                        & 67.35±0.00                        & 81.16±0.29                        & 84.36±0.00                        & 79.55±0.00                        & 76.92±0.00                        & 46.10±0.25                         & 74.61±0.00                        & 90.20±0.00                         & 75.63±0.34                        & 61.35±0.00                        & 63.98±0.32                        & 79.45±0.23                        & 74.59±0.00                        & 65.18±0.00                        & 42.16±0.35                        & 0             \\ % \cline{2-19} 
                    & FedL2P                               & \textbf{91.14±0.30}                & 71.20±0.32                         & 87.17±0.77                        & 87.20±0.67                         & 83.71±0.54                        & 79.32±0.25                        & \textbf{51.42±0.67}               & 79.94±0.26                        & 91.72±0.31                        & 77.73±0.91                        & 70.55±2.18                        & 75.17±3.34                        & 82.36±0.23                        & 79.46±0.44                        & 71.43±1.26                        & 55.64±4.81                        & 2             \\ % \cline{2-19} 
                    & \method{}                                 & 88.33±0.70                        & \textbf{74.83±0.56}               & \textbf{88.41±0.29}               & \textbf{93.68±0.23}               & \textbf{89.39±0.53}               & \textbf{80.51±0.00}                         & 51.06±0.44                        & \textbf{80.15±0.29}               & \textbf{96.51±0.31}               & \textbf{80.67±0.00}                & \textbf{78.94±1.04}               & \textbf{77.63±0.32}               & 82.20±0.23                         & \textbf{85.95±0.44}               & \textbf{77.38±0.84}               & \textbf{68.87±0.69}               & \textbf{12}   \\ \bottomrule
\end{tabular}
}
\vspace{-1.2em}
\end{table*}

\subsubsection{Baselines}\label{sec:baselines}

Given a \basemodel{}, we compare \method{} with existing fine-tuning and {\em learning to personalize} approaches. 

\noindent\textbf{LoRA PEFT.}~We deploy LoRA~\cite{hu2021lora} on all linear layers of the model with a fixed rank $r$. 

\noindent\textbf{Non-FL Rank Selection.}~We compare with AdaLoRA~\cite{adalora} and our proposed LoRA-variant of BayesTune~\cite{kim2023bayestune}, BayesTune-LoRA (Section~\ref{sec:personalized_peft}), which optimizes $\bm{\lambda}$ separately for each client. 

\noindent\textbf{FL to Personalize.}~We compare with FedL2P~\cite{royson2023fedl2p} which trains a MLP federatedly to output per-client learning rates for each LoRA module.

For each baseline, we either follow best practices recommended by the corresponding authors or employ a simple grid search and pick the best performing hyperparameters. Full details in Appendix.~\ref{appendix:experiments}.

\subsection{Results on Text Classification}\label{sec:text_class}

We evaluate our approach in a typical FL setup, where the pretrained model is first trained using Standard FL with full fine-tuning and the resulting \basemodel{} is then personalized to each client. Tables~\ref{tab:masakha_seen} \& \ref{tab:masakha_unseen} show the mean and standard deviation (SD) of the accuracy for each language in our MasakhaNEWS setup for \seen{} and \unseen{} pool respectively (similarly for XNLI in Appendix Tables~\ref{tab:xnli_seen} \& \ref{tab:xnli_unseen}). In addition, we also show the number of languages, labelled ``Wins", an approach is best performing for each budget $r$. 

The results in all four tables show that federated {\em learning to personalize} methods (FedL2P and \method{}) outperform the other baselines in most cases. Non-FL rank selection approaches (AdaLoRA and BayesTune-LoRA), on the other hand, tend to overfit and/or struggle to learn an optimal rank structure given the limited number of samples in each client. Comparing FedL2P and \method{}, \method{} largely surpass FedL2P with a few exceptions, indicating that learning to personalize LoRA rank structure is the better hyperparameter choice than personalizing learning rates; this finding is also aligned with recent LLM-based optimizer findings~\cite{zhao2025deconstructing}, which shows that Adam's performance is robust with respect to its learning rate.

\subsubsection{\method{}'s Complementability with Personalized FL Works.}\label{sec:eval_complement}

% \begin{table*}[]
% \centering
% \caption{Mean Accuracy ± standard deviation on each language and overall win rate on XNLI dataset, with FedPDA, seen pool}
% \label{tab:xnli_seen_fedpda}
% \begin{scriptsize}\resizebox{0.98\textwidth}{!}{
% \begin{tabular}{c|l|l|l|l|l|l|l|l|l|l|l|l|l|l|l|l|c}
% \toprule

\begin{table*}[]
\centering
\caption{Mean±SD Accuracy of each language across 3 different seeds for clients in the \seen{} pool of our XNLI setup. The pretrained model is trained using FedDPA-T and the resulting \basemodel{} is personalized to each client given a baseline approach.}
\label{tab:xnli_seen_feddpa}
\begin{scriptsize}\resizebox{0.98\textwidth}{!}{

\begin{tabular}{c|l|l|l|l|l|l|l|l|l|l|l|l|l|l|l|l|c}
\toprule
% \textbf{Lora Rank}  
\textbf{$\mathbf{r}$} & \multicolumn{1}{c|}{\textbf{Approach}} & \multicolumn{1}{c|}{\textbf{bg}} & \multicolumn{1}{c|}{\textbf{hi}} & \multicolumn{1}{c|}{\textbf{es}} & \multicolumn{1}{c|}{\textbf{el}} & \multicolumn{1}{c|}{\textbf{vi}} & \multicolumn{1}{c|}{\textbf{tr}} & \multicolumn{1}{c|}{\textbf{de}} & \multicolumn{1}{c|}{\textbf{ur}} & \multicolumn{1}{c|}{\textbf{en}} & \multicolumn{1}{c|}{\textbf{zh}} & \multicolumn{1}{c|}{\textbf{th}} & \multicolumn{1}{c|}{\textbf{sw}} & \multicolumn{1}{c|}{\textbf{ar}} & \multicolumn{1}{c|}{\textbf{fr}} & \multicolumn{1}{c|}{\textbf{ru}} & \textbf{Wins} \\ \midrule
% \multirow{5}{*}{1}  & LoRA                                   & 44.60±0.00                        & 41.73±0.09                       & 47.73±0.19                       & 50.20±0.00                        & 52.53±0.19                       & 47.00±0.16                        & 48.07±0.09                       & 40.60±0.00                        & 45.40±0.00                        & 43.33±0.09                       & 41.07±0.09                       & 50.80±0.00                        & 45.27±0.09                       & 46.33±0.19                       & 49.20±0.00                        & 0             \\ %\cline{2-18} 
%                     & AdaLoRA                              & 44.07±0.09                       & 41.20±0.00                        & 47.47±0.09                       & 50.20±0.00                        & 52.40±0.00                        & 46.53±0.09                       & 47.93±0.09                       & 38.87±0.09                       & 44.60±0.00                        & 42.27±0.09                       & 40.67±0.09                       & 50.40±0.00                        & 45.00±0.00                        & 46.00±0.00                        & 48.80±0.00                        & 0             \\ %\cline{2-18} 
%                     & BayesTune-LoRA                            & 43.67±0.09                       & 40.40±0.00                        & 47.20±0.00                        & 50.00±0.00                        & 52.33±0.09                       & 46.27±0.09                       & 47.40±0.16                        & 38.80±0.00                        & 43.93±0.09                       & 41.33±0.09                       & 40.53±0.09                       & 49.80±0.00                        & 44.40±0.00                        & 46.00±0.00                        & 47.93±0.09                       & 0             \\ %\cline{2-18} 
%                     & FedL2P                               & 44.33±0.19                       & 41.33±0.19                       & 47.67±0.09                       & 50.20±0.00                        & 52.47±0.09                       & 46.93±0.19                       & 47.93±0.19                       & 40.20±0.28                        & 45.20±0.00                        & 42.93±0.19                       & 40.93±0.09                       & 50.67±0.09                       & 45.27±0.09                       & 46.33±0.19                       & 48.80±0.00                        & 0             \\ %\cline{2-18} 
%                     & \method{}                                 & \textbf{58.00±0.16}               & \textbf{54.93±0.09}              & \textbf{55.33±0.09}              & \textbf{55.33±0.19}              & \textbf{54.80±0.33}               & \textbf{54.20±0.16}               & \textbf{55.80±0.16}               & \textbf{60.33±0.34}              & \textbf{55.53±0.25}              & \textbf{55.67±0.09}              & \textbf{58.07±0.62}              & \textbf{54.33±0.19}              & \textbf{54.93±0.09}              & \textbf{57.07±0.38}              & \textbf{54.53±0.34}              & \textbf{15}   \\ \hline
\multirow{5}{*}{2}  & LoRA                                   & 45.80±0.28                        & 42.80±0.16                        & 48.73±0.25                       & 50.87±0.09                       & 53.00±0.00                        & 48.00±0.33                        & 49.87±0.09                       & 41.93±0.09                       & 46.53±0.34                       & 44.40±0.16                        & 42.53±0.25                       & 51.80±0.16                        & 46.93±0.25                       & 48.07±0.19                       & 50.53±0.25                       & 0             \\ %\cline{2-18} 
                    & AdaLoRA                              & 44.07±0.09                       & 41.20±0.00                        & 47.47±0.09                       & 50.00±0.00                        & 52.40±0.00                        & 46.53±0.09                       & 48.00±0.00                        & 38.80±0.00                        & 44.67±0.09                       & 42.20±0.00                        & 40.67±0.09                       & 50.40±0.00                        & 45.00±0.00                        & 46.00±0.00                        & 48.80±0.00                        & 0             \\ %\cline{2-18} 
                    & BayesTune-LoRA                            & 43.80±0.00                        & 40.60±0.00                        & 47.20±0.00                        & 50.00±0.00                        & 52.40±0.00                        & 46.20±0.00                        & 47.40±0.16                        & 38.80±0.00                        & 44.27±0.09                       & 41.60±0.00                        & 40.53±0.09                       & 49.80±0.00                        & 44.73±0.09                       & 46.00±0.00                        & 48.07±0.19                       & 0             \\ %\cline{2-18} 
                    & FedL2P                               & 47.47±2.78                       & 44.53±3.18                       & 50.27±2.94                       & 51.47±1.39                       & 53.60±0.71                        & 50.07±2.81                       & 50.93±2.62                       & 44.80±4.53                        & 49.07±4.35                       & 46.53±4.01                       & 44.40±3.68                        & 52.27±1.27                       & 48.40±2.97                        & 49.47±2.96                       & 51.60±2.27                        & 0             \\ %\cline{2-18} 
                    & \method{}                                 & \textbf{64.40±0.16}               & \textbf{57.80±1.34}               & \textbf{58.53±1.32}              & \textbf{59.73±2.58}              & \textbf{60.80±0.71}               & \textbf{58.87±2.87}              & \textbf{55.00±0.16}               & \textbf{63.67±0.19}              & \textbf{55.93±0.19}              & \textbf{56.27±0.34}              & \textbf{58.33±0.25}              & \textbf{59.47±0.47}              & \textbf{55.20±0.16}               & \textbf{57.53±0.19}              & \textbf{55.20±0.43}               & \textbf{15}   \\ \hline
\multirow{5}{*}{4}  & LoRA                                   & 48.73±0.38                       & 46.07±0.66                       & 53.27±0.09                       & 52.53±0.09                       & 53.33±0.09                       & 50.27±0.25                       & 52.80±0.28                        & 46.13±0.25                       & 51.67±0.25                       & 48.53±0.19                       & 45.47±0.19                       & 53.33±0.09                       & 50.47±0.41                       & 50.27±0.09                       & 52.87±0.19                       & 0             \\ %\cline{2-18} 
                    & AdaLoRA                              & 44.00±0.00                        & 41.13±0.09                       & 47.40±0.00                        & 50.00±0.00                        & 52.40±0.00                        & 46.40±0.00                        & 47.67±0.09                       & 38.80±0.00                        & 44.47±0.09                       & 41.93±0.09                       & 40.60±0.00                        & 50.27±0.09                       & 44.87±0.09                       & 46.00±0.00                        & 48.80±0.00                        & 0             \\ %\cline{2-18} 
                    & BayesTune-LoRA                            & 44.00±0.00                        & 41.00±0.00                        & 47.20±0.00                        & 50.00±0.00                        & 52.40±0.00                        & 46.33±0.09                       & 47.67±0.09                       & 38.80±0.00                        & 44.47±0.09                       & 41.80±0.00                        & 40.60±0.00                        & 50.00±0.00                        & 44.80±0.00                        & 46.00±0.00                        & 48.67±0.09                       & 0             \\ %\cline{2-18} 
                    & FedL2P                               & 48.67±2.36                       & 45.20±2.26                        & 52.13±1.20                        & 51.87±0.52                       & 53.87±0.34                       & 50.27±0.84                       & 52.00±0.99                        & 47.00±3.69                        & 50.73±2.59                       & 47.27±1.95                       & 45.07±1.65                       & 52.67±0.52                       & 49.67±1.59                       & 51.20±2.26                        & 52.47±1.37                       & 0             \\ %\cline{2-18} 
                    & \method{}                                 & \textbf{64.47±1.18}              & \textbf{64.00±0.98}               & \textbf{64.40±0.91}               & \textbf{63.07±0.41}              & \textbf{64.13±0.34}              & \textbf{64.87±0.68}              & \textbf{63.33±1.67}              & \textbf{64.47±0.62}              & \textbf{56.60±0.28}               & \textbf{63.93±1.16}              & \textbf{62.60±1.73}               & \textbf{65.00±0.33}               & \textbf{64.13±0.81}              & \textbf{61.73±1.23}              & \textbf{61.80±0.65}               & \textbf{15}   \\ \hline
\multirow{5}{*}{8}  & LoRA                                   & 55.80±0.16                        & 51.73±0.25                       & 55.73±0.09                       & 55.07±0.25                       & 54.40±0.00                        & 52.80±0.16                        & 54.60±0.00                        & 57.47±0.09                       & 55.53±0.09                       & 54.00±0.33                        & 51.93±0.09                       & 54.13±0.09                       & 53.07±0.09                       & 56.27±0.09                       & 54.60±0.16                        & 0             \\ %\cline{2-18} 
                    & AdaLoRA                              & 43.80±0.00                        & 40.87±0.09                       & 47.27±0.09                       & 50.00±0.00                        & 52.40±0.00                        & 46.27±0.09                       & 47.73±0.09                       & 38.80±0.00                        & 44.40±0.00                        & 41.87±0.09                       & 40.60±0.00                        & 50.00±0.00                        & 44.80±0.00                        & 46.00±0.00                        & 48.40±0.16                        & 0             \\ %\cline{2-18} 
                    & BayesTune-LoRA                            & 44.13±0.09                       & 41.20±0.00                        & 47.47±0.09                       & 50.20±0.00                        & 52.40±0.00                        & 46.47±0.09                       & 47.93±0.09                       & 39.13±0.09                       & 44.93±0.09                       & 42.53±0.09                       & 40.87±0.09                       & 50.47±0.09                       & 45.13±0.09                       & 46.13±0.09                       & 48.80±0.00                        & 0             \\ %\cline{2-18} 
                    & FedL2P                               & 52.20±0.59                        & 48.73±0.25                       & 54.20±0.16                        & 53.13±0.19                       & 54.33±0.09                       & 51.53±0.19                       & 53.60±0.16                        & 51.00±0.86                        & 54.27±0.25                       & 51.73±0.25                       & 47.07±0.52                       & 53.60±0.16                        & 51.87±0.19                       & 52.93±0.34                       & 54.60±0.16                        & 0             \\ %\cline{2-18} 
                    & \method{}                                 & \textbf{67.27±0.25}              & \textbf{66.60±0.33}               & \textbf{68.87±0.41}              & \textbf{65.00±0.43}               & \textbf{66.73±0.09}              & \textbf{67.80±0.28}               & \textbf{67.13±0.19}              & \textbf{67.60±0.16}               & \textbf{64.00±0.99}               & \textbf{67.67±0.38}              & \textbf{66.53±0.09}              & \textbf{67.33±0.25}              & \textbf{67.60±0.33}               & \textbf{66.53±0.25}              & \textbf{69.33±0.19}              & \textbf{15}   \\ \hline
\multirow{5}{*}{16} & LoRA                                   & 64.73±0.09                       & 64.27±0.19                       & 65.60±0.28                        & \textbf{65.33±0.19}              & 63.80±0.00                        & 64.27±0.09                       & 64.20±0.71                        & 64.80±0.16                        & 57.93±2.46                       & 65.93±0.25                       & 58.33±0.52                       & 65.13±0.19                       & 60.53±0.62                       & 64.53±0.47                       & 61.73±0.74                       & 1             \\ %\cline{2-18} 
                    & AdaLoRA                              & 43.87±0.09                       & 40.73±0.09                       & 47.20±0.00                        & 50.00±0.00                        & 52.40±0.00                        & 46.33±0.09                       & 47.67±0.09                       & 38.80±0.00                        & 44.27±0.09                       & 41.80±0.00                        & 40.60±0.00                        & 50.00±0.00                        & 44.80±0.00                        & 46.00±0.00                        & 48.27±0.09                       & 0             \\ %\cline{2-18} 
                    & BayesTune-LoRA                            & 44.60±0.00                        & 41.67±0.19                       & 47.80±0.00                        & 50.20±0.00                        & 52.87±0.09                       & 47.13±0.09                       & 48.33±0.19                       & 40.60±0.00                        & 45.40±0.00                        & 43.60±0.00                        & 41.00±0.00                        & 50.80±0.00                        & 45.33±0.09                       & 46.60±0.00                        & 49.40±0.16                        & 0             \\ %\cline{2-18} 
                    & FedL2P                               & 54.47±1.86                       & 52.73±0.50                        & 55.80±0.43                        & 54.73±0.47                       & 55.67±1.09                       & 54.80±0.00                        & 54.73±0.25                       & 57.67±2.78                       & 55.67±0.09                       & 54.93±0.41                       & 52.60±2.79                        & 53.93±0.41                       & 53.47±0.25                       & 55.60±2.12                        & 55.20±0.00                        & 0             \\ %\cline{2-18} 
                    & \method{}                                 & \textbf{67.87±0.74}              & \textbf{67.00±0.43}               & \textbf{69.53±0.57}              & 64.93±0.41                       & \textbf{67.00±0.33}               & \textbf{68.00±0.00}                & \textbf{67.20±0.43}               & \textbf{68.00±0.59}               & \textbf{65.20±1.07}               & \textbf{67.60±0.16}               & \textbf{66.60±0.57}               & \textbf{67.33±0.09}              & \textbf{67.80±0.28}               & \textbf{66.13±0.41}              & \textbf{69.87±0.25}              & \textbf{14}   \\ \bottomrule
\end{tabular}
}
\end{scriptsize}
\vspace{-1.5em}
\end{table*}
\begin{table*}[]
\caption{Mean±SD Accuracy of each language across 3 different seeds for clients in the \seen{} pool of our XNLI setup. The pretrained model is trained using DEPT(SPEC) and the resulting \basemodel{} is personalized to each client given a baseline approach.}
\label{tab:xnli_seen_dept}
\begin{scriptsize}\resizebox{0.98\textwidth}{!}{
\begin{tabular}{c|l|l|l|l|l|l|l|l|l|l|l|l|l|l|l|l|c}
\toprule
\textbf{$\mathbf{r}$} & \multicolumn{1}{c|}{\textbf{Approach}} & \multicolumn{1}{c|}{\textbf{bg}} & \multicolumn{1}{c|}{\textbf{hi}} & \multicolumn{1}{c|}{\textbf{es}} & \multicolumn{1}{c|}{\textbf{el}} & \multicolumn{1}{c|}{\textbf{vi}} & \multicolumn{1}{c|}{\textbf{tr}} & \multicolumn{1}{c|}{\textbf{de}} & \multicolumn{1}{c|}{\textbf{ur}} & \multicolumn{1}{c|}{\textbf{en}} & \multicolumn{1}{c|}{\textbf{zh}} & \multicolumn{1}{c|}{\textbf{th}} & \multicolumn{1}{c|}{\textbf{sw}} & \multicolumn{1}{c|}{\textbf{ar}} & \multicolumn{1}{c|}{\textbf{fr}} & \multicolumn{1}{c|}{\textbf{ru}} & \textbf{Wins} \\ \midrule
% \multirow{5}{*}{1}  & LoRA                                   & 53.60±0.16                        & 54.87±0.09                       & 55.80±0.16                        & 53.60±0.16                        & 54.87±0.09                       & 53.20±0.00                        & 54.20±0.00                        & 53.33±0.34                       & 60.47±0.09                       & 53.53±0.09                       & 49.40±0.16                        & 51.60±0.16                        & 51.87±0.09                       & 53.13±0.09                       & 54.80±0.00                        & 0             \\ %\cline{2-18} 
%                     & AdaLoRA                              & 53.33±0.09                       & 54.73±0.09                       & 55.40±0.00                        & 53.07±0.09                       & 54.33±0.19                       & 52.87±0.09                       & 54.07±0.09                       & 53.20±0.00                        & 60.27±0.09                       & 53.07±0.09                       & 49.47±0.09                       & 50.87±0.09                       & 51.40±0.00                        & 52.87±0.19                       & 54.13±0.19                       & 0             \\ %\cline{2-18} 
%                     & BayesTune-LoRA                            & 53.27±0.09                       & 54.47±0.09                       & 55.40±0.16                        & 53.13±0.09                       & 54.00±0.00                        & 52.67±0.09                       & 53.93±0.09                       & 53.07±0.19                       & 60.00±0.16                        & 53.07±0.09                       & 49.40±0.00                        & 50.40±0.00                        & 51.40±0.16                        & 52.60±0.00                        & 54.20±0.16                        & 0             \\ %\cline{2-18} 
%                     & FedL2P                               & 60.80±0.60                       & 61.60±4.20                       & 65.70±0.90                        & 65.10±1.70                        & 65.80±0.00                        & 61.90±3.90                        & 64.40±0.40                        & 62.60±1.00                        & 70.20±0.00                        & 61.90±1.30                        & 59.50±1.70                        & 60.80±3.20                       & 63.00±1.60                       & 61.00±0.40                        & 64.00±0.40                        & 0             \\ %\cline{2-18} 
%                     & \method{}                                 & \textbf{68.67±0.9}               & \textbf{67.33±1.65}              & \textbf{71.73±1.31}              & \textbf{71.73±0.25}              & \textbf{70.73±0.57}              & \textbf{69.33±2.07}              & \textbf{70.80±0.16}               & \textbf{70.27±0.9}               & \textbf{73.67±0.41}              & \textbf{71.20±0.16}               & \textbf{63.40±2.2}                & \textbf{69.20±1.61}               & \textbf{70.93±0.77}              & \textbf{69.00±1.41}               & \textbf{70.67±0.82}              & \textbf{15}   \\ \hline
\multirow{5}{*}{2}  & LoRA                                   & 54.10±0.10                        & 55.80±0.00                        & 57.10±0.10                        & 55.10±0.30                        & 56.30±0.10                        & 54.30±0.10                        & 55.10±0.10                        & 53.60±0.20                       & 61.50±0.10                        & 54.80±0.00                        & 50.70±0.10                        & 52.50±0.10                        & 53.40±0.20                       & 53.10±0.10                        & 55.30±0.10                        & 0             \\ %\cline{2-18} 
                    & AdaLoRA                              & 53.33±0.09                       & 54.53±0.09                       & 55.33±0.09                       & 52.80±0.28                        & 54.07±0.09                       & 52.87±0.09                       & 53.93±0.09                       & 52.87±0.19                       & 60.40±0.16                        & 53.07±0.09                       & 49.40±0.16                        & 50.67±0.09                       & 51.40±0.16                        & 52.80±0.16                        & 54.20±0.16                        & 0             \\ %\cline{2-18} 
                    & BayesTune-LoRA                            & 53.40±0.00                        & 54.40±0.00                        & 55.40±0.00                        & 53.10±0.10                        & 54.00±0.20                       & 52.90±0.10                        & 54.10±0.10                        & 53.20±0.00                        & 60.00±0.00                        & 53.20±0.00                        & 49.30±0.10                        & 50.50±0.10                        & 51.50±0.10                        & 52.60±0.20                       & 54.00±0.00                        & 0             \\ %\cline{2-18} 
                    & FedL2P                               & 64.70±1.10                        & 64.20±2.80                        & 67.90±1.30                        & 68.50±1.10                        & 68.60±0.20                       & 65.20±4.00                        & 67.30±0.70                        & 67.00±0.40                        & 71.50±0.30                        & 64.90±0.50                        & 61.70±1.70                        & 64.60±2.80                        & 65.80±2.00                        & 63.40±0.00                        & 67.40±0.40                        & 0             \\ %\cline{2-18} 
                    & \method{}                                 & \textbf{70.67±0.77}              & \textbf{70.07±1.04}              & \textbf{73.87±1.54}              & \textbf{73.47±0.19}              & \textbf{73.40±0.59}               & \textbf{71.93±2.08}              & \textbf{73.27±0.57}              & \textbf{71.87±0.62}              & \textbf{74.53±0.52}              & \textbf{73.80±0.43}               & \textbf{65.60±2.97}               & \textbf{69.93±1.16}              & \textbf{74.40±0.49}               & \textbf{71.27±1.52}              & \textbf{72.73±1.88}              & \textbf{15}   \\ \hline
\multirow{5}{*}{4}  & LoRA                                   & 56.67±0.34                       & 59.07±0.25                       & 59.13±0.41                       & 57.33±0.25                       & 59.00±0.28                        & 56.93±0.25                       & 57.40±0.59                        & 55.93±0.09                       & 63.47±0.25                       & 57.27±0.09                       & 52.47±0.34                       & 54.53±0.25                       & 56.40±0.33                        & 55.93±0.09                       & 58.40±0.33                        & 0             \\ %\cline{2-18} 
                    & AdaLoRA                              & 53.33±0.09                       & 54.40±0.16                        & 55.27±0.09                       & 53.20±0.00                        & 54.13±0.19                       & 52.93±0.09                       & 54.00±0.00                        & 52.87±0.09                       & 60.33±0.25                       & 53.13±0.09                       & 49.20±0.16                        & 50.60±0.16                        & 51.27±0.09                       & 52.60±0.16                        & 54.00±0.00                        & 0             \\ %\cline{2-18} 
                    & BayesTune-LoRA                            & 53.27±0.09                       & 54.53±0.19                       & 55.40±0.16                        & 53.00±0.16                        & 54.53±0.25                       & 52.87±0.09                       & 54.00±0.00                        & 53.07±0.09                       & 60.60±0.16                        & 53.07±0.09                       & 49.33±0.19                       & 51.07±0.09                       & 51.47±0.09                       & 52.60±0.00                        & 54.53±0.19                       & 0             \\ %\cline{2-18} 
                    & FedL2P                               & 66.50±1.10                        & 65.40±1.80                        & 69.90±1.90                        & 70.50±0.90                        & 70.10±0.30                        & 66.90±4.10                        & 69.70±0.90                        & 68.20±0.60                       & 72.80±0.20                       & 67.40±1.20                       & 62.60±2.00                        & 65.70±2.90                        & 67.70±1.30                        & 66.50±1.10                        & 69.00±0.40                        & 0             \\ %\cline{2-18} 
                    & \method{}                                 & \textbf{71.33±0.34}              & \textbf{70.07±1.09}              & \textbf{73.27±2.29}              & \textbf{73.27±0.68}              & \textbf{72.60±0.28}               & \textbf{71.87±2.32}              & \textbf{74.60±0.16}               & \textbf{72.93±0.38}              & \textbf{75.00±0.28}               & \textbf{74.33±1.16}              & \textbf{66.13±3.21}              & \textbf{68.13±0.34}              & \textbf{75.13±1.65}              & \textbf{72.20±0.33}               & \textbf{73.47±1.60}               & \textbf{15}   \\ \hline
\multirow{5}{*}{8}  & LoRA                                   & 60.60±0.28                        & 62.47±0.09                       & 63.73±0.09                       & 62.60±0.33                        & 64.67±0.09                       & 61.93±0.41                       & 62.53±0.09                       & 60.33±0.19                       & 67.33±0.25                       & 61.40±0.16                        & 56.53±0.09                       & 59.27±0.09                       & 61.33±0.25                       & 58.87±0.09                       & 63.13±0.19                       & 0             \\ %\cline{2-18} 
                    & AdaLoRA                              & 53.33±0.09                       & 54.53±0.19                       & 55.27±0.09                       & 52.80±0.28                        & 54.27±0.25                       & 53.00±0.00                        & 53.93±0.09                       & 52.93±0.09                       & 60.13±0.34                       & 53.00±0.00                        & 49.20±0.16                        & 50.27±0.19                       & 51.20±0.16                        & 52.73±0.38                       & 53.93±0.09                       & 0             \\ %\cline{2-18} 
                    & BayesTune-LoRA                            & 53.33±0.09                       & 54.40±0.00                        & 55.53±0.19                       & 53.20±0.00                        & 54.67±0.09                       & 53.00±0.00                        & 54.07±0.09                       & 53.13±0.25                       & 60.60±0.00                        & 53.47±0.09                       & 49.47±0.09                       & 51.33±0.34                       & 51.73±0.19                       & 52.87±0.09                       & 54.40±0.16                        & 0             \\ %\cline{2-18} 
                    & FedL2P                               & 66.40±0.60                       & 65.50±1.10                        & 70.50±3.50                        & 70.10±0.50                        & 70.50±1.70                        & 68.20±2.20                       & 69.60±0.40                        & 67.50±1.30                        & 72.60±0.60                       & 67.10±1.30                        & 61.80±1.00                        & 65.60±1.80                        & 68.20±1.00                        & 67.20±1.40                        & 69.10±1.70                        & 0             \\ %\cline{2-18} 
                    & \method{}                                 & \textbf{70.53±0.25}              & \textbf{69.27±0.77}              & \textbf{73.33±1.79}              & \textbf{71.27±1.39}              & \textbf{71.33±0.52}              & \textbf{70.53±2.10}               & \textbf{75.33±0.41}              & \textbf{73.07±0.77}              & \textbf{74.40±0.75}               & \textbf{73.93±1.11}              & \textbf{66.73±3.80}               & \textbf{68.07±0.94}              & \textbf{74.67±1.18}              & \textbf{71.87±0.19}              & \textbf{72.93±0.96}              & \textbf{15}   \\ \hline
\multirow{5}{*}{16} & LoRA                                   & 67.13±0.34                       & 67.80±0.00                        & 69.47±0.09                       & 71.27±0.19                       & 69.20±0.00                        & 68.07±0.38                       & 69.00±0.33                        & 68.73±0.25                       & 71.47±0.25                       & 68.00±0.16                        & 62.80±0.00                        & 67.33±0.25                       & 66.80±0.16                        & 65.07±0.19                       & 67.33±0.09                       & \textbf{0}    \\ %\cline{2-18} 
                    & AdaLoRA                              & 53.40±0.00                        & 54.40±0.00                        & 55.40±0.00                        & 52.60±0.16                        & 54.00±0.00                        & 52.93±0.09                       & 53.93±0.09                       & 52.93±0.09                       & 60.27±0.41                       & 53.00±0.00                        & 49.27±0.09                       & 50.27±0.09                       & 51.20±0.33                        & 52.80±0.28                        & 53.80±0.16                        & 0             \\ %\cline{2-18} 
                    & BayesTune-LoRA                            & 53.47±0.09                       & 54.67±0.09                       & 55.80±0.00                        & 53.40±0.00                        & 54.67±0.09                       & 53.07±0.09                       & 54.47±0.19                       & 53.27±0.09                       & 60.67±0.09                       & 53.40±0.16                        & 49.40±0.00                        & 51.67±0.09                       & 52.07±0.25                       & 53.07±0.09                       & 54.80±0.00                        & 0             \\ %\cline{2-18} 
                    & FedL2P                               & 68.00±1.02                        & 65.87±1.33                       & 70.20±3.13                        & \textbf{71.73±0.81}              & 71.20±1.40                        & 69.60±1.66                        & 71.20±1.02                        & 69.00±0.99                        & 73.80±0.82                        & 69.13±1.54                       & 64.00±1.14                        & 67.80±1.82                        & 69.27±1.23                       & 68.53±1.32                       & 69.93±1.95                       & 1             \\ %\cline{2-18} 
                    & \method{}                                 & \textbf{69.80±0.16}               & \textbf{69.27±1.04}              & \textbf{73.47±1.51}              & 70.40±1.84                        & \textbf{71.27±0.57}              & \textbf{70.67±2.88}              & \textbf{74.80±0.16}               & \textbf{73.20±1.14}               & \textbf{74.53±1.25}              & \textbf{73.53±0.75}              & \textbf{65.93±2.64}              & \textbf{68.60±1.70}                & \textbf{73.67±1.84}              & \textbf{70.67±0.25}              & \textbf{72.80±0.71}               & \textbf{14}   \\ \bottomrule
\end{tabular}
}
\end{scriptsize}
\vspace{-1.2em}
\end{table*}
\begin{table*}[]
\small
\centering
\setlength{\tabcolsep}{3.5pt}
\renewcommand{\arraystretch}{0.8}
\begin{tabular}{@{}cl|ccccc@{}}
\toprule
\textbf{\# Topics} & \textbf{Model} & \multicolumn{1}{l}{\textbf{\# Input Tokens}} & \multicolumn{1}{l}{\textbf{\# Output Tokens}} & \multicolumn{1}{l}{\textbf{\# LLM Calls}} & \multicolumn{1}{l}{\textbf{Cost (GPT-4)}} & \multicolumn{1}{l}{\textbf{Time (seconds)}} \\ \midrule
\multirow{3}{*}{2} & \modelTopic & 21383.08 & 3412.02 & 25.45 & 0.32 & 117.60 \\
 & Hierarchical & 31130.02 & 2536.66 & 13.15 & 0.39 & 83.13 \\
 & Incremental-\textit{Topic} & 59010.66 & 6115.04 & 15.15 & 0.77 & 214.39 \\ \midrule
\multirow{3}{*}{3} & \modelTopic & 30208.20 & 5040.38 & 37.38 & 0.45 & 149.54 \\
 & Hierarchical & 31144.83 & 2649.78 & 13.15 & 0.39 & 68.60 \\
 & Incremental-\textit{Topic} & 61344.07 & 8442.54 & 16.15 & 0.87 & 197.33 \\ \midrule
\multirow{3}{*}{4} & \modelTopic & 38286.40 & 6440.23 & 47.91 & 0.58 & 163.91 \\
 & Hierarchical & 31144.31 & 2740.31 & 13.15 & 0.39 & 88.75 \\
 & Incremental-\textit{Topic} & 62877.46 & 9966.45 & 17.15 & 0.93 & 312.55 \\ \midrule
\multirow{3}{*}{5} & \modelTopic & 47008.59 & 7918.92 & 58.94 & 0.71 & 186.32 \\
 & Hierarchical & 31160.88 & 2850.24 & 13.15 & 0.40 & 61.70 \\
 & Incremental-\textit{Topic} & 64893.95 & 11965.84 & 18.15 & 1.01 & 262.07 \\ \bottomrule
\end{tabular}
\caption{\label{appendix:table:cost_cqa} Number of LLM input/output tokens, LLM calls, GPT-4 Cost (USD), and Time (seconds) needed to run inference on a single DFQS example on ConflictingQA with the top-3 models. We report 5 runs and 20 examples.}
\end{table*}

\begin{table*}[]
\small
\centering
\setlength{\tabcolsep}{3.5pt}
\renewcommand{\arraystretch}{0.8}
\begin{tabular}{@{}cl|ccccc@{}}
\toprule
\multicolumn{1}{l}{\textbf{Dataset}} & \textbf{Model} & \multicolumn{1}{l}{\textbf{\# Input Tokens}} & \multicolumn{1}{l}{\textbf{\# Output Tokens}} & \multicolumn{1}{l}{\textbf{\# LLM Calls}} & \multicolumn{1}{l}{\textbf{Cost (GPT-4)}} & \multicolumn{1}{l}{\textbf{Time (seconds)}} \\ \midrule
\multirow{3}{*}{2} & \modelTopic & 17183.75 & 2722.40 & 20.30 & 0.25 & 94.81 \\
 & Hierarchical & 19181.59 & 2040.39 & 10.25 & 0.25 & 63.68 \\
 & Incremental-\textit{Topic} & 41656.87 & 5062.44 & 12.25 & 0.57 & 182.19 \\ 
 \midrule
\multirow{3}{*}{3} & \modelTopic & 24801.22 & 4136.12 & 30.40 & 0.37 & 126.83 \\
 & Hierarchical & 19182.58 & 2141.91 & 10.25 & 0.26 & 53.32 \\
 & Incremental-\textit{Topic} & 43119.51 & 6532.92 & 13.25 & 0.63 & 152.44 \\ \midrule
\multirow{3}{*}{4} & \modelTopic & 30677.67 & 5037.31 & 38.00 & 0.46 & 120.64 \\
 & Hierarchical & 19203.30 & 2253.17 & 10.25 & 0.26 & 73.35 \\
 & Incremental-\textit{Topic} & 43922.02 & 7327.88 & 14.25 & 0.66 & 241.54 \\ \midrule
\multirow{3}{*}{5} & \modelTopic & 36988.41 & 6049.93 & 46.09 & 0.55 & 139.71 \\
 & Hierarchical & 19211.74 & 2356.01 & 10.25 & 0.26 & 49.41 \\
 & Incremental-\textit{Topic} & 45113.12 & 8504.59 & 15.25 & 0.71 & 186.40 \\ \bottomrule
\end{tabular}
\caption{\label{appendix:table:cost_debate} Number of LLM input/output tokens, LLM calls, GPT-4 Cost (USD), and Time (seconds) needed to run inference on a single DFQS example on DebateQFS with the top-3 models. We report 5 runs and 20 examples.}
\end{table*}

\begin{table*}[]
\small
\centering
\setlength{\tabcolsep}{3.5pt}
\renewcommand{\arraystretch}{0.8}
\begin{tabular}{@{}cl|ccccc@{}}
\toprule
\multicolumn{1}{l}{\textbf{\# Topics}} & \textbf{Model} & \multicolumn{1}{l}{\textbf{\# Input Tokens}} & \multicolumn{1}{l}{\textbf{\# Output Tokens}} & \multicolumn{1}{l}{\textbf{\# LLM Calls}} & \multicolumn{1}{l}{\textbf{Cost (GPT-4)}} & \multicolumn{1}{l}{\textbf{Time (seconds)}} \\ 
\midrule
\multirow{3}{*}{ConflictingQA} & \modelTopic & 47008.59 & 7918.92 & 58.94 & 0.71 & 186.32 \\
 & \modelTopic Pick All & 53733.70 & 9596.75 & 71.75 & 0.83 & 303.13 \\
 & Hierarchical-\emph{Topic} & 168160.85 & 7485.50 & 66.75 & 1.91 & 210.80 \\ \midrule
\multirow{3}{*}{DebateQFS} & \modelTopic & 36988.41 & 6049.93 & 46.09 & 0.55 & 139.71 \\
& \modelTopic Pick All & 43098.85 & 7612.45 & 57.25 & 0.66 & 242.35 \\
& Hierarchical-\emph{Topic} & 105237.25 & 5278.35 & 52.25 & 1.21 & 139.96 \\ \bottomrule
\end{tabular}
\caption{\label{appendix:table:cost_weird} Number of LLM input/output tokens, LLM calls, GPT-4 Cost (USD), and Time (seconds) needed to run inference on a single DFQS example on ConflictingQA and DebateQFS with \modelTopic, the version of \modelTopic with no Moderator, and the version of Hierarchical merging that runs on each topic paragraph ($m=5$). We report 5 runs and 20 examples.}
\end{table*}



Apart from Standard FL, we show that \method{} can be plugged into existing personalized FL works that trains both a subset of the pretrained model and personalized layers for each client. Tables~\ref{tab:xnli_seen_feddpa} and \ref{tab:xnli_seen_dept} show that \method{} outperforms baselines in almost all cases in our XNLI setup given a \basemodel{} trained using FedDPA-T~\cite{FedDPA} and DEPT(SPEC)~\cite{DEPT} respectively. In short, \method{} can be integrated into a larger family of existing personalized FL approaches, listed in Section~\ref{sec:related}, to further improve personalization performance. 

\begin{table*}[t]
\caption{Avg. METEOR/ROUGE-1/ROUGE-L for \seen{} clients in our Fed-Aya setup. {\em Base model} is off-the-shelf Llama-3.2-3B-Instruct.}
\vspace{0.5em}
\label{tab:lama_fedaya_seen}
\begin{scriptsize}\resizebox{0.98\textwidth}{!}{
\begin{tabular}{c|l|l|l|l|l|l|l|l|c}
\toprule
% \textbf{Lora Rank}  
\textbf{$\mathbf{r}$} & \multicolumn{1}{c|}{\textbf{Approach}} & \multicolumn{1}{c|}{\textbf{te}} & \multicolumn{1}{c|}{\textbf{ar}} & \multicolumn{1}{c|}{\textbf{es}} & \multicolumn{1}{c|}{\textbf{en}} & \multicolumn{1}{c|}{\textbf{fr}} & \multicolumn{1}{c|}{\textbf{zh}} & \multicolumn{1}{c|}{\textbf{pt}} 
& \textbf{Wins} \\ \midrule
% \multirow{5}{*}{1}  & LoRA                                   & 0.2372/0.1409/0.1368             & 0.3291/0.0625/0.0617             & 0.3863/0.4134/0.385              & 0.3229/0.3613/0.2965             & 0.2811/0.3446/0.2802             & 0.1022/0.1184/0.117              & 0.3765/0.4415/0.4023                                         & 1             \\ % \cline{2-11} 
%                     & AdaLoRA                              & 0.2344/0.1394/0.1356             & 0.3478/0.0698/0.0691             & 0.3957/0.4278/0.3989             & 0.3537/0.3999/0.3307             & 0.2951/0.3638/0.3017             & 0.1065/0.1215/0.1195             & 0.3901/0.4547/0.4135                                         & 1             \\ % \cline{2-11} 
%                     & BayesTune-LoRA                            & 0.235/0.1364/0.1335              & 0.3205/0.0618/0.0615             & 0.3729/0.3948/0.3671             & 0.2834/0.2611/0.2091             & 0.294/0.3511/0.2867              & 0.0935/0.1044/0.1044             & 0.3567/0.4194/0.3802                                         & 0             \\ % \cline{2-11} 
%                     & FedL2P                               & 0.2231/0.1355/0.1323             & 0.322/0.061/0.0608               & 0.3867/0.4124/0.3837             & 0.3076/0.3128/0.256              & 0.2943/0.3569/0.2949             & 0.0812/0.1208/0.12               & 0.3621/0.4242/0.3865                                         & 0             \\ % \cline{2-11} 
%                     & \method{}                                 & \textbf{0.2399/0.1371/0.1336}    & \textbf{0.3523/0.0668/0.066}     & \textbf{0.4022/0.4328/0.4031}    & \textbf{0.3578/0.4201/0.3477}    & \textbf{0.3264/0.3875/0.3117}    & \textbf{0.1134/0.1153/0.1134}    & \textbf{0.3978/0.4614/0.4187}                                & \textbf{4}    \\ \hline
\multirow{5}{*}{2}  & LoRA                                   & 0.2354/0.1383/0.1344             & 0.3364/0.0659/0.0656             & 0.3871/0.4142/0.3855             & 0.3345/0.3793/0.3102             & 0.2884/0.3569/0.2968             & 0.1078/0.1208/0.1194             & 0.3835/0.4478/0.4091                                         & 0             \\ % \cline{2-11} 
                    & AdaLoRA                              & 0.2373/0.1428/0.1391             & 0.3440/0.0668/0.0665              & \textbf{0.3944/0.4273/0.3994}    & 0.3536/0.4042/0.3334             & 0.2858/0.3528/0.2937             & 0.1078/\textbf{0.1226}/\textbf{0.1200}               & 0.3834/0.4514/0.4108                                         & 2             \\ % \cline{2-11} 
                    & BayesTune-LoRA                            & 0.2406/\textbf{0.1440/0.1410}               & 0.3240/0.0579/0.0576              & 0.3797/0.4065/0.3781             & 0.2922/0.2841/0.2302             & 0.2883/0.3535/0.2927             & 0.0946/0.1132/0.1119             & 0.3674/0.4327/0.3932                                         & 1             \\ % \cline{2-11} 
                    & FedL2P                               & 0.2291/0.1356/0.1322             & 0.3329/0.0687/0.0675             & 0.3783/0.4034/0.3762             & 0.3250/0.3667/0.3032              & 0.2944/0.3614/0.3004             & 0.0869/0.1173/0.1162             & 0.3776/0.4439/0.4047                                         & 0             \\ % \cline{2-11} 
                    & \method{}                                 & \textbf{0.2434}/0.1440/0.1403     & \textbf{0.3663/0.0785/0.0764}    & 0.3941/0.4224/0.3928             & \textbf{0.3746/0.4321/0.3610}     & \textbf{0.3442/0.4057/0.3318}    & \textbf{0.1144}/0.1171/0.1161    & \textbf{0.3987/0.4646/0.4201}                                & \textbf{4}    \\ \hline
\multirow{5}{*}{4}  & LoRA                                   & 0.2339/0.1317/0.1282             & 0.3497/0.0679/0.0667             & 0.4016/\textbf{0.4369/0.4077}             & 0.3458/0.3993/0.3282             & 0.2988/0.3698/0.3044             & \textbf{0.1139}/0.1206/0.1183    & 0.3955/0.4588/\textbf{0.4186}                                         & 1             \\ % \cline{2-11} 
                    & AdaLoRA                              & 0.2331/0.1404/0.1365             & 0.3350/0.0663/0.0659              & 0.3877/0.4182/0.3900               & 0.3482/0.3948/0.3252             & 0.2886/0.3516/0.2930              & 0.1091/0.1240/0.1215              & 0.3816/0.4493/0.4091                                         & 0             \\ % \cline{2-11} 
                    & BayesTune-LoRA                            & 0.2369/0.1426/0.1395             & 0.3324/0.0609/0.0600               & 0.3846/0.4111/0.3825             & 0.3121/0.3195/0.2553             & 0.2900/0.3557/0.2951               & 0.1104/0.1181/0.1172             & 0.3707/0.4386/0.4015                                         & 0             \\ % \cline{2-11} 
                    & FedL2P                               & 0.2298/0.1364/0.1327             & 0.3376/0.0711/0.0692             & 0.3797/0.4130/0.3836              & 0.3392/0.3787/0.3130              & 0.2938/0.3648/0.3053             & 0.0974/\textbf{0.1264/0.1240}              & 0.3876/0.4561/0.4159                                         & 1             \\ % \cline{2-11} 
                    & \method{}                                 & \textbf{0.2455/0.1495/0.1455}    & \textbf{0.3671/0.0749/0.0736}    & \textbf{0.4021}/0.4333/0.3994    & \textbf{0.3831/0.4400/0.3648}      & \textbf{0.3381/0.4004/0.3225}    & 0.1129/0.1212/0.1200               & \textbf{0.4018/0.4618}/0.4172                                & \textbf{5}    \\ \hline
\multirow{5}{*}{8}  & LoRA                                   & 0.2361/0.1368/0.1329             & 0.3573/0.0708/0.0695             & 0.4017/0.4341/0.4047             & 0.3586/0.4182/0.3480              & 0.3047/0.3667/0.3029             & 0.1156/\textbf{0.1260}/0.1237              & 0.3982/0.4605/0.4186                                         & 0             \\ % \cline{2-11} 
                    & AdaLoRA                              & 0.2353/0.1443/0.1399             & 0.3272/0.0648/0.0645             & 0.3863/0.4217/0.3922             & 0.3437/0.3876/0.3183             & 0.2855/0.3552/0.2929             & 0.1044/0.1242/0.1216             & 0.3740/0.4421/0.4038                                          & 0             \\ % \cline{2-11} 
                    & BayesTune-LoRA                            & 0.2397/0.1393/0.1355             & 0.3444/0.0687/0.0678             & 0.4031/0.4327/0.4032             & 0.3294/0.3521/0.2812             & 0.2962/0.3585/0.3008             & 0.1130/0.1213/0.1193              & 0.3844/0.4480/0.4084                                          & 0             \\ % \cline{2-11} 
                    & FedL2P                               & 0.2324/0.1352/0.1316             & 0.3446/0.0698/0.0681             & 0.3819/0.4153/0.3855             & 0.3547/0.4082/0.3362             & 0.3030/0.3700/0.3076                & 0.0988/0.1217/0.1199             & 0.3940/\textbf{0.4611/0.4201}                                          & 1             \\ % \cline{2-11} 
                    & \method{}                                 & \textbf{0.2431/0.1479/0.1442}    & \textbf{0.3713/0.0792/0.0779}    & \textbf{0.4077/0.4409/0.4063}    & \textbf{0.3844/0.4441/0.3687}    & \textbf{0.3440/0.4031/0.3222}     & \textbf{0.1156}/0.1246/\textbf{0.1240}     & \textbf{0.4009}/0.4567/0.4119                                & \textbf{6}    \\ \hline
\multirow{5}{*}{16} & LoRA                                   & 0.2413/0.1387/0.1355             & 0.3605/0.0711/0.0699             & 0.3864/0.4227/0.3897             & 0.3603/0.4248/0.3533             & 0.3275/0.3894/0.3178             & \textbf{0.1194}/0.1241/0.1227    & 0.4025/\textbf{0.4659/0.4225}                                         & 1             \\ % \cline{2-11} 
                    & AdaLoRA                              & 0.2349/0.1388/0.1348             & 0.3248/0.0659/0.0655             & 0.3805/0.4141/0.3861             & 0.3346/0.3702/0.3039             & 0.2818/0.3554/0.2973             & 0.1022/0.1207/0.1181             & 0.3686/0.4379/0.4012                                         & 0             \\ % \cline{2-11} 
                    & BayesTune-LoRA                            & 0.2374/0.1351/0.1310              & 0.3556/\textbf{0.0813/0.0795}             & 0.3985/0.4317/0.4013             & 0.3477/0.3998/0.3295             & 0.2995/0.3621/0.2965             & 0.1167/0.1205/0.1186             & 0.3974/0.4579/0.4153                                         & 1             \\ % \cline{2-11} 
                    & FedL2P                               & 0.2345/0.1417/0.1368             & 0.3457/0.0643/0.0633             & 0.3884/0.4185/0.3810              & 0.3740/0.4420/0.3667               & 0.3301/0.3752/0.2945             & 0.0930/0.1223/0.1211              & 0.3956/0.4543/0.4086                                         & 0             \\ % \cline{2-11} 
                    & \method{}                                 & \textbf{0.2444/0.1447/0.1406}    & \textbf{0.3735}/0.0750/0.0740      & \textbf{0.4160/0.4462/0.4105}     & \textbf{0.3920/0.4488/0.3725}     & \textbf{0.3435/0.3992/0.3204}    & 0.1103/\textbf{0.1289/0.1270}              & \textbf{0.4052}/0.4623/0.4177                                & \textbf{5}    \\ \bottomrule
\end{tabular}
}
\end{scriptsize}
\vspace{-1.5em}
\end{table*}
\begin{table*}[t]
\centering
\caption{Avg. METEOR/ROUGE-1/ROUGE-L for \unseen{} clients in our Fed-Aya setup. {\em Base model} is off-the-shelf Llama-3.2-3B-Instruct.}
\vspace{0.5em}
\label{tab:lama_fedaya_unseen}

\begin{scriptsize}\resizebox{0.98\textwidth}{!}{
\begin{tabular}{c|l|l|l|l|l|l|l|l|l|c}
\toprule

% \textbf{Lora Rank} 
\textbf{$\mathbf{r}$} & \multicolumn{1}{c|}{\textbf{Approach}} & \multicolumn{1}{c|}{\textbf{te}} & \multicolumn{1}{c|}{\textbf{ar}} & \multicolumn{1}{c|}{\textbf{es}} & \multicolumn{1}{c|}{\textbf{en}} & \multicolumn{1}{c|}{\textbf{fr}} & \multicolumn{1}{c|}{\textbf{zh}} & \multicolumn{1}{c|}{\textbf{pt}} & \multicolumn{1}{c|}{\textbf{ru}} & \textbf{Wins} \\ \midrule
% \multirow{5}{*}{1}  & LoRA                                   & 0.1614/0.1123/0.1107             & 0.2391/0.045/0.045               & 0.4175/0.4901/0.4458             & 0.3197/0.3135/0.2507             & 0.3998/0.3333/0.3333             & 0.233/0.002/0.002                & 0.3431/0.4039/0.3855             & 0.2368/0.1813/0.1728             & 1             \\ % \cline{2-11} 
%                     & AdaLoRA                              & 0.1592/0.0914/0.0898             & 0.2363/0.0384/0.0384             & \textbf{0.4315/0.4908/0.4397}    & 0.3169/0.3104/0.2486             & 0.3222/0.7222/0.7222             & 0.2561/0.002/0.002               & 0.3302/0.4122/0.395              & 0.2446/0.1969/0.1969             & 1             \\ % \cline{2-11} 
%                     & BayesTune-LoRA                            & 0.1526/0.0591/0.0575             & 0.207/0.0406/0.0377              & 0.385/0.4615/0.4123              & \textbf{0.3266/0.3085/0.2423}    & 0.3998/0.3333/0.3333             & 0.2322/0.002/0.002               & 0.3088/0.3505/0.3313             & 0.2329/0.1663/0.1608             & 0             \\ % \cline{2-11} 
%                     & FedL2P                               & 0.158/0.0747/0.0731              & 0.2621/0.0441/0.0412             & 0.4056/0.4811/0.4332             & 0.3208/0.3096/0.2478             & 0.3998/0.3333/0.3333             & 0.2313/0.002/0.002               & 0.2928/0.3719/0.3521             & \textbf{0.2577/0.184/0.184}      & 0             \\ % \cline{2-11} 
%                     & \method{}                                 & \textbf{0.1722/0.0767/0.0751}    & \textbf{0.2759/0.0474/0.0472}    & 0.4257/0.5306/0.4758             & 0.3127/0.3198/0.2695             & \textbf{0.5012/0.746/0.746}      & \textbf{0.2655/0.0066/0.0066}    & \textbf{0.3555/0.4287/0.4092}    & 0.2337/0.1236/0.1218             & \textbf{6}    \\ \hline
\multirow{5}{*}{2}  & LoRA                                   & 0.1553/0.0854/0.0838             & 0.2425/0.0458/0.0418             & 0.4275/0.4916/0.4406             & \textbf{0.3248}/0.3120/0.2494     & 0.5513/0.6667/0.6667    & 0.2489/0.0020/0.0020               & \textbf{0.3610}/0.4228/0.4033     & 0.2242/0.1797/0.1735             & 0             \\ % \cline{2-11} 
                    & AdaLoRA                              & 0.1595/0.1108/0.1092             & 0.2326/\textbf{0.0721/0.0721}             & 0.4340/0.4954/0.4435              & 0.3234/0.3102/0.2513             & 0.5513/0.6667/0.6667   & 0.2504/0.0020/0.0020               & 0.3338/0.4121/0.3970              & 0.2335/0.1758/0.1703             & 1             \\ % \cline{2-11} 
                    & BayesTune-LoRA                            & \textbf{0.1676}/0.0888/0.0858    & 0.2243/0.0487/0.0457             & 0.3821/0.4564/0.4089             & 0.3176/0.3069/0.2437             & 0.3998/0.3333/0.3333             & 0.2308/0.0020/0.0020               & 0.3052/0.3716/0.3553             & 0.2484/0.1779/0.1773             & 0             \\ % \cline{2-11} 
                    & FedL2P                               & 0.1568/\textbf{0.1156/0.1097}             & 0.2368/0.0496/0.0496             & 0.4134/0.4810/0.4350               & 0.3141/0.3050/0.2418              & 0.3998/0.3333/0.3333             & 0.2350/0.0020/0.0020               & 0.3442/0.4012/0.3853             & 0.2451/0.2010/0.2010               & 1             \\ % \cline{2-11} 
                    & \method{}                                 & 0.1674/0.0736/0.0720              & \textbf{0.2636}/0.0485/0.0485    & \textbf{0.4391/0.5335/0.4796}    & 0.3084/\textbf{0.3278/0.2718}             & 0.2155/0.2222/0.2222             & \textbf{0.2771/0.0066/0.0066}    & 0.3477/\textbf{0.4295/0.4110}              & \textbf{0.3413/0.2929/0.2662}    & \textbf{5}    \\ \hline
\multirow{5}{*}{4}  & LoRA                                   & 0.1463/0.0852/0.0758             & 0.2475/0.0332/0.0332             & 0.4178/0.4762/0.4269             & \textbf{0.3394}/0.3264/0.2648    & 0.5513/0.6667/0.6667             & 0.2566/0.0020/0.0020               & \textbf{0.3554}/0.425/0.4044     & 0.2368/0.1553/0.1530              & 0             \\ % \cline{2-11} 
                    & AdaLoRA                              & 0.1634/0.0798/0.0782             & 0.2229/\textbf{0.0575/0.0575}             & 0.4258/0.4871/0.4387             & 0.3191/0.3129/0.2519             & 0.5513/0.6667/0.6667             & 0.2458/0.0020/0.0020               & 0.3523/0.4242/0.4067             & 0.2427/\textbf{0.1716/0.1654}             & 2             \\ % \cline{2-11} 
                    & BayesTune-LoRA                            & \textbf{0.1689/0.1044/0.1027}    & 0.2308/0.0557/0.0557             & 0.3911/0.4617/0.4141             & 0.3219/0.3088/0.2445             & 0.3998/0.3333/0.3333             & 0.2349/0.0020/0.0020               & 0.3233/0.3798/0.3642             & 0.2437/0.1714/0.1653             & 1             \\ % \cline{2-11} 
                    & FedL2P                               & 0.1556/0.0627/0.0611             & 0.2465/0.0372/0.0331             & 0.4292/0.4984/0.4445             & 0.3237/0.3103/0.2498             & 0.5513/0.6667/0.6667             & 0.2454/0.0020/0.0020               & 0.3446/0.4223/0.4056             & 0.2329/0.1703/0.1625             & 0             \\ % \cline{2-11} 
                    & \method{}                                 & 0.1618/0.0706/0.0689             & \textbf{0.2621}/0.0386/0.0386    & \textbf{0.4619/0.5591/0.5061}    & 0.3194/\textbf{0.3381/0.2835}             & 0.3998/0.3333/0.3333             & \textbf{0.2898/0.0094/0.0094}    & 0.3508/\textbf{0.4306/0.4115}             & \textbf{0.2538}/0.1531/0.1531    & \textbf{4}    \\ \hline
\multirow{5}{*}{8}  & LoRA                                   & 0.1600/0.0835/0.0791               & 0.2669/0.0524/0.0484             & 0.4312/0.5136/0.4573             & 0.3352/0.3326/0.2706             & \textbf{0.5664}/0.6667/0.6667    & 0.2564/0.0045/0.0045             & 0.3465/0.4264/0.4090              & 0.2275/\textbf{0.1847/0.1847}             & 1             \\ % \cline{2-11} 
                    & AdaLoRA                              & 0.1604/\textbf{0.1045/0.1029}             & 0.2226/0.0606/0.0606             & 0.4136/0.4889/0.4350              & 0.3200/0.3082/0.2462               & 0.3164/0.5000/0.5000                   & 0.2503/0.0020/0.0020               & 0.3398/0.4203/0.4008             & 0.2370/0.1687/0.1625              & 1             \\ % \cline{2-11} 
                    & BayesTune-LoRA                            & 0.1607/0.0735/0.0688             & 0.2377/0.0731/0.0731             & 0.4102/0.4803/0.4352             & 0.3261/0.3114/0.2458             & 0.3998/0.3333/0.3333             & 0.2476/0.0020/0.0020               & 0.3483/0.3998/0.3817             & 0.2514/0.1668/0.1612             & 0             \\ % \cline{2-11} 
                    & FedL2P                               & 0.1602/0.0586/0.0570              & 0.2462/0.0447/0.0447             & 0.4339/0.5007/0.4513             & \textbf{0.3482}/0.3281/0.2642    & 0.3301/\textbf{0.7460/0.7460}               & 0.2617/0.0020/0.0020              & 0.3362/0.4262/0.4109             & 0.2300/0.1542/0.1542               & 1             \\ % \cline{2-11} 
                    & \method{}                                 & \textbf{0.1720}/0.0948/0.0890      & \textbf{0.2787/0.0804/0.0804}    & \textbf{0.5022/0.5825/0.5168}    & 0.3377/\textbf{0.3661/0.2984}             & 0.2730/0.2756/0.2756              & \textbf{0.3186/0.0249/0.0241}    & \textbf{0.3623/0.4375/0.4158}    & \textbf{0.2770}/0.1171/0.1171     & \textbf{5}    \\ \hline
\multirow{5}{*}{16} & LoRA                                   & 0.1610/\textbf{0.0901/0.0885}              & 0.2585/0.0403/0.0371             & 0.4259/0.5245/0.4700               & 0.3104/0.3302/0.2775             & \textbf{0.4077}/0.3571/0.3571    & 0.2651/0.0020/0.0020               & \textbf{0.3501}/0.4260/0.4065     & 0.2751/\textbf{0.2048/0.2048}             & 2             \\ % \cline{2-11} 
                    & AdaLoRA                              & 0.1412/0.0679/0.0659             & 0.2119/0.0439/0.0439             & 0.4149/0.4867/0.4382             & 0.3241/0.3123/0.2473             & 0.3764/0.2879/0.2879             & 0.2421/0.0020/0.0020               & 0.3308/0.4071/0.3882             & 0.2260/0.1723/0.1667              & 0             \\ % \cline{2-11} 
                    & BayesTune-LoRA                            & 0.1518/0.0641/0.0560              & 0.2448/0.0318/0.0318             & 0.4331/0.5021/0.4519             & 0.3187/0.3128/0.2524             & 0.3928/0.3148/0.3148             & 0.2546/0.0020/0.0020               & 0.3378/0.3921/0.3728             & 0.2566/0.1867/0.1860              & 0             \\ % \cline{2-11} 
                    & FedL2P                               & 0.1613/0.0726/0.0707             & 0.2605/\textbf{0.0855/0.0841}             & 0.4181/0.5049/0.4764             & 0.3383/0.3723/0.3079             & 0.4065/\textbf{0.3889/0.3889}             & 0.2547/0.0060/0.0060               & 0.3262/0.3915/0.3695             & 0.2574/0.1404/0.1404             & 2             \\ % \cline{2-11} 
                    & \method{}                                 & \textbf{0.1670}/0.0826/0.0810      & \textbf{0.2809}/0.0751/0.0751    & \textbf{0.4935/0.5715/0.5119}    & \textbf{0.3424/0.3800/0.3146}      & 0.2562/0.2626/0.2626             & \textbf{0.3293/0.0364/0.0330}     & 0.3452/\textbf{0.4280/0.4103}              & \textbf{0.3320}/0.1856/0.1589     & \textbf{4}    \\ \bottomrule

\end{tabular}
}

\end{scriptsize}
\vspace{-1.2em}
\end{table*}


\subsection{Results on Instruction-Tuning Generation}\label{sec:ift_gen}

We evaluate our approach on the more challenging real-world multilingual benchmark, Fed-Aya. Tables~\ref{tab:lama_fedaya_seen} and \ref{tab:lama_fedaya_unseen} show the average METEOR~\cite{meteor}/ROUGE-1/ROUGE-L~\cite{ROUGE} of each language given the off-the-shelf instruction finetuned Llama-3.2-3B (Llama-3.2-3B-Instruct) for \seen{} and \unseen{} clients respectively. Similarly, in Appendix Tables~\ref{tab:mobilellama_fedaya_seen} and \ref{tab:mobilellama_fedaya_unseen}, we show the same tables given a pretrained MobileLLaMA-1.4B model  trained using Standard FL with LoRA following the training recipe from FedLLM-Bench~\cite{fedllm-bench}. These two models represent scenarios where the \basemodel{} may or may not be trained using FL. Similarly to our text classification results, we report ``Wins" if an approach has a better performance in at least $2$ out of $3$ metrics. 

In all four tables, \method{} outperforms baselines in most scenarios. We also observe that FedL2P underperforms standard baselines in most cases, a phenomenon also observed for our XNLI setup when the \basemodel{} is trained with FedDPA-T (Tables~\ref{tab:xnli_seen_feddpa}). We hypothesize that the inner-loop optimization in FedL2P fail to reach a stationary point\footnote{FedL2P relies on the implicit function theorem for hypergradient computation.} due to the inherent task difficulty (Fed-Aya) or a less-performant \basemodel{}, resulting in a sub-optimal hypergradient and downstream performance. 

\subsubsection{Limitations of \method{}.} In some cases, \method{} falls short, especially in the recall performance (ROUGE), such as Russian (\textit{ru}) and French (\textit{fr}) for \unseen{} clients for both {\em base models}. These cases highlight a couple of limitations of our approach: \textit{i)} \textit{ru} is not seen by PSG during federated training; there are no \textit{ru} samples in any clients in the seen pool and \textit{ii)} none of the clients in the unseen pool have \textit{fr} as a predominant language (Fig.~\ref{fig:fed-aya}). In the case of \textit{ru}, there are no other languages that are similar to \textit{ru} in the seen pool, resulting in worse performance. Hence, we do not expect a similar outcome in datasets with a more diverse pool of clients.

For \textit{fr}, as the number of predominant language samples are orders of magnitude higher than that of \textit{fr} samples, the generated $\bm{\lambda}$ are catered towards the predominant language. A simple solution to counteract this limitation is to extend PSG to generate $\bm{\lambda}$ per instance, rather than per client. However, doing so is extremely costly, requiring a forward pass through the PSG for every sample. This calls for novel efficient solutions that can better handle each client's minority languages and is left as future work.

\subsection{Cost of \method{}}

Table~\ref{tab:cost} shows the mean latency, in seconds, and the peak memory usage across 100 runs on the first client in the \seen{} pool for $r=16$ using a single Nvidia A100 GPU. Non-FL baselines do not incur a federated training cost while FL approaches requires training the PSG. Comparing FedL2P and \method{}, \method{} does not require expensive second-order optimization, resulting in better efficiency. We also note that FedL2P needs to be run for every rank while \method{} runs once for all targeted ranks.

For communication costs, not shown in the table, \method{} is more costly as it predicts per LoRA rank while FedL2P predicts per layer. Nonetheless, these costs are negligible compared to running FL on the \basemodel{}; FedL2P uses 0.02\% and 0.002\% and \method{} uses 0.2\% and 0.16\% of the parameters of mBERT and Llama-3.2-3B respectively.

During inference, FL-based approaches incur an additional forward pass of \basemodel{} and the PSG compared to non-FL approaches. Memory-wise, \method{} results in the smallest memory footprint for autoregressive generation as the PSG learns not to attach LoRA modules $\lambda_l=0$ on some layers, skipping {\em matmul} operations entirely.

\begin{figure}[t]
    \vspace{-0.2em}
    \small
    \centering
    \includegraphics[width=0.94\columnwidth,trim={0cm 0cm 0cm 3.5cm},clip]{figures/xnli_out_0.5_seen.png}
    % \captionsetup{font=small,labelfont=bf}
    \vspace{-2em}
    \caption{$\bm{\lambda}$ distance among languages in our XNLI setup. Each block shows the log-scale normalized average Euclidean distances between all pairs of clients' $\bm{\lambda}$ for two languages (see text). The smaller the distance, the more similar $\bm{\lambda}$ is. }
    \label{fig:xnli_out}
    \vspace{-1.5em}
\end{figure}

\subsection{Further Analysis}\label{sec:analysis}

In this section, we further analyze $\bm{\lambda}$ and how they differ across languages. Surprisingly, we find that \method{} learns language-agnostic rank structures. In other words, depending on the task and the \basemodel{}, the rank structure of $\bm{\lambda}$ is fixed across languages. For instance, in the case where $r=2$, \method{} allocate ranks to dense layers instead of attention blocks. With more budget, \textit{e.g.},~$r=16$, \method{} allocates more rank to either the query attention layer or the value attention layer depending on the setup. We show these rank structures across all setups for $r=2$ and $r=16$ in Appendix Fig.~\ref{fig:xnli_fedavg_out_r16}-\ref{fig:llama3_fedavg_out_r2}.

While the rank structure is the same across languages, the rank-wise scales (absolute values of $\bm{\lambda}$) differ. Following FedL2P, we visualize the difference in $\bm{\lambda}$ for different languages using the normalized mean distance, $d(j,k)$, between all clients pairs holding data for languages $j$ and $k$. Fig.~\ref{fig:xnli_out} and Appendix Fig.~\ref{fig:masakha_out} show these distances for XNLI and MasakhaNEWS setup respectively. Specifically, the value of each block in each figure is computed as follows: $\log(\frac{d(j,k)}{\sqrt{d(j,j)}\sqrt{d(k,k)}})$. Hence, the smaller the distance, the more similar $\bm{\lambda}$ is between languages. The results are aligned with our intuition that similar languages have similar $\bm{\lambda}$. For instance, the closest language to Urdu (\textit{ur}) is Arabic (\textit{ar}), both of which have the closest $\bm{\lambda}$ similarity (Fig.~\ref{fig:xnli_out}); likewise, for Tigrinya (\textit{tir}) and Amharic (\textit{amh}) in Appendix Fig.~\ref{fig:masakha_out}. We also observe that unrelated languages have similar $\bm{\lambda}$, \textit{e.g.},~Mandarin (\textit{zh}) and Vietnamese (\textit{vi}) share similar $\bm{\lambda}$ with the Indo-European languages (Fig.~\ref{fig:xnli_out}). This finding adds to existing evidence that leveraging dissimilar languages can sometimes benefit particular languages~\cite{fedllm-bench}.


% \qy{In this paper, we propose an efficient single-stage framework called \nickname{} for 3D object detection. Considering the task of object detection inherently focuses on the foreground points, we propose an instance-aware learning-based downsampling way to automatically select the sparse yet important instance points. In addition, a dedicated contextual centroid perception module is proposed to fully exploit the geometrical structure around the bounding boxes. Extensive experiments conducted on the KITTI detection benchmark demonstrated the superior efficiency and accuracy of the proposed \nickname{}. \revise{In future work, we will further tackle extreme cases such as overlapped bounding boxes.}}

%This paper presents a new point-based single-stage 3D object detection networks, named \nickname{}. With novel instance-aware downsampling strategy and centroid rally module, we can effectively and efficiently achieve muti-class 3D object detection in a bottom-up manner.  Our \nickname{} achieves the best results among pure point-based methods, and provides a state-of-the-art efficiency than existing LiDAR detectors. In the future, we will focus on designing an efficient network to achieve real-time and robust 3D detection in 360-degree LiDAR scenes.

\qy{In this paper, we propose an efficient solution termed \nickname{} for point-based 3D object detection in LiDAR point clouds. Considering the task of object detection inherently focuses on the foreground information, we propose an instance-aware learning-based downsampling way to automatically select the sparse yet important instance points. Additionally, a dedicated contextual centroid perception module is proposed to fully exploit the geometrical structure around the bounding boxes. Extensive experiments conducted on three detection benchmarks demonstrated the superior efficiency and accuracy of the proposed \nickname{}. 
}

\smallskip\noindent\textbf{Limitations.} Although the proposed \nickname{} can achieve remarkable efficiency in object detection of large-scale LiDAR points clouds, it also has limitations. \textit{e.g.,} the instance-aware sampling relies on the semantic prediction of each point, which is susceptible to class imbalances distribution. For future work, we will further explore advanced techniques to alleviate the imbalanced issue.



% In the unusual situation where you want a paper to appear in the
% references without citing it in the main text, use \nocite
% \nocite{langley00}

% \section*{Impact Statement} % doesnt count towards the 8 pages
% This paper presents work whose goal is to address the critical challenges of data heterogeneity and language diversity in federated learning. There are many potential societal consequences as well as threats and attacks that may increase privacy risks, none of which we feel must be specifically highlighted here as our work is orthogonal to many of the current security measures in place.

\bibliography{main}
\bibliographystyle{icml2025}


%%%%%%%%%%%%%%%%%%%%%%%%%%%%%%%%%%%%%%%%%%%%%%%%%%%%%%%%%%%%%%%%%%%%%%%%%%%%%%%
%%%%%%%%%%%%%%%%%%%%%%%%%%%%%%%%%%%%%%%%%%%%%%%%%%%%%%%%%%%%%%%%%%%%%%%%%%%%%%%
% APPENDIX
%%%%%%%%%%%%%%%%%%%%%%%%%%%%%%%%%%%%%%%%%%%%%%%%%%%%%%%%%%%%%%%%%%%%%%%%%%%%%%%
%%%%%%%%%%%%%%%%%%%%%%%%%%%%%%%%%%%%%%%%%%%%%%%%%%%%%%%%%%%%%%%%%%%%%%%%%%%%%%%
\newpage
\appendix
\newpage
\lstset{language=Python,basicstyle=\small\ttfamily,columns=fullflexible}
\section{Detailed Experimental Setup}\label{appendix:experiments}

For reproducibility and completeness, we provide comprehensive details of all setups, datasets, tasks, models, baselines, and hyperparameters. Code is in the process of being released.

\subsection{Tasks, Datasets, and Data Partitioning}

\noindent\textbf{XNLI~\cite{XNLI}}~A natural language inference benchmark dataset for evaluating cross-lingual understanding covering 15 diverse languages including both high- and low-resources languages: English, French, Spanish, German, Greek, Bulgarian, Russian, Turkish, Arabic, Vietnamese, Thai, Chinese, Hindi, Swahili and Urdu. XNLI consists of premise-hypothesis pairs, labeled as entailment, contradiction, or neutral across different languages. We sample 2k instances from the XNLI train split and 500 instances from the test split for each pool. The data is then divided equally among 20 clients for each language using the latent Dirichlet allocation (LDA) partition with $\alpha=0.5$. Hence, the total number of clients is 600 (15 languages $\cdot$ 20 clients per language $\cdot$ 2 pools).

\noindent\textbf{MasakhaNEWS~\cite{MasakhaNEWS}}~A news topic classification benchmark designed to address the lack of resources for African languages. It covers 2 high-resource languages, English and French, and 14 low-resource languages, namely Amharic, Hausa, Igbo, Lingala, Luganda, Naija, Oromo, Rundi, chiShona, Somali, Kiswahili, Tigrinya, isiXhosa, and Yorùbá. Each sample contains a headline, the body text, and one of the 7 labels: business, entertainment, health, politics, religion, sports, or technology. We first combine all samples from the MasakhaNEWS train and validation split to form our train set, and use the MasakhaNEWS test split as our test set. We then split both train and test in each of the 16 languages by half for each pool. Following our XNLI setup, we adopt LDA $\alpha=0.5$ and divide each language's data equally into 10 clients. Hence, the total number of clients is 320 (16 languages $\cdot$ 10 clients per language $\cdot$ 2 pools). Note that unlike XNLI, the number of samples for each language differs, hence there is quantity skew across clients. 

\noindent\textbf{Fed-Aya~\cite{fedllm-bench}}~A federated instruction tuning benchmark, based on Aya~\cite{singh2024aya}, where the data is annotated by contributors and partitioned by annotator ID. Following FedLLM-Bench~\cite{fedllm-bench}, we focus on 6 high-resource languages, English, Spanish, French, Russian, Portuguese, Chinese, and 2 low-resource languages, Arabic and Telugu. Additionally, we filter out the other languages from the dataset. Out of 38 clients, we select 8 for our \unseen{} pool, $\text{client\_ids}=[21, 22, 23, 24, 25, 26, 27, 34]$ and the rest goes into our \seen{} pool. Each client can have up to 4 languages where the number of data samples can range from a hundred to over a thousand samples per client.

\subsection{Models, Tokenizers, and Data Preprocessing}

\noindent\textbf{mBERT~\cite{BERT}.}~We use the pretrained multilingual BERT with its WordPiece tokenizer for all sequence classification experiments, namely all XNLI and MasakhaNEWS setups with various {\em base models}. For both datasets, we use a training batch size of 32 and pad input tokens on the right to a max token length of 128 and 256 respectively.

\noindent\textbf{MobileLLaMA-1.4B~\cite{mobilellama}.}~We train a \basemodel{} with a pretrained MobileLLaMA-1.4B with Standard FL using LoRA in our Fed-Aya setup. We use the default LLaMA tokenizer which is a BPE model based on sentencepiece~\cite{Kudo2018SentencePieceAS} and adopt the UNK token as the PAD token. During training, we use an effective batch size of $16$ and pad right to the longest token in the batch up to a max token length of 1024. For evaluation, we use a batch size of 8, padding left instead, with greedy sampling up to a max new token length of 1024. We use the Alpaca template to format each prompt:

\begin{lstlisting}[linewidth=\columnwidth,breaklines=true]
alpaca_template = """Below is an instruction that describes a task. Write a response that appropriately completes the request.

### Instruction:
{} 

### Response: {}{}"""
\end{lstlisting}

\noindent\textbf{Llama-3.2-3B~\cite{llama3}.}~We use the off-the-shelf Llama-3.2-3B-Instruct model as our \basemodel{} and its default tokenizer which is a BPE model based on tiktoken\footnote{https://github.com/openai/tiktoken}. Training and evaluation hyperparameters are the same as the ones we use for MobileLLaMA. The only two differences are \textit{1)} we add a PAD token `\verb|<pad>|', and \textit{2)} we use the Llama 3 instruction template instead:

\begin{lstlisting}[linewidth=\columnwidth,breaklines=true]
llama3_instruct_template = """<|begin_of_text|><|start_header_id|>user<|end_header_id|>

{}<|eot_id|><|start_header_id|>assistant<|end_header_id|>

{}{}
"""
\end{lstlisting}

\subsection{Complementary Approaches and Base Models}

In this work, we experiment with different {\em base models} to show that \method{} is complementary to a range of off-the-shelf models and models trained using existing FL approaches. In this section, we detail the different approaches we used to obtain these {\em base models}.

\noindent\textbf{Standard FL.} Standard FL involves training a single global model. Given a pretrained LLM, we run FedAvg on the \seen{} pool of clients, where $10\%$ of participating clients are sampled every round to train the model before sending the weights back for aggregation. In our XNLI and MasakhaNEWS setup, we do full fine-tuning of mBERT, setting each client's learning rate to $5e-5$ and running FedAvg for 100 rounds. In our Fed-Aya setup, we adopt the training recipe from FedLLM-Bench~\cite{fedllm-bench} for MobileLLaMA-1.4B, where we do PEFT with LoRA applied to query and value attention weights ($r=16$, $\alpha_{lora}=32$, dropout$=0.05$) for 200 rounds. We use the cosine learning rate decay over 200 rounds with initial learning rate $2e^{-5}$ and minimum learning rate $1e^{-6}$.

\noindent\textbf{Personalized FL.} We train personalized {\em base models} using FedDPA-T~\cite{FedDPA} and DEPT(SPEC)~\cite{DEPT} in our XNLI setup. FedDPA-T proposed having two separate LoRA adapters, one of which is shared (global) and the other is kept locally for each client (local). 
We adopted their training recipe for sequence classification, where the classifier is shared together with the global LoRA modules and the local LoRA modules stay local. LoRA modules are only applied to query and value attention weights (r=8, $\alpha_{lora}=8$, dropout=$0.05$). We set the learning rate to $5e^{-5}$.

DEPT (SPEC), on the other hand, proposed having personalized token and positional embeddings for each client. As DEPT was proposed for cross-silo FL, while we target cross-device FL, they assumed that each data source has an abundance of data to retrain the newly initialized embeddings. Hence, to adapt to the cross-device FL setting, we did not reinitialize the embeddings; each client fine-tunes their own embeddings starting from the pretrained mBERT embeddings. In other words, for each FL round, each client does full fine-tuning, sending weights of all layers except their own embeddings back to the server for aggregation. As with FedDPA-T, the learning rate is set to $5e^{-5}$.

\noindent\textbf{Off-the-shelf.} In our Fed-Aya setup, we use an off-the-shelf instruction finetuned Llama-3.2-3B-Instruct as our \basemodel{}.

\subsection{Baselines and \method{}}

To avoid an exponentially big search space, all hyperparameter tuning is done using simple grid search on our XNLI setup, with mBERT, and Fed-Aya setup, with MobileLLaMA as the \basemodel{}. The best hyperparameters found are then used for MasakhaNEWS and Fed-Aya with Llama3 respectively.

\noindent\textbf{LoRA PEFT~\cite{hu2021lora}.}~We search for the learning rate $[1e^{-5},1e^{-4},1e^{-3}]$ and the number of epochs $[1,2,3]$ and find that the learning rate $1e^{-4}$ with $2$ epochs had the best performance on the train set. We fixed $\alpha_{lora}=2r$. To ensure a similar inference budget across baselines, we set the number of epochs to $2$ for all our experiments.

\noindent\textbf{AdaLoRA~\cite{adalora}.}~Similarly to LoRA, we search for the learning rate $[1e^{-5},1e^{-4},1e^{-3},1e^{-2}]$, the time interval between two budget allocations, $\Delta_T$, $[1,10,100]$ and the coefficient of the orthogonal regularization, $\gamma$, $[0.1,0.5]$. Within our search space, we find learning rate$=1e^{-3}$, $\Delta_T=1.0$, and $\gamma=0.1$ to be the best performing one. We run AdaLoRA once per resource budget $r$, setting the initial rank to be $1.5\times$ of $r$, as recommended. We set the initial fine-tuning warmup steps and final fine-tuning steps to be $10\%$ and $30\%$ of the total steps respectively. 

\noindent\textbf{BayesTune-LoRA (Section\ref{sec:personalized_peft}).}~For fair comparison with \method{}, we use the same hyperparameters as \method{}. This baseline, hence, is an ablation study of how much performance collaboratively learning a PSG adds.

\noindent\textbf{FedL2P~\cite{royson2023fedl2p}}~As FedL2P requires a validation set for outer-loop bi-level optimization and federated early stopping, we split the train set of every client $80\%$ train and $20\%$ validation. Following FedL2P, we set the federated early stopping patience to 50 rounds, MLP hidden dimension is set to 100, the inner-loop learning rate to be the same as finetuning, $1e^{-4}$, and the hypergradient hyperparameters, $Q=3, \psi=0.1$ with hypergradient clamping of $[-1,1]$. 
We use Adam for both the inner-loop and outer-loop optimizers and search for the learning rate for the MLP (LRNet) $[1e^{-5},1e^{-4},1e^{-3}]$ and the learnable post-multiplier learning rate $\tilde{\eta}$ $[1e^{-4},1e^{-5},1e^{-6}]$, picking $1e^{-4}$ and $1e^{-6}$ to be the best respectively. Finally, we use $3$ outer-loop steps with an effective outer-loop batch size of $16$.

\noindent\textbf{\method{} (Section~\ref{sec:main_method})}~We set $\alpha_{r\_mul}=2$, our resulting $r_{init}$ is, hence, $32$ since our $r_{\text{max target}}=16$ for all experiments. Following our standard LoRA fine-tuning baseline, we adopt the same learning rate and $\alpha_{lora}$, $1e-4$ and $2r_{init}$ respectively. The learning rate of $\bm{\lambda}$, on the other hand, is searched $[1e^{-1},1e^{-2},1e^{-3},1e^{-4}]$, and we pick $1e^{-2}$ for all experiments.
All $\lambda$ values are initialized to $1e^{-4}$. The MLP hidden dimension is set to $2\times$ the input dimension, which is model dependent. We clamp the output of the MLP as well as $\lambda$ with a minimum value of $1e-4$ in the forward pass during training. Following FedL2P, we use a straight-through estimator~\cite{bengio2013estimating} after clamping to propagate gradients. We initialize the weights of MLP with Xavier initialization~\cite{glorot2010understanding} using the normal distribution with a gain value of $1e-6$ and the bias with a constant $1e-4$. Lastly, we set $\alpha_s=1e^{+2}$ and $\alpha_p=1e^{-2}$ for all experiments.


\section{Additional Results}

This section contains supplementary results and analyses, omitted from the main paper due to space limitations, that complement the presented findings.
Fig.~\ref{fig:xnli_dept_out_r16}-\ref{fig:llama3_fedavg_out_r2} show the language-agnostic rank structures under different budgets ($r=2$ and $r=16$) learnt by \method{} for different setups as mentioned in Section~\ref{sec:analysis}. These plots illustrate the prioritization of layers for LoRA fine-tuning. 

Note that while the rank structure is the same across languages, the strength of personalization, absolute value of $\bm{\lambda}$, differs, as shown in Fig.~\ref{fig:masakha_out} in this Section and Fig.~\ref{fig:xnli_out} in the main paper. These two figures show the difference in $\bm{\lambda}$ across languages as described in Section~\ref{sec:analysis}. To sum up, the smaller the distance between two languages, represented as a block in the figure, the more similar their generated $\bm{\lambda}$ are. The results show that while similar languages sometimes exhibit similar $\bm{\lambda}$ values, unrelated languages also occasionally share similar $\bm{\lambda}$, consistent with findings in the literature that leveraging dissimilar languages can be beneficial.

Lastly, Tables~\ref{tab:xnli_seen} and \ref{tab:xnli_unseen} contain results for our XNLI setup where the \basemodel{} is fine-tuned from the pretrained mBERT with Standard FL using full fine-tuning and is used to complement results and findings in Section~\ref{sec:text_class}. Similarly, Tables~\ref{tab:mobilellama_fedaya_seen} and \ref{tab:mobilellama_fedaya_unseen} complement the results and findings of our Fed-Aya setup described in Section~\ref{sec:ift_gen}.


\begin{figure}[t]
    \small
    \centering
    \includegraphics[width=1.0\columnwidth]{figures/masakha_out_0.5_seen.png}
    % \captionsetup{font=small,labelfont=bf}
    \vspace{-3em}
    \caption{$\bm{\lambda}$ distance among languages in our MasakhaNEWS setup. Each block shows the log-scale normalized average Euclidean distances between all pairs of clients' $\bm{\lambda}$ for two languages (see text). The smaller the distance, the more similar $\bm{\lambda}$ is. }
    \label{fig:masakha_out}
    % \vspace{-2em}
\end{figure}

\begin{table*}[t]
\centering
\caption{Mean±SD Accuracy of each language across 3 different seeds for clients in the \seen{} pool of our XNLI setup. The pretrained model is trained using Standard FL with full fine-tuning and the resulting \basemodel{} is personalized to each client given a baseline approach.}
\label{tab:xnli_seen}
\begin{scriptsize}\resizebox{0.98\textwidth}{!}{

\begin{tabular}{c|l|l|l|l|l|l|l|l|l|l|l|l|l|l|l|l|c}
\toprule
% \textbf{Lora Rank}  

\textbf{$\mathbf{r}$} & \multicolumn{1}{c|}{\textbf{Approach}} & \multicolumn{1}{c|}{\textbf{bg}} & \multicolumn{1}{c|}{\textbf{hi}} & \multicolumn{1}{c|}{\textbf{es}} & \multicolumn{1}{c|}{\textbf{el}} & \multicolumn{1}{c|}{\textbf{vi}} & \multicolumn{1}{c|}{\textbf{tr}} & \multicolumn{1}{c|}{\textbf{de}} & \multicolumn{1}{c|}{\textbf{ur}} & \multicolumn{1}{c|}{\textbf{en}} & \multicolumn{1}{c|}{\textbf{zh}} & \multicolumn{1}{c|}{\textbf{th}} & \multicolumn{1}{c|}{\textbf{sw}} & \multicolumn{1}{c|}{\textbf{ar}} & \multicolumn{1}{c|}{\textbf{fr}} & \multicolumn{1}{c|}{\textbf{ru}} & \textbf{Wins} \\ \midrule
% \multirow{5}{*}{1}  & LoRA                                   & 46.60±0.28                        & 45.20±0.16                        & 50.00±0.28                        & 49.60±0.16                        & 48.73±0.19                       & 47.67±0.19                       & 48.93±0.09                       & 48.33±0.19                       & 51.27±0.09                       & 49.27±0.09                       & 44.60±0.16                        & 43.20±0.33                        & 44.40±0.28                        & 49.53±0.19                       & 47.13±0.09                       & 0             \\ %\cline{2-18} 
%                     & AdaLoRA                              & 45.67±0.09                       & 44.47±0.09                       & 48.80±0.16                        & 49.07±0.09                       & 48.27±0.25                       & 47.07±0.50                       & 48.53±0.25                       & 47.93±0.09                       & 51.00±0.16                        & 48.53±0.09                       & 43.87±0.34                       & 42.80±0.33                        & 44.20±0.28                        & 48.93±0.34                       & 46.20±0.28                        & 0             \\ %\cline{2-18} 
%                     & BayesTune-LoRA                            & 45.67±0.09                       & 44.33±0.09                       & 48.53±0.09                       & 48.80±0.16                        & 47.80±0.00                        & 46.80±0.16                        & 48.33±0.09                       & 47.87±0.19                       & 50.93±0.09                       & 48.07±0.25                       & 43.60±0.16                        & 42.27±0.25                       & 44.00±0.16                        & 48.67±0.09                       & 45.60±0.16                        & 0             \\ %\cline{2-18} 
%                     & FedL2P                               & \textbf{65.47±1.39}              & \textbf{69.47±1.43}              & \textbf{73.53±1.27}              & \textbf{69.33±1.06}              & \textbf{70.13±1.47}              & \textbf{71.00±1.34}               & \textbf{72.00±0.99}               & \textbf{72.13±0.25}              & \textbf{74.73±0.93}              & \textbf{67.40±1.07}               & \textbf{67.80±0.75}               & \textbf{71.93±0.34}              & \textbf{70.80±1.45}               & \textbf{72.67±0.66}              & \textbf{72.20±1.02}               & \textbf{15}   \\ %\cline{2-18} 
%                     & \method{}                                 & 62.93±1.11                       & 63.60±1.4                         & 70.40±1.82                        & 65.80±1.61                        & 67.67±1.05                       & 65.07±2.17                       & 67.67±0.66                       & 67.00±1.42                        & 73.53±0.62                       & 62.33±0.77                       & 62.93±0.50                       & 69.13±0.41                       & 64.93±1.89                       & 67.47±1.25                       & 66.87±1.15                       & 0             \\ \hline
\multirow{5}{*}{2}  & LoRA                                   & 47.47±0.19                       & 45.93±0.34                       & 50.80±0.16                        & 50.80±0.16                        & 50.80±0.28                        & 48.80±0.59                        & 50.07±0.66                       & 49.67±0.47                       & 53.13±0.41                       & 50.00±0.28                        & 45.47±0.41                       & 44.33±0.09                       & 45.33±0.25                       & 51.00±0.16                        & 48.33±0.74                       & 0             \\ %\cline{2-18} 
                    & AdaLoRA                              & 45.73±0.09                       & 44.40±0.00                        & 49.00±0.28                        & 49.13±0.19                       & 48.00±0.16                        & 47.07±0.34                       & 48.27±0.09                       & 47.87±0.09                       & 50.93±0.09                       & 48.20±0.16                        & 43.80±0.16                        & 43.00±0.00                        & 44.20±0.16                        & 48.87±0.09                       & 46.27±0.25                       & 0             \\ %\cline{2-18} 
                    & BayesTune-LoRA                            & 45.67±0.19                       & 44.40±0.00                        & 48.33±0.25                       & 48.80±0.16                        & 48.13±0.19                       & 46.80±0.28                        & 48.13±0.09                       & 47.80±0.00                        & 50.87±0.19                       & 48.00±0.00                        & 43.53±0.25                       & 42.13±0.19                       & 44.13±0.09                       & 48.93±0.09                       & 45.67±0.09                       & 0             \\ %\cline{2-18} 
                    & FedL2P                               & 66.67±0.90                       & 69.47±0.77                       & 74.33±0.93                       & 70.73±1.00                        & 71.27±0.82                       & 71.27±0.57                       & 72.27±1.32                       & 73.20±0.28                        & 75.27±0.81                       & 68.27±1.32                       & 67.93±0.75                       & 73.47±1.31                       & 71.47±0.94                       & 72.80±0.28                        & 73.07±0.25                       & 0             \\ %\cline{2-18} 
                    & \method{}                                 & \textbf{71.73±0.41}              & \textbf{72.33±0.25}              & \textbf{75.40±0.59}               & \textbf{73.73±0.84}              & \textbf{74.80±0.43}               & \textbf{74.73±0.41}              & \textbf{75.00±0.71}               & \textbf{74.33±0.57}              & \textbf{75.47±0.19}              & \textbf{70.53±0.52}              & \textbf{71.33±0.09}              & \textbf{73.73±0.41}              & \textbf{75.60±0.57}               & \textbf{74.60±0.59}               & \textbf{74.13±0.68}              & \textbf{15}   \\ \hline
\multirow{5}{*}{4}  & LoRA                                   & 49.40±0.16                        & 50.00±0.86                        & 54.47±0.62                       & 53.33±0.41                       & 53.00±0.28                        & 51.53±0.66                       & 53.27±0.62                       & 52.93±1.09                       & 56.27±0.47                       & 52.20±0.43                        & 49.07±0.62                       & 48.80±0.00                        & 49.20±0.33                        & 54.27±0.34                       & 51.73±1.11                       & 0             \\ %\cline{2-18} 
                    & AdaLoRA                              & 45.87±0.19                       & 44.33±0.09                       & 48.60±0.16                        & 48.93±0.09                       & 48.27±0.34                       & 46.87±0.09                       & 48.47±0.09                       & 47.80±0.00                        & 51.27±0.09                       & 48.07±0.09                       & 43.67±0.09                       & 42.73±0.25                       & 44.20±0.28                        & 48.73±0.09                       & 45.93±0.19                       & 0             \\ %\cline{2-18} 
                    & BayesTune-LoRA                            & 45.60±0.00                        & 44.33±0.09                       & 48.87±0.19                       & 49.13±0.09                       & 48.00±0.28                        & 47.00±0.33                        & 48.47±0.09                       & 47.80±0.16                        & 51.13±0.09                       & 48.13±0.09                       & 43.73±0.19                       & 42.40±0.16                        & 44.20±0.00                        & 48.93±0.09                       & 46.00±0.16                        & 0             \\ %\cline{2-18} 
                    & FedL2P                               & 67.40±1.42                        & 69.73±1.37                       & 74.47±0.34                       & 70.20±0.85                        & 71.53±0.94                       & 71.27±0.90                       & 72.73±0.94                       & 73.07±0.34                       & 75.27±0.47                       & 68.73±1.16                       & 68.27±0.47                       & 73.73±0.66                       & 71.47±1.64                       & 73.47±0.50                       & 72.87±0.52                       & 0             \\ %\cline{2-18} 
                    & \method{}                                 & \textbf{72.80±0.33}               & \textbf{72.13±0.41}              & \textbf{75.40±0.28}               & \textbf{74.27±0.34}              & \textbf{74.93±0.09}              & \textbf{75.20±0.28}               & \textbf{75.80±0.16}               & \textbf{75.07±0.25}              & \textbf{75.93±0.25}              & \textbf{71.60±0.16}               & \textbf{70.80±0.28}               & \textbf{73.07±0.25}              & \textbf{75.53±0.62}              & \textbf{75.87±0.34}              & \textbf{74.87±0.19}              & \textbf{15}   \\ \hline
\multirow{5}{*}{8}  & LoRA                                   & 55.33±1.60                       & 54.13±0.09                       & 59.93±0.50                       & 58.20±1.28                        & 58.07±0.25                       & 56.53±0.25                       & 59.73±0.09                       & 58.47±1.33                       & 63.40±0.71                        & 57.13±0.34                       & 56.13±0.19                       & 56.13±0.50                       & 55.93±0.09                       & 59.20±0.33                        & 58.20±2.01                        & 0             \\ %\cline{2-18} 
                    & AdaLoRA                              & 45.80±0.00                        & 44.40±0.00                        & 48.47±0.09                       & 49.00±0.16                        & 48.13±0.09                       & 46.93±0.09                       & 48.27±0.09                       & 48.00±0.16                        & 50.80±0.16                        & 48.07±0.09                       & 43.73±0.09                       & 42.40±0.16                        & 44.20±0.16                        & 48.93±0.09                       & 45.93±0.09                       & 0             \\ %\cline{2-18} 
                    & BayesTune-LoRA                            & 45.87±0.25                       & 44.33±0.09                       & 49.13±0.47                       & 49.27±0.09                       & 48.20±0.00                        & 47.07±0.09                       & 48.60±0.16                        & 47.80±0.00                        & 51.00±0.28                        & 48.60±0.16                        & 44.13±0.09                       & 42.93±0.09                       & 44.40±0.00                        & 49.20±0.43                        & 46.20±0.16                        & 0             \\ %\cline{2-18} 
                    & FedL2P                               & 64.73±1.27                       & 66.07±2.19                       & 72.13±1.80                       & 67.60±1.42                        & 68.80±1.85                        & 68.47±1.89                       & 69.40±1.99                        & 70.93±2.22                       & 73.47±1.18                       & 65.93±1.86                       & 65.73±1.98                       & 70.73±2.96                       & 67.73±1.93                       & 70.53±1.93                       & 70.47±1.89                       & 0             \\ %\cline{2-18} 
                    & \method{}                                 & \textbf{73.93±0.34}              & \textbf{70.67±0.09}              & \textbf{75.87±0.19}              & \textbf{73.80±0.16}               & \textbf{74.33±0.47}              & \textbf{75.60±0.16}               & \textbf{74.93±0.09}              & \textbf{74.27±0.19}              & \textbf{76.47±0.09}              & \textbf{72.53±0.52}              & \textbf{70.67±0.09}              & \textbf{71.67±0.34}              & \textbf{76.87±0.25}              & \textbf{75.87±0.25}              & \textbf{75.00±0.57}               & \textbf{15}   \\ \hline
\multirow{5}{*}{16} & LoRA                                   & 63.93±1.46                       & 64.20±0.00                        & 70.40±0.16                        & 67.07±1.32                       & 68.53±0.25                       & 66.07±0.66                       & 68.13±0.25                       & 69.67±0.62                       & 72.47±0.90                       & 64.67±1.24                       & 65.47±1.11                       & 69.87±0.90                       & 67.20±0.16                        & 68.73±0.09                       & 68.00±0.85                        & 0             \\ %\cline{2-18} 
                    & AdaLoRA                              & 45.80±0.16                        & 44.40±0.00                        & 48.73±0.25                       & 48.93±0.19                       & 48.07±0.19                       & 47.00±0.33                        & 48.27±0.09                       & 47.73±0.09                       & 50.93±0.19                       & 48.20±0.16                        & 43.47±0.09                       & 42.53±0.09                       & 44.20±0.00                        & 48.93±0.09                       & 45.80±0.00                        & 0             \\ %\cline{2-18} 
                    & BayesTune-LoRA                            & 46.27±0.09                       & 44.93±0.19                       & 49.67±0.25                       & 49.60±0.00                        & 48.60±0.16                        & 47.47±0.19                       & 49.00±0.00                        & 48.00±0.00                        & 51.47±0.19                       & 49.00±0.16                        & 44.67±0.09                       & 43.07±0.25                       & 44.33±0.25                       & 49.13±0.52                       & 46.67±0.34                       & 0             \\ %\cline{2-18} 
                    & FedL2P                               & 61.53±2.88                       & 62.33±3.44                       & 68.73±3.60                       & 65.07±1.88                       & 65.13±3.10                        & 65.07±3.20                        & 65.80±3.59                        & 67.73±2.87                       & 71.33±2.13                       & 62.73±2.78                       & 62.27±3.27                       & 67.20±3.92                        & 63.93±3.77                       & 68.07±2.88                       & 67.20±3.43                        & 0             \\ %\cline{2-18} 
                    & \method{}                                 & \textbf{73.87±0.09}              & \textbf{70.80±0.16}               & \textbf{75.87±0.09}              & \textbf{73.73±0.68}              & \textbf{74.93±0.25}              & \textbf{75.33±0.25}              & \textbf{74.40±0.16}               & \textbf{74.13±0.19}              & \textbf{76.53±0.19}              & \textbf{72.40±0.28}               & \textbf{70.80±0.28}               & \textbf{71.13±0.62}              & \textbf{76.20±0.43}               & \textbf{75.87±0.09}              & \textbf{75.13±0.09}              & \textbf{15}   \\ \bottomrule
\end{tabular}
}
\end{scriptsize}
\vspace{-1.5em}
\end{table*}

\begin{table*}[t]
\centering
\caption{Mean±SD Accuracy of each language across 3 different seeds for clients in the \unseen{} pool of our XNLI setup. The pretrained model is trained using Standard FL with full fine-tuning and the resulting \basemodel{} is personalized to each client given a baseline approach.}
\label{tab:xnli_unseen}
\begin{scriptsize}\resizebox{0.98\textwidth}{!}{

\begin{tabular}{c|l|l|l|l|l|l|l|l|l|l|l|l|l|l|l|l|c}
\toprule
% \textbf{Lora Rank}  
\textbf{$\mathbf{r}$} & \multicolumn{1}{c|}{\textbf{Approach}} & \multicolumn{1}{c|}{\textbf{bg}} & \multicolumn{1}{c|}{\textbf{hi}} & \multicolumn{1}{c|}{\textbf{es}} & \multicolumn{1}{c|}{\textbf{el}} & \multicolumn{1}{c|}{\textbf{vi}} & \multicolumn{1}{c|}{\textbf{tr}} & \multicolumn{1}{c|}{\textbf{de}} & \multicolumn{1}{c|}{\textbf{ur}} & \multicolumn{1}{c|}{\textbf{en}} & \multicolumn{1}{c|}{\textbf{zh}} & \multicolumn{1}{c|}{\textbf{th}} & \multicolumn{1}{c|}{\textbf{sw}} & \multicolumn{1}{c|}{\textbf{ar}} & \multicolumn{1}{c|}{\textbf{fr}} & \multicolumn{1}{c|}{\textbf{ru}} & \textbf{Wins} \\ \hline
% \multirow{5}{*}{1}  & LoRA                                   & 49.27±0.19                       & 45.47±0.09                       & 50.33±0.09                       & 48.20±0.00                        & 47.60±0.00                        & 44.87±0.09                       & 49.47±0.25                       & 46.60±0.00                        & 55.33±0.19                       & 49.60±0.43                        & 44.53±0.09                       & 40.27±0.09                       & 46.33±0.09                       & 48.53±0.25                       & 44.80±0.16                        & 0             \\ %\cline{2-18} 
%                     & AdaLoRA                              & 49.20±0.00                        & 45.40±0.16                        & 50.33±0.09                       & 48.27±0.09                       & 47.13±0.09                       & 44.67±0.25                       & 49.33±0.09                       & 46.53±0.09                       & 54.80±0.28                        & 48.53±0.25                       & 44.87±0.09                       & 40.07±0.09                       & 46.67±0.25                       & 48.80±0.00                        & 44.47±0.09                       & 0             \\ %\cline{2-18} 
%                     & BayesTune-LoRA                            & 49.00±0.00                        & 45.20±0.00                        & 50.07±0.09                       & 47.93±0.09                       & 47.07±0.09                       & 44.53±0.09                       & 49.27±0.25                       & 46.20±0.00                        & 54.60±0.16                        & 48.53±0.34                       & 44.73±0.09                       & 39.93±0.09                       & 46.73±0.09                       & 48.80±0.00                        & 44.53±0.19                       & 0             \\ %\cline{2-18} 
%                     & FedL2P                               & \textbf{54.67±0.68}              & \textbf{50.80±0.16}               & \textbf{56.40±0.57}               & \textbf{50.40±0.59}               & \textbf{53.73±0.62}              & 48.00±0.16                        & \textbf{53.73±0.19}              & \textbf{49.47±0.66}              & \textbf{59.87±0.25}              & \textbf{52.00±0.59}               & \textbf{47.93±0.41}              & 43.73±0.50                       & 50.33±0.57                       & 53.07±0.41                       & 48.47±0.25                       & \textbf{10}   \\ %\cline{2-18} 
%                     & \method{}                                 & 52.93±0.25                       & 48.80±1.23                        & 55.73±0.77                       & 49.93±0.19                       & 52.80±0.71                        & \textbf{48.47±0.47}              & 52.60±0.49                        & 48.93±0.25                       & 57.80±0.49                        & 50.80±0.16                        & 47.27±0.52                       & \textbf{44.07±0.50}              & \textbf{49.07±0.47}              & \textbf{53.67±0.25}              & \textbf{48.60±0.16}               & 5             \\ \hline
\multirow{5}{*}{2}  & LoRA                                   & 49.33±0.09                       & 45.93±0.25                       & 51.13±0.09                       & 48.53±0.25                       & 48.53±0.25                       & 45.27±0.25                       & 49.80±0.16                        & 46.87±0.34                       & 55.40±0.28                        & 49.33±0.25                       & 44.40±0.00                        & 41.20±0.16                        & 46.87±0.09                       & 48.33±0.09                       & 45.13±0.09                       & 0             \\ %\cline{2-18} 
                    & AdaLoRA                              & 49.33±0.09                       & 45.27±0.09                       & 50.27±0.19                       & 47.80±0.16                        & 47.27±0.09                       & 44.60±0.00                        & 49.33±0.34                       & 46.60±0.16                        & 54.87±0.25                       & 48.73±0.19                       & 44.73±0.09                       & 40.07±0.09                       & 46.67±0.09                       & 48.67±0.09                       & 44.53±0.09                       & 0             \\ %\cline{2-18} 
                    & BayesTune-LoRA                            & 49.13±0.09                       & 45.13±0.19                       & 50.20±0.00                        & 48.00±0.16                        & 47.07±0.19                       & 44.67±0.09                       & 49.07±0.25                       & 46.47±0.19                       & 54.67±0.09                       & 48.73±0.19                       & 44.80±0.16                        & 40.07±0.09                       & 46.60±0.16                        & 48.73±0.09                       & 44.27±0.09                       & 0             \\ %\cline{2-18} 
                    & FedL2P                               & \textbf{54.40±0.43}               & 49.93±0.25                       & 56.80±0.33                        & \textbf{51.40±0.33}               & 54.00±0.71                        & 48.27±0.50                       & \textbf{54.00±0.59}               & 50.13±0.41                       & \textbf{59.93±0.41}              & \textbf{52.47±0.25}              & 48.00±0.28                        & 44.53±0.34                       & \textbf{50.33±0.41}              & 53.00±0.28                        & 49.13±0.74                       & 6             \\ %\cline{2-18} 
                    & \method{}                                 & 54.00±0.59                        & \textbf{51.13±0.34}              & \textbf{57.67±0.09}              & 51.13±0.68                       & \textbf{54.53±0.25}              & \textbf{49.73±0.09}              & 53.33±0.66                       & \textbf{53.13±0.98}              & 59.60±0.33                        & 51.53±0.38                       & \textbf{48.87±0.19}              & \textbf{46.53±0.66}              & 49.00±0.43                        & \textbf{56.33±0.68}              & \textbf{52.47±0.41}              & \textbf{9}    \\ \hline
\multirow{5}{*}{4}  & LoRA                                   & 49.87±0.09                       & 47.07±0.09                       & 52.27±0.34                       & 49.67±0.19                       & 49.87±0.25                       & 46.47±0.34                       & 50.33±0.09                       & 48.20±0.16                        & 56.53±0.62                       & 49.80±0.16                        & 45.00±0.43                        & 41.40±0.00                        & 47.93±0.09                       & 48.73±0.09                       & 45.80±0.16                        & 0             \\ %\cline{2-18} 
                    & AdaLoRA                              & 49.27±0.09                       & 45.20±0.16                        & 50.07±0.25                       & 47.93±0.25                       & 47.27±0.09                       & 44.60±0.00                        & 49.27±0.19                       & 46.53±0.09                       & 54.73±0.25                       & 48.60±0.16                        & 44.60±0.16                        & 40.07±0.19                       & 46.47±0.09                       & 48.80±0.00                        & 44.53±0.34                       & 0             \\ %\cline{2-18} 
                    & BayesTune-LoRA                            & 49.33±0.19                       & 45.13±0.09                       & 50.00±0.16                        & 48.20±0.16                        & 47.13±0.09                       & 44.67±0.09                       & 49.40±0.00                        & 46.40±0.16                        & 54.93±0.09                       & 48.87±0.09                       & 44.60±0.00                        & 40.13±0.09                       & 46.60±0.00                        & 48.53±0.09                       & 44.40±0.00                        & 0             \\ %\cline{2-18} 
                    & FedL2P                               & 54.73±0.38                       & \textbf{50.93±0.66}              & 57.27±0.34                       & \textbf{51.47±0.34}              & \textbf{54.67±1.09}              & 47.67±0.25                       & \textbf{54.20±0.43}               & 50.93±0.25                       & \textbf{59.80±0.59}               & \textbf{52.73±0.09}              & 47.87±0.50                       & 44.80±0.00                        & 50.20±0.28                        & 53.47±0.50                       & 49.53±0.93                       & 6             \\ %\cline{2-18} 
                    & \method{}                                 & \textbf{55.33±0.34}              & 50.60±0.91                        & \textbf{58.20±0.16}               & 51.33±0.41                       & 54.60±0.28                        & \textbf{48.80±0.28}               & 52.73±0.57                       & \textbf{53.47±0.47}              & 59.13±0.68                       & 51.53±0.34                       & \textbf{48.47±0.41}              & \textbf{47.20±0.59}               & \textbf{49.00±0.16}               & \textbf{56.73±0.66}              & \textbf{52.13±0.41}              & \textbf{9}    \\ \hline
\multirow{5}{*}{8}  & LoRA                                   & 51.00±0.33                        & 48.13±0.50                       & 54.00±0.33                        & 50.87±0.34                       & 51.20±0.28                        & 46.93±0.25                       & 52.13±1.05                       & 48.27±0.25                       & 58.33±0.19                       & 50.13±0.19                       & 45.80±0.16                        & 42.40±0.00                        & 48.13±1.05                       & 50.93±0.09                       & 46.80±0.59                        & 0             \\ %\cline{2-18} 
                    & AdaLoRA                              & 49.20±0.16                        & 45.40±0.16                        & 50.13±0.19                       & 48.13±0.09                       & 47.07±0.09                       & 44.73±0.09                       & 49.13±0.09                       & 46.33±0.09                       & 54.93±0.19                       & 48.47±0.34                       & 44.60±0.00                        & 40.13±0.09                       & 46.67±0.09                       & 48.67±0.19                       & 44.33±0.25                       & 0             \\ %\cline{2-18} 
                    & BayesTune-LoRA                            & 49.27±0.09                       & 45.07±0.19                       & 50.27±0.09                       & 47.93±0.09                       & 47.20±0.16                        & 44.73±0.09                       & 49.27±0.19                       & 46.33±0.19                       & 54.93±0.09                       & 48.53±0.09                       & 44.53±0.09                       & 40.33±0.09                       & 46.67±0.09                       & 48.53±0.09                       & 44.53±0.38                       & 0             \\ %\cline{2-18} 
                    & FedL2P                               & \textbf{54.13±0.25}              & 49.60±1.14                        & 57.47±0.82                       & 50.87±0.50                       & 53.20±0.28                        & 47.60±0.43                        & \textbf{54.20±0.16}               & 49.87±0.09                       & \textbf{59.53±0.62}              & \textbf{52.27±0.90}              & 47.47±0.25                       & 43.60±0.71                        & \textbf{50.27±0.52}              & 52.60±0.57                        & 48.80±0.57                        & 5             \\ %\cline{2-18} 
                    & \method{}                                 & 53.40±0.00                        & \textbf{50.60±0.33}               & \textbf{57.67±0.62}              & 50.53±0.34                       & \textbf{55.13±0.38}              & \textbf{48.00±0.75}               & 52.53±0.34                       & \textbf{51.87±0.19}              & 57.00±1.02                        & 50.67±0.09                       & \textbf{47.60±0.00}               & \textbf{46.67±0.52}              & 48.47±0.66                       & \textbf{55.00±0.33}               & \textbf{50.73±0.75}              & \textbf{9}    \\ \hline
\multirow{5}{*}{16} & LoRA                                   & 52.73±0.25                       & 48.80±0.00                        & 55.47±0.25                       & \textbf{51.13±1.05}              & 53.53±0.41                       & \textbf{48.87±0.75}              & 54.40±0.91                        & 49.40±0.99                        & 58.87±1.11                       & \textbf{51.80±0.49}               & 46.93±0.09                       & 43.60±0.99                        & 49.13±1.05                       & 52.13±0.81                       & 48.93±2.03                       & 3             \\ %\cline{2-18} 
                    & AdaLoRA                              & 49.13±0.09                       & 45.20±0.00                        & 50.20±0.16                        & 47.93±0.25                       & 46.93±0.19                       & 44.53±0.09                       & 49.33±0.19                       & 46.53±0.09                       & 54.80±0.16                        & 48.73±0.09                       & 44.73±0.19                       & 40.13±0.09                       & 46.73±0.25                       & 48.67±0.09                       & 44.40±0.28                        & 0             \\ %\cline{2-18} 
                    & BayesTune-LoRA                            & 49.33±0.25                       & 45.33±0.19                       & 50.20±0.00                        & 47.93±0.09                       & 47.53±0.25                       & 44.80±0.16                        & 49.40±0.33                        & 46.53±0.09                       & 55.27±0.09                       & 48.87±0.34                       & 44.47±0.09                       & 40.20±0.28                        & 46.53±0.09                       & 48.73±0.25                       & 44.73±0.25                       & 0             \\ %\cline{2-18} 
                    & FedL2P                               & 52.53±1.16                       & 49.00±0.28                        & 56.40±1.40                        & 50.47±0.96                       & 52.47±0.68                       & 47.80±0.28                        & \textbf{53.93±0.38}              & 48.80±0.85                        & \textbf{59.07±0.41}              & 51.73±0.75                       & 47.13±0.82                       & 43.00±1.14                        & \textbf{49.27±0.84}              & 52.07±1.16                       & 48.33±0.74                       & 3             \\ %\cline{2-18} 
                    & \method{}                                 & \textbf{53.73±0.52}              & \textbf{49.87±0.50}              & \textbf{57.07±0.52}              & 49.87±0.25                       & \textbf{54.67±0.90}              & 47.67±0.41                       & 52.27±0.38                       & \textbf{52.27±0.52}              & 57.87±0.68                       & 50.33±0.66                       & \textbf{47.60±0.33}               & \textbf{46.20±0.16}               & 49.00±0.49                        & \textbf{54.60±0.71}               & \textbf{49.67±0.57}              & \textbf{9}    \\ \bottomrule
\end{tabular}
}
\end{scriptsize}
\vspace{-1.5em}
\end{table*}

\begin{table*}[t]
\npdecimalsign{.}
\nprounddigits{2}
\caption{Average METEOR/ROUGE-1/ROUGE-L of each language for \seen{} clients of our Fed-Aya setup. The pretrained MobileLLaMA-1.4B model is trained using Standard FL with LoRA following FedLLM-Bench~\cite{fedllm-bench} and the resulting \basemodel{} is personalized to each client given a baseline approach.}
\vspace{0.5em}
\label{tab:mobilellama_fedaya_seen}
\begin{scriptsize}\resizebox{0.98\textwidth}{!}{
\begin{tabular}{c|l|l|l|l|l|l|l|l|c}
\toprule
% \textbf{Lora}\\\textbf{Rank}  

\textbf{$\mathbf{r}$}& \multicolumn{1}{c|}{\textbf{Approach}} & \multicolumn{1}{c|}{\textbf{te}} & \multicolumn{1}{c|}{\textbf{ar}} & \multicolumn{1}{c|}{\textbf{es}} & \multicolumn{1}{c|}{\textbf{en}} & \multicolumn{1}{c|}{\textbf{fr}} & \multicolumn{1}{c|}{\textbf{zh}} & \multicolumn{1}{c|}{\textbf{pt}} 
& \textbf{Wins} \\ \midrule
% \multirow{5}{*}{1}  & LoRA                                   & 0.125/0.0637/0.0613              & 0.1723/0.0208/0.0203             & 0.3159/0.3469/0.3203             & 0.2458/0.3066/0.2478             & 0.1878/0.2455/0.1952             & \textbf{0.0649/0.084/0.084}      & 0.2527/0.3116/0.2792                                         & 0             \\ % \cline{2-11} 
%                     & AdaLoRA                              & 0.1218/0.0557/0.0537             & 0.1899/0.0173/0.0166             & 0.3241/0.3478/0.3218             & 0.2434/0.3058/0.2452             & 0.2048/0.2445/0.1988             & 0.0633/0.0841/0.0841             & 0.2692/0.329/0.2959                                          & 0             \\ % \cline{2-11} 
%                     & BayesTune-LoRA                            & 0.12/0.0557/0.0541               & 0.1582/0.0226/0.0223             & 0.2863/0.3269/0.2991             & 0.2326/0.2848/0.2319             & 0.1742/0.2298/0.184              & 0.06/0.0723/0.0723               & 0.2289/0.2837/0.2526                                         & 1             \\ % \cline{2-11} 
%                     & FedL2P                               & \textbf{0.1396/0.0769/0.0746}    & \textbf{0.2109/0.0207/0.0197}    & 0.3218/0.3527/0.3252             & \textbf{0.2692/0.3293/0.2697}    & \textbf{0.2365/0.2652/0.2105}    & 0.0527/0.0866/0.0865             & \textbf{0.2797/0.3228/0.2916}                                & 2             \\ % \cline{2-11} 
%                     & \method{}                                 & 0.126/0.0627/0.0603              & 0.2039/0.0221/0.0216             & \textbf{0.3335/0.3624/0.3346}    & 0.2645/0.3369/0.2748             & 0.2181/0.2667/0.2145             & 0.0616/0.0842/0.0842             & 0.2792/0.3373/0.3035                                         & \textbf{4}    \\ \hline
\multirow{5}{*}{2}  & LoRA                                   & 0.1207/0.0545/0.0524             & 0.1835/0.0210/0.0205              & 0.3228/0.3519/0.3251             & 0.2457/0.3121/0.2525             & 0.2049/0.2484/0.2002             & 0.0616/\textbf{0.0874/0.0873}             & 0.2631/0.3266/0.2922                                         & 1             \\ % \cline{2-11} 
                    & AdaLoRA                              & 0.1238/0.0607/0.0586             & 0.1819/0.0223/0.0218             & 0.3288/0.3573/0.3315             & 0.2459/0.3172/0.2524             & 0.1963/0.2422/0.1915             & \textbf{0.0689}/0.0832/0.0832    & 0.2745/0.3327/0.2986                                         & 0             \\ % \cline{2-11} 
                    & BayesTune-LoRA                            & 0.1245/0.0605/0.0580              & 0.1813/0.0181/0.0178             & 0.2941/0.3317/0.3063             & 0.2345/0.2892/0.2367             & 0.1885/0.2430/0.1970               & 0.0643/0.0805/0.0805             & 0.2408/0.2971/0.2645                                         & 0             \\ % \cline{2-11} 
                    & FedL2P                               & \textbf{0.1451/0.0747/0.0725}    & 0.2017/0.0219/0.0213             & 0.3321/0.3523/0.3245             & 0.2635/0.3307/0.2692             & 0.2298/0.2467/0.2034             & 0.0544/0.0803/0.0803             & 0.2780/0.3133/0.2832                                          & 1             \\ % \cline{2-11} 
                    & \method{}                                 & 0.1266/0.0629/0.0606             & \textbf{0.2081/0.0254/0.0248}    & \textbf{0.3425/0.3663/0.3376}    & \textbf{0.2745/0.3469/0.2831}    & \textbf{0.2342/0.2766/0.2246}    & 0.0524/0.0777/0.0777             & \textbf{0.2846/0.3403/0.3066}                                & \textbf{5}    \\ \hline
\multirow{5}{*}{4}  & LoRA                                   & 0.1232/0.0570/0.0537              & 0.1861/0.0202/0.0197             & 0.3284/0.3555/0.3291             & 0.2541/0.3202/0.2585             & 0.2037/0.2475/0.1990              & 0.0559/\textbf{0.0886/0.0886}             & 0.2734/0.3314/0.2975                                         & 1             \\ % \cline{2-11} 
                    & AdaLoRA                              & 0.1240/0.0596/0.0574              & 0.1858/0.0180/0.0177              & 0.3310/0.3548/0.3287              & 0.2448/0.3111/0.2500               & 0.1892/0.2331/0.1859             & 0.0617/0.0852/0.0851             & 0.2640/0.3250/0.2934                                           & 0             \\ % \cline{2-11} 
                    & BayesTune-LoRA                            & 0.1214/0.0548/0.0532             & 0.1912/0.0201/0.0195             & 0.3042/0.3405/0.3150              & 0.2405/0.3022/0.2440              & 0.1973/0.2393/0.1925             & \textbf{0.0670}/0.0806/0.0806     & 0.2468/0.3057/0.2737                                         & 0             \\ % \cline{2-11} 
                    & FedL2P                               & 0.1258/0.0616/0.0592             & 0.1805/0.0217/0.0208             & 0.3260/0.3568/0.3298              & 0.2519/0.3108/0.2498             & 0.2037/0.2493/0.2027             & 0.0484/0.0798/0.0798             & 0.2626/0.3174/0.2836                                         & 0             \\ % \cline{2-11} 
                    & \method{}                                 & \textbf{0.1350/0.0722/0.0692}     & \textbf{0.2218/0.0275/0.0269}    & \textbf{0.3448/0.3712/0.3419}    & \textbf{0.2796/0.3516/0.2883}    & \textbf{0.2469/0.2753/0.2244}    & 0.0554/0.0843/0.0843             & \textbf{0.2911/0.3397/0.3060}                                 & \textbf{6}    \\ \hline
\multirow{5}{*}{8}  & LoRA                                   & 0.1241/0.0522/0.0493             & 0.2059/0.0189/0.0184             & 0.3435/0.3688/0.3427             & 0.2686/0.3400/0.2771               & 0.2197/0.2556/0.2037             & 0.0583/\textbf{0.0886}/0.0884             & 0.2822/0.3418/0.3076                                         & 0             \\ % \cline{2-11} 
                    & AdaLoRA                              & 0.1245/0.0623/0.0599             & 0.1771/0.0189/0.0184             & 0.3215/0.3485/0.3225             & 0.2461/0.3101/0.2512             & 0.1799/0.2287/0.1839             & \textbf{0.0613}/0.0806/0.0806    & 0.2607/0.3194/0.2879                                         & 0             \\ % \cline{2-11} 
                    & BayesTune-LoRA                            & 0.1246/0.0591/0.0572             & 0.2046/0.0192/0.0189             & 0.3247/0.3520/0.3284              & 0.2430/0.3095/0.2496              & 0.2132/0.2542/0.2059             & 0.0601/0.0818/\textbf{0.0818}             & 0.2588/0.3171/0.2871                                         & 0             \\ % \cline{2-11} 
                    & FedL2P                               & 0.1316/\textbf{0.0687/0.0661}             & 0.1855/0.0218/0.0215             & 0.3272/0.3535/0.3280              & 0.2696/0.3304/0.2711             & 0.2109/0.2558/0.2059             & 0.0510/0.0816/0.0816              & 0.2750/0.3252/0.2897                                          & 1             \\ % \cline{2-11} 
                    & \method{}                                 & \textbf{0.1327}/0.0662/0.0644    & \textbf{0.2304/0.0253/0.0246}    & \textbf{0.3474/0.3847/0.3531}    & \textbf{0.2941/0.3656/0.2996}    & \textbf{0.2553/0.2829/0.2268}    & 0.0538/0.0814/0.0814             & \textbf{0.2945/0.3452/0.3108}                                & \textbf{5}    \\ \hline
\multirow{5}{*}{16} & LoRA                                   & 0.1217/0.0562/0.0536             & 0.2080/0.0221/0.0218              & 0.3387/0.3616/0.3352             & 0.2757/0.3431/0.2807             & \textbf{0.2497/0.2880/0.2337}     & 0.0553/\textbf{0.0844/0.0844}             & 0.2902/0.3449/0.3106                                         & 2             \\ % \cline{2-11} 
                    & AdaLoRA                              & 0.1251/0.0624/0.0602             & 0.1676/0.0198/0.0192             & 0.3048/0.3329/0.3037             & 0.2391/0.2985/0.2416             & 0.1821/0.2309/0.1866             & \textbf{0.0575}/0.0815/0.0815    & 0.2530/0.3099/0.2784                                          & 0             \\ % \cline{2-11} 
                    & BayesTune-LoRA                            & 0.1374/0.0745/0.0720              & 0.2119/0.0175/0.0172             & 0.3358/0.3649/0.3397             & 0.2587/0.3189/0.2577             & 0.2222/0.2603/0.2113             & 0.0520/0.0824/0.0824              & 0.2862/0.3450/0.3109                                          & 0             \\ % \cline{2-11} 
                    & FedL2P                               & \textbf{0.1559/0.0827/0.0799}    & 0.2013/0.0228/0.0226             & 0.3278/0.3541/0.3268             & 0.2772/0.3278/0.2693             & 0.2306/0.2346/0.1925             & 0.0506/0.0838/0.0838             & 0.2817/0.3179/0.2851                                         & 1             \\ % \cline{2-11} 
                    & \method{}                                 & 0.1309/0.0663/0.0638             & \textbf{0.2359/0.0258/0.0252}    & \textbf{0.3447/0.3778/0.3463}    & \textbf{0.2802/0.3485/0.2858}    & 0.2473/0.2775/0.2208             & 0.0538/0.0835/0.0835             & \textbf{0.2975/0.3491/0.3157}                                & \textbf{4}    \\ \bottomrule
\end{tabular}

}
\npnoround
\end{scriptsize}
\vspace{-1.5em}
\end{table*}


\begin{table*}[t]
\caption{Average METEOR/ROUGE-1/ROUGE-L of each language for \seen{} clients of our Fed-Aya setup. The pretrained MobileLLaMA-1.4B model is trained using Standard FL with LoRA following FedLLM-Bench~\cite{fedllm-bench} and the resulting \basemodel{} is personalized to each client given a baseline approach.}
\vspace{0.5em}
\label{tab:mobilellama_fedaya_unseen}
\begin{scriptsize}\resizebox{0.98\textwidth}{!}{
\begin{tabular}{c|l|l|l|l|l|l|l|l|l|c}
\toprule
\textbf{$\mathbf{r}$}  & \multicolumn{1}{c|}{\textbf{Approach}} & \multicolumn{1}{c|}{\textbf{te}} & \multicolumn{1}{c|}{\textbf{ar}} & \multicolumn{1}{c|}{\textbf{es}} & \multicolumn{1}{c|}{\textbf{en}} & \multicolumn{1}{c|}{\textbf{fr}} & \multicolumn{1}{c|}{\textbf{zh}} & \multicolumn{1}{c|}{\textbf{pt}} & \multicolumn{1}{c|}{\textbf{ru}} & \textbf{Wins} \\ \midrule
% \multirow{5}{*}{1}  & LoRA                                   & 0.0344/0.0000/0.0000                   & 0.1004/0.0564/0.0539             & 0.3329/0.4252/0.3809             & 0.1987/0.2339/0.1988             & 0.0505/0.0000/0.0000                   & 0.1457/0.0041/0.0034             & 0.1714/0.1907/0.1792             & \textbf{0.1174/0.0556/0.0556}    & 1             \\ % \cline{2-11} 
%                     & AdaLoRA                              & 0.0342/0.0042/0.0042             & 0.0941/0.0581/0.0556             & 0.3607/0.4708/0.4256             & 0.196/0.232/0.2011               & 0.049/0.0000/0.0000                    & \textbf{0.154/0.0092/0.0092}     & 0.1804/0.2222/0.2102             & 0.1075/0.0556/0.0556             & 2             \\ % \cline{2-11} 
%                     & BayesTune-LoRA                            & 0.0202/0.0000/0.0000                   & 0.0741/0.0461/0.0461             & 0.3357/0.3817/0.349              & 0.1912/0.2252/0.1858             & 0.0521/0.0000/0.0000                   & 0.1156/0.002/0.002               & 0.1592/0.164/0.151               & 0.0963/0.0556/0.0556             & 0             \\ % \cline{2-11} 
%                     & FedL2P                               & \textbf{0.1107/0.0075/0.0075}    & \textbf{0.1695/0.0391/0.0391}    & 0.3691/0.4312/0.4055             & 0.2432/0.2668/0.2291             & \textbf{0.0595/0.019/0.019}      & 0.1016/0.0101/0.0101             & \textbf{0.2085/0.2337/0.2237}    & 0.0746/0.0556/0.0556             & \textbf{3}    \\ % \cline{2-11} 
%                     & \method{}                                 & 0.1057/0.0042/0.0042             & 0.1191/0.0587/0.0587             & \textbf{0.3718/0.4349/0.4045}    & \textbf{0.2993/0.3342/0.297}     & 0.049/0.0000/0.0000                    & 0.1153/0.0049/0.0049             & 0.1928/0.2301/0.2169             & 0.0912/0.0556/0.0556             & 2             \\ \hline
\multirow{5}{*}{2}  & LoRA                                   & 0.0531/0.0042/0.0042             & 0.1032/\textbf{0.0524/0.0524}            & 0.3547/0.4490/0.4020               & 0.2385/0.2713/0.2356             & 0.0490/0.0000/0.0000                    & 0.1381/0.0088/0.0088             & 0.1865/0.2184/0.2070              & 0.0921/0.0556/0.0556             & 1             \\ % \cline{2-11} 
                    & AdaLoRA                              & 0.0312/0.0000/0.0000                   & 0.1000/0.0520/0.0520                  & 0.3244/0.4250/0.3821              & 0.2165/0.2606/0.2197             & 0.0490/\textbf{0.0513/0.0513}              & \textbf{0.1609}/0.0089/0.0089    & 0.1904/0.2258/0.2129             & \textbf{0.0992}/0.0556/0.0556    & 1             \\ % \cline{2-11} 
                    & BayesTune-LoRA                            & 0.0276/0.0000/0.0000                   & 0.0934/0.0501/0.0476             & 0.3560/0.4084/0.3716              & 0.1786/0.2024/0.1688             & 0.0450/0.0000/0.0000                    & 0.1566/0.0109/0.0092             & 0.1469/0.1628/0.1500               & 0.0873/0.0556/0.0556             & 0             \\ % \cline{2-11} 
                    & FedL2P                               & \textbf{0.1199}/0.0042/0.0042    & \textbf{0.1399}/0.0206/0.0206    & \textbf{0.3923/0.4587}/0.4194    & 0.2688/0.3049/0.2650              & 0.0024/0.0011/0.0011             & 0.0888/\textbf{0.0121/0.0121}             & \textbf{0.2022}/0.2170/0.2063     & 0.0850/0.0556/0.0556              & 2             \\ % \cline{2-11} 
                    & \method{}                                 & 0.1105/\textbf{0.0132/0.0114}             & 0.1127/0.0483/0.0483             & 0.3812/0.4546/\textbf{0.4209}             & \textbf{0.3047/0.3354/0.3006}    & 0.0490/0.0000/0.0000                    & 0.0650/0.0069/0.0069              & 0.1997/\textbf{0.2505/0.2384}             & 0.0956/0.0556/0.0556             & \textbf{3}    \\ \hline
\multirow{5}{*}{4}  & LoRA                                   & 0.0687/0.0075/0.0075             & 0.0989/0.0581/0.0556             & 0.3244/0.4086/0.3749             & 0.2297/0.2726/0.2370              & 0.0490/0.0513/0.0513              & 0.1530/0.0089/0.0089              & 0.1872/0.2267/0.2124             & 0.0974/0.0556/0.0556             & 0             \\ % \cline{2-11} 
                    & AdaLoRA                              & 0.0263/0.0000/0.0000                   & 0.0894/\textbf{0.0634}/0.0610              & 0.3336/0.4322/0.3883             & 0.1762/0.2105/0.1765             & 0.0490/0.0000/0.0000                    & \textbf{0.1630}/0.0066/0.0049     & 0.1832/0.2096/0.1980              & 0.1156/0.0556/0.0556             & 0             \\ % \cline{2-11} 
                    & BayesTune-LoRA                            & 0.0471/0.0075/0.0075             & 0.0938/0.0544/0.0544             & 0.3357/0.4058/0.3629             & 0.1657/0.1964/0.1647             & 0.1155/0.0635/0.0635             & 0.1557/\textbf{0.0115/0.0099}             & 0.1511/0.1636/0.1517             & 0.0858/0.0222/0.0222             & 1             \\ % \cline{2-11} 
                    & FedL2P                               & 0.0569/0.0075/0.0075             & 0.1002/0.0610/\textbf{0.0610}               & 0.3098/0.4066/0.3837             & 0.2494/0.2901/0.2569             & \textbf{0.1157/0.0741/0.0741}    & 0.1585/0.0041/0.0041             & 0.1724/0.1920/0.1806              & 0.0804/0.0556/0.0556             & 1             \\ % \cline{2-11} 
                    & \method{}                                 & \textbf{0.1121/0.0212/0.0212}    & \textbf{0.1537}/0.0333/0.0333    & \textbf{0.3725/0.4622/0.4273}    & \textbf{0.2945/0.3276/0.2821}    & 0.0490/0.0000/0.0000                    & 0.0722/0.0069/0.0069             & \textbf{0.1903/0.2547/0.2438}    & \textbf{0.1258/0.0556/0.0556}    & \textbf{5}    \\ \hline
\multirow{5}{*}{8}  & LoRA                                   & 0.1022/0.0042/0.0042             & 0.1206/0.0549/0.0549             & 0.3401/0.4382/0.3977             & 0.2722/\textbf{0.3127/0.2726}             & 0.0490/0.0000/0.0000                    & 0.1241/0.0089/0.0089             & 0.1879/0.2331/0.2202             & 0.0761/0.0556/0.0556             & 1             \\ % \cline{2-11} 
                    & AdaLoRA                              & 0.0218/0.0000/0.0000                   & 0.0897/0.0364/0.0364             & 0.3383/0.4295/0.3920              & 0.1786/0.2150/0.1795              & 0.1221/\textbf{0.0833/0.0833}             & 0.1470/0.0066/0.0049              & 0.1773/0.1909/0.1770              & 0.1124/0.0556/0.0556             & 1             \\ % \cline{2-11} 
                    & BayesTune-LoRA                            & 0.0529/0.0042/0.0042             & 0.0970/\textbf{0.0557/0.0557}              & 0.3544/0.4257/0.3789             & 0.1891/0.2235/0.1860              & \textbf{0.1262}/0.0784/0.0784    & \textbf{0.1522}/0.0068/0.0068    & 0.1365/0.1603/0.1483             & 0.0895/0.0556/0.0556             & 1             \\ % \cline{2-11} 
                    & FedL2P                               & 0.0934/0.0042/0.0042             & 0.1138/0.0495/0.0495             & 0.3237/0.4366/0.4068             & 0.2438/0.2849/0.2451             & 0.0490/0.0000/0.0000                    & 0.1408/\textbf{0.0139/0.0139}             & 0.1946/0.2280/0.2130               & \textbf{0.1187}/0.0556/0.0556    & 1             \\ % \cline{2-11} 
                    & \method{}                                 & \textbf{0.1099/0.0132/0.0114}    & \textbf{0.1565}/0.0082/0.0082    & \textbf{0.4471/0.5240/0.4876}     & \textbf{0.2798}/0.2851/0.2475    & 0.0490/0.0000/0.0000                    & 0.1071/0.0069/0.0069             & \textbf{0.2160/0.2700/0.2559}       & 0.0862/0.0465/0.0465             & \textbf{3}    \\ \hline
\multirow{5}{*}{16} & LoRA                                   & 0.1193/0.0042/0.0042             & \textbf{0.1418/0.0587/0.0587}    & 0.3647/0.4240/0.3927              & 0.2939/\textbf{0.3164/0.2753}             & 0.0490/0.0000/0.0000                    & 0.0962/0.0089/0.0089             & 0.1916/0.2438/0.2320              & \textbf{0.1403/0.0556/0.0556}    & \textbf{3}    \\ % \cline{2-11} 
                    & AdaLoRA                              & 0.0079/0.0000/0.0000                   & 0.0848/0.0386/0.0361             & 0.3548/0.4216/0.3864             & 0.1814/0.2158/0.1750              & 0.1443/0.1333/0.1333             & 0.1408/\textbf{0.0115/0.0099}             & 0.1828/0.1892/0.1793             & 0.0957/0.0556/0.0556             & 1             \\ % \cline{2-11} 
                    & BayesTune-LoRA                            & 0.0674/0.0075/0.0075             & 0.0887/0.0420/0.0420               & 0.3508/0.4174/0.3788             & 0.2088/0.2442/0.2106             & 0.1681/\textbf{0.1905/0.1905}             & \textbf{0.1594}/0.0089/0.0089    & 0.1877/0.2362/0.2225             & 0.0972/0.0556/0.0556             & 1             \\ % \cline{2-11} 
                    & FedL2P                               & 0.1093/\textbf{0.0673/0.0673}             & 0.1171/0.0320/0.0296              & 0.3895/0.4515/0.4131             & 0.2629/0.2606/0.2150              & 0.1349/0.1026/0.1026             & 0.1335/0.0041/0.0020              & 0.2011/0.2149/0.2027             & 0.0645/0.0253/0.0253             & 1             \\ % \cline{2-11} 
                    & \method{}                                 & \textbf{0.1289}/0.0165/0.0147    & 0.1215/0.0166/0.0166             & \textbf{0.4048/0.4910/0.4520}      & \textbf{0.2968}/0.3065/0.2678    & \textbf{0.2887}/0.1667/0.1667    & 0.0960/0.0069/0.0069              & \textbf{0.2359/0.2863/0.2704}    & 0.0791/0.0222/0.0222             & 2             \\ \bottomrule
\end{tabular}
}
\end{scriptsize}
\vspace{-1.5em}
\end{table*}


\begin{figure*}
\centering
\begin{minipage}{.23\linewidth}
  \includegraphics[width=\linewidth]{figures/xnli_fedavg_out_0.5_seen_en_layerwiserank.png}
  \captionof{figure}{Language agnostic rank structure of mBERT in our XNLI setup where the \basemodel{} is trained with FedIFT full-finetuning ($r=16$).}
  \label{fig:xnli_fedavg_out_r16}
\end{minipage}
\hspace{.01\linewidth}
\begin{minipage}{.23\linewidth}
  \includegraphics[width=\linewidth]{figures/xnli_fedavg_out_0.0625_seen_en_layerwiserank.png}
  \captionof{figure}{Language agnostic rank structure of mBERT in our XNLI setup where the \basemodel{} is trained with FedIFT full-finetuning ($r=2$).}
  \label{fig:xnli_fedavg_out_r2}
\end{minipage}
\hspace{.01\linewidth}
\begin{minipage}{.23\linewidth}
  \includegraphics[width=\linewidth]{figures/xnli_dept_out_0.5_seen_en_layerwiserank.png}
  \captionof{figure}{Language agnostic rank structure of mBERT in our XNLI setup where the \basemodel{} is trained with DEPT(SPEC) ($r=16$).}
  \label{fig:xnli_dept_out_r16}
\end{minipage}
\hspace{.01\linewidth}
\begin{minipage}{.23\linewidth}
  \includegraphics[width=\linewidth]{figures/xnli_dept_out_0.0625_seen_en_layerwiserank.png}
  \captionof{figure}{Language agnostic rank structure of mBERT in our XNLI setup where the \basemodel{} is trained with DEPT(SPEC) ($r=2$).}
  \label{fig:xnli_dept_out_r2}
\end{minipage}
\end{figure*}

\begin{figure*}
\centering
\begin{minipage}{.24\linewidth}
  \includegraphics[width=\linewidth]{figures/masakha_out_0.5_seen_eng_layerwiserank.png}
  \captionof{figure}{Language agnostic rank structure of mBERT in our MasakhaNEWS setup where the \basemodel{} is trained with FedIFT full-finetuning ($r=16$).}
  \label{fig:masakha_fedavg_out_r16}
\end{minipage}
\hspace{.2\linewidth}
\begin{minipage}{.24\linewidth}
  \includegraphics[width=\linewidth]{figures/masakha_out_0.0625_seen_eng_layerwiserank.png}
  \captionof{figure}{Language agnostic rank structure of mBERT in our MasakhaNEWS setup where the \basemodel{} is trained with FedIFT full-finetuning ($r=2$).}
  \label{fig:masakha_fedavg_out_r2}
\end{minipage}
\end{figure*}


\begin{figure*}
\centering
\begin{minipage}{.23\linewidth}
  \includegraphics[width=\linewidth]{figures/mobilellama_out_0.5_seen_en_layerwiserank.png}
  \captionof{figure}{Language agnostic rank structure of MobileLLaMA-1.4B in our Fed-Aya setup where the \basemodel{} is trained with FedIFT LoRA ($r=16$). Zoom in for best results.}
  \label{fig:mobilellama_fedavg_out_r16}
\end{minipage}
\hspace{.01\linewidth}
\begin{minipage}{.23\linewidth}
  \includegraphics[width=\linewidth]{figures/mobilellama_out_0.0625_seen_en_layerwiserank.png}
  \captionof{figure}{Language agnostic rank structure of MobileLLaMA-1.4B in our Fed-Aya setup where the \basemodel{} is trained with FedIFT LoRA ($r=2$). Zoom in for best results.}
  \label{fig:mobilellama_fedavg_out_r2}
\end{minipage}
\hspace{.01\linewidth}
\begin{minipage}{.23\linewidth}
  \includegraphics[width=\linewidth]{figures/llama3_out_0.5_seen_en_layerwiserank.png}
  \captionof{figure}{Language agnostic rank structure of Llama-3.2-3B in our Fed-Aya setup where the \basemodel{} is an off-the-shelf instruction tuned Llama-3.2-3B-Instruct ($r=16$). Zoom in for best results.}
  \label{fig:llama3_fedavg_out_r16}
\end{minipage}
\hspace{.01\linewidth}
\begin{minipage}{.23\linewidth}
  \includegraphics[width=\linewidth]{figures/llama3_out_0.0625_seen_en_layerwiserank.png}
  \captionof{figure}{Language agnostic rank structure of Llama-3.2-3B in our Fed-Aya setup where the \basemodel{} is an off-the-shelf instruction tuned Llama-3.2-3B-Instruct ($r=2$). Zoom in for best results.}
  \label{fig:llama3_fedavg_out_r2}
\end{minipage}
\end{figure*}




% \appendix
% \onecolumn
% \section{You \emph{can} have an appendix here.}

% You can have as much text here as you want. The main body must be at most $8$ pages long.
% For the final version, one more page can be added.
% If you want, you can use an appendix like this one.  

% The $\mathtt{\backslash onecolumn}$ command above can be kept in place if you prefer a one-column appendix, or can be removed if you prefer a two-column appendix.  Apart from this possible change, the style (font size, spacing, margins, page numbering, etc.) should be kept the same as the main body.
%%%%%%%%%%%%%%%%%%%%%%%%%%%%%%%%%%%%%%%%%%%%%%%%%%%%%%%%%%%%%%%%%%%%%%%%%%%%%%%
%%%%%%%%%%%%%%%%%%%%%%%%%%%%%%%%%%%%%%%%%%%%%%%%%%%%%%%%%%%%%%%%%%%%%%%%%%%%%%%


\end{document}


% This document was modified from the file originally made available by
% Pat Langley and Andrea Danyluk for ICML-2K. This version was created
% by Iain Murray in 2018, and modified by Alexandre Bouchard in
% 2019 and 2021 and by Csaba Szepesvari, Gang Niu and Sivan Sabato in 2022.
% Modified again in 2023 and 2024 by Sivan Sabato and Jonathan Scarlett.
% Previous contributors include Dan Roy, Lise Getoor and Tobias
% Scheffer, which was slightly modified from the 2010 version by
% Thorsten Joachims & Johannes Fuernkranz, slightly modified from the
% 2009 version by Kiri Wagstaff and Sam Roweis's 2008 version, which is
% slightly modified from Prasad Tadepalli's 2007 version which is a
% lightly changed version of the previous year's version by Andrew
% Moore, which was in turn edited from those of Kristian Kersting and
% Codrina Lauth. Alex Smola contributed to the algorithmic style files.
