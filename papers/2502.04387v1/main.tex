%%%%%%%% ICML 2025 EXAMPLE LATEX SUBMISSION FILE %%%%%%%%%%%%%%%%%
%%% TO REMOVE ARXIV VERSION, ADD BACK \newcommand{\ICML@appearing}in icml2025.sty and impact statement
\documentclass{article}

% Recommended, but optional, packages for figures and better typesetting:
\usepackage{microtype}
\usepackage{tikz}
\usepackage{graphicx}
\usepackage{subfigure}
\usepackage{booktabs} % for professional tables
\usepackage{multirow} 
\usepackage{makecell}
% hyperref makes hyperlinks in the resulting PDF.
% If your build breaks (sometimes temporarily if a hyperlink spans a page)
% please comment out the following usepackage line and replace
% \usepackage{icml2025} with \usepackage[nohyperref]{icml2025} above.
\usepackage{hyperref}
\usepackage[justification=centering]{caption}
\usepackage[flushleft]{threeparttable}

% Attempt to make hyperref and algorithmic work together better:
\newcommand{\theHalgorithm}{\arabic{algorithm}}

% Use the following line for the initial blind version submitted for review:
\usepackage[accepted]{icml2025}
\usepackage{numprint} 
% If accepted, instead use the following line for the camera-ready submission:
%\usepackage[accepted]{icml2025}

% For theorems and such
\usepackage{amsmath}
\usepackage{amssymb}
\usepackage{mathtools}
\usepackage{bm}
\usepackage{amsthm}
\usepackage{xspace}
\usepackage{xcolor}
\usepackage{enumitem} 
\usepackage{listings}

% if you use cleveref..
\usepackage[capitalize,noabbrev]{cleveref}

%%%%%%%%%%%%%%%%%%%%%%%%%%%%%%%%
% THEOREMS
%%%%%%%%%%%%%%%%%%%%%%%%%%%%%%%%
\theoremstyle{plain}
\newtheorem{theorem}{Theorem}[section]
\newtheorem{proposition}[theorem]{Proposition}
\newtheorem{lemma}[theorem]{Lemma}
\newtheorem{corollary}[theorem]{Corollary}
\theoremstyle{definition}
\newtheorem{definition}[theorem]{Definition}
\newtheorem{assumption}[theorem]{Assumption}
\theoremstyle{remark}
\newtheorem{remark}[theorem]{Remark}

% Todonotes is useful during development; simply uncomment the next line
%    and comment out the line below the next line to turn off comments
%\usepackage[disable,textsize=tiny]{todonotes}
\usepackage[textsize=tiny]{todonotes}
% \usepackage{siunitx}
\DeclareMathOperator*{\argmin}{arg\,min}
% The \icmltitle you define below is probably too long as a header.
% Therefore, a short form for the running title is supplied here:
\icmltitlerunning{Federated Learning to Personalize PEFT for Multilingual LLMs}
\begin{document}

\twocolumn[
\icmltitle{FedP$^2$EFT: Federated Learning to Personalize \\ Parameter Efficient Fine-Tuning for Multilingual LLMs}

% It is OKAY to include author information, even for blind
% submissions: the style file will automatically remove it for you
% unless you've provided the [accepted] option to the icml2025
% package.

% List of affiliations: The first argument should be a (short)
% identifier you will use later to specify author affiliations
% Academic affiliations should list Department, University, City, Region, Country
% Industry affiliations should list Company, City, Region, Country

% You can specify symbols, otherwise they are numbered in order.
% Ideally, you should not use this facility. Affiliations will be numbered
% in order of appearance and this is the preferred way.
\icmlsetsymbol{equal}{*}

\begin{icmlauthorlist}
\icmlauthor{Royson Lee}{comp}
\icmlauthor{Minyoung Kim}{comp}
\icmlauthor{Fady Rezk}{sch}
\icmlauthor{Rui Li}{comp}
\icmlauthor{Stylianos I. Venieris}{comp}
\icmlauthor{Timothy Hospedales}{comp,sch}
\end{icmlauthorlist}

% \icmlaffiliation{yyy}{Department of XXX, University of YYY, Location, Country}
\icmlaffiliation{comp}{Samsung AI Center, Cambridge, UK}
\icmlaffiliation{sch}{University of Edinburgh, UK}

\icmlcorrespondingauthor{Royson Lee}{royson.lee@samsung.com}
% \icmlcorrespondingauthor{Firstname2 Lastname2}{first2.last2@www.uk}

% You may provide any keywords that you
% find helpful for describing your paper; these are used to populate
% the "keywords" metadata in the PDF but will not be shown in the document
\icmlkeywords{Machine Learning, ICML}

\vskip 0.3in
]

%%% SOME KEYWORDS WE CAN ALL USE :D
\newcommand{\method}{FedP\textsuperscript{2}EFT\xspace}
\newcommand{\seen}{{\em seen}\xspace}
\newcommand{\unseen}{{\em unseen}\xspace}
\newcommand{\basemodel}{{\em base model}\xspace}


\newcommand*\circled[1]{\tikz[baseline=(char.base)]{
            \node[shape=circle,fill,inner sep=1.0pt,scale=0.8] (char) {\textcolor{white}{#1}};}}
\interfootnotelinepenalty=10000
\setlength{\belowdisplayskip}{4.0pt} \setlength{\belowdisplayshortskip}{4.0pt}
\setlength{\abovedisplayskip}{4.0pt} \setlength{\abovedisplayshortskip}{4.0pt}

\renewcommand\theadfont{}
% this must go after the closing bracket ] following \twocolumn[ ...

% This command actually creates the footnote in the first column
% listing the affiliations and the copyright notice.
% The command takes one argument, which is text to display at the start of the footnote.
% The \icmlEqualContribution command is standard text for equal contribution.
% Remove it (just {}) if you do not need this facility.

\printAffiliationsAndNotice{}  % leave blank if no need to mention equal contribution
% \printAffiliationsAndNotice{\icmlEqualContribution} % otherwise use the standard text.

\begin{abstract}
Federated learning (FL) has enabled the training of multilingual large language models (LLMs) on diverse and decentralized multilingual data, especially on low-resource languages. To improve client-specific performance, personalization via the use of parameter-efficient fine-tuning (PEFT) modules such as LoRA is common. 
This involves a {\em personalization strategy} (PS), such as the design of the PEFT adapter structures (\textit{e.g.}, in which layers to add LoRAs and what ranks) and choice of hyperparameters (\textit{e.g.}, learning rates) for fine-tuning. Instead of manual PS configuration, we propose \method{}, a federated {\em learning-to-personalize} method for multilingual LLMs in cross-device FL settings. Unlike most existing PEFT structure selection methods, which are prone to overfitting low-data regimes, \method{} collaboratively learns the optimal personalized PEFT structure for each client via Bayesian sparse rank selection. Evaluations on both simulated and real-world multilingual FL benchmarks demonstrate that \method{} largely outperforms existing personalized fine-tuning methods, while complementing a range of existing FL methods.


%%%% VERSION WITHOUT LATEX FOR OPENREVIEW (30 Jan 2025 20:56)
% Federated learning (FL) has enabled the training of multilingual large language models (LLMs) on diverse and decentralized multilingual data, especially on low-resource languages. To improve client-specific performance, personalization via the use of parameter-efficient fine-tuning (PEFT) modules such as LoRA is common. This involves a personalization strategy (PS), such as the design of the PEFT adapter structures (e.g., in which layers to add LoRAs and what ranks) and choice of hyperparameters (e.g., learning rates) for fine-tuning. Instead of manual PS configuration, we propose FedP$^2$EFT, a federated learning-to-personalize method for multilingual LLMs in cross-device FL settings. Unlike most existing PEFT structure selection methods, which are prone to overfitting low-data regimes, FedP$^2$EFT collaboratively learns the optimal personalized PEFT structure for each client via Bayesian sparse rank selection. Evaluations on both simulated and real-world multilingual FL benchmarks demonstrate that FedP$^2$EFT largely outperforms existing personalized fine-tuning methods, while complementing a range of existing FL methods.

% OLD
% We consider the problem of {\em learning to personalize} a large-scale deep model driven by Federated Learning (FL). Personalization of deep networks typically amounts to adding parameter-efficient fine-tuning (PEFT) modules such as LoRAs and fine-tuning them on private client data. Hence, personalization involves a {\em personalization strategy} (PS) such as the choice of the PEFT adapter structures (\textit{e.g.}, in which layers to add LoRAs and what ranks) and hyperparameter choices for fine-tuning (\textit{e.g.}, learning rates). Conventional practice is to define the PS manually, but in this paper we aim to solve it in an FL manner. 
% We specifically focus on PEFT adapter structure learning through FL, more concretely FL-based optimal LoRA rank selection. While most existing rank selection methods are prone to overfitting, especially in low-data regimes, we propose a more reliable and robust approach that collaboratively learns the optimal personalized LoRA ranks for each client from a Bayesian sparse model selection perspective.
% We empirically demonstrate our method complements a range of popular FL methods, largely outperforming existing non-FL and FL personalization baselines on both simulated and real-world multilingual FL benchmarks. 
\end{abstract}



\section{Introduction}\label{sec:Intro} 


Novel view synthesis offers a fundamental approach to visualizing complex scenes by generating new perspectives from existing imagery. 
This has many potential applications, including virtual reality, movie production and architectural visualization \cite{Tewari2022NeuRendSTAR}. 
An emerging alternative to the common RGB sensors are event cameras, which are  
 bio-inspired visual sensors recording events, i.e.~asynchronous per-pixel signals of changes in brightness or color intensity. 

Event streams have very high temporal resolution and are inherently sparse, as they only happen when changes in the scene are observed. 
Due to their working principle, event cameras bring several advantages, especially in challenging cases: they excel at handling high-speed motions 
and have a substantially higher dynamic range of the supported signal measurements than conventional RGB cameras. 
Moreover, they have lower power consumption and require varied storage volumes for captured data that are often smaller than those required for synchronous RGB cameras \cite{Millerdurai_3DV2024, Gallego2022}. 

The ability to handle high-speed motions is crucial in static scenes as well,  particularly with handheld moving cameras, as it helps avoid the common problem of motion blur. It is, therefore, not surprising that event-based novel view synthesis has gained attention, although color values are not directly observed.
Notably, because of the substantial difference between the formats, RGB- and event-based approaches require fundamentally different design choices. %

The first solutions to event-based novel view synthesis introduced in the literature demonstrate promising results \cite{eventnerf, enerf} and outperform non-event-based alternatives for novel view synthesis in many challenging scenarios. 
Among them, EventNeRF \cite{eventnerf} enables novel-view synthesis in the RGB space by assuming events associated with three color channels as inputs. 
Due to its NeRF-based architecture \cite{nerf}, it can handle single objects with complete observations from roughly equal distances to the camera. 
It furthermore has limitations in training and rendering speed: 
the MLP used to represent the scene requires long training time and can only handle very limited scene extents or otherwise rendering quality will deteriorate. 
Hence, the quality of synthesized novel views will degrade for larger scenes. %

We present Event-3DGS (E-3DGS), i.e.,~a new method for novel-view synthesis from event streams using 3D Gaussians~\cite{3dgs} 
demonstrating fast reconstruction and rendering as well as handling of unbounded scenes. 
The technical contributions of this paper are as follows: 
\begin{itemize}
\item With E-3DGS, we introduce the first approach for novel view synthesis from a color event camera that combines 3D Gaussians with event-based supervision. 
\item We present frustum-based initialization, adaptive event windows, isotropic 3D Gaussian regularization and 3D camera pose refinement, and demonstrate that high-quality results can be obtained. %

\item Finally, we introduce new synthetic and real event datasets for large scenes to the community to study novel view synthesis in this new problem setting. 
\end{itemize}
Our experiments demonstrate systematically superior results compared to EventNeRF \cite{eventnerf} and other baselines. 
The source code and dataset of E-3DGS are released\footnote{\url{https://4dqv.mpi-inf.mpg.de/E3DGS/}}. 





\section{Related Work}
\label{sec:related}


% \subsection{Radiance Fields for Novel View Synthesis}
\noindent {\bf NeRF.}
Neural Radiance Fields (NeRF)\citep{mildenhall2020nerf} revolutionized novel view synthesis via differentiable volume rendering\citep{tulsiani2017mvsupervision, henzler2019platonicgan} and positional encoding~\citep{vaswani2017attentionisallyouneed, gehring2017convolutional}. NeRF models improved in efficiency~\citep{liu2020neural, Garbin_2021_ICCV, chen2024improving}, rendering quality~\citep{mipnerf, zhang2020nerf++,meuleman2023progressively}, handling dynamic scenes~\cite{liu2023robust}, and data efficiency~\citep{pixel.nerf, ibrnet, lin2024frugalnerf,su2024boostmvsnerfs}. Despite excelling at view synthesis, NeRF’s implicit representation complicates scene editing. Recent work on object manipulation~\citep{yang2021learning}, stylization~\citep{wang2023nerf, haque2023instruct}, and inpainting~\citep{nerf.in, spinnerf, mirzaei2023reference} struggles with 3D consistency and structural priors, especially in unbounded scenes.
% Neural Radiance Fields (NeRF)~\citep{original.nerf} revolutionized novel view synthesis, enabling photorealistic scene reconstruction via differentiable volume rendering~\citep{tulsiani2017mvsupervision, henzler2019platonicgan} and positional encoding~\citep{vaswani2017attentionisallyouneed, gehring2017convolutional}. NeRF-based models have since improved in efficiency~\citep{liu2020neural, Garbin_2021_ICCV, chen2024improving}, rendering quality~\citep{mipnerf, zhang2020nerf++,su2024boostmvsnerfs}, and data efficiency~\citep{pixel.nerf, ibrnet, lin2024frugalnerf}. While NeRF excels in view synthesis, its implicit volumetric representation complicates scene editing. Recent works on object manipulation~\citep{yang2021learning}, stylization~\citep{wang2023nerf, haque2023instruct}, and inpainting~\citep{nerf.in, spinnerf, mirzaei2023reference} face challenges in 3D inpainting in unbounded environments, as NeRF struggles with 3D consistency and leveraging explicit structural priors.


\vspace{3pt}
\noindent {\bf 3D Gaussian Splatting.}
3D Gaussian Splatting (3DGS)~\cite{kerbl20233d} efficiently represents scenes with explicit 3D Gaussians, enabling faster rendering, easier training, and flexible editing\citep{chen2024gaussianeditor}. Recent extensions like Scaffold-GS~\citep{scaffoldgs} enhance efficiency with dynamic anchors, while 2DGS~\citep{huang20242d} refines multi-view geometry. 3DGS has also expanded to dynamic scenes~\citep{yang2024deformable, luiten2023dynamic, Wu_2024_CVPR,fan2025spectromotion} and semantic representations~\citep{ye2023gaussian, qin2023langsplat}, supporting advanced editing and novel view synthesis~\citep{qiu-2024-featuresplatting, huang20242d}. Gaussian-based methods thus offer strong potential for explicit 3D inpainting.
% 3D Gaussian Splatting (3DGS)~\citep{kerbl20233d} is an efficient alternative to NeRF, representing scenes with explicit 3D Gaussians for faster rendering, easier training, and more flexible scene editing~\citep{chen2024gaussianeditor}. Recent extensions include Scaffold-GS~\citep{scaffoldgs}, which improves rendering efficiency with dynamic anchor points, and 2DGS~\citep{Huang2DGS2024}, which refines multi-view reconstructions for view-consistent geometry. 3DGS has also been extended to dynamic environments~\citep{yang2024deformable, luiten2023dynamic, Wu_2024_CVPR} and semantic-aware representations~\citep{ye2023gaussian, qin2023langsplat}, advancing scene manipulation and novel view synthesis~\citep{qiu-2024-featuresplatting, huang20242d}. These advancements highlight the potential of Gaussian-based representations for explicit scene editing, making them well-suited for 3D inpainting tasks.

% \subsection{2D Image Inpainting}
% \paragraph{Traditional methods.}
% Image inpainting has evolved from early PDE-based techniques \citep{bertalmio2000image} to exemplar-based methods \citep{criminisi2004region}. Texture synthesis \citep{efros1999texture} and patch-based approaches like PatchMatch \citep{barnes2009patchmatch} further advanced the field. Despite limitations with large missing regions and complex textures \citep{jam2021comprehensive,liu2018image}, these methods established principles now incorporated into learning-based approaches \citep{liu2018image,yu2019free}. Their computational efficiency remains valuable in resource-constrained scenarios \citep{jam2021comprehensive}.

% \vspace{-3mm}
% \paragraph{Deep learning-based methods.}
% Deep learning has revolutionized image inpainting, with CNNs like Context Encoders \citep{feature.learning.by.inpainting} pioneering the field. GANs \citep{gan} and models like DeepFillv2 \citep{yu2019free} further improved results. Large Mask Inpainting (LaMa) \citep{lama} addressed large missing regions. Recently, diffusion models\citep{NEURIPS2020_4c5bcfec}, particularly Stable Diffusion\citep{rombach2022high}, have demonstrated remarkable inpainting capabilities, leveraging complex data distributions\citep{dhariwal2021diffusion}. Beyond text-to-image generation, diffusion models are commonly used for image-to-image tasks, including image editing and inpainting. SDEdit\citep{meng2022sdedit} leverages diffusion models for semantic editing by injecting Gaussian noise into input images and performing iterative denoising, ensuring structural coherence while modifying visual content. To further improve image manipulation fidelity, Noise Inversion techniques such as DDIM Inversion~\citep{song2021denoising} enable precise latent code inference through deterministic reverse diffusion sampling. This approach retains finer details of the original image, making it particularly effective for manipulating real images within diffusion-based generative models. In the context of inpainting, models like SDXL-Inpainting have been developed by fine-tuning diffusion models specifically for inpainting tasks. While these methods have significantly improved image inpainting quality, Stable Diffusion-based inpainting often introduces scene-inconsistent artifacts within the inpainted regions. This challenge becomes even more pronounced when leveraging 2D diffusion priors for 3D inpainting, as it can lead to multi-view inconsistencies—a major limitation for 3D scene reconstruction~\citep{li2023diffusion}. The success of diffusion-based inpainting has inspired extensions to 3D inpainting tasks\citep{nerf.in, inpaint3d}, though adapting 2D approaches to 3D presents additional challenges, such as geometry misalignment, depth inconsistencies, and occlusion handling\citep{mirzaei2023reference}.



% \vspace{-3mm}
% \paragraph{Reference-based methods.}
% Reference-based inpainting methods \citep{zhao2022geofill} enhance traditional inpainting by incorporating visual context from reference images, improving content accuracy and consistency. LeftRefill \citep{tang2023realfill} uses a two-stage architecture with feature matching and refinement networks, enabling inpainting from different viewpoints based on reference information \citep{zhao2022geofill}. While these methods show promise in various applications \citep{jam2021comprehensive}, challenges remain in seamless integration and reference selection \citep{li2023diffusion}, particularly when views diverge significantly from the reference. The success of these approaches has also inspired 3D inpainting extensions \citep{nerf.in, inpaint3d}, though adapting to 3D introduces additional complexities \citep{mirzaei2023reference}.

% \vspace{-2mm}

% \subsection{Image Inpainting}
\vspace{3pt}
\noindent {\bf Traditional and learning-based image inpainting.}
Early image inpainting techniques, including PDE-based~\citep{bertalmio2000image}, exemplar-based~\citep{criminisi2004region}, and PatchMatch~\citep{barnes2009patchmatch}, were effective for small regions but struggled with complex textures and large gaps~\citep{jam2021comprehensive, liu2018image}. Deep learning advanced the field significantly, starting with Context Encoders~\citep{feature.learning.by.inpainting} and GAN-based methods like DeepFill~\citep{generative.inpainting, yu2019free}, improving content synthesis and coherence. Recent models such as LaMa~\citep{lama} use Fourier convolutional networks to address large masks. Diffusion models~\citep{NEURIPS2020_4c5bcfec}, notably Stable Diffusion~\citep{rombach2022high}, introduced iterative refinement capabilities, providing more flexible and structurally consistent inpainting compared to GANs~\citep{dhariwal2021diffusion}.
% Early image inpainting techniques, including PDE-based~\citep{bertalmio2000image}, exemplar-based~\citep{criminisi2004region}, and patch-based methods like PatchMatch~\citep{barnes2009patchmatch}, were effective for small missing regions but struggled with complex textures and large gaps~\citep{jam2021comprehensive, liu2018image}. Deep learning brought significant advancements, starting with CNN-based models like Context Encoders~\citep{feature.learning.by.inpainting} and GANs such as DeepFill~\citep{generative.inpainting, yu2019free}, which improved content synthesis and structural coherence. More recent models like Large Mask Inpainting (LaMa)~\citep{lama} further enhanced quality by using Fourier convolutional networks for large masked regions. The rise of diffusion models~\citep{NEURIPS2020_4c5bcfec}, notably Stable Diffusion~\citep{rombach2022high}, introduced powerful text-to-image and image-to-image capabilities, enabling more flexible and structurally consistent inpainting by iteratively refining missing regions, unlike GANs~\citep{dhariwal2021diffusion}.

% \vspace{-4mm}

\vspace{3pt}
\noindent {\bf Diffusion models for image editing and inpainting.}
Beyond direct inpainting, diffusion models are widely used for image editing. SDEdit~\citep{meng2022sdedit} injects Gaussian noise and iteratively denoises, enabling semantic edits while preserving global structure. Noise inversion techniques~\cite{mokady2022null, miyake2024negativepromptinversionfastimage}, such as DDIM Inversion~\citep{song2020denoising}, further improve editing fidelity by enabling precise latent inference through deterministic reverse diffusion.
Inpainting-specific diffusion models like SDXL-Inpainting~\cite{podell2023sdxlimprovinglatentdiffusion} enhance image reconstruction by fine-tuning Stable Diffusion. Reference-based methods~\cite{tang2023realfill}, such as LeftRefill~\citep{cao2024leftrefill}, use diffusion models for reference-guided synthesis but struggle in regions distant from reference views.
Despite advancements, Stable Diffusion-based inpainting~\cite{inpaint3d} still suffers from inconsistent artifacts in scene-dependent contexts, causing multi-view inconsistencies problematic for 3D scenes~\citep{li2023diffusion}. This motivates our use of SDEdit and DDIM Inversion to preserve structural information and ensure multi-view coherence.
% Beyond direct inpainting, diffusion models are widely used for image editing. SDEdit~\citep{meng2022sdedit} injects controlled Gaussian noise followed by iterative denoising, enabling semantic modifications while preserving global structure. To improve editing fidelity, Noise Inversion techniques~\cite{, mokady2022null, miyake2024negativepromptinversionfastimage} like DDIM Inversion~\citep{song2020denoising} allow precise latent code inference through deterministic reverse diffusion sampling. By inverting an image to a specific noise level and denoising it back, Noise-Inversion provides finer control over content preservation, making it highly effective for inpainting real images while minimizing distortion during denoising.
% % 
% Inpainting-specific diffusion models like SDXL-Inpainting~\cite{podell2023sdxlimprovinglatentdiffusion} enhance the process by fine-tuning Stable Diffusion models for image reconstruction. Reference-based method~\cite{tang2023realfill} such as LeftRefill~\citep{cao2024leftrefill} uses pre-trained diffusion models for reference-guided synthesis, stitching reference and target views to enable contextual inpainting, view synthesis, and image completion via task-specific prompt tuning. However, LeftRefill struggles in regions far from the reference view, where alignment becomes less reliable.
% % 
% Despite these advancements, Stable Diffusion-based inpainting~\cite{inpaint3d} often produces inconsistent artifacts, particularly in scene-dependent contexts. When applied to 3D inpainting, these artifacts lead to multi-view inconsistencies, a critical limitation for scene reconstruction and object removal~\citep{li2023diffusion}. This motivates our use of SDEdit and DDIM Inversion for 3D inpainting, ensuring that denoising preserves critical structural information while maintaining coherence across viewpoints.  

% Although diffusion-based inpainting has inspired 3D inpainting extensions\citep{nerf.in, inpaint3d}, adapting 2D methods to 3D introduces challenges like geometry misalignment, depth inconsistencies, and occlusion handling.



% \subsection{3D Scene Inpainting}
% \paragraph{Methods without multi-view background knowledge.}
% As 3D scene representation and reconstruction techniques have advanced, the need for 3D inpainting methods has grown. Early approaches to 3D scene inpainting often relied on single-view or limited-view information, attempting to extend 2D inpainting concepts into the 3D domain without leveraging extensive multi-view knowledge.
% One category of methods focuses on direct 3D shape completion. These approaches typically operate on point clouds or voxel representations. For instance, PCN (Point Completion Network) introduced by \citet{yuan2018pcn} uses an encoder-decoder architecture to complete partial point clouds. While effective for object-level completion, these methods often struggle with complex, large-scale scene inpainting.
% Another approach involves using 2.5D representations, where depth information is incorporated alongside RGB data. Depth-aware inpainting methods, such as the work by \citet{3d.photography}, extend 2D inpainting techniques by considering depth as an additional channel. These methods can produce more geometrically consistent results but are limited by their reliance on a single viewpoint.
% Some researchers have explored the use of generative models for 3D inpainting. 3D-GAN, proposed by \citet{wu2016learning}, generates 3D shapes from a probabilistic space, which can be adapted for inpainting tasks. However, these methods often struggle with fine details and scene-level consistency.
% In the context of neural rendering, early attempts at NeRF editing and inpainting also fall into this category. Methods like EditNeRF by \citet{liu2021editing} allow for object-level editing in NeRF scenes but are limited in their ability to handle large-scale scene modifications or inpainting of complex structures.
% Standalone NeRF inpainting methods, such as NeRF-In by \citet{nerf.in}, attempt to inpaint 3D scenes represented as Neural Radiance Fields. These approaches often rely on 2D inpainting results as supervision, projecting them back into the 3D space. While they can produce plausible results for small edits, they struggle with view consistency and large-scale modifications.
% A common limitation of these single-view or limited-view methods is their inability to fully leverage the 3D structure of the scene. They often produce results that are inconsistent across different viewpoints or fail to capture the true geometry of occluded regions \citep{spinnerf}. Additionally, these methods may struggle with understanding the global context of the scene, leading to inpainted content that doesn't align well with the overall scene structure \citep{wang2023inpaintnerf360}.
% Despite these limitations, these methods have laid important groundwork for 3D scene inpainting. They have highlighted the challenges specific to 3D inpainting, such as maintaining geometric consistency and handling occlusions, which have informed the development of more advanced, multi-view aware techniques \citep{mirzaei2023reference}.
% Existing 3D inpainting approaches~\cite{weder2022removing, spinner, nerfin} extended 2D concepts to 3D without extensive multi-view knowledge. These include direct 3D shape completion methods like PCN \citep{yuan2018pcn}, 2.5D representations \citep{3d.photography}, and generative models like 3D-GAN \citep{wu2016learning}. In the field of neural rendering, EditNeRF \citep{liu2021editing} and NeRF-In \citep{nerf.in} pioneered NeRF editing and inpainting. These methods often struggle with view consistency \citep{spinnerf} and global context \citep{wang2023inpaintnerf360}. Despite limitations, they laid groundwork for more advanced, multi-view aware techniques \citep{mirzaei2023reference}. 

% Existing 3D inpainting approaches用在NeRf上~\cite{weder2022removing, spinnerf, nerfin} 因為nerf implicit representation的特性,通常都是leverage 2d inpainter to 3d, 像spinnerf~\cite{spinnerf}用一個lpips loss來減緩muli-view inpainting 的inconcsitency. 而reference-based method, 為了要進一步解決multi-view inconsitency的問題,他們提出想要只使用少數的reference image來代表要inpaint的區域。然而他們通常只能能render的novel view角度很小, 局限在forward facing scene, 很能利用到unbounded 360 scene上。InNeRF360~\cite{wang2023inpaintnerf360}雖然可以使用在360場景,使用Hallucinating Density Removal來清除inconsistency造成的flaoters, 但卻一樣是利用將object先inpaint掉在拿去train nerf的方式, 沒辦法利用原有場景資訊。

% 而得天GaussianSplatting explict的特性, GaussianGrouping可以將semantic 資訊加到每一顆Gaussian上, InFusion [citation] approaches 3D Gaussian inpainting by leveraging depth completion and progressive reference view synthesis. While achieving efficient results, the method's limitations include manual view selection requirements and potential inaccuracies in depth completion for complex geometries, 而且他們的depth completion model需奧finetuning. GScream leverages 3D Gaussian Splatting for object removal by integrating monocular depth guidance and cross-attention feature propagation between visible and in-painted regions to achieve consistent geometry and textures. However, it's hard to extend to unbounded 360 scene 因為他一樣是reference-based method. 而我們的方法...


\vspace{3pt}
\noindent {\bf 3D scene inpainting.}
Existing 3D inpainting methods for NeRF~\cite{weder2022removing, spinnerf, shen2023nerfin, yin2023or} typically adapt 2D models to NeRF’s implicit representation. For instance, SPIn-NeRF~\cite{spinnerf} employs perceptual loss to improve multi-view consistency. Reference-based methods~\cite{mirzaei2023reference, mirzaei2024reffusionreferenceadapteddiffusion, wang2024gscream} enhance consistency using reference images but remain limited to small-angle view rendering, restricting their use in 360° scenes. NeRFiller~\cite{weber2023nerfiller} iteratively refines consistency with grid prior but struggles with fine-grained textures due to image downsampling. InNeRF360~\cite{wang2023inpaintnerf360} handles 360° scenes via density hallucination but has limited scene utilization.
% 
Gaussian Splatting-based methods like Gaussian Grouping~\cite{ye2023gaussian} inject semantic information, while InFusion~\cite{liu2024infusion} employs depth completion but requires manual view selection. GScream~\cite{scaffoldgs} integrates Scaffold-GS but faces difficulties in unbounded 360° scenes. Our method addresses these issues by enhancing multi-view consistency and depth-aware inpainting in 360° scenarios using Gaussian Splatting.
% Existing 3D inpainting approaches for NeRF~\cite{weder2022removing, spinnerf, shen2023nerfin, yin2023or} often extend 2D inpainting models into 3D due to NeRF’s implicit representation. For example, SPIn-NeRF~\cite{spinnerf} uses perceptual loss to reduce multi-view inpainting inconsistencies. Reference-based methods~\cite{mirzaei2023reference, mirzaei2024reffusionreferenceadapteddiffusion, wang2024gscream} aim to further reduce these inconsistencies by using a few reference images to represent the inpainting area. However, these methods are generally restricted to rendering novel views from small angles, making them less suitable for unbounded 360° environments. NeRFiller~\cite{weber2023nerfiller} improves multi-view consistency with Grid Prior, extends it to Joint Multi-View Inpainting, and refines missing regions iteratively via Dataset Update, ensuring 3D structural coherence without reference images or object masks. Yet, its reliance on image downsampling limits high-frequency detail reconstruction, reducing effectiveness for fine-grained textures. InNeRF360~\cite{wang2023inpaintnerf360} adapts to 360° scenes using Hallucinating Density Removal to address view inconsistencies but remains limited by its object inpainting approach before NeRF training, restricting full scene utilization.
% % 
% Gaussian Splatting enables precise inpainting, as seen in Gaussian Grouping~\cite{ye2023gaussian}, which injects semantic information into each Gaussian. InFusion~\cite{liu2024infusion} enhances 3D Gaussian inpainting with depth completion and progressive reference synthesis but is limited by manual view selection and fine-tuning. GScream integrates Scaffold-GS~\cite{scaffoldgs} for object removal, using monocular depth and cross-attention for consistency but struggles with 360° unbounded scenes due to fixed reference views. Our method addresses these challenges by improving multi-view consistency in 360° environments, leveraging Gaussian Splatting for explicit scene manipulation and depth-aware inpainting.




% Previous work such as SPIn-NeRF that integrates 2d inpainting model with perceptual loss and depth inpainting guidance to reconstruct Inpainted Neural Radiance Fields (NeRF). OR-NeRF Removing Objects From NeRF use confidence-based view selection automatically removes inconsistent views from the optimization preventing unwanted artefacts in the final result.MVIP-NeRF utilizes a multi-view approach to perform 3D inpainting on NeRF scenes by employing diffusion priors, where it jointly optimizes RGB and normal map completion through an iterative Score Distillation Sampling (SDS) process, ensuring consistent appearance and geometry alignment across multiple views while leveraging a multi-view scoring mechanism to distill generative priors from different perspectives. 然而這些方法並沒有透過leverage 已有場景資訊來進行inpainting. Reference-guided controllable inpainting, as presented by Mirzaei et al. (2023), leverages reference images and view-dependent effects to guide the inpainting process in 3D scenes, enabling consistent and visually coherent completions across multiple perspectives while handling challenges like disocclusions and geometric consistency. GScream introduces a robust framework for object removal in 3D scenes by optimizing Gaussian primitives' positions for geometric consistency and utilizing a cross-attention feature propagation mechanism to enhance texture coherence, effectively restoring both geometry and texture across visible and occluded areas. 
% Reference-guided inpainting in NeRF  

% 先nerf方法, 指出雖然..., 他們1需要非常精確的object mask來remov場景的objec, 但又inplcit-method, 所以沒辦法很好的透出場景資訊
% 在講道






% \vspace{-3mm}
% \paragraph{Methods leveraging multi-view information.}
% As the limitations of single-view 3D inpainting methods became apparent, researchers began to explore approaches that leverage multi-view information. These methods aim to produce more consistent and geometrically accurate results by utilizing the rich information available from multiple viewpoints of a scene.
% One of the pioneering works in this direction is SPIn-NeRF by \citet{spinnerf}. This method combines Neural Radiance Fields (NeRF) with multi-view image inpainting to remove objects from 3D scenes. SPIn-NeRF uses a two-stage approach: first, it inpaints each input view using a 2D inpainting method, then it optimizes a NeRF to fit these inpainted views. By leveraging multi-view consistency, SPIn-NeRF can produce more coherent results across different viewpoints than single-view methods.
% Another significant contribution in this area is the work by \citet{philip2018plane} on object removal for image-based rendering. Their method uses multi-view stereo to reconstruct the scene geometry and then performs inpainting in both color and depth spaces across multiple views. This approach demonstrates the importance of considering both appearance and geometry in multi-view 3D inpainting.
% Inpaint3D, proposed by \citet{inpaint3d}, takes a different approach by training a 3D-aware inpainting network on a large dataset of indoor scenes. This method can leverage the learned 3D priors to produce geometrically consistent inpaintings across multiple views, even for large missing regions.
% Recent advancements in NeRF-based representations have led to more sophisticated multi-view inpainting methods. For instance, InpaintNeRF360 by \citet{wang2023inpaintnerf360} extends inpainting capabilities to 360-degree scenes. This method uses a combination of 2D inpainting guidance and 3D consistency optimization to handle the challenges of inpainting in omnidirectional environments.
% Gaussian Grouping, introduced by \citet{ye2023gaussian}, presents a novel approach to 3D scene editing using 3D Gaussian Splatting. While not specifically designed for inpainting, this method demonstrates how multi-view information can be leveraged to segment and manipulate 3D scenes represented by Gaussians, opening new possibilities for 3D inpainting tasks.
% A common thread among these multi-view methods is their ability to maintain consistency across different viewpoints, a crucial aspect of 3D scene inpainting. By leveraging information from multiple views, these approaches can better understand the underlying 3D structure of the scene and produce inpainted results that are coherent with the global scene geometry \citep{mirzaei2023reference}.
% However, challenges remain. Many of these methods still struggle with large-scale occlusions or complex geometric structures \citep{weder2022removing}. The computational cost of processing multiple views can also be significant, especially for high-resolution or large-scale scenes \citep{barron2023zip}. Additionally, balancing the influence of different views and handling potential inconsistencies between them remains an active area of research \citep{yin2023or}.
% Despite these challenges, multi-view 3D inpainting methods have significantly advanced the state of the art, enabling more realistic and consistent scene editing and completion. As research progresses, we can expect to see further improvements in the quality and efficiency of these techniques, potentially leading to new applications in fields such as virtual reality, augmented reality, and digital twin technologies \citep{bommasani2021opportunities}.

% Multi-view 3D inpainting methods address the limitations of single-view approaches. SPIn-NeRF \citep{spinnerf} combines NeRF with multi-view image inpainting. \citet{philip2018plane} use multi-view stereo for object removal in image-based rendering. Inpaint3D \citep{inpaint3d} leverages learned 3D priors. InpaintNeRF360 \citep{wang2023inpaintnerf360} extends to 360-degree scenes, while Gaussian Grouping \citep{ye2023gaussian} uses 3D Gaussian Splatting. These methods maintain consistency across viewpoints \citep{mirzaei2023reference} but face challenges with large-scale occlusions \citep{weder2022removing}, computational costs \citep{barron2023zip}, and view inconsistencies \citep{yin2023or}. Despite challenges, they advance scene editing and completion, potentially leading to new applications \citep{bommasani2021opportunities}.










\section{Our Approach}\label{sec:main}


\subsection{Preliminaries \& Motivation}

In personalized FL, the goal is to minimize each client's local objective $\mathbb{E}_{(x,y) \sim P^i}\mathcal{L}^i(\Phi^i;x,y)$ where $P^i$ represents the data distribution of the $i$-th client, $x$ and $y$ are the input data and labels, respectively, and $\mathcal{L}^i(\Phi^i;x,y)$ is the loss function for client $i$ given model parameters $\Phi^i$. This is typically achieved via fine-tuning~\cite{matsuda2022empirical,chen2022pfl} a \basemodel{}, with parameters $\Phi^i_{BM}$ and a set of hyperparameters, \textit{e.g.}~learning rate. Note that $\Phi^i_{BM}$ may differ across clients if it is already personalized, \textit{e.g.}~if $\Phi^i_{BM}$ is obtained using a personalized FL algorithm.

Fine-tuning LLMs, however, is unprecedentedly compute and memory intensive, and prone to overfitting. As such, the majority of existing federated LLM works~\cite{zhao2023breaking,fedpeft} rely on PEFT methods, with LoRA~\cite{hu2021lora} being a prevalent choice due to its efficiency and performance. Specifically, for a frozen weight matrix $W \in \mathbb{R}^{d\times e}$, LoRA introduces low-rank matrices $B \in \mathbb{R}^{d \times r}$ and $A \in \mathbb{R}^{r \times e}$ where $r \ll \min(d, e)$. The adapted weights are then expressed as: $W' = W + \frac{\alpha_{\text{lora}}}{r}BA$ where $\alpha_{\text{lora}}$ is a hyperparameter and only $B$ and $A$ are trained during fine-tuning.
Although effective, these FL works rely on a fixed hand-crafted PS, \textit{e.g.},~a manually defined LoRA rank on hand-picked layers, for all clients, leading to suboptimal personalized models. 

\subsection{Personalized PEFT}\label{sec:personalized_peft}

We, instead, propose using a different PS for each client. Common hyperparameter choices from previous federated HPO approaches (Section~\ref{sec:related}) include learning rates and batch normalization (BN) hyperparameters. While these hyperparameters have been shown to be effective for handling data heterogeneity in popular vision and speech benchmarks~\cite{fedbn,li2016revisiting,fedper}, they are less consequential or not applicable when fine-tuning LLMs. This stems from the fact that LLMs are often fine-tuned using adaptive optimizers, \textit{e.g.}~Adam, which are more robust to the learning rate~\cite{zhao2025deconstructing}, and BN layers are not typically used. 
A more critical hyperparameter choice shown to be effective, especially for cross-lingual transfer learning~\cite{pfeiffer2020mad}, is the PEFT adapter structure; specifically which layers to introduce LoRAs in and what ranks to utilize~\cite{adalora,autolora}. 

\noindent\textbf{Adapting BayesTune for LoRA Rank Selection.}~
Building upon BayesTune~\cite{kim2023bayestune}, a Bayesian sparse model selection approach, we formulate PEFT personalization as a sparse LoRA rank selection problem and propose BayesTune-LoRA. 
Concretely, we introduce rank-wise latent variables $\lambda \in \mathbb{R}^r, \; \lambda_i > 0, \; \forall i=1,2,\cdots,r$ for each LoRA matrix: $B\lambda A$. Let $\bm{\lambda}= \{\lambda_{l,\cdot}\}_{l=1}^L$ be the set of all $\lambda$ where $\lambda_{l,\cdot}$ represents the rank-wise scales for layer $l$ in a model with $L$ LoRA modules (similarly for $\bm{A}$ and $\bm{B}$). Using BayesTune, the 
values for $\theta=(\bm{\lambda}$,$\bm{A}$,$\bm{B})$ are optimized as:
% 
% \vspace{-2em}
\begin{align}
\label{eq:bayestune}
\theta^* =& \argmin_{\theta} \mathcal{L}_{\text{CE}}(\theta;D) + \frac{\alpha_{s}}{N} \mathcal{L}_{s}(\bm{\lambda},\bm{B}) + \frac{\alpha_{p}}{N}\mathcal{L}_{p}(\bm{\lambda}) 
\end{align}
where $D$$=$$\{(x_i, y_i)\}_{i=1}^N$ is the train dataset, $N$ the size of $D$, $\mathcal{L}_{\text{CE}}(\theta;D)$ the cross-entropy loss, $\alpha_{p}$ and $\alpha_{s}$ hyperparameters, $\mathcal{L}_s$ the logarithm of the Laplace distribution (prior imposed on $p(B|\lambda)$\footnote{Unlike BayesTune, where every parameter is associated with its own prior scale, we use an ``independent'' Laplace prior where each $\lambda_{l,i}$ applies to all entries of $B_{l,i}$}), $f(\|B_{l,i}\|_1;\mu,b)= \frac{1}{2b} \exp\left(-\frac{|\|B_{l,i}\|_1 - \mu|}{b}\right)
$ with $\mu=0$ ($B$ is initialized to 0 in LoRA) and $b=\lambda_{l,i}$:
\begin{align}
\mathcal{L}_s(\bm{\lambda}, \bm{B}) = \sum^L_l \sum^r_i \left(\log \lambda_{l,i} + \frac{\|B_{l,i}\|_1}{\lambda_{l,i}} + \log2\right)
\end{align}
and $\mathcal{L}_p$ is the logarithm of the Gamma distribution (hyper-prior imposed on $\lambda$), $\mathcal{G}(\lambda_{l,i};\alpha_g,\beta_g)= \frac{\beta_g^{\alpha_g}}{\Gamma(\alpha_g)} \lambda_{l,i}^{\alpha_g-1} e^{-\beta_g \lambda_{l,i}}$ where $\alpha_g=0.01,\beta_g=100$ following the hyperparameters set by the original authors:
\begin{align}
\mathcal{L}_p(\bm{\lambda}) = \sum^L_l \sum^r_i &(0.99\cdot \log \lambda_{l,i} + 100 \cdot \lambda_{l,i} \nonumber \\
& - 0.01\log(100) + \log\Gamma(0.01)) 
\end{align}
In practice, we can save computations by removing all constants and the duplicate term $\log \lambda$, resulting in the following approximated penalty losses:
\begin{align}
\mathcal{L}_s(\bm{\lambda}, \bm{B}) &= \sum^L_l \sum^r_i \frac{\|B_{l,i}\|_1}{\lambda_{l,i}} \\
\mathcal{L}_p(\bm{\lambda}) &= \sum^L_l \sum^r_i (\log \lambda_{l,i} + 100 \cdot \lambda_{l,i}) 
\end{align}
Roughly speaking, $\mathcal{L}_p$ encourages small $\lambda$ while $\mathcal{L}_s$ encourages larger $\lambda$ for larger LoRA $B$ (per column) updates. Hence, minimizing the losses in Eq.~(\ref{eq:bayestune}) encourages larger $\lambda$ in more significant ranks. 

\noindent\textbf{Personalizing PEFT with BayesTune-LoRA.}~For each client, we attach BayesTune-LoRA modules, $\theta$, to all linear layers of its \basemodel{} with rank $r_{\text{init}} = \alpha_{r\_mul} \cdot r_{\text{max target}}$ where $r_{\text{max target}}$ is the maximum inference resource budget and $r_{\text{init}}$ is the initial rank before pruning. $\theta$ is then optimized using Adam~\cite{Kingma_2014} as per Eq.~(\ref{eq:bayestune}).\footnote{BayesTune proposed using SGLD~\cite{welling2011bayesian}, adding Gaussian noise to the gradient updates and sampling from the posterior distribution. Due to the challenges of estimating the full posterior distribution in FL settings, particularly with limited client data, we opt to find a point estimate.}

After training, we freeze the resulting $\bm{\lambda}$ and use it for personalization. Specifically, given a resource budget (total rank budget) of $r \cdot L$, we prune $\bm{\lambda}$ by taking the top-$(r \cdot L)$ largest ranks, along with the corresponding rows of $\bm{A}$ and columns of $\bm{B}$.\footnote{The LoRA module is discarded for layers where $\|\lambda_l\|_1 = 0$} We then reinitialize the pruned $\bm{A}$ and $\bm{B}$ and perform standard fine-tuning on $\mathcal{L}_{\text{CE}}$ with the frozen pruned $\bm{\lambda}$ to obtain the personalized model. Note that we only have to train $\bm{\lambda}$ once for all ranks $\leq r_{\text{max target}}$.

\subsection{\method{}: FL to Personalize PEFT}\label{sec:main_method}

The limited data available to each client in FL makes it difficult to train an effective PS in isolation, frequently resulting in overfitting. Following FedL2P~\cite{royson2023fedl2p}, we mitigate this by federatedly learning a common PSG that generates client-wise PS. Concretely, we use a small, one hidden layer multilayer perceptron (MLP) with parameters $\phi$ that takes as input the client meta-data and outputs an estimated PS as follows:
% 
% \vspace{-2em}
\begin{align}\label{eq:mlp}
\bm{\hat{\lambda}} = \text{MLP}(\phi; \:\: &E(h_0),SD(h_0),E(h_1),SD(h_1), \nonumber \\
&\cdots,E(h_{L-1}),SD(h_{L-1}))
\end{align}
where $h_{l-1}$ is the input feature to the $l$-th layer in the \basemodel{}, and $E(\cdot)$ and $SD(\cdot)$ are the mean and standard deviation (SD), respectively.

In contrast to FedL2P, which adopts a computationally demanding meta-learning approach to train MLP, we take a two-stage strategy for each client: \textit{1)} first, learn $\bm{\lambda}$, followed by \textit{2)} regression learning of MLP to target the learned $\bm{\lambda}$. 

\noindent\textbf{Federated Training of \method{}.}~Fig.~\ref{fig:approach} shows the entire \method{} algorithm during federated training. For each federated round, each sampled participating client $i$ receives $\phi$ from the server and loads them into its MLP. They then \circled{1} perform a forward pass of the local train dataset on their \basemodel{} and a forward pass of the MLP with the resulting features as per Eq.~(\ref{eq:mlp}). \circled{2} The estimated $\bm{\hat{\lambda}}^i$ is plugged into our proposed BayesTune-LoRA (Section~\ref{sec:personalized_peft}) and \circled{3} fine-tuning is performed as per Eq.~(\ref{eq:bayestune}) for $s$ steps (Stage 1). \circled{4} The resulting $\hat{\bm{\lambda}}^{i,s}$
is used as an approximated ground-truth for regression learning of MLP to target the learned $\hat{\bm{\lambda}}^{i,s}$, where $\mathcal{L}_1$ is the L1 loss (Stage 2). Finally, \circled{5} $\phi$ is sent back to the server for aggregation. As there is no single aggregation method that outperforms all others in every situation~\cite{matsuda2022empirical,chen2022pfl,fedllm-bench}, we utilize FedAvg~\cite{fedavg}. The aggregated $\phi$ is then sent to clients for the next round.

At the end of federated training, the learned $\phi$ can be deployed to any client, \seen{} or \unseen{} during federated training. Note that unlike FedL2P, which requires federated training for every target rank, \method{} inherits the property of BayesTune-LoRA; federated training is a one-time cost for all ranks $\leq r_{\text{max target}}$.

% \vspace{-0.3cm}
\noindent\textbf{Inference with \method{}.}~Fig.~\ref{fig:intro}(a) shows how \method{} personalizes PEFT for each client upon deployment. Given the learned MLP and the client's \basemodel{}, \circled{1} the client meta-data are retrieved (Eq.~\ref{eq:mlp}) and used to generate the client's PS, $\bm{\lambda}$. \circled{2} Given the client's resource budget of total rank $r \cdot L$, we take the top-$(r \cdot L)$ largest ranks in $\bm{\lambda}$, freeze them, and initialize our proposed BayesTune-LoRA modules for all layers where $\|\lambda_l\|_1$$>$$0$. \circled{3} The personalized LoRA ranks are used for fine-tuning before merging back to the \basemodel{} to obtain the final personalized model.

\begin{figure}[t]
    \includegraphics[width=1.0\columnwidth,trim={4.5cm 2cm 12cm 3.8cm},clip]{figures/fedppeft_approach.pdf} 
    \vspace{-0.9cm}
    % \captionsetup{font=footnotesize	,labelfont=bf}
    \caption{\method{}'s federated training of PSG for each federated round (See text in Section.~\ref{sec:main_method}).}
    \vspace{-0.5cm}
    \label{fig:approach}
    \vspace{-0.2cm}
\end{figure}






% \newpage
\section{Evaluation}\label{sec:expmts}

\subsection{Experimental Setup}

We conduct experiments on multilingual scenarios, where clients with diverse high- and low-resource languages can collaboratively learn how to personalize a given base model to better cater to their language preferences. In all experiments, we divide clients in two pools, \seen{} and \unseen{}, where only the clients in the \seen{} pool actively participate in federated training. We set the maximum number of communication rounds for training the PSG to $150$, randomly sampling $10\%$ of participating clients every round. We use Adam~\cite{Kingma_2014} as the default optimizer for all our experiments. We evaluate on resource budgets $r=2,4,8,16$ where the total rank budget is $r \cdot L$. We summarize the FL scenarios that we consider in our experiments, leaving comprehensive details in Appendix~\ref{appendix:experiments}.

\subsubsection{Tasks, Models, and Datasets}

\noindent\textbf{Text Classification.}~We adopt the pretrained multilingual BERT~\cite{BERT} (mBERT) for all text classification experiments. For datasets, we introduce additional data heterogeneity to the simulated FL setups, XNLI~\cite{XNLI} and MasakhaNEWS~\cite{MasakhaNEWS}, proposed in PE\_FL~\cite{zhao2023breaking}. 

For our XNLI setup, we sample 2k instances for train and $500$ for test in each pool. In contrast to PE\_FL, which had $15$ clients ($1$ language per client), we divide the data equally among $20$ clients for each language. We then adopt the latent Dirichlet allocation (LDA) partition method~\cite{hsu2019measuring, yurochkin2019bayesian}, $y \sim Dir(\alpha)$, to simulate non-IID label shifts among these clients, with $\alpha=0.5$. Hence, there is a total of $600$ clients ($15$ languages $\cdot$ $20$ clients per language $\cdot$ $2$ pools), consisting of both label and feature heterogeneity.

For MasakhaNEWS, we first split the data in each of the $16$ languages by half for each pool. Similar to our XNLI setup, we divide each language's data equally among 10 clients and adopt LDA with $\alpha=0.5$, resulting in $320$ clients in total. Differing from our XNLI setup, each language varies in the amount of samples, adding another layer of data heterogeneity to the setup: quantity skew.

\noindent\textbf{Instruction-Tuning Generation.}~We adopt pretrained MobileLLaMA-1.4B~\cite{mobilellama} and Llama-3.2-3B~\cite{llama3}, which are representative of commonly supported model sizes on recent high-end edge devices~\cite{openelm2024icmlw,2024_mobilequant,edgellm2024tmc}. For each model, we run experiments on the recent Fed-Aya dataset. Fed-Aya is a real-world FL dataset naturally partitioned by annotator ID and each client has data with up to $4$ languages. Out of a total of $38$ clients, we select $8$ clients for our \unseen{} pool. We also split each client's data $80\%/20\%$ for train and test, respectively. Fig.~\ref{fig:fed-aya} shows the distribution of predominant languages, where predominant refers to the client's most commonly used language, in our setup.

\begin{figure}
    \small
    \centering
    \includegraphics[width=0.9\columnwidth]{figures/fedaya_pool.png}
    % \captionsetup{font=small,labelfont=bf}
    \vspace{-1.5em}
    \caption{The number of clients in each predominant language in our Fed-Aya setup.}
    \label{fig:fed-aya}
    \vspace{-2em}
\end{figure}

\subsubsection{Complementary Approaches}\label{sec:complementary}

We show \method{}'s compatibility with both off-the-shelf models and models trained using existing FL methods. Concretely, given a pretrained model, we obtain a \basemodel{} using one of the following approaches: 

\noindent\textbf{Standard FL.} We further train the pretrained model federatedly on the \seen{} pool, either using existing PEFT methods or full fine-tuning~\cite{fedllm-bench, fedpeft},  

\noindent\textbf{Personalized FL.} We adopt two recent personalized FL works: \textit{i)} FedDPA-T~\cite{FedDPA}, which learns per-client personalized LoRA modules in addition to global LoRA modules, and \textit{ii)} DEPT (SPEC)~\cite{DEPT}, which learns per-client personalized token and positional embeddings while keeping the rest of the model shared. The \basemodel{} hence differs for each client.

\noindent\textbf{Off-the-shelf.} We use the pretrained model as the \basemodel{} without additional training.

\begin{table*}[t]
\centering
% \captionsetup{justification=centering}
\caption{Mean±SD Accuracy of each language across 3 different seeds for \seen{} clients of our MasakhaNEWS setup. The pretrained model is trained using Standard FL with full fine-tuning and the resulting \basemodel{} is personalized to each client given a baseline approach.}
\label{tab:masakha_seen}
\begin{scriptsize}\resizebox{0.98\textwidth}{!}{
\begin{tabular}{c|l|l|l|l|l|l|l|l|l|l|l|l|l|l|l|l|l|c}
\toprule
% \textbf{Lora Rank}  
\textbf{$\mathbf{r}$} & \multicolumn{1}{c|}{\textbf{Approach}} & \multicolumn{1}{c|}{\textbf{eng}} & \multicolumn{1}{c|}{\textbf{som}} & \multicolumn{1}{c|}{\textbf{run}} & \multicolumn{1}{c|}{\textbf{fra}} & \multicolumn{1}{c|}{\textbf{lin}} & \multicolumn{1}{c|}{\textbf{ibo}} & \multicolumn{1}{c|}{\textbf{amh}} & \multicolumn{1}{c|}{\textbf{hau}} & \multicolumn{1}{c|}{\textbf{pcm}} & \multicolumn{1}{c|}{\textbf{swa}} & \multicolumn{1}{c|}{\textbf{orm}} & \multicolumn{1}{c|}{\textbf{xho}} & \multicolumn{1}{c|}{\textbf{yor}} & \multicolumn{1}{c|}{\textbf{sna}} & \multicolumn{1}{c|}{\textbf{lug}} & \multicolumn{1}{c|}{\textbf{tir}} & \textbf{Wins} 
\\ \midrule
% \multirow{5}{*}{1}  & LoRA                                   & 90.01±0.10                        & 59.41±0.32                        & 81.16±0.29                        & 88.63±0.00                        & 83.53±0.54                        & 78.97±0.00                        & 45.74±0.00                        & 75.58±0.15                        & 96.05±0.00                        & 78.99±0.34                        & 64.20±0.00                         & 69.37±0.32                        & 79.02±0.00                        & 78.80±0.00                         & 67.57±0.00                        & 44.85±0.00                        & 0             \\ % \cline{2-19} 
%                     & AdaLoRA                              & 89.87±0.00                        & 59.63±0.32                        & 80.96±0.29                        & 88.63±0.00                        & 83.14±0.54                        & 78.97±0.00                        & 45.21±0.00                        & 75.16±0.00                        & 96.05±0.00                        & 78.99±0.34                        & 64.20±0.00                         & 69.59±0.00                        & 79.02±0.00                        & 78.80±0.00                         & 67.87±0.42                        & 44.85±0.00                        & 0             \\ % \cline{2-19} 
%                     & BayesTune-LoRA                            & 89.87±0.00                        & 59.63±0.32                        & 81.16±0.29                        & 88.63±0.00                        & 83.91±0.00                        & 78.97±0.00                        & 45.21±0.00                        & 75.05±0.15                        & 96.05±0.00                        & 78.71±0.20                        & 64.20±0.00                         & 69.37±0.32                        & 78.86±0.23                        & 78.80±0.00                         & 67.57±0.00                        & 44.85±0.00                        & 0             \\ % \cline{2-19} 
%                     & FedL2P                               & 89.87±0.00                        & 59.86±0.00                        & 81.37±0.00                        & 88.63±0.00                        & 82.76±0.00                        & 79.49±0.00                        & 45.21±0.00                        & 75.68±0.15                        & 96.05±0.00                        & 78.85±0.20                        & 64.20±0.00                         & 69.37±0.32                        & 79.02±0.00                        & 78.80±0.00                         & 67.57±0.00                        & 44.85±0.00                        & \textbf{0}    \\ % \cline{2-19} 
%                     & \method{}                                 & \textbf{92.05±0.10}                & \textbf{64.63±0.56}               & \textbf{85.71±0.00}                & \textbf{91.31±0.22}               & \textbf{87.36±0.00}                & \textbf{80.69±0.49}               & \textbf{52.48±0.50}                & \textbf{79.35±0.30}                & 96.05±0.00                        & \textbf{84.87±0.69}               & \textbf{68.72±0.77}               & \textbf{73.42±0.64}               & \textbf{82.93±0.00}                & 78.08±0.25                        & \textbf{69.07±0.42}               & \textbf{52.45±0.34}               & 14            \\ \hline
\multirow{5}{*}{2}  & LoRA                                   & 90.44±0.10                        & 60.09±0.32                        & 81.37±0.51                        & 88.63±0.00                        & 83.53±0.54                        & 79.83±0.24                        & 45.74±0.00                        & 75.79±0.00                        & 96.05±0.00                        & 78.99±0.00                        & 64.20±0.00                         & 69.14±0.32                        & 79.18±0.23                        & 78.80±0.00                         & 67.57±0.00                        & 44.85±0.00                        & 0             \\ % \cline{2-19} 
                    & AdaLoRA                              & 89.87±0.00                        & 59.41±0.32                        & 81.16±0.29                        & 88.63±0.00                        & 82.76±0.00                        & 78.97±0.00                        & 45.21±0.00                        & 75.05±0.15                        & 96.05±0.00                        & 78.57±0.00                        & 63.99±0.29                        & 69.59±0.00                        & 78.86±0.23                        & 78.80±0.00                         & 67.57±0.00                        & 44.85±0.00                        & 0             \\ % \cline{2-19} 
                    & BayesTune-LoRA                            & 89.87±0.00                        & 59.63±0.32                        & 81.37±0.00                        & 88.63±0.00                        & 83.14±0.54                        & 78.97±0.00                        & 45.21±0.00                        & 75.16±0.00                        & 96.05±0.00                        & 78.57±0.00                        & 63.99±0.29                        & 69.59±0.00                        & 79.02±0.00                        & 78.80±0.00                         & 67.57±0.00                        & 44.85±0.00                        & 0             \\ % \cline{2-19} 
                    & FedL2P                               & 90.72±0.59                        & 61.00±1.16                         & 81.99±0.88                        & 89.10±0.67                         & 83.91±0.00                        & 79.66±0.24                        & 45.74±0.00                        & 76.73±1.12                        & 96.05±0.00                        & 79.69±0.40                        & 64.40±0.29                         & 69.14±0.32                        & 79.67±0.61                        & 78.80±0.00                         & 67.87±0.42                        & 45.34±0.69                        & 0             \\ % \cline{2-19} 
                    & \method{}                                 & \textbf{91.98±0.00}                & \textbf{65.54±1.16}               & \textbf{87.79±0.29}               & \textbf{93.52±0.23}               & \textbf{88.51±0.94}               & \textbf{82.56±0.00}                & \textbf{51.95±0.25}               & \textbf{79.66±0.29}               & \textbf{97.59±0.31}               & \textbf{84.73±0.52}               & \textbf{72.22±1.01}               & \textbf{76.35±0.55}               & \textbf{82.11±0.23}               & \textbf{81.34±0.68}               & \textbf{69.07±0.42}               & \textbf{63.48±0.34}               & \textbf{16}   \\ \hline
\multirow{5}{*}{4}  & LoRA                                   & 91.35±0.00                        & 61.90±0.00                         & 83.23±0.00                        & 89.10±0.00                         & 84.68±0.54                        & 80.17±0.24                        & 47.34±0.00                        & 77.57±0.15                        & 96.05±0.00                        & 79.83±0.34                        & 64.61±0.29                        & 69.37±0.32                        & 80.49±0.00                        & 78.80±0.00                         & 67.87±0.42                        & 45.83±0.34                        & 0             \\ % \cline{2-19} 
                    & AdaLoRA                              & 89.87±0.00                        & 59.63±0.32                        & 80.96±0.29                        & 88.63±0.00                        & 83.53±0.54                        & 78.97±0.00                        & 45.21±0.00                        & 75.47±0.00                        & 96.05±0.00                        & 78.57±0.00                        & 63.79±0.29                        & 69.59±0.00                        & 79.02±0.00                        & 78.80±0.00                         & 67.57±0.00                        & 44.85±0.00                        & 0             \\ % \cline{2-19} 
                    & BayesTune-LoRA                            & 89.87±0.00                        & 59.41±0.32                        & 80.96±0.29                        & 88.63±0.00                        & 82.76±0.00                        & 78.97±0.00                        & 45.21±0.00                        & 75.47±0.26                        & 96.05±0.00                        & 78.71±0.20                        & 64.20±0.00                         & 69.59±0.00                        & 79.02±0.00                        & 78.80±0.00                         & 67.57±0.00                        & 44.85±0.00                        & 0             \\ % \cline{2-19} 
                    & FedL2P                               & 91.70±0.40                         & 63.49±1.16                        & 83.23±0.51                        & 90.84±0.97                        & 83.91±0.94                        & 80.00±0.42                         & 47.69±1.40                        & 79.45±1.07                        & 96.05±0.00                        & 81.51±1.03                        & 64.81±0.50                        & 71.62±1.99                        & 81.46±0.80                        & 78.80±0.00                         & 68.17±0.85                        & 46.32±0.60                        & 0             \\ % \cline{2-19} 
                    & \method{}                                 & \textbf{92.05±0.44}               & \textbf{67.12±0.64}               & \textbf{86.34±0.00}                & \textbf{93.36±0.39}               & \textbf{91.19±1.43}               & \textbf{82.56±0.84}               & \textbf{53.90±0.25}                & \textbf{81.66±0.65}               & \textbf{98.46±0.31}               & \textbf{83.89±1.10}                & \textbf{72.84±0.51}               & \textbf{77.70±0.00}                 & \textbf{82.11±0.23}               & \textbf{83.70±0.00}                 & \textbf{74.47±0.42}               & \textbf{67.65±0.00}                & \textbf{16}   \\ \hline
\multirow{5}{*}{8}  & LoRA                                   & 91.56±0.00                        & 63.95±0.00                        & 82.82±0.29                        & 90.84±0.23                        & 86.21±0.00                        & 79.83±0.24                        & 49.82±0.25                        & 78.83±0.15                        & 96.05±0.00                        & 81.79±0.52                        & 65.64±0.29                        & 72.30±0.55                         & 81.79±0.23                        & 79.89±0.00                        & 67.87±0.42                        & 44.85±0.00                        & 0             \\ % \cline{2-19} 
                    & AdaLoRA                              & 89.87±0.00                        & 59.86±0.00                        & 80.96±0.29                        & 88.63±0.00                        & 83.14±0.54                        & 78.97±0.00                        & 45.21±0.00                        & 75.37±0.15                        & 96.05±0.00                        & 78.85±0.20                        & 64.20±0.00                         & 69.59±0.00                        & 78.86±0.23                        & 78.80±0.00                         & 67.57±0.00                        & 44.85±0.00                        & 0             \\ % \cline{2-19} 
                    & BayesTune-LoRA                            & 89.94±0.10                        & 59.41±0.32                        & 80.75±0.00                        & 88.63±0.00                        & 83.14±0.54                        & 78.97±0.00                        & 45.21±0.00                        & 75.47±0.26                        & 96.05±0.00                        & 78.71±0.20                        & 64.20±0.00                         & 69.59±0.00                        & 79.02±0.00                        & 78.80±0.00                         & 67.57±0.00                        & 44.85±0.00                        & 0             \\ % \cline{2-19} 
                    & FedL2P                               & \textbf{91.91±0.26}               & 64.18±0.32                        & 83.02±0.58                        & 91.15±0.59                        & 84.29±0.54                        & 80.51±0.00                        & 48.58±0.67                        & 79.98±0.82                        & 96.27±0.31                        & 82.49±0.79                        & 66.25±1.62                        & 72.07±1.39                        & 81.95±1.05                        & 79.16±0.51                        & 68.47±1.27                        & 47.79±3.65                        & 1             \\ % \cline{2-19} 
                    & \method{}                                 & 91.42±0.20                        & \textbf{68.03±0.96}               & \textbf{85.09±0.00}                & \textbf{92.42±0.39}               & \textbf{91.95±0.00}                & \textbf{81.88±0.24}               & \textbf{55.32±0.43}               & \textbf{82.70±0.26}                & \textbf{98.68±0.00}                & \textbf{84.31±0.20}                & \textbf{73.25±0.58}               & \textbf{78.15±0.32}               & \textbf{84.72±0.46}               & \textbf{85.15±0.26}               & \textbf{74.17±0.42}               & \textbf{68.38±0.60}                & \textbf{15}   \\ \hline
\multirow{5}{*}{16} & LoRA                                   & \textbf{91.91±0.10}                & 64.63±0.00                        & 84.47±0.00                        & 91.47±0.00                        & 86.21±0.00                        & \textbf{81.54±0.00}                         & \textbf{55.85±0.43}               & 81.87±0.15                        & 96.27±0.31                        & \textbf{84.45±0.00}                & 73.66±0.77                        & 73.20±0.32                         & 82.60±0.61                         & 80.80±0.26                         & 70.87±0.42                        & 52.94±0.00                        & 3             \\ % \cline{2-19} 
                    & AdaLoRA                              & 89.87±0.00                        & 59.86±0.00                        & 80.96±0.29                        & 88.63±0.00                        & 82.76±0.00                        & 78.97±0.00                        & 45.21±0.00                        & 75.05±0.39                        & 96.05±0.00                        & 78.85±0.20                        & 63.79±0.29                        & 69.59±0.00                        & 79.02±0.00                        & 78.80±0.00                         & 67.57±0.00                        & 44.85±0.00                        & 0             \\ % \cline{2-19} 
                    & BayesTune-LoRA                            & 90.08±0.18                        & 59.63±0.32                        & 81.16±0.29                        & 88.63±0.00                        & 83.53±0.54                        & 78.97±0.00                        & 45.21±0.00                        & 75.58±0.15                        & 96.05±0.00                        & 78.57±0.00                        & 64.20±0.00                         & 69.37±0.32                        & 79.02±0.00                        & 78.80±0.00                         & 67.57±0.00                        & 44.85±0.00                        & 0             \\ % \cline{2-19} 
                    & FedL2P                               & 91.70±0.10                         & 66.22±0.32                        & \textbf{85.09±0.51}               & \textbf{92.10±0.23}                & 86.21±0.00                        & 80.34±0.24                        & 54.79±1.15                        & 81.13±0.26                        & 96.71±0.00                        & 83.89±0.40                        & 71.81±2.04                        & 75.68±0.00                        & 84.06±0.23                        & 80.61±0.68                        & 70.87±0.85                        & 57.84±4.86                        & 2             \\ % \cline{2-19} 
                    & \method{}                                 & 91.14±0.30                        & \textbf{67.80±0.85}                & 84.47±0.00                        & 91.31±0.22                        & \textbf{91.95±0.00}                & \textbf{81.54±0.42}                        & 51.95±0.25                        & \textbf{83.44±0.39}               & \textbf{98.25±0.31}               & 82.49±0.71                        & \textbf{76.54±0.50}                & \textbf{79.50±0.32}                & \textbf{84.39±0.00}                & \textbf{85.87±0.00}                & \textbf{75.38±0.43}               & \textbf{68.38±0.00}                & \textbf{10}   \\ \bottomrule
\end{tabular}
}
\end{scriptsize}
\vspace{-1.5em}
\end{table*}


\begin{table*}[t]
\centering
\caption{Mean±SD Accuracy of each language for \unseen{} clients of our MasakhaNEWS setup. The pretrained model is trained using Standard FL with full fine-tuning and the resulting \basemodel{} is personalized to each client given a baseline approach.}
\label{tab:masakha_unseen}
\resizebox{0.99\textwidth}{!}{
\begin{tabular}{c|l|l|l|l|l|l|l|l|l|l|l|l|l|l|l|l|l|c}
\toprule
% \textbf{Lora Rank} 
\textbf{$\mathbf{r}$} & \multicolumn{1}{c|}{\textbf{Approach}} & \multicolumn{1}{c|}{\textbf{eng}} & \multicolumn{1}{c|}{\textbf{som}} & \multicolumn{1}{c|}{\textbf{run}} & \multicolumn{1}{c|}{\textbf{fra}} & \multicolumn{1}{c|}{\textbf{lin}} & \multicolumn{1}{c|}{\textbf{ibo}} & \multicolumn{1}{c|}{\textbf{amh}} & \multicolumn{1}{c|}{\textbf{hau}} & \multicolumn{1}{c|}{\textbf{pcm}} & \multicolumn{1}{c|}{\textbf{swa}} & \multicolumn{1}{c|}{\textbf{orm}} & \multicolumn{1}{c|}{\textbf{xho}} & \multicolumn{1}{c|}{\textbf{yor}} & \multicolumn{1}{c|}{\textbf{sna}} & \multicolumn{1}{c|}{\textbf{lug}} & \multicolumn{1}{c|}{\textbf{tir}} & \textbf{Wins}
\\ \midrule
% \multirow{5}{*}{1}  & LoRA                                   & \textbf{90.58±0.20}                & 67.80±0.32                         & 81.37±0.00                        & 84.36±0.00                        & 79.55±0.00                        & 76.92±0.00                        & 46.28±0.00                        & 74.82±0.15                        & 90.20±0.00                         & 75.63±0.00                        & 61.96±0.00                        & 63.76±0.00                        & 79.61±0.00                        & 74.59±0.00                        & 65.18±0.00                        & 41.91±0.00                        & 1             \\ % \cline{2-19} 
%                     & AdaLoRA                              & 90.44±0.27                        & 67.12±0.32                        & 80.96±0.29                        & 84.36±0.00                        & 79.55±0.00                        & 76.92±0.00                        & 46.28±0.00                        & 74.82±0.15                        & 90.20±0.00                         & 75.77±0.20                        & 61.96±0.00                        & 63.98±0.32                        & 79.61±0.00                        & 74.59±0.00                        & 65.18±0.00                        & 41.91±0.00                        & 0             \\ % \cline{2-19} 
%                     & BayesTune-LoRA                            & 90.30±0.00                         & 67.35±0.00                        & 80.75±0.00                        & 84.36±0.00                        & 79.55±0.00                        & 76.92±0.00                        & 46.10±0.25                         & 74.82±0.15                        & 90.20±0.00                         & 75.63±0.00                        & 61.55±0.29                        & 63.76±0.00                        & 79.61±0.00                        & 74.59±0.00                        & 65.18±0.00                        & 41.91±0.00                        & 0             \\ % \cline{2-19} 
%                     & FedL2P                               & 90.15±0.27                        & 67.35±0.56                        & 81.58±0.29                        & 84.36±0.00                        & 79.55±0.00                        & 76.92±0.00                        & 46.28±0.00                        & 75.24±0.00                        & 90.20±0.00                         & 75.49±0.20                        & 61.96±0.00                        & 63.76±0.00                        & 79.94±0.23                        & 74.59±0.00                        & 65.48±0.42                        & 42.16±0.35                        & 0             \\ % \cline{2-19} 
%                     & \method{}                                 & 90.01±0.10                        & \textbf{69.84±0.32}               & \textbf{84.47±0.00}                & \textbf{86.73±0.00}                & \textbf{84.85±0.54}               & \textbf{80.52±0.73}               & \textbf{50.53±1.15}               & \textbf{77.53±0.15}               & \textbf{91.50±0.00}                 & \textbf{76.05±0.34}               & \textbf{69.94±0.87}               & \textbf{71.14±0.55}               & \textbf{82.69±0.61}               & \textbf{75.50±0.25}                & \textbf{66.66±0.42}               & \textbf{48.77±0.34}               & \textbf{15}   \\ \hline
\multirow{5}{*}{2}  & LoRA                                   & \textbf{90.72±0.00}                & 68.48±0.32                        & 81.99±0.51                        & 84.36±0.00                        & 79.55±0.00                        & 76.75±0.24                        & 46.28±0.00                        & 75.45±0.15                        & 90.20±0.00                         & 75.63±0.34                        & 61.96±0.00                        & 64.21±0.63                        & 80.10±0.00                         & 74.59±0.00                        & 65.18±0.00                        & 41.91±0.00                        & 1             \\ % \cline{2-19} 
                    & AdaLoRA                              & 90.30±0.00                         & 67.35±0.00                        & 80.75±0.00                        & 84.36±0.00                        & 79.55±0.00                        & 76.92±0.00                        & 46.28±0.00                        & 74.82±0.15                        & 90.20±0.00                         & 75.91±0.20                        & 61.55±0.29                        & 64.21±0.32                        & 79.61±0.00                        & 74.59±0.00                        & 65.18±0.00                        & 41.91±0.00                        & 0             \\ % \cline{2-19} 
                    & BayesTune-LoRA                            & 90.30±0.00                         & 67.35±0.00                        & 80.75±0.00                        & 84.36±0.00                        & 79.55±0.00                        & 76.92±0.00                        & 45.92±0.25                        & 74.61±0.00                        & 90.20±0.00                         & 75.49±0.20                        & 61.55±0.29                        & 63.76±0.00                        & 79.45±0.23                        & 74.41±0.25                        & 65.18±0.00                        & 42.65±0.00                        & 0             \\ % \cline{2-19} 
                    & FedL2P                               & 90.58±0.10                        & 68.48±0.32                        & 82.61±0.00                        & 84.36±0.00                        & 79.55±0.00                        & 76.41±0.42                        & 46.81±0.75                        & 75.97±0.59                        & 90.20±0.00                         & 75.91±0.40                        & 62.17±0.29                        & 64.43±0.95                        & 80.26±0.23                        & 74.77±0.26                        & 66.07±0.00                        & 41.91±0.00                        & 0             \\ % \cline{2-19} 
                    & \method{}                                 & 90.65±0.10                        & \textbf{72.11±0.56}               & \textbf{88.41±0.29}               & \textbf{88.15±0.39}               & \textbf{86.36±0.00}                & \textbf{81.54±0.42}               & \textbf{51.06±0.00}                & \textbf{79.52±0.30}                & \textbf{93.68±0.31}               & \textbf{78.01±0.40}                & \textbf{73.01±0.50}                & \textbf{78.52±1.10}                & \textbf{83.66±0.23}               & \textbf{80.00±0.44}                & \textbf{69.64±0.00}                & \textbf{58.33±0.34}               & \textbf{15}   \\ \hline
\multirow{5}{*}{4}  & LoRA                                   & 90.79±0.10                        & 68.71±0.00                        & 82.61±0.00                        & 84.36±0.00                        & 78.41±0.00                        & 76.41±0.00                        & 47.87±0.00                        & 75.76±0.30                        & 90.20±0.00                         & 76.05±0.00                        & 63.60±0.29                         & 65.55±0.32                        & 80.58±0.00                        & 74.77±0.26                        & 65.48±0.42                        & 41.91±0.00                        & 0             \\ % \cline{2-19} 
                    & AdaLoRA                              & 90.15±0.10                        & 67.35±0.00                        & 81.58±0.29                        & 84.20±0.22                         & 79.55±0.00                        & 76.92±0.00                        & 46.46±0.25                        & 74.71±0.15                        & 90.20±0.00                         & 75.77±0.20                        & 61.76±0.29                        & 63.76±0.00                        & 79.61±0.00                        & 74.59±0.00                        & 65.18±0.00                        & 42.16±0.35                        & 0             \\ % \cline{2-19} 
                    & BayesTune-LoRA                            & 90.23±0.10                        & 67.12±0.32                        & 80.75±0.00                        & 84.36±0.00                        & 79.55±0.00                        & 76.92±0.00                        & 46.28±0.00                        & 74.71±0.15                        & 90.20±0.00                         & 75.21±0.00                        & 61.35±0.00                        & 63.76±0.00                        & 79.61±0.00                        & 74.59±0.00                        & 65.18±0.00                        & 42.40±0.35                         & 0             \\ % \cline{2-19} 
                    & FedL2P                               & 90.82±0.11                        & 68.37±0.34                        & 83.54±0.93                        & 85.54±1.19                        & 80.12±0.57                        & 76.41±0.51                        & 48.67±0.80                        & 77.43±0.63                        & 90.20±0.00                         & 76.68±0.21                        & 63.80±1.23                         & 67.78±2.02                        & 81.31±0.73                        & 75.95±0.81                        & 66.51±0.45                        & 43.75±1.84                        & 0             \\ % \cline{2-19} 
                    & \method{}                                 & \textbf{90.86±0.36}               & \textbf{72.56±0.32}               & \textbf{86.96±0.00}                & \textbf{91.00±0.39}                & \textbf{87.88±0.54}               & \textbf{81.37±0.48}               & \textbf{52.13±0.00}                & \textbf{81.61±0.15}               & \textbf{94.77±0.00}                & \textbf{79.13±0.20}                & \textbf{74.03±0.29}               & \textbf{80.32±0.32}               & \textbf{83.50±0.00}                 & \textbf{84.86±0.00}                & \textbf{77.08±0.84}               & \textbf{68.87±0.35}               & \textbf{16}   \\ \hline
\multirow{5}{*}{8}  & LoRA                                   & 90.65±0.10                        & 68.71±0.00                        & 83.02±0.29                        & 84.83±0.00                        & 80.68±0.00                        & 77.95±0.00                        & 50.00±0.00                         & 77.43±0.00                        & 90.20±0.00                         & 76.89±0.00                        & 64.42±0.00                        & 66.89±0.32                        & 82.04±0.00                        & 76.58±0.25                        & 66.07±0.00                        & 43.38±0.00                        & 0             \\ % \cline{2-19} 
                    & AdaLoRA                              & 90.23±0.10                        & 67.35±0.00                        & 80.96±0.29                        & 84.36±0.00                        & 79.55±0.00                        & 76.92±0.00                        & 46.28±0.00                        & 74.71±0.15                        & 90.20±0.00                         & 75.63±0.00                        & 61.76±0.29                        & 63.54±0.32                        & 79.45±0.23                        & 74.59±0.00                        & 65.18±0.00                        & 42.16±0.35                        & 0             \\ % \cline{2-19} 
                    & BayesTune-LoRA                            & 90.23±0.27                        & 67.35±0.00                        & 81.37±0.51                        & 84.20±0.22                         & 79.55±0.00                        & 76.92±0.00                        & 46.28±0.00                        & 74.61±0.00                        & 90.20±0.00                         & 75.35±0.20                        & 61.55±0.29                        & 63.98±0.32                        & 79.29±0.23                        & 74.59±0.00                        & 65.18±0.00                        & 42.40±0.35                         & 0             \\ % \cline{2-19} 
                    & FedL2P                               & \textbf{90.79±0.10}                & 68.94±0.32                        & 83.85±0.51                        & 85.62±0.81                        & 80.31±1.07                        & 77.43±0.73                        & 49.29±1.25                        & 77.95±0.97                        & 90.42±0.31                        & 76.89±0.00                        & 65.03±1.80                        & 68.46±1.45                        & 81.72±0.46                        & 76.94±1.02                        & 66.67±0.84                        & 45.10±1.93                         & 1             \\ % \cline{2-19} 
                    & \method{}                                 & 90.37±0.36                        & \textbf{73.70±0.32}                & \textbf{89.23±0.29}               & \textbf{93.36±0.39}               & \textbf{89.39±0.53}               & \textbf{79.66±0.64}               & \textbf{51.42±0.67}               & \textbf{81.09±0.29}               & \textbf{96.08±0.00}                & \textbf{78.99±0.34}               & \textbf{77.10±0.29}                & \textbf{78.52±0.55}               & \textbf{82.85±0.23}               & \textbf{86.49±0.00}                & \textbf{77.38±0.42}               & \textbf{67.65±0.00}                & \textbf{15}   \\ \hline
\multirow{5}{*}{16} & LoRA                                   & 90.93±0.00                        & 70.75±0.00                        & 85.09±0.00                        & 87.36±0.23                        & 82.95±0.00                        & \textbf{80.51±0.00}                         & 50.35±0.25                        & 79.83±0.15                        & 90.85±0.00                        & 76.89±0.00                        & 71.78±0.00                        & 71.81±0.95                        & \textbf{83.17±0.23}               & 80.18±0.67                        & 67.86±0.00                        & 51.96±0.69                        & 1             \\ % \cline{2-19} 
                    & AdaLoRA                              & 90.15±0.10                        & 67.58±0.32                        & 80.75±0.00                        & 84.36±0.00                        & 79.55±0.00                        & 76.92±0.00                        & 46.10±0.25                         & 74.71±0.15                        & 90.20±0.00                         & 75.63±0.34                        & 61.55±0.29                        & 63.98±0.32                        & 79.61±0.00                        & 74.59±0.00                        & 65.18±0.00                        & 42.40±0.35                         & 0             \\ % \cline{2-19} 
                    & BayesTune-LoRA                            & 90.58±0.10                        & 67.35±0.00                        & 81.16±0.29                        & 84.36±0.00                        & 79.55±0.00                        & 76.92±0.00                        & 46.10±0.25                         & 74.61±0.00                        & 90.20±0.00                         & 75.63±0.34                        & 61.35±0.00                        & 63.98±0.32                        & 79.45±0.23                        & 74.59±0.00                        & 65.18±0.00                        & 42.16±0.35                        & 0             \\ % \cline{2-19} 
                    & FedL2P                               & \textbf{91.14±0.30}                & 71.20±0.32                         & 87.17±0.77                        & 87.20±0.67                         & 83.71±0.54                        & 79.32±0.25                        & \textbf{51.42±0.67}               & 79.94±0.26                        & 91.72±0.31                        & 77.73±0.91                        & 70.55±2.18                        & 75.17±3.34                        & 82.36±0.23                        & 79.46±0.44                        & 71.43±1.26                        & 55.64±4.81                        & 2             \\ % \cline{2-19} 
                    & \method{}                                 & 88.33±0.70                        & \textbf{74.83±0.56}               & \textbf{88.41±0.29}               & \textbf{93.68±0.23}               & \textbf{89.39±0.53}               & \textbf{80.51±0.00}                         & 51.06±0.44                        & \textbf{80.15±0.29}               & \textbf{96.51±0.31}               & \textbf{80.67±0.00}                & \textbf{78.94±1.04}               & \textbf{77.63±0.32}               & 82.20±0.23                         & \textbf{85.95±0.44}               & \textbf{77.38±0.84}               & \textbf{68.87±0.69}               & \textbf{12}   \\ \bottomrule
\end{tabular}
}
\vspace{-1.2em}
\end{table*}

\subsubsection{Baselines}\label{sec:baselines}

Given a \basemodel{}, we compare \method{} with existing fine-tuning and {\em learning to personalize} approaches. 

\noindent\textbf{LoRA PEFT.}~We deploy LoRA~\cite{hu2021lora} on all linear layers of the model with a fixed rank $r$. 

\noindent\textbf{Non-FL Rank Selection.}~We compare with AdaLoRA~\cite{adalora} and our proposed LoRA-variant of BayesTune~\cite{kim2023bayestune}, BayesTune-LoRA (Section~\ref{sec:personalized_peft}), which optimizes $\bm{\lambda}$ separately for each client. 

\noindent\textbf{FL to Personalize.}~We compare with FedL2P~\cite{royson2023fedl2p} which trains a MLP federatedly to output per-client learning rates for each LoRA module.

For each baseline, we either follow best practices recommended by the corresponding authors or employ a simple grid search and pick the best performing hyperparameters. Full details in Appendix.~\ref{appendix:experiments}.

\subsection{Results on Text Classification}\label{sec:text_class}

We evaluate our approach in a typical FL setup, where the pretrained model is first trained using Standard FL with full fine-tuning and the resulting \basemodel{} is then personalized to each client. Tables~\ref{tab:masakha_seen} \& \ref{tab:masakha_unseen} show the mean and standard deviation (SD) of the accuracy for each language in our MasakhaNEWS setup for \seen{} and \unseen{} pool respectively (similarly for XNLI in Appendix Tables~\ref{tab:xnli_seen} \& \ref{tab:xnli_unseen}). In addition, we also show the number of languages, labelled ``Wins", an approach is best performing for each budget $r$. 

The results in all four tables show that federated {\em learning to personalize} methods (FedL2P and \method{}) outperform the other baselines in most cases. Non-FL rank selection approaches (AdaLoRA and BayesTune-LoRA), on the other hand, tend to overfit and/or struggle to learn an optimal rank structure given the limited number of samples in each client. Comparing FedL2P and \method{}, \method{} largely surpass FedL2P with a few exceptions, indicating that learning to personalize LoRA rank structure is the better hyperparameter choice than personalizing learning rates; this finding is also aligned with recent LLM-based optimizer findings~\cite{zhao2025deconstructing}, which shows that Adam's performance is robust with respect to its learning rate.

\subsubsection{\method{}'s Complementability with Personalized FL Works.}\label{sec:eval_complement}

% \begin{table*}[]
% \centering
% \caption{Mean Accuracy ± standard deviation on each language and overall win rate on XNLI dataset, with FedPDA, seen pool}
% \label{tab:xnli_seen_fedpda}
% \begin{scriptsize}\resizebox{0.98\textwidth}{!}{
% \begin{tabular}{c|l|l|l|l|l|l|l|l|l|l|l|l|l|l|l|l|c}
% \toprule

\begin{table*}[]
\centering
\caption{Mean±SD Accuracy of each language across 3 different seeds for clients in the \seen{} pool of our XNLI setup. The pretrained model is trained using FedDPA-T and the resulting \basemodel{} is personalized to each client given a baseline approach.}
\label{tab:xnli_seen_feddpa}
\begin{scriptsize}\resizebox{0.98\textwidth}{!}{

\begin{tabular}{c|l|l|l|l|l|l|l|l|l|l|l|l|l|l|l|l|c}
\toprule
% \textbf{Lora Rank}  
\textbf{$\mathbf{r}$} & \multicolumn{1}{c|}{\textbf{Approach}} & \multicolumn{1}{c|}{\textbf{bg}} & \multicolumn{1}{c|}{\textbf{hi}} & \multicolumn{1}{c|}{\textbf{es}} & \multicolumn{1}{c|}{\textbf{el}} & \multicolumn{1}{c|}{\textbf{vi}} & \multicolumn{1}{c|}{\textbf{tr}} & \multicolumn{1}{c|}{\textbf{de}} & \multicolumn{1}{c|}{\textbf{ur}} & \multicolumn{1}{c|}{\textbf{en}} & \multicolumn{1}{c|}{\textbf{zh}} & \multicolumn{1}{c|}{\textbf{th}} & \multicolumn{1}{c|}{\textbf{sw}} & \multicolumn{1}{c|}{\textbf{ar}} & \multicolumn{1}{c|}{\textbf{fr}} & \multicolumn{1}{c|}{\textbf{ru}} & \textbf{Wins} \\ \midrule
% \multirow{5}{*}{1}  & LoRA                                   & 44.60±0.00                        & 41.73±0.09                       & 47.73±0.19                       & 50.20±0.00                        & 52.53±0.19                       & 47.00±0.16                        & 48.07±0.09                       & 40.60±0.00                        & 45.40±0.00                        & 43.33±0.09                       & 41.07±0.09                       & 50.80±0.00                        & 45.27±0.09                       & 46.33±0.19                       & 49.20±0.00                        & 0             \\ %\cline{2-18} 
%                     & AdaLoRA                              & 44.07±0.09                       & 41.20±0.00                        & 47.47±0.09                       & 50.20±0.00                        & 52.40±0.00                        & 46.53±0.09                       & 47.93±0.09                       & 38.87±0.09                       & 44.60±0.00                        & 42.27±0.09                       & 40.67±0.09                       & 50.40±0.00                        & 45.00±0.00                        & 46.00±0.00                        & 48.80±0.00                        & 0             \\ %\cline{2-18} 
%                     & BayesTune-LoRA                            & 43.67±0.09                       & 40.40±0.00                        & 47.20±0.00                        & 50.00±0.00                        & 52.33±0.09                       & 46.27±0.09                       & 47.40±0.16                        & 38.80±0.00                        & 43.93±0.09                       & 41.33±0.09                       & 40.53±0.09                       & 49.80±0.00                        & 44.40±0.00                        & 46.00±0.00                        & 47.93±0.09                       & 0             \\ %\cline{2-18} 
%                     & FedL2P                               & 44.33±0.19                       & 41.33±0.19                       & 47.67±0.09                       & 50.20±0.00                        & 52.47±0.09                       & 46.93±0.19                       & 47.93±0.19                       & 40.20±0.28                        & 45.20±0.00                        & 42.93±0.19                       & 40.93±0.09                       & 50.67±0.09                       & 45.27±0.09                       & 46.33±0.19                       & 48.80±0.00                        & 0             \\ %\cline{2-18} 
%                     & \method{}                                 & \textbf{58.00±0.16}               & \textbf{54.93±0.09}              & \textbf{55.33±0.09}              & \textbf{55.33±0.19}              & \textbf{54.80±0.33}               & \textbf{54.20±0.16}               & \textbf{55.80±0.16}               & \textbf{60.33±0.34}              & \textbf{55.53±0.25}              & \textbf{55.67±0.09}              & \textbf{58.07±0.62}              & \textbf{54.33±0.19}              & \textbf{54.93±0.09}              & \textbf{57.07±0.38}              & \textbf{54.53±0.34}              & \textbf{15}   \\ \hline
\multirow{5}{*}{2}  & LoRA                                   & 45.80±0.28                        & 42.80±0.16                        & 48.73±0.25                       & 50.87±0.09                       & 53.00±0.00                        & 48.00±0.33                        & 49.87±0.09                       & 41.93±0.09                       & 46.53±0.34                       & 44.40±0.16                        & 42.53±0.25                       & 51.80±0.16                        & 46.93±0.25                       & 48.07±0.19                       & 50.53±0.25                       & 0             \\ %\cline{2-18} 
                    & AdaLoRA                              & 44.07±0.09                       & 41.20±0.00                        & 47.47±0.09                       & 50.00±0.00                        & 52.40±0.00                        & 46.53±0.09                       & 48.00±0.00                        & 38.80±0.00                        & 44.67±0.09                       & 42.20±0.00                        & 40.67±0.09                       & 50.40±0.00                        & 45.00±0.00                        & 46.00±0.00                        & 48.80±0.00                        & 0             \\ %\cline{2-18} 
                    & BayesTune-LoRA                            & 43.80±0.00                        & 40.60±0.00                        & 47.20±0.00                        & 50.00±0.00                        & 52.40±0.00                        & 46.20±0.00                        & 47.40±0.16                        & 38.80±0.00                        & 44.27±0.09                       & 41.60±0.00                        & 40.53±0.09                       & 49.80±0.00                        & 44.73±0.09                       & 46.00±0.00                        & 48.07±0.19                       & 0             \\ %\cline{2-18} 
                    & FedL2P                               & 47.47±2.78                       & 44.53±3.18                       & 50.27±2.94                       & 51.47±1.39                       & 53.60±0.71                        & 50.07±2.81                       & 50.93±2.62                       & 44.80±4.53                        & 49.07±4.35                       & 46.53±4.01                       & 44.40±3.68                        & 52.27±1.27                       & 48.40±2.97                        & 49.47±2.96                       & 51.60±2.27                        & 0             \\ %\cline{2-18} 
                    & \method{}                                 & \textbf{64.40±0.16}               & \textbf{57.80±1.34}               & \textbf{58.53±1.32}              & \textbf{59.73±2.58}              & \textbf{60.80±0.71}               & \textbf{58.87±2.87}              & \textbf{55.00±0.16}               & \textbf{63.67±0.19}              & \textbf{55.93±0.19}              & \textbf{56.27±0.34}              & \textbf{58.33±0.25}              & \textbf{59.47±0.47}              & \textbf{55.20±0.16}               & \textbf{57.53±0.19}              & \textbf{55.20±0.43}               & \textbf{15}   \\ \hline
\multirow{5}{*}{4}  & LoRA                                   & 48.73±0.38                       & 46.07±0.66                       & 53.27±0.09                       & 52.53±0.09                       & 53.33±0.09                       & 50.27±0.25                       & 52.80±0.28                        & 46.13±0.25                       & 51.67±0.25                       & 48.53±0.19                       & 45.47±0.19                       & 53.33±0.09                       & 50.47±0.41                       & 50.27±0.09                       & 52.87±0.19                       & 0             \\ %\cline{2-18} 
                    & AdaLoRA                              & 44.00±0.00                        & 41.13±0.09                       & 47.40±0.00                        & 50.00±0.00                        & 52.40±0.00                        & 46.40±0.00                        & 47.67±0.09                       & 38.80±0.00                        & 44.47±0.09                       & 41.93±0.09                       & 40.60±0.00                        & 50.27±0.09                       & 44.87±0.09                       & 46.00±0.00                        & 48.80±0.00                        & 0             \\ %\cline{2-18} 
                    & BayesTune-LoRA                            & 44.00±0.00                        & 41.00±0.00                        & 47.20±0.00                        & 50.00±0.00                        & 52.40±0.00                        & 46.33±0.09                       & 47.67±0.09                       & 38.80±0.00                        & 44.47±0.09                       & 41.80±0.00                        & 40.60±0.00                        & 50.00±0.00                        & 44.80±0.00                        & 46.00±0.00                        & 48.67±0.09                       & 0             \\ %\cline{2-18} 
                    & FedL2P                               & 48.67±2.36                       & 45.20±2.26                        & 52.13±1.20                        & 51.87±0.52                       & 53.87±0.34                       & 50.27±0.84                       & 52.00±0.99                        & 47.00±3.69                        & 50.73±2.59                       & 47.27±1.95                       & 45.07±1.65                       & 52.67±0.52                       & 49.67±1.59                       & 51.20±2.26                        & 52.47±1.37                       & 0             \\ %\cline{2-18} 
                    & \method{}                                 & \textbf{64.47±1.18}              & \textbf{64.00±0.98}               & \textbf{64.40±0.91}               & \textbf{63.07±0.41}              & \textbf{64.13±0.34}              & \textbf{64.87±0.68}              & \textbf{63.33±1.67}              & \textbf{64.47±0.62}              & \textbf{56.60±0.28}               & \textbf{63.93±1.16}              & \textbf{62.60±1.73}               & \textbf{65.00±0.33}               & \textbf{64.13±0.81}              & \textbf{61.73±1.23}              & \textbf{61.80±0.65}               & \textbf{15}   \\ \hline
\multirow{5}{*}{8}  & LoRA                                   & 55.80±0.16                        & 51.73±0.25                       & 55.73±0.09                       & 55.07±0.25                       & 54.40±0.00                        & 52.80±0.16                        & 54.60±0.00                        & 57.47±0.09                       & 55.53±0.09                       & 54.00±0.33                        & 51.93±0.09                       & 54.13±0.09                       & 53.07±0.09                       & 56.27±0.09                       & 54.60±0.16                        & 0             \\ %\cline{2-18} 
                    & AdaLoRA                              & 43.80±0.00                        & 40.87±0.09                       & 47.27±0.09                       & 50.00±0.00                        & 52.40±0.00                        & 46.27±0.09                       & 47.73±0.09                       & 38.80±0.00                        & 44.40±0.00                        & 41.87±0.09                       & 40.60±0.00                        & 50.00±0.00                        & 44.80±0.00                        & 46.00±0.00                        & 48.40±0.16                        & 0             \\ %\cline{2-18} 
                    & BayesTune-LoRA                            & 44.13±0.09                       & 41.20±0.00                        & 47.47±0.09                       & 50.20±0.00                        & 52.40±0.00                        & 46.47±0.09                       & 47.93±0.09                       & 39.13±0.09                       & 44.93±0.09                       & 42.53±0.09                       & 40.87±0.09                       & 50.47±0.09                       & 45.13±0.09                       & 46.13±0.09                       & 48.80±0.00                        & 0             \\ %\cline{2-18} 
                    & FedL2P                               & 52.20±0.59                        & 48.73±0.25                       & 54.20±0.16                        & 53.13±0.19                       & 54.33±0.09                       & 51.53±0.19                       & 53.60±0.16                        & 51.00±0.86                        & 54.27±0.25                       & 51.73±0.25                       & 47.07±0.52                       & 53.60±0.16                        & 51.87±0.19                       & 52.93±0.34                       & 54.60±0.16                        & 0             \\ %\cline{2-18} 
                    & \method{}                                 & \textbf{67.27±0.25}              & \textbf{66.60±0.33}               & \textbf{68.87±0.41}              & \textbf{65.00±0.43}               & \textbf{66.73±0.09}              & \textbf{67.80±0.28}               & \textbf{67.13±0.19}              & \textbf{67.60±0.16}               & \textbf{64.00±0.99}               & \textbf{67.67±0.38}              & \textbf{66.53±0.09}              & \textbf{67.33±0.25}              & \textbf{67.60±0.33}               & \textbf{66.53±0.25}              & \textbf{69.33±0.19}              & \textbf{15}   \\ \hline
\multirow{5}{*}{16} & LoRA                                   & 64.73±0.09                       & 64.27±0.19                       & 65.60±0.28                        & \textbf{65.33±0.19}              & 63.80±0.00                        & 64.27±0.09                       & 64.20±0.71                        & 64.80±0.16                        & 57.93±2.46                       & 65.93±0.25                       & 58.33±0.52                       & 65.13±0.19                       & 60.53±0.62                       & 64.53±0.47                       & 61.73±0.74                       & 1             \\ %\cline{2-18} 
                    & AdaLoRA                              & 43.87±0.09                       & 40.73±0.09                       & 47.20±0.00                        & 50.00±0.00                        & 52.40±0.00                        & 46.33±0.09                       & 47.67±0.09                       & 38.80±0.00                        & 44.27±0.09                       & 41.80±0.00                        & 40.60±0.00                        & 50.00±0.00                        & 44.80±0.00                        & 46.00±0.00                        & 48.27±0.09                       & 0             \\ %\cline{2-18} 
                    & BayesTune-LoRA                            & 44.60±0.00                        & 41.67±0.19                       & 47.80±0.00                        & 50.20±0.00                        & 52.87±0.09                       & 47.13±0.09                       & 48.33±0.19                       & 40.60±0.00                        & 45.40±0.00                        & 43.60±0.00                        & 41.00±0.00                        & 50.80±0.00                        & 45.33±0.09                       & 46.60±0.00                        & 49.40±0.16                        & 0             \\ %\cline{2-18} 
                    & FedL2P                               & 54.47±1.86                       & 52.73±0.50                        & 55.80±0.43                        & 54.73±0.47                       & 55.67±1.09                       & 54.80±0.00                        & 54.73±0.25                       & 57.67±2.78                       & 55.67±0.09                       & 54.93±0.41                       & 52.60±2.79                        & 53.93±0.41                       & 53.47±0.25                       & 55.60±2.12                        & 55.20±0.00                        & 0             \\ %\cline{2-18} 
                    & \method{}                                 & \textbf{67.87±0.74}              & \textbf{67.00±0.43}               & \textbf{69.53±0.57}              & 64.93±0.41                       & \textbf{67.00±0.33}               & \textbf{68.00±0.00}                & \textbf{67.20±0.43}               & \textbf{68.00±0.59}               & \textbf{65.20±1.07}               & \textbf{67.60±0.16}               & \textbf{66.60±0.57}               & \textbf{67.33±0.09}              & \textbf{67.80±0.28}               & \textbf{66.13±0.41}              & \textbf{69.87±0.25}              & \textbf{14}   \\ \bottomrule
\end{tabular}
}
\end{scriptsize}
\vspace{-1.5em}
\end{table*}
\begin{table*}[]
\caption{Mean±SD Accuracy of each language across 3 different seeds for clients in the \seen{} pool of our XNLI setup. The pretrained model is trained using DEPT(SPEC) and the resulting \basemodel{} is personalized to each client given a baseline approach.}
\label{tab:xnli_seen_dept}
\begin{scriptsize}\resizebox{0.98\textwidth}{!}{
\begin{tabular}{c|l|l|l|l|l|l|l|l|l|l|l|l|l|l|l|l|c}
\toprule
\textbf{$\mathbf{r}$} & \multicolumn{1}{c|}{\textbf{Approach}} & \multicolumn{1}{c|}{\textbf{bg}} & \multicolumn{1}{c|}{\textbf{hi}} & \multicolumn{1}{c|}{\textbf{es}} & \multicolumn{1}{c|}{\textbf{el}} & \multicolumn{1}{c|}{\textbf{vi}} & \multicolumn{1}{c|}{\textbf{tr}} & \multicolumn{1}{c|}{\textbf{de}} & \multicolumn{1}{c|}{\textbf{ur}} & \multicolumn{1}{c|}{\textbf{en}} & \multicolumn{1}{c|}{\textbf{zh}} & \multicolumn{1}{c|}{\textbf{th}} & \multicolumn{1}{c|}{\textbf{sw}} & \multicolumn{1}{c|}{\textbf{ar}} & \multicolumn{1}{c|}{\textbf{fr}} & \multicolumn{1}{c|}{\textbf{ru}} & \textbf{Wins} \\ \midrule
% \multirow{5}{*}{1}  & LoRA                                   & 53.60±0.16                        & 54.87±0.09                       & 55.80±0.16                        & 53.60±0.16                        & 54.87±0.09                       & 53.20±0.00                        & 54.20±0.00                        & 53.33±0.34                       & 60.47±0.09                       & 53.53±0.09                       & 49.40±0.16                        & 51.60±0.16                        & 51.87±0.09                       & 53.13±0.09                       & 54.80±0.00                        & 0             \\ %\cline{2-18} 
%                     & AdaLoRA                              & 53.33±0.09                       & 54.73±0.09                       & 55.40±0.00                        & 53.07±0.09                       & 54.33±0.19                       & 52.87±0.09                       & 54.07±0.09                       & 53.20±0.00                        & 60.27±0.09                       & 53.07±0.09                       & 49.47±0.09                       & 50.87±0.09                       & 51.40±0.00                        & 52.87±0.19                       & 54.13±0.19                       & 0             \\ %\cline{2-18} 
%                     & BayesTune-LoRA                            & 53.27±0.09                       & 54.47±0.09                       & 55.40±0.16                        & 53.13±0.09                       & 54.00±0.00                        & 52.67±0.09                       & 53.93±0.09                       & 53.07±0.19                       & 60.00±0.16                        & 53.07±0.09                       & 49.40±0.00                        & 50.40±0.00                        & 51.40±0.16                        & 52.60±0.00                        & 54.20±0.16                        & 0             \\ %\cline{2-18} 
%                     & FedL2P                               & 60.80±0.60                       & 61.60±4.20                       & 65.70±0.90                        & 65.10±1.70                        & 65.80±0.00                        & 61.90±3.90                        & 64.40±0.40                        & 62.60±1.00                        & 70.20±0.00                        & 61.90±1.30                        & 59.50±1.70                        & 60.80±3.20                       & 63.00±1.60                       & 61.00±0.40                        & 64.00±0.40                        & 0             \\ %\cline{2-18} 
%                     & \method{}                                 & \textbf{68.67±0.9}               & \textbf{67.33±1.65}              & \textbf{71.73±1.31}              & \textbf{71.73±0.25}              & \textbf{70.73±0.57}              & \textbf{69.33±2.07}              & \textbf{70.80±0.16}               & \textbf{70.27±0.9}               & \textbf{73.67±0.41}              & \textbf{71.20±0.16}               & \textbf{63.40±2.2}                & \textbf{69.20±1.61}               & \textbf{70.93±0.77}              & \textbf{69.00±1.41}               & \textbf{70.67±0.82}              & \textbf{15}   \\ \hline
\multirow{5}{*}{2}  & LoRA                                   & 54.10±0.10                        & 55.80±0.00                        & 57.10±0.10                        & 55.10±0.30                        & 56.30±0.10                        & 54.30±0.10                        & 55.10±0.10                        & 53.60±0.20                       & 61.50±0.10                        & 54.80±0.00                        & 50.70±0.10                        & 52.50±0.10                        & 53.40±0.20                       & 53.10±0.10                        & 55.30±0.10                        & 0             \\ %\cline{2-18} 
                    & AdaLoRA                              & 53.33±0.09                       & 54.53±0.09                       & 55.33±0.09                       & 52.80±0.28                        & 54.07±0.09                       & 52.87±0.09                       & 53.93±0.09                       & 52.87±0.19                       & 60.40±0.16                        & 53.07±0.09                       & 49.40±0.16                        & 50.67±0.09                       & 51.40±0.16                        & 52.80±0.16                        & 54.20±0.16                        & 0             \\ %\cline{2-18} 
                    & BayesTune-LoRA                            & 53.40±0.00                        & 54.40±0.00                        & 55.40±0.00                        & 53.10±0.10                        & 54.00±0.20                       & 52.90±0.10                        & 54.10±0.10                        & 53.20±0.00                        & 60.00±0.00                        & 53.20±0.00                        & 49.30±0.10                        & 50.50±0.10                        & 51.50±0.10                        & 52.60±0.20                       & 54.00±0.00                        & 0             \\ %\cline{2-18} 
                    & FedL2P                               & 64.70±1.10                        & 64.20±2.80                        & 67.90±1.30                        & 68.50±1.10                        & 68.60±0.20                       & 65.20±4.00                        & 67.30±0.70                        & 67.00±0.40                        & 71.50±0.30                        & 64.90±0.50                        & 61.70±1.70                        & 64.60±2.80                        & 65.80±2.00                        & 63.40±0.00                        & 67.40±0.40                        & 0             \\ %\cline{2-18} 
                    & \method{}                                 & \textbf{70.67±0.77}              & \textbf{70.07±1.04}              & \textbf{73.87±1.54}              & \textbf{73.47±0.19}              & \textbf{73.40±0.59}               & \textbf{71.93±2.08}              & \textbf{73.27±0.57}              & \textbf{71.87±0.62}              & \textbf{74.53±0.52}              & \textbf{73.80±0.43}               & \textbf{65.60±2.97}               & \textbf{69.93±1.16}              & \textbf{74.40±0.49}               & \textbf{71.27±1.52}              & \textbf{72.73±1.88}              & \textbf{15}   \\ \hline
\multirow{5}{*}{4}  & LoRA                                   & 56.67±0.34                       & 59.07±0.25                       & 59.13±0.41                       & 57.33±0.25                       & 59.00±0.28                        & 56.93±0.25                       & 57.40±0.59                        & 55.93±0.09                       & 63.47±0.25                       & 57.27±0.09                       & 52.47±0.34                       & 54.53±0.25                       & 56.40±0.33                        & 55.93±0.09                       & 58.40±0.33                        & 0             \\ %\cline{2-18} 
                    & AdaLoRA                              & 53.33±0.09                       & 54.40±0.16                        & 55.27±0.09                       & 53.20±0.00                        & 54.13±0.19                       & 52.93±0.09                       & 54.00±0.00                        & 52.87±0.09                       & 60.33±0.25                       & 53.13±0.09                       & 49.20±0.16                        & 50.60±0.16                        & 51.27±0.09                       & 52.60±0.16                        & 54.00±0.00                        & 0             \\ %\cline{2-18} 
                    & BayesTune-LoRA                            & 53.27±0.09                       & 54.53±0.19                       & 55.40±0.16                        & 53.00±0.16                        & 54.53±0.25                       & 52.87±0.09                       & 54.00±0.00                        & 53.07±0.09                       & 60.60±0.16                        & 53.07±0.09                       & 49.33±0.19                       & 51.07±0.09                       & 51.47±0.09                       & 52.60±0.00                        & 54.53±0.19                       & 0             \\ %\cline{2-18} 
                    & FedL2P                               & 66.50±1.10                        & 65.40±1.80                        & 69.90±1.90                        & 70.50±0.90                        & 70.10±0.30                        & 66.90±4.10                        & 69.70±0.90                        & 68.20±0.60                       & 72.80±0.20                       & 67.40±1.20                       & 62.60±2.00                        & 65.70±2.90                        & 67.70±1.30                        & 66.50±1.10                        & 69.00±0.40                        & 0             \\ %\cline{2-18} 
                    & \method{}                                 & \textbf{71.33±0.34}              & \textbf{70.07±1.09}              & \textbf{73.27±2.29}              & \textbf{73.27±0.68}              & \textbf{72.60±0.28}               & \textbf{71.87±2.32}              & \textbf{74.60±0.16}               & \textbf{72.93±0.38}              & \textbf{75.00±0.28}               & \textbf{74.33±1.16}              & \textbf{66.13±3.21}              & \textbf{68.13±0.34}              & \textbf{75.13±1.65}              & \textbf{72.20±0.33}               & \textbf{73.47±1.60}               & \textbf{15}   \\ \hline
\multirow{5}{*}{8}  & LoRA                                   & 60.60±0.28                        & 62.47±0.09                       & 63.73±0.09                       & 62.60±0.33                        & 64.67±0.09                       & 61.93±0.41                       & 62.53±0.09                       & 60.33±0.19                       & 67.33±0.25                       & 61.40±0.16                        & 56.53±0.09                       & 59.27±0.09                       & 61.33±0.25                       & 58.87±0.09                       & 63.13±0.19                       & 0             \\ %\cline{2-18} 
                    & AdaLoRA                              & 53.33±0.09                       & 54.53±0.19                       & 55.27±0.09                       & 52.80±0.28                        & 54.27±0.25                       & 53.00±0.00                        & 53.93±0.09                       & 52.93±0.09                       & 60.13±0.34                       & 53.00±0.00                        & 49.20±0.16                        & 50.27±0.19                       & 51.20±0.16                        & 52.73±0.38                       & 53.93±0.09                       & 0             \\ %\cline{2-18} 
                    & BayesTune-LoRA                            & 53.33±0.09                       & 54.40±0.00                        & 55.53±0.19                       & 53.20±0.00                        & 54.67±0.09                       & 53.00±0.00                        & 54.07±0.09                       & 53.13±0.25                       & 60.60±0.00                        & 53.47±0.09                       & 49.47±0.09                       & 51.33±0.34                       & 51.73±0.19                       & 52.87±0.09                       & 54.40±0.16                        & 0             \\ %\cline{2-18} 
                    & FedL2P                               & 66.40±0.60                       & 65.50±1.10                        & 70.50±3.50                        & 70.10±0.50                        & 70.50±1.70                        & 68.20±2.20                       & 69.60±0.40                        & 67.50±1.30                        & 72.60±0.60                       & 67.10±1.30                        & 61.80±1.00                        & 65.60±1.80                        & 68.20±1.00                        & 67.20±1.40                        & 69.10±1.70                        & 0             \\ %\cline{2-18} 
                    & \method{}                                 & \textbf{70.53±0.25}              & \textbf{69.27±0.77}              & \textbf{73.33±1.79}              & \textbf{71.27±1.39}              & \textbf{71.33±0.52}              & \textbf{70.53±2.10}               & \textbf{75.33±0.41}              & \textbf{73.07±0.77}              & \textbf{74.40±0.75}               & \textbf{73.93±1.11}              & \textbf{66.73±3.80}               & \textbf{68.07±0.94}              & \textbf{74.67±1.18}              & \textbf{71.87±0.19}              & \textbf{72.93±0.96}              & \textbf{15}   \\ \hline
\multirow{5}{*}{16} & LoRA                                   & 67.13±0.34                       & 67.80±0.00                        & 69.47±0.09                       & 71.27±0.19                       & 69.20±0.00                        & 68.07±0.38                       & 69.00±0.33                        & 68.73±0.25                       & 71.47±0.25                       & 68.00±0.16                        & 62.80±0.00                        & 67.33±0.25                       & 66.80±0.16                        & 65.07±0.19                       & 67.33±0.09                       & \textbf{0}    \\ %\cline{2-18} 
                    & AdaLoRA                              & 53.40±0.00                        & 54.40±0.00                        & 55.40±0.00                        & 52.60±0.16                        & 54.00±0.00                        & 52.93±0.09                       & 53.93±0.09                       & 52.93±0.09                       & 60.27±0.41                       & 53.00±0.00                        & 49.27±0.09                       & 50.27±0.09                       & 51.20±0.33                        & 52.80±0.28                        & 53.80±0.16                        & 0             \\ %\cline{2-18} 
                    & BayesTune-LoRA                            & 53.47±0.09                       & 54.67±0.09                       & 55.80±0.00                        & 53.40±0.00                        & 54.67±0.09                       & 53.07±0.09                       & 54.47±0.19                       & 53.27±0.09                       & 60.67±0.09                       & 53.40±0.16                        & 49.40±0.00                        & 51.67±0.09                       & 52.07±0.25                       & 53.07±0.09                       & 54.80±0.00                        & 0             \\ %\cline{2-18} 
                    & FedL2P                               & 68.00±1.02                        & 65.87±1.33                       & 70.20±3.13                        & \textbf{71.73±0.81}              & 71.20±1.40                        & 69.60±1.66                        & 71.20±1.02                        & 69.00±0.99                        & 73.80±0.82                        & 69.13±1.54                       & 64.00±1.14                        & 67.80±1.82                        & 69.27±1.23                       & 68.53±1.32                       & 69.93±1.95                       & 1             \\ %\cline{2-18} 
                    & \method{}                                 & \textbf{69.80±0.16}               & \textbf{69.27±1.04}              & \textbf{73.47±1.51}              & 70.40±1.84                        & \textbf{71.27±0.57}              & \textbf{70.67±2.88}              & \textbf{74.80±0.16}               & \textbf{73.20±1.14}               & \textbf{74.53±1.25}              & \textbf{73.53±0.75}              & \textbf{65.93±2.64}              & \textbf{68.60±1.70}                & \textbf{73.67±1.84}              & \textbf{70.67±0.25}              & \textbf{72.80±0.71}               & \textbf{14}   \\ \bottomrule
\end{tabular}
}
\end{scriptsize}
\vspace{-1.2em}
\end{table*}
\begin{table*}[]
\small
\centering
\setlength{\tabcolsep}{3.5pt}
\renewcommand{\arraystretch}{0.8}
\begin{tabular}{@{}cl|ccccc@{}}
\toprule
\textbf{\# Topics} & \textbf{Model} & \multicolumn{1}{l}{\textbf{\# Input Tokens}} & \multicolumn{1}{l}{\textbf{\# Output Tokens}} & \multicolumn{1}{l}{\textbf{\# LLM Calls}} & \multicolumn{1}{l}{\textbf{Cost (GPT-4)}} & \multicolumn{1}{l}{\textbf{Time (seconds)}} \\ \midrule
\multirow{3}{*}{2} & \modelTopic & 21383.08 & 3412.02 & 25.45 & 0.32 & 117.60 \\
 & Hierarchical & 31130.02 & 2536.66 & 13.15 & 0.39 & 83.13 \\
 & Incremental-\textit{Topic} & 59010.66 & 6115.04 & 15.15 & 0.77 & 214.39 \\ \midrule
\multirow{3}{*}{3} & \modelTopic & 30208.20 & 5040.38 & 37.38 & 0.45 & 149.54 \\
 & Hierarchical & 31144.83 & 2649.78 & 13.15 & 0.39 & 68.60 \\
 & Incremental-\textit{Topic} & 61344.07 & 8442.54 & 16.15 & 0.87 & 197.33 \\ \midrule
\multirow{3}{*}{4} & \modelTopic & 38286.40 & 6440.23 & 47.91 & 0.58 & 163.91 \\
 & Hierarchical & 31144.31 & 2740.31 & 13.15 & 0.39 & 88.75 \\
 & Incremental-\textit{Topic} & 62877.46 & 9966.45 & 17.15 & 0.93 & 312.55 \\ \midrule
\multirow{3}{*}{5} & \modelTopic & 47008.59 & 7918.92 & 58.94 & 0.71 & 186.32 \\
 & Hierarchical & 31160.88 & 2850.24 & 13.15 & 0.40 & 61.70 \\
 & Incremental-\textit{Topic} & 64893.95 & 11965.84 & 18.15 & 1.01 & 262.07 \\ \bottomrule
\end{tabular}
\caption{\label{appendix:table:cost_cqa} Number of LLM input/output tokens, LLM calls, GPT-4 Cost (USD), and Time (seconds) needed to run inference on a single DFQS example on ConflictingQA with the top-3 models. We report 5 runs and 20 examples.}
\end{table*}

\begin{table*}[]
\small
\centering
\setlength{\tabcolsep}{3.5pt}
\renewcommand{\arraystretch}{0.8}
\begin{tabular}{@{}cl|ccccc@{}}
\toprule
\multicolumn{1}{l}{\textbf{Dataset}} & \textbf{Model} & \multicolumn{1}{l}{\textbf{\# Input Tokens}} & \multicolumn{1}{l}{\textbf{\# Output Tokens}} & \multicolumn{1}{l}{\textbf{\# LLM Calls}} & \multicolumn{1}{l}{\textbf{Cost (GPT-4)}} & \multicolumn{1}{l}{\textbf{Time (seconds)}} \\ \midrule
\multirow{3}{*}{2} & \modelTopic & 17183.75 & 2722.40 & 20.30 & 0.25 & 94.81 \\
 & Hierarchical & 19181.59 & 2040.39 & 10.25 & 0.25 & 63.68 \\
 & Incremental-\textit{Topic} & 41656.87 & 5062.44 & 12.25 & 0.57 & 182.19 \\ 
 \midrule
\multirow{3}{*}{3} & \modelTopic & 24801.22 & 4136.12 & 30.40 & 0.37 & 126.83 \\
 & Hierarchical & 19182.58 & 2141.91 & 10.25 & 0.26 & 53.32 \\
 & Incremental-\textit{Topic} & 43119.51 & 6532.92 & 13.25 & 0.63 & 152.44 \\ \midrule
\multirow{3}{*}{4} & \modelTopic & 30677.67 & 5037.31 & 38.00 & 0.46 & 120.64 \\
 & Hierarchical & 19203.30 & 2253.17 & 10.25 & 0.26 & 73.35 \\
 & Incremental-\textit{Topic} & 43922.02 & 7327.88 & 14.25 & 0.66 & 241.54 \\ \midrule
\multirow{3}{*}{5} & \modelTopic & 36988.41 & 6049.93 & 46.09 & 0.55 & 139.71 \\
 & Hierarchical & 19211.74 & 2356.01 & 10.25 & 0.26 & 49.41 \\
 & Incremental-\textit{Topic} & 45113.12 & 8504.59 & 15.25 & 0.71 & 186.40 \\ \bottomrule
\end{tabular}
\caption{\label{appendix:table:cost_debate} Number of LLM input/output tokens, LLM calls, GPT-4 Cost (USD), and Time (seconds) needed to run inference on a single DFQS example on DebateQFS with the top-3 models. We report 5 runs and 20 examples.}
\end{table*}

\begin{table*}[]
\small
\centering
\setlength{\tabcolsep}{3.5pt}
\renewcommand{\arraystretch}{0.8}
\begin{tabular}{@{}cl|ccccc@{}}
\toprule
\multicolumn{1}{l}{\textbf{\# Topics}} & \textbf{Model} & \multicolumn{1}{l}{\textbf{\# Input Tokens}} & \multicolumn{1}{l}{\textbf{\# Output Tokens}} & \multicolumn{1}{l}{\textbf{\# LLM Calls}} & \multicolumn{1}{l}{\textbf{Cost (GPT-4)}} & \multicolumn{1}{l}{\textbf{Time (seconds)}} \\ 
\midrule
\multirow{3}{*}{ConflictingQA} & \modelTopic & 47008.59 & 7918.92 & 58.94 & 0.71 & 186.32 \\
 & \modelTopic Pick All & 53733.70 & 9596.75 & 71.75 & 0.83 & 303.13 \\
 & Hierarchical-\emph{Topic} & 168160.85 & 7485.50 & 66.75 & 1.91 & 210.80 \\ \midrule
\multirow{3}{*}{DebateQFS} & \modelTopic & 36988.41 & 6049.93 & 46.09 & 0.55 & 139.71 \\
& \modelTopic Pick All & 43098.85 & 7612.45 & 57.25 & 0.66 & 242.35 \\
& Hierarchical-\emph{Topic} & 105237.25 & 5278.35 & 52.25 & 1.21 & 139.96 \\ \bottomrule
\end{tabular}
\caption{\label{appendix:table:cost_weird} Number of LLM input/output tokens, LLM calls, GPT-4 Cost (USD), and Time (seconds) needed to run inference on a single DFQS example on ConflictingQA and DebateQFS with \modelTopic, the version of \modelTopic with no Moderator, and the version of Hierarchical merging that runs on each topic paragraph ($m=5$). We report 5 runs and 20 examples.}
\end{table*}



Apart from Standard FL, we show that \method{} can be plugged into existing personalized FL works that trains both a subset of the pretrained model and personalized layers for each client. Tables~\ref{tab:xnli_seen_feddpa} and \ref{tab:xnli_seen_dept} show that \method{} outperforms baselines in almost all cases in our XNLI setup given a \basemodel{} trained using FedDPA-T~\cite{FedDPA} and DEPT(SPEC)~\cite{DEPT} respectively. In short, \method{} can be integrated into a larger family of existing personalized FL approaches, listed in Section~\ref{sec:related}, to further improve personalization performance. 

\begin{table*}[t]
\caption{Avg. METEOR/ROUGE-1/ROUGE-L for \seen{} clients in our Fed-Aya setup. {\em Base model} is off-the-shelf Llama-3.2-3B-Instruct.}
\vspace{0.5em}
\label{tab:lama_fedaya_seen}
\begin{scriptsize}\resizebox{0.98\textwidth}{!}{
\begin{tabular}{c|l|l|l|l|l|l|l|l|c}
\toprule
% \textbf{Lora Rank}  
\textbf{$\mathbf{r}$} & \multicolumn{1}{c|}{\textbf{Approach}} & \multicolumn{1}{c|}{\textbf{te}} & \multicolumn{1}{c|}{\textbf{ar}} & \multicolumn{1}{c|}{\textbf{es}} & \multicolumn{1}{c|}{\textbf{en}} & \multicolumn{1}{c|}{\textbf{fr}} & \multicolumn{1}{c|}{\textbf{zh}} & \multicolumn{1}{c|}{\textbf{pt}} 
& \textbf{Wins} \\ \midrule
% \multirow{5}{*}{1}  & LoRA                                   & 0.2372/0.1409/0.1368             & 0.3291/0.0625/0.0617             & 0.3863/0.4134/0.385              & 0.3229/0.3613/0.2965             & 0.2811/0.3446/0.2802             & 0.1022/0.1184/0.117              & 0.3765/0.4415/0.4023                                         & 1             \\ % \cline{2-11} 
%                     & AdaLoRA                              & 0.2344/0.1394/0.1356             & 0.3478/0.0698/0.0691             & 0.3957/0.4278/0.3989             & 0.3537/0.3999/0.3307             & 0.2951/0.3638/0.3017             & 0.1065/0.1215/0.1195             & 0.3901/0.4547/0.4135                                         & 1             \\ % \cline{2-11} 
%                     & BayesTune-LoRA                            & 0.235/0.1364/0.1335              & 0.3205/0.0618/0.0615             & 0.3729/0.3948/0.3671             & 0.2834/0.2611/0.2091             & 0.294/0.3511/0.2867              & 0.0935/0.1044/0.1044             & 0.3567/0.4194/0.3802                                         & 0             \\ % \cline{2-11} 
%                     & FedL2P                               & 0.2231/0.1355/0.1323             & 0.322/0.061/0.0608               & 0.3867/0.4124/0.3837             & 0.3076/0.3128/0.256              & 0.2943/0.3569/0.2949             & 0.0812/0.1208/0.12               & 0.3621/0.4242/0.3865                                         & 0             \\ % \cline{2-11} 
%                     & \method{}                                 & \textbf{0.2399/0.1371/0.1336}    & \textbf{0.3523/0.0668/0.066}     & \textbf{0.4022/0.4328/0.4031}    & \textbf{0.3578/0.4201/0.3477}    & \textbf{0.3264/0.3875/0.3117}    & \textbf{0.1134/0.1153/0.1134}    & \textbf{0.3978/0.4614/0.4187}                                & \textbf{4}    \\ \hline
\multirow{5}{*}{2}  & LoRA                                   & 0.2354/0.1383/0.1344             & 0.3364/0.0659/0.0656             & 0.3871/0.4142/0.3855             & 0.3345/0.3793/0.3102             & 0.2884/0.3569/0.2968             & 0.1078/0.1208/0.1194             & 0.3835/0.4478/0.4091                                         & 0             \\ % \cline{2-11} 
                    & AdaLoRA                              & 0.2373/0.1428/0.1391             & 0.3440/0.0668/0.0665              & \textbf{0.3944/0.4273/0.3994}    & 0.3536/0.4042/0.3334             & 0.2858/0.3528/0.2937             & 0.1078/\textbf{0.1226}/\textbf{0.1200}               & 0.3834/0.4514/0.4108                                         & 2             \\ % \cline{2-11} 
                    & BayesTune-LoRA                            & 0.2406/\textbf{0.1440/0.1410}               & 0.3240/0.0579/0.0576              & 0.3797/0.4065/0.3781             & 0.2922/0.2841/0.2302             & 0.2883/0.3535/0.2927             & 0.0946/0.1132/0.1119             & 0.3674/0.4327/0.3932                                         & 1             \\ % \cline{2-11} 
                    & FedL2P                               & 0.2291/0.1356/0.1322             & 0.3329/0.0687/0.0675             & 0.3783/0.4034/0.3762             & 0.3250/0.3667/0.3032              & 0.2944/0.3614/0.3004             & 0.0869/0.1173/0.1162             & 0.3776/0.4439/0.4047                                         & 0             \\ % \cline{2-11} 
                    & \method{}                                 & \textbf{0.2434}/0.1440/0.1403     & \textbf{0.3663/0.0785/0.0764}    & 0.3941/0.4224/0.3928             & \textbf{0.3746/0.4321/0.3610}     & \textbf{0.3442/0.4057/0.3318}    & \textbf{0.1144}/0.1171/0.1161    & \textbf{0.3987/0.4646/0.4201}                                & \textbf{4}    \\ \hline
\multirow{5}{*}{4}  & LoRA                                   & 0.2339/0.1317/0.1282             & 0.3497/0.0679/0.0667             & 0.4016/\textbf{0.4369/0.4077}             & 0.3458/0.3993/0.3282             & 0.2988/0.3698/0.3044             & \textbf{0.1139}/0.1206/0.1183    & 0.3955/0.4588/\textbf{0.4186}                                         & 1             \\ % \cline{2-11} 
                    & AdaLoRA                              & 0.2331/0.1404/0.1365             & 0.3350/0.0663/0.0659              & 0.3877/0.4182/0.3900               & 0.3482/0.3948/0.3252             & 0.2886/0.3516/0.2930              & 0.1091/0.1240/0.1215              & 0.3816/0.4493/0.4091                                         & 0             \\ % \cline{2-11} 
                    & BayesTune-LoRA                            & 0.2369/0.1426/0.1395             & 0.3324/0.0609/0.0600               & 0.3846/0.4111/0.3825             & 0.3121/0.3195/0.2553             & 0.2900/0.3557/0.2951               & 0.1104/0.1181/0.1172             & 0.3707/0.4386/0.4015                                         & 0             \\ % \cline{2-11} 
                    & FedL2P                               & 0.2298/0.1364/0.1327             & 0.3376/0.0711/0.0692             & 0.3797/0.4130/0.3836              & 0.3392/0.3787/0.3130              & 0.2938/0.3648/0.3053             & 0.0974/\textbf{0.1264/0.1240}              & 0.3876/0.4561/0.4159                                         & 1             \\ % \cline{2-11} 
                    & \method{}                                 & \textbf{0.2455/0.1495/0.1455}    & \textbf{0.3671/0.0749/0.0736}    & \textbf{0.4021}/0.4333/0.3994    & \textbf{0.3831/0.4400/0.3648}      & \textbf{0.3381/0.4004/0.3225}    & 0.1129/0.1212/0.1200               & \textbf{0.4018/0.4618}/0.4172                                & \textbf{5}    \\ \hline
\multirow{5}{*}{8}  & LoRA                                   & 0.2361/0.1368/0.1329             & 0.3573/0.0708/0.0695             & 0.4017/0.4341/0.4047             & 0.3586/0.4182/0.3480              & 0.3047/0.3667/0.3029             & 0.1156/\textbf{0.1260}/0.1237              & 0.3982/0.4605/0.4186                                         & 0             \\ % \cline{2-11} 
                    & AdaLoRA                              & 0.2353/0.1443/0.1399             & 0.3272/0.0648/0.0645             & 0.3863/0.4217/0.3922             & 0.3437/0.3876/0.3183             & 0.2855/0.3552/0.2929             & 0.1044/0.1242/0.1216             & 0.3740/0.4421/0.4038                                          & 0             \\ % \cline{2-11} 
                    & BayesTune-LoRA                            & 0.2397/0.1393/0.1355             & 0.3444/0.0687/0.0678             & 0.4031/0.4327/0.4032             & 0.3294/0.3521/0.2812             & 0.2962/0.3585/0.3008             & 0.1130/0.1213/0.1193              & 0.3844/0.4480/0.4084                                          & 0             \\ % \cline{2-11} 
                    & FedL2P                               & 0.2324/0.1352/0.1316             & 0.3446/0.0698/0.0681             & 0.3819/0.4153/0.3855             & 0.3547/0.4082/0.3362             & 0.3030/0.3700/0.3076                & 0.0988/0.1217/0.1199             & 0.3940/\textbf{0.4611/0.4201}                                          & 1             \\ % \cline{2-11} 
                    & \method{}                                 & \textbf{0.2431/0.1479/0.1442}    & \textbf{0.3713/0.0792/0.0779}    & \textbf{0.4077/0.4409/0.4063}    & \textbf{0.3844/0.4441/0.3687}    & \textbf{0.3440/0.4031/0.3222}     & \textbf{0.1156}/0.1246/\textbf{0.1240}     & \textbf{0.4009}/0.4567/0.4119                                & \textbf{6}    \\ \hline
\multirow{5}{*}{16} & LoRA                                   & 0.2413/0.1387/0.1355             & 0.3605/0.0711/0.0699             & 0.3864/0.4227/0.3897             & 0.3603/0.4248/0.3533             & 0.3275/0.3894/0.3178             & \textbf{0.1194}/0.1241/0.1227    & 0.4025/\textbf{0.4659/0.4225}                                         & 1             \\ % \cline{2-11} 
                    & AdaLoRA                              & 0.2349/0.1388/0.1348             & 0.3248/0.0659/0.0655             & 0.3805/0.4141/0.3861             & 0.3346/0.3702/0.3039             & 0.2818/0.3554/0.2973             & 0.1022/0.1207/0.1181             & 0.3686/0.4379/0.4012                                         & 0             \\ % \cline{2-11} 
                    & BayesTune-LoRA                            & 0.2374/0.1351/0.1310              & 0.3556/\textbf{0.0813/0.0795}             & 0.3985/0.4317/0.4013             & 0.3477/0.3998/0.3295             & 0.2995/0.3621/0.2965             & 0.1167/0.1205/0.1186             & 0.3974/0.4579/0.4153                                         & 1             \\ % \cline{2-11} 
                    & FedL2P                               & 0.2345/0.1417/0.1368             & 0.3457/0.0643/0.0633             & 0.3884/0.4185/0.3810              & 0.3740/0.4420/0.3667               & 0.3301/0.3752/0.2945             & 0.0930/0.1223/0.1211              & 0.3956/0.4543/0.4086                                         & 0             \\ % \cline{2-11} 
                    & \method{}                                 & \textbf{0.2444/0.1447/0.1406}    & \textbf{0.3735}/0.0750/0.0740      & \textbf{0.4160/0.4462/0.4105}     & \textbf{0.3920/0.4488/0.3725}     & \textbf{0.3435/0.3992/0.3204}    & 0.1103/\textbf{0.1289/0.1270}              & \textbf{0.4052}/0.4623/0.4177                                & \textbf{5}    \\ \bottomrule
\end{tabular}
}
\end{scriptsize}
\vspace{-1.5em}
\end{table*}
\begin{table*}[t]
\centering
\caption{Avg. METEOR/ROUGE-1/ROUGE-L for \unseen{} clients in our Fed-Aya setup. {\em Base model} is off-the-shelf Llama-3.2-3B-Instruct.}
\vspace{0.5em}
\label{tab:lama_fedaya_unseen}

\begin{scriptsize}\resizebox{0.98\textwidth}{!}{
\begin{tabular}{c|l|l|l|l|l|l|l|l|l|c}
\toprule

% \textbf{Lora Rank} 
\textbf{$\mathbf{r}$} & \multicolumn{1}{c|}{\textbf{Approach}} & \multicolumn{1}{c|}{\textbf{te}} & \multicolumn{1}{c|}{\textbf{ar}} & \multicolumn{1}{c|}{\textbf{es}} & \multicolumn{1}{c|}{\textbf{en}} & \multicolumn{1}{c|}{\textbf{fr}} & \multicolumn{1}{c|}{\textbf{zh}} & \multicolumn{1}{c|}{\textbf{pt}} & \multicolumn{1}{c|}{\textbf{ru}} & \textbf{Wins} \\ \midrule
% \multirow{5}{*}{1}  & LoRA                                   & 0.1614/0.1123/0.1107             & 0.2391/0.045/0.045               & 0.4175/0.4901/0.4458             & 0.3197/0.3135/0.2507             & 0.3998/0.3333/0.3333             & 0.233/0.002/0.002                & 0.3431/0.4039/0.3855             & 0.2368/0.1813/0.1728             & 1             \\ % \cline{2-11} 
%                     & AdaLoRA                              & 0.1592/0.0914/0.0898             & 0.2363/0.0384/0.0384             & \textbf{0.4315/0.4908/0.4397}    & 0.3169/0.3104/0.2486             & 0.3222/0.7222/0.7222             & 0.2561/0.002/0.002               & 0.3302/0.4122/0.395              & 0.2446/0.1969/0.1969             & 1             \\ % \cline{2-11} 
%                     & BayesTune-LoRA                            & 0.1526/0.0591/0.0575             & 0.207/0.0406/0.0377              & 0.385/0.4615/0.4123              & \textbf{0.3266/0.3085/0.2423}    & 0.3998/0.3333/0.3333             & 0.2322/0.002/0.002               & 0.3088/0.3505/0.3313             & 0.2329/0.1663/0.1608             & 0             \\ % \cline{2-11} 
%                     & FedL2P                               & 0.158/0.0747/0.0731              & 0.2621/0.0441/0.0412             & 0.4056/0.4811/0.4332             & 0.3208/0.3096/0.2478             & 0.3998/0.3333/0.3333             & 0.2313/0.002/0.002               & 0.2928/0.3719/0.3521             & \textbf{0.2577/0.184/0.184}      & 0             \\ % \cline{2-11} 
%                     & \method{}                                 & \textbf{0.1722/0.0767/0.0751}    & \textbf{0.2759/0.0474/0.0472}    & 0.4257/0.5306/0.4758             & 0.3127/0.3198/0.2695             & \textbf{0.5012/0.746/0.746}      & \textbf{0.2655/0.0066/0.0066}    & \textbf{0.3555/0.4287/0.4092}    & 0.2337/0.1236/0.1218             & \textbf{6}    \\ \hline
\multirow{5}{*}{2}  & LoRA                                   & 0.1553/0.0854/0.0838             & 0.2425/0.0458/0.0418             & 0.4275/0.4916/0.4406             & \textbf{0.3248}/0.3120/0.2494     & 0.5513/0.6667/0.6667    & 0.2489/0.0020/0.0020               & \textbf{0.3610}/0.4228/0.4033     & 0.2242/0.1797/0.1735             & 0             \\ % \cline{2-11} 
                    & AdaLoRA                              & 0.1595/0.1108/0.1092             & 0.2326/\textbf{0.0721/0.0721}             & 0.4340/0.4954/0.4435              & 0.3234/0.3102/0.2513             & 0.5513/0.6667/0.6667   & 0.2504/0.0020/0.0020               & 0.3338/0.4121/0.3970              & 0.2335/0.1758/0.1703             & 1             \\ % \cline{2-11} 
                    & BayesTune-LoRA                            & \textbf{0.1676}/0.0888/0.0858    & 0.2243/0.0487/0.0457             & 0.3821/0.4564/0.4089             & 0.3176/0.3069/0.2437             & 0.3998/0.3333/0.3333             & 0.2308/0.0020/0.0020               & 0.3052/0.3716/0.3553             & 0.2484/0.1779/0.1773             & 0             \\ % \cline{2-11} 
                    & FedL2P                               & 0.1568/\textbf{0.1156/0.1097}             & 0.2368/0.0496/0.0496             & 0.4134/0.4810/0.4350               & 0.3141/0.3050/0.2418              & 0.3998/0.3333/0.3333             & 0.2350/0.0020/0.0020               & 0.3442/0.4012/0.3853             & 0.2451/0.2010/0.2010               & 1             \\ % \cline{2-11} 
                    & \method{}                                 & 0.1674/0.0736/0.0720              & \textbf{0.2636}/0.0485/0.0485    & \textbf{0.4391/0.5335/0.4796}    & 0.3084/\textbf{0.3278/0.2718}             & 0.2155/0.2222/0.2222             & \textbf{0.2771/0.0066/0.0066}    & 0.3477/\textbf{0.4295/0.4110}              & \textbf{0.3413/0.2929/0.2662}    & \textbf{5}    \\ \hline
\multirow{5}{*}{4}  & LoRA                                   & 0.1463/0.0852/0.0758             & 0.2475/0.0332/0.0332             & 0.4178/0.4762/0.4269             & \textbf{0.3394}/0.3264/0.2648    & 0.5513/0.6667/0.6667             & 0.2566/0.0020/0.0020               & \textbf{0.3554}/0.425/0.4044     & 0.2368/0.1553/0.1530              & 0             \\ % \cline{2-11} 
                    & AdaLoRA                              & 0.1634/0.0798/0.0782             & 0.2229/\textbf{0.0575/0.0575}             & 0.4258/0.4871/0.4387             & 0.3191/0.3129/0.2519             & 0.5513/0.6667/0.6667             & 0.2458/0.0020/0.0020               & 0.3523/0.4242/0.4067             & 0.2427/\textbf{0.1716/0.1654}             & 2             \\ % \cline{2-11} 
                    & BayesTune-LoRA                            & \textbf{0.1689/0.1044/0.1027}    & 0.2308/0.0557/0.0557             & 0.3911/0.4617/0.4141             & 0.3219/0.3088/0.2445             & 0.3998/0.3333/0.3333             & 0.2349/0.0020/0.0020               & 0.3233/0.3798/0.3642             & 0.2437/0.1714/0.1653             & 1             \\ % \cline{2-11} 
                    & FedL2P                               & 0.1556/0.0627/0.0611             & 0.2465/0.0372/0.0331             & 0.4292/0.4984/0.4445             & 0.3237/0.3103/0.2498             & 0.5513/0.6667/0.6667             & 0.2454/0.0020/0.0020               & 0.3446/0.4223/0.4056             & 0.2329/0.1703/0.1625             & 0             \\ % \cline{2-11} 
                    & \method{}                                 & 0.1618/0.0706/0.0689             & \textbf{0.2621}/0.0386/0.0386    & \textbf{0.4619/0.5591/0.5061}    & 0.3194/\textbf{0.3381/0.2835}             & 0.3998/0.3333/0.3333             & \textbf{0.2898/0.0094/0.0094}    & 0.3508/\textbf{0.4306/0.4115}             & \textbf{0.2538}/0.1531/0.1531    & \textbf{4}    \\ \hline
\multirow{5}{*}{8}  & LoRA                                   & 0.1600/0.0835/0.0791               & 0.2669/0.0524/0.0484             & 0.4312/0.5136/0.4573             & 0.3352/0.3326/0.2706             & \textbf{0.5664}/0.6667/0.6667    & 0.2564/0.0045/0.0045             & 0.3465/0.4264/0.4090              & 0.2275/\textbf{0.1847/0.1847}             & 1             \\ % \cline{2-11} 
                    & AdaLoRA                              & 0.1604/\textbf{0.1045/0.1029}             & 0.2226/0.0606/0.0606             & 0.4136/0.4889/0.4350              & 0.3200/0.3082/0.2462               & 0.3164/0.5000/0.5000                   & 0.2503/0.0020/0.0020               & 0.3398/0.4203/0.4008             & 0.2370/0.1687/0.1625              & 1             \\ % \cline{2-11} 
                    & BayesTune-LoRA                            & 0.1607/0.0735/0.0688             & 0.2377/0.0731/0.0731             & 0.4102/0.4803/0.4352             & 0.3261/0.3114/0.2458             & 0.3998/0.3333/0.3333             & 0.2476/0.0020/0.0020               & 0.3483/0.3998/0.3817             & 0.2514/0.1668/0.1612             & 0             \\ % \cline{2-11} 
                    & FedL2P                               & 0.1602/0.0586/0.0570              & 0.2462/0.0447/0.0447             & 0.4339/0.5007/0.4513             & \textbf{0.3482}/0.3281/0.2642    & 0.3301/\textbf{0.7460/0.7460}               & 0.2617/0.0020/0.0020              & 0.3362/0.4262/0.4109             & 0.2300/0.1542/0.1542               & 1             \\ % \cline{2-11} 
                    & \method{}                                 & \textbf{0.1720}/0.0948/0.0890      & \textbf{0.2787/0.0804/0.0804}    & \textbf{0.5022/0.5825/0.5168}    & 0.3377/\textbf{0.3661/0.2984}             & 0.2730/0.2756/0.2756              & \textbf{0.3186/0.0249/0.0241}    & \textbf{0.3623/0.4375/0.4158}    & \textbf{0.2770}/0.1171/0.1171     & \textbf{5}    \\ \hline
\multirow{5}{*}{16} & LoRA                                   & 0.1610/\textbf{0.0901/0.0885}              & 0.2585/0.0403/0.0371             & 0.4259/0.5245/0.4700               & 0.3104/0.3302/0.2775             & \textbf{0.4077}/0.3571/0.3571    & 0.2651/0.0020/0.0020               & \textbf{0.3501}/0.4260/0.4065     & 0.2751/\textbf{0.2048/0.2048}             & 2             \\ % \cline{2-11} 
                    & AdaLoRA                              & 0.1412/0.0679/0.0659             & 0.2119/0.0439/0.0439             & 0.4149/0.4867/0.4382             & 0.3241/0.3123/0.2473             & 0.3764/0.2879/0.2879             & 0.2421/0.0020/0.0020               & 0.3308/0.4071/0.3882             & 0.2260/0.1723/0.1667              & 0             \\ % \cline{2-11} 
                    & BayesTune-LoRA                            & 0.1518/0.0641/0.0560              & 0.2448/0.0318/0.0318             & 0.4331/0.5021/0.4519             & 0.3187/0.3128/0.2524             & 0.3928/0.3148/0.3148             & 0.2546/0.0020/0.0020               & 0.3378/0.3921/0.3728             & 0.2566/0.1867/0.1860              & 0             \\ % \cline{2-11} 
                    & FedL2P                               & 0.1613/0.0726/0.0707             & 0.2605/\textbf{0.0855/0.0841}             & 0.4181/0.5049/0.4764             & 0.3383/0.3723/0.3079             & 0.4065/\textbf{0.3889/0.3889}             & 0.2547/0.0060/0.0060               & 0.3262/0.3915/0.3695             & 0.2574/0.1404/0.1404             & 2             \\ % \cline{2-11} 
                    & \method{}                                 & \textbf{0.1670}/0.0826/0.0810      & \textbf{0.2809}/0.0751/0.0751    & \textbf{0.4935/0.5715/0.5119}    & \textbf{0.3424/0.3800/0.3146}      & 0.2562/0.2626/0.2626             & \textbf{0.3293/0.0364/0.0330}     & 0.3452/\textbf{0.4280/0.4103}              & \textbf{0.3320}/0.1856/0.1589     & \textbf{4}    \\ \bottomrule

\end{tabular}
}

\end{scriptsize}
\vspace{-1.2em}
\end{table*}


\subsection{Results on Instruction-Tuning Generation}\label{sec:ift_gen}

We evaluate our approach on the more challenging real-world multilingual benchmark, Fed-Aya. Tables~\ref{tab:lama_fedaya_seen} and \ref{tab:lama_fedaya_unseen} show the average METEOR~\cite{meteor}/ROUGE-1/ROUGE-L~\cite{ROUGE} of each language given the off-the-shelf instruction finetuned Llama-3.2-3B (Llama-3.2-3B-Instruct) for \seen{} and \unseen{} clients respectively. Similarly, in Appendix Tables~\ref{tab:mobilellama_fedaya_seen} and \ref{tab:mobilellama_fedaya_unseen}, we show the same tables given a pretrained MobileLLaMA-1.4B model  trained using Standard FL with LoRA following the training recipe from FedLLM-Bench~\cite{fedllm-bench}. These two models represent scenarios where the \basemodel{} may or may not be trained using FL. Similarly to our text classification results, we report ``Wins" if an approach has a better performance in at least $2$ out of $3$ metrics. 

In all four tables, \method{} outperforms baselines in most scenarios. We also observe that FedL2P underperforms standard baselines in most cases, a phenomenon also observed for our XNLI setup when the \basemodel{} is trained with FedDPA-T (Tables~\ref{tab:xnli_seen_feddpa}). We hypothesize that the inner-loop optimization in FedL2P fail to reach a stationary point\footnote{FedL2P relies on the implicit function theorem for hypergradient computation.} due to the inherent task difficulty (Fed-Aya) or a less-performant \basemodel{}, resulting in a sub-optimal hypergradient and downstream performance. 

\subsubsection{Limitations of \method{}.} In some cases, \method{} falls short, especially in the recall performance (ROUGE), such as Russian (\textit{ru}) and French (\textit{fr}) for \unseen{} clients for both {\em base models}. These cases highlight a couple of limitations of our approach: \textit{i)} \textit{ru} is not seen by PSG during federated training; there are no \textit{ru} samples in any clients in the seen pool and \textit{ii)} none of the clients in the unseen pool have \textit{fr} as a predominant language (Fig.~\ref{fig:fed-aya}). In the case of \textit{ru}, there are no other languages that are similar to \textit{ru} in the seen pool, resulting in worse performance. Hence, we do not expect a similar outcome in datasets with a more diverse pool of clients.

For \textit{fr}, as the number of predominant language samples are orders of magnitude higher than that of \textit{fr} samples, the generated $\bm{\lambda}$ are catered towards the predominant language. A simple solution to counteract this limitation is to extend PSG to generate $\bm{\lambda}$ per instance, rather than per client. However, doing so is extremely costly, requiring a forward pass through the PSG for every sample. This calls for novel efficient solutions that can better handle each client's minority languages and is left as future work.

\subsection{Cost of \method{}}

Table~\ref{tab:cost} shows the mean latency, in seconds, and the peak memory usage across 100 runs on the first client in the \seen{} pool for $r=16$ using a single Nvidia A100 GPU. Non-FL baselines do not incur a federated training cost while FL approaches requires training the PSG. Comparing FedL2P and \method{}, \method{} does not require expensive second-order optimization, resulting in better efficiency. We also note that FedL2P needs to be run for every rank while \method{} runs once for all targeted ranks.

For communication costs, not shown in the table, \method{} is more costly as it predicts per LoRA rank while FedL2P predicts per layer. Nonetheless, these costs are negligible compared to running FL on the \basemodel{}; FedL2P uses 0.02\% and 0.002\% and \method{} uses 0.2\% and 0.16\% of the parameters of mBERT and Llama-3.2-3B respectively.

During inference, FL-based approaches incur an additional forward pass of \basemodel{} and the PSG compared to non-FL approaches. Memory-wise, \method{} results in the smallest memory footprint for autoregressive generation as the PSG learns not to attach LoRA modules $\lambda_l=0$ on some layers, skipping {\em matmul} operations entirely.

\begin{figure}[t]
    \vspace{-0.2em}
    \small
    \centering
    \includegraphics[width=0.94\columnwidth,trim={0cm 0cm 0cm 3.5cm},clip]{figures/xnli_out_0.5_seen.png}
    % \captionsetup{font=small,labelfont=bf}
    \vspace{-2em}
    \caption{$\bm{\lambda}$ distance among languages in our XNLI setup. Each block shows the log-scale normalized average Euclidean distances between all pairs of clients' $\bm{\lambda}$ for two languages (see text). The smaller the distance, the more similar $\bm{\lambda}$ is. }
    \label{fig:xnli_out}
    \vspace{-1.5em}
\end{figure}

\subsection{Further Analysis}\label{sec:analysis}

In this section, we further analyze $\bm{\lambda}$ and how they differ across languages. Surprisingly, we find that \method{} learns language-agnostic rank structures. In other words, depending on the task and the \basemodel{}, the rank structure of $\bm{\lambda}$ is fixed across languages. For instance, in the case where $r=2$, \method{} allocate ranks to dense layers instead of attention blocks. With more budget, \textit{e.g.},~$r=16$, \method{} allocates more rank to either the query attention layer or the value attention layer depending on the setup. We show these rank structures across all setups for $r=2$ and $r=16$ in Appendix Fig.~\ref{fig:xnli_fedavg_out_r16}-\ref{fig:llama3_fedavg_out_r2}.

While the rank structure is the same across languages, the rank-wise scales (absolute values of $\bm{\lambda}$) differ. Following FedL2P, we visualize the difference in $\bm{\lambda}$ for different languages using the normalized mean distance, $d(j,k)$, between all clients pairs holding data for languages $j$ and $k$. Fig.~\ref{fig:xnli_out} and Appendix Fig.~\ref{fig:masakha_out} show these distances for XNLI and MasakhaNEWS setup respectively. Specifically, the value of each block in each figure is computed as follows: $\log(\frac{d(j,k)}{\sqrt{d(j,j)}\sqrt{d(k,k)}})$. Hence, the smaller the distance, the more similar $\bm{\lambda}$ is between languages. The results are aligned with our intuition that similar languages have similar $\bm{\lambda}$. For instance, the closest language to Urdu (\textit{ur}) is Arabic (\textit{ar}), both of which have the closest $\bm{\lambda}$ similarity (Fig.~\ref{fig:xnli_out}); likewise, for Tigrinya (\textit{tir}) and Amharic (\textit{amh}) in Appendix Fig.~\ref{fig:masakha_out}. We also observe that unrelated languages have similar $\bm{\lambda}$, \textit{e.g.},~Mandarin (\textit{zh}) and Vietnamese (\textit{vi}) share similar $\bm{\lambda}$ with the Indo-European languages (Fig.~\ref{fig:xnli_out}). This finding adds to existing evidence that leveraging dissimilar languages can sometimes benefit particular languages~\cite{fedllm-bench}.


\section{Conclusion}
In this work we show that training high quality \slms with a very modest compute budget, is feasible. We give these main guidelines: (i) \textbf{Do not skimp on the model} - not all model families are born equal and the TWIST initialisation exaggerates this, thus it is worth selecting a stronger / bigger text-LM even if it means less tokens. we found Qwen$2.5$ to be a good choice; (ii) \textbf{Utilise synthetic training data} - pre-training on data generated with TTS helps a lot; (iii) \textbf{Go beyond next token prediction} - we found that DPO boosts performance notably even when using synthetic data, and as little as $30$ minutes training massively improves results; (iv) \textbf{Optimise hyper-parameters} - as researchers we often dis-regard this stage, yet we found that tuning learning rate schedulers and optimising code efficiency can improve results notably. We hope that these insights, and open source resources will be of use to the research community in furthering research into remaining open questions in \slms.

% In the unusual situation where you want a paper to appear in the
% references without citing it in the main text, use \nocite
% \nocite{langley00}

% \section*{Impact Statement} % doesnt count towards the 8 pages
% This paper presents work whose goal is to address the critical challenges of data heterogeneity and language diversity in federated learning. There are many potential societal consequences as well as threats and attacks that may increase privacy risks, none of which we feel must be specifically highlighted here as our work is orthogonal to many of the current security measures in place.

\bibliography{main}
\bibliographystyle{icml2025}


%%%%%%%%%%%%%%%%%%%%%%%%%%%%%%%%%%%%%%%%%%%%%%%%%%%%%%%%%%%%%%%%%%%%%%%%%%%%%%%
%%%%%%%%%%%%%%%%%%%%%%%%%%%%%%%%%%%%%%%%%%%%%%%%%%%%%%%%%%%%%%%%%%%%%%%%%%%%%%%
% APPENDIX
%%%%%%%%%%%%%%%%%%%%%%%%%%%%%%%%%%%%%%%%%%%%%%%%%%%%%%%%%%%%%%%%%%%%%%%%%%%%%%%
%%%%%%%%%%%%%%%%%%%%%%%%%%%%%%%%%%%%%%%%%%%%%%%%%%%%%%%%%%%%%%%%%%%%%%%%%%%%%%%
\newpage
\newcommand{\gradientcellAA}[1]{\gradientcell{#1}{50}{76}}
\newcommand{\gradientcellBB}[1]{\gradientcell{#1}{36}{61}}
\newcommand{\gradientcellCC}[1]{\gradientcell{#1}{32}{47}}
\newcommand{\gradientcellDD}[1]{\gradientcell{#1}{24}{71}}
\newcommand{\gradientcellEE}[1]{\gradientcell{#1}{46}{83}}
\newcommand{\gradientcellFF}[1]{\gradientcell{#1}{49}{67}}

\begin{table*}[t]
\centering
\resizebox{0.9\textwidth}{!}{  
\begin{tabular}{cl|ccc|cccc}
 & & \multicolumn{3}{c|}{\textbf{Knowledge-Based}} & \multicolumn{3}{c|}{\textbf{Reasoning-Based}} \\

& \textbf{Option} & \textbf{MMLU} & \textbf{MMLU-Pro} & \textbf{GPQA} & \textbf{SQuAD} & \textbf{StrategyQA} & \textbf{MuSR} \\ \midrule
& Baseline                    & \gradientcellAA{72.7} \tiny{$\pm$ 0.8} & \gradientcellBB{57.5} \tiny{$\pm$ 0.7} & \gradientcellCC{45.2} \tiny{$\pm$ 2.0} & \gradientcellDD{69.5} \tiny{$\pm$ 0.8} & \gradientcellEE{76.6} \tiny{$\pm$ 2.3} & \gradientcellFF{63.2} \tiny{$\pm$ 1.2}  \\
& Baseline with \ac{CoT}      & \textbf{\gradientcellAA{75.5} \tiny{$\pm$ 0.8}} & \textbf{\gradientcellBB{60.8} \tiny{$\pm$ 1.0}} & \gradientcellCC{45.9} \tiny{$\pm$ 1.5} & \gradientcellDD{68.0} \tiny{$\pm$ 1.4} & \gradientcellEE{78.2} \tiny{$\pm$ 1.0} & \textbf{\gradientcellFF{66.8} \tiny{$\pm$ 1.6}} \\
\midrule
\raisebox{-1.4cm}[0pt][0pt]{\rotatebox{90}{\textbf{Voting}}} & Simple               & \gradientcellAA{75.5} \tiny{$\pm$ 1.3} & \gradientcellBB{56.5} \tiny{$\pm$ 5.4} & \gradientcellCC{45.7} \tiny{$\pm$ 0.3} & \gradientcellDD{69.5} \tiny{$\pm$ 0.5} & \gradientcellEE{81.2} \tiny{$\pm$ 1.4} & \gradientcellFF{59.3 }\tiny{$\pm$ 3.0}\\
& Ranked               & \gradientcellAA{73.3} \tiny{$\pm$ 1.9} & \gradientcellBB{54.3} \tiny{$\pm$ 0.9} & \gradientcellCC{44.2} \tiny{$\pm$ 2.1} & \textbf{\gradientcellDD{70.6} \tiny{$\pm$ 1.1}} & \gradientcellEE{80.3} \tiny{$\pm$ 1.0} & \gradientcellFF{59.5} \tiny{$\pm$ 0.5}\\
& Cumulative           & \gradientcellAA{72.5} \tiny{$\pm$ 1.9} & \gradientcellBB{53.0} \tiny{$\pm$ 0.4} & \gradientcellCC{43.9} \tiny{$\pm$ 3.1} & \gradientcellDD{69.7} \tiny{$\pm$ 1.1} & \gradientcellEE{80.0} \tiny{$\pm$ 0.8} & \gradientcellFF{60.0} \tiny{$\pm$ 1.7}\\
& Approval             & \gradientcellAA{50.2} \tiny{$\pm$ 2.1} & \gradientcellBB{36.3} \tiny{$\pm$ 3.7} & \gradientcellCC{33.0} \tiny{$\pm$ 3.4} & \gradientcellDD{24.3} \tiny{$\pm$ 5.0} & \gradientcellEE{46.4} \tiny{$\pm$ 13.9} & \gradientcellFF{49.1} \tiny{$\pm$ 12.4}  \\
& Average\footnotemark 
                            & \gradientcellAA{73.8} \tiny{$\pm$ 1.7} & \gradientcellBB{54.6} \tiny{$\pm$ 2.2} & \gradientcellCC{44.6} \tiny{$\pm$ 1.8} & \gradientcellDD{69.9} \tiny{$\pm$ 0.9} & \gradientcellEE{80.5} \tiny{$\pm$ 0.7} & \gradientcellFF{59.6} \tiny{$\pm$ 1.7}\\
\midrule
\raisebox{-1.4cm}[0pt][0pt]{\rotatebox{90}{\textbf{Consensus}}} &Majority          & \gradientcellAA{74.0} \tiny{$\pm$ 1.4} & \gradientcellBB{57.3} \tiny{$\pm$ 3.0} & \gradientcellCC{43.7} \tiny{$\pm$ 1.0} & \gradientcellDD{58.2} \tiny{$\pm$ 1.0} & \gradientcellEE{80.1} \tiny{$\pm$ 0.3} & \gradientcellFF{61.3} \tiny{$\pm$ 3.3} \\
& Supermajority        & \gradientcellAA{71.9} \tiny{$\pm$ 1.2} & \gradientcellBB{57.0} \tiny{$\pm$ 1.7} & \textbf{\gradientcellCC{46.6} \tiny{$\pm$ 0.8}} & \gradientcellDD{54.3} \tiny{$\pm$ 1.9} & \gradientcellEE{80.3} \tiny{$\pm$ 1.3} & \gradientcellFF{60.2} \tiny{$\pm$ 0.3} \\
& Unanimity             & \gradientcellAA{72.2} \tiny{$\pm$ 2.4} & \gradientcellBB{57.3} \tiny{$\pm$ 1.5} & \gradientcellCC{45.3} \tiny{$\pm$ 2.5} & \gradientcellDD{56.7} \tiny{$\pm$ 2.2} & \gradientcellEE{78.1} \tiny{$\pm$ 2.3} & \gradientcellFF{61.3} \tiny{$\pm$ 0.8}  \\
& Average                     & \gradientcellAA{72.7} \tiny{$\pm$ 1.7} & \gradientcellBB{57.2} \tiny{$\pm$ 2.1} & \gradientcellCC{45.2} \tiny{$\pm$ 1.4} & \gradientcellDD{56.4} \tiny{$\pm$ 1.7} & \gradientcellEE{79.5} \tiny{$\pm$ 1.3} & \gradientcellFF{60.9} \tiny{$\pm$ 1.5}\\
\midrule
& \textbf{Metric} & \textbf{Accuracy} & \textbf{Accuracy} & \textbf{Accuracy} & \textbf{F1 Score} & \textbf{Accuracy} & \textbf{Accuracy} \\
\end{tabular}}
\caption{Task performance{\tiny$\pm$std} for seven decision protocols (voting and consensus-based) on six tasks (knowledge and reasoning) based on agents with Llama 70B. \textbf{Bold} indicates the highest results per dataset. Standard deviation for three runs.}

\label{tab:results_70b}
\end{table*}
\footnotetext{Approval Voting is left out as it consistently fails to reach a voting decision as described in \Cref{sec:experiment1}.}

\section{Additional Results}

Additional results for the first experiment to provide further information.
\subsection{Task Performance with Llama 3 70B}
\label{app:decision_70b}

Compared to the results of the Llama 3 8B model, the larger Llama 3 70B model performs much better overall, as seen in \Cref{tab:results_70b}. Most of the results are a bit better than the baseline, but the multi-agent discussions are only in a few cases able to outperform the \ac{CoT} baseline. This model does not fail to follow the prompt for the MuSR baseline and consensus-based decision protocols. Therefore, the big performance gain from the smaller model cannot be observed here. SQuAD 2.0 and StrategyQA had the largest performance gains, even outperforming the \ac{CoT} baseline, similar to the results from the smaller model. This difference in task performance can have many reasons. As \citet{li_dawn_2024} showed, smaller models are more likely to hallucinate, which reduces task performance. This can be mitigated by using multiple agents because it is less likely that two agents hallucinate the same things. Larger models tend to hallucinate less, reducing this effect for the Llama 3 70B model \citep{li_dawn_2024}. In general, Llama 3 70B has a much higher baseline for task performance, making it more difficult to improve baseline results. Many of the improvements by the Llama 3 8B model are quite small, except for the ones where the Llama 3 70B model also outperforms the \ac{CoT} baseline. This can be taken as evidence that these multi-agent discussions require specific problem structures, or else the agents are just talking about the same results for multiple rounds and agreeing with each other. If these discussions continue too long, they can drift away from the original task, which reduces task performance. This has also been observed by \citet{becker_multi-agent_2024} and an example can be seen in \Cref{example:failed_discussion}. A positive example of how discussion can help task performance can be seen in \Cref{example:good_discussion}.

\subsection{Termination percent}
\label{app:termination_percent}
The data in \Cref{tab:termination_percentages} shows the number of turns that are needed for each decision protocol to reach a final decision for the MMLU dataset. Most of the voting decision protocols are able to vote for a final answer already in the first round in which they are allowed to vote. Simple voting has the highest agreement rate, but also cumulative and ranked voting only need in a few cases another round. In contrast, the approval decision protocol only achieves this in $\sim27\%$ of the cases. About $14\%$ need another round and the rest is canceled after the fifth round. This happens because these models like to agree with each other, and therefore they tend to vote for many of the answers, which often leads to a tie. Therefore, more restrictive voting decision protocols can reach a decision more easily, as a tie is less likely. 
The consensus decision protocols require only one to two rounds to reach consensus and still achieve a higher task performance because these results are based on the MMLU dataset. The decision protocols tend to behave similarly in terms of rounds needed to create a final answer, independent of task and model.

\begin{table*}[ht]
\centering
\resizebox{0.9\textwidth}{!}{  
\begin{tabular}{l|ccccc|c}
\toprule
\textbf{Group} & \textbf{Turn 1} & \textbf{Turn 2} & \textbf{Turn 3} & \textbf{Turn 4} & \textbf{Turn 5\footnotemark} & \textbf{Task Performance Score} \\
\midrule
Voting         & 0.00\%      & 0.00\%      & 99.33\%  & 0.50\%   & 0.17\%   & \gradientcellA{53.3} \tiny{$\pm$ 1.8} \\
Cumulative     & 0.00\%      & 0.00\%      & 94.00\%  & 5.50\%   & 0.50\%   & \gradientcellA{52.6} \tiny{$\pm$ 4.0} \\
Ranked         & 0.00\%      & 0.00\%      & 91.17\%  & 7.83\%   & 1.00\%   & \gradientcellA{49.2} \tiny{$\pm$ 1.5} \\
Approval       & 0.00\%      & 0.00\%      & 26.67\%  & 14.33\%  & 59.00\%  & \gradientcellA{43.0} \tiny{$\pm$ 2.1} \\
\midrule
Majority       & 80.00\%  & 13.67\%  & 4.83\%   & 1.00\%   & 0.50\%   & \gradientcellA{53.2} \tiny{$\pm$ 2.5} \\
Supermaj.  & 79.33\%  & 14.33\%  & 4.83\%   & 1.00\%   & 0.50\%   & \gradientcellA{54.6} \tiny{$\pm$ 3.6} \\
Unanimity      & 59.50\%  & 21.67\%  & 12.67\%  & 3.50\%   & 2.67\%   & \gradientcellA{54.2} \tiny{$\pm$ 1.0} \\
\bottomrule
\end{tabular}}
\caption{Number of rounds needed for each decision protocol to reach a final decision for the MMLU dataset.}
\label{tab:termination_percentages}
\end{table*}

\footnotetext{In this round the discussion is terminated.}


\section{Additional Details on Datasets}
\label{appendix:datasets}
\begin{table*}[ht]
    \footnotesize
    \centering
    \renewcommand{\arraystretch}{1.1} % Adjusts the row spacing
    \resizebox{16cm}{!} 
    { 
    \begin{tblr}{hline{1,2,Z} = 0.8pt, hline{3-Y} = 0.2pt,
                 colspec = {Q[l,m, 13em] Q[l,m, 6em] Q[c,m, 8em] Q[c,m, 5em] Q[l,m, 14em]},
                 colsep  = 4pt,
                 row{1}  = {0.4cm, font=\bfseries, bg=gray!30},
                 row{2-Z} = {0.2cm},
                 }
\textbf{Dataset}       & \textbf{Table Source} & \textbf{\# Tables / Statements} & \textbf{\# Words / Statement} & \textbf{Explicit Control}\\ 
\SetCell[c=5]{c} \textit{Single-sentence Table-to-Text}\\
ToTTo \cite{parikh2020tottocontrolledtabletotextgeneration}   & Wikipedia        & 83,141 / 83,141                  & 17.4                          & Table region      \\
LOGICNLG \cite{chen2020logicalnaturallanguagegeneration} & Wikipedia        & 7,392 / 36,960                  & 14.2                          & Table regions      \\ 
HiTab \cite{cheng-etal-2022-hitab}   & Statistics web   & 3,597 / 10,672                  & 16.4                          & Table regions \& reasoning operator \\ 
\SetCell[c=5]{c} \textit{Generic Table Summarization}\\
ROTOWIRE \cite{wiseman2017challengesdatatodocumentgeneration} & NBA games      & 4,953 / 4,953                   & 337.1                         & \textbf{\textit{X}}                   \\
SciGen \cite{moosavi2021scigen} & Sci-Paper      & 1,338 / 1,338                   & 116.0                         & \textbf{\textit{X}}                   \\
NumericNLG \cite{suadaa-etal-2021-towards} & Sci-Paper   & 1,355 / 1,355                   & 94.2                          & \textbf{\textit{X}}                    \\
\SetCell[c=5]{c} \textit{Table Question Answering}\\
FeTaQA \cite{nan2021fetaqafreeformtablequestion}     & Wikipedia      & 10,330 / 10,330                 & 18.9                          & Queries rewritten from ToTTo \\
\SetCell[c=5]{c} \textit{Query-Focused Table Summarization}\\
QTSumm \cite{zhao2023qtsummqueryfocusedsummarizationtabular}                        & Wikipedia      & 2,934 / 7,111                   & 68.0                          & Queries from real-world scenarios\\ 
\textbf{eC-Tab2Text} (\textit{ours})                           & e-Commerce products      & 1,452 / 3,354                   & 56.61                          & Queries from e-commerce products\\
    \end{tblr}
    }
\caption{Comparison between \textbf{eC-Tab2Text} (\textit{ours}) and existing table-to-text generation datasets. Statements and queries are used interchangeably. Our dataset specifically comprises tables from the e-commerce domain.}
\label{tab:datasets}
\end{table*}
The dataset selection is very important for this work. It needs to be tested whether decision protocols perform well in multiple domains and whether some protocols perform better with specific tasks than others. Therefore, we selected datasets from different domains and divided them into two groups:

\begin{itemize}
    \item \textbf{Knowledge-based Datasets}: MMLU, MMLU-Pro, and GPQA. These still require some reasoning and domain knowledge.
    \item \textbf{Reasoning-based Datasets}: StrategyQA, MuSR, and SQuAD 2.0. These emphasize multistep reasoning and textual comprehension.
\end{itemize}

An overview of all these datasets can be found in \Cref{tab:datasets} with a description and the number of samples used for evaluation.

\subsection{Sampling Strategy}
Because multi-agent discussions are expensive, we use a small subset of each dataset that still represents the dataset effectively. This follows approaches used by \citet{yin_exchange--thought_2023, chen_reconcile_2024, becker_multi-agent_2024} and ensures a 95\% confidence level with a 5\% margin of error:

\[
n_0 = \frac{Z^2 \cdot p \cdot (1 - p)}{d^2},
\]
where \(Z = 1.96\), \(p = 0.5\), and \(d = 0.05\) \citep{thompson_sampling_2012}. For finite datasets, a finite population correction is applied:
\[
n = \frac{n_0}{1 + \frac{n_0 - 1}{N}},
\]
where \(N\) is the total number of samples in each dataset \citep{cochran_sampling_1953}. The specific sample sizes reflecting this calculation are listed in \Cref{tab:datasets}.

\subsection{Repeatability}
Each dataset was tested three times to obtain a standard deviation of the results \citep{reimers_reporting_2017, chen_reconcile_2024, becker_multi-agent_2024}, ensuring reliable and robust performance estimates across multiple evaluations.



\section{Multi-Agent Framework}
\label{sec:mallm}
For our experiments, we use the \ac{MALLM} framework\footnote{Available here: \href{https://anonymous.4open.science/r/mallm}{anonymous.4open.science/r/mallm}}.

\subsection{Architecture Overview}
\label{sec:architecture}
To better understand the different modules, we take a closer look at each component and what role it plays in creating multi-agent discussions. An overview can be found in \Cref{fig:mallm_overview} as it provides an example workflow for the framework and how a discussion is created. The discussion starts with generating personas relevant to the given task and assigning them to the participating agents. The personas are generated using the same \ac{LLM} which is later used for the agents. After that the agents start to generate solutions and improve the suggestions from the other agents. The turn order of the agents is defined by the \textit{discussion paradigm}. This also defines which answers are visible to other agents and who can talk to whom. The \textit{response generator} defines how an agent receives the other answers and also the way it responds. After a certain number of rounds or when enough agents agree, a \textit{decision protocol} is used to select the best answer either via voting or just by looking for a certain consensus threshold. If the decision protocol fails, for example, due to a tied vote, the discussion continues for another round. In the framework a parameter can be defined to terminate discussions after a certain number of rounds to make sure they do not communicate forever.

\begin{figure*}
    \centering
    \includegraphics[width=0.95\linewidth]{pdf/MALLM-Overview.drawio.pdf}
    \caption{Example multi-agent discussion conducted in the \ac{MALLM} framework. It showcases the functionality of the four modules and how they work together to get an improved final solution.}
    \label{fig:mallm_overview}
\end{figure*}

\paragraph{Agent Personalities.}
The first step of the discussion is the generation of agent personas. Each of the agents participating in the discussion has a certain persona assigned to them. This can unlock more knowledge for the \ac{LLM} on a specific topic \citep{kim_persona_2024}. To get the best results, we want as diverse personas as possible while still maintaining them to be relevant to the task. The default setting for the framework is to prompt a \ac{LLM} and ask for a persona relevant to the given task \citep{wang_unleashing_2024}. After each generation it also provides the generated personas to avoid duplication. This way of generating personas provides a good starting point, but as this is built as a modular component, it can be swapped out with another function, which, for example, generates half of the agents with this method and initializes the other ones as neutral agents without a persona.
\paragraph{Response Generators.}
Another important part of multi-agent discussions is how each agent responds to the previous responses. Do we use \ac{CoT} to improve performance, or does this result in too long answers? By changing the way an agent is prompted, a lot of performance can be gained or lost. Therefore, it is key to make this as customizable as possible. The researcher has the possibility to change the default behavior (neutral answers), for example, by prompting the agent to be more critical or changing the way the discussion history is presented. The system prompt for the agent's persona can also be adjusted. \ac{MALLM} already has many different built-in response generators. The ones relevant for this work are the following.
\begin{itemize}
    \item \textbf{Free Text} is the most basic form of the agent prompt. Each agent gets a predefined number of discussion history rounds as memory. The prompt language is neutral, and the task is presented each round to mitigate the potential drift from the topic of the discussion \citep{becker_multi-agent_2024}. In addition, the agent is always asked to agree or disagree with the answer of the previous agent.
    \item \textbf{Simple} behaves very similar to the Free Text response generator, but the prompt is a bit simpler to make it easier to understand for the \ac{LLM} and reduce the context length.
    \item \textbf{Critical} forces the agent to respond very critically to the previous answer and try to find new solutions. Some studies have shown that \acp{LLM} can show some form of sycophancy, which is not helpful for a constructive discussion \citep{sharma_towards_2023}. Encouraging them to be more critical may reduce this.
    \item \textbf{Reasoning} doesn't allow the agents to communicate their final solution with the other agents. They can only share reasoning that can be used to find a final solution. In the end, each agent has to come up with its own solution without being directly influenced by other agents.
\end{itemize}
\paragraph{Discussion Paradigms.}
These paradigms define the discussion format for the entire task. They can control the order in which the agents communicate with each other, and which answers are visible only to certain agents. Currently, all the built-in discussion paradigms are static, meaning that the turn order is predefined and cannot be changed based on specific events during the discussion. However, due to the modular nature of \ac{MALLM}, a new discussion paradigm can be added, for example using an \ac{LLM} as a moderator to dynamically decide which agent should respond next. Current research by \citet{yin_exchange--thought_2023} and \citet{becker_multi-agent_2024} shows that discussion protocols have little impact on downstream task performance. \ac{MALLM} includes the following discussion paradigms, which are illustrated in \Cref{fig:discussion_paradigms}. The first four paragdims are inspired by the work of \citet{yin_exchange--thought_2023}, while the fifth was developed as part of this work.
\begin{figure}[H]
    \centering
    \includegraphics[width=\linewidth]{pdf/Memory.drawio-1.pdf}
    \caption{Illustration of Discussion Paradigms available for use in \ac{MALLM}}
    \label{fig:discussion_paradigms}
\end{figure}

\begin{itemize}
    \item \textbf{Memory} is the most basic discussion paradigm. The agents respond to the solution of the previous agents with feedback or an improved solution. All answers are visible to the other agents. 
    \item \textbf{Relay} behaves similarly to the memory paradigm. The turn order is the same, but each agent can only see the answer from the previous agent.
    \item \textbf{Report} introduces one agent as a moderator that can communicate with other agents. The other agents can only communicate with the moderator and have access to these messages only. Only the moderator can see all messages.
    \item \textbf{Debate} is similar to the report paradigm, as it also needs a moderator. Here, the other agents can communicate for a predefined number of rounds before they forward their reasoning to the moderator agent, which starts the next round of debate.
    \item \textbf{Collective Refinement} In this protocol, each agent first generates an answer independently. In each subsequent round, every agent receives the responses from all other agents at the same time. Using this shared information, each agent refines their own answer. This process continues throughout the rounds, helping agents gradually reach a shared and improved solution. There is no turn order, and all agents have the same level of knowledge in each round.
\end{itemize}

\paragraph{Decision Protocols.}
These are crucial for the framework as they decide which answer gets presented as the final answer to the problem. Multi-agent discussions produce multiple results for the same problem because each agent has its own reasoning and ideas on how to solve the problem. Therefore, some process is needed to decide which answer looks the most promising. We divide these decision protocols into three subtypes that we want to analyze. An overview of how each of these decision protocols works theoretically can be found in \Cref{fig:decision_protocols} and all prompts used for them can be found in \Cref{sec:app_prompts}.

\begin{figure*}[H]
    \centering
    \includegraphics[width=0.95\linewidth]{pdf/MALLM-Overview.drawio.pdf}
    \caption{Example multi-agent discussion conducted in the \ac{MALLM} framework. It showcases the functionality of the four modules and how they work together to get an improved final solution.}
    \label{fig:mallm_overview}
\end{figure*}

\subparagraph{Consensus Based Decision Protocols.} 
These are the simplest kinds of decision protocols. After each answer, the next agent has to agree or disagree with the previous statement. Depending on the response generator, this happens in the same message, and the agreement is extracted with a regular expression, or this is split into multiple answers. If enough agents agree in order, there is a consensus. The final answer is extracted by instructing the last agent to solve the given task with the information available in the latest messages. The prompt used for this can be found in \Cref{sec:app_final_answer_prompt}. There are several types of consensus decision protocols available in \ac{MALLM}. \textbf{Majority consensus} requires 50\% of the agents to agree. \textbf{Supermajority consensus} requires 66\% of the agents to agree, and \textbf{unanimity consensus} requires all agents to agree.

\subparagraph{Voting Based Decision Protocols.}
For voting based decision protocols, the process differs slightly compared to consensus-based decision protocols. The agents are forced to discuss for a predefined number of turns and afterward create a final solution. In the default setting, they have to discuss for three rounds, as current research such as \citet{du_improving_2023} shows that this allows for reasonable strong improvements considering computing resources. If there happens to be a tie in the voting, the agents have to discuss it for another round, and after that, they are asked to vote again. If they do not reach a final decision before exceeding the maximum number of rounds (defined in the discussion configuration), the solution of the first agent is used. To analyze the impact of the voting procedure, different processes similar to the work of \citet{yang_llm_2024} are implemented.
\begin{itemize}
    \item \textbf{Simple Voting} Each of the agents has only one vote. They can vote for any other agent or for themselves. The agent with the most votes wins.
    \item \textbf{Ranked Voting} The agents have to rank all final answers. The best solution is chosen by adding the ranking indices for a given agent and then selecting the answer with the best cumulative rank.
    \item \textbf{Cumulative Voting} Each agent has to distribute up to 25 points to all possible answers. They can also give fewer points and freely divide the points between all agents (even themselves). The winner is selected by adding all the points for a given agent and selecting the final answer with the most points.
    \item \textbf{Approval Voting} The agent has to provide a list of solutions that it approves. After that, the approvals from all agents are counted, and the answer with the most approvals wins the vote.
\end{itemize}
\clearpage




\section{Prompts}
\label{sec:app_prompts}
\subsection{Final Answer Extraction}
\label{sec:app_final_answer_prompt}
\medskip
\begin{figure}[H]
    \centering
    \begin{combinedprompt}
    \textbf{System Prompt:} \\
    \begingroup
    \colorbox{systemcolor}{\parbox{\dimexpr\linewidth-2\fboxsep\relax}{
    Your role: \texttt{<persona>} (\texttt{<persona description>})
    }}
    \endgroup

    \vspace{0.4em} %

    \textbf{User Prompt:} \\
    \begingroup
    \colorbox{usercolor}{\parbox{\dimexpr\linewidth-2\fboxsep\relax}{
    You are tasked with creating a final solution based on the given input and your previous response.\\
    Task: \texttt{<task>}\\
    Input: \texttt{<input sample>}\\
    Your previous response: \texttt{<previous answer>}\\
    Extract the final solution to the task from the provided text. Remove statements of agreement, disagreement, and explanations. Do not modify the text. Do not output any text besides the solution. If there is no solution provided, just copy the previous response.
    }}
    \endgroup
\end{combinedprompt}
    \caption{Prompt used to extract the final answer of a given agent from its previous response.}
    \label{fig:extract_prompt}
\end{figure}

\subsection{Voting Prompts}
\label{app:voting_prompts}

\begin{figure}[H]
    \centering
    \begin{combinedprompt}
    \textbf{System Prompt:} \\
    \begingroup
    \colorbox{systemcolor}{\parbox{\dimexpr\linewidth-2\fboxsep\relax}{
    Your role: \texttt{<persona>} (\texttt{<persona description>})
    }}
    \endgroup

    \vspace{0.4em} %

    \textbf{User Prompt:} \\
    \begingroup
    \colorbox{usercolor}{\parbox{\dimexpr\linewidth-2\fboxsep\relax}{
    You are tasked with voting for the best solution from the list provided below based on the given task.\\
    Task: \texttt{<task>}\\
    Question: \texttt{<input sample>}\\
    Here are the possible solutions:\\
    Solution 1: \texttt{<agent 1 final answer>}\\
    Solution 2: \texttt{<agent 2 final answer>}\\
    Solution 3: \texttt{<agent 3 final answer>}\\
    Based on the above solutions, please provide the number of the solution you are voting for. Answer only with the number.
    }}
    \endgroup
\end{combinedprompt}
    \caption{Prompt used to get a vote from each agent for the Simple Voting decision protocol.}
    \label{fig:simple_voting_prompt}
\end{figure}

\begin{figure}[H]
    \centering
    \begin{combinedprompt}
    \textbf{System Prompt:} \\
    \begingroup
    \colorbox{systemcolor}{\parbox{\dimexpr\linewidth-2\fboxsep\relax}{
    Your role: \texttt{<persona>} (\texttt{<persona description>})
    }}
    \endgroup

    \vspace{0.4em} %

    \textbf{User Prompt:} \\
    \begingroup
    \colorbox{usercolor}{\parbox{\dimexpr\linewidth-2\fboxsep\relax}{
    You are tasked with approving any number of solutions from the list provided below based on the given task.\\
    Task: \texttt{<task>}\\
    Question: \texttt{<input sample>}\\
    Here are the possible solutions:\\
    Solution 1: \texttt{<agent 1 final answer>}\\
    Solution 2: \texttt{<agent 2 final answer>}\\
    Solution 3: \texttt{<agent 3 final answer>}\\
    Based on the above solutions, please provide the numbers of the solutions you are approving, separated by commas. Answer only with the numbers.
    }}
    \endgroup
\end{combinedprompt}
    \caption{Prompt used to get a vote from each agent for the Approval Voting decision protocol.}
    \label{fig:approval_voting_prompt}
\end{figure}

\begin{figure}[H]
    \centering
    \begin{combinedprompt}
    \textbf{System Prompt:} \\
    \begingroup
    \colorbox{systemcolor}{\parbox{\dimexpr\linewidth-2\fboxsep\relax}{
    Your role: \texttt{<persona>} (\texttt{<persona description>})
    }}
    \endgroup

    \vspace{0.4em} %

    \textbf{User Prompt:} \\
    \begingroup
    \colorbox{usercolor}{\parbox{\dimexpr\linewidth-2\fboxsep\relax}{
    You are tasked with distributing 10 points among the provided solutions based on the given task.\\
    Task: \texttt{<task>}\\
    Question: \texttt{<input sample>}\\
    Here are the possible solutions:\\
    Solution 1: \texttt{<agent 1 final answer>}\\
    Solution 2: \texttt{<agent 2 final answer>}\\
    Solution 3: \texttt{<agent 3 final answer>}\\
    Based on the above solutions, please distribute 10 points among the solutions. Provide your points allocation as a JSON dictionary where keys are solution numbers (as int) and values are the points. The total points should sum up to 10. Answer only with the JSON dictionary.
    }}
    \endgroup
\end{combinedprompt}
    \caption{Prompt used to get a vote from each agent for the Cumulative Voting decision protocol.}
    \label{fig:cumulative_voting_prompt}
\end{figure}

\begin{figure}[H]
    \centering
    \begin{combinedprompt}
    \textbf{System Prompt:} \\
    \begingroup
    \colorbox{systemcolor}{\parbox{\dimexpr\linewidth-2\fboxsep\relax}{
    Your role: \texttt{<persona>} (\texttt{<persona description>})
    }}
    \endgroup

    \vspace{0.4em} %

    \textbf{User Prompt:} \\
    \begingroup
    \colorbox{usercolor}{\parbox{\dimexpr\linewidth-2\fboxsep\relax}{
    You are tasked with ranking the solutions from the most preferred to the least preferred based on the given task.\\
    Task: \texttt{<task>}\\
    Question: \texttt{<input sample>}\\
    Here are the possible solutions:\\
    Solution 1: \texttt{<agent 1 final answer>}\\
    Solution 2: \texttt{<agent 2 final answer>}\\
    Solution 3: \texttt{<agent 3 final answer>}\\
    Based on the above solutions, please provide the rankings of the solutions separated by spaces. Example: '0 2 1' if you prefer Solution 0 the most, then Solution 2, and finally Solution 1. Provide up to 5 rankings. Only answer with the rankings.
    }}
    \endgroup
\end{combinedprompt}
    \caption{Prompt used to get a vote from each agent for the Ranked Voting decision protocol.}
    \label{fig:ranked_voting_prompt}
\end{figure}







\subsection{Challenge Prompt}
\label{app:challenge_prompt}
\begin{figure}[H]
    \centering
    \begin{combinedprompt}
    \textbf{System Prompt:} \\
    \begingroup
    \colorbox{systemcolor}{\parbox{\dimexpr\linewidth-2\fboxsep\relax}{
    You are a participant in a group discussion. \\
    Your role: \texttt{<persona>} (\texttt{<persona description>})
    }}
    \endgroup

    \vspace{0.4em} %

    \textbf{User Prompt:} \\
    \begingroup
    \colorbox{usercolor}{\parbox{\dimexpr\linewidth-2\fboxsep\relax}{
    The task is: \texttt{<task>}. The question is: \texttt{<question>}. \\
    This is the final answer generated by the discussion: \texttt{<final\_answer>}. \\
    Please critically evaluate this answer. If you agree with the final answer, respond with the exact word 'AGREE' to confirm. If you do not agree, respond with the exact word 'DISAGREE' to challenge the answer.
    }}
    \endgroup
\end{combinedprompt}
    \caption{Prompt used to challenge the final answer.}
    \label{fig:challenge_prompt}
\end{figure}

\begin{figure}[H]
    \centering
    \begin{combinedprompt}
    \textbf{System Prompt:} \\
    \begingroup
    \colorbox{systemcolor}{\parbox{\dimexpr\linewidth-2\fboxsep\relax}{
    You are a participant in a group discussion. \\
    Your role: \texttt{<persona>} (\texttt{<persona description>})
    }}
    \endgroup

    \vspace{0.4em} %

    \textbf{User Prompt:} \\
    \begingroup
    \colorbox{usercolor}{\parbox{\dimexpr\linewidth-2\fboxsep\relax}{
    The task is: \texttt{<task>}. The question is: \texttt{<question>}. \\
    This is the final answer generated by the discussion: \texttt{<final\_answer>}. \\
    You dont agree with the final answer. Please provide a new answer to the question. Include the letter corresponding to your answer in the solution.
    }}
    \endgroup
\end{combinedprompt}
    \caption{Prompt used to generate a new answer in case the final answer got challenged.}
    \label{fig:challenge_new_prompt}
\end{figure}
\newpage
\onecolumn
\section{MALLM Setup}
\label{sec:mallm_setup_app}

    
\begin{configpython}[Default Parameters used for each experiment]{lst:config_default}
input_json_file_path: str = None
output_json_file_path: str = None
task_instruction_prompt: str = None
task_instruction_prompt_template: Optional[str] = None
endpoint_url: str = "https://api.openai.com/v1"
model_name: str = "gpt-3.5-turbo"
api_key: str = "-"
max_turns: int = 10
skip_decision_making: bool = False
discussion_paradigm: str = "memory"
response_generator: str = "simple"
decision_protocol: str = "hybrid_consensus"
visible_turns_in_memory: int = 2
debate_rounds: int = 2
concurrent_api_requests: int = 100
use_baseline: bool = False
use_chain_of_thought: bool = True
num_agents: int = 3
num_neutral_agents: int = 0
agent_generator: str = "expert"
agent_generators_list: list = []
trust_remote_code: bool = False
num_samples: Optional[int] = None
hf_dataset_split: Optional[str] = "test"
hf_token: Optional[str] = None
hf_dataset_version: Optional[str] = None
hf_dataset_input_column: Optional[str] = None
hf_dataset_reference_column: Optional[str] = None
hf_dataset_context_column: Optional[str] = None
use_ablation: bool = False
shuffle_input_samples: bool = False
all_agents_generate_first_draft: bool = False
all_agents_generate_draft: bool = False
policy: Optional[str] = None
voting_protocols_with_alterations: bool = False
calculate_persona_diversity: bool = False
\end{configpython}
\newpage
\section{Example Discussions}
\subsection{Successfull Voting Discussion}
All decision protocols are attached as an example to this discussion. The original discussion was created using simple voting.

\vspace{5mm}

\definecolor{lightyellow}{RGB}{255, 255, 204}

\begin{tcolorbox}[title=Overview, colframe=black!10, coltitle=black, fonttitle=\bfseries, boxrule=0.5mm, width=\textwidth]

\textbf{Task:} Answer the following question. If the question is not answerable with the provided information, write '[UNKNOWN]'.

\textbf{Input:} What choir from the area has sung at the Guildhall?

\textbf{Context:} Southampton has two large live music venues, the Mayflower Theatre (formerly the Gaumont Theatre) and the Guildhall. The Guildhall has seen concerts from a wide range of popular artists including Pink Floyd, David Bowie, Delirious?, Manic Street Preachers, The Killers, The Kaiser Chiefs, Amy Winehouse, Lostprophets, The Midnight Beast, Modestep, and All Time Low. It also hosts classical concerts presented by the Bournemouth Symphony Orchestra, City of Southampton Orchestra, Southampton Concert Orchestra, Southampton Philharmonic Choir and Southampton Choral Society.  

\begin{tcolorbox}[coltitle=black, fonttitle=\bfseries, boxrule=0.5mm, width=\textwidth]
\textbf{MALLM solution:} Southampton Philharmonic Choir \\
\textbf{Real solution:} Southampton Philharmonic Choir
\end{tcolorbox}
\end{tcolorbox}


\begin{tcolorbox}[breakable, enhanced jigsaw, title=Turn 1, colframe=black!10, coltitle=black, fonttitle=\bfseries, boxrule=0.5mm, width=\textwidth]

\begin{tcolorbox}[title=Music Connoisseur \hfill \tiny{\textnormal{A knowledgeable musician who is familiar with various music genres and has attended concerts at the Guildhall.}}, colback=lightyellow, colframe=orange!30, coltitle=black, fonttitle=\bfseries, boxrule=0.5mm, width=\textwidth]
\begin{minipage}{1.0\textwidth}
Based on the information provided, I propose the following solution: Southampton Philharmonic Choir. The text states: "It also hosts classical concerts presented by the Bournemouth Symphony Orchestra, City of Southampton Orchestra, Southampton Concert Orchestra, Southampton Philharmonic Choir and Southampton Choral Society." This suggests that the Southampton Philharmonic Choir is one of the choirs that has performed at the Guildhall, as it is mentioned alongside the other organizations that host classical concerts at the venue.
\end{minipage}
\end{tcolorbox}

\begin{tcolorbox}[title=Choir Conductor \hfill \tiny{\textnormal{A dedicated choral conductor with extensive experience in leading local choirs and orchestras, familiar with the performances at the Guildhall.}}, colback=lightyellow, colframe=orange!30, coltitle=black, fonttitle=\bfseries, boxrule=0.5mm, width=\textwidth]
\begin{minipage}{1.0\textwidth}
[AGREE] I agree with the current solution that the Southampton Philharmonic Choir has sung at the Guildhall. The text explicitly mentions it as one of the organizations that hosts classical concerts at the Guildhall, which indicates that the choir has performed at the venue.
\end{minipage}
\end{tcolorbox}

\begin{tcolorbox}[title=Music Journalist \hfill \tiny{\textnormal{A music enthusiast who has written articles and reviews about various concerts and events at the Guildhall, including performances by local choirs.}}, colback=lightyellow, colframe=orange!30, coltitle=black, fonttitle=\bfseries, boxrule=0.5mm, width=\textwidth]
\begin{minipage}{1.0\textwidth}
[AGREE] I completely agree with the current solution. The text explicitly mentions the Southampton Philharmonic Choir as one of the organizations that hosts classical concerts at the Guildhall, making it the most likely choice.
\end{minipage}
\end{tcolorbox}

\end{tcolorbox}

\begin{tcolorbox}[breakable, enhanced jigsaw, title=Turn 2, colframe=black!10, coltitle=black, fonttitle=\bfseries, boxrule=0.5mm, width=\textwidth]

\begin{tcolorbox}[title=Music Connoisseur \hfill \tiny{\textnormal{A knowledgeable musician who is familiar with various music genres and has attended concerts at the Guildhall.}}, colback=lightyellow, colframe=orange!30, coltitle=black, fonttitle=\bfseries, boxrule=0.5mm, width=\textwidth]
\begin{minipage}{1.0\textwidth}
[AGREE] I'm in agreement with the current solution, as the text explicitly mentions the Southampton Philharmonic Choir as one of the organizations that hosts classical concerts at the Guildhall. This makes it the most likely candidate to have sung at the venue.
\end{minipage}
\end{tcolorbox}

\begin{tcolorbox}[title=Choir Conductor \hfill \tiny{\textnormal{A dedicated choral conductor with extensive experience in leading local choirs and orchestras, familiar with the performances at the Guildhall.}}, colback=lightyellow, colframe=orange!30, coltitle=black, fonttitle=\bfseries, boxrule=0.5mm, width=\textwidth]
\begin{minipage}{1.0\textwidth}
[AGREE] I agree with the current solution that the Southampton Philharmonic Choir has sung at the Guildhall. The text explicitly mentions it as one of the organizations that hosts classical concerts at the Guildhall, which indicates that the choir has performed at the venue.
\end{minipage}
\end{tcolorbox}

\begin{tcolorbox}[title=Music Journalist \hfill \tiny{\textnormal{A music enthusiast who has written articles and reviews about various concerts and events at the Guildhall, including performances by local choirs.}}, colback=lightyellow, colframe=orange!30, coltitle=black, fonttitle=\bfseries, boxrule=0.5mm, width=\textwidth]
\begin{minipage}{1.0\textwidth}
[AGREE] I completely agree with the current solution. The text explicitly mentions the Southampton Philharmonic Choir as one of the organizations that hosts classical concerts at the Guildhall, making it the most likely choice.
\end{minipage}
\end{tcolorbox}

\end{tcolorbox}


\begin{tcolorbox}[breakable, enhanced jigsaw, title=Turn 3, colframe=black!10, coltitle=black, fonttitle=\bfseries, boxrule=0.5mm, width=\textwidth]

\begin{tcolorbox}[title=Music Connoisseur \hfill \tiny{\textnormal{A knowledgeable musician who is familiar with various music genres and has attended concerts at the Guildhall.}}, colback=lightyellow, colframe=orange!30, coltitle=black, fonttitle=\bfseries, boxrule=0.5mm, width=\textwidth]
\begin{minipage}{1.0\textwidth}
[AGREE] I agree with the current solution and the assessment that there is no need to improve it. The text provides explicit evidence that the Southampton Philharmonic Choir hosts classical concerts at the Guildhall, making it the most likely correct answer.
\end{minipage}
\end{tcolorbox}

\begin{tcolorbox}[title=Choir Conductor \hfill \tiny{\textnormal{A dedicated choral conductor with extensive experience in leading local choirs and orchestras, familiar with the performances at the Guildhall.}}, colback=lightyellow, colframe=orange!30, coltitle=black, fonttitle=\bfseries, boxrule=0.5mm, width=\textwidth]
\begin{minipage}{1.0\textwidth}
[AGREE] I agree with the current solution that the Southampton Philharmonic Choir has sung at the Guildhall. The text explicitly mentions it as one of the organizations that hosts classical concerts at the Guildhall, which indicates that the choir has performed at the venue.
\end{minipage}
\end{tcolorbox}

\begin{tcolorbox}[title=Music Journalist \hfill \tiny{\textnormal{A music enthusiast who has written articles and reviews about various concerts and events at the Guildhall, including performances by local choirs.}}, colback=lightyellow, colframe=orange!30, coltitle=black, fonttitle=\bfseries, boxrule=0.5mm, width=\textwidth]
\begin{minipage}{1.0\textwidth}
[AGREE] I completely agree with the current solution. The text explicitly mentions the Southampton Philharmonic Choir as one of the organizations that hosts classical concerts at the Guildhall, making it the most likely choice.
\end{minipage}
\end{tcolorbox}

\end{tcolorbox}


\begin{tcolorbox}[title=Result Simple Voting, colframe=black!10, coltitle=black, fonttitle=\bfseries, boxrule=0.5mm, width=\textwidth]
\textbf{Music Connoisseur final answer:} Southampton Philharmonic Choir

\textbf{Choir Conductor final answer:} Southampton Choral Society

\textbf{Music Journalist final answer:} Southampton Philharmonic Choir

\bigskip

\textbf{Music Connoisseur} voted for answer from \textbf{Music Journalist}

\textbf{Choir Conductor} voted for answer from \textbf{Music Journalist}

\textbf{Music Journalist} voted for answer from \textbf{Music Journalist}


\end{tcolorbox}

\begin{tcolorbox}[title=Result Approval Voting, colframe=black!10, coltitle=black, fonttitle=\bfseries, boxrule=0.5mm, width=\textwidth]
\textbf{Music Critic final answer:} Southampton Philharmonic Choir

\textbf{Local Music Event Coordinator final answer:} Southampton Choral Society and Southampton Philharmonic Choir

\textbf{Information Architect final answer:} Southampton Philharmonic Choir

Southampton Choral Society

\bigskip

\textbf{Music Critic} approved answers from \textbf{Information Architect}, \textbf{Music Critic}

\textbf{Local Music Event Coordinator} approved answers from \textbf{Information Architect}

\textbf{Information Architect} approved answers from \textbf{Information Architect}


\end{tcolorbox}


\begin{tcolorbox}[title=Result Cumulative Voting, colframe=black!10, coltitle=black, fonttitle=\bfseries, boxrule=0.5mm, width=\textwidth]
\textbf{Local Music Enthusiast final answer:} Southampton Choral Society

Southampton Philharmonic Choir

\textbf{Archivist final answer:} Southampton Philharmonic Choir

Southampton Choral Society

\textbf{Music Journalist final answer:} Southampton Philharmonic Choir and Southampton Choral Society

\bigskip

\textbf{Local Music Enthusiast} allocated points: \\
\textbf{Archivist}: 9\\
\textbf{Music Journalist}: 1
\bigskip

\textbf{Archivist} allocated points: \\
\textbf{Local Music Enthusiast}: 2\\
\textbf{Archivist}: 4\\
\textbf{Music Journalist}: 4
\bigskip

\textbf{Music Journalist} allocated points: \\
\textbf{Local Music Enthusiast}: 1\\
\textbf{Archivist}: 2\\
\textbf{Music Journalist}: 7


\end{tcolorbox}

\begin{tcolorbox}[title=Result Ranked Voting, colframe=black!10, coltitle=black, fonttitle=\bfseries, boxrule=0.5mm, width=\textwidth]
\textbf{Music Journalist final answer:} Southampton Philharmonic Choir and Southampton Choral Society.

\textbf{Local Music Expert final answer:} Southampton Philharmonic Choir and Southampton Choral Society

\textbf{Concert Promoter final answer:} Southampton Philharmonic Choir and Southampton Choral Society

\bigskip

\textbf{Music Journalist} ranked answers: \\
1. Music Journalist\\
2. Local Music Expert\\
3. Concert Promoter
\bigskip

\textbf{Local Music Expert} ranked answers:\\
1. Music Journalist\\
2. Local Music Expert\\
3. Concert Promoter
\bigskip

\textbf{Concert Promoter} ranked answers:\\
1. Music Journalist\\
2. Local Music Expert\\
3. Concert Promoter


\end{tcolorbox}















\label{example:good_discussion}



\newpage
\subsection{Agents Tricked Discussion}
In this discussion, the agents were tricked by information provided in the context.

\vspace{5mm}

\definecolor{lightyellow}{RGB}{255, 255, 204}


\begin{tcolorbox}[title=Overview, colframe=black!10, coltitle=black, fonttitle=\bfseries, boxrule=0.5mm, width=\textwidth]

\textbf{Task:} Answer the following question. If the question is not answerable with the provided information, write '[UNKNOWN]'.

\textbf{Input:} What provides critical support for drug discovery and the availability of economic resources?

\textbf{Context:} Biodiversity provides critical support for drug discovery and the availability of medicinal resources. A significant proportion of drugs are derived, directly or indirectly, from biological sources: at least 50\% of the pharmaceutical compounds on the US market are derived from plants, animals, and micro-organisms, while about 80\% of the world population depends on medicines from nature (used in either modern or traditional medical practice) for primary healthcare. Only a tiny fraction of wild species has been investigated for medical potential. Biodiversity has been critical to advances throughout the field of bionics. Evidence from  analysis and biodiversity science indicates that the decline in output from the pharmaceutical sector since the mid-1980s can be attributed to a move away from natural product exploration ("bioprospecting") in favor of genomics and synthetic chemistry, indeed claims about the value of undiscovered pharmaceuticals may not provide enough incentive for companies in free markets to search for them because of the high cost of development; meanwhile, natural products have a long history of supporting significant economic and health innovation. Marine ecosystems are particularly important, although inappropriate bioprospecting can increase biodiversity loss, as well as violating the laws of the communities and states from which the resources are taken.

\begin{tcolorbox}[coltitle=black, fonttitle=\bfseries, boxrule=0.5mm, width=\textwidth]
\textbf{MALLM solution:} Biodiversity provides critical support for drug discovery and the availability of economic resources. \\
\textbf{Real solution:} Not included in context!
\end{tcolorbox}

\end{tcolorbox}


\begin{tcolorbox}[breakable, enhanced jigsaw, title=Turn 1, colframe=black!10, coltitle=black, fonttitle=\bfseries, boxrule=0.5mm, width=\textwidth]

\begin{tcolorbox}[title=Economist \hfill \tiny{\textnormal{A researcher who studies the impact of biodiversity on economic systems, with expertise in market analysis and the economic benefits of natural resources.}}, colback=lightyellow, colframe=orange!30, coltitle=black, fonttitle=\bfseries, boxrule=0.5mm, width=\textwidth]
\begin{minipage}{1.0\textwidth}
As an economist, I'd like to propose a solution that addresses the decline in output from the pharmaceutical sector since the mid-1980s and leverages the untapped potential of biodiversity for drug discovery and economic innovation. Here's a multi-faceted approach: **Solution:** Establish a Public-Private Partnership (PPP) to incentivize the exploration and development of natural products for pharmaceuticals, while ensuring responsible and sustainable bioprospecting practices.
\end{minipage}
\end{tcolorbox}

\begin{tcolorbox}[title=Ethnobotanist \hfill \tiny{\textnormal{A scientist who studies the traditional uses of plants by indigenous communities and their potential for medicinal applications, with expertise in deciphering the cultural and historical context of natural resources.}}, colback=lightyellow, colframe=orange!30, coltitle=black, fonttitle=\bfseries, boxrule=0.5mm, width=\textwidth]
\begin{minipage}{1.0\textwidth}
As an ethnobotanist, I agree with the current solution in principle. The Public-Private Partnership (PPP) framework has the potential to bring together the necessary stakeholders and incentivize the exploration and development of natural products for pharmaceuticals while ensuring responsible and sustainable bioprospecting practices. The multistep approach outlined is comprehensive and addresses various aspects of the issue.
\end{minipage}
\end{tcolorbox}

\begin{tcolorbox}[title=Research Biologist \hfill \tiny{\textnormal{A scientist who studies the properties and potential applications of natural products from a wide range of living organisms, with expertise in the discovery and development of new medicines from biodiversity sources.}}, colback=lightyellow, colframe=orange!30, coltitle=black, fonttitle=\bfseries, boxrule=0.5mm, width=\textwidth]
\begin{minipage}{1.0\textwidth}
I, as a Research Biologist, AGREE with the current solution and the additional suggestions proposed by the Ethnobotanist. The Public-Private Partnership (PPP) framework is a well-rounded approach that addresses the decline in output from the pharmaceutical sector by providing a framework for responsible and sustainable bioprospecting, while promoting economic innovation and job creation. I particularly appreciate the emphasis on community engagement and empowerment, benefit-sharing agreements, traditional knowledge protection, and capacity building for ecosystem-based medicine. 
\end{minipage}
\end{tcolorbox}

\end{tcolorbox}


\begin{tcolorbox}[breakable, enhanced jigsaw, title=Turn 2, colframe=black!10, coltitle=black, fonttitle=\bfseries, boxrule=0.5mm, width=\textwidth]

\begin{tcolorbox}[title=Economist \hfill \tiny{\textnormal{A researcher who studies the impact of biodiversity on economic systems, with expertise in market analysis and the economic benefits of natural resources.}}, colback=lightyellow, colframe=orange!30, coltitle=black, fonttitle=\bfseries, boxrule=0.5mm, width=\textwidth]
\begin{minipage}{1.0\textwidth}
    The current solution has been improved with additional considerations from the ethnobotanist and research biologist. I notice that the additional suggestions enhance the comprehensive approach, addressing community engagement, benefit-sharing, and traditional knowledge protection.
\end{minipage}
\end{tcolorbox}

\begin{tcolorbox}[title=Ethnobotanist \hfill \tiny{\textnormal{A scientist who studies the traditional uses of plants by indigenous communities and their potential for medicinal applications, with expertise in deciphering the cultural and historical context of natural resources.}}, colback=lightyellow, colframe=orange!30, coltitle=black, fonttitle=\bfseries, boxrule=0.5mm, width=\textwidth]
\begin{minipage}{1.0\textwidth}
As an ethnobotanist, I agree with the current solution in principle. The Public-Private Partnership (PPP) framework has the potential to bring together the necessary stakeholders and incentivize the exploration and development of natural products for pharmaceuticals while ensuring responsible and sustainable bioprospecting practices. The multistep approach outlined is comprehensive and addresses various aspects of the issue.
\end{minipage}
\end{tcolorbox}

\begin{tcolorbox}[title=Research Biologist \hfill \tiny{\textnormal{A scientist who studies the properties and potential applications of natural products from a wide range of living organisms, with expertise in the discovery and development of new medicines from biodiversity sources.}}, colback=lightyellow, colframe=orange!30, coltitle=black, fonttitle=\bfseries, boxrule=0.5mm, width=\textwidth]
\begin{minipage}{1.0\textwidth}
I, as a Research Biologist, AGREE with the current solution and the additional suggestions proposed by the Ethnobotanist. The Public-Private Partnership (PPP) framework is a well-rounded approach that addresses the decline in output from the pharmaceutical sector by providing a framework for responsible and sustainable bioprospecting, while promoting economic innovation and job creation. I particularly appreciate the emphasis on community engagement and empowerment, benefit-sharing agreements, traditional knowledge protection, and capacity building for ecosystem-based medicine. These considerations will ensure that local communities are involved in the decision-making process, receive fair compensation for their involvement, and that their traditional knowledge and cultural heritage are respected.

\end{minipage}
\end{tcolorbox}

\end{tcolorbox}


\begin{tcolorbox}[breakable, enhanced jigsaw, title=Turn 3, colframe=black!10, coltitle=black, fonttitle=\bfseries, boxrule=0.5mm, width=\textwidth]

\begin{tcolorbox}[title=Economist \hfill \tiny{\textnormal{A researcher who studies the impact of biodiversity on economic systems, with expertise in market analysis and the economic benefits of natural resources.}}, colback=lightyellow, colframe=orange!30, coltitle=black, fonttitle=\bfseries, boxrule=0.5mm, width=\textwidth]
\begin{minipage}{1.0\textwidth}
[AGREE] The revised solution has effectively incorporated the suggestions from the ethnobotanist and research biologist, addressing community engagement, benefit-sharing, traditional knowledge protection, and other crucial aspects of bioprospecting. The emphasis on responsible and sustainable bioprospecting ensures that the development of novel pharmaceuticals benefits both humans and the environment. To further refine the solution, I agree with the research biologist's suggestion to develop a clear and measurable framework for assessing the impact of the Public-Private Partnership (PPP) on the pharmaceutical sector, local communities, and the environment. Establishing Key Performance Indicators (KPIs) and a monitoring and evaluation process will help ensure that the PPP is held accountable for its impact and remains a responsible and effective approach to bioprospecting.
\end{minipage}
\end{tcolorbox}

\begin{tcolorbox}[title=Ethnobotanist \hfill \tiny{\textnormal{A scientist who studies the traditional uses of plants by indigenous communities and their potential for medicinal applications, with expertise in deciphering the cultural and historical context of natural resources.}}, colback=lightyellow, colframe=orange!30, coltitle=black, fonttitle=\bfseries, boxrule=0.5mm, width=\textwidth]
\begin{minipage}{1.0\textwidth}
As an ethnobotanist, I agree with the current solution in principle. The Public-Private Partnership (PPP) framework has the potential to bring together the necessary stakeholders and incentivize the exploration and development of natural products for pharmaceuticals while ensuring responsible and sustainable bioprospecting practices. The multistep approach outlined is comprehensive and addresses various aspects of the issue.
\end{minipage}
\end{tcolorbox}

\begin{tcolorbox}[title=Research Biologist \hfill \tiny{\textnormal{A scientist who studies the properties and potential applications of natural products from a wide range of living organisms, with expertise in the discovery and development of new medicines from biodiversity sources.}}, colback=lightyellow, colframe=orange!30, coltitle=black, fonttitle=\bfseries, boxrule=0.5mm, width=\textwidth]
\begin{minipage}{1.0\textwidth}
I, as a Research Biologist, AGREE with the current solution and the additional suggestions proposed by the Ethnobotanist. The Public-Private Partnership (PPP) framework is a well-rounded approach that addresses the decline in output from the pharmaceutical sector by providing a framework for responsible and sustainable bioprospecting, while promoting economic innovation and job creation. I particularly appreciate the emphasis on community engagement and empowerment, benefit-sharing agreements, traditional knowledge protection, and capacity building for ecosystem-based medicine. These considerations will ensure that local communities are involved in the decision-making process, receive fair compensation for their involvement, and that their traditional knowledge and cultural heritage are respected.

\end{minipage}
\end{tcolorbox}



\end{tcolorbox}


\begin{tcolorbox}[title=Result, colframe=black!10, coltitle=black, fonttitle=\bfseries, boxrule=0.5mm, width=\textwidth]
\textbf{Economist final answer:} Biodiversity

\textbf{Ethnobotanist final answer:} Biodiversity provides critical support for drug discovery and the availability of economic resources.

\textbf{Research Biologist final answer:} Biodiversity

\bigskip

\textbf{Economist} voted for answer from \textbf{Ethnobotanist}

\textbf{Ethnobotanist} voted for answer from \textbf{Ethnobotanist}

\textbf{Research Biologist} voted for answer from \textbf{Ethnobotanist}


\end{tcolorbox}

\label{example:trick_discussion}

\newpage
\subsection{Bad Voting Discussion}
In this discussion, the agents were tricked by one agent who came up with a solution not provided in the context.

\vspace{5mm}

\definecolor{lightyellow}{RGB}{255, 255, 204}

\begin{tcolorbox}[title=Overview, colframe=black!10, coltitle=black, fonttitle=\bfseries, boxrule=0.5mm, width=\textwidth]

\textbf{Task:} Answer the following question. If the question is not answerable with the provided information, write '[UNKNOWN]'.

\textbf{Input:} Where must enclitics appear in an English sentence?

\textbf{Context:} Because Czech uses grammatical case to convey word function in a sentence (instead of relying on word order, as English does), its word order is flexible. As a pro-drop language, in Czech an intransitive sentence can consist of only a verb; information about its subject is encoded in the verb. Enclitics (primarily auxiliary verbs and pronouns) must appear in the second slot of a sentence, after the first stressed unit. The first slot must contain a subject and object, a main form of a verb, an adverb or a conjunction (except for the light conjunctions a, "and", i, "and even" or ale, "but").

\begin{tcolorbox}[coltitle=black, fonttitle=\bfseries, boxrule=0.5mm, width=\textwidth]
\textbf{MALLM solution:} C) In an English sentence, enclitics (auxiliary verbs and pronouns) tend to appear next to the main verb, usually before any other adverbs or modifiers, but their placement can vary depending on sentence structure and context. \\
\textbf{Real solution:} Not included in context!
\end{tcolorbox}

\end{tcolorbox}

\begin{tcolorbox}[breakable, enhanced jigsaw, title=Turn 1, colframe=black!10, coltitle=black, fonttitle=\bfseries, boxrule=0.5mm, width=\textwidth]

\begin{tcolorbox}[title=Linguist \hfill \tiny{\textnormal{A scholar specializing in Slavic languages, experienced in analyzing the grammatical structures of Czech and other languages.}}, colback=lightyellow, colframe=orange!30, coltitle=black, fonttitle=\bfseries, boxrule=0.5mm, width=\textwidth]
\begin{minipage}{1.0\textwidth}
As a linguist, I'd like to analyze the information given about Czech grammar. We know that in Czech, enclitics (auxiliary verbs and pronouns) must appear in the second syntactic slot of a sentence, after the first stressed unit. Now, we're trying to apply this rule to an English sentence. English does not have a similar grammatical case system, and its word order is more fixed than Czech. However, we can try to find a rough equivalent. So, my proposed solution is that in an English sentence, enclitics (auxiliary verbs and pronouns) tend to appear next to the main verb, and usually before any other adverbs or modifiers. What do you think? Would you like to refine this proposal or add any additional considerations?
\end{minipage}
\end{tcolorbox}

\begin{tcolorbox}[title=English Language Teacher \hfill \tiny{\textnormal{An expert in teaching English grammar and language structures, familiar with the challenges of translating and comparing languages like Czech.}}, colback=lightyellow, colframe=orange!30, coltitle=black, fonttitle=\bfseries, boxrule=0.5mm, width=\textwidth]
\begin{minipage}{1.0\textwidth}
[DISAGREE] While I understand the thought process behind the proposed solution, I think it's essential to consider the unique characteristics of the English language. The idea that enclitics in English tend to appear next to the main verb is accurate, but it's not as rigid as the rule in Czech. In Czech, enclitics must appear in the second syntactic slot, which implies a more fixed position, whereas in English, the placement of auxiliary verbs and pronouns can vary depending on sentence structure and context.
\end{minipage}
\end{tcolorbox}

\begin{tcolorbox}[title=Grammar Analyst \hfill \tiny{\textnormal{A meticulous researcher who has studied the intricacies of languages and their grammatical structures, with a focus on comparing and contrasting language features.}}, colback=lightyellow, colframe=orange!30, coltitle=black, fonttitle=\bfseries, boxrule=0.5mm, width=\textwidth]
\begin{minipage}{1.0\textwidth}
I agree with the English Language Teacher's revised solution. The initial proposal tried to apply the Czech rule to English, but the English language has its unique characteristics that require a more nuanced approach. The revised solution accurately captures the flexibility of English grammar, acknowledging that the placement of enclitics can vary depending on sentence structure and context.
\end{minipage}
\end{tcolorbox}

\end{tcolorbox}

\begin{tcolorbox}[breakable, enhanced jigsaw, title=Turn 2, colframe=black!10, coltitle=black, fonttitle=\bfseries, boxrule=0.5mm, width=\textwidth]

\begin{tcolorbox}[title=Linguist \hfill \tiny{\textnormal{A scholar specializing in Slavic languages, experienced in analyzing the grammatical structures of Czech and other languages.}}, colback=lightyellow, colframe=orange!30, coltitle=black, fonttitle=\bfseries, boxrule=0.5mm, width=\textwidth]
\begin{minipage}{1.0\textwidth}
I agree with the revised solution proposed by the English Language Teacher and the Grammar Analyst. The revised solution accurately captures the flexibility of English grammar and acknowledges that the placement of enclitics (auxiliary verbs and pronouns) can vary depending on sentence structure and context. * The placement of enclitics in English is not as rigid as in Czech, and their placement can vary depending on sentence structure and context. * English sentence structure is more complex, including indirect objects, passives, and compound verb forms, which can affect the placement of auxiliary verbs and pronouns.
\end{minipage}
\end{tcolorbox}

\begin{tcolorbox}[title=English Language Teacher \hfill \tiny{\textnormal{An expert in teaching English grammar and language structures, familiar with the challenges of translating and comparing languages like Czech.}}, colback=lightyellow, colframe=orange!30, coltitle=black, fonttitle=\bfseries, boxrule=0.5mm, width=\textwidth]
\begin{minipage}{1.0\textwidth}
While I understand the thought process behind the proposed solution, I think it's essential to consider the unique characteristics of the English language. The idea that enclitics in English tend to appear next to the main verb is accurate, but it's not as rigid as the rule in Czech. In Czech, enclitics must appear in the second syntactic slot, which implies a more fixed position, whereas in English, the placement of auxiliary verbs and pronouns can vary depending on sentence structure and context. Additionally, the provided examples, while correct in themselves, don't fully demonstrate the complexity of English sentence structure. For instance, consider sentences with indirect objects, passives, or compound verb forms, where the placement of auxiliary verbs and pronouns can be more nuanced.
\end{minipage}
\end{tcolorbox}

\begin{tcolorbox}[title=Grammar Analyst \hfill \tiny{\textnormal{A meticulous researcher who has studied the intricacies of languages and their grammatical structures, with a focus on comparing and contrasting language features.}}, colback=lightyellow, colframe=orange!30, coltitle=black, fonttitle=\bfseries, boxrule=0.5mm, width=\textwidth]
\begin{minipage}{1.0\textwidth}
I agree with the English Language Teacher's revised solution. The initial proposal tried to apply the Czech rule to English, but the English language has its unique characteristics that require a more nuanced approach. The revised solution accurately captures the flexibility of English grammar, acknowledging that the placement of enclitics can vary depending on sentence structure and context. C) In an English sentence, enclitics (auxiliary verbs and pronouns) tend to appear next to the main verb, usually before any other adverbs or modifiers, but their placement can vary depending on sentence structure and context. I believe this revised solution provides a more accurate and practical approximation of the English language's enclitic behavior, considering both the unique characteristics of English and the complexity of sentence structure.
\end{minipage}
\end{tcolorbox}

\end{tcolorbox}

\begin{tcolorbox}[breakable, enhanced jigsaw, title=Turn 3, colframe=black!10, coltitle=black, fonttitle=\bfseries, boxrule=0.5mm, width=\textwidth]

\begin{tcolorbox}[title=Linguist \hfill \tiny{\textnormal{A scholar specializing in Slavic languages, experienced in analyzing the grammatical structures of Czech and other languages.}}, colback=lightyellow, colframe=orange!30, coltitle=black, fonttitle=\bfseries, boxrule=0.5mm, width=\textwidth]
\begin{minipage}{1.0\textwidth}
The current solution accurately captures the flexibility of English grammar and acknowledges that the placement of enclitics (auxiliary verbs and pronouns) can vary depending on sentence structure and context. There's no need to improve it further. The revised solution: C) In an English sentence, enclitics (auxiliary verbs and pronouns) tend to appear next to the main verb, usually before any other adverbs or modifiers, but their placement can vary depending on sentence structure and context. is a clear and accurate representation of the English language's enclitic behavior.
\end{minipage}
\end{tcolorbox}

\begin{tcolorbox}[title=English Language Teacher \hfill \tiny{\textnormal{An expert in teaching English grammar and language structures, familiar with the challenges of translating and comparing languages like Czech.}}, colback=lightyellow, colframe=orange!30, coltitle=black, fonttitle=\bfseries, boxrule=0.5mm, width=\textwidth]
\begin{minipage}{1.0\textwidth}
While I understand the thought process behind the proposed solution, I think it's essential to consider the unique characteristics of the English language. The idea that enclitics in English tend to appear next to the main verb is accurate, but it's not as rigid as the rule in Czech. In Czech, enclitics must appear in the second syntactic slot, which implies a more fixed position, whereas in English, the placement of auxiliary verbs and pronouns can vary depending on sentence structure and context. Here's a revised solution: C) In an English sentence, enclitics (auxiliary verbs and pronouns) tend to appear next to the main verb, usually before any other adverbs or modifiers, but their placement can vary depending on sentence structure and context. What do you think?
\end{minipage}
\end{tcolorbox}

\begin{tcolorbox}[title=Grammar Analyst \hfill \tiny{\textnormal{A meticulous researcher who has studied the intricacies of languages and their grammatical structures, with a focus on comparing and contrasting language features.}}, colback=lightyellow, colframe=orange!30, coltitle=black, fonttitle=\bfseries, boxrule=0.5mm, width=\textwidth]
\begin{minipage}{1.0\textwidth}
I agree with the English Language Teacher's revised solution. The initial proposal tried to apply the Czech rule to English, but the English language has its unique characteristics that require a more nuanced approach. The revised solution accurately captures the flexibility of English grammar, acknowledging that the placement of enclitics can vary depending on sentence structure and context. The revised solution is an improvement over the initial proposal because it takes into account the complexity of English sentence structure, including indirect objects, passives, and compound verb forms, which can affect the placement of auxiliary verbs and pronouns. Additionally, the revised solution provides a more accurate description of the typical placement of enclitics in English sentences, namely next to the main verb, usually before any adverbs or modifiers.
\end{minipage}
\end{tcolorbox}

\end{tcolorbox}

\begin{tcolorbox}[title=Result, colframe=black!10, coltitle=black, fonttitle=\bfseries, boxrule=0.5mm, width=\textwidth]
\textbf{Linguist final answer:} C) In an English sentence, enclitics (auxiliary verbs and pronouns) tend to appear next to the main verb, usually before any other adverbs or modifiers, but their placement can vary depending on sentence structure and context.

\textbf{English Language Teacher final answer:} [UNKNOWN]

\textbf{Grammar Analyst final answer:} [UNKNOWN]

\bigskip

\textbf{Linguist} voted for answer from \textbf{Linguist}

\textbf{English Language Teacher} voted for answer from \textbf{Linguist}

\textbf{Grammar Analyst} voted for answer from \textbf{Linguist}


\end{tcolorbox}

\label{example:failed_discussion}

\newpage
\subsection{Majority Consensus Discussion}
In this discussion, the agents discussed only one round, as they already had a high enough agreement score.

\vspace{5mm}

\definecolor{lightyellow}{RGB}{255, 255, 204}


\begin{tcolorbox}[title=Overview, colframe=black!10, coltitle=black, fonttitle=\bfseries, boxrule=0.5mm, width=\textwidth]

\textbf{Task:} Answer the following question. If the question is not answerable with the provided information, write '[UNKNOWN]'.

\textbf{Input:} What choir from the area has sung at the Guildhall?

\textbf{Context:} Southampton has two large live music venues, the Mayflower Theatre (formerly the Gaumont Theatre) and the Guildhall. The Guildhall has seen concerts from a wide range of popular artists including Pink Floyd, David Bowie, Delirious?, Manic Street Preachers, The Killers, The Kaiser Chiefs, Amy Winehouse, Lostprophets, The Midnight Beast, Modestep, and All Time Low. It also hosts classical concerts presented by the Bournemouth Symphony Orchestra, City of Southampton Orchestra, Southampton Concert Orchestra, Southampton Philharmonic Choir and Southampton Choral Society.  

\begin{tcolorbox}[coltitle=black, fonttitle=\bfseries, boxrule=0.5mm, width=\textwidth]
\textbf{MALLM solution:} Southampton Choral Society \\
\textbf{Real solution:} Southampton Philharmonic Choir
\end{tcolorbox}
\end{tcolorbox}


\begin{tcolorbox}[breakable, enhanced jigsaw, title=Turn 1, colframe=black!10, coltitle=black, fonttitle=\bfseries, boxrule=0.5mm, width=\textwidth]

\begin{tcolorbox}[title=Music Connoisseur (SOLUTION) \hfill \tiny{\textnormal{A knowledgeable musician who is familiar with various music genres and has attended concerts at the Guildhall.}}, colback=lightyellow, colframe=orange!30, coltitle=black, fonttitle=\bfseries, boxrule=0.5mm, width=\textwidth]
\begin{minipage}{1.0\textwidth}
As a Music Historian, I've been through the archives and checked the event schedules of the Guildhall. According to my research, one of the choirs that has sung at the Guildhall is... Southampton Choral Society!
\end{minipage}
\end{tcolorbox}

\begin{tcolorbox}[title=Choir Conductor (AGREED) \hfill \tiny{\textnormal{A dedicated choral conductor with extensive experience in leading local choirs and orchestras, familiar with the performances at the Guildhall.}}, colback=lightyellow, colframe=orange!30, coltitle=black, fonttitle=\bfseries, boxrule=0.5mm, width=\textwidth]
\begin{minipage}{1.0\textwidth}
[AGREE] As a Choir Member of the Southampton Philharmonic Choir who has performed at the Guildhall, I can confirm that the Southampton Choral Society is indeed one of the choirs that have sung at the Guildhall. Our experiences have shown that we often share the stage with the Southampton Choral Society during joint performances with orchestras and other ensembles. So, I'm happy to confirm that Southampton Choral Society is a correct answer. 
\end{minipage}
\end{tcolorbox}
\end{tcolorbox}


\label{example:majority_discussion}

\newpage
\section{AI Usage Card}
{\sffamily
    \centering
    \tcbset{colback=white!10!white}
    \begin{tcolorbox}[
        title={\large \textbf{AI Usage Card} \hfill \makebox{\qrcode[height=1cm]{https://ai-cards.org}}},
        breakable,
        boxrule=0.7pt,
        width=\textwidth,
        center,
        before lower={\footnotesize{AI Usage Card v1.1 \hfill \url{https://ai-cards.org} \hfill \href{https://jpwahle.com/ai-cards-preprint}{PDF} | \href{https://jpwahle.com/cite/jcdl2023wahle.bib}{BibTeX}}},
        segmentation empty,
        halign lower=center,
        collower=black,
        coltitle=black, %
        colbacklower=gray!20, %
        colbacktitle=gray!20  %
        ]
        \vspace{-10pt}
        \footnotesize{
            \begin{longtable}{>
            {\raggedright\arraybackslash}p{.15\textwidth}>{\raggedright\arraybackslash}p{.25\textwidth}>{\raggedright\arraybackslash}p{.25\textwidth}>{\raggedright\arraybackslash}p{.25\textwidth}}
              
                {\color{LightBlue} \MakeUppercase{Project Details}} \newline 
                & {\color{LightBlue} \MakeUppercase{Project Name}} \newline Voting or Consensus? Decision-Making in Multi-Agent Debate
                & {\color{LightBlue} \MakeUppercase{Domain}} \newline Paper
                & {\color{LightBlue} \MakeUppercase{Key Application}} \newline Natural Language Processing
                
                \\
                
                {\color{LightBlue} \MakeUppercase{Contact(s)}}  
                & {\color{LightBlue} \MakeUppercase{Name(s)}} 
                & {\color{LightBlue} \MakeUppercase{Email(s)}} 
                & {\color{LightBlue} \MakeUppercase{Affiliation(s)}} 
                \\ & Lars Benedikt Kaesberg & larsbenedikt.kaesberg@uni-goettingen.de & University Göttingen
                
                \\
                
                {\color{LightBlue} \MakeUppercase{Model(s)}} 
                & {\color{LightBlue} \MakeUppercase{Model Name(s)}} 
                & {\color{LightBlue} \MakeUppercase{Version(s)}} 
                \\ & Llama & 3 70b, 3 8b
                \\ & ChatGPT & 4o, o1
                \\ & Github Copilot & latest

                \\
                \cmidrule{1-4}
                \\
                {\color{LightBlue} \MakeUppercase{Literature Review}} \newline 
                & {\color{LightBlue} \MakeUppercase{Finding literature}} \newline ChatGPT 
                & {\color{LightBlue} \MakeUppercase{Finding examples from known literature or adding literature for existing statements}} \newline ChatGPT  
                & {\color{gray} \MakeUppercase{Comparing literature}}  
                \\
                \cmidrule{2-4}
                \\        
                {\color{LightBlue} \MakeUppercase{Writing}} \newline    
                & {\color{LightBlue} \MakeUppercase{Generating new text based on instructions}} \newline Llama 
                & {\color{LightBlue} \MakeUppercase{Assisting in improving own content or Paraphrasing related work}} \newline ChatGPT 
                & {\color{gray} \MakeUppercase{Putting other works in perspective}}  
                \\
                \cmidrule{2-4}
                \\
                {\color{LightBlue} \MakeUppercase{Coding}} \newline 
                & {\color{LightBlue} \MakeUppercase{Generating new code based on descriptions or existing code}} \newline ChatGPT \newline Github Copilot 
                & {\color{LightBlue} \MakeUppercase{Refactoring and optimizing existing code}} \newline ChatGPT \newline Github Copilot 
                & {\color{gray} \MakeUppercase{Comparing aspects of existing code}}  
                \\
                \cmidrule{1-4}
                \\
        
                {\color{LightBlue} \MakeUppercase{Ethics}} \newline    
                & {\color{LightBlue} \MakeUppercase{Why did we use AI for this project?}} \newline Efficiency / Speed \newline Scalability \newline Expertise Access  
                & {\color{LightBlue} \MakeUppercase{What steps are we taking to mitigate errors of AI?}} \newline None
                & {\color{LightBlue} \MakeUppercase{What steps are we taking to minimize the chance of harm or inappropriate use of AI?}} \newline None

                \\
                \cmidrule{1-4}
                \\
            \end{longtable}
        
        

        \medskip
        
        \textbf{\color{LightBlue} \MakeUppercase{The corresponding authors verify and agree with the modifications or generations of their  used AI-generated content}}
        }
        
        \tcblower
    \end{tcolorbox}
}


% \appendix
% \onecolumn
% \section{You \emph{can} have an appendix here.}

% You can have as much text here as you want. The main body must be at most $8$ pages long.
% For the final version, one more page can be added.
% If you want, you can use an appendix like this one.  

% The $\mathtt{\backslash onecolumn}$ command above can be kept in place if you prefer a one-column appendix, or can be removed if you prefer a two-column appendix.  Apart from this possible change, the style (font size, spacing, margins, page numbering, etc.) should be kept the same as the main body.
%%%%%%%%%%%%%%%%%%%%%%%%%%%%%%%%%%%%%%%%%%%%%%%%%%%%%%%%%%%%%%%%%%%%%%%%%%%%%%%
%%%%%%%%%%%%%%%%%%%%%%%%%%%%%%%%%%%%%%%%%%%%%%%%%%%%%%%%%%%%%%%%%%%%%%%%%%%%%%%


\end{document}


% This document was modified from the file originally made available by
% Pat Langley and Andrea Danyluk for ICML-2K. This version was created
% by Iain Murray in 2018, and modified by Alexandre Bouchard in
% 2019 and 2021 and by Csaba Szepesvari, Gang Niu and Sivan Sabato in 2022.
% Modified again in 2023 and 2024 by Sivan Sabato and Jonathan Scarlett.
% Previous contributors include Dan Roy, Lise Getoor and Tobias
% Scheffer, which was slightly modified from the 2010 version by
% Thorsten Joachims & Johannes Fuernkranz, slightly modified from the
% 2009 version by Kiri Wagstaff and Sam Roweis's 2008 version, which is
% slightly modified from Prasad Tadepalli's 2007 version which is a
% lightly changed version of the previous year's version by Andrew
% Moore, which was in turn edited from those of Kristian Kersting and
% Codrina Lauth. Alex Smola contributed to the algorithmic style files.
