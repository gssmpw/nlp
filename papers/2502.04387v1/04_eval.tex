% \newpage
\section{Evaluation}\label{sec:expmts}

\subsection{Experimental Setup}

We conduct experiments on multilingual scenarios, where clients with diverse high- and low-resource languages can collaboratively learn how to personalize a given base model to better cater to their language preferences. In all experiments, we divide clients in two pools, \seen{} and \unseen{}, where only the clients in the \seen{} pool actively participate in federated training. We set the maximum number of communication rounds for training the PSG to $150$, randomly sampling $10\%$ of participating clients every round. We use Adam~\cite{Kingma_2014} as the default optimizer for all our experiments. We evaluate on resource budgets $r=2,4,8,16$ where the total rank budget is $r \cdot L$. We summarize the FL scenarios that we consider in our experiments, leaving comprehensive details in Appendix~\ref{appendix:experiments}.

\subsubsection{Tasks, Models, and Datasets}

\noindent\textbf{Text Classification.}~We adopt the pretrained multilingual BERT~\cite{BERT} (mBERT) for all text classification experiments. For datasets, we introduce additional data heterogeneity to the simulated FL setups, XNLI~\cite{XNLI} and MasakhaNEWS~\cite{MasakhaNEWS}, proposed in PE\_FL~\cite{zhao2023breaking}. 

For our XNLI setup, we sample 2k instances for train and $500$ for test in each pool. In contrast to PE\_FL, which had $15$ clients ($1$ language per client), we divide the data equally among $20$ clients for each language. We then adopt the latent Dirichlet allocation (LDA) partition method~\cite{hsu2019measuring, yurochkin2019bayesian}, $y \sim Dir(\alpha)$, to simulate non-IID label shifts among these clients, with $\alpha=0.5$. Hence, there is a total of $600$ clients ($15$ languages $\cdot$ $20$ clients per language $\cdot$ $2$ pools), consisting of both label and feature heterogeneity.

For MasakhaNEWS, we first split the data in each of the $16$ languages by half for each pool. Similar to our XNLI setup, we divide each language's data equally among 10 clients and adopt LDA with $\alpha=0.5$, resulting in $320$ clients in total. Differing from our XNLI setup, each language varies in the amount of samples, adding another layer of data heterogeneity to the setup: quantity skew.

\noindent\textbf{Instruction-Tuning Generation.}~We adopt pretrained MobileLLaMA-1.4B~\cite{mobilellama} and Llama-3.2-3B~\cite{llama3}, which are representative of commonly supported model sizes on recent high-end edge devices~\cite{openelm2024icmlw,2024_mobilequant,edgellm2024tmc}. For each model, we run experiments on the recent Fed-Aya dataset. Fed-Aya is a real-world FL dataset naturally partitioned by annotator ID and each client has data with up to $4$ languages. Out of a total of $38$ clients, we select $8$ clients for our \unseen{} pool. We also split each client's data $80\%/20\%$ for train and test, respectively. Fig.~\ref{fig:fed-aya} shows the distribution of predominant languages, where predominant refers to the client's most commonly used language, in our setup.

\begin{figure}
    \small
    \centering
    \includegraphics[width=0.9\columnwidth]{figures/fedaya_pool.png}
    % \captionsetup{font=small,labelfont=bf}
    \vspace{-1.5em}
    \caption{The number of clients in each predominant language in our Fed-Aya setup.}
    \label{fig:fed-aya}
    \vspace{-2em}
\end{figure}

\subsubsection{Complementary Approaches}\label{sec:complementary}

We show \method{}'s compatibility with both off-the-shelf models and models trained using existing FL methods. Concretely, given a pretrained model, we obtain a \basemodel{} using one of the following approaches: 

\noindent\textbf{Standard FL.} We further train the pretrained model federatedly on the \seen{} pool, either using existing PEFT methods or full fine-tuning~\cite{fedllm-bench, fedpeft},  

\noindent\textbf{Personalized FL.} We adopt two recent personalized FL works: \textit{i)} FedDPA-T~\cite{FedDPA}, which learns per-client personalized LoRA modules in addition to global LoRA modules, and \textit{ii)} DEPT (SPEC)~\cite{DEPT}, which learns per-client personalized token and positional embeddings while keeping the rest of the model shared. The \basemodel{} hence differs for each client.

\noindent\textbf{Off-the-shelf.} We use the pretrained model as the \basemodel{} without additional training.

\begin{table*}[t]
\centering
% \captionsetup{justification=centering}
\caption{Mean±SD Accuracy of each language across 3 different seeds for \seen{} clients of our MasakhaNEWS setup. The pretrained model is trained using Standard FL with full fine-tuning and the resulting \basemodel{} is personalized to each client given a baseline approach.}
\label{tab:masakha_seen}
\begin{scriptsize}\resizebox{0.98\textwidth}{!}{
\begin{tabular}{c|l|l|l|l|l|l|l|l|l|l|l|l|l|l|l|l|l|c}
\toprule
% \textbf{Lora Rank}  
\textbf{$\mathbf{r}$} & \multicolumn{1}{c|}{\textbf{Approach}} & \multicolumn{1}{c|}{\textbf{eng}} & \multicolumn{1}{c|}{\textbf{som}} & \multicolumn{1}{c|}{\textbf{run}} & \multicolumn{1}{c|}{\textbf{fra}} & \multicolumn{1}{c|}{\textbf{lin}} & \multicolumn{1}{c|}{\textbf{ibo}} & \multicolumn{1}{c|}{\textbf{amh}} & \multicolumn{1}{c|}{\textbf{hau}} & \multicolumn{1}{c|}{\textbf{pcm}} & \multicolumn{1}{c|}{\textbf{swa}} & \multicolumn{1}{c|}{\textbf{orm}} & \multicolumn{1}{c|}{\textbf{xho}} & \multicolumn{1}{c|}{\textbf{yor}} & \multicolumn{1}{c|}{\textbf{sna}} & \multicolumn{1}{c|}{\textbf{lug}} & \multicolumn{1}{c|}{\textbf{tir}} & \textbf{Wins} 
\\ \midrule
% \multirow{5}{*}{1}  & LoRA                                   & 90.01±0.10                        & 59.41±0.32                        & 81.16±0.29                        & 88.63±0.00                        & 83.53±0.54                        & 78.97±0.00                        & 45.74±0.00                        & 75.58±0.15                        & 96.05±0.00                        & 78.99±0.34                        & 64.20±0.00                         & 69.37±0.32                        & 79.02±0.00                        & 78.80±0.00                         & 67.57±0.00                        & 44.85±0.00                        & 0             \\ % \cline{2-19} 
%                     & AdaLoRA                              & 89.87±0.00                        & 59.63±0.32                        & 80.96±0.29                        & 88.63±0.00                        & 83.14±0.54                        & 78.97±0.00                        & 45.21±0.00                        & 75.16±0.00                        & 96.05±0.00                        & 78.99±0.34                        & 64.20±0.00                         & 69.59±0.00                        & 79.02±0.00                        & 78.80±0.00                         & 67.87±0.42                        & 44.85±0.00                        & 0             \\ % \cline{2-19} 
%                     & BayesTune-LoRA                            & 89.87±0.00                        & 59.63±0.32                        & 81.16±0.29                        & 88.63±0.00                        & 83.91±0.00                        & 78.97±0.00                        & 45.21±0.00                        & 75.05±0.15                        & 96.05±0.00                        & 78.71±0.20                        & 64.20±0.00                         & 69.37±0.32                        & 78.86±0.23                        & 78.80±0.00                         & 67.57±0.00                        & 44.85±0.00                        & 0             \\ % \cline{2-19} 
%                     & FedL2P                               & 89.87±0.00                        & 59.86±0.00                        & 81.37±0.00                        & 88.63±0.00                        & 82.76±0.00                        & 79.49±0.00                        & 45.21±0.00                        & 75.68±0.15                        & 96.05±0.00                        & 78.85±0.20                        & 64.20±0.00                         & 69.37±0.32                        & 79.02±0.00                        & 78.80±0.00                         & 67.57±0.00                        & 44.85±0.00                        & \textbf{0}    \\ % \cline{2-19} 
%                     & \method{}                                 & \textbf{92.05±0.10}                & \textbf{64.63±0.56}               & \textbf{85.71±0.00}                & \textbf{91.31±0.22}               & \textbf{87.36±0.00}                & \textbf{80.69±0.49}               & \textbf{52.48±0.50}                & \textbf{79.35±0.30}                & 96.05±0.00                        & \textbf{84.87±0.69}               & \textbf{68.72±0.77}               & \textbf{73.42±0.64}               & \textbf{82.93±0.00}                & 78.08±0.25                        & \textbf{69.07±0.42}               & \textbf{52.45±0.34}               & 14            \\ \hline
\multirow{5}{*}{2}  & LoRA                                   & 90.44±0.10                        & 60.09±0.32                        & 81.37±0.51                        & 88.63±0.00                        & 83.53±0.54                        & 79.83±0.24                        & 45.74±0.00                        & 75.79±0.00                        & 96.05±0.00                        & 78.99±0.00                        & 64.20±0.00                         & 69.14±0.32                        & 79.18±0.23                        & 78.80±0.00                         & 67.57±0.00                        & 44.85±0.00                        & 0             \\ % \cline{2-19} 
                    & AdaLoRA                              & 89.87±0.00                        & 59.41±0.32                        & 81.16±0.29                        & 88.63±0.00                        & 82.76±0.00                        & 78.97±0.00                        & 45.21±0.00                        & 75.05±0.15                        & 96.05±0.00                        & 78.57±0.00                        & 63.99±0.29                        & 69.59±0.00                        & 78.86±0.23                        & 78.80±0.00                         & 67.57±0.00                        & 44.85±0.00                        & 0             \\ % \cline{2-19} 
                    & BayesTune-LoRA                            & 89.87±0.00                        & 59.63±0.32                        & 81.37±0.00                        & 88.63±0.00                        & 83.14±0.54                        & 78.97±0.00                        & 45.21±0.00                        & 75.16±0.00                        & 96.05±0.00                        & 78.57±0.00                        & 63.99±0.29                        & 69.59±0.00                        & 79.02±0.00                        & 78.80±0.00                         & 67.57±0.00                        & 44.85±0.00                        & 0             \\ % \cline{2-19} 
                    & FedL2P                               & 90.72±0.59                        & 61.00±1.16                         & 81.99±0.88                        & 89.10±0.67                         & 83.91±0.00                        & 79.66±0.24                        & 45.74±0.00                        & 76.73±1.12                        & 96.05±0.00                        & 79.69±0.40                        & 64.40±0.29                         & 69.14±0.32                        & 79.67±0.61                        & 78.80±0.00                         & 67.87±0.42                        & 45.34±0.69                        & 0             \\ % \cline{2-19} 
                    & \method{}                                 & \textbf{91.98±0.00}                & \textbf{65.54±1.16}               & \textbf{87.79±0.29}               & \textbf{93.52±0.23}               & \textbf{88.51±0.94}               & \textbf{82.56±0.00}                & \textbf{51.95±0.25}               & \textbf{79.66±0.29}               & \textbf{97.59±0.31}               & \textbf{84.73±0.52}               & \textbf{72.22±1.01}               & \textbf{76.35±0.55}               & \textbf{82.11±0.23}               & \textbf{81.34±0.68}               & \textbf{69.07±0.42}               & \textbf{63.48±0.34}               & \textbf{16}   \\ \hline
\multirow{5}{*}{4}  & LoRA                                   & 91.35±0.00                        & 61.90±0.00                         & 83.23±0.00                        & 89.10±0.00                         & 84.68±0.54                        & 80.17±0.24                        & 47.34±0.00                        & 77.57±0.15                        & 96.05±0.00                        & 79.83±0.34                        & 64.61±0.29                        & 69.37±0.32                        & 80.49±0.00                        & 78.80±0.00                         & 67.87±0.42                        & 45.83±0.34                        & 0             \\ % \cline{2-19} 
                    & AdaLoRA                              & 89.87±0.00                        & 59.63±0.32                        & 80.96±0.29                        & 88.63±0.00                        & 83.53±0.54                        & 78.97±0.00                        & 45.21±0.00                        & 75.47±0.00                        & 96.05±0.00                        & 78.57±0.00                        & 63.79±0.29                        & 69.59±0.00                        & 79.02±0.00                        & 78.80±0.00                         & 67.57±0.00                        & 44.85±0.00                        & 0             \\ % \cline{2-19} 
                    & BayesTune-LoRA                            & 89.87±0.00                        & 59.41±0.32                        & 80.96±0.29                        & 88.63±0.00                        & 82.76±0.00                        & 78.97±0.00                        & 45.21±0.00                        & 75.47±0.26                        & 96.05±0.00                        & 78.71±0.20                        & 64.20±0.00                         & 69.59±0.00                        & 79.02±0.00                        & 78.80±0.00                         & 67.57±0.00                        & 44.85±0.00                        & 0             \\ % \cline{2-19} 
                    & FedL2P                               & 91.70±0.40                         & 63.49±1.16                        & 83.23±0.51                        & 90.84±0.97                        & 83.91±0.94                        & 80.00±0.42                         & 47.69±1.40                        & 79.45±1.07                        & 96.05±0.00                        & 81.51±1.03                        & 64.81±0.50                        & 71.62±1.99                        & 81.46±0.80                        & 78.80±0.00                         & 68.17±0.85                        & 46.32±0.60                        & 0             \\ % \cline{2-19} 
                    & \method{}                                 & \textbf{92.05±0.44}               & \textbf{67.12±0.64}               & \textbf{86.34±0.00}                & \textbf{93.36±0.39}               & \textbf{91.19±1.43}               & \textbf{82.56±0.84}               & \textbf{53.90±0.25}                & \textbf{81.66±0.65}               & \textbf{98.46±0.31}               & \textbf{83.89±1.10}                & \textbf{72.84±0.51}               & \textbf{77.70±0.00}                 & \textbf{82.11±0.23}               & \textbf{83.70±0.00}                 & \textbf{74.47±0.42}               & \textbf{67.65±0.00}                & \textbf{16}   \\ \hline
\multirow{5}{*}{8}  & LoRA                                   & 91.56±0.00                        & 63.95±0.00                        & 82.82±0.29                        & 90.84±0.23                        & 86.21±0.00                        & 79.83±0.24                        & 49.82±0.25                        & 78.83±0.15                        & 96.05±0.00                        & 81.79±0.52                        & 65.64±0.29                        & 72.30±0.55                         & 81.79±0.23                        & 79.89±0.00                        & 67.87±0.42                        & 44.85±0.00                        & 0             \\ % \cline{2-19} 
                    & AdaLoRA                              & 89.87±0.00                        & 59.86±0.00                        & 80.96±0.29                        & 88.63±0.00                        & 83.14±0.54                        & 78.97±0.00                        & 45.21±0.00                        & 75.37±0.15                        & 96.05±0.00                        & 78.85±0.20                        & 64.20±0.00                         & 69.59±0.00                        & 78.86±0.23                        & 78.80±0.00                         & 67.57±0.00                        & 44.85±0.00                        & 0             \\ % \cline{2-19} 
                    & BayesTune-LoRA                            & 89.94±0.10                        & 59.41±0.32                        & 80.75±0.00                        & 88.63±0.00                        & 83.14±0.54                        & 78.97±0.00                        & 45.21±0.00                        & 75.47±0.26                        & 96.05±0.00                        & 78.71±0.20                        & 64.20±0.00                         & 69.59±0.00                        & 79.02±0.00                        & 78.80±0.00                         & 67.57±0.00                        & 44.85±0.00                        & 0             \\ % \cline{2-19} 
                    & FedL2P                               & \textbf{91.91±0.26}               & 64.18±0.32                        & 83.02±0.58                        & 91.15±0.59                        & 84.29±0.54                        & 80.51±0.00                        & 48.58±0.67                        & 79.98±0.82                        & 96.27±0.31                        & 82.49±0.79                        & 66.25±1.62                        & 72.07±1.39                        & 81.95±1.05                        & 79.16±0.51                        & 68.47±1.27                        & 47.79±3.65                        & 1             \\ % \cline{2-19} 
                    & \method{}                                 & 91.42±0.20                        & \textbf{68.03±0.96}               & \textbf{85.09±0.00}                & \textbf{92.42±0.39}               & \textbf{91.95±0.00}                & \textbf{81.88±0.24}               & \textbf{55.32±0.43}               & \textbf{82.70±0.26}                & \textbf{98.68±0.00}                & \textbf{84.31±0.20}                & \textbf{73.25±0.58}               & \textbf{78.15±0.32}               & \textbf{84.72±0.46}               & \textbf{85.15±0.26}               & \textbf{74.17±0.42}               & \textbf{68.38±0.60}                & \textbf{15}   \\ \hline
\multirow{5}{*}{16} & LoRA                                   & \textbf{91.91±0.10}                & 64.63±0.00                        & 84.47±0.00                        & 91.47±0.00                        & 86.21±0.00                        & \textbf{81.54±0.00}                         & \textbf{55.85±0.43}               & 81.87±0.15                        & 96.27±0.31                        & \textbf{84.45±0.00}                & 73.66±0.77                        & 73.20±0.32                         & 82.60±0.61                         & 80.80±0.26                         & 70.87±0.42                        & 52.94±0.00                        & 3             \\ % \cline{2-19} 
                    & AdaLoRA                              & 89.87±0.00                        & 59.86±0.00                        & 80.96±0.29                        & 88.63±0.00                        & 82.76±0.00                        & 78.97±0.00                        & 45.21±0.00                        & 75.05±0.39                        & 96.05±0.00                        & 78.85±0.20                        & 63.79±0.29                        & 69.59±0.00                        & 79.02±0.00                        & 78.80±0.00                         & 67.57±0.00                        & 44.85±0.00                        & 0             \\ % \cline{2-19} 
                    & BayesTune-LoRA                            & 90.08±0.18                        & 59.63±0.32                        & 81.16±0.29                        & 88.63±0.00                        & 83.53±0.54                        & 78.97±0.00                        & 45.21±0.00                        & 75.58±0.15                        & 96.05±0.00                        & 78.57±0.00                        & 64.20±0.00                         & 69.37±0.32                        & 79.02±0.00                        & 78.80±0.00                         & 67.57±0.00                        & 44.85±0.00                        & 0             \\ % \cline{2-19} 
                    & FedL2P                               & 91.70±0.10                         & 66.22±0.32                        & \textbf{85.09±0.51}               & \textbf{92.10±0.23}                & 86.21±0.00                        & 80.34±0.24                        & 54.79±1.15                        & 81.13±0.26                        & 96.71±0.00                        & 83.89±0.40                        & 71.81±2.04                        & 75.68±0.00                        & 84.06±0.23                        & 80.61±0.68                        & 70.87±0.85                        & 57.84±4.86                        & 2             \\ % \cline{2-19} 
                    & \method{}                                 & 91.14±0.30                        & \textbf{67.80±0.85}                & 84.47±0.00                        & 91.31±0.22                        & \textbf{91.95±0.00}                & \textbf{81.54±0.42}                        & 51.95±0.25                        & \textbf{83.44±0.39}               & \textbf{98.25±0.31}               & 82.49±0.71                        & \textbf{76.54±0.50}                & \textbf{79.50±0.32}                & \textbf{84.39±0.00}                & \textbf{85.87±0.00}                & \textbf{75.38±0.43}               & \textbf{68.38±0.00}                & \textbf{10}   \\ \bottomrule
\end{tabular}
}
\end{scriptsize}
\vspace{-1.5em}
\end{table*}


\begin{table*}[t]
\centering
\caption{Mean±SD Accuracy of each language for \unseen{} clients of our MasakhaNEWS setup. The pretrained model is trained using Standard FL with full fine-tuning and the resulting \basemodel{} is personalized to each client given a baseline approach.}
\label{tab:masakha_unseen}
\resizebox{0.99\textwidth}{!}{
\begin{tabular}{c|l|l|l|l|l|l|l|l|l|l|l|l|l|l|l|l|l|c}
\toprule
% \textbf{Lora Rank} 
\textbf{$\mathbf{r}$} & \multicolumn{1}{c|}{\textbf{Approach}} & \multicolumn{1}{c|}{\textbf{eng}} & \multicolumn{1}{c|}{\textbf{som}} & \multicolumn{1}{c|}{\textbf{run}} & \multicolumn{1}{c|}{\textbf{fra}} & \multicolumn{1}{c|}{\textbf{lin}} & \multicolumn{1}{c|}{\textbf{ibo}} & \multicolumn{1}{c|}{\textbf{amh}} & \multicolumn{1}{c|}{\textbf{hau}} & \multicolumn{1}{c|}{\textbf{pcm}} & \multicolumn{1}{c|}{\textbf{swa}} & \multicolumn{1}{c|}{\textbf{orm}} & \multicolumn{1}{c|}{\textbf{xho}} & \multicolumn{1}{c|}{\textbf{yor}} & \multicolumn{1}{c|}{\textbf{sna}} & \multicolumn{1}{c|}{\textbf{lug}} & \multicolumn{1}{c|}{\textbf{tir}} & \textbf{Wins}
\\ \midrule
% \multirow{5}{*}{1}  & LoRA                                   & \textbf{90.58±0.20}                & 67.80±0.32                         & 81.37±0.00                        & 84.36±0.00                        & 79.55±0.00                        & 76.92±0.00                        & 46.28±0.00                        & 74.82±0.15                        & 90.20±0.00                         & 75.63±0.00                        & 61.96±0.00                        & 63.76±0.00                        & 79.61±0.00                        & 74.59±0.00                        & 65.18±0.00                        & 41.91±0.00                        & 1             \\ % \cline{2-19} 
%                     & AdaLoRA                              & 90.44±0.27                        & 67.12±0.32                        & 80.96±0.29                        & 84.36±0.00                        & 79.55±0.00                        & 76.92±0.00                        & 46.28±0.00                        & 74.82±0.15                        & 90.20±0.00                         & 75.77±0.20                        & 61.96±0.00                        & 63.98±0.32                        & 79.61±0.00                        & 74.59±0.00                        & 65.18±0.00                        & 41.91±0.00                        & 0             \\ % \cline{2-19} 
%                     & BayesTune-LoRA                            & 90.30±0.00                         & 67.35±0.00                        & 80.75±0.00                        & 84.36±0.00                        & 79.55±0.00                        & 76.92±0.00                        & 46.10±0.25                         & 74.82±0.15                        & 90.20±0.00                         & 75.63±0.00                        & 61.55±0.29                        & 63.76±0.00                        & 79.61±0.00                        & 74.59±0.00                        & 65.18±0.00                        & 41.91±0.00                        & 0             \\ % \cline{2-19} 
%                     & FedL2P                               & 90.15±0.27                        & 67.35±0.56                        & 81.58±0.29                        & 84.36±0.00                        & 79.55±0.00                        & 76.92±0.00                        & 46.28±0.00                        & 75.24±0.00                        & 90.20±0.00                         & 75.49±0.20                        & 61.96±0.00                        & 63.76±0.00                        & 79.94±0.23                        & 74.59±0.00                        & 65.48±0.42                        & 42.16±0.35                        & 0             \\ % \cline{2-19} 
%                     & \method{}                                 & 90.01±0.10                        & \textbf{69.84±0.32}               & \textbf{84.47±0.00}                & \textbf{86.73±0.00}                & \textbf{84.85±0.54}               & \textbf{80.52±0.73}               & \textbf{50.53±1.15}               & \textbf{77.53±0.15}               & \textbf{91.50±0.00}                 & \textbf{76.05±0.34}               & \textbf{69.94±0.87}               & \textbf{71.14±0.55}               & \textbf{82.69±0.61}               & \textbf{75.50±0.25}                & \textbf{66.66±0.42}               & \textbf{48.77±0.34}               & \textbf{15}   \\ \hline
\multirow{5}{*}{2}  & LoRA                                   & \textbf{90.72±0.00}                & 68.48±0.32                        & 81.99±0.51                        & 84.36±0.00                        & 79.55±0.00                        & 76.75±0.24                        & 46.28±0.00                        & 75.45±0.15                        & 90.20±0.00                         & 75.63±0.34                        & 61.96±0.00                        & 64.21±0.63                        & 80.10±0.00                         & 74.59±0.00                        & 65.18±0.00                        & 41.91±0.00                        & 1             \\ % \cline{2-19} 
                    & AdaLoRA                              & 90.30±0.00                         & 67.35±0.00                        & 80.75±0.00                        & 84.36±0.00                        & 79.55±0.00                        & 76.92±0.00                        & 46.28±0.00                        & 74.82±0.15                        & 90.20±0.00                         & 75.91±0.20                        & 61.55±0.29                        & 64.21±0.32                        & 79.61±0.00                        & 74.59±0.00                        & 65.18±0.00                        & 41.91±0.00                        & 0             \\ % \cline{2-19} 
                    & BayesTune-LoRA                            & 90.30±0.00                         & 67.35±0.00                        & 80.75±0.00                        & 84.36±0.00                        & 79.55±0.00                        & 76.92±0.00                        & 45.92±0.25                        & 74.61±0.00                        & 90.20±0.00                         & 75.49±0.20                        & 61.55±0.29                        & 63.76±0.00                        & 79.45±0.23                        & 74.41±0.25                        & 65.18±0.00                        & 42.65±0.00                        & 0             \\ % \cline{2-19} 
                    & FedL2P                               & 90.58±0.10                        & 68.48±0.32                        & 82.61±0.00                        & 84.36±0.00                        & 79.55±0.00                        & 76.41±0.42                        & 46.81±0.75                        & 75.97±0.59                        & 90.20±0.00                         & 75.91±0.40                        & 62.17±0.29                        & 64.43±0.95                        & 80.26±0.23                        & 74.77±0.26                        & 66.07±0.00                        & 41.91±0.00                        & 0             \\ % \cline{2-19} 
                    & \method{}                                 & 90.65±0.10                        & \textbf{72.11±0.56}               & \textbf{88.41±0.29}               & \textbf{88.15±0.39}               & \textbf{86.36±0.00}                & \textbf{81.54±0.42}               & \textbf{51.06±0.00}                & \textbf{79.52±0.30}                & \textbf{93.68±0.31}               & \textbf{78.01±0.40}                & \textbf{73.01±0.50}                & \textbf{78.52±1.10}                & \textbf{83.66±0.23}               & \textbf{80.00±0.44}                & \textbf{69.64±0.00}                & \textbf{58.33±0.34}               & \textbf{15}   \\ \hline
\multirow{5}{*}{4}  & LoRA                                   & 90.79±0.10                        & 68.71±0.00                        & 82.61±0.00                        & 84.36±0.00                        & 78.41±0.00                        & 76.41±0.00                        & 47.87±0.00                        & 75.76±0.30                        & 90.20±0.00                         & 76.05±0.00                        & 63.60±0.29                         & 65.55±0.32                        & 80.58±0.00                        & 74.77±0.26                        & 65.48±0.42                        & 41.91±0.00                        & 0             \\ % \cline{2-19} 
                    & AdaLoRA                              & 90.15±0.10                        & 67.35±0.00                        & 81.58±0.29                        & 84.20±0.22                         & 79.55±0.00                        & 76.92±0.00                        & 46.46±0.25                        & 74.71±0.15                        & 90.20±0.00                         & 75.77±0.20                        & 61.76±0.29                        & 63.76±0.00                        & 79.61±0.00                        & 74.59±0.00                        & 65.18±0.00                        & 42.16±0.35                        & 0             \\ % \cline{2-19} 
                    & BayesTune-LoRA                            & 90.23±0.10                        & 67.12±0.32                        & 80.75±0.00                        & 84.36±0.00                        & 79.55±0.00                        & 76.92±0.00                        & 46.28±0.00                        & 74.71±0.15                        & 90.20±0.00                         & 75.21±0.00                        & 61.35±0.00                        & 63.76±0.00                        & 79.61±0.00                        & 74.59±0.00                        & 65.18±0.00                        & 42.40±0.35                         & 0             \\ % \cline{2-19} 
                    & FedL2P                               & 90.82±0.11                        & 68.37±0.34                        & 83.54±0.93                        & 85.54±1.19                        & 80.12±0.57                        & 76.41±0.51                        & 48.67±0.80                        & 77.43±0.63                        & 90.20±0.00                         & 76.68±0.21                        & 63.80±1.23                         & 67.78±2.02                        & 81.31±0.73                        & 75.95±0.81                        & 66.51±0.45                        & 43.75±1.84                        & 0             \\ % \cline{2-19} 
                    & \method{}                                 & \textbf{90.86±0.36}               & \textbf{72.56±0.32}               & \textbf{86.96±0.00}                & \textbf{91.00±0.39}                & \textbf{87.88±0.54}               & \textbf{81.37±0.48}               & \textbf{52.13±0.00}                & \textbf{81.61±0.15}               & \textbf{94.77±0.00}                & \textbf{79.13±0.20}                & \textbf{74.03±0.29}               & \textbf{80.32±0.32}               & \textbf{83.50±0.00}                 & \textbf{84.86±0.00}                & \textbf{77.08±0.84}               & \textbf{68.87±0.35}               & \textbf{16}   \\ \hline
\multirow{5}{*}{8}  & LoRA                                   & 90.65±0.10                        & 68.71±0.00                        & 83.02±0.29                        & 84.83±0.00                        & 80.68±0.00                        & 77.95±0.00                        & 50.00±0.00                         & 77.43±0.00                        & 90.20±0.00                         & 76.89±0.00                        & 64.42±0.00                        & 66.89±0.32                        & 82.04±0.00                        & 76.58±0.25                        & 66.07±0.00                        & 43.38±0.00                        & 0             \\ % \cline{2-19} 
                    & AdaLoRA                              & 90.23±0.10                        & 67.35±0.00                        & 80.96±0.29                        & 84.36±0.00                        & 79.55±0.00                        & 76.92±0.00                        & 46.28±0.00                        & 74.71±0.15                        & 90.20±0.00                         & 75.63±0.00                        & 61.76±0.29                        & 63.54±0.32                        & 79.45±0.23                        & 74.59±0.00                        & 65.18±0.00                        & 42.16±0.35                        & 0             \\ % \cline{2-19} 
                    & BayesTune-LoRA                            & 90.23±0.27                        & 67.35±0.00                        & 81.37±0.51                        & 84.20±0.22                         & 79.55±0.00                        & 76.92±0.00                        & 46.28±0.00                        & 74.61±0.00                        & 90.20±0.00                         & 75.35±0.20                        & 61.55±0.29                        & 63.98±0.32                        & 79.29±0.23                        & 74.59±0.00                        & 65.18±0.00                        & 42.40±0.35                         & 0             \\ % \cline{2-19} 
                    & FedL2P                               & \textbf{90.79±0.10}                & 68.94±0.32                        & 83.85±0.51                        & 85.62±0.81                        & 80.31±1.07                        & 77.43±0.73                        & 49.29±1.25                        & 77.95±0.97                        & 90.42±0.31                        & 76.89±0.00                        & 65.03±1.80                        & 68.46±1.45                        & 81.72±0.46                        & 76.94±1.02                        & 66.67±0.84                        & 45.10±1.93                         & 1             \\ % \cline{2-19} 
                    & \method{}                                 & 90.37±0.36                        & \textbf{73.70±0.32}                & \textbf{89.23±0.29}               & \textbf{93.36±0.39}               & \textbf{89.39±0.53}               & \textbf{79.66±0.64}               & \textbf{51.42±0.67}               & \textbf{81.09±0.29}               & \textbf{96.08±0.00}                & \textbf{78.99±0.34}               & \textbf{77.10±0.29}                & \textbf{78.52±0.55}               & \textbf{82.85±0.23}               & \textbf{86.49±0.00}                & \textbf{77.38±0.42}               & \textbf{67.65±0.00}                & \textbf{15}   \\ \hline
\multirow{5}{*}{16} & LoRA                                   & 90.93±0.00                        & 70.75±0.00                        & 85.09±0.00                        & 87.36±0.23                        & 82.95±0.00                        & \textbf{80.51±0.00}                         & 50.35±0.25                        & 79.83±0.15                        & 90.85±0.00                        & 76.89±0.00                        & 71.78±0.00                        & 71.81±0.95                        & \textbf{83.17±0.23}               & 80.18±0.67                        & 67.86±0.00                        & 51.96±0.69                        & 1             \\ % \cline{2-19} 
                    & AdaLoRA                              & 90.15±0.10                        & 67.58±0.32                        & 80.75±0.00                        & 84.36±0.00                        & 79.55±0.00                        & 76.92±0.00                        & 46.10±0.25                         & 74.71±0.15                        & 90.20±0.00                         & 75.63±0.34                        & 61.55±0.29                        & 63.98±0.32                        & 79.61±0.00                        & 74.59±0.00                        & 65.18±0.00                        & 42.40±0.35                         & 0             \\ % \cline{2-19} 
                    & BayesTune-LoRA                            & 90.58±0.10                        & 67.35±0.00                        & 81.16±0.29                        & 84.36±0.00                        & 79.55±0.00                        & 76.92±0.00                        & 46.10±0.25                         & 74.61±0.00                        & 90.20±0.00                         & 75.63±0.34                        & 61.35±0.00                        & 63.98±0.32                        & 79.45±0.23                        & 74.59±0.00                        & 65.18±0.00                        & 42.16±0.35                        & 0             \\ % \cline{2-19} 
                    & FedL2P                               & \textbf{91.14±0.30}                & 71.20±0.32                         & 87.17±0.77                        & 87.20±0.67                         & 83.71±0.54                        & 79.32±0.25                        & \textbf{51.42±0.67}               & 79.94±0.26                        & 91.72±0.31                        & 77.73±0.91                        & 70.55±2.18                        & 75.17±3.34                        & 82.36±0.23                        & 79.46±0.44                        & 71.43±1.26                        & 55.64±4.81                        & 2             \\ % \cline{2-19} 
                    & \method{}                                 & 88.33±0.70                        & \textbf{74.83±0.56}               & \textbf{88.41±0.29}               & \textbf{93.68±0.23}               & \textbf{89.39±0.53}               & \textbf{80.51±0.00}                         & 51.06±0.44                        & \textbf{80.15±0.29}               & \textbf{96.51±0.31}               & \textbf{80.67±0.00}                & \textbf{78.94±1.04}               & \textbf{77.63±0.32}               & 82.20±0.23                         & \textbf{85.95±0.44}               & \textbf{77.38±0.84}               & \textbf{68.87±0.69}               & \textbf{12}   \\ \bottomrule
\end{tabular}
}
\vspace{-1.2em}
\end{table*}

\subsubsection{Baselines}\label{sec:baselines}

Given a \basemodel{}, we compare \method{} with existing fine-tuning and {\em learning to personalize} approaches. 

\noindent\textbf{LoRA PEFT.}~We deploy LoRA~\cite{hu2021lora} on all linear layers of the model with a fixed rank $r$. 

\noindent\textbf{Non-FL Rank Selection.}~We compare with AdaLoRA~\cite{adalora} and our proposed LoRA-variant of BayesTune~\cite{kim2023bayestune}, BayesTune-LoRA (Section~\ref{sec:personalized_peft}), which optimizes $\bm{\lambda}$ separately for each client. 

\noindent\textbf{FL to Personalize.}~We compare with FedL2P~\cite{royson2023fedl2p} which trains a MLP federatedly to output per-client learning rates for each LoRA module.

For each baseline, we either follow best practices recommended by the corresponding authors or employ a simple grid search and pick the best performing hyperparameters. Full details in Appendix.~\ref{appendix:experiments}.

\subsection{Results on Text Classification}\label{sec:text_class}

We evaluate our approach in a typical FL setup, where the pretrained model is first trained using Standard FL with full fine-tuning and the resulting \basemodel{} is then personalized to each client. Tables~\ref{tab:masakha_seen} \& \ref{tab:masakha_unseen} show the mean and standard deviation (SD) of the accuracy for each language in our MasakhaNEWS setup for \seen{} and \unseen{} pool respectively (similarly for XNLI in Appendix Tables~\ref{tab:xnli_seen} \& \ref{tab:xnli_unseen}). In addition, we also show the number of languages, labelled ``Wins", an approach is best performing for each budget $r$. 

The results in all four tables show that federated {\em learning to personalize} methods (FedL2P and \method{}) outperform the other baselines in most cases. Non-FL rank selection approaches (AdaLoRA and BayesTune-LoRA), on the other hand, tend to overfit and/or struggle to learn an optimal rank structure given the limited number of samples in each client. Comparing FedL2P and \method{}, \method{} largely surpass FedL2P with a few exceptions, indicating that learning to personalize LoRA rank structure is the better hyperparameter choice than personalizing learning rates; this finding is also aligned with recent LLM-based optimizer findings~\cite{zhao2025deconstructing}, which shows that Adam's performance is robust with respect to its learning rate.

\subsubsection{\method{}'s Complementability with Personalized FL Works.}\label{sec:eval_complement}

% \begin{table*}[]
% \centering
% \caption{Mean Accuracy ± standard deviation on each language and overall win rate on XNLI dataset, with FedPDA, seen pool}
% \label{tab:xnli_seen_fedpda}
% \begin{scriptsize}\resizebox{0.98\textwidth}{!}{
% \begin{tabular}{c|l|l|l|l|l|l|l|l|l|l|l|l|l|l|l|l|c}
% \toprule

\begin{table*}[]
\centering
\caption{Mean±SD Accuracy of each language across 3 different seeds for clients in the \seen{} pool of our XNLI setup. The pretrained model is trained using FedDPA-T and the resulting \basemodel{} is personalized to each client given a baseline approach.}
\label{tab:xnli_seen_feddpa}
\begin{scriptsize}\resizebox{0.98\textwidth}{!}{

\begin{tabular}{c|l|l|l|l|l|l|l|l|l|l|l|l|l|l|l|l|c}
\toprule
% \textbf{Lora Rank}  
\textbf{$\mathbf{r}$} & \multicolumn{1}{c|}{\textbf{Approach}} & \multicolumn{1}{c|}{\textbf{bg}} & \multicolumn{1}{c|}{\textbf{hi}} & \multicolumn{1}{c|}{\textbf{es}} & \multicolumn{1}{c|}{\textbf{el}} & \multicolumn{1}{c|}{\textbf{vi}} & \multicolumn{1}{c|}{\textbf{tr}} & \multicolumn{1}{c|}{\textbf{de}} & \multicolumn{1}{c|}{\textbf{ur}} & \multicolumn{1}{c|}{\textbf{en}} & \multicolumn{1}{c|}{\textbf{zh}} & \multicolumn{1}{c|}{\textbf{th}} & \multicolumn{1}{c|}{\textbf{sw}} & \multicolumn{1}{c|}{\textbf{ar}} & \multicolumn{1}{c|}{\textbf{fr}} & \multicolumn{1}{c|}{\textbf{ru}} & \textbf{Wins} \\ \midrule
% \multirow{5}{*}{1}  & LoRA                                   & 44.60±0.00                        & 41.73±0.09                       & 47.73±0.19                       & 50.20±0.00                        & 52.53±0.19                       & 47.00±0.16                        & 48.07±0.09                       & 40.60±0.00                        & 45.40±0.00                        & 43.33±0.09                       & 41.07±0.09                       & 50.80±0.00                        & 45.27±0.09                       & 46.33±0.19                       & 49.20±0.00                        & 0             \\ %\cline{2-18} 
%                     & AdaLoRA                              & 44.07±0.09                       & 41.20±0.00                        & 47.47±0.09                       & 50.20±0.00                        & 52.40±0.00                        & 46.53±0.09                       & 47.93±0.09                       & 38.87±0.09                       & 44.60±0.00                        & 42.27±0.09                       & 40.67±0.09                       & 50.40±0.00                        & 45.00±0.00                        & 46.00±0.00                        & 48.80±0.00                        & 0             \\ %\cline{2-18} 
%                     & BayesTune-LoRA                            & 43.67±0.09                       & 40.40±0.00                        & 47.20±0.00                        & 50.00±0.00                        & 52.33±0.09                       & 46.27±0.09                       & 47.40±0.16                        & 38.80±0.00                        & 43.93±0.09                       & 41.33±0.09                       & 40.53±0.09                       & 49.80±0.00                        & 44.40±0.00                        & 46.00±0.00                        & 47.93±0.09                       & 0             \\ %\cline{2-18} 
%                     & FedL2P                               & 44.33±0.19                       & 41.33±0.19                       & 47.67±0.09                       & 50.20±0.00                        & 52.47±0.09                       & 46.93±0.19                       & 47.93±0.19                       & 40.20±0.28                        & 45.20±0.00                        & 42.93±0.19                       & 40.93±0.09                       & 50.67±0.09                       & 45.27±0.09                       & 46.33±0.19                       & 48.80±0.00                        & 0             \\ %\cline{2-18} 
%                     & \method{}                                 & \textbf{58.00±0.16}               & \textbf{54.93±0.09}              & \textbf{55.33±0.09}              & \textbf{55.33±0.19}              & \textbf{54.80±0.33}               & \textbf{54.20±0.16}               & \textbf{55.80±0.16}               & \textbf{60.33±0.34}              & \textbf{55.53±0.25}              & \textbf{55.67±0.09}              & \textbf{58.07±0.62}              & \textbf{54.33±0.19}              & \textbf{54.93±0.09}              & \textbf{57.07±0.38}              & \textbf{54.53±0.34}              & \textbf{15}   \\ \hline
\multirow{5}{*}{2}  & LoRA                                   & 45.80±0.28                        & 42.80±0.16                        & 48.73±0.25                       & 50.87±0.09                       & 53.00±0.00                        & 48.00±0.33                        & 49.87±0.09                       & 41.93±0.09                       & 46.53±0.34                       & 44.40±0.16                        & 42.53±0.25                       & 51.80±0.16                        & 46.93±0.25                       & 48.07±0.19                       & 50.53±0.25                       & 0             \\ %\cline{2-18} 
                    & AdaLoRA                              & 44.07±0.09                       & 41.20±0.00                        & 47.47±0.09                       & 50.00±0.00                        & 52.40±0.00                        & 46.53±0.09                       & 48.00±0.00                        & 38.80±0.00                        & 44.67±0.09                       & 42.20±0.00                        & 40.67±0.09                       & 50.40±0.00                        & 45.00±0.00                        & 46.00±0.00                        & 48.80±0.00                        & 0             \\ %\cline{2-18} 
                    & BayesTune-LoRA                            & 43.80±0.00                        & 40.60±0.00                        & 47.20±0.00                        & 50.00±0.00                        & 52.40±0.00                        & 46.20±0.00                        & 47.40±0.16                        & 38.80±0.00                        & 44.27±0.09                       & 41.60±0.00                        & 40.53±0.09                       & 49.80±0.00                        & 44.73±0.09                       & 46.00±0.00                        & 48.07±0.19                       & 0             \\ %\cline{2-18} 
                    & FedL2P                               & 47.47±2.78                       & 44.53±3.18                       & 50.27±2.94                       & 51.47±1.39                       & 53.60±0.71                        & 50.07±2.81                       & 50.93±2.62                       & 44.80±4.53                        & 49.07±4.35                       & 46.53±4.01                       & 44.40±3.68                        & 52.27±1.27                       & 48.40±2.97                        & 49.47±2.96                       & 51.60±2.27                        & 0             \\ %\cline{2-18} 
                    & \method{}                                 & \textbf{64.40±0.16}               & \textbf{57.80±1.34}               & \textbf{58.53±1.32}              & \textbf{59.73±2.58}              & \textbf{60.80±0.71}               & \textbf{58.87±2.87}              & \textbf{55.00±0.16}               & \textbf{63.67±0.19}              & \textbf{55.93±0.19}              & \textbf{56.27±0.34}              & \textbf{58.33±0.25}              & \textbf{59.47±0.47}              & \textbf{55.20±0.16}               & \textbf{57.53±0.19}              & \textbf{55.20±0.43}               & \textbf{15}   \\ \hline
\multirow{5}{*}{4}  & LoRA                                   & 48.73±0.38                       & 46.07±0.66                       & 53.27±0.09                       & 52.53±0.09                       & 53.33±0.09                       & 50.27±0.25                       & 52.80±0.28                        & 46.13±0.25                       & 51.67±0.25                       & 48.53±0.19                       & 45.47±0.19                       & 53.33±0.09                       & 50.47±0.41                       & 50.27±0.09                       & 52.87±0.19                       & 0             \\ %\cline{2-18} 
                    & AdaLoRA                              & 44.00±0.00                        & 41.13±0.09                       & 47.40±0.00                        & 50.00±0.00                        & 52.40±0.00                        & 46.40±0.00                        & 47.67±0.09                       & 38.80±0.00                        & 44.47±0.09                       & 41.93±0.09                       & 40.60±0.00                        & 50.27±0.09                       & 44.87±0.09                       & 46.00±0.00                        & 48.80±0.00                        & 0             \\ %\cline{2-18} 
                    & BayesTune-LoRA                            & 44.00±0.00                        & 41.00±0.00                        & 47.20±0.00                        & 50.00±0.00                        & 52.40±0.00                        & 46.33±0.09                       & 47.67±0.09                       & 38.80±0.00                        & 44.47±0.09                       & 41.80±0.00                        & 40.60±0.00                        & 50.00±0.00                        & 44.80±0.00                        & 46.00±0.00                        & 48.67±0.09                       & 0             \\ %\cline{2-18} 
                    & FedL2P                               & 48.67±2.36                       & 45.20±2.26                        & 52.13±1.20                        & 51.87±0.52                       & 53.87±0.34                       & 50.27±0.84                       & 52.00±0.99                        & 47.00±3.69                        & 50.73±2.59                       & 47.27±1.95                       & 45.07±1.65                       & 52.67±0.52                       & 49.67±1.59                       & 51.20±2.26                        & 52.47±1.37                       & 0             \\ %\cline{2-18} 
                    & \method{}                                 & \textbf{64.47±1.18}              & \textbf{64.00±0.98}               & \textbf{64.40±0.91}               & \textbf{63.07±0.41}              & \textbf{64.13±0.34}              & \textbf{64.87±0.68}              & \textbf{63.33±1.67}              & \textbf{64.47±0.62}              & \textbf{56.60±0.28}               & \textbf{63.93±1.16}              & \textbf{62.60±1.73}               & \textbf{65.00±0.33}               & \textbf{64.13±0.81}              & \textbf{61.73±1.23}              & \textbf{61.80±0.65}               & \textbf{15}   \\ \hline
\multirow{5}{*}{8}  & LoRA                                   & 55.80±0.16                        & 51.73±0.25                       & 55.73±0.09                       & 55.07±0.25                       & 54.40±0.00                        & 52.80±0.16                        & 54.60±0.00                        & 57.47±0.09                       & 55.53±0.09                       & 54.00±0.33                        & 51.93±0.09                       & 54.13±0.09                       & 53.07±0.09                       & 56.27±0.09                       & 54.60±0.16                        & 0             \\ %\cline{2-18} 
                    & AdaLoRA                              & 43.80±0.00                        & 40.87±0.09                       & 47.27±0.09                       & 50.00±0.00                        & 52.40±0.00                        & 46.27±0.09                       & 47.73±0.09                       & 38.80±0.00                        & 44.40±0.00                        & 41.87±0.09                       & 40.60±0.00                        & 50.00±0.00                        & 44.80±0.00                        & 46.00±0.00                        & 48.40±0.16                        & 0             \\ %\cline{2-18} 
                    & BayesTune-LoRA                            & 44.13±0.09                       & 41.20±0.00                        & 47.47±0.09                       & 50.20±0.00                        & 52.40±0.00                        & 46.47±0.09                       & 47.93±0.09                       & 39.13±0.09                       & 44.93±0.09                       & 42.53±0.09                       & 40.87±0.09                       & 50.47±0.09                       & 45.13±0.09                       & 46.13±0.09                       & 48.80±0.00                        & 0             \\ %\cline{2-18} 
                    & FedL2P                               & 52.20±0.59                        & 48.73±0.25                       & 54.20±0.16                        & 53.13±0.19                       & 54.33±0.09                       & 51.53±0.19                       & 53.60±0.16                        & 51.00±0.86                        & 54.27±0.25                       & 51.73±0.25                       & 47.07±0.52                       & 53.60±0.16                        & 51.87±0.19                       & 52.93±0.34                       & 54.60±0.16                        & 0             \\ %\cline{2-18} 
                    & \method{}                                 & \textbf{67.27±0.25}              & \textbf{66.60±0.33}               & \textbf{68.87±0.41}              & \textbf{65.00±0.43}               & \textbf{66.73±0.09}              & \textbf{67.80±0.28}               & \textbf{67.13±0.19}              & \textbf{67.60±0.16}               & \textbf{64.00±0.99}               & \textbf{67.67±0.38}              & \textbf{66.53±0.09}              & \textbf{67.33±0.25}              & \textbf{67.60±0.33}               & \textbf{66.53±0.25}              & \textbf{69.33±0.19}              & \textbf{15}   \\ \hline
\multirow{5}{*}{16} & LoRA                                   & 64.73±0.09                       & 64.27±0.19                       & 65.60±0.28                        & \textbf{65.33±0.19}              & 63.80±0.00                        & 64.27±0.09                       & 64.20±0.71                        & 64.80±0.16                        & 57.93±2.46                       & 65.93±0.25                       & 58.33±0.52                       & 65.13±0.19                       & 60.53±0.62                       & 64.53±0.47                       & 61.73±0.74                       & 1             \\ %\cline{2-18} 
                    & AdaLoRA                              & 43.87±0.09                       & 40.73±0.09                       & 47.20±0.00                        & 50.00±0.00                        & 52.40±0.00                        & 46.33±0.09                       & 47.67±0.09                       & 38.80±0.00                        & 44.27±0.09                       & 41.80±0.00                        & 40.60±0.00                        & 50.00±0.00                        & 44.80±0.00                        & 46.00±0.00                        & 48.27±0.09                       & 0             \\ %\cline{2-18} 
                    & BayesTune-LoRA                            & 44.60±0.00                        & 41.67±0.19                       & 47.80±0.00                        & 50.20±0.00                        & 52.87±0.09                       & 47.13±0.09                       & 48.33±0.19                       & 40.60±0.00                        & 45.40±0.00                        & 43.60±0.00                        & 41.00±0.00                        & 50.80±0.00                        & 45.33±0.09                       & 46.60±0.00                        & 49.40±0.16                        & 0             \\ %\cline{2-18} 
                    & FedL2P                               & 54.47±1.86                       & 52.73±0.50                        & 55.80±0.43                        & 54.73±0.47                       & 55.67±1.09                       & 54.80±0.00                        & 54.73±0.25                       & 57.67±2.78                       & 55.67±0.09                       & 54.93±0.41                       & 52.60±2.79                        & 53.93±0.41                       & 53.47±0.25                       & 55.60±2.12                        & 55.20±0.00                        & 0             \\ %\cline{2-18} 
                    & \method{}                                 & \textbf{67.87±0.74}              & \textbf{67.00±0.43}               & \textbf{69.53±0.57}              & 64.93±0.41                       & \textbf{67.00±0.33}               & \textbf{68.00±0.00}                & \textbf{67.20±0.43}               & \textbf{68.00±0.59}               & \textbf{65.20±1.07}               & \textbf{67.60±0.16}               & \textbf{66.60±0.57}               & \textbf{67.33±0.09}              & \textbf{67.80±0.28}               & \textbf{66.13±0.41}              & \textbf{69.87±0.25}              & \textbf{14}   \\ \bottomrule
\end{tabular}
}
\end{scriptsize}
\vspace{-1.5em}
\end{table*}
\begin{table*}[]
\caption{Mean±SD Accuracy of each language across 3 different seeds for clients in the \seen{} pool of our XNLI setup. The pretrained model is trained using DEPT(SPEC) and the resulting \basemodel{} is personalized to each client given a baseline approach.}
\label{tab:xnli_seen_dept}
\begin{scriptsize}\resizebox{0.98\textwidth}{!}{
\begin{tabular}{c|l|l|l|l|l|l|l|l|l|l|l|l|l|l|l|l|c}
\toprule
\textbf{$\mathbf{r}$} & \multicolumn{1}{c|}{\textbf{Approach}} & \multicolumn{1}{c|}{\textbf{bg}} & \multicolumn{1}{c|}{\textbf{hi}} & \multicolumn{1}{c|}{\textbf{es}} & \multicolumn{1}{c|}{\textbf{el}} & \multicolumn{1}{c|}{\textbf{vi}} & \multicolumn{1}{c|}{\textbf{tr}} & \multicolumn{1}{c|}{\textbf{de}} & \multicolumn{1}{c|}{\textbf{ur}} & \multicolumn{1}{c|}{\textbf{en}} & \multicolumn{1}{c|}{\textbf{zh}} & \multicolumn{1}{c|}{\textbf{th}} & \multicolumn{1}{c|}{\textbf{sw}} & \multicolumn{1}{c|}{\textbf{ar}} & \multicolumn{1}{c|}{\textbf{fr}} & \multicolumn{1}{c|}{\textbf{ru}} & \textbf{Wins} \\ \midrule
% \multirow{5}{*}{1}  & LoRA                                   & 53.60±0.16                        & 54.87±0.09                       & 55.80±0.16                        & 53.60±0.16                        & 54.87±0.09                       & 53.20±0.00                        & 54.20±0.00                        & 53.33±0.34                       & 60.47±0.09                       & 53.53±0.09                       & 49.40±0.16                        & 51.60±0.16                        & 51.87±0.09                       & 53.13±0.09                       & 54.80±0.00                        & 0             \\ %\cline{2-18} 
%                     & AdaLoRA                              & 53.33±0.09                       & 54.73±0.09                       & 55.40±0.00                        & 53.07±0.09                       & 54.33±0.19                       & 52.87±0.09                       & 54.07±0.09                       & 53.20±0.00                        & 60.27±0.09                       & 53.07±0.09                       & 49.47±0.09                       & 50.87±0.09                       & 51.40±0.00                        & 52.87±0.19                       & 54.13±0.19                       & 0             \\ %\cline{2-18} 
%                     & BayesTune-LoRA                            & 53.27±0.09                       & 54.47±0.09                       & 55.40±0.16                        & 53.13±0.09                       & 54.00±0.00                        & 52.67±0.09                       & 53.93±0.09                       & 53.07±0.19                       & 60.00±0.16                        & 53.07±0.09                       & 49.40±0.00                        & 50.40±0.00                        & 51.40±0.16                        & 52.60±0.00                        & 54.20±0.16                        & 0             \\ %\cline{2-18} 
%                     & FedL2P                               & 60.80±0.60                       & 61.60±4.20                       & 65.70±0.90                        & 65.10±1.70                        & 65.80±0.00                        & 61.90±3.90                        & 64.40±0.40                        & 62.60±1.00                        & 70.20±0.00                        & 61.90±1.30                        & 59.50±1.70                        & 60.80±3.20                       & 63.00±1.60                       & 61.00±0.40                        & 64.00±0.40                        & 0             \\ %\cline{2-18} 
%                     & \method{}                                 & \textbf{68.67±0.9}               & \textbf{67.33±1.65}              & \textbf{71.73±1.31}              & \textbf{71.73±0.25}              & \textbf{70.73±0.57}              & \textbf{69.33±2.07}              & \textbf{70.80±0.16}               & \textbf{70.27±0.9}               & \textbf{73.67±0.41}              & \textbf{71.20±0.16}               & \textbf{63.40±2.2}                & \textbf{69.20±1.61}               & \textbf{70.93±0.77}              & \textbf{69.00±1.41}               & \textbf{70.67±0.82}              & \textbf{15}   \\ \hline
\multirow{5}{*}{2}  & LoRA                                   & 54.10±0.10                        & 55.80±0.00                        & 57.10±0.10                        & 55.10±0.30                        & 56.30±0.10                        & 54.30±0.10                        & 55.10±0.10                        & 53.60±0.20                       & 61.50±0.10                        & 54.80±0.00                        & 50.70±0.10                        & 52.50±0.10                        & 53.40±0.20                       & 53.10±0.10                        & 55.30±0.10                        & 0             \\ %\cline{2-18} 
                    & AdaLoRA                              & 53.33±0.09                       & 54.53±0.09                       & 55.33±0.09                       & 52.80±0.28                        & 54.07±0.09                       & 52.87±0.09                       & 53.93±0.09                       & 52.87±0.19                       & 60.40±0.16                        & 53.07±0.09                       & 49.40±0.16                        & 50.67±0.09                       & 51.40±0.16                        & 52.80±0.16                        & 54.20±0.16                        & 0             \\ %\cline{2-18} 
                    & BayesTune-LoRA                            & 53.40±0.00                        & 54.40±0.00                        & 55.40±0.00                        & 53.10±0.10                        & 54.00±0.20                       & 52.90±0.10                        & 54.10±0.10                        & 53.20±0.00                        & 60.00±0.00                        & 53.20±0.00                        & 49.30±0.10                        & 50.50±0.10                        & 51.50±0.10                        & 52.60±0.20                       & 54.00±0.00                        & 0             \\ %\cline{2-18} 
                    & FedL2P                               & 64.70±1.10                        & 64.20±2.80                        & 67.90±1.30                        & 68.50±1.10                        & 68.60±0.20                       & 65.20±4.00                        & 67.30±0.70                        & 67.00±0.40                        & 71.50±0.30                        & 64.90±0.50                        & 61.70±1.70                        & 64.60±2.80                        & 65.80±2.00                        & 63.40±0.00                        & 67.40±0.40                        & 0             \\ %\cline{2-18} 
                    & \method{}                                 & \textbf{70.67±0.77}              & \textbf{70.07±1.04}              & \textbf{73.87±1.54}              & \textbf{73.47±0.19}              & \textbf{73.40±0.59}               & \textbf{71.93±2.08}              & \textbf{73.27±0.57}              & \textbf{71.87±0.62}              & \textbf{74.53±0.52}              & \textbf{73.80±0.43}               & \textbf{65.60±2.97}               & \textbf{69.93±1.16}              & \textbf{74.40±0.49}               & \textbf{71.27±1.52}              & \textbf{72.73±1.88}              & \textbf{15}   \\ \hline
\multirow{5}{*}{4}  & LoRA                                   & 56.67±0.34                       & 59.07±0.25                       & 59.13±0.41                       & 57.33±0.25                       & 59.00±0.28                        & 56.93±0.25                       & 57.40±0.59                        & 55.93±0.09                       & 63.47±0.25                       & 57.27±0.09                       & 52.47±0.34                       & 54.53±0.25                       & 56.40±0.33                        & 55.93±0.09                       & 58.40±0.33                        & 0             \\ %\cline{2-18} 
                    & AdaLoRA                              & 53.33±0.09                       & 54.40±0.16                        & 55.27±0.09                       & 53.20±0.00                        & 54.13±0.19                       & 52.93±0.09                       & 54.00±0.00                        & 52.87±0.09                       & 60.33±0.25                       & 53.13±0.09                       & 49.20±0.16                        & 50.60±0.16                        & 51.27±0.09                       & 52.60±0.16                        & 54.00±0.00                        & 0             \\ %\cline{2-18} 
                    & BayesTune-LoRA                            & 53.27±0.09                       & 54.53±0.19                       & 55.40±0.16                        & 53.00±0.16                        & 54.53±0.25                       & 52.87±0.09                       & 54.00±0.00                        & 53.07±0.09                       & 60.60±0.16                        & 53.07±0.09                       & 49.33±0.19                       & 51.07±0.09                       & 51.47±0.09                       & 52.60±0.00                        & 54.53±0.19                       & 0             \\ %\cline{2-18} 
                    & FedL2P                               & 66.50±1.10                        & 65.40±1.80                        & 69.90±1.90                        & 70.50±0.90                        & 70.10±0.30                        & 66.90±4.10                        & 69.70±0.90                        & 68.20±0.60                       & 72.80±0.20                       & 67.40±1.20                       & 62.60±2.00                        & 65.70±2.90                        & 67.70±1.30                        & 66.50±1.10                        & 69.00±0.40                        & 0             \\ %\cline{2-18} 
                    & \method{}                                 & \textbf{71.33±0.34}              & \textbf{70.07±1.09}              & \textbf{73.27±2.29}              & \textbf{73.27±0.68}              & \textbf{72.60±0.28}               & \textbf{71.87±2.32}              & \textbf{74.60±0.16}               & \textbf{72.93±0.38}              & \textbf{75.00±0.28}               & \textbf{74.33±1.16}              & \textbf{66.13±3.21}              & \textbf{68.13±0.34}              & \textbf{75.13±1.65}              & \textbf{72.20±0.33}               & \textbf{73.47±1.60}               & \textbf{15}   \\ \hline
\multirow{5}{*}{8}  & LoRA                                   & 60.60±0.28                        & 62.47±0.09                       & 63.73±0.09                       & 62.60±0.33                        & 64.67±0.09                       & 61.93±0.41                       & 62.53±0.09                       & 60.33±0.19                       & 67.33±0.25                       & 61.40±0.16                        & 56.53±0.09                       & 59.27±0.09                       & 61.33±0.25                       & 58.87±0.09                       & 63.13±0.19                       & 0             \\ %\cline{2-18} 
                    & AdaLoRA                              & 53.33±0.09                       & 54.53±0.19                       & 55.27±0.09                       & 52.80±0.28                        & 54.27±0.25                       & 53.00±0.00                        & 53.93±0.09                       & 52.93±0.09                       & 60.13±0.34                       & 53.00±0.00                        & 49.20±0.16                        & 50.27±0.19                       & 51.20±0.16                        & 52.73±0.38                       & 53.93±0.09                       & 0             \\ %\cline{2-18} 
                    & BayesTune-LoRA                            & 53.33±0.09                       & 54.40±0.00                        & 55.53±0.19                       & 53.20±0.00                        & 54.67±0.09                       & 53.00±0.00                        & 54.07±0.09                       & 53.13±0.25                       & 60.60±0.00                        & 53.47±0.09                       & 49.47±0.09                       & 51.33±0.34                       & 51.73±0.19                       & 52.87±0.09                       & 54.40±0.16                        & 0             \\ %\cline{2-18} 
                    & FedL2P                               & 66.40±0.60                       & 65.50±1.10                        & 70.50±3.50                        & 70.10±0.50                        & 70.50±1.70                        & 68.20±2.20                       & 69.60±0.40                        & 67.50±1.30                        & 72.60±0.60                       & 67.10±1.30                        & 61.80±1.00                        & 65.60±1.80                        & 68.20±1.00                        & 67.20±1.40                        & 69.10±1.70                        & 0             \\ %\cline{2-18} 
                    & \method{}                                 & \textbf{70.53±0.25}              & \textbf{69.27±0.77}              & \textbf{73.33±1.79}              & \textbf{71.27±1.39}              & \textbf{71.33±0.52}              & \textbf{70.53±2.10}               & \textbf{75.33±0.41}              & \textbf{73.07±0.77}              & \textbf{74.40±0.75}               & \textbf{73.93±1.11}              & \textbf{66.73±3.80}               & \textbf{68.07±0.94}              & \textbf{74.67±1.18}              & \textbf{71.87±0.19}              & \textbf{72.93±0.96}              & \textbf{15}   \\ \hline
\multirow{5}{*}{16} & LoRA                                   & 67.13±0.34                       & 67.80±0.00                        & 69.47±0.09                       & 71.27±0.19                       & 69.20±0.00                        & 68.07±0.38                       & 69.00±0.33                        & 68.73±0.25                       & 71.47±0.25                       & 68.00±0.16                        & 62.80±0.00                        & 67.33±0.25                       & 66.80±0.16                        & 65.07±0.19                       & 67.33±0.09                       & \textbf{0}    \\ %\cline{2-18} 
                    & AdaLoRA                              & 53.40±0.00                        & 54.40±0.00                        & 55.40±0.00                        & 52.60±0.16                        & 54.00±0.00                        & 52.93±0.09                       & 53.93±0.09                       & 52.93±0.09                       & 60.27±0.41                       & 53.00±0.00                        & 49.27±0.09                       & 50.27±0.09                       & 51.20±0.33                        & 52.80±0.28                        & 53.80±0.16                        & 0             \\ %\cline{2-18} 
                    & BayesTune-LoRA                            & 53.47±0.09                       & 54.67±0.09                       & 55.80±0.00                        & 53.40±0.00                        & 54.67±0.09                       & 53.07±0.09                       & 54.47±0.19                       & 53.27±0.09                       & 60.67±0.09                       & 53.40±0.16                        & 49.40±0.00                        & 51.67±0.09                       & 52.07±0.25                       & 53.07±0.09                       & 54.80±0.00                        & 0             \\ %\cline{2-18} 
                    & FedL2P                               & 68.00±1.02                        & 65.87±1.33                       & 70.20±3.13                        & \textbf{71.73±0.81}              & 71.20±1.40                        & 69.60±1.66                        & 71.20±1.02                        & 69.00±0.99                        & 73.80±0.82                        & 69.13±1.54                       & 64.00±1.14                        & 67.80±1.82                        & 69.27±1.23                       & 68.53±1.32                       & 69.93±1.95                       & 1             \\ %\cline{2-18} 
                    & \method{}                                 & \textbf{69.80±0.16}               & \textbf{69.27±1.04}              & \textbf{73.47±1.51}              & 70.40±1.84                        & \textbf{71.27±0.57}              & \textbf{70.67±2.88}              & \textbf{74.80±0.16}               & \textbf{73.20±1.14}               & \textbf{74.53±1.25}              & \textbf{73.53±0.75}              & \textbf{65.93±2.64}              & \textbf{68.60±1.70}                & \textbf{73.67±1.84}              & \textbf{70.67±0.25}              & \textbf{72.80±0.71}               & \textbf{14}   \\ \bottomrule
\end{tabular}
}
\end{scriptsize}
\vspace{-1.2em}
\end{table*}
\begin{table*}[]
\small
\centering
\setlength{\tabcolsep}{3.5pt}
\renewcommand{\arraystretch}{0.8}
\begin{tabular}{@{}cl|ccccc@{}}
\toprule
\textbf{\# Topics} & \textbf{Model} & \multicolumn{1}{l}{\textbf{\# Input Tokens}} & \multicolumn{1}{l}{\textbf{\# Output Tokens}} & \multicolumn{1}{l}{\textbf{\# LLM Calls}} & \multicolumn{1}{l}{\textbf{Cost (GPT-4)}} & \multicolumn{1}{l}{\textbf{Time (seconds)}} \\ \midrule
\multirow{3}{*}{2} & \modelTopic & 21383.08 & 3412.02 & 25.45 & 0.32 & 117.60 \\
 & Hierarchical & 31130.02 & 2536.66 & 13.15 & 0.39 & 83.13 \\
 & Incremental-\textit{Topic} & 59010.66 & 6115.04 & 15.15 & 0.77 & 214.39 \\ \midrule
\multirow{3}{*}{3} & \modelTopic & 30208.20 & 5040.38 & 37.38 & 0.45 & 149.54 \\
 & Hierarchical & 31144.83 & 2649.78 & 13.15 & 0.39 & 68.60 \\
 & Incremental-\textit{Topic} & 61344.07 & 8442.54 & 16.15 & 0.87 & 197.33 \\ \midrule
\multirow{3}{*}{4} & \modelTopic & 38286.40 & 6440.23 & 47.91 & 0.58 & 163.91 \\
 & Hierarchical & 31144.31 & 2740.31 & 13.15 & 0.39 & 88.75 \\
 & Incremental-\textit{Topic} & 62877.46 & 9966.45 & 17.15 & 0.93 & 312.55 \\ \midrule
\multirow{3}{*}{5} & \modelTopic & 47008.59 & 7918.92 & 58.94 & 0.71 & 186.32 \\
 & Hierarchical & 31160.88 & 2850.24 & 13.15 & 0.40 & 61.70 \\
 & Incremental-\textit{Topic} & 64893.95 & 11965.84 & 18.15 & 1.01 & 262.07 \\ \bottomrule
\end{tabular}
\caption{\label{appendix:table:cost_cqa} Number of LLM input/output tokens, LLM calls, GPT-4 Cost (USD), and Time (seconds) needed to run inference on a single DFQS example on ConflictingQA with the top-3 models. We report 5 runs and 20 examples.}
\end{table*}

\begin{table*}[]
\small
\centering
\setlength{\tabcolsep}{3.5pt}
\renewcommand{\arraystretch}{0.8}
\begin{tabular}{@{}cl|ccccc@{}}
\toprule
\multicolumn{1}{l}{\textbf{Dataset}} & \textbf{Model} & \multicolumn{1}{l}{\textbf{\# Input Tokens}} & \multicolumn{1}{l}{\textbf{\# Output Tokens}} & \multicolumn{1}{l}{\textbf{\# LLM Calls}} & \multicolumn{1}{l}{\textbf{Cost (GPT-4)}} & \multicolumn{1}{l}{\textbf{Time (seconds)}} \\ \midrule
\multirow{3}{*}{2} & \modelTopic & 17183.75 & 2722.40 & 20.30 & 0.25 & 94.81 \\
 & Hierarchical & 19181.59 & 2040.39 & 10.25 & 0.25 & 63.68 \\
 & Incremental-\textit{Topic} & 41656.87 & 5062.44 & 12.25 & 0.57 & 182.19 \\ 
 \midrule
\multirow{3}{*}{3} & \modelTopic & 24801.22 & 4136.12 & 30.40 & 0.37 & 126.83 \\
 & Hierarchical & 19182.58 & 2141.91 & 10.25 & 0.26 & 53.32 \\
 & Incremental-\textit{Topic} & 43119.51 & 6532.92 & 13.25 & 0.63 & 152.44 \\ \midrule
\multirow{3}{*}{4} & \modelTopic & 30677.67 & 5037.31 & 38.00 & 0.46 & 120.64 \\
 & Hierarchical & 19203.30 & 2253.17 & 10.25 & 0.26 & 73.35 \\
 & Incremental-\textit{Topic} & 43922.02 & 7327.88 & 14.25 & 0.66 & 241.54 \\ \midrule
\multirow{3}{*}{5} & \modelTopic & 36988.41 & 6049.93 & 46.09 & 0.55 & 139.71 \\
 & Hierarchical & 19211.74 & 2356.01 & 10.25 & 0.26 & 49.41 \\
 & Incremental-\textit{Topic} & 45113.12 & 8504.59 & 15.25 & 0.71 & 186.40 \\ \bottomrule
\end{tabular}
\caption{\label{appendix:table:cost_debate} Number of LLM input/output tokens, LLM calls, GPT-4 Cost (USD), and Time (seconds) needed to run inference on a single DFQS example on DebateQFS with the top-3 models. We report 5 runs and 20 examples.}
\end{table*}

\begin{table*}[]
\small
\centering
\setlength{\tabcolsep}{3.5pt}
\renewcommand{\arraystretch}{0.8}
\begin{tabular}{@{}cl|ccccc@{}}
\toprule
\multicolumn{1}{l}{\textbf{\# Topics}} & \textbf{Model} & \multicolumn{1}{l}{\textbf{\# Input Tokens}} & \multicolumn{1}{l}{\textbf{\# Output Tokens}} & \multicolumn{1}{l}{\textbf{\# LLM Calls}} & \multicolumn{1}{l}{\textbf{Cost (GPT-4)}} & \multicolumn{1}{l}{\textbf{Time (seconds)}} \\ 
\midrule
\multirow{3}{*}{ConflictingQA} & \modelTopic & 47008.59 & 7918.92 & 58.94 & 0.71 & 186.32 \\
 & \modelTopic Pick All & 53733.70 & 9596.75 & 71.75 & 0.83 & 303.13 \\
 & Hierarchical-\emph{Topic} & 168160.85 & 7485.50 & 66.75 & 1.91 & 210.80 \\ \midrule
\multirow{3}{*}{DebateQFS} & \modelTopic & 36988.41 & 6049.93 & 46.09 & 0.55 & 139.71 \\
& \modelTopic Pick All & 43098.85 & 7612.45 & 57.25 & 0.66 & 242.35 \\
& Hierarchical-\emph{Topic} & 105237.25 & 5278.35 & 52.25 & 1.21 & 139.96 \\ \bottomrule
\end{tabular}
\caption{\label{appendix:table:cost_weird} Number of LLM input/output tokens, LLM calls, GPT-4 Cost (USD), and Time (seconds) needed to run inference on a single DFQS example on ConflictingQA and DebateQFS with \modelTopic, the version of \modelTopic with no Moderator, and the version of Hierarchical merging that runs on each topic paragraph ($m=5$). We report 5 runs and 20 examples.}
\end{table*}



Apart from Standard FL, we show that \method{} can be plugged into existing personalized FL works that trains both a subset of the pretrained model and personalized layers for each client. Tables~\ref{tab:xnli_seen_feddpa} and \ref{tab:xnli_seen_dept} show that \method{} outperforms baselines in almost all cases in our XNLI setup given a \basemodel{} trained using FedDPA-T~\cite{FedDPA} and DEPT(SPEC)~\cite{DEPT} respectively. In short, \method{} can be integrated into a larger family of existing personalized FL approaches, listed in Section~\ref{sec:related}, to further improve personalization performance. 

\begin{table*}[t]
\caption{Avg. METEOR/ROUGE-1/ROUGE-L for \seen{} clients in our Fed-Aya setup. {\em Base model} is off-the-shelf Llama-3.2-3B-Instruct.}
\vspace{0.5em}
\label{tab:lama_fedaya_seen}
\begin{scriptsize}\resizebox{0.98\textwidth}{!}{
\begin{tabular}{c|l|l|l|l|l|l|l|l|c}
\toprule
% \textbf{Lora Rank}  
\textbf{$\mathbf{r}$} & \multicolumn{1}{c|}{\textbf{Approach}} & \multicolumn{1}{c|}{\textbf{te}} & \multicolumn{1}{c|}{\textbf{ar}} & \multicolumn{1}{c|}{\textbf{es}} & \multicolumn{1}{c|}{\textbf{en}} & \multicolumn{1}{c|}{\textbf{fr}} & \multicolumn{1}{c|}{\textbf{zh}} & \multicolumn{1}{c|}{\textbf{pt}} 
& \textbf{Wins} \\ \midrule
% \multirow{5}{*}{1}  & LoRA                                   & 0.2372/0.1409/0.1368             & 0.3291/0.0625/0.0617             & 0.3863/0.4134/0.385              & 0.3229/0.3613/0.2965             & 0.2811/0.3446/0.2802             & 0.1022/0.1184/0.117              & 0.3765/0.4415/0.4023                                         & 1             \\ % \cline{2-11} 
%                     & AdaLoRA                              & 0.2344/0.1394/0.1356             & 0.3478/0.0698/0.0691             & 0.3957/0.4278/0.3989             & 0.3537/0.3999/0.3307             & 0.2951/0.3638/0.3017             & 0.1065/0.1215/0.1195             & 0.3901/0.4547/0.4135                                         & 1             \\ % \cline{2-11} 
%                     & BayesTune-LoRA                            & 0.235/0.1364/0.1335              & 0.3205/0.0618/0.0615             & 0.3729/0.3948/0.3671             & 0.2834/0.2611/0.2091             & 0.294/0.3511/0.2867              & 0.0935/0.1044/0.1044             & 0.3567/0.4194/0.3802                                         & 0             \\ % \cline{2-11} 
%                     & FedL2P                               & 0.2231/0.1355/0.1323             & 0.322/0.061/0.0608               & 0.3867/0.4124/0.3837             & 0.3076/0.3128/0.256              & 0.2943/0.3569/0.2949             & 0.0812/0.1208/0.12               & 0.3621/0.4242/0.3865                                         & 0             \\ % \cline{2-11} 
%                     & \method{}                                 & \textbf{0.2399/0.1371/0.1336}    & \textbf{0.3523/0.0668/0.066}     & \textbf{0.4022/0.4328/0.4031}    & \textbf{0.3578/0.4201/0.3477}    & \textbf{0.3264/0.3875/0.3117}    & \textbf{0.1134/0.1153/0.1134}    & \textbf{0.3978/0.4614/0.4187}                                & \textbf{4}    \\ \hline
\multirow{5}{*}{2}  & LoRA                                   & 0.2354/0.1383/0.1344             & 0.3364/0.0659/0.0656             & 0.3871/0.4142/0.3855             & 0.3345/0.3793/0.3102             & 0.2884/0.3569/0.2968             & 0.1078/0.1208/0.1194             & 0.3835/0.4478/0.4091                                         & 0             \\ % \cline{2-11} 
                    & AdaLoRA                              & 0.2373/0.1428/0.1391             & 0.3440/0.0668/0.0665              & \textbf{0.3944/0.4273/0.3994}    & 0.3536/0.4042/0.3334             & 0.2858/0.3528/0.2937             & 0.1078/\textbf{0.1226}/\textbf{0.1200}               & 0.3834/0.4514/0.4108                                         & 2             \\ % \cline{2-11} 
                    & BayesTune-LoRA                            & 0.2406/\textbf{0.1440/0.1410}               & 0.3240/0.0579/0.0576              & 0.3797/0.4065/0.3781             & 0.2922/0.2841/0.2302             & 0.2883/0.3535/0.2927             & 0.0946/0.1132/0.1119             & 0.3674/0.4327/0.3932                                         & 1             \\ % \cline{2-11} 
                    & FedL2P                               & 0.2291/0.1356/0.1322             & 0.3329/0.0687/0.0675             & 0.3783/0.4034/0.3762             & 0.3250/0.3667/0.3032              & 0.2944/0.3614/0.3004             & 0.0869/0.1173/0.1162             & 0.3776/0.4439/0.4047                                         & 0             \\ % \cline{2-11} 
                    & \method{}                                 & \textbf{0.2434}/0.1440/0.1403     & \textbf{0.3663/0.0785/0.0764}    & 0.3941/0.4224/0.3928             & \textbf{0.3746/0.4321/0.3610}     & \textbf{0.3442/0.4057/0.3318}    & \textbf{0.1144}/0.1171/0.1161    & \textbf{0.3987/0.4646/0.4201}                                & \textbf{4}    \\ \hline
\multirow{5}{*}{4}  & LoRA                                   & 0.2339/0.1317/0.1282             & 0.3497/0.0679/0.0667             & 0.4016/\textbf{0.4369/0.4077}             & 0.3458/0.3993/0.3282             & 0.2988/0.3698/0.3044             & \textbf{0.1139}/0.1206/0.1183    & 0.3955/0.4588/\textbf{0.4186}                                         & 1             \\ % \cline{2-11} 
                    & AdaLoRA                              & 0.2331/0.1404/0.1365             & 0.3350/0.0663/0.0659              & 0.3877/0.4182/0.3900               & 0.3482/0.3948/0.3252             & 0.2886/0.3516/0.2930              & 0.1091/0.1240/0.1215              & 0.3816/0.4493/0.4091                                         & 0             \\ % \cline{2-11} 
                    & BayesTune-LoRA                            & 0.2369/0.1426/0.1395             & 0.3324/0.0609/0.0600               & 0.3846/0.4111/0.3825             & 0.3121/0.3195/0.2553             & 0.2900/0.3557/0.2951               & 0.1104/0.1181/0.1172             & 0.3707/0.4386/0.4015                                         & 0             \\ % \cline{2-11} 
                    & FedL2P                               & 0.2298/0.1364/0.1327             & 0.3376/0.0711/0.0692             & 0.3797/0.4130/0.3836              & 0.3392/0.3787/0.3130              & 0.2938/0.3648/0.3053             & 0.0974/\textbf{0.1264/0.1240}              & 0.3876/0.4561/0.4159                                         & 1             \\ % \cline{2-11} 
                    & \method{}                                 & \textbf{0.2455/0.1495/0.1455}    & \textbf{0.3671/0.0749/0.0736}    & \textbf{0.4021}/0.4333/0.3994    & \textbf{0.3831/0.4400/0.3648}      & \textbf{0.3381/0.4004/0.3225}    & 0.1129/0.1212/0.1200               & \textbf{0.4018/0.4618}/0.4172                                & \textbf{5}    \\ \hline
\multirow{5}{*}{8}  & LoRA                                   & 0.2361/0.1368/0.1329             & 0.3573/0.0708/0.0695             & 0.4017/0.4341/0.4047             & 0.3586/0.4182/0.3480              & 0.3047/0.3667/0.3029             & 0.1156/\textbf{0.1260}/0.1237              & 0.3982/0.4605/0.4186                                         & 0             \\ % \cline{2-11} 
                    & AdaLoRA                              & 0.2353/0.1443/0.1399             & 0.3272/0.0648/0.0645             & 0.3863/0.4217/0.3922             & 0.3437/0.3876/0.3183             & 0.2855/0.3552/0.2929             & 0.1044/0.1242/0.1216             & 0.3740/0.4421/0.4038                                          & 0             \\ % \cline{2-11} 
                    & BayesTune-LoRA                            & 0.2397/0.1393/0.1355             & 0.3444/0.0687/0.0678             & 0.4031/0.4327/0.4032             & 0.3294/0.3521/0.2812             & 0.2962/0.3585/0.3008             & 0.1130/0.1213/0.1193              & 0.3844/0.4480/0.4084                                          & 0             \\ % \cline{2-11} 
                    & FedL2P                               & 0.2324/0.1352/0.1316             & 0.3446/0.0698/0.0681             & 0.3819/0.4153/0.3855             & 0.3547/0.4082/0.3362             & 0.3030/0.3700/0.3076                & 0.0988/0.1217/0.1199             & 0.3940/\textbf{0.4611/0.4201}                                          & 1             \\ % \cline{2-11} 
                    & \method{}                                 & \textbf{0.2431/0.1479/0.1442}    & \textbf{0.3713/0.0792/0.0779}    & \textbf{0.4077/0.4409/0.4063}    & \textbf{0.3844/0.4441/0.3687}    & \textbf{0.3440/0.4031/0.3222}     & \textbf{0.1156}/0.1246/\textbf{0.1240}     & \textbf{0.4009}/0.4567/0.4119                                & \textbf{6}    \\ \hline
\multirow{5}{*}{16} & LoRA                                   & 0.2413/0.1387/0.1355             & 0.3605/0.0711/0.0699             & 0.3864/0.4227/0.3897             & 0.3603/0.4248/0.3533             & 0.3275/0.3894/0.3178             & \textbf{0.1194}/0.1241/0.1227    & 0.4025/\textbf{0.4659/0.4225}                                         & 1             \\ % \cline{2-11} 
                    & AdaLoRA                              & 0.2349/0.1388/0.1348             & 0.3248/0.0659/0.0655             & 0.3805/0.4141/0.3861             & 0.3346/0.3702/0.3039             & 0.2818/0.3554/0.2973             & 0.1022/0.1207/0.1181             & 0.3686/0.4379/0.4012                                         & 0             \\ % \cline{2-11} 
                    & BayesTune-LoRA                            & 0.2374/0.1351/0.1310              & 0.3556/\textbf{0.0813/0.0795}             & 0.3985/0.4317/0.4013             & 0.3477/0.3998/0.3295             & 0.2995/0.3621/0.2965             & 0.1167/0.1205/0.1186             & 0.3974/0.4579/0.4153                                         & 1             \\ % \cline{2-11} 
                    & FedL2P                               & 0.2345/0.1417/0.1368             & 0.3457/0.0643/0.0633             & 0.3884/0.4185/0.3810              & 0.3740/0.4420/0.3667               & 0.3301/0.3752/0.2945             & 0.0930/0.1223/0.1211              & 0.3956/0.4543/0.4086                                         & 0             \\ % \cline{2-11} 
                    & \method{}                                 & \textbf{0.2444/0.1447/0.1406}    & \textbf{0.3735}/0.0750/0.0740      & \textbf{0.4160/0.4462/0.4105}     & \textbf{0.3920/0.4488/0.3725}     & \textbf{0.3435/0.3992/0.3204}    & 0.1103/\textbf{0.1289/0.1270}              & \textbf{0.4052}/0.4623/0.4177                                & \textbf{5}    \\ \bottomrule
\end{tabular}
}
\end{scriptsize}
\vspace{-1.5em}
\end{table*}
\begin{table*}[t]
\centering
\caption{Avg. METEOR/ROUGE-1/ROUGE-L for \unseen{} clients in our Fed-Aya setup. {\em Base model} is off-the-shelf Llama-3.2-3B-Instruct.}
\vspace{0.5em}
\label{tab:lama_fedaya_unseen}

\begin{scriptsize}\resizebox{0.98\textwidth}{!}{
\begin{tabular}{c|l|l|l|l|l|l|l|l|l|c}
\toprule

% \textbf{Lora Rank} 
\textbf{$\mathbf{r}$} & \multicolumn{1}{c|}{\textbf{Approach}} & \multicolumn{1}{c|}{\textbf{te}} & \multicolumn{1}{c|}{\textbf{ar}} & \multicolumn{1}{c|}{\textbf{es}} & \multicolumn{1}{c|}{\textbf{en}} & \multicolumn{1}{c|}{\textbf{fr}} & \multicolumn{1}{c|}{\textbf{zh}} & \multicolumn{1}{c|}{\textbf{pt}} & \multicolumn{1}{c|}{\textbf{ru}} & \textbf{Wins} \\ \midrule
% \multirow{5}{*}{1}  & LoRA                                   & 0.1614/0.1123/0.1107             & 0.2391/0.045/0.045               & 0.4175/0.4901/0.4458             & 0.3197/0.3135/0.2507             & 0.3998/0.3333/0.3333             & 0.233/0.002/0.002                & 0.3431/0.4039/0.3855             & 0.2368/0.1813/0.1728             & 1             \\ % \cline{2-11} 
%                     & AdaLoRA                              & 0.1592/0.0914/0.0898             & 0.2363/0.0384/0.0384             & \textbf{0.4315/0.4908/0.4397}    & 0.3169/0.3104/0.2486             & 0.3222/0.7222/0.7222             & 0.2561/0.002/0.002               & 0.3302/0.4122/0.395              & 0.2446/0.1969/0.1969             & 1             \\ % \cline{2-11} 
%                     & BayesTune-LoRA                            & 0.1526/0.0591/0.0575             & 0.207/0.0406/0.0377              & 0.385/0.4615/0.4123              & \textbf{0.3266/0.3085/0.2423}    & 0.3998/0.3333/0.3333             & 0.2322/0.002/0.002               & 0.3088/0.3505/0.3313             & 0.2329/0.1663/0.1608             & 0             \\ % \cline{2-11} 
%                     & FedL2P                               & 0.158/0.0747/0.0731              & 0.2621/0.0441/0.0412             & 0.4056/0.4811/0.4332             & 0.3208/0.3096/0.2478             & 0.3998/0.3333/0.3333             & 0.2313/0.002/0.002               & 0.2928/0.3719/0.3521             & \textbf{0.2577/0.184/0.184}      & 0             \\ % \cline{2-11} 
%                     & \method{}                                 & \textbf{0.1722/0.0767/0.0751}    & \textbf{0.2759/0.0474/0.0472}    & 0.4257/0.5306/0.4758             & 0.3127/0.3198/0.2695             & \textbf{0.5012/0.746/0.746}      & \textbf{0.2655/0.0066/0.0066}    & \textbf{0.3555/0.4287/0.4092}    & 0.2337/0.1236/0.1218             & \textbf{6}    \\ \hline
\multirow{5}{*}{2}  & LoRA                                   & 0.1553/0.0854/0.0838             & 0.2425/0.0458/0.0418             & 0.4275/0.4916/0.4406             & \textbf{0.3248}/0.3120/0.2494     & 0.5513/0.6667/0.6667    & 0.2489/0.0020/0.0020               & \textbf{0.3610}/0.4228/0.4033     & 0.2242/0.1797/0.1735             & 0             \\ % \cline{2-11} 
                    & AdaLoRA                              & 0.1595/0.1108/0.1092             & 0.2326/\textbf{0.0721/0.0721}             & 0.4340/0.4954/0.4435              & 0.3234/0.3102/0.2513             & 0.5513/0.6667/0.6667   & 0.2504/0.0020/0.0020               & 0.3338/0.4121/0.3970              & 0.2335/0.1758/0.1703             & 1             \\ % \cline{2-11} 
                    & BayesTune-LoRA                            & \textbf{0.1676}/0.0888/0.0858    & 0.2243/0.0487/0.0457             & 0.3821/0.4564/0.4089             & 0.3176/0.3069/0.2437             & 0.3998/0.3333/0.3333             & 0.2308/0.0020/0.0020               & 0.3052/0.3716/0.3553             & 0.2484/0.1779/0.1773             & 0             \\ % \cline{2-11} 
                    & FedL2P                               & 0.1568/\textbf{0.1156/0.1097}             & 0.2368/0.0496/0.0496             & 0.4134/0.4810/0.4350               & 0.3141/0.3050/0.2418              & 0.3998/0.3333/0.3333             & 0.2350/0.0020/0.0020               & 0.3442/0.4012/0.3853             & 0.2451/0.2010/0.2010               & 1             \\ % \cline{2-11} 
                    & \method{}                                 & 0.1674/0.0736/0.0720              & \textbf{0.2636}/0.0485/0.0485    & \textbf{0.4391/0.5335/0.4796}    & 0.3084/\textbf{0.3278/0.2718}             & 0.2155/0.2222/0.2222             & \textbf{0.2771/0.0066/0.0066}    & 0.3477/\textbf{0.4295/0.4110}              & \textbf{0.3413/0.2929/0.2662}    & \textbf{5}    \\ \hline
\multirow{5}{*}{4}  & LoRA                                   & 0.1463/0.0852/0.0758             & 0.2475/0.0332/0.0332             & 0.4178/0.4762/0.4269             & \textbf{0.3394}/0.3264/0.2648    & 0.5513/0.6667/0.6667             & 0.2566/0.0020/0.0020               & \textbf{0.3554}/0.425/0.4044     & 0.2368/0.1553/0.1530              & 0             \\ % \cline{2-11} 
                    & AdaLoRA                              & 0.1634/0.0798/0.0782             & 0.2229/\textbf{0.0575/0.0575}             & 0.4258/0.4871/0.4387             & 0.3191/0.3129/0.2519             & 0.5513/0.6667/0.6667             & 0.2458/0.0020/0.0020               & 0.3523/0.4242/0.4067             & 0.2427/\textbf{0.1716/0.1654}             & 2             \\ % \cline{2-11} 
                    & BayesTune-LoRA                            & \textbf{0.1689/0.1044/0.1027}    & 0.2308/0.0557/0.0557             & 0.3911/0.4617/0.4141             & 0.3219/0.3088/0.2445             & 0.3998/0.3333/0.3333             & 0.2349/0.0020/0.0020               & 0.3233/0.3798/0.3642             & 0.2437/0.1714/0.1653             & 1             \\ % \cline{2-11} 
                    & FedL2P                               & 0.1556/0.0627/0.0611             & 0.2465/0.0372/0.0331             & 0.4292/0.4984/0.4445             & 0.3237/0.3103/0.2498             & 0.5513/0.6667/0.6667             & 0.2454/0.0020/0.0020               & 0.3446/0.4223/0.4056             & 0.2329/0.1703/0.1625             & 0             \\ % \cline{2-11} 
                    & \method{}                                 & 0.1618/0.0706/0.0689             & \textbf{0.2621}/0.0386/0.0386    & \textbf{0.4619/0.5591/0.5061}    & 0.3194/\textbf{0.3381/0.2835}             & 0.3998/0.3333/0.3333             & \textbf{0.2898/0.0094/0.0094}    & 0.3508/\textbf{0.4306/0.4115}             & \textbf{0.2538}/0.1531/0.1531    & \textbf{4}    \\ \hline
\multirow{5}{*}{8}  & LoRA                                   & 0.1600/0.0835/0.0791               & 0.2669/0.0524/0.0484             & 0.4312/0.5136/0.4573             & 0.3352/0.3326/0.2706             & \textbf{0.5664}/0.6667/0.6667    & 0.2564/0.0045/0.0045             & 0.3465/0.4264/0.4090              & 0.2275/\textbf{0.1847/0.1847}             & 1             \\ % \cline{2-11} 
                    & AdaLoRA                              & 0.1604/\textbf{0.1045/0.1029}             & 0.2226/0.0606/0.0606             & 0.4136/0.4889/0.4350              & 0.3200/0.3082/0.2462               & 0.3164/0.5000/0.5000                   & 0.2503/0.0020/0.0020               & 0.3398/0.4203/0.4008             & 0.2370/0.1687/0.1625              & 1             \\ % \cline{2-11} 
                    & BayesTune-LoRA                            & 0.1607/0.0735/0.0688             & 0.2377/0.0731/0.0731             & 0.4102/0.4803/0.4352             & 0.3261/0.3114/0.2458             & 0.3998/0.3333/0.3333             & 0.2476/0.0020/0.0020               & 0.3483/0.3998/0.3817             & 0.2514/0.1668/0.1612             & 0             \\ % \cline{2-11} 
                    & FedL2P                               & 0.1602/0.0586/0.0570              & 0.2462/0.0447/0.0447             & 0.4339/0.5007/0.4513             & \textbf{0.3482}/0.3281/0.2642    & 0.3301/\textbf{0.7460/0.7460}               & 0.2617/0.0020/0.0020              & 0.3362/0.4262/0.4109             & 0.2300/0.1542/0.1542               & 1             \\ % \cline{2-11} 
                    & \method{}                                 & \textbf{0.1720}/0.0948/0.0890      & \textbf{0.2787/0.0804/0.0804}    & \textbf{0.5022/0.5825/0.5168}    & 0.3377/\textbf{0.3661/0.2984}             & 0.2730/0.2756/0.2756              & \textbf{0.3186/0.0249/0.0241}    & \textbf{0.3623/0.4375/0.4158}    & \textbf{0.2770}/0.1171/0.1171     & \textbf{5}    \\ \hline
\multirow{5}{*}{16} & LoRA                                   & 0.1610/\textbf{0.0901/0.0885}              & 0.2585/0.0403/0.0371             & 0.4259/0.5245/0.4700               & 0.3104/0.3302/0.2775             & \textbf{0.4077}/0.3571/0.3571    & 0.2651/0.0020/0.0020               & \textbf{0.3501}/0.4260/0.4065     & 0.2751/\textbf{0.2048/0.2048}             & 2             \\ % \cline{2-11} 
                    & AdaLoRA                              & 0.1412/0.0679/0.0659             & 0.2119/0.0439/0.0439             & 0.4149/0.4867/0.4382             & 0.3241/0.3123/0.2473             & 0.3764/0.2879/0.2879             & 0.2421/0.0020/0.0020               & 0.3308/0.4071/0.3882             & 0.2260/0.1723/0.1667              & 0             \\ % \cline{2-11} 
                    & BayesTune-LoRA                            & 0.1518/0.0641/0.0560              & 0.2448/0.0318/0.0318             & 0.4331/0.5021/0.4519             & 0.3187/0.3128/0.2524             & 0.3928/0.3148/0.3148             & 0.2546/0.0020/0.0020               & 0.3378/0.3921/0.3728             & 0.2566/0.1867/0.1860              & 0             \\ % \cline{2-11} 
                    & FedL2P                               & 0.1613/0.0726/0.0707             & 0.2605/\textbf{0.0855/0.0841}             & 0.4181/0.5049/0.4764             & 0.3383/0.3723/0.3079             & 0.4065/\textbf{0.3889/0.3889}             & 0.2547/0.0060/0.0060               & 0.3262/0.3915/0.3695             & 0.2574/0.1404/0.1404             & 2             \\ % \cline{2-11} 
                    & \method{}                                 & \textbf{0.1670}/0.0826/0.0810      & \textbf{0.2809}/0.0751/0.0751    & \textbf{0.4935/0.5715/0.5119}    & \textbf{0.3424/0.3800/0.3146}      & 0.2562/0.2626/0.2626             & \textbf{0.3293/0.0364/0.0330}     & 0.3452/\textbf{0.4280/0.4103}              & \textbf{0.3320}/0.1856/0.1589     & \textbf{4}    \\ \bottomrule

\end{tabular}
}

\end{scriptsize}
\vspace{-1.2em}
\end{table*}


\subsection{Results on Instruction-Tuning Generation}\label{sec:ift_gen}

We evaluate our approach on the more challenging real-world multilingual benchmark, Fed-Aya. Tables~\ref{tab:lama_fedaya_seen} and \ref{tab:lama_fedaya_unseen} show the average METEOR~\cite{meteor}/ROUGE-1/ROUGE-L~\cite{ROUGE} of each language given the off-the-shelf instruction finetuned Llama-3.2-3B (Llama-3.2-3B-Instruct) for \seen{} and \unseen{} clients respectively. Similarly, in Appendix Tables~\ref{tab:mobilellama_fedaya_seen} and \ref{tab:mobilellama_fedaya_unseen}, we show the same tables given a pretrained MobileLLaMA-1.4B model  trained using Standard FL with LoRA following the training recipe from FedLLM-Bench~\cite{fedllm-bench}. These two models represent scenarios where the \basemodel{} may or may not be trained using FL. Similarly to our text classification results, we report ``Wins" if an approach has a better performance in at least $2$ out of $3$ metrics. 

In all four tables, \method{} outperforms baselines in most scenarios. We also observe that FedL2P underperforms standard baselines in most cases, a phenomenon also observed for our XNLI setup when the \basemodel{} is trained with FedDPA-T (Tables~\ref{tab:xnli_seen_feddpa}). We hypothesize that the inner-loop optimization in FedL2P fail to reach a stationary point\footnote{FedL2P relies on the implicit function theorem for hypergradient computation.} due to the inherent task difficulty (Fed-Aya) or a less-performant \basemodel{}, resulting in a sub-optimal hypergradient and downstream performance. 

\subsubsection{Limitations of \method{}.} In some cases, \method{} falls short, especially in the recall performance (ROUGE), such as Russian (\textit{ru}) and French (\textit{fr}) for \unseen{} clients for both {\em base models}. These cases highlight a couple of limitations of our approach: \textit{i)} \textit{ru} is not seen by PSG during federated training; there are no \textit{ru} samples in any clients in the seen pool and \textit{ii)} none of the clients in the unseen pool have \textit{fr} as a predominant language (Fig.~\ref{fig:fed-aya}). In the case of \textit{ru}, there are no other languages that are similar to \textit{ru} in the seen pool, resulting in worse performance. Hence, we do not expect a similar outcome in datasets with a more diverse pool of clients.

For \textit{fr}, as the number of predominant language samples are orders of magnitude higher than that of \textit{fr} samples, the generated $\bm{\lambda}$ are catered towards the predominant language. A simple solution to counteract this limitation is to extend PSG to generate $\bm{\lambda}$ per instance, rather than per client. However, doing so is extremely costly, requiring a forward pass through the PSG for every sample. This calls for novel efficient solutions that can better handle each client's minority languages and is left as future work.

\subsection{Cost of \method{}}

Table~\ref{tab:cost} shows the mean latency, in seconds, and the peak memory usage across 100 runs on the first client in the \seen{} pool for $r=16$ using a single Nvidia A100 GPU. Non-FL baselines do not incur a federated training cost while FL approaches requires training the PSG. Comparing FedL2P and \method{}, \method{} does not require expensive second-order optimization, resulting in better efficiency. We also note that FedL2P needs to be run for every rank while \method{} runs once for all targeted ranks.

For communication costs, not shown in the table, \method{} is more costly as it predicts per LoRA rank while FedL2P predicts per layer. Nonetheless, these costs are negligible compared to running FL on the \basemodel{}; FedL2P uses 0.02\% and 0.002\% and \method{} uses 0.2\% and 0.16\% of the parameters of mBERT and Llama-3.2-3B respectively.

During inference, FL-based approaches incur an additional forward pass of \basemodel{} and the PSG compared to non-FL approaches. Memory-wise, \method{} results in the smallest memory footprint for autoregressive generation as the PSG learns not to attach LoRA modules $\lambda_l=0$ on some layers, skipping {\em matmul} operations entirely.

\begin{figure}[t]
    \vspace{-0.2em}
    \small
    \centering
    \includegraphics[width=0.94\columnwidth,trim={0cm 0cm 0cm 3.5cm},clip]{figures/xnli_out_0.5_seen.png}
    % \captionsetup{font=small,labelfont=bf}
    \vspace{-2em}
    \caption{$\bm{\lambda}$ distance among languages in our XNLI setup. Each block shows the log-scale normalized average Euclidean distances between all pairs of clients' $\bm{\lambda}$ for two languages (see text). The smaller the distance, the more similar $\bm{\lambda}$ is. }
    \label{fig:xnli_out}
    \vspace{-1.5em}
\end{figure}

\subsection{Further Analysis}\label{sec:analysis}

In this section, we further analyze $\bm{\lambda}$ and how they differ across languages. Surprisingly, we find that \method{} learns language-agnostic rank structures. In other words, depending on the task and the \basemodel{}, the rank structure of $\bm{\lambda}$ is fixed across languages. For instance, in the case where $r=2$, \method{} allocate ranks to dense layers instead of attention blocks. With more budget, \textit{e.g.},~$r=16$, \method{} allocates more rank to either the query attention layer or the value attention layer depending on the setup. We show these rank structures across all setups for $r=2$ and $r=16$ in Appendix Fig.~\ref{fig:xnli_fedavg_out_r16}-\ref{fig:llama3_fedavg_out_r2}.

While the rank structure is the same across languages, the rank-wise scales (absolute values of $\bm{\lambda}$) differ. Following FedL2P, we visualize the difference in $\bm{\lambda}$ for different languages using the normalized mean distance, $d(j,k)$, between all clients pairs holding data for languages $j$ and $k$. Fig.~\ref{fig:xnli_out} and Appendix Fig.~\ref{fig:masakha_out} show these distances for XNLI and MasakhaNEWS setup respectively. Specifically, the value of each block in each figure is computed as follows: $\log(\frac{d(j,k)}{\sqrt{d(j,j)}\sqrt{d(k,k)}})$. Hence, the smaller the distance, the more similar $\bm{\lambda}$ is between languages. The results are aligned with our intuition that similar languages have similar $\bm{\lambda}$. For instance, the closest language to Urdu (\textit{ur}) is Arabic (\textit{ar}), both of which have the closest $\bm{\lambda}$ similarity (Fig.~\ref{fig:xnli_out}); likewise, for Tigrinya (\textit{tir}) and Amharic (\textit{amh}) in Appendix Fig.~\ref{fig:masakha_out}. We also observe that unrelated languages have similar $\bm{\lambda}$, \textit{e.g.},~Mandarin (\textit{zh}) and Vietnamese (\textit{vi}) share similar $\bm{\lambda}$ with the Indo-European languages (Fig.~\ref{fig:xnli_out}). This finding adds to existing evidence that leveraging dissimilar languages can sometimes benefit particular languages~\cite{fedllm-bench}.

