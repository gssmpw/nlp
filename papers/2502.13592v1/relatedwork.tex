\section{Related Work}
\citet{mahajan-shaikh-2021-need} categorize MPC corpora into three types: Spoken Unscripted, Spoken Scripted, and Written. In this work, we address Spoken Unscripted and Written MPCs. Spoken Unscripted MPCs typically occur in in-person discussions, with spontaneous interactions. Written MPCs, on the other hand, are characteristic of online platforms: here conversations unfold asynchronously, often due to platform-specific constraints. While social media allow for rapid collection of large-scale written MPCs, these datasets often come with incomplete interaction metadata. Many datasets record only explicit reply-to relationships, neglecting implicit addressees and richer conversational dynamics \cite{ouchi-tsuboi-2016-addressee, zhang-etal-2018-conversations, chang-danescu-niculescu-mizil-2019-trouble}. \citet{weiEtAl2023} point out that well-known MPC corpora \cite{ritter-etal-2010-unsupervised, baumgartner2020pushshiftredditdataset, lowe-etal-2015-ubuntu} are useful for response generation, but not for more interactive tasks. Only most recent efforts focus on capturing conversational dynamics (i.e., go beyond text content) \cite{hua-etal-2024-get}.

Structural analyses of social communication networks have primarily focused on interaction patterns across multiple conversations \cite{panzarasa2009patterns, coletto2017motif, 10.1145/3178876.3186139, felmlee2021dyads}. This confirms the relevance of such structures in studying conversation dynamics. However, our focus is on interactions emerging within a single conversation rather than across multiple discussions, applying the same structural analysis techniques. 

To the best of our knowledge, the only existing attempt at generating synthetic MPCs was made by \citet{chen-etal-2023-places}. However, their work primarily focused on conversations involving at most three participants, limiting the complexity of interactions. In contrast, our study explores the generation of MPCs with four or more participants, leading to more elaborate discussion dynamics. While this increased complexity allows for richer conversational structures, it also introduces a higher likelihood of generation errors, necessitating a rigorous evaluation process to assess the quality and consistency of the generated dialogues.