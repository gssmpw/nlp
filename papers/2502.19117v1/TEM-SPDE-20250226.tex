      %------------------------------------------------------------------------------
% Beginning of mcom_sample.tex
%------------------------------------------------------------------------------
%
% AMS-LaTeX version 2 sample file for journals, based on amsart.cls.
%
%        ***     DO NOT USE THIS FILE AS A STARTER.      ***
%        ***       USE THE MCOM-L-TEMPLATE FILE.         ***
%
% Please see the README-MCOM-OVERLEAF.TXT file for more information.
%
\documentclass{mcom-l}

\newtheorem{tm}{Theorem}[section]
\newtheorem{ap}{Assumption}[section]
\newtheorem{df}{Definition}[section]
\newtheorem{prop}{Proposition}[section]
\newtheorem{lm}{Lemma}[section]
\newtheorem{cor}{Corollary}[section]
% \newtheorem{lemma}[theorem]{Lemma}

% \theoremstyle{definition}
% \newtheorem{definition}[theorem]{Definition}
% \newtheorem{example}[theorem]{Example} 
% \newtheorem{exercise}[theorem]{Exercise}

\theoremstyle{remark}
% \newtheorem{remark}[theorem]{Remark}
\newtheorem{rk}{Remark}[section]
\newtheorem{ex}{Example}[section] 

\numberwithin{equation}{section}


\usepackage{graphicx} 
\usepackage{extarrows}
\usepackage{latexsym}
%\usepackage{showkeys}
\usepackage{amsmath}
\usepackage{amssymb}
\usepackage{amsfonts}
\usepackage{verbatim}
\usepackage{mathrsfs}
\usepackage[modulo]{lineno} % 行数
%\usepackage[latin1]{inputenc}
\usepackage{color}
\usepackage{xcolor}
\usepackage[colorlinks,citecolor=blue,urlcolor=blue]{hyperref}
\usepackage{soul}

% \newtheorem{tm}{Theorem}[section]


%    Several Notations

\newcommand{\tr}{\top}
\newcommand{\Tr}{\operatorname{Tr}}

\newcommand{\cc}{\mathbb C} 
\newcommand{\ee}{\mathbb E}
\newcommand{\ff}{\mathbb F}
\newcommand{\ii}{\mathbb I}
\newcommand{\pp}{\mathbb P}
\newcommand{\nn}{\mathbb N}
\newcommand{\rr}{\mathbb R}
\newcommand{\hh}{\mathbb H}
\newcommand{\ww}{\mathbb W} 
\newcommand{\zz}{\mathbb Z}

\newcommand{\br}{\mathbf r}
\newcommand{\bs}{\mathbf s}

\newcommand{\AAA}{\mathcal A}
\newcommand{\BB}{\mathcal B}
\newcommand{\CC}{\mathcal C}
\newcommand{\LL}{\mathcal L}
\newcommand{\TT}{\mathcal T}
\newcommand{\OO}{\mathcal O}
\newcommand{\PP}{\mathcal P}
\newcommand{\RR}{\mathcal R}
\newcommand{\HH}{\mathcal H}
\newcommand{\EE}{\mathcal E}
\newcommand{\WW}{\mathcal W}

\newcommand{\OOO}{\mathscr O}
\newcommand{\FFF}{\mathscr F}
\newcommand{\MMM}{\mathscr M}
\newcommand{\HHH}{\mathscr H}

\newcommand{\<}{\langle}
\renewcommand{\>}{\rangle}
\allowdisplaybreaks \allowdisplaybreaks[4] 
\newcommand{\Var}{\operatorname{Var}}
\newcommand{\Cov}{\operatorname{Cov}}
\newcommand{\e}{\mathrm{e}}
\newcommand{\dd}{\mathrm{d}}


\newcommand{\abs}[1]{\left\lvert #1 \right\rvert}
\newcommand{\norm}[1]{\left\lVert #1 \right\rVert}
\newcommand{\innerproduct}[2]{\left\langle #1,#2 \right\rangle}
\newcommand{\proj}{\operatorname{proj}}
\newcommand{\rank}{\operatorname{rank}}
 

%    Blank box placeholder for figures (to avoid requiring any
%    particular graphics capabilities for printing this document).
\newcommand{\blankbox}[2]{%
  \parbox{\columnwidth}{\centering
%    Set fboxsep to 0 so that the actual size of the box will match the
%    given measurements more closely.
    \setlength{\fboxsep}{0pt}%
    \fbox{\raisebox{0pt}[#2]{\hspace{#1}}}%
  }%
}

\begin{document}

\title[Ergodicity and Estimates for Tamed Schemes of Super-linear SPDEs]
{Geometric Ergodicity and Optimal Error Estimates for a Class of Novel Tamed Schemes to Super-linear Stochastic PDEs}
 
%    Information for the first author
\author{Zhihui LIU}
\address{Department of Mathematics \& National Center for Applied Mathematics Shenzhen (NCAMS) \& Shenzhen International Center for Mathematics, Southern University of Science and Technology, Shenzhen 518055, People's Republic of China}
% \curraddr{}
\email{liuzh3@sustech.edu.cn (Corresponding author)}
 
%    Information for the second author
\author{Jie SHEN}
%    Address of record for the research reported here
\address{Eastern Institute of Technology, Ningbo, Zhejiang 315200, People's Republic of China}
%    Current address
% \curraddr{}
\email{jshen@eitech.edu.cn}

%    \thanks will become a 1st page footnote.
\thanks{} 
 

%    General info

\subjclass[2010]{Primary 60H35; Secondary 60H15, 65M60}
 
\keywords{super-linear stochastic partial differential equation, 
stochastic Allen--Cahn equation,
numerical invariant measure, 
numerical ergodicity, 
strong error estimate}

\begin{abstract} 
We construct a class of novel tamed schemes that can preserve the original Lyapunov functional for super-linear stochastic PDEs (SPDEs), including the stochastic Allen--Cahn equation, driven by multiplicative or additive noise, and provide a rigorous analysis of their long-time unconditional stability. We also show that the corresponding Galerkin-based fully discrete tamed schemes inherit the geometric ergodicity of the SPDEs and establish their convergence towards the SPDEs with optimal strong rates in both the multiplicative and additive noise cases.   
\end{abstract}

\maketitle
 

\section{Introduction}

 
The long-time numerical behavior of super-linear stochastic partial differential equations (SPDEs), especially the numerical ergodicity of SPDEs whose coefficients violate the classical Lipschitz continuity and linear growth assumptions, is an intriguing question.
As a fundamental long-term behavior, ergodicity characterizes the equivalence between temporal and spatial averages, also known as the ergodic limit, for Markov processes or chains generated by SPDEs and their numerical discretizations. This concept is crucial in quantum mechanics, fluid dynamics, financial mathematics, and many others \cite{DZ96, HW19}.

In practical applications, computing the ergodic limit, i.e., the mean of a given function with respect to the equilibrium distribution, also known as the invariant measure, is often desired. However, obtaining an explicit expression for the invariant measure of an infinite-dimensional stochastic system is rarely feasible. As a result, it is generally impossible to directly compute the ergodic limit for nonlinear SPDEs driven by multiplicative noise. This challenge has driven significant research in the past decade, inspiring the development of numerical algorithms designed to preserve the ergodicity of the original system and efficiently approximate the ergodic limit.

In this paper, we investigate the following super-linear SPDE:
\begin{align}\label{see-fg} 
& {\rm d} X(t, \xi)  
=(\Delta X(t, \xi)+f(X(t, \xi))) {\rm d}t
+g(X(t, \xi)) {\rm d}W(t, \xi), 
\quad (t, \xi) \in \rr_+\times \OOO,
\end{align}
with (homogenous) Dirichlet boundary condition (DBC) $X(t, \xi)=0$, $(t, \xi) \in \rr_+\times \partial \OOO$, and initial  condition $X(0, \xi)=X_0(\xi)$, $\xi \in \OOO$. In the above, 
 the physical domain $\OOO \subset \rr^d$ ($d=1,2,3$) is a bounded open set with a piecewise smooth boundary, 
 $f$ and $g$ are assumed to be monotone-type with polynomial growth, and $W$ is an infinite-dimensional Wiener process (cf. Section \ref{sec2} for more details).
 
It is clear that Eq. \eqref{see-fg} includes the following stochastic Allen--Cahn equation (SACE), arising from phase transition in materials science by stochastic perturbation, as a special case:
\begin{align} \label{ac}
{\rm d} X(t, \xi)= \Delta X(t, \xi)  {\rm d}t + \epsilon^{-2} (X(t, \xi)-X(t, \xi)^3) {\rm d}t + g(X(t, \xi)) {\rm d}W(t, \xi), 
\end{align}
 where the positive index $\epsilon \ll 1$ is the interface thickness; see, e.g., \cite{BGJK23, BJ19, BP24, CH19, FLZ17, LQ20, LQ21, OPW23} and references therein.  
We aim at constructing efficient numerical integrators for  \eqref{see-fg}, whose solution is uniquely ergodic with respect to an equilibrium distribution that
\begin{enumerate}
\item
are ergodic with respect to an approximate equilibrium distribution of Eq. \eqref{see-fg} on infinite time intervals; 

\item
strongly converge with optimal convergence rate to the solutions of Eq. \eqref{see-fg} on any finite time intervals. 
\end{enumerate}   

Significant progress has been made in the design and analysis of numerical approximations for ergodic limits in finite-dimensional cases, particularly for stochastic ordinary differential equations (SODEs) \cite{LV22, LL24, LW24, MSH02}. In contrast, the construction and analysis of numerical ergodic limits for SPDEs, even those driven by additive noise, remain in the early stages of development.

The backward Euler method, along with its Galerkin-based full discretizations, satisfies properties (1) and (2) above; however, these methods often incur high computational costs. For example, \cite{Liu24a} discusses the exponential ergodicity of SACE driven by multiplicative trace-class noise with moderate interface thickness, while \cite{Liu24b} establishes the unique ergodicity of SACE driven by multiplicative white noise, irrespective of the interface thickness. Additionally, \cite{Bre22, BK17, BV16, CHW17, CHS21, WC24} explore approaches for approximating the invariant measures of SPDEs driven by additive white or colored noise.

To reduce computational costs, nonlinearity-explicit schemes are often preferred.  
In order to construct nonlinearity-explicit schemes that can inherit the unique ergodicity of Eq. \eqref{see-fg}, as in the tamed methods studied in the finite-dimensional case, e.g.,  \cite{HJK12, LNSZ24, NNZ25}, the numerical Lyapunov structure plays a crucial role.
However, \cite{BHJKLS19} demonstrated that the classical Euler-Maruyama (EM) scheme and its Galerkin-based full discretizations, when applied to Eq. \eqref{see-fg} with super-linear growth coefficients, would lead to a blow-up in the $p$-th moment for all $p \geq 2$. In particular, the square function, although a natural Lyapunov function for the monotone equation  \eqref{see-fg}, is not a suitable Lyapunov function for the EM scheme.

To address this, we propose a family of novel, nonlinearity-tamed schemes that achieve the two main objectives  (1) and (2) for super-linear SPDEs \eqref{see-fg} driven by more subtle and challenging multiplicative noise. Compared to existing tamed schemes in the literature, such as those in \cite{Bre22, HS23, WC24}, our schemes are simpler as we tame the values of the numerical solutions rather than their norms at each step.

A key step of our approach is the rigorous analysis of long-time unconditional stability for these temporal-tamed schemes and their Galerkin-based full discretizations. This analysis shows that the tamed schemes preserve the same Lyapunov structure as the super-linear equation  \eqref{see-fg}. Combined with the equivalence of transition probabilities, these temporal-tamed schemes are shown to inherit the geometric ergodicity of  \eqref{see-fg} under certain non-degenerate conditions. 
To establish the second property, it is crucial to estimate the effect of these tamed methods on the dynamics of  \eqref{see-fg} through strong error estimates. 

Our main contributions include:
\begin{itemize}
\item constructed a class of novel tamed schemes for super-linear SPDEs that preserve the Lyapunov structure and can be efficiently implemented; 
\item proved that the solutions of these new schemes are unconditionally stable at any time interval;
\item established the strong convergence of these schemes with optimal spatial and temporal orders for both the multiplicative and additive noise cases.
\end{itemize}
To our knowledge, this is the first construction of nonlinearity-explicit schemes that preserve the unique ergodicity of super-linear SPDEs driven by either additive or multiplicative noise.
We emphasize that all of our results are valid for one, two, and three space dimensions.

The structure of the paper is as follows. We introduce some necessary preliminaries and propose the tamed schemes in Section \ref{sec2}. In Section \ref{sec3}, we develop the Lyapunov structure and probabilistic regularity properties of the tamed schemes with Galerkin-based full discretizations, from which we establish their geometric ergodicity. We establish in Sections \ref{sec4} and \ref{sec5}  the strong error estimates between these tamed full discretizations and Eq. \eqref{see-fg} in the multiplicative and additive noise cases, respectively.
 


\section{Preliminaries}
\label{sec2}


In this section, we first introduce some notations and main assumptions and then propose the tamed schemes. 
  
  \subsection{Notations}
  \label{sec2.1}

  
For $q\in [1,\infty]$, we denote by $(L_\xi^q,\|\cdot\|_{L_\xi^q})$, $(L_t^q,\|\cdot\|_{L_t^q})$, $(L_\omega^q,\|\cdot\|_{L_\omega^q})$  the usual real-valued Lebesgue spaces in $\OOO$, $(0, T)$, and $\Omega$, respectively.
For convenience, sometimes we use the temporal, sample path, and spatially mixed norm $\|\cdot\|_{L_\omega^p L_t^r L_\xi^q}$ in different orders, such as
\begin{align*}
\|X\|_{L_\omega^p L_t^r L_\xi^q}
:=\Big(\int_\Omega \Big(\int_0^T \Big(\int_0^1 |X(t,\xi,\omega)|^q {\rm d} \xi\Big)^\frac rq {\rm d}t\Big)^\frac pr \pp({\rm d}\omega)\Big)^\frac 1p
\end{align*}
for $X\in L_\omega^p L_t^r L_\xi^q$,
with the usual modification for $r=\infty$ or $q=\infty$. 

Let $H:=\{u \in L_\xi^2: u|_{\partial \OOO}=0\}$ with its inner product and norm  denoted by $\|\cdot\|$ (or $\|\cdot\|_{L_\xi^2}$) and $\<\cdot, \cdot\>$, respectively. We denote by $A: {\rm Dom}(A)\subset H\rightarrow H$ the Dirichlet Laplacian on $H$.
Then $A$ is the infinitesimal generator of an analytic $C_0$-semigroup $S(\cdot)=e^{A\cdot}$ on $H$, and  one can define the fractional powers $(-A)^\theta$ for $\theta\in \rr$ of the self-adjoint and positive operator $-A$.
Let $\dot H^\theta$ be the domain of 
$(-A)^{\theta/2}$ equipped with the norm $\|\cdot\|_\theta$:
\begin{align} \label{h-theta}
\|u\|_\theta:=\|(-A)^\frac\theta2 u\|,
\quad u \in \dot H^\theta.
\end{align}
In particular, one has $\dot H^0=H$.
The inner product in $V:=\dot H^1$ is denoted by $\<\cdot, \cdot\>_1$. 
It is clear that $V^*:=\dot H^{-1}$ is the dual space of $V$ (with respect to $\<\cdot, \cdot\>$); the dualization between them is denoted by ${_{-1}}\<\cdot, \cdot\>_1$. 
We use ${\rm Id}$ to denote the identity operator on various Hilbert spaces if there is no confusion. 

Let $U$ be a separable Hilbert space and ${\bf Q}$ be a self-adjoint and nonnegative definite operator on $U$.
Denote by $U_0:={\bf Q}^{1/2} U$ and 
$(\LL_2^\theta:=HS(U_0; \dot H^\theta), \|\cdot\|_{\LL_2^\theta})$ the space of Hilbert--Schmidt operators from $U_0$ to $\dot H^\theta$ for $\theta\in \rr_+$.
% The spaces $U$, $U_0$, and $\LL_2^\theta$ are equipped with Borel $\sigma$-algebras $\BB(U)$, $\BB(U_0)$, and $\BB(\LL_2^\theta)$, respectively.
Let $W:=\{W(t):\ t\in [0,T]\}$ be a $U$-valued ${\bf Q}$-Wiener process in the stochastic basis $(\Omega,\FFF, \ff,\pp)$, i.e., there exists an orthonormal basis $\{q_k\}_{k=1}^\infty$ of $U$ which forms the eigenvectors of $\bf Q$ with the eigenvalues $\{\lambda^{\bf Q}_k\}_{k=1}^\infty$, and a sequence of mutually independent Brownian motions $\{\beta_k\}_{k=1}^\infty $ such that (see \cite[Proposition 2.1.10]{LR15})
\begin{align*}
W(t)=\sum_{k\in \nn_+} \sqrt{\lambda^{\bf Q}_k} q_k \beta_k(t),
\quad t \ge 0.
\end{align*} 
We mainly focus on the case of trace operator, i.e., $\sum_{k\in \nn_+} \lambda^{\bf Q}_k<\infty$; we refer to \cite{Liu24b} for the numerical ergodicity of an implicit scheme for Eq. \eqref{see-fg} driven by white noise.
  
  

 
 
   \subsection{Assumptions}
  
   
For the study of the Lyapunov structure of numerical schemes, our central assumption on the coefficients $f$ and $g$ in \eqref{see-fg} is the following coupled coercive and polynomial growth conditions. 
 
 
\begin{ap} \label{ap}   
There exist positive constants $L_i$, $i=1,2,3,4$, and $q$ such that for all $\xi \in \rr$ and $u \in L_\xi^{q+2}$, 
\begin{align} 
u(\xi) f(u(\xi))+ |g(u(\xi))|^2 \sum_{k \in \nn} \lambda^{\bf Q}_k |q_k(\xi)|^2 
& \le L_1 - L_2 |u(\xi)|^{q+2}, \label{coe} \\
|f(\xi)| & \le L_3+ L_4 |\xi|^{q+1}. \label{f-grow}
\end{align}  
 \end{ap}
 
 
 \begin{rk} 
The coupled coercive condition \eqref{coe} implies the following classical coercive condition in the infinite-dimensional sense:
\begin{align*} 
\<u, f(u)\> + \|g(u)\|_{\LL_2^0} ^2 
\le L_1 |\OOO| - L_2 \|u\|_{L_\xi^{q+2}}^{q+2},
\quad u \in L_\xi^{q+2},
 \end{align*} 
 where $|\OOO|$ denotes the area of $\OOO$.
 In the case 
\begin{align} \label{cq}
C_{\bf Q}:=\sum_{k \in \nn} \lambda^{\bf Q}_k \|q_k\|_\infty^2<\infty,
\end{align} 
which holds when $\sup_{k \in \nn} \|q_k\|_\infty<\infty$, \eqref{coe} is equivalent to the following type of the coupled coercive condition:
 \begin{align*} 
\xi f(\xi)+ C_{\bf Q} |g(\xi)|^2 
& \le L_1 - L_2 |\xi|^{q+2}, \quad \xi \in \rr.
\end{align*} 
 \end{rk}

 \begin{rk}
The conditions \eqref{coe}-\eqref{f-grow} yield a polynomial growth condition on $g$, i.e., there exist positive constants $L_5$ and $L_6$ such that 
 \begin{align*}
|g(u(\xi))|^2 \sum_{k \in \nn} \lambda^{\bf Q}_k |q_k(\xi)|^2 
\le L_5+L_6 |u(\xi)|^{q+2},
\quad \xi \in \rr, ~  u \in L_\xi^{q+2},
 \end{align*} 
which shows that 
\begin{align*}
\|g(u)\|_{\LL_2^0} ^2 \le L_5 |\OOO| +L_6 \|u\|_{L_\xi^{q+2}}^{q+2},
\quad u \in L_\xi^{q+2}. 
 \end{align*}  
\end{rk}

\begin{rk} \label{rk-ap-pol} 
Assumption \ref{ap} particularly includes the polynomial drift (with a negative leading coefficient) and diffusion functions.
Indeed, let $k \in \nn$ and assume 
 \begin{align}\label{bs-ex}
f(\xi)=\sum_{i=0}^{2k+1} a_i \xi^i, \quad
g(\xi)=\sum_{j=0}^{k+1} c_j \xi^j,
\quad \xi \in \rr,
 \end{align} 
 where $a_i, c_j \in \rr$, $i=0,1,\cdots,2k+1$, $j=0,1,\cdots,k+1$.  
Through elementary algebraic calculations, one derives \eqref{coe}-\eqref{f-grow} for positive constants $L_i$ depending on the coefficients of the polynomials in \eqref{bs-ex}, $i=1,2,3,4$, and $q=2k$ provided \eqref{cq} holds and  
 \begin{align*}
a_{2k+1}+c_{k+1}^2 C_{\bf Q}<0.
 \end{align*} 
 In particular, this includes the double-well potential drift corresponding to $f(\xi) = (1-\xi^2) \xi$, $\xi \in \rr$, and quadratic growth diffusion function $g$ with small absolute leading coefficients.
\end{rk}

\begin{rk} \label{rk-ap-lin}
If $g$ is of linear growth (including the additive noise case), then the condition \eqref{coe} is decomposed into
\begin{align} 
\xi f(\xi) & \le L_1'-L_2' |\xi|^{q+2}, \label{f-coe} \\ 
|g(\xi)|^2 \sum_{k \in \nn} \lambda^{\bf Q}_k |q_k(\xi)|^2 
& \le L_5'+ L_6' |\xi|^2, \label{s-grow}
\end{align}  
for all $\xi \in \rr$ with some positive constants $L_1'$, $L_2'$, $L_5'$, and $L_6' \ge 0$.
\end{rk}

Throughout the paper, we assume that $q>0$ when $d=1,2$ and $q \in (0,2]$ when $d=3$, 
so that the Sobolev embeddings 
\begin{align} \label{emb}
\dot H^1\hookrightarrow L_\xi^{2(q+1)} % \hookrightarrow L_\xi^{q+2}
\hookrightarrow L_\xi^2 \hookrightarrow \dot H^{-1} \hookrightarrow L_\xi^{[2(q+1)]'},
\quad d=1,2,3, 
\end{align}
hold; see \cite{LQ21} for the reason for choosing this range of $q$.
Then we can define the Nemytskii operators $F: \dot H^1 \rightarrow \dot H^{-1}$ and $G: H \rightarrow \LL_2^0$ associated with $f$ and $g$, respectively, by
\begin{align} 
F(u)(\xi):=& f(u(\xi)),\quad u \in \dot H^1,\ \xi \in \OOO, \label{df-F}\\
G(u) q_k(\xi):=& g(u(\xi)) q_k(\xi), \quad u \in H,~ k \in \nn,~ \xi \in \OOO.
\label{df-G}
\end{align}
 



With these preliminaries, Eq. \eqref{see-fg} is equivalent to the following infinite-dimensional stochastic evolution equation:
\begin{align} \label{see}
{\rm d} X(t)=(\Delta X(t)+F(X(t))) {\rm d}t+G(X(t)) {\rm d}W, \quad  t \ge 0;
\quad X(0)=X_0 \in H.
\end{align} 
 
 
    \subsection{Tamed Schemes}
  
   
   
To construct nonlinearity-explicit schemes of Eq. \eqref{see}, as the coercive condition \eqref{coe} includes super-linear drift and diffusion coefficients, we utilize a tamed technique that tames both the nonlinear drift and diffusion coefficients.
More precisely, we consider the following nonlinearity-tamed Euler--Maruyama (TEM) scheme applied to Eq. \eqref{see}:
 \begin{align} \label{tem}
 	Z_n=Z_{n-1}+\Delta Z_n \tau
 	+ F_\tau (Z_{n-1}) \tau 
 	+ G_\tau (Z_{n-1})\delta_{n-1} W,
~ n \in \nn_+; \quad Z_0=X_0,
 \end{align}
 where $\tau \in (0, 1]$ denotes the temporal step-size, $\delta_{n-1} W:=W(t_n) -W(t_{n-1})$, $n \in \nn_+$, and the tamed drift and diffusion functions are defined, respectively, by 
 \begin{align} \label{bs-tau}
 f_\tau (\xi):=\frac{f(\xi)}{(1+\tau|\xi|^{2q})^{1/2}}, \quad 
 g_\tau(\xi):=\frac{g(\xi)}{(1+\sqrt{\tau}|\xi|^q)^{1/2}}, \quad 
\xi \in \rr^d,
\end{align} 
or 
\begin{align} \label{bs-tau+}
f_\tau (\xi):=\frac{f(\xi)}{(1+\tau|\xi|^{2q})^{1/2}}, \quad 
g_\tau(\xi):=\frac{g(\xi)}{(1+\sqrt{\tau}|\xi|^{2q})^{1/2}}, \quad 
\xi \in \rr^d.
\end{align}  
We refer to \cite{LW24} for an analogous tamed technique used in the finite-dimensional case.

The TEM scheme \eqref{tem} is linearly implicit, so it can be uniquely solved pathwise.
 Moreover, it is clear that $\{Z_n, \FFF_{t_n}\}_{n \in \nn}$ is a (time-homogeneous) Markov chain.  
 
 
  
 
\section{Geometric Ergodicity of Tamed Schemes}
 \label{sec3}
 
 This section will investigate the numerical ergodicity of the TEM scheme \eqref{tem} and its Galerkin-based full discretizations applied to \eqref{see} by exploring their Lyapunov structures and probabilistic regularity.
 
\subsection{Lyapunov Structure}

 
 The following one-step Lyapunov estimate indicates that the TEM scheme \eqref{tem} inherits the original Lyapunov function $V$ of Eq. \eqref{see} defined in \eqref{lya-spde}, even though $V$ is not a Lyapunov function of the classical EM scheme as indicated in \cite{BHJKLS19}.
 This mainly shows that the TEM scheme \eqref{tem} is unconditionally stable in an infinite time horizon, which improves the finite-time unconditionally stability result derived in \cite{HS23} for another type of tamed scheme. 
 
 The main observation is the following coercive estimate about the tamed drift and diffusion coefficients.
Here and after, we omit the integration variable when there is an integration to simplify the notations.
 
\begin{lm} 
Under the conditions \eqref{coe}-\eqref{f-grow},
for all $\beta \in (0, L_2)$, $\tau \in (0, (L_2-\beta)^2 L_4^{-4}/8]$,  and $\alpha>0$, there exist positive constants ${\widetilde L}_3$ and ${\widetilde L}_4$ depending on $L_3$ and $L_4$ such that for any $u \in \dot H^1$, 
\begin{align} \label{coe-tau}
\tau \|F_\tau(u)\|^2+ \<u, F_\tau(u) \>+\|G_\tau(u)\|_{\LL_2^0}^2 
& \le (L_1+2 L_3^2) |\OOO| -\beta \alpha^q/\sqrt{1+\alpha^{2q}} \|u\|^2.
\end{align} 
\end{lm}

\begin{proof}
By the representation \eqref{bs-tau} and the conditions \eqref{coe}-\eqref{f-grow}, we have 
\begin{align*}
& \tau \|F_\tau(u)\|^2+ \<u, F_\tau(u) \>+\|G_\tau(u)\|_{\LL_2^0}^2 \\
& =\int_\OOO  \frac{\tau |f(u)|^2}{1+\tau|u|^{2q}}
+ \frac{u f(u)}{(1+\tau|u|^{2q})^{1/2}}
+ \frac{\sum_{k \in \nn} \lambda^{\bf Q}_k |g(u) q_k |^2}{(1+\sqrt \tau |u|^q)} {\rm d}\xi \\
% & \le \int_\OOO \frac{2\tau (L_3^2+L_4^2 |u|^{2q+2})}{1+\tau|u|^{2q}}+ \frac{u f(u)+\sum_{k \in \nn} |g(u) {\bf Q}^{1/2} q_k |^2}{(1+\tau|u|^{2q})^{1/2}} {\rm d}\xi  \\ 
& \le \int_\OOO \frac{2\tau (L_3^2+L_4^2 |u|^{2q+2})}{1+\tau|u|^{2q}}
+ \frac{u f(u)+|g(u)|^2  \sum_{k \in \nn} \lambda^{\bf Q}_k |q_k |^2}{(1+\tau|u|^{2q})^{1/2}} {\rm d}\xi  \\ 
& \le \int_\OOO \frac{2\tau (L_3^2+L_4^2 |u|^{2q+2})}{1+\tau|u|^{2q}}
+ \frac{L_1 - L_2 |u|^{q+2}}{(1+\tau|u|^{2q})^{1/2}} {\rm d}\xi  \\ 
& \le (L_1+2 L_3^2) |\OOO| + \int_\OOO \frac{2 L_4^2 \tau |u|^{2q+2}}{1+\tau|u|^{2q}}
+ \frac{- L_2 |u|^{q+2}}{(1+\tau|u|^{2q})^{1/2}} {\rm d}\xi  \\ 
& \le (L_1+2 L_3^2) |\OOO| -\beta \alpha^q/\sqrt{1+\alpha^{2q}} \|u\|^2. 
\end{align*}
In the last inequality above, we used an estimate given in \cite[Lemma 2.1]{LW24}.
This completes the proof of \eqref{coe-tau}.
\end{proof}
  

 
 
 	
 \begin{tm} \label{tm-lya}
Let $X_0$ be an $H$-valued, $\FFF_0$-measurable r.v. such that $\ee \|X_0\|^2 < \infty$ and Assumption \ref{ap} hold.
 Then there exist positive constants $K_1$, $K_2$, and $\tau_{\max} \le 1$ such that for any $\tau \in (0, \tau_{\max}]$ and $n \in \nn_+$,  
 	\begin{align} \label{lya}
\ee [\|Z_n\|^2 | \FFF_{n-1}]
+2 \tau \ee[ \|\nabla Z_n\|^2|\FFF_{n-1}]
 	\le  (1- K_1 \tau) \|Z_{n-1}\|^2+ K_2 \tau.
 	\end{align}
 \end{tm}
 
 
 \begin{proof}
 Let $n \in \nn_+$.
Testing \eqref{tem} with $Z_n$ and using the elementary equality 
\begin{align} \label{ab}
\<\alpha-\beta, \alpha\>
=(|\alpha|^2-|\beta|^2)/2+|\alpha-\beta|^2/2,
\quad \alpha, \beta\in \rr,
\end{align}  
and Cauchy--Schwarz inequality, we obtain  
\begin{align*}
& \|Z_n\|^2-\|Z_{n-1}\|^2 
+  \|Z_n-Z_{n-1}\|^2 +2 \tau \|\nabla Z_n\|^2 \nonumber  \\
&=2\tau \<Z_n-Z_{n-1}, F_\tau(Z_{n-1}) \> 
+ 2\tau \<Z_{n-1}, F_\tau(Z_{n-1})\> \nonumber  \\
& \quad + 2 \<Z_n-Z_{n-1}, G_\tau (Z_{n-1}) \delta_{n-1} W\>
+ 2 \<Z_{n-1}, G_\tau (Z_{n-1}) \delta_{n-1} W\> \nonumber \\
& \le 2 \tau^2 \|F_\tau(Z_{n-1})\|^2
+ 2\tau \<Z_{n-1}, F_\tau(Z_{n-1})\> + \|Z_n-Z_{n-1}\|^2 \nonumber  \\
& \quad + 2 \|G_\tau (Z_{n-1}) \delta_{n-1} W\|^2
+ 2 \<Z_{n-1}, G_\tau (Z_{n-1}) \delta_{n-1} W\>.
\end{align*}  
This shows 
\begin{align} \label{lya-err}
& \|Z_n\|^2+2 \tau \|\nabla Z_n\|^2  \nonumber  \\
&\le \|Z_{n-1}\|^2 + 2\tau^2 \|F_\tau(Z_{n-1})\|^2
+ 2\tau \<Z_{n-1}, F_\tau(Z_{n-1}) \> \nonumber  \\
& \quad + 2 \|G_\tau (Z_{n-1}) \delta_{n-1} W\|^2
+ 2 \<Z_{n-1}, G_\tau (Z_{n-1}) \delta_{n-1} W\>.
\end{align}  
  
Note that $Z_{n-1}$ is $\FFF_{n-1}$-measurable and $\delta_{n-1} W$ is independent of $\FFF_{n-1}$. 
Then taking the conditional expectation  $\ee[\cdot |\FFF_{n-1}]$
 on both sides of the above equation and using It\^o isometry, the fact 
 \begin{align*} 
\ee[ \|G_\tau (Z_{n-1}) \delta_{n-1} W\|^2 |\FFF_{n-1}]
= \tau \|G_\tau (Z_{n-1})\|_{\LL_2^0}^2,
\end{align*}
and the estimate \eqref{coe-tau}, we derive  
 \begin{align*} 
& \ee[ \|Z_n\|^2|\FFF_{n-1}]+2 \tau \ee[ \|\nabla Z_n\|^2|\FFF_{n-1}]  \\
&\le \|Z_{n-1}\|^2  
+ 2\tau [\tau \|F_\tau(Z_{n-1})\|^2+ \<Z_{n-1}, F_\tau(Z_{n-1}) \>+\|G_\tau (Z_{n-1})\|_{\LL_2^0}^2] \\
& \le (1- 2 \beta \alpha^q/\sqrt{1+\alpha^{2q}} \tau) \|Z_{n-1}\|^2 + 
2 (L_1+2 L_3^2) |\OOO| \tau.
\end{align*}     
This completes the proof of \eqref{lya} with $K_1:=2 \beta \alpha^q/\sqrt{1+\alpha^{2q}}$ and $K_2:=2 (L_1+ 2 L_3^2) |\OOO|$ for all $\tau \in (0, \tau_{\max}]$ with $\tau_{\max}:=\min \{(L_2-\beta)^2 L_4^{-4}/8,1\}$.
\end{proof}
 

 \begin{rk} 
The estimate \eqref{lya} indicates that the Lyapunov function $V: \dot H^1 \to [0,\infty)$ of Eq. \eqref{see} defined by
\begin{equation}\label{lya-spde}
          V(x)=\|x\|^2 + 2 \tau \|\nabla x\|^2, \quad x \in \dot H^1, 
      \end{equation}
      is also a Lyapunov function of the TEM scheme \eqref{tem}:
       \begin{equation*}
          \ee [V(Z_n)\mid \FFF_{n-1}] \leq 
          (1-K_1 \tau) V(Z_{n-1}) +  K_2 \tau, \quad n \in \nn.
      \end{equation*}   
Moreover, \eqref{tem} is unconditionally stable in an infinite time horizon:
       \begin{equation*}
\ee \|Z_n\|^2 
+2 \tau \sum_{k \in \zz_n^*} \ee \|\nabla Z_k\|^2|\FFF_{k-1}]
 	\le  K_2/K_1+e^{-K_1 \tau n} \ee |X_0|^2, \quad \forall~ n \in \nn_+.
      \end{equation*} 
       \end{rk}


 \begin{rk}\label{rk-lya} 
 If the diffusion function $g$ is of linear growth, then we do not need to tame $g$ and consider the following drift-TEM scheme:
\begin{align} \label{tem-}
 	Y_n=Y_{n-1}+ \Delta Y_n
	+ f_\tau (Y_{n-1}) \tau
 	+g(Y_{n-1})\delta_{n-1} W, 
	~ n \in \nn_+; \quad Z_0=X_0.
 \end{align}  
 Similar arguments used in the proof of Theorem \ref{tm-lya} yield the following Lyapunov structure of the drift-TEM scheme \eqref{tem-}: 
 	\begin{align} \label{lya-}
\ee[\|Y_n\|^2 | \FFF_{n-1}]
&\le (1- K_1' \tau) \|Y_{n-1}\|^2 + K_2' \tau , \quad n \in \nn_+,
 	\end{align}   
for any $\tau \in (0, {\widetilde \tau}_{\max}]$ with certain ${\widetilde \tau}_{\max} \in (0, 1]$, and for some	positive constants $K_1'$ and $K_2'$, 
provided $X_0$ is $\FFF_0$-measurable such that $\ee|X_0|^2 < \infty$ and the conditions \eqref{f-grow}, \eqref{f-coe}, and \eqref{s-grow} hold with $L_6'<L_2'$.
In the additive noise case, one has $L_6'=0<L_2'$, so the previous restriction $L_6'<L_2'$ always vanishes. 
Under the conditions in Remark \ref{rk-ap-lin}, if considering the TEM scheme \eqref{tem}, one derives a similar Lyapunov estimate as \eqref{lya-} without the restriction $L_6'<L_2'$, since the primary coercive condition \eqref{coe} in Theorem \ref{tm-lya} follows from \eqref{f-coe}-\eqref{s-grow}, in combination with Young inequality, with $L_2 \in (0, L_2')$. 
 \end{rk}

 




\subsection{Application to Galerkin-based tamed full discretizations}
 

In this part, we apply the arguments and results in the previous part to tamed full discretizations by  Galerkin finite elements without any additional conditions.
%Even though our method can be generalized to spectral Galerkin approximations, 
%We focus on Galerkin finite element TEM (GTEM) schemes of Eq. \eqref{see}.

Let $h\in (0,1)$, $\TT_h$ be a regular family of partitions of $\OOO$ with maximal length $h$, and $V_h \subset V$ be the space of continuous functions on $\bar \OOO$ which are piecewise linear over $\TT_h$ and vanish on the boundary $\partial \OOO$.
Denote by $\BB(V_h)$ the Borel $\sigma$-algebra generated by $V_h$.
Let $A_h: V_h \rightarrow V_h$ and $\PP_h: V^* \rightarrow V_h$ be the discrete Laplacian and generalized orthogonal projection operators, respectively, defined by 
\begin{align*}  
\<A_h u^h, v^h\> & =-\<\nabla u^h, \nabla v^h\>,
\quad u^h, v^h\in V_h,  \\
\<\PP_h u, v^h\> & =_1\<v^h, u\>_{-1},
\quad u\in V^*,\ v^h\in V_h. 
\end{align*}

We discretize the TEM schemes \eqref{tem} and \eqref{tem-} in space with Galerkin finite element approximations, i.e., find a sequence of $\ff$-adapted $V_h$-valued Markov chains $\{Z_n^h:\ n \in \zz_N\}$ such that 
 \begin{align} \label{Gtem}
 	Z_n^h=Z_{n-1}^h+\Delta_h Z_n^h \tau
 	+ \PP_h F_\tau (Z_{n-1}^h) \tau 
 	+ \PP_h G_\tau(Z_{n-1}^h)\delta_{n-1} W,  
	\quad n \in \zz^*_N,
 \end{align} 
 with initial datum $Z^h_0=\PP_h X_0$.
We call it the Galerkin TEM (GTEM) scheme.  
It can also be  equivalently stated as  
\begin{align}\label{full}
Z_n^h=S_{h,\tau} Z_{n-1}^h+\tau S_{h,\tau} \PP_h F_\tau(Z_{n-1}^h)
+S_{h,\tau} \PP_h G_\tau(Z_{n-1}^h) \delta_{n-1} W,
\quad n \in \zz^*_N,
\end{align}
where $S_{h,\tau}:=({\rm Id}-\tau A_h)^{-1}$ is a space-time approximation of the continuous semigroup $S$ in one step.
Iterating \eqref{full} for $n-1$ times, we obtain  
\begin{align}\label{full-sum}
Z^h_n 
=S_{h,\tau}^n  X^h_0+\tau \sum_{i=0}^{n-1}  S_{h,\tau}^{n-i} \PP_h F_\tau(Z^h_i)
+\sum_{i=0}^{n-1}  S_{h,\tau}^{n-i} \PP_h G_\tau (Z^h_i) \delta_i W,
\quad n \in \zz^*_N.
\end{align}
Similarly, in the linear growth diffusion case, we have the following equivalent drift-GTEM schemes:
 \begin{align}  
Y_n^h=Y_{n-1}^h + \Delta_h Y_n^h \tau
+ \PP_h F_\tau (Y_{n-1}^h) \tau
 	+ \PP_h G(Y_{n-1}^h) \delta_{n-1} W,   \label{Gtem-} \\
Y^h_n 
=S_{h,\tau}^n  X^h_0+\tau \sum_{i=0}^{n-1}  S_{h,\tau}^{n-i} \PP_h F_\tau(Y^h_i)
+\sum_{i=0}^{n-1}  S_{h,\tau}^{n-i} \PP_h G(Y^h_i) \delta_i W, 	\label{full-sum}
 \end{align} 
for $n \in \zz^*_N$, with $Y_0^h=\PP_h X_0$.
 

Similarly to the proof of Theorem \ref{tm-lya}, and in light of Remark \ref{rk-lya}, we can establish the following Lyapunov estimates of the GTEM scheme \eqref{Gtem} and the drift-GTEM scheme \eqref{Gtem-}, respectively, where the constants $K_1$, $K_2$, $\tau_{\max}$, and $K_1'$, $K_2'$, ${\widetilde \tau}_{\max}$ are given in \eqref{lya} and \eqref{lya-}, respectively.


  	
 \begin{tm} \label{tm-lya-gtem}
Let $X_0$ be an $H$-valued, $\FFF_0$-measurable r.v. such that $\ee \|X_0\|^2 < \infty$ and Assumption \ref{ap} hold.
 Then for any $\tau \in (0, \tau_{\max}]$, $h \in (0, 1)$, and $n \in \nn_+$, the GTEM scheme \eqref{Gtem} satisfies
 	\begin{align*}
\ee [\|Z_n^h\|^2 | \FFF_{n-1}]
+2 \tau \ee[ \|\nabla Z_n^h\|^2|\FFF_{n-1}]
 	\le  (1- K_1 \tau) \|Z_{n-1}^h\|^2+ K_2 \tau.  
 	\end{align*}  
\qed
 \end{tm}
 
 
 

  \begin{tm} \label{tm-lya-gtem-}
Let $X_0$ be an $H$-valued, $\FFF_0$-measurable r.v. such that $\ee \|X_0\|^2 < \infty$ and the conditions \eqref{f-grow}, \eqref{f-coe}, and \eqref{s-grow} hold with $L_6<L_2'$.
 Then for any $\tau \in (0, \tau_{\max}]$, $h \in (0, 1)$, and $n \in \nn_+$, the drift-GTEM scheme \eqref{Gtem-} satisfies
 	\begin{align*} 
	\ee [\|Y_n^h\|^2 | \FFF_{n-1}]
+2 \tau \ee[ \|\nabla Y_n^h\|^2|\FFF_{n-1}]
 	\le  (1- K_1' \tau) \|Y_{n-1}^h\|^2+ K_2' \tau. 
 	\end{align*}
\qed
 \end{tm} 
 
 
\subsection{Geometric Ergodicity of GTEM and Drift-GTEM}
 

In this part, we focus on the probabilistic regularity and geometric ergodicity of the GTEM scheme \eqref{Gtem} and the drift-GTEM scheme \eqref{Gtem-} under the following additional nondegenerate condition.


\begin{ap} \label{ap-non} 
$f$ and $g$ are continuous and for any $x \in H$, $g(u) g(u)^\tr$ is a positive definite operator in $\LL(H)$, i.e., 
\[
\< g(u) g(u)^\tr h, h\> >0, \quad \forall~ h\in H.
\]
\end{ap}
Recall the assumption that $\bf Q$ is self-adjoint and positive definite at the end of Section \ref{sec2.1}. It is well-known that ${\bf Q}^{1/2}$ is also self-adjoint and positive definite. One can show the equivalence between the positive definiteness of $g(u)g(u)^\tr$ and $g(u){\bf Q}^{1/2} [g(u){\bf Q}^{1/2}]^\tr$ with little efforts.


As at the beginning of this section, $\{Z_n^h, \FFF_{t_n}\}_{n \in \nn}$ and $\{Y_n^h, \FFF_{t_n}\}_{n \in \nn}$ generated by the GTEM scheme \eqref{Gtem} and the drift-GTEM scheme \eqref{Gtem-}, respectively, are two (time-homogeneous) Markov chains.  
Denote by $P: \rr^d \times \BB(V_h) \to [0, 1]$ and ${\widetilde P}: \rr^d \times \BB(V_h) \to [0, 1]$ the transition kernels of the Markov chain $\{Z_n, \FFF_{t_n}\}_{n \in \nn}$ and $\{Y_n, \FFF_{t_n}\}_{n \in \nn}$, respectively.
 Then for any $n \in \nn$, $x \in \rr^d$, and $A \in \BB(V_h)$,
 \begin{align} 
 P(x,A)=\mathbb{P}(Z_{n+1}^h \in A|Z_n^h=x), \label{kernel} \\
{\widetilde P}(x,A)=\mathbb{P}(Y_{n+1}^h \in A|Y_n^h=x). \label{kernel-} 
 \end{align}
 We refer to \cite{LL24} for the definitions and properties of invariant measure and ergodicity for Markov chains.

We have the following probabilistic regular property for the transition kernels $P$ and ${\widetilde P}$ defined in \eqref{kernel} and \eqref{kernel-} associated with the GTEM scheme \eqref{Gtem} and the drift-GTEM scheme \eqref{Gtem-}, respectively.
 
 \begin{prop} \label{prop-reg}
 Let Assumption \ref{ap-non} hold.
 For any $\tau \in (0,1)$, $P$ and ${\widetilde P}$ are regular in $\BB(V_h)$, i.e., $P(x, \cdot)$ or ${\widetilde P}(x, \cdot)$ are equivalent for all $x \in V_h$. 
 Consequently, there exists at most, if it exists, one invariant measure of the GTEM scheme \eqref{Gtem} and the drift-GTEM scheme \eqref{Gtem-}, respectively, in $\BB(V_h)$.
 \end{prop}
 
 \begin{proof}
For $ x \in V_h$ and $A \in \BB(V_h)$, by the GTEM scheme \eqref{Gtem}, 
\begin{align*}
P(x,A)&=\mathbb{P} (Z_{n+1}^h \in A |Z_n^h=x)\\
&=\mathbb{P} (x+ \PP_h f_\tau(x)
+\PP_h g_\tau(x) \delta_n W) \in ({\rm Id}- \tau\Delta_h) (A) )\\
&=\mu_{x+\PP_h f_\tau(x),  [\PP_h g_\tau(x)] [\PP_h g_\tau(x)]^* \tau}(({\rm Id}- \tau\Delta_h) (A)),
 \end{align*}
 as $x+\PP_h f_\tau(x)+ \PP_h g_\tau(x) \delta_n W$ is normally distributed with mean $x+\PP_h f_\tau(x)$ and variance operator $[\PP_h g_\tau(x)] [\PP_h g_\tau(x)]^* \tau$, where $\mu_{z, C}$ denotes Gaussian measure in $V_h$ with mean $z \in V_h$ and variance operator $C \in \mathcal{L}(V_h)$, respectively. 

Under Assumption \ref{ap-non}, we have that $[\PP_h g_\tau(x)] [\PP_h g_\tau(x)]^* \tau$ is nondegenerate in $\mathcal{L}(V_h)$, as noted in \cite[Remark 4.1(ii)]{LL24}, so that the family of Gaussian measures $\{\mu_{x, [\PP_h G(x) {{\bf Q}^{1/2}}] [\PP_h G(x) {{\bf Q}^{1/2}}]^* \tau}: x\in V_h\}$ are all equivalent by applying Feldman--Hajek theorem in the finite dimensional space $V_h$.
This shows that $P$ is regular.
The regularity for ${\widetilde P}$ is analogous, and we complete the proof.
\end{proof}
 

In combination with the Lyapunov structure in Theorem \ref{tm-lya} and Remark \ref{rk-lya}, the probabilistic regular property in Proposition \ref{prop-reg}, and the proof of \cite[Corollary 3.2]{LL24}, we have the following geometric ergodicity of the GTEM scheme \eqref{Gtem} and the drift-GTEM scheme \eqref{Gtem-}, respectively, where the constants $K_1$, $K_2$, $\tau_{\max}$, and ${\widetilde \tau}_{\max}$ are given in \eqref{lya} and \eqref{lya-}, respectively.

\begin{tm}  
Let Assumptions \ref{ap} and \ref{ap-non} hold.  
The GTEM scheme \eqref{Gtem} is geometrically ergodic in $\BB(V_h)$ for any $\tau \in (0, \tau_{\max}]$.   
\qed 
\end{tm}

\begin{tm}  
Let the conditions \eqref{f-grow}, \eqref{f-coe}, \eqref{s-grow} with $L_6<L_2'$, and Assumption \ref{ap-non} hold.   
The drift-GTEM scheme \eqref{Gtem-} is geometrically ergodic in $\BB(V_h)$ for any $\tau \in (0, {\widetilde \tau}_{\max}]$.  
\qed 
\end{tm}

 
 
 

 \section{Strong Error Estimates in Multiplicative Noise Case}
 \label{sec4}
 
In this section, we carry out a quantitative error analysis between the drift-GTEM scheme  \eqref{Gtem-} and Eq. \eqref{see} under the $L^{2p}(\Omega)$-norm for $p \ge 2$.
To this end, we need the following coupled monotone and coercive conditions and polynomial growth conditions on the continuous modulus of the coefficients in Eq. \eqref{see}.

 \begin{ap} \label{ap-f}
 $f \in \CC^1(\rr; \rr)$ and there exist positive constants $K_i$, $i=1,2,3$, such that for all $\xi, \eta \in \rr$ and $\tau \in (0, 1)$,  
\begin{align} 
(f(\xi)-f(\eta)) (\xi-\eta) & \le K_1 |\xi-\eta|^2, \label{f-mon} \\ 
|f'(\xi)| & \le K_2 (1+|\xi|^q), \label{f'} \\ 
[1-\tau |\xi|^{2q}]f'(\xi)+2q \tau |\xi|^{2(q-1)} \xi f(\xi) 
& \le K_3 (1+\tau|\xi|^{2q})^{3/2}. \label{con-f'up} 
\end{align} 
\end{ap}

The one-sided Lipschitz condition \eqref{f-mon} implies that $f'$ has an upper bound $K_1$:
\begin{align}  
f'(\xi) & \le K_1, \quad \xi \in \rr, \label{f-up} 
\end{align} 
 and yields the following coercive condition with some constant $\widetilde{K}_1$ depending on $K_1$ and $f(0)$:
\begin{align}   
\xi f(\xi) & \le \widetilde{K}_1 (1+ \xi^2),  \quad \xi \in \rr. \label{f-coe-} 
\end{align} 
In the case $f \in \CC^1(\rr; \rr)$, \eqref{f'} is equivalent to 
\begin{align}  \label{f-lip}
|f(\xi)-f(\eta)|& \le \widetilde{K}_2 |\xi-\eta|(1+|\xi|^q+|\eta|^q), \quad \forall~ \xi, \eta \in \rr, 
\end{align}
for some positive constant $\widetilde{K}_2$ depending on $K_2$ and $f(0)$.
 It follows from the monotone condition \eqref{f-mon} and the coercive condition \eqref{f-coe-}, respectively, that the operator $F$ defined in \eqref{df-F} satisfies 
\begin{align*} 
_{-1}\<F(u)-F(v), u-v\>_1 & \le K_1 \|u-v\|^2,   \\
_{-1}\<F(u), u\>_1 & \le  \widetilde{K}_1(1+ \|u\|^2),  
\end{align*}
for all $u, v \in \dot H^1$.
Moreover, the polynomial growth condition \eqref{f'} or \eqref{f-lip} yields that (see \cite[(4.16)]{LQ21}) 
 \begin{align}  
\|F(u)-F(v)\|_{-1} & \le C (1+\|u\|^q_1+\|v\|^q_1 ) \|u-v\|, 
\quad u, v \in \dot H^1. \label{F-}  
\end{align}



\begin{rk}
Condition \eqref{con-f'up} includes all odd polynomials with negative leading coefficients, as in Remark \ref{rk-ap-pol} (i.e., \eqref{bs-ex} with $k \in \nn_+$ and $a_{2k+1}<0$). 
In this case, direct calculations yield that   
$[1-\tau |\xi|^{2q}]f'(\xi)+2q \tau |\xi|^{2(q-1)} \xi f(\xi)$ is a polynomial with degree $3q=6k$ and leading coefficient given by $\tau a_{2k+1} (2k-1)<0$, which has an upper bound, so that \eqref{con-f'up} holds true. 
\iffalse
 In particular, this includes the double-well potential drift:
\begin{align*}
f_\tau'(\xi)
=\frac{-\xi^2(\tau \xi^4+3)+3 \tau \xi^4+1} 
{(1+\tau \xi^4)^{3/2}}
\le 4, \quad \forall~ \xi \in \rr.
\end{align*}
\fi
\end{rk}


\begin{ap} \label{ap-g}
\begin{enumerate}
\item
The operator $G: H\rightarrow \LL_2^0$ defined in \eqref{df-G}  is Lipschitz continuous, i.e., there exists a positive constant $K_5$ such that for all $u,v\in H$,
\begin{align}   
\|G(u)-G(v)\|_{\LL_2^0}& \le K_5 \|u-v\|. \label{g-lip}
\end{align}

\item
$G$ grows linearly in $\dot H^1_x$, i.e., $G(\dot H^1_x)\subset \LL_2^1$ and there exists a positive constant $K_6$ such that for all $z \in \dot H^1$,
\begin{align} \label{g1}
\|G(z) \|_{\LL_2^1} \le K_6 (1+\|z\|_1),
\quad z \in \dot H^1.  
\end{align}

\item 
There exist positive constants $\theta$ and $K_7$ such that $G(\dot H^{1+\theta}_x)\subset \LL_2^{1+\theta}$ and 
\begin{align} \label{g2}
\|G(z) \|_{\LL_2^{1+\theta}}
\le K_7 (1+ \|z\|_{1+\theta}),
\quad z \in \dot H^{1+\theta}.
\end{align}
\end{enumerate}
\end{ap}


The Lipschitz condition \eqref{g-lip} implies that $G: H\rightarrow \LL_2^0$ is of linear growth with some constant $\widetilde{K}_5$ depending on $K_5$ and $G(0)$:
\begin{align}   
\|G(u)\|_{\LL_2^0}& \le \widetilde{K}_5(1 + \|u\|),
\quad u \in H. \label{g-grow}  
\end{align}  


\begin{rk}
Assumptions \ref{ap-f} and \ref{ap-g} are consistent with Assumptions \ref{ap} and \ref{ap-non}.
\end{rk} 

\iffalse
\begin{rk}
All the convergence results in Section \ref{sec4} hold true for those $G$ that satisfy Assumption \eqref{ap-g}(1) and grow polynomially in $\dot H^{1+\theta}$ with some $\theta \in [0, 1)$.
For simplicity, we only analyze the linear growth case for $G$.
\end{rk}
 \fi
 

 

It is known that under Assumption \ref{ap-g}(1),  Eq. \eqref{see} with $X_0 \in \dot H^1$ admits a unique mild solution in $L^p(\Omega; \CC([0, T]; \dot H^1)$ for arbitrary $p \ge 1$ that satisfy (see \cite{LQ21})
\begin{align} \label{mild}
X(t)
&=S(t) X_0+\int_0^t S(t-r) F(u) {\rm d}r
+\int_0^t S(t-r) G(u) {\rm d}W(r),
\quad t \ge 0.
\end{align}
We will need the following ultracontractive and smoothing properties of the analytic $\CC_0$-semigroup $S$ (see, e.g., \cite[Theorem 6.13 in Chapter 2]{Paz83}):
\begin{align}
\|S(t) u\|_\mu \le C t^{-\frac{\mu-\nu}2} \|u\|_\nu,   
& \quad \forall~ t > 0, ~ 0 \le \nu \le \mu \le 2,~ u \in \dot H^\nu, \label{ana} \\
\|(S(t)-{\rm Id}_H) u\| \le C t^{\frac\rho2} \|u\|_\rho, \label{ana-hol} 
& \quad \forall~ t > 0, ~ 0 \le \rho \le 2, ~ u \in \dot H^\rho,  \\
\|S(t) u\|_{L_\xi^p} \le C \|u\|_{L_\xi^p},   
& \quad \forall~ t > 0, ~ p \in [1, \infty],~ u \in L_\xi^p. \label{ana-lp}
\end{align}   
Here and what follows, we use $C$ to denote generic constants independent of various discrete parameters that may differ in each appearance. 
Similarly to \eqref{ana} (with $\mu=\nu=0$), there holds that  
\begin{align} \label{sht1}
\|S_{h,\tau}^k \PP_h u\|
\le \|u\|,
& \quad \forall~ k \in \nn_+, ~ x\in H. 
\end{align}  
 \iffalse
Riesz--Torin interpolation theorem yields (the case $p=2$ is trivial and $p=\infty$ is derived in \cite[Lemma 6.1]{Tho06})
\begin{align} \label{sht2}
\| S_{h,\tau} \PP_h u\|_{L_\xi^p}
\le C \|u\|_{L_\xi^p}, 
& \quad \forall~ p \in [2, \infty],~ u \in L_\xi^p.  
\end{align}
 \fi


Let us first recall the following regularity of the exact solution $X$ to Eq. \eqref{see}; we also need an $L_\xi^{2(3q+1)}$-estimate of $X$ to derive the estimate of $\|F(X)-F_\tau(X)\|_{L_\omega^p L_\xi^2}$ in \eqref{j23}, according to the estimate \eqref{F-Ftau+}.
Here and after we denote by $\CC_t^\alpha L_\omega^p \dot H^\beta$ the space of $\ff$-adapted stochastic processes $Z$ such that their $\|\cdot\|_{\CC_t^\alpha L_\omega^p \dot H^\beta}$-seminorm are finite:
\begin{align*}   
\|Z\|_{\CC_t^\alpha L_\omega^p \dot H^\beta}
:= \sup_{0\le s < t \le T} \frac{\|Z(t)-Z(s)\|_{L_\omega^p \dot H^\beta}} 
{|t-s|^\alpha} 
& <\infty.
\end{align*}
 


\begin{lm} \label{lm-u}
\begin{enumerate}
\item
Let $\gamma\in [0,1]$, $X_0 \in \dot H^{1+\gamma}$, and Assumptions \ref{ap-f}-\ref{ap-g}(1)+(2) hold. 
Assume that Assumption \ref{ap-g}(3) holds if $\gamma=1$.
Then for any $p \ge 2$ and $\beta\in [0,1]$, there exists a constant $C$ such that 
\begin{align}  
\|X\|_{L_\omega^p L_t^\infty \dot H^{1+\gamma}}
& \le 
\begin{cases}
C e^{CT} (1+\|X_0\|_1), \quad \gamma=0; \\
C e^{CT} (1+\|X_0\|^{q+1}_{1+\gamma}), \quad \gamma\in [0,1); \\
C e^{CT} (1+\|X_0\|^{(q+1)^2}_2), \quad \gamma=1,
\end{cases}  \label{reg-u} \\
\|X\|_{\CC_t^{\frac{1+\gamma-\beta}2\wedge \frac12} L_\omega^p \dot H^\beta}
& \le 
\begin{cases}
C e^{CT} (1+\|X_0\|^{q+1}_{1+\gamma}), \quad \gamma\in [0,1); \\
C e^{CT} (1+\|X_0\|^{(q+1) ^2}_2), \quad \gamma=1.
\end{cases}  \label{hol-u}
\end{align}  

\item
Let $X_0 \in \dot H^1 \cap L_\xi^{2(3q+1)}$, and Assumptions \ref{ap-f}-\ref{ap-g}(1)+(2) hold. 
Then for any $p \ge 2$, there exists a constant $C$ such that  
\begin{align}   \label{reg-u+}
\ee \sup_{t\in [0,T]} \|X(t)\|^p_{L_\xi^{2(3q+1)}}
& \le  C e^{CT} (1+\|X_0\|_1^p+\|X_0\|_{L_\xi^{2(3q+1)}}^{p(q+1)}).
\end{align}
\end{enumerate}  
\end{lm} 


\begin{proof}
(1) The moment's estimate \eqref{reg-u} and the H\"older estimate \eqref{hol-u} had been shown in Theorem 3.1, Proposition 3.1, Theorem 3.3 and, respectively, Corollary 3.2 of \cite{LQ21}.

(2) It was shown in \cite[Proposition 2.1]{LQ21} that 
\begin{align*}
& \Big\|\int_0^t S_{t-r} F(X(r)) {\rm d}r
+\int_0^t S_{t-r} G(X(r)) {\rm d}W(r)\|_{L_\omega^p L_t^\infty \dot H^{1+\beta}} \\
& \le C (\|F(X)\|_{L_\omega^p L_t^\infty L_\xi^2}
+ \|G(X)\|_{L_\omega^p L_t^\infty \LL_2^1}) \\
& \le C e^{CT} (1+\|X_0\|^{q+1}_1), \quad \forall~ \beta \in [0,1 ).
\end{align*}
This inequality, in combination with the embedding $\dot H^{1+\beta} \subset L^\infty$ for sufficiently large $\beta \in (1/2, 1)$ and the estimation on $J^0$ that  
\begin{align*} 
\|S_t X_0\|_{L_\omega^p L_t^\infty L_\xi^{2(3q+1)}}
\le C \|X_0\|_{L_\xi^{2(3q+1)}}, 
\end{align*} 
followed by \eqref{ana-lp} with $p=2(3q+1)$, completes the proof of \eqref{reg-u+}.  
\end{proof}



\subsection{Auxiliary Process and Moment's Estimates}


As the drift-GTEM \eqref{Gtem-} is linearly implicit, unlike the finite-dimensional case studied in \cite{LW24}, there seems to be no corresponding continuous-time interpolation such that it is an It\^o process. 
Consequently, one could not utilize the tool of the It\^o formula.
Instead, as in \cite{LQ21}, we introduce the auxiliary process 
\begin{align}\label{aux}
{\widehat Y}_n^h 
=S_{h,\tau}^n \PP_h X_0
+\tau \sum_{i=0}^{n-1} S_{h,\tau}^{n-i} \PP_h F_\tau(X(t_i))
+\sum_{i=0}^{n-1} S_{h,\tau}^{n-i} \PP_h G(X(t_i)) \delta_i W,
\end{align}
for $n \in \zz^*_N$, with ${\widehat Y}_0^h=\PP_h X_0$, where the terms $Y_i^h$ in the discrete deterministic and stochastic convolutions of \eqref{full-sum} are both replaced by $X(t_i)$.
It is clear that 
\begin{align} \label{aux+} 
{\widehat Y}_n^h
&={\widehat Y}_{n-1}^h+\tau A_h {\widehat Y}_n^h 
+\tau \PP_h F_\tau(X(t_{n-1}))
+\PP_h G(X(t_{n-1})) \delta_{n-1} W, \quad n \in \zz^*_N.
\end{align}


In the following estimates, we will frequently use the facts about the tamed functions $f_\tau$ defined by \eqref{bs-tau} and its corresponding Nemytskii operator $F_\tau$.


\begin{lm} \label{lm-ftau} 
Let the conditions \eqref{f-mon} and \eqref{f'} hold.
\begin{enumerate}
\item
There exists a constant $C$ such that 
\begin{align} 
|f_\tau'(\xi)| & \le  C \tau^{-1/2}, \quad \xi \in \rr, \label{ftau'-grow}  \\  
\tau \|F_\tau(u)\|_1^2
& \le C(1+\|u\|_1^2), \quad u \in \dot H^1, \label{tauFtau1}  \\
\|F_\tau(u)-F_\tau(v)\|_{-1} 
& \le C (1+\|u\|^q_1+\|v\|^q_1 ) \|u-v\|, 
\quad u, v \in \dot H^1, \label{Ftau-} \\ 
\|F(z)-F_\tau(z)\|_{-1} 
& \le C \tau^{1/2} (1+\|z\|_1^{2q+1}), \quad z \in \dot H^1, \label{F-Ftau} \\
\|F(z)-F_\tau(z)\|
& \le C \tau (1+\|z\|_{L_\xi^{2(3q+1)}}^{3q+1}), \quad z \in L_\xi^{2(3q+1)}. \label{F-Ftau+} 
\end{align} 
 
\item
Assume \eqref{con-f'up} holds, then $f_\tau'$ has an upper bound so that  
\begin{align}  
\<F_\tau(u), u\>_1 & \le C (1+\|u\|_1^2),
\quad u \in \dot H^1. \label{Ftau-coe1} 
\end{align} 
Assume furthermore that \eqref{f-mon} holds, then  
\begin{align}   \label{Ftau-mon}
_{-1}\<F_\tau(u)-F_\tau(v), u-v\>_1 
& \le C \|u-v\|^2, \quad u, v \in \dot H^1.
\end{align}
\end{enumerate}
\end{lm}

\begin{proof}   
 Let $\xi, \eta \in \rr$, $u, v, z \in \dot H^1$, and $w \in L_\xi^{2(3q+1)}$.  
By direct calculations, we have 
\begin{align*}
f_\tau'(\xi)=\frac{[1-\tau |\xi|^{2q}]f'(\xi)+2q \tau |\xi|^{2(q-1)} \xi f(\xi)} 
{(1+\tau|\xi|^{2q})^{3/2}}.
\end{align*}
Then the condition \eqref{f'} and Young inequality imply \eqref{ftau'-grow}: 
\begin{align*}
|f_\tau'(\xi)|
& \le \frac{K_2 (1+\tau |\xi|^{2q}) (1+|\xi|^q)} 
{(1+\tau|\xi|^{2q})^{3/2}}
+ \frac{C \tau (1+|\xi|^{3q})} 
{(1+\tau|\xi|^{2q})^{3/2}}
\le C \tau^{-1/2}.
\end{align*}
This bound, in combination with the fact $\tau |f_\tau(\xi)|^2 \le C(1+|\xi|^2)$, yields \eqref{tauFtau1}:
\begin{align*} 
\tau \|F_\tau(u)\|_1^2
&=\tau \|F_\tau(u)\|^2+\tau \|F_\tau'(u) \nabla u\|_1^2 \\
& \le C (1+\|u\|^2)+ C \|\nabla u\|_1^2 \\
& \le C(1+\|u\|_1^2). 
\end{align*} 

When \eqref{con-f'up} holds, it is clear that $f_\tau'$ has an upper bound:
\begin{align*} 
f_\tau'(\xi) & \le K_3, \quad \xi \in \rr.
\end{align*} 
Consequently, we derive \eqref{Ftau-coe1} and \eqref{Ftau-mon}, respectively:
\begin{align*} 
\<F_\tau(u), u\>_1
& =\<F_\tau(u), u\>+\<F_\tau'(u) \nabla u, \nabla u\> \\
& =\int_D \Big(\frac{u f(u)}{(1+\tau|u|^{2q})^{1/2}}+f_\tau'(u) |\nabla u|^2 \Big){\rm d}\xi   \\
 & \le \widetilde{K}_1 (1+\|u\|^2) + K_3 \|\nabla u\|^2 \\
& \le (\widetilde{K}_1 \vee K_3) (1+\|u\|_1^2), \\
_{-1}\<F_\tau(u)-F_\tau(v), u-v\>_1 
& =\int_D f_\tau'(u(\xi)+\lambda (v(\xi)-u(\xi)))) |u(\xi)-v(\xi)|^2 {\rm d}\xi \\
& \le K_3 \|u-v\|^2, 
\end{align*}
for some $\lambda  \in (0, 1)$. 

To show \eqref{Ftau-}-\eqref{F-Ftau}, we use the representation \eqref{bs-tau} to get 
\begin{align*}
f_\tau(\xi)-f_\tau(\eta)  
% &=\frac{f(\xi)}{(1+\tau|\xi|^{2q})^{1/2}}-\frac{f(\eta)}{(1+\tau|\eta|^{2q})^{1/2}}  \\
&= \frac{f(\xi) (1+\tau|\eta|^{2q})^{1/2}-f(\eta)(1+\tau|\xi|^{2q})^{1/2}}{(1+\tau|\xi|^{2q})^{1/2} (1+\tau|\eta|^{2q})^{1/2}} \nonumber  \\
% &\le \frac{|f(\xi)-f(\eta)| (1+\tau|\eta|^{2q})^{1/2}}{(1+\tau|\xi|^{2q})^{1/2} (1+\tau|\eta|^{2q})^{1/2}} \\
% & \quad + \frac{|f(\eta)| \cdot |(1+\tau|\eta|^{2q})^{1/2}-(1+\tau|\xi|^{2q})^{1/2}]|}{(1+\tau|\xi|^{2q})^{1/2} (1+\tau|\eta|^{2q})^{1/2}}   \\
&= \frac{f(\xi)-f(\eta)}{(1+\tau|\xi|^{2q})^{1/2}} 
+ \frac{f(\eta) [(1+\tau|\eta|^{2q})^{1/2}-(1+\tau|\xi|^{2q})^{1/2}]}{(1+\tau|\xi|^{2q})^{1/2} (1+\tau|\eta|^{2q})^{1/2}}  \nonumber   \\
&= \frac{f(\xi)-f(\eta)}{(1+\tau|\xi|^{2q})^{1/2}} 
+\tau f(\xi, \eta),
\end{align*}
where  
\begin{align*}
f(\xi, \eta):=\frac{f(\eta) (|\eta|^{2q}-|\xi|^{2q})} 
{(1+\tau|\xi|^{2q})^{1/2} (1+\tau|\eta|^{2q})^{1/2}[(1+\tau|\eta|^{2q})^{1/2}+(1+\tau|\xi|^{2q})^{1/2}]},
\end{align*}
$\xi, \eta \in \rr$.
As $f$ satisfies \eqref{f-grow} and \eqref{f-lip}, we have 
\begin{align*}
|f_\tau(\xi)-f_\tau(\eta)|   
& \le |f(\xi)-f(\eta)|
+ \frac{C \tau (1+ |\xi|^{3q}+ |\eta|^{3q})}{1+\tau|\xi|^{2q}+ \tau |\eta|^{2q}} \cdot |\xi-\eta| \\ 
&\le C |\xi-\eta| (1+ |\eta|^q+ |\xi|^q).
\end{align*}
By the embedding \eqref{emb}, we get \eqref{Ftau-}:
\begin{align*} 
\|F_\tau(u)-F_\tau(v)\|_{-1}
& \le C \|F_\tau(u)-F_\tau(v)\|_{L_\xi^{[2(q+1)]'}}
\le C \|u-v\| (1+\|u\|_1^q+\|u\|_1^q).
\end{align*} 

\iffalse
Then, using the condition \eqref{f-mon}, mean value theorem, and Young inequality, we get 
\begin{align*} 
(f_\tau(\xi)-f_\tau(\eta))(\xi-\eta)
% & = \frac{(f(\xi)-f(\eta))(\xi-\eta)}{(1+\tau|\xi|^{2q})^{1/2}} + \frac{\tau |f(\eta) (|\eta|^{2q}-|\xi|^{2q}) (\xi-\eta)|} {(1+\tau|\xi|^{2q})^{1/2} (1+\tau|\eta|^{2q})^{1/2}[(1+\tau|\eta|^{2q})^{1/2}+(1+\tau|\xi|^{2q})^{1/2}]} \\
& \le \frac{(f(\xi)-f(\eta))(\xi-\eta)}{(1+\tau|\xi|^{2q})^{1/2}} 
+ \tau f(\xi, \eta) (\xi-\eta)  \\
& \le K_1 |\xi-\eta|^2 + \frac12 \|\xi-\eta\|^2 + \frac12 \tau^2 |f(\xi, \eta)|^2 \\
& \le C |\xi-\eta|^2 + C \tau^2 (1+|\xi|^{2(3q+1)}+|\eta|^{2(3q+1)}),
\end{align*}
and thus
\begin{align*} 
_{-1}\<F_\tau(u)-F_\tau(v), u-v\>_1 
\le C \|u-v\|^2 + C \tau^2 (1+\|u\|_{L_\xi^{2(3q+1)}}^{2(3q+1)}+\|v\|_{L_\xi^{2(3q+1)}}^{2(3q+1)}).    
\end{align*}  
\fi

Finally, It was shown in \cite[Remark 3.2]{LW24} that for all $\xi \in \rr$, 
\begin{align*}
|f(\xi)-f_\tau(\xi)| 
& \le (\frac12 \tau|\xi|^{2q}|f(\xi)|) \wedge (\frac{\sqrt2}2 \tau^{1/2}|\xi|^q|f(\xi)|).
\end{align*} 
Then we have  
\begin{align*} 
\|F(z)-F_\tau(z)\|
 & \le C \tau \|1+|z|^{3q+1}\|_{L_\xi^2}
\le C \tau (1+\|z\|_{L_\xi^{2(3q+1)}}^{3q+1}),
\end{align*}
which shows \eqref{F-Ftau+}.
Moreover, by the embedding \eqref{emb}, we have 
\begin{align*} 
\|F(z)-F_\tau(z)\|_{-1}  
\le C \tau^{1/2} \|1+|z|^{2q+1}\|_{L_\xi^{[2(q+1)]'}}
\le C \tau^{1/2} (1+\|z\|_{L_\xi^{2(q+1)}}^{2q+1}).
\end{align*}
This shows \eqref{F-Ftau}, and we complete the proof.  
\end{proof}
 

With more effort using the idea developed in \cite[Proposition 3.1]{Liu22}, we have the following general moment's estimate.
This moment's estimate under $\dot H^1$-norm is crucial to derive enhanced moment estimates needed in the error estimations of the following two parts. 

 
  \begin{prop} \label{prop-reg-Y}
Let $p \ge 2$, $X_0$ be an $\FFF_0$-measurable r.v. such that $X_0 \in L^{2p}(\Omega; \dot H^1)$, and Assumptions \ref{ap-f} and \ref{ap-g}(1) hold.
 Then there exists a positive constant $C$ independent of $T$ such that for any $N \in \nn_+$ with $\tau=T/N \in (0, 1)$,
 	\begin{align} \label{znh-h1}
& \ee [\sup_{n \in \zz^*_N} \|Y_n^h\|_1^{2p}]
+2 \ee \Big(\sum_{n \in \zz_N^*} \|\nabla Y_n^h\|_1^2 \tau \Big)^p 
\le e^{C T} (1+ \ee \|X_0\|_1^{2p}).
 	\end{align} 
 \end{prop}
 
 
 \begin{proof} 
 Let $p \ge 2$ and $n \in \nn_+$. 
 As in the proof of Theorem \ref{tm-lya}, we replace the norm and inner product of $H$ in \eqref{lya-err} by those in $\dot H^1$, respectively, to derive   
\begin{align}  \label{yn-1}
& \|Y_n^h\|_1^2+2 \tau \|\nabla Y_n^h\|_1^2 \nonumber  \\
&\le \|Y_{n-1}^h\|_1^2 + 2\tau^2 \|F_\tau(Y_{n-1}^h)\|_1^2
+ 2\tau \<F_\tau(Y_{n-1}^h), Y_{n-1}^h\>_1  \nonumber  \\
& \quad + 2 \|g_\tau(Y_{n-1}^h) \delta_{n-1} W\|_1^2
+ M_{n-1}^h \nonumber  \\
&\le C \tau + (1+ C \tau) \|Y_{n-1}^h\|_1^2 
+ 2 \|g(Y_{n-1}^h) \delta_{n-1} W\|_1^2
+ M_{n-1}^h,
\end{align}
where $M_{n-1}^h:=2 \<Y_{n-1}^h, G(Y_{n-1}^h) \delta_{n-1} W\>_1$, and in the last inequality, we used the estimates \eqref{tauFtau1} and \eqref{Ftau-coe1}.   
It follows that 
\begin{align*}  
& \|Y_n^h\|_1^2-\|Y_{n-1}^h\|_1^2 + 2 \tau \|\nabla Y_n^h\|_1^2
\le C \tau (1+ \|Y_{n-1}^h\|_1^2)
+ 2 \|g_\tau(Y_{n-1}^h) \delta_{n-1} W\|_1^2
+ M_{n-1}^h. 
\end{align*} 
Summing up the above inequality yields that  
\begin{align}  \label{lp-}
& \|Y_n^h\|_1^2 + 2 \tau \sum_{k=1}^n \|\nabla Y_k^h \|_1^2 \nonumber \\
& \le \|X_0\|_1^2 +C \tau \sum_{k=1}^n (1+\|Y_{k-1}^h\|_1^2) 
+ 2 \sum_{k=1}^n \|g(Y_{k-1}^h) \delta_{k-1} W\|_1^2
+ \sum_{k=1}^n M_{k-1}^h.
\end{align}
Then, we use the elementary inequality 
\begin{align} \label{in-sum}
\Big(\sum_{i=0}^{N-1} a_i \Big)^\gamma 
\le N^{\gamma-1} \sum_{i=0}^{N-1} a_i^\gamma,
\quad \forall~\gamma \ge 1, ~ a_i \ge 0,
\end{align} 
to derive \begin{align*}  
& \ee\Big[C \tau \sum_{k=1}^N (1+\|Y_{k-1}^h\|_1^2) \Big]^p  
\le C \sum_{k=1}^N (1+\ee \|Y_{k-1}^h\|_1^{2p}) \tau.
\end{align*}
Similarly, using both the continuous and discrete Burkholder--Davis--Gundy inequalities, Cauchy--Schwarz and Young inequalities, and the condition \eqref{g1}, we have  
\begin{align}  \label{est-sto}
\ee\Big[2 \sum_{k=1}^N \|g(Y_{k-1}^h) \delta_{k-1} W\|_1^2 \Big]^p  
+ \ee \sup_{n \in \zz_N^*} \Big|\sum_{k=1}^n M_{k-1}^h\Big|^p
& \le C \sum_{k=1}^n (1+\ee \|Y_{k-1}^h\|_1^{2p}) \tau.
\end{align}  
Taking $L^p_\omega$-norm on both sides of \eqref{lp-}, we obtain
\begin{align*}  
& \ee \sup_{n \in \zz_N^*} \|Y_n^h\|_1^{2p}
+ 2^p \ee \Big(\sum_{k=1}^N \|\nabla Y_k^h \|_1^2 \tau\Big)^p
\le \ee \|X_0\|_1^{2p} +C \sum_{k=1}^N (1+\ee \|Y_{k-1}^h\|_1^{2p}) \tau,
\end{align*}
from which we conclude \eqref{znh-h1}. 
\end{proof}


\begin{rk}
If we consider the condition \eqref{f-coe} rather than \eqref{f-coe-}, one can get an estimate of $\ee (\sum_{k \in \zz_{n-1}^*} \|Y_k\|_{L_\xi^{q+2}}^{q+2} \tau )^p$ for any $p \ge 2$ and $n \in \nn_+$:
\begin{align*}
& \ee \sup_{n \in \zz_N^*} \|Y_n\|^{2p}
+2^p \ee \Big(\sum_{n \in \zz_N^*} \|\nabla Y_n\|^2 \tau \Big)^p
+2^p \ee \Big(\sum_{n \in \zz_N^*} \|Y_n\|_{L_\xi^{q+2}}^{q+2} \tau \Big)^p \\
& \le e^{C N \tau} (1+ \ee \|X_0\|^{2p}).
 	\end{align*}
We note that such an estimate was used in \cite{HS23} to control the $L_\xi^6$-norm of a tamed scheme applied to SACE in 1D in deriving the overall strong convergence rate. 
Our methodology directly establishes the $\dot H^1$-norm and $L_\xi^{2(3q+1)}$-norm of the proposed GTEM schemes, eliminating the dimension restriction.    
\end{rk}




We also need regularity estimates for the auxiliary process \eqref{aux}.


\begin{lm} \label{lm-reg-aux}
Let $X_0 \in \dot H^1$ and Assumptions \ref{ap-f} and \ref{ap-g}(1) hold.  
Then for all $p \ge 2$, there exists a constant $C$ such that 
\begin{align}  \label{reg-aux} 
 \sup_{n \in \zz_N} \ee \|{\widehat Y}_n^h\|_1^{2p}  
& \le C e^{CT} (1+\|X_0\|_1^{2p(q+1)}).
\end{align}   
\end{lm}

\begin{proof} 
 Let $p \ge 2$ and $n \in \nn_+$.
 Testing \eqref{aux+} with ${\widehat Y}_n^h$ under the $\dot H^1$-norm and using the elementary equality \eqref{ab}, Cauchy--Schwarz inequality, integration by parts formula, we obtain  
\begin{align*}
& \|{\widehat Y}_n^h\|_1^2-\|{\widehat Y}_{n-1}^h\|_1^2 
+  \|{\widehat Y}_n^h-{\widehat Y}_{n-1}^h\|_1^2 
+2 \tau \|\nabla {\widehat Y}_n^h\|_1^2 \nonumber  \\
&=2\tau \<{\widehat Y}_n^h, F_\tau(X(t_{n-1}))\>_1
+ 2 \<{\widehat Y}_n^h-{\widehat Y}_{n-1}^h, G(X(t_{n-1})) \delta_{n-1} W\>_1
+ {\widehat M}_{n-1}^h \\
& \le \tau \|\nabla {\widehat Y}_n^h\|_1^2 + 2\tau \|F_\tau(X(t_{n-1}))\|^2 \\
& \quad + \|{\widehat Y}_n^h-{\widehat Y}_{n-1}^h\|^2+ \|G (X(t_{n-1})) \delta_{n-1} W\|^2 + {\widehat M}_{n-1}^h,
\end{align*}   
where ${\widehat M}_{n-1}^h:=2 \<{\widehat Y}_{n-1}^h, G(X(t_{n-1})) \delta_{n-1} W\>_1$. 
It follows that 
\begin{align*}  
& \|{\widehat Y}_n^h\|_1^2-\|{\widehat Y}_{n-1}^h\|_1^2 + \tau \|\nabla {\widehat Y}_n^h\|_1^2 \\
& \le 2\tau \|F_\tau(X(t_{n-1}))\|^2 
+ \|G (X(t_{n-1})) \delta_{n-1} W\|^2 + {\widehat M}_{n-1}^h.
\end{align*}
Similar arguments used in the proof of Propostion \ref{prop-reg-Y} imply
\begin{align*}  
& \ee \sup_{n \in \zz_N^*} \|{\widehat Y}_n\|_1^{2p}
+ \tau^p \ee \Big(\sum_{k=1}^N \|\nabla {\widehat Y}_k \|_1^2 \tau\Big)^p 
\\
& \le \ee \|X_0\|_1^{2p} +C \sum_{k=1}^N (1+\ee \|{\widehat Y}_{k-1}^h\|_1^{2p}) \tau 
+C T^p (\|F_\tau(X)\|_{L_t^\infty L_\omega^{2p} L_\xi^2}^{2p} 
+ \|G (X)\|_{L_t^\infty L_\omega^p \LL_2^0}^{2p}).
\end{align*}
Then we conclude \eqref{reg-aux} through the regularity \eqref{reg-u} and the conditions \eqref{f'} and \eqref{g-grow}. 
\iffalse
 For $\beta \in (0, 2)$, by Minkovski inequality and the estimate \eqref{sht1} with $\mu=\beta$, we have   
\begin{align*}
& \| \tau \sum_{i=0}^{n-1} S_{h,\tau}^{n-i} \PP_h F_\tau(X(t_i))\|_{L_\omega^p \dot H^\beta}
\le C \tau \sum_{i=0}^{n-1} t_{n-i}^{-\beta/2}  \|F_\tau(X(t_i))\|_{L_\omega^p L_\xi^2}.
\end{align*}
It follows from the fact $|f_\tau (\xi)| \le |f(\xi)| $, $\xi \in \rr$, with $f$ satisying \eqref{f'}, the embedding \eqref{emb}, and the estimate \eqref{reg-u} that 
\begin{align} \label{reg-sf}
\| \tau \sum_{i=0}^{n-1} S_{h,\tau}^{n-i} \PP_h F_\tau(X(t_i))\|_{L_\omega^p \dot H^\beta} 
& \le C (1+ \|X\|_{L_t^\infty L_\omega^{p(q+1)} \dot H_x^1}^{q+1})  \sum_{i=0}^{n-1} t_{n-i}^{-\beta/2} \tau  \nonumber \\
& \le C T^{1-\beta/2}e^{CT} (1+ \|X_0\|_1^{q+1}).
\end{align} 

Similarly, for the discrete stochastic convolution term, we have 
\begin{align}  \label{reg-sg}
\| \tau \sum_{i=0}^{n-1} S_{h,\tau}^{n-i} \PP_h G(X(t_i)) \delta_i W\|_{L_\omega^p \dot H^\beta}
& \le C \tau \sum_{i=0}^{n-1}  t_{n-i}^{-\frac{\beta-1}2} \|G(X(t_i)) \delta_i W\|_{L_\omega^p \dot H^1} \nonumber \\
& \le C (1+ \|X\|_{L_t^\infty L_\omega^p \dot H_x^1})  \sum_{i=0}^{n-1} t_{n-i}^{-\frac{\beta-1}2} \tau  \nonumber \\
& \le C T^{\frac{3-\beta}2} e^{CT}(1+ \|X_0\|_1),
\end{align}
where we used the condition \eqref{g1} and the estimate \eqref{reg-u}.
Combining the above two estimates \eqref{reg-sf} and \eqref{reg-sg} with the embedding $\dot H^\beta \subset L_\xi^{2(3q+1)}$ for $\beta \in (d/2, 2)$ and the estimate    
\begin{align*}  
& \| S_{h,\tau}^n X_0^h\|_{L_\omega^p \dot H^1}
\le C \|X_0\|_1, 
\end{align*}
followed from the estimate \eqref{sht1} with $\mu=\nu=1$, we conclude \eqref{reg-aux}.
\fi
\end{proof}
 




\subsection{Strong Error Estimate}


In this part, we establish the convergence with finite time strong error estimate of the drift-GTEM scheme \eqref{Gtem-} towards Eq. \eqref{see}. 

Denote by $E_{h,\tau}(t)=S(t)-S_{h,\tau}^n  \PP_h$ for $t\in (t_{n-1}, t_n ]$ with $n \in \zz^*_N$.
We will use the following estimates of $E_{h,\tau}$ in the error analysis of the GTEM scheme \eqref{Gtem-}; see, e.g., \cite[Lemmas 4.1]{LQ21}.


\begin{lm} \label{lm-eht}
Let $t>0$.
\begin{enumerate}
\item
For $0\le \nu\le \mu\le 2$ and $x\in \dot H^\nu$,
\begin{align} \label{eht1}
\|E_{h,\tau} (t) x\|\le C (h^\mu+\tau^\frac\mu2) t^{-\frac{\mu-\nu}2} \|x\|_\nu.
\end{align}

\item
For $0\le \mu\le 1$ and $x\in \dot H^\mu$,
\begin{align} \label{eht2}
\Big( \int_0^t \|E_{h,\tau}(r) x\|^2 {\rm d}r \Big)^{1/2} 
\le C (h^{1+\mu}+\tau^\frac{1+\mu}2) \|x\|_\mu.
\end{align}
\end{enumerate}
\end{lm}
 

\iffalse
\begin{rk}
If we consider the $L^\infty$ case, we have that 
$S_{h,\tau}^n \PP_h$ (semi-discrete case) is a bounded linear operator in $\CC$ (see \cite[Theorem 6.7]{Tho06}) (when $S_h$ is the piecewise linear finite element spaces based on quasiuniform triangulations):  
\begin{align*}
& \| S_{h,\tau}^n \PP_h u\|_{L_\xi^\infty}
\le C \|u\|_{L_\xi^\infty}, \quad u \in L_\xi^\infty.
\end{align*}
% with an algorithmic factor (see \cite[Theorem 6.1]{Tho06}) 
% \PP_h is a bounded linear operator in $L_\xi^\infty$ (see \cite[Lemma 6.1]{Tho06})
% backward Euler FEM is a type I rational approximation of heat semigroup \cite[Lemma 7.3]{Tho06}+P.116+defines a type I′ scheme under the condition $k/h^2 \le C$.  
\end{rk}
\fi
 
Next, we show the strong error estimation between the exact solution $X$ of Eq. \eqref{see} and the auxiliary process $\{{\widehat Y}_n^h\}_{n \in \zz_N}$ defined by \eqref{aux}.
For simplicity, we assume that the initial datum $X_0$ is a deterministic element in $\dot H^1$ in the rest of the paper.
All the results in these two sections hold for $\FFF_0$-measurable $X_0$ with certain bounded high-order moments.


\begin{lm} \label{lm-aux} 
Let $\gamma\in [0,1)$, $X_0 \in \dot H^{1+\gamma}$, and Assumptions \ref{ap-f} and \ref{ap-g}(1)+(2) hold. 
Then for any $p\ge 1$, there exists a constant $C$ such that 
\begin{align} \label{err-aux}
\sup_{n \in \zz_N} \|X(t_n)-{\widehat Y}_n^h\|_{L_\omega^p L_\xi^2} 
\le C e^{CT} (1+\|X_0\|_1^{2q+1}+\|X_0\|_{1+\gamma}) (h^{1+\gamma}+\tau^{1/2} ).
\end{align}  
Assume furthermore that Assumption \ref{ap-g}(3) holds if $\gamma=1$.
Then 
\begin{align} \label{err-aux2}
\sup_{n \in \zz_N} \|X(t_n)-{\widehat Y}_n^h\|_{L_\omega^p L_\xi^2} 
\le C e^{CT} (1+\|X_0\|_2^{(q+1)^2}) (h^2+\tau^{1/2} ).
\end{align} 
\end{lm}

\begin{proof}
Let $n \in \nn_+$.
Subtracting the auxiliary process ${\widehat Y}_n^h$ defined by \eqref{aux} from the mild formulation \eqref{mild} with $t=t_n$, we get
\begin{align} \label{j}
J^n  :=\|X(t_n )-{\widehat Y}_n^h \|_{L_\omega^p L_\xi^2}
\le \sum_{i=1}^3 J^n  _i,
\end{align}
where 
\begin{align*} 
J^n  _1 & =\| E_{h,\tau}(t_n ) X_0\|_{L_\omega^p L_\xi^2}, \\
J^n  _2 & =\|\sum_{i=0}^{n-1} \int_{t_i}^{t_{i+1}} [S_{t_n-r} F(X)-S_{h,\tau}^{n-i} \PP_h F_\tau(X(t_i))] {\rm d}r \|_{L_\omega^p L_\xi^2},   \\
J^n_3 & =\|\sum_{i=0}^{n-1} \int_{t_i}^{t_{i+1}} [S_{t_n-r} G(X)-S_{h,\tau}^{n-i} \PP_h G(X(t_i))] {\rm d}W(r) \|_{L_\omega^p L_\xi^2}.
\end{align*}
In the sequel, we treat the above three terms one by one.

The estimation \eqref{eht1} with $\mu=\nu=1+\gamma$ yields that 
\begin{align}  \label{j1}
J^n _1
\le C (h^{1+\gamma}+\tau^\frac{1+\gamma}2 ) \|X_0\|_{1+\gamma},
\quad \gamma\in [0,1].
\end{align}
To deal with the second term, we decompose it into the following three terms by Minkovskii inequality:
\begin{align*} 
J^n  _2 
&\le \sum_{i=0}^{n-1} \int_{t_i}^{t_{i+1}} 
\|S_{t_n-r} [F(X)-F(X(t_i))] \|_{L_\omega^p L_\xi^2} {\rm d}r  \\
&\quad + \sum_{i=0}^{n-1} \int_{t_i}^{t_{i+1}} 
\|E_{h,\tau}(t_n-r) F(X(t_i))] \|_{L_\omega^p L_\xi^2} {\rm d}r \\
&\quad + \sum_{i=0}^{n-1} \int_{t_i}^{t_{i+1}} 
\|S_{h,\tau}^{n-i} [F(X(t_i))-F_\tau(X(t_i))]\|_{L_\omega^p L_\xi^2} {\rm d}r 
=: \sum_{i=1}^3 J^n  _{2i}.
\end{align*} 
By the uniform boundedness \eqref{ana} with $(\mu, \nu)=(1/2, 0)$ and  the dual estimation \eqref{F-}, we have  
\begin{align*} 
J^n  _{21} 
&\le C \sum_{i=0}^{n-1} \int_{t_i}^{t_{i+1}} (t_n-r)^{-1/2}
\|F(X)-F(X(t_i)) \|_{L_\omega^p \dot H^{-1}} {\rm d}r \\ 
&\le C \sum_{i=0}^{n-1} \int_{t_i}^{t_{i+1}} (t_n-r)^{-1/2}
\| (1+\|X\|^q_1+\|X(t_i)\|^q_1 ) \|X-X(t_i)\|\|_{L_\omega^p} {\rm d}r \\
& \le C \tau^{1/2} (1+\|X\|^q_{L_t^\infty L_\omega^{2pq} \dot H^1} )
\|X\|_{\CC_t^{1/2} L_\omega^{2p} L_\xi^2} 
\cdot \Big(\sum_{i=0}^{n-1} \int_{t_i}^{t_{i+1}} \sqrt{\frac{r-t_i}{t_n-r}}  {\rm d}r \Big) \\  
& \le C T \tau^{1/2} (1+\|X\|^q_{L_t^\infty L_\omega^{2pq} \dot H^1} )
\|X\|_{\CC_t^{1/2} L_\omega^{2p} L_\xi^2}.
\end{align*}  
Using \eqref{eht1} with $\mu=1+\gamma$ for $\gamma\in [0,1)$ and $\nu=0$ and the embedding \eqref{emb}, we derive  
\begin{align} \label{j22}
J^n _{22}
&\le C (h^{1+\gamma}+\tau^\frac{1+\gamma}2) 
\|F(X)\|_{L_t^\infty L_\omega^p L_\xi^2} 
\Big(\sum_{j=0}^n  \int_{t_j}^{t_{j+1}} \sigma^{-\frac{1+\gamma}2} {\rm d}\sigma\Big)  \nonumber \\
&\le C T^{\frac{1-\gamma}2} (h^{1+\gamma}+\tau^\frac{1+\gamma}2) 
(1+\|X\|_{L_t^\infty L_\omega^p \dot H^1}^{q+1}),
\quad \gamma\in [0,1).
\end{align}
% When $1+\gamma >d/2$, $\dot H^{1+\gamma}$ is an algebra. So in the double well potential case $f(u)=x-x^3$, $\|F(u)\|_{1+\gamma} \le \|u\|_{1+\gamma}+\|u\|_{1+\gamma}^3$. 
The stability property \eqref{sht1} and the estimates \eqref{F-Ftau+} and \eqref{reg-u+} imply 
\begin{align} \label{j23} 
J^n  _{23} 
& \le \sum_{i=0}^{n-1} \int_{t_i}^{t_{i+1}} 
\|F(X(t_i))-F_\tau(X(t_i))\|_{L_\omega^p L_\xi^2} {\rm d}r \nonumber  \\
&\le C \tau T (1+\|X\|_{L_t^\infty L_\omega^{p(3q+1)} L_\xi^{2(3q+1)}}^{3q+1}).
\end{align} 

Combining the above three estimations and the estimates \eqref{reg-u} and \eqref{hol-u} implies 
\begin{align}  \label{j2}
J^n _2
& \le C T^{\frac{1-\gamma}2} (h^{1+\gamma}+\tau^{1/2}) 
[(1+\|X\|^q_{L_t^\infty L_\omega^{2pq} \dot H^1} )
\|X\|_{\CC_t^{1/2} L_\omega^{2p} L_\xi^2} \nonumber \\
& \quad +(1+\|X\|_{L_t^\infty L_\omega^p \dot H^1}^{q+1})
+\|X\|_{L_t^\infty L_\omega^{p(2q+1)} \dot H^1}^{2q+1}]  \nonumber \\ 
& \le C e^{CT} (h^{1+\gamma}+\tau^{1/2} )  
(1+ \|X_0\|_1^{2q+1}),
\quad \gamma\in [0,1).
\end{align}

For the last term $J^n _3$, which can be controlled by using both the continuous and discrete Burkholder--Davis--Gundy inequalities (see, e.g., \cite[(2.22) and (2.23)]{LQ21}) as  
\begin{align*} 
(J^n _3)^2  
& \le C \sum_{i=0}^{n-1} \int_{t_i}^{t_{i+1}} 
\|S_{t_n-r} g(u)-S_{h,\tau}^{n-i} \PP_h G(X(t_i))] \|_{L_\omega^p \LL_2^0}^2 {\rm d}r  \\
& \le C \sum_{i=0}^{n-1} \int_{t_i}^{t_{i+1}} 
\|S_{h,\tau}^{m+1-i} \PP_h [g(u)-G(X(t_i))]\|_{L_\omega^p \LL_2^0}^2 {\rm d}r  \\
& \quad + C \sum_{i=0}^{n-1} \int_{t_i}^{t_{i+1}} 
\|E_{h,\tau}(t_n-r) g(u)] \|_{L_\omega^p \LL_2^0}^2 {\rm d}r
:=\sum_{i=1}^2 (J^n _{3i})^2.
\end{align*}
Using the stability property \eqref{sht1}, the conditions \eqref{g-lip} and \eqref{g1}, and \eqref{eht1} with $(\mu,\nu)=(1+\gamma,1)$ for $\gamma\in [0,1)$, we get 
\begin{align*} 
(J^n _3)^2
&\le C \|X\|_{\CC_t^{1/2} L_\omega^{2p} L_\xi^2}^2
\times \sum_{i=0}^{m-1}  \int_{t_i}^{t_{i+1}} (r-t_i) {\rm d}r \nonumber  \\ 
&\quad +C (h^{1+\gamma}+\tau^\frac{1+\gamma}2)^2 
(1+\|X\|_{L_t^\infty L_\omega^p \dot H^1})^2 
\times  \int_0^{t_n } r^{-\gamma} {\rm d}r.
\end{align*} 
It follows from the estimations \eqref{reg-u} and \eqref{hol-u} that 
\begin{align}  \label{j3} 
J^n _3
&\le C T^{1/2 \wedge (1-\gamma)} (h^{1+\gamma}+\tau^{1/2}) 
(1+\|X\|_{\CC_t^{1/2} L_\omega^{2p} L_\xi^2}
+\|X\|_{L_t^\infty L_\omega^p \dot H^1})  \nonumber  \\
& \le C e^{CT} (1+\|X_0\|_1^{q+1} ) (h^{1+\gamma}+\tau^{1/2}), 
\quad \gamma\in [0,1).
\end{align} 
Putting the estimations \eqref{j1}, \eqref{j2}, and \eqref{j3} together results in
\begin{align*} 
J^n 
\le C e^{CT} (1+\|X_0\|_1^{2q+1} )
(h^{1+\gamma}+\tau^{1/2} ), 
\quad \gamma\in [0,1).
\end{align*}
This inequality, in combination with the estimation on $J^0$ that  
\begin{align*} 
J^0=\|X_0-X_0^h\|_{L_\omega^p L_\xi^2}
\le C h^{1+\gamma} \|X_0\|_{1+\gamma},
\quad \gamma\in [0,1],
\end{align*}
completes the proof of \eqref{err-aux} with $\gamma\in [0,1)$. 

To show \eqref{err-aux2} for $\gamma=1$, we just need to give refined estimations for the $J^n _{22}$ and $J^n _{32}$ provided $X_0 \in \dot H^2$ and \eqref{g2} holds.
Applying Minkovskii inequality and using \eqref{eht1} with $\mu=2$ and $\nu=\beta\in (0,1/2)$, the embedding $\dot H^2 \hookrightarrow \dot H^\beta \cap L_\xi^\infty$, the estimate \eqref{reg-u}, and the fact  (see \cite[Lemma 3.2]{LQ21}) 
\begin{align*}
\|F(u)\|_\beta
&\le C (1+\|u\|^{q+1}_{L_\xi^\infty} +\|u\|^{q+1}_\beta ),
\quad \forall ~ u\in \dot H^\beta\cap L_\xi^\infty,
\end{align*}
we derive  
\begin{align} \label{j22+}
J^n _{22}
&\le C (h^2+\tau) 
(1+\|X\|^{q+1}_{L_t^\infty L_\omega^p \dot H^\beta}
+\|X\|^{q+1}_{L_t^\infty L_\omega^p L_\xi^\infty}) 
\times \Big(\sum_{j=0}^n  \int_{t_j}^{t_{j+1}} \sigma^{-\frac{2-\beta}2} {\rm d}\sigma\Big)  \nonumber \\
&\le C e^{CT}(1+\|X_0\|_2^{(q+1)^2}) (h^2+\tau).
\end{align}
On the other hand, \eqref{eht1} with $\mu=2$ and $\nu=1+\theta$, the condition \eqref{g2} and the estimation \eqref{reg-u} imply that 
\begin{align}  \label{j32} 
J^n _{32}
& \le C (h^2+\tau) \|G(u)\|_{L_t^\infty L_\omega^p \LL_2^{1+\theta}} 
\Big(\int_0^T \sigma^{-(1-\theta)} {\rm d}\sigma \Big)^{1/2} \nonumber \\
& \le C T^{\theta/2} (h^2+\tau) (1+\|X\|_{L_t^\infty L_\omega^p \dot H^{1+\theta}})  \nonumber \\ 
& \le C e^{CT} (1+\|X_0\|^{q+1}_2 ) (h^2+\tau ).
\end{align}
This completes the proof.
\end{proof}
 

Combining Lemma \ref{lm-aux} with a variational approach used in the proof of Theorem \ref{tm-lya}, we have the following strong convergence rate between the solution $X$ of  \eqref{see} and the drift-GTEM scheme $\{Y_n^h\}$ of \eqref{Gtem-}.



 \begin{tm}  \label{tm-err}
Let $X_0 \in \dot H^{1+\gamma}$ with $\gamma\in [0,1)$ and Assumptions \ref{ap-f} and \ref{ap-g}(1)+(2) hold. 
Then for any $p\ge 2$, there exists a constant $C$ such that  
\begin{align} \label{err} 
\sup_{n \in \zz_N} \|X(t_n) - Y_n^h \|_{L_\omega^{2p} L_\xi^2}
\le C e^{CT} (1+\|X_0\|_1^{q^2+3q+1}+\|X_0\|_{1+\gamma}^{\frac{q^2+3q+1}{2q+1}}) (h^{1+\gamma}+\tau^{1/2} ).
\end{align}
Assume furthermore that Assumption \ref{ap-g}(3) holds if $\gamma=1$.
Then
\begin{align} \label{err+} 
\sup_{n \in \zz_N} \|X(t_n) - Y_n^h\|_{L_\omega^{2p} L_\xi^2}
\le C e^{CT} (1+\|X_0\|_2^{(2q+1)(q+1)}) (h^2+\tau^{1/2} ).
\end{align}
 \end{tm}
 
 
 \begin{proof} 
Let $n \in \zz^*_N$ and denote  
$e_n^h: ={\widehat Y}_n^h -Y_n^h$. 
Then $e_n^h  \in V_h$
with vanishing initial datum $e_h^0=0$.
In terms of \eqref{aux} and \eqref{full}, it is not difficult to show that 
\begin{align*} 
{\widehat Y}_n^h 
&={\widehat Y}_{n-1}^h+\tau A_h {\widehat Y}_n^h 
+\tau \PP_h F_\tau(X(t_{n-1}))
+\PP_h G(X(t_{n-1})) \delta_{n-1} W, \\
Y_n^h  
&=Y_{n-1}^h +\tau A_h Y_n^h  
+\tau \PP_h F_\tau(Y_{n-1}^h)
+\PP_h G(Y_{n-1}^h) \delta_{n-1} W.
\end{align*}
Consequently, 
\begin{align*}
e_n^h  -e_{n-1}^h 
&=\tau A_h e_n^h  
+\tau \PP_h (F_\tau(X(t_{n-1}))-F_\tau(Y_{n-1}^h)) \nonumber \\
&\quad +\PP_h (G(X(t_{n-1}))-G(Y_{n-1}^h)) \delta_{n-1} W.
\end{align*}
Testing with $e_n^h$ on both sides of the above equation and using the elementary equality \eqref{ab}, the estimate \eqref{Ftau-mon}, and the condition \eqref{g-lip}, we obtain
\begin{align*}
& \|e_n^h  \|^2-\|e_{n-1}^h\|^2 + \|e_n^h -e_{n-1}^h\|^2
+2\tau \|\nabla e_n^h  \|^2 \nonumber \\
&= 2\tau ~ {_{-1}}\<F_\tau(X(t_{n-1}))-F_\tau({\widehat Y}_{n-1}^h), e_n^h\>_1 \\
& \quad +2\tau ~{_{-1}}\<F_\tau({\widehat Y}_{n-1}^h)-F_\tau(Y_{n-1}^h), e_n^h -e_{n-1}^h\>_1 \nonumber \\
&\quad + 2\tau ~ {_{-1}}\<F_\tau({\widehat Y}_{n-1}^h)-F_\tau(Y_{n-1}^h), e_{n-1}^h \>_1 \\
&\quad + 2\<e_n^h  -e_{n-1}^h, (G(X(t_{n-1}))-G(Y_{n-1}^h))\delta_{n-1} W\>+ R_{n-1}^h \\
& \le C\tau  \|e_{n-1}^h\|^2+ \tau \|\nabla e_n^h  \|^2
+ \tau \|F_\tau(X(t_{n-1}))-F_\tau({\widehat Y}_{n-1}^h)\|_{-1}^2 \\
& \quad +2\tau^2 \|F_\tau({\widehat Y}_{n-1}^h)-F_\tau(Y_{n-1}^h)\|^2 + \|e_n^h  -e_{n-1}^h \|^2 \\ 
& \quad + 4 \|(G(X(t_{n-1}))-G(Y_{n-1}^h))\delta_{n-1} W\|^2
+ R_{n-1}^h,
\end{align*}  
where $R_{n-1}^h:=2\<e_{n-1}^h , (G(X(t_{n-1}) )-G(Y_{n-1}^h))\delta_{n-1} W\>$. 
It follows that 
\begin{align*}  
& \|e^h_n\|^2 -(1+C \tau) \|e^h_{n-1}\|^2 + \tau \|\nabla e_n^h  \|^2 \\
& \le \tau \|F_\tau(X(t_{n-1}))-F_\tau({\widehat Y}_{n-1}^h)\|_{-1}^2
+2\tau^2 \|F_\tau({\widehat Y}_{n-1}^h)-F_\tau(Y_{n-1}^h)\|^2 \\
& \quad + 4 \|(G(X(t_{n-1}))-G(Y_{n-1}^h))\delta_{n-1} W\|^2
+ R_{n-1}^h. 
\end{align*}

Summing up the above inequality, in combination with the estimate \eqref{ftau'-grow}, yields that  
\begin{align*}  
& \|e^h_n\|^2 + \tau \sum_{k=1}^n \|\nabla e_k^h  \|^2 \\
& \le C \tau \sum_{k \in \zz_{n-1}} \|e^h_k\|^2
+ \tau \sum_{k \in \zz_{n-1}} \|F_\tau(X(t_k))-F_\tau({\widehat Y}_k^h)\|_{-1}^2 \\
& \quad + 4 \sum_{k \in \zz_{n-1}} \|(G(X(t_k))-G(Y_k^h))\delta_k W\|^2 + \sum_{k \in \zz_{n-1}} |R_k^h|. 
\end{align*}
As in \eqref{est-sto}, we apply both the continuous and discrete Burkholder--Davis--Gundy inequalities, Cauchy--Schwarz and Young inequalities, and the condition \eqref{g-lip}, to obatin 
\begin{align*}  
& \ee\Big( \sum_{k \in \zz_{N-1}} \|(G(X(t_k))-G(Y_k^h))\delta_k W\|^2\Big)^p + \ee \sup_{n \in \zz_N^*} \Big|\sum_{k \in \zz_{n-1}} R_k^h \Big|^p \\
& \le C \tau \sum_{k \in \zz_{N-1}} \ee \|e^h_k\|^{2p} 
+ C \sup_{k \in \zz_{N-1}} \ee \|X(t_k)-{\widehat Y}_k^h\|^{2p}.
\end{align*}  

Then, we use the elementary inequality \eqref{in-sum} to derive 
\begin{align*}  
\ee \sup_{n \in \zz_N^*} \|e^h_n\|^{2p}  
& \le C \tau \sum_{k \in \zz_{N-1}} \ee \|e^h_k\|^{2p} 
+ C \sup_{k \in \zz_{N-1}} \ee \|X(t_k)-{\widehat Y}_k^h\|^{2p} \\
& \quad + C \sup_{k \in \zz_{N-1}} \ee \|F_\tau(X(t_k))-F_\tau({\widehat Y}_k^h)\|_{-1}^{2p}.
\end{align*}
By discrete Gr\"onwall inequality and the estimate \eqref{Ftau-}, we obtain
\begin{align}  \label{err-xy}
& \ee \sup_{n \in \zz_N^*} \|e^h_n\|^{2p}  \\
& \le C e^{CT} \sup_{k \in \zz_{N-1}} (\ee \|X(t_k)-{\widehat Y}_k^h\|^{4p})^{1/2}
(1+ \|X\|_{L_t^\infty L_\omega^{4p q} \dot H^1}^{2pq} 
+\sup_{n \in \zz_{N-1}} \|{\widehat Y}_k^h\|_{L_\omega^{4p q} \dot H^1}^{2pq}). 
\nonumber 
\end{align} 
By Minkovski inequality, we conclude \eqref{err}-\eqref{err+} from the estimates \eqref{err-aux}-\eqref{err-aux2}, \eqref{reg-u}, \eqref{reg-aux}, and \eqref{err-xy}.
 \end{proof}
  
  
   \begin{rk}
 One can apply our method to derive the same type of strong convergence rate between the solution of Eq. \eqref{see-fg} (or equivalently, Eq. \eqref{see}) with super-linear growth diffusion function $g$ and the GTEM scheme \eqref{Gtem}, provided one can derive analogous well-posedness and moment's estimate in Lemma \ref{lm-u}; the corresponding tamed diffusion function $g_\tau$ defined in \eqref{bs-tau} would satisfy similar estimates in Lemma \ref{lm-ftau}.
 \end{rk}
 

\section{High-order Convergence Rate in Additive Noise Case}
\label{sec5}


In this section, we consider the additive noise case, i.e., the diffusion coefficient of Eq. \eqref{see} is a constant operator $G(\cdot) \equiv G$: 
\begin{align} \label{see-}
{\rm d} X(t)=(\Delta X(t)+F(X(t))) {\rm d}t+ G{\rm d}W, \quad  t \ge 0,
\end{align}
 In this case, the conditions \eqref{g-lip}-\eqref{g1} are equivalent to the assumption $G\in \LL_2^1$ which was imposed in \cite{QW19} as $\|(-A)^{1/2} {\bf Q}^{1/2} \|_{HS(H;H)}<\infty$ with $g(\cdot) \equiv 1$.
 
We consider the following version of the drift-GTEM scheme \eqref{Gtem-} applied to Eq. \eqref{see} (i.e., \eqref{Gtem-} with constant diffusion operator $G(\cdot) \equiv G$):
 \begin{align} \label{tem*}
Y_n^h=Y_{n-1}^h + \Delta_h Y_n^h \tau
+ \PP_h F_\tau (Y_{n-1}^h) \tau
 	+ \PP_h G \delta_{n-1} W,
~ n \in \nn_+; \quad Y_0^h=\PP_h X_0. 
 \end{align} 
Our main purpose is to establish a high-order convergence result under additional regularity and polynomial growth assumptions on the drift function $f$:
 \begin{align}\label{f''}
|f''(\xi)| \le (L_3' +L_4' |\xi|^{q-1}),
\quad \xi \in \rr^d,
  \end{align}
  with $q \ge 1$,  for some positive constants $L_3'$ and $L_4'$, where $f''$ denotes the second derivative of $f$.
Clearly, the condition \eqref{f''} is consistent with Assumption \ref{ap-f}. 

Now, we can state and prove the higher-order convergence for the drift-GTEM scheme \eqref{tem*} applied to Eq. \eqref{see-}, i.e., the drift-GTEM scheme \eqref{Gtem-} applied to Eq. \eqref{see} driven by additive noise.
 
 
 
 \begin{tm}  \label{tm-err-} 
Let $X_0 \in \dot H^{1+\gamma} \cap L_\xi^{2(3q+1)}$ with $\gamma\in [0,1]$, $G(\cdot) \equiv G \in \LL_2^1$, and Assumption \ref{ap-f} hold. 
Then for any $p\ge 2$, there exists a constant $C$ such that  
\begin{align} \label{err-} 
\sup_{n \in \zz_N} \|X(t_n) - Y_n^h \|_{L_\omega^{2p} L_\xi^2}
\le C e^{CT} (1+\|X_0\|_{L_\xi^{2(3q+1)}}^{2(2q+1)}
+\|X_0\|_{1+\gamma}^{2(2q+1)}) (h^{1+\gamma}+\tau^\frac{1+\gamma}2 ).
\end{align}
Assume furthermore that there exists a positive constant $\theta$ such that $G \in \LL_2^{1+\theta}$ if $\gamma=1$.
Then
\begin{align} \label{err-+} 
\sup_{n \in \zz_N} \|X(t_n) - Y_n^h\|_{L_\omega^{2p} L_\xi^2}
\le C e^{CT} (1+\|X_0\|_2^{(q+1)(q+2)}) (h^2+\tau).
\end{align}
 \end{tm}
  

\begin{proof}
Let $p \ge 2$ and $\gamma\in [0,1]$.
As \eqref{err-xy} in the proof in Theorem \ref{tm-err}, we have 
\begin{align} \label{err-xy+}  
& \ee \sup_{n \in \zz_N^*} \|X(t_n) - Y_n^h\|_{L_\omega^{2p} L_\xi^2}   \\
& \le C e^{CT} \sup_{k \in \zz_{N-1}} \|X(t_k)-{\widehat Y}_k^h\|_{L_\omega^{4p} L_\xi^2}
(1+ \|X\|_{L_t^\infty L_\omega^{4p q} \dot H^1}^q 
+\sup_{n \in \zz_{N-1}} \|{\widehat Y}_k^h\|_{L_\omega^{4p q} \dot H^1}^q). 
\nonumber 
\end{align} 
So it suffices to show the following refined estimation for any $p \ge 2$ between the exact solution $X$ and the auxiliary process ${\widehat Y}_n^h$ defined by \eqref{aux} with $G(\cdot)\equiv G$.
By means of the proof in Lemma \ref{lm-aux}, we only need to give refined estimations for $J^n  _{21}$ and $J^n_3$ under the additional condition \eqref{f''}.

To show a refined estimation of $J^n  _{21}$, note that for $r\in [t_i,t_{i+1})$,
\begin{align*} 
X(t_{i+1})
&=S_{t_{i+1}-r} X(r)
+\int_r^{t_{i+1}} S_{t_{i+1}-\sigma} F(X(\sigma)) {\rm d}\sigma
+\int_r^{t_{i+1}} S_{t_{i+1}-\sigma} G {\rm d}W(\sigma).
\end{align*}
Then, using the Taylor formula leads to the splitting of $J^n  _{21}$ into 
\begin{align*} 
J^n  _{21} 
&=\|\sum_{i=0}^{n-1} \int_{t_i}^{t_{i+1}} S_{t_n-r} 
[F(X(r))-F(X(t_{i+1}))] {\rm d}r \|_{L_\omega^p L_\xi^2} \\
&\le \Big \|\sum_{i=0}^{n-1} \int_{t_i}^{t_{i+1}} S_{t_n-r} 
DF(X(r)) (S_{t_{i+1}-r}-{\rm Id}) X(r) {\rm d}r \Big\|_{L_\omega^p L_\xi^2} \\
&\quad + \Big \|\sum_{i=0}^{n-1} \int_{t_i}^{t_{i+1}} S_{t_n-r} 
DF(X(r)) \Big(\int_r^{t_{i+1}} S_{t_{i+1}-\sigma} F(X(\sigma)) {\rm d}\sigma\Big) {\rm d}r \Big\|_{L_\omega^p L_\xi^2} \\
&\quad + \Big \|\sum_{i=0}^{n-1} \int_{t_i}^{t_{i+1}} S_{t_n-r} 
DF(X(r)) \Big(\int_r^{t_{i+1}} S_{t_{i+1}-\sigma} G {\rm d}W(\sigma) \Big)  {\rm d}r \Big\|_{L_\omega^p L_\xi^2} \\
&\quad +\Big \|\sum_{i=0}^{n-1} \int_{t_i}^{t_{i+1}} S_{t_n-r} 
R_F(X(r), X(t_{i+1})) {\rm d}r \Big\|_{L_\omega^p L_\xi^2}
=:\sum_{i=1}^4 J^n  _{21i}, 
\end{align*}
where $R_F$ denotes the remainder term
\begin{align*} 
& R_F(X(r), X(t_{i+1})) \\
&:=\int_0^1 D^2 F (X(r)+\lambda (X(t_i)-X(r)) )
 (X(t_i)-X(r), X(t_i)-X(r)) (1-\lambda) {\rm d}\lambda \\
&=\int_0^1 f'' (X(r)+\lambda (X(t_i)-X(r)) )
 (X(t_i)-X(r))^2 (1-\lambda) {\rm d}\lambda.
\end{align*}
We shall estimate $J^n  _{21i}$, $i=1,2,3,4$, successively.

Since for $\delta\in (3/2, 2)$, $\dot H^\delta\hookrightarrow \CC$, it follows by the dual argument that
\begin{align} \label{l1} 
\|x\|_{-\delta}\le C \|x\|_{L_\xi^1},
\quad x\in L_\xi^1.
\end{align}
This inequality, in conjunction with Minkovskii and Cauchy--Schwarz inequalities, the condition \eqref{f'}, the embedding \eqref{emb}, and the estimations \eqref{ana} with $(\mu, \nu)=(\delta, 0)$ and \eqref{ana-hol} with $\rho=1+\gamma$, yields that 
\begin{align} \label{j211} 
J^n  _{211} 
% &\le \sum_{i=0}^{n-1} \int_{t_i}^{t_{i+1}}  \|(-A)^\frac\delta2 S_{t_n-r} \|_{\LL(H)} \|DF(X(r)) (S_{t_{i+1}-r}-{\rm Id}) X(r)\|_{L_\omega^p \dot H^{-\delta}} {\rm d}r \nonumber \\
&\le C \sum_{i=0}^{n-1} \int_{t_i}^{t_{i+1}} (t_n-r)^{-\frac\delta2}
\|DF(X(r)) (S_{t_{i+1}-r}-{\rm Id}) X(r)\|_{L_\omega^p L_\xi^1}  {\rm d}r \nonumber \\
&\le C \sum_{i=0}^{n-1} \int_{t_i}^{t_{i+1}} (t_n-r)^{-\frac\delta2}
\|f'(X(r))\|_{L_\omega^{2p} L_\xi^2} 
\|(S_{t_{i+1}-r}-{\rm Id}) X(r)\|_{L_\omega^{2p} L_\xi^2}  {\rm d}r \nonumber \\
&\le C \tau^\frac{1+\gamma}2 \|f'(X)\|_{L_t^\infty L_\omega^{2p} L_\xi^2} 
\|X\|_{L_t^\infty L_\omega^{2p} \dot H^{1+\gamma}} 
\Big(\int_0^{t_n} r^{-\frac\delta2}  {\rm d}r \Big) \nonumber \\
&\le C \tau^\frac{1+\gamma}2 T^{1-\delta/2} (1+\|X\|_{L_t^\infty L_\omega^{2pq} \dot H^1}^q) \|X\|_{L_t^\infty L_\omega^{2pq} \dot H^{1+\gamma}},
\quad \gamma \in [0, 1].
\end{align}
A similar argument implies that 
\begin{align} \label{j212} 
J^n  _{212} 
&\le C \sum_{i=0}^{n-1} \int_{t_i}^{t_{i+1}} \int_r^{t_{i+1}} 
(t_n-r)^{-\frac\delta2}
\|f'(X(r))\|_{L_\omega^{2p} L_\xi^2} 
\|f(X(\sigma))\|_{L_\omega^{2p} L_\xi^2} {\rm d}\sigma {\rm d}r \nonumber \\
&\le C \tau T^{1-\delta/2} (1+\|X\|_{L_t^\infty L_\omega^{2pq} \dot H^1}^{q+1}).
\end{align}

To estimate the third term $J^n  _{213}$, we apply the stochastic Fubini theorem, the discrete and continuous Burkholder--Davis--Gundy inequalities, and the Cauchy--Schwarz inequality to derive 
\begin{align*}
(J^n  _{213})^2 
&=\Big\| \sum_{i=0}^{n-1} \int_{t_i}^{t_{i+1}} \int_{t_i}^{t_{i+1}} 
\chi_{[r,t_{i+1})}(\sigma) S_{t_n-r} 
DF(X(r)) S_{t_{i+1}-\sigma} G {\rm d}r {\rm d}W(\sigma) \Big\|_{L_\omega^p L_\xi^2}^2 \\ 
&\le C \sum_{i=0}^{n-1} \int_{t_i}^{t_{i+1}} 
\Big\| \int_{t_i}^{t_{i+1}} \chi_{[r,t_{i+1})}(\sigma) S_{t_n-r} 
DF(X(r)) S_{t_{i+1}-\sigma} G {\rm d}r \Big\|^2_{L_\omega^p \LL_2^0} 
{\rm d}\sigma \\
&\le C \tau \sum_{i=0}^{n-1} \int_{t_i}^{t_{i+1}} 
\int_{t_i}^{t_{i+1}} \chi_{[r,t_{i+1})}(\sigma) \|S_{t_n-r} DF(X(r)) S_{t_{i+1}-\sigma} G\|_{L_\omega^p \LL_2^0}^2 {\rm d}r  
{\rm d}\sigma   \nonumber \\
&\le C \tau \sum_{i=0}^{n-1} 
\Big(\int_{t_i}^{t_{i+1}} \|f'(X(r))\|^2_{L_\omega^p \dot H^{-1}}  {\rm d}r \Big)
\Big(\int_{t_i}^{t_{i+1}}  \|S(\sigma) G\|_{\LL_2^1}^2  {\rm d}\sigma \Big).
\end{align*}
This, in combination with the condition \eqref{f'} and the assumption $G\in \LL_2^1$, shows that 
\begin{align} \label{j213}
J^n  _{213} \le C \tau (1+\|X\|_{L_t^\infty L_\omega^{pq} \dot H^1}^q).
\end{align}
Finally, in terms of the estimations \eqref{ana} with $(\mu, \nu)=(\delta, 0)$ and \eqref{l1}, Minkovskii and H\"older inequalities, the condition \eqref{f''}, and the embeddings \eqref{emb} and $\dot H^{1/2} \hookrightarrow L_\xi^3$, we estimate the last term $J^n  _{214}$ similarly to $J^n  _{211}$ by
\begin{align*}
J^n  _{214} 
&\le C \sum_{i=0}^{n-1} \int_{t_i}^{t_{i+1}} (t_n-r)^{-\frac\delta2}
\|R_F(X, X(t_{i+1}))\|_{L_\omega^p L_\xi^1} {\rm d}r \nonumber \\
&\le C \sum_{i=0}^{n-1} \int_{t_i}^{t_{i+1}} (t_n-r)^{-\frac\delta2}
\Big(\int_0^1 \|f''(X+\lambda (X(t_i)-X) ) \|_{L_\omega^{3p} L_\xi^3} \\
& \qquad \qquad \qquad \qquad \qquad \qquad \qquad \times 
\|X(t_i)-X \|^2_{L_\omega^{3p} L_\xi^3}{\rm d}\lambda\Big) {\rm d}r \nonumber \\
&\le C \tau (1+\|X\|^{q-1}_{L_t^\infty L_\omega^{3p(q-1)} L_\xi^{3(q-1)}} )\|X\|^2_{\CC_t^{(1/2+\gamma) \wedge 1} L_\omega^{3p} \dot H^{1/2}}\Big(\int_0^{t_n} r^{-\frac\delta2} {\rm d}r \Big).
\end{align*} 
When $d=3$, $3(q-1)<2(q+1)$ (as $q \le 3$); when $d=1, 2$, $\dot H^1 \subset L_\xi^p$ for any $p \in [1, \infty)$.
Consequently,   
\begin{align} \label{j214}
J^n  _{214} 
\le C \tau^{(1/2+\gamma) \wedge 1} T^{1-\delta/2} (1+\|X\|^{q-1}_{L_t^\infty L_\omega^{3p(q-1)} \dot H^1} )\|X\|^2_{\CC_t^{(1/2+\gamma) \wedge 1} L_\omega^{3p} \dot H^{1/2}}.
\end{align} 

Collecting the above four estimations \eqref{j211}-\eqref{j214} together, 
we have 
\begin{align} \label{j21}
J^n  _{21} 
& \le C \tau^\frac{1+\gamma}2 T^{1-\delta/2} [1+\|X\|_{L_t^\infty L_\omega^{2pq} \dot H^1}^{q+1}
+ (1+\|X\|_{L_t^\infty L_\omega^{2pq} \dot H^1}^q) \|X\|_{L_t^\infty L_\omega^{2pq} \dot H^{1+\gamma}} \nonumber  \\
& \qquad \qquad \qquad \qquad + (1+\|X\|^{q-1}_{L_t^\infty L_\omega^{3p(q-1)} \dot H^1} )\|X\|^2_{\CC_t^{(1/2+\gamma) \wedge 1} L_\omega^{3p} \dot H^{1/2}}], 
\end{align}
for all $\gamma \in [0, 1]$.
  
Another term $J^n_3$ ($J^n_{31}$ vanishes in the additive noise case) can be handled by the Burkholder--Davis--Gundy inequality and \eqref{eht2} with $\mu=1$:
\begin{align}  \label{j3+} 
J^n_3 
&=\Big\|\int_0^{t_n} 
E_{h,\tau}(t_n-r) G {\rm d}W(r) \Big \|_{L_\omega^p L_\xi^2}  \nonumber  \\
& \le C \Big(\int_0^{t_n}  \|E_{h,\tau}(\sigma) G \|^2_{\LL_2^0} {\rm d}\sigma \Big)^{1/2} \nonumber  \\
& \le C T\|G \|_{\LL_2^1} (h^2+\tau).
\end{align}
Now we can conclude \eqref{err-} and \eqref{err-+}, taking into account of \eqref{err-xy+}, the error estimates \eqref{j1}, \eqref{j21}, \eqref{j22}+\eqref{j22+},  \eqref{j23}, and \eqref{j3+} for $J^n_1$, $J^n  _{21}$, $J^n_{22}$, $J^n_{23}$, and $J^n_3$, respectively, and the moment estimates \eqref{reg-u}, \eqref{hol-u}, \eqref{reg-u+}, and \eqref{reg-aux}.
\end{proof} 
 
 
 \begin{rk}
 When $d=1, 2$, as $\dot H^1 \subset L_\xi^p$ for any $p \in [1, \infty)$, the assumption that $X_0 \in L_\xi^{2(3q+1)}$ is always valid, so \eqref{err-}-\eqref{err-+} hold true provided $X_0 \in \dot H^{1+\gamma}$ with $\gamma \in [0, 1]$. On the other hand, when
 $d=3$, \eqref{err-}-\eqref{err-+} hold true provided $X_0 \in \dot H^{1+\gamma}$ with $\gamma \in [0, 1]$ as $\dot H^{1+\gamma} \subset L_\xi^{2(3q+1)}$.


 The requirement of the $L_\xi^{2(3q+1)}$-regularity of the exact solution comes from the third term $J_{23}^n $ in \eqref{j23}, which is used to control the $L^2$-norm of $|\cdot|^{3q+1}$ to derive high-order of convergence; this term can not be controlled by its $\dot H^1$-norm for SACE corresponding to $q=2$ when $d=3$. 
 
 \section*{Acknowledgements}
 
 
The first author is supported by the National Natural Science Foundation of China (NNSFC), No. 12101296, Basic and Applied Basic Research Foundation of Guangdong Province, No. 2024A1515012348, and Shenzhen Basic Research Special Project (Natural Science Foundation) Basic Research (General Project), Nos. JCYJ20220530112814033 and JCYJ20240813094919026; the second author is supported by NNSFC, Nos. 12101296, 12371409, and W2431008. 

 \end{rk}
 
 
 \bibliographystyle{plain}
  \bibliography{bib.bib}
    

\end{document}

%------------------------------------------------------------------------------
% End of mcom_sample.tex
%------------------------------------------------------------------------------