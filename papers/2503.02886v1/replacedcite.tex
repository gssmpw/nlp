\section{Background and Related Work}
\label{background_motivation}

\subsection{Historical Context}
\label{2024elec_info}
\label{2024elec_context}

The 2024 U.S. presidential election featured Kamala Harris, the Democratic nominee and former Vice President, running with Tim Walz against Donald Trump, the Republican nominee, and his running mate J.D. Vance ____. This election set new records in campaign advertising, with total costs exceeding \$10.53 billion, up from \$9 billion in 2020 ____. The Harris campaign spent approximately \$880 million on ads, while the Trump campaign spent approximately \$425 million ____. These figures underscore the increasing dominance of online ads in electoral campaigns, reinforcing the need for continued research to better understand their influence and implications. Our study seeks to contribute to this. 


As context for 2024, the prior 2020 U.S. election saw a significant percentage of Americans doubting its legitimacy ____. Despite extensive investigations disproving these claims and identifying fewer than 475 cases of voter fraud out of over 25 million votes cast in six battleground states ____, skepticism remains widespread. A July 2023 poll found that 38\% of Americans still believed Biden did not legitimately win the presidency and among registered voters who supported Trump in 2020, 75\% continued to doubt Biden’s legitimacy ____. Our study examines how these narratives and opinions are leveraged by advertisers in political campaigns and how such campaigns shape and reinforce public perceptions.

\subsection{Targeted Advertising}
\label{microtargeting}
``Microtargeting'' is defined as ``a marketing strategy that uses consumer data and demographics to identify the interests and preferences of specific individuals or small groups to send targeted advertisements that align with their interests'' ____. The rise of microtargeting in digital advertising, particularly for political campaigns, enables advertisers to specifically tailor their ads based on behavioral data, interests, and demographics ____. Political advertisers leverage these capabilities to amplify campaign messaging and influence voter turnout or preferences ____. 

Previous research has highlighted ethical and regulatory challenges associated with microtargeting ____ as well as studied its impacts. For example, one study found that microtargeting political ads can be up to 70\% more effective in swaying policy support than using a generic ad designed to appeal to the entire population____.

Our research builds on these findings by analyzing 2024 election ads across three regions: Atlanta, GA; Seattle, WA; and Los Angeles, CA. These locations provide a diverse sample to investigate how regional factors, such as political party affiliation and swing-state status, influence advertising techniques and themes.

\subsection{Regulation and Transparency for Online Political Ads}
\label{regulation_transparency}
With the increase in political ads, platforms have implemented measures to enhance regulation and transparency. Some social media companies, including Twitter, TikTok, LinkedIn, and Pinterest, have banned political ads entirely ____. In contrast, Meta (formerly Facebook, which also owns Instagram) and Google remain dominant players in the political advertising space, adopting distinct regulatory approaches.

Meta mandates an authorization process for ads about social issues, elections, or politics, requiring disclaimers and adherence to regional regulations ____. Political ads are also archived in the Facebook Ad Library which provides details about sponsors and targeting criteria ____. However, the effectiveness of this system has been criticized due to its reliance on advertisers to self-declare political ads, potentially leaving gaps for undisclosed content ____. Research has also show that Meta has ``spotty and inconsistent enforcement'' and that deceptive advertising networks were successfully able to issue ads on Facebook pages ____. In addition, researchers have raised concerns about the Ad Library’s API, citing its inadequacy in providing comprehensive data for analysis ____.\footnote{Indeed, we also initially planned to incorporate Facebook’s Ad Library data into our analysis, specifically using the Facebook/Meta Ad Library API ____, but obtaining access proved excessively cumbersome within the project timeline. Facebook’s process required waiting for a physical letter with a verification code (which took approximately two weeks), and despite submitting valid U.S. identification (driver's license and passport), our application was rejected three times. Ultimately, a notarized letter was required for verification, delaying access to the point where it was no longer feasible to include data from the Facebook/Meta Ad Library in our analysis.}

Similarly, Google requires political advertisers to complete a verification process, limits targeting options for election ads, and mandates disclosures about ad sponsors ____. However, enforcement and oversight have been criticized, with reports suggesting insufficient action against disinformation and improper ad placements ____.

In our work, we thus (1) focus on web ads, which can be collected using a web crawler based infrastucture, and (2) consider not only explicitly categorized political ads but also those with implicit political themes that technically do not fall under Facebook and Google's policies. 

\subsection{Related Research}
\label{prior_studies}
There exists extensive research into the ad ecosystem and its implications. This research spans various domains, including computer security, privacy, and political science. Within computer science, specifically within security and privacy, studies have largely focused on the privacy-invasive mechanisms enabling ad tracking and delivery ____. Our work diverges by emphasizing the content of ads and the contextual targeting strategies that influence them, rather than the mechanisms behind ad delivery and tracking.

Recent studies have also examined a range of problematic ad content ____, such as clickbait ____, monetization from misinformation ____, and malicious ads propagating malware ____. Building on this, role of ads has been studied in the context of several political events, including the 2020 U.S. elections____ and the Russian invasion of Ukraine in 2022 ____.
Emerging analyses of ads from the 2024 U.S. election cycle have also already been presented____.

Building on this foundation, our research focuses on ads displayed across the web specifically during the month leading up to and briefly following the 2024 U.S. elections. We analyze both officially declared political ads and those with implicit political themes, addressing limitations identified in previous studies ____. 
In particular, our work is modeled after the framework we established in prior work____. We utilize our website crawler, ad scraper, site seed list, and codebook, adapting these resources to examine political advertising during the 2024 U.S. elections.