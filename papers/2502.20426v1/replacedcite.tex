\section{Related works}
\label{sec:related}

The study of the persuasion capabilities of LLMs, and more broadly generative AI models, is an emerging field of research with a rapidly growing literature base that already includes surveys____. 
We can roughly divide this diverse body of research into two categories: (1) evaluation of persuasion abilities of LLMs and (2) investigation into the factors that make LLMs persuasion effective.

Into the first category will fall the works in which persuasion is evaluated solely in terms of final success____.
For example, Wongkamjan \textit{et al.}____ studied the ability of LLMs to use persuasion and deception in Diplomacy, a strategy game with a large negotiation component.
In contrast to our approach, persuasion was recognized by a change of the agent's initial intended action to the one suggested by some other player during negotiations, but there was no attempt to further characterize particular persuasion techniques used.
Other works in this category differ in how they define and measure the success of persuasion but are similar in that they leave for others the challenge of relating tokens returned by a language model to results from human psychology.

More in line with our work are studies in the second category, which try to investigate the role of different factors in successful LLM persuasion, including the use of persuasion strategies. For example, Breum~\textit{et al.}____ based their analysis on linguistic dimensions of social pragmatics (e.g., status, similarity, identity)____, Jin~\textit{et al.}____ analyzed persuasion factors described by Cialdini____, whereas others____ manually curated lists of factors.
Our work also considers factors of persuasion success in a particular social scenario (in our case, Among Us), and we base these factors on a wide selection of persuasion techniques compiled from various sources (Section~\ref{sec:methods:persuasion}).

Because of their universality and ability to understand language, LLMs are often used as game agents____, for example in Avalon____, Diplomacy____, Minecraft____, and even Red Dead Redemption 2____. In these studies, the emphasis is placed on efficient and human-competitive gameplay.
In the context of this work, the most relevant is the recent application of LLM agents in Among Us____, where the focus was on evaluating the reasoning capability of a single LLM (GPT-3.5 Turbo), configured into several distinct personas based on the high-level strategies for crewmates and impostors. The importance of deception and manipulation to effectively play the game is recognized, and in one of the experiments, conversations were tagged by a different LLM with five non-mutually exclusive categories, of which the most relevant for us are `deception', `leadership \& influence', and `suspicion {\&} defense'. Beyond these high-level categories, however, no attempt was made to investigate the particular deception and defense strategies used. In contrast, our work aims at a fine-grained analysis of persuasion techniques employed by a broad selection of state-of-the-art LLMs. To the best of our knowledge, our work is the first comprehensive attempt at examining persuasion techniques employed by LLM game agents at this level of detail.