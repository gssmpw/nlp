\section{Artifact}

\subsection{Abstract}

This artifact provides the source code for generating, simulating, and evaluating the Micro Blossom design on FPGAs, with software in Rust and hardware description in Scala (which generates Verilog).
The artifact includes the following four experiments:
\begin{enumerate}
    \item Generating synthesizable Verilog modules from arbitrary decoding graphs.
    \item Correctness verification using Verilator~\cite{Snyder_Verilator_v5014} (\S9.1).
    \item Resource estimation with the Vivado Design Suite~\cite{vivado2023} (reproducing \autoref{tab:resource-usage} in \S9.4).
    \item Decoding speed on Xilinx VMK180 board~\cite{vmk180} (reproducing \autoref{fig:eva-decoding-latency} in \S9.2 and \autoref{fig:individual-techniques} in \S9.3).
\end{enumerate}

\subsection{Artifact check-list (meta-information)}

{\small
\begin{itemize}
  \item {\bf Program: Micro Blossom~\cite{micro-blossom}}
  \item {\bf Compilation: Docker~\cite{merkel2014docker}, (optional) Vivado Design Suite~\cite{vivado2023}}
  \item {\bf Hardware: x86-64 PC, (optional) VMK180 Evaluation Board~\cite{vmk180}}
  \item {\bf How much disk space required (approximately)?: 50GB}
  \item {\bf How much time is needed to prepare workflow (approximately)?: 20 minutes}
  \item {\bf How much time is needed to complete experiments (approximately)?: 2 hours}
  \item {\bf Publicly available?: Yes~\cite{micro-blossom}}
  \item {\bf Code licenses (if publicly available)?: MIT License}
  \item {\bf Archived (provide DOI)?: \href{https://doi.org/10.5281/zenodo.14773458}{10.5281/zenodo.14773458}}
\end{itemize}
}

\subsection{Description}

\subsubsection{How to access}

The source code and original data are available at {\url{https://doi.org/10.5281/zenodo.14773458}}.
Alternatively, one can find the source code without trace data or Vivado projects at {\url{https://github.com/yuewuo/micro-blossom/releases/tag/ae}}.

\subsubsection{Hardware dependencies}

Any x86-64 PC can perform Verilog generation, correctness verification, and data plotting from trace files.
A full rerun of all Vivado projects and data traces requires at least 64 GB of memory and a VMK180 Evaluation Board~\cite{vmk180}.

\subsubsection{Software dependencies}

Only Docker is required.
Optionally, Vivado Design Suite v2023.2~\cite{vivado2023} can be installed to build FPGA projects from scratch.

\subsection{Installation}

\begin{enumerate}
    \item Download at {\url{https://doi.org/10.5281/zenodo.14773458}}.
    \item Extract (\mintinline[bgcolor=commandbkg,fontsize=\small]{shell}{tar -xvzf micro-blossom.tar.gz}) (5min).
    \item Move onto the folder (\mintinline[bgcolor=commandbkg,fontsize=\small]{shell}{cd micro-blossom}).
    \item Build the image (\mintinline[bgcolor=commandbkg,fontsize=\small]{shell}{docker build -t 'mbi' .}) (10min).
    \item Create a container (\mintinline[bgcolor=commandbkg,fontsize=\small]{shell}{docker run -itd --name 'mb'}\\\mintinline[bgcolor=commandbkg,fontsize=\small]{shell}{ -v .:/root/micro-blossom 'mbi'}).
    \item Access the container (\mintinline[bgcolor=commandbkg,fontsize=\small]{shell}{docker exec -it mb bash}). All the following experiments assume the docker bash environment.
\end{enumerate}

\subsection{Experiment 1. Verilog Generation}

\nosection{Experiment workflow}

\begin{minted}[xleftmargin=18pt,breaklines,linenos,tabsize=2,fontsize=\small,bgcolor=commandbkg]{shell}
# 1. generate some decoding graphs (2min)
cd $HOME/micro-blossom/src/cpu/blossom/
cargo run -r --bin generate_example_graphs
# 2. generate Verilog given a graph (1min)
cd $HOME/micro-blossom/
sbt "runMain microblossom.MicroBlossomBusGenerator --graph resources/graphs/example_d3.json"
# 3. open the generated Verilog module
less ./gen/MicroBlossomBus.v
\end{minted}

\nosection{Evaluation and expected results}

The experiment generates a Verilog file from a decoding graph described in a JSON file.
Users can specify other example decoding graphs from the ``resources/graphs'' folder or provide any custom decoding graph in the same format.
The expected Verilog file follows this structure:

\begin{minted}[xleftmargin=18pt,breaklines,linenos,tabsize=2,fontsize=\scriptsize,bgcolor=outputbkg,frame=lines]{verilog}
// Generator : SpinalHDL v1.9.3    git head : 029104c7...
// Component : MicroBlossomBus
`timescale 1ns/1ps

module MicroBlossomBus (
  input               s0_awvalid,
  output reg          s0_awready,
  input      [22:0]   s0_awaddr,
  ...
); ...

module Edge_1 (
  input               io_message_valid,
  input      [11:0]   io_message_instruction,
  ...
  output              io_conflict_valid
); ...

module Vertex_1 (
  input               io_message_valid,
  input      [11:0]   io_message_instruction,
  ...
); ...
\end{minted}

\subsection{Experiment 2. Correctness Verification}\label{ssec:ae-correctness}

We implement our design in both Rust and Scala.
To verify correctness, each test command runs randomized verification across various QEC configurations:
\begin{itemize}
    \item QEC codes: quantum repetition code and rotated surface code.
    \item code distances: between 3 and 19.
    \item noise models: code-capacity noise, phenomenological noise, and circuit-level noise.
    \item physical error rates: $0.1\%, 0.3\%, 1\%, 3\%, 0.1, 0.3$, $0.499$.
\end{itemize}

\nosection{Experiment workflow}

\begin{minted}[xleftmargin=18pt,breaklines,linenos,tabsize=2,bgcolor=commandbkg,fontsize=\small]{shell}
cd $HOME/micro-blossom/src/cpu/blossom/
# Rust simulator (cycle-accurate, but no AXI4)
cargo run -r test paper-section5 -r5  # 5 min
cargo run -r test paper-section6 -r5  # 8 min
cargo run -r test paper-section7 -r5  # 12 min
# Scala design > Verilog > Verilator simulator
cargo run -r test embedded-axi4 -r20  # 80 min
\end{minted}

\nosection{Evaluation and expected results}

The test command panics if an incorrect or suboptimal MWPM solution is detected.
Otherwise, it displays a progress bar for each QEC configuration, all of which should complete at 100\%.
The expected output from the Verilator simulator should look like:

\begin{minted}[xleftmargin=18pt,breaklines,linenos,tabsize=2,fontsize=\scriptsize,bgcolor=outputbkg,frame=lines]{shell}
root@...:~/micro-blossom/src/cpu/blossom# cargo run -r test embedded-axi4 -r20
    Finished release [optimized] target(s) in 0.07s
     Running `target/release/micro_blossom test embedded-axi4 -r20`
Starting Scala simulator host... this may take a while (listening on 127.0.0.1:36463)
[info] welcome to sbt 1.9.6 (Ubuntu Java 11.0.25)                                                                          ...
[success] Total time: 2 s, completed Jan 30, 2025, 5:05:55 AM
...
[Runtime] SpinalHDL v1.9.3    git head : 029104c77a54c53f1edda327a3bea333f7d65fd9
...
[Progress] Verilator compilation done in 435.736 ms
[Progress] Start MicroBlossomBus hosted simulation with seed 476234681
Simulation started
[EmbeddedAxi4] repetition 3 0.01 20 / 20 [============] 100.00 % 7.97/s 
requested quit, aborting...
[Done] Simulation done in 2578.465 ms
Scala process quit normally
Successfully remove build folder
\end{minted}

\subsection{Experiment 3. Resource Estimation}\label{ssec:ae-exp3}

\nosection{Experiment workflow}

\begin{minted}[xleftmargin=18pt,breaklines,linenos,tabsize=2,fontsize=\small,bgcolor=commandbkg]{shell}
cd $HOME/micro-blossom/artifact
python3 table_4_resource_usage.py  # 3 min
\end{minted}

\nosection{Evaluation and expected results}

The script builds the Vivado projects or reuses existing ones located at \mintinline[fontsize=\small]{shell}{$MB_VIVADO_PROJECTS}, generating table\_4.pdf in the same folder.
For ease of artifact evaluation, we include the Vivado projects.
Removing them (see \S\ref{ssec:ae-custom}) will result in a full rebuild from scratch (Vivado v2023.2~\cite{vivado2023} required).

\subsection{Experiment 4. Decoding Speed Evaluation}\label{ssec:ae-exp4}

\nosection{Experiment workflow}

\begin{minted}[xleftmargin=18pt,breaklines,linenos,tabsize=2,fontsize=\small,bgcolor=commandbkg]{shell}
cd $HOME/micro-blossom/artifact
python3 figure_8_decoding_latency.py  # 7 min
python3 figure_9a_improvement.py  # 7 min
\end{minted}

\nosection{Evaluation and expected results}

The script runs the evaluation or reuses existing trace files to generate figure\_8\_*.pdf and figure\_9a.pdf in the same folder.
For ease of artifact evaluation, we include the original trace files collected from FPGA hardware.
If these trace files are removed (see \S\ref{ssec:ae-custom}), rerunning the script will automatically execute the experiments on actual hardware (VMK180 Evaluation Board~\cite{vmk180} required).

\subsection{Experiment customization}\label{ssec:ae-custom}

Our tools automate the entire process, from generating decoding graphs to building Vivado projects and experimenting on FPGA hardware.
If any intermediate files are deleted, the script will automatically regenerate the missing components.
The following commands can be used to remove specific intermediate files:

\begin{minted}[xleftmargin=18pt,breaklines,linenos,tabsize=2,fontsize=\small,bgcolor=commandbkg]{shell}
cd $HOME/micro-blossom/artifact
# remove only plots and temporary build files
make partial-clean
# remove trace data (+7 hours to rerun)
make clean-speed-data
# remove Vivado projects (+23 hours to rerun)
make complete-clean 
\end{minted}

Note that the provided Docker image does not include Vivado Design Suite~\cite{vivado2023} and cannot be used to rerun trace data on hardware or rebuild the Vivado projects.
To set up an environment for hardware evaluation, follow the steps in the Dockerfile on a PC with Vivado installed and connected to the evaluation board~\cite{vmk180}.
Before exiting the Docker environment, perform a partial clean to avoid permission issues.

\subsection{Notes}

To set up a new machine and evaluation board, grant user access to the USB driver to allow our tool to use the Xilinx debugger (xsdb) for programming the FPGA device.

\begin{minted}[xleftmargin=18pt,breaklines,linenos,tabsize=2,fontsize=\small,bgcolor=commandbkg]{shell}
sudo adduser $USER dialout
sudo rm /tmp/digilent-adept2-*
\end{minted}

Additionally, one needs to identify the USB TTY device where the FPGA outputs data.
Our tool will then read from the log file that captures this TTY output.

\begin{minted}[xleftmargin=18pt,breaklines,linenos,tabsize=2,fontsize=\small,bgcolor=commandbkg]{shell}
cd $HOME/micro-blossom/src/fpga/utils/
touch ttymicroblossom  # create TTY log file
sudo picocom /dev/ttyUSB1 -b 115200 --imap lfcrlf -g ./ttymicroblossom  # log to file
export MICRO_BLOSSOM_HARDWARE_CONNECTED=1
\end{minted}

\subsection{Methodology}

Submission, reviewing and badging methodology:

\begin{itemize}
  \item \small\url{https://www.acm.org/publications/policies/artifact-review-and-badging-current}
  \item \url{https://cTuning.org/ae}
\end{itemize}
