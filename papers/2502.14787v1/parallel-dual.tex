\section{Accelerating Dual Phase}\label{sec:parallel-dual}

\begin{table*}[hbt]
    \centering
    {
    \caption{Dual-phase operations in Parity Blossom and its local algorithm in Micro Blossom. 
    Column 3 describes each of the dual-phase operations as explicated by~\cite{wu2023qce}.
    Column 4 (local algorithm per \pu) explains how Micro Blossom implements each dual-phase operation with a process that only uses information local to the vertex or edge, states of itself and its neighbors.}
    \label{tab:dual-phase-operations}
        % \small % 9pt
        \footnotesize
\begin{tabular}{@{}c|c|l||l@{}}
\toprule
Operation &
  \!\!Arguments\!\! &
  Description in Parity Blossom &
  Local Algorithm on Parallel \pus in Micro Blossom \\ \hline\hline
\code{set Direction} &
  $S$, $\Delta y_S$ &
  update the direction of node $S$, $\Delta y_S$ &
  each \puv $v$ updates the direction $\Delta y_S$ if $S \in N_v$ \\\hline
\code{grow} &
  length $l$ &
  grow the \emph{Covers} of nodes $S$ by $l \times \Delta y_S$ &
  \begin{tabular}{@{}l@{}} each \puv $v$ sets $r_v \coloneq \max(0, r_v + l \times \max_{S \in N_v} \Delta y_S)$,\\ $\quad$ followed by one or more ``\code{update Cover}'' operations until stable \end{tabular} \\\hline
\code{update Cover} &
  --- &
  update the boundary of the \emph{Covers} &
  \begin{tabular}{@{}l@{}} each \puv $v$ updates $T_v$ and its corresponding $N_v$ as follows: \\ $\quad$ remove any $t \in T_v$ if $(v \neq t) \land (\nexists e = (u, v): t \in T_u \land r_u = r_v + w_e)$; \\ $\quad$ then  insert $T_v \coloneq T_v \cup T_u$ for any $e = (u, v) \in E$ where $r_u = r_v + w_e$. \\ It uses states from both \puv ($T_v, N_v, r_v$) and \pue ($w_e$).
  \end{tabular} \\\hline
\code{merge Cover} &
  \!\!nodes $\{ C_i \}$\!\! &
  merge the \emph{Covers} of $\{ C_i \}$ as the \emph{Cover} of $S$ &
  each \puv $v$ replaces every $C_i$ with $S$ in $N_v$ \\\hline
\code{split Cover} &
  node $S$ &
  split $S$'s \emph{Cover} into individual \emph{Covers} of $\{ C_i \}$ &
  \begin{tabular}{@{}l@{}} each \puv $v$ removes $S$ from $N_v$; \\ $\quad$ then for each $u \in S$, if $u \in T_v$, insert $\text{Root}(u)$ into $N_v$ \end{tabular} \\\hline
\code{detect Conflict}\!\! &
  --- &
  \begin{tabular}{@{}l@{}}find a \emph{Conflict} between overlapping \emph{Covers}; \\ $\quad$ otherwise find length $l \ge 0$ to grow\end{tabular} &
  \begin{tabular}{@{}l@{}} each \pue uses \ref{theorem:conflict-detection-disklet} to detect \conf on $e$; \\ each \puv/\pue uses \ref{theorem:local-length-to-grow-disklet} \\ $\quad$ to find a maximum length $l$ to grow without violating any constraint \end{tabular} \\
\bottomrule
\end{tabular}
        \vspace{1ex}
    }
\end{table*}


In this section, we describe how to accelerate the dual phase using parallel \pus.
The key insight is that certain operations in Parity Blossom~\cite{wu2023qce} can be parallelized using the \emph{Covers} of the nodes.
We can use parallel \pus to (1) maintain the \emph{Covers} when performing a dual-phase operation, and (2) implement these operations using the local state.

We first provide background on Parity Blossom in \S\ref{sec:background-parity}, describe a new algorithm design that parallelizes dual-phase operations at the vertex level in \S\ref{ssec:vertex-level-parallel-algorithm}, and present a more resource-efficient version of the algorithm in \S\ref{ssec:resource-efficient-algorithm}.

\subsection{Background}
\label{sec:background-parity}

Parity Blossom~\cite{wu2023qce} is based on a geometric interpretation of the decoding graph in which any two \emph{points} in the graph have a non-negative distance.

The algorithm associates each defect vertex $v$ with a set of points called \emph{Circle}, $C(v,\sum_{A \in \mathcal{A}(v)} y_A)$, with $v$ being the center and a radius of $\sum_{A \in \mathcal{A}(v)} y_A$ (Appendix D~\cite{wu2023qce}).
We say two \covs are neighbors if their associated vertices are neighbors on the decoding graph.

Parity Blossom associates a node $S$ with a set of points called \cov, which is the union of the above \emph{Circle}s of defect vertices in the node, i.e., $\text{\cov}(S)=\cup_{v\in S} C(v,\sum_{A \in \mathcal{A}(v)} y_A)$.

Finally, the algorithm reduces the dual-phase operations to the manipulations of its \covs.
We have \textbf{Theorem: Tight Edge Detection (\cov)}~\cite{wu2023qce} that  $\text{\cov}(S_1) \cap \text{\cov}(S_2) \neq \varnothing$ implies there exists a tight edge between nodes $S_1$ and $S_2$.
The dual phase of Parity Blossom maintains the \emph{\cov}s of the nodes and uses them to detect situations when it can no longer adjust the dual variables without violating any dual problem constraint, called \emph{Obstacles}.
Once the dual phase detects an \emph{Obstacle}, the algorithm switches to the primal phase to resolve it.

There are two types of \emph{Obstacles}, as defined in~\cite{wu2023qce} as (2a) and (2b).
The first corresponds to the constraint (2a) when the dual variable of a node $S \in \mathcal{O}$ is already 0 but is still shrinking: $y_S = 0 \land \Delta y_S < 0$.
This type of \emph{Obstacle} occurs rarely and can be handled efficiently using a priority queue on the CPU~\cite{kolmogorov2009blossom,higgott2025sparse}.
The second type corresponds to constraint (2b) and occurs more frequently, when two nodes ($S_1$ and $S_2$) are growing toward each other ($\Delta y_{S_1} + \Delta y_{S_2} > 0$) while there is already a tight edge between them in the syndrome graph ($\text{\cov}(S_1) \cap \text{\cov}(S_2) \neq \varnothing$).
We call this type of \emph{Obstacle}  \emph{Conflict}.
A \emph{Conflict} indicates an edge connecting the two nodes in the syndrome graph has become tight. 

The first three columns of \autoref{tab:dual-phase-operations} describe all dual-phase operations in \cite{wu2023qce}.
The first five operations update the \covs while the last operation detects \confs.

An astute reader may realize that Parity Blossom is \cov\emph{-parallel}, although the implementation reported in~\cite{wu2023qce} did not exploit this.
Because \covs are only associated with defect vertices and defect vertices appear randomly on the decoding graph, manipulations of a \cov may require information about non-neighboring \covs, which would require non-local connectivity between vertices if one tries to achieve vertex-parallelism by associating a \PU with each vertex like \arch.


\subsection{Algorithm of Parallel Dual-phase Operation}\label{ssec:vertex-level-parallel-algorithm}

The key idea of \arch toward achieving vertex-level parallelism is to maintain per-vertex information that is updated locally so that manipulations of a \cov will only require information associated with the vertices and edges within the \cov.
We use vertices to store the information of \covs in a distributed manner while only storing the edge weights on the edges.
The algorithm is formally described in Column 4 of \autoref{tab:dual-phase-operations}.

\smallskip\noindent \textbf{Algorithm: Parallel Dual Operations}~~
Dual-phase operations can be implemented in parallel using information local to \pus, according to \autoref{tab:dual-phase-operations} (Column 4).

We next succinctly present the mathematical notions and theorems behind this algorithm.

\smallskip
\textbf{Definition: Residual Distance}. Given a vertex $v \in V$ and a defect vertex $u \in D$ from a decoding graph, we define the \emph{Residual Distance} as
\begin{align*}
    d_r(v, u) = \sum\nolimits_{A \in \mathcal{A}(u)} y_A - \text{Dist}(u, v)
\end{align*}

\textbf{Definition: Residue, Touches and Nodes}.
Given a vertex $v \in V$, we define the following states: \emph{Residue} $r_v$, \emph{Touches} $T_v \subseteq D$, and \emph{Nodes} $N_v \subseteq \mathcal{O}^*$.
\begin{gather*}
    r_v = \max(0, \max_{u \in D}~d_r(v, u)) \\
    T_v = \argmax_{u \in D | d_r(v, u) \ge 0} d_r(v, u) \\
    N_v = \{\ \text{Root}(u)\ |\ u \in T_v\ \}\\
\end{gather*}

In \arch, each \puv maintains the Residue, Touches and Nodes of the corresponding vertex.
In addition, it also records all the directions $\Delta y_S, \forall S \in N_v$.
The complete \pu state is shown in \autoref{tab:states}.

Together, the per-vertex states maintained by all \puvs constitute a distributed description of the \emph{Covers}.
Each \puv knows whether its vertex belongs to the \emph{Cover} of a node $S$ ($S \in N_v$), and if so, how far it is from the nearest boundary ($r_v$).
An example is shown in \autoref{fig:vertex-state}, where the vertex's state provides information on \emph{Covers} that includes it.

\begin{table}[t]
    \centering
    \caption{The \pu states of both the original algorithm \S\ref{ssec:vertex-level-parallel-algorithm} and a more resource efficient algorithm \S\ref{ssec:resource-efficient-algorithm}.}
    \label{tab:states}
    {
        % \small % 9pt
        \footnotesize
\begin{tabular}{@{}cc|cc@{}}
\toprule
\pu &
  Full State (\S\ref{ssec:vertex-level-parallel-algorithm}) &
  Compact State (\S\ref{ssec:resource-efficient-algorithm}) &
  Compact Values \\ \midrule
Vertex ($v$)&
  Touches ($T_v$) &
  \emph{unique}-Touch ($t_v$) &
  $[0, |V|)$ or $\varnothing$ \\ 
 &
  Nodes ($N_v$) &
  \emph{unique}-Node ($n_v$) &
   $[0, 2|V|)$ or $\varnothing$ \\ 
 &
  \multicolumn{2}{c}{Residue ($r_v$)} &
  $[0, \max \sum w_e]$ \\
 &
  \begin{tabular}{@{}c@{}} Directions \\ ($\Delta y_S, \forall S \in N_v$) \end{tabular} &
  Direction ($s_v = \Delta y_{n_v}$) &
   $\{$+1, -1, 0$\}$ \\ 
 &
  \multicolumn{2}{c}{Is Defect ($d_v = (v \in D)$)} &
  $\{\text{true}, \text{false}\}$ \\ 
 &
  \multicolumn{2}{c}{Is Boundary ($b_v$)} &
  $\{\text{true}, \text{false}\}$  \\
 &
  \multicolumn{2}{c}{Index ($i_v$)} &
  $[0, |V|)$  \\ \midrule
Edge ($e$) &
  \multicolumn{2}{c}{Weight ($w_e$)} &
  $[0, \max w_e]$ \\ \bottomrule
\end{tabular}
        \vspace{1ex}
    }
\end{table}

We prove the following theorems. The first implies that all \emph{Conflicts} can be detected locally on each \pue of $e \in E$ only using information local to $e$ and its incident vertices.
The second implies that the length of growth can be found locally for each \puv and \pue.

\vspace{1ex}
\theoremconflictdetectiondisklet{theorem:conflict-detection-disklet}
\vspace{1ex}

The accelerator computes 
the right side of $\Longleftrightarrow$ in the above theorem in two steps: (\textit{i}) each \pue computes $(r_{v_1} + r_{v_2} \ge w_e) \land \Delta y_{S_1} + \Delta y_{S_2} > 0$ for every $S_1 \in N_{v_1}, S_2 \in N_{v_2}$ in parallel and reports the result; (\textit{ii}) the convergecast tree aggregates results from all \pues. 
We note that $w_e$ is local to the \pue associated with $e$; $r_{v_i}$, $N_{v_i}$, and $\Delta y_{S_i}$ are local to the vPU associated with $v_i$, which is incident to $e$.

When an \pue detects a \emph{Conflict} in Step (\textit{i}), it reports $(v_1$, $v_2$, $S_1$, $S_2$, $t_1$, $t_2)$ where $t_1 \in T_{v_1}, \text{Root}(t_1) = S_1$ and $t_2 \in T_{v_2}, \text{Root}(t_2) = S_2$. $t_1$ and $t_2$ exist due to the definition of $N_v$.
The convergecast tree picks an arbitrary \emph{Conflict} (and its report) out of all reported.
The convergecast tree consists of $|E|-1$ multiplexers and incurs an $O(\log |E|)$ latency.
In this way, when there exists any \emph{Conflict}, the accelerator reports at least one of them and thus implements the dual phase~\cite{wu2023qce}.

\vspace{1ex}
\theoremlocallengthtogrowdisklet{theorem:local-length-to-grow-disklet}
 
In the accelerator, \puvs compute the left of $\bigcup$  in the above theorem while \pues compute the right, both using local information.
The \pus report their results via the convergecast tree to compute the minimum $l$, which consists of $|V|+|E|-1$ comparators and incurs an $O(\log|E|)$ latency.

Together, the above two theorems indicate that per-vertex states can replace \emph{Covers} to detect \emph{Conflicts} in the implementation of the blossom algorithm.
Concretely, they lead to the algorithm for the six dual-phase operations with local information described in \autoref{tab:dual-phase-operations}.

\subsection{A More Resource-Efficient Algorithm}\label{ssec:resource-efficient-algorithm}

Although the local algorithm in \autoref{tab:dual-phase-operations} implements all dual-phase operations with vertex and edge-level parallelism and using only local information, it requires a large per-vertex state for hardware implementation: both $T_v$ and $N_v$ require $O(|V|)$ memory.
In order to reduce resource usage for each \puv, we further simplify the algorithm so that a \puv only stores a unique \emph{Touch} and \emph{Node} for its vertex.

\textbf{Definition: unique Touch and Node}.
The \emph{unique-Touch} $t_v \in D \cup \{ \varnothing \}$ is a touch vertex $u \in T_v$ whose node has the maximum $\Delta y_{\text{Root}(u)}$, and the \emph{unique-Node} $n_v = \text{Root}(t_v) \in \mathcal{O}^* \cup \{ \varnothing \}$ is the node of $t_v$.
\begin{align*}
t_v = \begin{cases}
    v, & \text{if $v \in D$},\\
    \argmax_{u \in T_v} \Delta y_{~\text{Root}(u)}, & \text{else if $T_v \neq \varnothing$},\\
    \varnothing, & \text{otherwise}.
\end{cases}
\end{align*}

By maintaining a single touch $t_v$ and node $n_v$, we reduce the state of \puv as summarized in \autoref{tab:states} and the instruction set that operates on it in \autoref{tab:instruction-set}.

\begin{figure}[t]
    \centering
    \includegraphics[width=\linewidth,trim={0 3.5cm 0 0},clip]{figures/code/vertex-state.pdf}
    \caption{Example of per-vertex states: $r_v$, $T_v$ and $N_v$. These local vertex states allow individual vertex to know its position relative to the \emph{Covers}. A vertex $v$ is outside of any \text{Cover} if and only if $N_v = T_v = \varnothing$. When $T_v \neq \varnothing$, then $r_v \ge 0$ is the distance from the vertex to the nearest \emph{Cover} boundary.}
    \label{fig:vertex-state}
\end{figure}

\paragraph{Instruction set}
As shown in \autoref{tab:instruction-set}, we design a compact instruction set that represents each dual-phase operation in \autoref{tab:dual-phase-operations}.
Another inefficiency in \S\ref{ssec:vertex-level-parallel-algorithm} is that the ``\code{split Cover}'' operation requires each \puv to maintain the hierarchical structure of the blossom.
We mitigate this problem by storing the blossom structure in the CPU and using a single \code{setCover}($C$, $S$) instruction to implement both ``\code{merge Cover}'' and ``\code{split Cover}'' operations.
Each \puv simply set $n_v \coloneq S$ if $\{t_v\} = C$ or $n_v = C$ upon receiving ``\code{set Cover}''.


\paragraph{Compact PUs}
We implement this resource-efficient algorithm using a compact state per vertex and edge as shown in \autoref{tab:states} and combinational logic with a fixed CPI (clock per instruction) of 1.
This combinational logic takes the \emph{current} state from the registers of its neighboring vertices and edges.
Then it outputs the \emph{next} state that is captured by the registers and fed as input in the next clock cycle.
All \pus share the same clock and run synchronously.
In surface code, the number of registers or gates scales with $O(|V|\ \text{polylog}|V|)$, as detailed in \S\ref{ssec:eva-resource}.

\begin{table}[t]
    \caption{Instruction set of the dual-phase accelerator. The indices of the blossoms are encoded in $\qty{15}{bits}$, which supports $2^{14} = 16384$ vertices ($d \le 31$). The instruction word can be further extended.}
    \label{tab:instruction-set}
    \centering
    {
        % \small % 9pt
        \footnotesize
\begin{tabular}{@{}ccc@{}}
\toprule
Instruction &
  $\qns$Arguments$\qns$ &
  Instruction Word (32 bits)
  \\ \midrule
\code{reset} &
  --- &
  \verb#|                        |1001|00|# \\
\code{set Direction} &
  $S, \Delta y_S$ &
  \verb#|   S [31:17]  |dir [16:15]  0|00|# \\
\code{grow} &
  $l$ &
  \verb#|         l [31:6]       |1101|00|# \\
\code{set Cover} &
  $C, S$ &
  \verb#|   C [31:17]  |   S [16:2]   |01|# \\
\code{find Conflict} &
  --- &
  \verb#|                        |0001|00|# \\
\code{load Defects} &
   custom &
   \verb#|      custom[31:6]      |0111|00|# \\
\bottomrule
\end{tabular}
        \vspace{1ex}
    }
\end{table}

