\section{Background}\label{sec:background}

\paragraph{Quantum Error Correction (QEC) Codes}
A QEC code encodes a logical qubit using multiple physical qubits.
The surface code is one of the most promising QEC codes, featuring a universal fault-tolerant gate set that enables fault-tolerant quantum computation~\cite{fowler2012surface}.
As shown in \autoref{fig:bkgd-surface-code}, a distance-$d$ surface code consists of $d^2$ data qubits and $d^2 - 1$ stabilizers.

\paragraph{Decoding Graph}

As shown in \autoref{fig:bkgd-code-capacity}, a decoding graph is a weighted graph $G = (V, E, W)$ derived from a code and noise model.
Each vertex corresponds to a stabilizer measurement.
If the measurement result is flipped, we call the corresponding vertex a \textit{defect vertex}, marked in red; otherwise, it is a \textit{regular vertex} marked in white.
$D \subseteq V$ represents the set of defect vertices.
Each edge corresponds to a potential quantum error, with weight $w_e$ calculated from the error probability $p_e$ by $w_e = \log ((1 - p_e) / p_e)$.
An edge connects the vertices that can detect the corresponding error.
\textit{Virtual vertices} (in yellow) along the code boundary represents the unknown measurements~\cite{wu2023qce}.

Since stabilizer measurements are noisy, we need $\Theta(d)$ measurement rounds to achieve fault tolerance~\cite{gottesman2013fault}.
The resulting decoding graph is a three-dimensional (3D) graph that contains $|V| = \Theta(d^3)$ vertices, as shown in \autoref{fig:bkgd-circuit-level}.
Each horizontal layer, which is two-dimensional (2D), corresponds to a measurement round.
The edges between layers represent the measurement errors.

\paragraph{Syndrome graph and MWPM Decoder}
The set of defect vertices $D$ is known as the \emph{syndrome}.
Given the decoding graph and a syndrome, one can derive the \emph{syndrome graph}, which is a fully connected graph with the vertices being the defect vertices from the decoding graph.
The weight $w_e$ of an edge $e$ is computed as the minimum weight of paths in the decoding graph between the two incident defect vertices.
The Minimum-Weight Perfect Matching (MWPM) decoder finds the most likely error by finding an MWPM in the syndrome graph.
We denote the edges of the syndrome graph as $E'$.

\paragraph{Blossom Algorithm}
The blossom algorithm leverages Linear Programming (LP) to solve the minimum-weight perfect matching (MWPM) problem~\cite{edmonds1969blossom}.
It considers the original MWPM problem as the primal problem: minimizing the total weight $\sum_{e\in E'} w_e x_e$ where the primal variable $x_e$ is either 1 or 0, indicating whether edge $e$ is in the matching.
Accordingly, the dual problem maximizes the sum of the dual variables $\sum_{S\in \mathcal{O}^*} y_S$ where $\mathcal{O}^* = \{ S | S\subseteq D, |S|\ \text{is odd}\}$~\cite{wu2023qce}.
In this dual problem, there is a constraint corresponding to each primal variable (and therefore each edge). When the equality for this constraint holds,  the corresponding edge is considered \emph{tight}.

The blossom algorithm works on both the primal and dual problems. The dual phase, solving the dual problem, changes $y_S$ and creates tight edges; the primal phase, solving the primal problem, changes $x_e$ and determines which tight edges will be in the matching.
Intuitively, the blossom algorithm gradually increases the ``available budget'' ($\sum y_S$) while checking whether a perfect matching solution with a ``cost'' ($\sum w_e x_e$) exists within the ``budget''.
The dual phase increases the ``budget'' while ensuring it does not exceed the minimum required ``cost'' (by enforcing dual constraints).
The primal phase seeks a perfect matching within the allocated ``budget'', and negotiates with the dual phase to increase it as needed.
The optimality is achieved when the ``cost'' matches the ``budget''.

Three notions from the blossom algorithm are important in this work: blossom, node, and alternating tree.
In the context of QEC~\cite{wu2023qce}, we can inductively define a \emph{blossom} on the syndrome graph as follows, adapted from~\cite{wu2023qce} (Page 4): 
(\textit{i}) a defect vertex is a blossom; (\textit{ii})  an odd number of blossoms connected by tight edges in a circle form a blossom.
Apparently, each blossom $S \in \mathcal{O}^*$ is associated with a dual variable $y_S$.
A \emph{node} is a blossom that does not belong to any other blossom according to (\textit{ii}). 
Each defect vertex $v \in D$ belongs to a unique node, denoted $\text{Root}(v)$.

The primal phase organizes nodes (and tight edges) into matched pairs and alternating trees. A matched pair consists of two nodes connected by a tight edge $e$ with $x_e=1$.
An \emph{alternating tree} consists of an odd number of nodes that are connected by tight edges in a tree.
In the tree, there are an odd number of tight edges connecting any leaf node to the root.
It is called an alternating tree because alternating edges along the path from a leaf to the root are included in the matching with $x_e=1$.
A single node not matched is also an alternating tree.
An example of a matched pair and an alternating tree is shown in \autoref{fig:example-alternating-tree}.

With alternating trees and matched pairs, the primal phase determines how to adjust a dual variable $y_S$ in the dual phase by assigning a \emph{direction} $\Delta y_S \in \{ 0, +1, -1 \}$.
When $S$ is matched, it sets $\Delta y_S$ as $0$; 
otherwise, it determines $\Delta y_S$ by $S$'s position in the alternating tree which alternates between $\pm 1$ from the root ($+1$) to the leaf (also $+1$) as shown in \autoref{fig:example-alternating-tree}.
The dual phase adjusts $y_S$ based on $\Delta y_S$. When $\Delta y_S=+1$, we say it \emph{grows} $y_S$ (and the node $S$); when $\Delta y_S=-1$, we say it \emph{shrinks} $y_S$ (and the node $S$).
When $\Delta y_{S_1}+\Delta y_{S_2}>0$, we say nodes $S_1$ and $S_2$ are growing toward each other. 

\begin{figure}[!t]
    \centering
    \includegraphics[width=\linewidth]{figures/slides/example-alternating-tree.pdf}
    \caption{A node is either matched or in an alternating tree. The primal phase maintains the tight edges of both the solid lines ($x_e = 1$) and dotted lines ($x_e = 0$). The radius of a blossom $S$ represents the corresponding dual variable $y_S$. The direction of each node $\Delta y_S \in \{ 0, +1, -1 \}$ is marked, with different colors.}
    \label{fig:example-alternating-tree}
    \Description[example alternating tree]{}
\end{figure}

\paragraph{Fast MWPM Decoders}
The blossom algorithm solves the MWPM decoding problem using the syndrome graph~\cite{wu2022interpretation}, which is dense with $O(|V|^2)$ edges.
Sparse Blossom~\cite{higgott2025sparse} and Parity Blossom~\cite{wu2023qce} speed up MWPM decoding by adapting the blossom algorithm to solve the MWPM problem on the decoding graph, which is adopted by \arch.

Parity Blossom conveniently decomposes the blossom algorithm into the primal and dual phases, which is also adopted by \arch.
Parity Blossom implements these in the decoding graph, improving the asymptotic average time complexity to almost linear, but still falls short of sub-$\mu s$ decoding latency.
As shown in \autoref{fig:amdahls-law}, the dual phase takes most of the decoding time of Parity Blossom.
Improving the speed of the dual phase is critical for further improving the decoding speed.

\begin{figure}[!t]
    \centering
    \includegraphics[width=0.83\linewidth]{figures/slides/real-time-decoder-feedback.pdf}
    \caption{Fault-tolerant logical $\hat{T}$ gate on the target qubit is implemented using a resource qubit in the magic $|T\rangle$ state~\cite{bravyi2005universal} and a circuit consisting of fault-tolerant Clifford gates and a conditional logical $\hat{S}$ gate with decoder feedforward.}
    \label{fig:real-time-decoding}
    \Description[real time decoding]{}
\end{figure}

\paragraph{Why Decoding Latency Matters}
Nontrivial fault-tolerant quantum computing requires feedforward from the decoder with a soft deadline~\cite{terhal2015quantum}, making the decoding a soft real-time problem.
The soft deadline comes from implementing logical non-Clifford gates, which are necessary for any universal quantum gate set.
For example, implementing a logical $\hat{T}$ gate requires a logical qubit readout as a feedforward signal, as shown in \autoref{fig:real-time-decoding}.
While waiting for the decoded readout, the target qubit accumulates more logical errors.
When there is no decoding latency, the data qubit only goes through $d$ stabilizer measurement rounds and suffers from a total logical error rate of $p^\text{MWPM}_L$.
If the decoding latency is $L$ measured in the number of stabilizer measurement cycles, then the logical error rate of the target qubit effectively becomes roughly $p^\text{MWPM}_L (1 + L/d)$.
Taking $d=21$ surface code with $\qty{1}{\mu s}$ measurement cycle~\cite{google2023suppressing} as an example, the effective logical error rate is $34 p^\text{MWPM}_L$ for Fusion Blossom~\cite{wu2023qce}, and $5 p^\text{MWPM}_L$ for the faster but less accurate Union-Find decoder~\cite{liyanage2024fpga}.
In this case, although Union-Find decoding has 5x more logical errors than MWPM decoding, it causes fewer logical errors when including latency-induced idle errors.
