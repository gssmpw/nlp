\section{Related Work}
\label{sec:related}

\paragraph{Hardware Union-Find (UF) decoder.} 
We are inspired by Helios~\cite{liyanage2023qce}, a low-latency UF decoder implemented on FPGA.
Helios also associates a processing unit with each vertex and therefore achieves the same level of parallelism as \arch.
However, the UF decoder approximates the blossom algorithm~\cite{wu2022interpretation}.
Such approximation makes it much more amenable to a complete parallel realization at the cost of lower decoding accuracy. 
CC Decoder~\cite{barber2023realtime} is another hardware UF decoder that is highly resource-efficient, using a dedicated hardware unit to speed up a sequential algorithm.
There are other hardware UF decoders, but they lack evaluation data from actual implementation~\cite{das2022afs}.

\paragraph{Hardware approximate MWPM decoder.}
Astrea~\cite{vittal2023astrea} and Promatch~\cite{alavisamani2024promatch} are hardware-accelerated MWPM decoders that achieve roughly the same accuracy of MWPM decoding under certain conditions.
They decode simple syndromes below certain Hamming weights, which implicitly assumes small code sizes and low physical error rates.
For example, at $d=13$ and $p=0.1\%$, their approximation leads to more than $13.9\times$ higher logical error rate~\cite{alavisamani2024promatch}.
Achieving high decoding accuracy and low decoding latency at the same time is challenging.
An important insight from~\cite{vittal2023astrea} is that the blossom algorithm is too complex for hardware implementation.
Instead of implementing a full hardware blossom algorithm, \system accelerates part of the algorithm in hardware.


\paragraph{Exploiting locality in syndrome data.} Micro Blossom leverages the fact that defect vertices are often matched locally in an MWPM solution (\S\ref{sec:pre-matching}).
The Clique decoder~\cite{ravi2023better} and the Lazy decoder~\cite{delfosse2020hierarchical} also exploit it to lower the bandwidth requirement and to accelerate MWPM decoding.
However, both decoders treat the MWPM decoder as a black box. As a result, Lazy becomes less effective at reducing bandwidth for $d \ge 15$ with $p=0.1\%$~\cite{delfosse2020hierarchical}, 
while Clique sacrifices logical error rate by over 10x at $d \ge 11$ with $p=0.1\%$~\cite{ravi2023better}.
In contrast, Micro Blossom achieves better scalability and optimal decoding accuracy by integrating the locality-aware optimizations into the blossom algorithm itself.
Notably, Micro Blossom achieves a reduced bandwidth of $O(p^2 |V| + 1)$ compared to the original syndrome data of $O(p|V| + 1)$, and remains effective at reducing bandwidth requirements for arbitrarily large code distances.

\paragraph{Software MWPM decoder.}
Sparse Blossom~\cite{higgott2025sparse} and Fusion Blossom~\cite{wu2023qce} achieve an almost linear average decoding time of $O(p |V|)$.
They maintain clusters on the decoding graph instead of on the syndrome graph~\cite{wu2022interpretation}, similar to that in the Union-Find decoder~\cite{delfosse2021almost}.
Although coarse-grained parallelization~\cite{wu2023qce} of the blossom algorithm improves the throughput of MWPM decoding, it cannot lower the decoding latency at the microsecond level.