\usepackage[utf8]{inputenc}
\usepackage{wasysym} % although not listed in https://www.scomminc.com/pp/acmsig/ACM-Template-Accepted-Packages.txt, this package is merely just some symbols and I couldn't find a good alternative (although mathabx does include similar symbols, importing it causes LaTeX error of command \bigtimes already defined.)
\usepackage{amsfonts}
\usepackage{amsmath}
\usepackage{graphicx}
\usepackage{amsthm}
\usepackage{color}
\usepackage[]{hyperref}
\usepackage{siunitx} % only "SIunits" is listed in https://www.scomminc.com/pp/acmsig/ACM-Template-Accepted-Packages.txt, however, this package is superseded by siunitx
\usepackage{soul}
\usepackage{float}
\usepackage{subcaption} % only "subfigure" is listed in https://www.scomminc.com/pp/acmsig/ACM-Template-Accepted-Packages.txt, however, that package is deprecated and recommended using subcaption

\usepackage[frozencache=true,cachedir=minted-cache]{minted} % fix arXiv minted error

% final solution: https://tex.stackexchange.com/questions/280590/work-around-for-minted-code-highlighting-in-arxiv/725077#comment1701405_659452

% when generating the minted cache files, using the following lines; once generated, use the above
% \usepackage[cachedir=minted-cache]{minted}

\usemintedstyle{vs}
\setminted[r]{fontfamily=lmtt}

\definecolor{outputbkg}{gray}{0.95}
\definecolor{commandbkg}{rgb}{1, 0.95, 0.95}

\usepackage[ruled,linesnumbered]{algorithm2e}
\renewcommand{\algorithmautorefname}{Algorithm}

\newcommand{\code}[1]{\texttt{#1}}

\DeclareMathOperator*{\argmin}{arg\,min}
\DeclareMathOperator*{\argmax}{arg\,max}

\newcommand{\papertitle}{\system: Accelerated Minimum-Weight Perfect Matching Decoding for Quantum Error Correction}

% commands for nosection label
\makeatletter
\newcommand{\nosectionlabel}[2]{%
   \protected@write \@auxout {}{\string \newlabel {#1}{{\textbf{#2}}{\thepage}{#2}{#1}{}} }%
   \hypertarget{#1}{\noindent\textbf{#2.}}
}
\makeatother

\newcommand{\dns}{\!\!}  % dual negative space
\newcommand{\qns}{\!\!\!\!}  % quad negative space
\newcommand{\dqns}{\qns\qns}  % 2 qns
\newcommand{\tqns}{\qns\qns\qns}  % 3 qns
\newcommand{\qqns}{\qns\qns\qns\qns}  % 4 qns
\newcommand{\compactsetminus}{\!\setminus\!}
\newcommand{\spaciousland}{\;\land\;}
\newcommand{\dqquad}{\qquad\qquad}
\newcommand{\tqquad}{\qquad\qquad}
\newcommand{\qqquad}{\qquad\qquad\qquad\qquad}


\newcommand{\system}{Micro Blossom\xspace}
\newcommand{\arch}{Micro Blossom\xspace}
\newcommand{\round}{round-wise\xspace}
\newcommand{\Round}{Round-wise\xspace}
\newcommand{\obstacle}{obstacle\xspace}
\newcommand{\obstacles}{obstacles\xspace}
\newcommand{\pu}{PU\xspace}
\newcommand{\pus}{PUs\xspace}
\newcommand{\puv}{vPU\xspace}
\newcommand{\puvs}{vPUs\xspace}
\newcommand{\pue}{ePU\xspace}
\newcommand{\pues}{ePUs\xspace}
\newcommand{\PU}{\pu\xspace}
\newcommand{\PUs}{\pus\xspace}
\newcommand{\PUV}{Vertex PU\xspace}
\newcommand{\PUVs}{Vertex PUs\xspace}
\newcommand{\PUE}{Edge PU\xspace}
\newcommand{\PUEs}{Edge PUs\xspace}
\newcommand{\cov}{\textit{Cover}\xspace}
\newcommand{\covs}{\textit{Covers}\xspace}
\newcommand{\conf}{\textit{Conflict}\xspace}
\newcommand{\confs}{\textit{Conflicts}\xspace}
\newcommand{\loc}{local\xspace}
\newcommand{\isl}{isolated\xspace}
\newcommand{\Loc}{Local\xspace}
\newcommand{\Isl}{Isolated\xspace}
\newcommand{\tight}{tight\xspace}

\newcommand{\nosection}[1]{\vspace{3pt}\noindent\textbf{#1}}

% \begin{CCSXML}
% <ccs2012>
%    <concept>
%        <concept_id>10010583.10010786.10010813.10011726.10011728</concept_id>
%        <concept_desc>Hardware~Quantum error correction and fault tolerance</concept_desc>
%        <concept_significance>500</concept_significance>
%        </concept>
%    <concept>
%        <concept_id>10010520.10010521.10010542.10010546</concept_id>
%        <concept_desc>Computer systems organization~Heterogeneous (hybrid) systems</concept_desc>
%        <concept_significance>500</concept_significance>
%        </concept>
%    <concept>
%        <concept_id>10010520.10010570.10010574</concept_id>
%        <concept_desc>Computer systems organization~Real-time system architecture</concept_desc>
%        <concept_significance>500</concept_significance>
%        </concept>
%    <concept>
%        <concept_id>10002950.10003624.10003633.10003642</concept_id>
%        <concept_desc>Mathematics of computing~Matchings and factors</concept_desc>
%        <concept_significance>500</concept_significance>
%        </concept>
%  </ccs2012>
% \end{CCSXML}

% \ccsdesc[500]{Computer systems organization~Heterogeneous (hybrid) systems}
% \ccsdesc[500]{Computer systems organization~Real-time system architecture}
% \ccsdesc[500]{Mathematics of computing~Matchings and factors}
% \ccsdesc[500]{Hardware~Quantum error correction and fault tolerance}

\keywords{Minimum-Weight Perfect Matching (MWPM) Decoder, Quantum Error Correction, Heterogeneous Architecture}
% it is typical to use comma in the paper to separate keywords, but remember to use semicolon when submitting the camera ready version: see https://www.scomminc.com/pp/acmsig/asplos.htm#4:~:text=Separating%20your%20keywords%3B%20with%20semi%2Dcolons%3B%20is%20mandatory
