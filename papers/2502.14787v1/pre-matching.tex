\section{Accelerating Primal Phase}\label{sec:pre-matching}

As described in \S\ref{sec:parallel-dual}, \arch accelerates the dual phase with parallel \pus while keeping the primal phase in software, running on a CPU. 
As a result,  the CPU and accelerator frequently interact during the decoding process, which significantly contributes to the decoding latency.
One major source of such interactions is when the dual phase (running on the accelerator) detects a \conf, the primal phase (running on the CPU) must resolve it. 
We reduce the frequency of such interactions by postponing the handling of a common type of \confs and derive their matchings in parallel.
We first provide background (\S\ref{ssec:background-matching-patterns}) and demonstrate a resource-efficient, $O(1)$ time implementation of detecting and handling such \confs on parallel \pus (\S\ref{ssec:first-order-pre-matching}).

\subsection{Background}
\label{ssec:background-matching-patterns}

When the dual phase detects a \conf, the primal phase manipulates the alternating trees and the matched pairs following a set of rules~\cite{kolmogorov2009blossom}.
It does three things.
(\emph{i}) it changes some primal variables $x_e$. (\textit{ii}) It changes the direction ($\Delta y_S$) for the nodes involved (See \S\ref{sec:background} Blossom Algorithm). (\textit{iii}) it may create new blossoms or expand existing blossoms.
In \arch, the primal phase (running the CPU) sends the ``\code{set Direction}'' and ``\code{set Cover}'' instructions to the accelerator as a consequence of (\textit{ii}) and (\textit{iii}), respectively.

There is a special type of \conf that requires much simpler primal phase operations.
We call them \isl \confs.
\Isl \confs are results of \emph{isolated} errors whose neighboring error sources are normal, as shown in \autoref{fig:randomly-distributed-errors} where errors are represented by blue edges in the decoding graph.  
With ever-improving physical qubits, isolated errors are common.
Importantly, when the physical error rates vary little, i.e., $\min p > (\max p)^2$, an isolated physical error always leads to one isolated \conf.
An \isl \conf happens when the \cov of a node $\{v\}$ incident of an isolated error (represented by edge $e$) grows into that of the other node $\{u\}$ incident of $e$, which can be either a defect vertex (\autoref{fig:local-conflict-regular}) or a virtual vertex (\autoref{fig:local-conflict-boundary}), before they touch any other \covs.
When an \isl \conf happens, the primal phase's three operations are much simpler.
(\textit{i}) it sets $x_g$ to $1$, where $g$ is the edge between $u$ and $v$ in the syndrome graph. (\textit{ii}) it sets $\Delta y_{\{v\}}$ (and $\Delta y_{\{u\}}$ in the case of \autoref{fig:local-conflict-regular}) to $0$. The primal phase does not do anything for (\textit{iii}).
Our key insight is that if an \isl \conf remains \isl at the end of the algorithm, it can be incorporated into the final MWPM trivially: $g$, the edge between $u$ and $v$ in the syndrome graph, should be in the MWPM.
That is, the accelerator does not need to report an \isl \conf at all. When one of the involved \covs touches another \cov during the course of the algorithm, leading to a non-isolated \conf, the accelerator will report this new \conf to the CPU, which will discover the previously unreported \conf from the report.

\subsection{Accelerated Handling of \Isl \conf}\label{ssec:first-order-pre-matching}
\begin{figure}[t]
    \centering
         \begin{subfigure}{.33\linewidth}
             \centering
         \includegraphics[width=1\textwidth,page=1]{figures/slides/randomly-distributed-errors.pdf}
            \caption{Random Errors.}
            \label{fig:randomly-distributed-errors}
        \end{subfigure}
         \begin{subfigure}{.32\linewidth}
             \centering
         \includegraphics[width=1\textwidth,page=1]{figures/slides/matching.pdf}
            \caption{Regular Edge.}
            \label{fig:local-conflict-regular}
        \end{subfigure}
        \begin{subfigure}{.32\linewidth}
            \centering
            \includegraphics[width=1\textwidth,page=2]{figures/slides/matching.pdf}
            \caption{Boundary Edge.}
            \label{fig:local-conflict-boundary}
          \end{subfigure}  
    \caption{(a) \Isl errors often (but not always) lead to \isl \confs.
    (b) When the isolated error does not happen on the boundary, it results in a pair of defect vertices, $u$ and $v$. 
    (c) When the isolated error happens at the boundary of the surface code, it results in a defect vertex $v$, next to a virtual vertex $u$ (marked yellow).
    When the dual phase grows $\text{Cover}(\{v\})$ and ($\text{Cover}(\{u\})$ in (b)),  they are likely to touch each other, before \covs of any other defect vertices, which are farther away. This results in an \isl \conf, which \arch treats efficiently.
    Parity Blossom treats a virtual vertex as a defect vertex whose \cov never grows.}
    \label{fig:conflict-resolution}
\end{figure}

We next describe how \arch detects and handles some common \isl \confs using parallel \pus, without invoking the CPU.
Because not all \isl \confs can be detected by \pus using local information only, we must narrow down to those that can be.

\paragraph{Parallel \pue Logic for Detecting \Isl \confs}
For an \pue to detect if an \isl \conf happens on its edge $e=(u,v)$,
it must tell if $\{u\}$ and $\{v\}$ are both nodes and $Cover(\{u\})$ and $Cover(\{v\})$ touch other \covs.
In general, this is impossible only with information local to the \pue (the local states of \puvs of its incident vertices).
Our key idea is to only consider those \isl \confs in which whether $Cover(\{u\})$ and $Cover(\{v\})$ touch other \covs can be detected using information local to each \pue.
We have identified two kinds of such \isl \confs.

The first kind is illustrated by \autoref{fig:local-conflict-regular} where an \isl error happens away from the surface code boundary and leads to a pair of defect vertices whose other neighbors are non-defect.
We define $t_e \coloneq (r_u + r_v \ge w_e)$ indicates whether $e$ is \tight.
$q_v$ indicates whether $e$ is the only \tight edge incident to $v$, which can be computed by $v$'s \puv.
We detect such \isl \confs using the following condition.
\begin{equation}
m^\text{r}_e \coloneq t_e \land (d_u \land q_u \land s_u > 0) \land (d_v \land q_v \land s_v > 0) \label{eq:regular-matching-condition}
\end{equation}

However, the above condition is not effective for an edge $e= (u,v)$ on the boundary of the surface code, as illustrated by \autoref{fig:local-conflict-boundary}.
This is because $\text{Cover}(\{v\})$ likely also touches the neighboring regular vertices $\{u_i\}$.
Thus, we have to detect \isl \confs using the \puv states from all $\{u_i\}$, which is more expensive than \autoref{eq:regular-matching-condition}.
\begin{equation}
m^\text{b}_e \coloneq t_e \land b_u \land (s_v > 0) \land d_v \dns\underset{\substack{e' \in E(v) - e\\ (u', v) = e'}}{\land}\dns\left[\neg t_{e'} \lor (\neg d_{u'} \land q_{u'})\right] \label{eq:boundary-matching-condition}
\end{equation}

In summary, in \arch, \pues corresponding to regular and boundary edges use \autoref{eq:regular-matching-condition} and \autoref{eq:boundary-matching-condition} as sufficient conditions to detect \isl \confs in parallel.

\paragraph{Parallel Handling of \Isl \conf}
Once an \pue detects an \isl \conf, its detection logic $m_e$ is true, which temporarily signifies that the corresponding edge $g$ in the syndrome graph should be included in the matching and $x_g$  should be set as 1. It sets  $\Delta y_{\{v\}}$ and $\Delta y_{\{u\}}$ to $0$ so there will be no change to  $y_{\{v\}}$ and $y_{\{u\}}$. These correspond to primal operations (\textit{i}) and (\textit{ii}) described in \S\ref{ssec:background-matching-patterns}.

During the process of the algorithm, if another \cov touches that of $\{v\}$ (or that of $\{u\}$ in the case of \autoref{fig:local-conflict-regular}), a non-\isl \conf happens. $m_e$ will turn false and some \pue will report the new \conf to the primal phase on the CPU, which will also find out the unreported \conf on $e$.

When the algorithm ends, if the \conf on edge $e=(u,v)$ remains \isl, the accelerator trivially includes $g$, the edge between $u$ and $v$ in the syndrome graph, in the MWPM.
