\section{Introduction}

\begin{figure*}[t]
\newcommand{\subfiglinewidth}{0.325\linewidth}
\newcommand{\codelinewidth}{\linewidth}
\begin{minipage}{0.7\linewidth}
    \centering
    \begin{subfigure}{\subfiglinewidth}
        \centering
        \includegraphics[width=\codelinewidth]{figures/code/code.pdf}
        \caption{Surface Code $d=5$.}
        \label{fig:bkgd-surface-code}
    \end{subfigure}
    \begin{subfigure}{\subfiglinewidth}
        \centering
        \includegraphics[width=\codelinewidth]{figures/code/micro_paper_code_capacity_demo.png}
        \caption{Decoding Graph.}
        \label{fig:bkgd-code-capacity}
    \end{subfigure}
    \begin{subfigure}{\subfiglinewidth}
        \centering
        \includegraphics[width=\codelinewidth]{figures/code/circuit-level-decoding-graph.png}
        \caption{3D Decoding Graph.}
        \label{fig:bkgd-circuit-level}
    \end{subfigure}
    \caption{Surface code and decoding graph. (a) The surface code interleaves data qubits ($\CIRCLE$) with stabilizer qubits ($\Circle$). Here we only show $\hat{Z}$-type stabilizer qubits that detect $\hat{X}$ errors. The $\hat{X}$-type stabilizes can be decoded likewise independently. (b) The decoding graph of (a). Each vertex represents a stabilizer measurement; each edge represents a potential error. Stabilizers with flipped measurement and their vertices (defect vertices) are marked in red in both figures. (c) A decoding graph from a circuit-level implementation of the surface code with $d$ rounds of measurements.}
    \label{fig:bkgd}
\end{minipage}
\hfill
\begin{minipage}{0.27\linewidth}
    \begin{minipage}{\linewidth}
        \centering
        \includegraphics[width=0.96\codelinewidth]{figures/Amdahls-law/Amdahls-law-2.pdf}
    \end{minipage}
    \caption{Potential speed up according to Amdahl's Law, sampled from the Fusion Blossom~\cite{wu2023qce} running on Apple M1 Max. The potential speedup is the theoretical upper bound of optimizing the dual phase.}
    \label{fig:amdahls-law}
    \Description[potential speed up]{}
\end{minipage}
\end{figure*}

Quantum error correction (QEC) plays an important role in building fault-tolerant quantum computers.
It requires a decoder that processes syndromes, potentially measured $10^6$ times per second, to discover qubit errors timely, within a microsecond~\cite{google2023suppressing} in the case of implementing fault-tolerant logical $\hat{T}$ gates~\cite{holmes2020nisq+}.
For surface codes, an important class of QEC codes, the most-likely error decoding problem can be formulated as a Minimum-Weight Perfect Matching (MWPM) problem~\cite{dennis2002topological} associated with a \emph{syndrome graph}~\cite{wu2022interpretation}, 
under the assumption of a noise model of independently distributed bit and phase flips, and can be solved by the famous blossom algorithm~\cite{edmonds1965paths,kolmogorov2009blossom}. 

In recent years, many MPWM decoder implementations have been implemented in response to high throughput and low latency requirements. Sparse Blossom~\cite{higgott2025sparse} and Parity Blossom~\cite{wu2023qce} reduce the time complexity of MWPM decoding by adapting the blossom algorithm to work on the \emph{decoding graph}. Fusion Blossom~\cite{wu2023qce} improves throughput by using multiple cores to process blocks of syndrome measurement rounds in parallel. 
None of these exact MWPM decoders achieve a decoding latency close to a microsecond or leverage any hardware acceleration. In fact, it is believed~\cite{vittal2023astrea} that the blossom algorithm~\cite{edmonds1965paths,kolmogorov2009blossom}, which underpins all these MWPM decoders, is too complex for hardware acceleration. Not surprisingly, many have resorted to less accurate decoders that approximate MWPM decoding~\cite{vittal2023astrea,liyanage2023qce,alavisamani2024promatch}, causing 1.7x~\cite{liyanage2023qce} or even 13.9x~\cite{alavisamani2024promatch} more logical errors at code distance $d=13$ and physical error rate $p=0.1\%$. 
    
In this work, we present the first publicly known exact MWPM decoder that achieves sub-microsecond latency, called \system. It is also the first publicly known exact MWPM decoder with hardware acceleration. \system builds on ideas from recent fast decoders, namely~\cite{wu2023qce,higgott2025sparse}, to implement the blossom algorithm and support stream decoding.
\system carefully partitions the blossom algorithm between software and a programmable accelerator of parallel processing units (\pus), one corresponding to each vertex and edge in the decoding graph (\S\ref{sec:overview}). 
It implements the primal phase of the blossom algorithm in software that flexibly handles complex data structures. It implements the dual phase in the programmable accelerator that exploits the finest-grained parallelism of the decoding graph in vertices and edges (\S\ref{sec:parallel-dual}).
To reduce the expensive software-hardware interaction, \system uses parallel \pus to resolve common \confs without involving the primal phase on the CPU (\S\ref{sec:pre-matching}). To support low-latency stream decoding, \system performs round-wise fusion with the parallel \pus in the accelerator (\S\ref{sec:fusion}).

Using SpinalHDL~\cite{spinalhdlv193}, we implement a parameterized prototype of \system that takes an arbitrary decoding graph as input and produces synthesizable Verilog code (\S\ref{sec:implementation}).
When running the accelerator on a Xilinx Versal VMK180 FPGA, our prototype achieves an average decoding latency of $\qty{0.8}{\mu s}$ for code distance $d=13$ and physical error rate $p=0.1\%$ in a circuit-level noise (\S\ref{sec:evaluation}).
It is the first publicly known MWPM decoder with sub-$\mu s$ latency, 8x shorter than the best reported~\cite{higgott2025sparse}, for the same code distance and noise model.
We note that the largest surface code reported in QEC physical experiments is $d=7$~\cite{acharya2024quantum}.
We also note that \system can support even larger $d$ for sub-$\mu s$ latency if more resources are available, e.g., larger FPGA, and if higher clock frequency is feasible, e.g., using ASIC, as analyzed in \S\ref{ssec:eva-resource}.

\begin{comment}
By solving exact MWPM solutions in sub-$\mu s$ latency, we show that some nontrivial global optimization problems could be solved in real time if they exhibit certain structures and statistical attributes.
We hope that our work inspires other real-time systems to employ advanced algorithms.
\end{comment}

In summary, we make the following contributions:
\begin{itemize}
    \item We design a heterogeneous architecture for scalable, real-time, and exact MWPM decoding for QEC.
    \item We parallelize the blossom algorithm at the vertex and edge levels, which improves both the worst-case and average decoding latency.
    \item We demonstrate the first publicly known sub-$\mu s$ average latency of MWPM decoding on a prototype using FPGA for surface code of $d=13$.
\end{itemize}

Our implementation of \system is open-source and available from~\cite{micro-blossom}.
