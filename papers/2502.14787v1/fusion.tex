\section{Round-wise Fusion for Stream Decoding}\label{sec:fusion}

To further reduce decoding latency, \arch processes the syndrome in a stream manner, starting the decoding process as soon as a measurement round arrives.
We provide background on the fusion operation in \S\ref{ssec:background-fusion}, which enables stream decoding.
We describe \emph{round-wise fusion} that performs intra-block fusion to further lower decoding latency, as detailed in \S\ref{ssec:round-wise-fusion}.
Additionally, we extend the local \conf detection algorithm to support round-wise fusion in \S\ref{ssec:micro-fusion-extension}.

\subsection{Background}\label{ssec:background-fusion}

As shown in \autoref{fig:architecture}, syndrome data for a $d\times d$ surface code comes in a series of $d$ measurement rounds with a fixed interval (about $1\mu s$ for superconducting qubits).
Stream decoding treats the decoding problem as an incremental optimization problem: instead of waiting for all $d$ rounds to solve them as a single batch, stream decoding can start decoding as soon as the first round arrives and incrementally incorporates a newly arrived round into the solution.
If properly designed, a stream decoder finds a final solution identical to that of a batch decoder.

Fusion Blossom~\cite{wu2023qce} is such a stream decoder based on Parity Blossom and the concept of \emph{fusion}.
It divides the decoding graph $G= (V, E)$ into two non-overlapping subgraphs $G_1=(V_1, E_1)$ and $G_2=(V_2, E_2)$,  and a set of vertices $V_b = V \setminus V_1 \setminus V_2$, which form a boundary between $G_1$ and $G_2$.
It finds the MWPMs for $G_1$ and $G_2$ separately using Parity Blossom and fuses them into the MWPM for $G$ by considering the boundary vertices $V_b$.
This divide-fusion process can be applied recursively so that Parity Blossom is only invoked on a subgraph of a manageable size.
Fusion Blossom differs from other stream decoding methods, such as window decoding~\cite{dennis2002topological,iyengar2011windowed,skoric2023parallel,tan2022scalable,bombin2023modular}, in that it does not compromise decoding accuracy or introduce redundant computation.
In~\cite{wu2023qce}, the authors considered the smallest subgraphs that consist of multiple measurement rounds.

\arch adopts the same approach to implement stream decoding but considers a novel, extreme case where $G_1=(V_1,E_1)$ consists of all measurement rounds received before the latest, $G_2=(V_2,E_2)$ is empty, and the boundary $V_b$ includes vertices in the latest measurement round. We call this strategy \emph{\round fusion}.

\subsection{\Round Fusion}\label{ssec:round-wise-fusion}
\Round fusion incrementally finds the global MWPM by fusing a newly arrived measurement round into the partial MWPM of previously received rounds.
Its global optimality is guaranteed per Fusion Blossom~\cite{wu2023qce}.

\begin{comment}
\begin{figure*}[t]
    \centering
    \subfloat[$t = \qty{1}{\mu s}$]{\includegraphics[width=0.16\linewidth,trim={0 0 0 15cm},clip]{figures/web-plot/fusion-process/fusion_demo_1.png}}
    \hfill
    \subfloat[$t = \qty{2}{\mu s}$]{\includegraphics[width=0.16\linewidth,trim={0 0 0 15cm},clip]{figures/web-plot/fusion-process/fusion_demo_3.png}}
    \hfill
    \subfloat[$t = \qty{3}{\mu s}$]{\includegraphics[width=0.16\linewidth,trim={0 0 0 15cm},clip]{figures/web-plot/fusion-process/fusion_demo_5.png}}
    %\\
    \hfill
    \subfloat[$t = \qty{4}{\mu s}$]{\includegraphics[width=0.16\linewidth]{figures/web-plot/fusion-process/fusion_demo_7.png}}
    \hfill
    \subfloat[$t = \qty{5}{\mu s}$]{\includegraphics[width=0.16\linewidth]{figures/web-plot/fusion-process/fusion_demo_9.png}}
    \hfill
    \subfloat[MWPM]{\includegraphics[width=0.16\linewidth]{figures/web-plot/fusion-process/fusion_demo_10.png}}
    \caption{An example of stream decoding based on \round fusion. (a) It starts with the first measurement round $G_1$ and the second measurement round as the boundary (blue). 
    \lina{What is G1 in (a)? the red vertices are defect? If so, G1 also includes a round?}
    \lin{What are the red edges and green edges in (a)? How do they change in (b)?}
    \lin{The rest of the caption explains fusion. IMHO, this is not necessary. Instead, you should use (b) to (f) to explain the round-wise fusion process.}
    When fusing a measurement round, we first load the defects and then recover the boundary vertices as regular vertices. The Covers evolve as new measurements come, but it is statistically unlikely to affect the Covers away from the fusion boundary. One can imagine that the ``active'' region of the decoder is only a few measurement rounds near the fusion boundary, although in the worst case it can still update the whole decoding graph.
    }
    \label{fig:fusion-process}
\end{figure*}
\end{comment}

As the accelerator features a $d\times d\times d$ array of \puvs, when the first round arrives, the accelerator loads it into the first layer ($1\times d\times d$) and solves it. When the second round arrives, the accelerator loads it into the second layer ($2\times d\times d$). It treats the first layer as $G_1$, the second layer as the boundary $V_b$, and $G_2=\varnothing$, and performs the fusion. This process continues until all $d$ rounds have been processed.

\arch implements the above process in $O(1)$ time simply by manipulating the \puv state $b_v$, which indicates whether the associated vertex $v$ is a boundary vertex.
It sets $b_v$ to \code{true} if the \puv is unloaded; as a result, it treats $v$ as a virtual vertex in the fusion operation~\cite{wu2023qce}.
Each \puv is assigned a layer ID according to its position in the 3D array.
At the outset, $b_v$ of all \puvs is set to \code{true}.
When a new round of syndrome is available, the controller issues a ``\code{load Defects}'' instruction with the layer ID.
Upon receiving this instruction, the \puvs with the matching layer ID load the defect bit $d_v$ and set $b_v$ to \code{false}.

To perform the fusion, the only additional logic is in the software (CPU) that breaks all existing matchings with the fusion boundary, which are originally virtual vertices in the newly arrived layer~\cite{wu2023qce}. Then \arch continues to resolve the new \confs and to find the optimal solution for the fused decoding graph.

\subsection{Handling Isolated Conflict}\label{ssec:micro-fusion-extension}

We next extend the parallel detection of isolated \confs (\S\ref{ssec:first-order-pre-matching}) to round-wise fusion.
From \S\ref{ssec:first-order-pre-matching}, we know that detecting isolated \confs for boundary edges (\autoref{eq:boundary-matching-condition}) is more expensive than that for regular edges (\autoref{eq:regular-matching-condition}).
Without round-wise fusion, boundary edges are relatively rare and as a result, add little to the overall overhead.
However, with round-wise fusion, every edge must be a boundary edge at least once. As a result, the high cost of detecting isolated \confs on the boundary edges becomes problematic.

Our key insight toward mitigating this problem is that the logic for boundary edges is complicated because $\text{Cover}(v)$ may touch other vertices, i.e., $u_2$ and $u_3$ in \autoref{fig:local-conflict-boundary}, before touching the \cov of a virtual vertex ($u$). 
By temporarily reducing the weights of edges connected to the fusion boundary $V_b$, isolated \conf likely happens on these edges without $\text{Cover}(v)$ interacting with other vertices.
Once $b_v$ becomes false, the \pues restore the original edge weights, ensuring that the final solution remains intact after all \puvs have loaded the syndrome.

Based on the above idea, we design a more resource-efficient logic to detect isolated \confs during round-wise fusion, replacing that of \autoref{eq:boundary-matching-condition}.
For an edge $e = (u, v)$ where vertex $u$ has a layer ID higher than that of $v$, we define the isolated \conf condition $m^{\text{f}}_e$.
First, we define the concept of non-volatile tightness $t'_e$, which ignores temporary tight edges when $u$ is still virtual: $t'_e \coloneq \neg b_u \land (r_u + r_v \ge w_e)$.
Next, we define $\emptyset_v \coloneq (0 = \sum_{e\in E(v)} t'_e)$, which checks if vertex $v$ has no non-volatile tight edges surrounding it.
The condition $m^{\text{f}}_e$ holds when $u$ is virtual and $\{v\}$ is a growing node without any non-volatile tight edge incident to $v$.
\begin{equation}
m^{\text{f}}_e := t_e \land b_u \land (s_v > 0) \land d_v \land \emptyset_v \label{eq:fusion-matching-condition}
\end{equation}

When the physical error rates vary little, like those in \S\ref{ssec:background-matching-patterns}, we can again demonstrate that every physical error results in one isolated \conf.
By combining accelerator-based handling of isolated \confs with round-wise fusion, the average number of defects processed by the CPU is reduced from $O(p^2 d^3)$ to $O(p^2 d^2)$, assuming $p$ is small and the decoding speed exceeds the measurement rate.
The $d$-fold reduction results from the fact that the decoder focuses on a constant number of recent measurement rounds.
