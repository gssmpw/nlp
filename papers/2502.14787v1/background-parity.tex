\subsection{Background}
\label{sec:background-parity}

Parity Blossom~\cite{wu2023qce} is based on a geometric interpretation of the decoding graph in which any two \emph{points} in the graph have a non-negative distance.

The algorithm associates each defect vertex $v$ with a set of points called \emph{Circle}, $C(v,\sum_{A \in \mathcal{A}(v)} y_A)$, with $v$ being the center and a radius of $\sum_{A \in \mathcal{A}(v)} y_A$ (Appendix D~\cite{wu2023qce}).
We say two \covs are neighbors if their associated vertices are neighbors on the decoding graph.

Parity Blossom associates a node $S$ with a set of points called \cov, which is the union of the above \emph{Circle}s of defect vertices in the node, i.e., $\text{\cov}(S)=\cup_{v\in S} C(v,\sum_{A \in \mathcal{A}(v)} y_A)$.

Finally, the algorithm reduces the dual-phase operations to the manipulations of its \covs.
We have \textbf{Theorem: Tight Edge Detection (\cov)}~\cite{wu2023qce} that  $\text{\cov}(S_1) \cap \text{\cov}(S_2) \neq \varnothing$ implies there exists a tight edge between nodes $S_1$ and $S_2$.
The dual phase of Parity Blossom maintains the \emph{\cov}s of the nodes and uses them to detect situations when it can no longer adjust the dual variables without violating any dual problem constraint, called \emph{Obstacles}.
Once the dual phase detects an \emph{Obstacle}, the algorithm switches to the primal phase to resolve it.

There are two types of \emph{Obstacles}, as defined in~\cite{wu2023qce} as (2a) and (2b).
The first corresponds to the constraint (2a) when the dual variable of a node $S \in \mathcal{O}$ is already 0 but is still shrinking: $y_S = 0 \land \Delta y_S < 0$.
This type of \emph{Obstacle} occurs rarely and can be handled efficiently using a priority queue on the CPU~\cite{kolmogorov2009blossom,higgott2025sparse}.
The second type corresponds to constraint (2b) and occurs more frequently, when two nodes ($S_1$ and $S_2$) are growing toward each other ($\Delta y_{S_1} + \Delta y_{S_2} > 0$) while there is already a tight edge between them in the syndrome graph ($\text{\cov}(S_1) \cap \text{\cov}(S_2) \neq \varnothing$).
We call this type of \emph{Obstacle}  \emph{Conflict}.
A \emph{Conflict} indicates an edge connecting the two nodes in the syndrome graph has become tight. 

The first three columns of \autoref{tab:dual-phase-operations} describe all dual-phase operations in \cite{wu2023qce}.
The first five operations update the \covs while the last operation detects \confs.

An astute reader may realize that Parity Blossom is \cov\emph{-parallel}, although the implementation reported in~\cite{wu2023qce} did not exploit this.
Because \covs are only associated with defect vertices and defect vertices appear randomly on the decoding graph, manipulations of a \cov may require information about non-neighboring \covs, which would require non-local connectivity between vertices if one tries to achieve vertex-parallelism by associating a \PU with each vertex like \arch.
