
\documentclass[11pt]{article}

\usepackage{microtype}
\usepackage{graphicx}
\usepackage{subcaption}
\usepackage{booktabs}
\usepackage[numbers]{natbib}
\usepackage{xfrac}
\usepackage{nicefrac}
\usepackage{aligned-overset}
\usepackage{algorithm}
\usepackage{algpseudocode}

\usepackage[colorlinks=true, linkcolor=blue!50!black, citecolor=green!50!black]{hyperref}
\usepackage{enumitem}

\newcommand{\theHalgorithm}{\arabic{algorithm}}

\usepackage{amsmath}
\usepackage{amssymb}
\usepackage{mathtools}
\usepackage{amsthm} 


\usepackage[capitalize,noabbrev]{cleveref}

\theoremstyle{plain}
\newtheorem{theorem}{Theorem}[section]
\newtheorem{proposition}[theorem]{Proposition}
\newtheorem{lemma}[theorem]{Lemma}
\newtheorem{corollary}[theorem]{Corollary}
\theoremstyle{definition}
\newtheorem{definition}[theorem]{Definition}
\newtheorem{assumption}[theorem]{Assumption}
\theoremstyle{remark}
\newtheorem{remark}[theorem]{Remark}

\Crefname{assumption}{Assumption}{Assumptions}

\usepackage[textsize=tiny]{todonotes}


%
\setlength\unitlength{1mm}
\newcommand{\twodots}{\mathinner {\ldotp \ldotp}}
% bb font symbols
\newcommand{\Rho}{\mathrm{P}}
\newcommand{\Tau}{\mathrm{T}}

\newfont{\bbb}{msbm10 scaled 700}
\newcommand{\CCC}{\mbox{\bbb C}}

\newfont{\bb}{msbm10 scaled 1100}
\newcommand{\CC}{\mbox{\bb C}}
\newcommand{\PP}{\mbox{\bb P}}
\newcommand{\RR}{\mbox{\bb R}}
\newcommand{\QQ}{\mbox{\bb Q}}
\newcommand{\ZZ}{\mbox{\bb Z}}
\newcommand{\FF}{\mbox{\bb F}}
\newcommand{\GG}{\mbox{\bb G}}
\newcommand{\EE}{\mbox{\bb E}}
\newcommand{\NN}{\mbox{\bb N}}
\newcommand{\KK}{\mbox{\bb K}}
\newcommand{\HH}{\mbox{\bb H}}
\newcommand{\SSS}{\mbox{\bb S}}
\newcommand{\UU}{\mbox{\bb U}}
\newcommand{\VV}{\mbox{\bb V}}


\newcommand{\yy}{\mathbbm{y}}
\newcommand{\xx}{\mathbbm{x}}
\newcommand{\zz}{\mathbbm{z}}
\newcommand{\sss}{\mathbbm{s}}
\newcommand{\rr}{\mathbbm{r}}
\newcommand{\pp}{\mathbbm{p}}
\newcommand{\qq}{\mathbbm{q}}
\newcommand{\ww}{\mathbbm{w}}
\newcommand{\hh}{\mathbbm{h}}
\newcommand{\vvv}{\mathbbm{v}}

% Vectors

\newcommand{\av}{{\bf a}}
\newcommand{\bv}{{\bf b}}
\newcommand{\cv}{{\bf c}}
\newcommand{\dv}{{\bf d}}
\newcommand{\ev}{{\bf e}}
\newcommand{\fv}{{\bf f}}
\newcommand{\gv}{{\bf g}}
\newcommand{\hv}{{\bf h}}
\newcommand{\iv}{{\bf i}}
\newcommand{\jv}{{\bf j}}
\newcommand{\kv}{{\bf k}}
\newcommand{\lv}{{\bf l}}
\newcommand{\mv}{{\bf m}}
\newcommand{\nv}{{\bf n}}
\newcommand{\ov}{{\bf o}}
\newcommand{\pv}{{\bf p}}
\newcommand{\qv}{{\bf q}}
\newcommand{\rv}{{\bf r}}
\newcommand{\sv}{{\bf s}}
\newcommand{\tv}{{\bf t}}
\newcommand{\uv}{{\bf u}}
\newcommand{\wv}{{\bf w}}
\newcommand{\vv}{{\bf v}}
\newcommand{\xv}{{\bf x}}
\newcommand{\yv}{{\bf y}}
\newcommand{\zv}{{\bf z}}
\newcommand{\zerov}{{\bf 0}}
\newcommand{\onev}{{\bf 1}}

% Matrices

\newcommand{\Am}{{\bf A}}
\newcommand{\Bm}{{\bf B}}
\newcommand{\Cm}{{\bf C}}
\newcommand{\Dm}{{\bf D}}
\newcommand{\Em}{{\bf E}}
\newcommand{\Fm}{{\bf F}}
\newcommand{\Gm}{{\bf G}}
\newcommand{\Hm}{{\bf H}}
\newcommand{\Id}{{\bf I}}
\newcommand{\Jm}{{\bf J}}
\newcommand{\Km}{{\bf K}}
\newcommand{\Lm}{{\bf L}}
\newcommand{\Mm}{{\bf M}}
\newcommand{\Nm}{{\bf N}}
\newcommand{\Om}{{\bf O}}
\newcommand{\Pm}{{\bf P}}
\newcommand{\Qm}{{\bf Q}}
\newcommand{\Rm}{{\bf R}}
\newcommand{\Sm}{{\bf S}}
\newcommand{\Tm}{{\bf T}}
\newcommand{\Um}{{\bf U}}
\newcommand{\Wm}{{\bf W}}
\newcommand{\Vm}{{\bf V}}
\newcommand{\Xm}{{\bf X}}
\newcommand{\Ym}{{\bf Y}}
\newcommand{\Zm}{{\bf Z}}

% Calligraphic

\newcommand{\Ac}{{\cal A}}
\newcommand{\Bc}{{\cal B}}
\newcommand{\Cc}{{\cal C}}
\newcommand{\Dc}{{\cal D}}
\newcommand{\Ec}{{\cal E}}
\newcommand{\Fc}{{\cal F}}
\newcommand{\Gc}{{\cal G}}
\newcommand{\Hc}{{\cal H}}
\newcommand{\Ic}{{\cal I}}
\newcommand{\Jc}{{\cal J}}
\newcommand{\Kc}{{\cal K}}
\newcommand{\Lc}{{\cal L}}
\newcommand{\Mc}{{\cal M}}
\newcommand{\Nc}{{\cal N}}
\newcommand{\nc}{{\cal n}}
\newcommand{\Oc}{{\cal O}}
\newcommand{\Pc}{{\cal P}}
\newcommand{\Qc}{{\cal Q}}
\newcommand{\Rc}{{\cal R}}
\newcommand{\Sc}{{\cal S}}
\newcommand{\Tc}{{\cal T}}
\newcommand{\Uc}{{\cal U}}
\newcommand{\Wc}{{\cal W}}
\newcommand{\Vc}{{\cal V}}
\newcommand{\Xc}{{\cal X}}
\newcommand{\Yc}{{\cal Y}}
\newcommand{\Zc}{{\cal Z}}

% Bold greek letters

\newcommand{\alphav}{\hbox{\boldmath$\alpha$}}
\newcommand{\betav}{\hbox{\boldmath$\beta$}}
\newcommand{\gammav}{\hbox{\boldmath$\gamma$}}
\newcommand{\deltav}{\hbox{\boldmath$\delta$}}
\newcommand{\etav}{\hbox{\boldmath$\eta$}}
\newcommand{\lambdav}{\hbox{\boldmath$\lambda$}}
\newcommand{\epsilonv}{\hbox{\boldmath$\epsilon$}}
\newcommand{\nuv}{\hbox{\boldmath$\nu$}}
\newcommand{\muv}{\hbox{\boldmath$\mu$}}
\newcommand{\zetav}{\hbox{\boldmath$\zeta$}}
\newcommand{\phiv}{\hbox{\boldmath$\phi$}}
\newcommand{\psiv}{\hbox{\boldmath$\psi$}}
\newcommand{\thetav}{\hbox{\boldmath$\theta$}}
\newcommand{\tauv}{\hbox{\boldmath$\tau$}}
\newcommand{\omegav}{\hbox{\boldmath$\omega$}}
\newcommand{\xiv}{\hbox{\boldmath$\xi$}}
\newcommand{\sigmav}{\hbox{\boldmath$\sigma$}}
\newcommand{\piv}{\hbox{\boldmath$\pi$}}
\newcommand{\rhov}{\hbox{\boldmath$\rho$}}
\newcommand{\upsilonv}{\hbox{\boldmath$\upsilon$}}

\newcommand{\Gammam}{\hbox{\boldmath$\Gamma$}}
\newcommand{\Lambdam}{\hbox{\boldmath$\Lambda$}}
\newcommand{\Deltam}{\hbox{\boldmath$\Delta$}}
\newcommand{\Sigmam}{\hbox{\boldmath$\Sigma$}}
\newcommand{\Phim}{\hbox{\boldmath$\Phi$}}
\newcommand{\Pim}{\hbox{\boldmath$\Pi$}}
\newcommand{\Psim}{\hbox{\boldmath$\Psi$}}
\newcommand{\Thetam}{\hbox{\boldmath$\Theta$}}
\newcommand{\Omegam}{\hbox{\boldmath$\Omega$}}
\newcommand{\Xim}{\hbox{\boldmath$\Xi$}}


% Sans Serif small case

\newcommand{\Gsf}{{\sf G}}

\newcommand{\asf}{{\sf a}}
\newcommand{\bsf}{{\sf b}}
\newcommand{\csf}{{\sf c}}
\newcommand{\dsf}{{\sf d}}
\newcommand{\esf}{{\sf e}}
\newcommand{\fsf}{{\sf f}}
\newcommand{\gsf}{{\sf g}}
\newcommand{\hsf}{{\sf h}}
\newcommand{\isf}{{\sf i}}
\newcommand{\jsf}{{\sf j}}
\newcommand{\ksf}{{\sf k}}
\newcommand{\lsf}{{\sf l}}
\newcommand{\msf}{{\sf m}}
\newcommand{\nsf}{{\sf n}}
\newcommand{\osf}{{\sf o}}
\newcommand{\psf}{{\sf p}}
\newcommand{\qsf}{{\sf q}}
\newcommand{\rsf}{{\sf r}}
\newcommand{\ssf}{{\sf s}}
\newcommand{\tsf}{{\sf t}}
\newcommand{\usf}{{\sf u}}
\newcommand{\wsf}{{\sf w}}
\newcommand{\vsf}{{\sf v}}
\newcommand{\xsf}{{\sf x}}
\newcommand{\ysf}{{\sf y}}
\newcommand{\zsf}{{\sf z}}


% mixed symbols

\newcommand{\sinc}{{\hbox{sinc}}}
\newcommand{\diag}{{\hbox{diag}}}
\renewcommand{\det}{{\hbox{det}}}
\newcommand{\trace}{{\hbox{tr}}}
\newcommand{\sign}{{\hbox{sign}}}
\renewcommand{\arg}{{\hbox{arg}}}
\newcommand{\var}{{\hbox{var}}}
\newcommand{\cov}{{\hbox{cov}}}
\newcommand{\Ei}{{\rm E}_{\rm i}}
\renewcommand{\Re}{{\rm Re}}
\renewcommand{\Im}{{\rm Im}}
\newcommand{\eqdef}{\stackrel{\Delta}{=}}
\newcommand{\defines}{{\,\,\stackrel{\scriptscriptstyle \bigtriangleup}{=}\,\,}}
\newcommand{\<}{\left\langle}
\renewcommand{\>}{\right\rangle}
\newcommand{\herm}{{\sf H}}
\newcommand{\trasp}{{\sf T}}
\newcommand{\transp}{{\sf T}}
\renewcommand{\vec}{{\rm vec}}
\newcommand{\Psf}{{\sf P}}
\newcommand{\SINR}{{\sf SINR}}
\newcommand{\SNR}{{\sf SNR}}
\newcommand{\MMSE}{{\sf MMSE}}
\newcommand{\REF}{{\RED [REF]}}

% Markov chain
\usepackage{stmaryrd} % for \mkv 
\newcommand{\mkv}{-\!\!\!\!\minuso\!\!\!\!-}

% Colors

\newcommand{\RED}{\color[rgb]{1.00,0.10,0.10}}
\newcommand{\BLUE}{\color[rgb]{0,0,0.90}}
\newcommand{\GREEN}{\color[rgb]{0,0.80,0.20}}

%%%%%%%%%%%%%%%%%%%%%%%%%%%%%%%%%%%%%%%%%%
\usepackage{hyperref}
\hypersetup{
    bookmarks=true,         % show bookmarks bar?
    unicode=false,          % non-Latin characters in AcrobatÕs bookmarks
    pdftoolbar=true,        % show AcrobatÕs toolbar?
    pdfmenubar=true,        % show AcrobatÕs menu?
    pdffitwindow=false,     % window fit to page when opened
    pdfstartview={FitH},    % fits the width of the page to the window
%    pdftitle={My title},    % title
%    pdfauthor={Author},     % author
%    pdfsubject={Subject},   % subject of the document
%    pdfcreator={Creator},   % creator of the document
%    pdfproducer={Producer}, % producer of the document
%    pdfkeywords={keyword1} {key2} {key3}, % list of keywords
    pdfnewwindow=true,      % links in new window
    colorlinks=true,       % false: boxed links; true: colored links
    linkcolor=red,          % color of internal links (change box color with linkbordercolor)
    citecolor=green,        % color of links to bibliography
    filecolor=blue,      % color of file links
    urlcolor=blue           % color of external links
}
%%%%%%%%%%%%%%%%%%%%%%%%%%%%%%%%%%%%%%%%%%%



\usepackage[margin=1.1in]{geometry}
\captionsetup{labelfont=bf}
\usepackage{authblk}

\title{Solving Inverse Problems with Deep Linear Neural Networks: Global Convergence Guarantees for Gradient Descent with Weight Decay}
\author[1,2]{Hannah Laus\thanks{Corresponding author. Email: \href{mailto:hannah.laus@tum.de}{\texttt{hannah.laus@tum.de}}}}
\author[3]{Suzanna Parkinson}
\author[4]{Vasileios Charisopoulos}
\author[1,2,5]{Felix Krahmer}
\author[3,4,6,7,8,9]{Rebecca Willett}
\affil[1]{Department of Mathematics, Technical University of Munich}
\affil[2]{Munich Center for Machine Learning (MCML)}
\affil[3]{Committee on Computational and Applied Mathematics, University of Chicago}
\affil[4]{Data Science Institute, University of Chicago}
\affil[5]{Munich Data Science Institute (MDSI), Technical University of Munich}
\affil[6]{Department of Computer Science, University of Chicago}
\affil[7]{Department of Statistics, University of Chicago}
\affil[8]{NSF-Simons National Institute for Theory and Mathematics in Biology (NITMB)}
\affil[9]{NSF-Simons National Institute for AI in the Sky (SkAI)}
\renewcommand{\Affilfont}{\fontsize{9}{10.8}\footnotesize}
\date{}

\allowdisplaybreaks

\begin{document}


\maketitle

\begin{abstract}


	Machine learning methods are commonly used to solve inverse problems, wherein an unknown signal must be estimated from few measurements generated via a known acquisition procedure. In particular, neural networks perform well empirically but have limited theoretical guarantees. In this work, we study an underdetermined linear inverse problem that admits several possible solution mappings. A standard remedy (e.g., in compressed sensing) establishing uniqueness of the solution mapping is to assume knowledge of latent low-dimensional structure in the source signal. We ask the following question: do deep neural networks adapt to this low-dimensional structure when trained by gradient descent with weight decay regularization? We prove that mildly overparameterized deep linear networks trained in this manner converge to an approximate solution that accurately solves the inverse problem while implicitly encoding latent subspace structure. To our knowledge, this is the first result to rigorously show that deep linear networks trained with weight decay automatically adapt to latent subspace structure in the data under practical stepsize and weight initialization schemes. Our work highlights that regularization and overparameterization improve generalization, while overparameterization also accelerates convergence during training.


\end{abstract}

\section{Introduction}
\label{sec:submission}
\section{Introduction}
\label{sec:introduction}
The business processes of organizations are experiencing ever-increasing complexity due to the large amount of data, high number of users, and high-tech devices involved \cite{martin2021pmopportunitieschallenges, beerepoot2023biggestbpmproblems}. This complexity may cause business processes to deviate from normal control flow due to unforeseen and disruptive anomalies \cite{adams2023proceddsriftdetection}. These control-flow anomalies manifest as unknown, skipped, and wrongly-ordered activities in the traces of event logs monitored from the execution of business processes \cite{ko2023adsystematicreview}. For the sake of clarity, let us consider an illustrative example of such anomalies. Figure \ref{FP_ANOMALIES} shows a so-called event log footprint, which captures the control flow relations of four activities of a hypothetical event log. In particular, this footprint captures the control-flow relations between activities \texttt{a}, \texttt{b}, \texttt{c} and \texttt{d}. These are the causal ($\rightarrow$) relation, concurrent ($\parallel$) relation, and other ($\#$) relations such as exclusivity or non-local dependency \cite{aalst2022pmhandbook}. In addition, on the right are six traces, of which five exhibit skipped, wrongly-ordered and unknown control-flow anomalies. For example, $\langle$\texttt{a b d}$\rangle$ has a skipped activity, which is \texttt{c}. Because of this skipped activity, the control-flow relation \texttt{b}$\,\#\,$\texttt{d} is violated, since \texttt{d} directly follows \texttt{b} in the anomalous trace.
\begin{figure}[!t]
\centering
\includegraphics[width=0.9\columnwidth]{images/FP_ANOMALIES.png}
\caption{An example event log footprint with six traces, of which five exhibit control-flow anomalies.}
\label{FP_ANOMALIES}
\end{figure}

\subsection{Control-flow anomaly detection}
Control-flow anomaly detection techniques aim to characterize the normal control flow from event logs and verify whether these deviations occur in new event logs \cite{ko2023adsystematicreview}. To develop control-flow anomaly detection techniques, \revision{process mining} has seen widespread adoption owing to process discovery and \revision{conformance checking}. On the one hand, process discovery is a set of algorithms that encode control-flow relations as a set of model elements and constraints according to a given modeling formalism \cite{aalst2022pmhandbook}; hereafter, we refer to the Petri net, a widespread modeling formalism. On the other hand, \revision{conformance checking} is an explainable set of algorithms that allows linking any deviations with the reference Petri net and providing the fitness measure, namely a measure of how much the Petri net fits the new event log \cite{aalst2022pmhandbook}. Many control-flow anomaly detection techniques based on \revision{conformance checking} (hereafter, \revision{conformance checking}-based techniques) use the fitness measure to determine whether an event log is anomalous \cite{bezerra2009pmad, bezerra2013adlogspais, myers2018icsadpm, pecchia2020applicationfailuresanalysispm}. 

The scientific literature also includes many \revision{conformance checking}-independent techniques for control-flow anomaly detection that combine specific types of trace encodings with machine/deep learning \cite{ko2023adsystematicreview, tavares2023pmtraceencoding}. Whereas these techniques are very effective, their explainability is challenging due to both the type of trace encoding employed and the machine/deep learning model used \cite{rawal2022trustworthyaiadvances,li2023explainablead}. Hence, in the following, we focus on the shortcomings of \revision{conformance checking}-based techniques to investigate whether it is possible to support the development of competitive control-flow anomaly detection techniques while maintaining the explainable nature of \revision{conformance checking}.
\begin{figure}[!t]
\centering
\includegraphics[width=\columnwidth]{images/HIGH_LEVEL_VIEW.png}
\caption{A high-level view of the proposed framework for combining \revision{process mining}-based feature extraction with dimensionality reduction for control-flow anomaly detection.}
\label{HIGH_LEVEL_VIEW}
\end{figure}

\subsection{Shortcomings of \revision{conformance checking}-based techniques}
Unfortunately, the detection effectiveness of \revision{conformance checking}-based techniques is affected by noisy data and low-quality Petri nets, which may be due to human errors in the modeling process or representational bias of process discovery algorithms \cite{bezerra2013adlogspais, pecchia2020applicationfailuresanalysispm, aalst2016pm}. Specifically, on the one hand, noisy data may introduce infrequent and deceptive control-flow relations that may result in inconsistent fitness measures, whereas, on the other hand, checking event logs against a low-quality Petri net could lead to an unreliable distribution of fitness measures. Nonetheless, such Petri nets can still be used as references to obtain insightful information for \revision{process mining}-based feature extraction, supporting the development of competitive and explainable \revision{conformance checking}-based techniques for control-flow anomaly detection despite the problems above. For example, a few works outline that token-based \revision{conformance checking} can be used for \revision{process mining}-based feature extraction to build tabular data and develop effective \revision{conformance checking}-based techniques for control-flow anomaly detection \cite{singh2022lapmsh, debenedictis2023dtadiiot}. However, to the best of our knowledge, the scientific literature lacks a structured proposal for \revision{process mining}-based feature extraction using the state-of-the-art \revision{conformance checking} variant, namely alignment-based \revision{conformance checking}.

\subsection{Contributions}
We propose a novel \revision{process mining}-based feature extraction approach with alignment-based \revision{conformance checking}. This variant aligns the deviating control flow with a reference Petri net; the resulting alignment can be inspected to extract additional statistics such as the number of times a given activity caused mismatches \cite{aalst2022pmhandbook}. We integrate this approach into a flexible and explainable framework for developing techniques for control-flow anomaly detection. The framework combines \revision{process mining}-based feature extraction and dimensionality reduction to handle high-dimensional feature sets, achieve detection effectiveness, and support explainability. Notably, in addition to our proposed \revision{process mining}-based feature extraction approach, the framework allows employing other approaches, enabling a fair comparison of multiple \revision{conformance checking}-based and \revision{conformance checking}-independent techniques for control-flow anomaly detection. Figure \ref{HIGH_LEVEL_VIEW} shows a high-level view of the framework. Business processes are monitored, and event logs obtained from the database of information systems. Subsequently, \revision{process mining}-based feature extraction is applied to these event logs and tabular data input to dimensionality reduction to identify control-flow anomalies. We apply several \revision{conformance checking}-based and \revision{conformance checking}-independent framework techniques to publicly available datasets, simulated data of a case study from railways, and real-world data of a case study from healthcare. We show that the framework techniques implementing our approach outperform the baseline \revision{conformance checking}-based techniques while maintaining the explainable nature of \revision{conformance checking}.

In summary, the contributions of this paper are as follows.
\begin{itemize}
    \item{
        A novel \revision{process mining}-based feature extraction approach to support the development of competitive and explainable \revision{conformance checking}-based techniques for control-flow anomaly detection.
    }
    \item{
        A flexible and explainable framework for developing techniques for control-flow anomaly detection using \revision{process mining}-based feature extraction and dimensionality reduction.
    }
    \item{
        Application to synthetic and real-world datasets of several \revision{conformance checking}-based and \revision{conformance checking}-independent framework techniques, evaluating their detection effectiveness and explainability.
    }
\end{itemize}

The rest of the paper is organized as follows.
\begin{itemize}
    \item Section \ref{sec:related_work} reviews the existing techniques for control-flow anomaly detection, categorizing them into \revision{conformance checking}-based and \revision{conformance checking}-independent techniques.
    \item Section \ref{sec:abccfe} provides the preliminaries of \revision{process mining} to establish the notation used throughout the paper, and delves into the details of the proposed \revision{process mining}-based feature extraction approach with alignment-based \revision{conformance checking}.
    \item Section \ref{sec:framework} describes the framework for developing \revision{conformance checking}-based and \revision{conformance checking}-independent techniques for control-flow anomaly detection that combine \revision{process mining}-based feature extraction and dimensionality reduction.
    \item Section \ref{sec:evaluation} presents the experiments conducted with multiple framework and baseline techniques using data from publicly available datasets and case studies.
    \item Section \ref{sec:conclusions} draws the conclusions and presents future work.
\end{itemize}

\subsection{Related work}
\label{sec:subsec:related-work}

%\section{Related Work}
%\label{sec:related-work}

%\subsection{Background}

%Defect detection is critical to ensure the yield of integrated circuit manufacturing lines and reduce faults. Previous research has primarily focused on wafer map data, which engineers produce by marking faulty chips with different colors based on test results. The specific spatial distribution of defects on a wafer can provide insights into the causes, thereby helping to determine which stage of the manufacturing process is responsible for the issues. Although such research is relatively mature, the continual miniaturization of integrated circuits and the increasing complexity and density of chip components have made chip-level detection more challenging, leading to potential risks\cite{ma2023review}. Consequently, there is a need to combine this approach with magnified imaging of the wafer surface using scanning electron microscopes (SEMs) to detect, classify, and analyze specific microscopic defects, thus helping to identify the particular process steps where defects originate.

%Previously, wafer surface defect classification and detection were primarily conducted by experienced engineers. However, this method relies heavily on the engineers' expertise and involves significant time expenditure and subjectivity, lacking uniform standards. With the ongoing development of artificial intelligence, deep learning methods using multi-layer neural networks to extract and learn target features have proven highly effective for this task\cite{gao2022review}.

%In the task of defect classification, it is typical to use a model structure that initially extracts features through convolutional and pooling layers, followed by classification via fully connected layers. Researchers have recently developed numerous classification model structures tailored to specific problems. These models primarily focus on how to extract defect features effectively. For instance, Chen et al. presented a defect recognition and classification algorithm rooted in PCA and classification SVM\cite{chen2008defect}. Chang et al. utilized SVM, drawing on features like smoothness and texture intricacy, for classifying high-intensity defect images\cite{chang2013hybrid}. The classification of defect images requires the formulation of numerous classifiers tailored for myriad inspection steps and an Abundance of accurately labeled data, making data acquisition challenging. Cheon et al. proposed a single CNN model adept at feature extraction\cite{cheon2019convolutional}. They achieved a granular classification of wafer surface defects by recognizing misclassified images and employing a k-nearest neighbors (k-NN) classifier algorithm to gauge the aggregate squared distance between each image feature vector and its k-neighbors within the same category. However, when applied to new or unseen defects, such models necessitate retraining, incurring computational overheads. Moreover, with escalating CNN complexity, the computational demands surge.

%Segmentation of defects is necessary to locate defect positions and gather information such as the size of defects. Unlike classification networks, segmentation networks often use classic encoder-decoder structures such as UNet\cite{ronneberger2015u} and SegNet\cite{badrinarayanan2017segnet}, which focus on effectively leveraging both local and global feature information. Han Hui et al. proposed integrating a Region Proposal Network (RPN) with a UNet architecture to suggest defect areas before conducting defect segmentation \cite{han2020polycrystalline}. This approach enables the segmentation of various defects in wafers with only a limited set of roughly labeled images, enhancing the efficiency of training and application in environments where detailed annotations are scarce. Subhrajit Nag et al. introduced a new network structure, WaferSegClassNet, which extracts multi-scale local features in the encoder and performs classification and segmentation tasks in the decoder \cite{nag2022wafersegclassnet}. This model represents the first detection system capable of simultaneously classifying and segmenting surface defects on wafers. However, it relies on extensive data training and annotation for high accuracy and reliability. 

%Recently, Vic De Ridder et al. introduced a novel approach for defect segmentation using diffusion models\cite{de2023semi}. This approach treats the instance segmentation task as a denoising process from noise to a filter, utilizing diffusion models to predict and reconstruct instance masks for semiconductor defects. This method achieves high precision and improved defect classification and segmentation detection performance. However, the complex network structure and the computational process of the diffusion model require substantial computational resources. Moreover, the performance of this model heavily relies on high-quality and large amounts of training data. These issues make it less suitable for industrial applications. Additionally, the model has only been applied to detecting and segmenting a single type of defect(bridges) following a specific manufacturing process step, limiting its practical utility in diverse industrial scenarios.

%\subsection{Few-shot Anomaly Detection}
%Traditional anomaly detection techniques typically rely on extensive training data to train models for identifying and locating anomalies. However, these methods often face limitations in rapidly changing production environments and diverse anomaly types. Recent research has started exploring effective anomaly detection using few or zero samples to address these challenges.

%Huang et al. developed the anomaly detection method RegAD, based on image registration technology. This method pre-trains an object-agnostic registration network with various images to establish the normality of unseen objects. It identifies anomalies by aligning image features and has achieved promising results. Despite these advancements, implementing few-shot settings in anomaly detection remains an area ripe for further exploration. Recent studies show that pre-trained vision-language models such as CLIP and MiniGPT can significantly enhance performance in anomaly detection tasks.

%Dong et al. introduced the MaskCLIP framework, which employs masked self-distillation to enhance contrastive language-image pretraining\cite{zhou2022maskclip}. This approach strengthens the visual encoder's learning of local image patches and uses indirect language supervision to enhance semantic understanding. It significantly improves transferability and pretraining outcomes across various visual tasks, although it requires substantial computational resources.
%Jeong et al. crafted the WinCLIP framework by integrating state words and prompt templates to characterize normal and anomalous states more accurately\cite{Jeong_2023_CVPR}. This framework introduces a novel window-based technique for extracting and aggregating multi-scale spatial features, significantly boosting the anomaly detection performance of the pre-trained CLIP model.
%Subsequently, Li et al. have further contributed to the field by creating a new expansive multimodal model named Myriad\cite{li2023myriad}. This model, which incorporates a pre-trained Industrial Anomaly Detection (IAD) model to act as a vision expert, embeds anomaly images as tokens interpretable by the language model, thus providing both detailed descriptions and accurate anomaly detection capabilities.
%Recently, Chen et al. introduced CLIP-AD\cite{chen2023clip}, and Li et al. proposed PromptAD\cite{li2024promptad}, both employing language-guided, tiered dual-path model structures and feature manipulation strategies. These approaches effectively address issues encountered when directly calculating anomaly maps using the CLIP model, such as reversed predictions and highlighting irrelevant areas. Specifically, CLIP-AD optimizes the utilization of multi-layer features, corrects feature misalignment, and enhances model performance through additional linear layer fine-tuning. PromptAD connects normal prompts with anomaly suffixes to form anomaly prompts, enabling contrastive learning in a single-class setting.

%These studies extend the boundaries of traditional anomaly detection techniques and demonstrate how to effectively address rapidly changing and sample-scarce production environments through the synergy of few-shot learning and deep learning models. Building on this foundation, our research further explores wafer surface defect detection based on the CLIP model, especially focusing on achieving efficient and accurate anomaly detection in the highly specialized and variable semiconductor manufacturing process using a minimal amount of labeled data.


\subsection{Notation and basic constructions}
\label{sec:subsec:notation}
% !TeX root = main.tex 


\newcommand{\lnote}{\textcolor[rgb]{1,0,0}{Lydia: }\textcolor[rgb]{0,0,1}}
\newcommand{\todo}{\textcolor[rgb]{1,0,0.5}{To do: }\textcolor[rgb]{0.5,0,1}}


\newcommand{\state}{S}
\newcommand{\meas}{M}
\newcommand{\out}{\mathrm{out}}
\newcommand{\piv}{\mathrm{piv}}
\newcommand{\pivotal}{\mathrm{pivotal}}
\newcommand{\isnot}{\mathrm{not}}
\newcommand{\pred}{^\mathrm{predict}}
\newcommand{\act}{^\mathrm{act}}
\newcommand{\pre}{^\mathrm{pre}}
\newcommand{\post}{^\mathrm{post}}
\newcommand{\calM}{\mathcal{M}}

\newcommand{\game}{\mathbf{V}}
\newcommand{\strategyspace}{S}
\newcommand{\payoff}[1]{V^{#1}}
\newcommand{\eff}[1]{E^{#1}}
\newcommand{\p}{\vect{p}}
\newcommand{\simplex}[1]{\Delta^{#1}}

\newcommand{\recdec}[1]{\bar{D}(\hat{Y}_{#1})}





\newcommand{\sphereone}{\calS^1}
\newcommand{\samplen}{S^n}
\newcommand{\wA}{w}%{w_{\mathfrak{a}}}
\newcommand{\Awa}{A_{\wA}}
\newcommand{\Ytil}{\widetilde{Y}}
\newcommand{\Xtil}{\widetilde{X}}
\newcommand{\wst}{w_*}
\newcommand{\wls}{\widehat{w}_{\mathrm{LS}}}
\newcommand{\dec}{^\mathrm{dec}}
\newcommand{\sub}{^\mathrm{sub}}

\newcommand{\calP}{\mathcal{P}}
\newcommand{\totspace}{\calZ}
\newcommand{\clspace}{\calX}
\newcommand{\attspace}{\calA}

\newcommand{\Ftil}{\widetilde{\calF}}

\newcommand{\totx}{Z}
\newcommand{\classx}{X}
\newcommand{\attx}{A}
\newcommand{\calL}{\mathcal{L}}



\newcommand{\defeq}{\mathrel{\mathop:}=}
\newcommand{\vect}[1]{\ensuremath{\mathbf{#1}}}
\newcommand{\mat}[1]{\ensuremath{\mathbf{#1}}}
\newcommand{\dd}{\mathrm{d}}
\newcommand{\grad}{\nabla}
\newcommand{\hess}{\nabla^2}
\newcommand{\argmin}{\mathop{\rm argmin}}
\newcommand{\argmax}{\mathop{\rm argmax}}
\newcommand{\Ind}[1]{\mathbf{1}\{#1\}}

\newcommand{\norm}[1]{\left\|{#1}\right\|}
\newcommand{\fnorm}[1]{\|{#1}\|_{\text{F}}}
\newcommand{\spnorm}[2]{\left\| {#1} \right\|_{\text{S}({#2})}}
\newcommand{\sigmin}{\sigma_{\min}}
\newcommand{\tr}{\text{tr}}
\renewcommand{\det}{\text{det}}
\newcommand{\rank}{\text{rank}}
\newcommand{\logdet}{\text{logdet}}
\newcommand{\trans}{^{\top}}
\newcommand{\poly}{\text{poly}}
\newcommand{\polylog}{\text{polylog}}
\newcommand{\st}{\text{s.t.~}}
\newcommand{\proj}{\mathcal{P}}
\newcommand{\projII}{\mathcal{P}_{\parallel}}
\newcommand{\projT}{\mathcal{P}_{\perp}}
\newcommand{\projX}{\mathcal{P}_{\mathcal{X}^\star}}
\newcommand{\inner}[1]{\langle #1 \rangle}

\renewcommand{\Pr}{\mathbb{P}}
\newcommand{\Z}{\mathbb{Z}}
\newcommand{\N}{\mathbb{N}}
\newcommand{\R}{\mathbb{R}}
\newcommand{\E}{\mathbb{E}}
\newcommand{\F}{\mathcal{F}}
\newcommand{\var}{\mathrm{var}}
\newcommand{\cov}{\mathrm{cov}}


\newcommand{\calN}{\mathcal{N}}

\newcommand{\jccomment}{\textcolor[rgb]{1,0,0}{C: }\textcolor[rgb]{1,0,1}}
\newcommand{\fracpar}[2]{\frac{\partial #1}{\partial  #2}}

\newcommand{\A}{\mathcal{A}}
\newcommand{\B}{\mat{B}}
%\newcommand{\C}{\mat{C}}

\newcommand{\I}{\mat{I}}
\newcommand{\M}{\mat{M}}
\newcommand{\D}{\mat{D}}
%\newcommand{\U}{\mat{U}}
\newcommand{\V}{\mat{V}}
\newcommand{\W}{\mat{W}}
\newcommand{\X}{\mat{X}}
\newcommand{\Y}{\mat{Y}}
\newcommand{\mSigma}{\mat{\Sigma}}
\newcommand{\mLambda}{\mat{\Lambda}}
\newcommand{\e}{\vect{e}}
\newcommand{\g}{\vect{g}}
\renewcommand{\u}{\vect{u}}
\newcommand{\w}{\vect{w}}
\newcommand{\x}{\vect{x}}
\newcommand{\y}{\vect{y}}
\newcommand{\z}{\vect{z}}
\newcommand{\fI}{\mathfrak{I}}
\newcommand{\fS}{\mathfrak{S}}
\newcommand{\fE}{\mathfrak{E}}
\newcommand{\fF}{\mathfrak{F}}

\newcommand{\Risk}{\mathcal{R}}

\renewcommand{\L}{\mathcal{L}}
\renewcommand{\H}{\mathcal{H}}

\newcommand{\cn}{\kappa}
\newcommand{\nn}{\nonumber}


\newcommand{\Hess}{\nabla^2}
\newcommand{\tlO}{\tilde{O}}
\newcommand{\tlOmega}{\tilde{\Omega}}

\newcommand{\calF}{\mathcal{F}}
\newcommand{\fhat}{\widehat{f}}
\newcommand{\calS}{\mathcal{S}}

\newcommand{\calX}{\mathcal{X}}
\newcommand{\calY}{\mathcal{Y}}
\newcommand{\calD}{\mathcal{D}}
\newcommand{\calZ}{\mathcal{Z}}
\newcommand{\calA}{\mathcal{A}}
\newcommand{\fbayes}{f^B}
\newcommand{\func}{f^U}


\newcommand{\bayscore}{\text{calibrated Bayes score}}
\newcommand{\bayrisk}{\text{calibrated Bayes risk}}

\newtheorem{example}{Example}[section]
\newtheorem{exc}{Exercise}[section]
%\newtheorem{rem}{Remark}[section]

\newtheorem{theorem}{Theorem}[section]
\newtheorem{definition}{Definition}
\newtheorem{proposition}[theorem]{Proposition}
\newtheorem{corollary}[theorem]{Corollary}

\newtheorem{remark}{Remark}[section]
\newtheorem{lemma}[theorem]{Lemma}
\newtheorem{claim}[theorem]{Claim}
\newtheorem{fact}[theorem]{Fact}
\newtheorem{assumption}{Assumption}

\newcommand{\iidsim}{\overset{\mathrm{i.i.d.}}{\sim}}
\newcommand{\unifsim}{\overset{\mathrm{unif}}{\sim}}
\newcommand{\sign}{\mathrm{sign}}
\newcommand{\wbar}{\overline{w}}
\newcommand{\what}{\widehat{w}}
\newcommand{\KL}{\mathrm{KL}}
\newcommand{\Bern}{\mathrm{Bernoulli}}
\newcommand{\ihat}{\widehat{i}}
\newcommand{\Dwst}{\calD^{w_*}}
\newcommand{\fls}{\widehat{f}_{n}}


\newcommand{\brpi}{\pi^{br}}
\newcommand{\brtheta}{\theta^{br}}

% \newcommand{\M}{\mat{M}}
% \newcommand\Mmh{\mat{M}^{-1/2}}
% \newcommand{\A}{\mat{A}}
% \newcommand{\B}{\mat{B}}
% \newcommand{\C}{\mat{C}}
% \newcommand{\Et}[1][t]{\mat{E_{#1}}}
% \newcommand{\Etp}{\Et[t+1]}
% \newcommand{\Errt}[1][t]{\mat{\bigtriangleup_{#1}}}
% \newcommand\cnM{\kappa}
% \newcommand{\cn}[1]{\kappa\left(#1\right)}
% \newcommand\X{\mat{X}}
% \newcommand\fstar{f_*}
% \newcommand\Xt[1][t]{\mat{X_{#1}}}
% \newcommand\ut[1][t]{{u_{#1}}}
% \newcommand\Xtinv{\inv{\Xt}}
% \newcommand\Xtp{\mat{X_{t+1}}}
% \newcommand\Xtpinv{\inv{\left(\mat{X_{t+1}}\right)}}
% \newcommand\U{\mat{U}}
% \newcommand\UTr{\trans{\mat{U}}}
% \newcommand{\Ut}[1][t]{\mat{U_{#1}}}
% \newcommand{\Utinv}{\inv{\Ut}}
% \newcommand{\UtTr}[1][t]{\trans{\mat{U_{#1}}}}
% \newcommand\Utp{\mat{U_{t+1}}}
% \newcommand\UtpTr{\trans{\mat{U}_{t+1}}}
% \newcommand\Utptild{\mat{\widetilde{U}_{t+1}}}
% \newcommand\Us{\mat{U^*}}
% \newcommand\UsTr{\trans{\mat{U^*}}}
% \newcommand{\Sigs}{\mat{\Sigma}}
% \newcommand{\Sigsmh}{\Sigs^{-1/2}}
% \newcommand{\eye}{\mat{I}}
% \newcommand{\twonormbound}{\left(4+\DPhi{\M}{\Xt[0]}\right)\twonorm{\M}}
% \newcommand{\lamj}{\lambda_j}

% \renewcommand\u{\vect{u}}
% \newcommand\uTr{\trans{\vect{u}}}
% \renewcommand\v{\vect{v}}
% \newcommand\vTr{\trans{\vect{v}}}
% \newcommand\w{\vect{w}}
% \newcommand\wTr{\trans{\vect{w}}}
% \newcommand\wperp{\vect{w}_{\perp}}
% \newcommand\wperpTr{\trans{\vect{w}_{\perp}}}
% \newcommand\wj{\vect{w_j}}
% \newcommand\vj{\vect{v_j}}
% \newcommand\wjTr{\trans{\vect{w_j}}}
% \newcommand\vjTr{\trans{\vect{v_j}}}

% \newcommand{\DPhi}[2]{\ensuremath{D_{\Phi}\left(#1,#2\right)}}
% \newcommand\matmult{{\omega}}


\section{Main result}
\label{sec:main-result}
In this section, we present our main result as well as a proof sketch focusing
on the depth $L = 2$ case. Recall that we are interested in solving~\eqref{eq:l2regprob1},
for the special case where $f_{W_{L:1}}$ is a deep linear network, using gradient
descent (\cref{alg:gradient-descent}). Concretely,
we want to minimize the following loss function:
\begin{equation}
	\cL(\set{W_{\ell}}_{\ell = 1, \dots, L}; (X, Y)) :=
	\frac{1}{2} \frobnorm{W_{L} \cdots W_{1} Y - X}^2 + \frac{\lambda}{2} \sum_{\ell=1}^{L} \frobnorm{W_{\ell}}^2.
	\label{eq:loss-function}
\end{equation}
\begin{algorithm}[tb]
	\caption{Gradient descent}
	\label{alg:gradient-descent}
	\begin{algorithmic}
		\State \textbf{Input}: data $X$, $Y$, step-size $\eta > 0$, iterations $T$.
		\State \textbf{Initialize} weights $\set{W_{\ell}(0)}_{\ell=1}^{L}$.
		\For{$t = 0, 1, \dots, T-1$}
		\State $W_{\ell}(t+1) = W_{\ell}(t) - \eta \grad \cL(\set{W_{\ell}(t)}_{\ell=1}^{L}; (X, Y))$
		\EndFor
		\State \textbf{return} $\set{W_{\ell}(T)}_{\ell=1}^{L}$.
	\end{algorithmic}
\end{algorithm}
We consider weight matrices of the following sizes:
\begin{itemize}[itemsep=0ex]
	\item The weight matrix of the input layer $W_{1} \in \Rbb^{\dhid \times \din}$,
	      where $\dhid$ is a width common to all hidden layers.
	\item The weight matrix of the output layer $W_{L} \in \Rbb^{\dout \times \dhid}$.
	\item All other weight matrices $W_{2}, \dots, W_{L-1} \in \Rbb^{\dhid \times \dhid}$.
\end{itemize}
We will also write $W_{j:i}(t)$ for the following product of weight matrices at the $t^{\text{th}}$ iteration:
\begin{equation}
	W_{j:i}(t) := \prod_{\ell = j}^{i} W_{\ell}(t).
	\label{eq:folded-product}
\end{equation}
Having fixed the architecture, we introduce two mild assumptions under
which our results hold.
\begin{assumption}[Restricted Isometry Property]
	\label{assumption:rip}
	The measurement matrix $A$ from~\eqref{eq:dataset-measurements}
	satisfies the following: there exists $\delta > 0$ such that,
	for all vectors $x \in \range(R)$,
	\begin{equation}
		(1 - \delta) \norm{x}^2 \leq \norm{Ax}^2 \leq (1 + \delta) \norm{x}^2.
		\label{eq:rip}
	\end{equation}
\end{assumption}
\Cref{assumption:rip} is standard
in the compressed sensing literature~\cite{foucart2013invitation}, as it is
a sufficient condition that enables the solution of high-dimensional linear inverse
problems from few measurements. In our context,
\cref{assumption:rip} essentially states that the
training data has been sampled from inverse problems that are identifiable.

Our next assumption relates to the network initialization:
\begin{assumption}[Initialization]
	\label{assumption:initialization}
	The weight matrices $W_1, \dots, W_{L}$ at initialization are sampled from a scaled
	(``fan-in'') normal distribution:
	\begin{equation}
		[W_{\ell}(0)]_{ij} \iid \begin{cases}
			\cN\left(0, \frac{1}{\din}\right), & \ell = 1,           \\
			\cN(0, \frac{1}{\dhid}),           & \ell = 2, \dots, L.
		\end{cases}
		\label{eq:fan-in-initialization}
	\end{equation}
\end{assumption}
\Cref{assumption:initialization} is by no means restrictive: it
was introduced by~\cite{HZRS15} as a heuristic for stabilizing neural network training
and enjoys widespread adoption.\footnote{
	See, e.g., the \texttt{torch.nn.init.kaiming\_normal\_} initialization
	method in \texttt{Pytorch}.
}

We now present an informal version of our main result. The formal statement can be found in \cref{thm:mainresult-formal}, and the proof comprises \cref{sec:subsec:prelim,sec:subsec:main result formal,sec:subsec:Lemmas used for the proof,sec:subsec:Properties at initialization,sec:subsec:Step 1: Rapid early convergence,sec:subsec:Step 2: he error stays small,sec:subsec:Step 3: Convergence off the subspace}.
\begin{theorem}[Informal]
	\label{theorem:main-informal}
	Let~\cref{assumption:rip,assumption:initialization} hold and set
	the step size $\eta$ and weight decay parameter $\lambda$ as
	\begin{equation}
		\eta := \sfrac{\din}{L \cdot \sigma_{\max}^2(X)}, \quad
		\lambda := \gamma \sigma_{\min}^2(X) \sqrt{\sfrac{\din}{\dout}},
		\label{eq:informal-theorem-stepsize-and-wd}
	\end{equation}
	where $\gamma \in (0, 1]$ is a user-specified accuracy parameter. Moreover, define
	the times
	\begin{subequations}
		\begin{align}
			\tau & = \inf\set*{
				t \in \mathbb{N} \mid
				\frobnorm{W_{L:1}(t)Y - X} \leq
				\frac{80 \gamma \frobnorm{X}}{L}
			},             \label{eq:tau def}                                                            \\
			T    & = \frac{2 L \kappa^2 \log(\dhid)}{\gamma} \sqrt{\frac{\dout}{\din}}, \label{eq:T def}
		\end{align}
	\end{subequations}
	where $\kappa := \opnorm{X} \opnorm{X^{\dag}}$ denotes the condition number  of $X$.
	Finally, let $\sr(X) := \nicefrac{\frobnorm{X}^2}{\opnorm{X}^2}$ denote
	the \emph{stable rank} of $X$.
	Then, as long as the hidden layer width satisfies
	\[
		\dhid \gtrsim \dout \cdot \sr(X) \cdot \mathrm{poly}(L, \kappa),
	\]
	gradient descent (\cref{alg:gradient-descent})
	produces iterates that satisfy
	\begin{align}
		\frobnorm{W_{L:1}(t+1)Y - X}
		                                          & \leq
		\begin{cases}
			\left(1 - \frac{1}{32 \kappa^2}\right) \frobnorm{W_{L:1}(t)Y - X}, & t < \tau;         \\
			C_1 \gamma \frobnorm{X},                                           & \tau \le t \leq T
		\end{cases} \label{eq:thm-regression-error} \\
		\opnorm{W_{L:1}(T) P_{\range(Y)}^{\perp}} & \leq \left( \frac{1}{\dhid} \right)^{C_{2}},
		\label{eq:thm-generalization-error}
	\end{align}
	where $C_{1}$ and $C_{2}$ are universal positive constants. These guarantees hold with high probability
	over the random initialization.
\end{theorem}
\cref{eq:thm-regression-error} in \cref{theorem:main-informal} demonstrates that the reconstruction of $X$ from $Y$ can be made arbitrarily accurate using a suitably small choice of regularization parameter $\lambda$.
On the other hand, \cref{eq:thm-generalization-error} ensures that the component of the weights acting on the orthogonal complement of the signal subspace can be made small by increasing the hidden width of the model; this ensures robustness to noisy test data as discussed below in \cref{sec:subsec:robustness}.
\cref{theorem:main-informal} also highlights two distinct phases of gradient descent; during the first $\tau$ iterations, \cref{eq:thm-regression-error} suggests that the error in the reconstruction converges linearly up to the threshold specified in~\eqref{eq:tau def}. Upon reaching that threshold, the behavior changes:
while the reconstruction error can increase mildly from
iteration $\tau$ to $T$, the component of the weights acting on the orthogonal complement of the signal subspace shrinks to the level shown in \cref{eq:thm-generalization-error}.
The number of iterations $T$ of gradient descent required to achieve this behavior grows only logarithmically with the hidden width, but is highly sensitive to the targeted reconstruction accuracy -- and therefore the weight decay parameter $\lambda$.

\begin{remark}
	Plugging $\eta$ and $\lambda$ into~\cref{eq:T def} implies that $T = O(\nicefrac{1}{\eta \lambda})$; this is consistent with the
	results of~\citet{lewkowycz2020training,wang2024implicit}. The
	former work observes empirically that SGD without momentum
	attains maximum performance at roughly $O(\nicefrac{1}{\eta \lambda})$ iterations,
	while the latter work~\citep[Theorem B.2]{wang2024implicit}
	suggests that stochastic gradient descent requires a similar number of iterations to find a low-rank solution --- albeit one that might be a poor data fit.
\end{remark}

\begin{remark}
	\label{remark:small-eta}
	\cref{theorem:main-informal} remains valid when the step size $\eta$ is chosen to be smaller than the value specified in the theorem, albeit at the expense of an increased number of iterations $T$.
\end{remark}


\subsection{Proof sketch}
\label{sec:subsec:proof-sketch}
In this section, we provide a proof sketch for~\cref{theorem:main-informal}; full
proofs are deferred to the Appendix. For simplicity, the proof sketch focuses on the case
\[
	L = 2, \;\; \kappa = 1, \;\; \delta = \frac{1}{10}.
\]
Since normalization at initialization (\cref{assumption:initialization})
is essential to the proof, it is convenient to be explicit about normalization factors. We consider
the equivalent loss
\begin{equation}
	\cL(W_1, W_2) := \frac{1}{2} \frobnorm[\Big]{
		\frac{1}{\sqrt{\dhid \din}} W_{2} W_{1} Y - X
	}^2
	+ \frac{\lambda}{2} \left(
	\frac{\frobnorm{W_1}^2}{\din} + \frac{\frobnorm{W_2}^2}{\dhid}
	\right),
	\label{eq:reformulated-loss}
\end{equation}
under the assumption that $(W_1(0))_{ij}, (W_2(0))_{ij} \iid \cN(0, 1)$.
We will also use the shorthand notation
\[
	\Phi(t) := \frac{1}{\sqrt{\dhid \din}} W_{2:1}(t) Y - X.
\]
We refer to $\frobnorm{\Phi(t)}$ as the regression error.
Note that the gradient descent updates lead to the following decomposition:
\begin{align*}
	 & W_{2:1}(t+1)                                                                                                        \\
	 & = \begin{aligned}[t]
		      & \left(1 - \frac{\eta \lambda}{\dhid}\right)\left(1 - \frac{\eta \lambda}{\din}\right) W_{2:1}(t) + E_0(t) \\
		      & - \frac{\eta}{\sqrt{\dhid \din}}
		     \left(1 - \frac{\eta \lambda}{\dhid}\right) W_{2}(t) (W_{2}(t))^{\T} \Phi(t) Y^{\T}                          \\
		      & - \frac{\eta}{\sqrt{\dhid \din}}
		     \left(1 - \frac{\eta \lambda}{\din}\right) \Phi(t) Y^{\T} (W_{1}(t))^{\T} W_{1}(t),
	     \end{aligned}
\end{align*}
where $E_0(t) \in O(\eta^2)$ contains high-order terms. Multiplying both sides from the right by $(1 / \sqrt{\dhid \din}) Y$, subtracting
$X$ and taking norms, we obtain the following bound on the regression error at time $t+1$:
\begin{align}
	\frobnorm{\Phi(t+1)}
	\leq
	\opnorm{I - \eta P(t)} \frobnorm{\Phi(t)}
	+ O\left(\frac{\eta \lambda}{\din}\right) \cdot \frobnorm[\Big]{\frac{W_{2:1}(t) Y}{\sqrt{\dhid \din}}}
	+ \frobnorm[\Big]{\frac{1}{\sqrt{\dhid \din}} E_0(t) Y},
	\label{eq:proof-sketch-error-decomposition}
\end{align}
where $P(t)$ is an operator acting on matrix space whose matrix representation in terms of the vectorization is given by
\[
	P(t) :=
	\frac{1}{\dhid \din}
	\left(1 - \frac{\eta \lambda}{\din}\right) (W_1(t) Y)^{\T} (W_{1}(t) Y) \\
	+ \frac{1}{\dhid \din}
	\left(1 - \frac{\eta \lambda}{\dhid}\right) (Y^{\T} Y) \otimes ((W_{2}(t))^{\T} W_{2}(t)).
\]
In particular, one can show that the high-order terms from $E_0(t)$
can be ``folded'' into the first term in~\eqref{eq:proof-sketch-error-decomposition}, since
\begin{equation}
	\frobnorm[\Big]{\frac{1}{\sqrt{\dhid \din}} E_0(t) Y} \leq
	\frac{\eta \opnorm{P(t)}}{4} \frobnorm{\Phi(t)}.
\end{equation}
Consequently, it is the spectrum of $P(t)$ that controls the rate of convergence (up to error that vanishes as $\lambda \rightarrow 0$). Thanks to properties of the Kronecker product, controlling the
spectrum of $P(t)$ can be reduced to controlling the extremal singular values of $W_{1} Y$ and $W_2$ (see~\cref{lemma:P-k-spectrum} for the full statement).

The remainder of the proof outlines two phases for the convergence behavior of gradient descent. In the first phase,
the regression error is driven rapidly to a level that depends
on the regularization strength $\lambda$; in the second phase, the ``off-subspace'' error decreases while the regression error can fluctuate, albeit in a controlled manner.
\vspace*{-1em}
\paragraph{Phase 1: Rapid linear convergence.}
In the first phase, we show that the following properties hold by induction for $t < \tau$:
\begin{itemize}
	\item (\textbf{Singular value control}): For all $i$, it holds that
	      \begin{align*}
		      \frac{3}{4} \sqrt{\dhid}                  & \leq \sigma_{i}(W_{2}(t)) \leq \frac{5}{4} \sqrt{\dhid}                     \\
		      \frac{3}{4} \sqrt{\dhid} \sigma_{\min}(X) & \leq \sigma_{i}(W_{1}(t) Y) \leq \frac{5}{4} \sqrt{\dhid} \sigma_{\max}(X).
	      \end{align*}
	\item (\textbf{Small displacement}): We have
	      \begin{align*}
		      \opnorm{W_{1}(t) - \left(1 - \frac{\eta \lambda}{\din}\right)^{t} W_{1}(0)}  & \lesssim \sqrt{\dout \sr(X)}; \\
		      \opnorm{W_{2}(t) - \left(1 - \frac{\eta \lambda}{\dhid}\right)^{t} W_{2}(0)} & \lesssim \sqrt{\dout \sr(X)}.
	      \end{align*}
	\item (\textbf{Sufficient decrease in regression error}): We have
	      \begin{align}
		       & \frobnorm{\Phi(t+1)}                                                                                                                                  \label{eq:proof-outline-suff-decrease} \\
		       & \leq \left(1 - \frac{\eta \sigma_{\min}^2(X)}{8 \din}\right) \frobnorm{\Phi(t)} + \frac{5 \eta \lambda}{2\din} \sqrt{\frac{\dout}{\din}} \frobnorm{X}. \notag
	      \end{align}
\end{itemize}
A detailed argument can be found in~\cref{sec:subsec:Step 1: Rapid early convergence}.

While $t < \tau$, where $\tau$ is defined in~\cref{theorem:main-informal}, the second
term in the right-hand side of~\eqref{eq:proof-outline-suff-decrease} satisfies
\[
	\frac{5 \eta \lambda}{2 \din} \sqrt{\frac{\dout}{\din}} \frobnorm{X} \leq
	\frac{\eta \sigma_{\min}^2(X)}{16 \din} \frobnorm{\Phi(t)}.
\]
Consequently, we obtain the following recurrence for $t < \tau$:
\begin{equation}
	\frobnorm{\Phi(t+1)} \leq \left(1 - \frac{\eta \sigma_{\min}^2(X)}{16 \din}\right) \frobnorm{\Phi(t)}.
	\label{eq:phase-1-decrease}
\end{equation}
Plugging $\eta = \nicefrac{\din}{2 \sigma_{\max}^2(X)}$ into~\eqref{eq:phase-1-decrease} yields the bound~\eqref{eq:thm-regression-error} for $t < \tau$.
Iterating~\eqref{eq:phase-1-decrease} until the condition in the definition of $\tau$ fails reveals that the length of
phase 1 is at most
\begin{equation}
	\tau \lesssim \log\left(\frac{1}{\gamma} \sqrt{\frac{\dout}{\din}}\right) \;\;
	\text{iterations.}
	\label{eq:proof-sketch-tau-bound}
\end{equation}
Consequently, we achieve regression error $O(\gamma)$ within $O(\log \frac{1}{\gamma})$ iterations, a rate
commensurate with that achieved by gradient descent when minimizing the \emph{convex} objective
$\min_{W} \frobnorm{WY - X}^2$~\cite{du2019width}. Reducing the \emph{off-subspace} error,
$\opnorm{W_{2:1}(t) P_{\range(Y)}^{\perp}}$,
requires further work as outlined below.
\vspace*{-1em}
\paragraph{Phase 2: Off-subspace component reduction.}
During the first phase, the off-subspace error will generally
decrease but remain nontrivial, requiring additional iterations to bring to acceptable levels.
The challenge is that when $t > \tau$, the regression error $\frobnorm{\Phi(t)}$ is no longer monotonic.
To that end, we argue that the regression error \emph{remains} small
(up to a constant multiplicative factor) for the next $O\big(\frac{1}{\gamma}\big)$ steps,
subject to the same stepsize requirements; in turn, these steps are sufficient to reduce
the off-subspace error to $O(\mathrm{poly}(\dhid^{-1}))$.
Specifically, we argue that the following properties
hold (see~\cref{sec:subsec:Step 2: he error stays small}) for all iterations $t$ satisfying $\tau \le t \le O(\log(\dhid) / \lambda)$:
\begin{itemize}
	\item (\textbf{Singular value control II}): For all $i$, it holds that
	      \begin{align*}
		      \frac{5}{7} \sqrt{\dhid}  & \leq \sigma_{i}(W_{2}(t)) \leq \frac{9}{7} \sqrt{\dhid}; \\
		      \sigma_{\max}(W_{1}(t) Y) & \leq \frac{9}{7} \sqrt{\dhid} \sigma_{\max}(X).
	      \end{align*}
	\item (\textbf{Small displacement II}): We have that
	      \begin{align*}
		      \opnorm{W_{1}(t) - \left(1 - \frac{\eta \lambda}{\din}\right)^{t - \tau} W_{1}(\tau)}  & \lesssim \sqrt{\dout \sr(X)} \log(\dhid); \\
		      \opnorm{W_{2}(t) - \left(1 - \frac{\eta \lambda}{\dhid}\right)^{t - \tau} W_{2}(\tau)} & \lesssim \sqrt{\dout \sr(X)} \log(\dhid).
	      \end{align*}
	\item (\textbf{Small regression error}): We have
	      \begin{equation*}
		      \frobnorm{\Phi(t)} \lesssim
		      \frac{\lambda \frobnorm{X}}{\sigma_{\min}^2(X)} \sqrt{\frac{\dout}{\din}}
		      = O(\gamma \frobnorm{X}).
	      \end{equation*}
\end{itemize}
Equipped with the properties above, we show that the off-subspace error
satisfies the bound:
\begin{equation}
	\opnorm{W_{2:1}(t) P_{\range(Y)}^{\perp}} \lesssim \left(1 - \frac{\lambda}{2}\right)^{t}
	\sqrt{\frac{\dhid}{\din}},
	\label{eq:proof-sketch-off-subspace-bound}
\end{equation}
which is at most $\dhid^{-C_2}$ when
$t \geq \Omega\left(\frac{\log(\dhid)}{\lambda}\right)$.
See~\cref{sec:subsec:Step 3: Convergence off the subspace} for details.


\subsection{Robustness to noisy test data}
\label{sec:subsec:robustness}

\section{Loss Robustness}
\label{sec:robustness}

% We extend the concept of label noise to the autoregressive language modeling domain, focusing on asymmetric or class-conditional noise. Specifically, at each step $t$, the label $\xbm_t$ in the training data of the black-box model is flipped to  $\tilde \xbm_t \in V$ with probability $p^*(\tilde \xbm_t|\xbm_t)$, while the feature vectors or preceding tokens $(\xbm_{t-1:1})$ remain unchanged. Consequently, the black-box model observes samples from a noisy distribution given by  $p^*(\tilde \xbm_t, \xbm_{t-1:1}) = \sum_{\xbm_t}p^*(\tilde \xbm_t | \xbm_t)p^*(\xbm_t|\xbm_{t-1:1})p^*(\xbm_{t-1:1})$.

% Denote by $T_t  \in [0, 1]^{|V|\times |V|}$, the noise transition matrix at step $t$ specifying the probability of one label being flipped to another, so that $\forall i, j \;\; T_{t_{ij}}=p^*(\tilde \xbm_t = \ebm^j | \xbm_t = \ebm^i)$. The matrix is row-stochastic and not necessarily symmetric across the classes. 

% To address asymmetric label noise, we modify the loss $\bm{\ell}$ to ensure robustness. Initially, assuming the noise transition matrix $T_t$ is known, we apply a loss correction inspired by prior work~\citep{patrini2017making, sukhbaatar2015training}. Subsequently, we relax this assumption and estimate $T_t$ directly, forming the foundation of our plugin model approach.

We extend label noise modeling to the autoregressive language setting, focusing on asymmetric or class-conditional noise. At each step $t$, the label $\xbm_t$ in the black-box model’s training data is flipped to $\tilde \xbm_t \in V$ with probability $p^*(\tilde \xbm_t|\xbm_t)$, while preceding tokens $(\xbm_{t-1:1})$ remain unchanged. As a result, the black-box model observes samples from a noisy distribution: $p^*(\tilde \xbm_t, \xbm_{t-1:1}) = \sum_{\xbm_t} p^*(\tilde \xbm_t | \xbm_t) p^*(\xbm_t|\xbm_{t-1:1}) p^*(\xbm_{t-1:1}).$

We define the noise transition matrix $T_t \in [0,1]^{|V|\times |V|}$ at step $t$, where each entry $T_{t_{ij}} = p^*(\tilde \xbm_t = \ebm^j | \xbm_t = \ebm^i)$ represents the probability of label flipping. This matrix is row-stochastic but not necessarily symmetric.

To handle asymmetric label noise, we modify the loss $\bm{\ell}$ for robustness. Initially, assuming a known $T_t$, we apply a loss correction inspired by~\citep{patrini2017making, sukhbaatar2015training}. We then relax this assumption by estimating $T_t$ directly, forming the basis of our \textit{Plugin} model approach.

We observe that a language model trained with no loss correction would result in a predictor for noisy labels $b(\tilde \xbm_t | \xbm_{t-1:1})$. We can make explicit the dependence on $T_t$. For example, with cross-entropy we have:

\begin{align*}
&\ell(\ebm^i, b(\tilde \xbm_t | \xbm_{t-1:1})) = -\log b(\tilde\xbm_t = \ebm^i | \xbm_{t-1:1}) \\
&= -\log \sum_{j=1}^{|V|} p^*(\tilde\xbm_t = \ebm^i | \xbm_t = \ebm^j) b(\xbm_t = \ebm^j | \xbm_{t-1:1}) \\ 
&= -\log \sum_{j=1}^{|V|} T_{t_{ji}} {b}(\xbm_t = \ebm^j | \xbm_{t-1:1}), \numberthis
\label{eq:fc}
\end{align*}
or in matrix form
\begin{equation}
    \label{eq:fc-mat}
    \bm{\ell}(b(\tilde \xbm_t|\xbm_{t-1:1})) = -\log T_t^\top b(\xbm_t|\xbm_{t-1:1}).
\end{equation}

% This loss compares the noisy label $\tilde \xbm_t$ to the noisy predictions averaged using the transition matrix $T_t$ at step $t$. Note that the cross-entropy loss is commonly employed for next-token prediction tasks. Cross-entropy is a \emph{proper composite loss} with the softmax function as its \emph{inverse link function}~\citep{patrini2017making}. Consequently, from Theorem 2 of~\citep{patrini2017making}, the minimizer of the \emph{forwardly-corrected} loss in Equation~\eqref{eq:fc-mat} for noisy data corresponds to the minimizer of the actual loss for clean data. Formally, this can be expressed as:

% \begin{align*}
%     \label{eq:loss-minimizers}
%     & \argmin_{w} E^*_{\tilde \xbm_t,\xbm_{t-1:1}}\Big[\bm{\ell}(\xbm_t, T_t^T b(\xbm_t|\xbm_{t-1:1})) \Big] \\ &= 
%     \argmin_{w} E^*_{\xbm_t,\xbm_{t-1:1}}\Big[\bm{\ell}(\xbm_t, b(\xbm_t|\xbm_{t-1:1})) \Big],
% \end{align*}
% where $w$ are the weights of the language model, and their dependence is implicitly embedded in the definition of the softmax output $b$ from the black-box language model. This result indicates that if the transition matrix $T_t$ were known, we could transform the softmax output $b(\bm{x}_t \mid \bm{x}_{t-1:1})$ using $T_t^T$, use the transformed predictions as the final outputs, and re-train the black-box model accordingly with the corrected loss. However, the transition matrix $T_t$ is not known a priori, and we do not have access to the training data. Thus, estimating $T_t$ from clean data becomes a crucial step in our approach.

This loss compares the noisy label $\tilde \xbm_t$ to the noisy predictions averaged via the transition matrix $T_t$ at step $t$. Cross-entropy loss, commonly used for next-token prediction, is a \emph{proper composite loss} with the softmax function as its \emph{inverse link function}~\citep{patrini2017making}. Consequently, from Theorem 2 of~\citet{patrini2017making}, the minimizer of the \emph{forwardly-corrected} loss in Equation~\eqref{eq:fc-mat} on noisy data aligns with the minimizer of the true loss on clean data, i.e., 
\begin{align*}
    \label{eq:loss-minimizers}
    & \argmin_{w} E^*_{\tilde \xbm_t,\xbm_{t-1:1}}\Big[\bm{\ell}(\xbm_t, T_t^\top b(\xbm_t|\xbm_{t-1:1})) \Big] \\ &= 
    \argmin_{w} E^*_{\xbm_t,\xbm_{t-1:1}}\Big[\bm{\ell}(\xbm_t, b(\xbm_t|\xbm_{t-1:1})) \Big],
\end{align*}
where $w$ are the language model’s weights, implicitly embedded in the softmax output $b$ from the black-box model. This result suggests that if $T_t$ were known, we could transform the softmax output $b(\xbm_t \mid \xbm_{t-1:1})$ using $T_t^T$, use the transformed predictions as final outputs, and retrain the model accordingly. However, since $T_t$ is unknown and training data is inaccessible, estimating $T_t$ from clean data is essential to our approach.


\subsection{Estimation of Transition Matrix}
\label{ssec:estimatingT}

% In our problem setup, we assume access to a small amount of clean language data for the task. Under the assumption that the black-box model is expressive enough to model $p^*(\tilde{\bm{x}}_t \mid \bm{x}_{t-1:1})$ (Assumption (2) in Theorem 3 of~\citep{patrini2017making}), the transition matrix $T_t$ can be estimated using this clean data. Considering the supervised classification problem at step $t$, let $\mathcal{X}_t^i$ denote all samples in the clean data where $\bm{x}_t = \bm{e}^i$ and the preceding tokens are $(\bm{x}_{t-1}, \dots, \bm{x}_1)$. A naive estimate of the transition matrix can be computed as follows:

We assume access to a small amount of target language data for the task. Given that the black-box model is expressive enough to approximate $p^*(\tilde{\xbm}_t \mid \xbm_{t-1:1})$ (Assumption (2) in Theorem 3 of~\citet{patrini2017making}), the transition matrix $T_t$ can be estimated from this target data. Considering the supervised classification setting at step $t$, let $\mathcal{X}_t^i$ represent all target data samples where $\xbm_t = \ebm^i$ and the preceding tokens are $(\xbm_{t-1:1})$. A naive estimate of the transition matrix is: $\hat T_{t_{ij}}=b(\tilde \xbm_t = \ebm^j|\xbm_t=\ebm^i)=\frac{1}{|\mathcal{X}_t^i|}\sum_{x\in\mathcal{X}_t^i}b(\tilde \xbm_t = \ebm^j|\xbm_{t-1:1})$.


While this setup works for a single step $t$, there are two key challenges in extending it across all steps in the token prediction task:

\begin{enumerate}[leftmargin=0.4cm]
    \item \textbf{Limited sample availability:} The number of samples where $\bm{x}_t = \bm{e}^i$ and the preceding tokens $(\bm{x}_{t-1}, \dots, \bm{x}_1)$ match exactly is limited in the clean data, especially with large vocabulary sizes (e.g., $|V| = O(100K)$ for LLaMA~\citep{dubey2024llama}). This necessitates modeling the transition matrix as a function of features derived from $\bm{x}_{t-1:1}$, akin to text-based autoregressive models.
    \item \textbf{Large parameter space:} With a vocabulary size of $|V| = O(100K)$, the transition matrix $T_t$ at step $t$ contains approximately 10 billion parameters. This scale may exceed the size of the closed-source LLM and cannot be effectively learned from limited target data. Therefore, structural restrictions must be imposed on $T_t$ to reduce its complexity.
\end{enumerate}

To address these challenges, we impose the restriction that the transition matrix $T_t$ is diagonal. While various constraints could be applied to simplify the problem, assuming $T_t$ is diagonal offers two key advantages. First, it allows the transition matrix—effectively a vector in this case—to be modeled using standard autoregressive language models, such as a \emph{GPT-2 model with $k$ transformer blocks}, a \emph{LLaMA model with $d$-dimensional embeddings}, or a fine-tuned \emph{GPT-2-small} model. These architectures can be adjusted based on the size of the target data. Second, a diagonal transition matrix corresponds to a symmetric or class-independent label noise setup, where $\xbm_t = \ebm^i$ flips to any other class with equal probability in the training data. This assumption, while simplifying, remains realistic within the framework of label noise models.

By enforcing this diagonal structure, we ensure efficient estimation of the transition matrix while maintaining practical applicability within our framework. In the next section, we outline our approach for adapting closed-source language models to target data.














\section{Numerical experiments}
\label{sec:numerics}
\section{Numerical results}\label{sec:numerical_results}
We will use Algorithm~\ref{alg:bae} to perform the numerical tests.

The finite auxiliary domain is slightly larger than the torus and is chosen to be $[-1-\ell,1+\ell]^3$ and the grid size is taken as
\begin{align}
h = \frac{2+2\ell}{N},\quad N = 2^n-1
\end{align}
which gives a total of $N^3$ degrees of freedom in the auxiliary domain and the number of unknowns in the boundary equations is in the scale of $\mathcal{O}(N^2)$.

The Lattice Green's functions are precomputed on mesh $\mathbb{Z}^3$ and stored in a lookup matrix of size $N\times N\times N$ for only positive index values, and can be reused in different tests. Due to the symmetry of the Lattice Green's function, only 1/48 of the values are stored in 3D.

\subsection{Poisson's equation}
The geometry is an ellipsoid given implicitly by
\begin{align}
\phi(x,y,z) = \frac{x^2}{a^2}+\frac{y^2}{b^2}+\frac{z^2}{c^2}-1=0
\end{align}
with $a=1,b=0.8,c=0.4$. We choose $\ell=0.25$ for the auxiliary domain in this test. The spherical harmonics spectral approach developed in \cite{epshteyn2019efficient} will not work well for this ellipsoidal geometry.

The errors are computed against the exact solution
\begin{align}
u(x,y,z) = x^2+y^2+z^2
\end{align}
and boundary data $g(x,y,z)$ and the source term $f(x,y,z)$ are computed using the exact $u(x,y,z)$.

Due to memory limit, we do not store matrices $D_{\pm}$ or $S_\pm$, and we refer to \cite{martinsson2009boundary} for behaviors of singular values of matrices $S_-$ and $D_-$. In Figure~\ref{fig:poisson_res}, we present the relative residual of solving 
\begin{align}
(\Phi_+D_+ + \Phi_-D_-) q_d = b
\end{align}
using vanilla GMRES with no preconditioner and a zero initial guess. The tolerance is set to be $10^{-14}$ for test purpose, whereas a larger tolerance such as $10^{-8}$ would suffice for accuracy. It can be observed that for $N=31,63,127,255$, where the length of $q_d$ approximately quadruples over mesh refinement, the number of iterations grow mildly, even though not as good as the double layer formulation in the boundary integral method. This indicates that preconditioning techniques should be explored.


\begin{figure}[H]
    \centering
    % trim=left bottom right top
    \includegraphics[width=0.5\textwidth, trim = 0cm 5cm 0cm 5.5cm]{fig/Poisson3D_residuals}
    \caption{Relative residual of GMRES using double layer formulation for Poisson equations}
    \label{fig:poisson_res}
\end{figure}

In Figure~\ref{fig:poisson_results}, we plot the numerical solution and the error patterns on the finest mesh $255\times255\times255$ in our simulation. The error patterns hint that it might benefit from post-processing techniques studied such as in \cite{mirzaee2011smoothness}.

\begin{figure}[H]
    \centering
    \begin{subfigure}{0.45\textwidth}
        \centering
        % trim=left bottom right top
        \includegraphics[width=\textwidth]{fig/Poisson_3D_Solution_N256}
        % \vspace*{0.5mm}
        \caption{Numerical Solution $(N=255)$}
        \label{fig:poisson_solution}
    \end{subfigure}
    \quad
    \begin{subfigure}{0.45\textwidth}
        \centering
        \includegraphics[width=\textwidth]{fig/Poisson_3D_Error_N256}
        % \vspace*{0.5mm}
        \caption{Pointwise errors $(N=255)$}
        \label{fig:poisson_error}
    \end{subfigure}
    \caption{Numerical solution and pointwise errors for Poisson equation}\label{fig:poisson_results}
\end{figure}

In Table~\ref{table:poisson_convergence}, the max-norm error and convergence rates are presented, which shows the designed second order convergence.

\begin{table}[htbp]
    \centering
    \begin{tabular}{@{} r S[table-format=1.4e-1] S[table-format=1.2] @{}}
        \toprule
        {$N$} & {Max Error} & {Rate} \\
        \midrule
        31   & 1.3222e-03 & {--} \\
        63   & 3.5101e-04 & 1.91 \\
        127  & 9.1688e-05 & 1.94 \\
        255  & 2.2926e-05 & 2.00 \\
        \bottomrule
    \end{tabular}
    \caption{Max error and convergence rates for Poisson's equation in an ellipsoid}
    \label{table:poisson_convergence}
\end{table}

\subsection{Modified Helmholtz equation}

For modified Helmholtz equation, we test with $\sigma=10$. For the geometry, we use a multi-connected torus defined by the implicit function
\begin{align}
\phi(x,y,z) = (\sqrt{x^2+y^2}-R)^2+z^2-r^2 = 0
\end{align}
where $R=0.6$ is the distance between the center of the tube and the center of the torus and $r=0.3$ is the radius of the tube. The interior of the torus is categorized by $\phi(x,y,z)<0$. The auxiliary domain takes $\ell=0.1$.

\begin{remark}
A torus is simple enough, yet presenting sufficient numerical challenges for unfitted boundary methods. We chose this geometry to demonstrate the effectiveness of combing lattice Green's function in free space with local basis functions, where global basis functions will be difficult to construct for spectral approaches. We should also mention that using Non-Uniform Rational B-Splines (NURBS) on patches for 3D CAD geometry is also possible and has been studied in the difference potentials framework in \cite{PETROPAVLOVSKY2024112705}.
\end{remark}

The exact solution is
\begin{align}
u(x,y,z) = \sin(x)\cos(y)\sin(z),
\end{align}
and $f$ and $g$ are computed using the exact solution.

It can be seen in Figure~\ref{fig:poisson_res} and Figure~\ref{fig:mod_res} that the iteration numbers are similar regardless of the equation types $\sigma=0$ or $\sigma=10$ or different types of geometry or lattice Green's functions, corroborating the robustness of the boundary algebraic linear system.

The GMRES convergence behavior shown in Figure~\ref{fig:mod_res} demonstrates the effectiveness of the double layer formulation for the modified Helmholtz equation. The similar convergence patterns observed for different mesh sizes $(N = 31, 63, 127, 255)$ suggest that the conditioning of the system remains relatively stable under mesh refinement. This behavior is particularly noteworthy given the complex geometry of the torus and the challenging nature of the modified Helmholtz operator.

\begin{figure}[H]
    \centering
    % trim=left bottom right top
    \includegraphics[width=0.5\textwidth, trim = 0cm 5cm 0cm 5.5cm]{fig/Modified3D_residuals}
    \caption{Relative residual of GMRES using double layer formulation for modified Helmholtz equations}
    \label{fig:mod_res}
\end{figure}

In Figure~\ref{fig:mod_results}, the numerical solution and the error patterns on the finest mesh $255\times255\times255$ are presented. The large errors occur around where $u$ is large in magnitude.

\begin{figure}[H]
    \centering
    \begin{subfigure}{0.45\textwidth}
        \centering
        % trim=left bottom right top
        \includegraphics[width=\textwidth]{fig/Modified_3D_Solution_N256}
        % \vspace*{0.5mm}
        \caption{Numerical Solution $(N=255)$}
        \label{fig:mod_solution}
    \end{subfigure}
    \quad
    \begin{subfigure}{0.45\textwidth}
        \centering
        \includegraphics[width=\textwidth]{fig/Modified_3D_Error_N256}
        % \vspace*{0.5mm}
        \caption{Pointwise errors $(N=255)$}
        \label{fig:mod_error}
    \end{subfigure}
    \caption{Numerical solution and point errors for modified Helmholtz equation}\label{fig:mod_results}
\end{figure}

In Table~\ref{table:mod_convergence}, the max-norm error and convergence rates also show that the designed second order convergence is achieved.

\begin{table}[htbp]
    \centering
    \begin{tabular}{@{} r S[table-format=1.4e-1] S[table-format=1.2] @{}}
        \toprule
        {$N$} & {Max Error} & {Rate} \\
        \midrule
        31   & 7.5002e-05 & {--} \\
        63   & 2.0682e-05 & 1.86 \\
        127  & 5.6077e-06 & 1.88 \\
        255  & 1.3978e-06 & 2.00 \\
        \bottomrule
    \end{tabular}
    \caption{Max Error and Convergence Rates for modified Helmholtz equation in a torus}
    \label{table:mod_convergence}
\end{table}

\section{Limitations and future directions}
\label{sec:discussion}

\paragraph{Depth and generalization.}
Our experiments in~\cref{fig:errors-by-depth} suggest that depth is beneficial for both the regression and the ``off-subspace'' errors: larger depth, at least up to a certain point, leads to faster convergence. This phenomenon is not covered by our main theoretical result, but constitutes an interesting direction for future work.

\vspace*{-1em}

\paragraph{Near-singular matrices and conditioning.}
Our main result (\cref{theorem:main-informal}) does not provide meaningful insights
for \emph{approximately} low-rank data; e.g., inputs $X$ that can
be decomposed as the sum of a well-conditioned low-rank component and a full-rank component with relatively small singular values, a pervasive property in data science applications~\cite{UT19}.
For such inputs, it is plausible that gradient descent with weight decay is able to rapidly converge to a solution mapping that provides a good approximation
to the ``low-rank'' component of the input $X$. We leave such an investigation
to future work.

\vspace*{-1em}

\paragraph{Adaptivity of deep non-linear networks.}
Our experiments in~\cref{fig:nonlinear} suggest that weight decay can lead to robust solutions beyond the simple linear inverse problem setting.
In particular, a natural next step would be to study the training dynamics of $\ell_2$-regularized
gradient descent for deep networks with several linear layers and a ReLU layer
(as considered in~\cite{parkinson2023linear}) under the assumption that the input
data is generated by the ``union-of-subspaces'' model used in~\cref{fig:nonlinear}.

\section*{Acknowledgements}
SP gratefully acknowledges the support of the NSF Graduate Research Fellowship Program NSF DGE-2140001. VC and RW gratefully acknowledge the support of NSF DMS-2023109, the NSF-Simons National Institute for Theory and Mathematics in Biology (NITMB) through NSF (DMS-2235451) and Simons Foundation (MP-TMPS-00005320), and the Margot and Tom Pritzker Foundation. FK gratefully acknowledges the support of the German Science Foundation (DFG) in the context of the priority program Theoretical Foundations of Deep Learning
(project KR 4512/6-1). HL and FK gratefully acknowledge the support of the Munich Center for Machine Learning (MCML).




\bibliographystyle{unsrtnat}
\bibliography{main}

\newpage
\appendix
\onecolumn
\section{Main result and proof}
\label{sec:mainresultandproof-appendix}
This section presents the formal version of the main result and the full proof. 
We start with fixing some notation and assumptions in \cref{sec:subsec:prelim}, and then state the main result in \cref{sec:subsec:main result formal}.
\cref{sec:subsec:Lemmas used for the proof} shows some general lemmas used in multiple proof steps. 
We show in \cref{sec:subsec:Properties at initialization} that certain properties hold at initialization.
The main proof then involves three steps.
First, we prove by induction in \cref{sec:subsec:Step 1: Rapid early convergence} that the regression error rapidly decreases during an initial phase of gradient descent.
Second, in \cref{sec:subsec:Step 2: he error stays small}, we again use induction to show that the regression error remains small during a subsequent phase of gradient descent.
Third, in \cref{sec:subsec:Step 3: Convergence off the subspace} we show that the ``off-subspace'' error becomes small during this period. 
We conclude by showing that this method is robust at test time in \cref{sec:subsec:Robustness at test time}.

Let us note that \cref{sec:subsec:Lemmas used for the proof,sec:subsec:Properties at initialization,sec:subsec:Step 1: Rapid early convergence} are based on the proof in \cite{du2019width} of the convergence of gradient descent for the convex problem $\min_W \frobnorm{ WY-X}$. Because of the additional regularization term in our setting, the proof is significantly different. For example, we cannot prove that the error converges towards $0$ or stays small for all iterations. Instead, we show in \cref{sec:subsec:Step 2: he error stays small,sec:subsec:Step 3: Convergence off the subspace} that the error stays less than $O(\lambda)$ for many iterations, during which time the ``off-subspace'' error shrinks, leading to good generalization.
\subsection{Preliminaries}
\label{sec:subsec:prelim}

\subsubsection{Notation}
\label{sec:subsec:subsubsec:notation}
We first establish some notation; let
\begin{subequations}
	\begin{align}
		W_{j:i} & = \prod_{t=i}^j W_{t} = \begin{cases}
			                                  W_j W_{j-1} \ldots W_i, & \text{if $i \leq j$}, \\
			                                  I,                      & \text{otherwise};
		                                  \end{cases} \\
		U       & = d_w^{-\frac{L-1}{2}}m^{-\frac{1}{2}} W_{L:1}Y.                        \\
		\Phi    & = U - X.
	\end{align}
\end{subequations}

The matrix $U$ corresponds to the network predictions,
while $\Phi$ corresponds to the matrix of training residuals.
To refer to a matrix at iteration $t$ of gradient descent, we write $W_{i}(t)$, $W_{j:i}(t), U(t), \Phi(t)$, etc.
When convenient, we write
\begin{equation}
	d_i = \begin{cases}
		d_w & \text{for $2\le i \le L$} \\
		m   & \text{for $i=1$.}
	\end{cases}
\end{equation}
With this notation at hand, our loss function becomes
\begin{equation}
	f(W_1, \ldots, W_L)
	= \frac{1}{2} \frobnorm{\Phi}^2
	+ \frac{\lambda}{2} \sum_{j=1}^{L}
    \frobnorm*{\frac{1}{d_i} W_i}^2 \\
	\label{eq:loss}
\end{equation}
We shall also write $\cprod{i}$ and $\call$ for the following products appearing in our proofs:
\begin{equation}
	\cprod{i} := \prod_{\substack{j = 1\\j \neq i}}^{L} \left(1 - \frac{\eta \lambda}{d_i} \right) \quad \text{and} \quad
	\call := \prod_{i = 1}^{L} \left(1 - \frac{\eta \lambda}{d_i}\right).
	\label{eq:c-prod-i}
\end{equation}
Finally, we let $\sr(X) \in [1, \rank(X)]$ denote the \emph{stable rank} of $X$, defined as 
\begin{equation}
  \sr(X) := \left( \frac{\frobnorm{X}}{\opnorm{X}} \right)^2 =
  \sum_{i \geq 1} \left( \frac{\sigma_{i}(X)}{\sigma_{1}(X)} \right)^2.
  \label{eq:stable-rank}
\end{equation}

\subsubsection{Initialization} \label{sec:subsec:subsubsec:initialization}
\begin{assumption}[Initialization]
	All the weight matrices $W_\ell$ are initialized according to:
	\[
		(W_{\ell})_{ij} \iid \cN(0, 1)
	\]
	with dimensions $W_{1} \in \Rbb^{d_w \times m}$, $W_{2}, \dots, W_{L-1} \in \Rbb^{d_w \times d_w}$,
	and $W_{L} \in \Rbb^{d \times d_w}$.
\end{assumption}
This is the same as \cref{assumption:initialization} since in the optimization problem in \cref{eq:loss} we have explicitly pulled out the normalization factor that comes from the ``fan-in" initialization.

\subsubsection{Gradient Descent Updates}
\label{sec:subsec:subsubsec:gd-updates}
The gradient of the regression error with respect to $W_{i}$ is equal to
\begin{equation}
	\grad_{W_i} \left[\frac{1}{2}\frobnorm{\Phi}^2 \right]
	= d_w^{-\frac{L-1}{2}}m^{-\frac{1}{2}} W_{L:i+1}^{\T} \Phi Y^{\T} W_{i-1:1}^{\T}.
	\label{eq:mse-gradient}
\end{equation}
The gradient of the $\ell_2$-regularization term is
\begin{equation}
	\grad_{W_i} \left[ \frac{\lambda}{2}\frobnorm*{\frac{1}{\sqrt{d_i}}W_i}^2 \right]
	= \frac{\lambda}{d_i} W_i.
\end{equation}
Hence the gradient descent iteration is as follows:
\begin{equation}
	W_i(t+1)
	= \left(1-\frac{\eta\lambda}{d_i}\right) W_i(t) - \eta
	d_w^{-\frac{L-1}{2}}m^{-\frac{1}{2}} W_{L:i+1}(t)^{\T} \Phi(t) Y^{\T} W_{i-1:1}(t)^{\T},
  \quad \text{for $1 \leq i \leq L$.}
	\label{eq:Wi-update}
\end{equation}

\subsubsection{Simplifying the number of samples}
\label{sec:subsec:subsubsec:number-of-samples}
We may assume that we have exactly $s$ input samples for the purpose of analysis.
Indeed,~\cref{claim:exact-rank-X} below shows that the gradient descent trajectories remain
unchanged when the number of samples $n > s$.
\begin{claim}
	\label{claim:exact-rank-X}
	Without loss of generality, we may assume $Z \in \Rbb^{s \times s}$, with $\rank(Z) = s$.
\end{claim}
\begin{proof}
	Since $X = RZ$ where $Z \in \Rbb^{s \times n}$ and $n \geq s$, the economic SVD of $Z$ yields
	\[
		X = R U_{Z} \Sigma_{Z} V_{Z}^{\T}, \quad
		U_{Z} \in O(s), \; \Sigma_{Z} = \diag(\sigma_1, \dots, \sigma_{s}),
		\; V_{Z} \in O(n, s).
	\]
	Since the Frobenius norm is unitarily invariant,
	\begin{align*}
		\frobnorm{\Phi}
		 & = \frobnorm{d_w^{-\frac{L-1}{2}}m^{-\frac{1}{2}} W_{L:1}Y - X}
		\\&= \frobnorm{(d_w^{-\frac{L-1}{2}}m^{-\frac{1}{2}}W_{L:1}AR - R) U_{Z} \Sigma_{Z} V_{Z}^\T}
		\\&= \frobnorm{d_w^{-\frac{L-1}{2}}m^{-\frac{1}{2}}W_{L:1}ARU_{Z}\Sigma_{Z} - RU_{Z} \Sigma_{Z}}.
	\end{align*}
	Moreover,
	\begin{align*}
		\grad_{W_i} \left[\frac{1}{2}\frobnorm{\Phi}^2 \right]
		 & = d_w^{-\frac{L-1}{2}}m^{-\frac{1}{2}} W_{L:i+1}^{\T} \Phi Y^{\T} W_{i-1:1}^{\T}                                                                                                                              \\
		 & = d_w^{-\frac{L-1}{2}}m^{-\frac{1}{2}} W_{L:i+1}^{\T} (d_w^{-\frac{L-1}{2}}m^{-\frac{1}{2}} W_{L:1}A - I)R U_{Z} \Sigma_{Z} \underbrace{V_{Z}^{\T} V_{Z}}_{I_{s}} \Sigma_{Z} U_{Z}^{\T} R^{\T} A^{\T} W_{i-1:1}^{\T} \\
		 & = d_w^{-\frac{L-1}{2}}m^{-\frac{1}{2}} W_{L:i+1}^{\T} (d_w^{-\frac{L-1}{2}}m^{-\frac{1}{2}}  W_{L:1}A - I)R U_{Z} \Sigma_{Z}^2 U_{Z}^{\T} R^{\T} A W_{i-1:1}^{\T}.
	\end{align*}
	Thus, without loss of generality, we can assume that $X = R U_{Z} \Sigma_{Z} \in \Rbb^{\dout \times s}$ since this assumption does not change the gradient descent trajectory or value of the loss function.
\end{proof}

Throughout the remainder of this section, we will assume that $n = s$.

\subsection{Main result}
\label{sec:subsec:main result formal}
Given the above assumptions and notation, we can now state the formal version of our main result.
\begin{theorem}
  \label{thm:mainresult-formal}
  Let~\cref{assumption:rip,assumption:initialization} hold
  with $\delta = \frac{1}{10}$. Furthermore, suppose the following conditions are true:
  \begin{equation}
      \lambda \leq
      \frac{L \sigma_{\min}^2(X)}{400 \cdot 35}, \;\;
      \dhid \gtrsim \dout \cdot \sr(X) \cdot \mathrm{poly}(L, \kappa), \;\;
      \eta \leq \frac{m}{L \sigma^2_{\max}(X)}, \;\; \text{and} \;\; \lambda = \gamma \sigma_{\min}^2(X) \sqrt{\frac{m}{d}},
      \label{eq:main-thm-assumptions}
  \end{equation}
  where $\gamma \in (0, 1]$ is a user-specified accuracy parameter.
  Moreover, define the times
	\begin{subequations}
		\begin{align}
			\tau & = \inf\set*{
				t \in \mathbb{N} \mid
				\frobnorm{\Phi(t)} \leq
				\frac{80 \gamma\frobnorm{X}}{L}
			},                                                                       \label{eq:tau def phi}\\
			T    & = \frac{2\log(\dhid)\sqrt{dm}}{\eta\gamma \sigma_{\min}^2(X)}. \label{eq:T def appendix}
		\end{align}
	\end{subequations}
Then with probability of at least $1-c_1 e^{-c_2 \dout}$  over the random initialization,
\begin{align}
		 \frobnorm{W_{L:1}(t+1)Y - X} \label{eq:thm-regression-error-formal}
		 & \leq
		\begin{cases}
			\left(1 - \frac{\eta L \sigma_{\min}^2(X)}{32 \din}\right) \frobnorm{W_{L:1}(t)Y - X}, & t < \tau;        \\
			C_1 \gamma \frobnorm{X} ,                                           & \tau \leq t \leq T.
		\end{cases}
		\\
		 \opnorm{W_{L:1}(T) P_{\range(Y)}^{\perp}}  
     &\leq \left( \frac{1}{\dhid} \right)^{C_2},
		\label{eq:thm-generalization-error-formal}
	\end{align}
    where $c_1, c_2$, $C_1$ and $C_2$ are positive universal constants.
\end{theorem}
\begin{remark}
    Throughout~\cref{sec:subsec:Lemmas used for the proof,sec:subsec:Properties at initialization,sec:subsec:Step 1: Rapid early convergence,sec:subsec:Step 2: he error stays small,sec:subsec:Step 3: Convergence off the subspace}, \cref{assumption:initialization,assumption:rip,eq:main-thm-assumptions} are in force.
\end{remark}
\begin{remark}
    The condition $\lambda \leq \nicefrac{L \sigma_{\min}^2(X)}{400 \cdot 35}$ is automatically satisfied for small enough $\gamma$.
\end{remark}

\subsection{Lemmas used for the proof}
\label{sec:subsec:Lemmas used for the proof}
The following lemma bounds the deviation of $W_{i}(t)$ from $\left(1-\frac{\eta \lambda}{d_i} \right)^t W_{i}(0)$.
\begin{lemma}
	\label{lemma:difference-norm}
  For any $i \in [L]$, any $t \in \mathbb{N}$, and any matrix norm $\norm{\cdot}$, we have
	\begin{align*}
     & \norm{W_{i}(t)- \left(1-\frac{\eta \lambda}{d_i} \right)^t W_{i}(0)} \\
		 & \leq
		\eta d_w^{-\frac{L-1}{2}}m^{-\frac{1}{2}} \sum_{j=0}^{t-1}
		\left(1 - \frac{\eta \lambda}{d_i}\right)^{t-1-j}
		\norm{W_{L:i+1}(j)^{\T} \Phi(j) (W_{i-1:1}(j) Y)^{\T}}
		\label{eq:difference-norm-scaled}
	\end{align*}
\end{lemma}
\begin{proof}
	The proof follows from the update formula for $W_{i}$ in \cref{eq:Wi-update}.
	Writing
	\begin{equation}
		B_{t} := d_w^{-\frac{L-1}{2}}m^{-\frac{1}{2}} W_{L:i+1}(t)^{\T} \Phi(t) Y^{\T} W_{i-1:1}(t)^{\T},
	\end{equation}
	we rewrite~\cref{eq:Wi-update} as the recursion
	\begin{align*}
		W_i(t) & = \left(1 - \frac{\eta \lambda}{d_i}\right) W_i(t-1) - \eta B_{t-1} \notag \\
		       & = \left(1 - \frac{\eta \lambda}{d_i}\right)^2 W_i(t-2)
		- \eta \left(1 - \frac{\eta \lambda}{d_i}\right) B_{t-2} - \eta B_{t-1} \notag      \\
		       & \qquad \qquad \vdots \notag                                                \\
		       & = \left(1 - \frac{\eta \lambda}{d_i}\right)^t W_i(0)
		- \eta \sum_{j=0}^{t-1} \left(1 - \frac{\eta \lambda}{d_i}\right)^{t-1-j} B_{j}.
	\end{align*}
	Rearranging, taking norms and applying the triangle inequality yields the result.
\end{proof}
\subsubsection{Evolution of Product Matrix} \label{sec:iteration of WL1}
\begin{lemma}
	\label{lemma:prod-evolution-I}
	For an arbitrary iteration index $t$, it holds that
	\begin{align}
		W_{L:1}(t+1) &=
    \begin{aligned}[t]
         & \call W_{L:1}(t) + E_0(t) \\
         & - \eta d_w^{-\frac{L-1}{2}}m^{-\frac{1}{2}}
			     \sum_{i=1}^L
			     \cprod{i}
			     W_{L:i+1}(t) W_{L:i+1}^{\T}(t) \Phi(t) Y^{\T} W_{i-1:1}^{\T}(t) W_{i-1:1}(t),
    \end{aligned}
    \label{eq:prod-evolution-I}
	\end{align}
	with $E_0(t)$ containing all $O(\eta^2)$ terms.
\end{lemma}
\begin{proof}
	This is essentially the decomposition in~\citep[Section 5]{du2019width}, modified since
	\begin{equation}
		W_i(t+1)
		= W_i(t)\left(1-\frac{\eta\lambda}{d_i}\right) - \eta
		d_w^{-\frac{L-1}{2}}m^{-\frac{1}{2}} W_{L:i+1}(t)^{\T} \Phi(t) Y^{\T} W_{i-1:1}(t)^{\T}.
	\end{equation}
    For the sake of brevity, we do not repeat the argument here.
\end{proof}

\paragraph{Evolution of the Network Outputs and Residuals.}
\label{sec:evolution of res}
Armed with~\cref{lemma:prod-evolution-I}, we right-multiply
both sides of~\eqref{eq:prod-evolution-I} by $d_w^{-\frac{L-1}{2}}m^{-\frac{1}{2}} Y$ to obtain
\begin{equation}
	U(t+1) = \begin{aligned}[t] 
    &\call U(t) + E(t)  \\
    & - \eta d_w^{-(L-1)}m^{-1}
	  \sum_{i=1}^L
	\cprod{i}
    W_{L:i+1}(t) W_{L:i+1}(t)^{\T} \Phi(t) Y^{\T} W_{i-1:1}^{\T}(t) W_{i-1:1}(t)Y
  \end{aligned}
	\label{eq:prod-evolution-I-U}
\end{equation}
where $E(t) := d_w^{-\frac{L-1}{2}}m^{-\frac{1}{2}} E_0(t)Y$.

Vectorizing both sides and using the identity $\vect(AXB) = (B^{\T} \otimes A) \vect(X)$ yields
\begin{align}
	\vect (U(t+1))
	= \call \vect (U(t))
	-\eta P(t) \vect(\Phi(t))
	+ \vect(E(t)),
	\label{eq:prod-evolution-I-vec}
\end{align}
where we write $P(t)$ for the following matrix (dropping the
time index $t$ for brevity):
\begin{align}
	P
	= d_w^{-(L-1)}m^{-1}
	\sum_{i=1}^L
	\cprod{i}
	\left(Y^{\T} W_{i-1:1}^{\T} W_{i-1:1}Y \right)
	\otimes
	\left(W_{L:i+1} W_{L:i+1}^{\T}\right)
    \in \R^{sd \times sd}
	\label{eq:coeff-matrix}
\end{align}
We subtract $\vect(X)$ from both sides of
\cref{eq:prod-evolution-I-vec}; using the notation from~\cref{eq:coeff-matrix}, the result is equal to
\begin{align}
    \notag
	\vect(\Phi(t+1))
	 & =
	\begin{aligned}[t]
		 & \call \vect (U(t))
		- \vect(X)
		-\eta P(t) \vect(\Phi(t))
		+ \vect(E(t))
	\end{aligned} \\
	 & =
	\begin{aligned}[t]
		 & \left(\call - 1\right) \vect (U(t))
		+ (I-\eta P(t)) \vect(\Phi(t))
		+ \vect(E(t))
	\end{aligned}
	\label{eq:error-evolution-I}
\end{align}
Taking the Frobenius norm on both sides of~\eqref{eq:error-evolution-I}
and using the triangle inequality yields
\begin{align}
    \notag
	\frobnorm{\Phi(t+1)}
	 & \leq
	\opnorm{I - \eta P(t)} \frobnorm{\Phi(t)}
	+ \left|\call-1\right| \frobnorm{U(t)}
	+ \frobnorm{E(t)} \\
	 & \leq
	\left(1 - \eta \lambda_{\min}(P(t))\right)
	\frobnorm{\Phi(t)}
	+ \frobnorm{U(t)} \callbound
	+ \frobnorm{E(t)},
	\label{eq:error-evolution-II}
\end{align}
as long as $\eta \le \frac{1}{\lambda_{\max}(P(t))}$, using
\cref{lemma:one-minus-folded-product} in the second inequality.
Intuitively, \cref{eq:error-evolution-II} suggests that
bounding the spectrum of $P$ will allow us to get a recursive bound on the norm of the residual.


\subsubsection{Bounds on the spectrum of $P$}
In this paragraph, we furnish bounds on the spectrum of $P$ in terms of the spectrum of $W_{L:i+1}$ and $W_{i-1:1} Y$, for $i = 1 \ldots L$. In the following lemma, we drop the time index $t$ for
simplicity.
\label{sec:bounding spectrum of p}
\begin{lemma}
	\label{lemma:P-k-spectrum}
	We have the following inequalities:
	\begin{align}
		\lambda_{\max}(P)
		 & \leq d_w^{-(L-1)}m^{-1}
		\sum_{i=1}^L
		\cprod{i} \sigma_{\max}^2(W_{i-1:1} Y) \sigma_{\max}^2(W_{L:i+1}); \label{eq:lambda-max-pk} \\
		\lambda_{\min}(P)
		 & \geq d_w^{-(L-1)}m^{-1}
		\sum_{i=1}^L
		\cprod{i} \sigma_{\min}^2(W_{i-1:1} Y) \sigma_{\min}^2(W_{L:i+1}). \label{eq:lambda-min-pk}
	\end{align}
\end{lemma}
\begin{proof}
	The inequalities are straightforward to prove using the definition of $P$ in \cref{eq:coeff-matrix} and the following facts:
	\begin{enumerate}
		\item The largest (or smallest) eigenvalue of a sum of matrices is bounded above (or below) by the sum of the largest (or smallest) eigenvalues.
		\item The eigenvalues of a Kronecker product are the products of the eigenvalues of the individual factors.
		\item For any matrix $A$, $\lambda_{\max}(A^{\T} A) = \sigma_{\max}^2(A)$.
	\end{enumerate}
	Using these facts, the result is immediate.
\end{proof}
\subsection{Properties at initialization}
\label{sec:subsec:Properties at initialization}
Let us bound the extremal singular values of $W_{j:1}Y$, $W_{L:i}$, $W_{i:j}$ at initialization and bound $\frobnorm{\Phi}$ and $\frobnorm{U}$ at initialization.
\begin{lemma}	\label{lemma:restricted-singular-values}
	There are universal constants $c_1, c_2 > 0$ such that 
	\begin{subequations}
		\begin{align*}
			\prob{\max_{1 \leq i < L} d_{w}^{-\frac{i}{2}} \sigma_{\max}(W_{i:1}(0) Y) \leq \frac{6}{5} \sigma_{\max}(X)}
			 & \geq 1 - c_1 \exp\left(-\frac{c_2 \dhid}{L} \right), \\
			\prob{\min_{1 \leq i < L} d_{w}^{-\frac{i}{2}} \sigma_{\min}(W_{i:1}(0) Y) \geq \frac{4}{5} \sigma_{\min}(X)}
			 & \geq 1 - c_1 \exp\left(-\frac{c_2 \dhid}{L} \right).
		\end{align*}
	\end{subequations}
\end{lemma}
\begin{proof}
    Let $U \Sigma V^{\T}$ be the economic SVD of $Y = AX$; since
    $X \in \range(R)$, where $\dim(\range(R)) = s$, this implies
    $U \in O(m, s)$, $\Sigma = \diag(\sigma_1, \dots, \sigma_{s})$
    and $V \in O(n, s)$. Consequently, for all $1 \leq i < L$ we have
    \begin{align}
        \sigma_{\max}(W_{i:1}(0) Y) &=
        \sigma_{\max}(W_{i:1}(0) AX) \notag \\ 
        & \leq \sigma_{\max}(W_{i:1}(0) U) \cdot \sigma_{\max}(\Sigma V^{\T}) \notag \\
        &= \sigma_{\max}(W_{i:1}(0) U) \opnorm{U \Sigma V^{\T}} \notag \\
        &\leq \sqrt{1 + \delta} \cdot \sigma_{\max}(W_{i:1}(0) U) \cdot \sigma_{\max}(X),
        \label{eq:w1y-ub}
    \end{align}
    where the last inequality follows from~\cref{assumption:rip}.
    Similarly, we have
    \begin{align}
        \sigma_{\min}(W_{i:1}(0) Y) &=
        \sigma_{\min}(W_{i:1}(0) AX) \notag \\ 
        & \geq \sigma_{\min}(W_{i:1}(0) U) \cdot \sigma_{\min}(\Sigma V^{\T}) \notag \\
        &= \sigma_{\min}(W_{i:1}(0) U) \cdot \sigma_{\min}(AX) \notag \\
        &\geq \sqrt{1 - \delta} \cdot \sigma_{\min}(W_{i:1}(0) U) \cdot \sigma_{\min}(X).
        \label{eq:w1y-lb}
    \end{align}
	We proceed with bounding the singular values of $W_{i:1}(0)U$. Note that
	\begin{equation}
		W_{1}(0) U = \begin{bmatrix}
			\ip{(W_{1}(0))_{1, :}, U_{:, 1}} & \dots & \ip{(W_{1}(0))_{1, :}, U_{:, s}} \\
			\ip{(W_{1}(0))_{2, :}, U_{:, 1}} & \dots & \ip{(W_{1}(0))_{2, :}, U_{:, s}} \\
			\vdots                  &       & \vdots                  \\
			\ip{(W_{1}(0))_{\dhid, :}, U_{:, 1}} & \dots & \ip{(W_{1}(0))_{\dhid, :}, U_{:, s}}
		\end{bmatrix}
		\overset{(d)}{=} G \in \Rbb^{\dhid \times s}, \;\;
		G_{ij} \iid \cN(0, 1),
		\label{eq:AR-product}
	\end{equation}
	since any two components are Gaussian and uncorrelated.
    Indeed, we have that
	\begin{align*}
		\expec{\ip{(W_1(0))_{i, :}, U_{:, j}} \ip{(W_{1}(0))_{k, :}, U_{:, \ell}}} & =
		\mathsf{tr}\Big( U_{:, j}^{\T}\expec{(W_{1}(0))_{i, :} (W_{1}(0))_{k, :}^{\T}} U_{:, \ell} \Big) \\
		                                                           & =
		\begin{cases}
			0,                              & i \neq k \\
			\ip{U_{:, j}, U_{:, \ell}} = 0, & i = k
		\end{cases},
	\end{align*}
	using the fact that $W_{1}(0)$ has isotropic and
	$U$ has orthogonal columns. We now apply
    \cref{lemma:wide-gaussian-prod-tail} with
    \[
        A_1 = W_{1}(0) U, A_2 = W_2, \dots, A_{i} = W_{i}, \;\; \text{and} \;\;
        n_1 = n_{2} = \dots = n_{i} = \dhid.
    \]
    For these parameter choices, we have
    \(
        \sum_{j = 1}^{i} \frac{1}{n_j} =
        \frac{i}{\dhid}.
    \)
    Thus, for any fixed $y \in \Sbb^{s-1}$ and $i < L$,
    \cref{lemma:wide-gaussian-prod-tail} yields
	\begin{equation}
		\prob{\abs{ \norm{W_{i:1}(0) Uy}^2 - d_w^i} \geq \frac{1}{10} d_w^i}
		\leq c_1 \exp\left(-\frac{c_2 \dhid}{i} \right).
	\end{equation}
	Taking an $\varepsilon$-net $\cN_{\varepsilon}$ of $\Sbb^{s-1}$
	and using~\citep[Exercise 4.3.4]{Ver18}, we obtain for $i<L$
	\begin{align}
		\sup_{y \in \Sbb^{s-1}} \abs{\norm{W_{i:1}(0) U y}^2 - d_w^{i}} & \leq
		\frac{1}{1 - 2 \varepsilon} \sup_{y \in \cN_{\varepsilon}} \abs{\norm{W_{i:1}(0)Uy}^2 - d_w^i} \notag \\ &\leq
		d_w^i \cdot \frac{1}{10(1 - 2 \varepsilon)}, \label{eq:epsilon-net-bound}
	\end{align}
	where the last inequality holds with probability at least $1 - c_1 \abs{\cN_{\varepsilon}} \exp\left(-\frac{c_2 \dhid}{i} \right)$
	as a result of a union bound over $\cN_{\varepsilon}$.
	Hence for $i<L$,
	\begin{align}
		\sigma_{\max}^2(W_{i:1}(0) U) & = \sup_{x \in \Sbb^{s-1}}
		\norm{W_{i:1}(0)U x}^2 \notag                                                                                         \\
		                               & \leq d_w^i + \sup_{x \in \Sbb^{s-1}} \abs{\norm{W_{i:1}(0) Ux}^2 - d_w^i} \notag     \\
		                               & \leq d_w^i \left( 1 + \frac{1}{10(1 - 2 \varepsilon)} \right). \label{eq:w1y-ub-epsnet}
	\end{align}
	Similarly for $i<L$,
	\begin{align}
		\sigma_{\min}^2(W_{i:1}(0) U) & = \inf_{x \in \Sbb^{s-1}}
		\norm{W_{i:1}(0) U x}^2 \notag                                                                                       \\
		                               & \geq d_w^i - \sup_{x \in \Sbb^{s-1}} \abs{\norm{W_{i:1}(0) U x}^2 - d_w^i} \notag   \\
		                               & \geq d_w^i \left(1 - \frac{1}{10(1 - 2 \varepsilon)}\right).
                                       \label{eq:w1y-lb-epsnet}
	\end{align}
	Letting $\varepsilon = \sfrac{1}{10}$ in~\cref{eq:w1y-lb-epsnet,eq:w1y-ub-epsnet} and applying the bounds of~\cref{eq:w1y-ub,eq:w1y-lb} shows that the bound holds for each individual $i$ with probability at least
	\begin{align*}
		1 - c_1 \exp\left\{-\frac{c_2 \dhid}{i} + \log \abs{\cN_{\varepsilon}}\right\} & \geq
		1 - c_1 \exp\left\{-\frac{c_2 \dhid}{i} + s \log \left(1 + \frac{2}{\varepsilon}\right)\right\} \\
		                                                                         & \geq
		1 - c_1 \exp\left(-\frac{c_2 \dhid}{2i}\right),
	\end{align*}
	as long as $\dhid \gtrsim L s$, using the bound~\citep[Corollary 4.2.13]{Ver18}:
	\[
		\abs{\cN_{\varepsilon}} \leq \left(1 + \frac{2}{\varepsilon}\right)^{s}.
	\]
	Taking an additional union bound over $i = 1, \dots, L-1$ combined with the condition
    $\dhid \gtrsim L s \log(L)$ yields the claim.
\end{proof}

\begin{lemma}
	\label{lemma:norm-product-bounded}
	There exist constants $c,C > 0$ such that
	\begin{equation}
		\prob{
			\max_{1 < k \leq j < L}
			\dhid^{-\frac{j - k + 1}{2}} \opnorm{W_{j:k}(0)} \leq
			\sqrt{\frac{L}{c}}
		} \geq
		1 - \exp\left(
		-\frac{C d_w}{L}
		\right).
		\label{eq:norm-product-bounded-uniform}
	\end{equation}
\end{lemma}
\begin{proof}
	Since $W_{i} \in \Rbb^{n_i \times n_{i-1}}=\Rbb^{d_w \times d_w}$ for all $1 <i < L$,
	Lemma \ref{lemma:wide-gaussian-prod-tail} implies
	\[
		\prob{0.9 d_w^{j-k+1} \norm{y}^2 \leq \norm{W_j \dots W_k y}^2 \leq 1.1 d_w^{j-k+1} \norm{y}^2} \geq 1 - 2 \exp\left(
		-\frac{c_1 d_w}{j - k + 1}
		\right).
	\]
	In the following choose $y \in \Sbb^{d_w-1}$ and a small constant $c_2 < c_1$. We can partition $[d_w]$ into $\frac{L}{c_2}$ sets, each of size $ \frac{c_2 d_w}{L}$. Therefore we can write 
	\[
		[d_w] = S_1 \cup \dots \cup S_{\frac{L}{c_2}}.
	\]
  Let $\supp(u):=\{i \mid u_i \neq 0 \}$ and $U_{S_\ell} := \set{u \in \Sbb^{d_w-1} \mid \supp(u) \subset S_\ell}$.
  Taking a $\frac{1}{2}$-net $\mathcal{N}_{\ell}$ of $U_{S_{\ell}}$, we obtain:
	\begin{align*}
		\norm{W_{j:k}(0)u_{\ell}} \leq \sqrt{1.1} d_w^{\frac{j-k+1}{2}} \cdot \frac{3}{2} \leq 2 d_w^{\frac{j-k+1}{2}}, \;\; \text{for all $u_{\ell} \in U_{S_\ell}$},
	\end{align*}
	with the probability of failure at most
	\begin{align*}
		\abs{\cN_{\ell}} \exp\left(-\frac{c_1 d_{w}}{j - k + 1}\right) & \leq
		\log\left(1 + \frac{2}{\sfrac{1}{2}}\right)^{\abs{S_{\ell}}}
		\exp\left(-\frac{c_1 d_{w}}{j - k + 1}\right)                         \\
		                                                               & \leq
		\exp\left(
		- \frac{c_1 d_{w}}{j - k + 1} + \frac{c_2 d_{w}}{L} \log(5)
		\right)                                                               \\
		                                                               & \leq
		\exp\left(
		- \frac{d_{w}}{L} \left(c_1 - c_2 \log(5)\right)
		\right).
	\end{align*}
	The above inequality holds for all $\ell$ at the same time with probability of at least
	\begin{align*}
		1 - \frac{L}{c_2} \exp\left(-\frac{d_{w}}{L}(c_1 - c_2 \log(5))\right)
		\geq 1 - \exp\left(-\frac{C d_w}{L}\right),
	\end{align*}
	for some small constant $C > 0$, as long as $d_{w} \gtrsim L \log \frac{L}{c_2}$ and
	$c_2 \leq \frac{c_1}{2 \log(5)}$ (as a result of a union bound).
    
	Finally, note that we can write any unit vector $y \in \Sbb^{d_w - 1}$ as
	\[
		y = \sum_{\ell} \alpha_{\ell} u_{\ell}, \;\; u_{\ell} \in U_{S_{\ell}}, \;\;
		\sum_{\ell} \alpha_{\ell}^2 = 1.
	\]
	Using the triangle inequality and conditioning on the previous event, we obtain
	\begin{equation}
		\norm{W_{j:i}(0)y} \leq \sum_{\ell} \norm{ W_{j:i}(0)\alpha_{\ell} u_{\ell}} \leq 2  d_w^{\frac{j-k+1}{2}}\sum_{\ell} \abs{\alpha_{\ell}} \leq 2  d_w^{\frac{j-k+1}{2}} \sqrt{\frac{L}{c_1}\sum_{\ell} \alpha_{\ell}^2} \leq d_w^{j-k+1} \sqrt{\frac{L}{c}},
	\end{equation}
	where the last step is using norm equivalence,
	and relabeling $c := \frac{c_1}{4}$. This
	completes the proof of the first display in~\cref{lemma:norm-product-bounded}.

	Finally, to prove~\cref{eq:norm-product-bounded-uniform},
	we apply the union bound over at most $\binom{L}{2} = O(L^2)$
	pairs of indices $i, j$ and use the fact that $d_{w}
		\gtrsim L \log(\frac{L}{c_1})$.
\end{proof}

\begin{lemma}
	\label{lemma:restricted-WL_singular_values}
    There is
	a universal constant $C > 0$ such that
	\begin{subequations}
		\begin{align}
			\prob{
			\max_{1 < i \leq L}
			d_{w}^{-\frac{L - i + 1}{2}}
			\sigma_{\max}(W_{L:i}(0))
			\leq \frac{6}{5}
			} & \geq 1 - \exp\left(-\frac{C d_w}{L}\right),
			\label{eq:restricted-WL-sval-ub-unif}
			\\
			\prob{
			\min_{1 < i \leq L}
			d_{w}^{-\frac{L - i + 1}{2}} \sigma_{\min}(W_{L:i}(0))
			\geq \frac{4}{5}
			} & \geq 1 - \exp\left(-\frac{C d_w}{L} \right).
			\label{eq:restricted-WL-sval-lb-unif}
		\end{align}
	\end{subequations}
\end{lemma}
\begin{proof}
	Since $W_{i}^{\T} \in \Rbb^{d_w \times d_w}$ for $1< i < L$ and $W_{L}^{\T} \in \Rbb^{d_w \times d}$, it follows from Lemma \ref{lemma:wide-gaussian-prod-tail} that
	\begin{align*}
		 & \prob{\abs*{\norm{W_{L:i}^{\T}(0) y}^2 - d_w^{L-i+1}} \geq d_w^{L-i+1}\frac{1}{10}}
		\leq
		2\exp\left(-\frac{c_1 d_w}{L - i + 1}\right)
	\end{align*}
	for any $y \in \Sbb^{d-1}$ and some $c_1 > 0$. Taking an $\varepsilon$-net $\cN_{\varepsilon}$ of
	$\Sbb^{d-1}$ and using~\citep[Exercise 4.3.4]{Ver18}, we have
	\begin{equation}
		\sup_{y \in \Sbb^{d-1}} \abs*{\norm{W_{L:i}^{\T}(0) y}^2 - d_w^{L-i+1}} \leq
		\frac{1}{1 - 2 \varepsilon} \sup_{y \in \cN_{\varepsilon}} \abs*{\norm{W_{L:i}^{\T}(0)y}^2 - d_w^{L-i+1}} \leq
		\frac{d_w^{L-i+1}}{10  \cdot (1 - 2 \varepsilon)}, \label{eq:epsilon-net-bound-upper}
	\end{equation}
	where the last inequality holds with probability at least $1 - 2 \abs{\cN_{\varepsilon}} \exp\left\{-\frac{c d_w}{L - i + 1}\right\}$
	as a result of~\cref{lemma:wide-gaussian-prod-tail} and a union bound over $\cN_{\varepsilon}$.
	In light of~\cref{eq:epsilon-net-bound-upper}, we have
	\begin{align}
		\sigma_{\max}^2(W_{L:i}^{\T}(0)) & = \sup_{x \in \Sbb^{d-1}}
		\norm{W_{L:i}^{\T}(0) x}^2 \notag                                                                                                    \\
		                                 & \leq d_w^{L-i+1} + \sup_{x \in \Sbb^{d-1}} \abs*{\norm{W_{L:i}^{\T}(0) x}^2 - d_w^{L-i+1}} \notag \\
		                                 & \leq d_w^{L-i+1} \cdot \left(1 + \frac{1}{10(1 - 2 \varepsilon)}\right). \label{eq:wLi-ub-epsnet}
	\end{align}
	At the same time,~\cref{eq:epsilon-net-bound-upper} leads to the lower bound
	\begin{align}
		\sigma_{\min}^2(W_{L:i}^{\T}(0)) & = \inf_{x \in \Sbb^{d-1}}
		\norm{W_{L:i}^{\T}(0) x}^2 \notag                                                                                                      \\
		                                 & \geq d_w^{L-i+1} - \sup_{x \in \Sbb^{d-1}} \abs*{\norm{W_{L:i}^{\T}(0) x}^2 - d_w^{L-i+1}} \notag   \\
		                                 & \geq d_w^{L-i+1} \cdot \left( 1 - \frac{1}{10(1 - 2 \varepsilon)} \right). \label{eq:wLi-lb-epsnet}
	\end{align}
	Letting $\varepsilon = 0.25$ in~\cref{eq:wLi-lb-epsnet,eq:wLi-ub-epsnet} we obtain the bound for each individual $i$ with probability of at least
	\begin{align*}
		1 - 2 \exp\left\{-\frac{c d_w}{L-i+1} + \log \abs{\cN_{\varepsilon}}\right\} & \geq
		1 - 2 \exp\left\{-\frac{c d_w}{L-i+1} + d \log \left(1 + \frac{2}{\varepsilon}\right)\right\} \\
		                                                                           & \geq
		1 - 2\exp\left(-\frac{c d_w}{2L}\right),
	\end{align*}
	as long as $d_w \gtrsim L d$, using the bound from in~\citep[Corollary 4.2.13]{Ver18}:
	\[
		\abs{\cN_{\varepsilon}} \leq \left(1 + \frac{2}{\varepsilon}\right)^{d}.
	\]
    To prove \cref{eq:restricted-WL-sval-ub-unif,eq:restricted-WL-sval-lb-unif,}
	we apply a union bound over $1 < i \leq L$ and require that
	$\dhid \gtrsim L \dout \log (L)$.
\end{proof}

\begin{lemma}
	\label{lemma:initial-regression-error}
	At initialization, it holds that
	\begin{align}
		\frobnorm{\Phi(0)} \leq \left(\frac{6}{5}  \sqrt{\frac{d}{m}}  +1 \right) \frobnorm{X} \leq
		\left(\frac{11}{5}  \sqrt{\frac{\dout}{\din}}  \right) \frobnorm{X}
		\quad \text{ and } \quad
		\frobnorm{U(0)} \leq \frac{6}{5}  \sqrt{\frac{\dout}{\din}}  \frobnorm{X}
		\label{eq:initial-regression-error}
	\end{align}
	with probability at least $1 - c_1 \exp\left(-c_2 \dout\right)$ as long
	as $\din \gtrsim s$ and $\dhid \gtrsim L \din$. 
\end{lemma}
\begin{proof}
	Let $\bar{U} \bar{\Sigma} \bar{V}^{\T}$ denote the economic SVD of $AX$, with $\bar{U} \in O(m, s)$. We have
	\begin{align*}
		\frobnorm{U(0)}
		 & = \frobnorm{d_w^{-\frac{L-1}{2}}m^{-\frac{1}{2}} W_{L:1}(0)Y} \\
    &= d_w^{-\frac{L-1}{2}}m^{-\frac{1}{2}} \frobnorm{W_{L:1}(0)AX} \\
    &\leq \dhid^{-\frac{L-1}{2}} \din^{-\frac{1}{2}}
    \opnorm{W_{L:1}(0)\bar{U}}
    \frobnorm{\bar{\Sigma} \bar{V}^{\T}} \\
    &\leq \sqrt{1 + \delta} \cdot \dhid^{-\frac{L-1}{2}} \din^{-\frac{1}{2}}
    \opnorm{W_{L:1}(0)\bar{U}}
    \frobnorm{X},
	\end{align*}
	where the last inequality follows from~\cref{assumption:rip} and unitary invariance
    of the norm. Similarly, we have
	\begin{align*}
		\frobnorm{\Phi(0)}
		 & = \frobnorm{U(0) - X}
		\leq \frobnorm{U(0)} + \frobnorm{X}.
	\end{align*}
	Consequently, it suffices to bound
	$d_w^{-\frac{L-1}{2}}m^{-\frac{1}{2}}\opnorm{W_{L:1}(0)\bar{U}}$. To that end,
    we invoke~\cref{lemma:wide-gaussian-prod-tail} with
    \[
        A_{1} = W_{1}(0) \bar{U}, A_{2} = W_2, \dots, A_{L} = W_{L}, \;\; \text{with} \;\;
        n_{1} = n_{2} = \dots = n_{L} = \dhid, \text{ and } n_{L+1} = \dout.
    \]
    For these choices, the failure probability will depend on the term
    \[
        \sum_{i = 1}^L \frac{1}{n_i}
        = \frac{L-1}{\dhid} + \frac{1}{\dout} 
        \leq \frac{2}{\dout},
    \]
    under the assumption $\dhid \gtrsim L \cdot \dout$. Indeed,
    \cref{lemma:wide-gaussian-prod-tail} yields (for any fixed $y \in \Rbb^s$):
    \begin{equation}
        \prob{\abs{\norm{W_{L:1}(0)Uy}^2 - \dout \cdot \dhid^{L - 1} \norm{y}^2}
        \geq \frac{1}{10} \dout \cdot \dhid^{L-1} \norm{y}^2}
        \leq c_1 \exp(-c_2 \dout).
    \end{equation}
    Taking an $\varepsilon$-net of $\mathbb{S}^{s-1}$ and proceeding as in the proof of~\cref{lemma:restricted-singular-values}, we obtain
    \[
        \opnorm{W_{L:1}(0)U}^2 \leq \dout \cdot \dhid^{L-1} \left(
        1 + \frac{1}{10(1 - 2 \varepsilon)}
        \right)
    \]
    with probability at least $1 - c_1 \exp\left(-c_2 \dout + s \log(1 + \frac{2}{\varepsilon})\right) \geq
    1 - \exp\left(-c_2 \sfrac{\dout}{2}\right)$, since
    $\dout \geq \din \gtrsim s$ for~\cref{assumption:rip} to be valid. Finally, letting
    $\varepsilon = \sfrac{1}{10} = \delta$, we obtain
    \[
        \sqrt{\frac{1 + \delta}{\dhid^{L - 1} \din}} \opnorm{W_{L:1}(0) \bar{U}}
        \leq \frac{6}{5} \sqrt{\frac{\dout}{\din}},
    \]
    as expected. This completes the proof.
\end{proof}



Before we proceed with the proof, we note that a simple union bound shows that
all the bounds in~\cref{lemma:initial-regression-error,lemma:restricted-WL_singular_values,lemma:restricted-singular-values,lemma:norm-product-bounded} are fulfilled simultaneously with probability at least
$1 - c_1 \exp(-c_2 \dout)$, for appropriate universal constants $c_1, c_2 > 0$.


\subsection{Step 1: Rapid early convergence}
\label{sec:subsec:Step 1: Rapid early convergence}
The first step of our convergence analysis is 
showing a sufficient decrease in the regression error until time $\tau$ as defined in \cref{eq:tau def phi}.
We will prove the following theorem in this section. 
\begin{theorem}
\label{thm:step1-induction}
    For all $0 \leq t \leq \tau$, the following events hold with probability of at least $1-c_1 e^{-c_2 d}$, where $c_1, c_2 > 0$ are
    universal constants, over the random initialization:
    \begin{subequations}
      \begin{align}
          \cA(t) &:= \set*{
              \frobnorm{\Phi(t+1)} \leq
              \left(1 - \frac{\eta L \sigma_{\min}^2(X)}{16 \din} \right)
              \frobnorm{\Phi(t)}
              + \frac{5 \eta \lambda}{2 \din} \sqrt{\frac{\dout}{\din}} \frobnorm{X}
          }
          \label{eq:event-A-prestop}
          \\
          \cB(t) &:= \left\{
          \begin{array}{lcll}
              \sigma_{\max}(W_{j:i}(t)) &\leq&
              2 \sqrt{\frac{L}{c}} \dhid^{\frac{j - i + 1}{2}}, &
              1 < i \leq j < L \\
              \sigma_{\max}(W_{i:1}(t) Y) &\leq&
              \frac{5}{4} \dhid^{\frac{i}{2}} \cdot \sigma_{\max}(X), &
              1 \leq i < L, \\
              \sigma_{\max}(W_{L:i}(t)) &\leq&
              \frac{5}{4} \dhid^{\frac{L - i + 1}{2}}, & 1 < i \leq L, \\
              \sigma_{\min}(W_{i:1}(t) Y) &\geq&
              \frac{3}{4} \dhid^{\frac{i}{2}} \cdot \sigma_{\min}(X), & 1 \leq i < L, \\
              \sigma_{\min}(W_{L:i}(t)) &\geq&
              \frac{3}{4} \dhid^{\frac{L - i + 1}{2}}, & 1 \leq i < L.
              \\
          \end{array} \right\},
          \label{eq:event-B-prestop}
          \\
          \cC(t) &:= \set*{
              \opnorm{W_{i}(t) - \left(
              1 - \frac{\eta \lambda}{d_i}
              \right)^t W_{i}(0)}
              \lesssim
              R
              \mid
              1 \leq i \leq L
          },
          \;\; \text{where} \;\;
          R := \frac{\kappa^2 \sqrt{d \sr(X)}}{L}.
          \label{eq:event-C-prestop}
      \end{align}
    \end{subequations}
\end{theorem}
We will prove the above theorem by induction, starting with $t=0$ (\cref{lemma:step-i-properties-at-initialization}).
We then proceed by showing that:
\begin{itemize}
  \item $\set{\cA(j)}_{j < t}$ and $\cB(t)$ imply $\cA(t)$ (\cref{lemma:step-i-all-else-implies-A,lemma:Bt-implies-bound-on-E});
  \item $\set{\cA(j), \cB(j)}_{j < t}$ imply $\cC(t)$ (\cref{lemma:Bj-implies-Ct-phase-1});
  \item $\cC(t)$ implies $\cB(t)$ (\cref{lemma:Ct-implies-Bt-phase-1}).
\end{itemize}
The proof of~\cref{thm:step1-induction} follows by iterating the above implications
until the stopping time $\tau$ is reached.
\begin{lemma}[Initialization]
    \label{lemma:step-i-properties-at-initialization}
    The events $\cA(0)$, $\cB(0)$ and $\cC(0)$ hold with probability at least $1-c_1 e^{-c_2 \dout}$, where $c_1, c_2 > 0$ are universal constants.
\end{lemma}
\begin{proof}
    The base case $\cC(0)$ is trivial. On the other hand, $\cB(0)$ follows from
    \cref{lemma:restricted-singular-values,lemma:norm-product-bounded,,lemma:restricted-WL_singular_values}. 
    Finally, we show in \cref{lemma:step-i-all-else-implies-A}
    that $\cB(t)$ implies $\cA(t)$ for all $t,$ including $t = 0$.
\end{proof}


\begin{lemma}
    \label{lemma:Bt-implies-bound-on-E}
    Fix $t \leq \tau$ and suppose that $\{\cA(j)\}_{j \leq t - 1}$ and $\{\cB(j)\}_{j \leq t}$ hold.
    Then
    \begin{equation}
        \frobnorm{E(t)}
        \leq \frac{17 \eta L \sigma_{\min}^2(X)}{1024 \din}
        \cdot \frobnorm{\Phi(t)}.
    \end{equation}
\end{lemma}
\begin{proof}
Note that each term
in $E(t)$ is the product of $2$ or more terms of the form
$\grad_{W_i} \frac{1}{2} \frobnorm{\Phi}^2$
and $L-2$ or fewer terms of the form $W_{i}(t)(1 - \nicefrac{\eta \lambda}{d_i})$.
When $\ell$ of these terms are from the former category, there are $\binom{L}{\ell}$
ways to choose their indices $(s_1, \dots, s_{\ell})$. Each such choice induces
a term $C_{(s_1, \dots, s_{\ell})}$, defined by
\begin{align*}
  & C_{(s_1, \dots, s_{\ell})}\\
  & := \eta^{\ell} 
  \widetilde{W}_{L:(s_{\ell}+1)} \left( \grad_{W_{s_{\ell}}} \frac{1}{2} \frobnorm{\Phi}^2 \right)
  \widetilde{W}_{(s_{\ell}-1):(s_{\ell-1} + 1)} \left(\grad_{W_{s_{\ell-1}}} \frac{1}{2} \frobnorm{\Phi}^2\right)
  \dots
  \left( \grad_{W_{s_1}} \frac{1}{2} \frobnorm{\Phi}^2 \right)
  \widetilde{W}_{(s_1-1):1}
\end{align*}
where we define the products $\widetilde{W}_{i:j}$ as 
$\widetilde{W}_{i:j} = W_{i:j} \prod_{k = i}^{j} \left(1 - \frac{\eta \lambda}{d_k}\right).
  \label{eq:tildeW-defn}
$
Each factor of the form $\grad_{W_k} \frac{1}{2} \frobnorm{\Phi}^2$ satisfies
\begin{align}
  \frobnorm{\grad_{W_k} \frac{1}{2} \frobnorm{\Phi}^2} &\leq
  d_{w}^{-\frac{L-1}{2}} \din^{-\frac{1}{2}}
  \opnorm{W_{L:(k+1)}(t)} \frobnorm{\Phi(t)} \opnorm{W_{(k-1):1}(t) Y} \notag \\
                                                   &\leq
  \frac{5}{4} d_{w}^{-\frac{L-1}{2}} \din^{-\frac{1}{2}}
  \cdot d_{w}^{\frac{L - k}{2}} \frobnorm{\Phi(t)}
  \frac{5}{4} d_{w}^{\frac{k-1}{2}} \opnorm{X} \notag \\
                                                   &=
                                                   \frac{25}{16 \sqrt{\din}} \frobnorm{\Phi(t)} \opnorm{X}.
  \label{eq:Et-decomp-loss-term-bound}
\end{align}
From $\cB(t)$, the factors $W_{(s_{\ell}-1):(s_{\ell-1} + 1)}$ satisfy
$
\frobnorm{W_{(s_{\ell}-1):(s_{\ell-1} + 1)}}
\le 2\sqrt{\frac{L}{c}} \dhid^{\frac{s_\ell - s_{\ell-1}-1}{2}}.
$
From $\cB(t)$ and \cref{assumption:rip}, we also get 
$
\frobnorm{W_{(s_{1}-1):1}Y}
\le \frac{5}{4} \dhid^{\frac{s_1-1}{2}} \sigma_{\max}(X).
$
Similarly, we have
$
\frobnorm{W_{L:s_{\ell} + 1}}
\le 2\sqrt{\frac{L}{c}} \dhid^{\frac{L - s_{\ell}}{2}}.
$
Consequently, $C_{(s_1, \dots, s_{\ell})}Y$
admits the following bound:
\begin{align*}
  & \frobnorm{C_{(s_1, \dots, s_{\ell})}Y} \\
  &\leq
  \eta^{\ell} 
  \left( \prod_{k \notin \set{s_1, \dots, s_{\ell}}} \left(1 - \frac{\eta \lambda}{d_k}\right) \right) \cdot
  \left( \frac{25}{16 \sqrt{\din}} \frobnorm{\Phi(t)} \opnorm{X}\right)^\ell \cdot
  \left[
    \frac{5}{4} d_{w}^{\frac{s_1 - 1}{2}} \opnorm{X}
    \cdot 2\sqrt{\frac{L}{c}} \dhid^{\frac{L - s_{\ell}}{2}}
    \prod_{k = 1}^{\ell-1}
    2 \sqrt{\frac{L}{c}} d_{w}^{\frac{s_{k+1} - s_{k} - 1}{2}} 
  \right].
\end{align*}
Note that the last
term equals
\begin{equation}
  \frac{5}{4} d_{w}^{\frac{s_1 - 1}{2}} \opnorm{X}
    \cdot 2\sqrt{\frac{L}{c}} \dhid^{\frac{L - s_{\ell}}{2}}
    \prod_{k = 1}^{\ell-1}
    2 \sqrt{\frac{L}{c}} d_{w}^{\frac{s_{k+1} - s_{k} - 1}{2}} 
    = \frac{5}{4} \opnorm{X}
  \left(2 \sqrt{\frac{L}{c}}\right)^{\ell} \cdot
  d_{w}^{\frac{L - \ell}{2}},
  \label{eq:Et-decomp-last-term-bound}
\end{equation}
and the first term, comprising products for indices different from $\set{s_1, \dots, s_{\ell}}$, satisfies:
\begin{equation}
  \prod_{k \notin \set{s_1, \dots, s_{\ell}}} \left(1 - \frac{\eta \lambda}{d_k}\right)
  \leq \left(1 - \frac{\eta \lambda}{d_w}\right)^{L - \ell},
  \label{eq:Et-decomp-first-term-bound}
\end{equation}
since $\dhid \geq \din$ by assumption.
Putting~\cref{eq:Et-decomp-first-term-bound,eq:Et-decomp-loss-term-bound,eq:Et-decomp-last-term-bound}
together, we obtain
\begin{align*}
  \frobnorm{E(t)} &=
  \frobnorm{d_{w}^{-\frac{L-1}{2}} \din^{-\frac{1}{2}} E_{0}(t) Y} \\
                 &\leq
  \frac{5}{4} \opnorm{X}  d_{w}^{-\frac{L-1}{2}} \din^{-\frac{1}{2}}
  \sum_{\ell = 2}^{L}
  \eta^{\ell}
  \binom{L}{\ell} \left(1 - \frac{\eta \lambda}{d_w}\right)^{L - \ell}
  \left(2 \sqrt{\frac{L}{c}}\right)^{\ell} d_{w}^{\frac{L - \ell}{2}}
  \left(\frac{25}{16 \sqrt{\din}} \frobnorm{\Phi(t)} \opnorm{X}\right)^{\ell}  \\
                 &\leq
 \frac{5}{4} \opnorm{X}  \sqrt{\frac{\dhid}{\din}}
  \sum_{\ell=2}^{L} \left(\frac{C \eta L^{\frac{3}{2}} \opnorm{X} \frobnorm{\Phi(t)}}{\sqrt{\din \cdot d_w} \cdot (1 - \frac{\eta \lambda}{d_w})}\right)^{\ell} \\
                 &\leq
                 \frac{5C \eta L^{\frac{3}{2}} \opnorm{X}^2 \frobnorm{\Phi(t)}}{4\din \cdot (1 - \frac{\eta \lambda}{d_w})}
                 \sum_{\ell = 1}^{L - 1} \left(
                   \frac{C \eta L^{\frac{3}{2}} \opnorm{X} \frobnorm{\Phi(t)}}{(\din d_w)^{1/2} (1 - \nicefrac{\eta \lambda}{d_w})}
                 \right)^{\ell},
\end{align*}
where the second to last inequality was obtained  from the following bounds:
\begin{itemize}
  \item for any $j \in \mathbb{N}$, we have $\binom{L}{j} \leq L^j$;
  \item for any $j \in \mathbb{N}$, we have $\left(1 - \nicefrac{\eta \lambda}{d_w}\right)^{j} \leq 1$;
  \item finally, we relabel $C := 2 \sqrt{\frac{1}{c}} \cdot \frac{25}{16} $ for simplicity.
\end{itemize}
Note that $\eta \lambda \leq \frac{d_w}{2}$ implies $\nicefrac{\eta}{(1 - \frac{\eta \lambda}{d_w})} \leq 2\eta$. Consequently,
\begin{align*}
 \frac{2C \eta L^{3/2} \opnorm{X} \frobnorm{\Phi(t)}}{\sqrt{\din \dhid}} &\leq
     \frac{2C L^{1/2} \sqrt{\din} \frobnorm{\Phi(t)}}{\sqrt{\dhid} \sigma_{\max}(X)} \\ 
                                                                         &\lesssim
    \frac{\sqrt{L} \frobnorm{X} \sqrt{d}}{\sigma_{\max}(X) \sqrt{d_w}} \\
     &\lesssim
 \sqrt{\frac{L \dout \sr(X)}{d_w}} \\
     &\leq \frac{1}{2},
\end{align*}
where the first inequality follows from the upper bound on $\eta$, the second inequality
follows from $\cA(0), \dots, \cA(t-1)$, which together with the definition of $\tau$
imply that $\frobnorm{\Phi(t)} \leq \frobnorm{\Phi(0)} \lesssim \sqrt{\frac{\dout}{\din}} \frobnorm{X}$,
the penultimate inequality follows from the definition of $\sr(X)$ and the last inequality follows
from the lower bound on $\dhid$.
Therefore, the sum is bounded by $1$, which we use in the second inequality in the following.
Putting everything together, we obtain
\begin{align*}
  \frobnorm{E(t)}
  &\leq
  \frac{2 C \eta L^{3/2} \opnorm{X}^2 \frobnorm{\Phi(t)}}{m}
  \sum_{\ell = 1}^{L - 1} \left(
    \frac{2 C \eta L^{3/2} \opnorm{X} \frobnorm{\Phi(t)}}{\sqrt{\din \dhid}}
  \right)^{\ell}
  \\
  &\leq
  \frac{4 C^2 \eta^2 L^3 \opnorm{X}^3 \frobnorm{\Phi(t)}^2}{\din^{3/2} \dhid^{1/2}}
  \cdot \frac{1}{1 - \frac{2 C \eta L^{3/2} \opnorm{X} \frobnorm{\Phi(t)}}{\sqrt{\din \dhid}}}
  \\
  &\lesssim
  \frac{ \eta^2 L^{3} \opnorm{X}^4 \frobnorm{\Phi(t)}}{ \din^2} \sqrt{\frac{\dout \sr(X)}{\dhid}}
  \\
  &\leq
  \frac{ \eta L^2 \opnorm{X}^2 \frobnorm{\Phi(t)}}{ \din} \sqrt{\frac{\dout \sr(X)}{\dhid}} \\
  &\leq
  \frac{17 \eta L \sigma_{\min}^2(X)}{1024 \din} \cdot \frobnorm{\Phi(t)},
\end{align*}
by using the bound on $\eta$ and
after choosing $\dhid$ to satisfy
\[
  \dhid \gtrsim \dout \cdot \sr(X) \cdot L^2 \cdot \kappa^4.
\]
This completes the proof of the Lemma.
\end{proof}
Note that $\cprod{i} \leq 1$ is trivially true. On the other hand, we have the
following lower bound.
\begin{lemma}
    \label{lemma:cprod-i-lb}
    We have that
    $\cprod{i} \geq \frac{1}{4}$ for all $1 \leq i \leq L$.
\end{lemma}
\begin{proof}
    From the definition of $\cprod{i}$ in \cref{eq:c-prod-i} and~\cref{thm:weierstrass}, we have $\cprod{i} = \prod_{j \neq i}^{L} \left(1 - \frac{\eta \lambda}{d_j}\right) \geq
    1 - \sum_{j \neq i}^{L} \frac{\eta \lambda}{d_j}$. Moreover, we have that
    \begin{align*}
      \sum_{j \neq i} \frac{\eta \lambda}{d_j} &\leq
      \sum_{j \neq i} \frac{\din}{d_j} \cdot \frac{\lambda}{L \sigma_{\max}^2(X)} \\
                                               &\leq
                                               \frac{\lambda}{L \sigma_{\max}^2(X)} + \sum_{j \notin \set{i, 1}} \frac{\lambda}{L \sigma_{\max}^2(X)} \frac{\din}{\dhid} \\
                                               &\leq
      \frac{\gamma}{L \kappa^2} \sqrt{\frac{\din}{\dout}} \cdot \left(
        1 + \sum_{j \notin \set{i, 1}} \frac{1}{L^2}
      \right) \\
                                               &\leq
                                               \frac{\gamma}{L} \sqrt{\frac{\din}{\dout}} \cdot \left(1 + \frac{1}{L}\right) \\
                                               &\leq \frac{3 \gamma}{4} \sqrt{\frac{\din}{\dout}} \\
                                               &\leq \frac{3}{4},
    \end{align*}
    where the first inequality follows from the upper bound on $\eta$, the second inequality follows
    from the fact that $d_j = \dhid$ for all $j > 1$ and $d_1 = \din$, with $\dhid > \din$, the
    third inequality follows from the lower bound $\dhid \geq L \cdot \din$, the penultimate inequality
    follows from the fact the function $L \mapsto \frac{1}{L} \left(1 + \frac{1}{L}\right)$ is
    decreasing in $L$ and equal to $\frac{3}{4}$ for $L = 2$, and the last inequality
    follows from the assumption that $\din \leq \dout$ and $\gamma \leq 1$.
\end{proof}
Before we prove the event $\cA(t)$, we prove the following Lemma.
\begin{lemma}
  \label{lemma:Pt-extreme-eigenvalues}
  Under the event $\cB(t)$, the following holds:
  \begin{align*}
    \lambda_{\min}(P(t)) &\geq
    \left(\frac{9}{32}\right)^2 \cdot \frac{L \sigma_{\min}^2(X)}{\din}, \quad
    \lambda_{\max}(P(t)) \leq
    \frac{3 L \sigma_{\max}^2(X)}{\din}.
  \end{align*}
\end{lemma}
\begin{proof}
  From~\cref{lemma:cprod-i-lb}, it follows that $\cprod{i} \in [\frac{1}{4}, 1]$. Consequently \cref{lemma:P-k-spectrum} yields
  \begin{align*}
    \lambda_{\min}(P(t)) &\geq \frac{1}{4 d_{w}^{L-1} \din} \sum_{i = 1}^L \sigma_{\min}^2(W_{L:(i+1)}(t)) \sigma_{\min}^2(W_{(i-1):1}(t)Y) \\
                         &\geq
                         \frac{1}{4 d_w^{L-1} \din} \sum_{i = 1}^L \left(\frac{3}{4} d_{w}^{\frac{L - i}{2}}\right)^2 \left(\frac{3}{4} d_{w}^{\frac{i-1}{2}} \sigma_{\min}(X) \right)^2 \\
                         &= \frac{1}{4 d_{w}^{L-1} \din} \sum_{i = 1}^L \left( \frac{9}{16} \right)^2 d_{w}^{L - 1} \sigma_{\min}^2(X) \\
                         &\geq \frac{81 L \sigma_{\min}^2(X)}{1024 \din},
  \end{align*}
  where the second inequality uses~\cref{eq:event-B-prestop}, the assumption that the event $\cB(t)$ holds.
 
  Similarly, we have
  \begin{align*}
    \lambda_{\max}(P(t)) &\leq
    \frac{1}{d_{w}^{L-1} \din} \sum_{i = 1}^L \sigma_{\max}^2(W_{L:(i+1)}(t)) \sigma_{\max}^2(W_{(i-1):1}(t)Y) \\
                         &\leq
    \frac{1}{d_{w}^{L-1} \din} \sum_{i = 1}^L
    \left(\frac{5}{4} d_w^{\frac{L - i}{2}}\right)^2
    \left(\frac{5}{4} d_w^{\frac{i-1}{2}} \sigma_{\max}(X) \right)^2 \\
                         &\leq
    \frac{L \sigma_{\max}^2(X)}{\din} \cdot \left( \frac{25}{16} \right)^2 \\
                         &\leq
    \frac{3L \sigma_{\max}^2(X)}{\din},
  \end{align*}
  which completes the proof.
\end{proof}



\begin{lemma}
    \label{lemma:step-i-all-else-implies-A}
    For any $0 \leq t < \tau$, $\set{\set{\cA(j)}_{j < t}, \set{\cB(j)}_{j \leq t}} \implies \cA(t)$. Moreover, we have
    \begin{equation}
        \cA(t) \implies
        \frobnorm{\Phi(t+1)} \leq
        \left(
        1 - \frac{\eta L \sigma_{\min}^2(X)}{32\din}
        \right) \frobnorm{\Phi(t)}.
        \label{eq:At-strengthened-step-1}
    \end{equation}
\end{lemma}
\begin{proof}
    Recall the decomposition of the error from \cref{eq:error-evolution-I}:
    \[
        \mathrm{vec}(\Phi(t+1)) = (I - \eta P(t))
        \mathrm{vec}(\Phi(t)) +
        \mathrm{vec}(E(t)) +
        (\call - 1) \mathrm{vec}(U(t)).
    \]
    Taking norms on both sides and invoking the
    bound on $\frobnorm{E(t)}$ from~\cref{lemma:Bt-implies-bound-on-E}, we obtain
 \begin{align}
      \|\mathrm{vec}(\Phi(t+1))\| &=
      \|(I - \eta P(t)) \mathrm{vec}(\Phi(t)) +
      \mathrm{vec}(E(t)) +
    (\call - 1) \mathrm{vec}(U(t))\| \notag \\
                                     &\leq
                                     \opnorm{(I - \eta P(t))} \frobnorm{\Phi(t)}
                                     + \frobnorm{E(t)} + \abs{\call - 1} \frobnorm{U(t)} \notag \\
                                     &\leq \left(1 - \left(\frac{9}{32}\right)^2 \frac{\eta L \sigma^2_{\min}(X)}{\din} + 
                                     \frac{17 \eta L \sigma^2_{\min}(X)}{1024 \din}\right) \frobnorm{\Phi(t)}
      + \abs{\call - 1} \frobnorm{U(t)} \notag \\
      &\leq \left(1 - \frac{\eta L \sigma_{\min}^2(X)}{16\din}\right) \frobnorm{\Phi(t)}
      + \left[\frac{(L - 1) \eta \lambda}{\dhid} + \frac{\eta \lambda}{\din}\right] \frobnorm{U(t)} \notag \\
      &\leq \left(1 - \frac{\eta L \sigma_{\min}^2(X)}{16\din}\right) \frobnorm{\Phi(t)}
      + \frac{5 \eta \lambda}{2\din} \cdot \frobnorm{X} \sqrt{\frac{\dout}{\din}},
      \label{eq:loss-evolution-decomposition}
    \end{align}
    where the penultimate inequality follows from~\cref{lemma:one-minus-folded-product} and
    \eqref{eq:loss-evolution-decomposition} follows from
    \[
      d_{w} \geq L \din \implies \frac{(L-1) \eta \lambda}{d_w} \leq \frac{\eta \lambda}{\din},
      \quad \text{and} \quad
      \frobnorm{U(t)} \leq \frac{5}{4} \frobnorm{X} \sqrt{\frac{\dout}{\din}}
    \]
    If $t < \tau$, then from the definition of the stopping time $\tau$ in~\eqref{eq:tau def phi} and the identity $\lambda = \gamma\sigma_{\min}^2(X) \sqrt{\sfrac{m}{d}}$,
    it follows that
    \begin{align*}
      \frobnorm{\Phi(t+1)}  &\leq
      \left(1 - \frac{\eta L \sigma_{\min}^2(X)}{16 \din}\right) \frobnorm{\Phi(t)}  +
      \frac{5 \eta \lambda \sqrt{\dout}}{2\din \sqrt{\din}} \frobnorm{X} \\
      &\leq
      \left(1 - \frac{\eta L \sigma_{\min}^2(X)}{32 \din}\right) \frobnorm{\Phi(t)},
     \end{align*}
    which proves the inequality in~\eqref{eq:At-strengthened-step-1}.
\end{proof}


\begin{corollary}
    \label{corollary:length-of-step-1}
    With high probability, the stopping time $\tau$ satisfies
    \[
        \tau \leq
        \frac{32 \din}{\eta L \sigma_{\min}^2(X)}
        \log\left(
          \frac{L \sigma_{\min}^2(X)}{35 \lambda}
        \right).
    \]
\end{corollary}
\begin{proof}
    For any $t < \tau$, \cref{lemma:step-i-all-else-implies-A} implies
    \begin{align*}
        \frobnorm{\Phi(t)} &\leq
        \left(
        1 - \frac{\eta L \sigma_{\min}^2(X)}{32\din}
      \right) \frobnorm{\Phi(t-1)} \\ &\leq
        \left(1 - \frac{\eta L \sigma_{\min}^2(X)}{32\din} \right)^{t} \frobnorm{\Phi(0)} \\
        &\leq
        \exp\left(
        -\frac{t \eta L \sigma_{\min}^2(X)}{32 \din}
    \right) \frobnorm{\Phi(0)} \\ &\leq
        \exp\left(
        -\frac{t \eta L \sigma_{\min}^2(X)}{32 \din}
      \right) \frac{11}{5} \sqrt{\frac{\dout}{\din}} \frobnorm{X},
    \end{align*}
    where the penultimate inequality follows from the identity $1 - x \leq \exp(-x)$ and
    the last inequality follows from~\cref{lemma:initial-regression-error}. Finally,
    we obtain
    \[
      t \geq \frac{32 \din}{\eta L \sigma^2_{\min}(X)}
        \log\left(
          \frac{L \sigma_{\min}^2(X)}{35 \lambda}
        \right) \implies
        \frobnorm{\Phi(t)} \leq
        \frac{80 \lambda \frobnorm{X}}{
        L \sigma_{\min}^2(X)
      } \sqrt{\frac{\dout}{\din}}
      = \frac{80 \gamma \frobnorm{X}}{L},
    \]
    which implies the stated upper bound on
    $\tau$.
\end{proof}
Next we will prove the event $\cC(t)$ (\cref{eq:event-C-prestop}).
\begin{lemma}
  \label{lemma:Bj-implies-Ct-phase-1}
  For any $t \leq \tau$, we have that
  $\set{\cA(j), \cB(j)}_{j < t} \implies \cC(t)$: 
  \begin{equation}
    \frobnorm{W_{i}(t) - \left(1 - \frac{\eta \lambda}{d_i}\right)^{t} W_{i}(0)}
    \lesssim
    \frac{\kappa^2 \sqrt{d \sr(X)}}{L} := R
    , \quad
    \text{for all $i = 1, \dots, L$.}
    \label{eq:Wi-travel-distance-improved}
  \end{equation}
\end{lemma}
\begin{proof}
Given \cref{lemma:difference-norm} we obtain the bound
  \begin{align*}
     &\frobnorm{W_{i}(t) - \left(1 - \frac{\eta \lambda}{d_i}\right)^t W_{i}(0)}\\
    &\leq \eta \sum_{j = 0}^{t-1} \left(1 - \frac{\eta \lambda}{d_i}\right)^{t - 1 - j}
    d_{w}^{-\frac{L - 1}{2}} \din^{-\frac{1}{2}}
    \frobnorm{W_{L:(i+1)}(j)\Phi(j) W_{(i-1):1}(j) Y} \\
    &\leq \eta \sum_{j = 0}^{t-1} \left(1 - \frac{\eta \lambda}{d_i}\right)^{t - 1 - j}
    d_{w}^{-\frac{L - 1}{2}} \din^{-\frac{1}{2}}
    \opnorm{W_{L:(i+1)}(j)} \frobnorm{\Phi(j)} \opnorm{W_{(i-1):1}(j) Y} \\
    &\leq
    \eta \sum_{j = 0}^{t-1} \left(1 - \frac{\eta \lambda}{d_i}\right)^{t-1-j}
    \left(1 - \frac{\eta L \sigma_{\min}^2(X)}{32 \din}\right)^{j}
    \left( \frac{5}{4} d_{w}^{\frac{L-i}{2}} \right) \left(\frac{5}{4} d_{w}^{\frac{i-1}{2}} \sigma_{\max}(X)\right)
    \frac{\frobnorm{\Phi(0)}}{d_{w}^{\frac{L-1}{2}} \din^{\frac{1}{2}}} \\
    &\leq
    \eta \frac{25 \frobnorm{\Phi(0)} \opnorm{X}}{16 \sqrt{\din}}
    \sum_{j = 0}^{t - 1} \left(1 - \frac{\eta L \sigma_{\min}^2(X)}{32 \din}\right)^{j} \\
    &\leq
    \eta \frac{25 \frobnorm{\Phi(0)} \opnorm{X}}{16 \sqrt{\din}}
    \cdot \frac{32 \din}{\eta L \sigma_{\min}^2(X)} \\
    &=
    \frac{C \frobnorm{\Phi(0)} \opnorm{X} \sqrt{\din} }{L \sigma_{\min}^2(X)}\\
    & \leq \frac{C\sqrt{\dout} \cdot \frobnorm{X} \opnorm{X}}{L \sigma^2_{\min}(X)}\\
    & \leq \frac{\kappa^2 \sqrt{d \sr(X)}}{L},
  \end{align*}
  with $C=50$,
  which is independent of the choice of layer $i$. This completes the proof.
\end{proof}
Finally we prove the event $\cB(t)$ (\cref{eq:event-B-prestop}).
\begin{lemma}\label{lem:step1-b(t)-proof}
  \label{lemma:Ct-implies-Bt-phase-1}
    We have that $\cC(t) \implies
    \cB(t)$ for any $t \leq \tau$.
\end{lemma}
\begin{proof}
  To prove $\cB(t)$, we need to control the extremal singular values of several matrix products.

  \paragraph{Bounding $\opnorm{W_{j:i}(t)}$.}
  Fix any $i > 1$ and $j \geq i$. We start with the following decomposition:
  \begin{align*}
    W_{j:i}(t) &=
    \prod_{\ell = j}^{i} W_{\ell}(t) \\ 
               &=
               \prod_{\ell=j}^{i} \bigg( \Big( 1 - \frac{\eta \lambda}{d_{\ell}} \Big)^{t} W_{\ell}(0) + \underbrace{W_{\ell}(t) - \Big(1 - \frac{\eta \lambda}{d_{\ell}}\Big)^{t} W_{\ell}(0)}_{\Delta_{\ell}(t)} \bigg) \\
               &= \begin{aligned}[t]
      & W_{j:i}(0) \cdot \prod_{\ell = j}^{i} \left(1 - \frac{\eta \lambda}{d_{\ell}}\right)^{t}  + \sum_{s = 1}^{j - i + 1} \sum_{i \leq k_{1}, \dots, k_{s} \leq j} \widetilde{W}_{j:(k_{s} + 1)}(0) \Delta_{k_s}(t) \dots \Delta_{k_1}(t) \widetilde{W}_{(k_1 - 1):i}(0),
      \end{aligned}
  \end{align*}
  using a slight abuse of the notation for $\widetilde{W}_{j:i}$ introduced in~\eqref{eq:tildeW-defn}:
  \[
    \widetilde{W}_{j:i}(0) = W_{j:i} \cdot \prod_{\ell = j}^{i} \left(1 - \frac{\eta \lambda}{d_{\ell}}\right)^{t}.
  \]
  Continuing, we have the following upper bound:
  \begin{align*}
     \opnorm{\widetilde{W}_{j:(k_{s}+1)}(0) \Delta_{k_{s}}(t) \dots \Delta_{k_{1}}(t) \widetilde{W}_{(k_1 - 1):i}(0)} &\leq
    R^{s}
    \bigg[ \prod_{\substack{\ell = j\\\ell \notin \set{k_{1}, \dots, k_{s}}}}^{i} \left(1 - \frac{\eta \lambda}{d_{\ell}}\right)^{t} \bigg]
    \left( \sqrt{\frac{L}{c}}\right)^{s + 1} \dhid^{\frac{j - i + 1 - s}{2}} \\
    &\leq
     \sqrt{\frac{L}{c}} \dhid^{\frac{j - i + 1}{2}} \cdot
    \left( \frac{C R \sqrt{L}}{\sqrt{d_w}} 
     \right)^{s} 
  \end{align*}
  Summing up over all possible $k_{1}, \dots, k_{s}$ for all possible $s = 1$ to $s = j - i + 1$,
  we have
  \begin{align*}
      & \sum_{s = 1}^{j - i + 1} \sum_{i \leq k_{1}, \dots, k_{s} \leq j} \opnorm{ \widetilde{W}_{j:(k_{s} + 1)}(0) \Delta_{k_s}(t) \dots \Delta_{k_1}(t) \widetilde{W}_{(k_1 - 1):i}(0)} \\
      & \leq
       \sqrt{\frac{L}{c}} \dhid^{\frac{ j - i + 1 }{2}}
      \sum_{s = 1}^{j - i + 1} \binom{j - i + 1}{s} \cdot \left( \frac{C R \sqrt{L}}{\sqrt{d_w}} 
     \right)^{s}  \\
      &\leq
       \sqrt{\frac{L}{c}} \dhid^{\frac{j - i + 1}{2}}
      \sum_{s = 1}^{j - i + 1}
     \left( \frac{C R \sqrt{L}}{\sqrt{d_w}} 
     \right)^{s} ,
  \end{align*}
  where the last inequality follows from the bound
  \(
    \binom{j - i + 1}{s} \leq \binom{L}{s} \leq L^{s}.
  \)
  Finally, by~\cref{lemma:truncated-geometric-series},
  \begin{align*}
    \sum_{s = 1}^{j - i + 1}
     \left( \frac{C R L^{3/2}}{\sqrt{d_w}} 
     \right)^{s} 
    & \lesssim
    \left( \frac{C R L^{3/2}}{\sqrt{d_w}} 
     \right) \cdot
    \frac{1}{1 -\left( \frac{C RL^{3/2}}{\sqrt{d_w}} 
     \right) }
    \leq \frac{1}{3},
  \end{align*}
  as long as we choose $\dhid$ such that
  \[
    \dhid \gtrsim R^2 L^{3} \asymp
    \frac{L \dout \kappa^4}{\sigma_{\max}^2(X)}
    \Leftrightarrow
    \left( \frac{ R L^{3/2}}{\sqrt{d_w}} 
     \right) \lesssim \frac{1}{4}.
  \]
  Putting everything together, we arrive at
  \begin{align*}
    \opnorm{W_{j:i}(t)} & \leq
    \opnorm{W_{j:i}(0)} \prod_{\ell = j}^{i} \left(1 - \frac{\eta \lambda}{d_{\ell}}\right)^{t}
    + \frac{1}{3} \sqrt{\frac{L}{c}} \dhid^{\frac{j - i + 1}{2}} \leq
    \frac{4}{3} \sqrt{\frac{L}{c}} \dhid^{\frac{j - i + 1}{2}},
  \end{align*}
  using $(1 - \nicefrac{\eta \lambda}{d_{\ell}}) \leq 1$ and~\cref{lemma:norm-product-bounded} in the last inequality. This proves the
  first bound in the definition of $\cB(t)$. \\
  \paragraph{Bounding $\sigma_{\min}(W_{i:1}Y)$ and $\opnorm{W_{i:1}Y}$.}
  To control the singular values of $W_{i:1}(t) Y$ for $i < L$, we write
  \begin{align*}
      W_{i:1}(t) Y &=
      \Big( \prod_{ \ell = i }^{1} W_{\ell}(t)
      \Big) Y \\
      &=
      \bigg[ \prod_{\ell = i}^{1} 
      \bigg( \Big(1 - \frac{\eta \lambda}{d_{\ell}}\Big)^{t} W_{\ell}(0) +
      \Big[ W_{\ell}(t) - \Big(1 - \frac{\eta \lambda}{d_{\ell}}\Big)^{t} W_{\ell}(0)
      \Big] \bigg) \bigg] Y \\
      &=
      \begin{aligned}[t]
        & W_{i:1}(0) Y \cdot
        \prod_{\ell = i}^{1} \Big(1 - \frac{\eta \lambda}{d_{\ell}}\Big)^{t}  +
        \sum_{s = 1}^{i}
        \sum_{1 \leq k_{1}, \dots, k_{s} \leq i}
        \widetilde{W}_{i:(k_{s}+1)}(0)
        \Delta_{k_{s}} \dots \Delta_{k_1}
        \widetilde{W}_{(k_1 - 1):1}(0) Y
      \end{aligned}
  \end{align*}
  From the above decomposition and Weyl's inequality, it follows that
  \begin{equation}
      \abs{\sigma_{j}(W_{i:1}(t) Y) - \sigma_{j}(\widetilde{W}_{i:1}(0) Y)}
      \leq \sum_{s = 1}^{i} \sum_{(k_1, \dots, k_{s})} \opnorm{
        \widetilde{W}_{i:(k_{s}+1)}(0)
        \Delta_{k_{s}} \dots \Delta_{k_1}
        \widetilde{W}_{(k_{1} - 1):1}(0) Y
      }.
      \label{eq:singular-value-bound-phase-I}
  \end{equation}
  We now turn to bound the terms in the sum on the RHS of~\eqref{eq:singular-value-bound-phase-I}.
  First, note that
  \begin{align*}
    &\opnorm{
      \widetilde{W}_{i:(k_{s}+1)}(0)
      \Delta_{k_{s}} \dots \Delta_{k_1}
      \widetilde{W}_{(k_{1} - 1):1}(0) Y
    }\\
    &=
    \prod_{\substack{\ell = i\\\ell \notin \set{k_1, \dots, k_{s}}}}^{1}
    \left(1 - \frac{\eta \lambda}{d_{\ell}}\right)^{t}
    \opnorm{
      W_{i:(k_{s}+1)}(0) \Delta_{k_{s}} \dots \Delta_{k_{1}} W_{(k_{1} - 1):1}(0) Y
    } \\
    &\leq
    \opnorm{W_{i:(k_{s}+1)}(0) \Delta_{k_{s}} \dots \Delta_{k_{1}} W_{(k_{1}-1):1}(0) Y} \\
    &\leq
    \left(R \cdot 2 \sqrt{\frac{L}{c}}\right)^{s} \dhid^{\frac{i - k_{1} + 1 - s}{2}} \cdot
    \frac{6}{5} \dhid^{\frac{k_{1} - 1}{2}} \sigma_{\max}(X) \\
    &=
    \frac{6}{5} \cdot \left(R \cdot 2 \sqrt{\frac{L}{c}}\right)^{s} \dhid^{\frac{i - s}{2}} \sigma_{\max}(X) \\
    &=
    \frac{6}{5} \left(R \cdot 2 \sqrt{\frac{L}{c \dhid}}\right)^{s} \dhid^{\frac{i}{2}} \cdot \sigma_{\max}(X).
  \end{align*}
  Again, summing over all possible $(k_{1}, \dots, k_{s})$ for $s = 1$ to $i$
  yields the upper bound
  \begin{align*}
     \frac{6\sigma_{\max}(X) \cdot \dhid^{\frac{i}{2}}}{5} \cdot \sum_{s = 1}^{i} \binom{i}{s}
    \left(R \cdot 2 \sqrt{\frac{L}{c \dhid}}\right)^{s} 
    &\leq
    \frac{6 \sigma_{\max}(X) \cdot \dhid^{\frac{i}{2}}}{5}
    \sum_{s = 1}^{i} \left(R \cdot 2 i \sqrt{\frac{L}{c\dhid}}\right)^{s} \\
    &\leq
    \frac{6 \sigma_{\max}(X) \cdot \dhid^{\frac{i}{2}}}{5}
    \frac{R \cdot 2 L \sqrt{\frac{L}{c \dhid}}}{1 - R \cdot  2L \sqrt{\frac{L}{c \dhid}}} \\ 
    &\leq
    \frac{6 \sigma_{\max}(X) \cdot \dhid^{\frac{i}{2}}}{5}
    \frac{R \cdot 4 L \sqrt{\frac{L}{c \dhid}}}{3} \\
    &\leq
    c_{\mathsf{b}} \cdot \sigma_{\min}(X) \cdot \dhid^{\frac{i}{2}},
  \end{align*}
  valid for any $\dhid$ satisfying the following identity:
  \[
    \frac{R \cdot 8L}{5} \sqrt{\frac{L}{c \dhid}} \leq \frac{c_{\mathsf{b}}}{\kappa}
    \Leftrightarrow
    \dhid \gtrsim  \kappa^2 R^2 L^3 c_{\mathsf{b}}^2
    \asymp
    L \dout \cdot \frac{\kappa^6 \sr(X)}{c_{\mathsf{b}}^2},
  \]
   where $c_{\mathsf{b}}$ is a free parameter. Plugging
  the derived bound into~\eqref{eq:singular-value-bound-phase-I} yields
  \begin{align*}
    \sigma_{\max}(W_{i:1}(t)Y) &\leq
    \sigma_{\max}(\widetilde{W}_{i:1}(0) Y) + c_{\mathsf{b}} \cdot \sigma_{\min}(X) \cdot \dhid^{\frac{i}{2}} \\
                               &\leq
                               \sigma_{\max}(W_{i:1}(0) Y) + c_{\mathsf{b}} \cdot \sigma_{\max}(X) \cdot \dhid^{\frac{i}{2}} \\
                               &\leq
                               \left( \frac{6}{5} + c_{\mathsf{b}} \right) \sigma_{\max}(X) \cdot \dhid^{\frac{i}{2}} \\
                               &\leq
                               \frac{5}{4} \sigma_{\max}(X) \cdot \dhid^{\frac{i}{2}},
  \end{align*}
  after choosing $c_{\mathsf{b}} \leq \frac{1}{20}$. This proves the second bound
  in the definition of $\cB(t)$.


  Similarly, we have the following lower bound:
  \begin{align*}
    \sigma_{\min}(W_{i:1}(t) Y) &\geq
    \sigma_{\min}(\widetilde{W}_{i:1}(0) Y) - c_{\mathsf{b}} \cdot \sigma_{\min}(X) \cdot \dhid^{\frac{i}{2}} \\
                                &\geq
                                \sigma_{\min}(W_{i:1}(0) Y) \cdot \prod_{\ell = i}^{1} \left(1 - \frac{\eta \lambda}{d_{\ell}}\right)^{t}
    - c_{\mathsf{b}} \cdot \sigma_{\min}(X) \cdot \dhid^{\frac{i}{2}} \\
                                &\geq
                                \left[ \left(1 - \frac{1}{20L}\right)^{i} \cdot \frac{4}{5} - c_{\mathsf{b}} \right] \cdot \sigma_{\min}(X) \cdot \dhid^{\frac{i}{2}} \\
                                &\geq
                                \left[ \left(1 - \frac{1}{20L}\right)^{L} \cdot \frac{4}{5} - c_{\mathsf{b}} \right] \cdot \sigma_{\min}(X) \cdot \dhid^{\frac{i}{2}} \\
                                &\geq
    \left[ \frac{19}{20} \cdot \frac{4}{5} - c_{\mathsf{b}} \right] \cdot \sigma_{\min}(X) \cdot \dhid^{\frac{i}{2}} \\
                                &\geq
                                \frac{3 \sigma_{\min}(X)}{4} \cdot \dhid^{\frac{i}{2}},
  \end{align*}
  where the third inequality follows from~\cref{lemma:contraction-factor-small-phase-1}, the second to last inequality follows from \cref{lemma:small-contraction-factor-lower-bound} and the last
  inequality follows from choosing $c_{\mathsf{b}} \leq \frac{1}{100}$. This proves the fourth bound
  in the definition of $\cB(t)$.\\
  
  \paragraph{Bounding $\sigma_{\min}(W_{L:i})$ and $\opnorm{W_{L:i}}$.}
  We now furnish upper and lower bounds for singular values of $W_{L:i}(t)$, for $i > 1$. By an analogous
  argument to the one employed for $W_{j:i}(t)$, when $j < L$, we arrive at
  \begin{align}
      W_{L:i}(t) &= {W}_{L:i}(0) \prod_{\ell = L}^{i} \left(1 - \frac{\eta \lambda}{d_{\ell}}\right)^{t} +
      \sum_{s = 1}^{L - i} \sum_{i \leq k_{1}, \dots, k_{s} \leq L}
      \widetilde{W}_{L:(k_{s}+1)}(0) \Delta_{k_{s}} \dots \Delta_{k_{1}}
      \widetilde{W}_{(k_{1} - 1):i}(0)
      \label{eq:step-I-decomp-WL-i}
  \end{align}
  As before, we bound each summand on the RHS of~\eqref{eq:step-I-decomp-WL-i}. We have
  \begin{align*}
      \opnorm{\widetilde{W}_{L:(k_{s}+1)}(0) \Delta_{k_{s}} \dots \Delta_{k_{1}}
      \widetilde{W}_{(k_{1} - 1):i}(0)}  &\leq
     R^s
      \cdot \frac{6}{5} \dhid^{\frac{L - k_{s}}{2}}
      \cdot \left(2 \sqrt{\frac{L}{c}} \right)^{s}
      \dhid^{\frac{k_{s} - i + 1 - s}{2}} \\ &=
      \frac{6}{5} \cdot
      \left( \frac{R L^{1/2}}{\sqrt{d_w}} \right)^{s} \cdot \dhid^{\frac{L - i + 1}{2}}
  \end{align*}
  Adding up all the summands yields the upper bound
  \begin{align*}
       \sum_{s = 1}^{L - i} \sum_{i \leq k_{1}, \dots, k_{s} \leq L}
      \opnorm{\widetilde{W}_{L:(k_{s}+1)}(0) \Delta_{k_{s}} \dots \Delta_{k_{1}}
      \widetilde{W}_{(k_{1} - 1):i}(0)}  &\leq
      \frac{6 \dhid^{\frac{L - i + 1}{2}}}{5} \cdot \sum_{s = 1}^{L - i} \binom{L - i}{s}
      \left( \frac{R L^{1/2}}{\sqrt{d_w}} \right)^{s} \\
      &\leq
      \frac{6 \dhid^{\frac{L - i + 1}{2}}}{5} \cdot
      \sum_{s = 1}^{L - i}
      \left( \frac{R L^{3/2}}{\sqrt{d_w}} \right)^{s} \\ &\leq
      \frac{6 \dhid^{\frac{L - i + 1}{2}}}{5} \cdot
      \frac{R L^{3/2}}{\sqrt{d_w}},
  \end{align*}
  where the penultimate inequality follows from $\binom{k}{i} \leq k^{i}$ and the last
  inequality follows from~\cref{lemma:truncated-geometric-series}. Again, we introduce a free parameter
  $\bar{c}_{\mathsf{b}}$ and require
  \[ \frac{R L^{3/2}}{\sqrt{d_w}} = 
    \frac{\kappa^2 \sqrt{L \dout \sr(X)}}{\sqrt{ \dhid}} 
    \lesssim \bar{c}_{\mathsf{b}} \Leftrightarrow
    d_{w} \gtrsim \frac{L \dout \sr(X) \kappa^4}{\bar{c}_{\mathsf{b}}^2}.
  \]
  Returning to~\eqref{eq:step-I-decomp-WL-i}, we obtain the upper bound
  \begin{align*}
      \sigma_{\max}(W_{L:i}(t))
      &\leq \sigma_{\max}(W_{L:i}(0)) +
      \bar{c}_{\mathsf{b}} \cdot \dhid^{\frac{L - i + 1}{2}} \leq \left( \frac{6}{5} + \bar{c}_{\mathsf{b}} \right) \cdot \dhid^{\frac{L - i + 1}{2}} \leq \frac{5}{4} \cdot \dhid^{\frac{L - i + 1}{2}},
  \end{align*}
  choosing $\bar{c}_{\mathsf{b}} \leq \frac{1}{20}$. This proves the third inequality
  in $\cB(t)$; similarly by using ~\cref{lemma:contraction-factor-small-phase-1} and \cref{lemma:small-contraction-factor-lower-bound} we get the lower bound
  \begin{align*}
     \sigma_{\min}(W_{L:i}(t)) &\geq 
     \sigma_{\min}(W_{L:i}(0)) \cdot \left(1 - \frac{1}{20 L}\right)^{L - i + 1}
     - \bar{c}_{\mathsf{b}} \cdot \dhid^{\frac{L - i + 1}{2}} \\
     & \geq
    \sigma_{\min}(W_{L:i}(0)) \cdot \left(1 - \frac{1}{20 L}\right)^{L}
     - \bar{c}_{\mathsf{b}} \cdot \dhid^{\frac{L - i + 1}{2}} \\
     &\geq
     \left(0.95 \cdot \frac{4}{5} - \bar{c}_{\mathsf{b}}\right) \cdot \dhid^{\frac{L - i + 1}{2}} \\
     &\geq \frac{3}{4} \cdot \dhid^{\frac{L - i + 1}{2}},
  \end{align*}
  after choosing $\bar{c}_{\mathsf{b}} \leq \frac{1}{100}$. This proves the final inequality
  making up the event $\cB(t)$.
\end{proof}




\begin{lemma}
  \label{lemma:contraction-factor-small-phase-1}
    For any $t \leq \tau$, it follows that
    \[
        \left(1 - \frac{\eta \lambda}{d_i} \right)^t \geq 1 - \frac{1}{20L}.
    \]
\end{lemma}
\begin{proof}
    From~\cref{thm:weierstrass}, it follows that
    \[
        \left( 1 - \frac{\eta \lambda}{d_i} \right)^{t} \geq
        1 - \frac{t \eta \lambda}{d_i}
        \geq 1 - \frac{\tau \eta \lambda}{d_i}.
    \]
    From~\cref{corollary:length-of-step-1}, the quantity above is at least
    \begin{align*}
        1 - \frac{\tau \eta \lambda}{d_i} &\geq
        1 - \frac{32 \din}{\eta \sigma_{\min}^2(X) L} \log\left(
          \frac{L \sigma_{\min}^2(X)}{35 \lambda}
        \right) \cdot \frac{\eta \lambda}{d_i} \\
        &=
        1 - 32 \cdot \frac{\lambda}{L \sigma_{\min}^2(X)} \cdot
        \frac{\din}{d_i} \cdot
        \log\left(
            \frac{L \sigma_{\min}^2(X)}{35 \lambda}
        \right).
    \end{align*}
    We now argue that for small enough $\lambda$,
    the last term is at most $1 - \frac{1}{20L}$.
    Indeed,
    \[
        \frac{35 \lambda}{L \sigma_{\min}^2(X)} \log \left(
          \frac{L \sigma_{\min}^2(X)}{35 \lambda}
        \right)
        \leq \frac{1}{20}
        \Leftrightarrow
        20
        \leq
        \frac{20 \cdot 35 \lambda}{L \sigma_{\min}^2(X)} \exp\left(\frac{L \sigma_{\min}^2(X)}{20 \cdot 35 \lambda}\right).
    \]
    The above inequality is itself implied by the assumption that $\lambda \leq 
        \frac{L \sigma_{\min}^2(X)}{400 \cdot 35}$, which implies
    \[
        \frac{20 \cdot 35 \lambda}{L \sigma_{\min}^2(X)} \exp\left(\frac{L \sigma_{\min}^2(X)}{20 \cdot 35 \lambda}\right)
        \geq
        \frac{L \sigma_{\min}^2(X)}{20 \cdot 35 \lambda} \geq 20,
    \]
    using the inequality $x \exp(1 / x) \geq \frac{1}{x}$ for all $x > 0$ above.
    Therefore,
    \[
        1 - \frac{\tau \eta \lambda}{d_i} \geq
        1 - \frac{32}{35} \frac{35 \lambda}{L \sigma_{\min}^2(X)} \log\left(\frac{L \sigma_{\min}^2(X)}{35 \lambda}\right)
        \frac{\din}{d_i}
        \geq
        1 - \frac{32}{35 \cdot 20 L} \geq 1 - \frac{1}{20 L},
    \]
    which completes the proof of the claim.
\end{proof}
\begin{lemma}
    \label{lemma:small-contraction-factor-lower-bound}
    For any $L \geq 2$, we have that
    \[
        \left(1 - \frac{1}{20 L} \right)^{L}
        \geq 0.95.
    \]
\end{lemma}
\begin{proof}
    The function $x \mapsto \left(1 - \frac{1}{20x}\right)^{x}$ is monotone increasing for
    all $x \geq 1$. Therefore, 
    \[
        \left(1 - \frac{1}{20L}\right)^{L} \geq \left( 1 - \frac{1}{40} \right)^{2}
        = \left( \frac{39}{40} \right)^2 > 0.95.
    \]
\end{proof}

\begin{proof}[Proof of \cref{thm:step1-induction}]
We prove this theorem by induction. The base case follows from~\cref{lemma:step-i-properties-at-initialization}.
Now, suppose that all events $\cA(t)$, $\cB(t)$ and $\cC(t)$ hold up to some arbitrary index $t < \tau$. Then:
\begin{itemize}
  \item The event $\cC(t+1)$ holds by~\cref{lemma:Bj-implies-Ct-phase-1};
  \item The event $\cB(t+1)$ holds by~\cref{lemma:Ct-implies-Bt-phase-1} and the previous item;
  \item Finally, the event $\cA(t+1)$ holds by~\cref{lemma:step-i-all-else-implies-A} and the preceding item.
\end{itemize}
This completes the proof of the theorem.
\end{proof}


\subsection{Step 2: Regression error stays small}
\label{sec:subsec:Step 2: he error stays small}
In \cref{sec:subsec:Step 1: Rapid early convergence} we have shown that after $\tau$ iterations our regression error is small; namely,
$
  \frobnorm{\Phi(\tau)} \leq \frac{80 \gamma}{L} \frobnorm{X}.
$
We now want to show that the regression error remains small until at least iteration $T$.
In particular, we will show that
$
  \frobnorm{\Phi(t)} \leq C_1 \gamma\frobnorm{X}
$
for all $\tau \leq t \leq T$.
This we will show again by induction over the events stated in the following theorem.
\begin{theorem}\label{thm:step2-induction}
 Given $\tau$ defined in \eqref{eq:tau def phi} and $T$ defined in \eqref{eq:T def appendix}, then for all $\tau \leq t \leq T$ the following events hold with probability of at least $1-e^{-\Omega(d)}$ over the random initialization,
    \begin{subequations}
	\begin{align}
    \cA(t) & := \set*{\frobnorm{\Phi(t)} \leq C_1\gamma\frobnorm{X}}
		\label{eq:induction-a} \\
		\cB(t) & := \left\{ \begin{array}{rclrr}
			            \sigma_{\max}(W_{j:i}(t))   & \leq & \left( 2\sqrt{\frac{L}{c}} \right) d_w^{\frac{j-i+1}{2}}, \;\; \forall 1 < i \leq j < L                 \\
			            \sigma_{\max}(W_{i:1}(t) Y)  & \leq & \frac{9}{7} d_w^{\frac{i}{2}} \sigma_{\max}(X),  \;\; \forall 1 \leq i < L\\
                        \sigma_{\max}(W_{L:i}(t) ) &\leq& \frac{9}{7} d_w^{\frac{L-i+1}{2}} , \;\; \forall 1 < i \leq  L\\
                        \sigma_{\min}(W_{L:i}(t) ) & \geq& \frac{5}{7} d_w^{\frac{L-i+1}{2}}  , \;\; \forall 1 < i \leq  L
                        \end{array} \right\} \label{eq:induction-b} \\
      \cC(t) & := \set*{
        \frobnorm{W_{i}(t) - \left(1 - \frac{\eta \lambda}{d_i}\right)^{t - \tau} W_{i}(\tau)}
        \leq \Delta_{\infty}
      },
      \quad
      \Delta_{\infty} := C \kappa^2 \sqrt{\dout \sr(X)} \log(\dhid) .
      \label{eq:induction-d-after-tao}
	\end{align}
\end{subequations}
where $C_1 > 0$ is a universal constant and $c>0$ is the constant from \cref{lemma:norm-product-bounded}.
\end{theorem}
The events are similar to those in the first phase. In this phase, the difference is that we cannot guarantee anymore that the smallest singular value of $\sigma_{\min}(W_{i:1}Y)$ gets arbitrarily small. 
Note that $\cA(\tau), \cB(\tau)$ are true by~\cref{thm:step1-induction} and $\cC(\tau)$ is trivially true.
Throughout this section, we will require $\dhid$ to satisfy the following inequality
\begin{equation}
  \dhid \gtrsim \Delta_{\infty}^2 L^3 = \mathcal{O}\left(L^3 \kappa^4 \dout \sr(X) \log^2(\dhid) \right).
  \label{eq:dhid-lb-step-2}
\end{equation}

\paragraph{Proof of $\cC(t)$.}
We start by proving $\cC(t)$ given $\set{\cA(j), \cB(j)}_{j < t}$.
\begin{lemma}
Given that the set  of events
$\set{\cA(j), \cB(j)}_{j = \tau}^{t-1}$ for $\tau \leq t \leq T$ hold, then $\cC(t)$ holds:
\[
        \frobnorm{W_{i}(t) - \left(1 - \frac{\eta \lambda}{d_i}\right)^{t - \tau} W_{i}(\tau)}
        \lesssim \kappa^2 \sqrt{\dout \sr(X)} \log(\dhid)
      .
\]
\end{lemma}
\begin{proof}
  From~\cref{lemma:difference-norm} and the trivial bound $(1 - \nicefrac{\eta \lambda}{d_i}) \leq 1$, we deduce that
  \begin{align*}
    & \frobnorm{W_{i}(t) - \left(1 - \frac{\eta \lambda}{d_i}\right)^{t - \tau}
    W_{i}(\tau)} \\
    &\leq
    \eta \sum_{j = 0}^{t - \tau - 1}
    \left(1 - \frac{\eta \lambda}{d_i}\right)^{t - \tau - j}
    \frac{1}{\sqrt{\dhid^{L - 1} \din}}
    \opnorm{W_{L:(i+1)}(j + \tau)}
    \frobnorm{\Phi(j + \tau)} \opnorm{W_{(i-1):1}(j + \tau) Y} \\
    &\leq
    \frac{\eta}{\sqrt{\dhid^{L-1} \din}}
    \sum_{j = 0}^{t - \tau - 1}
    \frac{9}{7} \dhid^{\frac{L - i}{2}}
    \cdot \frobnorm{\Phi(j + \tau)} \cdot
    \frac{9}{7} \dhid^{\frac{i - 1}{2}} \sigma_{\max}(X) \\
    &\lesssim
    \frac{ \eta \opnorm{X}}{\sqrt{\din}}
    \sum_{j = 0}^{t - \tau - 1}
    \frobnorm{\Phi(j + \tau)} \\
    &\lesssim
    \frac{ \eta \opnorm{X}}{ \sqrt{\din}}
    \cdot T\gamma\frobnorm{X}  \\
    &\lesssim \kappa^2 \sqrt{\dout \sr(X)} \cdot \log(\dhid),
  \end{align*}
  where the second inequality follows from $\set{\cB(j)}_{j = \tau}^{t-1}$, the fourth inequality
  follows from $\set{\cA(j)}_{j = \tau}^{t-1}$, and the last inequality follows from the
  upper bound on $\eta$ and the identity $\frobnorm{X} \opnorm{X} = \opnorm{X}^2 \sqrt{\sr(X)}$.
\end{proof}
We next prove that $\cB(t)$ is implied by $\cC(t)$.
\begin{lemma}
  Fix $t \in [\tau, T]$. Then $\cC(t) \implies \cB(t)$.
\end{lemma}
\begin{proof}
As in the proof of \cref{lem:step1-b(t)-proof}, we have the decomposition
\begin{align*}
  W_{j:i}(t) & =
  \prod_{\ell = i}^j \left(\left(1 - \frac{\eta \lambda}{d_{\ell}}\right)^{t - \tau} W_{\ell}^{(\tau)} + \left(W_{\ell}^{(t)} - \left(1 - \frac{\eta \lambda}{d_{\ell}}\right)^{t - \tau} W_{\ell}^{(\tau)} \right)\right) \\
  &=
  \begin{aligned}[t]
    & W_{j:i}(\tau) \prod_{\ell = j}^{i} \left(1 - \frac{\eta \lambda}{d_{\ell}}\right)^{t - \tau} \\ 
    & + \sum_{i \leq k_1, \dots, k_s \leq j} \left[ \prod_{\ell \notin \set{k_1, \dots, k_{s}}} \left(1 - \frac{\eta \lambda}{d_{\ell}}\right)^{t - \tau} \right] W_{j:(k_s + 1)}(\tau) \Delta_{k_s}
		\dots \Delta_{k_1} W_{(k_1 - 1):i}(\tau),
  \end{aligned}
\end{align*}
where each term satisfies $\opnorm{\Delta_{k_{i}}} \leq \Delta_{\infty}$. Therefore,
\begin{align}
  \opnorm[\Big]{W_{j:i}(t) - W_{j:i}(\tau) \prod_{\ell = j}^{i} \left(1 - \frac{\eta \lambda}{d_{\ell}}\right)^{t - \tau}} & \leq
  \sum_{s = 1}^{j-i+1} \binom{j-i+1}{s} \left(\Delta_{\infty}\right)^{s} \left(2 \sqrt{\frac{L}{c}}\right)^{s+1} d_w^{\frac{j-i+1-s}{2}}                                                  \notag \\
		                                     & \leq
		\left(2\sqrt{\frac{L}{c}}\right) d_w^{\frac{j-i+1}{2}}
    \sum_{s = 1}^{j-i} \left(\frac{C \Delta_{\infty} L^{3/2}}{\sqrt{\dhid}}\right)^{s} \notag \\
		                                     & \leq
       \left(2\sqrt{\frac{L}{c}}\right) d_w^{\frac{j-i+1}{2}}
		\frac{1}{31}, 
	\end{align}
  using~\cref{lemma:truncated-geometric-series} and the assumed bound~\eqref{eq:dhid-lb-step-2}.
  Therefore,
  \begin{align*}
    \opnorm{W_{j:i}(t)} &\leq \opnorm{W_{j:i}(\tau)} \prod_{\ell = j}^{i} \left(1 - \frac{\eta \lambda}{d_{\ell}}\right)^{t - \tau}
  + \opnorm[\Big]{W_{j:i}(t) - W_{j:i}(\tau) \prod_{i \leq \ell \leq j} \left(1 - \frac{\eta \lambda}{d_{\ell}}\right)^{t - \tau}} \\
    & \leq  \left(2\sqrt{\frac{L}{c}}\right) d_w^{\frac{j-i+1}{2}} + \left(2\sqrt{\frac{L}{c}}\right) d_w^{\frac{j-i+1}{2}}
		\frac{1}{31} \\
    &\leq  \left(2\sqrt{\frac{L}{c}}\right) d_w^{\frac{j-i+1}{2}},
  \end{align*}
  relabeling $c$ appropriately in the last step to absorb the $1 + \frac{1}{31}$ term.
  This proves the first bound in the event $\cB(t)$. Continuing with $W_{L:i}(t)$, we have
  \begin{align}
    \opnorm[\Big]{W_{L:i}(t) - W_{L:i}(\tau) \prod_{i \leq \ell \leq L} \left(1 - \frac{\eta \lambda}{d_{\ell}}\right)^{t - \tau}}
    & \leq
    \sum_{s = 1}^{L-i+1} \binom{i}{\ell}\left(\Delta_{\infty}\right)^{s}
    \left(2 \sqrt{\frac{L}{c}}\right)^{s} \dhid^{\frac{L-i+1-s}{2}} \frac{5}{4}
            \notag \\
                                          & \leq
      \frac{5}{4}  d_w^{\frac{L-i+1}{2}} 
      \sum_{s = 1}^{L-i+1} \left(\frac{C \Delta_{\infty} L^{3/2}}{\sqrt{\dhid}}\right)^{s} \notag \\
                                          & \leq
      \frac{1}{63}  \frac{5}{4}  d_w^{\frac{L-i+1}{2}}.
  \end{align}
  Again using the bound from~\eqref{eq:dhid-lb-step-2}, we deduce that
  \begin{align*}
    \opnorm{W_{L:i}(t)} &\leq \opnorm{W_{L:i}(\tau)} + \opnorm[\Big]{W_{L:i}(t) - W_{L:i}(\tau) \prod_{i \leq \ell \leq L}\left(1 - \frac{\eta \lambda}{d_{\ell}}\right)^{t - \tau}} \\
                      &  \leq  \frac{5}{4} d_w^{\frac{L-i+1}{2}} + \frac{1}{63} \frac{5}{4}  d_w^{\frac{L-i+1}{2}} \\ 
                      & = \frac{80}{63}  d_w^{\frac{L-i+1}{2}} \\
                      & \leq \frac{9}{7}  d_w^{\frac{L-i+1}{2}},
  \end{align*}
  which proves the second bound from the event $\cB(t)$, as well as
\begin{align*}
  \sigma_{\min}(W_{L:i}(t)) &\geq  \sigma_{\min}(W_{L:i}(\tau)) \prod_{i \leq \ell \leq L} \left(1 - \frac{\eta \lambda}{d_{\ell}}\right)^{t - \tau} - 
  \opnorm[\Big]{W_{L:i}(t) - W_{L:i}(\tau) \prod_{i \leq \ell \leq L}\left(1 - \frac{\eta \lambda}{d_{\ell}}\right)^{t - \tau}} \\
                            & \geq  \frac{3}{4} d_w^{\frac{L-i+1}{2}} \cdot \prod_{i \leq \ell \leq L} \left(1 - \frac{\eta \lambda}{d_{\ell}}\right)^{t - \tau}
                            - \frac{1}{63} \frac{3}{4} d_w^{\frac{L-i+1}{2}} \\
                            &\geq \frac{3}{4} \dhid^{\frac{L - i + 1}{2}} \left[
                              \exp\Big(-2 \cdot \frac{(t - \tau)\eta \lambda}{d_{\ell}}\Big)^{L - i + 1} - \frac{1}{63}
                            \right] \\
                            &\geq \frac{3}{4} \dhid^{\frac{L - i + 1}{2}}
                            \left[
                              \exp\left(-2 \cdot \frac{L \cdot \log(\dhid) \cdot \din}{d_{\ell}}\right) - \frac{1}{63}
                            \right] \\
                            &\geq
                            \frac{3}{4} \dhid^{\frac{L - i + 1}{2}} \cdot \frac{60}{63} \\
                            &= \frac{5}{7} \dhid^{\frac{L - i + 1}{2}},
\end{align*}
using $1 - x \geq \exp(-2x)$ in the third inequality, $t - \tau \leq \nicefrac{ \log(\dhid) \cdot \din}{\eta \lambda}$
in the penultimate inequality, the fact that $d_{\ell} = \dhid$ for $\ell > 1$,
and choosing $d_{w} \geq \nicefrac{4L \log(\dhid) \cdot \din}{\log(\frac{63}{61})}$ in the last
inequality. This proves the third bound from $\cB(t)$.

Finally, we have the upper bound
\begin{align*}
   \opnorm{W_{i:1}(t)Y} &\leq \opnorm{W_{i:1}(\tau)Y} \prod_{1 \leq \ell \leq i} \left(1-\frac{\eta \lambda}{d_{\ell}}\right)^{t-\tau} +
  \opnorm[\Big]{W_{i:1}(t)Y - W_{i:1}(\tau)Y \prod_{1 \leq \ell \leq i} \left(1 - \frac{\eta \lambda}{d_{\ell}}\right)^{t - \tau}} \\
                        & \leq  \frac{5}{4} \dhid^{\frac{i}{2}} \opnorm{X}   +
  \frac{1}{63} \frac{5}{4}  \dhid^{\frac{i}{2}}  \sigma_{\max}(X) \\
                        &= \frac{80}{63}  \dhid^{\frac{i}{2}} \opnorm{X} \\
                        &\leq \frac{9}{7}  d_w^{\frac{i}{2}} \opnorm{X}.
\end{align*}
This proves the last bound from the event $\cB(t)$.
\end{proof}
\paragraph{Proof of $\cA(t)$.} We show in the following that the events $\cB(t), \cA(t)$ imply $\cA(t+1)$. Let us start by stating the Lemma.
\begin{lemma}
  For any $\tau \leq t \leq T$, we have that $\set{\set{\cA(j)}_{\tau \leq j \leq t-1}, \set{\cB(j)}_{\tau \leq j \leq t}} \implies \cA(t)$.
\end{lemma}
\begin{proof}
From~\cref{lemma:cprod-i-lb}, it follows that $\cprod{i} \in [\frac{1}{4}, 1]$. From this and $\mathcal{B}(t)$, it follows that
\begin{align*}
    \lambda_{\min}(P(t)) &\geq \frac{1}{4 d_w^{L-1}m} \sum_{i=1}^L  \sigma^2_{\min}(W_{L:(i+1)}(t)) \sigma^2_{\min}(W_{i-1:1}(t)Y)  \\
    &\geq \frac{1}{4 d_w^{L-1}m}  \sigma^2_{\min}(W_{L:2}(t)) \sigma^2_{\min}(Y)\\
    &\geq \frac{1}{4 d_w^{L-1}m}  d_w^{L-1} \left(\frac{5}{7}\right)^2 (1-\delta)^2 \sigma_{\min}^2(X)\\
    &\geq  \frac{1}{4 d_w^{L-1}m}  \frac{1}{2} d_w^{L-1} \frac{8}{10} \sigma_{\min}^2(X)\\
    &\geq \frac{ \sigma_{\min}^2(X) }{10 \din},
\end{align*}
given $\delta = \frac{1}{10}$.
Similarly, for the upper bound on $\lambda_{\max}(P(t))$, we get
\begin{align*}
    \lambda_{\max}(P(t)) &\leq \frac{1}{d_w^{L-1}m} \sum_{i=1}^L \sigma^2_{\max}(W_{L:(i+1)}(t)) \sigma^2_{\max}(W_{i-1:1}(t)Y)  \\
    &\leq \frac{1}{d_w^{L-1}m} \sum_{i=1}^L \left(\frac{9}{7}\right)^2 d_w^{L-i} \left(\frac{9}{7}\right)^2 d_w^{i-1} \sigma_{\max}^2(X)\\ 
    &\leq \frac{2 L  \sigma_{\max}^2(X) }{m}.
\end{align*}
Similarly to the first $\tau$ iterations, we obtain a bound on the higher-order terms:
\begin{align*}
    & \frobnorm{E(t)Y} \\
    &= \frobnorm{d_w^{-\frac{L-1}{2}}m^{-\frac{1}{2}}E_0(t)Y} \\
    & \leq d_w^{-\frac{L-1}{2}}m^{-\frac{1}{2}} \sum_{\ell=2}^L \eta^{\ell} \binom{L}{\ell} \left(1-\frac{\eta \lambda}{d_w}\right)^{L-\ell} \left(2 \sqrt{\frac{L}{c}} \right)^{\ell -1} d_w^{\frac{L-\ell}{2}} \left(\frac{3}{ \sqrt{m}}\frobnorm{\Phi(t)}\opnorm{X} \right)^{\ell} \opnorm{Y}\\
    &\leq
                 \frac{C \eta L^{\frac{3}{2}} \opnorm{X}^2 \frobnorm{\Phi(t)}}{\din}
                 \sum_{\ell = 1}^{L - 1} \left(
                   \frac{C \eta L^{\frac{3}{2}} \opnorm{X} \frobnorm{\Phi(t)}}{(\din d_w)^{1/2}}
                 \right)^{\ell} \\
    &\lesssim
  \frac{\eta^2 L^3 \opnorm{X}^3 \frobnorm{\Phi(t)}^2}{\din^{3/2} \dhid^{1/2}} \\
  &\lesssim
  \frac{\eta^2 L^{3} \lambda \opnorm{X}^4 \frobnorm{\Phi(t)}}{\din^2 \sigma_{\min}^2(X)} \sqrt{\frac{\dout}{\dhid}}
  \\
  &\leq
  \frac{\eta L^{2} \opnorm{X}^2  \frobnorm{\Phi(t)}}{\din} \sqrt{\frac{\din}{\dhid}} \\
  &\leq
  \frac{3 \eta \sigma_{\min}^2(X)}{80 \din} \cdot \frobnorm{\Phi(t)},
\end{align*}
where the second inequality follows by imitating the
argument in~\cref{lemma:Bt-implies-bound-on-E}, the third inequality follows from
\cref{lemma:truncated-geometric-series} and the inequality
 \begin{align*}
   \frac{C \eta L^{3/2} \opnorm{X} \frobnorm{\Phi(t)}}{\sqrt{\din \dhid}} \overset{(\eta \leq \nicefrac{\din}{L \sigma_{\max}^2(X)})}&{\leq}
     \frac{C L^{1/2} \sqrt{\din} \frobnorm{\Phi(t)}}{\sqrt{\dhid} \sigma_{\max}(X)} \\ 
                                                                          \overset{(\cA(t))}&{\leq}
    \frac{C \lambda L^{1/2} \sqrt{\sr(X)}}{\sigma_{\min}^2(X)} \sqrt{\frac{\dout}{\dhid}} \\
   \overset{(\lambda \lesssim L \sigma_{\min}^2(X))}&{\lesssim}
     C L^{3/2} \sqrt{\frac{\dout \sr(X)}{\dhid}} \\ 
   \overset{(\dhid \gtrsim L^3 \dout \sr(X))}&{\leq} \frac{1}{2},
 \end{align*}
the second to last inequality uses that $\eta \leq \frac{m}{L \sigma^2_{\max}(X)}$ and that $\lambda \leq \sigma^2_{\min}(X) \sqrt{\frac{m}{d}}$
and the last inequality
follows from $\dhid \gtrsim \din L^4 \kappa^4$.
Therefore, we arrive at the following bound on the regression error:
\begin{align*}
  \frobnorm{\mathrm{vec}(\Phi(t+1))} &= \opnorm{(I-\eta P(t))} \frobnorm{\Phi(t)} + \frobnorm{E(t)} + \abs{\call - 1} \frobnorm{U(t)}\\
                                 &\leq \left( 1- \frac{ \eta \sigma^2_{\min}(X)}{10\din} +\frac{3 \eta \sigma_{\min}^2(X)}{80\din}  \right) \frobnorm{\Phi(t)} + \abs{\call - 1} \frobnorm{U(t)}\\
    & \leq \left( 1- \frac{\eta \sigma^2_{\min}(X)}{16\din} \right)  \frobnorm{\Phi(t)} + \left[\frac{(L-1)\eta \lambda}{d_w} +\frac{\eta \lambda}{\din} \right] \left(\frac{5}{4}  \sqrt{\frac{d}{m}}  \right) \frobnorm{X} \\
   & \leq \left( 1- \frac{\eta \sigma^2_{\min}(X)}{16m} \right)  \frobnorm{\Phi(t)} +
   \frac{5 \sqrt{d} \eta \lambda}{2\din \sqrt{\din}} \frobnorm{X}.
\end{align*}
We can split the remaining analysis into two cases:
\begin{enumerate}
  \item If $\frac{40  \lambda \frobnorm{X}}{\sigma_{\min}^2(X)} \sqrt{\frac{\dout}{\din}} \leq \frobnorm{\Phi(t)} \leq \frac{80  \lambda \frobnorm{X}}{\sigma_{\min}^2(X)} \sqrt{\frac{\dout}{\din}}$, then
  \begin{align*}
    \frobnorm{\Phi(t+1)} &\leq \left( 1-  \frac{ \eta \sigma^2_{\min}(X)}{16\din} \right) \frobnorm{\Phi(t)}+ \frac{5  \sqrt{\dout} \eta \lambda}{2\din \sqrt{\din}} \frobnorm{X}\\
                       &\leq \frobnorm{\Phi(t)} -\left( \frac{ \eta \sigma^2_{\min}(X)}{16\din} \right)  \frac{40  \sqrt{\dout}\lambda \frobnorm{X}}{ \sigma_{\min}^2(X) \sqrt{\din}}
                       + \frac{5  \sqrt{\dout} \eta \lambda}{2\din \sqrt{\din}} \opnorm{X}\\
                       &\leq \frobnorm{\Phi(t)} -\left( \frac{40 \eta \lambda \frobnorm{X} \sqrt{\dout}}{ 16\din \sqrt{\din}} \right)  + \frac{5   \sqrt{d} \eta \lambda}{2 \din \sqrt{\din}} \frobnorm{X}\\
      &= \frobnorm{\Phi(t)}
      - \frac{\eta \lambda \frobnorm{X} \sqrt{\dout}}{\din \sqrt{\din}}
      \left[\frac{5}{2}- \frac{5}{2}\right] \\
      & \leq \frobnorm{\Phi(t)}.
  \end{align*}
  \item On the other hand, if
  $\frobnorm{\Phi(t)} \leq \frac{40 \lambda \frobnorm{X}}{\sigma_{\min}^2(X)} \sqrt{\frac{\dout}{\din}}$, then
  \begin{align*}
     \frobnorm{\Phi(t+1)}
      &\leq \left( 1- \frac{\eta \sigma^2_{\min}(X)}{16m} \right) \frobnorm{\Phi(t)} + \frac{5 \sqrt{d} \eta \lambda}{2\din \sqrt{\din}} \frobnorm{X}\\
      &\leq \left( 1- \frac{\eta \sigma^2_{\min}(X)}{16m} \right)
      \frac{40 \lambda \frobnorm{X}}{\sigma_{\min}^2(X)} \sqrt{\frac{\dout}{\din}}
      + \frac{5  \sqrt{d} \eta \lambda}{2\din \sqrt{\din}} \frobnorm{X}\\
      & \leq \frac{40 \lambda \frobnorm{X}}{ \sigma_{\min}^2(X)}
      \sqrt{\frac{\dout}{\din}} + \frac{5 \sqrt{\dout} \eta \lambda}{2\din \sqrt{\din}} \frobnorm{X}\\
      & \leq \frac{41\lambda \frobnorm{X}}{\sigma_{\min}^2(X)}
\sqrt{\frac{\dout}{\din}} \\
&\leq \frac{80\lambda \frobnorm{X}}{\sigma_{\min}^2(X)}
\sqrt{\frac{\dout}{\din}} ,
  \end{align*}
  where the penultimate inequality follows from the requirement $\eta \leq \frac{2m}{5\sigma_{\min}^2(X)}$.
  In particular, since we assumed $\cA(t)$ holds, which means $\frobnorm{\Phi(t)} \leq \frac{80 \lambda \frobnorm{X}}{\sigma^2_{\min}(X)} \sqrt{\frac{\dout}{\din}}$ then this also holds for $\frobnorm{\Phi(t+1)}$. 
\end{enumerate}
This shows that the event $\cA(t+1)$ holds.
\end{proof}
\begin{proof}[Proof of \cref{thm:step2-induction}]
    Taking the above Lemmas together, we have shown that the base case for all three events $\cA, \cB, \cC$ holds. Further we have shown by induction that $\set{\cA(j), \cB(j)}_{\tau \leq j<t} \implies \cC(t+1)$, $\cC(t) \implies \cB(t)$ and $\cA(t), \cB(t) \implies \cA(t+1)$. From this, the theorem follows.
\end{proof}

\subsection{Step 3: Convergence off the subspace}
\label{sec:subsec:Step 3: Convergence off the subspace}
In this section, we show that the off-subspace error depends on the hidden width. 
The on-subspace components of the weights act on the image of the subspace; the off-subspace components are in the orthogonal complement of the on-subspace components.
More formally, the projection onto the subspace is defined as
 $P^{\perp}_{\range(Y)} := YY^{\ddag}$. 
To determine the behavior off the subspace, we must consider the projection onto ${\range(Y)}^{\perp}$, which we denote $P_{\range(Y)}^{\perp}.$
Note that
\begin{align*}
	W_{1}(t+1) P_{\range(Y)}^{\perp} & =
	W_{1}(t)\left(1 - \frac{\eta \lambda}{\din}\right) P_{\range(Y)}^{\perp} -
	\eta \cdot \frac{1}{\sqrt{d_w^{L-1}m}} W_{L:2}^{\T} \Phi(t) Y^{\T} P_{\range(Y)}^{\perp}                    \\
	                        & =
	\left(1 - \frac{\eta \lambda}{\din}\right) W_{1}(t)P_{\range(Y)}^{\perp}                                    \\
	                        & = \left(1 - \frac{\eta \lambda}{\din}\right)^{t+1} W_{1}(0) P_{\range(Y)}^{\perp}
\end{align*}
using $Y^{\T} P_{\range(Y)}^{\perp} = Y^{\T} P_{\range(Y)^{\perp}} = Y^{\T} P_{\ker(Y^{\T})} = 0$.
By event $\cB(t)$ from~\cref{eq:induction-b}, we have
\begin{align}
	\opnorm{W_{L:1}(t) P_{\range(Y)}^{\perp}} & \leq
    \opnorm{W_{L:2}(t)} \opnorm{W_{1}(t) P_{\range(Y)}^{\perp}}
 \notag \\
    &\leq
    \frac{9}{7} \dhid^{\frac{L - 1}{2}}
    \cdot
	\left(1 - \frac{\eta \lambda}{\din}\right)^t \opnorm{W_{1}{(0)} P_{\range(Y)}^{\perp}}
	\label{eq:W-perp-bound-refine-i}
\end{align}
Normalizing on both sides we obtain
\begin{align}
	\opnorm[\bigg]{\frac{1}{\sqrt{\dhid^{L-1} \din}}W_{L:1}(t) P_{\range(Y)}^{\perp}} & \leq  2 \left(1 - \frac{\eta \lambda}{\din}\right)^t \frac{1}{\sqrt{\din}} \opnorm{W_{1}(0) P_{\range(Y)}^{\perp}}.
\end{align}
We now turn to bounding $\opnorm{W_{1}(0) P_{\range(Y)}^{\perp}}$. 
Let $V_{\perp} \in O(m, m-s)$ be a matrix whose columns span ${\range(Y)}^{\perp}$;
by orthogonal invariance of the operator norm and
the Gaussian distribution, we have
\begin{align*}
    \opnorm{W_{1}(0) P_{\range(Y)}^{\perp}} &=
    \opnorm{W_{1}(0) V_{\perp} V_{\perp}^{\T}} \\
    &=
    \opnorm{W_{1}(0) V_{\perp}},
\end{align*}
where $W_{1}(0) V_{\perp} \in \Rbb^{\dhid \times (\din - s)}$ is a matrix with standard Gaussian elements; indeed,
\[
    W_{1}{(0)} V_{\perp} = \begin{bmatrix}
        W_{1}(0) (V_{\perp})_{:, 1} &
        \dots &
        W_{1}(0) (V_{\perp})_{:, \din - s}
    \end{bmatrix} \overset{(d)}{=}
    \begin{bmatrix}
        \bar{g}_1 & \dots \bar{g}_{\din - s}
    \end{bmatrix},
    \quad \text{where} \;\;
    \bar{g}_i \iid \cN(0, I_{\dhid}).
\]
Therefore by~\citep[Corollary 7.3.3]{Ver18}, the following holds
with probability $1-2\exp(-c \dhid^2)$:
\begin{align*}
	\opnorm{W_{1}(0) P_{\range(Y)}^{\perp}} & \leq
    2\sqrt{\dhid} + \sqrt{\din - s}
    \lesssim \sqrt{\dhid}.
\end{align*}
By the preceding displays,
\begin{align}
	\opnorm[\bigg]{\frac{1}{\sqrt{\dhid^{L-1} \din}} W_{L:1}(T) P_{\range(Y)}^{\perp}}
	 & \lesssim \left(1 - \frac{\eta \lambda}{\din}\right)^{T}
	\cdot \sqrt{\frac{\dhid}{\din}}         \notag \\
	 & =  \left(1 - \frac{\eta \lambda}{\din}\right)^{T}
	\exp\left(\frac{1}{2} \log(\dhid / \din)\right) \notag \\
	 & \leq
	 \exp\left(-\frac{T \eta \lambda}{\din}
    + \frac{1}{2} \log(\dhid)
	\right)                                                       \notag \\
    &=  \dhid^{-\frac{3}{2}},
	\label{eq:W-perp-bound-refined}
\end{align} 
where the second inequality follows from the identity $1 - x \leq \exp(-x)$,
the penultimate inequality follows from the choice of $T = \frac{2\log(\dhid)\sqrt{dm}}{\eta\gamma \sigma_{\min}^2(X)}$ from \cref{eq:T def appendix} and the choice of $\lambda = \gamma \sigma_{\min}^2(X) \sqrt{\frac{m}{d}}$.


\subsection{Robustness at test time}
\label{sec:subsec:Robustness at test time}
Suppose we have trained our model for $T$ steps and $W_{L:1}(T) = W_{L}(T) \cdots W_{1}(T)$ are the weights of the model at the end of training.
In what follows, we suppress the iteration index $T$ for simplicity. By~\cref{theorem:main-informal},
\begin{subequations}
\begin{align}
    \frobnorm{W_{L:1}Y-X} &\leq
    C_1 \gamma \frobnorm{X};
    \label{eq:robust-on} \\
    \opnorm{W_{L:1} P_{\range(Y)}^{\perp}} &\leq
    \dhid^{-C_2}
    \label{eq:robust-off}
\end{align}
\end{subequations}
for universal constants $C_1, C_2 > 0$. Suppose that we receive a new test pair $(x, y)$ satisfying
\begin{align}
    y=Ax+\epsilon, \;\; x \in \range(R), \;\;
    \epsilon \sim \cN(0, \sigma^2 I_{m}).
    \label{eq:new-test-point}
\end{align}
The next corollary characterizes the estimation error $\norm{W_{L:1}y - x}$.
\begin{corollary}
    Let $(W_1, \dots, W_{L})$ be the weight matrices of a deep linear 
    network trained for $T$ iterations in the setting of~\cref{theorem:main-informal}.
    Consider a new data point $(x, y)$ satisfying~\eqref{eq:new-test-point}.
    Then the output of the network, $W_{L:1}y$, satisfies
    \begin{equation}
        \norm{W_{L:1}y - x} \lesssim
        \gamma \kappa \sqrt{\sr(X)}  + \frac{1}{\dhid^{C_2}}
        + \sigma \sqrt{s}
    \end{equation}
    with probability of at least $1 - c_1 \exp(-c_2 \dout) - \exp(-c_3 s)$.
    Conversely, let $(W^{\lambda=0}_1(t),...,W^{\lambda=0}_L(t))$ be the weight matrices of a deep linear network trained in the setting of~\cref{theorem:main-informal} with $\lambda = 0$. Then for any $\beta >0$, there exists an iteration $T$ such that the reconstruction error $\frobnorm{W^{\lambda=0}_{L:1}(t)Y-X} \leq \beta \frobnorm{X}$ for all $t>T$. Moreover, with probability at least $1 - c_1 \exp(-c_2 \dout) - \exp(-c_3 s)$, the test error satisfies
    \[
     \norm{W^{\lambda=0}_{L:1}(t)y-x} \gtrsim \sigma \left(
    \sqrt{\frac{\dout(\din - s)}{\din}} - \sqrt{s}
    \right) -
    \beta \kappa \sqrt{\sr(X)} \norm{y}.
    \]
\end{corollary}
\begin{proof}
    We start by bounding the error between the ``oracle'' solution mapping $XY^{\dag}$ and the trained neural network. We have
    \begin{align*}
        \opnorm{ W_{L:1}-XY^{\dag}} &=  \opnorm{ W_{L:1}YY^{\dag}-XY^{\dag}+  W_{L:1}(I-YY^{\dag})}\\
        & \leq \opnorm{ W_{L:1}Y-X} \opnorm{Y^{\dag}}+ \opnorm{ W_{L:1}(I-YY^{\dag})}\\
        & \leq \frobnorm{ W_{L:1}Y-X} \opnorm{Y^{\dag}}+ \opnorm{ W_{L:1}P_{\range(Y)}^{\perp}} \\
        & \leq
        C_1 \gamma \frobnorm{X} \opnorm{Y^{\dag}} + \dhid^{-C_2} \\
        & = \frac{C_1 \gamma \sqrt{\sr(X)} \opnorm{X}}{\sigma_{\min}(Y)}
        + \dhid^{-C_2} \\
        & \leq \frac{C_1 \gamma \sqrt{\sr(X)} \sigma_{\max}(X)}{(1 - \delta) \sigma_{\min}(X)}
        + \dhid^{-C_2} \\
        & \lesssim \gamma \kappa \sqrt{\sr(X)} + \dhid^{-C_2},
    \end{align*}
    where the third inequality follows from~\cref{eq:robust-on,eq:robust-off},
    the second equality follows from the definition of $\sr(X)$ and the identity
    $\opnorm{Y^{\dag}} = \frac{1}{\sigma_{\min}(Y)}$, the penultimate inequality
    follows from~\cref{assumption:rip} and the last inequality follows by
    substituting $\delta = \frac{1}{10}$. Consequently, we have
    \begin{align}
        \norm{W_{L:1}y - x} &=
        \norm{(W_{L:1} - XY^{\dag})y + XY^{\dag}y - x} \notag \\
        &\leq
        \opnorm{W_{L:1} - XY^{\dag}} \norm{y} +
        \norm{XY^{\dag}y - x} \notag \\
        &\lesssim
        \left(\gamma \kappa \sqrt{\sr(X)} + \dhid^{-C_2}\right) \norm{y}
        + \norm{XY^{\dag} y - x}.
        \label{eq:robust-proof-decomposition}
    \end{align}
    We now argue that the second term in~\eqref{eq:robust-proof-decomposition}
    is bounded by $\sigma \sqrt{s}$.
    Recall that $Y = AX$, $X = RZ$ for some $Z \in \mathbb{R}^{s \times n}$ with full row rank, and $x = Rz$ for some $z \in \Rbb^{s}$.
    Therefore, we have
    \begin{align*}
        XY^{\dag} y &=
        RZ (ARZ)^{\dag} y \\
        &=
        R ZZ^{\dag} (AR)^{\dag} y \\
        &=
        R (AR)^{\dag} (AR) z + R(AR)^{\dag} \epsilon \\
        &= Rz + R(AR)^{\dag} \epsilon \\
        &= x + R(AR)^{\dag} \epsilon.
    \end{align*}
    The second equality in the preceding display follows from the fact
    that $(M_1 M_2)^{\dag} = M_2^{\dag} M_1^{\dag}$ when $M_1$ and $M_2$ are full column-rank and full row-rank respectively; indeed, here $M_1 \equiv AR$ is
    full column-rank by~\cref{assumption:rip} and $M_2 \equiv Z$ is full
    row-rank by assumption. Similarly, the third and fourth inequalities
    follow from the full row rankness and full column rankness of $Z$ and $AR$, respectively. Consequently, we have the bound
    \begin{align*}
        \norm{XY^{\dag} y - x} &=
        \norm{R(AR)^{\dag} \epsilon} =
        \norm{(AR)^{\dag} \epsilon},
    \end{align*}
    since $R$ is a matrix with orthogonal columns. We now write
    \[
        AR = \bar{U} \bar{\Sigma} \bar{V}^{\T}, \quad \text{where} \;\;
        \bar{U} \in O(m, s), \;
        \bar{V} \in O(s), \;
        1 - \delta \leq \Sigma_{ii} \leq 1 + \delta,
    \]
    for the economic SVD of $AR$, where the inequalities on the singular
    values follow from~\cref{assumption:rip}. In particular,
    \begin{align*}
        \norm{(AR)^{\dag} \epsilon} &=
        \norm{\bar{V} \bar{\Sigma}^{-1} \bar{U}^{\T} \epsilon}
        \leq
        \frac{1}{\sigma_{\min}(\bar{\Sigma})}
        \norm{\bar{U}^{\T} \epsilon}
        \lesssim
        \norm{\bar{U}^{\T} \epsilon},
    \end{align*}
    using $\delta = \frac{1}{10}$ in the last inequality.
    Finally, by standard properties of the multivariate normal distribution,
    \[
        \bar{U}^{\T} \epsilon \sim \cN(0, \sigma^2 I_{s}) \implies
        \norm{\bar{U}^{\T} \epsilon} \lesssim
        \sigma \sqrt{s},
    \]
    with probability at least $1 - \exp(-c s^2)$~\citep[Theorem 3.1.1]{Ver18},
    for a universal constant $c > 0$.
    Returning to~\eqref{eq:robust-proof-decomposition}, we conclude that
    \[
        \norm{W_{L:1}y - x} \lesssim 
        \left( \gamma \kappa \sqrt{\sr(X)} + \dhid^{-C_2} \right) \norm{y}
        + \sigma \sqrt{s},
    \]
    with probability at least $1 - c_1 \exp(-c_2 \dout) - \exp(-c_3 s)$. This proves the first of the two claims.

    We now prove the lower bound for the reconstruction error for the weights $W_{i}^{\lambda = 0}(t)$. For simplicity, we write $\bar{W}_{L:1} := W_{L}^{\lambda = 0}(t) \dots W_{1}^{\lambda = 0}(t)$ and suppress the dependence on $t$. We obtain
    \begin{align}
        \norm{\bar W_{L:1}y-XY^{\dag}y} &=
        \norm{\bar W_{L:1}(I-YY^{\dag})y+(\bar W_{L:1}Y-X)Y^{\dag} y} \notag \\
        & \geq \norm{\bar W_{L:1}P_{\range(Y)}^{\perp}y}- \frac{\frobnorm{(\bar W_{L:1}Y-X)}} {\sigma_{\min}(Y)} \norm{y} \notag \\
        & \geq \norm{\bar W_{L:1}P_{\range(Y)}^{\perp}\epsilon}- \frac{\beta \frobnorm{X}} {\sigma_{\min}(Y)} \norm{y} \notag \\
        & \gtrsim  \sqrt{\frac{\dout}{\din}} \norm{P_{\range(Y)}^{\perp} \epsilon}- \frac{\beta \sqrt{\sr(X)} \sigma_{\max}(X)} {(1  - \delta) \sigma_{\min}(X)} \norm{y} \notag \\
        & \gtrsim \sigma \sqrt{\frac{\dout (\din - s)}{\din}}  - \beta \kappa \sqrt{\sr(X)} \norm{y},
        \label{eq:test-error-lb-1}
    \end{align}
    where the first inequality follows from the reverse triangle inequality and the identity $\norm{Y^{\dag}} = \nicefrac{1}{\sigma_{\min}(Y)}$,
    the second inequality follows by the assumption that $t > T$,
    the third inequality follows from~\cref{assumption:rip}, the definition of $\sr(X)$ and~\cref{lemma:wide-gaussian-prod-tail} combined with property $\cC(t)$ from~\cref{sec:subsec:Step 1: Rapid early convergence}, and the last inequality follows from the fact that
    \begin{equation}
        \label{eq:projected-noise}
        \norm{P_{\range(Y)}^{\perp} \epsilon}
        \gtrsim \sigma \sqrt{\din - s}, \quad
        \text{with probability at least $1 - \exp(-c(\din - s)^2)$.}
    \end{equation}
    To see~\eqref{eq:projected-noise}, let $V_{\perp} \in O(\din, \din - s)$ be a matrix whose columns span $\range(Y)^{\perp}$
    such that $P_{\range(Y)}^{\perp} = V_{\perp} V_{\perp}^{\T}$.
    By orthogonal invariance of the Gaussian distribution,
    \[
        V_{\perp}^{\T} \epsilon \overset{(d)}{=}
        \cN(0, \sigma^2 I_{\din - s}).
    \]
    Moreover, by orthogonal invariance of the Euclidean norm,
    \[
        \norm{P_{\range(Y)}^{\perp} \epsilon} =
        \norm{V_{\perp}^{\T} \epsilon}.
    \]
    Combining the two preceding displays with~\citep[Theorem 3.1.1]{Ver18} yields the inequality~\eqref{eq:projected-noise}.

Altogether, we get the following lower bound
\begin{align*}
    \norm{\bar W_{L:1}(y)-x}&= \norm{\bar W_{L:1}y-XY^{\dag}y+XY^{\dag}y-x} \\
    &\geq \norm{(\bar W_{L:1}-XY^{\dag}) y} - \norm{XY^{\dag}y-x}\\
    & \gtrsim
    \sigma \sqrt{\frac{\dout(\din - s)}{\din}}
    - \beta \kappa \sqrt{s} \norm{y} -
    \norm{XY^{\dag} Ax - x} - \norm{XY^{\dag} \epsilon} \\
    & \gtrsim \sigma \sqrt{\frac{\dout (\din - s)}{\din}} - \beta \kappa \sqrt{\sr(X)} \norm{y} - \sigma \sqrt{s} \\
    &= \sigma \left(
    \sqrt{\frac{\dout(\din - s)}{\din}} - \sqrt{s}
    \right) -
    \beta \kappa \sqrt{\sr(X)} \norm{y},
\end{align*}
where the first inequality follows from the reverse triangle inequality, the second inequality follows from
the bound~\eqref{eq:test-error-lb-1} and the last inequality follows from the fact that $XY^{\dag} Ax = x$ and the upper bound $\norm{XY^{\dag} \epsilon} \lesssim \sigma \sqrt{s}$, which follows from standard properties of the multivariate Gaussian distribution. This lower bound holds
with probability at least $1 - c_1 \exp(-c_2 \dout) - \exp(-c_3 s)$.
This concludes the proof of the lower bound.
\end{proof}


\section{Auxiliary results}
In this section, we state and prove results used to prove the main result \cref{thm:mainresult-formal} or mentioned in the introduction. We start with a result showing that a global minimizer solution of the regularized optimization problem is zero on the orthogonal complement of the image.
\begin{lemma}\label{lem:robustsolutionofoptproblem}
    Suppose $f_{W_{L:1}}$ is a global minimizer of the regularized optimization problem~\eqref{eq:l2regprob1}.
    Then $f_{W_{L:1}}$ satisfies $W_1P_{\range(Y)}^{\perp}=0$, where $P_{\range(Y)}^{\perp}$ is the projection onto the orthogonal complement of $\range(Y)$.
\end{lemma}
\begin{proof}
Suppose that $f_{W_{L:1}}$ is a minimizer with $W_1P_{\range(Y)}^{\perp} \neq 0$. Then consider $f_{W_{L:1}P_{\range(Y)}}$, the neural network that coincides with $f_{W_{L:1}}$ except that its first-layer weights are right-multiplied by $P_{\range(Y)}$. We have
\begin{align*}
\frobnorm{f_{W_{L:1}P_{\range(Y)}}(Y)-X} 
= \frobnorm{f_{W_{L:1}}(P_{\range(Y)}Y)-X} 
= \frobnorm{f_{W_{L:1}}(Y)-X}. 
\end{align*}
Hence the first term in the objective in~\eqref{eq:l2regprob1} is the same for $f_{W_{L:1}}$ and $f_{W_{L:1}P_{\range(Y)}}$.
By the Pythagorean theorem, we have that
\begin{align*}
    \frobnorm{W_1}^2 &=
    \frobnorm{W_1 P_{\range(Y)} }^2 +
    \frobnorm{W_1 P_{\range(Y)}^{\perp}}^2
    > \frobnorm{W_1 P_{\range(Y)} }^2
\end{align*}
since $W_1P_{\range(Y)}^{\perp} \neq 0$ by
assumption. Thus the regularization term in the objective~\eqref{eq:l2regprob1} is strictly larger for $f_{W_{L:1}}$ than for $f_{W_{L:1}P_{\range(Y)}}$. Therefore $f_{W_{L:1}}$ cannot be the minimal-norm solution.
\end{proof}

\begin{lemma}
    \label{lemma:wide-gaussian-prod-tail}
    Let $A_1, A_2, \dots, A_{q}$ have i.i.d. Gaussian elements with $A_{i} \in \Rbb^{n_{i} \times n_{i-1}}$, $n_{0} = n$,
	and $n_{i} \gtrsim q$. Then
    \begin{align}
    	\expec{\norm{A_{q} \dots A_1 y}^2} & = \norm{y}^2 \cdot \prod_{i=1}^q n_{i},
			\label{eq:expec-prod}  \\
        \prob{\abs[\big]{\norm{A_{q} \cdots A_1 y}^2 - \norm{y}^2 \prod_{i=1}^q n_i}
        \geq 0.1 \norm{y}^2 \prod_{i=1}^q n_i}
        &\leq c_1 \exp\left(-\frac{c_2}{\sum_{ i = 1 }^q n_i^{-1}} \right),
        \label{eq:tail-bound-prod}
    \end{align}
    where $c_1, c_2 > 0$ are universal constants and $y$ is any fixed vector.
\end{lemma}
\begin{proof}
    We start with~\eqref{eq:expec-prod}. Note that for any $A_{i}$, we have
	\begin{align*}
		\norm{A_{i} y}^2 & = \sum_{j = 1}^{n_i} \ip{(A_{i})_{j, :}, y}^2                                      \\
		                 & \overset{(d)}{=} \sum_{j = 1}^{n_i} \norm{y}^2 g_{i}^2 \qquad (g_i \sim \cN(0, 1)) \\
		                 & \overset{(d)}{=} \norm{y}^2 Z_{i},
	\end{align*}
	where $Z_i \sim \chi^2_{n_i}$, a $\chi^2$-random variable with $n_i$ degrees of freedom. As a result,
	\[
		\expec{\norm{A_i y}^2} = \norm{y}^2 \expec{Z_{i}} = \norm{y}^2 \cdot n_{i}.
	\]
	Moreover, since $A_{1}, \dots, A_{q}$ are independent, we have
	\begin{align*}
		\expec{\norm{A_{q} \dots A_1 y}^2} & =
		\expec{\expec{\norm{A_{q} (A_{q-1} \dots A_1 y)}^2 \mid A_{1}, \dots, A_{q-1}}} \\
		                                   & =
		n_{q} \expec{\norm{A_{q-1} \dots A_{1} y}^2}                                    \\
		                                   & =
		n_{q} \expec{\expec{\norm{A_{q-1} \dots A_1 y}^2 \mid A_{1}, \dots, A_{q-2}}}   \\
		                                   & =
		n_{q} \cdot n_{q-1} \expec{\norm{A_{q-2} \dots A_1 y}^2}                        \\
		                                   & = \dots                                    \\
		                                   & = \prod_{j = 1}^q n_{i} \cdot \norm{y}^2,
	\end{align*}
	by iterating the above construction; this proves~\cref{eq:expec-prod}.

	\vspace{11pt}

	\noindent We now prove~\cref{eq:tail-bound-prod}. Let $\norm{y} = 1$ for simplicity;
	then $\norm{A_q \dots A_1 y}^2 \sim Z_{q} Z_{q-1} \dots Z_{1}$, where $Z_{i} \sim \chi^2_{n_i}$.
	The moments of a random variable $X \sim \chi^2_{k}$ satisfy
	\begin{align*}
		\mathbb{E}[X^{\lambda}] & = \frac{2^{\lambda} \Gamma(\frac{k}{2} + \lambda)}{\Gamma(\frac{k}{2})} \\
		                        & =
		\frac{2^{\lambda} \sqrt{\frac{4 \pi}{k + 2 \lambda}} \left(\frac{k + 2\lambda}{2e}\right)^{\frac{k}{2} + \lambda}}{
		\sqrt{\frac{4 \pi}{k}} \left(\frac{k}{2e}\right)^{\frac{k}{2}}
		} \left(1 + O(1 / k)\right),
	\end{align*}
	for all $\lambda > -k/2$, with the second equality furnished by a Stirling approximation.
	Following [Eq. (20)] in \cite{du2019width}, we obtain the following upper bound:
	\begin{equation}
		\mathbb{E}[X^{\lambda}] \leq
		\exp\left(
		\frac{2 \lambda^2}{k} - \frac{1}{2} \log\left(1 + \frac{2\lambda}{k}\right)
		+ \lambda \log k
		\right) \cdot \left(1 + O\left(\frac{1}{k}\right)\right),
		\quad \forall \lambda \geq -\frac{k}{4}.
		\label{eq:stirling-ub-chi2}
	\end{equation}
	To bound the upper tail in~\cref{eq:tail-bound-prod}, we argue that for any $\lambda > 0$,
	\begin{align}
		 & \prob{Z_q \dots Z_1 \geq \exp(c) \prod_{i=1}^q n_i}                                                           \notag \\
		 & \leq
		\exp\left(-\lambda c\right) \left(\prod_{i=1}^q n_i\right)^{-\lambda} \cdot
		\mathbb{E}[(Z_q \dots Z_1)^{\lambda}]                                                                            \notag \\
		 & =
		\exp\left(-\lambda c - \lambda \log\left( \prod_{i=1}^q n_i \right)\right) \mathbb{E}[(Z_q \dots Z_1)^{\lambda}] \notag \\
		 & \leq
		\exp\left(-\lambda c - \lambda \log\left( \prod_{i=1}^q n_i \right)
		+ \sum_{i=1}^q \frac{2 \lambda^2}{n_i} + \lambda \log(n_i) - \frac{1}{2} \log\left(1 + \frac{2\lambda}{n_i}\right)
		\right) \prod_{j=1}^q \left(1 + O\left(\frac{1}{n_i}\right)\right).
		\label{eq:Z-chernoff-ub}
	\end{align}
	Under our assumption that $n_i \gtrsim q$, the last term above satisfies
	\begin{align}
		\prod_{j=1}^q \left(1 + O\left(\frac{1}{n_i}\right)\right) & \lesssim
		\prod_{j=1}^q \left(1 + \frac{1}{q}\right) \notag                       \\
		                                                           & =
		\left(1 + \frac{1}{q}\right)^q \notag                                   \\
		                                                           & \leq
		\lim_{n \to \infty} \left(1 + \frac{1}{n}\right)^{n} \notag             \\
		                                                           & = \exp(1),
		\label{eq:exp-defn-ub}
	\end{align}
	using the formal definition of the exponential. 
    For the upper tail, the exponent in~\eqref{eq:Z-chernoff-ub} can be simplified as follows:
    \begin{align*}
        -\lambda c + \sum_{i = 1}^q \frac{2 \lambda^2}{n_i} - \frac{1}{2} \log\left(1 + \frac{2 \lambda}{n_i}\right)
        \leq 
        -\lambda c + 2 \lambda^2 \sum_{i = 1}^q \frac{1}{n_i},
    \end{align*}
    using $\log(1 + u) \geq 0$ for any $u \geq 0$.
    Maximizing over $\lambda \geq 0$ yields
    \[
        \lambda_{\star} = \frac{c}{4 \cdot \sum_{i = 1}^q \frac{1}{n_i}}.
    \]
    Plugging the value of $\lambda_{\star}$ into the upper bound for the exponent leads to
    \begin{align*}
        & -\lambda_{\star} c + 2 \lambda_{\star}^2
        \sum_{i = 1}^q \frac{1}{n_i} \\
        &=
        -\frac{c^2}{4} \frac{1}{\sum_{i = 1}^q n_i^{-1}} +
        \frac{c^2}{8} \cdot
        \frac{\sum_{i = 1}^q \frac{1}{n_i}}{
        \left(\sum_{i = 1}^q \frac{1}{n_i}\right)^2
        } \\
        &=
        -\frac{c^2}{8} \cdot \frac{1}{\sum_{i = 1}^q n_i^{-1}}.
    \end{align*}
    Setting $c = \log(1.1)$ completes the proof.
    
    We now derive the lower bound in~\cref{eq:tail-bound-prod}.
    Given $\lambda < 0$, we have
	\begin{align*}
		 &\prob{Z_{q} \dots Z_1 \leq \exp(-c) \prod_{i=1}^q n_i} \\
		 & =\prob{(Z_{q} \dots Z_1)^{\lambda} \geq \exp(-\lambda c) \Big(\prod_{i=1}^q n_i\Big)^{\lambda}} \\
		 & \leq
		\exp\left(\lambda c - \lambda \log\Big(\prod_{i=1}^q n_i\Big)\right) \mathbb{E}[(Z_q \dots Z_1)^{\lambda}]
		\\
		 & \leq
		C_1 \exp\left(\lambda c - \lambda \sum_{i=1}^q \log(n_i)
		+ \sum_{i=1}^q \frac{2\lambda^2}{n_i} - \frac{1}{2} \log\left(1 + \frac{2\lambda}{n_i}\right)
		+ \lambda \log(n_i)
		\right)                                                                                            \\
		 & =
		C_1 \exp\left(
		\lambda c + 2 \lambda^2 \sum_{i=1}^q \frac{1}{n_i} - \frac{1}{2} \log\left(1 + \frac{2 \lambda}{n_i}\right)
		\right)
	\end{align*}
    In particular, the exponent in the preceding display satisfies
    \begin{align*}
        \lambda c + \sum_{i = 1}^q \frac{2 \lambda^2}{n_i} - \frac{1}{2} \log\Big(1 + \frac{2 \lambda}{n_i}\Big)
        & \leq
        \lambda c + 2 \lambda^2 \sum_{i = 1}^q 
        \frac{1}{n_i} - 2 \lambda \sum_{i=1}^q \frac{1}{n_i},
    \end{align*}
    using the inequaliy $\log(1 + 2x) \geq 4x$ valid for any $x > -\frac{1}{4}$. Setting $\lambda = -\frac{c}{4 \sum_{i=1}^q n_i^{-1}}$ yields
    \[
        -\frac{c^2}{4 \sum_{i=1}^q n_i^{-1}} +
        \frac{c^2}{8} \frac{\sum_{i=1}^q n_i^{-1}}{(\sum_{i=1}^q n_i^{-1})^2}
        +\frac{c}{2}
        =
        -\frac{c^2}{4 \sum_{i = 1}^q n_i^{-1}} + \frac{c}{2}.
    \]
    Setting $c = -\log(0.9)$ completes the proof.
\end{proof}

\begin{theorem}[Weierstrass]
	\label{thm:weierstrass}
	The following inequality holds:
	\begin{equation}
		1 - \sum_{i=1}^n w_{i} x_i \leq \prod_{i = 1}^n \left(1 - x_i\right)^{w_i}, \quad
		\text{for all $x \in [0, 1]$ and $w_i \geq 1$}.
		\label{eq:weierstrass}
	\end{equation}
\end{theorem}
\begin{proof}
    We prove the inequality by induction on the number of
    terms. For the base case $n = 1$, consider the function
    \[
        h(w) = (1 - x_1)^w - (1 - w x_1), \quad
        \text{with} \;\;
        h'(w) = (1 - x_1)^{w} \log(1 - x_1) + x_1
    \]
    Clearly $h(1) = 0$, so it suffices to show $h$ is increasing on $[1, \infty)$. Starting from the inequality
    \(
        \log(1 - x_1) \geq \frac{x_1}{x_1 - 1},
    \)
    we have
    \begin{align*}
        h'(w) &\geq
        \frac{x_1 (1 - x_1)^{w}}{x_1 - 1} + x_1 \\
        &=
        \frac{x_1 (1 - x_1)^{w} + x_1 (x_1 - 1)}{x_1 - 1} \\
        &=
        \frac{x_1 \left[(1 - x_1) - (1 - x_1)^w\right]}{1 - x_1} \\
        &\geq 0, \quad \text{for all $w \geq 1$,}
    \end{align*}
    since $(1 - x_1) \in (0, 1)$. This proves the claim
    for $n = 1$.

    Now suppose the claim holds up to some $n \in \mathbb{N}$. We have
    \begin{align*}
        \prod_{j = 1}^{n+1} (1 - x_j)^{w_j} &=
        (1 - x_{n+1})^{w_{n+1}} \prod_{j = 1}^n (1 - x_j)^{w_j} \\
        &\geq
        (1 - w_{n+1} x_{n+1}) \prod_{j = 1}^{n}
        (1 - x_j)^{w_j} \\
        &\geq
        (1 - w_{n+1} x_{n+1}) \left(1 - \sum_{j = 1}^n w_j x_j \right) \\
        &=
        1 - \sum_{j = 1}^{n+1} w_j x_j +
        w_{n+1} x_{n+1} \cdot \sum_{j = 1}^{n} w_j x_j \\
        &\geq
        1 - \sum_{j = 1}^{n+1} w_j x_j,
    \end{align*}
    where the first inequality follows from the base
    case, the second inequality follows by the inductive
    hypothesis and the last inequality follows from
    nonnegativity of $\set{w_j}_{j \geq 1}$ and $\set{x_j}_{j \geq 1}$. This completes the proof.
\end{proof}

\begin{lemma}
	\label{lemma:one-minus-folded-product}
	Under the assumptions of \cref{thm:mainresult-formal}, we have that
	\begin{equation}
		|1 - \call|
		\leq
		\frac{(L - 1) \eta \lambda}{d_w} +
		\frac{\eta \lambda}{m} \leq \frac{2 \eta \lambda}{m}.
		\label{eq:one-minus-folded-product}
	\end{equation}
\end{lemma}
\begin{proof}
	Since $\call < 1$, we have
	$|\call - 1| =
		1 - \prod_{i = 1}^{L}
		\left(1 - \frac{\eta \lambda}{d_i}\right)$.
	Now, let $x_i := \frac{\eta \lambda}{d_i}$ and $w_i := 1$ for $i = 1, \dots, L$. From~\cref{thm:weierstrass},
	it follows that
	\begin{align*}
		1 - \prod_{i = 1}^L
		\left(1 - \frac{\eta \lambda}{d_i}\right) & \leq
		\sum_{i = 1}^L \frac{\eta \lambda}{d_i} =
		\frac{(L - 1) \eta \lambda}{d_{w}} +
		\frac{\eta \lambda}{m}
		\leq \frac{2 \eta \lambda}{m},
	\end{align*}
    under the assumption that $\dhid \ge m(L-1)$.
\end{proof}
\begin{lemma}
    \label{lemma:truncated-geometric-series}
    For any $\alpha \leq \frac{1}{2}$ and $j, k \in \mathbb{N}$, it holds that
    \begin{equation}
        \sum_{i = j}^k \alpha^{i} \leq
        2 \alpha^j (1 - \alpha^{k - j + 1}).
    \end{equation}
\end{lemma}
\begin{proof}
    The claim follows from the geometric series formula:
    \begin{align*}
       \sum_{i = j}^k \alpha^i &=
       \alpha^j \sum_{i = 0}^{k- j } \alpha^i
       =
       \alpha^j \cdot \frac{1 - \alpha^{k - j + 1}}{1 - \alpha}
       \leq
       2 \alpha^j (1 - \alpha^{k - j + 1}),
    \end{align*}
    where the last inequality follows from  $\nicefrac{1}{1 - \alpha} \leq 2$.
\end{proof}


\section{Information on numerics for the union of subspaces model}
\label{sec:appendix numerical description}
\paragraph{Data generation for the union of subspaces experiments.}
The union-of-subspaces model stipulates that each vector in the input
data belongs to one of $k$ subspaces. Formally, there exists a
collection $\mathcal{R} := \set{R_1, \dots, R_k}$, where $R_{i} \in O(\dout, s)$,
such that $x^{i} \in \bigcup_{j = 1}^k \range(R_{j})$ for all $i$.
In our experiments, we generate training samples from the union-of-subspaces model in the following manner:
\begin{itemize}
    \item Sample $Z \in \mathbb{R}^{\dout \times n}$ according to
    the procedure described in~\cref{sec:numerics}.
    \item For each $i = 1, \dots, n$:
    \begin{enumerate}
        \item Sample $R \sim \mathrm{Unif}(\mathcal{R})$.
        \item Set $X_{:, i} = R Z_{:, i}$.
    \end{enumerate}
\end{itemize}

\paragraph{Neural network architecture for the union-of-subspaces model.}
The inverse mapping for linear inverse problems with data from a union-of-subspaces model is in general nonlinear for $k > 1$ --- as
a result, deep linear networks are not a suitable choice for learning
the inverse mapping.
Nevertheless, it is known that the inverse mapping is approximated to arbitrary accuracy
by a piecewise-linear mapping~(see~\cite{GOW20}), which can be realized as a multi-index model of the form $g(V^{\T} x)$ for
suitable $V$ and vector-valued mapping $g$. Guided by this, we use a
neural network architecture defined as follows:
\begin{equation}
    f_{W_1, \dots, W_L}(x) =
    W_L \left( W_{L-1} W_{L-2} \cdots W_{1} x\right)_+,
    \label{eq:mixed-linear-relu-network}
\end{equation}
where $W_1, \dots, W_{L}$ are learnable weight matrices and
$[\cdot]_+$ denotes the (elementwise) positive part, equivalent to using a ReLU activation at the $(L-1)^{\text{th}}$ hidden layer.
Indeed, recent results~\cite{parkinson2023linear} suggest that neural
networks of the form~\eqref{eq:mixed-linear-relu-network} are biased
towards multi-index models such as the one sought to approximate the
inverse mapping. Finally, all the networks from~\cref{fig:gauss_noise_wd} were trained for $100000$ iterations
with learning rate $\eta = 10^{-3}$.




\end{document}
