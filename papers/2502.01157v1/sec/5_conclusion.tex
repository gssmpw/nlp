\section{Conclusions}
% 
We have introduced Radiant Foam, a novel representation that allows real-time differentiable ray tracing.
The core of our method is a foam structure of polyhedral cells, which allows efficient volumetric mesh ray tracing algorithms to be applied \textit{without} relying on dedicated hardware such as NVIDIA RT cores.
We allow these cells to be continuously optimized by parameterizing them as a Voronoi diagram, which we show to be differentiable under volume rendering.
By doing so, we have shown that one can achieve similar modeling quality as 3D Gaussian Splatting, but without sacrificing the benefits of a true ray tracing-based volume renderer, nor the fast rendering speed of rasterization-based renderers.

\paragraph{Limitations and future work}
While the Voronoi-based representation we have proposed is very effective at constructing foam models through continuous optimization, the space of possible foam models which could be used in our rendering pipeline is much larger than what is parameterized by Voronoi.
Most notably, our current model always requires that cell boundaries be equidistant between neighbouring points, which leads to many small, empty cells being needed to define a surface.
Future work could potentially relax this requirement by generalizing beyond Voronoi diagrams.
\quad
Other open research questions include how to compose multiple foam models together efficiently and accounting for varying illumination, how to model dynamic content instead of static scenes, how to enable editing of scenes, and how to integrate generative modeling with our representation.
Progress in these directions could make foam models relevant in real-time ray tracing applications currently dominated by triangle meshes, as we have already found that foam-based ray tracing can exceed the performance of dedicated ray tracing hardware.

\paragraph{Acknowledgements}
%
This work was supported in part by the Natural Sciences and Engineering Research Council of Canada (NSERC) Discovery Grant [2023-05617], NSERC Collaborative Research and Development Grant, the SFU Visual Computing Research Chair, Google Research, Digital Research Alliance of Canada, and Advanced Research Computing at the University of British Columbia.
\quad
We would like to thank George Kopanas, Lily Goli, Alex Evans, Thomas Muller, Bernhard Kerbl, Vincent Sitzmann, Forrester Cole, Or Litany, and David Fleet for their feedback and/or early research discussions.