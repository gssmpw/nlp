\begin{abstract}
Research on differentiable scene representations is consistently moving towards more efficient, real-time models.
Recently, this has led to the popularization of splatting methods, which eschew the traditional ray-based rendering of radiance fields in favor of rasterization.
This has yielded a significant improvement in rendering speeds due to the efficiency of rasterization algorithms and hardware, but has come at a cost: the approximations that make rasterization efficient also make implementation of light transport phenomena like reflection and refraction much more difficult.
We propose a novel scene representation which avoids these approximations, but keeps the efficiency and reconstruction quality of splatting by leveraging a decades-old efficient volumetric mesh ray tracing algorithm which has been largely overlooked in recent computer vision research.
The resulting model, which we name \textit{Radiant Foam}, achieves rendering speed and quality comparable to Gaussian Splatting, without the constraints of rasterization.
Unlike ray traced Gaussian models that use \textit{hardware} ray tracing acceleration, our method requires no special hardware or APIs beyond the standard features of a programmable GPU.
\end{abstract}