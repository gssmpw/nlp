\begin{figure*}
\centering
\includegraphics[width=.32\linewidth]{fig/delaunay_voronoi/f3_1.png}
\hfill
\includegraphics[width=.32\linewidth]{fig/delaunay_voronoi/f3_2.png}
\hfill
\includegraphics[width=.32\linewidth]{fig/delaunay_voronoi/f3_3.png}
\caption{
{\bf Delaunay and its dual Voronoi -- }
% 
(left) Given a set of points {\color{figred}(red)} in $\real^N$, we can find circumspheres {\color{figblue}(blue)} which each pass through $N{+}1$ points.
(center) The set of all circumspheres which contain no points on their interior defines the Delaunay triangulation, where the $N{+}1$ points tangent to each circumsphere form a simplex.
(right) These circumspheres also describe the Delaunay triangulation's dual, the Voronoi diagram: the centers of circumspheres tangent to a point become the vertices of the Voronoi cell containing that point.
}
\label{fig:delaunay_voronoi}
\end{figure*}