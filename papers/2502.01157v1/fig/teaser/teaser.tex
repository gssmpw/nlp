\twocolumn[{%
\renewcommand\twocolumn[1][]{#1}%
\maketitle
% 
\dropshadow{\includegraphics[height=.28\linewidth]{fig/teaser/teaser_1.png}}
\hfill
\dropshadow{\includegraphics[height=.28\linewidth]{fig/teaser/teaser_2.png}}
% 
\captionof{figure}{
\textbf{Teaser --}
We introduce the Radiant Foam differentiable 3D representation, which can be used to learn accurate radiance fields for any novel view synthesis applications (left).
If we slice our foam along the plane highlighted by the red ``laser'', we expose (right) the internal structure of our representation: a polyhedral mesh that provides an injective parameterization of the 3D domain.
% 
Our representation is a \textit{foam}, as the polyhedral cell structure is analogous to a closed-cell foam which partitions space into regions physically separated by thin, flat walls.
It is \textit{radiant}, as each foam bubble emits a view-dependent radiance that can be used to model the plenoptic function.
% 
% 
% 
\vspace{1em}
}
\label{fig:teaser}
}]