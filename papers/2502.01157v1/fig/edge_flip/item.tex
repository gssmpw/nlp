\begin{figure*}
\centering
\includegraphics[width=.31\linewidth]{fig/edge_flip/f4_1.png}
\hfill
\includegraphics[width=.31\linewidth]{fig/edge_flip/f4_2.png}
\hfill
\includegraphics[width=.31\linewidth]{fig/edge_flip/f4_3.png}
% 
\caption{
{\bf Edge flips -- }
The connectivity of the Delaunay graph {\color{figgreen}(green)} is sensitive to small positional perturbations, leading to ``edge flips'' in the triangulation.
These discrete changes occur at configurations where the circumspheres {\color{figblue}(blue)} of two neighbouring simplices become identical.
In the dual Voronoi diagram, this configuration also corresponds to a discrete change, but the cell boundary which changes has an area of zero at the moment of the flip (center).
Consequently, while there are still discrete changes in the adjacency structure of the Voronoi diagram, the \textit{shapes of the cells} vary continuously with the positions of the points {\color{figred}(red)}, which enables gradient-based optimization.}
\label{fig:edge_flip}
\end{figure*}