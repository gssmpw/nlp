%%%%%%%% ICML 2025 EXAMPLE LATEX SUBMISSION FILE %%%%%%%%%%%%%%%%%

\documentclass{article}

% Recommended, but optional, packages for figures and better typesetting:
\usepackage{microtype}
\usepackage{graphicx}
\usepackage{subfigure}
\usepackage{booktabs} % for professional tables
\usepackage{multirow}
\usepackage{colortbl}
\usepackage{array}
%\usepackage{arrowsc}
% Define grey color for highlighting
\definecolor{lightgrey}{rgb}{0.9,0.9,0.9}


% hyperref makes hyperlinks in the resulting PDF.
% If your build breaks (sometimes temporarily if a hyperlink spans a page)
% please comment out the following usepackage line and replace
% \usepackage{icml2025} with \usepackage[nohyperref]{icml2025} above.

% Attempt to make hyperref and algorithmic work together better:
\newcommand{\theHalgorithm}{\arabic{algorithm}}

% Use the following line for the initial blind version submitted for review:

%\usepackage{icml2025}

% If accepted, instead use the following line for the camera-ready submission:
\usepackage[accepted]{icml2025}

% For theorems and such
\usepackage{inconsolata}
\usepackage{hyperref}       % hyperlinks
\usepackage{url}            % simple URL typesetting
\usepackage{booktabs}       % professional-quality tables
\usepackage{amsfonts}       % blackboard math symbols
\usepackage{nicefrac}       % compact symbols for 1/2, etc.
\usepackage{microtype}      % microtypography
\usepackage{xcolor}         % colors
\usepackage{graphicx}

\usepackage[font=small]{caption}
\usepackage{subcaption}
\usepackage{wrapfig}
\usepackage{multirow}
\usepackage{amsmath}
\usepackage{amsfonts}
\usepackage{bm}
\usepackage{amssymb}
\usepackage{amsthm}
\usepackage{verbatim}
\usepackage{longtable}
\usepackage{arydshln}
\usepackage{algorithm}
% \usepackage{algpseudocode}
\usepackage{mathtools, nccmath}

\usepackage[skins]{tcolorbox}%
\tcbuselibrary{breakable}%
\tcbuselibrary{hooks}%%
\DeclarePairedDelimiter{\nint}\lfloor\rceil
\DeclareMathOperator*{\argmax}{arg\,max}
\DeclareMathOperator*{\argmin}{arg\,min}
% if you use cleveref..
\usepackage[capitalize,noabbrev]{cleveref}
\newcommand{\ours}{\texttt{ParetoQ}}
\theoremstyle{plain}
\newtheorem{theorem}{Theorem}[section]
\newtheorem{proposition}[theorem]{Proposition}
\newtheorem{lemma}[theorem]{Lemma}
\newtheorem{corollary}[theorem]{Corollary}
\theoremstyle{definition}
\newtheorem{definition}[theorem]{Definition}
\newtheorem{assumption}[theorem]{Assumption}
\theoremstyle{remark}
\newtheorem{remark}[theorem]{Remark}

\usepackage[textsize=tiny]{todonotes}

\icmltitlerunning{ParetoQ}
\begin{document}

\twocolumn[
\icmltitle{ParetoQ: Scaling Laws in Extremely Low-bit LLM Quantization}
\vspace{-1.5em}
\begin{center}
\textbf{Zechun Liu} \hspace{0.5em} \textbf{Changsheng Zhao} \hspace{0.5em} \textbf{Hanxian Huang} \hspace{0.5em} \textbf{Sijia Chen} \hspace{0.5em} \textbf{Jing Zhang} \hspace{0.5em} \textbf{Jiawei Zhao} \hspace{0.5em} \textbf{Scott Roy} \\
\textbf{Lisa Jin} \hspace{0.5em} \textbf{Yunyang Xiong} \hspace{0.5em} \textbf{Yangyang Shi} \textbf{Lin Xiao} \hspace{0.5em} \textbf{Yuandong Tian} \hspace{0.5em} \textbf{Bilge Soran} \hspace{0.5em} \\ \textbf{Raghuraman Krishnamoorthi} \hspace{0.5em} \textbf{Tijmen Blankevoort} \hspace{0.5em} \textbf{Vikas Chandra}  \\
\vspace{0.5em}
\large{Meta}
\end{center}
\vspace{3em}
]


% \icmlsetsymbol{equal}{*}

% \begin{icmlauthorlist}

% \icmlauthor{Zechun Liu}{meta}
% \icmlauthor{Changsheng Zhao}{meta}
% \icmlauthor{Hanxian Huang}{meta}
% \icmlauthor{Sijia Chen}{meta}
% \icmlauthor{Jing Zhang}{meta}
% \icmlauthor{Jiawei Zhao}{meta}
% \icmlauthor{Scott Roy}{meta}
% \icmlauthor{Lisa Jin}{meta}
% \icmlauthor{Yunyang Xiong}{meta}
% \icmlauthor{Yangyang Shi}{meta}
% \icmlauthor{Lin Xiao}{meta}
% \icmlauthor{Yuandong Tian}{meta}
% \icmlauthor{Bilge Soran}{meta}
% \icmlauthor{Raghuraman Krishnamoorthi}{meta}
% \icmlauthor{Tijmen Blankevoort}{meta}
% \icmlauthor{Vikas Chandra}{meta}

% %\icmlauthor{}{sch}
% \end{icmlauthorlist}

% \icmlaffiliation{meta}{Meta}
% \icmlcorrespondingauthor{Zechun Liu}{zechunliu@meta.com}
% \vskip 0.3in
% ]


% \printAffiliationsAndNotice{}

\begin{abstract}
  In this work, we present a novel technique for GPU-accelerated Boolean satisfiability (SAT) sampling. Unlike conventional sampling algorithms that directly operate on conjunctive normal form (CNF), our method transforms the logical constraints of SAT problems by factoring their CNF representations into simplified multi-level, multi-output Boolean functions. It then leverages gradient-based optimization to guide the search for a diverse set of valid solutions. Our method operates directly on the circuit structure of refactored SAT instances, reinterpreting the SAT problem as a supervised multi-output regression task. This differentiable technique enables independent bit-wise operations on each tensor element, allowing parallel execution of learning processes. As a result, we achieve GPU-accelerated sampling with significant runtime improvements ranging from $33.6\times$ to $523.6\times$ over state-of-the-art heuristic samplers. We demonstrate the superior performance of our sampling method through an extensive evaluation on $60$ instances from a public domain benchmark suite utilized in previous studies. 


  
  % Generating a wide range of diverse solutions to logical constraints is crucial in software and hardware testing, verification, and synthesis. These solutions can serve as inputs to test specific functionalities of a software program or as random stimuli in hardware modules. In software verification, techniques like fuzz testing and symbolic execution use this approach to identify bugs and vulnerabilities. In hardware verification, stimulus generation is particularly vital, forming the basis of constrained-random verification. While generating multiple solutions improves coverage and increases the chances of finding bugs, high-throughput sampling remains challenging, especially with complex constraints and refined coverage criteria. In this work, we present a novel technique that enables GPU-accelerated sampling, resulting in high-throughput generation of satisfying solutions to Boolean satisfiability (SAT) problems. Unlike conventional sampling algorithms that directly operate on conjunctive normal form (CNF), our method refines the logical constraints of SAT problems by transforming their CNF into simplified multi-level Boolean expressions. It then leverages gradient-based optimization to guide the search for a diverse set of valid solutions.
  % Our method specifically takes advantage of the circuit structure of refined SAT instances by using GD to learn valid solutions, reinterpreting the SAT problem as a supervised multi-output regression task. This differentiable technique enables independent bit-wise operations on each tensor element, allowing parallel execution of learning processes. As a result, we achieve GPU-accelerated sampling with significant runtime improvements ranging from $10\times$ to $1000\times$ over state-of-the-art heuristic samplers. Specifically, we demonstrate the superior performance of our sampling method through an extensive evaluation on $60$ instances from a public domain benchmark suite utilized in previous studies.

\end{abstract}

\begin{IEEEkeywords}
Boolean Satisfiability, Gradient Descent, Multi-level Circuits, Verification, and Testing.
\end{IEEEkeywords}
\section{Introduction}
\label{section:introduction}

% redirection is unique and important in VR
Virtual Reality (VR) systems enable users to embody virtual avatars by mirroring their physical movements and aligning their perspective with virtual avatars' in real time. 
As the head-mounted displays (HMDs) block direct visual access to the physical world, users primarily rely on visual feedback from the virtual environment and integrate it with proprioceptive cues to control the avatar’s movements and interact within the VR space.
Since human perception is heavily influenced by visual input~\cite{gibson1933adaptation}, 
VR systems have the unique capability to control users' perception of the virtual environment and avatars by manipulating the visual information presented to them.
Leveraging this, various redirection techniques have been proposed to enable novel VR interactions, 
such as redirecting users' walking paths~\cite{razzaque2005redirected, suma2012impossible, steinicke2009estimation},
modifying reaching movements~\cite{gonzalez2022model, azmandian2016haptic, cheng2017sparse, feick2021visuo},
and conveying haptic information through visual feedback to create pseudo-haptic effects~\cite{samad2019pseudo, dominjon2005influence, lecuyer2009simulating}.
Such redirection techniques enable these interactions by manipulating the alignment between users' physical movements and their virtual avatar's actions.

% % what is hand/arm redirection, motivation of study arm-offset
% \change{\yj{i don't understand the purpose of this paragraph}
% These illusion-based techniques provide users with unique experiences in virtual environments that differ from the physical world yet maintain an immersive experience. 
% A key example is hand redirection, which shifts the virtual hand’s position away from the real hand as the user moves to enhance ergonomics during interaction~\cite{feuchtner2018ownershift, wentzel2020improving} and improve interaction performance~\cite{montano2017erg, poupyrev1996go}. 
% To increase the realism of virtual movements and strengthen the user’s sense of embodiment, hand redirection techniques often incorporate a complete virtual arm or full body alongside the redirected virtual hand, using inverse kinematics~\cite{hartfill2021analysis, ponton2024stretch} or adjustments to the virtual arm's movement as well~\cite{li2022modeling, feick2024impact}.
% }

% noticeability, motivation of predicting a probability, not a classification
However, these redirection techniques are most effective when the manipulation remains undetected~\cite{gonzalez2017model, li2022modeling}. 
If the redirection becomes too large, the user may not mitigate the conflict between the visual sensory input (redirected virtual movement) and their proprioception (actual physical movement), potentially leading to a loss of embodiment with the virtual avatar and making it difficult for the user to accurately control virtual movements to complete interaction tasks~\cite{li2022modeling, wentzel2020improving, feuchtner2018ownershift}. 
While proprioception is not absolute, users only have a general sense of their physical movements and the likelihood that they notice the redirection is probabilistic. 
This probability of detecting the redirection is referred to as \textbf{noticeability}~\cite{li2022modeling, zenner2024beyond, zenner2023detectability} and is typically estimated based on the frequency with which users detect the manipulation across multiple trials.

% version B
% Prior research has explored factors influencing the noticeability of redirected motion, including the redirection's magnitude~\cite{wentzel2020improving, poupyrev1996go}, direction~\cite{li2022modeling, feuchtner2018ownershift}, and the visual characteristics of the virtual avatar~\cite{ogawa2020effect, feick2024impact}.
% While these factors focus on the avatars, the surrounding virtual environment can also influence the users' behavior and in turn affect the noticeability of redirection.
% One such prominent external influence is through the visual channel - the users' visual attention is constantly distracted by complex visual effects and events in practical VR scenarios.
% Although some prior studies have explored how to leverage user blindness caused by visual distractions to redirect users' virtual hand~\cite{zenner2023detectability}, there remains a gap in understanding how to quantify the noticeability of redirection under visual distractions.

% visual stimuli and gaze behavior
Prior research has explored factors influencing the noticeability of redirected motion, including the redirection's magnitude~\cite{wentzel2020improving, poupyrev1996go}, direction~\cite{li2022modeling, feuchtner2018ownershift}, and the visual characteristics of the virtual avatar~\cite{ogawa2020effect, feick2024impact}.
While these factors focus on the avatars, the surrounding virtual environment can also influence the users' behavior and in turn affect the noticeability of redirection.
This, however, remains underexplored.
One such prominent external influence is through the visual channel - the users' visual attention is constantly distracted by complex visual effects and events in practical VR scenarios.
We thus want to investigate how \textbf{visual stimuli in the virtual environment} affect the noticeability of redirection.
With this, we hope to complement existing works that focus on avatars by incorporating environmental visual influences to enable more accurate control over the noticeability of redirected motions in practical VR scenarios.
% However, in realistic VR applications, the virtual environment often contains complex visual effects beyond the virtual avatar itself. 
% We argue that these visual effects can \textbf{distract users’ visual attention and thus affect the noticeability of redirection offsets}, while current research has yet taken into account.
% For instance, in a VR boxing scenario, a user’s visual attention is likely focused on their opponent rather than on their virtual body, leading to a lower noticeability of redirection offsets on their virtual movements. 
% Conversely, when reaching for an object in the center of their field of view, the user’s attention is more concentrated on the virtual hand’s movement and position to ensure successful interaction, resulting in a higher noticeability of offsets.

Since each visual event is a complex choreography of many underlying factors (type of visual effect, location, duration, etc.), it is extremely difficult to quantify or parameterize visual stimuli.
Furthermore, individuals respond differently to even the same visual events.
Prior neuroscience studies revealed that factors like age, gender, and personality can influence how quickly someone reacts to visual events~\cite{gillon2024responses, gale1997human}. 
Therefore, aiming to model visual stimuli in a way that is generalizable and applicable to different stimuli and users, we propose to use users' \textbf{gaze behavior} as an indicator of how they respond to visual stimuli.
In this paper, we used various gaze behaviors, including gaze location, saccades~\cite{krejtz2018eye}, fixations~\cite{perkhofer2019using}, and the Index of Pupil Activity (IPA)~\cite{duchowski2018index}.
These behaviors indicate both where users are looking and their cognitive activity, as looking at something does not necessarily mean they are attending to it.
Our goal is to investigate how these gaze behaviors stimulated by various visual stimuli relate to the noticeability of redirection.
With this, we contribute a model that allows designers and content creators to adjust the redirection in real-time responding to dynamic visual events in VR.

To achieve this, we conducted user studies to collect users' noticeability of redirection under various visual stimuli.
To simulate realistic VR scenarios, we adopted a dual-task design in which the participants performed redirected movements while monitoring the visual stimuli.
Specifically, participants' primary task was to report if they noticed an offset between the avatar's movement and their own, while their secondary task was to monitor and report the visual stimuli.
As realistic virtual environments often contain complex visual effects, we started with simple and controlled visual stimulus to manage the influencing factors.

% first user study, confirmation study
% collect data under no visual stimuli, different basic visual stimuli
We first conducted a confirmation study (N=16) to test whether applying visual stimuli (opacity-based) actually affects their noticeability of redirection. 
The results showed that participants were significantly less likely to detect the redirection when visual stimuli was presented $(F_{(1,15)}=5.90,~p=0.03)$.
Furthermore, by analyzing the collected gaze data, results revealed a correlation between the proposed gaze behaviors and the noticeability results $(r=-0.43)$, confirming that the gaze behaviors could be leveraged to compute the noticeability.

% data collection study
We then conducted a data collection study to obtain more accurate noticeability results through repeated measurements to better model the relationship between visual stimuli-triggered gaze behaviors and noticeability of redirection.
With the collected data, we analyzed various numerical features from the gaze behaviors to identify the most effective ones. 
We tested combinations of these features to determine the most effective one for predicting noticeability under visual stimuli.
Using the selected features, our regression model achieved a mean squared error (MSE) of 0.011 through leave-one-user-out cross-validation. 
Furthermore, we developed both a binary and a three-class classification model to categorize noticeability, which achieved an accuracy of 91.74\% and 85.62\%, respectively.

% evaluation study
To evaluate the generalizability of the regression model, we conducted an evaluation study (N=24) to test whether the model could accurately predict noticeability with new visual stimuli (color- and scale-based animations).
Specifically, we evaluated whether the model's predictions aligned with participants' responses under these unseen stimuli.
The results showed that our model accurately estimated the noticeability, achieving mean squared errors (MSE) of 0.014 and 0.012 for the color- and scale-based visual stimili, respectively, compared to participants' responses.
Since the tested visual stimuli data were not included in the training, the results suggested that the extracted gaze behavior features capture a generalizable pattern and can effectively indicate the corresponding impact on the noticeability of redirection.

% application
Based on our model, we implemented an adaptive redirection technique and demonstrated it through two applications: adaptive VR action game and opportunistic rendering.
We conducted a proof-of-concept user study (N=8) to compare our adaptive redirection technique with a static redirection, evaluating the usability and benefits of our adaptive redirection technique.
The results indicated that participants experienced less physical demand and stronger sense of embodiment and agency when using the adaptive redirection technique. 
These results demonstrated the effectiveness and usability of our model.

In summary, we make the following contributions.
% 
\begin{itemize}
    \item 
    We propose to use users' gaze behavior as a medium to quantify how visual stimuli influences the noticebility of redirection. 
    Through two user studies, we confirm that visual stimuli significantly influences noticeability and identify key gaze behavior features that are closely related to this impact.
    \item 
    We build a regression model that takes the user's gaze behavioral data as input, then computes the noticeability of redirection.
    Through an evaluation study, we verify that our model can estimate the noticeability with new participants under unseen visual stimuli.
    These findings suggest that the extracted gaze behavior features effectively capture the influence of visual stimuli on noticeability and can generalize across different users and visual stimuli.
    \item 
    We develop an adaptive redirection technique based on our regression model and implement two applications with it.
    With a proof-of-concept study, we demonstrate the effectiveness and potential usability of our regression model on real-world use cases.

\end{itemize}

% \delete{
% Virtual Reality (VR) allows the user to embody a virtual avatar by mirroring their physical movements through the avatar.
% As the user's visual access to the physical world is blocked in tasks involving motion control, they heavily rely on the visual representation of the avatar's motions to guide their proprioception.
% Similar to real-world experiences, the user is able to resolve conflicts between different sensory inputs (e.g., vision and motor control) through multisensory integration, which is essential for mitigating the sensory noise that commonly arises.
% However, it also enables unique manipulations in VR, as the system can intentionally modify the avatar's movements in relation to the user's motions to achieve specific functional outcomes,
% for example, 
% % the manipulations on the avatar's movements can 
% enabling novel interaction techniques of redirected walking~\cite{razzaque2005redirected}, redirected reaching~\cite{gonzalez2022model}, and pseudo haptics~\cite{samad2019pseudo}.
% With small adjustments to the avatar's movements, the user can maintain their sense of embodiment, due to their ability to resolve the perceptual differences.
% % However, a large mismatch between the user and avatar's movements can result in the user losing their sense of embodiment, due to an inability to resolve the perceptual differences.
% }

% \delete{
% However, multisensory integration can break when the manipulation is so intense that the user is aware of the existence of the motion offset and no longer maintains the sense of embodiment.
% Prior research studied the intensity threshold of the offset applied on the avatar's hand, beyond which the embodiment will break~\cite{li2022modeling}. 
% Studies also investigated the user's sensitivity to the offsets over time~\cite{kohm2022sensitivity}.
% Based on the findings, we argue that one crucial factor that affects to what extent the user notices the offset (i.e., \textit{noticeability}) that remains under-explored is whether the user directs their visual attention towards or away from the virtual avatar.
% Related work (e.g., Mise-unseen~\cite{marwecki2019mise}) has showcased applications where adjustments in the environment can be made in an unnoticeable manner when they happen in the area out of the user's visual field.
% We hypothesize that directing the user's visual attention away from the avatar's body, while still partially keeping the avatar within the user's field-of-view, can reduce the noticeability of the offset.
% Therefore, we conduct two user studies and implement a regression model to systematically investigate this effect.
% }

% \delete{
% In the first user study (N = 16), we test whether drawing the user's visual attention away from their body impacts the possibility of them noticing an offset that we apply to their arm motion in VR.
% We adopt a dual-task design to enable the alteration of the user's visual attention and a yes/no paradigm to measure the noticeability of motion offset. 
% The primary task for the user is to perform an arm motion and report when they perceive an offset between the avatar's virtual arm and their real arm.
% In the secondary task, we randomly render a visual animation of a ball turning from transparent to red and becoming transparent again and ask them to monitor and report when it appears.
% We control the strength of the visual stimuli by changing the duration and location of the animation.
% % By changing the time duration and location of the visual animation, we control the strengths of attraction to the users.
% As a result, we found significant differences in the noticeability of the offsets $(F_{(1,15)}=5.90,~p=0.03)$ between conditions with and without visual stimuli.
% Based on further analysis, we also identified the behavioral patterns of the user's gaze (including pupil dilation, fixations, and saccades) to be correlated with the noticeability results $(r=-0.43)$ and they may potentially serve as indicators of noticeability.
% }

% \delete{
% To further investigate how visual attention influences the noticeability, we conduct a data collection study (N = 12) and build a regression model based on the data.
% The regression model is able to calculate the noticeability of the offset applied on the user's arm under various visual stimuli based on their gaze behaviors.
% Our leave-one-out cross-validation results show that the proposed method was able to achieve a mean-squared error (MSE) of 0.012 in the probability regression task.
% }

% \delete{
% To verify the feasibility and extendability of the regression model, we conduct an evaluation study where we test new visual animations based on adjustments on scale and color and invite 24 new participants to attend the study.
% Results show that the proposed method can accurately estimate the noticeability with an MSE of 0.014 and 0.012 in the conditions of the color- and scale-based visual effects.
% Since these animations were not included in the dataset that the regression model was built on, the study demonstrates that the gaze behavioral features we extracted from the data capture a generalizable pattern of the user's visual attention and can indicate the corresponding impact on the noticeability of the offset.
% }

% \delete{
% Finally, we demonstrate applications that can benefit from the noticeability prediction model, including adaptive motion offsets and opportunistic rendering, considering the user's visual attention. 
% We conclude with discussions of our work's limitations and future research directions.
% }

% \delete{
% In summary, we make the following contributions.
% }
% % 
% \begin{itemize}
%     \item 
%     \delete{
%     We quantify the effects of the user's visual attention directed away by stimuli on their noticeability of an offset applied to the avatar's arm motion with respect to the user's physical arm. 
%     Through two user studies, we identified gaze behavioral features that are indicative of the changes in noticeability.
%     }
%     \item 
%     \delete{We build a regression model that takes the user's gaze behavioral data and the offset applied to the arm motion as input, then computes the probability of the user noticing the offset.
%     Through an evaluation study, we verified that the model needs no information about the source attracting the user's visual attention and can be generalizable in different scenarios.
%     }
%     \item 
%     \delete{We demonstrate two applications that potentially benefit from the regression model, including adaptive motion offsets and opportunistic rendering.
%     }

% \end{itemize}

\begin{comment}
However, users will lose the sense of embodiment to the virtual avatars if they notice the offset between the virtual and physical movements.
To address this, researchers have been exploring the noticing threshold of offsets with various magnitudes and proposing various redirection techniques that maintain the sense of embodiment~\cite{}.

However, when users embody virtual avatars to explore virtual environments, they encounter various visual effects and content that can attract their attention~\cite{}.
During this, the user may notice an offset when he observes the virtual movement carefully while ignoring it when the virtual contents attract his attention from the movements.
Therefore, static offset thresholds are not appropriate in dynamic scenarios.

Past research has proposed dynamic mapping techniques that adapted to users' state, such as hand moving speed~\cite{frees2007prism} or ergonomically comfortable poses~\cite{montano2017erg}, but not considering the influence of virtual content.
More specifically, PRISM~\cite{frees2007prism} proposed adjusting the C/D ratio with a non-linear mapping according to users' hand moving speed, but it might not be optimal for various virtual scenarios.
While Erg-O~\cite{montano2017erg} redirected users' virtual hands according to the virtual target's relative position to reduce physical fatigue, neglecting the change of virtual environments. 

Therefore, how to design redirection techniques in various scenarios with different visual attractions remains unknown.
To address this, we investigate how visual attention affects the noticing probability of movement offsets.
Based on our experiments, we implement a computational model that automatically computes the noticing probability of offsets under certain visual attractions.
VR application designers and developers can easily leverage our model to design redirection techniques maintaining the sense of embodiment adapt to the user's visual attention.
We implement a dynamic redirection technique with our model and demonstrate that it effectively reduces the target reaching time without reducing the sense of embodiment compared to static redirection techniques.

% Need to be refined
This paper offers the following contributions.
\begin{itemize}
    \item We investigate how visual attractions affect the noticing probability of redirection offsets.
    \item We construct a computational model to predict the noticing probability of an offset with a given visual background.
    \item We implement a dynamic redirection technique adapting to the visual background. We evaluate the technique and develop three applications to demonstrate the benefits. 
\end{itemize}



First, we conducted a controlled experiment to understand how users perceived the movement offset while subjected to various distractions.
Since hand redirection is one of the most frequently used redirections in VR interactions, we focused on the dynamic arm movements and manually added angular offsets to the' elbow joint~\cite{li2022modeling, gonzalez2022model, zenner2019estimating}. 
We employed flashing spheres in the user's field of view as distractions to attract users' visual attention.
Participants were instructed to report the appearing location of the spheres while simultaneously performing the arm movements and reporting if they perceived an offset during the movement. 
(\zhipeng{Add the results of data collection. Analyze the influence of the distance between the gaze map and the offset.}
We measured the visual attraction's magnitude with the gaze distribution on it.
Results showed that stronger distractions made it harder for users to notice the offset.)
\zhipeng{Need to rewrite. Not sure to use gaze distribution or a metric obtained from the visual content.}
Secondly, we constructed a computational model to predict the noticing probability of offsets with given visual content.
We analyzed the data from the user studies to measure the influence of visual attractions on the noticing probability of offsets.
We built a statistical model to predict the offset's noticing probability with a given visual content.
Based on the model, we implement a dynamic redirection technique to adjust the redirection offset adapted to the user's current field of view.
We evaluated the technique in a target selection task compared to no hand redirection and static hand redirection.
\zhipeng{Add the results of the evaluation.}
Results showed that the dynamic hand redirection technique significantly reduced the target selection time with similar accuracy and a comparable sense of embodiment.
Finally, we implemented three applications to demonstrate the potential benefits of the visual attention adapted dynamic redirection technique.
\end{comment}

% This one modifies arm length, not redirection
% \citeauthor{mcintosh2020iteratively} proposed an adaptation method to iteratively change the virtual avatar arm's length based on the primary tasks' performance~\cite{mcintosh2020iteratively}.



% \zhipeng{TO ADD: what is redirection}
% Redirection enables novel interactions in Virtual Reality, including redirected walking, haptic redirection, and pseudo haptics by introducing an offset to users' movement.
% \zhipeng{TO ADD: extend this sentence}
% The price of this is that users' immersiveness and embodiment in VR can be compromised when they notice the offset and perceive the virtual movement not as theirs~\cite{}.
% \zhipeng{TO ADD: extend this sentence, elaborate how the virtual environment attracts users' attention}
% Meanwhile, the visual content in the virtual environment is abundant and consistently captures users' attention, making it harder to notice the offset~\cite{}.
% While previous studies explored the noticing threshold of the offsets and optimized the redirection techniques to maintain the sense of embodiment~\cite{}, the influence of visual content on the probability of perceiving offsets remains unknown.  
% Therefore, we propose to investigate how users perceive the redirection offset when they are facing various visual attractions.


% We conducted a user study to understand how users notice the shift with visual attractions.
% We used a color-changing ball to attract the user's attention while instructing users to perform different poses with their arms and observe it meanwhile.
% \zhipeng{(Which one should be the primary task? Observe the ball should be the primary one, but if the primary task is too simple, users might allocate more attention on the secondary task and this makes the secondary task primary.)}
% \zhipeng{(We need a good and reasonable dual-task design in which users care about both their pose and the visual content, at least in the evaluation study. And we need to be able to control the visual content's magnitude and saliency maybe?)}
% We controlled the shift magnitude and direction, the user's pose, the ball's size, and the color range.
% We set the ball's color-changing interval as the independent factor.
% We collect the user's response to each shift and the color-changing times.
% Based on the collected data, we constructed a statistical model to describe the influence of visual attraction on the noticing probability.
% \zhipeng{(Are we actually controlling the attention allocation? How do we measure the attracting effect? We need uniform metrics, otherwise it is also hard for others to use our knowledge.)}
% \zhipeng{(Try to use eye gaze? The eye gaze distribution in the last five seconds to decide the attention allocation? Basically constructing a model with eye gaze distribution and noticing probability. But the user's head is moving, so the eye gaze distribution is not aligned well with the current view.)}

% \zhipeng{Saliency and EMD}
% \zhipeng{Gaze is more than just a point: Rethinking visual attention
% analysis using peripheral vision-based gaze mapping}

% Evaluation study(ideal case): based on the visual content, adjusting the redirection magnitude dynamically.

% \zhipeng{(The risk is our model's effect is trivial.)}

% Applications:
% Playing Lego while watching demo videos, we can accelerate the reaching process of bricks, and forbid the redirection during the manipulation.

% Beat saber again: but not make a lot of sense? Difficult game has complicated visual effects, while allows larger shift, but do not need large shift with high difficulty



\section{A Better QAT Scheduling Strategy for Extreme Low-Bit LLMs}
\label{sec:scalinglaw}
\begin{figure}[t!]
    \centering
    \includegraphics[width=0.82\linewidth]{figures/1_QAT_portion_fig.pdf}
    \caption{\small{With a fixed total training budget of 100B tokens ($\mathcal{B}_\text{train}$), where $\mathcal{B}_\text{FP} + \mathcal{B}_\text{QAT} = \mathcal{B}_\text{train}$, we explore optimal allocation between full-precision pretraining ($\mathcal{B}_\text{FP}$) and QAT fine-tuning ($\mathcal{B}_\text{QAT}$). ``0.0'' represents QAT from scratch, while ``1.0'' indicates full-precision pretraining followed by PTQ. Results on MobileLLM-125M show peak accuracy with $\sim$90\% of the budget for full-precision pretraining and $\sim$10\% for QAT fine-tuning.}
}
    \label{fig:qat_proportion}
\end{figure}

In this work, we systematically investigate trade-offs involving bit precision ($\mathcal{P}$), quantization functions ($\mathcal{F}$), model size ($\mathcal{N}$), training strategies ($\mathcal{S}_{train}$) and training token ($\mathcal{D}$). 
\begin{equation}
\mathcal{L}(\mathcal{P}, \mathcal{F}, \mathcal{N}, \mathcal{S}_{train}, \mathcal{D}) 
\end{equation}
Given the vast search space defined by these variables, we first fix the quantization method ($\mathcal{F}$) and explore the dimensions of bit precision ($\mathcal{P}$), training strategies ($\mathcal{S}_{train}$) and training tokens ($\mathcal{D}$) in this section.


\begin{figure*}[thb]
    \centering
    \includegraphics[width=0.8\linewidth]{figures/2_finetune_vs_scratch_fig.pdf}
    \caption{\small{Analysis of training token requirements for quantization-aware fine-tuning and training from scratch across 1-bit, 1.58-bit, 2-bit, 3-bit, and 4-bit settings. Fine-tuning typically saturates at 10B tokens for 3-bit and 4-bit, and at 30B tokens for 1-bit, 1.58-bit, and 2-bit. Fine-tuning consistently outperforms training from scratch in both accuracy and token efficiency across all bit configurations.}}
    \label{fig:finetune_vs_scratch}
\end{figure*}


\subsection{Training Budget Allocation}

Post-Training Quantization (PTQ) and Quantization-Aware Training (QAT) are two primary quantization approaches. PTQ applies quantization after full-precision training, simplifying deployment but often leads to significant performance loss at bit widths below 4 bits. In contrast, QAT incorporates quantization during training to optimize model performance for low-bit-width representations.

Here we start by answering a key question:

\textbf{Given a fixed training budget (in \#tokens) $\mathcal{B}_{\text{train}} = \mathcal{B}_{\text{FPT}} + \mathcal{B}_{\text{QAT}}$, how should the budget be optimally allocated between full-precision training ($\mathcal{B}_{\text{FPT}}$) and quantization-aware training/fine-tuning ($\mathcal{B}_{\text{QAT}}$) to maximize the accuracy of the quantized model?}

This question is both technically intriguing and practically significant. Our approach begins with analyzing the pretraining phase to determine the optimal switching point from FPT to QAT, aiming to minimize the loss:
\begin{equation}
    \mathcal{B}_\text{FPT}^*, \mathcal{B}_\text{QAT}^* = \mathop{\arg \min}\limits_{\mathcal{B}_{\text{FPT}} + \mathcal{B}_{\text{QAT}}=\mathcal{B}_{\text{train}}} \mathcal{L}(\mathcal{B}_{\text{FPT}}, \mathcal{B}_{\text{QAT}} | \mathcal{N}, \mathcal{P})
\end{equation}
where $\mathcal{B}_\text{FPT}^*$ and $\mathcal{B}_\text{QAT}^*$ describe the optimal allocation of a computational budget $\mathcal{B}_{\text{train}}$. We utilize $\mathcal{B}_{\text{train}}$ to incorporate training tokens utilization ($\mathcal{D}$) into the training strategy ($\mathcal{S}$). 
Specifically, we evaluate various allocation ratios of \(\mathcal{B}_{\text{FPT}}\) and \(\mathcal{B}_{\text{QAT}}\) on MobileLLM-125M across four bit-widths ( 1.58-bit, 2-bit, 3-bit, and 4-bit). The FP models undergo a complete learning rate scheduling cycle for \(\mathcal{B}_{\text{FPT}}\) tokens, followed by another cycle for QAT for \(\mathcal{B}_{\text{QAT}}\) tokens. Detailed experimental settings are provided in the appendix.

Figure~\ref{fig:qat_proportion} reveals a distinct upward trend in the full-precision pre-training proportion versus accuracy curve. Notably, accuracy peaks at $\sim$ 90\% FPT allocation for almost every bit-width choice, then decline sharply when FPT exceeds 90\%, likely because this leaves insufficient tokens and training capacity for QAT. This leads to our first key finding:
\begin{tcolorbox}[
  enhanced,
  colback=blue!4!white,
  boxrule=0.8 pt, 
  boxsep=0pt, 
  left=2pt, 
right=2pt, 
  top=2pt,
  bottom=2pt, 
  drop fuzzy shadow=black!50
]
\textbf{Finding-1} QAT finetuning consistently surpasses both PTQ with $\mathcal{B}_\text{FPT} = \mathcal{B}_\text{train}$ and QAT from scratch with $\mathcal{B}_\text{QAT} = \mathcal{B}_\text{train}$. Optimal performance is nearly achieved by dedicating the majority of the training budget to full precision (FP) training and approximately 10\% to QAT.

\end{tcolorbox}


\subsection{Fine-tuning Characteristics}

Then we investigate the impact of finetuning tokens across various bit choices, spanning 7 architectures and 5 bit levels. Results in Figure~\ref{fig:finetune_vs_scratch} offer several key insights:

1. \textbf{Fine-tuning benefits across all bit-widths}: This observation challenges recent methodologies that trains ternary LLMs from scratch~\cite{kaushal2024spectra, ma2024era}. Instead, we suggest leveraging pre-trained full-precision models for initialization is a more effective approach for training quantized networks, including binary and ternary.

2. \textbf{Optimal fine-tuning budget and bit width}: Lower bit quantization (binary, ternary, 2-bit) requires more fine-tuning than higher bit quantization (3-bit, 4-bit). 3-bit and 4-bit reach near full precision accuracy after 10B tokens, while lower-bit quantization saturates around 30B tokens.

\begin{figure}[t!]
    \centering
    \includegraphics[width=0.8\linewidth]{figures/22_err_violin_fig.pdf}
    \caption{\small L1 norm difference between QAT-finetuned weights and full-precision initialization (\(||W_{\text{finetune}}\) \(- W_{\text{init}}||_{l1}\) \(/||W_{\text{init}}||_{l1}\)). Models quantized to 1, 1.58, and 2 bits show larger weight changes, indicating distinct `\textit{compensation}' behavior in higher-bit quantization and `\textit{reconstruction}' in lower-bit settings.}
    \label{fig:22_err_violin}
\end{figure}

\begin{figure*}[bth!]
    \centering
    \includegraphics[width=0.9\linewidth]{figures/31_quantization_function_compare_fig.pdf}
    \caption{\small Impact of quantization grid choice across bit widths. Binary quantization uses a sign function; Ternary and 2-bit prefer more balanced output levels and range coverage; For 3-bit and higher, including ``0" in quantization levels is more favorable.}
    \label{fig:31_quantization_function_compare}
\end{figure*}
3. \textbf{QAT behavior transition between bit-widths}: Networks quantized to 3-bit/4-bit recover near full-precision accuracy after fine-tuning, while binary, ternary, and 2-bit saturate before achieving full accuracy. We hypothesize that QAT acts as ``\textit{compensation}" for bit-widths above 2-bit, adjusting weights within adjacent quantization levels, and as ``\textit{reconstruction}" below 2-bit, where weights adapt beyond nearby grids to form new representations. This is supported by weight change analysis in Figure~\ref{fig:22_err_violin}, showing smaller adjustments in 3-bit/4-bit (10-20\%) and larger shifts in lower-bit quantization ($\sim$40\%), indicating substantial value reconstruction.

\begin{tcolorbox}[
  enhanced,
  colback=blue!4!white,
  boxrule=0.8 pt, 
  boxsep=0pt, 
  left=2pt, 
right=2pt, 
  top=2pt, 
  bottom=2pt, 
  drop fuzzy shadow=black!50
]
\textbf{Finding-2} 
While fine-tuning enhances performance across all bit-widths, even binary and ternary, optimal fine-tuning effort inversely correlates with bit-width. For 3-bit and 4-bit weights, fine-tuning adjusts within a nearby grid to mitigate accuracy loss, and requires less finetuning tokens. In contrast, binary and ternary weights break the grid, creating new semantic representations to maintain performance, requiring longer finetuning.
\end{tcolorbox}

\section{A Hitchhiker’s Guide to Quantization Method Choices}\label{sec:quantization_choice}

We have examined the impact of training strategy and budget allocations ($\mathcal{B}_\text{train}$, $\mathcal{B}_\text{QAT}$) on scaling laws. Building on the optimal training practices outlined in Section~\ref{sec:scalinglaw}, we focus on a critical yet often overlooked factor: the choice of quantization functions ($\mathcal{F}$).
\begin{equation}
   \mathcal{F}^* = \mathop{\arg \min}\limits_{\mathcal{F}} \mathcal{L}(\mathcal{F} | \mathcal{P}, \mathcal{B}_\text{QAT}^*)
\end{equation}
The significance of this choice has been largely underestimated in prior scaling law studies~\cite{kumar2024scaling}. Our results show that, especially at sub-4-bit quantization, the choice of function is highly sensitive and can drastically alter scaling law outcomes. An improper selection can distort performance and lead to entirely different conclusions, underscoring the need for a careful design of $\mathcal{F}$.

\subsection{Preliminary}
In general, a uniform quantization function is expressed as
\begin{equation}
\label{eq:minmax}
    \mathbf{W}_\mathbf{Q}^i = \alpha \nint{\frac{\mathbf{W}_\mathbf{R}^i - \beta}{\alpha}} + \beta \ 
\end{equation}
Here $\mathbf{W_Q}$ represents quantized weights, $\mathbf{W_R}$ denotes their real-valued counterparts~\citep{quantization_whitepaper, krishnamoorthi2018quantizing}. Key design choices focus on scale $\alpha$ and bias $\beta$. For symmetric min-max quantization, $\alpha = \frac{\max(|\mathbf{W_R}|)}{2^{N-1} - 1}$ and $\beta = 0$. In asymmetric min-max quantization, $\alpha = \frac{\max(\mathbf{W_R}) - \min(\mathbf{W_R})}{2^N - 1}$ and $\beta = \min(\mathbf{W_R})$. Symmetric min-max quantization is prevalent for weights $\geqslant$ 4 bits, while sub-4-bit quantization requires distinct functions.

For binary quantization, assigning the sign of full-precision weights ($\mathbf{W}_\mathbf{R}$) to binary weights ($\mathbf{W}_\mathbf{B}$) is a commonly used approach~\cite{rastegari2016xnor,liu2018bi}: $\mathbf{W}_\mathbf{B}^i = \alpha \cdot \text{Sign}(\mathbf{W}_\mathbf{R}^i)$, where $\alpha = \frac{||\mathbf{W_R}||_{l1}}{n_{\mathbf{W_R}}}$. 

In ternary quantization, ternary weights are often given by $\mathbf{W}_\mathbf{T}^i = \alpha \cdot \text{Sign}(\mathbf{W}_\mathbf{R}^i) \cdot \mathbf{1}_{|\mathbf{W}_\mathbf{R}^i| > \Delta}$, with $\Delta = \frac{0.7 \cdot ||\mathbf{W_R}||_{l1}}{n_{\mathbf{W_R}}}$ and $\alpha_{_\mathbf{T}} = \frac{\sum_i \mathbf{W}_\mathbf{R}^i \cdot \mathbf{1}_{| \mathbf{W}_\mathbf{R}^i| > \Delta}}{\sum_i \mathbf{1}_{| \mathbf{W}_\mathbf{R}^i| > \Delta}}$ ~\cite{TernaryBERT,liu2023binary}.
Besides binary and ternary quantization, there is less work targeting 2-bit or 3-bit integer quantization function design. Directly using min-max quantization for them will lead to performance collapse.

\subsection{Introducing \ours{}}
In sub-4-bit quantization, design requirements vary significantly across bit levels. Equal attention to each bit choice is crucial for accurate, reliable comparisons.

\subsubsection{Trade-offs}
We identify two key trade-offs in low-bit quantization for LLMs: (1) Outlier precision vs. intermediate value precision and (2) Symmetry vs. inclusion of ``0" at the output level.

\begin{figure*}[thb]
    \centering
    \includegraphics[width=\linewidth]{figures/32_quantization_method_results_fig.pdf}
    \caption{\small Comparison of quantization methods across different bit-widths. Extreme low-bit quantization is highly sensitive to quantization function selection. (b)-(e) show that the learnable policy with range clipping updated via final loss consistently outperforms stats-based methods with fixed range clipping. From (f)-(i), the SEQ works better for ternary and 2-bit quantization, while 3 and 4-bits favor LSQ.}
    \label{fig:quantization_method}
\end{figure*}


\textbf{(1) Range clipping}
Outliers challenge LLM quantization~\cite{lin2023awq,liu2024spinquant}, especially when using \textit{min-max} ranges for weight quantization for extremely low-bit quantization. As seen in Figure~\ref{fig:quantization_method} (b)-(e), \textit{min-max} quantization works at 4 bits but loses accuracy at lower bit-widths. On the other hand, range clipping improves lower-bit quantization but harms 4-bit accuracy. We refer to range-setting methods based on weight statistics as ``stats-based" approaches. The effectiveness of these quantization functions varies with different bit choices.

Learnable scales, however, optimize quantization ranges as network parameters, balancing outlier suppression and precision. Solutions like LSQ~\cite{esser2019learned} and its binary~\cite{liu2022bit} and ternary~\cite{liu2023binary} extensions exist. While prior work favored learnable policies for activations but used statistics-based quantization for weights~\cite{liu2023llm}, we find that, with appropriate gradient scaling, learnable scales yield stable, superior performance for weights. As shown in Figure~\ref{fig:quantization_method} (b)-(e), learnable policies consistently outperform stats-based methods across all bit widths.

\textbf{(2) Quantization grids}
Level symmetry in quantization grids is crucial for lower-bit quantization, yet it is rarely discussed. The ``0" in quantization output levels is essential for nullifying irrelevant information, but in even-level quantization (e.g., 2-bit, 3-bit, 4-bit), including ``0" results in imbalanced levels. For example, in 2-bit quantization, options like \((-2, -1, 0, 1)\) and \((-1.5, -0.5, 0.5, 1.5)\) exist. The former limits representation with only one positive level, while the latter offers a balanced distribution. Inspired by this, we propose Stretched Elastic Quant (SEQ), an amendment to LSQ for lower-bit scenarios:
\begin{equation}
   \!\!\!\mathbf{W}_\mathbf{Q}^i \!=\!\alpha\!\left(\!\nint{{\rm Clip}\!\left(\!\frac{\mathbf{W}_\mathbf{R}^i}{\alpha}, -1, 1\!\right)\!\!\times\!\! \frac{k}{2} -\! 0.5}\! +\! 0.5\right)\!\! / k \!\times\! 2 
\end{equation}
Here, \(k\) denotes the number of quantization levels. Figure~\ref{fig:31_quantization_function_compare} visualizes quantization grids, showing that SEQ not only balances output quantized levels but also evenly divides the full-precision weight span to quantization levels, which turns out to be crucial for extremely low-bit quantization. Figure~\ref{fig:quantization_method} (f)-(i) demonstrate SEQ's superiority in ternary and 2-bit quantization, while LSQ with ``0'' in output level slightly outperforms in 3 and 4-bit cases.

\begin{tcolorbox}[
  enhanced,
  colback=blue!4!white,
  boxrule=0.8 pt, 
  boxsep=0pt, 
  left=2pt, 
right=2pt, 
  top=2pt, 
  bottom=2pt, 
  drop fuzzy shadow=black!50
]
\textbf{Finding-3} Extreme low-bit quantization is highly sensitive to quantization function selection, with no single optimal function for all bit widths. Learnable range settings outperform statistics-based methods due to their flexibility in optimizing range parameters with respect to the final loss. Ternary and 2-bit quantization favor symmetric levels and balanced range coverage in quantization grid configuration, while imbalance levels with ``0" in output levels are more effective for 3 and 4-bit quantization.
\end{tcolorbox}

\begin{figure*}[t!]
    \centering
    \includegraphics[width=0.95\linewidth]{figures/pareto_curve_fig.pdf}
    \caption{\small (a) (b) In sub-4-bit regime, 1.58-bit, 2-bit, and 3-bit quantization outperform 4-bit in terms of the accuracy-model size trade-off. (c) Under hardware constraints, 2-bit quantization demonstrates superior accuracy-speed trade-offs compared to higher-bit schemes.}
    \label{fig:pareto_curve}
\end{figure*}

\subsubsection{Quantization Function}
Based on our analysis, we integrate the optimal quantization functions identified for each bit-width into one formula, denoted as \ours. This includes Elastic Binarization~\cite{liu2022bit} for 1-bit quantization, LSQ~\cite{esser2019learned} for 3 and 4-bit quantization, and the proposed SEQ for 1.58 and 2-bit quantization: 

\begin{equation}
\begin{split}
\label{eq:custom_quant_forward}
&\mathbf{W}_\mathbf{Q}^i = \alpha \mathbf{\hat{W}_Q}^i  \\
   & = \left\{  
         \begin{array}{lr}
         \!\!\!\alpha \! \cdot \! {\rm Sign}(\mathbf{W}_\mathbf{R}^i),  \ \ \ \ \ \ \ \ \ \ \ \ \ \ \ \ \ \ \ \ \ \ \ \ \ \ \ \ {\rm if} \ N_{bit}=1 \\ 
         \vspace{0.3em}
         \!\!\!\alpha(\nint{{\rm Clip}(\frac{\mathbf{W}_\mathbf{R}^i}{\alpha}, -1, 1) \times k/2 - 0.5} + 0.5) / k\times2,  \\ 
         \vspace{0.3em}
         \ \ \ \ \ \ \ \ \ \ \ \ \ \ \ \ \ \ \ \ \ \ \ \ \ \ \ \ \ \ \ \ \ \ \ \ \ \ \ \ \ \ \ \ \ \ \ \ \ \ \ {\rm if} \ N_{bit}=1.58, 2 \\ 
         \!\!\!\alpha \nint{{\rm Clip}(\frac{\mathbf{W}_\mathbf{R}^i}{\alpha}, n, p)},  \ \ \ \ \ \ \ \ \ \ \ \ \ \ \ \ \ \ \ \ {\rm if} \ N_{bit}=3, 4 \\ 
         \end{array} 
         \right.     
\end{split}
\end{equation}
Here $k$ equals $3$ in the ternary case and $2^{N_{bit}}$ otherwise; $n = -2^{N_{bit} -1}$ and $p =2^{N_{bit} -1} - 1$. In the backward pass, the gradients to the weights and scaling factor can be easily calculated using straight-through estimator:
\begin{equation}
\begin{split}
\label{eq:custom_quant_backward_w}
\frac{\partial\mathbf{W}_\mathbf{Q}^i}{\partial\mathbf{W}_\mathbf{R}^i} \overset{STE}{\approx}
& \left\{ 
        \begin{array}{lr} 
        \vspace{0.3em}
        \mathbf{\large{1}}_{|\frac{\mathbf{W}_\mathbf{R}^i}{\alpha}|< 1}, \ \ \ \ \ \ {\rm if} \ N_{bit}=1, 1.58, 2 \\ 
        \mathbf{\large{1}}_{n < \frac{\mathbf{W}_\mathbf{R}^i}{\alpha} < p}, \ \ \ \ {\rm if} \ N_{bit}=3,4 
        \end{array} 
        \right.     
\end{split}
\end{equation}
\begin{equation}
\begin{split}
\label{eq:custom_quant_backward_w}
\frac{\partial\mathbf{W}_\mathbf{Q}^i}{\alpha} \overset{STE}{\approx}
& \left\{ 
        \begin{array}{lr} 
        \vspace{0.6em}
        \!\!{\rm Sign}(\mathbf{W}_\mathbf{R}^i), \ \ \ \ \ \ \ \ \ \ \ \ \ \ \ \ \ \ \ \ \ \ {\rm if} \ N_{bit}=1 \\ 
        \vspace{0.5em}
        \!\!\mathbf{\hat{W}_R}^i -\! \frac{\mathbf{W}_\mathbf{R}^i}{\alpha} \cdot \mathbf{\large{1}}_{|\frac{\mathbf{W}_\mathbf{R}^i}{\alpha}|< 1}, \ \ \ \ {\rm if} \ N_{bit}=1.58, 2 \\
        \!\!\mathbf{\hat{W}_R}^i -\! \frac{\mathbf{W}_\mathbf{R}^i}{\alpha} \cdot \mathbf{\large{1}}_{n < \frac{\mathbf{W}_\mathbf{R}^i}{\alpha} < p}, \ \ {\rm if} \ N_{bit}=3,4 
        \end{array} 
        \right.     
\end{split}
\end{equation}

For the initialization of $\alpha$, we use $\alpha = \frac{||\mathbf{W}_\mathbf{R}||_{l1}}{n_{_{\mathbf{W}_\mathbf{R}}}}$ for the binary case, since the scaling factor has the closed-form solution to minimizing quantization error: $\mathcal{E} = || \alpha \hat{\mathbf{W}}_\mathbf{Q} - \mathbf{W}_\mathbf{R} ||_{l2}$. 
For the other cases, we simply initialize $\alpha$ as the maximum absolute value of the weights. For ternary and 2-bit quantization, $\alpha = \max(|\mathbf{W}_\mathbf{R}|)$, associated with SEQ quantizer, and for 3-bit and 4-bit cases, $\alpha =\frac{\max(|\mathbf{W}_\mathbf{R}|)}{p}$,  associated with LSQ quantizer. 

With \ours{}, we present a robust comparison framework across five bit-widths (1-bit, 1.58-bit, 2-bit, 3-bit, 4-bit), each achieving state-of-the-art accuracy. This facilitates direct, apple-to-apple comparisons to identify the most effective bit-width selection.







\section{Pareto-Optimality of Extremely Low-Bit LLM}\label{sec:pareto_frontier}
To ensure a consistent apples-to-apples performance comparison across different bit-width configurations, we first determined the optimal training setup ($\mathcal{B}_{train}^*$) in Section~\ref{sec:scalinglaw} and the quantization function ($\mathcal{F}^*$) for each bit in Section~\ref{sec:quantization_choice}. Using this unified framework for all bit widths, we examine the trade-off between model size and quantization bit: $\mathcal{L}(\mathcal{P}, \mathcal{N} | \mathcal{F}^*, \mathcal{B}_{train}^*)$.

\subsection{Accuracy-compression Trade-off}
In on-device deployment scenarios, such as wearables and portables, storage constraints often limit the capacity of large language models (LLMs). To optimize performance within these constraints, quantization is essential. A common dilemma is whether to train a larger model and quantize it to a lower bit-width or to train a smaller model and quantize it to a higher bit-width.

4-bit quantization-aware training (QAT) achieves near-lossless compression in many scenarios, making it widely adopted. However, the landscape below 4-bit remains unclear, with limited comparative analysis. Previous claims about ternary models matching 16-bit performance~\cite{ma2024era} were based on lower FP16 baselines than current standards. Spectra's comparisons between ternary QAT and 4-bit PTQ fall short of a fair evaluation due to inconsistencies in the training schemes used~\cite{spectra}.

With $\ours{}$, we are able to improve the analysis. Figure~\ref{fig:pareto_curve} (a) demonstrates that sub-4-bit quantization, including binary, ternary, 2-bit, and 3-bit, often surpasses 4-bit. Notably, 2-bit and ternary models reside on the Pareto frontier. For instance, a 2-bit MobileLLM-1B model achieves 1.8 points higher accuracy than a 4-bit MobileLLM-600M model, with even smaller model sizes. This trend persists across larger LLaMA models, as shown in Figure~\ref{fig:pareto_curve} (b), demonstrating the potential of lower-bit quantization for achieving both higher accuracy and compression. We calculate the effective quantized model size as $(\#\text{weights} \times \text{weight-bits}+ \#\text{embedding-weights} \times \text{embedding-bits})/8$. More comprehensive analysis is provided in the Appendix.

\subsection{Hardware Implementation Constraints}
\label{sec:on_device}

In practical deployment, both memory limitations and hardware constraints must be considered. While 2-bit and ternary quantization sit on the accuracy-size Pareto frontier, 2-bit quantization is generally more feasible due to practical challenges. 
Ternary quantization, using a 1.58-bit format with values  \(\{-1, 0, 1\}\), appears more storage-efficient but is inefficient in implementation.
Storing ternary values with sparsity exploitation is effective only when sparsity exceeds 90\%, due to high indexing costs. Packing ternary values into an Int32 offers limited compression but complicates GEMM. Some approaches~\cite{yang20241} even store ternary values as 2-bit signed integers, negating the expected storage benefits. In contrast, 2-bit quantization directly maps bit pairs to values, reducing unpacking and conversion overhead, which can be more efficient for custom GEMM kernels. As a result, 2-bit quantization is often a more practical choice for deployment.

\subsection{Accuracy-speed Trade-off}
To evaluate potential speedup benefits beyond memory reduction, we implemented 2-bit quantization kernels on the CPU and compared them with 4-bit quantization. The curves in Figure~\ref{fig:pareto_curve} (c) demonstrate that, within our experimental range, 2-bit quantized models consistently outperform 4-bit models in terms of accuracy-speed performance, positioning 2-bit quantization as a superior choice for on-device applications where both latency and storage are critical. Detailed settings are provided in the appendix.
\section{Experimental Evaluation}\label{section:experiments}
We already achieved our primary objective of deriving time-series-specific subsampling guarantees for DP-SGD adapted to forecasting.
Our main goal for this section is to investigate the trade-offs we discovered in discussing these guarantees.
In addition, we train common probabilistic forecasting architectures on standard datasets to verify the feasibility of training deep differentially private forecasting models while retaining meaningful utility.
The full experimental setup  is described in~\cref{appendix:experimental_setup}.
%An implementation will be made available upon publication.

\subsection{Trade-Offs in Structured Subsampling}

\begin{figure}
    \vskip 0.2in
    \centering
        \includegraphics[width=0.99\linewidth]{figures/experiments/eval_pld_deterministic_vs_random_top_level/daily_20_32_main.pdf}
        \vskip -0.3cm
        \caption{Top-level deterministic iteration (\cref{theorem:deterministic_top_level_wr}) vs top-level WOR sampling (\cref{theorem:wor_top_level_wr}) for $\numinstances=1$.
        Sampling is more private despite requiring more compositions.}
        \label{fig:deterministic_vs_random_top_level_daily_main}
    \vskip -0.2in
\end{figure}




For the following experiments, we assume that we have $N=320$ sequences, batch size $\batchsize = 32$, and noise scale $\sigma = 1$.
We further assume $L=10  (L_F + L_C) + L_F - 1$, so that 
the chance of bottom-level sampling a subsequence containing any specific element is 
$r=0.1$ when choosing $\numinstances = 1$ as the number of subsequences.
In~\cref{appendix:extra_experiments_eval_pld}, we repeat all experiments with a wider range of parameters.
All results are consistent with the ones shown here.

\textbf{Number of Subsequences $\bm{\numinstances}$.}
Let us begin with a trade-off inherent to bi-level subsampling:
We can achieve the same batch size $\batchsize$ with different $\numinstances$, each
leading to different top- and bottom-level amplification.
We claim that $\numinstances = 1$ (i.e., maximum bottom-level amplification) is preferable.
For a fair comparison, we compare our provably tight guarantee for $\numinstances=1$ (\cref{theorem:wor_top_level_wr})
with optimistic lower bounds for $\numinstances > 1$ (\cref{theorem:wor_top_wr_bottom_upper})
instead of our sound upper bounds (\cref{theorem:wor_top_level_wr_general}), i.e.,
we make the competitors stronger.
As shown in~\cref{fig:monotonicity_daily_main}, $\numinstances = 1$ only has smaller $\delta(\epsilon)$ for $\epsilon \geq 10^{-1}$ when considering a single training step.
However, after $100$-fold composition, $\numinstances = 1$ achieves smaller $\delta(\epsilon)$ even in $[10^{-3}, 10^{-1}]$ (see~\cref{fig:monotonicity_composed_daily_main}).
Our explanation is that $\numinstances > 1$ results in larger $\delta(\epsilon)$ for large $\epsilon$, i.e., is more likely to have a large privacy loss.
Because the privacy loss of a composed mechanism is the sum of component privacy losses~\cite{sommer2018privacy}, this is problematic when performing multiple training steps.
We shall thus later use $\numinstances=1$ for training.

%Intuitively, $\delta(\epsilon)$ can be interpreted as the probability that the log-likelihood ratio of $M_x$ and $M_{x'}$ (``privacy loss'') exceeds $\epsilon$.\footnote{For the formal relation between privay loss and privacy profiles, see~\cref{lemma:profile_from_pld} taken from~\cite{balle2018improving}}


\textbf{Step- vs Epoch-Level Accounting.}
Next, we show the benefit of top-level sampling sequences (\cref{theorem:wor_top_level_wr}) instead of deterministically iterating over them (\cref{theorem:deterministic_top_level_wr}), even though we risk privacy leakage at every training step.
For our parameterization and $\numinstances=1$, top-level sampling with replacement requires $10$ compositions per epoch.
As shown in~\cref{fig:deterministic_vs_random_top_level_daily_main}, the resultant epoch-level profile is nevertheless smaller, and remains so after $10$ epochs.
This is consistent with any work on DP-SGD (e.g., \cite{abadi2016deep}) that uses subsampling instead of deterministic iteration.

\textbf{Epoch Privacy vs Length.} In~\cref{appendix:extra_experiments_epoch_length} we additionally explore the fact that, if we wanted to use deterministic top-level iteration, 
the number of subsequences 
$\numinstances$ would affect epoch length.
As expected, we observe that composing many private mechanisms ($\numinstances=1$) is preferable to composing few much less private mechanisms ($\numinstances > 1$) 
when considering a fixed number of training steps.

\begin{figure}
    \vskip 0.2in
    \centering
        \includegraphics[width=0.99\linewidth]{figures/experiments/eval_pld_label_noise/daily_30_32_main.pdf}
        \vskip -0.3cm
        \caption{Varying label noise $\sigma_F$ for top-level WOR and bottom-level WR  (\cref{theorem:data_augmentation_general}) with $\sigma_C = 0, \numinstances=1$.
        Increasing $\sigma_F$ is equivalent to decreasing forecast length.
        }
        \label{fig:label_noise_daily_main}
    \vskip -0.2in
\end{figure}

\textbf{Amplification by Label Perturbation.}
Finally, because the way in which adding Gaussian noise to the context and/or forecast window 
amplifies privacy (\cref{theorem:data_augmentation_general}) 
may be somewhat opaque, let us consider top-level sampling without replacement, bottom-level sampling with replacement,
$\numinstances=1$, $\sigma_C=0$, and varying label noise standard deviations $\sigma_F$. 
As shown in~\cref{fig:label_noise_daily_main}, increasing $\sigma_F$ has the same effect as letting the forecast length $L_C$ go to zero, i.e., eliminates the risk of leaking private information if it appears in the forecast window.
Of course, this data augmentation 
will have an effect on model utility, which we investigate shortly.

\begin{figure*}
\centering
\vskip 0.2in
    \begin{subfigure}{0.49\textwidth}
        \includegraphics[]{figures/experiments/eval_pld_monotonicity_composed/daily_20_32_1_main.pdf}
        \caption{Training step $1$}\label{fig:monotonicity_daily_main}
    \end{subfigure}
    \hfill
    \begin{subfigure}{0.49\textwidth}
        \includegraphics[]{figures/experiments/eval_pld_monotonicity_composed/daily_20_32_100_main.pdf}
        \caption{Training step $100$}\label{fig:monotonicity_composed_daily_main}
    \end{subfigure}\caption{
    Top-level WOR and bottom-level WR sampling under varying number of subsequences.
    Under composition, even optimistic lower bounds (\cref{theorem:wor_top_wr_bottom_upper}) 
    indicate worse privacy for $\numinstances > 1$ than 
    our tight upper bound for $\numinstances=1$ (\cref{theorem:wor_top_level_wr}).}
    \label{fig:monotonicity_daily_main_container}
\vskip -0.2in
\end{figure*}


\subsection{Application to Probabilistic Forecasting}
While the contribution of our work lies in formally analyzing the privacy of DP-SGD adapted to forecasting, 
training models with this algorithm can serve as a sanity-check to verify that the guarantees are sufficiently strong to retain meaningful utility under non-trivial privacy budgets.


\begin{table}[b]
\vskip -0.38cm
\caption{Average CRPS on \texttt{traffic} for $\delta=10^{-7}$. Seasonal, AutoETS, and models with $\epsilon=\infty$ are without noise.}
\label{table:1_event_training_traffic_main}
\vskip 0.18cm
\begin{center}
\begin{small}
\begin{sc}
\begin{tabular}{lcccc}
\toprule
Model & $\epsilon = 0.5$ & $\epsilon = 1$ & $\epsilon = 2$ &  $\epsilon = \infty$ \\
\midrule
SimpleFF & $0.207$ & $0.195$ & $0.193$ & $0.136$ \\ 
DeepAR & $\mathbf{0.157}$ & $\mathbf{0.145}$ & $\mathbf{0.142}$ & $\mathbf{0.124}$ \\
iTransf. & $0.211$ & $0.193$ & $0.188$ & $0.135$ \\
DLinear & $0.204$ & $0.192$ & $0.188$ & $0.140$ \\
\midrule
Seasonal   & $0.251$ & $0.251$ & $0.251$ & $0.251$\\
AutoETS   & $0.407$ & $0.407$ & $0.407$ & $0.407$\\
\bottomrule
\end{tabular}
\end{sc}
\end{small}
\end{center}
\vskip -0.1in
\end{table}

\textbf{Datasets, Models, and Metrics.}
We consider three standard benchmarks: \texttt{traffic}, \texttt{electricity}, and \texttt{solar\_10\_minutes} as used in~\cite{Lai2018modeling}.
We further consider four common architectures: 
A two-layer feed-forward neural network (``SimpleFeedForward''), a recurrent neural network (``DeepAR''~\cite{salinas2020deepar}),
an encoder-only transformer (``iTransformer''~\cite{liu2024itransformer}), and a refined feed-forward network proposed to compete with attention-based models (``DLinear''~\cite{zeng2023transformers}).
We let these architectures parameterize elementwise $t$-distributions to obtain probabilistic forecasts.
We measure the quality of these probabilistic forecasts using continuous ranked probability scores (CRPS), which we approximate via mean weighted quantile losses (details in~\cref{appendix:metrics}).
As a reference for what constitutes ``meaningful utility'', we compare against seasonal na\"{i}ve forecasting and exponential smoothing (``AutoETS'') without introducing any noise.
All experiments are repeated with $5$ random seeds.


\textbf{Event-Level Privacy.} \cref{table:1_event_training_traffic_main} shows CRPS of all models on the \texttt{traffic} test set 
when setting $\delta=10^{-7}$, and training on the training set until reaching a pre-specified $\epsilon$
with $1$-event-level privacy. For the other datasets and standard deviations, see~\cref{appendix:privacy_utility_tradeoff_event_level_privacy}.
The column $\epsilon=\infty$ indicates non-DP training.
As can be seen, models can retain much of their utility and outperform the baselines, even for $\epsilon \leq 1$ which is generally considered a small privacy budget~\cite{ponomareva2023dp}.
For instance, the average CRPS of DeepAR on the traffic dataset is $0.124$ with non-DP training and $0.157$ for $\epsilon=0.5$.
Note that, since all models are trained using  our tight privacy analysis,
which specific model performs best  on which specific dataset is orthogonal to our contribution. 

\textbf{Other results.}
In~\cref{appendix:privacy_utility_tradeoff_user_level_privacy} we additionally train with $w$-event and $w$-user privacy.
In~\cref{appendix:privacy_utility_tradeoff_label_privacy}, we demonstrate that label perturbations can offer an improved privacy--utility trade-off. 
All results confirm that our guarantees for DP-SGD adapted to forecasting are strong enough to enable provably private training while retaining utility.

\section{Related Work}

\textbf{Mathematical Reasoning.}
%
Recent advancements in LLMs' mathematical reasoning capabilities have been driven by Chain-of-Thought (CoT) prompting \cite{og-cot-prompt, scratchpadNye}, which significantly outperforms direct-answer approaches by generating intermediate step-by-step reasoning.
%
Building on CoT, various enhancements have emerged, including self-consistency, which replaces greedy decoding with sampling-based inference to select the most consistent solution \cite{wang2023selfconsistency}.
 %
Meanwhile, PoT and PaL \cite{pot, pal} improve reasoning by delegating computation to a Python interpreter, reducing the task of translating problems into code.
%

Another key advancement is instruction fine-tuning on mathematical datasets.
%
\citet{metamath} introduced MetaMathQA, expanding existing datasets through diverse rephrasings, while \citet{mammoth} leveraged a hybrid MathInstruct dataset combining CoT’s generality with PoT’s computational precision.
%
Additionally, external tool integration has been explored \cite{mario, tora}, with curated tool-use datasets enhancing LLMs’ reasoning capabilities.
%

\textbf{Multilingual Mathematical Reasoning.}
%
Despite LLMs' advancements in English mathematical reasoning, their performance in other languages still lags.
%
Efforts to bridge this gap include sample translation for multilingual alignment \cite{mathoctopus, mcot, qalign} and multilingual preference optimization \cite{mapo}.
%
\citet{mathoctopus} created a multilingual mathematical dataset by translating GSM8K into ten languages, though accurate translations remain a costly and time-consuming endeavor.
%
To mitigate this, \citet{qalign} proposed a two-step approach: translating questions into English before fine-tuning on larger English datasets like MetaMathQA.
%
Alternatively, \citet{mapo} leveraged existing translation models as alignment signals for preference optimization.
%

Beyond dataset translation, prompting techniques offer a cost-effective alternative.
%
\citet{not-all-lang-cross-lingual-cot-prompt} introduced role-playing prompts where the model first translates questions into English before applying CoT reasoning.
%
\citet{tot-multi-prompt} proposed a Tree-of-Thought framework for structured, multi-step reasoning across languages.
%
\citet{crosspal} extended PoT with Cross-PAL, aligning reasoning across multiple languages through code generation.


\section{Conclusion}
We present live monitoring and mid-run interventions for multi-agent systems. We demonstrate that monitors based on simple statistical measures can effectively predict future agent failures, and these failures can be prevented by restarting the communication channel. Experiments across multiple environments and models show consistent gains of up to 17.4\%-20\% in system performance, with an addition in inference-time compute.
Our work also introduces \ourenv{}, a new environment for studying multi-agent cooperation.

\section*{Acknowledgment}

We sincerely appreciate Haicheng Wu and Alex Fu from NVIDIA for their dedicated support in the Cutlass kernel.


\bibliography{example_paper}
\bibliographystyle{icml2025}

\newpage
\subsection{Lloyd-Max Algorithm}
\label{subsec:Lloyd-Max}
For a given quantization bitwidth $B$ and an operand $\bm{X}$, the Lloyd-Max algorithm finds $2^B$ quantization levels $\{\hat{x}_i\}_{i=1}^{2^B}$ such that quantizing $\bm{X}$ by rounding each scalar in $\bm{X}$ to the nearest quantization level minimizes the quantization MSE. 

The algorithm starts with an initial guess of quantization levels and then iteratively computes quantization thresholds $\{\tau_i\}_{i=1}^{2^B-1}$ and updates quantization levels $\{\hat{x}_i\}_{i=1}^{2^B}$. Specifically, at iteration $n$, thresholds are set to the midpoints of the previous iteration's levels:
\begin{align*}
    \tau_i^{(n)}=\frac{\hat{x}_i^{(n-1)}+\hat{x}_{i+1}^{(n-1)}}2 \text{ for } i=1\ldots 2^B-1
\end{align*}
Subsequently, the quantization levels are re-computed as conditional means of the data regions defined by the new thresholds:
\begin{align*}
    \hat{x}_i^{(n)}=\mathbb{E}\left[ \bm{X} \big| \bm{X}\in [\tau_{i-1}^{(n)},\tau_i^{(n)}] \right] \text{ for } i=1\ldots 2^B
\end{align*}
where to satisfy boundary conditions we have $\tau_0=-\infty$ and $\tau_{2^B}=\infty$. The algorithm iterates the above steps until convergence.

Figure \ref{fig:lm_quant} compares the quantization levels of a $7$-bit floating point (E3M3) quantizer (left) to a $7$-bit Lloyd-Max quantizer (right) when quantizing a layer of weights from the GPT3-126M model at a per-tensor granularity. As shown, the Lloyd-Max quantizer achieves substantially lower quantization MSE. Further, Table \ref{tab:FP7_vs_LM7} shows the superior perplexity achieved by Lloyd-Max quantizers for bitwidths of $7$, $6$ and $5$. The difference between the quantizers is clear at 5 bits, where per-tensor FP quantization incurs a drastic and unacceptable increase in perplexity, while Lloyd-Max quantization incurs a much smaller increase. Nevertheless, we note that even the optimal Lloyd-Max quantizer incurs a notable ($\sim 1.5$) increase in perplexity due to the coarse granularity of quantization. 

\begin{figure}[h]
  \centering
  \includegraphics[width=0.7\linewidth]{sections/figures/LM7_FP7.pdf}
  \caption{\small Quantization levels and the corresponding quantization MSE of Floating Point (left) vs Lloyd-Max (right) Quantizers for a layer of weights in the GPT3-126M model.}
  \label{fig:lm_quant}
\end{figure}

\begin{table}[h]\scriptsize
\begin{center}
\caption{\label{tab:FP7_vs_LM7} \small Comparing perplexity (lower is better) achieved by floating point quantizers and Lloyd-Max quantizers on a GPT3-126M model for the Wikitext-103 dataset.}
\begin{tabular}{c|cc|c}
\hline
 \multirow{2}{*}{\textbf{Bitwidth}} & \multicolumn{2}{|c|}{\textbf{Floating-Point Quantizer}} & \textbf{Lloyd-Max Quantizer} \\
 & Best Format & Wikitext-103 Perplexity & Wikitext-103 Perplexity \\
\hline
7 & E3M3 & 18.32 & 18.27 \\
6 & E3M2 & 19.07 & 18.51 \\
5 & E4M0 & 43.89 & 19.71 \\
\hline
\end{tabular}
\end{center}
\end{table}

\subsection{Proof of Local Optimality of LO-BCQ}
\label{subsec:lobcq_opt_proof}
For a given block $\bm{b}_j$, the quantization MSE during LO-BCQ can be empirically evaluated as $\frac{1}{L_b}\lVert \bm{b}_j- \bm{\hat{b}}_j\rVert^2_2$ where $\bm{\hat{b}}_j$ is computed from equation (\ref{eq:clustered_quantization_definition}) as $C_{f(\bm{b}_j)}(\bm{b}_j)$. Further, for a given block cluster $\mathcal{B}_i$, we compute the quantization MSE as $\frac{1}{|\mathcal{B}_{i}|}\sum_{\bm{b} \in \mathcal{B}_{i}} \frac{1}{L_b}\lVert \bm{b}- C_i^{(n)}(\bm{b})\rVert^2_2$. Therefore, at the end of iteration $n$, we evaluate the overall quantization MSE $J^{(n)}$ for a given operand $\bm{X}$ composed of $N_c$ block clusters as:
\begin{align*}
    \label{eq:mse_iter_n}
    J^{(n)} = \frac{1}{N_c} \sum_{i=1}^{N_c} \frac{1}{|\mathcal{B}_{i}^{(n)}|}\sum_{\bm{v} \in \mathcal{B}_{i}^{(n)}} \frac{1}{L_b}\lVert \bm{b}- B_i^{(n)}(\bm{b})\rVert^2_2
\end{align*}

At the end of iteration $n$, the codebooks are updated from $\mathcal{C}^{(n-1)}$ to $\mathcal{C}^{(n)}$. However, the mapping of a given vector $\bm{b}_j$ to quantizers $\mathcal{C}^{(n)}$ remains as  $f^{(n)}(\bm{b}_j)$. At the next iteration, during the vector clustering step, $f^{(n+1)}(\bm{b}_j)$ finds new mapping of $\bm{b}_j$ to updated codebooks $\mathcal{C}^{(n)}$ such that the quantization MSE over the candidate codebooks is minimized. Therefore, we obtain the following result for $\bm{b}_j$:
\begin{align*}
\frac{1}{L_b}\lVert \bm{b}_j - C_{f^{(n+1)}(\bm{b}_j)}^{(n)}(\bm{b}_j)\rVert^2_2 \le \frac{1}{L_b}\lVert \bm{b}_j - C_{f^{(n)}(\bm{b}_j)}^{(n)}(\bm{b}_j)\rVert^2_2
\end{align*}

That is, quantizing $\bm{b}_j$ at the end of the block clustering step of iteration $n+1$ results in lower quantization MSE compared to quantizing at the end of iteration $n$. Since this is true for all $\bm{b} \in \bm{X}$, we assert the following:
\begin{equation}
\begin{split}
\label{eq:mse_ineq_1}
    \tilde{J}^{(n+1)} &= \frac{1}{N_c} \sum_{i=1}^{N_c} \frac{1}{|\mathcal{B}_{i}^{(n+1)}|}\sum_{\bm{b} \in \mathcal{B}_{i}^{(n+1)}} \frac{1}{L_b}\lVert \bm{b} - C_i^{(n)}(b)\rVert^2_2 \le J^{(n)}
\end{split}
\end{equation}
where $\tilde{J}^{(n+1)}$ is the the quantization MSE after the vector clustering step at iteration $n+1$.

Next, during the codebook update step (\ref{eq:quantizers_update}) at iteration $n+1$, the per-cluster codebooks $\mathcal{C}^{(n)}$ are updated to $\mathcal{C}^{(n+1)}$ by invoking the Lloyd-Max algorithm \citep{Lloyd}. We know that for any given value distribution, the Lloyd-Max algorithm minimizes the quantization MSE. Therefore, for a given vector cluster $\mathcal{B}_i$ we obtain the following result:

\begin{equation}
    \frac{1}{|\mathcal{B}_{i}^{(n+1)}|}\sum_{\bm{b} \in \mathcal{B}_{i}^{(n+1)}} \frac{1}{L_b}\lVert \bm{b}- C_i^{(n+1)}(\bm{b})\rVert^2_2 \le \frac{1}{|\mathcal{B}_{i}^{(n+1)}|}\sum_{\bm{b} \in \mathcal{B}_{i}^{(n+1)}} \frac{1}{L_b}\lVert \bm{b}- C_i^{(n)}(\bm{b})\rVert^2_2
\end{equation}

The above equation states that quantizing the given block cluster $\mathcal{B}_i$ after updating the associated codebook from $C_i^{(n)}$ to $C_i^{(n+1)}$ results in lower quantization MSE. Since this is true for all the block clusters, we derive the following result: 
\begin{equation}
\begin{split}
\label{eq:mse_ineq_2}
     J^{(n+1)} &= \frac{1}{N_c} \sum_{i=1}^{N_c} \frac{1}{|\mathcal{B}_{i}^{(n+1)}|}\sum_{\bm{b} \in \mathcal{B}_{i}^{(n+1)}} \frac{1}{L_b}\lVert \bm{b}- C_i^{(n+1)}(\bm{b})\rVert^2_2  \le \tilde{J}^{(n+1)}   
\end{split}
\end{equation}

Following (\ref{eq:mse_ineq_1}) and (\ref{eq:mse_ineq_2}), we find that the quantization MSE is non-increasing for each iteration, that is, $J^{(1)} \ge J^{(2)} \ge J^{(3)} \ge \ldots \ge J^{(M)}$ where $M$ is the maximum number of iterations. 
%Therefore, we can say that if the algorithm converges, then it must be that it has converged to a local minimum. 
\hfill $\blacksquare$


\begin{figure}
    \begin{center}
    \includegraphics[width=0.5\textwidth]{sections//figures/mse_vs_iter.pdf}
    \end{center}
    \caption{\small NMSE vs iterations during LO-BCQ compared to other block quantization proposals}
    \label{fig:nmse_vs_iter}
\end{figure}

Figure \ref{fig:nmse_vs_iter} shows the empirical convergence of LO-BCQ across several block lengths and number of codebooks. Also, the MSE achieved by LO-BCQ is compared to baselines such as MXFP and VSQ. As shown, LO-BCQ converges to a lower MSE than the baselines. Further, we achieve better convergence for larger number of codebooks ($N_c$) and for a smaller block length ($L_b$), both of which increase the bitwidth of BCQ (see Eq \ref{eq:bitwidth_bcq}).


\subsection{Additional Accuracy Results}
%Table \ref{tab:lobcq_config} lists the various LOBCQ configurations and their corresponding bitwidths.
\begin{table}
\setlength{\tabcolsep}{4.75pt}
\begin{center}
\caption{\label{tab:lobcq_config} Various LO-BCQ configurations and their bitwidths.}
\begin{tabular}{|c||c|c|c|c||c|c||c|} 
\hline
 & \multicolumn{4}{|c||}{$L_b=8$} & \multicolumn{2}{|c||}{$L_b=4$} & $L_b=2$ \\
 \hline
 \backslashbox{$L_A$\kern-1em}{\kern-1em$N_c$} & 2 & 4 & 8 & 16 & 2 & 4 & 2 \\
 \hline
 64 & 4.25 & 4.375 & 4.5 & 4.625 & 4.375 & 4.625 & 4.625\\
 \hline
 32 & 4.375 & 4.5 & 4.625& 4.75 & 4.5 & 4.75 & 4.75 \\
 \hline
 16 & 4.625 & 4.75& 4.875 & 5 & 4.75 & 5 & 5 \\
 \hline
\end{tabular}
\end{center}
\end{table}

%\subsection{Perplexity achieved by various LO-BCQ configurations on Wikitext-103 dataset}

\begin{table} \centering
\begin{tabular}{|c||c|c|c|c||c|c||c|} 
\hline
 $L_b \rightarrow$& \multicolumn{4}{c||}{8} & \multicolumn{2}{c||}{4} & 2\\
 \hline
 \backslashbox{$L_A$\kern-1em}{\kern-1em$N_c$} & 2 & 4 & 8 & 16 & 2 & 4 & 2  \\
 %$N_c \rightarrow$ & 2 & 4 & 8 & 16 & 2 & 4 & 2 \\
 \hline
 \hline
 \multicolumn{8}{c}{GPT3-1.3B (FP32 PPL = 9.98)} \\ 
 \hline
 \hline
 64 & 10.40 & 10.23 & 10.17 & 10.15 &  10.28 & 10.18 & 10.19 \\
 \hline
 32 & 10.25 & 10.20 & 10.15 & 10.12 &  10.23 & 10.17 & 10.17 \\
 \hline
 16 & 10.22 & 10.16 & 10.10 & 10.09 &  10.21 & 10.14 & 10.16 \\
 \hline
  \hline
 \multicolumn{8}{c}{GPT3-8B (FP32 PPL = 7.38)} \\ 
 \hline
 \hline
 64 & 7.61 & 7.52 & 7.48 &  7.47 &  7.55 &  7.49 & 7.50 \\
 \hline
 32 & 7.52 & 7.50 & 7.46 &  7.45 &  7.52 &  7.48 & 7.48  \\
 \hline
 16 & 7.51 & 7.48 & 7.44 &  7.44 &  7.51 &  7.49 & 7.47  \\
 \hline
\end{tabular}
\caption{\label{tab:ppl_gpt3_abalation} Wikitext-103 perplexity across GPT3-1.3B and 8B models.}
\end{table}

\begin{table} \centering
\begin{tabular}{|c||c|c|c|c||} 
\hline
 $L_b \rightarrow$& \multicolumn{4}{c||}{8}\\
 \hline
 \backslashbox{$L_A$\kern-1em}{\kern-1em$N_c$} & 2 & 4 & 8 & 16 \\
 %$N_c \rightarrow$ & 2 & 4 & 8 & 16 & 2 & 4 & 2 \\
 \hline
 \hline
 \multicolumn{5}{|c|}{Llama2-7B (FP32 PPL = 5.06)} \\ 
 \hline
 \hline
 64 & 5.31 & 5.26 & 5.19 & 5.18  \\
 \hline
 32 & 5.23 & 5.25 & 5.18 & 5.15  \\
 \hline
 16 & 5.23 & 5.19 & 5.16 & 5.14  \\
 \hline
 \multicolumn{5}{|c|}{Nemotron4-15B (FP32 PPL = 5.87)} \\ 
 \hline
 \hline
 64  & 6.3 & 6.20 & 6.13 & 6.08  \\
 \hline
 32  & 6.24 & 6.12 & 6.07 & 6.03  \\
 \hline
 16  & 6.12 & 6.14 & 6.04 & 6.02  \\
 \hline
 \multicolumn{5}{|c|}{Nemotron4-340B (FP32 PPL = 3.48)} \\ 
 \hline
 \hline
 64 & 3.67 & 3.62 & 3.60 & 3.59 \\
 \hline
 32 & 3.63 & 3.61 & 3.59 & 3.56 \\
 \hline
 16 & 3.61 & 3.58 & 3.57 & 3.55 \\
 \hline
\end{tabular}
\caption{\label{tab:ppl_llama7B_nemo15B} Wikitext-103 perplexity compared to FP32 baseline in Llama2-7B and Nemotron4-15B, 340B models}
\end{table}

%\subsection{Perplexity achieved by various LO-BCQ configurations on MMLU dataset}


\begin{table} \centering
\begin{tabular}{|c||c|c|c|c||c|c|c|c|} 
\hline
 $L_b \rightarrow$& \multicolumn{4}{c||}{8} & \multicolumn{4}{c||}{8}\\
 \hline
 \backslashbox{$L_A$\kern-1em}{\kern-1em$N_c$} & 2 & 4 & 8 & 16 & 2 & 4 & 8 & 16  \\
 %$N_c \rightarrow$ & 2 & 4 & 8 & 16 & 2 & 4 & 2 \\
 \hline
 \hline
 \multicolumn{5}{|c|}{Llama2-7B (FP32 Accuracy = 45.8\%)} & \multicolumn{4}{|c|}{Llama2-70B (FP32 Accuracy = 69.12\%)} \\ 
 \hline
 \hline
 64 & 43.9 & 43.4 & 43.9 & 44.9 & 68.07 & 68.27 & 68.17 & 68.75 \\
 \hline
 32 & 44.5 & 43.8 & 44.9 & 44.5 & 68.37 & 68.51 & 68.35 & 68.27  \\
 \hline
 16 & 43.9 & 42.7 & 44.9 & 45 & 68.12 & 68.77 & 68.31 & 68.59  \\
 \hline
 \hline
 \multicolumn{5}{|c|}{GPT3-22B (FP32 Accuracy = 38.75\%)} & \multicolumn{4}{|c|}{Nemotron4-15B (FP32 Accuracy = 64.3\%)} \\ 
 \hline
 \hline
 64 & 36.71 & 38.85 & 38.13 & 38.92 & 63.17 & 62.36 & 63.72 & 64.09 \\
 \hline
 32 & 37.95 & 38.69 & 39.45 & 38.34 & 64.05 & 62.30 & 63.8 & 64.33  \\
 \hline
 16 & 38.88 & 38.80 & 38.31 & 38.92 & 63.22 & 63.51 & 63.93 & 64.43  \\
 \hline
\end{tabular}
\caption{\label{tab:mmlu_abalation} Accuracy on MMLU dataset across GPT3-22B, Llama2-7B, 70B and Nemotron4-15B models.}
\end{table}


%\subsection{Perplexity achieved by various LO-BCQ configurations on LM evaluation harness}

\begin{table} \centering
\begin{tabular}{|c||c|c|c|c||c|c|c|c|} 
\hline
 $L_b \rightarrow$& \multicolumn{4}{c||}{8} & \multicolumn{4}{c||}{8}\\
 \hline
 \backslashbox{$L_A$\kern-1em}{\kern-1em$N_c$} & 2 & 4 & 8 & 16 & 2 & 4 & 8 & 16  \\
 %$N_c \rightarrow$ & 2 & 4 & 8 & 16 & 2 & 4 & 2 \\
 \hline
 \hline
 \multicolumn{5}{|c|}{Race (FP32 Accuracy = 37.51\%)} & \multicolumn{4}{|c|}{Boolq (FP32 Accuracy = 64.62\%)} \\ 
 \hline
 \hline
 64 & 36.94 & 37.13 & 36.27 & 37.13 & 63.73 & 62.26 & 63.49 & 63.36 \\
 \hline
 32 & 37.03 & 36.36 & 36.08 & 37.03 & 62.54 & 63.51 & 63.49 & 63.55  \\
 \hline
 16 & 37.03 & 37.03 & 36.46 & 37.03 & 61.1 & 63.79 & 63.58 & 63.33  \\
 \hline
 \hline
 \multicolumn{5}{|c|}{Winogrande (FP32 Accuracy = 58.01\%)} & \multicolumn{4}{|c|}{Piqa (FP32 Accuracy = 74.21\%)} \\ 
 \hline
 \hline
 64 & 58.17 & 57.22 & 57.85 & 58.33 & 73.01 & 73.07 & 73.07 & 72.80 \\
 \hline
 32 & 59.12 & 58.09 & 57.85 & 58.41 & 73.01 & 73.94 & 72.74 & 73.18  \\
 \hline
 16 & 57.93 & 58.88 & 57.93 & 58.56 & 73.94 & 72.80 & 73.01 & 73.94  \\
 \hline
\end{tabular}
\caption{\label{tab:mmlu_abalation} Accuracy on LM evaluation harness tasks on GPT3-1.3B model.}
\end{table}

\begin{table} \centering
\begin{tabular}{|c||c|c|c|c||c|c|c|c|} 
\hline
 $L_b \rightarrow$& \multicolumn{4}{c||}{8} & \multicolumn{4}{c||}{8}\\
 \hline
 \backslashbox{$L_A$\kern-1em}{\kern-1em$N_c$} & 2 & 4 & 8 & 16 & 2 & 4 & 8 & 16  \\
 %$N_c \rightarrow$ & 2 & 4 & 8 & 16 & 2 & 4 & 2 \\
 \hline
 \hline
 \multicolumn{5}{|c|}{Race (FP32 Accuracy = 41.34\%)} & \multicolumn{4}{|c|}{Boolq (FP32 Accuracy = 68.32\%)} \\ 
 \hline
 \hline
 64 & 40.48 & 40.10 & 39.43 & 39.90 & 69.20 & 68.41 & 69.45 & 68.56 \\
 \hline
 32 & 39.52 & 39.52 & 40.77 & 39.62 & 68.32 & 67.43 & 68.17 & 69.30  \\
 \hline
 16 & 39.81 & 39.71 & 39.90 & 40.38 & 68.10 & 66.33 & 69.51 & 69.42  \\
 \hline
 \hline
 \multicolumn{5}{|c|}{Winogrande (FP32 Accuracy = 67.88\%)} & \multicolumn{4}{|c|}{Piqa (FP32 Accuracy = 78.78\%)} \\ 
 \hline
 \hline
 64 & 66.85 & 66.61 & 67.72 & 67.88 & 77.31 & 77.42 & 77.75 & 77.64 \\
 \hline
 32 & 67.25 & 67.72 & 67.72 & 67.00 & 77.31 & 77.04 & 77.80 & 77.37  \\
 \hline
 16 & 68.11 & 68.90 & 67.88 & 67.48 & 77.37 & 78.13 & 78.13 & 77.69  \\
 \hline
\end{tabular}
\caption{\label{tab:mmlu_abalation} Accuracy on LM evaluation harness tasks on GPT3-8B model.}
\end{table}

\begin{table} \centering
\begin{tabular}{|c||c|c|c|c||c|c|c|c|} 
\hline
 $L_b \rightarrow$& \multicolumn{4}{c||}{8} & \multicolumn{4}{c||}{8}\\
 \hline
 \backslashbox{$L_A$\kern-1em}{\kern-1em$N_c$} & 2 & 4 & 8 & 16 & 2 & 4 & 8 & 16  \\
 %$N_c \rightarrow$ & 2 & 4 & 8 & 16 & 2 & 4 & 2 \\
 \hline
 \hline
 \multicolumn{5}{|c|}{Race (FP32 Accuracy = 40.67\%)} & \multicolumn{4}{|c|}{Boolq (FP32 Accuracy = 76.54\%)} \\ 
 \hline
 \hline
 64 & 40.48 & 40.10 & 39.43 & 39.90 & 75.41 & 75.11 & 77.09 & 75.66 \\
 \hline
 32 & 39.52 & 39.52 & 40.77 & 39.62 & 76.02 & 76.02 & 75.96 & 75.35  \\
 \hline
 16 & 39.81 & 39.71 & 39.90 & 40.38 & 75.05 & 73.82 & 75.72 & 76.09  \\
 \hline
 \hline
 \multicolumn{5}{|c|}{Winogrande (FP32 Accuracy = 70.64\%)} & \multicolumn{4}{|c|}{Piqa (FP32 Accuracy = 79.16\%)} \\ 
 \hline
 \hline
 64 & 69.14 & 70.17 & 70.17 & 70.56 & 78.24 & 79.00 & 78.62 & 78.73 \\
 \hline
 32 & 70.96 & 69.69 & 71.27 & 69.30 & 78.56 & 79.49 & 79.16 & 78.89  \\
 \hline
 16 & 71.03 & 69.53 & 69.69 & 70.40 & 78.13 & 79.16 & 79.00 & 79.00  \\
 \hline
\end{tabular}
\caption{\label{tab:mmlu_abalation} Accuracy on LM evaluation harness tasks on GPT3-22B model.}
\end{table}

\begin{table} \centering
\begin{tabular}{|c||c|c|c|c||c|c|c|c|} 
\hline
 $L_b \rightarrow$& \multicolumn{4}{c||}{8} & \multicolumn{4}{c||}{8}\\
 \hline
 \backslashbox{$L_A$\kern-1em}{\kern-1em$N_c$} & 2 & 4 & 8 & 16 & 2 & 4 & 8 & 16  \\
 %$N_c \rightarrow$ & 2 & 4 & 8 & 16 & 2 & 4 & 2 \\
 \hline
 \hline
 \multicolumn{5}{|c|}{Race (FP32 Accuracy = 44.4\%)} & \multicolumn{4}{|c|}{Boolq (FP32 Accuracy = 79.29\%)} \\ 
 \hline
 \hline
 64 & 42.49 & 42.51 & 42.58 & 43.45 & 77.58 & 77.37 & 77.43 & 78.1 \\
 \hline
 32 & 43.35 & 42.49 & 43.64 & 43.73 & 77.86 & 75.32 & 77.28 & 77.86  \\
 \hline
 16 & 44.21 & 44.21 & 43.64 & 42.97 & 78.65 & 77 & 76.94 & 77.98  \\
 \hline
 \hline
 \multicolumn{5}{|c|}{Winogrande (FP32 Accuracy = 69.38\%)} & \multicolumn{4}{|c|}{Piqa (FP32 Accuracy = 78.07\%)} \\ 
 \hline
 \hline
 64 & 68.9 & 68.43 & 69.77 & 68.19 & 77.09 & 76.82 & 77.09 & 77.86 \\
 \hline
 32 & 69.38 & 68.51 & 68.82 & 68.90 & 78.07 & 76.71 & 78.07 & 77.86  \\
 \hline
 16 & 69.53 & 67.09 & 69.38 & 68.90 & 77.37 & 77.8 & 77.91 & 77.69  \\
 \hline
\end{tabular}
\caption{\label{tab:mmlu_abalation} Accuracy on LM evaluation harness tasks on Llama2-7B model.}
\end{table}

\begin{table} \centering
\begin{tabular}{|c||c|c|c|c||c|c|c|c|} 
\hline
 $L_b \rightarrow$& \multicolumn{4}{c||}{8} & \multicolumn{4}{c||}{8}\\
 \hline
 \backslashbox{$L_A$\kern-1em}{\kern-1em$N_c$} & 2 & 4 & 8 & 16 & 2 & 4 & 8 & 16  \\
 %$N_c \rightarrow$ & 2 & 4 & 8 & 16 & 2 & 4 & 2 \\
 \hline
 \hline
 \multicolumn{5}{|c|}{Race (FP32 Accuracy = 48.8\%)} & \multicolumn{4}{|c|}{Boolq (FP32 Accuracy = 85.23\%)} \\ 
 \hline
 \hline
 64 & 49.00 & 49.00 & 49.28 & 48.71 & 82.82 & 84.28 & 84.03 & 84.25 \\
 \hline
 32 & 49.57 & 48.52 & 48.33 & 49.28 & 83.85 & 84.46 & 84.31 & 84.93  \\
 \hline
 16 & 49.85 & 49.09 & 49.28 & 48.99 & 85.11 & 84.46 & 84.61 & 83.94  \\
 \hline
 \hline
 \multicolumn{5}{|c|}{Winogrande (FP32 Accuracy = 79.95\%)} & \multicolumn{4}{|c|}{Piqa (FP32 Accuracy = 81.56\%)} \\ 
 \hline
 \hline
 64 & 78.77 & 78.45 & 78.37 & 79.16 & 81.45 & 80.69 & 81.45 & 81.5 \\
 \hline
 32 & 78.45 & 79.01 & 78.69 & 80.66 & 81.56 & 80.58 & 81.18 & 81.34  \\
 \hline
 16 & 79.95 & 79.56 & 79.79 & 79.72 & 81.28 & 81.66 & 81.28 & 80.96  \\
 \hline
\end{tabular}
\caption{\label{tab:mmlu_abalation} Accuracy on LM evaluation harness tasks on Llama2-70B model.}
\end{table}

%\section{MSE Studies}
%\textcolor{red}{TODO}


\subsection{Number Formats and Quantization Method}
\label{subsec:numFormats_quantMethod}
\subsubsection{Integer Format}
An $n$-bit signed integer (INT) is typically represented with a 2s-complement format \citep{yao2022zeroquant,xiao2023smoothquant,dai2021vsq}, where the most significant bit denotes the sign.

\subsubsection{Floating Point Format}
An $n$-bit signed floating point (FP) number $x$ comprises of a 1-bit sign ($x_{\mathrm{sign}}$), $B_m$-bit mantissa ($x_{\mathrm{mant}}$) and $B_e$-bit exponent ($x_{\mathrm{exp}}$) such that $B_m+B_e=n-1$. The associated constant exponent bias ($E_{\mathrm{bias}}$) is computed as $(2^{{B_e}-1}-1)$. We denote this format as $E_{B_e}M_{B_m}$.  

\subsubsection{Quantization Scheme}
\label{subsec:quant_method}
A quantization scheme dictates how a given unquantized tensor is converted to its quantized representation. We consider FP formats for the purpose of illustration. Given an unquantized tensor $\bm{X}$ and an FP format $E_{B_e}M_{B_m}$, we first, we compute the quantization scale factor $s_X$ that maps the maximum absolute value of $\bm{X}$ to the maximum quantization level of the $E_{B_e}M_{B_m}$ format as follows:
\begin{align}
\label{eq:sf}
    s_X = \frac{\mathrm{max}(|\bm{X}|)}{\mathrm{max}(E_{B_e}M_{B_m})}
\end{align}
In the above equation, $|\cdot|$ denotes the absolute value function.

Next, we scale $\bm{X}$ by $s_X$ and quantize it to $\hat{\bm{X}}$ by rounding it to the nearest quantization level of $E_{B_e}M_{B_m}$ as:

\begin{align}
\label{eq:tensor_quant}
    \hat{\bm{X}} = \text{round-to-nearest}\left(\frac{\bm{X}}{s_X}, E_{B_e}M_{B_m}\right)
\end{align}

We perform dynamic max-scaled quantization \citep{wu2020integer}, where the scale factor $s$ for activations is dynamically computed during runtime.

\subsection{Vector Scaled Quantization}
\begin{wrapfigure}{r}{0.35\linewidth}
  \centering
  \includegraphics[width=\linewidth]{sections/figures/vsquant.jpg}
  \caption{\small Vectorwise decomposition for per-vector scaled quantization (VSQ \citep{dai2021vsq}).}
  \label{fig:vsquant}
\end{wrapfigure}
During VSQ \citep{dai2021vsq}, the operand tensors are decomposed into 1D vectors in a hardware friendly manner as shown in Figure \ref{fig:vsquant}. Since the decomposed tensors are used as operands in matrix multiplications during inference, it is beneficial to perform this decomposition along the reduction dimension of the multiplication. The vectorwise quantization is performed similar to tensorwise quantization described in Equations \ref{eq:sf} and \ref{eq:tensor_quant}, where a scale factor $s_v$ is required for each vector $\bm{v}$ that maps the maximum absolute value of that vector to the maximum quantization level. While smaller vector lengths can lead to larger accuracy gains, the associated memory and computational overheads due to the per-vector scale factors increases. To alleviate these overheads, VSQ \citep{dai2021vsq} proposed a second level quantization of the per-vector scale factors to unsigned integers, while MX \citep{rouhani2023shared} quantizes them to integer powers of 2 (denoted as $2^{INT}$).

\subsubsection{MX Format}
The MX format proposed in \citep{rouhani2023microscaling} introduces the concept of sub-block shifting. For every two scalar elements of $b$-bits each, there is a shared exponent bit. The value of this exponent bit is determined through an empirical analysis that targets minimizing quantization MSE. We note that the FP format $E_{1}M_{b}$ is strictly better than MX from an accuracy perspective since it allocates a dedicated exponent bit to each scalar as opposed to sharing it across two scalars. Therefore, we conservatively bound the accuracy of a $b+2$-bit signed MX format with that of a $E_{1}M_{b}$ format in our comparisons. For instance, we use E1M2 format as a proxy for MX4.

\begin{figure}
    \centering
    \includegraphics[width=1\linewidth]{sections//figures/BlockFormats.pdf}
    \caption{\small Comparing LO-BCQ to MX format.}
    \label{fig:block_formats}
\end{figure}

Figure \ref{fig:block_formats} compares our $4$-bit LO-BCQ block format to MX \citep{rouhani2023microscaling}. As shown, both LO-BCQ and MX decompose a given operand tensor into block arrays and each block array into blocks. Similar to MX, we find that per-block quantization ($L_b < L_A$) leads to better accuracy due to increased flexibility. While MX achieves this through per-block $1$-bit micro-scales, we associate a dedicated codebook to each block through a per-block codebook selector. Further, MX quantizes the per-block array scale-factor to E8M0 format without per-tensor scaling. In contrast during LO-BCQ, we find that per-tensor scaling combined with quantization of per-block array scale-factor to E4M3 format results in superior inference accuracy across models. 

\end{document}
