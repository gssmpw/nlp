\appendix
\onecolumn
\section{Appendix / supplemental material}

\subsection{Complete Results of Figure~\ref{fig:52_main_result_ternary}}
Table~\ref{tab:appendix_ternary_table} presents the numerical results of Figure~\ref{fig:52_main_result_ternary}. We evaluate accuracy across eight zero-shot commonsense reasoning tasks: ARC-easy, ARC-challenge~\citep{clark2018arc}, BoolQ~\citep{clark2019boolq}, PIQA~\citep{bisk2020piqa}, SIQA~\citep{sap2019siqa}, HellaSwag~\citep{zellers2019hellaswag}, OBQA~\citep{mihaylov2018obqa}, and WinoGrande~\citep{sakaguchi2021winogrande}, along with perplexity on the WikiText2 test set~\citep{merity2016wiki2}. Our results are compared against prior state-of-the-art ternary quantization-aware training works, including 1-bit era~\cite{ma2024era} and Spectra~\cite{spectra}. We also include the comparison to LLM-QAT~\citep{liu2023llmqat}. Consistent with previous methodologies~\cite{ma2024era,spectra}, we quantize all weights to low-bit, excluding the embedding and output layers. The $\ours{}$ 3B ternary model is quantized from LLaMA3~\cite{llama3modelcard} 3B model, while other models are quantized from MobileLLM~\cite{liu2024mobilellm}. 
As Spectra did not report results on the SIQA and OBQA datasets, the values in Figure~\ref{fig:52_main_result_ternary} represent the average accuracy across the remaining six tasks.

\begin{table}[h]
\renewcommand\arraystretch{0.6}
\centering
\caption{\small Comparison of $\ours{}$ with QAT methods, including general LLM-QAT~\cite{liu2023llmqat} and ternary-specific QAT methods such as Spectra~\cite{spectra} and 1-bit Era~\cite{ma2024era}.}
\vspace{-30pt}
\label{tab:appendix_ternary_table}
\setlength{\tabcolsep}{1mm}
\resizebox{0.8\textwidth}{!}{
\begin{tabular}{c|c|cccccccc|c}
& & & & & & & & & &  \\
& & & & & & & & & &  \\
& & & & & & & & & &  \\
& & & & & & & & & &  \\
\hline\hline
\multirow{2}{*}{Method} & \multirow{2}{*}{\# Params} & ARC-e & ARC-c & BoolQ & PIQA & SIQA & HellaSwag & OBQA & WinoGrande & Wiki2 \\ 
 &  & ($\uparrow$)  & ($\uparrow$) & ($\uparrow$) & ($\uparrow$) & ($\uparrow$) & ($\uparrow$) & ($\uparrow$) & ($\uparrow$) & ($\downarrow$) \\ 
\midrule
RTN & 125M & 25.5 & 26.5 & 37.8 & 49.6 & 36.3 & 26.3 & 27.7 & 49.3 & 4.0e5 \\
LLM-QAT & 125M & 34.9 & 20.4 & 59.0 & 54.6 & 39.0 & 29.1 & 30.2 & 50.9 & 87.3 \\
Spectra & 190M & 43.9 & 19.5 & 46.8 & \textbf{62.3} & -- & 29.0 & -- & 51.7 & -- \\
\rowcolor{gray!20}  $\ours{}$ & 125M & \textbf{49.3} &\textbf{ 30.9} & \textbf{61.0} & 62.1 & \textbf{41.0} &\textbf{ 34.3} & \textbf{40.4 }& \textbf{52.9} & \textbf{19.8} \\
\midrule
RTN & 350M & 26.6 & 25.1 & 37.8 & 48.7 & 36.7 & 26.5 & 27.5 & 50.2 & 3.7e5 \\
LLM-QAT & 350M & 39.1 & 24.1 & 61.6 & 55.5 & 39.9 & 30.4 & 32.1 & 50.6 & 68.6 \\
Spectra & 390M & 48.6 & 21.2 & 55.1 & 65.0 & -- & 32.0 & -- & 52.2 & -- \\
\rowcolor{gray!20}  $\ours{}$ & 350M & \textbf{56.8} & \textbf{36.3} & \textbf{62.2 }& \textbf{67.1} &\textbf{43.5} & \textbf{44.0} & \textbf{46.3} & \textbf{55.2} & \textbf{14.4} \\
\midrule
RTN & 600M & 26.2 & 24.6 & 62.2 & 49.5 & 36.3 & 26.1 & 27.1 & 48.8 & 6.6e5 \\
LLM-QAT & 600M & 34.0 & 23.0 & 59.4 & 53.6 & 38.9 & 28.7 & 32.3 & 51.4 & 71.7 \\
1-bit era & 700M & 49.5 & 29.0 & 59.2 & 67.5 & 43.6 & 43.2 & 38.9 & 53.5 & 17.3 \\
Spectra & 560M & 50.2 & 21.0 & 57.3 & 67.5 & -- & 33.8 & -- & 53.1 & -- \\
\rowcolor{gray!20}  $\ours{}$ & 600M & \textbf{65.5} & \textbf{43.8} & \textbf{62.3} & \textbf{70.6} & \textbf{44.7} & \textbf{51.3} & \textbf{47.1} & \textbf{58.8 }& \textbf{11.4} \\
\midrule
RTN & 1B & 25.7 & 24.8 & 37.8 & 49.3 & 37.1 & 26.2 & 25.2 & 50.2 & 1.4e5 \\
LLM-QAT & 1B & 36.0 & 26.2 & 47.7 & 55.1 & 39.7 & 31.3 & 33.5 & 49.6 & 56.9 \\
1-bit era & 1.3B & 52.4 & 34.1 & 61.9 & 69.1 & 44.7 & 47.4 & 41.1 & 55.3 & 23.6 \\
Spectra & 1.1B & 56.3 & 24.6 & 59.1 & 69.3 & -- & 38.8 & -- & 55.5 & -- \\
\rowcolor{gray!20}  $\ours{}$ & 1B & \textbf{68.5} & \textbf{47.6 }& \textbf{62.8} & \textbf{72.1} & \textbf{45.3} & \textbf{57.4} & \textbf{52.9} & \textbf{61.3} & \textbf{10.0} \\
\midrule
RTN & 1.5B & 25.5 & 26.8 & 37.8 & 49.0 & 37.6 & 26.0 & 30.5 & 50.2 & 9.7e4 \\
LLM-QAT & 1.5B & 41.1 & 26.1 & 60.5 & 57.6 & 39.5 & 35.0 & 31.9 & 49.8 & 39.7 \\
Spectra & 1.5B & 59.0 & 24.7 & 54.1 & 70.3 & -- & 40.9 & -- & 56.1 & -- \\
\rowcolor{gray!20}  $\ours{}$ & 1.5B & \textbf{70.2 }& \textbf{48.0} &\textbf{ 65.8} & \textbf{73.4} & \textbf{47.3} & \textbf{61.8} & \textbf{55.3} & \textbf{62.4} & \textbf{9.0} \\
\midrule
RTN & 3B & 26.9 & 23.6 & 62.2 & 51.3 & 37.6 & 26.4 & 27.0 & 49.3 & 4.4e5 \\
LLM-QAT & 3B & 44.5 & 30.7 & 62.1 & 62.7 & 41.0 & 43.4 & 35.0 & 50.6 & 6.5e2 \\
1-bit era & 3B & 58.7 & 37.2 & 61.3 & 71.3 & 45.2 & 56.0 & 45.8 & 60.3 & 265.6 \\
Spectra & 3.9B & 66.0 & 31.9 & 66.5 & 74.4 & -- & 48.3 & -- & 62.1 & -- \\
\rowcolor{gray!20}  $\ours{}$ & 3B & \textbf{71.5} &\textbf{ 48.6} & \textbf{68.2} & \textbf{75.5 }& \textbf{46.4} & \textbf{67.9} & \textbf{54.3} & \textbf{63.1} & \textbf{9.9} \\
\hline\hline
\end{tabular}}
\end{table}

\subsection{Complete Results of Figure~\ref{fig:53_main_result_234bit}}
In Tables~\ref{tab:appendix_w2}, ~\ref{tab:appendix_w3}, and ~\ref{tab:appendix_w4}, we provide detailed results corresponding to Figure~\ref{fig:53_main_result_234bit}. We compare $\ours{}$ against LLM-QAT~\citep{liu2023llmqat}, GPTQ~\citep{frantar2022gptq}, AWQ~\citep{lin2023awq}, OmniQuant~\citep{shao2023omniquant}, and SpinQuant~\citep{liu2024spinquant}. Following the common practice~\cite{frantar2022gptq,liu2023llmqat}, we apply low-bit quantization to all weights, except for the embedding and output layers.
\begin{table}[h]
\renewcommand\arraystretch{0.6}
\centering
\caption{Complete results of \textbf{2-bit quantization} on WikiText2 and Zero-shot Common Sense Reasoning tasks.}
\vspace{-45pt}
\label{tab:appendix_w2}
\setlength{\tabcolsep}{1mm}
\resizebox{0.9\textwidth}{!}{%
\begin{tabular}{c|c|ccccccccc|c}
& & & & & & & & & & & \\
& & & & & & & & & & & \\
& & & & & & & & & & & \\
& & & & & & & & & & & \\
& & & & & & & & & & & \\
& & & & & & & & & & & \\
\hline\hline
\multirow{2}{*}{Model Name} & \multirow{2}{*}{Method} & ARC-e & ARC-c & BoolQ & PIQA & SIQA & HellaSwag & OBQA & WinoGrande & Avg. & Wiki2 \\ 
 &  & ($\uparrow$)  & ($\uparrow$) & ($\uparrow$) & ($\uparrow$) & ($\uparrow$) & ($\uparrow$) & ($\uparrow$) & ($\uparrow$) & ($\uparrow$) & ($\downarrow$) \\ \midrule
\multirow{9}{*}{MobileLLM-125M} & FP & 56.0 & 34.5 & 56.3 & 65.5 & 42.0 & 40.1 & 42.2 & 51.3 & 48.5 & 14.9 \\ 
\noalign{\vspace{0.1em}} \cdashline{2-12} \noalign{\vspace{0.2em}}
 & RTN & 26.1 & 24.1 & 62.2 & 50.3 & 37.1 & 26.6 & 28.9 & 49.4 & 38.1 & 6.4e5 \\ 
 & GPTQ & 28.9 & 26.2 & 44.2 & 51.1 & 39.1 & 28.1 & 33.2 & 48.0 & 37.3 & 2.4e2 \\ 
 & AWQ & 25.8 & 24.2 & 44.2 & 50.7 & 38.8 & 26.2 & 29.2 & 51.6 & 36.3 & 6.5e3 \\ 
 & OmniQ & 32.4 & 22.7 & 38.1 & 53.4 & 39.4 & 28.2 & 30.9 & 49.9 & 36.9 & 1.2e2 \\ 
 & LLM-QAT & 34.9 & 23.3 & 61.8 & 53.8 & 39.3 & 29.1 & 27.4 & 51.3 & 40.1 & 66.8 \\ 
 & SpinQuant & 31.6 & 23.3 & 40.3 & 52.2 & 40.5 & 28.6 & 28.9 & 50.1 & 36.9 & 68.7 \\ 
\rowcolor{gray!20}\cellcolor{white} & $\ours{}$ & 50.7 & 32.7 & 59.8 & 63.3 & 41.0 & 36.3 & 40.6 & 52.7 & 47.1 & 25.1 \\ 
\noalign{\vspace{0.1em}} \hdashline \noalign{\vspace{0.2em}}
\multirow{9}{*}{MobileLLM-350M} & FP & 65.5 & 42.3 & 57.4 & 71.0 & 43.5 & 53.3 & 47.3 & 58.3 & 54.8 & 10.4 \\ 
\noalign{\vspace{0.1em}} \cdashline{2-12} \noalign{\vspace{0.2em}}
 & RTN & 25.9 & 26.5 & 62.2 & 49.8 & 37.7 & 26.3 & 26.0 & 51.2 & 38.2 & 60.3 \\ 
 & GPTQ & 28.6 & 21.5 & 40.5 & 50.4 & 38.8 & 26.6 & 27.3 & 50.4 & 35.5 & 1.6e2 \\ 
 & AWQ & 27.0 & 23.5 & 47.6 & 49.4 & 38.2 & 26.4 & 26.2 & 49.5 & 36.0 & 7.2e4 \\ 
 & OmniQ & 33.9 & 23.4 & 39.6 & 54.9 & 38.4 & 28.6 & 29.4 & 49.7 & 37.2 & 80.8 \\ 
 & LLM-QAT & 40.6 & 25.9 & 62.0 & 55.6 & 40.0 & 31.8 & 31.1 & 52.6 & 42.5 & 8.2e4 \\ 
 & SpinQuant & 32.4 & 25.0 & 37.8 & 54.6 & 40.1 & 29.2 & 27.5 & 48.9 & 36.9 & 67.5 \\ 
\rowcolor{gray!20}\cellcolor{white} & $\ours{}$ & 59.0 & 39.4 & 63.5 & 68.8 & 43.1 & 47.3 & 44.1 & 57.5 & 52.8 & 17.7 \\ 
\noalign{\vspace{0.1em}} \hdashline \noalign{\vspace{0.2em}}
\multirow{9}{*}{MobileLLM-600M} & FP & 68.5 & 47.6 & 60.5 & 72.5 & 44.4 & 59.5 & 51.4 & 61.4 & 58.2 & 9.0 \\ 
\noalign{\vspace{0.1em}} \cdashline{2-12} \noalign{\vspace{0.2em}}
 & RTN & 25.8 & 26.2 & 37.8 & 49.8 & 37.6 & 25.9 & 26.8 & 50.9 & 35.1 & 2.7e2 \\ 
 & GPTQ & 27.9 & 26.6 & 48.2 & 49.5 & 39.0 & 25.9 & 26.8 & 49.4 & 36.6 & 3.4e2 \\ 
 & AWQ & 26.4 & 25.2 & 40.6 & 50.7 & 38.7 & 26.5 & 23.6 & 49.3 & 35.1 & 8.9e3 \\ 
 & OmniQ & 39.0 & 24.5 & 55.8 & 55.9 & 40.2 & 30.1 & 32.1 & 51.3 & 41.1 & 68.3 \\ 
 & LLM-QAT & 42.7 & 25.6 & 62.1 & 56.0 & 38.8 & 33.7 & 29.6 & 51.5 & 42.5 & 4.7e2 \\ 
 & SpinQuant & 28.2 & 22.4 & 39.8 & 52.0 & 38.0 & 27.9 & 22.1 & 49.1 & 34.9 & 2.7e2 \\ 
\rowcolor{gray!20}\cellcolor{white} & $\ours{}$ & 67.7 & 43.3 & 63.0 & 72.1 & 44.8 & 53.9 & 49.8 & 58.4 & 56.6 & 15.4 \\ 
\noalign{\vspace{0.1em}} \hdashline \noalign{\vspace{0.2em}}
\multirow{9}{*}{MobileLLM-1B} & FP & 73.4 & 50.8 & 67.6 & 74.1 & 46.7 & 64.7 & 56.6 & 62.7 & 62.1 & 8.0 \\ 
\noalign{\vspace{0.1em}} \cdashline{2-12} \noalign{\vspace{0.2em}}
 & RTN & 26.3 & 26.5 & 62.2 & 49.2 & 36.9 & 26.0 & 25.8 & 48.8 & 37.7 & 6.0e4 \\ 
 & GPTQ & 29.7 & 25.4 & 38.7 & 50.3 & 38.9 & 26.1 & 26.4 & 49.6 & 35.6 & 4.7e2 \\ 
 & AWQ & 26.6 & 26.8 & 59.1 & 50.2 & 37.1 & 26.0 & 24.0 & 50.4 & 37.5 & 1.5e5 \\ 
 & OmniQ & 38.0 & 26.1 & 41.7 & 54.6 & 40.1 & 31.1 & 33.3 & 51.4 & 39.5 & 46.3 \\ 
 & LLM-QAT & 42.6 & 26.7 & 49.7 & 57.7 & 40.4 & 34.9 & 31.4 & 49.2 & 41.6 & 1.9e5 \\ 
 & SpinQuant & 35.3 & 23.9 & 42.8 & 53.3 & 40.5 & 30.3 & 29.7 & 49.8 & 38.2 & 35.7 \\ 
\rowcolor{gray!20}\cellcolor{white} & $\ours{}$ & 73.3 & 49.3 & 65.7 & 74.2 & 45.9 & 60.3 & 57.4 & 61.6 & 61.0 & 13.4 \\ 
\noalign{\vspace{0.1em}} \hdashline \noalign{\vspace{0.2em}}
\multirow{9}{*}{MobileLLM-1.5B} & FP & 73.9 & 51.4 & 70.0 & 74.8 & 46.6 & 66.4 & 55.1 & 63.2 & 62.7 & 7.8 \\ 
\noalign{\vspace{0.1em}} \cdashline{2-12} \noalign{\vspace{0.2em}}
 & RTN & 25.2 & 25.3 & 37.8 & 49.3 & 36.0 & 26.4 & 25.0 & 48.5 & 34.2 & 1.7e2 \\ 
 & GPTQ & 29.8 & 22.3 & 45.3 & 53.4 & 39.3 & 27.0 & 25.8 & 51.4 & 36.8 & 1.7e2 \\ 
 & AWQ & 28.9 & 26.1 & 43.7 & 51.1 & 37.7 & 26.6 & 24.4 & 49.8 & 36.0 & 7.1e3 \\ 
 & OmniQ & 50.6 & 30.6 & 54.6 & 59.7 & 40.6 & 38.9 & 32.1 & 52.2 & 44.9 & 31.3 \\ 
 & LLM-QAT & 45.3 & 26.5 & 61.6 & 58.6 & 40.1 & 37.5 & 33.1 & 50.6 & 44.2 & 33.9 \\ 
 & SpinQuant & 34.0 & 21.6 & 52.3 & 54.1 & 39.4 & 29.5 & 29.9 & 50.5 & 38.9 & 37.4 \\ 
\rowcolor{gray!20}\cellcolor{white} & $\ours{}$ & 73.3 & 47.5 & 70.1 & 74.1 & 46.8 & 64.6 & 55.5 & 62.5 & 61.8 & 11.7 \\ 
\noalign{\vspace{0.1em}} \hdashline \noalign{\vspace{0.2em}}
\multirow{9}{*}{LLaMA-1B} & FP & 64.8 & 42.5 & 64.8 & 74.8 & 44.8 & 64.4 & 50.2 & 61.5 & 58.5 & 9.6 \\ 
\noalign{\vspace{0.1em}} \cdashline{2-12} \noalign{\vspace{0.2em}}
 & RTN & 26.5 & 26.8 & 62.2 & 51.0 & 36.8 & 25.9 & 28.5 & 50.2 & 38.5 & 1.5e6 \\ 
 & GPTQ & 29.3 & 27.6 & 37.8 & 51.5 & 38.6 & 26.5 & 32.0 & 50.8 & 36.8 & 3.3e2 \\ 
 & AWQ & 27.4 & 26.0 & 48.9 & 50.2 & 37.0 & 25.7 & 24.4 & 51.5 & 36.4 & 2.0e5 \\ 
 & OmniQ & 27.9 & 24.7 & 39.0 & 51.1 & 40.4 & 26.0 & 26.2 & 50.0 & 35.6 & 3.3e3 \\ 
 & LLM-QAT & 49.2 & 33.3 & 62.0 & 63.9 & 41.1 & 41.5 & 37.5 & 54.4 & 47.9 & 1.3e5 \\ 
 & SpinQuant & 25.6 & 24.6 & 62.4 & 51.6 & 36.1 & 25.8 & 29.1 & 50.8 & 38.3 & 46.7 \\ 
\rowcolor{gray!20}\cellcolor{white} & $\ours{}$ & 64.8 & 41.7 & 62.8 & 73.1 & 44.0 & 56.6 & 52.0 & 58.5 & 56.7 & 12.5 \\ 
\noalign{\vspace{0.1em}} \hdashline \noalign{\vspace{0.2em}}
\multirow{9}{*}{LLaMA-3B} & FP & 72.6 & 50.7 & 74.6 & 78.2 & 48.5 & 74.3 & 53.7 & 69.2 & 65.2 & 7.7 \\ 
\noalign{\vspace{0.1em}} \cdashline{2-12} \noalign{\vspace{0.2em}}
 & RTN & 26.9 & 25.1 & 37.8 & 50.1 & 37.9 & 25.7 & 26.6 & 49.6 & 35.0 & 7.8e5 \\ 
 & GPTQ & 28.6 & 22.9 & 46.4 & 50.0 & 38.4 & 27.1 & 30.1 & 50.1 & 36.7 & 2.7e2 \\ 
 & AWQ & 27.3 & 27.5 & 38.2 & 51.1 & 38.3 & 26.1 & 25.4 & 50.7 & 35.6 & 6.2e5 \\ 
 & OmniQ & 28.3 & 24.6 & 37.8 & 50.5 & 38.0 & 25.3 & 26.6 & 50.2 & 35.2 & 6.5e3 \\ 
 & LLM-QAT & 49.3 & 33.3 & 63.5 & 65.2 & 41.7 & 48.9 & 34.2 & 52.2 & 48.5 & 2.9e5 \\ 
 & SpinQuant & 28.3 & 23.7 & 53.2 & 51.1 & 38.8 & 26.1 & 25.8 & 49.0 & 37.0 & 57.4 \\ 
\rowcolor{gray!20}\cellcolor{white} & $\ours{}$ & 73.9 & 49.0 & 68.8 & 76.4 & 47.0 & 69.2 & 56.6 & 64.4 & 63.2 & 9.1 \\ 
\noalign{\vspace{0.1em}} \hdashline \noalign{\vspace{0.2em}}
\multirow{9}{*}{LLaMA-8B} & FP & 81.0 & 57.7 & 83.6 & 81.0 & 49.3 & 79.5 & 55.7 & 73.9 & 70.2 & 6.2 \\ 
\noalign{\vspace{0.1em}} \cdashline{2-12} \noalign{\vspace{0.2em}}
 & RTN & 27.2 & 25.1 & 37.8 & 49.7 & 37.4 & 26.1 & 26.2 & 50.5 & 35.0 & 1.2e6 \\ 
 & GPTQ & 27.0 & 26.1 & 61.6 & 50.5 & 37.4 & 26.0 & 27.5 & 49.7 & 38.2 & 1.6e2 \\ 
 & AWQ & 26.0 & 27.1 & 58.3 & 51.4 & 38.0 & 26.1 & 23.8 & 49.8 & 37.6 & 1.1e6 \\ 
 & OmniQ & 27.3 & 22.8 & 37.9 & 49.5 & 38.7 & 25.3 & 23.4 & 49.4 & 34.3 & 7.6e4 \\ 
 & LLM-QAT & 54.8 & 35.9 & 64.8 & 68.0 & 41.8 & 58.0 & 35.7 & 54.7 & 51.7 & 29.5 \\ 
 & SpinQuant & 32.4 & 22.0 & 59.0 & 53.2 & 38.4 & 31.9 & 28.0 & 49.9 & 39.3 & 31.2 \\ 
\rowcolor{gray!20}\cellcolor{white} & $\ours{}$ & 78.5 & 54.5 & 76.4 & 79.2 & 48.9 & 73.8 & 54.5 & 70.0 & 67.0 & 8.0 \\ 
\hline\hline
\end{tabular}}
\end{table}

\begin{table}[h]
\renewcommand\arraystretch{0.6}
\centering
\caption{Complete results of \textbf{3-bit quantization} on WikiText2 and Zero-shot Common Sense Reasoning tasks..}
\vspace{-45pt}
\label{tab:appendix_w3}
\setlength{\tabcolsep}{1mm}
\resizebox{0.9\textwidth}{!}{%
\begin{tabular}{c|c|ccccccccc|c}
& & & & & & & & & & & \\
& & & & & & & & & & & \\
& & & & & & & & & & & \\
& & & & & & & & & & & \\
& & & & & & & & & & & \\
& & & & & & & & & & & \\
\hline\hline
\multirow{2}{*}{Model Name} & \multirow{2}{*}{Method} & ARC-e & ARC-c & BoolQ & PIQA & SIQA & HellaSwag & OBQA & WinoGrande & Avg. & Wiki2 \\ 
 &  & ($\uparrow$)  & ($\uparrow$) & ($\uparrow$) & ($\uparrow$) & ($\uparrow$) & ($\uparrow$) & ($\uparrow$) & ($\uparrow$) & ($\uparrow$) & ($\downarrow$) \\ \midrule
\multirow{9}{*}{MobileLLM-125M} & FP & 56.0 & 34.5 & 56.3 & 65.5 & 42.0 & 40.1 & 42.2 & 51.3 & 48.5 & 14.9 \\ 
\noalign{\vspace{0.1em}} \cdashline{2-12} \noalign{\vspace{0.2em}}
 & RTN & 45.7 & 30.0 & 59.0 & 60.5 & 40.4 & 34.9 & 38.3 & 50.5 & 44.9 & 38.2 \\ 
 & GPTQ & 49.0 & 28.2 & 53.3 & 61.3 & 40.5 & 36.2 & 37.3 & 50.9 & 44.6 & 22.8 \\ 
 & AWQ & 48.5 & 27.8 & 52.7 & 62.3 & 40.1 & 35.6 & 35.3 & 50.4 & 44.1 & 27.1 \\ 
 & OmniQ & 50.2 & 29.4 & 53.9 & 61.5 & 41.6 & 36.4 & 43.2 & 50.2 & 45.8 & 20.5 \\ 
 & LLM-QAT & 44.7 & 28.7 & 53.7 & 60.6 & 41.1 & 34.6 & 34.9 & 50.2 & 43.5 & 37.5 \\ 
 & SpinQuant & 50.9 & 30.8 & 46.7 & 62.1 & 41.5 & 37.3 & 39.1 & 48.9 & 44.7 & 17.6 \\ 
\rowcolor{gray!20}\cellcolor{white} & $\ours{}$ & 53.5 & 33.7 & 56.1 & 65.6 & 41.7 & 40.0 & 41.2 & 51.3 & 47.9 & 21.6 \\ 
\noalign{\vspace{0.1em}} \hdashline \noalign{\vspace{0.2em}}
\multirow{9}{*}{MobileLLM-350M} & FP & 65.5 & 42.3 & 57.4 & 71.0 & 43.5 & 53.3 & 47.3 & 58.3 & 54.8 & 10.4 \\ 
\noalign{\vspace{0.1em}} \cdashline{2-12} \noalign{\vspace{0.2em}}
 & RTN & 58.8 & 35.9 & 59.5 & 65.0 & 41.8 & 43.9 & 39.1 & 53.8 & 49.7 & 37.4 \\ 
 & GPTQ & 59.8 & 34.0 & 60.6 & 67.5 & 42.1 & 46.5 & 38.7 & 53.9 & 50.4 & 14.0 \\ 
 & AWQ & 59.5 & 35.7 & 57.5 & 66.9 & 42.1 & 47.0 & 42.3 & 53.8 & 50.6 & 14.5 \\ 
 & OmniQ & 58.0 & 36.2 & 61.2 & 67.2 & 42.4 & 46.1 & 42.1 & 52.0 & 50.7 & 13.5 \\ 
 & LLM-QAT & 54.6 & 35.4 & 60.5 & 65.9 & 42.2 & 42.6 & 41.9 & 53.4 & 49.5 & 22.6 \\ 
 & SpinQuant & 57.9 & 35.3 & 59.3 & 67.0 & 41.4 & 47.5 & 43.2 & 54.3 & 50.7 & 12.1 \\ 
\rowcolor{gray!20}\cellcolor{white} & $\ours{}$ & 63.9 & 40.5 & 61.4 & 70.6 & 43.2 & 51.4 & 50.0 & 56.6 & 54.7 & 14.9 \\ 
\noalign{\vspace{0.1em}} \hdashline \noalign{\vspace{0.2em}}
\multirow{9}{*}{MobileLLM-600M} & FP & 68.5 & 47.6 & 60.5 & 72.5 & 44.4 & 59.5 & 51.4 & 61.4 & 58.2 & 9.0 \\ 
\noalign{\vspace{0.1em}} \cdashline{2-12} \noalign{\vspace{0.2em}}
 & RTN & 55.3 & 32.5 & 57.0 & 63.1 & 42.1 & 40.6 & 37.7 & 54.1 & 47.8 & 12.0 \\ 
 & GPTQ & 61.4 & 38.0 & 55.7 & 68.5 & 42.5 & 51.8 & 43.2 & 56.2 & 52.2 & 11.7 \\ 
 & AWQ & 63.6 & 39.5 & 55.6 & 70.0 & 43.1 & 53.0 & 45.0 & 58.0 & 53.5 & 12.9 \\ 
 & OmniQ & 64.9 & 41.6 & 63.4 & 69.8 & 42.1 & 53.0 & 45.4 & 58.2 & 54.8 & 11.3 \\ 
 & LLM-QAT & 61.8 & 38.0 & 62.1 & 68.5 & 43.6 & 48.9 & 44.2 & 54.6 & 52.7 & 19.0 \\ 
 & SpinQuant & 63.4 & 42.9 & 60.9 & 68.7 & 42.4 & 52.0 & 44.5 & 57.4 & 54.0 & 10.5 \\ 
\rowcolor{gray!20}\cellcolor{white} & $\ours{}$ & 68.2 & 47.4 & 64.2 & 73.1 & 44.2 & 58.1 & 50.2 & 62.4 & 58.5 & 13.2 \\ 
\noalign{\vspace{0.1em}} \hdashline \noalign{\vspace{0.2em}}
\multirow{9}{*}{MobileLLM-1B} & FP & 73.4 & 50.8 & 67.6 & 74.1 & 46.7 & 64.7 & 56.6 & 62.7 & 62.1 & 8.0 \\ 
\noalign{\vspace{0.1em}} \cdashline{2-12} \noalign{\vspace{0.2em}}
 & RTN & 59.7 & 36.6 & 58.9 & 67.2 & 40.8 & 45.0 & 44.3 & 53.4 & 50.7 & 19.1 \\ 
 & GPTQ & 66.7 & 43.0 & 63.5 & 72.3 & 42.9 & 57.8 & 49.2 & 59.4 & 56.8 & 10.2 \\ 
 & AWQ & 68.8 & 43.5 & 62.9 & 71.1 & 43.7 & 57.9 & 49.2 & 57.0 & 56.8 & 10.8 \\ 
 & OmniQ & 69.5 & 44.7 & 64.8 & 72.1 & 43.5 & 57.3 & 47.0 & 57.7 & 57.1 & 9.8 \\ 
 & LLM-QAT & 65.3 & 42.6 & 61.2 & 70.4 & 44.0 & 54.3 & 48.8 & 56.8 & 55.5 & 17.4 \\ 
 & SpinQuant & 68.2 & 44.0 & 63.5 & 71.1 & 43.9 & 57.2 & 45.7 & 59.0 & 56.6 & 8.9 \\ 
\rowcolor{gray!20}\cellcolor{white} & $\ours{}$ & 72.3 & 51.4 & 67.0 & 74.5 & 45.7 & 63.4 & 53.7 & 62.1 & 61.3 & 12.4 \\ 
\noalign{\vspace{0.1em}} \hdashline \noalign{\vspace{0.2em}}
\multirow{9}{*}{MobileLLM-1.5B} & FP & 73.9 & 51.4 & 70.0 & 74.8 & 46.6 & 66.4 & 55.1 & 63.2 & 62.7 & 7.8 \\ 
\noalign{\vspace{0.1em}} \cdashline{2-12} \noalign{\vspace{0.2em}}
 & RTN & 63.2 & 38.0 & 58.5 & 67.2 & 43.6 & 47.9 & 45.9 & 56.0 & 52.5 & 10.2 \\ 
 & GPTQ & 70.6 & 43.7 & 64.5 & 71.9 & 45.0 & 59.2 & 50.8 & 58.9 & 58.1 & 9.9 \\ 
 & AWQ & 72.6 & 46.8 & 66.0 & 71.7 & 44.6 & 61.7 & 52.0 & 62.4 & 59.7 & 9.6 \\ 
 & OmniQ & 71.8 & 46.4 & 67.4 & 72.9 & 46.2 & 60.9 & 50.2 & 61.9 & 59.7 & 9.1 \\ 
 & LLM-QAT & 68.6 & 44.4 & 62.4 & 71.8 & 45.4 & 57.8 & 49.2 & 57.2 & 57.1 & 15.4 \\ 
 & SpinQuant & 71.5 & 45.1 & 67.8 & 71.9 & 44.8 & 61.3 & 50.2 & 61.6 & 59.3 & 8.5 \\ 
\rowcolor{gray!20}\cellcolor{white} & $\ours{}$ & 72.6 & 49.9 & 70.6 & 75.7 & 47.7 & 66.0 & 56.2 & 64.5 & 62.9 & 11.4 \\ 
\noalign{\vspace{0.1em}} \hdashline \noalign{\vspace{0.2em}}
\multirow{9}{*}{LLaMA-1B} & FP & 64.8 & 42.5 & 64.8 & 74.8 & 44.8 & 64.4 & 50.2 & 61.5 & 58.5 & 9.6 \\ 
\noalign{\vspace{0.1em}} \cdashline{2-12} \noalign{\vspace{0.2em}}
 & RTN & 28.9 & 25.0 & 55.9 & 53.5 & 37.8 & 30.1 & 28.9 & 50.6 & 38.8 & 30.9 \\ 
 & GPTQ & 37.4 & 27.3 & 43.1 & 58.4 & 39.2 & 37.1 & 32.4 & 53.8 & 41.1 & 68.6 \\ 
 & AWQ & 41.5 & 26.7 & 49.2 & 58.0 & 41.4 & 34.9 & 31.8 & 52.8 & 42.0 & 1.5e2 \\ 
 & OmniQ & 39.0 & 28.8 & 61.3 & 58.8 & 40.0 & 36.3 & 32.9 & 52.7 & 43.7 & 63.4 \\ 
 & LLM-QAT & 52.7 & 32.4 & 60.5 & 66.6 & 44.0 & 43.2 & 40.2 & 53.8 & 49.2 & 20.7 \\ 
 & SpinQuant & 56.9 & 34.9 & 61.0 & 69.3 & 42.0 & 53.4 & 41.2 & 56.2 & 51.9 & 12.6 \\ 
\rowcolor{gray!20}\cellcolor{white} & $\ours{}$ & 65.3 & 41.9 & 64.2 & 73.8 & 43.9 & 61.3 & 47.7 & 59.5 & 57.2 & 10.9 \\ 
\noalign{\vspace{0.1em}} \hdashline \noalign{\vspace{0.2em}}
\multirow{9}{*}{LLaMA-3B} & FP & 72.6 & 50.7 & 74.6 & 78.2 & 48.5 & 74.3 & 53.7 & 69.2 & 65.2 & 7.7 \\ 
\noalign{\vspace{0.1em}} \cdashline{2-12} \noalign{\vspace{0.2em}}
 & RTN & 40.4 & 29.7 & 60.1 & 60.6 & 41.3 & 43.4 & 33.4 & 52.9 & 45.2 & 24.9 \\ 
 & GPTQ & 50.4 & 34.6 & 65.1 & 66.6 & 44.1 & 53.8 & 35.7 & 58.8 & 51.1 & 11.4 \\ 
 & AWQ & 58.5 & 36.5 & 65.4 & 70.8 & 43.1 & 54.8 & 44.6 & 59.3 & 54.1 & 37.7 \\ 
 & OmniQ & 59.7 & 38.6 & 47.6 & 73.5 & 45.9 & 62.4 & 49.8 & 61.8 & 54.9 & 12.7 \\ 
 & LLM-QAT & 64.4 & 40.1 & 62.0 & 71.7 & 45.0 & 58.2 & 44.7 & 59.9 & 55.8 & 13.4 \\ 
 & SpinQuant & 66.4 & 43.8 & 70.8 & 73.9 & 47.7 & 67.6 & 51.0 & 67.1 & 61.0 & 9.2 \\ 
\rowcolor{gray!20}\cellcolor{white} & $\ours{}$ & 72.3 & 49.8 & 73.3 & 76.7 & 48.8 & 71.9 & 56.2 & 67.3 & 64.5 & 8.4 \\ 
\noalign{\vspace{0.1em}} \hdashline \noalign{\vspace{0.2em}}
\multirow{9}{*}{LLaMA-8B} & FP & 81.0 & 57.7 & 83.6 & 81.0 & 49.3 & 79.5 & 55.7 & 73.9 & 70.2 & 6.2 \\ 
\noalign{\vspace{0.1em}} \cdashline{2-12} \noalign{\vspace{0.2em}}
 & RTN & 42.4 & 29.4 & 43.0 & 61.6 & 41.0 & 37.3 & 34.2 & 53.9 & 42.9 & 12.6 \\ 
 & GPTQ & 60.8 & 35.5 & 69.0 & 70.3 & 44.9 & 61.3 & 38.7 & 64.9 & 55.7 & 9.1 \\ 
 & AWQ & 72.3 & 46.1 & 74.9 & 75.9 & 48.2 & 70.8 & 52.0 & 66.8 & 63.4 & 16.6 \\ 
 & OmniQ & 68.0 & 45.4 & 68.3 & 73.9 & 46.0 & 68.7 & 50.4 & 62.3 & 60.4 & 12.1 \\ 
 & LLM-QAT & 68.8 & 48.8 & 71.1 & 75.9 & 46.8 & 67.8 & 48.2 & 65.1 & 61.6 & 10.5 \\ 
 & SpinQuant & 75.5 & 52.0 & 81.0 & 78.7 & 49.2 & 74.3 & 53.6 & 70.5 & 66.9 & 7.4 \\ 
\rowcolor{gray!20}\cellcolor{white} & $\ours{}$ & 78.2 & 55.7 & 80.6 & 80.2 & 50.1 & 76.5 & 55.1 & 70.9 & 68.4 & 7.0 \\ 
\hline\hline
\end{tabular}}
\end{table}




\begin{table}[h]
\renewcommand\arraystretch{0.6}
\centering
\caption{Complete results of \textbf{4-bit quantization} on WikiText2 and Zero-shot Common Sense Reasoning tasks.}
\vspace{-45pt}
\label{tab:appendix_w4}
\setlength{\tabcolsep}{1mm}
\resizebox{0.9\textwidth}{!}{%
\begin{tabular}{c|c|ccccccccc|c}
& & & & & & & & & & & \\
& & & & & & & & & & & \\
& & & & & & & & & & & \\
& & & & & & & & & & & \\
& & & & & & & & & & & \\
& & & & & & & & & & & \\
\hline\hline
\multirow{2}{*}{Model Name} & \multirow{2}{*}{Method} & ARC-e & ARC-c & BoolQ & PIQA & SIQA & HellaSwag & OBQA & WinoGrande & Avg. & Wiki2 \\ 
 &  & ($\uparrow$)  & ($\uparrow$) & ($\uparrow$) & ($\uparrow$) & ($\uparrow$) & ($\uparrow$) & ($\uparrow$) & ($\uparrow$) & ($\uparrow$) & ($\downarrow$) \\ \midrule
\multirow{9}{*}{MobileLLM-125M} & FP & 56.0 & 34.5 & 56.3 & 65.5 & 42.0 & 40.1 & 42.2 & 51.3 & 48.5 & 14.9 \\ 
\noalign{\vspace{0.1em}} \cdashline{2-12} \noalign{\vspace{0.2em}}
 & RTN & 53.4 & 33.3 & 53.9 & 64.7 & 41.5 & 39.7 & 40.2 & 51.8 & 47.3 & 9.2 \\ 
 & GPTQ & 53.4 & 33.5 & 54.7 & 64.4 & 42.5 & 39.2 & 43.8 & 52.2 & 48.0 & 16.1 \\ 
 & AWQ & 54.2 & 33.5 & 56.6 & 65.0 & 41.9 & 39.5 & 41.1 & 51.2 & 47.9 & 16.0 \\ 
 & OmniQ & 52.8 & 33.5 & 56.1 & 63.4 & 41.4 & 39.2 & 39.7 & 50.8 & 47.1 & 16.1 \\ 
 & LLM-QAT & 54.2 & 33.4 & 52.2 & 64.7 & 42.4 & 39.0 & 42.7 & 51.7 & 47.5 & 52.1 \\ 
 & SpinQuant & 55.2 & 33.7 & 58.1 & 65.0 & 42.5 & 39.7 & 40.6 & 49.8 & 48.1 & 15.4 \\ 
\rowcolor{gray!20}\cellcolor{white} & $\ours{}$ & 55.4 & 35.2 & 54.1 & 66.2 & 41.7 & 40.8 & 44.0 & 52.1 & 48.7 & 20.4 \\ 
\noalign{\vspace{0.1em}} \hdashline \noalign{\vspace{0.2em}}
\multirow{9}{*}{MobileLLM-350M} & FP & 65.5 & 42.3 & 57.4 & 71.0 & 43.5 & 53.3 & 47.3 & 58.3 & 54.8 & 10.4 \\ 
\noalign{\vspace{0.1em}} \cdashline{2-12} \noalign{\vspace{0.2em}}
 & RTN & 63.6 & 39.0 & 55.2 & 70.3 & 42.8 & 51.5 & 49.8 & 58.9 & 53.9 & 7.3 \\ 
 & GPTQ & 63.8 & 39.7 & 53.7 & 69.7 & 42.7 & 51.4 & 47.9 & 57.8 & 53.3 & 11.0 \\ 
 & AWQ & 63.0 & 38.5 & 57.1 & 70.7 & 43.6 & 51.6 & 45.8 & 55.2 & 53.2 & 11.2 \\ 
 & OmniQ & 63.9 & 37.4 & 56.2 & 69.8 & 42.4 & 50.9 & 46.6 & 54.2 & 52.7 & 11.1 \\ 
 & LLM-QAT & 63.4 & 42.0 & 59.8 & 70.1 & 43.6 & 51.5 & 47.0 & 57.5 & 54.4 & 17.1 \\ 
 & SpinQuant & 62.5 & 37.8 & 56.1 & 69.6 & 43.1 & 51.5 & 43.8 & 55.7 & 52.5 & 10.6 \\ 
\rowcolor{gray!20}\cellcolor{white} & $\ours{}$ & 64.9 & 41.6 & 57.8 & 71.3 & 44.4 & 53.5 & 48.2 & 57.9 & 55.0 & 14.2 \\ 
\noalign{\vspace{0.1em}} \hdashline \noalign{\vspace{0.2em}}
\multirow{9}{*}{MobileLLM-600M} & FP & 68.5 & 47.6 & 60.5 & 72.5 & 44.4 & 59.5 & 51.4 & 61.4 & 58.2 & 9.0 \\ 
\noalign{\vspace{0.1em}} \cdashline{2-12} \noalign{\vspace{0.2em}}
 & RTN & 67.8 & 45.1 & 48.5 & 71.6 & 44.0 & 57.8 & 49.8 & 59.6 & 55.5 & 15.4 \\ 
 & GPTQ & 68.5 & 47.0 & 50.2 & 72.3 & 43.8 & 57.7 & 49.6 & 58.9 & 56.0 & 9.4 \\ 
 & AWQ & 68.8 & 45.0 & 60.5 & 72.3 & 44.0 & 58.3 & 48.2 & 59.8 & 57.1 & 9.7 \\ 
 & OmniQ & 68.4 & 45.0 & 59.5 & 71.5 & 43.7 & 58.1 & 49.0 & 59.0 & 56.8 & 9.5 \\ 
 & LLM-QAT & 67.2 & 47.4 & 65.2 & 71.8 & 43.8 & 57.8 & 50.6 & 59.8 & 57.9 & 11.0 \\ 
 & SpinQuant & 69.1 & 44.7 & 64.3 & 71.5 & 43.0 & 57.4 & 49.0 & 57.1 & 57.0 & 9.2 \\ 
\rowcolor{gray!20}\cellcolor{white} & $\ours{}$ & 69.3 & 48.9 & 64.8 & 73.2 & 44.2 & 59.5 & 51.2 & 62.1 & 59.2 & 13.2 \\ 
\noalign{\vspace{0.1em}} \hdashline \noalign{\vspace{0.2em}}
\multirow{9}{*}{MobileLLM-1B} & FP & 73.4 & 50.8 & 67.6 & 74.1 & 46.7 & 64.7 & 56.6 & 62.7 & 62.1 & 8.0 \\ 
\noalign{\vspace{0.1em}} \cdashline{2-12} \noalign{\vspace{0.2em}}
 & RTN & 73.1 & 47.7 & 63.5 & 75.0 & 45.7 & 62.8 & 56.2 & 61.2 & 60.6 & 11.2 \\ 
 & GPTQ & 72.6 & 50.7 & 65.5 & 74.8 & 45.9 & 63.7 & 56.6 & 62.3 & 61.5 & 8.4 \\ 
 & AWQ & 73.7 & 48.6 & 65.3 & 73.5 & 45.6 & 62.5 & 49.4 & 60.6 & 59.9 & 8.5 \\ 
 & OmniQ & 72.5 & 49.3 & 66.0 & 74.3 & 45.0 & 62.5 & 52.2 & 62.1 & 60.5 & 8.4 \\ 
 & LLM-QAT & 72.1 & 49.5 & 66.1 & 73.9 & 46.2 & 63.0 & 55.4 & 63.7 & 61.2 & 10.0 \\ 
 & SpinQuant & 70.5 & 47.0 & 66.6 & 74.1 & 44.2 & 62.4 & 51.6 & 61.6 & 59.8 & 8.2 \\ 
\rowcolor{gray!20}\cellcolor{white} & $\ours{}$ & 74.7 & 52.1 & 67.9 & 74.8 & 46.9 & 64.8 & 56.2 & 62.1 & 62.5 & 11.7 \\ 
\noalign{\vspace{0.1em}} \hdashline \noalign{\vspace{0.2em}}
\multirow{9}{*}{MobileLLM-1.5B} & FP & 73.9 & 51.4 & 70.0 & 74.8 & 46.6 & 66.4 & 55.1 & 63.2 & 62.7 & 7.8 \\ 
\noalign{\vspace{0.1em}} \cdashline{2-12} \noalign{\vspace{0.2em}}
 & RTN & 73.7 & 49.5 & 66.0 & 74.5 & 46.4 & 65.5 & 52.7 & 62.0 & 61.3 & 9.4 \\ 
 & GPTQ & 73.9 & 49.9 & 68.9 & 73.7 & 46.6 & 64.9 & 54.5 & 62.0 & 61.8 & 8.2 \\ 
 & AWQ & 74.9 & 49.2 & 68.1 & 73.4 & 46.3 & 65.0 & 52.2 & 63.8 & 61.6 & 8.2 \\ 
 & OmniQ & 75.3 & 50.2 & 67.6 & 74.2 & 45.8 & 64.6 & 53.8 & 62.7 & 61.8 & 8.2 \\ 
 & LLM-QAT & 72.3 & 49.5 & 70.1 & 73.5 & 47.1 & 64.5 & 53.2 & 63.4 & 61.7 & 13.9 \\ 
 & SpinQuant & 73.8 & 48.9 & 68.6 & 73.9 & 45.8 & 64.8 & 52.3 & 63.9 & 61.5 & 7.9 \\ 
\rowcolor{gray!20}\cellcolor{white} & $\ours{}$ & 74.4 & 51.7 & 71.8 & 75.3 & 47.3 & 67.2 & 57.6 & 63.0 & 63.6 & 11.0 \\ 
\noalign{\vspace{0.1em}} \hdashline \noalign{\vspace{0.2em}}
\multirow{9}{*}{LLaMA-1B} & FP & 64.8 & 42.5 & 64.8 & 74.8 & 44.8 & 64.4 & 50.2 & 61.5 & 58.5 & 9.6 \\ 
\noalign{\vspace{0.1em}} \cdashline{2-12} \noalign{\vspace{0.2em}}
 & RTN & 55.7 & 36.3 & 61.9 & 70.4 & 43.0 & 56.9 & 39.3 & 55.5 & 52.4 & 8.9 \\ 
 & GPTQ & 55.2 & 38.8 & 57.9 & 70.5 & 43.5 & 55.4 & 43.2 & 58.0 & 52.8 & 13.4 \\ 
 & AWQ & 63.4 & 40.0 & 63.5 & 73.4 & 44.5 & 60.5 & 45.8 & 60.3 & 56.4 & 12.2 \\ 
 & OmniQ & 60.0 & 38.0 & 59.4 & 70.6 & 43.5 & 57.5 & 44.8 & 57.4 & 53.9 & 13.4 \\ 
 & LLM-QAT & 61.3 & 38.1 & 62.3 & 73.0 & 44.2 & 59.0 & 41.8 & 58.7 & 54.8 & 8.6 \\ 
 & SpinQuant & 62.2 & 40.3 & 64.1 & 72.3 & 44.0 & 61.6 & 47.9 & 59.8 & 56.5 & 10.3 \\ 
\rowcolor{gray!20}\cellcolor{white} & $\ours{}$ & 67.4 & 43.4 & 64.4 & 74.8 & 44.4 & 63.5 & 50.4 & 61.4 & 58.7 & 10.3 \\ 
\noalign{\vspace{0.1em}} \hdashline \noalign{\vspace{0.2em}}
\multirow{9}{*}{LLaMA-3B} & FP & 72.6 & 50.7 & 74.6 & 78.2 & 48.5 & 74.3 & 53.7 & 69.2 & 65.2 & 7.7 \\ 
\noalign{\vspace{0.1em}} \cdashline{2-12} \noalign{\vspace{0.2em}}
 & RTN & 59.0 & 40.2 & 57.5 & 74.5 & 46.5 & 65.5 & 44.9 & 64.9 & 56.6 & 13.1 \\ 
 & GPTQ & 64.7 & 46.7 & 66.5 & 75.3 & 47.0 & 64.7 & 50.0 & 66.7 & 60.2 & 11.1 \\ 
 & AWQ & 69.9 & 47.6 & 72.9 & 77.2 & 49.9 & 72.8 & 51.4 & 67.5 & 63.6 & 8.7 \\ 
 & OmniQ & 70.6 & 47.5 & 73.9 & 77.0 & 46.9 & 72.0 & 53.2 & 67.1 & 63.5 & 8.6 \\ 
 & LLM-QAT & 71.8 & 48.1 & 74.6 & 76.6 & 48.1 & 71.4 & 52.3 & 67.4 & 63.8 & 8.2 \\ 
 & SpinQuant & 70.2 & 47.9 & 73.8 & 76.4 & 47.8 & 71.9 & 54.3 & 68.0 & 63.8 & 8.0 \\ 
\rowcolor{gray!20}\cellcolor{white} & $\ours{}$ & 73.8 & 50.3 & 75.4 & 77.2 & 48.5 & 73.3 & 57.0 & 67.7 & 65.4 & 8.0 \\ 
\noalign{\vspace{0.1em}} \hdashline \noalign{\vspace{0.2em}}
\multirow{9}{*}{LLaMA-8B} & FP & 81.0 & 57.7 & 83.6 & 81.0 & 49.3 & 79.5 & 55.7 & 73.9 & 70.2 & 6.2 \\ 
\noalign{\vspace{0.1em}} \cdashline{2-12} \noalign{\vspace{0.2em}}
 & RTN & 75.8 & 50.7 & 77.8 & 78.5 & 48.1 & 74.7 & 53.9 & 71.6 & 66.4 & 7.9 \\ 
 & GPTQ & 77.7 & 51.9 & 80.6 & 79.4 & 50.8 & 76.7 & 51.8 & 71.6 & 67.6 & 7.0 \\ 
 & AWQ & 78.5 & 51.8 & 81.8 & 80.7 & 49.2 & 78.3 & 52.8 & 72.6 & 68.2 & 7.0 \\ 
 & OmniQ & 77.3 & 51.3 & 79.2 & 79.6 & 48.0 & 77.2 & 54.8 & 70.4 & 67.2 & 7.1 \\ 
 & LLM-QAT & 77.4 & 54.0 & 82.9 & 79.1 & 49.2 & 77.6 & 54.3 & 72.0 & 68.3 & 13.4 \\ 
 & SpinQuant & 78.8 & 56.0 & 82.5 & 79.7 & 49.5 & 78.5 & 54.6 & 71.5 & 68.9 & 6.5 \\ 
\rowcolor{gray!20}\cellcolor{white} & $\ours{}$ & 78.6 & 55.6 & 80.2 & 80.4 & 51.5 & 77.8 & 55.7 & 71.8 & 69.0 & 6.8 \\ 
\hline\hline
\end{tabular}}
\end{table}

\subsection{CPU Latency Experimental Setup}
We measure the CPU latency of five MobileLLM models on an Apple M1 MacBook Pro (32GB RAM) using 6 threads.  Each evaluation uses 5 prompt tokens and generates 122 tokens.  For the quantized models, embedding and output layers are quantized to 8-bit precision using channel-wise quantization, while weights in fully connected layers are quantized to 2-bit or 4-bit precision. Accuracy and decoding speed (in tokens/s) were measured under identical settings.

\subsection{GPU Latency Experimental Setup and Results}
\begin{figure*}[t!]
    \centering
    \includegraphics[width=\linewidth]{appendix_figures/gpu_fig.pdf}
    \caption{\small (a) Accuracy versus end-to-end GPU latency trade-off analysis. (b) Speedup in GPU kernel latency relative to BF16.}
    \label{fig:gpu}
\end{figure*}

We measured the latency of LLaMA 3.2 models (1B, 3B, 8B) on an H100 NVL GPU (94GB memory). The W4A16 kernel used the Machete kernel from vLLM~\cite{kwon2023efficient, machete}, while the W2A16 kernel was implemented based on the CUTLASS mixed precision backbone kernel. All tests were performed on a single GPU with a context length of 2048 tokens. For kernel-level latency, we compared the 2-bit kernel to the 4-bit Machete kernel across three weight shapes: (4096 $\times$ 4096), (8192 $\times$ 8192), and (16384 $\times$ 16384).

For smaller models (1B, 3B, 8B), the performance speed-up from reducing weight precision from 4-bit to 2-bit is minimal. This is due to the impact of conversion overhead, which becomes more pronounced when the weight size is small. Since the in-kernel conversion latency ratio is higher for smaller models, the benefits of 2-bit quantization are outweighed by the overhead. Consequently, 4-bit quantization achieves a more favorable speed-accuracy trade-off in these settings, offering better overall performance.
In comparison, for larger weight shapes (16384 $\times$ 16384), the 2-bit kernel provides a substantial speedup, achieving 4.14$\times$ faster performance than FP16 and 1.24$\times$ faster than the Machete 4-bit kernel.


\subsection{QAT Scheduling Experimental Setup}
The total training budget (\(\mathcal{B}_{\text{train}}\)) is set to 100B tokens. We vary the proportion of tokens allocated for full-precision training versus quantization-aware training (QAT) finetuning, sweeping the ratio across \([0, 0.01, 0.05, 0.1, 0.2, 0.4, 0.6, 0.8, 0.9, 0.95, 0.99, 1]\). Here, a ratio of 0 corresponds to QAT from scratch, while a ratio of 1 represents full-precision training followed by post-training quantization (PTQ). 

For full-precision training, we use 8×8 GPUs, a batch size of 16, a weight decay of 0.1, an initial learning rate of \(2.5 \times 10^{-3}\), and a linear learning rate decay to zero. For quantized network training, we also use 8×8 GPUs but with a batch size of 8, no weight decay, an initial learning rate of \(1 \times 10^{-4}\), and a linear learning rate decay to zero.

\subsection{Embedding Bit Precision vs. Accuracy Trade-off}
\begin{figure*}[t!]
    \centering
    \includegraphics[width=\linewidth]{appendix_figures/pareto_optimal_embedding_fig.pdf}
    \vspace{-1em}
    \caption{\small Trade-off between model size and accuracy for the optimal embedding bit width. ``W$x$E$y$'' indicates quantized weights into $x$-bits and embeddings into $y$-bits}
    \label{fig:pareto_optimal_embedding}
\end{figure*}

Despite the prevalent practice of not quantizing embedding and output layers, as noted in prior works such as Frantar et al.~\cite{frantar2022gptq} and Ma et al.~\cite{ma2024era}, our study extends the scaling law analysis by examining the impact of quantizing these layers. As illustrated in Figure~\ref{fig:pareto_optimal_embedding}, utilizing 4-bit embeddings or matching the bit precision of embeddings to that of weights positions these configurations on the Pareto front, in contrast to employing 8-bit or 16-bit embeddings.

\subsection{Weight Bit Precision vs. Accuracy Trade-off}
\begin{figure*}[t!]
    \centering
    \includegraphics[width=0.75\linewidth]{appendix_figures/pareto_optimal_weight_fig.pdf}
    \caption{\small Trade-off between model size and accuracy for the optimal weight bit width. ``W$x$E$y$'' indicates quantized weights into $x$-bits and embeddings into $y$-bits}
    \label{fig:pareto_optimal_weight}
\end{figure*}

For the trade-off between weight-bit precision and model accuracy, we consider two configurations: 4-bit embeddings and embeddings with the same bit precision as weights. In both scenarios, lower-bit quantization, such as 1.58-bit, 2-bit, and 3-bit, consistently outperforms 4-bit quantization, as depicted in Figure~\ref{fig:pareto_optimal_weight}.

\subsection{Pareto Curve in More Tasks}
\begin{figure*}[t!]
    \centering
    \includegraphics[width=\linewidth]{appendix_figures/pareto_curve_othertasks_fig.pdf}
    \caption{Trade-off between model size and accuracy on (a) TQA (b) $\text{Race}_\text{middle}$ and (c) $\text{Race}_\text{high}$.  ``W$x$E$y$'' denotes quantized weights into $x$-bits and embeddings into $y$-bits.}
    \label{fig:pareto_curve_othertasks}
\end{figure*}

\begin{figure*}[t!]
    \centering
    \includegraphics[width=0.7\linewidth]{appendix_figures/pareto_curve_othertasks2_fig.pdf}
    \caption{Trade-off between model size and accuracy on (a) LAMBADA (b) SciQ.  ``W$x$E$y$'' indicates quantized weights into $x$-bits and embeddings into $y$-bits.}
    \label{fig:pareto_curve_othertasks2}
\end{figure*}

Furthermore, we present results from a question-answering task, TriviaQA (TQA)~\cite{joshi2017triviaqa}, and a reading comprehension benchmark, RACE~\cite{lai2017race}, in Figures~\ref{fig:pareto_curve_othertasks} The findings are consistent across these tasks: 1-bit quantization yields the lowest performance, whereas 1.58-bit, 2-bit, and 3-bit quantization are comparable and generally surpass the performance of 4-bit quantization.

Additionally, for context-based word prediction (LAMBADA~\cite{paperno2016lambada}) and multiple-choice science questions (SciQ~\cite{welbl2017crowdsourcing}) in Figrue~\ref{fig:pareto_curve_othertasks2}, the results also shows a clear trend of 2-bit residing on the Pareto optimal frontier, outperforming 4-bit.

