\section{Related Work}
We review related works in the literature by categorizing them into two groups. See~\cite{survey_1} for a recent survey.

\subsection{Centralized VNE}

\cat{Generic VNE Algorithms}
In~\cite{chowdhury2011vineyard}, a heuristic approach based on linear programming and rounding was introduced. While their results regarding acceptance and the revenue-to-cost ratio were satisfactory, the algorithm suffered from a relatively high runtime.
%
To tackle the problem's scale and complexity, authors in~\cite{Kibalya_2020_CN} used dynamic programming principles. They decomposed virtual networks into a collection of edge-disjoint path segments and leveraged a multi-layer graph transformation to embed the resulting segments.
%
In~\cite{Cao_2020_WCNCW}, the authors considered a multi-dimensional scenario where each physical node and link is associated with a security feature. They implemented a greedy embedding approach to allocate resources while minimizing the likelihood of malicious attacks targeting virtual networks.
%
The authors in~\cite{Pentelas_2020_NOMS} discussed a multi-dimensional scenario in which each physical node and the link is associated with a security feature. They utilized a greedy embedding approach to allocate resources, aiming to minimize the potential of malicious attacks on virtual networks.
%
% In~\cite{Hosseini_2019_TNSM}, the authors focused on ensuring end-to-end delays for virtual links by representing the latency in physical links as random variables with known mean and variance.

\cat{Learning-based VNE Algorithms}
% In~\cite{graphNN_RL}, the authors utilized an asynchronous advantage actor-critic algorithm to automate the embedding process using exploration-exploitation techniques, where a spectral-based convolutional graph neural network is employed to extract features from the physical network that model the environment.
% 
To address the temporal dependency of the physical network state, which changes after serving requests, the authors in~\cite{Yao_2020_TNSM} framed the node embedding task as a time-series problem. They trained a recurrent neural network using the seq2seq model to learn the embedding location for virtual nodes.
%
In another study, referred to as GraphViNE~\cite{habibi2020graphvine}, the authors proposed a Graph Neural Network (GNN)-based approach that incorporates a BFS-like algorithm for solving the Virtual Network Embedding (VNE) problem. This method benefits from the ability of GNNs to capture complex relationships within both virtual and substrate networks. Despite using GPU resources to manage the runtime of the VNE algorithm, the runtime remains higher than many other methods.
%
In some works~\cite{graphNN_RL, dolati2019deepvine, elkael2022monkey}, reinforcement learning (RL) has been employed to automate the embedding process using exploration-exploitation techniques. In a particular study~\cite{elkael2022monkey}, the authors combined the Rollout Policy Adaptive Algorithm (NRPA) with a neighborhood search, proposing a Neighborhood Enhancement Policy Adaptive (NEPA) algorithm. This technique explores the policy branch of the search tree to find the optimal embedding.
%

All of the aforementioned studies adopt centralized approaches, with a single node responsible for embedding. However, centralized methods encounter two potential issues. First, their scalability is inherently limited, given the potential for the physical network to comprise over a million nodes. Second, the inability of multiple nodes to handle Virtual Network (VN) requests means that service providers must process these requests sequentially, not concurrently.

\subsection{Distributed VNE}
The first fully distributed VNE approach, named ADVNE~\cite{houidi2008distributed}, breaks each virtual network into several hub-and-spoke clusters, each of which is assigned to a substrate node in the physical network for the embedding task. Through the use of multi-agent systems, these clusters are embedded into substrate nodes in a distributed manner. However, due to the autonomous nature of each substrate node in ADVNE, unavoidable message overhead exists among the substrate nodes. As the size of the substrate network grows, the message overhead increases exponentially, potentially reducing ADVNE's efficiency and scalability in large-scale settings. Moreover, ADVNE doesn't consider embedding quality (like embedding costs) and substrate network constraints (like CPU and bandwidth limitations), thereby limiting its practicality in real-world applications.



To reduce message overhead and improve embedding quality, the authors in~\cite{beck2015distributed} introduced DPVNE. In this approach, the substrate network is first hierarchically divided into sub-SNs, which are organized as a binary tree. Specific substrate nodes are then selected as delegation nodes, responsible for managing external VN requests from customers. These delegation nodes receive VN requests and assign them to sub-SNs based on heuristic information. DPVNE employs a heuristic algorithm to embed VNs within the assigned sub-SNs, with the embedding process executed by the embedder node in each sub-SN. This method allows for the concurrent processing of multiple VN requests since several delegation nodes are selected. However, it doesn't suggest a method for communication between delegation nodes to select the best embedding.
%
Observing the promising performance of metaheuristic approaches in VNE, the authors in~\cite{song2019distributed} integrated metaheuristics into distributed VNE problems. They merged a distributed VNE system with their proposed distributed VNE algorithm to enhance the performance of distributed approaches.



While distributed algorithms have enhanced the performance and scalability of Virtual Network Embedding (VNE), they still encounter a lack of robustness due to the reliance on a standalone controller for VNE assignment. This approach introduces a single point of failure, compromising the system's resilience and scalability. 

