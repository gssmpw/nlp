% \begin{abstract}
% Virtual Network Embedding (VNE) is an essential component of network virtualization technology. 
% %
% Prior works on VNE mainly focused on resource efficiency and did not address the scalability as a first-grade objective. Consequently, the ever-increasing demand and size render them less-practical.
% %
% The few existing designs for mitigating this problem either do not extend to multi-resource settings or do not consider the physical servers and network simultaneously.
% %
% In this work, we develop \ourAlg, a parallelizable VNE solution based on spatial Graph Neural Networks (GNN) that clusters the servers to guide the embedding process towards an improved runtime and performance. 
% %
% Our experiments using simulations show that the parallelism of \ourAlg\ reduces its runtime by a factor of $8$. Also, \ourAlg\ improves the revenue-to-cost ratio by about $18\%$, compared to other simulated algorithms.
% \end{abstract}



\begin{abstract}
Virtual Network Embedding (VNE) is a technique for mapping virtual networks onto a physical network infrastructure, enabling multiple virtual networks to coexist on a shared physical network. 
%
Previous works focused on implementing centralized VNE algorithms, which suffer from lack of scalability and robustness. 
%
This project aims to implement a decentralized virtual network embedding algorithm that addresses the challenges of network virtualization, such as scalability,  single point of failure, and DoS attacks.
%
The proposed approach involves selecting L leaders from the physical nodes and embedding a virtual network request (VNR) in the local network of each leader using a simple algorithm like BFS. 
%
The algorithm then uses a leader-election mechanism for determining the node with the lowest cost and highest revenue and propagates the embedding to other leaders. 
%
By utilizing decentralization, we improve the scalability and robustness of the solution. 
% 
Additionally, we evaluate the effectiveness of our fully decentralized algorithm by comparing it with existing approaches. Our algorithm performs $12\%$ better in terms of acceptance rate and improves the revenue-to-cost ratio by roughly $21\%$ to compared approaches. 
\end{abstract}