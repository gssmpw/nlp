\section{Motivation}
\label{sec:motivation}
Virtual Network Embedding (VNE) has widespread applications in various fields, including cloud computing, data centers, and software-defined networks~\cite{survey_1}.
%
Traditionally, VNE is a centralized algorithm where a single controller handles all virtual network requests. However, as the physical network size has increased, centralized VNE has struggled to process requests in real time. This scalability issue has led to the proposition of distributed solutions.


These distributed solutions segment the network into smaller partitions, applying the embedding algorithm to each to minimize the challenge of searching the entire network. Despite their advantages, these solutions still require a centralized controller to assign requests to the sub-coordinators of each partition. While this central control requirement is suitable for some environments, such as data centers, it doesn't suit others, like edge IoT networks and mobile networks.


There are three key challenges in such environments that the current distributed VNE algorithms fail to address:

\begin{enumerate}
\item \textbf{Scalability:} Besides the increasing size of the physical networks, the number of requests is also growing, and the VNR can be sent from a wide range of physical locations across the world.
\item \textbf{Robustness:} A centralized controller has inherent vulnerabilities, including susceptibility to a single point of failure and Denial of Service (DoS) attacks.
\item \textbf{Privacy:} Users typically don't want their VNRs to be accessible to the entire network or a centralized authority.
\end{enumerate}

The objective of this paper is to develop a decentralized VNE algorithm designed to effectively overcome these challenges.