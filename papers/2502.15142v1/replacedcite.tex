\section{Related Work}
\label{sec: related work}

\subsection{Research on accessibility in mobile apps}\label{sub: research1}
More recently, there are various related researches on exploring the accessibility in mobile apps____________.
All of them comply with the WCAG____ standard, and aim to guide the development of app accessibility.

Exploring the quantity and types of accessibility issues in apps is a hot concern in the accessibility research community.
Thus, Alshayban et al.____ systematically analyzed the accessibility issues in current apps from the state of affairs, sentiments, and ways forward.
To improve the accessibility of apps, Chen et al.____ proposed a method to generate the labels of icons based on machine learning models.
Further, Mehralian et al.____ optimized this method by incorporating the contextual information of labels, and obtained more accurate results. 
Except for such methods of generating labels, other studies focused on designing and developing tools that could check accessibility issues. 
To that end, Chen et al.____ developed a tool named Xbot based on large-scale empirical research and the Accessibility Test Framework. 
This tool can capture the accessibility issues of small size and the lack of content-description in GUIs by optimizing the Monkey, and it can also collect more issues than other detection tools. 
Similar tools also include the Mobile Accessibility Test (MATE) by Eler et al.____ and Programmable UI-mobile Automation (PUMA) by Hao et al.____. 
Both of these methods rely on rule-based approaches to detect accessibility issues within GUIs. 
MATE demonstrates higher accuracy in identifying component size issues due to the diversification of rules specifically tailored for this problem. 
On the other hand, PUMA excels in optimizing various rules, allowing it to identify a broader range of accessibility issues, especially those related to image views. 
Besides, in the industry, Google Research developed Accessibility Scanner____.
This tool can scan the GUIs and provide suggestions to improve the accessibility of apps, based on content labels, touch target size, clickable items, as well as text and image contrast.
Also, IBM released Mobile Accessibility Checker____ to check accessibility issues in GUIs, but it is weak to process the color-related issues.
The biggest advantage of both industry tools is that they can run in real-time on mobile devices and provide reasonable problem reporting.
However, when compared to manual methods, these detection techniques may identify invisible views, while potentially overlooking certain problematic views and identifying invisible ones, such as `RecyclerViews' and `BoxLayouts'.

The above-mentioned studies primarily focus on detecting and reporting the GUI components experiencing issues, such as inconsistent color palettes, narrow intervals, and inappropriate sizing.
Nevertheless, none of them provides solutions to guide developers on tackling these issues.
Confronted with such outcomes, developers are still faced with the confusion of what adjustments should be made for specific components in the repair process.
In many instances, a multitude of accessibility issues can potentially discourage developers, or even abandon their remediation efforts.
As such, a method that involves offering adjustment strategies at a detailed level of component attributes is required for guiding developers in fixing accessibility issues, thereby enhancing the experience for low vision users.
This demand also underscores the necessity for the method proposed in this paper.

\subsection{Research on fixing accessibility issues}\label{sub: research2}
GUIs with unclear layouts and poor visual presentation would significantly impede low vision users from using apps.
Therefore, designing well-accessible GUIs and resolving accessibility issues are crucial means to improve the experience of low vision users.

Currently, there is a lack of research in this area, with only a few studies focused on exploring how to fix accessibility issues or provide practical solutions.
One such study by Li et al.____ proposed an automatic generation tool for the skeleton layout based on the Transformer model____, which solves three issues that developers may introduce in GUI design: improper component positioning, significant variations in component sizes, and disordered hierarchical structures. 
However, this approach can only guide users on how to design new GUIs and cannot be applied to already released GUIs with accessibility issues, nor can it directly repair these problems.
Apart from this tool, the method proposed by Alotaibi et al.____ in 2021 is the most relevant work to date.
Their method aimed at providing adjustments for size-based accessibility issues in GUIs. 
In more detail, they adopted the multi-objective optimization strategy in the genetic algorithm, and then, optimized four objectives: `Accessibility Heuristic,' `Relative Positioning and Alignment of Views,' `Minimum Spacing Between Views,' and `Amount of View Size Change,' to ensure that the repaired component keeps the visual consistency with other components in size. 
However, due to its exclusive focus on size-related issues, this approach may inadvertently introduce other problems during the fixing process (e.g., reducing the intervals after increasing component sizes). 
More details about this limitation can be found in our experiment of Section~\ref{sub: comparable}.
Such outcomes introduce uncertainty regarding whether the accessibility of the GUI is indeed improved.

The aforementioned methods illustrate the efforts made by researchers in fixing accessibility issues.
However, there is still a need to determine how to effectively fix a diverse array of GUI accessibility issues and to adopt a holistic approach to prevent the emergence of new issues during the repair process.
This is precisely the objective achieved by the approach proposed in this paper.


\begin{figure}
\centering
\includegraphics[width=8cm]{Figure_2.png}
\caption{The practical example of Whatsapp.}
\label{fig: whatsapp}
\end{figure}