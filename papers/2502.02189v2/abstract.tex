Novel materials drive progress across applications from energy storage to electronics. Automated characterization of material structures with machine learning methods offers a promising strategy for accelerating this key step in material design. In this work, we introduce an autoregressive language model that performs crystal structure prediction (CSP) from powder diffraction data. The presented model, deCIFer, generates crystal structures in the widely used Crystallographic Information File (CIF) format and can be conditioned on powder X-ray diffraction (PXRD) data. Unlike earlier works that primarily rely on high-level descriptors like composition, deCIFer performs CSP from diffraction data. We train deCIFer on nearly 2.3M unique crystal structures and validate on diverse sets of PXRD patterns for characterizing challenging inorganic crystal systems. Qualitative and quantitative assessments using the residual weighted profile and Wasserstein distance show that deCIFer produces structures that more accurately match the target diffraction data when conditioned, compared to the unconditioned case. Notably, deCIFer can achieve a 94\% match rate on unseen data. deCIFer bridges experimental diffraction data with computational CSP, lending itself as a powerful tool for crystal structure characterization and accelerating materials discovery.
