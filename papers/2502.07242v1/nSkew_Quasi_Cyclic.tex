\documentclass{amsart}
    \pagestyle{plain}
\usepackage{amsaddr}
\usepackage{amssymb}
\usepackage{amsmath,amsthm,nicematrix}
\usepackage{amsfonts}
\usepackage{graphicx}
\usepackage[top = 1in, right = 1in, left=1in, bottom = 1in]{geometry}
\usepackage{setspace}
    \onehalfspacing
\usepackage{xcolor}

\usepackage[
backend = biber,
sorting = nyt,
style = numeric
]{biblatex}
\addbibresource{aers.bib}

\usepackage{hyperref}

\newcommand{\N}{\mathbb{N}}
\newcommand{\Z}{\mathbb{Z}}
\newcommand{\I}{\mathbb{I}}
\newcommand{\R}{\mathbb{R}}
\newcommand{\Q}{\mathbb{Q}}
\newcommand{\C}{\mathcal{C}}
\newcommand{\F}{\mathbb{F}}
\newcommand{\bs}{\boldsymbol}

\DeclareMathOperator{\RS}{RS}
\DeclareMathOperator{\nRS}{nRS}
\DeclareMathOperator{\GRS}{GRS}
\DeclareMathOperator{\gcrd}{gcrd}
\DeclareMathOperator{\lclm}{lclm}


\newtheorem{theorem}{Theorem}[section]
\newtheorem{proposition}[theorem]{Proposition}
\newtheorem{corollary}[theorem]{Corollary}
\newtheorem{lemma}[theorem]{Lemma}
\newtheorem{conjecture}[theorem]{Conjecture}

\theoremstyle{definition}
\newtheorem{example}[theorem]{Example}
\newtheorem{definition}[theorem]{Definition}
\newtheorem{question}[theorem]{Question}

\theoremstyle{remark}
\newtheorem{remark}{Remark}
\newtheorem*{claim}{Claim}


\begin{document}
\title{Nonlinear Reed-Solomon codes and nonlinear skew quasi-cyclic codes}
\author{Daniel Bossaller}
\address[Bossaller]{University of Alabama in Huntsville, Huntsville, AL, USA}
\email{daniel.bossaller@uah.edu}
\author{Daniel Herden}
\address[Herden]{Baylor University, Waco, TX, USA}
\email{daniel\_herden@baylor.edu}
\author{Indalecio Ruiz-Bola\~nos}
\email{indalecio\_ruizbolan1@baylor.edu}
\address[Ruiz-Bola\~nos]{Baylor University, Waco, TX, USA}

\thanks{The second author has been partially supported by Simons Foundation grant MPS-TSM-00007788.}

%% Abstract
\begin{abstract}
%% Text of abstract
    This article begins with an exploration of nonlinear codes ($\F_q$-linear subspaces of $\F_{q^m}^n$) which are generalizations of the familiar Reed-Solomon codes. This then leads to a wider exploration of nonlinear analogues of the skew quasi-cyclic codes of index $\ell$ first explored in 2010 by Abualrub et al., i.e., $\F_{q^m}[x;\sigma]$-submodules of $\left(\F_{q^m}[x;\sigma]/(x^n - 1)\right)^\ell$. After introducing nonlinear skew quasi-cyclic codes, we then determine the module structure of these codes using a two-fold iteration of the Smith normal form of matrices over skew polynomial rings. Finally we show that in certain cases, a single use of the Smith normal form will suffice to determine the elementary divisors of the code.
\end{abstract}

\maketitle

\section{Introduction}\label{intro}
This article investigates nonlinear generalizations of Reed-Solomon codes and skew quasi-cyclic codes. The study of error-correcting codes is a centerpiece of modern digital communication and storage systems, ensuring reliable transmission and recovery of data. Among the most significant developments in this field are Reed-Solomon (RS) codes, renowned for their robust error-correcting capabilities. RS codes were introduced by Irving S. Reed and Gustave Solomon in 1960 \cite{rs1960} to address the need for robust error correction in digital communication and storage. Initially hampered by computational complexity, their adoption surged with the development of efficient decoding algorithms like the Berlekamp-Massey algorithm.

This paper introduces and explores nonlinear Reed-Solomon (nRS) codes, a generalization of classical RS codes, characterized by an evaluation set within an extension field rather than the base field. The nRS codes fall into the classification of broadly studied codes, such as evaluation codes \cite{Boucher-Ulmer:2014, Stichtenoth-2009}, skew cyclic \cite{Boucher-Geiselmann-Ulmer:2007, Boucher-Ulmer:2009, Gluesing-Luerssen:2021} and skew quasi-cyclic codes \cite{Abualrub-et-al:2010, Abualrub-et-al:2018, Bhaintwal:2012, Gao-Shen-Fu:2016}, and nonlinear codes \cite{Huffman:2013, huffmancyclic, Sharma-Kaur:2017}.

A linear code $\C$ of length $n\ell$ over $\F_{q^m}$ is called a \emph{quasi-cyclic code of index $\ell$} if  it is invariant under the shift of codewords by $\ell$ positions and $\ell$ is the minimum number with this property. These codes can be studied as $\F_{q^m}[X] / (X^n-1)$-modules by considering their polynomial representation, where $\F_{q^m}[X]$ is the ring of polynomials with coefficients in $\F_{q^m}$, with the usual commutative addition and multiplication.

Given an automorphism $\sigma: \F_{q^m}\rightarrow \F_{q^m}$, the ring of {skew polynomials $\F_{q^m}[X;\sigma]$} is the collection of polynomials with the usual addition but multiplication defined by $X\cdot a = \sigma(a)\cdot X$ for all $a\in \F_{q^m}$. The multiplication in $\F_{q^m}[X;\sigma]$ is not commutative. A linear code $\C$ of length $n$ is called \emph{skew cyclic} if $(c_1, \ldots, c_{n})\in \C$ implies that $\left(\sigma(c_n),\sigma(c_1) \ldots, \sigma(c_{n-1})\right)\in \C$. Skew cyclic codes can be studied as submodules of $\F_{q^m}[X;\sigma] / (X^n-1)$ when its codewords are regarded as skew polynomials, provided that $m\mid n$.

Given the field extension $\F_q\leq \F_{q^m}$, an \emph{$[n,k]$ additive code $\C$} is a $km$-dimensional $\F_q$-subspace of $\left( \F_q^m \right)^n$. We have that $\C$ has $q^{km}$ codewords. On the other hand, an \emph{$[n,k]$  $\F_q$-linear $\F_{q^m}$-code} is a $k$-dimensional $\F_q$-subspace of $\F_{q^m}^n$.

The latter half of Section \ref{sec:nrs} is concerned with defining a proper dual for our nonlinear Reed-Solomon codes. The classical Euclidean dual for such a nonlinear code results in a linear dual code. Thus, in general $\left( \C^\perp \right)^\perp \ne\C.$ To fix this issue, we present another dual, $\C_q^\perp$, with the help of \emph{orbit vectors}. This dual allows us to recover the nRS code $\C$ as the double dual whenever the evaluation vector is chosen to be an orbit vector.
%We provide two distinct duals for these nRS codes. The first dual is derived from the intersection of the classical dual with vectors possessing favorable orbital properties. This intersection approach highlights the intricate relationship between nRS codes and their classical counterparts, offering insights into their structural nuances.

We construct a second dual by exploring a generalization of skew quasi-cyclic codes. Unlike fully linear codes, these are linear only with respect to the base field $\F_q$, and are called \emph{$\F_q$-linear skew quasi-cyclic codes}. By representing these codes through their polynomial analogs, we show the equivalence of $\F_q$-linear skew quasi-cyclic codes and $P_n$-submodules of $R_n^\ell$, where $P_n$ and $R_n$ arise from  the quotients of skew polynomial rings by the ideal generated by $X^n-1$.

Utilizing this polynomial representation, we derive an algebraic structure for these codes, facilitating the construction of a stacked basis for both the code and the ideal generated by $X^n-1$. This foundation allows us to define a novel inner product and an optimal dual space for any $\F_q$-linear skew quasi-cyclic code, which allow us to recover the original code as the double dual. The problem of finding the algebraic structure of codes has also been tackled in \cite{Ou-azzou-et-al:2024, Lally-Fitzpatrick:2001, Ling-Sole:2006, Ling-Sole:2001, Ling-Sole:2003, Ling-Sole:2005}.

%To help obtain our results, we implemented the Smith normal form \cite{Berrick:2000} in python, for which we made use of some algorithms obtained from \cite{CarusoLeBorgne2017}.
%
%The concepts underpinning this research are multifaceted and encompass a range of mathematical constructs. Key among these are nonlinear codes, skew cyclic and quasicyclic codes, classical RS codes, skew evaluation codes, and the algebraic structures of skew polynomial rings. Additionally, the study involves left and right greatest common divisors, skew group rings, total divisors, and the Smith normal form.
%This paper aims to provide a thorough exposition of nRS codes, elucidate their dual properties, and establish a framework for their application in advanced coding systems. By bridging the gap between classical and modern coding theories, we seek to contribute to the ongoing evolution of error-correcting codes and their practical implementations.
Our article is organized as follows. Section~\ref{sec:prelim} reviews the necessary theory of Reed-Solomon codes as well as cyclic codes, quasi-cyclic codes, and their counterparts which utilize skew polynomials over a finite field. In \ref{sec:nrs} we introduce nonlinear generalizations of Reed-Solomon codes, provide bounds for their parameters, and conditions under which they meet the Singleton bound. Then, we describe their classical dual and define \emph{$(\sigma,\ell)$-orbit vectors}, which are then used to define the \emph{$q$-dual}. We complete the section by describing the $q$-dual for nRS codes and provide conditions under which the double $q$-dual recovers the original code. Then we make use of the field extensions $\F_q\leq \F_{q^a}\leq \F_{q^m}$ to introduce \emph{$\F_{q^a}$-linear skew cyclic and skew quasi-cyclic $\F_{q^m}$-codes} and motivate these definitions by showing that an appropriate choice of the evaluation vector makes the nRS codes fall within these classifications.

%%%  In Section~\ref{sec:nrs} we make use of the field extensions $\F_q\leq \F_{q^a}\leq \F_{q^m}$ to introduce \emph{$\F_{q^a}$-linear skew cyclic and skew quasi-cyclic $\F_{q^m}$-codes}. We then motivate these definitions by showing that an appropriate choice of the evaluation vector makes the nRS codes fall within these classifications.

In Section~\ref{sec:polycharacterization} we introduce the quotient skew polynomial rings $R_n$ and $P_n$, which are then used to alternatively describe $\F_{q^a}$-linear skew quasi-cyclic $\F_{q^m}$-codes as $P_n$-modules. Also, we provide an alternative description of $R_n$ whenever $a=1$ and $n=m$.

In Section~\ref{sec:structure1} we analyze the module structure of $\F_{q^a}[X;\sigma]$-modules using the Smith normal form at two different instances. Consider two $\F_{q^a}[X;\sigma]$-submodules $C$ and $D$ such that $D\leq C\leq \F_{q^m}[X;\sigma]^\ell$, and obtain a basis $\boldsymbol{c}_1^*, \boldsymbol{c}_2^*, \ldots, \boldsymbol{c}_{\xi}^*$ for $C$ such that $ \left\{  d_i \boldsymbol{c}_i^* : i =1, \ldots, \zeta  \right\} $ with $\zeta \le \xi$ is a basis of~$D$. Then we provide the $\F_{q^a}[X;\sigma]$-module structure of $C\big / D$.

The process of Section~\ref{sec:structure1} utilized two applications of the Smith normal form in order to find the structure constants of $\F_{q^a}$-linear skew quasi-cyclic $\F_{q^m}$-codes $\C$. However, the theory used was much more general than is necessary for the current setting of nonlinear skew quasi-cyclic codes. Section~\ref{sec:totaldivisors} is devoted to the classification of \emph{total divisors} of skew polynomials, culminating in Theorem~\ref{thm:grcdmakesmagic}, which allows us to, in Section~\ref{sec:structure2}, only require one application of the Smith normal form to find the structure constants.

In Section~\ref{sec:structure2} we analyze the algebraic structure of an $\F_{q^a}$-linear skew quasi-cyclic code $\C$ of length $\ell$. We consider the polynomial representation of the code, so that $\C$ is a $P_n$-submodule of $R_n^\ell$. We then use the Smith normal form to find a stacked basis $\boldsymbol{q}_1, \boldsymbol{q}_2,\ldots, \boldsymbol{q}_{r\ell}$ of $\F_{q^a}[X;\sigma]^{r\ell}$ and total divisors of $X^n-1$. These are used to obtain the $P_n$-module structure of $\C$ and to find an alternative generating set for $\C$. Besides, this stacked basis allow us to define a new inner product and dual space for $\C$.

\section{Preliminaries}\label{sec:prelim}
As the present article presents generalizations of cyclic codes and generalized Reed-Solomon codes, we will use this section to review the theory of Reed-Solomon codes and cyclic codes. We will, for the most part, state results without proof; the interested reader may refer to standard textbooks such as \cite{Macwilliams-Sloane:1977} or to the cited articles.

\subsection{Reed-Solomon Codes}
\begin{definition}
    Let $\bs \alpha = (\alpha_1,\alpha_2, \ldots, \alpha_n)$ be an ordered subset of distinct elements of the finite field $\F_q$, where $q$ is some prime power, and let $\bs \nu = (v_1, v_2, \ldots, v_n)$ be an ordered subset of nonzero elements of $\F_q$. The \emph{generalized Reed-Solomon code} of length $n$ and dimension $k$ and vector of multiples $\bs \nu$, denoted $\GRS_{\bs \nu}(\bs \alpha, k)$  is defined as
    \[\{\left(v_1 f(\alpha_1), v_2 f(\alpha_2), \ldots, v_n f(\alpha_n) \right) : f(X) \in \F_q[X] \text{ with } \deg(f) < k\}.\] In the case when $\bs \nu = (1, 1, \ldots, 1)$, we will call this simply a \emph{Reed-Solomon code}, which will be denoted by $\RS(\bs \alpha, k)$.
\end{definition}

These codes are widely studied and implemented since they meet the Singleton bound, i.e., there is an optimal trade-off between error-correcting capacity and dimension. Additionally, these codes have the following desirable properties. The proofs are standard and can be found in most coding theory textbooks.

\begin{lemma}
    Given a generalized Reed-Solomon code $\GRS_{\bs \nu}(\bs \alpha, k)$, the following hold.
    \begin{enumerate}
        \item[$(1)$] The minimum distance of $\GRS_{\bs \nu}(\bs \alpha, k)$ is $d = n - k + 1$, that is, it meets the Singleton bound.
        \item[$(2)$] The dual code of $\GRS_{\bs \nu}(\bs \alpha, k)$, with respect to the standard Euclidean inner product, is the generalized Reed-Solomon code with dimension $n-k$, the same evaluation vector $\bs \alpha$, but a different vector of multiples~$\bs{\nu'}$, $\GRS_{\bs{\nu'}}(\bs \alpha, n-k)$.
    \end{enumerate}
\end{lemma}

\subsection{Cyclic and Quasi-Cyclic Codes}
In this subsection let us fix a finite field $\F_q$ and some positive integer $n$ such that $\gcd(n,q) = 1$. Then we will fix the following notation $R_n := \F_q[X] / (X^n - 1)$ for the ring of degree $<n$ polynomials with the relation $X^n = 1$.

\begin{definition}
    A code $\C$ over $\F_q$ of length $n$ is called {\it cyclic} if whenever $(c_0, c_1, \ldots, c_n)$ is contained in $\C$, then $(c_n, c_1, \ldots, c_{n-1}) \in \C$, that is $\C$ is invariant under cyclic shifts. It is straightforward to show that cyclic codes are in one-to-one correspondence with the ideals of $R_n$.
\end{definition}

As $R_n$ is a commutative principal ideal ring, each cyclic code $\C$ contains a unique monic polynomial of smallest degree $g(X)$ (called the \emph{generator polynomial} of $\C$) such that the ideal generated by $g(X)$ is $\C$. It is straightforward (using the division algorithm) to see that $g(X) \mid (X^n - 1)$. Hence, define the polynomial $h(X) : = \frac{X^n -1}{g(X)}$. Note that $h$ is a polynomial of degree $n - \deg(g)$.

\begin{lemma}
    Let $\C$ be a cyclic code with generator polynomial $g(X)$ of degree $n-k$, and suppose that $h(X)$ is defined as above, then:
    \begin{enumerate}
        \item[$(1)$] $\C$ is a linear code of dimension $k$.
        \item[$(2)$] Let $\hat h(X) : = X^{k}h(X^{-1})$ be the \emph{reciprocal polynomial} of $h(X)$. Then $\hat h(X)$ is a generator polynomial for $\C^\perp$ under the standard Euclidean inner product.
    \end{enumerate}
\end{lemma}

There is an intimate connection between Reed-Solomon codes and cyclic codes (ideals of the ring $R_n = \F_{q}[X]/(X^n - 1)$) as shown in the following example.
\begin{example}
    Suppose that we have a finite field with $q$ elements, and choose $\bs \alpha = (1,\gamma, \gamma^2, \ldots, \gamma^{q-2})$ where $\gamma$ is a primitive element for the finite field $\F_q$ (note that in this case $n = q-1$). Then for $k < n$, $\RS(\bs \alpha, k)$ is a cyclic code.
\end{example}

\begin{definition}
\emph{Quasi-cyclic codes} are generalizations of cyclic codes which only repeat after cyclically shifting codewords some $\ell \geq 1$ positions, where this value $\ell$ is the smallest such number. In other words, for a word of length $n = m \cdot \ell$, if 
\[(c_0, c_1, \ldots, c_{\ell-1}, c_\ell, \ldots, c_{m \ell-1}) \in \C,\] then
\[(c_{(m-1) \ell}, \ldots, c_{m \ell-1}, c_0, \ldots, c_{(m-1) \ell - 1}) \in \C.\] 
\end{definition}
The reader is encouraged to consult \cite{Ling-Sole:2006, Ling-Sole:2001,Ling-Sole:2003, Ling-Sole:2005} for a wide-ranging overview of quasi-cyclic codes.

The invariance under $\ell$-shifts of the codewords permits a simpler notation of a word in a quasi-cyclic code as a matrix, as follows. 
\[\begin{pmatrix}
c_0 &c_1 &\cdots &c_{\ell-1}\\
c_{\ell} &c_{\ell+1} &\cdots &c_{2\ell - 1}\\
\vdots &\vdots &\ddots &\vdots\\
c_{(m-1)\ell} &c_{(m-1)\ell + 1} &\cdots &c_{m \ell - 1}
\end{pmatrix}\]
When the codewords are written in this way, it is evident that the code is invariant under cyclic ``row-shifts." This notation also suggests that every quasi-cyclic code is composed of cyclic codes, as is borne out in the following result.

\begin{lemma}
Let $R_m$ denote $\F_{q}[X]/(X^m -1)$. Then the quasi-cyclic codes of index $\ell$ and length $m \cdot \ell$ are in one-to-one correspondence with the $R_m$-submodules of the free $R_m$-module of rank $\ell$, $R_m^\ell$.
\end{lemma}

\subsection{Skew Polynomials and Skew (Quasi-)Cyclic Codes}
Skew polynomials are generalizations of polynomials first explored by Ore in \cite{Ore:1933}; in that article, the author showed that skew polynomials (to be formally defined below) are the most general polynomial-type ring for which the following two familiar properties of polynomials still hold:
\begin{enumerate}
    \item[$(1)$] $\deg(f(X) + g(X)) \leq \max\{\deg(f(X)), \deg(g(X))\}$,
    \item[$(2)$] $\deg(f(X)g(X)) = \deg(f(X)) + \deg(g(X))$.
\end{enumerate}
These two properties together assure that there must exist some type of division algorithm and Euclidean algorithm.

\begin{definition}
Let $R$ be a (possibly non-commutative) ring, $\sigma: R \to R$ an endomorphism of $R$, and $\delta: R \to R$ a $\sigma$-derivation of $R$ (i.e., an endomorphism such that $\delta(ab) = \sigma(a) \delta(b) + \delta(a) b$). Then the \emph{skew polynomial ring} $R[X; \sigma, \delta]$ is the (left) $R$-algebra of formal polynomials 
\[f(X) = \sum_{i = 0}^n a_i X^i\] with $a_i \in R$ with the addition defined component-wise and multiplication defined by the equation
\[Xa = \sigma(a)X + \delta(a).\]
\end{definition}

\begin{remark}
While the above is the original definition given by Ore, the present article assumes that for a prime-power $q = p^\ell$, $R = \F_{q^m}$ is a finite field, $\sigma$ is an $\F_{q}$-linear field automorphism, and $\delta = \mathbf{0}$ is the zero map. In this case, the skew polynomial ring $\F_{q^m}[X; \sigma]$ will often be considered as an $\F_{q}$-algebra.
\end{remark}

Skew polynomial rings have been extensively studied by many authors, particularly, Lam and Leroy in \cite{Lam-Leroy:1988}. The study of coding-theoretic applications of skew polynomials was initiated in 2009 by Boucher and Ulmer, \cite{Boucher-Ulmer:2009}, who examined skew polynomials as the basis for skew cyclic codes. Furthermore Liu, Kschischang, and Manganiello, \cite{Liu-Kschischang-Manganiello:2015}, and Martinez-Pe\~nas \cite{MartinezPenas:2018} studied Reed-Solomon type evaluation codes involving skew polynomial rings.

Skew polynomials retain many familiar properties of traditional polynomials; however, their non-commu\-ta\-ti\-vi\-ty necessitates more care in the implementation of those properties. For example,
\begin{proposition}{\cite{Ore:1933}}
    Let $R$ denote the skew polynomial ring $\F_{q^m}[X;\sigma]$, and say that $f(X)$ and $g(X)$ are skew polynomials with $g(X)$ nonzero. Then the following holds.
    \begin{enumerate}
        \item[$(1)$] $R$ is a right Euclidean domain. In other words, there are unique skew polynomials $q(X)$ and $r(X)$ where $\deg(r(X)) < \deg(g(X))$ such that 
        \[f(X) = q(X) g(X) + r(X).\] We say that $g(X)$ \emph{right divides} $f(X) - r(X)$, denoted $g(X) \mid_{r} (f(X) - r(X))$.

        \item[$(2)$] Given any two skew polynomials $f(X)$ and $g(X)$ in $R$, there is a unique monic \emph{greatest common right divisor} $\gcrd(f(X), g(X)) := d(X) \in R$ such that $d(X) \mid_r f(X)$ and $d(x) \mid_r g(x)$ and for any other $h(X)$ which simultaneously right divides $f(X)$ and $g(X)$, $h(X) \mid_r d(x)$. Moreover, there are polynomials $a(X)$ and $b(X)$ so that 
        \[d(X) = a(X) f(X) + b(X) g(X).\]

        \item[$(3)$] Given any two skew polynomials $f(X)$ and $g(X)$ in $R$, there is a unique monic \emph{least common left multiple} $\lclm(f(X), g(X)) := \ell(X)$ such that $f(x) \mid_r \ell(X)$ and $g(X) \mid_r \ell(X)$ such that for any other common left multiple, $h(X)$, $\ell(X) \mid_r h(X)$.

        \item[$(4)$] Given any left ideal $I$ of $R$, there is a polynomial $h(X)$ such that $R h(X) = I$, in other words, $R$ is a left principal ideal ring.
    \end{enumerate}
\end{proposition}

\begin{remark}
    In the above proposition, similar statements hold if we switch `left' for `right,' and vice-versa. An important property to note: even though $R$ is a left Euclidean domain and right Euclidean domain, when working with $R$ we must `choose a side.' For the purposes of this article, we will choose to work with $R$ as a right Euclidean domain as in the above proposition; however, the arguments of this paper also work mutatis mutandis for left Euclidean domains.
\end{remark}

Finally we note the following property of skew polynomials which will be useful in the exploration of skew quasi-cyclic codes.

\begin{lemma}
   Suppose that $n > 0$ such that $m \mid n$. Then the polynomial $X^n - 1$ commutes with every skew polynomial. As a result $R_n := \F_{q^m}[X; \sigma]/(X^n - 1)$ is a ring. 
\end{lemma}

Because skew polynomial rings admit a (one-sided) division algorithm, we may still define generalizations of cyclic codes and quasi-cyclic codes for $m\nmid n$; however we must navigate the submodules, rather than the ideals of the left $\F_q[X;\sigma]$-module \[R_n = \F_{q^m}[X;\sigma]/(X^n - 1),\] where $(X^n - 1)$ denotes the left $\F_{q^m}[X; \sigma]$-module generated by the skew polynomial $X^n - 1$.

\begin{definition}
    Let $\F_{q^m}$ be a finite field where $q$ is of prime power order. Define $R_n = \F_{q^m}[X, \sigma] / (X^n - 1)$, considered as a left $\F_{q^m}[X,\sigma]$-module. A \emph{skew cyclic code} $\C$ of length $n$ is then a left $\F_{q^m}[X,\sigma]$-submodule of $R_n$.
\end{definition}

\begin{remark}
    Cyclic codes are characterized by the property that whenever $(c_0, c_1, \ldots, c_{n-2}, c_{n-1})$ is a codeword, the cyclic shift $(c_{n-1}, c_0, c_1, \ldots, c_{n-2})$ is also a codeword. Using the relation $X^n = 1$ that is given by the left module quotient in the preceding definition, it is a simple exercise to see that if $(c_0, c_1, \ldots, c_{n-2}, c_{n-1})$ is a codeword in the skew cyclic code, then $\left(\sigma(c_{n-1}), \sigma(c_0), \sigma(c_1), \ldots, \sigma(c_{n-2})\right)$ must also be a codeword. 
\end{remark}
There are many aspects of skew cyclic codes that we leave unaddressed in this article; the interested reader may see Chapter 8 of the {\em Concise Encyclopedia of Coding Theory}, \cite{Gluesing-Luerssen:2021}.

A natural generalization of skew cyclic codes are skew quasi-cyclic codes, which were introduced by Abualrub, Ghrayeb, Aydin, and Siap in \cite{Abualrub-et-al:2010}.

\begin{definition}{(\cite{Abualrub-et-al:2010})}
Let $\F_{q^m}$ be a finite field of characteristic $p$, and let $\sigma$ be an automorphism of $\F_{q^m}$ with order $m$. A subset $\C$ of $\F_{q^m}^n$ is called a \emph{skew quasi-cyclic} code of length $n$, where $n = s\ell$, $m \mid s$, and index $\ell$ if 
\begin{enumerate}
    \item[$(1)$] $\C$ is a subspace of $\F_{q^m}^n$, and
    \item[$(2)$] if $\bs{c} = (a_{0,0}, a_{0,1}, \ldots, a_{0,\ell-1}, a_{1,0}, \ldots, a_{1,\ell-1}, \ldots, a_{s-1,\ell-1})\in \C$, then the word 
    \[T_{\ell,\sigma}(\bs{c}) = (\sigma(a_{s-1,0}), \ldots,\! \sigma(a_{s-1,\ell-1}),\! \sigma(a_{0,0}), \ldots,\! \sigma(a_{0,\ell-1}), \ldots,\! \sigma(a_{s-2,\ell-1})) \]
    belongs to $\C$. 
\end{enumerate}
\end{definition}
Property (2) above is more simply stated as that the codewords of a skew quasi-cyclic code are closed under a ``skew-shift" of index $\ell$. The authors assume $m \mid s$ so that $R_s = \F_{q^m}[X; \sigma]/(X^s - 1)$ is a ring in its own right, in which case we have the following characterization.

\begin{proposition}(\cite{Abualrub-et-al:2010}, Theorem 5)
    Say that $R_s = \F_{q^m}[X; \sigma]/(X^s - 1)$, and the parameters $s$, $\ell$, and $n$ are as above. Then the skew quasi-cyclic codes are in one-to-one correspondence with the $R_s$-submodules of $R_s^\ell$.
\end{proposition}

\section{Nonlinear Reed-Solomon Codes}\label{sec:nrs}
First we will, in a natural way, generalize the notion of Reed-Solomon code. We want to explore the changes in the Reed-Solomon codes when evaluation points are not taken in the base field but rather in a field extension.
%
\begin{definition}
    Let $\F_{q^m}$ be a field extension of $\F_q$, consider the evaluation vector $\boldsymbol{\alpha}=\left(\alpha_1, \alpha_2, \ldots, \alpha_n\right) \in \F_{q^m}^n$, and let $k>0$. We define the \emph{nonlinear Reed-Solomon code}, $\nRS_q(\boldsymbol{\alpha},k)$, of length $n$ and parameter $k$
    \[\{\left(f(\alpha_1), f(\alpha_2), \ldots, f(\alpha_n) \right) : f(X) \in \F_q[X] \text{ with } \deg(f) < k\}.\]
\end{definition}

\begin{remark}
    Given $f\in \F_q[X]$ with $\deg(f) < k$ and $\boldsymbol{\alpha}$ an evaluation vector, consider the minimal polynomial $\min(\alpha_i; \F_q)$ of each $\alpha_i$ over $\F_q$. Perform long division of $f(X)$ by $\min(\alpha_i; \F_q)$ and evaluate the residue at $\alpha_i$. The resulting codeword equals the one obtained in the preceding definition. This strategy to define an evaluation code is used in \cite{prcyu}.
\end{remark}

\begin{remark}
    To avoid creating a code with repetition, the length of the code $\C$ is restricted by $n\leq q^m$. For a meaningful  $\nRS_q(\boldsymbol{\alpha},k)$ code, we also need an upper bound on $k$. In particular, $k\leq mn$ is required if we are expecting $\F_q$-linear independent rows in our canonical generator matrix, so that $k$ matches with the dimension of the code. In the following, we will discuss a precise upper bound for $k$. Note that the generator matrix for $\nRS_q(\bs \alpha, k)$ is identical to the generator matrix of the conventional linear Reed-Solomon code. However, we only consider the $\F_q$-linear combinations of the rows, which are vectors of length $n$ in $\F_{q^m}$.
\end{remark}

\subsection{Bounds for \texorpdfstring{$k$}{k} and \texorpdfstring{$d$}{d}}

With $n$ fixed, we want to determine a bound for the dimension $k$. Notice that for $\alpha\in \F_{q^m}$, $\lambda_0, \ldots, \lambda_t\in \F_q$, and $a\geq 0$,
%
\begin{align*}
    \sum_{i=0}^t \lambda_i \alpha^i =0 \iff \left(\sum_{i=0}^t \lambda_i \alpha^i\right)^{q^a}=0
    \iff\!\! \sum_{i=0}^t \lambda_i \left( \alpha^{q^a} \right)^i= \sum_{i=0}^t \lambda_i^{q^a} \left( \alpha^{q^a} \right)^i =0.
\end{align*}
That is, $\C$ has the same dimension as the nRS code with the evaluation vector $\boldsymbol{\alpha}'$, where $\boldsymbol{\alpha}'$ results from shortening $\boldsymbol{\alpha}$ to have at most one representative from each cyclotomic coset.

Thus, assume that the entries of $\boldsymbol{\alpha}$ all come from distinct cyclotomic cosets, i.e., $\boldsymbol{\alpha}'=\boldsymbol{\alpha}$. Let $G$ be a generator matrix of $\C$ and $n$ the length of $\boldsymbol{\alpha}$. The entries of each column of $G$ span an $\F_q$-subspace of $\F_{q^m}$ of dimension at most $m$. Then the row reduction procedure described in Chapter 8 of \cite{Huffman:2013} provides a generator matrix $G'$ for $\C$ and a sequence $(k_i, i=1, \ldots, n)$ with $k_1+\ldots+k_n=k$ such that, in the $i$-th column, the $k_i$ entries from rows $k_1+\ldots+k_{i-1}+1$ to $k_1+\ldots+k_{i}$ are $\F_q$-linearly independent, and all entries in rows below are zero.

Let $i\in \{1, \ldots, n\}$. Notice that an $\F_q$-linear combination of the rows of the generator matrix $G$ outputs a vector where the $i$-th coordinate is the evaluation of some fixed polynomial $f$ of degree less than $k$ at $\alpha_i$. Recall that the minimal polynomial of $\alpha_i$ over $\F_q$, $\operatorname{min}(\alpha_i;\F_q)$ vanishes at $\alpha_i$, but not at any other $\alpha_j$. This provides the sharp bound $k_i \leq \deg\left( \operatorname{min}(\alpha_i;\F_q)  \right)$. Furthermore, since the product of distinct minimal polynomials is a divisor of $X^{q^m}-X$, we have that
%
\begin{eqnarray*}
k_1 + \ldots+k_n & \leq & \sum_{i=1}^n \deg\left( \operatorname{min}
    (\alpha_i;\F_q)  \right) = \deg\left( \prod_{i=1}^n\operatorname{min}
       (\alpha_i;\F_q)  \right)\\ & \leq & \deg \left(X^{q^m}-X  \right)=q^m.
\end{eqnarray*}
In the above, we may achieve equality iff $\boldsymbol{\alpha}$ is chosen to have one entry from each cyclotomic coset, in which case the product of minimal polynomials is $X^{q^m}-X$.
%, and we are allowed to choose $k_1, \ldots, k_n$ so that $k=q^m$.

Now consider the Hamming distance $d$ of $\nRS_q(\boldsymbol{\alpha},k)$. We first note that in the general case, i.e., multiple elements from the same cyclotomic coset are allowed in the evaluation vector, given the length $n$ and the dimension $k$, the Singleton bound from Chapter~8 of \cite{Huffman:2013} provides the bound
%
\[ d\leq n - \left\lceil\dfrac{k}{m}\right\rceil + 1. \]
We now give examples of evaluation vectors $\boldsymbol{\alpha}$ for which equality is attained.

\begin{example}
    Consider the extension $\F_2\leq \F_{16}$, $\alpha$ a primitive element of $\F_{16}$, and the evaluation vector $\boldsymbol{\alpha}= \left( \alpha, \alpha^2, \alpha^4 , \alpha^8 \right)$. Then  $\nRS_2(\boldsymbol{\alpha},k)$ with $k=4$ has length $n=4$, dimension $k=4$ and distance $d=4$, since the minimal polynomial of $\alpha$ over $\F_2$ has degree $4$. This code meets equality for the Singleton bound.
\end{example}

\begin{example}
    Let $m=1$ and $\boldsymbol{\alpha}\in \F_q^n$ an evaluation vector. Then $\nRS_q(\boldsymbol{\alpha},k)$ is a linear Reed-Solomon code, which has distance $d=n-k+1$. This code meets equality for the Singleton bound.
\end{example}

\begin{lemma}
    Suppose that $k\leq \min_{i=1, \ldots, n} \deg \left( \min(\alpha_i;\F_q) \right)$.\!\! Then $\nRS_q(\boldsymbol{\alpha},k)$ meets the Singleton bound.
\end{lemma}
\begin{proof}
    Since the minimum polynomials $\min(\alpha_i;\F_q)$ have degree at most $m$, it follows that $k\leq m$. Then, the Singleton bound establishes that $d\leq n - \left\lceil\frac{k}{m}\right\rceil + 1= n$.

    Moreover, any nonzero polynomial vanishing at $\alpha_i$ has degree higher or equal than $\deg \left( \min(\alpha_i;\F_q) \right)$. Hence no nonzero codeword can have a zero coordinate, thus ${d=n}$.
\end{proof}

\begin{theorem}
    Suppose that all coordinates of the evaluation vector $\boldsymbol{\alpha}$ come from different cyclotomic cosets with full size $m$. Then $\nRS_q(\boldsymbol{\alpha},k)$ meets the Singleton bound.
\end{theorem}

\begin{proof}
    Let $G$ be the generator matrix for  $\nRS_q(\boldsymbol{\alpha},k)$. Then performing the row reduction procedure described in \cite[Chapter 8]{Huffman:2013}, we may find an equivalent generator matrix $G'$ such that the minimum distance $d$ is less than or equal to $n$ minus the number of zeros in the bottom row of $G'$. Furthermore, this bottom row has a leading zero for every block of $G'$ minus the bottom most block. As every block has size less than or equal to $[\F_{q^m}:\F_q]=m$, this gives at least $\lceil \frac{k}{m} \rceil - 1$ leading zeros for the bottom row. This shows the Singleton bound $d\leq n - \left\lceil\frac{k}{m}\right\rceil + 1$.

    Let $\boldsymbol{c}=\left(f(\alpha_1),\ldots, f(\alpha_n) \right)$ with $\deg (f)<k$. If $\boldsymbol{c}$ contains $\lceil \frac{k}{m}\rceil$ zeros, then $f$ has $\lceil \frac{k}{m}\rceil$ of the $\alpha_i$ as zeros. Without loss of generality, let these zeros be $\alpha_1, \ldots, \alpha_r$ with $r=\lceil \frac{k}{m} \rceil$. Then
    $\deg (f)\ge \deg \left( \min\left(\alpha_1, \ldots, \alpha_r ; {\F_q}\right)\right).$
    Since all $\alpha_i$ belong to distinct cyclotomic cosets of maximum size $m$, we also have \[\min\left(\alpha_1, \ldots, \alpha_r ; {\F_q}\right)= \prod_{i=1}^r \min\left(\alpha_i; {\F_q} \right),\] and
    \begin{align*}
        \deg (f)\ge  \deg \left( \min\left(\alpha_1, \ldots, \alpha_r ; {\F_q}\right)\right)= &\sum_{i=1}^r \deg\left(\min\left(\alpha_i; {\F_q}\right) \right)\\ = &m \left\lceil \frac{k}{m}\right\rceil\geq m \frac{k}{m}=k.
    \end{align*}
    Thus, $f$ must have degree at least $k$, which is a contradiction. Therefore, no nonzero codeword $\boldsymbol{c}$ has more than $\lceil \frac{k}{m}\rceil-1$ zeros while the last row of $G'$ has exactly $\lceil \frac{k}{m}\rceil-1$ zeros, and $\nRS_q(\boldsymbol{\alpha},k)$ meets the Singleton bound.
\end{proof}
%
%\begin{conjecture}
%    The Singleton bound meets equality if and only if $\ell=1$ or $m=1$.
%\end{conjecture}
%
\subsection{Dual Codes and Nonlinear Skew Cyclic and Quasi-Cyclic Codes}

In this section we will study the dual codes of our $\nRS_q(\boldsymbol{\alpha},k)$. This will also motivate us to define $(\sigma, \ell)$-orbit vectors (Definition \ref{def:orbit}) and nonlinear skew cyclic and skew quasi-cyclic codes (Definition \ref{def:skewversions}).

Throughout this section we will assume that $\boldsymbol{\alpha}$ is a vector with distinct coordinates and that the code $\nRS_q(\boldsymbol{\alpha},k)$ has dimension $k$.

\begin{definition}
    Let $\C$ be an $\F_q$-linear $\F_{q^m}$-code of length $n$. The \emph{classical dual} (or just \emph{dual}) is the $\F_{q^m}$-linear code given by
    %
    \[\C^\perp =\left\{ \boldsymbol{x}\in \F_{q^m}^n : \langle \boldsymbol{c},\boldsymbol{x}\rangle=0 \mbox{ for all } \boldsymbol{c}\in \C \right\}. \]
\end{definition}

\begin{proposition} \label{prop3.7}
    The dual of the $\nRS_q(\boldsymbol{\alpha},k)$ code of length $n$ with $k\le n$ is the (linear) generalized Reed-Solomon code $GRS_{\boldsymbol{u}}(\boldsymbol{\alpha},n-k)$, where the vector of multiples $\boldsymbol{u}\in \F_q^{n}$ is given by
    %
    \[u_i = \left( \prod_{j\ne i} (\alpha_i-\alpha_j) \right)^{-1}.\]
\end{proposition}

\begin{proof}
    Consider the following polynomials:
    %
    \[L(X)=\prod_{i=1}^n (X-\alpha_i) , \qquad L_i(X)=\dfrac{L(X)}{(X-\alpha_i)},\,\, i=1,\ldots, n.\]
    By definition, $u_i = \left( \prod_{j\ne i} (\alpha_i-\alpha_j) \right)^{-1}=L_i(\alpha_i)^{-1}$.
    
    Consider $\boldsymbol{x}=f(\boldsymbol{\alpha})$ with $\deg (f)<k$ and $\boldsymbol{y}=\boldsymbol{u}g(\boldsymbol{\alpha})$ with $\deg (g)<n-k$. We want to show that $\langle\boldsymbol{x}, \boldsymbol{y}\rangle=0$.

    Note that $(fg)(X)$ is a polynomial of degree at most $n-2$. By the Lagrange interpolation method,
    \[f(X)g(X)=\sum_{i=1}^n \dfrac{L_i(X)}{L_i(\alpha_i)}f(\alpha_i)g(\alpha_i).\]
    Besides, for any $i$, $L_i(X)$ is a monic polynomial of degree $n-1$. So, by considering the coefficients of $X^{n-1}$ we get:
    %
    \[0=\sum_{i=1}^n \dfrac{1}{L_i(\alpha_i)}f(\alpha_i)g(\alpha_i)=\sum_{i=1}^n f(\alpha_i)\left(u_ig(\alpha_i)\right)= \langle\boldsymbol{x}, \boldsymbol{y}\rangle.\]
    As $\C^\perp$ in an $\F_{q^m}$-vector space with $\dim_{\F_{q^m}}(\C^\perp) = n-k$, this establishes $\C^\perp =GRS_{\boldsymbol{u}}(\boldsymbol{\alpha},n-k)$.
\end{proof}

\begin{remark}
    As an immediate consequence of Proposition \ref{prop3.7}, the double dual of $\C = \nRS_q(\boldsymbol{\alpha},k)$ is $(\C^\perp)^\perp = \RS(\bs \alpha, k)$, a (linear) Reed-Solomon code. Thus, in general, $(\C^\perp)^\perp =\C$ fails. In the following, we will try to fix this by means of an alternative definition of the dual code.
\end{remark}

We start with the definition of $(\sigma,\ell)$-orbit vectors.

\begin{definition}\label{def:orbit}
    Let $\sigma\in \operatorname{Aut}(\F_{q^m})$, $\ell\in \mathbb{Z}_{>0}$. A vector $\boldsymbol{c}=(c_1,\ldots,c_{n\ell})\in \F_{q^m}^{n\ell}$ is said to be a \emph{$(\sigma,\ell)$-orbit vector} if, up to a possible reordering of coordinates, $c_{i+\ell}=\sigma(c_i)$ for $i=1,\ldots,n\ell$, where indices are interpreted modulo $n\ell$. We can view $(\sigma,\ell)$-orbit vectors $\boldsymbol{c}$ as $n\times \ell$ matrices as follows:
    %
    %{\color{red} NOTE: $\ell$ is the number of cyclotomic cosets to take.} 
    %
    \[\boldsymbol{c}=\begin{pmatrix}
            \sigma^{n}(c_1)=c_1 & \cdots & \sigma^{n}(c_\ell)=c_\ell \\
            \sigma(c_1)         & \cdots & \sigma(c_\ell)            \\
            \sigma^2(c_1)       & \cdots & \sigma^2(c_\ell)          \\
            \vdots              &        & \vdots                    \\
            \sigma^{n-1}(c_1)   & \cdots & \sigma^{n-1}(c_\ell)      \\
        \end{pmatrix}.\]
    Since $\sigma^{n}(c_i)=c_i$ for all $i = 1, \ldots, n\ell$, the definition above requires that $n$ is a multiple of the order of $\sigma$.
    Unless otherwise stated, whenever we consider a collection of $(\sigma,\ell)$-orbit vectors, we will assume that the required reordering of coordinates is the same for every vector.
\end{definition}

In the construction of nonlinear Reed-Solomon codes, a useful choice of evaluation vector $\boldsymbol{\alpha}$ is the collection $\Gamma$ of all elements in $\F_{q^m}$ whose minimal polynomial has degree $m$. $\Gamma$ can be partitioned as $\Gamma=\Gamma_1\cup\ldots\cup\Gamma_\ell$, where each $\Gamma_i$ is a cyclotomic coset of size $m$. Furthermore, given representatives $\beta_i\in \Gamma_i$, we can describe $\Gamma_i$ as
%
\[\Gamma_i=\left\{\beta_i=\sigma^{m}(\beta_i), \sigma(\beta_i), \sigma^{2}(\beta_i) , \ldots, \sigma^{m-1}(\beta_i) \right\}.\]

Now we reorder the coordinates to write $\boldsymbol{\alpha}$ as below and exhibit $\boldsymbol{\alpha}$ as a  $(\sigma,\ell)$-orbit vector of length $m \ell$:
%
\[\boldsymbol{\alpha}=\begin{pmatrix}
        \beta_1               & \beta_2               & \cdots & \beta_\ell               \\
        \sigma(\beta_1)       & \sigma(\beta_2)       & \cdots & \sigma(\beta_\ell)       \\
        \sigma^2(\beta_1)     & \sigma^2(\beta_2)     & \cdots & \sigma^2(\beta_\ell)     \\
        \vdots                & \vdots                & \ddots & \vdots                   \\
        \sigma^{m-1}(\beta_1) & \sigma^{m-1}(\beta_2) & \cdots & \sigma^{m-1}(\beta_\ell)
    \end{pmatrix}.\]
This then motivates the following generalization of skew cyclic and skew quasi-cyclic codes.

\begin{definition}\label{def:skewversions}
    Let $\F_q\leq \F_{q^a} \leq \F_{q^m}$ be field extensions and $\sigma: \F_{q^m}\rightarrow \F_{q^m}$ an $\F_q$-automorphism. An \emph{$\F_{q^a}$-linear skew cyclic $\F_{q^m}$-code} is an $\F_{q^a}$-subspace, $\C$, of $\F_{q^m}^n$ with the property that
    %
    \begin{equation*}%\label{skew_cyclic_cond}
        (a_0,\ldots,a_{n-1})\in \C \implies (\sigma(a_{n-1}),\sigma(a_0),\ldots,\sigma(a_{n-2}))\in \C.
    \end{equation*}

    A code $\C$ of length $n\ell$ is called an \emph{$\F_{q^a}$-linear skew quasi-cyclic $\F_{q^m}$-code of index $\ell$} (for short, $\sigma$-QC code) if it is an $\F_{q^a}$-subspace of $\F_{q^m}^{n\ell}$ and, up to reordering coordinates, it is invariant under shift of codewords by $\ell$ positions composed with $\sigma$ coordinatewise, and this value $\ell$ is the smallest such number.
\end{definition}

\begin{remark}
    This invariance under a shift of codewords by $\ell$ positions composed with the field automorphism $\sigma$ applied coordinatewise is equivalent to
    %
    \[\sigma(s(\boldsymbol{c}))=\begin{pmatrix}
            \sigma(c_{(n-1)\ell}) & \cdots & \sigma(c_{n\ell-1}) \\
            \sigma(c_{0})   & \cdots & \sigma(c_{\ell-1})   \\
            \vdots            &        & \vdots                 \\
            \sigma(c_{(n-2)\ell}) & \cdots & \sigma(c_{(n-1)\ell-1})
        \end{pmatrix}\in \C \mbox{ for all } \boldsymbol{c} \in \C. \]

    Furthermore, these $\F_{q^a}$-linear skew quasi-cyclic $\F_{q^m}$-codes are related to codes studied extensively in the literature. For example, if $a=m$ and  $\sigma=\operatorname{id}$ we obtain quasi-cyclic codes, studied in \cite{Ling-Sole:2001}. If $a=m$ and $\sigma\ne \operatorname{id}$ we obtain linear skew QC-codes, that were introduced in \cite{Abualrub-et-al:2010}. %If $\ell=1$ we obtain $\F_q$-linear skew cyclic $\F_{q^m}$-codes, that were introduced in [to FILL].
\end{remark}

\begin{theorem}\label{thm-nrshasorbits}
    Suppose that $\boldsymbol{\alpha}$ is a $(\sigma,\ell)$-orbit vector of length $n\ell$. Then the nonlinear Reed-Solomon code
    %
    \[\nRS_q(\boldsymbol{\alpha},k) = \{\left(f(\alpha_1), f(\alpha_2), \ldots, f(\alpha_{n \ell}) \right) : f(X) \in \F_q[X] \text{ with } \deg(f) < k\}.\]
    %
    consists of $(\sigma,\ell)$-orbit vectors and is an $\F_{q}$-linear skew quasi-cyclic $\F_{q^m}$-code of index $\ell$.
    %is an $F_{q^t}$-linear skew quasi-cyclic code of index s.
\end{theorem}

\begin{proof}
    Let $f(X)\in \F_q[X]$ with degree less than $k$. Since $\boldsymbol{\alpha}$ is a $(\sigma,\ell)$-orbit vector, we have that $\alpha_{i+\ell}=\sigma(\alpha_i)$ for $i=1,\ldots,n\ell$. Then
%
    \begin{align*}
        f(\alpha_{i+\ell})=f(\sigma(\alpha_i))=\sigma(f(\alpha_i)),
    \end{align*}
    where the last equality holds because $\sigma$ fixes the coefficients of $f$, as these belong to $\F_q$.
\end{proof}

\begin{remark}
    For $\ell=1$, Theorem \ref{thm-nrshasorbits} means that if an $\nRS_q(\boldsymbol{\alpha},k)$ code $\C$ is constructed using a cyclotomic coset of full length $m$,
    %
    \[\boldsymbol{\alpha}=\left(\alpha, \alpha^q, \alpha^{q^2}, \ldots, \alpha^{q^{m-1}} \right),\]
    %
    then this $\nRS_q(\boldsymbol{\alpha},k)$ code is an $\F_q$-linear skew cyclic $\F_{q^m}$-code.
\end{remark}

We are now ready to propose an alternative dual code with the help of $(\sigma,\ell)$-orbit vectors.

\begin{definition}
    Let $\C$ be an $\F_q$-linear $\F_{q^m}$-code of length $n\ell$. The \emph{$q$-dual of $\C$} is the $\F_q$-linear vector space defined as
    %
    \[\C_q^\perp := \C^\perp \cap \mathcal{S}, \mbox{ where } \mathcal{S}:=\left \{ \boldsymbol{c}\in \F_{q^{m}}^{n\ell} :  \boldsymbol{c} \mbox{ is }(\sigma, \ell)\mbox{-orbit vector}\right\}.\]
\end{definition}

\begin{remark}
    The collection $\mathcal{S}$ of $(\sigma,\ell)$-orbit vectors is an $\F_q$-vector space since $\sigma$ fixes every element in $\F_q$, thus, given any $(\sigma,\ell)$-orbit vector $\boldsymbol{c}$, 
    %
    \[\sigma(k c_{i})= \sigma(k)\sigma(c_i)=kc_{i+\ell}, \mbox{ for all } k\in \F_q, i=1, \ldots, n\ell.\]
\end{remark}

%\begin{remark}
%    $m$ divides $n$...
%\end{remark}
\begin{proposition}\label{theorem-cqperpsubsetcperp}
    Let $\C= \nRS_q(\boldsymbol{\alpha},k)$ and suppose that $\boldsymbol{\alpha}$ is a $(\sigma,\ell)$-orbit vector of length $n\ell$. Then the $\F_q$-linear $\F_{q^m}$-code
    %
    \[\mathcal{V} := \{\left(u_1f(\alpha_1), u_2f(\alpha_2), \ldots, u_{n\ell} f(\alpha_{n\ell}) \right) : f(x) \in \F_q[X]\! \text{ with }\! \deg(f) < n\ell- k\},\]
    %
    consists of $(\sigma,\ell)$-orbit vectors and is an $\F_{q}$-linear skew quasi-cyclic $\F_{q^m}$-code of index~$\ell$. Here $u_i$ is defined as
    \[u_i = \left( \prod_{j\ne i} (\alpha_i-\alpha_j) \right)^{-1}, \quad i = 1,\ldots, n\ell.\]
    %
\end{proposition}

\begin{proof}
    %Recall that $\C_q^\perp$ is described as:
    For any $i\in\{1,\ldots,n\ell\}$ set $i'=i+\ell$. Since $\boldsymbol{\alpha}$ is a $(\sigma,\ell)$-orbit vector, we have that $\sigma(\alpha_i)=\alpha_{i'}$. Notice that the map $i\mapsto i'$ is a permutation in $S_{n \ell}$. Then, by using the automorphism~$\sigma$
    \begin{align*}
        \sigma(u_i) & = \sigma\left(\left( \prod_{j\ne i} (\alpha_i-\alpha_j) \right)^{-1}\right)=\left( \prod_{j\ne i} \sigma(\alpha_i-\alpha_j) \right)^{-1}\\ &=\left( \prod_{j\ne i} (\sigma(\alpha_i)-\sigma(\alpha_j)) \right)^{-1}
                    =\left( \prod_{j\ne i} \left(\alpha_{i'}-\alpha_{j'}\right) \right)^{-1}\\ &= \left( \prod_{j'\ne i'} \left(\alpha_{i'}-\alpha_{j'}\right) \right)^{-1}=u_{i'}.
    \end{align*}
    Hence,
    \[\sigma\left( u_i f(\alpha_i) \right)=\sigma\left( u_i \right)\sigma\left( f(\alpha_i) \right)=\sigma\left( u_i \right)f\left( \sigma(\alpha_i) \right)=u_{i'}f\left( \alpha_{i'} \right).\]
\end{proof}

\begin{proposition}
    Let $\C= \nRS_q(\boldsymbol{\alpha},k)$, $\sigma$ the Frobenius automorphism with $\sigma (\alpha) =\alpha^p$, and suppose that $\boldsymbol{\alpha}$ is a $(\sigma,\ell)$-orbit vector of length $n\ell$ with distinct coordinates and $k\le n\ell$. Then $\C_q^\perp = \mathcal{V}$, where $\mathcal{V}$ is defined as in Proposition~\ref{theorem-cqperpsubsetcperp}.
\end{proposition}

\begin{proof}
    By Propositions \ref{prop3.7} and \ref{theorem-cqperpsubsetcperp}, $\mathcal{V}
        \subseteq\C_q^\perp $. Let $\boldsymbol{c}\in \C_q^\perp$. There exists $f\in \F_{q^m}[X]$  such that $\boldsymbol{c}=\left(u_1f(\alpha_1),\ldots,u_{n\ell}f(\alpha_{n\ell})\right)$ and $\deg (f)<n\ell-k$. We want to show that $f\in \F_q[X]$. Write $f(X)=\sum_{j=0}^{n\ell-1}f_jX^j$. Let $i\in \{1, \ldots, n\ell\}$ and set $i'=i+\ell$. Notice that
    %
    \begin{align*}
        \boldsymbol{c}\mbox{ is a }(\sigma,\ell)\mbox{-orbit vector} & \implies \sigma\left( u_i f(\alpha_i) \right)=u_{i'}f\left( \alpha_{i'} \right)                                            \\
        & \implies \sigma\left( u_i \right)\sigma\left( f(\alpha_i) \right)=\sigma\left( u_i \right)f\left( \sigma(\alpha_i) \right) \\
        & \implies \sigma\left( f(\alpha_i) \right)=f\left( \sigma(\alpha_i) \right).
    \end{align*}
    %
    This fact combined with $f(X)=\sum_{j=0}^{n\ell -1}f_jX^j$ implies that for all $i=1, \ldots, n\ell $
    %
    \[\sum_{j=0}^{n\ell -1}f_j\sigma(\alpha_i)^j=f\left( \sigma(\alpha_i) \right)=\sigma\left( f(\alpha_i) \right)=\sum_{j=0}^{n\ell -1}\sigma(f_j)\sigma(\alpha_i)^j.\]

    By subtracting, we get that
    \[\sum_{j=0}^{n\ell-1}\left[\sigma(f_j)-f_j\right]\sigma(\alpha_i)^j=0.\]

    Thus, the polynomial $h(X)=\sum_{j=0}^{n\ell -1}[\sigma(f_j)-f_j]X^j$ has $\sigma(\alpha_1), \ldots, \sigma(\alpha_{n\ell})$ as roots, all of which are distinct, but $h(X)$ has degree less than $n \ell$. It follows that $h(X)$ is zero and $\sigma(f_j)=f_j$ for all $j=0, \ldots, n\ell -1$. Thus, $f_j$ belongs to the fixed field of $\sigma$, i.e., $f_j\in \F_q$ and $f(X)\in \F_q[X]$.%
\end{proof}

\begin{theorem}
    Let $\C= \nRS_q(\boldsymbol{\alpha},k)$, $\sigma$ the Frobenius automorphism with $\sigma (\alpha) =\alpha^p$, and suppose that $\boldsymbol{\alpha}$ is a $(\sigma,\ell)$-orbit vector of length $n\ell$ with distinct coordinates and $k\le n\ell$. Then %and suppose that $\boldsymbol{\alpha}$ is a $(\sigma,\ell)$-orbit vector.
    %
    \[ \left( \C_q^\perp \right)_q^\perp = \C. \]
\end{theorem}

\begin{proof}
    %By the calculations in the proof of \ref{theorem-cqperpsubsetcperp}, $\langle \boldsymbol{x},\boldsymbol{c}\rangle=0$, for all $\boldsymbol{c}\in \C$, $\boldsymbol{x}\in \C_q^\perp$. This is, $\C \subset \left( \C_q^\perp \right)_q^\perp.$
    Let $\boldsymbol{c}\in \C$. By Theorem \ref{thm-nrshasorbits}, $\boldsymbol{c}$ is a $(\sigma,\ell)$-orbit vector. Besides, for any $\boldsymbol{x}\in \C_q^\perp\subseteq \C^\perp$ we have $\langle \boldsymbol{c} , \boldsymbol{x}\rangle=0$. Hence, $\boldsymbol{c}\in \left( \C_q^\perp \right)_q^\perp$.

    Notice that
    %
    \begin{align*}
        & \langle \C_q^\perp \rangle_{\F_{q^m}}\\ & \quad =  \langle \left\{\left(u_1f(\alpha_1), u_2f(\alpha_2), \ldots, u_{n\ell}f(\alpha_{n \ell}) \right)\! :\! f(x) \in \F_q[X], \deg(f) < n\ell - k\right\} \rangle_{\F_{q^m}} \\   
        & \quad = \left\{\left(u_1f(\alpha_1), u_2f(\alpha_2), \ldots, u_{n\ell} f(\alpha_{n\ell}) \right) : f(x) \in \F_{q^m}[X], \deg(f) < n\ell- k\right\}                         \\
        & \quad = \C^\perp.
    \end{align*}

    This gives us a description for the dual and $q$-dual of $\C_q^\perp$:
    %
    \begin{align*}
        \left( \C_q^\perp \right)^\perp & = \left(\langle \C_q^\perp\rangle_{\F_{q^m}}\right)^\perp= \left(\C^\perp\right)^\perp=\langle \C\rangle_{\F_{q^m}} \\
                                        & = \left\{\left(f(\alpha_1), f(\alpha_2), \ldots, f(\alpha_{n\ell}) \right) : f(x) \in \F_{q^m}[X], \deg(f) < k\right\}.
    \end{align*}
    Thus,
    \begin{align*}
        \left( \C_q^\perp \right)_q^\perp & = \left( \C_q^\perp \right)^\perp \cap \left \{ \boldsymbol{c}\in \F_{q^{m}}^n :  \boldsymbol{c} \mbox{ is }(\sigma,\ell)\mbox{-orbit vector}\right\} \\
                                          & = \left\{\left(f(\alpha_1), f(\alpha_2), \ldots, f(\alpha_{n \ell}) \right) : f(x) \in \F_{q^m}[X], \deg(f) < k\right\} \cap \mathcal{S}.
    \end{align*}

    Let $\boldsymbol{c}\in \left( \C_q^\perp \right)_q^\perp$. There exists a polynomial $f\in \F_{q^m}[X]$ with $\deg (f)<k\le n\ell $ such that $\boldsymbol{c}=\left(f(\alpha_1),\ldots,f(\alpha_{n \ell})\right)$. We want to show that $f\in \F_q[X]$. Write $f(X)=\sum_{j=0}^{n\ell -1}f_jX^j$. Let $i\in \{1, \ldots, n\ell\}$. Notice that
    %
    \begin{align*}
        \boldsymbol{c}\mbox{ is a }(\sigma,\ell)\mbox{-orbit vector} \implies \sigma\left( f(\alpha_i) \right)=f\left( \sigma(\alpha_i) \right).
    \end{align*}
    %
    This fact combined with $f(X)=\sum_{j=0}^{n\ell -1}f_jX^j$ implies that for all $i=1, \ldots, n\ell$
    %
    \[\sum_{j=0}^{n\ell-1}f_j\sigma(\alpha_i)^j=f\left( \sigma(\alpha_i) \right)=\sigma\left( f(\alpha_i) \right)=\sum_{j=0}^{n\ell-1}\sigma(f_j)\sigma(\alpha_i)^j.\]
    By subtracting, we get that
    \[\sum_{j=0}^{n\ell -1}\left[\sigma(f_j)-f_j\right]\sigma(\alpha_i)^j=0.\]
    Hence, the polynomial $h(X)=\sum_{j=0}^{n\ell-1}[\sigma(f_j)-f_j]X^j$ has $\sigma(\alpha_1), \ldots, \sigma(\alpha_{n\ell})$ as roots, all of which are distinct, but $h(X)$ has degree less than $n\ell$. It follows that $h$ is zero and $\sigma(f_j)=f_j$ for all $j=0, \ldots, n\ell -1$. Thus, $f_j$ belongs to the fixed field of $\sigma$, i.e., $f_j\in \F_q$ and $f(X)\in \F_q[X]$.
\end{proof}

\section{Nonlinear Skew Cyclic and Nonlinear Skew Quasi-Cyclic Codes as Polynomials}\label{sec:polycharacterization}
%\section{Polynomial characterization of nonlinear skew cyclic and nonlinear skew quasi-cyclic codes}\label{sec:polycharacterization}
Consider the field extensions $\F_q \leq \F_{q^a}\leq \F_{q^m}$, $\sigma$ the Frobenius automorphism with $\sigma(\alpha)=\alpha^q$, and $n$ a multiple of $m$. Notice that $(X^n-1)$, the left-sided ideal generated by $X^n-1$, is a two-sided ideal of $\F_{q^a}[X;\sigma]$ and $\F_{q^m}[X;\sigma]$, respectively. Denote,
%Consider the field extensions $\F_q \leq \F_{q^a}\leq \F_{q^m}$ and $n$ a multiple of $m$. Notice that $_\bullet(X^n-1)$, the left-sided ideal generated by $X^n-1$, is a two-sided ideal of $\F_{q^a}[X;\sigma]$ and $\F_{q^m}[X;\sigma]$, respectively. We will write $_\bullet(X^n-1)$ as $(X^n-1)$. Denote,
\[R_n := \F_{q^m}[X;\sigma]/\left( X^n-1\right),\]
\[ P_n:=\F_{q^a}[X;\sigma]/\left( X^n-1\right).\]
%where $\sigma$ is the Frobenius automorphism given by $\sigma(\alpha)=\alpha^q$, for $\alpha \in \F_{q^m}$.

\begin{remark}
    Similar to the group algebras $R_n^{(q)}$ introduced in Huffman's study of $\F_q$-linear $\F_{q^m}$-cyclic codes, \cite{huffmancyclic}, we can express $R_n$ and $P_n$ as skew group rings
    %
    \[R_n= \F_{q^m}[\mathbb{Z}_n, \psi], \qquad P_n= \F_{q^a}[\mathbb{Z}_n, \psi], \]
    where $\psi: \mathbb{Z}_n \rightarrow \operatorname{Aut}(\F_{q^m})$ maps $k\in \mathbb{Z}_n $ to $\sigma^k$. Then $P_n$ is canonically embedded into $R_n$. Moreover, Theorem 0.1 in \cite{Mon1980} implies that $R_n$ and $P_n$ are semisimple. % {\color{red}  Include the argument, also skew group rings and skew Maschke}
\end{remark}

\begin{remark}
    If $a=1$, we have that $\sigma$ fixes $\F_{q^a}=\F_q$. Then $\F_{q}[X;\sigma]\simeq \F_q[X]$ is commutative under polynomial multiplication and
    \[ P_n=\F_{q}[X;\sigma]/\left( X^n-1\right)\simeq \F_{q}[X]/\left( X^n-1\right).\] Moreover, under the conditions $a=1$ and $n = m$, we have the following fact which was mentioned in \cite{Gluesing-Luerssen:2021} but whose proof seems to be folklore. We provide a constructive proof below.
\end{remark}

\begin{theorem}
    Let $q\geq2$ be a prime power and consider the extension $\F_q \leq \F_{q^m}$. Then $R_m = \F_{q^m}[X;\sigma]/\left( X^m-1\right)  \simeq \operatorname{Mat}_{m\times m}(\F_q)$.
\end{theorem}

\begin{proof}
    %Since $R_2$ is semisimple, we have that $R_2  \simeq \oplus S_i$, where $S_i$ are simple rings.
    %
    Since $R_m$ is semisimple and finite, the Artin-Wedderburn theorem gives
    \[R_m  \simeq \bigoplus_{i=1}^k \operatorname{Mat}_{n_i \times n_i}(F_i),\]
    where the $F_i$ are finite fields and the $n_i$ are positive integers. We will show that $R_m$ is simple, from which follows that $k=1$, since each direct summand is a two-sided ideal of $R_m$.

    Let $J\subseteq R_m$ be a two-sided nonzero ideal, which consists of elements of the form $f(X)+(X^m-1)$, with $\deg (f)<m$. We can pick $f(X)$ as a nonzero element in $J$ of minimal degree. As $J$ is closed under multiplication by $\alpha\in \F_{q^m}$, we can assume $f(X)$ to be monic.

    We will show that $f(X)$ is the constant polynomial $f(X)=1$, and thus $J=R_m$. Write
    %
    \[f(X)=X^d + a_{d-1}X^{d-1}+ \ldots + a_1 X + a_0, \quad d<m.\]
    %
    If $f(X)=X^d$, then
    %
    \[1+(X^m-1)=X^m+(X^m-1)=X^{m-d}\left[ X^d +(X^m-1) \right] \in X^{m-d} \cdot J \subseteq J,\]
    %
    i.e., $1\in J$ and we must have $f(X)=1$.

    If $f(X)\ne X^d$, there exists $i^*<d$ with $a_{i^*} \ne 0$, and $d\geq1$. Let $\alpha \in \F_{q^m} \setminus \{0\}$. Then $f(X)\alpha +(X^m-1)\in J$, and thus
    %\
    \begin{align*}
        & g(X): =  \left(X^d + a_{d-1}X^{d-1}+ \ldots + a_1 X + a_0\right) \alpha +(X^m-1)                                             \\
              & \quad =  \sigma^d(\alpha)X^d + \sigma^{d-1}(\alpha)a_{d-1}X^{d-1}+ \ldots + \sigma(\alpha)a_1 X + \alpha a_0 +(X^m-1) \in J.
    \end{align*}

    We have that $g(X)= \sigma^d(\alpha) f(X)$ as otherwise $h(X):=g(X)-\sigma^d(\alpha)f(X)$ $\ne 0$ is a polynomial of degree lower than $f(X)$ with $h(X)+(X^m-1)\in J$, which is a contradiction to the choice of $f(X)$. By comparing coefficients, we deduce that
    %
    \begin{align*}
        g(X)= \sigma^d(\alpha) f(X) & \implies \sigma^{i^*}(\alpha)a_{i^*} = \sigma^d(\alpha)a_{i^*} \mbox{ for all } \alpha \in \F_{q^m} \\
                                    & \implies \sigma^{i^*}(\alpha) = \sigma^d(\alpha) \mbox{ for all } \alpha \in \F_{q^m}              \\
                                    & \implies \sigma^{d-{i^*}}=\operatorname{id}_{\F_{q^m}}                                        \\
                                    & \implies m \mid d-i^*,
    \end{align*}
    %
    but the last consequence is false as $0<d-i^*\le d<m$. Therefore, $f(X)$ must be a monic monomial, which is the case previously discussed. Summarizing, there exist a field $F_1$ and an integer $n_1$ such that
    %
    \[R_m  \simeq  \operatorname{Mat}_{n_1 \times n_1}(F_1).\]

    Comparing the centers of $R_m$ and $\operatorname{Mat}_{n_1 \times n_1}(F_1)$ we get $F_1=\F_q$.
    As $R_m$ is an $\F_{q}$-vector space of dimension $m^2$ and has $q^{m^2}$ elements, we conclude $n_1=m$ and
    %
    \[R_m  \simeq  \operatorname{Mat}_{m\times m}(\F_q). \]
\end{proof}

Let $\boldsymbol{c}=\left( c_{i,j}\right)_{0,0}^{n-1,\ell-1}\in \F_{q^m}^{n\ell}$. Define a map $\phi: \F_{q^m}^{n\ell}\rightarrow R_n^\ell$ by
%
\[\phi(\boldsymbol{c})=\left(c_0(X),\ldots,c_{\ell-1}(X)\right),\]
%
where each column $j$ of $\boldsymbol{c}$ determines a polynomial
%
\[c_{j}(X)=\sum_{i=0}^{n-1} c_{i,j}X^i \in R_n.\]
% The map $\phi$ is $\F_{q^m}$-linear and is a one-to-one correspondence between the $\F_{q^m}$-vector spaces $\F_{q^m}^{n\ell}$ and $R_n^\ell$. Hence,  $\phi$ is an $\F_{q^m}$-linear isomorphism of $\F_{q^m}^{n\ell}$ and $R_n^\ell$.

The map $\phi$ gives an isomorphism of the $\F_{q^m}$-vector spaces $\F_{q^m}^{n\ell}$ and $R_n^\ell$. We can use $\phi$ to associate every $\sigma$-QC code with a $P_n$-module structure.
\begin{theorem}\label{thm:skewQCiffPnsubmodule}
    An $\F_{q^a}$-subspace $\C\subseteq \F_{q^{m}}^{n\ell}$ is a $\sigma$-QC code of length $n\ell$ and index $\ell$ if and only if $\phi(\C)$ is a left $P_n$-submodule of the ring $R_n^\ell$.
\end{theorem}

\begin{proof} Let $\C$ be a $\sigma$-QC code of length $n\ell$ and index $\ell$. We claim that $\phi(\C)$ is a $P_n$-submodule of $R_n^\ell$. Evidently, $\phi(\C)$ is closed under addition, and we are going to check that $\phi(\C)$ is also closed under scalar multiplication by elements of $P_n$. Let $\phi(\boldsymbol{c})=\left(c_0(X),\ldots,c_{\ell-1}(X)\right)$. Given $j\in \{0,\ldots,\ell-1\}$ we have that
    %
    \[Xc_j(X)=X\sum_{i=0}^{n-1} c_{i,j}X^i=\sum_{i=0}^{n-1} \sigma(c_{i,j})X^{i+1}\equiv \sum_{i=0}^{n-1} \sigma(c_{i-1,j})X^{i} \mod{(X^n-1)},\]
    hence
    %
    \begin{align*}
        X \phi(\textbf{c}) & =\left(Xc_0(X),\ldots,Xc_{\ell-1}(X)\right)                                                                 \\
                           & =\left(\sum_{i=0}^{n-1} \sigma(c_{i-1,0})X^{i}, \ldots, \sum_{i=0}^{n-1} \sigma(c_{i-1,\ell-1})X^{i}\right) \\
                           & =\phi \begin{pmatrix}
                                       \sigma(c_{n-1,0}) & \cdots & \sigma(c_{n-1,\ell-1}) \\
                                       \sigma(c_{0,0})   & \cdots & \sigma(c_{0,\ell-1})   \\
                                       \vdots            &        & \vdots                 \\
                                       \sigma(c_{n-2,0}) & \cdots & \sigma(c_{n-2,\ell-1})
                                   \end{pmatrix}\in \phi(\C).
    \end{align*}

    Then, by $\F_{q^a}$-linearity, $p(X)\phi(\boldsymbol{c})\in \phi(\C)$ for any $p(X)\in P_n$. Hence $\phi(\C)$ is a left $P_n$-submodule of $R_n^\ell$.

    Conversely, suppose that $D$ is a left $P_n$-submodule of $R_n^\ell$. Let $\C=\phi^{-1}(D)\subseteq \F_{q^{m}}^{n\ell}$. We claim that $\C$ is a $\sigma$-QC code of length $n\ell$ and index $\ell$.

    Since $\phi$ is an $\F_{q^a}$-vector space isomorphism, $\C$ is an $\F_{q^a}$-linear $\F_{q^m}$-code. To show that it is invariant under $\sigma\circ s$, let $\boldsymbol{c}=\left( c_{i,j}\right)_{0,0}^{n-1,\ell-1}$ denote the preimage of the polynomial vector $(c_0(X),\ldots,c_{\ell-1}(X))\in D$. It is enough to show that $\sigma(s(\boldsymbol{c}))\in \phi^{-1}(D)$; i.e., $\phi\left(\sigma(s(\boldsymbol{c}))\right)\in D$. Notice that
    %
    \begin{align*}
        \phi\left(\sigma(s(\boldsymbol{c}))\right)
        &=\phi\begin{pmatrix}
            \sigma(c_{n-1,0}) & \cdots & \sigma(c_{n-1,\ell-1}) \\
            \sigma(c_{0,0})   & \cdots & \sigma(c_{0,\ell-1})   \\
            \vdots            &        & \vdots                 \\
            \sigma(c_{n-2,0}) & \cdots & \sigma(c_{n-2,\ell-1})
        \end{pmatrix}             \\
        & = \left(Xc_0(X),\ldots,Xc_{\ell-1}(X)\right)                          \\
        & = X\left(c_0(X),\ldots,c_{\ell-1}(X)\right)\in X \cdot D \subseteq D.
    \end{align*}

    Therefore, $\C$ is a $\sigma$-QC code of length $n\ell$ and index $\ell$.
\end{proof}

\begin{remark}
    For $\ell=1$ and $a=1$, Theorem \ref{thm:skewQCiffPnsubmodule} means that an $\F_q$-subspace $\C\subseteq \F_{q^{m}}^{n}$ is an $\F_q$-linear skew cyclic code if and only if $\phi(\C)$ is a left $P_n$-submodule of the ring $R_n$. If in addition $n=m$, an equivalent description is given by the study of circulant matrices:

    A matrix $A=(a_{ij})\in \operatorname{Mat}_{n\times n}(\F_q)$ is called \emph{circulant} if $a_{i+1,j+1}=a_{ij}$ and $a_{i+1,1}=a_{i,n}$, for all $i,j\in \{1,\ldots,n-1\}$. Denote by $\operatorname{Circ}_n(\F_q)$ the collection of all circulant matrices in $\operatorname{Mat}_{n\times n}(\F_q)$. $\operatorname{Circ}_n(\F_q)$ is an $\F_q$-vector space of dimension $n$ and an $\F_q$-algebra.

    Given $g+(X^n-1)\in R_n$ with $g(X)=\sum_{i=0}^{n-1} g_i X^i$, let $\tilde{g}(X)=\sum_{i=0}^{n-1} g_i X^{q^i}$, its associated linearized polynomial. Let $\{\alpha, \alpha^q \ldots, \alpha^{q^{m-1}}\}$ be a normal basis of the extension $\F_q \leq \F_{q^m}$ and let $M_{\tilde{g}}$ denote the matrix of the $\F_q$-linear polynomial $\tilde{g}$ with respect to this basis. Set $\varphi(g+(X^n-1))=M_{\tilde{g}}$. Then $\varphi$ is a ring isomorphism between $R_n$ and $\operatorname{Mat}_{n\times n}(\F_q)$.

    We have that $\varphi\left( P_n\right)=\operatorname{Circ}_n(\F_q)$. Thus, if an $\F_q$-subspace $\C\subseteq \F_{q^{m}}^{n}$ is an $\F_q$-linear skew cyclic code, then $\varphi(\C)$ is a left $\operatorname{Circ}_n(\F_q)$-submodule of $\operatorname{Mat}_{n\times n}(\F_q)$.
\end{remark}

\section{Module Structure of \texorpdfstring{$\F_{q^a}[X;\sigma]$}{Skew-Polynomial}-Modules}\label{sec:structure1}

Consider the field extensions $\F_q \leq \F_{q^a}\leq \F_{q^m}$ with $r:=\frac{m}{a}$, the Frobenius automorphism $\sigma$ with $\sigma(\alpha)=\alpha^q$, and two $\F_{q^a}[X;\sigma]$-modules $C$ and $D$ such that $D\leq C\leq \F_{q^m}[X;\sigma]^\ell$. Note that $C$ and $D$ are free as $\F_{q^a}[X;\sigma]$-submodules of a free module. Recall that $\F_{q^m}$ is an $\F_{q^a}$-vector space.  We can choose a normal basis $b_1, \ldots, b_r$ of $\F_{q^m}$ over $\F_{q^a}$, i.e., there exists  $\alpha\in \F_{q^m}\setminus \{0\}$ such that $b_1= \alpha, b_2=\alpha^{q^a}, \ldots, b_r = \alpha^{q^{a(r-1)}}$. In fact, \cite[Satz 1.2]{BLESSENOHL-KARSTENJOHNSEN:1986} asserts that such an $\alpha$ can be chosen to simultaneously generate a normal basis of $\F_{q^m}$ over any field intermediate to $\F_q\leq \F_{q^m}$.

Let us start with the special case $a=1$. Given any $f(X)= \sum_{i} a_i X^i   \in \F_{q^m}[X;\sigma]$, we can represent
%
\[a_i=\sum_{j=1}^r k_{ij} b_j= \sum_{j=1}^r k_{ij} \alpha^{q^{j-1}}, \mbox{ where }k_{ij}\in \F_{q}. \]
%
Then, we can write
%
\begin{eqnarray}\label{FqXdecomposition2}
    f(X) = \sum_i a_i X^i & = & \sum_i \left( \sum_{j=1}^k k_{ij} b_j \right) X^i\nonumber \\ & = & \sum_{i,j}  k_{ij} X^i b_{j-i}  =\sum_{h=1}^k \left( \sum_i k_{i,h+i} X^i \right)  b_h,
\end{eqnarray}
where the index $h+i$ needs to be read modulo $m$.
%
If we consider $\F_{q^m}[X;\sigma]$ as an $\F_{q^a}[X;\sigma]$-module, it is now easy to see
%
\begin{equation}\label{Fqa-decomposition}
    \F_{q^m}[X;\sigma] = \bigoplus_{i=1}^{r} \F_{q^a}[X;\sigma] b_{i}.
\end{equation}

Decomposition \eqref{Fqa-decomposition} holds also for $a> 1$. However, the computationally efficient calculation in \eqref{FqXdecomposition2} has to be adjusted as follows: given any $f(X)= \sum_{i} a_i X^i   \in \F_{q^m}[X;\sigma]$, we express
%
\[\sigma^{-i}(a_i)=\sum_{j=1}^r k_{ij} b_j= \sum_{j=1}^r k_{ij} \alpha^{q^{a(j-1)}} \mbox{ with }k_{ij}\in \F_{q^a} \]
%
and write
%
\begin{eqnarray*}
    f(X) = \sum_i a_i X^i = \sum_i X^i \sigma^{-i}(a_i) & = & \sum_i X^i\left( \sum_{j=1}^k k_{ij} b_j \right) \\ & = & \sum_{j=1}^k \left( \sum_i \sigma^i(k_{ij}) X^i \right)  b_j.
\end{eqnarray*}

%Since $D$ is a free $\F_{q^a}[X;\sigma]$-submodule it has  
To continue with our discussion, let $\left\{\overline{r_1}, \ldots, \overline{r_k}\right\}$ be a generating set of $C$ as an $\F_{q^a}[X;\sigma]$-module, where
%
\[\overline{r_i} = \left( \widetilde{r_{i,1}} , \ldots, \widetilde{r_{i,\ell}} \right)\in \F_{q^m}[X;\sigma]^\ell.  \]

%Under the decomposition $\F_{q^m}[X;\sigma] = \bigoplus_{j=1}^{r} \F_{q^a}[X;\sigma] b_j$, write
Under the decomposition \eqref{Fqa-decomposition}, write each coordinate as
%
\[\widetilde{r_{i,\iota}}=\left( r_{i,(\iota-1)r+1}, r_{i,(\iota-1)r+2}, \ldots, r_{i,\iota r} \right) \in \F_{q^a} [X;\sigma]^{r}.\]

Then, $C= \langle \left( r_{i,1}, r_{i,2}, \ldots, r_{i,r\ell} \right): 1\le i\le k\rangle$ as an $\F_{q^a}[X;\sigma]$-module, and the rows of
%
\[ M= \begin{pmatrix}
        r_{1,1} & r_{1,2} & \ldots & r_{1,r\ell} \\
        r_{2,1} & r_{2,2} & \ldots & r_{2,r\ell} \\
        \vdots  & \vdots  & \ddots & \vdots      \\
        r_{k,1} & r_{k,2} & \ldots & r_{k,r\ell}
    \end{pmatrix}\]
% 
generate $C$ as an $\F_{q^a}[X;\sigma]$-module. \\

For the next step in our discussion, we need to review total divisors and the Smith normal form for non-commutative rings. For reference, see \cite{Berrick:2000}.

\begin{definition}
    Let $s, t\in R$. Recall that $s$  is a {\it left divisor} of $t$ if there exists $u\in R$ such that $t= su$, and $s$  is a {\it right divisor} of $t$ if there exists $u\in R$ such that $t= us$. Left and right divisors will be denoted by $s \mid_l t$ and $s \mid_r t$, respectively. A much stronger statement is that $s$ is a {\it total divisor} of $t$. We say that $s$ totally divides $t$ (or $s \mid \mid t$) if for all $u\in R$, $s \mid_l ut$ and $s\mid_r tu$.
\end{definition}

\begin{theorem}[Smith Normal Form]
    Let $R$ be a (two-sided) Euclidean ring without zero divisors and let $A=(a_{ij})$ be an $n\times p$ matrix over $R$. Then, there exist invertible matrices $P$ and $Q$ over $R$ such that
    %
    \[PAQ= \operatorname{diag}(e_1, \ldots, e_r, 0, \ldots, 0),\]
    a diagonal matrix with $e_1 \mid \mid e_2 \mid \mid \ldots \mid \mid e_{r}$ and $r\le \min\{n,p\}$. Moreover, $P$ may be obtained by applying a sequence of row operations to the $n\times n$ identity matrix, and $Q$ may be obtained by applying a sequence of column operations to the $p\times p$ identity matrix.
\end{theorem}
\begin{theorem}[The Invariant Factor Theorem]
    Let $M$ be a finitely generated right $R$-module, with $R$ a (two-sided) Euclidean ring without zero divisors. Then,
    \[ M \simeq R/e_1 R \oplus \ldots \oplus R/e_{r} R \oplus R^{s},\]
    %
    where $e_1, \ldots, e_r$ are nonunits in $R$, $e_1 \mid \mid e_2 \mid \mid \ldots \mid \mid e_r$ and $s\geq0$.
\end{theorem}
We use the Smith normal form of the matrix $M$ to find a basis $\boldsymbol{c}_1, \boldsymbol{c}_2,\ldots, \boldsymbol{c}_{\xi}$ of $C$. The Smith normal form provides us with invertible matrices $S$ and $T$ such that
%
\[SMT = J= \begin{pmatrix}
        h_1    & 0      & \ldots & 0      & 0      & \ldots \\
        0      & h_2    & \ldots & 0      & 0      & \ldots \\
        \vdots & \vdots & \ddots & \vdots & \vdots & \vdots \\
        0      & 0      & \ldots & h_\xi  & 0      & \ldots \\
        0      & 0      & \ldots & 0      & 0      & \ldots \\
        \vdots & \vdots & \vdots & \vdots & \vdots & \ddots \\
    \end{pmatrix}, \]
%Let $\xi$ be the largest index for which $h_\xi \ne 0$, then we have that 
where $h_1, h_2, \ldots, h_\xi$ are the nonzero diagonal entries of $J$ with $h_i$ a total divisor of $h_{i+1}$ for all $i=1, \ldots, \xi-1$. The basis elements $\boldsymbol{c}_i$ are the nonzero rows of $JT^{-1}$, while the rows of $T^{-1}$ form an associated basis of $\F_{q^a}[X;\sigma]^{r\ell}$.

Next, let $\left\{\overline{t_1}, \ldots, \overline{t_\delta}\right\}$ be a generating set of $D$ as an $\F_{q^a}[X;\sigma]$-module, where
%
\[\overline{t_i} = \left( \widetilde{t_{i,1}} , \ldots, \widetilde{t_{i,\ell}} \right)\in \F_{q^m}[X;\sigma]^\ell.  \]
%
Under the decomposition \eqref{Fqa-decomposition}, write
%
\[\widetilde{t_{i,\iota}}=\left( t_{i,(\iota-1)r+1}, t_{i,(\iota-1)r+2}, \ldots, t_{i,\iota r} \right) \in \F_{q^a} [X;\sigma]^{r}. \]
%
Then, $D= \langle \left( t_{i,1}, t_{i,2}, \ldots, t_{i,r\ell} \right): 1\le i \le \delta \rangle$ as an $\F_{q^a}[X;\sigma]$-module.
%
Express each row vector $\left( t_{i,1}, \ldots, t_{i,r\ell} \right)$ with respect to the basis $\boldsymbol{c}_1, \boldsymbol{c}_2,\ldots, \boldsymbol{c}_{\xi}$ of $C$ as
%
\[ \left( t_{i,1}, \ldots, t_{i,r\ell} \right) = \sum_{j=1}^\xi g_{i,j} \boldsymbol{c}_j, \; g_{i,j}  \in\F_{q^a} [X;\sigma]. \]
%\[\left( t_{i,1}, \ldots, t_{i,r\ell} \right)=\left( g_{i,1}, g_{i,2}, \ldots, g_{i,\xi} \right) \in \F_{q^a} [X;\sigma]^{\xi}. \]
%
Then the rows of
%
\[ N= \begin{pmatrix}
        g_{1,1}      & g_{1,2}      & \ldots & g_{1,\xi}      \\
        g_{2,1}      & g_{2,2}      & \ldots & g_{2,\xi}      \\
        \vdots       & \vdots       & \ddots & \vdots         \\
        g_{\delta,1} & g_{\delta,2} & \ldots & g_{\delta,\xi}
    \end{pmatrix}\]
%
generate $D$ as an $\F_{q^a}[X;\sigma]$-submodule of $C$, with respect to the basis vectors $\boldsymbol{c}_1, \boldsymbol{c}_2,\ldots, \boldsymbol{c}_{\xi}$ of~$C$.
%
Get the matrix $N$ into Smith normal form with nonzero diagonal entries $d_1, d_2, \ldots, d_{\zeta}$.
%
\[  \begin{pmatrix}
        d_1    & 0      & \ldots & 0      & 0      & \ldots \\
        0      & d_2    & \ldots & 0      & 0      & \ldots \\
        \vdots & \vdots & \ddots & \vdots & \vdots & \vdots \\
        0      & 0      & \ldots & d_\zeta  & 0      & \ldots \\
        0      & 0      & \ldots & 0      & 0      & \ldots \\
        \vdots & \vdots & \vdots & \vdots & \vdots & \ddots \\
    \end{pmatrix}.\]

This gives a new basis $\boldsymbol{c}_1^*, \boldsymbol{c}_2^*, \ldots, \boldsymbol{c}_{\xi}^*$ for $C$ such that $ \left\{  d_i \boldsymbol{c}_i^* : i =1, \ldots, \zeta  \right\} $ is a basis of~$D$. Set $d_i=0$ for all $i>\zeta$. We have that, as $\F_{q^a}[X;\sigma]$-modules,
%
\begin{align*}
    C / D & =\left(\bigoplus_{i=1}^\xi \F_{q^a} [X;\sigma]\boldsymbol{c}_i^* \right) \Big/ \left(\bigoplus_{i=1}^\xi  \F_{q^a} [X;\sigma]d_i\boldsymbol{c}_i^* \right) \\
              & \simeq \bigoplus_{i=1}^\xi \left(  \F_{q^a} [X;\sigma] /  \F_{q^a} [X;\sigma] d_i \right).
\end{align*}

\section{Total Divisors in Quotients of Skew Polynomial Rings}\label{sec:totaldivisors}

In Section \ref{sec:structure1} we described the module structure of $C/D$ using two applications  of the Smith normal form, where $C$ and $D$ are arbitrary $\F_{q^a}[X;\sigma]$-submodules such that $D\leq C\leq \F_{q^m}[X;\sigma]^\ell$. The nonlinear skew quasi-cyclic codes, the primary focus of this article, fall into the a special case where $D= (X^n-1)$, the two-sided ideal generated by $X^n-1$. In this special case, it is sufficient to use a single application of the Smith normal form. This reduction requires a rather deep result about total divisors which is presented in this section. 

Let $R=\F_{q^a}[X;\sigma]$ for the Frobenius automorphism $\sigma$ with $\sigma(\alpha)=\alpha^q$. Recall that $R$ has center $Z(R)= \F_q[X^a; \sigma]$ and that $R$ is a left and right Euclidean ring without zero divisors. In particular, $R$ is both a left and right principal ideal ring. 
For any $t \in R$ we will denote by $Rt$ the left ideal of $R$ generated by $t$, while $tR$ is the right ideal of $R$ generated by $t$, and
\[R t R=\left\{\sum_{i} r_{i} t r_{i}^{\prime} \mid r_{i}, r_{i}^{\prime} \in R\right\}\]
denotes the two-sided ideal of $R$ generated by $t$.

\begin{definition}
    For any left, right, or two-sided ideal $I\ne 0$ of $R$, denote by $g_I \in I$ the monic polynomial of smallest degree in $I$. This polynomial $g_I$ is unique and generates $I$. We can write:
    \begin{itemize}
        \item $I=Rg_I$, if $I$ is a left ideal,
        \item $I=g_IR$, if $I$ is a right ideal,
        \item $I=Rg_I=g_IR=Rg_IR$, if $I$ is a two-sided ideal.
    \end{itemize}
\end{definition}
%
%We recall the following definition from non-commutative ring theory.
We note some basic criteria for checking if a polynomial is a total divisor of another.
%
\begin{lemma}\label{thm:allb-fistcriteria}
Let $s,t\in R$. The polynomial $s$ is a total divisor of $t$ if and only if $s\mid_l \alpha X^{k} t,s\mid_r t \alpha X^{k}$ for all $\alpha \in \mathbb{F}_{q^a}$ and $k < \deg s$.
\end{lemma}
\begin{proof}
    We need to check that $s\mid_l g t$ and $s\mid_{r} t g$ for all $g \in R=\mathbb{F}_{q^a}[X; \sigma]$. If $\deg g \geq \deg s$, we can use left or right division algorithm of $g$ by $s$. In particular, we can write $g=q s+r$ with $q, r \in R$, $\deg r<\deg s$, and $s \mid_r t g$ if and only if $s\mid_{r} t r$. Thus, it suffices to check only those $g \in R$ with $\deg g<\deg s$. Moreover, by additivity, it suffices to check only $g=\alpha X^{k}$ with $\alpha \in \mathbb{F}_{q^a}, k < \deg s.$
\end{proof}

\begin{lemma}\label{thm:allb-secondcriteria}
    Let $s,t\in R$. The polynomial $s$ is a total divisor of $t$ if and only if $s\mid_l g_{I}$ and $s\mid_{r} g_{I}$, where $I=R t R$.
\end{lemma}

\begin{proof}
    Suppose that $s$ is a total divisor of $t$. Then $s$ is both a left and right divisor of every element of the form $\sum r_i t r_i^\prime$. That is, $s$ is both a left and right divisor of every element in $RtR$, in particular $s$ is both a left and right divisor of $g_I \in RtR$.

    Now, let $g\in R$. Then $tg\in RtR=Rg_I$, which implies that $g_I \mid_r tg$, and since $s\mid_r g_I$, we have  $s \mid_r tg$ by transitivity. Similarly, we can show that $s\mid_l gt$.
\end{proof}

\begin{theorem}\label{thm:selftotaldivisor}
    Let $s\in R$. Then $s$ is a total divisor of itself if and only if $s=\gamma c X^k$, where $\gamma \in \F_{q^a}$, $k \geq 0$, and $c\in Z(R)= \F_q[X^a; \sigma]$ a monic polynomial in the center of $R$.
\end{theorem}

\begin{proof}
    Let $s=\gamma c X^k$ with $\alpha \in \F_{q^a}$, $k \geq 0$, and $c\in Z(R)= \F_q[X^a; \sigma]$ a monic polynomial. Let $g\in R$. We need to show that $s \mid_r sg$. Notice that we can write $X^k g= h X^k$ for some $h\in R$. Then,
    %
    \[sg = \gamma c X^k g = \gamma c h X^k = \gamma h c X^k = \gamma h \gamma^{-1}\gamma c X^k = \gamma h \gamma^{-1} s. \]
%
    That is, $s\mid_r sg$. Similarly we have $s\mid_l gs$, and $s$ is a total divisor of itself.
    
    Conversely, suppose that $s=\sum_{i=0}^N a_i X^i\in R$ is a total divisor of itself. We assume that $a_N \ne 0$. Choose $\alpha \in \F_{q^a}$ such that $\alpha, \sigma(\alpha), \sigma^2(\alpha), \ldots, \sigma^{a-1}(\alpha)$ are all distinct and cover all the roots of $\min(\alpha, \F_q)$; for instance, we can take $\alpha$ to be a primitive element of the extension $\F_q\leq \F_{q^a}$.

    Since $s$ is a total divisor of itself, we have that 
    \[s\mid_r s\alpha= \left(\sum_{i=0}^N a_i X^i\right)\alpha = \sum_{i=0}^N a_i\sigma^i(\alpha) X^i.\]
    Since $s$ and $s\alpha$ have the same degree and $s\mid_r s\alpha$, there exists $\beta \in \F_{q^a}$ such that $s\alpha = \beta s$. Thus,
    %
    \[ \sum_{i=0}^N a_i \sigma^i(\alpha) X^i = s\alpha =\beta s =\sum_{i=0}^N \beta a_i X^i. \]

    By comparing coefficients we have $\sigma^i(\alpha)=\beta$ for all $a_i \ne 0$, but since $\alpha, \sigma(\alpha), \sigma^2(\alpha), \ldots, \sigma^{a-1}(\alpha)$ are distinct elements of $\F_{q^a}$, we have that all $i$ with $a_i \ne 0$ must belong to the same congruence class modulo $a$. In particular, $s= tX^k$ for some $k\ge 0, t\in \F_{q^a}[X^a;\sigma]$. Without loss of generality, we may factor out the leading coefficient of $t$ to write $t= \gamma c$ and $s= \gamma c X^k$ for some $\gamma \in \F_{q^a}$ and $c$ a monic polynomial in $\F_{q^a}[X^a;\sigma]$.

    We have $s=\gamma c X^k$ with $\gamma \in \F_{q^a}$, $k\ge 0$, and a monic polynomial $c= \sum_{i=0}^M c_i X^{ia} \in \F_{q^a}[X^a;\sigma]$. Since $s \mid\mid s$, we have that 
    \[s\mid_l Xs=X \gamma \left( \sum_{i=0}^M c_i X^{ia} \right) X^k = \sigma(\gamma) \sum_{i=0}^M \sigma({c_i}) X^{ia+k+1}. \]
%
Since $s\mid_l Xs$, there exists $\beta, \beta^\prime \in \F_{q^a}$ such that $Xs=s(\beta X +\beta^\prime)$. Thus,
%
\[\sigma(\gamma ) \sum_{i=0}^M \sigma(c_i)X^{ia+k+1}=Xs= s(\beta X+ \beta^\prime)= \gamma \left( \sum_{i=0}^M c_i X^{ia} \right)X^k (\beta  X+\beta^\prime). \]
Choose $i$ as the smallest integer such that $c_i \ne 0$. Then, comparing the coefficients of $X^{ia+k}$ on both sides, we obtain that $\gamma c_i \sigma^{ia+k}(\beta^\prime)=0$, and thus $\sigma^{ia+k}(\beta^\prime)=0$ and $\beta^\prime=0$. Therefore,
%Comparing lowest degree terms on this equation, we find that $\beta^\prime=0$. Thus, 
%
\begin{align}
    \sigma(\gamma ) \sum_{i=0}^M \sigma(c_i)X^{ia+k+1}
    &= \gamma \left( \sum_{i=0}^M c_i X^{ia} \right)X^k\beta  X\nonumber \\
    &= \gamma \left( \sum_{i=0}^M c_i X^{ia} \right) \sigma^k(\beta) X^{k+1}\nonumber \\
    &= \gamma \sigma^k(\beta) \sum_{i=0}^M c_i X^{ia+k+1}. \label{eq4}
\end{align}
Comparing leading coefficients, we have
%
\[ \sigma(\gamma) = \sigma(\gamma)\sigma(1)= \sigma(\gamma)\sigma(c_M)= \gamma \sigma^k(\beta)c_M =\gamma \sigma^k(\beta). \]
Plugging this back into Equation \eqref{eq4} gives
\[ \sigma(\gamma)\sum_{i=0}^M \sigma({c_i}) X^{ia+k+1} =\sigma(\gamma) \sum_{i=0}^M c_i X^{ia+k+1}. \]

We conclude that $\sigma(c_i)=c_i$ for all $i$, hence $c_i\in \F_q$. Therefore, $c\in \F_q[X^a;\sigma]$.
\end{proof}

\begin{corollary}\label{thm:form-of-gI}
    Let $I=RtR$. Then $g_I$ is a total divisor of itself and $g_I= c X^k$ with $c$ a monic polynomial in $Z(R)= \F_q[X^a; \sigma]$ and $k\ge 0$.
\end{corollary}

\begin{proof}
    Notice that $g_I \mid_l g_I$ and $g_I \mid_r g_I$. Then, by Lemma \ref{thm:allb-secondcriteria} it follows that $g_I$ is a total divisor of itself. Then, by Theorem \ref{thm:selftotaldivisor}, we have that $g_I = \gamma c X^k$, with $\gamma\in \F_{q^a},$ $k\ge 0$ and $c\in Z(R)$ monic. As $g_I$ is monic, we have that $\gamma=1$ and $g_I = c X^k$.
\end{proof}

\begin{remark}
    Let $I=RtR$ and $g_I=cX^k$ as in Theorem \ref{thm:form-of-gI}. Then $k$ is the maximal integer with $X^k \mid_r t$, so that $t= t^\prime X^k$, for some $t^\prime\in R$. The polynomial $c$ is the maximal monic polynomial in $Z(R)$ with $c\mid_r t^\prime$.
\end{remark}

\begin{proposition}\label{thm:intermediate}\ 
    \begin{enumerate}
        \item[$(1)$] Let $s\in R, t\in Z(R)$ with $s\mid_r t$. Then $s\mid_l t$ and $s$ is a total divisor of $t$.
        \item[$(2)$] Let $s\in Z(R), t\in R$ with $s\mid_r t$. Then $s\mid_l t$ and $s$ is a total divisor of $t$.
        \item[$(3)$] Let $s,t\in R$ and $c\in Z(R)$ with $s\mid_r c$ and $c\mid_r t$. Then $s$ is a total divisor of $t$.
    \end{enumerate}
\end{proposition}

\begin{proof}
    $(1)$ Notice that $s\mid_r t$ implies that $t=s^\prime s$ for some $s^\prime\in R$. Since $t\in Z(R)$ it follows that $t=s^\prime s=s s^\prime$. Thus $s\mid_l t$. Then, for $g\in R$ we have that $tg=gt=gs^\prime s$, that is, $s\mid_r tg$. Similarly, we can show that $s\mid_l gt$, and $s$ is a total divisor of $t$.

    $(2)$ Observe that $s \mid_r t$ implies that $t=s^\prime s$ for some $s^\prime \in R$. Since $s\in Z(R)$ it follows that $t=s^\prime s= s s^\prime$. Thus, $s\mid_l t$. Then, for $g\in R$ we have that $tg =s^\prime s g = s^\prime g s $, that is, $s \mid_r tg$. Similarly $s\mid_l gt$, and $s$ is a total divisor of $t$.

    $(3)$ Since $c\in Z(R)$, the conclusion of $(1)$ and $s\mid_r c$ imply that $s\mid \mid c$. Also, $(2)$ and $c\mid_r t$ imply that $c\mid \mid t$. By transitivity, $s\mid\mid t$.
\end{proof}
% \begin{proposition}\label{thm:intermediate2}
%     Let $s,t\in R$ and $c\in Z(R)$ with $s\mid_r c$ and $c\mid_r t$. Then $s$ is a total divisor of $t$.
% \end{proposition}

% \begin{proof}
%     Since $c\in Z(R)$, Theorem \eqref{thm:intermediate1}a) and $s\mid_r c$ imply that $s\mid \mid c$. Also, Theorem \eqref{thm:intermediate1}a) and $c\mid_r t$ imply that $c\mid \mid t$. By transitivity, $s\mid\mid t$.
% \end{proof}

\begin{theorem}\label{thm:grcdmakesmagic}
    If $s$ is a total divisor of $t$ and $a\mid n$, then $\operatorname{gcrd}(s,X^n-1)$ is a total divisor of $\operatorname{gcrd}(t,X^n-1)$.
\end{theorem}

\begin{proof}
    Suppose that $s\mid\mid t$ and let $I = RtR$. Then by Theorem \ref{thm:allb-secondcriteria}, $s\mid_r g_I$. Since $t\in I=Rg_I$, we have that $g_I\mid_r t$. Then, $s\mid_r g_I$ and $g_I\mid_r t$ imply that $\operatorname{gcrd}(s,X^n-1)\mid_r \operatorname{gcrd}(g_I,X^n-1)$ and $\operatorname{gcrd}(g_I,X^n-1)\mid_r \operatorname{gcrd}(t,X^n-1)$. If we can show that $\operatorname{gcrd}(g_I,X^n-1)$ belongs to $Z(R)$, then the desired conclusion follows from Theorem \ref{thm:intermediate}(3).

    To show that $\operatorname{gcrd}(g_I,X^n-1)\in Z(R)$, let $g_I= c X^k$ with $c\in Z(R)$ a monic polynomial and $k\ge 0$. We have that $\operatorname{gcrd}(g_I,X^n-1)=\operatorname{gcrd}(cX^k,X^n-1)=\operatorname{gcrd}(c,X^n-1)$, since the right Euclidean algorithm for calculating $\operatorname{gcrd}(cX^k, X^n-1)$ performs completely in the subring $\F_q[X;\sigma]=\F_q[X]$ of $R$. Furthermore, $\operatorname{gcrd}(c,X^n-1)\in Z(R)$ because $c, X^n-1\in Z(R)$ and the right Euclidean algorithm fully performs in the subring $Z(R)=\F_q[X^a;\sigma]= \F_q[X^a]$ of $R$. Hence, $\operatorname{gcrd}(g_I,X^n-1)\in Z(R)$.
\end{proof}

\section{Structure of \texorpdfstring{$P_n$}{Pn}-Submodules of \texorpdfstring{$R_n^\ell$}{Rnl}}\label{sec:structure2}
%\section{Structure of $P_n$-submodules of $R_n^\ell$}
Consider the field extensions $\F_q \leq \F_{q^a}\leq \F_{q^m}$ and $n$ a multiple of $m$.
%Notice that $_\bullet(X^n-1)$ is a two-sided ideal of $\F_{q^a}[X;\sigma]$ and $\F_{q^m}[X;\sigma]$. We will write $_\bullet(X^n-1)$ as $(X^n-1)$. 
Again, let
\[R_n = \F_{q^m}[X;\sigma]/\left( X^n-1\right),\]
\[ P_n=\F_{q^a}[X;\sigma]/\left( X^n-1\right),\]
where $\sigma$ is the Frobenius automorphism given by $\sigma(\alpha)=\alpha^q$, for $\alpha \in \F_{q^m}$.

% Notice that, in contrast with the group algebras $R_n^{(q)}$ introduced in Huffman's study of $\F_q$-linear $\F_{q^m}$-cyclic codes, \cite{huffmancyclic}, our $R_n$ is merely an abelian group under addition equipped with the action of $P_n$, turning $R_n$ into a $P_n$-module.

Let $\C$ be an $\F_{q^a}$-linear skew quasi-cyclic code of index $\ell$ with $\dim_{\F_{q^a}}(\C) = k$. We have that $\C$ is a $P_n$-submodule of $R_n^\ell$. This section is devoted to studying the $P_n$-module structure of  $\C$.

%Let $\pi: \F_{q^m}[X;\sigma]^\ell\rightarrow  R_n^\ell$ denote the canonical projection: coordinatewise division in $\F_{q^m}[X;\sigma]$ modulo $(X^n-1)$. Then $\pi^{-1}\left(\C \right)$, the preimage of $\C$ under $\pi$, is an $\F_{q^a}[X]$-module. The code $\C$ has a generator matrix of the form
Let $\pi: \F_{q^m}[X;\sigma]^\ell\rightarrow  R_n^\ell$ denote the canonical projection, i.e., identifying coordinatewise with the remainder in $\F_{q^m}[X;\sigma]$ modulo $(X^n-1)$. Then $\pi^{-1}\left(\C \right)$, the preimage of $\C$ under $\pi$, is an $\F_{q^a}[X;\sigma]$-module. The code $\C$ has a generator matrix of the form
%
\[ G= \begin{pmatrix}
        \overline{r_1} &
        \overline{r_2} &
        \ldots         &
        \overline{r_k}
    \end{pmatrix}^t\] 
    with $\overline{r_i} = \left( \widetilde{r_{i,1}} + (X^n-1), \ldots, \widetilde{r_{i,\ell}} + (X^n-1)\right)\in R_n^\ell$ and  $\widetilde{r_{i,\iota}} \in \F_{q^m}[X;\sigma]$,
%
where the rows $\overline{r_1} ,\overline{r_2},\ldots, \overline{r_k}$ generate $\C$ as a $P_n$-module.
%$\F_{q^a}[X;\sigma]$-module.

%As in Section 5, let $b_1, \ldots, b_r$ be a normal basis of $\F_{q^m}$ over $\F_{q^a}$. Then $b_1=\alpha$, $b_{1}=\alpha^{q^a}$, $\ldots$, $b_{r}=\alpha^{q^{m-a}}$ is a basis of $\F_{q^m}$ as an $\F_{q^a}$-vector space. Hence, if we consider $\F_{q^m}[X;\sigma]$ as an $\F_{q^a}[X;\sigma]$-module, we have from \eqref{Fqa-decomposition} that
As in Section 5, let $b_1, \ldots, b_r$ be a normal basis of $\F_{q^m}$ over $\F_{q^a}$, where $r=\frac{m}{a}$. If we consider $\F_{q^m}[X;\sigma]$ as an $\F_{q^a}[X;\sigma]$-module, we have as in \eqref{Fqa-decomposition} that
%
\begin{equation*}\label{Fqa2-decomposition}
    \F_{q^m}[X;\sigma] = \bigoplus_{i=1}^{r} \F_{q^a}[X;\sigma] b_{i }   .
\end{equation*}
%Notice that $P_n^r$ and $R_n$ are canonically isomorphic.
Under this decomposition, write
%
\[\widetilde{ {r}_{i,\iota}}=\left( r_{i,(\iota-1)r+1}, r_{i,(\iota-1)r+2}, \ldots, r_{i,\iota r} \right) \in \F_{q^a} [X;\sigma]^{r}. \]

Let $e_i$ be the element in $\F_{q^m}[X;\sigma]^\ell$ with entry $1$ as the $i$-th coordinate and entry $0$ in all the other coordinates. Notice that, as a left $\F_{q^a}[X;\sigma]$-module, the ideal $(X^n-1)\F_{q^m}[X;\sigma]^\ell$ has the basis
%
\begin{align*}
     & (X^n-1)b_1e_1, \ldots, (X^n-1)b_re_1,             \\
     & (X^n-1)b_1e_2, \ldots, (X^n-1)b_re_2,             \\
     & \qquad \qquad \qquad \ \ \vdots                       \\
     & (X^n-1)b_1e_\ell, \ldots, (X^n-1)b_re_\ell.
\end{align*}
%Then, $\pi^{-1}(\C)$ as an $\F_{q^a}[X;\sigma]$-module can be decomposed as
%\[\pi^{-1}(\C)= \left\langle  \left( r_{i,1}, r_{i,2}, \ldots, r_{i,r\ell} \right) : i=1, \ldots,  k \right\rangle +(X^n-1)\F_{q^a}[X;\sigma]^\ell\]
Then, \[\pi^{-1}(\C)= \left\langle  \left( r_{i,1}, r_{i,2}, \ldots, r_{i,r\ell} \right) : 1\le i\le k \right\rangle +(X^n-1)\F_{q^a}[X;\sigma]^{r\ell}\] as an $\F_{q^a}[X;\sigma]$-module, and the rows of
%Then, $\pi^{-1}(\C)= \left\langle \left\{ \left( r_{i,1}, r_{i,2}, \ldots, r_{i,r\ell} \right) \right\}_{i=1}^k \right\rangle +(X^n-1)\F_{q^a}[X;\sigma]^\ell$ as an $\F_{q^a}[X;\sigma]$-module, and the rows of 
%
\begin{equation}\label{Smith1}
        M= \begin{pmatrix}
        r_{1,1} & r_{1,2} & \ldots & r_{1,r\ell} \\
        r_{2,1} & r_{2,2} & \ldots & r_{2,r\ell} \\
        \vdots  & \vdots  & \ddots & \vdots      \\
        r_{k,1} & r_{k,2} & \ldots & r_{k,r\ell} \\ \hline
        X^n-1   & 0       & \ldots & 0           \\
        0       & X^n-1   & \ldots & 0           \\
        \vdots  & \vdots  & \ddots & \vdots      \\
        0       & 0       & \ldots & X^n-1
    \end{pmatrix}
\end{equation}
% 
generate $\pi^{-1}(\C)$ as an $\F_{q^a}[X;\sigma]$-module.

Recall that $\F_{q^m}[X;\sigma]$ is a free module over the two-sided Euclidean ring $\F_{q^a}[X;\sigma]$. Thus, $\pi^{-1}(\C)\subseteq \F_{q^m}[X;\sigma]^\ell$ is a free $\F_{q^a}[X;\sigma]$-module. %with a basis.

\begin{remark}
From here, one may apply the double Smith normal form approach from Section \ref{sec:structure1} to the $\F_{q^a}[X;\sigma]$-modules $D\leq C\leq \F_{q^m}[X;\sigma]^\ell$ with $C=\pi^{-1}(\C)$ and $D=(X^n-1)\F_{q^m}[X;\sigma]^\ell$ to obtain the $P_n$-module structure of  $\C \simeq C/D$. In the following, we will present a computationally more efficient alternative method that requires only one application of the Smith normal form.
\end{remark}

We use the Smith normal form of matrix $M$ to find a stacked basis $\boldsymbol{q}_1, \boldsymbol{q}_2,\ldots, \boldsymbol{q}_{r\ell}$ of $\F_{q^a}[X;\sigma]^{r\ell}$. The Smith normal form provides us with invertible matrices $S$ and $T$ such that
%
\begin{equation} \label{Smith2}
SMT = J =\begin{pmatrix}
        h_1    & 0      & \ldots & 0         \\
        0      & h_2    & \ldots & 0         \\
        \vdots & \vdots & \ddots & \vdots    \\
        0      & 0      & \ldots & h_{r\ell} \\
        0      & 0      & \ldots & 0         \\
        \vdots & \vdots & \ddots & \vdots    \\
        0      & 0      & \ldots & 0         \\
    \end{pmatrix}.
\end{equation}

Note that the polynomials $h_i$ must be nonzero as the matrix $M$ has full rank.
%Moreover, for the special case $a=1$ we have that $\F_{q^a}[X;\sigma]=\F_{q^a}[X]$ is a Euclidean domain, $h_i$ is determined as the greatest common divisor of the determinants of the $i\times i$ minors of $M$, and it is easy to see that $h_i$ must divide $X^n-1$.
In fact, we have the following much stronger general division result.
\begin{proposition}
    The polynomials $h_i$ on the main diagonal of $J$ are total divisors of $X^n-1$ in $\F_{q^a}[X;\sigma]$. In particular, $h_i d_i = d_i h_i = X^n-1$ for suitable $d_i\in \F_{q^a}[X;\sigma]$.
\end{proposition}
\begin{proof}
    Denote
    \[A= \begin{pmatrix}
            r_{1,1} & r_{1,2} & \ldots & r_{1,r\ell} \\
            r_{2,1} & r_{2,2} & \ldots & r_{2,r\ell} \\
            \vdots  & \vdots  & \ddots & \vdots      \\
            r_{k,1} & r_{k,2} & \ldots & r_{k,r\ell}
        \end{pmatrix}, \quad D = (X^n-1) \operatorname{Id}_{r\ell}, \]
    and obtain the Smith normal form of $A$. Let $\overline{S}, \overline{T}$ be invertible matrices such that
    %
    \[ \overline{S}A\overline{T}=\overline{J} = \begin{pmatrix}
            a_1    & \ldots & 0        & 0      & \ldots & 0      \\
            %0 & a_2 & \ldots & 0 & 0 & \ldots & 0\\
            \vdots & \ddots & \vdots   & \vdots &  & \vdots      \\
            0      & \ldots & a_\kappa & 0      & \ldots & 0      \\
            0      & \ldots & 0        & 0      & \ldots & 0      \\
            \vdots &  & \vdots   & \vdots & \ddots & \vdots \\
        \end{pmatrix}  ,\]
    where $a_1, a_2, \ldots, a_\kappa$ with $\kappa \le r\ell$ are the nonzero diagonal entries of $\overline{J}$ with $a_i$ a total divisor of $a_{i+1}$ for all $i=1, \ldots, \kappa-1$. Then,
    %
%    \begin{align*}
%        \begin{pNiceArray}{c|c}
%            S & \mathbf{0} \\
%            \hline
%            \mathbf{0} & T^{-1} \\
%        \end{pNiceArray} 
%        \begin{pNiceArray}{cccc}
%            \Block{2-4}<\Large>{ A } \\
%            \\ \hline
%            X^n-1 & 0 & \ldots & 0\\
%            0 & X^n-1 & \ldots & 0\\
%            \vdots & \vdots & \ddots & \vdots\\
%            0 & 0 & \ldots & X^n-1
%        \end{pNiceArray} T\\
%        = \begin{pNiceArray}{cccc}
%            \Block{2-4}<\Large>{ SAT } \\
%            \\ \hline
%            X^n-1 & 0 & \ldots & 0\\
%            0 & X^n-1 & \ldots & 0\\
%            \vdots & \vdots & \ddots & \vdots\\
%            0 & 0 & \ldots & X^n-1
%        \end{pNiceArray} \\
%        = \begin{pmatrix}
%            a_1    & \ldots & 0        & 0      & \ldots & 0      \\
%            \vdots & \ddots & \vdots   & \vdots & \vdots & 0      \\
%            0      & \ldots & a_\kappa & 0      & \ldots & 0      \\
%            0      & \ldots & 0        & 0      & \ldots & 0      \\
%            \vdots & \vdots & \vdots   & \vdots & \ddots & \vdots \\ \hline
%            X^n-1  & 0      & 0        & 0      & \ldots & 0      \\
%            0      & X^n-1  & 0        & 0      & \ldots & 0      \\
%            0      & 0      & X^n-1    & 0      & \ldots & 0      \\
%            0      & 0      & 0        & X^n-1  & \ldots & 0      \\
%            \vdots & \vdots & \vdots   & \vdots & \ddots & \vdots \\
%            0      & 0      & 0        & 0      & \ldots & X^n-1
%        \end{pmatrix}= : \widetilde{M}.
%    \end{align*}
    \begin{align*}
        \begin{pNiceArray}{c|c}
            \overline{S} & \mathbf{0} \\
            \hline
            \mathbf{0} & \overline{T}^{-1} \\
        \end{pNiceArray} & \begin{pmatrix}
                               A \\ \hline
                               D
                           \end{pmatrix}
        \overline{T} = \begin{pmatrix}
                \overline{S}A\overline{T} \\ \hline
                D
            \end{pmatrix}  = \begin{pmatrix}
                                 \overline{J} \\ \hline
                                 D
                             \end{pmatrix} \\
    \\
        & = \begin{pmatrix}
            a_1 & \ldots & 0        & 0      & \ldots & 0      \\
            \vdots                          & \ddots & \vdots   & \vdots & & \vdots      \\
            0                               & \ldots & a_\kappa & 0      & \ldots & 0      \\
            0                               & \ldots & 0        & 0      & \ldots & 0      \\
            \vdots                          &  & \vdots   & \vdots & \ddots & \vdots \\ \hline
            X^n-1                               & 0      & 0        & 0      & \ldots & 0      \\
            0                               & X^n-1  & 0        & 0      & \ldots & 0      \\
            0                               & 0      & X^n-1    & 0      & \ldots & 0      \\
            0                               & 0      & 0        & X^n-1  & \ldots & 0      \\
            \vdots                          & \vdots & \vdots   & \vdots & \ddots & \vdots \\
            0                               & 0      & 0        & 0      & \ldots & X^n-1
        \end{pmatrix}.
    \end{align*}
    %For each $i=1, \ldots, \kappa$, find $x_i, y_i, u_i\in \F_{q^a}[X;\sigma]$ such that
    %\[\operatorname{gcrd}(a_i, X^n-1)= x_i a_i + y_i (X^n-1) \mbox{ and } X^n-1= u_i \operatorname{gcrd}(a_i, X^n-1).\]
    Applying row operations to the first and $(k+1)$-st row of this matrix, we can implement the Euclidean algorithm for the greatest common right divisor of the elements $a_1$ and $X^n-1$ in the first column of this matrix. This process results in the equivalent matrix   
    \begin{align*}
        & = \begin{pmatrix}
            \operatorname{gcrd}(a_1, X^n-1) & \ldots & 0        & 0      & \ldots & 0      \\
            \vdots                          & \ddots & \vdots   & \vdots & & \vdots      \\
            0                               & \ldots & a_\kappa & 0      & \ldots & 0      \\
            0                               & \ldots & 0        & 0      & \ldots & 0      \\
            \vdots                          &  & \vdots   & \vdots & \ddots & \vdots \\ \hline
            0                               & 0      & 0        & 0      & \ldots & 0      \\
            0                               & X^n-1  & 0        & 0      & \ldots & 0      \\
            0                               & 0      & X^n-1    & 0      & \ldots & 0      \\
            0                               & 0      & 0        & X^n-1  & \ldots & 0      \\
            \vdots                          & \vdots & \vdots   & \vdots & \ddots & \vdots \\
            0                               & 0      & 0        & 0      & \ldots & X^n-1
        \end{pmatrix}.
    \end{align*}
    
    For $i=2, \ldots, \kappa$, %multiply iteratively on the left with the $(k+r\ell) \times (k+r\ell)$ invertible matrix consisting of the identity, except for the $i$-th and $(k+i)$-th row, which should be
    %
    %\[ (0, \ldots, 0,x_i , 0 , \ldots, 0, y_i , 0, \ldots) \mbox{ and }(0, \ldots ,0,-u_i x_i , 0 , \ldots ,0, 1-u_iy_i , 0, \ldots), \]
    %respectively, where the nonzero values are located in the $i$-th and $(k+i)$-th columns.
    repeat this process, applying row operations to the $i$-th and $(k+i)$-th row to reduce the entries
    $a_i$ and $X^n-1$ in column $i$ to $\operatorname{gcrd}(a_i, X^n-1)$ and $0$.
    This results in the equivalent matrix
    %
    \[
        \begin{pmatrix}
            \operatorname{gcrd}(a_1, X^n-1) & \ldots & 0                                    & 0      & \ldots & 0      \\
            \vdots                          & \ddots & \vdots                               & \vdots & & \vdots    \\
            0                               & \ldots & \operatorname{gcrd}(a_\kappa, X^n-1) & 0      & \ldots & 0      \\
            0                               & \ldots & 0                                    & 0      & \ldots & 0      \\
            \vdots                          & & \vdots                               & \vdots & \ddots & \vdots \\ \hline
            0                               & 0      & 0                                    & 0      & \ldots & 0      \\
            0                               & 0      & 0                                    & 0      & \ldots & 0      \\
            0                               & 0      & 0                                    & 0      & \ldots & 0      \\
            0                               & 0      & 0                                    & X^n-1  & \ldots & 0      \\
            \vdots                          & \vdots & \vdots                               & \vdots & \ddots & \vdots \\
            0                               & 0      & 0                                    & 0      & \ldots & X^n-1
        \end{pmatrix}.
    \]
    If necessary, we swap rows, so that we convert this matrix into
    \begin{align}\label{snf:6.1}
        \begin{pmatrix}
            \operatorname{gcrd}(a_1, X^n-1) & \ldots & 0                                    & 0      & \ldots & 0      \\
            \vdots                          & \ddots & \vdots                               & \vdots &  & \vdots      \\
            0                               & \ldots & \operatorname{gcrd}(a_\kappa, X^n-1) & 0      & \ldots & 0      \\
            0                               & \ldots & 0                                    & X^n-1  & \ldots & 0      \\
            \vdots                          & & \vdots                               & \vdots & \ddots & \vdots \\
            0                               & \ldots     & 0                                    & 0      & \ldots & X^n-1  \\
            0                               & \ldots      & 0                                    & 0      & \ldots & 0      \\
            \vdots                          & & \vdots                               & \vdots & \ddots & \vdots \\
        \end{pmatrix}.
    \end{align}
    Recall that $a_i$ is a total divisor of $a_{i+1}$ for all $i=1, \ldots, \kappa-1$. By Theorem~ \ref{thm:grcdmakesmagic}, $\operatorname{gcrd}(a_i, X^n-1)$ is a total divisor of $\operatorname{gcrd}(a_{i+1}, X^n-1)$, while $\operatorname{gcrd}(a_\kappa, X^n-1)\mid \mid X^n-1$ with Proposition \ref{thm:intermediate}(1) and $X^n-1\mid \mid X^n-1$ with Theorem \ref{thm:selftotaldivisor}. Thus, \eqref{snf:6.1} is the Smith normal form of $M$. Then, $h_i$ is either $\operatorname{gcrd}(a_{i}, X^n-1)$ or $X^n-1$. In either case, $h_{i}$ is a total divisor of $X^n-1$ by transitivity.

    Let $d_i\in \F_{q^a}[X;\sigma]$ with $h_i d_i = X^n-1$. Then $(X^n-1)= h_{i}d_{i}=d_{i}h_{i}$ since $(X^n-1)\in Z\left( \F_{q^a}[X;\sigma] \right)$.
    %we have that $(X^n-1)= h_{r\ell}d_{r\ell}=d_{r\ell}h_{r\ell}$. Let $g\in \F_{q^m}[X;\sigma]$. Then,
    %\[(X^n-1)g= g(X^n-1)= gd_{r\ell}h_{r\ell}, \mbox{  and  }g(X^n-1)= (X^n-1)g=   h_{r\ell}d_{r\ell}g. \]
    %Hence, $h_{r\ell}$ is right divisor of $(X^n-1)g$ and left divisor of $g(X^n-1)$. That is, $h_{r\ell}$ is a total divisor of $X^n-1$.
\end{proof}
%{\color{red}
\begin{remark}
    %Short proof for commutative case.
   Let $a=1$, i.e., consider the extension $\F_q\leq \F_{q^m}$. Then $\F_{q^a}[X;\sigma]=\F_q[X]$ is commutative under the multiplication and we can make use of the determinant of matrices to analyze the Smith normal form of the matrix $M$ in \eqref{Smith1} and \eqref{Smith2}. Notice that $h_1h_2\ldots h_i$ is the greatest common divisor of the determinants of all the $i\times i $ minors of $M$. Then $h_1$ divides every entry of $M$, since these are the determinants of the $1\times 1$ minors. Thus, $h_1$ divides $X^n-1$. Now, consider an $i\times i$ minor of $M$ with determinant $\delta$. Use a row from the bottom of $M$ to obtain an $(i+1)\times(i+1)$ minor of $M$ with determinant $\delta (X^n-1)$. It follows that $h_1 \ldots h_{i+1}$ divides the determinant of such a minor, regardless of $\delta$ (or the $i\times i$ minor chosen); that is, $h_1 \ldots, h_{i+1}$ divides \[\operatorname{gcd}(\mbox{determinants of } i\times i \mbox{ minors})\cdot (X^n-1)= h_1\ldots h_i\cdot (X^n-1).\] As $\F_q[X]$ is a domain, we get that $h_{i+1}$ divides $X^n-1$.
\end{remark}%}

Let $\boldsymbol{q}_i$ denote the rows of $T^{-1}$ in \eqref{Smith2}. Let $\boldsymbol{c}_i := h_i \boldsymbol{q}_i$. By the Smith normal form we have that $\boldsymbol{q}_1, \ldots, \boldsymbol{q}_{r\ell}$ is a basis of $\F_{q^a}[X;\sigma]^{r\ell}$ and that $\boldsymbol{c}_1, \ldots, \boldsymbol{c}_{r\ell}$ is a basis of $\pi^{-1}(\C)$.

Let $d_i$ be the right quotient of $X^n-1$ by $h_i$. Since $X^n-1 \in Z\left( \F_{q^a}[X;\sigma] \right)$, we have that $h_i d_i = d_i h_i = X^n-1$. Moreover, $d_i \boldsymbol{c}_i = d_i h_i\boldsymbol{q}_i =(X^n-1)\boldsymbol{q}_i$, and $d_1 \boldsymbol{c}_1, \ldots, d_{r\ell} \boldsymbol{c}_{r\ell}$ is a basis of $(X^n-1)\F_{q^a}[X;\sigma]^{r\ell}$.

\begin{theorem}\label{thm:main_theorem}
    Let $\C$ be an $\F_{q^a}$-linear skew quasi-cyclic $\F_{q^m}$-code of index $\ell$ and length $n\ell$. Then, both as an $\F_{q^a}[X;\sigma]$-module and as a $P_n$-module,
    %
    \begin{align*}
        \C & \simeq \pi^{-1}(\C) \Big/ \left((X^n-1)\F_{q^a}[X;\sigma]^{r\ell} \right)                                                                                                \\
           & =\left(\bigoplus_{i=1}^{r\ell} \F_{q^a} [X;\sigma]\boldsymbol{c}_i \right) \Big/ \left(\bigoplus_{i=1}^{r\ell}  \F_{q^a} [X;\sigma]d_i\boldsymbol{c}_i \right) \\
           & \simeq \bigoplus_{i=1}^{r\ell} \left(  \F_{q^a} [X;\sigma] \Big/  \F_{q^a} [X;\sigma]d_i  \right).
    \end{align*}
\end{theorem}
%\subsection{Dual for skew quasi}
%\subsection{Dual of an $\F_{q}$-linear quasi-cyclic $\F_{q^m}$-code}
Given the above decomposition, a natural question of the appropriate dual code for $\C$ arises.
%Let $a=1$. 
%Recall that
%
%\begin{equation}\label{cistardecomp}
%    \C \simeq \left(\bigoplus_{i=1}^{r\ell} \F_{q^a} [X;\sigma]\boldsymbol{c}_i  \right) \Big/ \left(\bigoplus_{i=1}^{r\ell}  \F_{q^a} [X;\sigma]d_i \boldsymbol{c}_i  \right).
%\end{equation}
For this, we introduce an inner product associated with the basis vectors $\boldsymbol{q}_i$.

\begin{definition}
    %Given the decomposition \eqref{cistardecomp}, d
    Denote
    %
    \[ Q= \begin{pmatrix}
            \boldsymbol{q}_1 \\
            \boldsymbol{q}_2 \\
            \vdots           \\
            \boldsymbol{q}_{r\ell}
        \end{pmatrix}.\]

    Since $\{\boldsymbol{q}_1, \ldots, \boldsymbol{q}_{r\ell} \}$ is a basis of $\F_{q^a}[X;\sigma]^{r\ell}$ as an $\F_{q^a}[X;\sigma]$-vector space, we have that $Q$ is invertible over $\F_{q^a}[X;\sigma]$. For row vectors $\boldsymbol{a},\boldsymbol{b}\in \F_{q^a}[X;\sigma]^{r\ell}$, define their \emph{$Q$-inner product} by
    %
    \[\langle\boldsymbol{a}, \boldsymbol{b}\rangle_Q:= \boldsymbol{a} Q^{-1} \left[ Q^{-1} \right]^T  \boldsymbol{b}^T =\langle \boldsymbol{a}Q^{-1}, \boldsymbol{b}Q^{-1}\rangle. \]

    This is a symmetric nondegenerate bilinear inner product. Furthermore, this inner product has good properties with respect to the dual code defined below.

    Similarly, for $\overline{\boldsymbol{a}},\overline{\boldsymbol{b}}\in  P_n^{r\ell}$, define their \emph{$Q$-inner product} by $\left\langle\overline{\boldsymbol{a}}, \overline{\boldsymbol{b}}\right\rangle_Q:= \langle\boldsymbol{a}, \boldsymbol{b}\rangle_Q+(X^n-1)\in P_n$, where $\pi: \F_{q^a}[X;\sigma]^{r\ell} \to P_n^{r\ell}$ denotes the canonical projection,
    $\overline{\boldsymbol{a}}=\pi(\bs{a})$ and $\overline{\boldsymbol{b}}=\pi(\bs{b})$ for some $\bs{a}, \bs{b}\in \F_{q^a}[X;\sigma]^{r\ell}$.
    %Notice that $P_n^r\simeq R_n$ and $P_n^{r\ell}\simeq R_n^\ell$.
    %$\overline{\boldsymbol{a}}=\bs{a}+(X^n-1)\F_{q^a}[X;\sigma]^{r\ell}$ and $\overline{\boldsymbol{b}}=\bs{b}+(X^n-1)\F_{q^a}[X;\sigma]^{r\ell}$.
    %
\end{definition}

%Let $i\in \{1, \ldots, r\ell\}$.
%We have $h_i \in \F_{q^a}[X;\sigma]$ such that $d_i h_i = h_id_i = X^n-1$. Besides, $d_i \boldsymbol{c}_i  \in (X^n-1)\F_{q^a}[X;\sigma]^{r\ell}$; i.e., there exists $\boldsymbol{q}_i\in \F_{q^a}[X;\sigma]^{r\ell}$ such that $d_i \boldsymbol{c}_i = (X^n-1)\boldsymbol{q}_i$.
%Then ${c}_i = h_i \boldsymbol{q}_i$ and $h_i d_i = d_i h_i = X^n-1$. Hence, if we define $\boldsymbol{c}_i' := d_i \boldsymbol{q}_i$,
%we obtain that:
%
%\begin{align*}
%    \left\langle h_1\boldsymbol{c}_1' , h_2\boldsymbol{c}_2' , \ldots, h_{r\ell}\boldsymbol{c}_{r\ell}' \right\rangle_{\F_{q^a}[X;\sigma]} & = \left\langle h_1d_1 \boldsymbol{q}_1, h_2d_2 \boldsymbol{q}_2, \ldots, h_{r\ell}d_{r\ell} \boldsymbol{q}_{r\ell}\right\rangle_{\F_{q^a}[X;\sigma]} \\
%    & = \left\langle d_1 \boldsymbol{c}_1 , d_2 \boldsymbol{c}_2 , \ldots, d_{r\ell} \boldsymbol{c}_{r\ell} \right\rangle_{\F_{q^a}[X;\sigma]} \\
%    & = (X^n-1)\F_{q^a}[X;\sigma]^{r\ell}.
%\end{align*}
%
%Moreover, we can write $\boldsymbol{c}_i' = \frac{d_i^2}{X^n-1}\boldsymbol{c}_i $.
%The vectors $\boldsymbol{q}_i$ provide us with a basis of $\F_{q^a}[X;\sigma]^{r\ell}$ as a free $\F_{q^a}[X;\sigma]$-module.

\begin{definition} \label{star}
    Let $\C$ be an $\F_{q^a}$-linear skew quasi-cyclic code of index $\ell$ and length $n\ell$ such that
    \begin{align}\label{codedecomposition}
        \C \simeq \left(\bigoplus_{i=1}^{r\ell} \F_{q^a} [X;\sigma]\boldsymbol{c}_i  \right) \Big/ \left(\bigoplus_{i=1}^{r\ell}  \F_{q^a} [X;\sigma]d_i\boldsymbol{c}_i  \right).
    \end{align}

    Define the \emph{$*$-dual of $\C$} (with respect to $\boldsymbol{q}_1, \ldots, \boldsymbol{q}_{r\ell}$) by
    \[\C_*^\perp := \left\langle \boldsymbol{c}_1'  , %\boldsymbol{c}_2' +(X^n-1)\F_{q^a}[X;\sigma]^{r\ell} , 
        \ldots, \boldsymbol{c}_{r\ell}'  \right\rangle_{\F_{q^a}[X;\sigma]} +(X^n-1)\F_{q^a}[X;\sigma]^{r\ell}, \]
    where ${c}_i = h_i \boldsymbol{q}_i$, $h_i d_i = d_i h_i = X^n-1$, and $\boldsymbol{c}_i' := d_i \boldsymbol{q}_i$.

    Note that the $\boldsymbol{c}_i'$ are linearly independent over $\F_{q^a}[X;\sigma]$ since the $\boldsymbol{q}_i$ form a basis for $\F_{q^a}[X;\sigma]^{r\ell}$.
\end{definition}

This definition provides us with a good dual, since it allows us to recover the original code if we keep track of the basis elements $\boldsymbol{q}_i$.

\begin{theorem}\label{thm:gen_dual}
    Let $\C$ be an $\F_{q^a}$-linear skew quasi-cyclic code of index $\ell$ and length $n\ell$, and suppose that $\C_{*}^\perp$ is defined with respect to the decomposition \eqref{codedecomposition} as above.
    \begin{enumerate}
        \item[$(1)$] If $\overline{\boldsymbol{c}}\in \C$ and $\overline{\boldsymbol{x}}\in  \C_*^\perp$, then $\left\langle\overline{\boldsymbol{c}}, \overline{\boldsymbol{x}}\right\rangle_Q=0_{P_n}$.
        \item[$(2)$] When constructed with respect to the same basis $\{\boldsymbol{q}_1, \ldots, \boldsymbol{q}_{r\ell} \}$, we have $\left( \C_*^\perp \right)_*^\perp =\C.$
              %\item[$(b)$] $\left( \C_*^\perp \right)_*^\perp =\C.$
    \end{enumerate}
\end{theorem}

\begin{proof}
    ${(1)}$ We have that $\bs{q}_i Q^{-1}=\bs{e}_i$, where $\bs{e}_i$ is the row vector with entry 1 in the $i$-th position, and entries $0$ elsewhere.
    Let $\overline{\bs{c}_i}= \bs{c}_i+(X^n-1)\F_{q^a}[X;\sigma]^{r\ell}$ and $\overline{ \bs{c}_j }'= \boldsymbol{c}_j'+(X^n-1)\F_{q^a}[X;\sigma]^{r\ell}$.
    Then for all $i,j\in \{1,\ldots, r\ell\}$,
    %
    \begin{align*}
        \left\langle\overline{\boldsymbol{c}_i},\overline{\boldsymbol{c}_j}'\right\rangle_Q & = \langle\boldsymbol{c}_i, \boldsymbol{c}_j'\rangle_Q + (X^n-1)     \\
        & = \langle h_i\boldsymbol{q}_iQ^{-1}, d_j\boldsymbol{q}_jQ^{-1}\rangle +(X^n-1) \\
        & = \langle h_i \bs{e}_i, d_j \bs{e}_j\rangle +(X^n-1) \\
        & = h_i d_j \delta_{ij} +(X^n-1) \\
        & =(X^n-1)\delta_{ij} +(X^n-1) = 0_{P_n}.
    \end{align*}
    %That is, $\boldsymbol{c}_i\cdot \boldsymbol{c}_j'=0$ in $P_n$ for all $i,j=1, \ldots, r\ell$. 
    Since $\C =\langle \overline{\bs{c}_1}, \ldots, \overline{\bs{c}_{r\ell}}\rangle_{\F_{q^a}[X;\sigma]}$ and $\C_*^\perp =\langle \overline{\bs{c}_1}', \ldots, \overline{\bs{c}_{r\ell}}'\rangle_{\F_{q^a}[X;\sigma]}$, the result follow by bilinearity of the inner product.

    ${(2)}$ Let $i\in \{1, \ldots, r\ell\}$. Recall that $\boldsymbol{c}_i = h_i \boldsymbol{q}_i$ and $\boldsymbol{c}_i' = d_i \boldsymbol{q}_i$. To construct $\left( \C_*^\perp \right)_*^\perp$,
    Definition \ref{star} instructs us to decompose $\boldsymbol{c}_i' = d_i \boldsymbol{q}_i$ to
    define new basis vectors $\boldsymbol{c}_i'' := h_i \boldsymbol{q}_i= \boldsymbol{c}_i$ for $\left( \C_*^\perp \right)_*^\perp$.
    %we must left divide $X^n-1$ by $d_i$. That is,
    %\[\bs{c}_i''= \frac{X^n-1}{d_i}\bs{q}_i= h_i \bs{q}_i =\bs{c}_i. \]
    Hence $\left( \C_*^\perp \right)_*^\perp$ and $\C$ have the same basis.
    %Consequence of $h_i d_i = d_ih_i=X^n-1$.
\end{proof}

\section{Conclusion and Future Work}\label{sec:future}
This article introduces a nonlinear generalization of Reed-Solomon codes. The case when these codes are of ``full length'' then leads to a generalization of the nonlinear cyclic codes introduced by Huffman in \cite{Huffman:2013}. These \emph{nonlinear skew quasi-cyclic codes} simultaneously generalize the quasi-cyclic codes explored by Ling, Sol\'e, and Niederreiter, \cite{Ling-Sole:2006, Ling-Sole:2001, Ling-Sole:2003, Ling-Sole:2005}, and Lally and Fitzpatrick, \cite{Lally-Fitzpatrick:2001}, and the skew cyclic and skew quasi-cyclic codes explored by Boucher and Ulmer in \cite{Boucher-Ulmer:2009} and Abualrub et al. in \cite{Abualrub-et-al:2010}. Further exploration of these codes shows that they correspond to the left $P_n$-submodules of the left $P_n$-module $R_n^\ell$. We then utilize an iterated Smith normal form to give a classification of nonlinear skew quasi-cyclic codes, namely, Theorem \ref{thm:main_theorem}. This module-theoretic classification leads to Theorem \ref{thm:gen_dual}, which provides a natural generalization of the standard dual code of a nonlinear skew quasi-cyclic code.

While the present article gives a definition for a dual code of a given nonlinear $\sigma$-QC code, we do not know if this is the ``best'' definition of a dual code. The forthcoming article \cite{BHR_2} will continue an exploration of this topic.

\printbibliography
\end{document}