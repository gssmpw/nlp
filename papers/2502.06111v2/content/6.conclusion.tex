\section{Conclusion} \label{conclusion}
\vspace{-0.5em}
%\subsection{Contributions}

The work introduces \model, a benchmark designed to evaluate the capabilities of LLM agents in automating the deployment of GitHub repositories for scientific research. Our study highlights that while LLMs show potential in handling tasks like environment setup and data preparation, they face challenges in complex tasks such as training and inference, where success rates are notably lower.

Our multi-agent framework, \agent, exemplifies how LLMs can collaborate to tackle deployment challenges, offering a promising approach to improving automation in software engineering. However, the results indicate that further advancements are needed to fully realize autonomous and reliable deployment processes.

Overall, \model serves as a crucial tool for assessing and improving LLM-driven deployment workflows in scientific research, paving the way for more efficient and automated computer science projects exploration.



%\subsection{Limitations}
%A large portion of repos are collections of multiple conferences. They occupy the priority of the science repos related to the conference.