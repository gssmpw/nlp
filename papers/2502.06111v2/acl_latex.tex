% This must be in the first 5 lines to tell arXiv to use pdfLaTeX, which is strongly recommended.
\pdfoutput=1
% In particular, the hyperref package requires pdfLaTeX in order to break URLs across lines.

\documentclass[11pt]{article}

% Change "review" to "final" to generate the final (sometimes called camera-ready) version.
% Change to "preprint" to generate a non-anonymous version with page numbers.
\usepackage[preprint]{acl}

% Standard package includes
\usepackage{times}
\usepackage{latexsym}

% For proper rendering and hyphenation of words containing Latin characters (including in bib files)
\usepackage[T1]{fontenc}
% For Vietnamese characters
% \usepackage[T5]{fontenc}
% See https://www.latex-project.org/help/documentation/encguide.pdf for other character sets

% This assumes your files are encoded as UTF8
\usepackage[utf8]{inputenc}

% This is not strictly necessary, and may be commented out,
% but it will improve the layout of the manuscript,
% and will typically save some space.
\usepackage{microtype}

% This is also not strictly necessary, and may be commented out.
% However, it will improve the aesthetics of text in
% the typewriter font.
\usepackage{inconsolata}

%Including images in your LaTeX document requires adding
%additional package(s)
\usepackage{graphicx}

%\usepackage[symbol]{footmisc}

%%%%% NEW MATH DEFINITIONS %%%%%

\usepackage{amsmath,amsfonts,bm}
\usepackage{derivative}
% Mark sections of captions for referring to divisions of figures
\newcommand{\figleft}{{\em (Left)}}
\newcommand{\figcenter}{{\em (Center)}}
\newcommand{\figright}{{\em (Right)}}
\newcommand{\figtop}{{\em (Top)}}
\newcommand{\figbottom}{{\em (Bottom)}}
\newcommand{\captiona}{{\em (a)}}
\newcommand{\captionb}{{\em (b)}}
\newcommand{\captionc}{{\em (c)}}
\newcommand{\captiond}{{\em (d)}}

% Highlight a newly defined term
\newcommand{\newterm}[1]{{\bf #1}}

% Derivative d 
\newcommand{\deriv}{{\mathrm{d}}}

% Figure reference, lower-case.
\def\figref#1{figure~\ref{#1}}
% Figure reference, capital. For start of sentence
\def\Figref#1{Figure~\ref{#1}}
\def\twofigref#1#2{figures \ref{#1} and \ref{#2}}
\def\quadfigref#1#2#3#4{figures \ref{#1}, \ref{#2}, \ref{#3} and \ref{#4}}
% Section reference, lower-case.
\def\secref#1{section~\ref{#1}}
% Section reference, capital.
\def\Secref#1{Section~\ref{#1}}
% Reference to two sections.
\def\twosecrefs#1#2{sections \ref{#1} and \ref{#2}}
% Reference to three sections.
\def\secrefs#1#2#3{sections \ref{#1}, \ref{#2} and \ref{#3}}
% Reference to an equation, lower-case.
\def\eqref#1{equation~\ref{#1}}
% Reference to an equation, upper case
\def\Eqref#1{Equation~\ref{#1}}
% A raw reference to an equation---avoid using if possible
\def\plaineqref#1{\ref{#1}}
% Reference to a chapter, lower-case.
\def\chapref#1{chapter~\ref{#1}}
% Reference to an equation, upper case.
\def\Chapref#1{Chapter~\ref{#1}}
% Reference to a range of chapters
\def\rangechapref#1#2{chapters\ref{#1}--\ref{#2}}
% Reference to an algorithm, lower-case.
\def\algref#1{algorithm~\ref{#1}}
% Reference to an algorithm, upper case.
\def\Algref#1{Algorithm~\ref{#1}}
\def\twoalgref#1#2{algorithms \ref{#1} and \ref{#2}}
\def\Twoalgref#1#2{Algorithms \ref{#1} and \ref{#2}}
% Reference to a part, lower case
\def\partref#1{part~\ref{#1}}
% Reference to a part, upper case
\def\Partref#1{Part~\ref{#1}}
\def\twopartref#1#2{parts \ref{#1} and \ref{#2}}

\def\ceil#1{\lceil #1 \rceil}
\def\floor#1{\lfloor #1 \rfloor}
\def\1{\bm{1}}
\newcommand{\train}{\mathcal{D}}
\newcommand{\valid}{\mathcal{D_{\mathrm{valid}}}}
\newcommand{\test}{\mathcal{D_{\mathrm{test}}}}

\def\eps{{\epsilon}}


% Random variables
\def\reta{{\textnormal{$\eta$}}}
\def\ra{{\textnormal{a}}}
\def\rb{{\textnormal{b}}}
\def\rc{{\textnormal{c}}}
\def\rd{{\textnormal{d}}}
\def\re{{\textnormal{e}}}
\def\rf{{\textnormal{f}}}
\def\rg{{\textnormal{g}}}
\def\rh{{\textnormal{h}}}
\def\ri{{\textnormal{i}}}
\def\rj{{\textnormal{j}}}
\def\rk{{\textnormal{k}}}
\def\rl{{\textnormal{l}}}
% rm is already a command, just don't name any random variables m
\def\rn{{\textnormal{n}}}
\def\ro{{\textnormal{o}}}
\def\rp{{\textnormal{p}}}
\def\rq{{\textnormal{q}}}
\def\rr{{\textnormal{r}}}
\def\rs{{\textnormal{s}}}
\def\rt{{\textnormal{t}}}
\def\ru{{\textnormal{u}}}
\def\rv{{\textnormal{v}}}
\def\rw{{\textnormal{w}}}
\def\rx{{\textnormal{x}}}
\def\ry{{\textnormal{y}}}
\def\rz{{\textnormal{z}}}

% Random vectors
\def\rvepsilon{{\mathbf{\epsilon}}}
\def\rvphi{{\mathbf{\phi}}}
\def\rvtheta{{\mathbf{\theta}}}
\def\rva{{\mathbf{a}}}
\def\rvb{{\mathbf{b}}}
\def\rvc{{\mathbf{c}}}
\def\rvd{{\mathbf{d}}}
\def\rve{{\mathbf{e}}}
\def\rvf{{\mathbf{f}}}
\def\rvg{{\mathbf{g}}}
\def\rvh{{\mathbf{h}}}
\def\rvu{{\mathbf{i}}}
\def\rvj{{\mathbf{j}}}
\def\rvk{{\mathbf{k}}}
\def\rvl{{\mathbf{l}}}
\def\rvm{{\mathbf{m}}}
\def\rvn{{\mathbf{n}}}
\def\rvo{{\mathbf{o}}}
\def\rvp{{\mathbf{p}}}
\def\rvq{{\mathbf{q}}}
\def\rvr{{\mathbf{r}}}
\def\rvs{{\mathbf{s}}}
\def\rvt{{\mathbf{t}}}
\def\rvu{{\mathbf{u}}}
\def\rvv{{\mathbf{v}}}
\def\rvw{{\mathbf{w}}}
\def\rvx{{\mathbf{x}}}
\def\rvy{{\mathbf{y}}}
\def\rvz{{\mathbf{z}}}

% Elements of random vectors
\def\erva{{\textnormal{a}}}
\def\ervb{{\textnormal{b}}}
\def\ervc{{\textnormal{c}}}
\def\ervd{{\textnormal{d}}}
\def\erve{{\textnormal{e}}}
\def\ervf{{\textnormal{f}}}
\def\ervg{{\textnormal{g}}}
\def\ervh{{\textnormal{h}}}
\def\ervi{{\textnormal{i}}}
\def\ervj{{\textnormal{j}}}
\def\ervk{{\textnormal{k}}}
\def\ervl{{\textnormal{l}}}
\def\ervm{{\textnormal{m}}}
\def\ervn{{\textnormal{n}}}
\def\ervo{{\textnormal{o}}}
\def\ervp{{\textnormal{p}}}
\def\ervq{{\textnormal{q}}}
\def\ervr{{\textnormal{r}}}
\def\ervs{{\textnormal{s}}}
\def\ervt{{\textnormal{t}}}
\def\ervu{{\textnormal{u}}}
\def\ervv{{\textnormal{v}}}
\def\ervw{{\textnormal{w}}}
\def\ervx{{\textnormal{x}}}
\def\ervy{{\textnormal{y}}}
\def\ervz{{\textnormal{z}}}

% Random matrices
\def\rmA{{\mathbf{A}}}
\def\rmB{{\mathbf{B}}}
\def\rmC{{\mathbf{C}}}
\def\rmD{{\mathbf{D}}}
\def\rmE{{\mathbf{E}}}
\def\rmF{{\mathbf{F}}}
\def\rmG{{\mathbf{G}}}
\def\rmH{{\mathbf{H}}}
\def\rmI{{\mathbf{I}}}
\def\rmJ{{\mathbf{J}}}
\def\rmK{{\mathbf{K}}}
\def\rmL{{\mathbf{L}}}
\def\rmM{{\mathbf{M}}}
\def\rmN{{\mathbf{N}}}
\def\rmO{{\mathbf{O}}}
\def\rmP{{\mathbf{P}}}
\def\rmQ{{\mathbf{Q}}}
\def\rmR{{\mathbf{R}}}
\def\rmS{{\mathbf{S}}}
\def\rmT{{\mathbf{T}}}
\def\rmU{{\mathbf{U}}}
\def\rmV{{\mathbf{V}}}
\def\rmW{{\mathbf{W}}}
\def\rmX{{\mathbf{X}}}
\def\rmY{{\mathbf{Y}}}
\def\rmZ{{\mathbf{Z}}}

% Elements of random matrices
\def\ermA{{\textnormal{A}}}
\def\ermB{{\textnormal{B}}}
\def\ermC{{\textnormal{C}}}
\def\ermD{{\textnormal{D}}}
\def\ermE{{\textnormal{E}}}
\def\ermF{{\textnormal{F}}}
\def\ermG{{\textnormal{G}}}
\def\ermH{{\textnormal{H}}}
\def\ermI{{\textnormal{I}}}
\def\ermJ{{\textnormal{J}}}
\def\ermK{{\textnormal{K}}}
\def\ermL{{\textnormal{L}}}
\def\ermM{{\textnormal{M}}}
\def\ermN{{\textnormal{N}}}
\def\ermO{{\textnormal{O}}}
\def\ermP{{\textnormal{P}}}
\def\ermQ{{\textnormal{Q}}}
\def\ermR{{\textnormal{R}}}
\def\ermS{{\textnormal{S}}}
\def\ermT{{\textnormal{T}}}
\def\ermU{{\textnormal{U}}}
\def\ermV{{\textnormal{V}}}
\def\ermW{{\textnormal{W}}}
\def\ermX{{\textnormal{X}}}
\def\ermY{{\textnormal{Y}}}
\def\ermZ{{\textnormal{Z}}}

% Vectors
\def\vzero{{\bm{0}}}
\def\vone{{\bm{1}}}
\def\vmu{{\bm{\mu}}}
\def\vtheta{{\bm{\theta}}}
\def\vphi{{\bm{\phi}}}
\def\va{{\bm{a}}}
\def\vb{{\bm{b}}}
\def\vc{{\bm{c}}}
\def\vd{{\bm{d}}}
\def\ve{{\bm{e}}}
\def\vf{{\bm{f}}}
\def\vg{{\bm{g}}}
\def\vh{{\bm{h}}}
\def\vi{{\bm{i}}}
\def\vj{{\bm{j}}}
\def\vk{{\bm{k}}}
\def\vl{{\bm{l}}}
\def\vm{{\bm{m}}}
\def\vn{{\bm{n}}}
\def\vo{{\bm{o}}}
\def\vp{{\bm{p}}}
\def\vq{{\bm{q}}}
\def\vr{{\bm{r}}}
\def\vs{{\bm{s}}}
\def\vt{{\bm{t}}}
\def\vu{{\bm{u}}}
\def\vv{{\bm{v}}}
\def\vw{{\bm{w}}}
\def\vx{{\bm{x}}}
\def\vy{{\bm{y}}}
\def\vz{{\bm{z}}}

% Elements of vectors
\def\evalpha{{\alpha}}
\def\evbeta{{\beta}}
\def\evepsilon{{\epsilon}}
\def\evlambda{{\lambda}}
\def\evomega{{\omega}}
\def\evmu{{\mu}}
\def\evpsi{{\psi}}
\def\evsigma{{\sigma}}
\def\evtheta{{\theta}}
\def\eva{{a}}
\def\evb{{b}}
\def\evc{{c}}
\def\evd{{d}}
\def\eve{{e}}
\def\evf{{f}}
\def\evg{{g}}
\def\evh{{h}}
\def\evi{{i}}
\def\evj{{j}}
\def\evk{{k}}
\def\evl{{l}}
\def\evm{{m}}
\def\evn{{n}}
\def\evo{{o}}
\def\evp{{p}}
\def\evq{{q}}
\def\evr{{r}}
\def\evs{{s}}
\def\evt{{t}}
\def\evu{{u}}
\def\evv{{v}}
\def\evw{{w}}
\def\evx{{x}}
\def\evy{{y}}
\def\evz{{z}}

% Matrix
\def\mA{{\bm{A}}}
\def\mB{{\bm{B}}}
\def\mC{{\bm{C}}}
\def\mD{{\bm{D}}}
\def\mE{{\bm{E}}}
\def\mF{{\bm{F}}}
\def\mG{{\bm{G}}}
\def\mH{{\bm{H}}}
\def\mI{{\bm{I}}}
\def\mJ{{\bm{J}}}
\def\mK{{\bm{K}}}
\def\mL{{\bm{L}}}
\def\mM{{\bm{M}}}
\def\mN{{\bm{N}}}
\def\mO{{\bm{O}}}
\def\mP{{\bm{P}}}
\def\mQ{{\bm{Q}}}
\def\mR{{\bm{R}}}
\def\mS{{\bm{S}}}
\def\mT{{\bm{T}}}
\def\mU{{\bm{U}}}
\def\mV{{\bm{V}}}
\def\mW{{\bm{W}}}
\def\mX{{\bm{X}}}
\def\mY{{\bm{Y}}}
\def\mZ{{\bm{Z}}}
\def\mBeta{{\bm{\beta}}}
\def\mPhi{{\bm{\Phi}}}
\def\mLambda{{\bm{\Lambda}}}
\def\mSigma{{\bm{\Sigma}}}

% Tensor
\DeclareMathAlphabet{\mathsfit}{\encodingdefault}{\sfdefault}{m}{sl}
\SetMathAlphabet{\mathsfit}{bold}{\encodingdefault}{\sfdefault}{bx}{n}
\newcommand{\tens}[1]{\bm{\mathsfit{#1}}}
\def\tA{{\tens{A}}}
\def\tB{{\tens{B}}}
\def\tC{{\tens{C}}}
\def\tD{{\tens{D}}}
\def\tE{{\tens{E}}}
\def\tF{{\tens{F}}}
\def\tG{{\tens{G}}}
\def\tH{{\tens{H}}}
\def\tI{{\tens{I}}}
\def\tJ{{\tens{J}}}
\def\tK{{\tens{K}}}
\def\tL{{\tens{L}}}
\def\tM{{\tens{M}}}
\def\tN{{\tens{N}}}
\def\tO{{\tens{O}}}
\def\tP{{\tens{P}}}
\def\tQ{{\tens{Q}}}
\def\tR{{\tens{R}}}
\def\tS{{\tens{S}}}
\def\tT{{\tens{T}}}
\def\tU{{\tens{U}}}
\def\tV{{\tens{V}}}
\def\tW{{\tens{W}}}
\def\tX{{\tens{X}}}
\def\tY{{\tens{Y}}}
\def\tZ{{\tens{Z}}}


% Graph
\def\gA{{\mathcal{A}}}
\def\gB{{\mathcal{B}}}
\def\gC{{\mathcal{C}}}
\def\gD{{\mathcal{D}}}
\def\gE{{\mathcal{E}}}
\def\gF{{\mathcal{F}}}
\def\gG{{\mathcal{G}}}
\def\gH{{\mathcal{H}}}
\def\gI{{\mathcal{I}}}
\def\gJ{{\mathcal{J}}}
\def\gK{{\mathcal{K}}}
\def\gL{{\mathcal{L}}}
\def\gM{{\mathcal{M}}}
\def\gN{{\mathcal{N}}}
\def\gO{{\mathcal{O}}}
\def\gP{{\mathcal{P}}}
\def\gQ{{\mathcal{Q}}}
\def\gR{{\mathcal{R}}}
\def\gS{{\mathcal{S}}}
\def\gT{{\mathcal{T}}}
\def\gU{{\mathcal{U}}}
\def\gV{{\mathcal{V}}}
\def\gW{{\mathcal{W}}}
\def\gX{{\mathcal{X}}}
\def\gY{{\mathcal{Y}}}
\def\gZ{{\mathcal{Z}}}

% Sets
\def\sA{{\mathbb{A}}}
\def\sB{{\mathbb{B}}}
\def\sC{{\mathbb{C}}}
\def\sD{{\mathbb{D}}}
% Don't use a set called E, because this would be the same as our symbol
% for expectation.
\def\sF{{\mathbb{F}}}
\def\sG{{\mathbb{G}}}
\def\sH{{\mathbb{H}}}
\def\sI{{\mathbb{I}}}
\def\sJ{{\mathbb{J}}}
\def\sK{{\mathbb{K}}}
\def\sL{{\mathbb{L}}}
\def\sM{{\mathbb{M}}}
\def\sN{{\mathbb{N}}}
\def\sO{{\mathbb{O}}}
\def\sP{{\mathbb{P}}}
\def\sQ{{\mathbb{Q}}}
\def\sR{{\mathbb{R}}}
\def\sS{{\mathbb{S}}}
\def\sT{{\mathbb{T}}}
\def\sU{{\mathbb{U}}}
\def\sV{{\mathbb{V}}}
\def\sW{{\mathbb{W}}}
\def\sX{{\mathbb{X}}}
\def\sY{{\mathbb{Y}}}
\def\sZ{{\mathbb{Z}}}

% Entries of a matrix
\def\emLambda{{\Lambda}}
\def\emA{{A}}
\def\emB{{B}}
\def\emC{{C}}
\def\emD{{D}}
\def\emE{{E}}
\def\emF{{F}}
\def\emG{{G}}
\def\emH{{H}}
\def\emI{{I}}
\def\emJ{{J}}
\def\emK{{K}}
\def\emL{{L}}
\def\emM{{M}}
\def\emN{{N}}
\def\emO{{O}}
\def\emP{{P}}
\def\emQ{{Q}}
\def\emR{{R}}
\def\emS{{S}}
\def\emT{{T}}
\def\emU{{U}}
\def\emV{{V}}
\def\emW{{W}}
\def\emX{{X}}
\def\emY{{Y}}
\def\emZ{{Z}}
\def\emSigma{{\Sigma}}

% entries of a tensor
% Same font as tensor, without \bm wrapper
\newcommand{\etens}[1]{\mathsfit{#1}}
\def\etLambda{{\etens{\Lambda}}}
\def\etA{{\etens{A}}}
\def\etB{{\etens{B}}}
\def\etC{{\etens{C}}}
\def\etD{{\etens{D}}}
\def\etE{{\etens{E}}}
\def\etF{{\etens{F}}}
\def\etG{{\etens{G}}}
\def\etH{{\etens{H}}}
\def\etI{{\etens{I}}}
\def\etJ{{\etens{J}}}
\def\etK{{\etens{K}}}
\def\etL{{\etens{L}}}
\def\etM{{\etens{M}}}
\def\etN{{\etens{N}}}
\def\etO{{\etens{O}}}
\def\etP{{\etens{P}}}
\def\etQ{{\etens{Q}}}
\def\etR{{\etens{R}}}
\def\etS{{\etens{S}}}
\def\etT{{\etens{T}}}
\def\etU{{\etens{U}}}
\def\etV{{\etens{V}}}
\def\etW{{\etens{W}}}
\def\etX{{\etens{X}}}
\def\etY{{\etens{Y}}}
\def\etZ{{\etens{Z}}}

% The true underlying data generating distribution
\newcommand{\pdata}{p_{\rm{data}}}
\newcommand{\ptarget}{p_{\rm{target}}}
\newcommand{\pprior}{p_{\rm{prior}}}
\newcommand{\pbase}{p_{\rm{base}}}
\newcommand{\pref}{p_{\rm{ref}}}

% The empirical distribution defined by the training set
\newcommand{\ptrain}{\hat{p}_{\rm{data}}}
\newcommand{\Ptrain}{\hat{P}_{\rm{data}}}
% The model distribution
\newcommand{\pmodel}{p_{\rm{model}}}
\newcommand{\Pmodel}{P_{\rm{model}}}
\newcommand{\ptildemodel}{\tilde{p}_{\rm{model}}}
% Stochastic autoencoder distributions
\newcommand{\pencode}{p_{\rm{encoder}}}
\newcommand{\pdecode}{p_{\rm{decoder}}}
\newcommand{\precons}{p_{\rm{reconstruct}}}

\newcommand{\laplace}{\mathrm{Laplace}} % Laplace distribution

\newcommand{\E}{\mathbb{E}}
\newcommand{\Ls}{\mathcal{L}}
\newcommand{\R}{\mathbb{R}}
\newcommand{\emp}{\tilde{p}}
\newcommand{\lr}{\alpha}
\newcommand{\reg}{\lambda}
\newcommand{\rect}{\mathrm{rectifier}}
\newcommand{\softmax}{\mathrm{softmax}}
\newcommand{\sigmoid}{\sigma}
\newcommand{\softplus}{\zeta}
\newcommand{\KL}{D_{\mathrm{KL}}}
\newcommand{\Var}{\mathrm{Var}}
\newcommand{\standarderror}{\mathrm{SE}}
\newcommand{\Cov}{\mathrm{Cov}}
% Wolfram Mathworld says $L^2$ is for function spaces and $\ell^2$ is for vectors
% But then they seem to use $L^2$ for vectors throughout the site, and so does
% wikipedia.
\newcommand{\normlzero}{L^0}
\newcommand{\normlone}{L^1}
\newcommand{\normltwo}{L^2}
\newcommand{\normlp}{L^p}
\newcommand{\normmax}{L^\infty}

\newcommand{\parents}{Pa} % See usage in notation.tex. Chosen to match Daphne's book.

\DeclareMathOperator*{\argmax}{arg\,max}
\DeclareMathOperator*{\argmin}{arg\,min}

\DeclareMathOperator{\sign}{sign}
\DeclareMathOperator{\Tr}{Tr}
\let\ab\allowbreak


\usepackage{hyperref}
\usepackage{url}

\usepackage{xspace}
\usepackage{amsmath}
\usepackage{graphicx}
\usepackage{hyperref}
\usepackage{booktabs}
\usepackage{adjustbox}
\usepackage{multirow}
\usepackage{tabularx}
\usepackage{booktabs}
\usepackage{caption}  % for captions in table*
\usepackage[inkscapelatex=false]{svg}
\usepackage{subfigure}
\usepackage{subcaption}
\usepackage{listings}

\newcommand{\llm}{LLM\xspace}
\newcommand{\codex}{\textsc{Codex}\xspace}
\newcommand{\codellama}{CodeLlama\xspace}
\newcommand{\starcoder}{{StarCoder}\xspace}
\newcommand{\deepseek}{DeepSeek-Coder\xspace}
\newcommand{\deepseekinstruct}{DeepSeek-Coder-Inst\xspace}
\newcommand{\codellamainstruct}{CodeLlama-Inst\xspace}
\newcommand{\wizardcoder}{{WizardCoder}\xspace}
\newcommand{\magicoder}{Magicoder\xspace}
\newcommand{\sweagent}{SWE-agent\xspace}
\newcommand{\autocoderover}{AutoCodeRover\xspace}

\newcommand{\evileval}{\textsc{EvoEval}\xspace}
\newcommand{\humaneval}{\textsc{HumanEval}\xspace}
\newcommand{\evalplus}{\textsc{EvalPlus}\xspace}
\newcommand{\apps}{{APPS}\xspace}
\newcommand{\mbpp}{{MBPP}\xspace}
\newcommand{\pie}{{\textsc{pie}}\xspace}
\newcommand{\swebench}{SWE-bench\xspace}
\newcommand{\swebenchlite}{SWE-bench Lite\xspace}
\newcommand{\swebenchlitefiltered}{SWE-bench Lite-$S$\xspace}


\newcommand{\parabf}[1]{\vspace{1mm}\noindent\textbf{#1} \hspace{0.5em}}


\newcommand{\model}{CSR-Bench\xspace}
\newcommand{\agent}{CSR-Agents\xspace}



% START APPENDIX
% Required packages
\usepackage{xcolor}      % For color support
\usepackage{tcolorbox}   % For tcolorbox
\usepackage{graphicx}    % Only needed if you plan to include images elsewhere
\usepackage{caption}     % For captioning tables/figures without floating them

% Define custom colors
\definecolor{myboxcolor}{RGB}{240, 240, 255}   % Light blue background
\definecolor{myframe}{RGB}{0, 0, 128}          % Navy blue frame
\definecolor{my_green}{RGB}{34, 139, 34}       % Forest green
\definecolor{my_yellow}{RGB}{255, 215, 0}      % Gold

% Create the custom tcolorbox
\newtcolorbox{mybody}{
  colback=myboxcolor,
  colframe=myframe,
  boxrule=1pt, % Adjust the border thickness
  left=1pt,
  right=1pt,
  top=1pt,
  bottom=1pt,
}

% Color commands
\newcommand{\red}[1]{\textcolor{red}{#1}}
\newcommand{\magenta}[1]{\textcolor{magenta}{#1}}
\newcommand{\green}[1]{\textcolor{my_green}{#1}}
\newcommand{\yellow}[1]{\textcolor{my_yellow}{#1}}
\newcommand{\blue}[1]{\textcolor{blue}{#1}}



% If the title and author information does not fit in the area allocated, uncomment the following
%
%\setlength\titlebox{<dim>}
%
% and set <dim> to something 5cm or larger.
% \title{\raisebox{-8pt}{\includegraphics[height=30pt]{assets/Logo.png}} \hspace{-0.9em} GSR-Bench: Benchmarking Deployment in GitHub Science Repositories with LLMs}

\title{\raisebox{-12pt}{\includegraphics[height=38pt]{assets/Logo.png}} \hspace{-0.2em} CSR-Bench: Benchmarking LLM Agents in Deployment of Computer Science Research Repositories}

% Author information can be set in various styles:
% For several authors from the same institution:
% \author{Author 1 \and ... \and Author n \\
%         Address line \\ ... \\ Address line}
% if the names do not fit well on one line use
%         Author 1 \\ {\bf Author 2} \\ ... \\ {\bf Author n} \\
% For authors from different institutions:
% \author{Author 1 \\ Address line \\  ... \\ Address line
%         \And  ... \And
%         Author n \\ Address line \\ ... \\ Address line}
% To start a separate ``row'' of authors use \AND, as in
% \author{Author 1 \\ Address line \\  ... \\ Address line
%         \AND
%         Author 2 \\ Address line \\ ... \\ Address line \And
%         Author 3 \\ Address line \\ ... \\ Address line}

\iffalse
\author{
    Yijia Xiao\thanks{Work done during an internship with Amazon.} \\
    University of California, Los Angeles \\
    %Los Angeles, CA, USA \\
    \texttt{yijia.xiao@cs.ucla.edu} 
    \And
    Runhui Wang \\
    Amazon Web Service \\
    %Seattle, USA \\
    \texttt{runhuiw@amazon.com} \\
    \And
    Luyang Kong \\
    Amazon Web Service \\
    %Seattle, USA \\
    \texttt{luyankon@amazon.com} 
    \AND
    Davor Golac \\
    Amazon Web Service \\
    %Seattle, USA \\
    \texttt{dgolac@amazon.com}
    \And
    Wei Wang \\
    University of California, Los Angeles \\
    %Los Angeles, CA, USA \\
    \texttt{weiwang@cs.ucla.edu}
}
\fi

\author{Yijia Xiao$^\dagger$*  \quad Runhui Wang$^\ddagger$*  \quad  \textbf{Luyang Kong}$^{\ddagger}$*  \quad 
\textbf{Davor Golac}$^\ddagger$ \quad \textbf{Wei Wang}$^\dagger$ 
  \\ 
$^\dagger$University of California, Los Angeles\\
$^\ddagger$ Amazon Web Services\\
{\small * Equal Contribution} \\
\texttt{\small \{yijia.xiao,weiwang\}@cs.ucla.edu} \\
\texttt{\small \{runhuiw,luyankon,dgolac\}@amazon.com}}


%\begingroup\def\thefootnote{*}\footnotetext{ Work perfomed during internship at Amazon.}\endgroup


%\author{
%  \textbf{First Author\textsuperscript{1}},
%  \textbf{Second Author\textsuperscript{1,2}},
%  \textbf{Third T. Author\textsuperscript{1}},
%  \textbf{Fourth Author\textsuperscript{1}},
%\\
%  \textbf{Fifth Author\textsuperscript{1,2}},
%  \textbf{Sixth Author\textsuperscript{1}},
%  \textbf{Seventh Author\textsuperscript{1}},
%  \textbf{Eighth Author \textsuperscript{1,2,3,4}},
%\\
%  \textbf{Ninth Author\textsuperscript{1}},
%  \textbf{Tenth Author\textsuperscript{1}},
%  \textbf{Eleventh E. Author\textsuperscript{1,2,3,4,5}},
%  \textbf{Twelfth Author\textsuperscript{1}},
%\\
%  \textbf{Thirteenth Author\textsuperscript{3}},
%  \textbf{Fourteenth F. Author\textsuperscript{2,4}},
%  \textbf{Fifteenth Author\textsuperscript{1}},
%  \textbf{Sixteenth Author\textsuperscript{1}},
%\\
%  \textbf{Seventeenth S. Author\textsuperscript{4,5}},
%  \textbf{Eighteenth Author\textsuperscript{3,4}},
%  \textbf{Nineteenth N. Author\textsuperscript{2,5}},
%  \textbf{Twentieth Author\textsuperscript{1}}
%\\
%\\
%  \textsuperscript{1}Affiliation 1,
%  \textsuperscript{2}Affiliation 2,
%  \textsuperscript{3}Affiliation 3,
%  \textsuperscript{4}Affiliation 4,
%  \textsuperscript{5}Affiliation 5
%\\
%  \small{
%    \textbf{Correspondence:} \href{mailto:email@domain}{email@domain}
%  }
%}

\newcommand{\fix}{\marginpar{FIX}}
\newcommand{\new}{\marginpar{NEW}}

\begin{document}
%\maketitle
{\makeatletter\acl@finalcopytrue
  \maketitle
}

\begin{abstract}
% \begin{abstract}
% Adversarial attacks pose significant threats to deploying Graph Neural Networks (GNNs) in real-world applications. Lines of studies have made progress in minimizing the influence of adversarial perturbations. However, existing methods often rely on fixed priors about the dataset or attacker, limiting their ability to generalize across diverse scenarios. These approaches cannot adaptively learn the intrinsic properties of the dataset.
% In this paper, we propose a novel framework, \ModelName (Graph \textbf{P}urification through t\textbf{R}ansfer \textbf{EN}tropy-guided \textbf{N}on-i\textbf{S}otropic Diffu\textbf{S}ion), which leverages a graph diffusion generative model to learn intrinsic properties and recover the clean structure of adversarial graphs. However, two key challenges arise: (1) The graph diffusion model’s uniform noise injection to all nodes during the forward process can over-perturb the graph, erasing valuable information and making recovery difficult, and (2) the diversity of the diffusion model in the reverse denoising process may cause the generated graph to deviate from the target clean structure.
% To address these challenges, we introduce a LID-Driven Non-Isotropic Diffusion process, which injects noise selectively, focusing on adversarial nodes while preserving the clean structure. Additionally, we propose a Graph Transfer Entropy-Guided Reverse Denoising process that maximizes transfer entropy to reduce uncertainty in the reverse process, ensuring that the generated graph remains aligned with the clean structure.
% Extensive experiments on both graph and node classification demonstrate our proposed \ModelName framework's robustness and superior generalization.
% Our code is available at \textcolor{mytablecolor}{\url{https:///}}
% \end{abstract}


% \begin{abstract}
% % priors and no priors-free 继续总结凝练
% % 鲁棒和各向同性有关
% Adversarial attacks pose significant threats to deploying Graph Neural Networks (GNNs) in real-world applications. Lines of studies have made progress in minimizing the influence of adversarial perturbations. 
% They often rely on priors such as neighbor similarity in clean graphs to restore the correct structure. However, this approach is less effective on datasets where these priors do not hold.
% % Their robustness methods often rely on priors of clean graphs or attacks.
% To achieve more generalized robustness, we need methods that can learn clean graph properties and recover the correct structure based on those learned properties, rather than depending on prior assumptions.
% Driven by this goal, in this work, we approach adversarial attacks from a distribution perspective: these attacks cause the graph distribution to deviate from the original clean distribution.
% % From this perspective, we propose using a graph generative model to learn the clean graph distribution without relying on priors and to purify adversarial graphs through distribution mapping.
% % 前面不要,直接提diffusion
% % Among various graph generative models, the diffusion model’s reverse denoising process naturally aligns with the removal of adversarial perturbations,
% % making it an ideal choice for mapping between adversarial and clean distributions. 
% From this perspective, we propose using the graph diffusion model to learn the clean graph distribution and purify adversarial graphs through distribution mapping.
% % The diffusion model’s reverse denoising process naturally aligns with the removal of adversarial perturbations, making it an ideal choice for mapping between adversarial and clean distributions.
% However, in graph diffusion models, 1) the indiscriminate noise injection across all nodes during the forward process can remove useful information still present in adversarial samples, making it difficult to recover the clean structure during reverse purification. 2) the diversity of the reverse denoising process may cause the generated graph to deviate from the target clean structure.
% To address these challenges, we propose a novel framework, \ModelName, to enhance gra\textbf{P}h rob\textbf{U}stness through t\textbf{R}ansfer \textbf{EN}tropy guid\textbf{E}d non-i\textbf{S}otropic diffu\textbf{S}ion purification.
% Our method introduces a LID-based Non-Isotropic Diffusion process, where we use local intrinsic dimensionality (LID) to estimate the adversarial degree of each node, enabling selective noise injection to focus on adversarial nodes while preserving the clean structure. Additionally, we propose a Graph Transfer Entropy-Guided Denoising process, which maximizes transfer entropy at each step to reduce uncertainty during the reverse process, 
% % ensuring the generated graph stays aligned with the clean structure without deviation.
% ensuring the generated graph matches the target clean graph without deviation.
% Extensive experiments on both graph and node classification tasks demonstrate the robustness of our \ModelName framework. Our code is available at \textcolor{mytablecolor}{\url{https:///}}.
% \end{abstract}

\begin{abstract}
Adversarial evasion attacks pose significant threats to graph learning, with lines of studies that have improved the robustness of Graph Neural Networks (GNNs).
However, existing works rely on priors about clean graphs or attacking strategies, which are often heuristic and inconsistent.
To achieve robust graph learning over different types of evasion attacks and diverse datasets, we investigate this problem from a prior-free structure purification perspective.
Specifically, we propose a novel \underline{\textbf{Diff}}usion-based \underline{\textbf{S}}tructure \underline{\textbf{P}}urification framework named \textbf{\ModelName}, which creatively incorporates the graph diffusion model to learn intrinsic distributions of clean graphs and purify the perturbed structures by removing adversaries under the direction of the captured predictive patterns without relying on priors.
\ModelName~is divided into the forward diffusion process and the reverse denoising process, during which structure purification is achieved.
To avoid valuable information loss during the forward process, we propose an LID-driven non-isotropic diffusion mechanism to selectively inject noise anisotropically.
To promote semantic alignment between the clean graph and the purified graph generated during the reverse process, we reduce the generation uncertainty by the proposed graph transfer entropy guided denoising mechanism.
Extensive experiments demonstrate the superior robustness of \ModelName~against evasion attacks.
% The reverse denoising process of diffusion models naturally aligns with removing graph adversarial perturbations, making them suitable for learning clean graph distribution and removing adversarial perturbations based on the learned distributional patterns without relying on priors.
% purifying adversarial graphs through distribution mapping.
% However, the indiscriminate noise injection in graph diffusion models can erase useful information, while the diversity of the reverse process may cause generated graphs to deviate from the target clean graph, making it difficult to directly apply them for purifying adversarial graph data.
% To address these challenges, 
% In this work, we propose a novel framework \ModelName, which introduces a LID-driven non-isotropic forward diffusion process and a transfer entropy-guided reverse denoising process to precisely remove adversarial perturbations and guide the generation toward the target clean graph.
% Our code is available at \textcolor{mytablecolor}{\url{https:///}}.
\end{abstract}


\keywords{robust graph learning, adversarial evasion attack, graph structure purification, graph diffuison}
\end{abstract}

\section{Introduction}
\label{sec::intro}

Embodied Question Answering (EQA) \cite{das2018embodied} represents a challenging task at the intersection of natural language processing, computer vision, and robotics, where an embodied agent (e.g., a UAV) must actively explore its environment to answer questions posed in natural language. While most existing research has concentrated on indoor EQA tasks \cite{gao2023room, pena2023visual}, such as exploring and answering questions within confined spaces like homes or offices \cite{liu2024aligning}, relatively little attention has been dedicated to EQA tasks in  open-ended city space. Nevertheless, extending EQA to city space is crucial for numerous real-world applications, including autonomous systems \cite{kalinowska2023embodied}, urban region profiling \cite{yan2024urbanclip}, and city planning \cite{gao2024embodiedcity}. 
% 1. 环境复杂性   
%    - 地标重复性问题(如区分相似建筑)  
%    - 动态干扰因素(交通流、行人)  
% 2. 行动复杂性  
%    - 长程导航路径规划  
%    - 移动控制、角度等  
% 3. 感知复杂性  
%    - 复合空间关系推理("A楼东侧商铺西边的车辆")  
%    - 时序依赖的观察结果整合

EQA tasks in city space (referred to as CityEQA) introduce a unique set of challenges that fundamentally differ from those encountered in indoor environments. Compared to indoor EQA, CityEQA faces three main challenges: 

1) \textbf{Environmental complexity with ambiguous objects}: 
Urban environments are inherently more complex,  featuring a diverse range of objects and structures, many of which are visually similar and difficult to distinguish without detailed semantic information (e.g., buildings, roads, and vehicles). This complexity makes it challenging to construct task instructions and specify the desired information accurately, as shown in Figure \ref{fig:example}. 

2) \textbf{Action complexity in cross-scale space}: 
The vast geographical scale of city space compels agents to adopt larger movement amplitudes to enhance exploration efficiency. However, it might risk overlooking detailed information within the scene. Therefore, agents require cross-scale action adjustment capabilities to effectively balance long-distance path planning with fine-grained movement and angular control.

3) \textbf{Perception complexity with observation dynamics}: 
% Rich semantic information in urban settings leads to varying observations depending on distance and orientation, which can impact the accuracy of answer generation. 
Observations can vary greatly depending on distance, orientation, and perspective. For example, an object may look completely different up close than it does from afar or from different angles. These differences pose challenges for consistency and can affect the accuracy of answer generation, as embodied agents must adapt to the dynamic and complex nature of urban environments.


\begin{table}
\centering
\caption{CityEQA-EC vs existing benchmarks.}
\label{table:dataset}
\renewcommand\arraystretch{1.2}
\resizebox{\linewidth}{!}{
\begin{tabular}{cccccc}
             & Place  & Open Vocab & Active & Platform  & Reference \\ \hline
EQA-v1      & Indoor & \textcolor{red}{\ding{55}}          & \textcolor{green}{\ding{51}}      & House3D      & \cite{das2018embodied}  \\
IQUAD        & Indoor & \textcolor{red}{\ding{55}}          & \textcolor{green}{\ding{51}}      & AI2-THOR     & \cite{gordon2018iqa} \\
MP3D-EQA     & Indoor & \textcolor{red}{\ding{55}}          & \textcolor{green}{\ding{51}}      & Matterport3D & \cite{wijmans2019embodied} \\
MT-EQA       & Indoor & \textcolor{red}{\ding{55}}          & \textcolor{green}{\ding{51}}      & House3D      & \cite{yu2019multi} \\
ScanQA       & Indoor & \textcolor{red}{\ding{55}}          & \textcolor{red}{\ding{55}}      & -            & \cite{azuma2022scanqa} \\
SQA3D        & Indoor & \textcolor{red}{\ding{55}}          & \textcolor{red}{\ding{55}}      & -            & \cite{masqa3d} \\
K-EQA        & Indoor & \textcolor{green}{\ding{51}}          & \textcolor{green}{\ding{51}}      & AI2-THOR     & \cite{tan2023knowledge} \\
OpenEQA      & Indoor & \textcolor{green}{\ding{51}}          & \textcolor{green}{\ding{51}}      & ScanNet/HM3D & \cite{majumdar2024openeqa} \\
 \hline
CityEQA-EC   & City (Outdoor)  & \textcolor{green}{\ding{51}}          & \textcolor{green}{\ding{51}}      & EmbodiedCity & - \\ \hline
\end{tabular}}
\end{table}

\begin{figure*}[!htb]
\centering
    \includegraphics[width=0.78\linewidth]{figures/example.pdf}
% \vspace{-0.2cm}
\caption{The typical workflow of the PMA to address City EQA tasks. There are two cars in this area, thus a valid question must contain landmarks and spatial relationships to specify a car. Given the task, PMA will sequentially complete multiple sub-tasks to find the answer.}
% \vspace{-0.2cm}
\label{fig:example}
\end{figure*}

As an initial step toward CityEQA, we developed \textbf{CityEQA-EC}, a benchmark dataset to evaluate embodied agents' performance on CityEQA tasks. The distinctions between this dataset and other EQA benchmarks are summarized in Table \ref{table:dataset}. CityEQA-EC comprises six task types characterized by open-vocabulary questions. These tasks utilize urban landmarks and spatial relationships to delineate the expected answer, adhering to human conventions while addressing object ambiguity. This design introduces significant complexity, turning CityEQA into long-horizon tasks that require embodied agents to identify and use landmarks, explore urban environments effectively, and refine observation to generate high-quality answers.

To address CityEQA tasks, we introduce the \textbf{Planner-Manager-Actor (PMA)}, a novel baseline agent powered by large models, designed to emulate human-like rationale for solving long-horizon tasks in urban environments, as illustrated in Figure \ref{fig:example}. PMA employs a hierarchical framework to generate actions and derive answers. The Planner module parses tasks and creates plans consisting of three sub-task types: navigation, exploration, and collection. The Manager oversees the execution of these plans while maintaining a global object-centric cognitive map \cite{deng2024opengraph}. This 2D grid-based representation enables precise object identification (retrieval) and efficient management of long-term landmark information. The Actor generates specific actions based on the Manager's instructions through its components: Navigator, Explorer, and Collector. Notably, the Collector integrates a Multi-Modal Large Language Model (MM-LLM) as its Vision-Language-Action (VLA) module to refine observations and generate high-quality answers.
PMA's performance is assessed against four baselines, including humans. 
Results show that humans perform best in CityEQA, while PMA achieves 60.73\% of human accuracy in answering questions, highlighting both the challenge and validity of the proposed benchmarks. 

% The Frontier-Based Exploration (FBE) Agent, widely used in indoor EQA tasks, performs worse than even a blind LLM. This underscores the importance of PMA's hierarchical framework and its use of landmarks and spatial relationships for tackling CityEQA tasks.

In summary, this paper makes the following significant contributions:
\vspace{-8pt}
\begin{itemize}[leftmargin=*]
    \item To the best of our knowledge, we present the first open-ended embodied question answering benchmark for city space, namely CityEQA-EC.
    \vspace{-7pt}
    \item We propose a novel baseline model, PMA, which is capable of solving long-horizon tasks for CityEQA tasks with a human-like rationale.
     \vspace{-7pt}
    \item Experimental results demonstrate that our approach outperforms existing baselines in tackling the CityEQA task. However, the gap with human performance highlights opportunities for future research to improve visual thinking and reasoning in embodied intelligence for city spaces.
\end{itemize}




\section{Related Works}


\noindent\textbf{3D Point Cloud Domain Adaptation and Generalization.}
Early endeavors within 3D domain adaptation (3DDA) focused on extending 2D adversarial methodologies~\cite{qin2019pointdan} to align point cloud features. Alternative methods have delved into geometry-aware self-supervised pre-tasks. Achituve \etal~\cite{achituve2021self} introduced DefRec, a technique employing self-complement tasks by reconstructing point clouds from a non-rigid distorted version, while Zou \etal~\cite{zou2021geometry} incorporating norm curves prediction as an auxiliary task. Liang \etal~\cite{liang2022point} put forth MLSP, focusing on point estimation tasks like cardinality, position, and normal. SDDA~\cite{cardace2023self} employs self-distillation to learn the point-based features. Additionally, post-hoc self-paced training~\cite{zou2021geometry,fan2022self,park2023pcadapter} has been embraced to refine adaptation to target distributions by accessing target data and conducting further finetuning based on prior knowledge from the source domain.
In contrast, the landscape of 3D domain generalization (3DDG) research remains nascent. Metasets~\cite{huang2021metasets} leverage meta-learning to address geometric variations, while PDG~\cite{wei2022learning} decomposes 3D shapes into part-based features to enhance generalization capabilities.
Despite the remarkable progress, existing studies assume that objects in both the source and target domains share the same orientation, limiting their practical application. This limitation propels our exploration into orientation-aware 3D domain generalization through intricate orientation learning.


\noindent\textbf{Rotation-generalizable Point Cloud Analysis.}
Previous works in point cloud analysis~\cite{qi2017pointnet, wang2019dynamic} enhance rotation robustness by introducing random rotations to augment point clouds. {However, generating a comprehensive set of rotated data is impractical, resulting in variable model performance across different scenes. To robustify the networks \wrt randomly rotated point clouds,} rotation-equivariance methods explore equivalent model architectures by incorporating equivalent operations~\cite{su2022svnet, Deng_2021_ICCV, luo2022equivariant} or steerable convolutions~\cite{chen2021equivariant, poulenard2021functional}.
Alternatively, rotation-invariance approaches aim to identify geometric descriptors invariant to rotations, such as distances and angles between local points~\cite{chen2019clusternet, zhang2020global} or point norms~\cite{zhao2019rotation, li2021rotation}. Besides, {Li \etal~\cite{li2021closer} have explored disambiguating the number of PCA-based canonical poses, while Kim \etal~\cite{kim2020rotation} and Chen \etal~\cite{chen2022devil} have transformed local point coordinates according to local reference frames to maintain rotation invariance. However, these methods focus on improving in-domain rotation robustness, neglecting domain shift and consequently exhibiting limited performance when applied to diverse domains. This study addresses the challenge of cross-domain generalizability together with rotation robustness and proposes novel solutions.} 

\noindent\textbf{Intricate Sample Mining}, aimed at identifying or synthesizing challenging samples that are difficult to classify correctly, seeks to rectify the imbalance between positive and negative samples for enhancing a model's discriminability. While traditional works have explored this concept in SVM optimization~\cite{felzenszwalb2009object}, shallow neural networks~\cite{dollar2009integral}, and boosted decision trees~\cite{yu2019unsupervised}, recent advances in deep learning have catalyzed a proliferation of researches in this area across various computer vision tasks. For instance, 
Lin \etal~\cite{lin2017focal} proposed a focal loss to concentrate training efforts on a selected group of hard examples in object detection, while Yu \etal~\cite{yu2019unsupervised} devised a soft multilabel-guided hard negative mining method to learn discriminative embeddings for person Re-ID. Schroff \etal~\cite{schroff2015facenet} introduced an online negative exemplar mining process to encourage spherical clusters in face embeddings for individual recognition, and Wang \etal~\cite{wang2021instance} designed an adversarially trained negative generator to yield instance-wise negative samples, bolstering the learning of unpaired image-to-image translation. In contrast to existing studies, our work presents the first attempt to mitigate the orientational shift in 3D point cloud domain generalization, by developing an effective intricate orientation mining strategy to achieve orientation-aware learning.


\section{\model} \label{gsrbench}
In this section, we provide the problem statement for code deployment in \model, introduce the code repository collection process of computer science research projects, and show their statistics.
% \vspace{-1em}
% \vspace{-0.5em}
\subsection{Problem Statement}
% \vspace{-0.5em}
The \model consists of a collection of computer science research repositories from GitHub and these repositories are used for evaluating the capabilities of LLMs in code deployment tasks. For each repository, the deployment tasks typically include: (1) setting up the environment; (2) preparing necessary data and model files; (3) conducting model training; (4) demonstration of inference; (5) performance evaluation. To complete these tasks, we prompt LLMs to generate executable bash commands by using the README file as the primary source of information and other repository contents (source code, bash scripts, directory structure, and etc.) as supplementary information.

%\textbf{Input Data.}
%The input data for the model is the GitHub repository itself, with a particular focus on the README file located in the root folder. The model has access to all files within the repository. The LLM agent can also leverage the repository's file structure to set up the correct path, rather than relying on placeholder paths.

%\textbf{Generation Goal.}
%The goal is to generate bash scripts for deploying and verifying the repositories and to produce a summary of the deployment process. In some instances, manual intervention may be required, such as setting up an API key or providing a download link for files requiring verification (e.g., for LLaMA models). The summary will be presented to users, allowing them to provide feedback on the large language model to address any issues or gaps.

%\textbf{Performance Evaluation.}
%For code repositories of empirical and experimental computer science research projects, typical deployment steps include environment setup, installation, data and model file preparation and download, training deployment, testing and evaluation, and demonstration or inference examples.

\textbf{Metric.}
During the evaluation in \model, the large language model will be prompted to generate executable commands for the corresponding sections for each repository. %Currently, large language model agents still require further development to fully and comprehensively understand and deploy repositories without errors. Therefore, 
we use the completion rate as a key metric, defined as the ratio between number of successfully executed commands %with a return code of zero 
and the total number of commands executed.
\subsection{Repository Collection}
% \vspace{-0.5em}
%\textbf{Construction} 
In \model, we aim to collect a diverse and comprehensive collection of code repositories of computer science-related research projects. GitHub is a good data source for this purpose and it provides tags for identifying most relevant repositories. Some example tags are “nlp”, “naacl”, and “emnlp2024”. Since \model focuses on computer science-related repositories, we filter the repositories by tags of various conference names and categories to ensure they include diverse topics. %, such as natural language processing, computer vision, data mining, machine learning and etc.

For repository selection, we use GitHub tags to obtain an initial set of over 1500 repositories that are relevant to computer science research topics and categorizing them into five areas: Natural Language Processing, Computer Vision, Large Language Models, Machine Learning, and Interdisciplinary topics. Notably, we collect repositories related to large language models because nowadays LLM-related research projects are increasingly popular due to its foundational impact in various areas of computer science. 

\begin{figure*}[htbp]
    \centering
    \begin{minipage}[b]{0.45\linewidth}
        \centering
        \includegraphics[width=\linewidth]{assets/GSRBench_Conf.pdf}
        \caption{Conf Distribution of \model}
        \label{fig:conf}
    \end{minipage}
    \hfill
    \begin{minipage}[b]{0.45\linewidth}
        \centering
        \includegraphics[width=\linewidth]{assets/GSRBench_Sunburst.pdf}
        \caption{Topic Distribution of \model}
        \label{fig:topic}
    \end{minipage}
    \vspace{-1em}
\end{figure*}


We obtain 100 high-quality code repositories for \model in the following steps. First, we categorize this initial set and sort them by the number of GitHub stars. Next, we manually check the content of each repository starting from the top of the sorted list. In this step, we only keep repositories that contain sufficient information in their README files. We also skip the repositories that do not contain deployable code. Finally, we check the licenses of the repositories and make sure they are permissive.
%We manually select the most representative, research-oriented large language model repositories from the top 50 repositories tagged with ``large language model".

% % \vspace{-1em}

% \vspace{-0.5em}
\subsection{Statistics of \model}

This section provides an in-depth analysis of the traits of repositories in \model. We examine the diversity and breadth of topics covered, as well as detailed statistics about the documentation and structure of these repositories.

% \textit{Category Distribution}

% Detailed analysis of the category distribution will illustrate how the repositories span across various scientific disciplines. This information helps to understand the interdisciplinary nature of the dataset and the predominant fields of study.
% % \vspace{-0.5em}


% \begin{figure}[htbp]
%     \centering
%     \includegraphics[width=0.95\linewidth]{assets/readme_word_count_distribution.pdf}
%     \caption{Distribution of Word Count in README Files}
%     \label{fig:readme_word_distribution}
% \end{figure}

% \begin{figure}[htbp]
%     \centering
%     \includegraphics[width=0.95\linewidth]{assets/repo_file_count_distribution.pdf}
%     \caption{Distribution of Repository File Count. This figure shows the distribution of repository file counts, with a histogram depicting the number of repositories against the count of files they contain.}
%     \label{fig:repo_file_distribution}
% \end{figure}

%% \vspace{-1.25em}





% \begin{figure}[ht]
%     \centering
%     \begin{minipage}[b]{0.95\linewidth}
%         \centering
%         \includegraphics[width=\linewidth]{assets/file_count_distribution.pdf}
%         \vspace{-1.5em}
%         \caption{Number of Files per 
%         Repository}
%         \label{fig:file_count_distribution}
%     \end{minipage}
%     % \hspace{0.05\linewidth}
%     \begin{minipage}[b]{0.95\linewidth}
%         \centering
%         \includegraphics[width=\linewidth]{assets/token_count_distribution.pdf}
%         \vspace{-1.5em}
%         \caption{Number of Tokens per README}
%         \label{fig:token_count_distribution}
%     \end{minipage}
% \end{figure}
% %% \vspace{-1.25em}
% \begin{figure}[htbp]
%     \centering
%     \begin{minipage}[b]{0.95\linewidth}
%         \centering
%         \includegraphics[width=\linewidth]{assets/star_count_distribution.pdf}
%         \vspace{-1.5em}
%         \caption{Stargazer Distribution}
%         \label{fig:stars_distribution}
%     \end{minipage}
%     % \hspace{0.05\linewidth}
%     \begin{minipage}[b]{0.95\linewidth}
%         \centering
%         \includegraphics[width=\linewidth]{assets/issue_count_distribution.pdf}
%         \vspace{-1.5em}
%         \caption{Size of Issue Database}
%         \label{fig:issue_count_distribution}
%     \end{minipage}
% \end{figure}


% %% \vspace{-1em}
% Figure \ref{fig:token_count_distribution} illustrates the distribution of word counts in README files, shedding light on the extent of documentation provided across the repositories. This metric is crucial for assessing the completeness of information necessary for users to understand and use the repositories effectively.

% Figure \ref{fig:file_count_distribution} shows the distribution of file counts of repositories, with a histogram depicting the number of repositories against the count of files they contain. The number of files can reflect the complexity and scale of its corresponding repository.

In \model, the README files and directory structures provide critical insights into the usability and organization of repositories. We use the following figures to analyze the lengths of README file and number of files, and offer a quantitative view of content complexity and organizational depth. The length of README file is an important metric because the most LLMs have limits on the input token length. The number of files indicate the complexity of the code repository.

\begin{figure}[ht]
    \centering
    \includegraphics[width=\linewidth]{assets/token_count_distribution.pdf}
    \vspace{-1.5em}
    \caption{Number of Tokens per README}
    \label{fig:token_count_distribution}
\end{figure}

Figure \ref{fig:token_count_distribution} shows the distribution of token counts in README files, highlighting the extent of documentation, which is essential for user understanding and repository usability. Since the mean token counts is just over 1000 and the maximum counts is around 3000, most LLMs can take the full README files as input.

\begin{figure}[ht]
    \centering
    \includegraphics[width=\linewidth]{assets/file_count_distribution.pdf}
    \vspace{-1.5em}
    \caption{Number of Files per Repository}
    \label{fig:file_count_distribution}
\end{figure}

Figure \ref{fig:file_count_distribution} depicts the distribution of file counts in repositories, reflecting their complexity and scale based on the number of files. Because the number of files in most repositories are in the lower hundreds, it is feasible to leverage directory structure for code deployment with LLMs.

\begin{figure}[ht]
    \centering
    \includegraphics[width=\linewidth]{assets/star_count_distribution.pdf}
    \vspace{-1.5em}
    \caption{Stargazer Distribution}
    \label{fig:stars_distribution}
\end{figure}


Figure \ref{fig:stars_distribution} shows the distribution of star counts in selected repositories. The average count is over 590, which means that these repositories receives significant attention and indicates they are generally well maintained.



\begin{figure}[ht]
    \centering
    \includegraphics[width=\linewidth]{assets/issue_count_distribution.pdf}
    \vspace{-1.5em}
    \caption{Size of Issue Database}
    \label{fig:issue_count_distribution}
\end{figure}

Figure \ref{fig:issue_count_distribution} shows the distribution of issues counts in selected repositories. On average, each repository contains over 48 issues, indicating these repositories have a good amount of engagements with the open-source community and sufficient support from the authors. Therefore, the information in the issues are valuable for the deployment tasks.



% \vspace{-1em}
% Note that the repository ``HIPT'' is excluded from these plots due to its outlier large size (`readme': 3117 words, `files': 14246, `folders': 970), which significantly skews the average. This exclusion ensures the data presented reflects a more typical repository profile within the dataset.



\subsection{Settings}

\begin{table*}[ht]
\centering
\tiny
% This must be in the first 5 lines to tell arXiv to use pdfLaTeX, which is strongly recommended.
\pdfoutput=1
% In particular, the hyperref package requires pdfLaTeX in order to break URLs across lines.

\documentclass[11pt]{article}

% Change "review" to "final" to generate the final (sometimes called camera-ready) version.
% Change to "preprint" to generate a non-anonymous version with page numbers.
\usepackage{acl}

% Standard package includes
\usepackage{times}
\usepackage{latexsym}

% Draw tables
\usepackage{booktabs}
\usepackage{multirow}
\usepackage{xcolor}
\usepackage{colortbl}
\usepackage{array} 
\usepackage{amsmath}

\newcolumntype{C}{>{\centering\arraybackslash}p{0.07\textwidth}}
% For proper rendering and hyphenation of words containing Latin characters (including in bib files)
\usepackage[T1]{fontenc}
% For Vietnamese characters
% \usepackage[T5]{fontenc}
% See https://www.latex-project.org/help/documentation/encguide.pdf for other character sets
% This assumes your files are encoded as UTF8
\usepackage[utf8]{inputenc}

% This is not strictly necessary, and may be commented out,
% but it will improve the layout of the manuscript,
% and will typically save some space.
\usepackage{microtype}
\DeclareMathOperator*{\argmax}{arg\,max}
% This is also not strictly necessary, and may be commented out.
% However, it will improve the aesthetics of text in
% the typewriter font.
\usepackage{inconsolata}

%Including images in your LaTeX document requires adding
%additional package(s)
\usepackage{graphicx}
% If the title and author information does not fit in the area allocated, uncomment the following
%
%\setlength\titlebox{<dim>}
%
% and set <dim> to something 5cm or larger.

\title{Wi-Chat: Large Language Model Powered Wi-Fi Sensing}

% Author information can be set in various styles:
% For several authors from the same institution:
% \author{Author 1 \and ... \and Author n \\
%         Address line \\ ... \\ Address line}
% if the names do not fit well on one line use
%         Author 1 \\ {\bf Author 2} \\ ... \\ {\bf Author n} \\
% For authors from different institutions:
% \author{Author 1 \\ Address line \\  ... \\ Address line
%         \And  ... \And
%         Author n \\ Address line \\ ... \\ Address line}
% To start a separate ``row'' of authors use \AND, as in
% \author{Author 1 \\ Address line \\  ... \\ Address line
%         \AND
%         Author 2 \\ Address line \\ ... \\ Address line \And
%         Author 3 \\ Address line \\ ... \\ Address line}

% \author{First Author \\
%   Affiliation / Address line 1 \\
%   Affiliation / Address line 2 \\
%   Affiliation / Address line 3 \\
%   \texttt{email@domain} \\\And
%   Second Author \\
%   Affiliation / Address line 1 \\
%   Affiliation / Address line 2 \\
%   Affiliation / Address line 3 \\
%   \texttt{email@domain} \\}
% \author{Haohan Yuan \qquad Haopeng Zhang\thanks{corresponding author} \\ 
%   ALOHA Lab, University of Hawaii at Manoa \\
%   % Affiliation / Address line 2 \\
%   % Affiliation / Address line 3 \\
%   \texttt{\{haohany,haopengz\}@hawaii.edu}}
  
\author{
{Haopeng Zhang$\dag$\thanks{These authors contributed equally to this work.}, Yili Ren$\ddagger$\footnotemark[1], Haohan Yuan$\dag$, Jingzhe Zhang$\ddagger$, Yitong Shen$\ddagger$} \\
ALOHA Lab, University of Hawaii at Manoa$\dag$, University of South Florida$\ddagger$ \\
\{haopengz, haohany\}@hawaii.edu\\
\{yiliren, jingzhe, shen202\}@usf.edu\\}



  
%\author{
%  \textbf{First Author\textsuperscript{1}},
%  \textbf{Second Author\textsuperscript{1,2}},
%  \textbf{Third T. Author\textsuperscript{1}},
%  \textbf{Fourth Author\textsuperscript{1}},
%\\
%  \textbf{Fifth Author\textsuperscript{1,2}},
%  \textbf{Sixth Author\textsuperscript{1}},
%  \textbf{Seventh Author\textsuperscript{1}},
%  \textbf{Eighth Author \textsuperscript{1,2,3,4}},
%\\
%  \textbf{Ninth Author\textsuperscript{1}},
%  \textbf{Tenth Author\textsuperscript{1}},
%  \textbf{Eleventh E. Author\textsuperscript{1,2,3,4,5}},
%  \textbf{Twelfth Author\textsuperscript{1}},
%\\
%  \textbf{Thirteenth Author\textsuperscript{3}},
%  \textbf{Fourteenth F. Author\textsuperscript{2,4}},
%  \textbf{Fifteenth Author\textsuperscript{1}},
%  \textbf{Sixteenth Author\textsuperscript{1}},
%\\
%  \textbf{Seventeenth S. Author\textsuperscript{4,5}},
%  \textbf{Eighteenth Author\textsuperscript{3,4}},
%  \textbf{Nineteenth N. Author\textsuperscript{2,5}},
%  \textbf{Twentieth Author\textsuperscript{1}}
%\\
%\\
%  \textsuperscript{1}Affiliation 1,
%  \textsuperscript{2}Affiliation 2,
%  \textsuperscript{3}Affiliation 3,
%  \textsuperscript{4}Affiliation 4,
%  \textsuperscript{5}Affiliation 5
%\\
%  \small{
%    \textbf{Correspondence:} \href{mailto:email@domain}{email@domain}
%  }
%}

\begin{document}
\maketitle
\begin{abstract}
Recent advancements in Large Language Models (LLMs) have demonstrated remarkable capabilities across diverse tasks. However, their potential to integrate physical model knowledge for real-world signal interpretation remains largely unexplored. In this work, we introduce Wi-Chat, the first LLM-powered Wi-Fi-based human activity recognition system. We demonstrate that LLMs can process raw Wi-Fi signals and infer human activities by incorporating Wi-Fi sensing principles into prompts. Our approach leverages physical model insights to guide LLMs in interpreting Channel State Information (CSI) data without traditional signal processing techniques. Through experiments on real-world Wi-Fi datasets, we show that LLMs exhibit strong reasoning capabilities, achieving zero-shot activity recognition. These findings highlight a new paradigm for Wi-Fi sensing, expanding LLM applications beyond conventional language tasks and enhancing the accessibility of wireless sensing for real-world deployments.
\end{abstract}

\section{Introduction}

In today’s rapidly evolving digital landscape, the transformative power of web technologies has redefined not only how services are delivered but also how complex tasks are approached. Web-based systems have become increasingly prevalent in risk control across various domains. This widespread adoption is due their accessibility, scalability, and ability to remotely connect various types of users. For example, these systems are used for process safety management in industry~\cite{kannan2016web}, safety risk early warning in urban construction~\cite{ding2013development}, and safe monitoring of infrastructural systems~\cite{repetto2018web}. Within these web-based risk management systems, the source search problem presents a huge challenge. Source search refers to the task of identifying the origin of a risky event, such as a gas leak and the emission point of toxic substances. This source search capability is crucial for effective risk management and decision-making.

Traditional approaches to implementing source search capabilities into the web systems often rely on solely algorithmic solutions~\cite{ristic2016study}. These methods, while relatively straightforward to implement, often struggle to achieve acceptable performances due to algorithmic local optima and complex unknown environments~\cite{zhao2020searching}. More recently, web crowdsourcing has emerged as a promising alternative for tackling the source search problem by incorporating human efforts in these web systems on-the-fly~\cite{zhao2024user}. This approach outsources the task of addressing issues encountered during the source search process to human workers, leveraging their capabilities to enhance system performance.

These solutions often employ a human-AI collaborative way~\cite{zhao2023leveraging} where algorithms handle exploration-exploitation and report the encountered problems while human workers resolve complex decision-making bottlenecks to help the algorithms getting rid of local deadlocks~\cite{zhao2022crowd}. Although effective, this paradigm suffers from two inherent limitations: increased operational costs from continuous human intervention, and slow response times of human workers due to sequential decision-making. These challenges motivate our investigation into developing autonomous systems that preserve human-like reasoning capabilities while reducing dependency on massive crowdsourced labor.

Furthermore, recent advancements in large language models (LLMs)~\cite{chang2024survey} and multi-modal LLMs (MLLMs)~\cite{huang2023chatgpt} have unveiled promising avenues for addressing these challenges. One clear opportunity involves the seamless integration of visual understanding and linguistic reasoning for robust decision-making in search tasks. However, whether large models-assisted source search is really effective and efficient for improving the current source search algorithms~\cite{ji2022source} remains unknown. \textit{To address the research gap, we are particularly interested in answering the following two research questions in this work:}

\textbf{\textit{RQ1: }}How can source search capabilities be integrated into web-based systems to support decision-making in time-sensitive risk management scenarios? 
% \sq{I mention ``time-sensitive'' here because I feel like we shall say something about the response time -- LLM has to be faster than humans}

\textbf{\textit{RQ2: }}How can MLLMs and LLMs enhance the effectiveness and efficiency of existing source search algorithms? 

% \textit{\textbf{RQ2:}} To what extent does the performance of large models-assisted search align with or approach the effectiveness of human-AI collaborative search? 

To answer the research questions, we propose a novel framework called Auto-\
S$^2$earch (\textbf{Auto}nomous \textbf{S}ource \textbf{Search}) and implement a prototype system that leverages advanced web technologies to simulate real-world conditions for zero-shot source search. Unlike traditional methods that rely on pre-defined heuristics or extensive human intervention, AutoS$^2$earch employs a carefully designed prompt that encapsulates human rationales, thereby guiding the MLLM to generate coherent and accurate scene descriptions from visual inputs about four directional choices. Based on these language-based descriptions, the LLM is enabled to determine the optimal directional choice through chain-of-thought (CoT) reasoning. Comprehensive empirical validation demonstrates that AutoS$^2$-\ 
earch achieves a success rate of 95–98\%, closely approaching the performance of human-AI collaborative search across 20 benchmark scenarios~\cite{zhao2023leveraging}. 

Our work indicates that the role of humans in future web crowdsourcing tasks may evolve from executors to validators or supervisors. Furthermore, incorporating explanations of LLM decisions into web-based system interfaces has the potential to help humans enhance task performance in risk control.






\section{Related Work}
\label{sec:relatedworks}

\input{tables/tab-related-full}

\noindent \textbf{Prompting-based LLM Agents.} Due to the lack of agent-specific pre-training corpus, existing LLM agents rely on either prompt engineering~\cite{hsieh2023tool,lu2024chameleon,yao2022react,wang2023voyager} or instruction fine-tuning~\cite{chen2023fireact,zeng2023agenttuning} to understand human instructions, decompose high-level tasks, generate grounded plans, and execute multi-step actions. 
However, prompting-based methods mainly depend on the capabilities of backbone LLMs (usually commercial LLMs), failing to introduce new knowledge and struggling to generalize to unseen tasks~\cite{sun2024adaplanner,zhuang2023toolchain}. 

\noindent \textbf{Instruction Finetuning-based LLM Agents.} Considering the extensive diversity of APIs and the complexity of multi-tool instructions, tool learning inherently presents greater challenges than natural language tasks, such as text generation~\cite{qin2023toolllm}.
Post-training techniques focus more on instruction following and aligning output with specific formats~\cite{patil2023gorilla,hao2024toolkengpt,qin2023toolllm,schick2024toolformer}, rather than fundamentally improving model knowledge or capabilities. 
Moreover, heavy fine-tuning can hinder generalization or even degrade performance in non-agent use cases, potentially suppressing the original base model capabilities~\cite{ghosh2024a}.

\noindent \textbf{Pretraining-based LLM Agents.} While pre-training serves as an essential alternative, prior works~\cite{nijkamp2023codegen,roziere2023code,xu2024lemur,patil2023gorilla} have primarily focused on improving task-specific capabilities (\eg, code generation) instead of general-domain LLM agents, due to single-source, uni-type, small-scale, and poor-quality pre-training data. 
Existing tool documentation data for agent training either lacks diverse real-world APIs~\cite{patil2023gorilla, tang2023toolalpaca} or is constrained to single-tool or single-round tool execution. 
Furthermore, trajectory data mostly imitate expert behavior or follow function-calling rules with inferior planning and reasoning, failing to fully elicit LLMs' capabilities and handle complex instructions~\cite{qin2023toolllm}. 
Given a wide range of candidate API functions, each comprising various function names and parameters available at every planning step, identifying globally optimal solutions and generalizing across tasks remains highly challenging.



\section{Preliminaries}
\label{Preliminaries}
\begin{figure*}[t]
    \centering
    \includegraphics[width=0.95\linewidth]{fig/HealthGPT_Framework.png}
    \caption{The \ourmethod{} architecture integrates hierarchical visual perception and H-LoRA, employing a task-specific hard router to select visual features and H-LoRA plugins, ultimately generating outputs with an autoregressive manner.}
    \label{fig:architecture}
\end{figure*}
\noindent\textbf{Large Vision-Language Models.} 
The input to a LVLM typically consists of an image $x^{\text{img}}$ and a discrete text sequence $x^{\text{txt}}$. The visual encoder $\mathcal{E}^{\text{img}}$ converts the input image $x^{\text{img}}$ into a sequence of visual tokens $\mathcal{V} = [v_i]_{i=1}^{N_v}$, while the text sequence $x^{\text{txt}}$ is mapped into a sequence of text tokens $\mathcal{T} = [t_i]_{i=1}^{N_t}$ using an embedding function $\mathcal{E}^{\text{txt}}$. The LLM $\mathcal{M_\text{LLM}}(\cdot|\theta)$ models the joint probability of the token sequence $\mathcal{U} = \{\mathcal{V},\mathcal{T}\}$, which is expressed as:
\begin{equation}
    P_\theta(R | \mathcal{U}) = \prod_{i=1}^{N_r} P_\theta(r_i | \{\mathcal{U}, r_{<i}\}),
\end{equation}
where $R = [r_i]_{i=1}^{N_r}$ is the text response sequence. The LVLM iteratively generates the next token $r_i$ based on $r_{<i}$. The optimization objective is to minimize the cross-entropy loss of the response $\mathcal{R}$.
% \begin{equation}
%     \mathcal{L}_{\text{VLM}} = \mathbb{E}_{R|\mathcal{U}}\left[-\log P_\theta(R | \mathcal{U})\right]
% \end{equation}
It is worth noting that most LVLMs adopt a design paradigm based on ViT, alignment adapters, and pre-trained LLMs\cite{liu2023llava,liu2024improved}, enabling quick adaptation to downstream tasks.


\noindent\textbf{VQGAN.}
VQGAN~\cite{esser2021taming} employs latent space compression and indexing mechanisms to effectively learn a complete discrete representation of images. VQGAN first maps the input image $x^{\text{img}}$ to a latent representation $z = \mathcal{E}(x)$ through a encoder $\mathcal{E}$. Then, the latent representation is quantized using a codebook $\mathcal{Z} = \{z_k\}_{k=1}^K$, generating a discrete index sequence $\mathcal{I} = [i_m]_{m=1}^N$, where $i_m \in \mathcal{Z}$ represents the quantized code index:
\begin{equation}
    \mathcal{I} = \text{Quantize}(z|\mathcal{Z}) = \arg\min_{z_k \in \mathcal{Z}} \| z - z_k \|_2.
\end{equation}
In our approach, the discrete index sequence $\mathcal{I}$ serves as a supervisory signal for the generation task, enabling the model to predict the index sequence $\hat{\mathcal{I}}$ from input conditions such as text or other modality signals.  
Finally, the predicted index sequence $\hat{\mathcal{I}}$ is upsampled by the VQGAN decoder $G$, generating the high-quality image $\hat{x}^\text{img} = G(\hat{\mathcal{I}})$.



\noindent\textbf{Low Rank Adaptation.} 
LoRA\cite{hu2021lora} effectively captures the characteristics of downstream tasks by introducing low-rank adapters. The core idea is to decompose the bypass weight matrix $\Delta W\in\mathbb{R}^{d^{\text{in}} \times d^{\text{out}}}$ into two low-rank matrices $ \{A \in \mathbb{R}^{d^{\text{in}} \times r}, B \in \mathbb{R}^{r \times d^{\text{out}}} \}$, where $ r \ll \min\{d^{\text{in}}, d^{\text{out}}\} $, significantly reducing learnable parameters. The output with the LoRA adapter for the input $x$ is then given by:
\begin{equation}
    h = x W_0 + \alpha x \Delta W/r = x W_0 + \alpha xAB/r,
\end{equation}
where matrix $ A $ is initialized with a Gaussian distribution, while the matrix $ B $ is initialized as a zero matrix. The scaling factor $ \alpha/r $ controls the impact of $ \Delta W $ on the model.

\section{HealthGPT}
\label{Method}


\subsection{Unified Autoregressive Generation.}  
% As shown in Figure~\ref{fig:architecture}, 
\ourmethod{} (Figure~\ref{fig:architecture}) utilizes a discrete token representation that covers both text and visual outputs, unifying visual comprehension and generation as an autoregressive task. 
For comprehension, $\mathcal{M}_\text{llm}$ receives the input joint sequence $\mathcal{U}$ and outputs a series of text token $\mathcal{R} = [r_1, r_2, \dots, r_{N_r}]$, where $r_i \in \mathcal{V}_{\text{txt}}$, and $\mathcal{V}_{\text{txt}}$ represents the LLM's vocabulary:
\begin{equation}
    P_\theta(\mathcal{R} \mid \mathcal{U}) = \prod_{i=1}^{N_r} P_\theta(r_i \mid \mathcal{U}, r_{<i}).
\end{equation}
For generation, $\mathcal{M}_\text{llm}$ first receives a special start token $\langle \text{START\_IMG} \rangle$, then generates a series of tokens corresponding to the VQGAN indices $\mathcal{I} = [i_1, i_2, \dots, i_{N_i}]$, where $i_j \in \mathcal{V}_{\text{vq}}$, and $\mathcal{V}_{\text{vq}}$ represents the index range of VQGAN. Upon completion of generation, the LLM outputs an end token $\langle \text{END\_IMG} \rangle$:
\begin{equation}
    P_\theta(\mathcal{I} \mid \mathcal{U}) = \prod_{j=1}^{N_i} P_\theta(i_j \mid \mathcal{U}, i_{<j}).
\end{equation}
Finally, the generated index sequence $\mathcal{I}$ is fed into the decoder $G$, which reconstructs the target image $\hat{x}^{\text{img}} = G(\mathcal{I})$.

\subsection{Hierarchical Visual Perception}  
Given the differences in visual perception between comprehension and generation tasks—where the former focuses on abstract semantics and the latter emphasizes complete semantics—we employ ViT to compress the image into discrete visual tokens at multiple hierarchical levels.
Specifically, the image is converted into a series of features $\{f_1, f_2, \dots, f_L\}$ as it passes through $L$ ViT blocks.

To address the needs of various tasks, the hidden states are divided into two types: (i) \textit{Concrete-grained features} $\mathcal{F}^{\text{Con}} = \{f_1, f_2, \dots, f_k\}, k < L$, derived from the shallower layers of ViT, containing sufficient global features, suitable for generation tasks; 
(ii) \textit{Abstract-grained features} $\mathcal{F}^{\text{Abs}} = \{f_{k+1}, f_{k+2}, \dots, f_L\}$, derived from the deeper layers of ViT, which contain abstract semantic information closer to the text space, suitable for comprehension tasks.

The task type $T$ (comprehension or generation) determines which set of features is selected as the input for the downstream large language model:
\begin{equation}
    \mathcal{F}^{\text{img}}_T =
    \begin{cases}
        \mathcal{F}^{\text{Con}}, & \text{if } T = \text{generation task} \\
        \mathcal{F}^{\text{Abs}}, & \text{if } T = \text{comprehension task}
    \end{cases}
\end{equation}
We integrate the image features $\mathcal{F}^{\text{img}}_T$ and text features $\mathcal{T}$ into a joint sequence through simple concatenation, which is then fed into the LLM $\mathcal{M}_{\text{llm}}$ for autoregressive generation.
% :
% \begin{equation}
%     \mathcal{R} = \mathcal{M}_{\text{llm}}(\mathcal{U}|\theta), \quad \mathcal{U} = [\mathcal{F}^{\text{img}}_T; \mathcal{T}]
% \end{equation}
\subsection{Heterogeneous Knowledge Adaptation}
We devise H-LoRA, which stores heterogeneous knowledge from comprehension and generation tasks in separate modules and dynamically routes to extract task-relevant knowledge from these modules. 
At the task level, for each task type $ T $, we dynamically assign a dedicated H-LoRA submodule $ \theta^T $, which is expressed as:
\begin{equation}
    \mathcal{R} = \mathcal{M}_\text{LLM}(\mathcal{U}|\theta, \theta^T), \quad \theta^T = \{A^T, B^T, \mathcal{R}^T_\text{outer}\}.
\end{equation}
At the feature level for a single task, H-LoRA integrates the idea of Mixture of Experts (MoE)~\cite{masoudnia2014mixture} and designs an efficient matrix merging and routing weight allocation mechanism, thus avoiding the significant computational delay introduced by matrix splitting in existing MoELoRA~\cite{luo2024moelora}. Specifically, we first merge the low-rank matrices (rank = r) of $ k $ LoRA experts into a unified matrix:
\begin{equation}
    \mathbf{A}^{\text{merged}}, \mathbf{B}^{\text{merged}} = \text{Concat}(\{A_i\}_1^k), \text{Concat}(\{B_i\}_1^k),
\end{equation}
where $ \mathbf{A}^{\text{merged}} \in \mathbb{R}^{d^\text{in} \times rk} $ and $ \mathbf{B}^{\text{merged}} \in \mathbb{R}^{rk \times d^\text{out}} $. The $k$-dimension routing layer generates expert weights $ \mathcal{W} \in \mathbb{R}^{\text{token\_num} \times k} $ based on the input hidden state $ x $, and these are expanded to $ \mathbb{R}^{\text{token\_num} \times rk} $ as follows:
\begin{equation}
    \mathcal{W}^\text{expanded} = \alpha k \mathcal{W} / r \otimes \mathbf{1}_r,
\end{equation}
where $ \otimes $ denotes the replication operation.
The overall output of H-LoRA is computed as:
\begin{equation}
    \mathcal{O}^\text{H-LoRA} = (x \mathbf{A}^{\text{merged}} \odot \mathcal{W}^\text{expanded}) \mathbf{B}^{\text{merged}},
\end{equation}
where $ \odot $ represents element-wise multiplication. Finally, the output of H-LoRA is added to the frozen pre-trained weights to produce the final output:
\begin{equation}
    \mathcal{O} = x W_0 + \mathcal{O}^\text{H-LoRA}.
\end{equation}
% In summary, H-LoRA is a task-based dynamic PEFT method that achieves high efficiency in single-task fine-tuning.

\subsection{Training Pipeline}

\begin{figure}[t]
    \centering
    \hspace{-4mm}
    \includegraphics[width=0.94\linewidth]{fig/data.pdf}
    \caption{Data statistics of \texttt{VL-Health}. }
    \label{fig:data}
\end{figure}
\noindent \textbf{1st Stage: Multi-modal Alignment.} 
In the first stage, we design separate visual adapters and H-LoRA submodules for medical unified tasks. For the medical comprehension task, we train abstract-grained visual adapters using high-quality image-text pairs to align visual embeddings with textual embeddings, thereby enabling the model to accurately describe medical visual content. During this process, the pre-trained LLM and its corresponding H-LoRA submodules remain frozen. In contrast, the medical generation task requires training concrete-grained adapters and H-LoRA submodules while keeping the LLM frozen. Meanwhile, we extend the textual vocabulary to include multimodal tokens, enabling the support of additional VQGAN vector quantization indices. The model trains on image-VQ pairs, endowing the pre-trained LLM with the capability for image reconstruction. This design ensures pixel-level consistency of pre- and post-LVLM. The processes establish the initial alignment between the LLM’s outputs and the visual inputs.

\noindent \textbf{2nd Stage: Heterogeneous H-LoRA Plugin Adaptation.}  
The submodules of H-LoRA share the word embedding layer and output head but may encounter issues such as bias and scale inconsistencies during training across different tasks. To ensure that the multiple H-LoRA plugins seamlessly interface with the LLMs and form a unified base, we fine-tune the word embedding layer and output head using a small amount of mixed data to maintain consistency in the model weights. Specifically, during this stage, all H-LoRA submodules for different tasks are kept frozen, with only the word embedding layer and output head being optimized. Through this stage, the model accumulates foundational knowledge for unified tasks by adapting H-LoRA plugins.
\input{tab/visual_comprehension_part1}
\input{tab/modality_transfer}

\noindent \textbf{3rd Stage: Visual Instruction Fine-Tuning.}  
In the third stage, we introduce additional task-specific data to further optimize the model and enhance its adaptability to downstream tasks such as medical visual comprehension (e.g., medical QA, medical dialogues, and report generation) or generation tasks (e.g., super-resolution, denoising, and modality conversion). Notably, by this stage, the word embedding layer and output head have been fine-tuned, only the H-LoRA modules and adapter modules need to be trained. This strategy significantly improves the model's adaptability and flexibility across different tasks.


\section{Experiment}
\label{s:experiment}

\subsection{Data Description}
We evaluate our method on FI~\cite{you2016building}, Twitter\_LDL~\cite{yang2017learning} and Artphoto~\cite{machajdik2010affective}.
FI is a public dataset built from Flickr and Instagram, with 23,308 images and eight emotion categories, namely \textit{amusement}, \textit{anger}, \textit{awe},  \textit{contentment}, \textit{disgust}, \textit{excitement},  \textit{fear}, and \textit{sadness}. 
% Since images in FI are all copyrighted by law, some images are corrupted now, so we remove these samples and retain 21,828 images.
% T4SA contains images from Twitter, which are classified into three categories: \textit{positive}, \textit{neutral}, and \textit{negative}. In this paper, we adopt the base version of B-T4SA, which contains 470,586 images and provides text descriptions of the corresponding tweets.
Twitter\_LDL contains 10,045 images from Twitter, with the same eight categories as the FI dataset.
% 。
For these two datasets, they are randomly split into 80\%
training and 20\% testing set.
Artphoto contains 806 artistic photos from the DeviantArt website, which we use to further evaluate the zero-shot capability of our model.
% on the small-scale dataset.
% We construct and publicly release the first image sentiment analysis dataset containing metadata.
% 。

% Based on these datasets, we are the first to construct and publicly release metadata-enhanced image sentiment analysis datasets. These datasets include scenes, tags, descriptions, and corresponding confidence scores, and are available at this link for future research purposes.


% 
\begin{table}[t]
\centering
% \begin{center}
\caption{Overall performance of different models on FI and Twitter\_LDL datasets.}
\label{tab:cap1}
% \resizebox{\linewidth}{!}
{
\begin{tabular}{l|c|c|c|c}
\hline
\multirow{2}{*}{\textbf{Model}} & \multicolumn{2}{c|}{\textbf{FI}}  & \multicolumn{2}{c}{\textbf{Twitter\_LDL}} \\ \cline{2-5} 
  & \textbf{Accuracy} & \textbf{F1} & \textbf{Accuracy} & \textbf{F1}  \\ \hline
% (\rownumber)~AlexNet~\cite{krizhevsky2017imagenet}  & 58.13\% & 56.35\%  & 56.24\%& 55.02\%  \\ 
% (\rownumber)~VGG16~\cite{simonyan2014very}  & 63.75\%& 63.08\%  & 59.34\%& 59.02\%  \\ 
(\rownumber)~ResNet101~\cite{he2016deep} & 66.16\%& 65.56\%  & 62.02\% & 61.34\%  \\ 
(\rownumber)~CDA~\cite{han2023boosting} & 66.71\%& 65.37\%  & 64.14\% & 62.85\%  \\ 
(\rownumber)~CECCN~\cite{ruan2024color} & 67.96\%& 66.74\%  & 64.59\%& 64.72\% \\ 
(\rownumber)~EmoVIT~\cite{xie2024emovit} & 68.09\%& 67.45\%  & 63.12\% & 61.97\%  \\ 
(\rownumber)~ComLDL~\cite{zhang2022compound} & 68.83\%& 67.28\%  & 65.29\% & 63.12\%  \\ 
(\rownumber)~WSDEN~\cite{li2023weakly} & 69.78\%& 69.61\%  & 67.04\% & 65.49\% \\ 
(\rownumber)~ECWA~\cite{deng2021emotion} & 70.87\%& 69.08\%  & 67.81\% & 66.87\%  \\ 
(\rownumber)~EECon~\cite{yang2023exploiting} & 71.13\%& 68.34\%  & 64.27\%& 63.16\%  \\ 
(\rownumber)~MAM~\cite{zhang2024affective} & 71.44\%  & 70.83\% & 67.18\%  & 65.01\%\\ 
(\rownumber)~TGCA-PVT~\cite{chen2024tgca}   & 73.05\%  & 71.46\% & 69.87\%  & 68.32\% \\ 
(\rownumber)~OEAN~\cite{zhang2024object}   & 73.40\%  & 72.63\% & 70.52\%  & 69.47\% \\ \hline
(\rownumber)~\shortname  & \textbf{79.48\%} & \textbf{79.22\%} & \textbf{74.12\%} & \textbf{73.09\%} \\ \hline
\end{tabular}
}
\vspace{-6mm}
% \end{center}
\end{table}
% 

\subsection{Experiment Setting}
% \subsubsection{Model Setting.}
% 
\textbf{Model Setting:}
For feature representation, we set $k=10$ to select object tags, and adopt clip-vit-base-patch32 as the pre-trained model for unified feature representation.
Moreover, we empirically set $(d_e, d_h, d_k, d_s) = (512, 128, 16, 64)$, and set the classification class $L$ to 8.

% 

\textbf{Training Setting:}
To initialize the model, we set all weights such as $\boldsymbol{W}$ following the truncated normal distribution, and use AdamW optimizer with the learning rate of $1 \times 10^{-4}$.
% warmup scheduler of cosine, warmup steps of 2000.
Furthermore, we set the batch size to 32 and the epoch of the training process to 200.
During the implementation, we utilize \textit{PyTorch} to build our entire model.
% , and our project codes are publicly available at https://github.com/zzmyrep/MESN.
% Our project codes as well as data are all publicly available on GitHub\footnote{https://github.com/zzmyrep/KBCEN}.
% Code is available at \href{https://github.com/zzmyrep/KBCEN}{https://github.com/zzmyrep/KBCEN}.

\textbf{Evaluation Metrics:}
Following~\cite{zhang2024affective, chen2024tgca, zhang2024object}, we adopt \textit{accuracy} and \textit{F1} as our evaluation metrics to measure the performance of different methods for image sentiment analysis. 



\subsection{Experiment Result}
% We compare our model against the following baselines: AlexNet~\cite{krizhevsky2017imagenet}, VGG16~\cite{simonyan2014very}, ResNet101~\cite{he2016deep}, CECCN~\cite{ruan2024color}, EmoVIT~\cite{xie2024emovit}, WSCNet~\cite{yang2018weakly}, ECWA~\cite{deng2021emotion}, EECon~\cite{yang2023exploiting}, MAM~\cite{zhang2024affective} and TGCA-PVT~\cite{chen2024tgca}, and the overall results are summarized in Table~\ref{tab:cap1}.
We compare our model against several baselines, and the overall results are summarized in Table~\ref{tab:cap1}.
We observe that our model achieves the best performance in both accuracy and F1 metrics, significantly outperforming the previous models. 
This superior performance is mainly attributed to our effective utilization of metadata to enhance image sentiment analysis, as well as the exceptional capability of the unified sentiment transformer framework we developed. These results strongly demonstrate that our proposed method can bring encouraging performance for image sentiment analysis.

\setcounter{magicrownumbers}{0} 
\begin{table}[t]
\begin{center}
\caption{Ablation study of~\shortname~on FI dataset.} 
% \vspace{1mm}
\label{tab:cap2}
\resizebox{.9\linewidth}{!}
{
\begin{tabular}{lcc}
  \hline
  \textbf{Model} & \textbf{Accuracy} & \textbf{F1} \\
  \hline
  (\rownumber)~Ours (w/o vision) & 65.72\% & 64.54\% \\
  (\rownumber)~Ours (w/o text description) & 74.05\% & 72.58\% \\
  (\rownumber)~Ours (w/o object tag) & 77.45\% & 76.84\% \\
  (\rownumber)~Ours (w/o scene tag) & 78.47\% & 78.21\% \\
  \hline
  (\rownumber)~Ours (w/o unified embedding) & 76.41\% & 76.23\% \\
  (\rownumber)~Ours (w/o adaptive learning) & 76.83\% & 76.56\% \\
  (\rownumber)~Ours (w/o cross-modal fusion) & 76.85\% & 76.49\% \\
  \hline
  (\rownumber)~Ours  & \textbf{79.48\%} & \textbf{79.22\%} \\
  \hline
\end{tabular}
}
\end{center}
\vspace{-5mm}
\end{table}


\begin{figure}[t]
\centering
% \vspace{-2mm}
\includegraphics[width=0.42\textwidth]{fig/2dvisual-linux4-paper2.pdf}
\caption{Visualization of feature distribution on eight categories before (left) and after (right) model processing.}
% 
\label{fig:visualization}
\vspace{-5mm}
\end{figure}

\subsection{Ablation Performance}
In this subsection, we conduct an ablation study to examine which component is really important for performance improvement. The results are reported in Table~\ref{tab:cap2}.

For information utilization, we observe a significant decline in model performance when visual features are removed. Additionally, the performance of \shortname~decreases when different metadata are removed separately, which means that text description, object tag, and scene tag are all critical for image sentiment analysis.
Recalling the model architecture, we separately remove transformer layers of the unified representation module, the adaptive learning module, and the cross-modal fusion module, replacing them with MLPs of the same parameter scale.
In this way, we can observe varying degrees of decline in model performance, indicating that these modules are indispensable for our model to achieve better performance.

\subsection{Visualization}
% 


% % 开始使用minipage进行左右排列
% \begin{minipage}[t]{0.45\textwidth}  % 子图1宽度为45%
%     \centering
%     \includegraphics[width=\textwidth]{2dvisual.pdf}  % 插入图片
%     \captionof{figure}{Visualization of feature distribution.}  % 使用captionof添加图片标题
%     \label{fig:visualization}
% \end{minipage}


% \begin{figure}[t]
% \centering
% \vspace{-2mm}
% \includegraphics[width=0.45\textwidth]{fig/2dvisual.pdf}
% \caption{Visualization of feature distribution.}
% \label{fig:visualization}
% % \vspace{-4mm}
% \end{figure}

% \begin{figure}[t]
% \centering
% \vspace{-2mm}
% \includegraphics[width=0.45\textwidth]{fig/2dvisual-linux3-paper.pdf}
% \caption{Visualization of feature distribution.}
% \label{fig:visualization}
% % \vspace{-4mm}
% \end{figure}



\begin{figure}[tbp]   
\vspace{-4mm}
  \centering            
  \subfloat[Depth of adaptive learning layers]   
  {
    \label{fig:subfig1}\includegraphics[width=0.22\textwidth]{fig/fig_sensitivity-a5}
  }
  \subfloat[Depth of fusion layers]
  {
    % \label{fig:subfig2}\includegraphics[width=0.22\textwidth]{fig/fig_sensitivity-b2}
    \label{fig:subfig2}\includegraphics[width=0.22\textwidth]{fig/fig_sensitivity-b2-num.pdf}
  }
  \caption{Sensitivity study of \shortname~on different depth. }   
  \label{fig:fig_sensitivity}  
\vspace{-2mm}
\end{figure}

% \begin{figure}[htbp]
% \centerline{\includegraphics{2dvisual.pdf}}
% \caption{Visualization of feature distribution.}
% \label{fig:visualization}
% \end{figure}

% In Fig.~\ref{fig:visualization}, we use t-SNE~\cite{van2008visualizing} to reduce the dimension of data features for visualization, Figure in left represents the metadata features before model processing, the features are obtained by embedding through the CLIP model, and figure in right shows the features of the data after model processing, it can be observed that after the model processing, the data with different label categories fall in different regions in the space, therefore, we can conclude that the Therefore, we can conclude that the model can effectively utilize the information contained in the metadata and use it to guide the model for classification.

In Fig.~\ref{fig:visualization}, we use t-SNE~\cite{van2008visualizing} to reduce the dimension of data features for visualization.
The left figure shows metadata features before being processed by our model (\textit{i.e.}, embedded by CLIP), while the right shows the distribution of features after being processed by our model.
We can observe that after the model processing, data with the same label are closer to each other, while others are farther away.
Therefore, it shows that the model can effectively utilize the information contained in the metadata and use it to guide the classification process.

\subsection{Sensitivity Analysis}
% 
In this subsection, we conduct a sensitivity analysis to figure out the effect of different depth settings of adaptive learning layers and fusion layers. 
% In this subsection, we conduct a sensitivity analysis to figure out the effect of different depth settings on the model. 
% Fig.~\ref{fig:fig_sensitivity} presents the effect of different depth settings of adaptive learning layers and fusion layers. 
Taking Fig.~\ref{fig:fig_sensitivity} (a) as an example, the model performance improves with increasing depth, reaching the best performance at a depth of 4.
% Taking Fig.~\ref{fig:fig_sensitivity} (a) as an example, the performance of \shortname~improves with the increase of depth at first, reaching the best performance at a depth of 4.
When the depth continues to increase, the accuracy decreases to varying degrees.
Similar results can be observed in Fig.~\ref{fig:fig_sensitivity} (b).
Therefore, we set their depths to 4 and 6 respectively to achieve the best results.

% Through our experiments, we can observe that the effect of modifying these hyperparameters on the results of the experiments is very weak, and the surface model is not sensitive to the hyperparameters.


\subsection{Zero-shot Capability}
% 

% (1)~GCH~\cite{2010Analyzing} & 21.78\% & (5)~RA-DLNet~\cite{2020A} & 34.01\% \\ \hline
% (2)~WSCNet~\cite{2019WSCNet}  & 30.25\% & (6)~CECCN~\cite{ruan2024color} & 43.83\% \\ \hline
% (3)~PCNN~\cite{2015Robust} & 31.68\%  & (7)~EmoVIT~\cite{xie2024emovit} & 44.90\% \\ \hline
% (4)~AR~\cite{2018Visual} & 32.67\% & (8)~Ours (Zero-shot) & 47.83\% \\ \hline


\begin{table}[t]
\centering
\caption{Zero-shot capability of \shortname.}
\label{tab:cap3}
\resizebox{1\linewidth}{!}
{
\begin{tabular}{lc|lc}
\hline
\textbf{Model} & \textbf{Accuracy} & \textbf{Model} & \textbf{Accuracy} \\ \hline
(1)~WSCNet~\cite{2019WSCNet}  & 30.25\% & (5)~MAM~\cite{zhang2024affective} & 39.56\%  \\ \hline
(2)~AR~\cite{2018Visual} & 32.67\% & (6)~CECCN~\cite{ruan2024color} & 43.83\% \\ \hline
(3)~RA-DLNet~\cite{2020A} & 34.01\%  & (7)~EmoVIT~\cite{xie2024emovit} & 44.90\% \\ \hline
(4)~CDA~\cite{han2023boosting} & 38.64\% & (8)~Ours (Zero-shot) & 47.83\% \\ \hline
\end{tabular}
}
\vspace{-5mm}
\end{table}

% We use the model trained on the FI dataset to test on the artphoto dataset to verify the model's generalization ability as well as robustness to other distributed datasets.
% We can observe that the MESN model shows strong competitiveness in terms of accuracy when compared to other trained models, which suggests that the model has a good generalization ability in the OOD task.

To validate the model's generalization ability and robustness to other distributed datasets, we directly test the model trained on the FI dataset, without training on Artphoto. 
% As observed in Table 3, compared to other models trained on Artphoto, we achieve highly competitive zero-shot performance, indicating that the model has good generalization ability in out-of-distribution tasks.
From Table~\ref{tab:cap3}, we can observe that compared with other models trained on Artphoto, we achieve competitive zero-shot performance, which shows that the model has good generalization ability in out-of-distribution tasks.


%%%%%%%%%%%%
%  E2E     %
%%%%%%%%%%%%


\section{Conclusion}
In this paper, we introduced Wi-Chat, the first LLM-powered Wi-Fi-based human activity recognition system that integrates the reasoning capabilities of large language models with the sensing potential of wireless signals. Our experimental results on a self-collected Wi-Fi CSI dataset demonstrate the promising potential of LLMs in enabling zero-shot Wi-Fi sensing. These findings suggest a new paradigm for human activity recognition that does not rely on extensive labeled data. We hope future research will build upon this direction, further exploring the applications of LLMs in signal processing domains such as IoT, mobile sensing, and radar-based systems.

\section*{Limitations}
While our work represents the first attempt to leverage LLMs for processing Wi-Fi signals, it is a preliminary study focused on a relatively simple task: Wi-Fi-based human activity recognition. This choice allows us to explore the feasibility of LLMs in wireless sensing but also comes with certain limitations.

Our approach primarily evaluates zero-shot performance, which, while promising, may still lag behind traditional supervised learning methods in highly complex or fine-grained recognition tasks. Besides, our study is limited to a controlled environment with a self-collected dataset, and the generalizability of LLMs to diverse real-world scenarios with varying Wi-Fi conditions, environmental interference, and device heterogeneity remains an open question.

Additionally, we have yet to explore the full potential of LLMs in more advanced Wi-Fi sensing applications, such as fine-grained gesture recognition, occupancy detection, and passive health monitoring. Future work should investigate the scalability of LLM-based approaches, their robustness to domain shifts, and their integration with multimodal sensing techniques in broader IoT applications.


% Bibliography entries for the entire Anthology, followed by custom entries
%\bibliography{anthology,custom}
% Custom bibliography entries only
\bibliography{main}
\newpage
\appendix

\section{Experiment prompts}
\label{sec:prompt}
The prompts used in the LLM experiments are shown in the following Table~\ref{tab:prompts}.

\definecolor{titlecolor}{rgb}{0.9, 0.5, 0.1}
\definecolor{anscolor}{rgb}{0.2, 0.5, 0.8}
\definecolor{labelcolor}{HTML}{48a07e}
\begin{table*}[h]
	\centering
	
 % \vspace{-0.2cm}
	
	\begin{center}
		\begin{tikzpicture}[
				chatbox_inner/.style={rectangle, rounded corners, opacity=0, text opacity=1, font=\sffamily\scriptsize, text width=5in, text height=9pt, inner xsep=6pt, inner ysep=6pt},
				chatbox_prompt_inner/.style={chatbox_inner, align=flush left, xshift=0pt, text height=11pt},
				chatbox_user_inner/.style={chatbox_inner, align=flush left, xshift=0pt},
				chatbox_gpt_inner/.style={chatbox_inner, align=flush left, xshift=0pt},
				chatbox/.style={chatbox_inner, draw=black!25, fill=gray!7, opacity=1, text opacity=0},
				chatbox_prompt/.style={chatbox, align=flush left, fill=gray!1.5, draw=black!30, text height=10pt},
				chatbox_user/.style={chatbox, align=flush left},
				chatbox_gpt/.style={chatbox, align=flush left},
				chatbox2/.style={chatbox_gpt, fill=green!25},
				chatbox3/.style={chatbox_gpt, fill=red!20, draw=black!20},
				chatbox4/.style={chatbox_gpt, fill=yellow!30},
				labelbox/.style={rectangle, rounded corners, draw=black!50, font=\sffamily\scriptsize\bfseries, fill=gray!5, inner sep=3pt},
			]
											
			\node[chatbox_user] (q1) {
				\textbf{System prompt}
				\newline
				\newline
				You are a helpful and precise assistant for segmenting and labeling sentences. We would like to request your help on curating a dataset for entity-level hallucination detection.
				\newline \newline
                We will give you a machine generated biography and a list of checked facts about the biography. Each fact consists of a sentence and a label (True/False). Please do the following process. First, breaking down the biography into words. Second, by referring to the provided list of facts, merging some broken down words in the previous step to form meaningful entities. For example, ``strategic thinking'' should be one entity instead of two. Third, according to the labels in the list of facts, labeling each entity as True or False. Specifically, for facts that share a similar sentence structure (\eg, \textit{``He was born on Mach 9, 1941.''} (\texttt{True}) and \textit{``He was born in Ramos Mejia.''} (\texttt{False})), please first assign labels to entities that differ across atomic facts. For example, first labeling ``Mach 9, 1941'' (\texttt{True}) and ``Ramos Mejia'' (\texttt{False}) in the above case. For those entities that are the same across atomic facts (\eg, ``was born'') or are neutral (\eg, ``he,'' ``in,'' and ``on''), please label them as \texttt{True}. For the cases that there is no atomic fact that shares the same sentence structure, please identify the most informative entities in the sentence and label them with the same label as the atomic fact while treating the rest of the entities as \texttt{True}. In the end, output the entities and labels in the following format:
                \begin{itemize}[nosep]
                    \item Entity 1 (Label 1)
                    \item Entity 2 (Label 2)
                    \item ...
                    \item Entity N (Label N)
                \end{itemize}
                % \newline \newline
                Here are two examples:
                \newline\newline
                \textbf{[Example 1]}
                \newline
                [The start of the biography]
                \newline
                \textcolor{titlecolor}{Marianne McAndrew is an American actress and singer, born on November 21, 1942, in Cleveland, Ohio. She began her acting career in the late 1960s, appearing in various television shows and films.}
                \newline
                [The end of the biography]
                \newline \newline
                [The start of the list of checked facts]
                \newline
                \textcolor{anscolor}{[Marianne McAndrew is an American. (False); Marianne McAndrew is an actress. (True); Marianne McAndrew is a singer. (False); Marianne McAndrew was born on November 21, 1942. (False); Marianne McAndrew was born in Cleveland, Ohio. (False); She began her acting career in the late 1960s. (True); She has appeared in various television shows. (True); She has appeared in various films. (True)]}
                \newline
                [The end of the list of checked facts]
                \newline \newline
                [The start of the ideal output]
                \newline
                \textcolor{labelcolor}{[Marianne McAndrew (True); is (True); an (True); American (False); actress (True); and (True); singer (False); , (True); born (True); on (True); November 21, 1942 (False); , (True); in (True); Cleveland, Ohio (False); . (True); She (True); began (True); her (True); acting career (True); in (True); the late 1960s (True); , (True); appearing (True); in (True); various (True); television shows (True); and (True); films (True); . (True)]}
                \newline
                [The end of the ideal output]
				\newline \newline
                \textbf{[Example 2]}
                \newline
                [The start of the biography]
                \newline
                \textcolor{titlecolor}{Doug Sheehan is an American actor who was born on April 27, 1949, in Santa Monica, California. He is best known for his roles in soap operas, including his portrayal of Joe Kelly on ``General Hospital'' and Ben Gibson on ``Knots Landing.''}
                \newline
                [The end of the biography]
                \newline \newline
                [The start of the list of checked facts]
                \newline
                \textcolor{anscolor}{[Doug Sheehan is an American. (True); Doug Sheehan is an actor. (True); Doug Sheehan was born on April 27, 1949. (True); Doug Sheehan was born in Santa Monica, California. (False); He is best known for his roles in soap operas. (True); He portrayed Joe Kelly. (True); Joe Kelly was in General Hospital. (True); General Hospital is a soap opera. (True); He portrayed Ben Gibson. (True); Ben Gibson was in Knots Landing. (True); Knots Landing is a soap opera. (True)]}
                \newline
                [The end of the list of checked facts]
                \newline \newline
                [The start of the ideal output]
                \newline
                \textcolor{labelcolor}{[Doug Sheehan (True); is (True); an (True); American (True); actor (True); who (True); was born (True); on (True); April 27, 1949 (True); in (True); Santa Monica, California (False); . (True); He (True); is (True); best known (True); for (True); his roles in soap operas (True); , (True); including (True); in (True); his portrayal (True); of (True); Joe Kelly (True); on (True); ``General Hospital'' (True); and (True); Ben Gibson (True); on (True); ``Knots Landing.'' (True)]}
                \newline
                [The end of the ideal output]
				\newline \newline
				\textbf{User prompt}
				\newline
				\newline
				[The start of the biography]
				\newline
				\textcolor{magenta}{\texttt{\{BIOGRAPHY\}}}
				\newline
				[The ebd of the biography]
				\newline \newline
				[The start of the list of checked facts]
				\newline
				\textcolor{magenta}{\texttt{\{LIST OF CHECKED FACTS\}}}
				\newline
				[The end of the list of checked facts]
			};
			\node[chatbox_user_inner] (q1_text) at (q1) {
				\textbf{System prompt}
				\newline
				\newline
				You are a helpful and precise assistant for segmenting and labeling sentences. We would like to request your help on curating a dataset for entity-level hallucination detection.
				\newline \newline
                We will give you a machine generated biography and a list of checked facts about the biography. Each fact consists of a sentence and a label (True/False). Please do the following process. First, breaking down the biography into words. Second, by referring to the provided list of facts, merging some broken down words in the previous step to form meaningful entities. For example, ``strategic thinking'' should be one entity instead of two. Third, according to the labels in the list of facts, labeling each entity as True or False. Specifically, for facts that share a similar sentence structure (\eg, \textit{``He was born on Mach 9, 1941.''} (\texttt{True}) and \textit{``He was born in Ramos Mejia.''} (\texttt{False})), please first assign labels to entities that differ across atomic facts. For example, first labeling ``Mach 9, 1941'' (\texttt{True}) and ``Ramos Mejia'' (\texttt{False}) in the above case. For those entities that are the same across atomic facts (\eg, ``was born'') or are neutral (\eg, ``he,'' ``in,'' and ``on''), please label them as \texttt{True}. For the cases that there is no atomic fact that shares the same sentence structure, please identify the most informative entities in the sentence and label them with the same label as the atomic fact while treating the rest of the entities as \texttt{True}. In the end, output the entities and labels in the following format:
                \begin{itemize}[nosep]
                    \item Entity 1 (Label 1)
                    \item Entity 2 (Label 2)
                    \item ...
                    \item Entity N (Label N)
                \end{itemize}
                % \newline \newline
                Here are two examples:
                \newline\newline
                \textbf{[Example 1]}
                \newline
                [The start of the biography]
                \newline
                \textcolor{titlecolor}{Marianne McAndrew is an American actress and singer, born on November 21, 1942, in Cleveland, Ohio. She began her acting career in the late 1960s, appearing in various television shows and films.}
                \newline
                [The end of the biography]
                \newline \newline
                [The start of the list of checked facts]
                \newline
                \textcolor{anscolor}{[Marianne McAndrew is an American. (False); Marianne McAndrew is an actress. (True); Marianne McAndrew is a singer. (False); Marianne McAndrew was born on November 21, 1942. (False); Marianne McAndrew was born in Cleveland, Ohio. (False); She began her acting career in the late 1960s. (True); She has appeared in various television shows. (True); She has appeared in various films. (True)]}
                \newline
                [The end of the list of checked facts]
                \newline \newline
                [The start of the ideal output]
                \newline
                \textcolor{labelcolor}{[Marianne McAndrew (True); is (True); an (True); American (False); actress (True); and (True); singer (False); , (True); born (True); on (True); November 21, 1942 (False); , (True); in (True); Cleveland, Ohio (False); . (True); She (True); began (True); her (True); acting career (True); in (True); the late 1960s (True); , (True); appearing (True); in (True); various (True); television shows (True); and (True); films (True); . (True)]}
                \newline
                [The end of the ideal output]
				\newline \newline
                \textbf{[Example 2]}
                \newline
                [The start of the biography]
                \newline
                \textcolor{titlecolor}{Doug Sheehan is an American actor who was born on April 27, 1949, in Santa Monica, California. He is best known for his roles in soap operas, including his portrayal of Joe Kelly on ``General Hospital'' and Ben Gibson on ``Knots Landing.''}
                \newline
                [The end of the biography]
                \newline \newline
                [The start of the list of checked facts]
                \newline
                \textcolor{anscolor}{[Doug Sheehan is an American. (True); Doug Sheehan is an actor. (True); Doug Sheehan was born on April 27, 1949. (True); Doug Sheehan was born in Santa Monica, California. (False); He is best known for his roles in soap operas. (True); He portrayed Joe Kelly. (True); Joe Kelly was in General Hospital. (True); General Hospital is a soap opera. (True); He portrayed Ben Gibson. (True); Ben Gibson was in Knots Landing. (True); Knots Landing is a soap opera. (True)]}
                \newline
                [The end of the list of checked facts]
                \newline \newline
                [The start of the ideal output]
                \newline
                \textcolor{labelcolor}{[Doug Sheehan (True); is (True); an (True); American (True); actor (True); who (True); was born (True); on (True); April 27, 1949 (True); in (True); Santa Monica, California (False); . (True); He (True); is (True); best known (True); for (True); his roles in soap operas (True); , (True); including (True); in (True); his portrayal (True); of (True); Joe Kelly (True); on (True); ``General Hospital'' (True); and (True); Ben Gibson (True); on (True); ``Knots Landing.'' (True)]}
                \newline
                [The end of the ideal output]
				\newline \newline
				\textbf{User prompt}
				\newline
				\newline
				[The start of the biography]
				\newline
				\textcolor{magenta}{\texttt{\{BIOGRAPHY\}}}
				\newline
				[The ebd of the biography]
				\newline \newline
				[The start of the list of checked facts]
				\newline
				\textcolor{magenta}{\texttt{\{LIST OF CHECKED FACTS\}}}
				\newline
				[The end of the list of checked facts]
			};
		\end{tikzpicture}
        \caption{GPT-4o prompt for labeling hallucinated entities.}\label{tb:gpt-4-prompt}
	\end{center}
\vspace{-0cm}
\end{table*}
% \section{Full Experiment Results}
% \begin{table*}[th]
    \centering
    \small
    \caption{Classification Results}
    \begin{tabular}{lcccc}
        \toprule
        \textbf{Method} & \textbf{Accuracy} & \textbf{Precision} & \textbf{Recall} & \textbf{F1-score} \\
        \midrule
        \multicolumn{5}{c}{\textbf{Zero Shot}} \\
                Zero-shot E-eyes & 0.26 & 0.26 & 0.27 & 0.26 \\
        Zero-shot CARM & 0.24 & 0.24 & 0.24 & 0.24 \\
                Zero-shot SVM & 0.27 & 0.28 & 0.28 & 0.27 \\
        Zero-shot CNN & 0.23 & 0.24 & 0.23 & 0.23 \\
        Zero-shot RNN & 0.26 & 0.26 & 0.26 & 0.26 \\
DeepSeek-0shot & 0.54 & 0.61 & 0.54 & 0.52 \\
DeepSeek-0shot-COT & 0.33 & 0.24 & 0.33 & 0.23 \\
DeepSeek-0shot-Knowledge & 0.45 & 0.46 & 0.45 & 0.44 \\
Gemma2-0shot & 0.35 & 0.22 & 0.38 & 0.27 \\
Gemma2-0shot-COT & 0.36 & 0.22 & 0.36 & 0.27 \\
Gemma2-0shot-Knowledge & 0.32 & 0.18 & 0.34 & 0.20 \\
GPT-4o-mini-0shot & 0.48 & 0.53 & 0.48 & 0.41 \\
GPT-4o-mini-0shot-COT & 0.33 & 0.50 & 0.33 & 0.38 \\
GPT-4o-mini-0shot-Knowledge & 0.49 & 0.31 & 0.49 & 0.36 \\
GPT-4o-0shot & 0.62 & 0.62 & 0.47 & 0.42 \\
GPT-4o-0shot-COT & 0.29 & 0.45 & 0.29 & 0.21 \\
GPT-4o-0shot-Knowledge & 0.44 & 0.52 & 0.44 & 0.39 \\
LLaMA-0shot & 0.32 & 0.25 & 0.32 & 0.24 \\
LLaMA-0shot-COT & 0.12 & 0.25 & 0.12 & 0.09 \\
LLaMA-0shot-Knowledge & 0.32 & 0.25 & 0.32 & 0.28 \\
Mistral-0shot & 0.19 & 0.23 & 0.19 & 0.10 \\
Mistral-0shot-Knowledge & 0.21 & 0.40 & 0.21 & 0.11 \\
        \midrule
        \multicolumn{5}{c}{\textbf{4 Shot}} \\
GPT-4o-mini-4shot & 0.58 & 0.59 & 0.58 & 0.53 \\
GPT-4o-mini-4shot-COT & 0.57 & 0.53 & 0.57 & 0.50 \\
GPT-4o-mini-4shot-Knowledge & 0.56 & 0.51 & 0.56 & 0.47 \\
GPT-4o-4shot & 0.77 & 0.84 & 0.77 & 0.73 \\
GPT-4o-4shot-COT & 0.63 & 0.76 & 0.63 & 0.53 \\
GPT-4o-4shot-Knowledge & 0.72 & 0.82 & 0.71 & 0.66 \\
LLaMA-4shot & 0.29 & 0.24 & 0.29 & 0.21 \\
LLaMA-4shot-COT & 0.20 & 0.30 & 0.20 & 0.13 \\
LLaMA-4shot-Knowledge & 0.15 & 0.23 & 0.13 & 0.13 \\
Mistral-4shot & 0.02 & 0.02 & 0.02 & 0.02 \\
Mistral-4shot-Knowledge & 0.21 & 0.27 & 0.21 & 0.20 \\
        \midrule
        
        \multicolumn{5}{c}{\textbf{Suprevised}} \\
        SVM & 0.94 & 0.92 & 0.91 & 0.91 \\
        CNN & 0.98 & 0.98 & 0.97 & 0.97 \\
        RNN & 0.99 & 0.99 & 0.99 & 0.99 \\
        % \midrule
        % \multicolumn{5}{c}{\textbf{Conventional Wi-Fi-based Human Activity Recognition Systems}} \\
        E-eyes & 1.00 & 1.00 & 1.00 & 1.00 \\
        CARM & 0.98 & 0.98 & 0.98 & 0.98 \\
\midrule
 \multicolumn{5}{c}{\textbf{Vision Models}} \\
           Zero-shot SVM & 0.26 & 0.25 & 0.25 & 0.25 \\
        Zero-shot CNN & 0.26 & 0.25 & 0.26 & 0.26 \\
        Zero-shot RNN & 0.28 & 0.28 & 0.29 & 0.28 \\
        SVM & 0.99 & 0.99 & 0.99 & 0.99 \\
        CNN & 0.98 & 0.99 & 0.98 & 0.98 \\
        RNN & 0.98 & 0.99 & 0.98 & 0.98 \\
GPT-4o-mini-Vision & 0.84 & 0.85 & 0.84 & 0.84 \\
GPT-4o-mini-Vision-COT & 0.90 & 0.91 & 0.90 & 0.90 \\
GPT-4o-Vision & 0.74 & 0.82 & 0.74 & 0.73 \\
GPT-4o-Vision-COT & 0.70 & 0.83 & 0.70 & 0.68 \\
LLaMA-Vision & 0.20 & 0.23 & 0.20 & 0.09 \\
LLaMA-Vision-Knowledge & 0.22 & 0.05 & 0.22 & 0.08 \\

        \bottomrule
    \end{tabular}
    \label{full}
\end{table*}




\end{document}

\caption{
EM (above) and F1 (below) of different models and baselines across languages on \ourdataset.
Avg. denotes the average performance of the baseline across all languages.
The best results of each model under each language are annotated in \textbf{bold}. 
}
\label{tab:main}
\end{table*}

\paragraph{Metrics}
% 我们使用Exact Match (EM)和F1衡量答案的正确性,follow前人工作
We use Exact Match (EM) and F1 score to evaluate the answers, following prior works \cite{chen-etal-2020-hybridqa,zhu-etal-2021-tat}. 
% EM是指模型的预测结果与目标答案完全一致的比例
EM refers to the proportion of predictions that exactly match the gold answer, and F1 measures the degree of overlap between the predicted and the gold answer in terms of their bag-of-words representation.
% F1则衡量了预测答案和真实答案之间词袋的重叠程度

\paragraph{Models}
% 我们使用了开源模型Llama3.1-Instruct (Llama3.1)和闭源模型gpt-4o来评测我们的数据集
We evaluate \ourdataset using the open-source model Llama3.1-Instruct (Llama3.1)~\cite{dubey-etal-2024-llama3.1} and the closed-source model \texttt{gpt-4o}~\cite{openai2024gpt4technicalreport}. 
% Llama3.1是目前表现最好的开源模型之一
Llama3.1 is currently one of the best-performing open-source models, and \texttt{gpt-4o} is considered one of the leading closed-source models.
% gpt-4o则是当前最优秀的闭源模型之一


\paragraph{Baselines}
% 我们将我们的方法和以下baseline比较,follow前人工作
We compare \ourmethod with the following baselines with three-shot prompts, following previous works \cite{shi2023MGSM,li-etal-2024-eliciting-Multilingual-code}.
\begin{itemize}
    % - Native-CoT:用原语言的CoT解答问题
    \item Native-CoT: solving the question using CoT~\cite{wei2022chain-of-thought} in the native language
    % - En-CoT:用英语CoT解答问题
    \item En-CoT: solving the question using CoT in English
    % - Native-PoT:用原语言的instruction提示LLM生成代码回答问题
    \item Native-PoT: prompting the LLM to generate code in the native language \cite{pal,pot}
    % - En-PoT:用英语的instruction提示LLM生成代码回答问题
    \item En-PoT: prompting the LLM to generate code in English 
    % Three-Agent是TAT-QA数据集上的SOTA方法,由3个agent构成:analyst agent负责抽取相关数据并计算,两个critic agent分别负责判断抽取和计算的正确性,并据此进行修改
    \item Three-Agent~\cite{fatemi2024three-agent} is the state-of-the-art method on the TAT-QA dataset. It consists of three agents: the analyst agent extracts relevant data and performs computations, and two critic agents evaluate the correctness of extraction and computation, respectively, and refine the results accordingly. 
    % 受限于计算资源,我们没有在gpt-4o上评测Three-Agent在我们数据集上的性能
    Due to computational resource limitations, we do not evaluate the performance of Three-Agent on \ourdataset using \texttt{gpt-4o}.
\end{itemize}
We present prompts for baselines and \ourmethod in Appendix~\ref{sec:prompt}.
% 我们还在附录中提供了直接回答和将输入翻译为英文之后推理的结果
Additionally, we provide results for both directly answering the question and reasoning after translating the input into English in Appendix~\ref{subsec:otherbaselines}.


\subsection{Main Experiments}
\label{subsec:main experiments}
% 在不同语言上我们的方法和其他baseline的对比如表所示
A comparison of \ourmethod with other baselines across different languages is presented in Table~\ref{tab:main}. 
% 可以发现
We observe that:
% 1. 非英语语言的TATQA性能相比英语性能平均下降%,证明了我们数据集的必要性
(\emph{i})~The performance on \ourdataset in non-English languages shows an average decrease of $19.4\%$ compared to English, underscoring the necessity of \ourdataset.
% 2. 我们的方法相比其他baseline平均提升%,弥合了%不同语言之间的性能差距,证明了我们方法的有效性
(\emph{ii})~\ourmethod demonstrates an average improvement of $3.3$ on EM and F1 over other baselines, reducing the performance gap between different languages by $23.2\%$, which validates the effectiveness.
% 3. 即使提升,所有baseline在所有语言上的EM和F1均低于40,证明了我们数据集的挑战性
(\emph{iii})~Despite these improvements, the EM and F1 of all baselines remain below $40$, highlighting the challenges of \ourdataset.


\paragraph{Baselines}
% 我们方法一致地超越了Three-Agent,是因为Three-Agent不完全适用于不需要计算的HybridQA数据集,和需要复杂计算很难依靠模型本身计算能力求解的SciTAT数据集,并且在非英语语言上,multi-agent的性能下降
(\emph{i})~\ourmethod consistently outperforms Three-Agent because Three-Agent is not fully suited to HybridQA, which does not require computations \cite{chen-etal-2020-hybridqa}, or SciTAT, which involves complex calculations that are challenging to the inherent capabilities of models \cite{zhang2024scitat}. 
Additionally, the performance of multi-agent declines in non-English languages \cite{beyer2024clembench,chen-etal-2024-Cross-Agent}.
% 用原语言推理和用英语推理之间的性能差异不大
(\emph{ii})~The performance difference between reasoning in the native language and English is minimal. 
% 虽然模型在英文上展现出更强的推理能力,但由于TATQA任务相比其他任务,如数值推理任务,更需要模型将问题中的实体对应到文本或表格中的实体的能力,而这一问题在跨语言推理时会更具有挑战
Although LLMs demonstrate stronger reasoning capabilities in English, the TATQA, compared to other tasks, relies more heavily on the capabilities of linking information, which presents greater challenges in cross-lingual reasoning \cite{min-etal-2019-cspider}. 
% 而我们的方法缓解了这一挑战,所以提升了性能
Therefore, \ourmethod mitigates this challenge, leading to improved performance. 
% 并且,PoT方法在我们数据集上的性能普遍优于CoT,因为我们数据集中数值推理类问题占较大比例(见表1),所以更适合PoT方法求解
(\emph{iii})~PoT consistently outperforms CoT because numerical reasoning questions constitute a significant proportion of \ourdataset (see Table~\ref{tab:answer_statistics}), making PoT more suitable for solving these questions \cite{pot,zhao-etal-2024-docmath}.

\paragraph{Languages}
% 模型普遍表现出在高资源语言,包括英语,德语,西班牙语,法语,俄语和中文上的性能普遍较高,而在低资源语言上的性能较差
The models generally exhibit high performance on high-resource languages, such as English, German, Spanish, French, Russian, and Chinese, while their performance on low-resource languages tends to be poor. 
% 而多语言能力越强的模型,在不同语言之间性能的差距越小,其中gpt-4o展现出了最强的性能
Moreover, models with stronger multilingual capabilities show smaller performance gaps across languages, with \texttt{gpt-4o} demonstrating the highest performance. 
% 这也证明了在有挑战的任务上评测模型多语言性能的必要性
This also underscores the necessity of evaluating multilingual performance on challenging tasks.

% \begin{figure*}[t]
%     \centering
%     \resizebox{0.75\linewidth}{!}{
\begin{tikzpicture}
\begin{polaraxis}[
    title={\empty},
    xlabel={\empty},
    ylabel={\empty},
    xtick={0,32.73,65.46,98.19,130.92,163.65,196.38,229.11,261.84,294.57,327.30},
    xticklabels={bn, de, en, es, fr, ja, ru, sw, te, th, zh},
    % xticklabel style={font=\small, yshift=2ex}, % 标签离圆圈远一些,调整 y 轴偏移
    ytick={10,20,30,40},
    yticklabels={10,20,30,40},
    grid=both,
    tick label style={font=\small},
    major grid style={solid, gray}, % 外层圆圈设为黑色实线
    minor grid style={solid, gray}, % 次要网格线也设为黑色实线(可选)
    % axis line style={solid, black},   % 径向线设为灰色实线
    axis line style={draw=none},
    axis on top=true,                % 确保轴线在顶部
    grid style={gray},              % 确保网格线颜色为黑色
    line join=bevel,
    tick align=outside,
    % major grid style={solid, gray},
    % minor grid style={dotted, gray},
    % axis line style={draw=none},
    % axis on top
]

% 第一组数据
\addplot[
    color=data_blue!300,
    mark=diamond,
    style={line width=1pt},
    fill=data_blue!250,fill opacity=0.2
] coordinates {
    (0,24.2)(32.73,24.2)(65.46,26.4)(98.19,24.2)(130.92,23.1)(163.65,26.4)(196.38,20.9)(229.11,24.2)(261.84,23.1)(294.57,25.3)(327.30,26.4)(0,24.2)
} -- cycle;

% 第二组数据
\addplot[
    color=reasoner_green!300,
    mark=square,
    style={line width=1pt},
    fill=reasoner_green!250,fill opacity=0.2
] coordinates {
    (0,13.0)(32.73,20.4)(65.46,27.8)(98.19,25.9)(130.92,25.9)(163.65,22.2)(196.38,22.2)(229.11,33.3)(261.84,14.8)(294.57,20.4)(327.30,18.5)(0,13.0)
} -- cycle;


% 第三组数据
\addplot[
    color=annotator_pink!150,
    mark=triangle,
    style={line width=1pt},
    fill=annotator_pink,fill opacity=0.2
] coordinates {
    (0,29.5)(32.73,35.2)(65.46,37.1)(98.19,37.1)(130.92,30.5)(163.65,29.5)(196.38,39.0)(229.11,35.2)(261.84,26.7)(294.57,31.4)(327.30,34.3)(0,29.5)
} -- cycle;

\legend{Table, Text, Hybrid}

\end{polaraxis}
\end{tikzpicture}
}

%     \vspace{-0.5em}
%     \caption{
%         The EM of \ourmethod across different answer sources on Llama3.1-70B.
%     }
%     \label{fig:answer_source}
%     \vspace{-1em}
% \end{figure*}

\begin{figure}[t]
    \centering
    \resizebox{0.75\linewidth}{!}{
\begin{tikzpicture}
\begin{polaraxis}[
    title={\empty},
    xlabel={\empty},
    ylabel={\empty},
    xtick={0,32.73,65.46,98.19,130.92,163.65,196.38,229.11,261.84,294.57,327.30},
    xticklabels={bn, de, en, es, fr, ja, ru, sw, te, th, zh},
    % xticklabel style={font=\small, yshift=2ex}, % 标签离圆圈远一些,调整 y 轴偏移
    ytick={10,20,30,40},
    yticklabels={10,20,30,40},
    grid=both,
    tick label style={font=\small},
    major grid style={solid, gray}, % 外层圆圈设为黑色实线
    minor grid style={solid, gray}, % 次要网格线也设为黑色实线(可选)
    % axis line style={solid, black},   % 径向线设为灰色实线
    axis line style={draw=none},
    axis on top=true,                % 确保轴线在顶部
    grid style={gray},              % 确保网格线颜色为黑色
    line join=bevel,
    tick align=outside,
    % major grid style={solid, gray},
    % minor grid style={dotted, gray},
    % axis line style={draw=none},
    % axis on top
]

% 第一组数据
\addplot[
    color=data_blue!300,
    mark=diamond,
    style={line width=1pt},
    fill=data_blue!250,fill opacity=0.2
] coordinates {
    (0,24.2)(32.73,24.2)(65.46,26.4)(98.19,24.2)(130.92,23.1)(163.65,26.4)(196.38,20.9)(229.11,24.2)(261.84,23.1)(294.57,25.3)(327.30,26.4)(0,24.2)
} -- cycle;

% 第二组数据
\addplot[
    color=reasoner_green!300,
    mark=square,
    style={line width=1pt},
    fill=reasoner_green!250,fill opacity=0.2
] coordinates {
    (0,13.0)(32.73,20.4)(65.46,27.8)(98.19,25.9)(130.92,25.9)(163.65,22.2)(196.38,22.2)(229.11,33.3)(261.84,14.8)(294.57,20.4)(327.30,18.5)(0,13.0)
} -- cycle;


% 第三组数据
\addplot[
    color=annotator_pink!150,
    mark=triangle,
    style={line width=1pt},
    fill=annotator_pink,fill opacity=0.2
] coordinates {
    (0,29.5)(32.73,35.2)(65.46,37.1)(98.19,37.1)(130.92,30.5)(163.65,29.5)(196.38,39.0)(229.11,35.2)(261.84,26.7)(294.57,31.4)(327.30,34.3)(0,29.5)
} -- cycle;

\legend{Table, Text, Hybrid}

\end{polaraxis}
\end{tikzpicture}
}

    \caption{
        The EM of \ourmethod across different answer sources on \ourdataset using Llama3.1-70B.
    }
    \label{fig:answer_source}
\end{figure}

\paragraph{Answer Source}
% 我们分析了我们的方法使用Llama3.1-70B时在不同答案来源上的性能,如图所示
We analyze the performance of \ourmethod using Llama3.1-70B across different answer sources, as shown in Figure~\ref{fig:answer_source}. 
% 使用其他模型时以及不同的基线在不同来源上的性能在附录中提供
The performance with other models and baselines across answer sources is provided in Appendix~\ref{subsec:appendix_answer_source}. 
% 可以发现
The results show that:
% 1. 答案来源为Hybrid的问题的性能整体上优于单一答案来源的问题
(\emph{i})~The performance of the hybrid answer source generally outperforms those with a single answer source. 
% 因为我们的方法相比基线(见表)可以同时从表格和文本中链接到相关信息,整合了异质的相关上下文,一定程度上缓解了混合的答案来源的挑战
Since \ourmethod, compared to other baselines (see Figure~\ref{fig:answer_sources_other_baselines}), enhances the links between the question and the context, integrating hybrid contextual information and alleviating the challenge.
% 2. 虽然整体上英语的性能最佳,但在不同的答案来源上不同语言展现出了不同的趋势
% While English performs best overall, different languages exhibit varying trends across answer sources. 
% 不同语言在答案来源上的性能除了与高资源或低资源有关,还与语言本身的特性有关
(\emph{ii})~The performance across answer sources is influenced not only by the availability of language-specific resources but also by the characteristics of the language.
% 比如,词法结构复杂的语言在答案来源是text上的性能较差,如德语、俄语等
For instance, languages with complex morphological structures, such as German and Russian, perform worse when the answer source is text. 
% 而斯瓦希里语在文本上的性能最高,因为其有着较为简单的词法结构,能相对容易地根据问题中的实体链接到文本中的实体
In contrast, Swahili shows the highest performance on text-based sources, as its simpler morphology allows for easier linking of entities in the text to those in question \cite{tuan-nguyen-etal-2020-Vietnamese,zhang-etal-2023-xsemplr}.

% \begin{figure*}[t]
%     \centering
%     \resizebox{0.75\linewidth}{!}{
\begin{tikzpicture}
\begin{polaraxis}[
    title={\empty},
    xlabel={\empty},
    ylabel={\empty},
    xtick={0,32.73,65.46,98.19,130.92,163.65,196.38,229.11,261.84,294.57,327.30},
    xticklabels={bn, de, en, es, fr, ja, ru, sw, te, th, zh},
    % xticklabel style={font=\small, yshift=2ex}, % 标签离圆圈远一些,调整 y 轴偏移
    ytick={20,40,60,80,100},
    yticklabels={20,40,60,80,100},
    grid=both,
    tick label style={font=\small},
    major grid style={solid, gray}, % 外层圆圈设为黑色实线
    minor grid style={solid, gray}, % 次要网格线也设为黑色实线(可选)
    % axis line style={solid, black},   % 径向线设为灰色实线
    axis line style={draw=none},
    axis on top=true,                % 确保轴线在顶部
    grid style={gray},              % 确保网格线颜色为黑色
    line join=bevel,
    tick align=outside,
    % major grid style={solid, gray},
    % minor grid style={dotted, gray},
    % axis line style={draw=none},
    % axis on top
]

% 第一组数据
\addplot[
    color=data_blue!300,
    mark=diamond,
    style={line width=1pt},
    fill=data_blue!250,fill opacity=0.2
] coordinates {
    (0,10.8)(32.73,12.1)(65.46,16.6)(98.19,16.6)(130.92,12.1)(163.65,13.4)(196.38,13.4)(229.11,15.9)(261.84,12.1)(294.57,14.0)(327.30,12.3)(0,10.8)
} -- cycle;

% 第二组数据
\addplot[
    color=reasoner_green!300,
    mark=square,
    style={line width=1pt},
    fill=reasoner_green!250,fill opacity=0.2
] coordinates {
    (0,44.6)(32.73,50.6)(65.46,51.8)(98.19,48.2)(130.92,47.0)(163.65,48.2)(196.38,51.8)(229.11,53.0)(261.84,41.0)(294.57,47.0)(327.30,52.5)(0,44.6)
} -- cycle;

% 第三组数据
\addplot[
    color=annotator_pink!150,
    mark=triangle,
    style={line width=1pt},
    fill=annotator_pink,fill opacity=0.2
] coordinates {
    (0,60.0)(32.73,90.0)(65.46,90.0)(98.19,90.0)(130.92,90.0)(163.65,60.0)(196.38,80.0)(229.11,80.0)(261.84,40.0)(294.57,60.0)(327.30,90.0)(0,60.0)
} -- cycle;

\legend{Span, Arithmetic, Count}

\end{polaraxis}
\end{tikzpicture}
}

%     \vspace{-0.5em}
%     \caption{
%         The EM of \ourmethod across different answer types on Llama3.1-70B.
%     }
%     \label{fig:answer_type}
%     \vspace{-1em}
% \end{figure*}

\begin{figure}[t]
    \centering
    \resizebox{0.75\linewidth}{!}{
\begin{tikzpicture}
\begin{polaraxis}[
    title={\empty},
    xlabel={\empty},
    ylabel={\empty},
    xtick={0,32.73,65.46,98.19,130.92,163.65,196.38,229.11,261.84,294.57,327.30},
    xticklabels={bn, de, en, es, fr, ja, ru, sw, te, th, zh},
    % xticklabel style={font=\small, yshift=2ex}, % 标签离圆圈远一些,调整 y 轴偏移
    ytick={20,40,60,80,100},
    yticklabels={20,40,60,80,100},
    grid=both,
    tick label style={font=\small},
    major grid style={solid, gray}, % 外层圆圈设为黑色实线
    minor grid style={solid, gray}, % 次要网格线也设为黑色实线(可选)
    % axis line style={solid, black},   % 径向线设为灰色实线
    axis line style={draw=none},
    axis on top=true,                % 确保轴线在顶部
    grid style={gray},              % 确保网格线颜色为黑色
    line join=bevel,
    tick align=outside,
    % major grid style={solid, gray},
    % minor grid style={dotted, gray},
    % axis line style={draw=none},
    % axis on top
]

% 第一组数据
\addplot[
    color=data_blue!300,
    mark=diamond,
    style={line width=1pt},
    fill=data_blue!250,fill opacity=0.2
] coordinates {
    (0,10.8)(32.73,12.1)(65.46,16.6)(98.19,16.6)(130.92,12.1)(163.65,13.4)(196.38,13.4)(229.11,15.9)(261.84,12.1)(294.57,14.0)(327.30,12.3)(0,10.8)
} -- cycle;

% 第二组数据
\addplot[
    color=reasoner_green!300,
    mark=square,
    style={line width=1pt},
    fill=reasoner_green!250,fill opacity=0.2
] coordinates {
    (0,44.6)(32.73,50.6)(65.46,51.8)(98.19,48.2)(130.92,47.0)(163.65,48.2)(196.38,51.8)(229.11,53.0)(261.84,41.0)(294.57,47.0)(327.30,52.5)(0,44.6)
} -- cycle;

% 第三组数据
\addplot[
    color=annotator_pink!150,
    mark=triangle,
    style={line width=1pt},
    fill=annotator_pink,fill opacity=0.2
] coordinates {
    (0,60.0)(32.73,90.0)(65.46,90.0)(98.19,90.0)(130.92,90.0)(163.65,60.0)(196.38,80.0)(229.11,80.0)(261.84,40.0)(294.57,60.0)(327.30,90.0)(0,60.0)
} -- cycle;

\legend{Span, Arithmetic, Count}

\end{polaraxis}
\end{tikzpicture}
}

    \vspace{-0.5em}
    \caption{
        The EM of \ourmethod across different answer types on \ourdataset using Llama3.1-70B.
    }
    \label{fig:answer_type}
    \vspace{-1em}
\end{figure}

\paragraph{Answer Type}
% 我们比较了我们方法在Llama3.1-70B上在不同答案类型上的性能,如图所示
We compare the performance of \ourmethod using Llama3.1-70B on different answer types, as shown in Figure~\ref{fig:answer_type}. 
% 我们在附录提供了其他模型和基线在不同答案类型上的性能
Results of other models and baselines across answer types are provided in Appendix~\ref{subsec:appendix_answer_types}. 
% 我们发现
We observe that:
% 1. 模型在Span类型和Arithmetic上的性能不好,而在Count类型上的性能最高
% (\emph{i})~The model performs poorly on Span and Arithmetic types, while it performs best on the Count type. 
(\emph{i})~The model performs best on the Count type. 
% 因为span类型的答案需要从表格和文本中抽取出短语,或用几句话总结分析,相比较Arithmetic和Count的答案都为数字,对词语的构成和顺序更为敏感
This is because Span answers require extracting short phrases or summarizing conclusions from tables and text, making them more sensitive to word composition and order. 
% 而Arithmetic类型相比较Count类型需要模型进行更加复杂的运算
Additionally, Arithmetic answers involve more complex computations than Count answers.
% 2. 模型在不同答案类型上的性能整体上呈现高资源语言优于低资源语言
(\emph{ii})~The model performs better on high-resource languages than low-resource languages across answer types overall. 
% 即使我们的方法减小了性能差距,但低资源语言和高资源语言之间仍在不同的答案类型上均存在较大的性能差距
Although \ourmethod narrows the performance gap, there remains a significant difference between high-resource and low-resource languages for all answer types.

\subsection{Analysis}
% 在本小节,我们主要分析模型在TATQA任务上的跨语言能力

\begin{table*}[ht]
\centering
\tiny
% \begin{tabular}{llcccccccccccc}
% \toprule
% \textbf{Instruction} & \textbf{Demo} & \textbf{bn} & \textbf{de} & \textbf{en} & \textbf{es} & \textbf{fr} & \textbf{ja} & \textbf{ru} & \textbf{sw} & \textbf{te} & \textbf{th} & \textbf{zh} & \textbf{Average} \\
% \midrule
% \multirow{3}{*}{Native} & Native & $20.0/\bm{23.8}$ & $28.4/\bm{33.9}$ & $28.4/\bm{33.8}$ & $29.2/\bm{35.8}$ & $29.2/\bm{34.0}$ & $27.6/\bm{30.1}$ & $27.6/\bm{31.7}$ & $\bm{32.0}/35.1$ & $20.4/\bm{24.2}$ & $25.2/\bm{28.3}$ & $28.8/\bm{37.4}$ & $27.0/\bm{31.7}$ \\
% & Multi & $22.0/\bm{24.6}$ & $\bm{30.0}/32.3$ & $30.4/\bm{35.4}$ & $30.4/\bm{35.0}$ & $28.4/\bm{31.6}$ & $26.0/\bm{27.6}$ & $26.4/\bm{28.8}$ & $28.8/\bm{30.7}$ & $24.4/\bm{26.3}$ & $24.4/\bm{26.7}$ & $24.8/\bm{30.7}$ & $26.9/\bm{30.0}$ \\
% & En & $20.8/\bm{24.4}$ & $29.2/\bm{33.6}$ & $28.4/\bm{33.8}$ & $24.8/\bm{30.6}$ & $27.2/\bm{32.0}$ & $24.0/\bm{22.8}$ & $28.4/\bm{31.8}$ & $29.2/\bm{31.6}$ & $19.6/\bm{22.3}$ & $21.2/\bm{23.5}$ & $24.4/\bm{30.7}$ & $24.9/\bm{28.8}$ \\
% \midrule
% \multirow{3}{*}{En} & Native & $\bm{27.6}/30.5$ & $28.6/\bm{30.3}$ & $28.4/\bm{33.8}$ & $29.6/\bm{34.1}$ & $25.2/\bm{29.6}$ & $25.6/\bm{28.3}$ & $29.2/\bm{33.0}$ & $30.0/\bm{33.6}$ & $28.0/\bm{31.4}$ & $26.8/\bm{34.1}$ & $27.6/\bm{31.5}$ \\
% & Multi & $26.4/\bm{28.9}$ & $27.2/\bm{29.9}$ & $\bm{30.4}/35.4$ & $30.8/\bm{34.1}$ & $29.6/\bm{32.2}$ & $29.2/\bm{31.7}$ & $30.0/\bm{32.6}$ & $30.0/\bm{33.6}$ & $27.2/\bm{29.8}$ & $27.2/\bm{31.2}$ & $28.8/\bm{34.1}$ & $\bm{28.8}/32.1$ \\
% & En & $24.0/\bm{26.3}$ & $28.0/\bm{31.3}$ & $31.2/\bm{35.3}$ & $29.2/\bm{34.1}$ & $26.8/\bm{31.1}$ & $26.8/\bm{29.4}$ & $28.8/\bm{33.5}$ & $30.8/\bm{34.7}$ & $22.8/\bm{25.9}$ & $26.8/\bm{30.5}$ & $28.0/\bm{34.9}$ & $27.6/\bm{31.6}$ \\
% \bottomrule
% \end{tabular}

% \begin{tabular}{llcccccc}
% \toprule
% \textbf{Instruction} & \textbf{Demo} & \textbf{bn} & \textbf{de} & \textbf{en} & \textbf{es} & \textbf{fr} & \textbf{ja}\\
% \midrule
% \multirow{3}{*}{Native} & Native & $20.0/23.8$ & $28.4/33.9$ & $28.4/33.8$ & $29.2/35.8$ & $29.2/\bm{34.0}$ & $27.6/30.1$ \\
% & Multi & $22.0/24.6$ & $\bm{30.0/32.3}$ & $30.4/\bm{35.4}$ & $30.4/\bm{35.0}$ & $28.4/\bm{31.6}$ & $26.0/27.6$ \\
% & En & $20.8/24.4$ & $29.2/33.6$ & $28.4/33.8$ & $24.8/30.2$ & $27.2/32.0$ & $24.0/22.8$ \\
% \midrule
% \multirow{3}{*}{En} & Native & $\bm{27.6/30.5}$ & $26.8/30.3$ & $28.4/33.8$ & $29.6/32.7$ & $25.2/29.6$ & $25.6/28.5$ \\
% & Multi & $26.4/28.9$ & $27.2/29.9$ & $30.4/\bm{35.4}$ & $\bm{30.8}/34.1$ & $\bm{29.6}/32.2$ & $\bm{29.2/31.7}$ \\
% & En & $24.0/26.3$ & $28.0/31.3$ & $\bm{31.2}/35.3$ & $29.2/34.6$ & $26.8/31.1$ & $26.8/29.4$ \\
% \bottomrule
% \end{tabular}

% \begin{tabular}{llcccccc}
% \toprule
% \textbf{Instruction} & \textbf{Demo} & \textbf{ru} & \textbf{sw} & \textbf{te} & \textbf{th} & \textbf{zh} & \textbf{Avg.} \\
% \midrule
% \multirow{3}{*}{Native} & Native & $27.6/31.7$ & $\bm{32.0/35.1}$ & $20.4/24.2$ & $25.2/28.3$ & $\bm{28.8/37.4}$ & $27.0/31.7$ \\
% & Multi & $26.4/28.8$ & $28.8/30.7$ & $24.4/26.3$ & $24.4/26.7$ & $24.8/30.7$ & $26.9/30.0$ \\
% & En & $28.4/31.8$ & $29.2/31.6$ & $19.6/22.3$ & $21.2/23.5$ & $24.4/30.7$ & $24.9/28.8$ \\
% \midrule
% \multirow{3}{*}{En} & Native & $29.2/33.0$ & $30.0/33.6$ & $26.0/28.8$ & $\bm{28.0/31.4}$ & $26.8/34.1$ & $27.6/31.5$ \\
% & Multi & $\bm{30.0}/32.6$ & $30.0/33.0$ & $\bm{27.2/29.8}$ & $27.2/31.2$ & $\bm{28.8}/34.7$ & $\bm{28.8/32.1}$ \\
% & En & $28.8/\bm{33.5}$ & $30.8/34.7$ & $22.8/25.9$ & $26.8/30.5$ & $28.0/34.9$ & $27.6/31.6$ \\
% \bottomrule
% \end{tabular}

\begin{tabular}{l|l|ccccccccccc|c}
\toprule
\textbf{Instruction} & \textbf{Demo} & \textbf{bn} & \textbf{de} & \textbf{en} & \textbf{es} & \textbf{fr} & \textbf{ja} & \textbf{ru} & \textbf{sw} & \textbf{te} & \textbf{th} & \textbf{zh} & \textbf{Avg.} \\
\midrule
\multirow{3}{*}{Native} & Native & $20.0$ & $28.4$ & $28.4$ & $29.2$ & $29.2$ & $27.6$ & $27.6$ & \bm{$32.0$} & $20.4$ & $25.2$ & \bm{$28.8$} & $27.0$ \\
& Multi & $22.0$ & \bm{$30.0$} & $30.4$ & $30.4$ & $28.4$ & $26.0$ & $26.4$ & $28.8$ & $24.4$ & $24.4$ & $24.8$ & $26.9$ \\
& En & $20.8$ & $29.2$ & $28.4$ & $24.8$ & $27.2$ & $24.0$ & $28.4$ & $29.2$ & $19.6$ & $21.2$ & $24.4$ & $24.9$ \\
\midrule
\multirow{3}{*}{En} & Native & \bm{$27.6$} & $26.8$ & $28.4$ & $29.6$ & $25.2$ & $25.6$ & $29.2$ & $30.0$ & $26.0$ & \bm{$28.0$} & $26.8$ & $27.6$ \\
& Multi & $26.4$ & $27.2$ & $30.4$ & \bm{$30.8$} & \bm{$29.6$} & \bm{$29.2$} & \bm{$30.0$} & $30.0$ & \bm{$27.2$} & $27.2$ & \bm{$28.8$} & \bm{$28.8$} \\
& En & $24.0$ & $28.0$ & \bm{$31.2$} & $29.2$ & $26.8$ & $26.8$ & $28.8$ & $30.8$ & $22.8$ & $26.8$ & $28.0$ & $27.6$ \\
\bottomrule
\end{tabular}

\begin{tabular}{l|l|ccccccccccc|c}
\toprule
\textbf{Instruction} & \textbf{Demo} & \textbf{bn} & \textbf{de} & \textbf{en} & \textbf{es} & \textbf{fr} & \textbf{ja} & \textbf{ru} & \textbf{sw} & \textbf{te} & \textbf{th} & \textbf{zh} & \textbf{Avg.} \\
\midrule
\multirow{3}{*}{Native} & Native & $23.8$ & \bm{$33.9$} & $33.8$ & \bm{$35.8$} & $\bm{34.0}$ & $30.1$ & $31.7$ & \bm{$35.1$} & $24.2$ & $28.3$ & \bm{$37.4$} & $31.7$ \\
& Multi & $24.6$ & $32.3$ & $\bm{35.4}$ & $35.0$ & $31.6$ & $27.6$ & $28.8$ & $30.7$ & $26.3$ & $26.7$ & $30.7$ & $30.0$ \\
& En & $24.4$ & $33.6$ & $33.8$ & $30.2$ & $32.0$ & $22.8$ & $31.8$ & $31.6$ & $22.3$ & $23.5$ & $30.7$ & $28.8$ \\
\midrule
\multirow{3}{*}{En} & Native & \bm{$30.5$} & $30.3$ & $33.8$ & $32.7$ & $29.6$ & $28.5$ & $33.0$ & $33.6$ & $28.8$ & \bm{$31.4$} & $34.1$ & $31.5$ \\
& Multi & $28.9$ & $29.9$ & $\bm{35.4}$ & $34.1$ & $32.2$ & $\bm{31.7}$ & $32.6$ & $33.0$ & \bm{$29.8$} & $31.2$ & $34.7$ & \bm{$32.1$} \\
& En & $26.3$ & $31.3$ & $35.3$ & $34.6$ & $31.1$ & $29.4$ & $\bm{33.5}$ & $34.7$ & $25.9$ & $30.5$ & $34.9$ & $31.6$ \\
\bottomrule
\end{tabular}


\caption{
EM (above) and F1 (below) of \ourmethod using the instructions and demonstrations of different languages on Llama3.1-70B.
The best results under each language are annotated in \textbf{bold}. 
% Multi指的是由多种语言(英语、西班牙语和中文)组成的示例
Demo refers to demonstrations. 
Multi refers to demonstrations composed of multiple languages (English, Spanish, and Chinese).
Avg. denotes the average performance of the baseline across all languages.
}
\label{tab:prompt_language}
\end{table*}

% prompt的语言如何影响我们方法的性能
% How does the language of prompt affect the performance of our method?
% \subsubsection{Prompt Language}
\subsubsection{How does the Prompt Language Affect \ourmethod?}
% 我们分析了使用不同语言的instruction和示例对我们方法性能的影响,如表所示
We analyze the impact of using instructions and demonstrations in different languages on the performance of \ourmethod, as shown in Table~\ref{tab:prompt_language}. 
% 其中,多语言示例我们选择了分别是英语、西班牙语和中文的各一个示例,因为模型在这三个高资源语言上的性能较高,且包含两个语系
For the multilingual demonstrations, we select one demonstration each from English, Spanish, and Chinese, as the models perform well on these three high-resource languages, which also cover two language families. 
% 而英文instruction和英文示例是我们主实验采用的设置
The English instruction and English demonstrations are the settings of \ourmethod used in the main experiments. 
% 可以发现,
The results indicate that:


% 1. 使用英文instruction整体上优于使用原语言instruction
(\emph{i})~Using English instructions generally outperforms using native instructions.
% 2. 使用多语言示例打败了原语言和英语示例,说明当在某些语言上没有足够的TATQA示例时,可以采用同一语系或高资源语言的示例来提升模型性能
(\emph{ii})~Multilingual demonstrations outperform both native language and English demonstrations, suggesting that when sufficient native demonstrations are not available on the TATQA task, using demonstrations from the same language family or high-resource languages can also enhance performance.
% 同时,斯瓦希里语在使用原语言的instruction和示例时达到最高性能,说明了这种语言的独特性,模型不容易将在高资源语言上的知识和推理能力迁移到斯瓦西里语上
Additionally, Swahili achieves the highest performance when using instructions and examples in the native language, highlighting its uniqueness. 
% This suggests that it is difficult for the model to transfer knowledge and reasoning capabilities from high-resource languages to Swahili.

\begin{figure}[t]
    \centering
    \includegraphics[width=0.95\linewidth]{fig/heatmap_cross.pdf}
    \caption{
    % 被各个表示正确解决的instance之间的重复率
    The EM/F1 of \ourmethod with questions and context (table and text) of different languages on \ourdataset using Llama3.1-70B. 
    }
    \label{fig:heatmap_cross}
\end{figure}

% 跨语言
% \subsubsection{Cross-lingual QA}
\subsubsection{How does the Language Affect \ourmethod in the Cross-lingual Setting?}
% 我们评测了在跨语言QA,即问题和文本、表格的语言不一致的设置下我们的方法在我们数据集上的性能,如图所示
We evaluate the performance of \ourmethod in the cross-lingual setting, where the languages of the question and context are inconsistent, with results in Figure~\ref{fig:heatmap_cross}. 
% 我们分别选取了高资源语言法语和中文,以及低资源语言孟加拉语、斯瓦希里语和泰卢固语,涉及4个语系
We select high-resource languages (French and Chinese), and low-resource languages (Bengali, Swahili, and Telugu), covering $4$ language families. 
% 我们发现
Our findings include:
% 1. 普遍来讲,模型从低资源跨到高资源时会带来性能提升,反过来则下降
(\emph{i})~Generally, \ourmethod shows improved performance when transitioning from low-resource to high-resource languages, while the opposite results in a decline. 
% 比如用法语和中文对法语上下文提问的性能较高,而用三种低资源语言提问的性能较低
For instance, the performances on the French context with French and Chinese questions are relatively high, whereas the performances with three low-resource languages are lower.
% 2. 斯瓦希里语在跨语言的设置上表现出较为稳定的性能,且在同语言QA上达到最佳性能
(\emph{ii})~The model achieves the best performance when the question and context are both Swahili. 
% 因为斯瓦希里语较为规则的语法和词法结构,使其在我们的任务上受益,尤其是需要链接相关信息时相比其他语言更加容易
This can be attributed to its relatively regular grammatical and lexical structures, which provide advantages when linking related information.
% making it easier compared to other languages.

% \begin{table*}[ht]
% \centering
% \small
% 
\begin{lemma}\label{Lemma:multi1} 
   Fixing the number of data contributor $i$ collects $n_i$, and others' strategies $\strategy_{-i}$, $\hat{\mu}\left(X_i\right)$ is the minimax estimator for the Normal distribution class $\Normaldistrib := \left\{\mathcal{N}(\mu,\sigma^2) \;\middle|\; \mu \in \mathbb{R}\right\}$,
    \begin{align*}
       \hat{\mu}(X_i)  = \underset{\hat{\mu}}{\arg\min} \sbr{\sup _\mu \mathbb{E}\left[(\hat{\mu}( Y_i)- \hat{\mu}( Y_{-i}) )^2 \;\middle|\;  \mu \right] }
    \end{align*} 
     
\end{lemma}


\begin{proof}

\begin{align*}
    & \ \mathbb{E}\left[ \left( \hat{\mu}\left( Y_i \right)-\hat{\mu}\left( Y_{-i} \right)  \right)^2 \right] \\ =  & \ \mathbb{E}\left[ \left( (\hat{\mu}\left(  Y_i \right)-\mu) -(\hat{\mu}\left(  Y_{-i} \right) -\mu) \right)^2   \right] \\ =  & \ A_0 + \mathbb{E}\left[ (\hat{\mu}\left(  Y_i \right)-\mu)^2  \right]
\end{align*}
where $A_0$ is a positive coefficient.

Thus the maximum risk can be written as:

\begin{align*}
    \sup _\mu \mathbb{E}\left[A_0 + \left(\hat{\mu}\left( Y_i\right)-\mu\right)^{2} \;\middle|\;  \mu \right]
\end{align*}


We construct a lower bound on the maximum risk using a sequence of Bayesian risks. Let $\Lambda_{\ell}:=\mathcal{N}\left(0, \ell^2\right), \ell=1,2, \ldots$ be a sequence of prior for $\mu$. For fixed $\ell$, the posterior distribution is:
$$
\begin{aligned}
p\left(\mu \;\middle|\;  X_i\right) & \propto p\left(X_i \;\middle|\;  \mu\right) p(\mu) \\ & \propto \exp \left(-\frac{1}{2 \sigma^2} \sum_{x \in X_i}(x-\mu)^2\right) \exp \left(-\frac{1}{2 \ell^2} \mu^2\right) \\
& \propto \exp \left(-\frac{1}{2}\left(\frac{n_i}{\sigma^2}+\frac{1}{\ell^2}\right) \mu^2+\frac{1}{2} 2 \frac{\sum_{x \in X_i} x}{\sigma^2} \mu\right) .
\end{aligned}
$$

This means the posterior of $\mu$ given $X_i$ is Gaussian with:

\begin{align*}
    \mu \lvert\, X_i & \sim \mathcal{N}\left(\frac{n_i \hat{\mu}\left(X_i\right) / \sigma^2}{n_i / \sigma^2+1 / \ell^2}, \frac{1}{n_i / \sigma^2+1 / \ell^2}\right) 
    \\ & =: \mathcal{N}\left(\mu_{\ell}, \sigma_{\ell}^2\right).
\end{align*}



Therefore, the posterior risk is: 
$$
\begin{aligned}
&   \mathbb{E}\left[A_0 + \left(\hat{\mu}\left( Y_i\right)-\mu\right)^{2}  \;\middle|\;  X_i\right] \\ = &  \mathbb{E}\left[A_0 +  \left(\left(\hat{\mu}\left( Y_i\right)-\mu_{\ell}\right)-\left(\mu-\mu_{\ell}\right)\right)^{2 j} \;\middle|\;  X_i\right] \\ =
& A_0+\int_{-\infty}^{\infty} \underbrace{\left(e-\left(\hat{\mu}\left( Y_i\right)-\mu_{\ell}\right)\right)^2}_{=: F_1\left(e-\left(\hat{\mu}\left(Y_i\right)-\mu_{\ell}\right)\right)} \underbrace{\frac{1}{\sigma_{\ell} \sqrt{2 \pi}} \exp \left(-\frac{e^2}{2 \sigma_{\ell}^2}\right)}_{=: F_2(e)} d e
\end{aligned}
$$

Because:
\begin{itemize}
    \item $F_1(\cdot)$ is even function and increases on $[0, \infty)$;
    \item $F_2(\cdot)$ is even function and decreases on $\left[0, \infty \right)$, and $\int_{\mathbb{R}} F_2(e) de<\infty$
    \item For any $a \in \mathbb{R}, \int_{\mathbb{R}} F_1(e-a) F_2(e) de<\infty$
\end{itemize}

By the corollary of Hardy-Littlewood inequality in Lemma \ref{lemmaHardy},
$$
\int_{\mathbb{R}} F_1(e-a) F_2(e) d e \geq \int_{\mathbb{R}} F_1(e) F_2(e) d e
$$
which means the posterior risk is minimized when $\hat{\mu}\left(Y_i\right)=\mu_{\ell}$. We then write the Bayes risk as, the Bayes risk is minimized by the posterior mean $\mu_{\ell}$:

\begin{align*}    
R_{\ell}:= & \mathbb{E}\left[ A_0+\mathbb{E}\left[\left(\mu-\mu_{\ell}\right)^{2 } \;\middle|\;  X_i\right]\right] \\ = & A_0 + \sigma_{\ell}^{2}
\end{align*}

and the limit of Bayesian risk as $\ell \rightarrow \infty$ is
$$
R_{\infty}:= A_0 + \frac{\sigma^{2}}{n_i}.
$$

When $\hat{\mu}\left(Y_i\right)=\hat{\mu}\left(X_i\right)$, i.e, the contributor submit a set of size $n_i$ with each element equal to $ \hat{\mu}\left(X_i\right)$, the maximum risk is:

\begin{align*}
& \sup _\mu \mathbb{E}\left[A_0+\left(\mu- 
\hat{\mu}\left(Y_i\right) \right)^{2 } \;\middle|\;  \mu \right] \\
= & \sup _\mu \mathbb{E}\left[A_0+\left(\mu- 
\hat{\mu}\left(X_i\right) \right)^{2 } \;\middle|\;  \mu \right]  \\
= & A_0+ \sigma^{2 } n_i^{-1}  \\
= &  R_{\infty}.
\end{align*}

This implies that,
\begin{align*}
    & \underset{\mu}{\sup}\; \mathbb{E} \sbr{ \rbr{\hat{\mu}\left( Y_i \right)-\hat{\mu}\left( Y_{-i} \right)  }^2 \;\middle|\;  \mu }  \\ \geq & \; R_{\infty} =  \sup _\mu \; \mathbb{E}\left[A_0+\left(\mu- 
\hat{\mu}\left(X_i\right) \right)^{2 } \;\middle|\;  \mu \right]
\end{align*}

Therefore, the recommended strategy $\hat{\mu}(Y_i) =\hat{\mu}( X_i)$ has a smaller maximum risk than other strategies. 

\end{proof}




1. The payment from the buyer a constant $v(n^{\star})$.


2. If the payment for every seller is a fixed constant, then sellers can fabricate data without actually collecting data.\\


%$p_1 = b/2 +(\hat{\mu}(Y_1)- \hat{\mu}(Y_2))^2$, $p_2 = b/2 -(\hat{\mu}(Y_1)- \hat{\mu}(Y_2))^2$, seller 1 can choose ${\mu}' = u + \epsilon$, expected payment for seller 1 is larger than $b/2$. %NIC for seller 1: $g({\mu}',\mu ) < b/2$ for all ${\mu}' \neq \mu$. NIC for seller 2: $g( \mu, {\mu}') >  b/2 $ for all ${\mu}' \neq \mu$.

To demonstrate that no truthful mechanism (NIC) satisfies all desired properties in a two-seller setting, we use proof by contradiction.  

Suppose that there is a NIC mechanism $M$ satisfying property 1-5. Under this mechanism, the best strategy for each seller is to collect $N_i^{\star}$ amount of data and submit truthfully, where $N_1^{\star}+N_2^{\star} = n^{\star} $. Since $M$ is NIC for strategy space $\left\{ (f_i,N_i)\right\}_{i=1,2}$, it must be NIC for the sub strategy space $\left\{ (f_i, N_i^{\star})\right\}_{i=1,2}$. 


Consider the case in which everyone collects $N_i^{\star}$ data point and submits $N_i^{\star}$ data point. Assume that the true mean is $\mu$, seller $1$ submit $N({\mu}', \sigma^2),\ {\mu}' = f(\mu) $ while seller 2 submit $N({\mu}, \sigma^2) $. We denote seller 1's expected payment as  $\mathbb{E}\left[ p_1(M,\strategy) \right] = g({\mu}', \mu)$.
Seller 1's utility is then:
\[ u_1(M,f) = \underset{\mu}{\inf} \ g({\mu}', \mu) -c\times N_1^{\star}\] where $c$ is the cost for collecting one data point.


The total payment from the buyer is $v(n^{\star})$, hence by budget balance, \[p_2 (M,f)  = v(n^{\star}) -  p_1(M,f), \ \mathbb{E}\left[ p_2(M,f) \right] = v(n^{\star}) - g({\mu}', \mu) \]

By NIC, we have,
\[ \underset{\mu}{\inf} \ g({\mu}', \mu) -c\times N_1^{\star} \leq \underset{\mu}{\inf} \ g({\mu}, \mu) -c\times N_1^{\star} \] \[ \underset{\mu}{\inf} \ (v(n^{\star})- g( \mu, {\mu}')) -c\times N_2^{\star} \leq \underset{\mu}{\inf} \ (v(n^{\star})-g( \mu, {\mu})) -c\times N_2^{\star}  \]

%Using the fact that $\underset{\mu}{\inf} \ g({\mu}, \mu) = \underset{\mu}{\sup} \ g({\mu}, \mu) = {v(n^{\star})}/2 $.
We obtain that for any ${\mu}'$ and $\mu$,
\[  \underset{\mu}{\inf} \ g({\mu}', \mu) \leq \underset{\mu}{\inf} \ g({\mu}, \mu)   \] \[  \underset{\mu}{\sup} \  g( \mu, {\mu})  \leq \underset{\mu}{\sup } \ g( \mu, {\mu}')  \]

We next show that the inequalities are strict. Assume, for contradiction there exists ${\mu}'$, for any $\mu$, $g({\mu}', 
\mu) \geq \underset{\mu}{\inf} \ g({\mu}, \mu)$. It then follows that $ \underset{\mu}{\inf} \ g({\mu}', \mu) \geq \underset{\mu}{\inf} \ g({\mu}, \mu)$. Under this assumption, seller 1 could fabricate data by submitting $N({\mu}', \sigma^2)$ without collecting any actual data. This contradicts with the fact that $(f_1 = I, N_1 = N_1^{\star})$ is the best strategy for seller 1. Hence, for any ${\mu}'$, there exists some $\mu$ such that $g({\mu}', 
\mu) < \underset{\mu}{\inf} \ g({\mu}, \mu)$. Therefore, for any ${\mu}'$, \[ \underset{\mu}{\inf} \ g({\mu}', \mu) < \underset{\mu}{\inf} \ g({\mu}, \mu). \]Similarly, we also have \[ \underset{\mu}{\sup} \  g( \mu, {\mu})  < \underset{\mu}{\sup } \ g( \mu, {\mu}').  \] 


For any $ {\mu}'$, let $f({\mu}') =  \underset{\mu}{\arg\sup}\, g(\mu, {\mu}')$, then we have for any ${\mu}'$,  $f({\mu}') \neq {\mu}'$ and $ g(f({\mu}'), {\mu}') > \underset{\mu}{\sup} \  g( \mu, {\mu}) \geq  \underset{\mu}{\inf} \  g( \mu, {\mu})$. This implies that seller 1 could fabricate data based on function $f$, this contradicts with the fact that the mechanism is NIC.


pay the seller $v(n^{\star})/2 - \beta (\hat{\mu}(Y_1)-\hat{\mu}(Y_2))^2$, charge buyer $v(n^{\star}) - 2\beta (\hat{\mu}(Y_1)-\hat{\mu}(Y_2))^2$


Buyer utility: $v(n^*)$-payment
Seller utility: payment - $cn^*$.

Sellers utility is positive?

Seller payment $(v(n^*)/2)-\beta (\hat{\mu}(Y_1)-\hat{\mu}(Y_2))^2 $, $\beta = (v(n^*)-cn^*)c(n^*)^2 / 4\sigma^2$, 



\section{Multiple buyers}
\subsection{}
Question 1: Do we fix the amount of data for sale ahead of time?


Assume we fix $N$, the amount of data for sale. The goal of mechanism is to maximize the sellers' revenue. According to previous paper, there exists at least one type who purchase at the amount $N$. Suppose that in offline setting, i.e., when the mechanism knows the buyer valuation and type distribution, the optimal revenue is $\text{OPT} $. 


We ask $d$ sellers to collect $N$ data points, and split $\text{OPT} $ revenue among sellers. (data can be duplicated). 

\[ p_i(M,s) =\mathbb{I}\left( \left| Y_i \right| = \frac{N}{d} \right) \rbr{\frac{\text{OPT}}{d}+d_i \frac{\sigma^2}{N_{-i}^{\star}} +d_i \frac{\sigma^2}{N_i^{\star}} }- d_i \rbr{\hat{\mu}(Y_i)-\hat{\mu}(Y_{-i}) }^2  \]

Buyer's expected utility is non negative. Next, we discuss sellers' expected utility $T\mathbb{E}[p_i]- cn_i$ (over $T$ roundsm\, maybe $T$ is fixed). Let $N_i^{\star} = \frac{N}{d}$.

\[ u_i(M,s) = \mathbb{I}\left( \left| Y_i \right| = \frac{N}{d} \right) \rbr{\frac{\text{OPT}}{d}+d_i \frac{\sigma^2}{N_{-i}^{\star}} +d_i \frac{\sigma^2}{N_i^{\star}} }T- Td_i \mathbb{E}\rbr{\hat{\mu}(Y_i)-\hat{\mu}(Y_{-i}) }^2 
 -  cn_i \]
Choose $d_i = \frac{c(N_i^{\star})^2}{T\sigma^2}$.


If we do not fix \( T \) in advance, let \( T_0 \) represent the time at which the cumulative utility over at least \( T_0 \) rounds is non-negative. We can select \( d_i \) such that \( d_i \geq \frac{c(N/d)^2}{T_0 \sigma^2} \). This ensures that the seller will never choose to collect less than \( N/d \) amount of data.








\section{Single buyer} \label{section: singlebuyer}


Each contributor \( i \) incurs a cost \( c_i \) to collect data,  without loss of generality, we assume \( c_1 \leq c_2 \leq \dots \leq c_d \). The broker is assumed to have full knowledge of the buyer's valuation curve \( \val(n) \), as well as the contributors’ costs $ c_{ i \in \contributors}$  for collecting each data point.

The maximum total profit for the contributors, assuming no constraints on truthful submissions, is given by:
\[
\mathrm{profit}^\star = \underset{\datanum_1, \dots, \datanum_d}{\max} \left( \val \left(\sum_{i=1}^{d} \datanum_i\right) - \sum_{i=1}^{d} c_i \datanum_i \right),
\]

where \( \datanum_i \) represents the number of data points collected by contributor \( i \). In this unconstrained scenario, since contributor 1 has the lowest collection cost, the optimal strategy is for contributor 1 to collect all the required data points while other contributors collect none. This approach maximizes total profit without considering the incentive for truthful submissions.

However, when truthful submission is taken into account, at least two contributors are needed because we need to use one contributor's data to verify the other's. We demonstrate that the maximum profit achievable under Nash Equilibrium is:
\[
\mathrm{profit}^\star + (c_1 - c_2),
\]
where \( c_1 - c_2 \) represents the additional cost differential caused by enforcing truthful behavior among contributors. 

\begin{algorithm}[H]
    \caption{Process of mechanism.}
    \begin{algorithmic}
        \STATE {\bfseries Input:} A population of buyers $\buyers$.
        \STATE The broker chooses the optimal data allocation to maximize contributors' profit:
        $$
        \{ \datanum_i^{\star} \}_{i=1}^d = \underset{\datanum_1,\dots,\datanum_d}{\arg\max}\  \rbr{v\rbr{\sum_i \datanum_i}-\sum_i \cost_i \datanum_i  }
      $$
       
        \STATE The broker recommend a strategy to each contributor: $\strategy_i^{\star} = (\datanum_i^{\star}, \mathbf{I})$.
        \STATE Each contributor selects a strategy $\strati = (\datanum_i, f_i)$, collects $\datanum_i$ data points $X_i$, and submits $Y_i = f_i(X_i)$.
        \STATE The mechanism generates an estimator $\hat{\mu}(M,\strategy)$ for the buyer, and charge her $\price_{j \in \buyers}$. \COMMENT{See (\ref{eq:buyer_pay}) }
        \STATE Each contributor is paid $\payi$.    \COMMENT{See (\ref{eq:seller_pay}) }
    \end{algorithmic}   
\end{algorithm}




\begin{theorem}
    there exists NIC mechanism satisfying the following properties (1) $\strategy^{\star}$ is Nash equilibrium. (2) The mechanism is individually rational at $\strategy^{\star}$ for both buyers and sellers. (3) Budget balance. (4) Under strategy $\strategy^{\star}$, the expected profit of buyers approximates the optimal profit $ \mathrm{profit}^{\star}$ within an additive error $\cost_2 - \cost_1$. 
\end{theorem}


Let $n^{\star}$ denote the optimal total number of data to be collected, $\datanum_1^{\star}=n^{\star}-1$, and $\datanum_2^{\star}=1$. Let $w=\val(n^{\star})-cn^{\star}$ denote the social welfare. One option for payment function is

\begin{align*}
    &\; \pay_i(M,\strategy^{\star}) \\  
    = & \;\mathbb{I}\left( \left| Y_i \right| = \datanum_i^{\star} \right) \rbr{\frac{\datanum_1^{\star}}{n^{\star}}\val(n^{\star})+d_i \frac{\sigma^2}{\datanum_{-i}^{\star}} +d_i \frac{\sigma^2}{\datanum_i^{\star}} } \\ & - d_i \rbr{\hat{\mu}(Y_i)-\hat{\mu}(Y_{-i}) }^2, \\[20pt] % Adds vertical space between equations
    &\; \price(M,\strategy^{\star}) \\  
    = & \; \sum_{i=1}^{2}\mathbb{I}\left(  \left| Y_i \right| = \datanum_i^{\star} \right) \rbr{\frac{\datanum_1^{\star}}{n^{\star}}\val(n^{\star}) +d_i \frac{\sigma^2}{\datanum_{-i}^{\star}} +d_i \frac{\sigma^2}{\datanum_i^{\star}} } \\ 
    & - \sum_{i=1}^{2} d_i \rbr{\hat{\mu}(Y_i)-\hat{\mu}(Y_{-i}) }^2, \\[20pt] % Adds vertical space between equations
    &\;  \utilityb (M,\strategy^{\star}) \\ 
   = & \; v(\datanum^{\star}) -\mathbb{E}[\price(M,\strategy^{\star})] \\ 
    = & \; - \sum_{i=1}^{d} \rbr{d_i \frac{\sigma^2}{\datanum_{-i}^{\star}} +d_i \frac{\sigma^2}{\datanum_i^{\star}}  } + \sum_{i=1}^{d}d_i \mathbb{E} \rbr{\hat{\mu}(Y_i)-\hat{\mu}(Y_{-i}) }^2 \\ 
    = & \; 0.
\end{align*}



Then contributors i's expected ptofit under strategy $\strategy^{\star}$ is 

\begin{align*}
& \; \utilci \rbr{\mechspace, \strategy^{\star} } \\ = &  \; \mathbb{E}\sbr{\pay_i(M,\strategy^{\star})} - \cost_i n_i^{\star} \\ = &  \; \mathbb{I}\left(\left| Y_i \right| = \datanum_i^{\star} \right) \rbr{\frac{\datanum_1^{\star}}{n^{\star}}\val(n^{\star})  +d_i \frac{\sigma^2}{\datanum_{-i}^{\star}} +d_i \frac{\sigma^2}{\datanum_i^{\star}} }\\  & -  d_i \mathbb{E}\rbr{\hat{\mu}(Y_i)-\hat{\mu}(Y_{-i}) }^2  -\cost n_i^{\star} \\ = &  \;  \frac{w}{\numcontributors}
\end{align*}
where $d_i = c(\datanum_i^*/d)^2 $, 

%\textcolor{red}{Buyer payment $\pi(M,s)$ can ve negative? Can it be interpreted as when the quality of data is bad, the mechanism pays money to the buyer as compensate, the contributor pays money to the mechanism as a penalty. }\textcolor{red}{Buyer pays $v(n^*)$, $p_i = v(n^*) \frac{\rbr{\hat{\mu}(Y_i)-\hat{\mu}(Y_{-i}) }^{-2}}{\sum{\rbr{\hat{\mu}(Y_i)-\hat{\mu}(Y_{-i}) }^{-2}}}$ }


We prove the NIC in three steps.


\textbf{First step} \textcolor{red}{to be fixed}: Giving others submitting truthfully, we know that when fixing $n_i$, submitting $\left| Y_i \right| = \frac{n^{\star}}{d}$ is the best strategy, otherwise, $\pay_i <0$ when $\left| Y_i \right| \neq \frac{n^{\star}}{d}$. Therefore, for any $n_i$ and $f_i$, we have for any $\mu$,
\begin{align*}
& u_i\rbr{\mechspace, (n_i,f_i, \left| Y_i \right| =\datanum_i^{\star}),\strategy_{-i}^{\star} } \\  \geq \  & u_i\rbr{\mechspace, \strategy_{-i}^{\star}} 
\end{align*}

\textbf{Second step}: Fixing $n_i$ and $\left| Y_i \right|$, sample mean $\hat{\mu}(X_i)$ is minimax estimator of $\mathbb{E}\rbr{\rbr{\hat{\mu}(Y_i)-\hat{\mu}(Y_{-i})}^2 \;\middle|\; P} $, i.e., \[ \hat{\mu}(X_i) = \underset{\hat{\mu}}{\inf} \  \underset{\distrifamily}{\sup}\ \mathbb{E}\sbr{\rbr{\hat{\mu}(Y_i)-\hat{\mu}(Y_{-i})}^2  \;\middle|\; \distri} \]Therefore we have 
\begin{align*}
     & \underset{\distrifamily}{\inf}\;u_i\rbr{\mechspace, (n_i,\hat{\mu}(Y_i)=\hat{\mu}(X_i), \left| Y_i \right|),\strategy_{-i}^{\star} } \\  \geq \  & \underset{\distrifamily}{\inf}\;u_i\rbr{\mechspace, (n_i,\hat{\mu}(Y_i), \left| Y_i \right|),\strategy_{-i}^{\star} } 
\end{align*}


\textbf{Third step}: By setting constant $d_i= c\rbr{\frac{\datanum_i^{\star}}{\sigma}}^2$, when fixing $\hat{\mu}(Y_i)=\hat{\mu}(X_i)$ and $\left| Y_i \right| = \datanum_i^{\star}$, collecting $n_i = \datanum_i^{\star}$ amount of data maximize the contributor utility 
\begin{align*}
    \underset{\distrifamily}{\inf}\;u_i\rbr{\mechspace, (n_i= \datanum_i^{\star},\hat{\mu}(Y_i)=\hat{\mu}(X_i), \left| Y_i \right|=\strategy_{-i}^{\star} } \\ \geq \underset{\distrifamily}{\inf}\;u_i\rbr{\mechspace, (n_i,\hat{\mu}(Y_i)=\hat{\mu}(X_i), \left| Y_i \right|=s_{-i}^{\star}}  
\end{align*}
 

\begin{align*}
    & u_i\rbr{\mechspace, (n_i,\hat{\mu}(Y_i)=\hat{\mu}(X_i), \left| Y_i \right|=\datanum_i^{\star}),\strategy_{-i}^{\star} }  \\ = & \rbr{\frac{w}{\numcontributors}+ c \datanum_i^{\star} +d_i \frac{\sigma^2}{\datanum_{-i}^{\star}} +d_i \frac{\sigma^2}{\datanum_i^{\star}} } - d_i \rbr{ \frac{\sigma^2}{\datanum_{-i}^{\star}} +\frac{\sigma^2}{\datanum_i^{\star}} } - cn_i
\end{align*}

Therefore, we have 
\begin{align*}
    &\underset{\distrifamily}{\inf}\;u_i \rbr{\mechspace, (n_i= \datanum_i^{\star},\hat{\mu}(Y_i)=\hat{\mu}(X_i), \left| Y_i \right|=\datanum_i^{\star}),\strategy_{-i}^{\star} } \\  = & \underset{\distrifamily}{\inf}\;u_i\rbr{\mechspace, (\datanum_i^{\star},f_i^{\star}),\strategy_{-i}^{\star} } \\ \geq &   \underset{\distrifamily}{\inf}\; u_i\rbr{\mechspace, (n_i,f_i ),\strategy_{-i}^{\star}} 
\end{align*}

When following the best strategy, properties 1-5 are all satisfied.



% \caption{
%     % 我们方法上非英语相比英语性能落后的错误原因,及比例
%     The error types and the proportion of non-English performance in \ourmethod are lagging behind that of English.
% }
% \label{tab:error}
% \end{table*}

\begin{figure}[t]
    \centering
    
\begin{lemma}\label{Lemma:multi1} 
   Fixing the number of data contributor $i$ collects $n_i$, and others' strategies $\strategy_{-i}$, $\hat{\mu}\left(X_i\right)$ is the minimax estimator for the Normal distribution class $\Normaldistrib := \left\{\mathcal{N}(\mu,\sigma^2) \;\middle|\; \mu \in \mathbb{R}\right\}$,
    \begin{align*}
       \hat{\mu}(X_i)  = \underset{\hat{\mu}}{\arg\min} \sbr{\sup _\mu \mathbb{E}\left[(\hat{\mu}( Y_i)- \hat{\mu}( Y_{-i}) )^2 \;\middle|\;  \mu \right] }
    \end{align*} 
     
\end{lemma}


\begin{proof}

\begin{align*}
    & \ \mathbb{E}\left[ \left( \hat{\mu}\left( Y_i \right)-\hat{\mu}\left( Y_{-i} \right)  \right)^2 \right] \\ =  & \ \mathbb{E}\left[ \left( (\hat{\mu}\left(  Y_i \right)-\mu) -(\hat{\mu}\left(  Y_{-i} \right) -\mu) \right)^2   \right] \\ =  & \ A_0 + \mathbb{E}\left[ (\hat{\mu}\left(  Y_i \right)-\mu)^2  \right]
\end{align*}
where $A_0$ is a positive coefficient.

Thus the maximum risk can be written as:

\begin{align*}
    \sup _\mu \mathbb{E}\left[A_0 + \left(\hat{\mu}\left( Y_i\right)-\mu\right)^{2} \;\middle|\;  \mu \right]
\end{align*}


We construct a lower bound on the maximum risk using a sequence of Bayesian risks. Let $\Lambda_{\ell}:=\mathcal{N}\left(0, \ell^2\right), \ell=1,2, \ldots$ be a sequence of prior for $\mu$. For fixed $\ell$, the posterior distribution is:
$$
\begin{aligned}
p\left(\mu \;\middle|\;  X_i\right) & \propto p\left(X_i \;\middle|\;  \mu\right) p(\mu) \\ & \propto \exp \left(-\frac{1}{2 \sigma^2} \sum_{x \in X_i}(x-\mu)^2\right) \exp \left(-\frac{1}{2 \ell^2} \mu^2\right) \\
& \propto \exp \left(-\frac{1}{2}\left(\frac{n_i}{\sigma^2}+\frac{1}{\ell^2}\right) \mu^2+\frac{1}{2} 2 \frac{\sum_{x \in X_i} x}{\sigma^2} \mu\right) .
\end{aligned}
$$

This means the posterior of $\mu$ given $X_i$ is Gaussian with:

\begin{align*}
    \mu \lvert\, X_i & \sim \mathcal{N}\left(\frac{n_i \hat{\mu}\left(X_i\right) / \sigma^2}{n_i / \sigma^2+1 / \ell^2}, \frac{1}{n_i / \sigma^2+1 / \ell^2}\right) 
    \\ & =: \mathcal{N}\left(\mu_{\ell}, \sigma_{\ell}^2\right).
\end{align*}



Therefore, the posterior risk is: 
$$
\begin{aligned}
&   \mathbb{E}\left[A_0 + \left(\hat{\mu}\left( Y_i\right)-\mu\right)^{2}  \;\middle|\;  X_i\right] \\ = &  \mathbb{E}\left[A_0 +  \left(\left(\hat{\mu}\left( Y_i\right)-\mu_{\ell}\right)-\left(\mu-\mu_{\ell}\right)\right)^{2 j} \;\middle|\;  X_i\right] \\ =
& A_0+\int_{-\infty}^{\infty} \underbrace{\left(e-\left(\hat{\mu}\left( Y_i\right)-\mu_{\ell}\right)\right)^2}_{=: F_1\left(e-\left(\hat{\mu}\left(Y_i\right)-\mu_{\ell}\right)\right)} \underbrace{\frac{1}{\sigma_{\ell} \sqrt{2 \pi}} \exp \left(-\frac{e^2}{2 \sigma_{\ell}^2}\right)}_{=: F_2(e)} d e
\end{aligned}
$$

Because:
\begin{itemize}
    \item $F_1(\cdot)$ is even function and increases on $[0, \infty)$;
    \item $F_2(\cdot)$ is even function and decreases on $\left[0, \infty \right)$, and $\int_{\mathbb{R}} F_2(e) de<\infty$
    \item For any $a \in \mathbb{R}, \int_{\mathbb{R}} F_1(e-a) F_2(e) de<\infty$
\end{itemize}

By the corollary of Hardy-Littlewood inequality in Lemma \ref{lemmaHardy},
$$
\int_{\mathbb{R}} F_1(e-a) F_2(e) d e \geq \int_{\mathbb{R}} F_1(e) F_2(e) d e
$$
which means the posterior risk is minimized when $\hat{\mu}\left(Y_i\right)=\mu_{\ell}$. We then write the Bayes risk as, the Bayes risk is minimized by the posterior mean $\mu_{\ell}$:

\begin{align*}    
R_{\ell}:= & \mathbb{E}\left[ A_0+\mathbb{E}\left[\left(\mu-\mu_{\ell}\right)^{2 } \;\middle|\;  X_i\right]\right] \\ = & A_0 + \sigma_{\ell}^{2}
\end{align*}

and the limit of Bayesian risk as $\ell \rightarrow \infty$ is
$$
R_{\infty}:= A_0 + \frac{\sigma^{2}}{n_i}.
$$

When $\hat{\mu}\left(Y_i\right)=\hat{\mu}\left(X_i\right)$, i.e, the contributor submit a set of size $n_i$ with each element equal to $ \hat{\mu}\left(X_i\right)$, the maximum risk is:

\begin{align*}
& \sup _\mu \mathbb{E}\left[A_0+\left(\mu- 
\hat{\mu}\left(Y_i\right) \right)^{2 } \;\middle|\;  \mu \right] \\
= & \sup _\mu \mathbb{E}\left[A_0+\left(\mu- 
\hat{\mu}\left(X_i\right) \right)^{2 } \;\middle|\;  \mu \right]  \\
= & A_0+ \sigma^{2 } n_i^{-1}  \\
= &  R_{\infty}.
\end{align*}

This implies that,
\begin{align*}
    & \underset{\mu}{\sup}\; \mathbb{E} \sbr{ \rbr{\hat{\mu}\left( Y_i \right)-\hat{\mu}\left( Y_{-i} \right)  }^2 \;\middle|\;  \mu }  \\ \geq & \; R_{\infty} =  \sup _\mu \; \mathbb{E}\left[A_0+\left(\mu- 
\hat{\mu}\left(X_i\right) \right)^{2 } \;\middle|\;  \mu \right]
\end{align*}

Therefore, the recommended strategy $\hat{\mu}(Y_i) =\hat{\mu}( X_i)$ has a smaller maximum risk than other strategies. 

\end{proof}




1. The payment from the buyer a constant $v(n^{\star})$.


2. If the payment for every seller is a fixed constant, then sellers can fabricate data without actually collecting data.\\


%$p_1 = b/2 +(\hat{\mu}(Y_1)- \hat{\mu}(Y_2))^2$, $p_2 = b/2 -(\hat{\mu}(Y_1)- \hat{\mu}(Y_2))^2$, seller 1 can choose ${\mu}' = u + \epsilon$, expected payment for seller 1 is larger than $b/2$. %NIC for seller 1: $g({\mu}',\mu ) < b/2$ for all ${\mu}' \neq \mu$. NIC for seller 2: $g( \mu, {\mu}') >  b/2 $ for all ${\mu}' \neq \mu$.

To demonstrate that no truthful mechanism (NIC) satisfies all desired properties in a two-seller setting, we use proof by contradiction.  

Suppose that there is a NIC mechanism $M$ satisfying property 1-5. Under this mechanism, the best strategy for each seller is to collect $N_i^{\star}$ amount of data and submit truthfully, where $N_1^{\star}+N_2^{\star} = n^{\star} $. Since $M$ is NIC for strategy space $\left\{ (f_i,N_i)\right\}_{i=1,2}$, it must be NIC for the sub strategy space $\left\{ (f_i, N_i^{\star})\right\}_{i=1,2}$. 


Consider the case in which everyone collects $N_i^{\star}$ data point and submits $N_i^{\star}$ data point. Assume that the true mean is $\mu$, seller $1$ submit $N({\mu}', \sigma^2),\ {\mu}' = f(\mu) $ while seller 2 submit $N({\mu}, \sigma^2) $. We denote seller 1's expected payment as  $\mathbb{E}\left[ p_1(M,\strategy) \right] = g({\mu}', \mu)$.
Seller 1's utility is then:
\[ u_1(M,f) = \underset{\mu}{\inf} \ g({\mu}', \mu) -c\times N_1^{\star}\] where $c$ is the cost for collecting one data point.


The total payment from the buyer is $v(n^{\star})$, hence by budget balance, \[p_2 (M,f)  = v(n^{\star}) -  p_1(M,f), \ \mathbb{E}\left[ p_2(M,f) \right] = v(n^{\star}) - g({\mu}', \mu) \]

By NIC, we have,
\[ \underset{\mu}{\inf} \ g({\mu}', \mu) -c\times N_1^{\star} \leq \underset{\mu}{\inf} \ g({\mu}, \mu) -c\times N_1^{\star} \] \[ \underset{\mu}{\inf} \ (v(n^{\star})- g( \mu, {\mu}')) -c\times N_2^{\star} \leq \underset{\mu}{\inf} \ (v(n^{\star})-g( \mu, {\mu})) -c\times N_2^{\star}  \]

%Using the fact that $\underset{\mu}{\inf} \ g({\mu}, \mu) = \underset{\mu}{\sup} \ g({\mu}, \mu) = {v(n^{\star})}/2 $.
We obtain that for any ${\mu}'$ and $\mu$,
\[  \underset{\mu}{\inf} \ g({\mu}', \mu) \leq \underset{\mu}{\inf} \ g({\mu}, \mu)   \] \[  \underset{\mu}{\sup} \  g( \mu, {\mu})  \leq \underset{\mu}{\sup } \ g( \mu, {\mu}')  \]

We next show that the inequalities are strict. Assume, for contradiction there exists ${\mu}'$, for any $\mu$, $g({\mu}', 
\mu) \geq \underset{\mu}{\inf} \ g({\mu}, \mu)$. It then follows that $ \underset{\mu}{\inf} \ g({\mu}', \mu) \geq \underset{\mu}{\inf} \ g({\mu}, \mu)$. Under this assumption, seller 1 could fabricate data by submitting $N({\mu}', \sigma^2)$ without collecting any actual data. This contradicts with the fact that $(f_1 = I, N_1 = N_1^{\star})$ is the best strategy for seller 1. Hence, for any ${\mu}'$, there exists some $\mu$ such that $g({\mu}', 
\mu) < \underset{\mu}{\inf} \ g({\mu}, \mu)$. Therefore, for any ${\mu}'$, \[ \underset{\mu}{\inf} \ g({\mu}', \mu) < \underset{\mu}{\inf} \ g({\mu}, \mu). \]Similarly, we also have \[ \underset{\mu}{\sup} \  g( \mu, {\mu})  < \underset{\mu}{\sup } \ g( \mu, {\mu}').  \] 


For any $ {\mu}'$, let $f({\mu}') =  \underset{\mu}{\arg\sup}\, g(\mu, {\mu}')$, then we have for any ${\mu}'$,  $f({\mu}') \neq {\mu}'$ and $ g(f({\mu}'), {\mu}') > \underset{\mu}{\sup} \  g( \mu, {\mu}) \geq  \underset{\mu}{\inf} \  g( \mu, {\mu})$. This implies that seller 1 could fabricate data based on function $f$, this contradicts with the fact that the mechanism is NIC.


pay the seller $v(n^{\star})/2 - \beta (\hat{\mu}(Y_1)-\hat{\mu}(Y_2))^2$, charge buyer $v(n^{\star}) - 2\beta (\hat{\mu}(Y_1)-\hat{\mu}(Y_2))^2$


Buyer utility: $v(n^*)$-payment
Seller utility: payment - $cn^*$.

Sellers utility is positive?

Seller payment $(v(n^*)/2)-\beta (\hat{\mu}(Y_1)-\hat{\mu}(Y_2))^2 $, $\beta = (v(n^*)-cn^*)c(n^*)^2 / 4\sigma^2$, 



\section{Multiple buyers}
\subsection{}
Question 1: Do we fix the amount of data for sale ahead of time?


Assume we fix $N$, the amount of data for sale. The goal of mechanism is to maximize the sellers' revenue. According to previous paper, there exists at least one type who purchase at the amount $N$. Suppose that in offline setting, i.e., when the mechanism knows the buyer valuation and type distribution, the optimal revenue is $\text{OPT} $. 


We ask $d$ sellers to collect $N$ data points, and split $\text{OPT} $ revenue among sellers. (data can be duplicated). 

\[ p_i(M,s) =\mathbb{I}\left( \left| Y_i \right| = \frac{N}{d} \right) \rbr{\frac{\text{OPT}}{d}+d_i \frac{\sigma^2}{N_{-i}^{\star}} +d_i \frac{\sigma^2}{N_i^{\star}} }- d_i \rbr{\hat{\mu}(Y_i)-\hat{\mu}(Y_{-i}) }^2  \]

Buyer's expected utility is non negative. Next, we discuss sellers' expected utility $T\mathbb{E}[p_i]- cn_i$ (over $T$ roundsm\, maybe $T$ is fixed). Let $N_i^{\star} = \frac{N}{d}$.

\[ u_i(M,s) = \mathbb{I}\left( \left| Y_i \right| = \frac{N}{d} \right) \rbr{\frac{\text{OPT}}{d}+d_i \frac{\sigma^2}{N_{-i}^{\star}} +d_i \frac{\sigma^2}{N_i^{\star}} }T- Td_i \mathbb{E}\rbr{\hat{\mu}(Y_i)-\hat{\mu}(Y_{-i}) }^2 
 -  cn_i \]
Choose $d_i = \frac{c(N_i^{\star})^2}{T\sigma^2}$.


If we do not fix \( T \) in advance, let \( T_0 \) represent the time at which the cumulative utility over at least \( T_0 \) rounds is non-negative. We can select \( d_i \) such that \( d_i \geq \frac{c(N/d)^2}{T_0 \sigma^2} \). This ensures that the seller will never choose to collect less than \( N/d \) amount of data.








\section{Single buyer} \label{section: singlebuyer}


Each contributor \( i \) incurs a cost \( c_i \) to collect data,  without loss of generality, we assume \( c_1 \leq c_2 \leq \dots \leq c_d \). The broker is assumed to have full knowledge of the buyer's valuation curve \( \val(n) \), as well as the contributors’ costs $ c_{ i \in \contributors}$  for collecting each data point.

The maximum total profit for the contributors, assuming no constraints on truthful submissions, is given by:
\[
\mathrm{profit}^\star = \underset{\datanum_1, \dots, \datanum_d}{\max} \left( \val \left(\sum_{i=1}^{d} \datanum_i\right) - \sum_{i=1}^{d} c_i \datanum_i \right),
\]

where \( \datanum_i \) represents the number of data points collected by contributor \( i \). In this unconstrained scenario, since contributor 1 has the lowest collection cost, the optimal strategy is for contributor 1 to collect all the required data points while other contributors collect none. This approach maximizes total profit without considering the incentive for truthful submissions.

However, when truthful submission is taken into account, at least two contributors are needed because we need to use one contributor's data to verify the other's. We demonstrate that the maximum profit achievable under Nash Equilibrium is:
\[
\mathrm{profit}^\star + (c_1 - c_2),
\]
where \( c_1 - c_2 \) represents the additional cost differential caused by enforcing truthful behavior among contributors. 

\begin{algorithm}[H]
    \caption{Process of mechanism.}
    \begin{algorithmic}
        \STATE {\bfseries Input:} A population of buyers $\buyers$.
        \STATE The broker chooses the optimal data allocation to maximize contributors' profit:
        $$
        \{ \datanum_i^{\star} \}_{i=1}^d = \underset{\datanum_1,\dots,\datanum_d}{\arg\max}\  \rbr{v\rbr{\sum_i \datanum_i}-\sum_i \cost_i \datanum_i  }
      $$
       
        \STATE The broker recommend a strategy to each contributor: $\strategy_i^{\star} = (\datanum_i^{\star}, \mathbf{I})$.
        \STATE Each contributor selects a strategy $\strati = (\datanum_i, f_i)$, collects $\datanum_i$ data points $X_i$, and submits $Y_i = f_i(X_i)$.
        \STATE The mechanism generates an estimator $\hat{\mu}(M,\strategy)$ for the buyer, and charge her $\price_{j \in \buyers}$. \COMMENT{See (\ref{eq:buyer_pay}) }
        \STATE Each contributor is paid $\payi$.    \COMMENT{See (\ref{eq:seller_pay}) }
    \end{algorithmic}   
\end{algorithm}




\begin{theorem}
    there exists NIC mechanism satisfying the following properties (1) $\strategy^{\star}$ is Nash equilibrium. (2) The mechanism is individually rational at $\strategy^{\star}$ for both buyers and sellers. (3) Budget balance. (4) Under strategy $\strategy^{\star}$, the expected profit of buyers approximates the optimal profit $ \mathrm{profit}^{\star}$ within an additive error $\cost_2 - \cost_1$. 
\end{theorem}


Let $n^{\star}$ denote the optimal total number of data to be collected, $\datanum_1^{\star}=n^{\star}-1$, and $\datanum_2^{\star}=1$. Let $w=\val(n^{\star})-cn^{\star}$ denote the social welfare. One option for payment function is

\begin{align*}
    &\; \pay_i(M,\strategy^{\star}) \\  
    = & \;\mathbb{I}\left( \left| Y_i \right| = \datanum_i^{\star} \right) \rbr{\frac{\datanum_1^{\star}}{n^{\star}}\val(n^{\star})+d_i \frac{\sigma^2}{\datanum_{-i}^{\star}} +d_i \frac{\sigma^2}{\datanum_i^{\star}} } \\ & - d_i \rbr{\hat{\mu}(Y_i)-\hat{\mu}(Y_{-i}) }^2, \\[20pt] % Adds vertical space between equations
    &\; \price(M,\strategy^{\star}) \\  
    = & \; \sum_{i=1}^{2}\mathbb{I}\left(  \left| Y_i \right| = \datanum_i^{\star} \right) \rbr{\frac{\datanum_1^{\star}}{n^{\star}}\val(n^{\star}) +d_i \frac{\sigma^2}{\datanum_{-i}^{\star}} +d_i \frac{\sigma^2}{\datanum_i^{\star}} } \\ 
    & - \sum_{i=1}^{2} d_i \rbr{\hat{\mu}(Y_i)-\hat{\mu}(Y_{-i}) }^2, \\[20pt] % Adds vertical space between equations
    &\;  \utilityb (M,\strategy^{\star}) \\ 
   = & \; v(\datanum^{\star}) -\mathbb{E}[\price(M,\strategy^{\star})] \\ 
    = & \; - \sum_{i=1}^{d} \rbr{d_i \frac{\sigma^2}{\datanum_{-i}^{\star}} +d_i \frac{\sigma^2}{\datanum_i^{\star}}  } + \sum_{i=1}^{d}d_i \mathbb{E} \rbr{\hat{\mu}(Y_i)-\hat{\mu}(Y_{-i}) }^2 \\ 
    = & \; 0.
\end{align*}



Then contributors i's expected ptofit under strategy $\strategy^{\star}$ is 

\begin{align*}
& \; \utilci \rbr{\mechspace, \strategy^{\star} } \\ = &  \; \mathbb{E}\sbr{\pay_i(M,\strategy^{\star})} - \cost_i n_i^{\star} \\ = &  \; \mathbb{I}\left(\left| Y_i \right| = \datanum_i^{\star} \right) \rbr{\frac{\datanum_1^{\star}}{n^{\star}}\val(n^{\star})  +d_i \frac{\sigma^2}{\datanum_{-i}^{\star}} +d_i \frac{\sigma^2}{\datanum_i^{\star}} }\\  & -  d_i \mathbb{E}\rbr{\hat{\mu}(Y_i)-\hat{\mu}(Y_{-i}) }^2  -\cost n_i^{\star} \\ = &  \;  \frac{w}{\numcontributors}
\end{align*}
where $d_i = c(\datanum_i^*/d)^2 $, 

%\textcolor{red}{Buyer payment $\pi(M,s)$ can ve negative? Can it be interpreted as when the quality of data is bad, the mechanism pays money to the buyer as compensate, the contributor pays money to the mechanism as a penalty. }\textcolor{red}{Buyer pays $v(n^*)$, $p_i = v(n^*) \frac{\rbr{\hat{\mu}(Y_i)-\hat{\mu}(Y_{-i}) }^{-2}}{\sum{\rbr{\hat{\mu}(Y_i)-\hat{\mu}(Y_{-i}) }^{-2}}}$ }


We prove the NIC in three steps.


\textbf{First step} \textcolor{red}{to be fixed}: Giving others submitting truthfully, we know that when fixing $n_i$, submitting $\left| Y_i \right| = \frac{n^{\star}}{d}$ is the best strategy, otherwise, $\pay_i <0$ when $\left| Y_i \right| \neq \frac{n^{\star}}{d}$. Therefore, for any $n_i$ and $f_i$, we have for any $\mu$,
\begin{align*}
& u_i\rbr{\mechspace, (n_i,f_i, \left| Y_i \right| =\datanum_i^{\star}),\strategy_{-i}^{\star} } \\  \geq \  & u_i\rbr{\mechspace, \strategy_{-i}^{\star}} 
\end{align*}

\textbf{Second step}: Fixing $n_i$ and $\left| Y_i \right|$, sample mean $\hat{\mu}(X_i)$ is minimax estimator of $\mathbb{E}\rbr{\rbr{\hat{\mu}(Y_i)-\hat{\mu}(Y_{-i})}^2 \;\middle|\; P} $, i.e., \[ \hat{\mu}(X_i) = \underset{\hat{\mu}}{\inf} \  \underset{\distrifamily}{\sup}\ \mathbb{E}\sbr{\rbr{\hat{\mu}(Y_i)-\hat{\mu}(Y_{-i})}^2  \;\middle|\; \distri} \]Therefore we have 
\begin{align*}
     & \underset{\distrifamily}{\inf}\;u_i\rbr{\mechspace, (n_i,\hat{\mu}(Y_i)=\hat{\mu}(X_i), \left| Y_i \right|),\strategy_{-i}^{\star} } \\  \geq \  & \underset{\distrifamily}{\inf}\;u_i\rbr{\mechspace, (n_i,\hat{\mu}(Y_i), \left| Y_i \right|),\strategy_{-i}^{\star} } 
\end{align*}


\textbf{Third step}: By setting constant $d_i= c\rbr{\frac{\datanum_i^{\star}}{\sigma}}^2$, when fixing $\hat{\mu}(Y_i)=\hat{\mu}(X_i)$ and $\left| Y_i \right| = \datanum_i^{\star}$, collecting $n_i = \datanum_i^{\star}$ amount of data maximize the contributor utility 
\begin{align*}
    \underset{\distrifamily}{\inf}\;u_i\rbr{\mechspace, (n_i= \datanum_i^{\star},\hat{\mu}(Y_i)=\hat{\mu}(X_i), \left| Y_i \right|=\strategy_{-i}^{\star} } \\ \geq \underset{\distrifamily}{\inf}\;u_i\rbr{\mechspace, (n_i,\hat{\mu}(Y_i)=\hat{\mu}(X_i), \left| Y_i \right|=s_{-i}^{\star}}  
\end{align*}
 

\begin{align*}
    & u_i\rbr{\mechspace, (n_i,\hat{\mu}(Y_i)=\hat{\mu}(X_i), \left| Y_i \right|=\datanum_i^{\star}),\strategy_{-i}^{\star} }  \\ = & \rbr{\frac{w}{\numcontributors}+ c \datanum_i^{\star} +d_i \frac{\sigma^2}{\datanum_{-i}^{\star}} +d_i \frac{\sigma^2}{\datanum_i^{\star}} } - d_i \rbr{ \frac{\sigma^2}{\datanum_{-i}^{\star}} +\frac{\sigma^2}{\datanum_i^{\star}} } - cn_i
\end{align*}

Therefore, we have 
\begin{align*}
    &\underset{\distrifamily}{\inf}\;u_i \rbr{\mechspace, (n_i= \datanum_i^{\star},\hat{\mu}(Y_i)=\hat{\mu}(X_i), \left| Y_i \right|=\datanum_i^{\star}),\strategy_{-i}^{\star} } \\  = & \underset{\distrifamily}{\inf}\;u_i\rbr{\mechspace, (\datanum_i^{\star},f_i^{\star}),\strategy_{-i}^{\star} } \\ \geq &   \underset{\distrifamily}{\inf}\; u_i\rbr{\mechspace, (n_i,f_i ),\strategy_{-i}^{\star}} 
\end{align*}

When following the best strategy, properties 1-5 are all satisfied.



    % \vspace{-0.5em}
    \caption{
    % 我们方法上非英语相比英语性能落后的错误原因,及比例
    The error types and their proportion of non-English performance in \ourmethod are inferior compared with English. 
    \textbf{Linking} refers to mapping entities in the question with incorrect information in the table or text. 
    \textbf{Formula} refers to using an incorrect formula. 
    \textbf{Redundancy} refers to outputting irrelevant information beyond the correct answer. 
    }
    \label{fig:error}
    \vspace{-1em}
\end{figure}

\subsection{Error Analysis}
\label{subsec:Error Analysis}
% 我们分析了我们方法在非英语语言上相比英语落后的原因,如图所示
We analyze the reasons for the inferior performance of \ourmethod on non-English languages compared to English, as shown in Figure~\ref{fig:error}. 
% 具体来说,我们选取Llama3.1-70B上我们的方法在英语上达到EM为1,而在非英语上EM为0的问题,每种语言随机sample 5个,共50个错误进行对比分析
Specifically, we select instances where \ourmethod achieved an EM of $1$ in English using Llama3.1-70B, but an EM of $0$ in non-English languages. 
For each language, we randomly sample five instances, with a total of $50$ errors for comparative analysis. 
% 我们在附录中展示了每种类型对应错误的示例
Examples of errors corresponding to each type are provided in Appendix~\ref{subsec:case study}. 
% 下面我们具体介绍每种错误
Below, we present a detailed discussion of each error type:

% 1. 链接:指模型将问题中的实体对应到了不想关的表格或文本中的信息,导致模型回答错误
(\emph{i})~\textbf{Linking}: 
% refers to associating entities in the question with incorrect information in tables or text. 
% 由于模型在非英语语言上的理解以及推理能力落后于英语,即使我们方法首先令模型专注于链接,模型仍然在链接上展现出了非常的挑战
Due to the relatively weaker abilities in non-English languages compared to English, even though \ourmethod initially prompts the model to focus on linking, the model still faces significant challenges in linking. 
% 尤其是在一些语言上,比如日语的平假名和片假名,或法语、德语等词法变化复杂的语言上,链接的挑战进一步加剧
These challenges are particularly pronounced in languages with complex orthographies, such as Japanese (with its hiragana and katakana scripts), or morphologically rich languages like French and German.
% 2. 公式:指模型在找到相关信息后,使用了错误的公式,导致代码返回了错误的结果
(\emph{ii})~\textbf{Formula} 
% pertains to situations where, after identifying relevant information, the model uses an incorrect formula, resulting in erroneous output. 
% 这也表明了模型在非英语语言上的数值推理能力相比英语仍存在差距
highlights the gap in the numerical reasoning abilities between non-English languages and English.
% 3. 多余信息:指模型没有按照指令的要求,输出了除正确答案之外的多余信息,导致EM为0
(\emph{iii})~\textbf{Redundancy}  
% refers to outputting irrelevant information beyond the correct answer, not as instructed, leading to an EM of $0$. 
% 这体现了模型相对较差的指令遵循能力
reflects the relatively weaker ability of instruction-following.

% 总之,模型在非英语语言上的能力落后,以及语言特殊的属性,令我们的方法在非英语语言上落后于英语,也证明了我们数据集的挑战性
In summary, the inferior performance on non-English languages and the specific properties of languages leads to the lower performance of \ourmethod on non-English languages, which also demonstrates the necessity of \ourdataset.
\subsection{Results Interpretation} \label{results}
%  \vspace{-1em}
The evaluation of the \model involved assessing the performance of various large language models (LLMs) across key tasks: Setup, Download, Training, Inference, and Evaluation. These tasks are essential for deploying repositories within the \model.





% \begin{table*}[h!]
% \centering
% \caption{Performance of Different Models Across Stages (Drafter Only)}
% \label{tab:performance_models_stages_drafter_only}
% \begin{tabular}{lcccccc}
% \toprule
% \textbf{Model} & \textbf{Environment Preparation} & \textbf{Data/Checkpoint} & \textbf{Training} & \textbf{Evaluation} & \textbf{Inference/Demo} \\
% \midrule
% LLama2         & X1 & Y1 & Z1 & W1 & V1 \\
% LLama3         & X2 & Y2 & Z2 & W2 & V2 \\
% Claude 3       & X3 & Y3 & Z3 & W3 & V3 \\
% \bottomrule
% \end{tabular}
% \end{table*}


% \begin{figure}[htbp]
%     \centerline{\includesvg[width=0.75\linewidth]{assets/GSRBench_Sunburst.svg}}
%     \caption{\model Topic Distribution}
%     \label{fig:sunburst}
% \end{figure}

% \begin{figure}[htbp]
%     \centerline{\includesvg[width=0.75\linewidth]{assets/GSRBench_Conf.svg}}
%     \caption{\model Topic Distribution}
%     \label{fig:sunburst}
% \end{figure}



% \begin{figure*}[htbp]
%     \centering
%     \includegraphics[width=0.7\linewidth]{assets/success_rate_histograms.pdf}
%     \caption{Success Ratio Histogram}
%     \label{fig:success_ratio}
% \end{figure*}




% \begin{figure*}[htbp]
%     \centering
%     \includegraphics[width=0.7\linewidth]{assets/agg/Claude_3_Sonnet.pdf}
%     \caption{Claude 3 Sonnet}
%     \label{fig:Claude_3_Sonnet}
% \end{figure*}

% \begin{figure*}[htbp]
%     \centering
%     \includegraphics[width=0.7\linewidth]{assets/agg/Llama_3_8b_Instruct.pdf}
%     \caption{Llama 3 8B Instruct}
%     \label{fig:Llama_3_8b_Instruct}
% \end{figure*}

% \begin{figure*}[htbp]
%     \centering
%     \includegraphics[width=0.7\linewidth]{assets/agg/Mistral_Large.pdf}
%     \caption{Mistral Large}
%     \label{fig:Mistral_Large}
% \end{figure*}




\subsubsection{Task-Specific Outcomes}
% % \vspace{-0.5em}
\parabf{Setup and Download:} Most models consistently performed well, reflecting their capability to initiate and manage basic deployment processes.



\parabf{Inference and Evaluation:} Performance was less consistent, with some models demonstrating moderate success, but generally struggling with the complexity of these tasks.

% % \vspace{-0.5em}
\parabf{Training:} Training tasks are particularly challenging, with lower success rates across the board, indicating that current LLMs require further refinement to handle training processes effectively.


% % \vspace{-1em}
\subsubsection{Overall Performance}





The success metrics across different tasks and models indicate a wide variability in performance. Generally, models showed higher success rates in Setup and Download tasks, with performance tapering off in more complex tasks such as Inference, Evaluation, and Training. This pattern highlights the challenges LLMs face in handling the full deployment process autonomously.

The results demonstrate that while LLMs have made significant strides in automating repository deployment, their ability to manage complex tasks remains limited. Improvements are needed, particularly in the areas of Inference and Training, to achieve fully autonomous and reliable deployment of science repositories.

However, there is still a large gap between LLMs and real scientists even if the advnanced tools are provided to the LLMs. To explain, it is not trivial to handle the nuances in the experiement environment setup for the science repositories. For example, the hardware and software compatibility issues are very common in code deployment and often causes confusions even for domain experts.
% % % \vspace{-1em}
% \begin{table*}[htbp]
%   \centering
%   \caption{Success Metrics Across Different Tasks}
%   \resizebox{\linewidth}{!}{%
%   \begin{tabular}{ll ccccc}
%   \toprule
%   \textbf{Model Type} & \textbf{Setup} & \textbf{Download} & \textbf{Inference} & \textbf{Evaluation} & \textbf{Training} \\
%   \midrule
%   Claude & 0.232--0.467 & 0.189--0.467 & 0.000--0.190 & 0.000--0.194 & 0.007--0.291 \\
%   Llama-2 & 0.152--0.840 & 0.182--1.000 & 0.000--0.467 & 0.000--0.700 & 0.000--1.000 \\
%   Llama-3 & 0.251--0.444 & 0.200--0.629 & 0.007--0.268 & 0.019--0.238 & 0.019--0.259 \\
%   Mistral & 0.170--0.427 & 0.178--0.495 & 0.029--0.291 & 0.024--0.317 & 0.040--0.444 \\
%   \bottomrule
%   \end{tabular}%
%   }
%   \label{tab:success_summary}
% \end{table*}


% % % \vspace{-1em}
% \begin{figure*}[htbp]
%     \centering
%     \begin{minipage}{0.49\linewidth}
%         \centering
%         \includegraphics[width=\linewidth]{assets/agg/Mistral_Large.pdf}
%         \caption{Mistral Large}
%         \label{fig:Mistral_Large}
%     \end{minipage}
%     \hfill
%     \begin{minipage}{0.49\linewidth}
%         \centering
%         \includegraphics[width=\linewidth]{assets/agg/Llama_3.1_8b_Instruct.pdf}
%         \caption{Llama 3.1 8B Instruct}
%         \label{fig:Llama_3.1_8b_Instruct}
%     \end{minipage}
%     % % \vspace{-1em}
% \end{figure*}

\vspace{-5pt}
\section{Discussion}
\textbf{Conclusion.}
In this work, we propose the \textit{\methodname{}} metric, $M_{AP}$, to evaluate preference data quality in alignment.
By measuring the gap from the model's current implicit reward margin to the target explicit reward margin, $M_{AP}$ quantifies the discrepancy between the current model and the aligned optimum, thereby indicating the potential for alignment enhancement.
Extensive experiments validate the efficacy of $M_{AP}$ across various training settings under offline and self-play preference learning scenarios.

\textbf{Limitations and future work}. 
Despite the performance improvements, $M_{AP}$ requires tuning a parameter $\beta$ to combine the explicit and implicit margins; future work could explore how to set this ratio automatically.
Additionally, while our experiments focus on the widely applied DPO and SimPO objectives, a broader investigation with alternative preference learning methods is crucial in future works.

% \section{Conclusion}
% In this paper, we introduce the \methodname{} metric to evaluate preference data quality in LLM alignment.
% By measuring the discrepancy between the model's current implicit reward margin to the target explicit reward margin, this metric quantifies the gap between the current model and the aligned optimum, thereby indicating the potential for alignment enhancement.
% Empirical results demonstrate that training on data selected by our metric consistently improves alignment performance, outperforming existing metrics across different base models and training objectives.
% Moreover, this metric extends to data generation scenarios (\ie self-play alignment): by identifying high-quality data from the intrinsic self-generated context, our metric yields superior results across various training settings, providing a comprehensive solution for enhancing LLM alignment through optimized
% preference data generation, selection, and utilization.


\section*{Impact Statement}
This paper presents work whose goal is to advance the field of Machine Learning. There are many potential societal consequences of our work, none which we feel must be specifically highlighted here.
\section*{Limitations}
Although our benchmark framework supports several tools to facilitate large language model agents in the code deployment task, it does not actually improve the original reasoning capabilities of the large language models that are used in the agents. To improve LLMs' reasoning capabilities for this specific task, the community may resort to techniques like RLHF, which is orthogonal to this work. Our benchmark only focuses on code repositories that related to computer science research topics, and does not involve other types of repositories. Although this framework can be reused for other types of the repositories, we do not explore that direction in this work, and leave it to future works.
\newpage
% Bibliography entries for the entire Anthology, followed by custom entries
%\bibliography{anthology,custom}
% Custom bibliography entries only
\bibliography{custom}
\clearpage
\newpage
\appendix

\subsection{Lloyd-Max Algorithm}
\label{subsec:Lloyd-Max}
For a given quantization bitwidth $B$ and an operand $\bm{X}$, the Lloyd-Max algorithm finds $2^B$ quantization levels $\{\hat{x}_i\}_{i=1}^{2^B}$ such that quantizing $\bm{X}$ by rounding each scalar in $\bm{X}$ to the nearest quantization level minimizes the quantization MSE. 

The algorithm starts with an initial guess of quantization levels and then iteratively computes quantization thresholds $\{\tau_i\}_{i=1}^{2^B-1}$ and updates quantization levels $\{\hat{x}_i\}_{i=1}^{2^B}$. Specifically, at iteration $n$, thresholds are set to the midpoints of the previous iteration's levels:
\begin{align*}
    \tau_i^{(n)}=\frac{\hat{x}_i^{(n-1)}+\hat{x}_{i+1}^{(n-1)}}2 \text{ for } i=1\ldots 2^B-1
\end{align*}
Subsequently, the quantization levels are re-computed as conditional means of the data regions defined by the new thresholds:
\begin{align*}
    \hat{x}_i^{(n)}=\mathbb{E}\left[ \bm{X} \big| \bm{X}\in [\tau_{i-1}^{(n)},\tau_i^{(n)}] \right] \text{ for } i=1\ldots 2^B
\end{align*}
where to satisfy boundary conditions we have $\tau_0=-\infty$ and $\tau_{2^B}=\infty$. The algorithm iterates the above steps until convergence.

Figure \ref{fig:lm_quant} compares the quantization levels of a $7$-bit floating point (E3M3) quantizer (left) to a $7$-bit Lloyd-Max quantizer (right) when quantizing a layer of weights from the GPT3-126M model at a per-tensor granularity. As shown, the Lloyd-Max quantizer achieves substantially lower quantization MSE. Further, Table \ref{tab:FP7_vs_LM7} shows the superior perplexity achieved by Lloyd-Max quantizers for bitwidths of $7$, $6$ and $5$. The difference between the quantizers is clear at 5 bits, where per-tensor FP quantization incurs a drastic and unacceptable increase in perplexity, while Lloyd-Max quantization incurs a much smaller increase. Nevertheless, we note that even the optimal Lloyd-Max quantizer incurs a notable ($\sim 1.5$) increase in perplexity due to the coarse granularity of quantization. 

\begin{figure}[h]
  \centering
  \includegraphics[width=0.7\linewidth]{sections/figures/LM7_FP7.pdf}
  \caption{\small Quantization levels and the corresponding quantization MSE of Floating Point (left) vs Lloyd-Max (right) Quantizers for a layer of weights in the GPT3-126M model.}
  \label{fig:lm_quant}
\end{figure}

\begin{table}[h]\scriptsize
\begin{center}
\caption{\label{tab:FP7_vs_LM7} \small Comparing perplexity (lower is better) achieved by floating point quantizers and Lloyd-Max quantizers on a GPT3-126M model for the Wikitext-103 dataset.}
\begin{tabular}{c|cc|c}
\hline
 \multirow{2}{*}{\textbf{Bitwidth}} & \multicolumn{2}{|c|}{\textbf{Floating-Point Quantizer}} & \textbf{Lloyd-Max Quantizer} \\
 & Best Format & Wikitext-103 Perplexity & Wikitext-103 Perplexity \\
\hline
7 & E3M3 & 18.32 & 18.27 \\
6 & E3M2 & 19.07 & 18.51 \\
5 & E4M0 & 43.89 & 19.71 \\
\hline
\end{tabular}
\end{center}
\end{table}

\subsection{Proof of Local Optimality of LO-BCQ}
\label{subsec:lobcq_opt_proof}
For a given block $\bm{b}_j$, the quantization MSE during LO-BCQ can be empirically evaluated as $\frac{1}{L_b}\lVert \bm{b}_j- \bm{\hat{b}}_j\rVert^2_2$ where $\bm{\hat{b}}_j$ is computed from equation (\ref{eq:clustered_quantization_definition}) as $C_{f(\bm{b}_j)}(\bm{b}_j)$. Further, for a given block cluster $\mathcal{B}_i$, we compute the quantization MSE as $\frac{1}{|\mathcal{B}_{i}|}\sum_{\bm{b} \in \mathcal{B}_{i}} \frac{1}{L_b}\lVert \bm{b}- C_i^{(n)}(\bm{b})\rVert^2_2$. Therefore, at the end of iteration $n$, we evaluate the overall quantization MSE $J^{(n)}$ for a given operand $\bm{X}$ composed of $N_c$ block clusters as:
\begin{align*}
    \label{eq:mse_iter_n}
    J^{(n)} = \frac{1}{N_c} \sum_{i=1}^{N_c} \frac{1}{|\mathcal{B}_{i}^{(n)}|}\sum_{\bm{v} \in \mathcal{B}_{i}^{(n)}} \frac{1}{L_b}\lVert \bm{b}- B_i^{(n)}(\bm{b})\rVert^2_2
\end{align*}

At the end of iteration $n$, the codebooks are updated from $\mathcal{C}^{(n-1)}$ to $\mathcal{C}^{(n)}$. However, the mapping of a given vector $\bm{b}_j$ to quantizers $\mathcal{C}^{(n)}$ remains as  $f^{(n)}(\bm{b}_j)$. At the next iteration, during the vector clustering step, $f^{(n+1)}(\bm{b}_j)$ finds new mapping of $\bm{b}_j$ to updated codebooks $\mathcal{C}^{(n)}$ such that the quantization MSE over the candidate codebooks is minimized. Therefore, we obtain the following result for $\bm{b}_j$:
\begin{align*}
\frac{1}{L_b}\lVert \bm{b}_j - C_{f^{(n+1)}(\bm{b}_j)}^{(n)}(\bm{b}_j)\rVert^2_2 \le \frac{1}{L_b}\lVert \bm{b}_j - C_{f^{(n)}(\bm{b}_j)}^{(n)}(\bm{b}_j)\rVert^2_2
\end{align*}

That is, quantizing $\bm{b}_j$ at the end of the block clustering step of iteration $n+1$ results in lower quantization MSE compared to quantizing at the end of iteration $n$. Since this is true for all $\bm{b} \in \bm{X}$, we assert the following:
\begin{equation}
\begin{split}
\label{eq:mse_ineq_1}
    \tilde{J}^{(n+1)} &= \frac{1}{N_c} \sum_{i=1}^{N_c} \frac{1}{|\mathcal{B}_{i}^{(n+1)}|}\sum_{\bm{b} \in \mathcal{B}_{i}^{(n+1)}} \frac{1}{L_b}\lVert \bm{b} - C_i^{(n)}(b)\rVert^2_2 \le J^{(n)}
\end{split}
\end{equation}
where $\tilde{J}^{(n+1)}$ is the the quantization MSE after the vector clustering step at iteration $n+1$.

Next, during the codebook update step (\ref{eq:quantizers_update}) at iteration $n+1$, the per-cluster codebooks $\mathcal{C}^{(n)}$ are updated to $\mathcal{C}^{(n+1)}$ by invoking the Lloyd-Max algorithm \citep{Lloyd}. We know that for any given value distribution, the Lloyd-Max algorithm minimizes the quantization MSE. Therefore, for a given vector cluster $\mathcal{B}_i$ we obtain the following result:

\begin{equation}
    \frac{1}{|\mathcal{B}_{i}^{(n+1)}|}\sum_{\bm{b} \in \mathcal{B}_{i}^{(n+1)}} \frac{1}{L_b}\lVert \bm{b}- C_i^{(n+1)}(\bm{b})\rVert^2_2 \le \frac{1}{|\mathcal{B}_{i}^{(n+1)}|}\sum_{\bm{b} \in \mathcal{B}_{i}^{(n+1)}} \frac{1}{L_b}\lVert \bm{b}- C_i^{(n)}(\bm{b})\rVert^2_2
\end{equation}

The above equation states that quantizing the given block cluster $\mathcal{B}_i$ after updating the associated codebook from $C_i^{(n)}$ to $C_i^{(n+1)}$ results in lower quantization MSE. Since this is true for all the block clusters, we derive the following result: 
\begin{equation}
\begin{split}
\label{eq:mse_ineq_2}
     J^{(n+1)} &= \frac{1}{N_c} \sum_{i=1}^{N_c} \frac{1}{|\mathcal{B}_{i}^{(n+1)}|}\sum_{\bm{b} \in \mathcal{B}_{i}^{(n+1)}} \frac{1}{L_b}\lVert \bm{b}- C_i^{(n+1)}(\bm{b})\rVert^2_2  \le \tilde{J}^{(n+1)}   
\end{split}
\end{equation}

Following (\ref{eq:mse_ineq_1}) and (\ref{eq:mse_ineq_2}), we find that the quantization MSE is non-increasing for each iteration, that is, $J^{(1)} \ge J^{(2)} \ge J^{(3)} \ge \ldots \ge J^{(M)}$ where $M$ is the maximum number of iterations. 
%Therefore, we can say that if the algorithm converges, then it must be that it has converged to a local minimum. 
\hfill $\blacksquare$


\begin{figure}
    \begin{center}
    \includegraphics[width=0.5\textwidth]{sections//figures/mse_vs_iter.pdf}
    \end{center}
    \caption{\small NMSE vs iterations during LO-BCQ compared to other block quantization proposals}
    \label{fig:nmse_vs_iter}
\end{figure}

Figure \ref{fig:nmse_vs_iter} shows the empirical convergence of LO-BCQ across several block lengths and number of codebooks. Also, the MSE achieved by LO-BCQ is compared to baselines such as MXFP and VSQ. As shown, LO-BCQ converges to a lower MSE than the baselines. Further, we achieve better convergence for larger number of codebooks ($N_c$) and for a smaller block length ($L_b$), both of which increase the bitwidth of BCQ (see Eq \ref{eq:bitwidth_bcq}).


\subsection{Additional Accuracy Results}
%Table \ref{tab:lobcq_config} lists the various LOBCQ configurations and their corresponding bitwidths.
\begin{table}
\setlength{\tabcolsep}{4.75pt}
\begin{center}
\caption{\label{tab:lobcq_config} Various LO-BCQ configurations and their bitwidths.}
\begin{tabular}{|c||c|c|c|c||c|c||c|} 
\hline
 & \multicolumn{4}{|c||}{$L_b=8$} & \multicolumn{2}{|c||}{$L_b=4$} & $L_b=2$ \\
 \hline
 \backslashbox{$L_A$\kern-1em}{\kern-1em$N_c$} & 2 & 4 & 8 & 16 & 2 & 4 & 2 \\
 \hline
 64 & 4.25 & 4.375 & 4.5 & 4.625 & 4.375 & 4.625 & 4.625\\
 \hline
 32 & 4.375 & 4.5 & 4.625& 4.75 & 4.5 & 4.75 & 4.75 \\
 \hline
 16 & 4.625 & 4.75& 4.875 & 5 & 4.75 & 5 & 5 \\
 \hline
\end{tabular}
\end{center}
\end{table}

%\subsection{Perplexity achieved by various LO-BCQ configurations on Wikitext-103 dataset}

\begin{table} \centering
\begin{tabular}{|c||c|c|c|c||c|c||c|} 
\hline
 $L_b \rightarrow$& \multicolumn{4}{c||}{8} & \multicolumn{2}{c||}{4} & 2\\
 \hline
 \backslashbox{$L_A$\kern-1em}{\kern-1em$N_c$} & 2 & 4 & 8 & 16 & 2 & 4 & 2  \\
 %$N_c \rightarrow$ & 2 & 4 & 8 & 16 & 2 & 4 & 2 \\
 \hline
 \hline
 \multicolumn{8}{c}{GPT3-1.3B (FP32 PPL = 9.98)} \\ 
 \hline
 \hline
 64 & 10.40 & 10.23 & 10.17 & 10.15 &  10.28 & 10.18 & 10.19 \\
 \hline
 32 & 10.25 & 10.20 & 10.15 & 10.12 &  10.23 & 10.17 & 10.17 \\
 \hline
 16 & 10.22 & 10.16 & 10.10 & 10.09 &  10.21 & 10.14 & 10.16 \\
 \hline
  \hline
 \multicolumn{8}{c}{GPT3-8B (FP32 PPL = 7.38)} \\ 
 \hline
 \hline
 64 & 7.61 & 7.52 & 7.48 &  7.47 &  7.55 &  7.49 & 7.50 \\
 \hline
 32 & 7.52 & 7.50 & 7.46 &  7.45 &  7.52 &  7.48 & 7.48  \\
 \hline
 16 & 7.51 & 7.48 & 7.44 &  7.44 &  7.51 &  7.49 & 7.47  \\
 \hline
\end{tabular}
\caption{\label{tab:ppl_gpt3_abalation} Wikitext-103 perplexity across GPT3-1.3B and 8B models.}
\end{table}

\begin{table} \centering
\begin{tabular}{|c||c|c|c|c||} 
\hline
 $L_b \rightarrow$& \multicolumn{4}{c||}{8}\\
 \hline
 \backslashbox{$L_A$\kern-1em}{\kern-1em$N_c$} & 2 & 4 & 8 & 16 \\
 %$N_c \rightarrow$ & 2 & 4 & 8 & 16 & 2 & 4 & 2 \\
 \hline
 \hline
 \multicolumn{5}{|c|}{Llama2-7B (FP32 PPL = 5.06)} \\ 
 \hline
 \hline
 64 & 5.31 & 5.26 & 5.19 & 5.18  \\
 \hline
 32 & 5.23 & 5.25 & 5.18 & 5.15  \\
 \hline
 16 & 5.23 & 5.19 & 5.16 & 5.14  \\
 \hline
 \multicolumn{5}{|c|}{Nemotron4-15B (FP32 PPL = 5.87)} \\ 
 \hline
 \hline
 64  & 6.3 & 6.20 & 6.13 & 6.08  \\
 \hline
 32  & 6.24 & 6.12 & 6.07 & 6.03  \\
 \hline
 16  & 6.12 & 6.14 & 6.04 & 6.02  \\
 \hline
 \multicolumn{5}{|c|}{Nemotron4-340B (FP32 PPL = 3.48)} \\ 
 \hline
 \hline
 64 & 3.67 & 3.62 & 3.60 & 3.59 \\
 \hline
 32 & 3.63 & 3.61 & 3.59 & 3.56 \\
 \hline
 16 & 3.61 & 3.58 & 3.57 & 3.55 \\
 \hline
\end{tabular}
\caption{\label{tab:ppl_llama7B_nemo15B} Wikitext-103 perplexity compared to FP32 baseline in Llama2-7B and Nemotron4-15B, 340B models}
\end{table}

%\subsection{Perplexity achieved by various LO-BCQ configurations on MMLU dataset}


\begin{table} \centering
\begin{tabular}{|c||c|c|c|c||c|c|c|c|} 
\hline
 $L_b \rightarrow$& \multicolumn{4}{c||}{8} & \multicolumn{4}{c||}{8}\\
 \hline
 \backslashbox{$L_A$\kern-1em}{\kern-1em$N_c$} & 2 & 4 & 8 & 16 & 2 & 4 & 8 & 16  \\
 %$N_c \rightarrow$ & 2 & 4 & 8 & 16 & 2 & 4 & 2 \\
 \hline
 \hline
 \multicolumn{5}{|c|}{Llama2-7B (FP32 Accuracy = 45.8\%)} & \multicolumn{4}{|c|}{Llama2-70B (FP32 Accuracy = 69.12\%)} \\ 
 \hline
 \hline
 64 & 43.9 & 43.4 & 43.9 & 44.9 & 68.07 & 68.27 & 68.17 & 68.75 \\
 \hline
 32 & 44.5 & 43.8 & 44.9 & 44.5 & 68.37 & 68.51 & 68.35 & 68.27  \\
 \hline
 16 & 43.9 & 42.7 & 44.9 & 45 & 68.12 & 68.77 & 68.31 & 68.59  \\
 \hline
 \hline
 \multicolumn{5}{|c|}{GPT3-22B (FP32 Accuracy = 38.75\%)} & \multicolumn{4}{|c|}{Nemotron4-15B (FP32 Accuracy = 64.3\%)} \\ 
 \hline
 \hline
 64 & 36.71 & 38.85 & 38.13 & 38.92 & 63.17 & 62.36 & 63.72 & 64.09 \\
 \hline
 32 & 37.95 & 38.69 & 39.45 & 38.34 & 64.05 & 62.30 & 63.8 & 64.33  \\
 \hline
 16 & 38.88 & 38.80 & 38.31 & 38.92 & 63.22 & 63.51 & 63.93 & 64.43  \\
 \hline
\end{tabular}
\caption{\label{tab:mmlu_abalation} Accuracy on MMLU dataset across GPT3-22B, Llama2-7B, 70B and Nemotron4-15B models.}
\end{table}


%\subsection{Perplexity achieved by various LO-BCQ configurations on LM evaluation harness}

\begin{table} \centering
\begin{tabular}{|c||c|c|c|c||c|c|c|c|} 
\hline
 $L_b \rightarrow$& \multicolumn{4}{c||}{8} & \multicolumn{4}{c||}{8}\\
 \hline
 \backslashbox{$L_A$\kern-1em}{\kern-1em$N_c$} & 2 & 4 & 8 & 16 & 2 & 4 & 8 & 16  \\
 %$N_c \rightarrow$ & 2 & 4 & 8 & 16 & 2 & 4 & 2 \\
 \hline
 \hline
 \multicolumn{5}{|c|}{Race (FP32 Accuracy = 37.51\%)} & \multicolumn{4}{|c|}{Boolq (FP32 Accuracy = 64.62\%)} \\ 
 \hline
 \hline
 64 & 36.94 & 37.13 & 36.27 & 37.13 & 63.73 & 62.26 & 63.49 & 63.36 \\
 \hline
 32 & 37.03 & 36.36 & 36.08 & 37.03 & 62.54 & 63.51 & 63.49 & 63.55  \\
 \hline
 16 & 37.03 & 37.03 & 36.46 & 37.03 & 61.1 & 63.79 & 63.58 & 63.33  \\
 \hline
 \hline
 \multicolumn{5}{|c|}{Winogrande (FP32 Accuracy = 58.01\%)} & \multicolumn{4}{|c|}{Piqa (FP32 Accuracy = 74.21\%)} \\ 
 \hline
 \hline
 64 & 58.17 & 57.22 & 57.85 & 58.33 & 73.01 & 73.07 & 73.07 & 72.80 \\
 \hline
 32 & 59.12 & 58.09 & 57.85 & 58.41 & 73.01 & 73.94 & 72.74 & 73.18  \\
 \hline
 16 & 57.93 & 58.88 & 57.93 & 58.56 & 73.94 & 72.80 & 73.01 & 73.94  \\
 \hline
\end{tabular}
\caption{\label{tab:mmlu_abalation} Accuracy on LM evaluation harness tasks on GPT3-1.3B model.}
\end{table}

\begin{table} \centering
\begin{tabular}{|c||c|c|c|c||c|c|c|c|} 
\hline
 $L_b \rightarrow$& \multicolumn{4}{c||}{8} & \multicolumn{4}{c||}{8}\\
 \hline
 \backslashbox{$L_A$\kern-1em}{\kern-1em$N_c$} & 2 & 4 & 8 & 16 & 2 & 4 & 8 & 16  \\
 %$N_c \rightarrow$ & 2 & 4 & 8 & 16 & 2 & 4 & 2 \\
 \hline
 \hline
 \multicolumn{5}{|c|}{Race (FP32 Accuracy = 41.34\%)} & \multicolumn{4}{|c|}{Boolq (FP32 Accuracy = 68.32\%)} \\ 
 \hline
 \hline
 64 & 40.48 & 40.10 & 39.43 & 39.90 & 69.20 & 68.41 & 69.45 & 68.56 \\
 \hline
 32 & 39.52 & 39.52 & 40.77 & 39.62 & 68.32 & 67.43 & 68.17 & 69.30  \\
 \hline
 16 & 39.81 & 39.71 & 39.90 & 40.38 & 68.10 & 66.33 & 69.51 & 69.42  \\
 \hline
 \hline
 \multicolumn{5}{|c|}{Winogrande (FP32 Accuracy = 67.88\%)} & \multicolumn{4}{|c|}{Piqa (FP32 Accuracy = 78.78\%)} \\ 
 \hline
 \hline
 64 & 66.85 & 66.61 & 67.72 & 67.88 & 77.31 & 77.42 & 77.75 & 77.64 \\
 \hline
 32 & 67.25 & 67.72 & 67.72 & 67.00 & 77.31 & 77.04 & 77.80 & 77.37  \\
 \hline
 16 & 68.11 & 68.90 & 67.88 & 67.48 & 77.37 & 78.13 & 78.13 & 77.69  \\
 \hline
\end{tabular}
\caption{\label{tab:mmlu_abalation} Accuracy on LM evaluation harness tasks on GPT3-8B model.}
\end{table}

\begin{table} \centering
\begin{tabular}{|c||c|c|c|c||c|c|c|c|} 
\hline
 $L_b \rightarrow$& \multicolumn{4}{c||}{8} & \multicolumn{4}{c||}{8}\\
 \hline
 \backslashbox{$L_A$\kern-1em}{\kern-1em$N_c$} & 2 & 4 & 8 & 16 & 2 & 4 & 8 & 16  \\
 %$N_c \rightarrow$ & 2 & 4 & 8 & 16 & 2 & 4 & 2 \\
 \hline
 \hline
 \multicolumn{5}{|c|}{Race (FP32 Accuracy = 40.67\%)} & \multicolumn{4}{|c|}{Boolq (FP32 Accuracy = 76.54\%)} \\ 
 \hline
 \hline
 64 & 40.48 & 40.10 & 39.43 & 39.90 & 75.41 & 75.11 & 77.09 & 75.66 \\
 \hline
 32 & 39.52 & 39.52 & 40.77 & 39.62 & 76.02 & 76.02 & 75.96 & 75.35  \\
 \hline
 16 & 39.81 & 39.71 & 39.90 & 40.38 & 75.05 & 73.82 & 75.72 & 76.09  \\
 \hline
 \hline
 \multicolumn{5}{|c|}{Winogrande (FP32 Accuracy = 70.64\%)} & \multicolumn{4}{|c|}{Piqa (FP32 Accuracy = 79.16\%)} \\ 
 \hline
 \hline
 64 & 69.14 & 70.17 & 70.17 & 70.56 & 78.24 & 79.00 & 78.62 & 78.73 \\
 \hline
 32 & 70.96 & 69.69 & 71.27 & 69.30 & 78.56 & 79.49 & 79.16 & 78.89  \\
 \hline
 16 & 71.03 & 69.53 & 69.69 & 70.40 & 78.13 & 79.16 & 79.00 & 79.00  \\
 \hline
\end{tabular}
\caption{\label{tab:mmlu_abalation} Accuracy on LM evaluation harness tasks on GPT3-22B model.}
\end{table}

\begin{table} \centering
\begin{tabular}{|c||c|c|c|c||c|c|c|c|} 
\hline
 $L_b \rightarrow$& \multicolumn{4}{c||}{8} & \multicolumn{4}{c||}{8}\\
 \hline
 \backslashbox{$L_A$\kern-1em}{\kern-1em$N_c$} & 2 & 4 & 8 & 16 & 2 & 4 & 8 & 16  \\
 %$N_c \rightarrow$ & 2 & 4 & 8 & 16 & 2 & 4 & 2 \\
 \hline
 \hline
 \multicolumn{5}{|c|}{Race (FP32 Accuracy = 44.4\%)} & \multicolumn{4}{|c|}{Boolq (FP32 Accuracy = 79.29\%)} \\ 
 \hline
 \hline
 64 & 42.49 & 42.51 & 42.58 & 43.45 & 77.58 & 77.37 & 77.43 & 78.1 \\
 \hline
 32 & 43.35 & 42.49 & 43.64 & 43.73 & 77.86 & 75.32 & 77.28 & 77.86  \\
 \hline
 16 & 44.21 & 44.21 & 43.64 & 42.97 & 78.65 & 77 & 76.94 & 77.98  \\
 \hline
 \hline
 \multicolumn{5}{|c|}{Winogrande (FP32 Accuracy = 69.38\%)} & \multicolumn{4}{|c|}{Piqa (FP32 Accuracy = 78.07\%)} \\ 
 \hline
 \hline
 64 & 68.9 & 68.43 & 69.77 & 68.19 & 77.09 & 76.82 & 77.09 & 77.86 \\
 \hline
 32 & 69.38 & 68.51 & 68.82 & 68.90 & 78.07 & 76.71 & 78.07 & 77.86  \\
 \hline
 16 & 69.53 & 67.09 & 69.38 & 68.90 & 77.37 & 77.8 & 77.91 & 77.69  \\
 \hline
\end{tabular}
\caption{\label{tab:mmlu_abalation} Accuracy on LM evaluation harness tasks on Llama2-7B model.}
\end{table}

\begin{table} \centering
\begin{tabular}{|c||c|c|c|c||c|c|c|c|} 
\hline
 $L_b \rightarrow$& \multicolumn{4}{c||}{8} & \multicolumn{4}{c||}{8}\\
 \hline
 \backslashbox{$L_A$\kern-1em}{\kern-1em$N_c$} & 2 & 4 & 8 & 16 & 2 & 4 & 8 & 16  \\
 %$N_c \rightarrow$ & 2 & 4 & 8 & 16 & 2 & 4 & 2 \\
 \hline
 \hline
 \multicolumn{5}{|c|}{Race (FP32 Accuracy = 48.8\%)} & \multicolumn{4}{|c|}{Boolq (FP32 Accuracy = 85.23\%)} \\ 
 \hline
 \hline
 64 & 49.00 & 49.00 & 49.28 & 48.71 & 82.82 & 84.28 & 84.03 & 84.25 \\
 \hline
 32 & 49.57 & 48.52 & 48.33 & 49.28 & 83.85 & 84.46 & 84.31 & 84.93  \\
 \hline
 16 & 49.85 & 49.09 & 49.28 & 48.99 & 85.11 & 84.46 & 84.61 & 83.94  \\
 \hline
 \hline
 \multicolumn{5}{|c|}{Winogrande (FP32 Accuracy = 79.95\%)} & \multicolumn{4}{|c|}{Piqa (FP32 Accuracy = 81.56\%)} \\ 
 \hline
 \hline
 64 & 78.77 & 78.45 & 78.37 & 79.16 & 81.45 & 80.69 & 81.45 & 81.5 \\
 \hline
 32 & 78.45 & 79.01 & 78.69 & 80.66 & 81.56 & 80.58 & 81.18 & 81.34  \\
 \hline
 16 & 79.95 & 79.56 & 79.79 & 79.72 & 81.28 & 81.66 & 81.28 & 80.96  \\
 \hline
\end{tabular}
\caption{\label{tab:mmlu_abalation} Accuracy on LM evaluation harness tasks on Llama2-70B model.}
\end{table}

%\section{MSE Studies}
%\textcolor{red}{TODO}


\subsection{Number Formats and Quantization Method}
\label{subsec:numFormats_quantMethod}
\subsubsection{Integer Format}
An $n$-bit signed integer (INT) is typically represented with a 2s-complement format \citep{yao2022zeroquant,xiao2023smoothquant,dai2021vsq}, where the most significant bit denotes the sign.

\subsubsection{Floating Point Format}
An $n$-bit signed floating point (FP) number $x$ comprises of a 1-bit sign ($x_{\mathrm{sign}}$), $B_m$-bit mantissa ($x_{\mathrm{mant}}$) and $B_e$-bit exponent ($x_{\mathrm{exp}}$) such that $B_m+B_e=n-1$. The associated constant exponent bias ($E_{\mathrm{bias}}$) is computed as $(2^{{B_e}-1}-1)$. We denote this format as $E_{B_e}M_{B_m}$.  

\subsubsection{Quantization Scheme}
\label{subsec:quant_method}
A quantization scheme dictates how a given unquantized tensor is converted to its quantized representation. We consider FP formats for the purpose of illustration. Given an unquantized tensor $\bm{X}$ and an FP format $E_{B_e}M_{B_m}$, we first, we compute the quantization scale factor $s_X$ that maps the maximum absolute value of $\bm{X}$ to the maximum quantization level of the $E_{B_e}M_{B_m}$ format as follows:
\begin{align}
\label{eq:sf}
    s_X = \frac{\mathrm{max}(|\bm{X}|)}{\mathrm{max}(E_{B_e}M_{B_m})}
\end{align}
In the above equation, $|\cdot|$ denotes the absolute value function.

Next, we scale $\bm{X}$ by $s_X$ and quantize it to $\hat{\bm{X}}$ by rounding it to the nearest quantization level of $E_{B_e}M_{B_m}$ as:

\begin{align}
\label{eq:tensor_quant}
    \hat{\bm{X}} = \text{round-to-nearest}\left(\frac{\bm{X}}{s_X}, E_{B_e}M_{B_m}\right)
\end{align}

We perform dynamic max-scaled quantization \citep{wu2020integer}, where the scale factor $s$ for activations is dynamically computed during runtime.

\subsection{Vector Scaled Quantization}
\begin{wrapfigure}{r}{0.35\linewidth}
  \centering
  \includegraphics[width=\linewidth]{sections/figures/vsquant.jpg}
  \caption{\small Vectorwise decomposition for per-vector scaled quantization (VSQ \citep{dai2021vsq}).}
  \label{fig:vsquant}
\end{wrapfigure}
During VSQ \citep{dai2021vsq}, the operand tensors are decomposed into 1D vectors in a hardware friendly manner as shown in Figure \ref{fig:vsquant}. Since the decomposed tensors are used as operands in matrix multiplications during inference, it is beneficial to perform this decomposition along the reduction dimension of the multiplication. The vectorwise quantization is performed similar to tensorwise quantization described in Equations \ref{eq:sf} and \ref{eq:tensor_quant}, where a scale factor $s_v$ is required for each vector $\bm{v}$ that maps the maximum absolute value of that vector to the maximum quantization level. While smaller vector lengths can lead to larger accuracy gains, the associated memory and computational overheads due to the per-vector scale factors increases. To alleviate these overheads, VSQ \citep{dai2021vsq} proposed a second level quantization of the per-vector scale factors to unsigned integers, while MX \citep{rouhani2023shared} quantizes them to integer powers of 2 (denoted as $2^{INT}$).

\subsubsection{MX Format}
The MX format proposed in \citep{rouhani2023microscaling} introduces the concept of sub-block shifting. For every two scalar elements of $b$-bits each, there is a shared exponent bit. The value of this exponent bit is determined through an empirical analysis that targets minimizing quantization MSE. We note that the FP format $E_{1}M_{b}$ is strictly better than MX from an accuracy perspective since it allocates a dedicated exponent bit to each scalar as opposed to sharing it across two scalars. Therefore, we conservatively bound the accuracy of a $b+2$-bit signed MX format with that of a $E_{1}M_{b}$ format in our comparisons. For instance, we use E1M2 format as a proxy for MX4.

\begin{figure}
    \centering
    \includegraphics[width=1\linewidth]{sections//figures/BlockFormats.pdf}
    \caption{\small Comparing LO-BCQ to MX format.}
    \label{fig:block_formats}
\end{figure}

Figure \ref{fig:block_formats} compares our $4$-bit LO-BCQ block format to MX \citep{rouhani2023microscaling}. As shown, both LO-BCQ and MX decompose a given operand tensor into block arrays and each block array into blocks. Similar to MX, we find that per-block quantization ($L_b < L_A$) leads to better accuracy due to increased flexibility. While MX achieves this through per-block $1$-bit micro-scales, we associate a dedicated codebook to each block through a per-block codebook selector. Further, MX quantizes the per-block array scale-factor to E8M0 format without per-tensor scaling. In contrast during LO-BCQ, we find that per-tensor scaling combined with quantization of per-block array scale-factor to E4M3 format results in superior inference accuracy across models. 


\end{document}
