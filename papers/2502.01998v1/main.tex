% VLDB template version of 2020-08-03 enhances the ACM template, version 1.7.0:
% https://www.acm.org/publications/proceedings-template
% The ACM Latex guide provides further information about the ACM template

\documentclass[sigconf, nonacm]{acmart}
\usepackage{tikz}
\usetikzlibrary{positioning}
\usetikzlibrary{calc}
\usepackage{array}
\usepackage{xcolor}
\usepackage{listings}
\usepackage{colortbl} % For coloring table rows
\usepackage{graphicx}
\usepackage{subcaption}
\usepackage{makecell} % For enabling word wrapping in table cells
\usepackage{enumitem}
\usepackage{amsthm}
\usepackage{amsmath}
\usepackage{algorithm}
\usepackage{algorithmic}
\usepackage{multirow} % For multi-row cells
%\usepackage{lmodern}

\newtheorem{definition}{Definition}  % This creates the 'definition' environment

%% The following content must be adapted for the final version
% paper-specific
\newcommand\vldbdoi{XX.XX/XXX.XX}
\newcommand\vldbpages{XXX-XXX}
% issue-specific
\newcommand\vldbvolume{14}
\newcommand\vldbissue{1}
\newcommand\vldbyear{2020}
% should be fine as it is
\newcommand\vldbauthors{\authors}
\newcommand\vldbtitle{\shorttitle} 
% leave empty if no availability url should be set
\newcommand\vldbavailabilityurl{https://github.com/linkedin/dataguard-udfs}
% whether page numbers should be shown or not, use 'plain' for review versions, 'empty' for camera ready
\newcommand\vldbpagestyle{plain} 

\begin{document}
\title{Data Guard: A Fine-grained Purpose-based Access Control System for Large Data Warehouses}

%%
%% The "author" command and its associated commands are used to define the authors and their affiliations.
\author{Khai Tran$^{*}$, Sudarshan Vasudevan$^{*}$, Pratham Desai$^{*}$, Alex Gorelik, Mayank Ahuja,
Athrey Yadatore Venkateshababu, Mohit Verma, Dichao Hu, Walaa Eldin Moustafa, Vasanth Rajamani$^{\dagger \ddagger}$, Ankit Gupta, Issac Buenrostro, Kalinda Raina}

\affiliation{LinkedIn Coporation, OpenAI Inc$^{\dagger}$ \hspace{1cm}}

\email{{khtran,suvasudevan,padesai,agorelik,mahuja,avenkateshababu,moverma,dihu,wmoustafa}@linkedin.com}
\email{vasanth@openai.com, {aigupta,ibuenros,kraina}@linkedin.com}

%%
%% The abstract is a short summary of the work to be presented in the
%% article.
\begin{abstract}
The last few years have witnessed a spate of data protection regulations in conjunction with an ever-growing appetite for data usage in large businesses, 
thus presenting significant challenges for businesses to maintain compliance. 
To address this conflict, we present \emph{Data Guard} - a fine-grained, purpose-based access control system for large data warehouses. 
Data Guard enables authoring policies based on semantic descriptions of data and  purpose of data access. Data Guard then translates these policies into SQL views that mask data from the underlying warehouse tables. 
At access time, Data Guard ensures compliance by transparently routing each table access to the appropriate data-masking view based on the purpose of the access,  
thus minimizing the effort of adopting Data Guard in existing applications.
Our enforcement solution allows masking data at much finer granularities than what traditional solutions allow. 
In addition to row and column level data masking, Data Guard can mask data at the sub-cell level for columns with non-atomic data types such as structs, arrays, and maps. 
This fine-grained masking allows Data Guard to preserve data utility for consumers while ensuring compliance. We implemented a number of performance optimizations to minimize the overhead of data masking operations. We perform numerous experiments to identify the key factors that influence the data masking overhead and demonstrate the efficiency of our implementation. 
\end{abstract}
\maketitle
%%% do not modify the following VLDB block %%
%%% VLDB block start %%%
\pagestyle{\vldbpagestyle}
\begingroup\small\noindent\raggedright\textbf{PVLDB Reference Format:}\\
Khai Tran, Sudarshan Vasudevan, Pratham Desai, Alex
Gorelik, Mayank Ahuja, Athrey Yadatore Venkateshababu, Mohit Verma, Dichao Hu, Walaa Eldin Moustafa, Vasanth Rajamani, 
Ankit Gupta, Issac Buenrostro, Kalinda Raina. \vldbtitle. PVLDB, \vldbvolume(\vldbissue): \vldbpages, \vldbyear.\\
\href{https://doi.org/\vldbdoi}{doi:\vldbdoi}
\endgroup
\begingroup
\renewcommand\thefootnote{}\footnote{\noindent
$^{*}$Equal contributors.\\
$^{\ddagger}$Work done while the author was at LinkedIn Coporation.\\
This work is licensed under the Creative Commons BY-NC-ND 4.0 International License. Visit \url{https://creativecommons.org/licenses/by-nc-nd/4.0/} to view a copy of this license. For any use beyond those covered by this license, obtain permission by emailing \href{mailto:info@vldb.org}{info@vldb.org}. Copyright is held by the owner/author(s). Publication rights licensed to the VLDB Endowment. \\
\raggedright Proceedings of the VLDB Endowment, Vol. \vldbvolume, No. \vldbissue\ %
ISSN 2150-8097. \\
\href{https://doi.org/\vldbdoi}{doi:\vldbdoi} \\
}\addtocounter{footnote}{-1}\endgroup
%%% VLDB block end %%%

%%% do not modify the following VLDB block %%
%%% VLDB block start %%%
\ifdefempty{\vldbavailabilityurl}{}{
\vspace{.3cm}
\begingroup\small\noindent\raggedright\textbf{PVLDB Artifact Availability:}\\
The source code, data, and/or other artifacts are available at \url{https://github.com/linkedin/dataguard-udfs}.
\endgroup
}
%%% VLDB block end %%%

% 
% 
The widespread integration of communication networks and smart devices in modern control systems has increased the vulnerability of industrial systems to online cyber-attacks, e.g., Industroyer, Blackenergy, etc \citep{osti_1505628}.
% Modern control systems have seen a large push to include communication networks and smart devices to increase performance, made possible by improvements in communication device cost and energy consumption. This trend has been coupled with the usage of open-standard communication protocols among industrial control systems, making them vulnerable to online cyber-attacks such as Industroyer, Blackenergy, etc \citep{osti_1505628}. 
To counter this, methods have been developed to improve security by achieving attack detection, mitigation, and monitoring, among others \citep{sandberg2022secure}. This paper focuses on active attack diagnosis to mitigate stealthy attacks. 
%
%\subsection{Literature review}

Active diagnosis techniques rely on the inclusion of additional moduli to control systems
% inclusion within the control system of additional moduli 
to alter the behavior of the system compared to information known by the attacker. 
For instance, the concept of additive watermarking was introduced in \cite{mo2015physical}, where noise signals of known mean and variance are added at the plant and compensated for it at the controller. 
This compensation, however, is not exact, causing some performance degradation. Thus, trade-offs between performance and detectability  are necessary \citep{zhu2023detection}.
% A later work \citep{zhu2023detection} designs the watermark signal by trading performance for detection. Thus, although additive watermarking serves as a good detection scheme, they endure performance losses even in the nominal case. 

In encrypted control \citep{darup2021encrypted}, the sensor data is encrypted, sent to the controller, and then operated on directly. Encrypted input signals are sent back to the plant for decryption. Although encryption is widespread in IT security, in control systems it presents some concerns, such as the introduction of time delays \citep{stabile2024verifiable}, while it may present inherent weaknesses \citep{alisic2023model}.
% they are not preferred as they introduce time delays \citep{stabile2024verifiable} which can cause instability, and some encryption schemes can be very weak  \citep{alisic2023model}. 

In moving target defense \citep{griffioen2020moving}, the plant is augmented with fictitious dynamics, known to the controller. The plant output is transmitted to the controller along with the fictitious states over a network under attack. 
The additional measurements then aide in the detection of attacks. 
This comes at the cost of higher communication bandwidth needs, which increases rapidly with the dimension of the augmented systems.
% Since the dynamics of the fictitious dynamics are exactly known to the controller, the attack is detected easily. However, when the scale of the system increases, the communication bandwidth used by moving the target defense approach increases rapidly. 

Other recently proposed works include two-way coding \citep{fang2019two}, a weak encryuption technique, and dynamic masking \citep{abdalmoaty2023privacy}, which enhances privacy as well as security, have been shown to be effective against zero-dynamics attacks.
% Two-way coding \citep{fang2019two} and dynamic masking \citep{abdalmoaty2023privacy} are other recently proposed approaches. Two-way coding is another form of weak encryption technique whilst dynamic masking proposes an architecture that enhances both privacy and security. These schemes are shown to be effective against zero dynamics attacks but remain to be studied for other classes of attacks. 
% Recent extensions include \citep{mukherjee2021secure,ramos2024privacy}.
% Some other works which are related are \citep{mukherjee2021secure}, an extension of \cite{fang2019two}. The work \citep{ramos2024privacy} is an extension of moving target defense for multi-agent systems. 
Furthermore, filtering techniques for attack detection are proposed by \cite{murguia2020security,hashemi2022codesign,escudero2023safety}, while not focusing on stealthy attacks.
% The works \citep{murguia2020security,hashemi2022codesign,escudero2023safety} develop filtering techniques to guarantee safety, without being focused on stealthy covert attacks.

Multiplicative watermarking (mWM) has been proposed by the authors as a diagnosis technique \citep{ferrari2020switching}. mWM consists of a pair of filters on each communication channel between the plant and its controller; the scheme is affine to weak encryption, whereby ``encoding'' and ``decoding'' are done by changing signals' dynamic characteristics through inverse pairs of filters. This enables original signals to be recovered exactly, and thus does not lead to performance degradation.
% A multiplicative watermark is an affine to a weak encryption technique, through which the signal is ``encoded'' by a filter, changing its dynamic behavior. The use of inverse pairs means that the original signal can be recovered, through ``decoding'' via an inverse filter. As such, differently to techniques based on additive watermarking, no performance is lost due to the injection of noise, and there are no bandwidth limitations.

%\subsection{Contributions}
One of the critical features of multiplicative watermarking is that to detect stealthy attacks, the mWM filter parameters must be switched over time. In this paper, an algorithm to optimally design the mWM parameters after a switching event is presented, enhancing detection performance, without changing the switching time.
% This is done without changing the switching time, which is taken as given.

\textcolor{black}{
To formalize the filter design problem, we suppose the defender is interested in optimal performance against adversaries injecting covert attacks with matched system parameters \citep{smith2015covert}, including the mWM parameters prior to the switch. This scenario represents a worst case where malicious agents can take full control of the system while remaining undetected.
Thus, the attack strategy is explicitly included within the formulation of the closed-loop system, and the mWM filters are chosen by solving an optimization problem minimizing the attack-energy-constrained output-to-output gain (AEC-OOG) \citep{anand2023risk}, a variation of the output-to-output gain proposed in  \cite{teixeira2015strategic}.
}
The main contributions of this paper are:
% We consider an adversary injecting a covert attack with matched system parameters \citep{smith2015covert}, i.e., an attacker with full knowledge of the control system parameters, including those of the mWM filters before the switch. This scenario is taken as a worst case, as it has been shown that this class of attacks can be made stealthy. To quantitatively define a cost, the output-to-output gain (OOG) \citep{teixeira2015strategic} is leveraged,
% a metric introduced to evaluate the impact of an additive attack in a control system. %Specifically, OOG evaluates the worst-case performance loss that an attacker injecting an undetectable attack can obtain. 
% Here, the maximum performance loss caused by a stealthy adversary with limited energy is taken, the attack-energy-constrained OOG (AEC-OOG) \citep{anand2023risk}. The main contributions of this paper are:
\begin{enumerate}
%[label=\alph*.]
\item The problem of optimally designing the switching mWM filters is formulated as an optimization problem, with the AEC-OOG is taken as the objective;%where the AEC-OOG is taken as the impact metric; 
\item The worst-case scenario of a covert attack with exact knowledge of plant and mWM filter parameters is embedded within the design problem;
% The optimization problem is defined to incorporate the worst-case scenario of a covert attack with exact knowledge of plant and mWM filter parameters;
\item The feasibility of the optimization problem is shown to be dependent only on stability conditions; 
\item A solution scheme is proposed to promote randomization of the mWM filter parameters such that an eavesdropping adversary cannot remain stealthy.
\end{enumerate} 

This builds on the results of \cite{ferrari2020switching}, where the focus was on the design of the switching protocols, rather than the parameters themselves.
Compared to previous work \citep{gallo2021design}, this paper introduces an optimization problem which is always feasible (thanks to the use of AEC-OOG in the objective), while also considering a more sophisticated class of covert attacks, where the presence of watermark is known to the adversary. 
Moreover, this paper poses a different objective than \citep{zhang2023hybrid}; indeed, while \citep{zhang2023hybrid} provided a design strategy to ensure certain privacy properties, in this paper we address the problem of optimal parameter design following a switching event.


%\subsection{Organization}
The rest of the paper is organized as follows. 
After formulating the problem in Section~\ref{sec:PF}, we propose our design algorithm in Section~\ref{sec:main}, and analyze its properties. It is then evaluated through a numerical example in Section~\ref{sec:NE}, and concluding remarks are given Section~\ref{sec:Con}.
% We provide the problem background in Section~\ref{sec:PF}. We formulate the design problem in Section~\ref{sec:main}, together with an analysis of its properties. The proposed algorithm is evaluated through a numerical example in Section \ref{sec:NE}. Concluding remarks are offered in Section \ref{sec:Con}.
\section{Related Work}

\subsection{View-Dependent Control}
View-dependent representations have a long history in computer graphics.
In his pioneering work, Rademacher proposed interpolating between \textit{key viewpoints} and associated \textit{key deformations} to manipulate 3D models~\cite{rademacher1999view}.
Other researchers have extended the idea to create 3D animation systems~\cite{10.1111:j.1467-8659.2004.00772.x}, streamline the modeling process~\cite{DBLP:journals/corr/abs-2103-15472}, and integrate physical simulation\cite{koyama2013view}.
Of particular note, Rivers et al.~\cite{rivers25Dcartoonmodels} introduced \textit{2.5D Cartoon Models}, a combination of planar meshes transformed, based upon view angle, so as to appears three dimensional.
Our work draws upon these works but is, to our knowledge, the first work to attempt to use view-dependent techniques to retarget 3D motion onto 2D characters.   

\subsection{Animation from 2D Images}

% output is still 2D
Many researchers have proposed different methods for creating animations from 2D images. Hornung et al.~\cite{Hornung2007anim2Dpicmotion} presented a method to deform a character from a photograph given user-provided joint annotations.
\textit{Toonsynth}~\cite{Dvoroznak18-SIG} and \textit{Neural Puppet}~\cite{poursaeed2020neural} both present methods to create new images of hand-drawn characters from examples.
% output is 3D model
Other researchers have proposed methods of obtaining 3D geometry from 2D sketches~\cite{igarashi2006teddy, Dvoroznak20-SA} and images~\cite{ArtiSketch,weng2019photo}.
% focus on sketches specifically
A number of works have specifically focused on childlike drawings.
Lingens et al.~\cite{lingens2020towards} proposed an evolutionary algorithm for animating children's drawings. 
\textit{MagicToon}~\cite{feng2017magictoon} creates a 3D model from childlike drawings for AR applications.
Similar to our work, Smith et al.~\cite{SmithHodgins} proposed a method for animating childlike drawings using 3D skeletal motion. 
However, the resulting animations are only suitable for use in 2D applications and the type of motions it supports are limited.

Unlike these previous works, our solution can be used in 3D contexts but does not create a 3D model. We instead relying upon a view-dependent formulation of the animated character.
\section{Architecture Overview} \label{sec:overview}
\begin{figure}[ht]  % 'ht' means here or at the top
\vspace{-0.4em}
\includegraphics[width=0.5\textwidth]{figures/SystemOverview.jpg} 
\caption{Data Guard System Architecture.}  % Adding a caption
\label{fig:SystemOverview}  % Adding a label for referencing
\vspace{-1em}
\end{figure}
Figure~\ref{fig:SystemOverview} shows the overall system architecture of Data Guard. 
Data producers provide semantic descriptions of data in the form of {\em policy labels} that are assigned to {\em fields} in a table. 
{\em Fields} refer to data elements in a table and represent the unit at which data masking is applied. 
Thus, a field can reference a column, a row, a cell or portion of data within a cell. 
The mapping of policy labels to fields is stored in a {\em Data Catalog} as shown in Figure~\ref{fig:SystemOverview}. 
Policies in Data Guard are authored in a domain-specific language (DSL) and stored in a {\em Policy Catalog} which provides APIs to query policies from the catalog. 
Each policy is assigned to a purpose and policy label pair. A policy definition has an associated boolean condition that determines whether a data element with the corresponding policy label is masked or preserved during data access.       
Figure~\ref{fig:SystemOverview} shows an example policy that is assigned to $ads$ purpose and $employer$ label. 
The {\em Policy Compiler} consumes the policy label-field mapping and policy definitions as inputs and generates purpose-specific static views which are registered with the data warehouse. 
User queries submitted via Spark or Trino that access tables (e.g. $T_1$ and $T_2$ in Figure~\ref{fig:SystemOverview}) are transparently redirected by a routing module called {\em ViewShift} to the appropriate data-masking views ($jobs.V_1$ and $ads.V_2$ respectively in Figure~\ref{fig:SystemOverview}) based on the accessor's purpose (i.e $jobs$ and $ads$ in Figure~\ref{fig:SystemOverview}). 
We next describe the access control model and system implementation details of Data Guard architecture.

\section{Access Control Model}
In this section, we introduce the main abstractions that enable purpose-limitation and fine-grained access control capabilities of Data Guard: (i) data access policies, (ii) data-masking views, and (iii) field paths. 

\subsection{Data Access Policies}
A data access policy in Data Guard specifies: (i) a {\em purpose} under which the policy is applicable, (ii) a {\em label} that determines the data elements to which the policy applies, and (iii) a set of access control {\em rules} that specify the conditions under which access to a data element with a given label is allowed or denied. We next introduce each of these concepts followed by a formal definition of a data access policy. 
 
\subsubsection{Purpose}
The access control model described in this paper is motivated by the {\em purpose limitation} principle introduced in the GDPR regulation~\cite{GDPR16}. Purpose limitation requires that data is collected from {\em data subjects} (i.e. individuals whose personal data is collected) for an explicitly declared purpose and is only processed for use cases compatible with the purpose under which it is collected. Further, usage of an individual's data for newer purposes is driven based on consents from the individual. At LinkedIn, a purpose is used to indicate the business justification for data collection or data processing. Purposes can either map to an external product such as advertising and jobs, or correspond to internal use cases such as security and business analytics. 

% Data Guard organizes purposes into a directed acyclic graph where a directed edge from purpose $A$ to purpose $B$ indicates that purpose $B$ is a child of parent purpose $A$.  A child purpose can have multiple parent purposes and each child purpose inherits policies assigned to each of its parent purposes. In case of a parent-child policy conflict, the more restrictive policy is given precedence. Organizing purposes into a hierarchy and enabling policy inheritance minimizes policy duplication across related purposes. 
At LinkedIn, purposes are organized as a directed acyclic graph which enables policy inheritance across purposes and minimizes policy duplication. For simplicity of discussion however, we assume henceforth that purposes are arbitrary string literals (such as {\em ads} and {\em jobs}) for the remainder of the paper. 
%Data Guard organizes purposes into a directed acyclic graph where a directed edge from purpose $A$ to purpose $B$ indicates that purpose $B$ is a child of parent purpose $A$.  A child purpose can have multiple parent purposes and each child purpose inherits policies assigned to each of its parent purposes. In case of a parent-child policy conflict, the more restrictive policy is given precedence. Organizing purposes into a hierarchy and enabling policy inheritance minimizes policy duplication across related purposes. 

Purpose limitation is ensured by requiring each data access to specify a valid purpose. At LinkedIn, all production workloads access warehouse data using service accounts. Similar to role-based access control systems where roles are assigned to a user, we assign a purpose to each service account. The purpose of the service account determines the access control policies that are enforced by Data Guard on the accessed tables. 

\subsubsection{Policy Labels}
Data producers provide semantic description of data in the tables they produce using {\em policy labels}. Figure~\ref{fig:SystemOverview} shows a data producer assigning policy labels $employer$ and $education$ assigned to $col_1$ and $col_2$ of table $T_1$. In general, data producers can assign policy labels either to:
(i) a table, (ii) column in a table (either a top-level or a nested column), (iii) cells, or (iv) sub-cell-level data elements within arrays, structs and maps. As we will see in Section~\ref{subsec:fieldpaths}, this fine-grained assignment of policy labels allows Data Guard to apply data masking at much finer granularities than existing solutions. 

Figure~\ref{fig:SystemOverview} also shows that policy labels are stored as metadata inside a {\em Data Catalog}. This design choice is motivated by recent trends in data architecture where data catalogs such as Horizon~\cite{Horizon}, Unity~\cite{Unity} and DataHub~\cite{Datahub} have emerged as the foundational infrastructure component to address data discovery and data governance needs of large organizations. These catalogs serve as the inventory of the critical data assets (e.g. warehouse tables) of an organization and store a variety of metadata such as schemas, lineages and ownership associated with these assets. These catalogs are therefore well-suited for storing compliance metadata such as {\em policy labels}. 

At LinkedIn, policy labels are carefully curated and organized into a hierarchical structure referred to as {\em ontology}. An ontology-based organization not only classifies data in our ecosystem, but also captures relationships between these data classes. Ontology also supports advanced use cases at LinkedIn related to reasoning and inference on compliance metadata. A detailed discussion on the design of our ontology is beyond the scope of this paper. We note however that the access control model proposed in this paper does not require an ontology-based organization of policy labels. For simplicity, we therefore assume that policy labels are arbitrary string values for the remainder of the paper. 

Policy labels link policies to the data elements (e.g. tables, columns, and cells) to which policies apply. The cardinality of policy labels is typically much smaller than the number of data assets and fields across all tables in a warehouse. Defining policies using policy labels significantly reduces policy duplication and allows policies to be consistently applied across tables in a warehouse as well as across non-warehouse data assets such as online data stores and data streams. Further, it automatically ensures compliance of newly created tables containing data elements tagged with one or more previously defined labels. 

\subsubsection{Policy Language} \label{subsubsec:policies}
Data masking in Data Guard is controlled using rules that are assigned to a purpose and policy label pair. 
\begin{definition} \label{def:policy}
Let $\mathcal{P}$ denote the set of purposes and $\mathcal{L}$ denote the set of policy labels. A policy in Data Guard is a tuple $(p,l, \langle c, a \rangle)$ where $p \in \mathcal{P}$ and $l \in \mathcal{L}$. $\langle c, a \rangle$ is a condition and action pair (or, a rule) where:
\begin{itemize}
\item $c$ is a compound SQL predicate composed of simple predicates connected using the boolean operators $\lbrace AND, OR, NOT \rbrace$. Each simple predicate in $c$ is of the form $(x \circ y)$ where:
\begin{enumerate}
\item where $x \in \mathcal{X}$ is a set of attributes (defined below),
\item $y \in dom(x)$ is the domain of values of attribute $x$.
\item $\circ$ represents a relational operator from the set of relational operators $\lbrace =, \neq, <, >, \leq, \geq, BETWEEN, IN, \\ LIKE, IS NULL \rbrace$.
\end{enumerate}
\item $a \in \mathcal{A}$ represents an action from the set of actions $\mathcal{A}$ that the access control system must take when $c$ is true.
\end{itemize} 
\end{definition}
Thus, the example policy shown in Figure~\ref{fig:SystemOverview} can be represented as a tuple: $(ads, employer, \langle allowEmplForAds=true, KEEP \rangle)$.
 
The set $\mathcal{X}$ of attributes over which policy conditions are defined is a union of data subject attributes (e.g. opt-in/opt-out consents, user location), data accessor attributes (e.g. accessor's role and location), and system attributes (e.g. availability/security zone where the system is located in). The set of actions $\mathcal{A}$ of interest in this paper is the set $\lbrace KEEP, MASK \rbrace$, where the action $KEEP$ allows access to the target data while $MASK$ redacts the data. For simplicity, we only consider a specific form of data redaction in this paper, where the target data element is either replaced with $NULL$ value or the tuple containing the target data element is filtered from the result set. Assignment of non-null replacement values to target data elements is an area of ongoing work and will be discussed in greater detail in Section~\ref{sec:considerations}. 

At LinkedIn, policies are managed like source code and policy changes are manually reviewed before they are committed to a code repository. Further, we run a number of validation checks on policies to avoid duplication and ensure that policies can be successfully parsed and compiled by the policy compiler. The changes committed to the repository are pushed to a {\em Policy Catalog}  which provides APIs to query policies by purpose and policy labels.

Without loss of generality, all policies considered in the remainder of the paper are $KEEP$ policies i.e. the data element referenced by a policy label is preserved if the policy condition evaluates to true and masked otherwise. We further restrict our attention to policies based on data subject consents for the remainder of this paper. In doing so, we aim to highlight:
\begin{enumerate}
\item the scalability challenges that need to be addressed to enforce consents of a large population of data subjects during data accesses, and 
\item the unique challenges such policies introduce for enabling fine-grained data masking. 
\end{enumerate}

\subsection{Data-Masking Views} \label{subsec:views}
Data Guard uses views as the interface for access control. Each data-masking view is a SQL representation of applicable data policies on a table for a given purpose. The applicable policies for a given purpose are found by matching policy labels assigned to fields in a table against the labels assigned to policies. The matched policies are then translated to SQL by the {\em Policy Compiler}, the implementation of which will be discussed in detail in Section~\ref{subsubsec:pc}. 

We illustrate an example data-masking view created by Data Guard in Figure~\ref{fig:SystemOverview}. 
Let us assume that the table $T_1$ shown in the figure is keyed by an $id$ column containing data subject identifiers. 
The figure also shows that $T_1$ contains a column $col_2$ that has $education$ data with an applicable policy for the $ads$ purpose. 
For simplicity, let us assume that all data subject attributes referenced in policy definitions are stored in the $member\_settings$ table which is also keyed by the data subject identifier. 
A naively implemented policy compiler generates the following view SQL for $ads$ purpose which masks $col_2$ based on the value of the attribute $allowGenderForAds$ in the $member\_settings$ table:
% Define SQL code style
\lstdefinestyle{sqlstyle}{
    language=SQL,
    basicstyle=\ttfamily\footnotesize,
    keywordstyle=\color{blue},
    commentstyle=\color{gray},
    stringstyle=\color{orange},
    showstringspaces=false,
    morekeywords={SELECT, FROM, WHERE, AND, OR, INSERT, INTO, UPDATE, DELETE}, % Add SQL keywords here
    frame=single, % Add a border around the code
    breaklines=true, % Automatically break long lines
}
\lstset{style=sqlstyle}
% Example SQL code
\begin{figure}[ht] 
\vspace{-1em}
\begin{lstlisting} 
SELECT T1.id, T1.col1, CASE WHEN allowGenderForAds = true THEN T1.col2 ELSE NULL END AS col2
FROM T1 JOIN member_settings T2 ON T1.id = T2.id;
\end{lstlisting}
\vspace{-1em}
 \caption{A data-masking view for $ads$ purpose}
\label{fig:viewsql}
\vspace{-1em}
\end{figure}

The view SQL shown in Figure~\ref{fig:viewsql}, while inefficient, has the important property that it is {\em schema-preserving} i.e. it has the same schema as the underlying table $T_1$. This invariant is maintained across all views created by Data Guard. The schema-preserving property ensures that applications can switch consumption from tables to views without needing to make code changes. Each view generated by Data Guard is registered within a purpose-specific database (e.g. $ads.V_1$ in Figure~\ref{fig:SystemOverview}), which facilitates search and discovery of views in the data catalog. In Section~\ref{sec:impl}, we discuss a number of optimizations which will be used to rewrite the view SQL in Figure~\ref{fig:viewsql} in order to make it performant.

The decision to use SQL views as the interface for access control was motivated by the following reasons: 
\begin{itemize}
\item {\em Portability}:  Views are engine-agnostic and work out of the box with engines like Spark and Trino.
\item {\em Debuggability}: Views make masking logic visible to end users and allow consumers to reason about {\em what} data is being filtered and {\em why}.
\item {\em Version control}: Views can be versioned in the same manner as software artifacts. Thus, changes to view logic (due to policy and label changes) can be tested by data consumers before they are deployed in production. 
\item {\em Agility:} Implementing changes to policy compiler is much faster than making changes to the compute engine code, which caters to a much larger set of use cases beyond access control.
\item {\em Optimizations:} A dedicated policy compilation layer provides the flexibility of implementing custom optimizations (e.g. bitmap optimization discussed in Section~\ref{subsubsec:bitmap}) when generating view SQL. Such use-case specific optimizations are otherwise non-trivial to implement inside general-purpose compute engines. 
\end{itemize}
As shown in Figure~\ref{fig:SystemOverview}, accesses to tables are routed to an appropriate masking view based on the purpose of access using a component called {\em ViewShift}.  ViewShift along with the schema-preserving property of the masking views allows applications to seamlessly switch consumption from tables to views without incurring significant migration costs. 

%We note an interesting parallel between our system and object-oriented programming. Raw data in tables can be seen as analogous to private data members of an object encapsulated inside data-masking views which constitute the public interface, while the routing of table accesses to views is similar to the concept of dynamic method dispatching central to runtime polymorphism implemented in languages like Java and C++.

\subsection{Field Paths} \label{subsec:fieldpaths}
\subsubsection{Motivation}
Existing solutions~\cite{Stonebraker74, Rizvi04,Xue23, Agrawal05} support data masking at row, column and cell-level granularity. These masking operations are typically accomplished using traditional projection and selection operators. 
%\begin{figure}[htbp]
%    \centering
 %   \includegraphics[width=0.5\textwidth]{figures/relation_slice.png}  % Replace with your image file
%    \caption{User data presented in a relation}
 %   \label{fig:relation_slice}  % This is the label for referencing the figure
%\end{figure}
As mentioned in Section~\ref{intro}, Data Guard's access control model supports masking data at much finer granularities than is possible with previous solutions. This need for finer grained data masking is motivated by prevalence of non-relational data in LinkedIn's warehouse (and indeed, many modern data warehouses~\cite{Snowflake} and lakehouses~\cite{Databricks}). It is common to model data using relation-valued attributes and {\em collection} types such as arrays and maps. Masking data selectively inside such data structures cannot be done using relational operators such as projection and selection alone.  

Let us consider an example relation $\mathcal{R}$ with the following schema:
\begin{tabbing}
    \hspace{1cm} \= \hspace{5cm} \= \kill
    \textbf{$col_1$}: \> \texttt{VARCHAR} \\
    \textbf{$col_2$}: \> \texttt{STRUCT<$field_{21}$:BIGINT, $field_{22}$:VARCHAR>} \\
    \textbf{$col_3$}: \> \texttt{ARRAY<STRUCT<$field_{31}$:VARCHAR, $field_{32}$:DOUBLE>>} \\
    \textbf{$col_4$}: \> \texttt{MAP<VARCHAR, ARRAY<STRUCT<$field_{41}$:VARCHAR, } \\
                    \> \texttt{$field_{42}$:BOOLEAN>>>}
\end{tabbing}
Figure~\ref{fig:relationexample} shows an instance of $\mathcal{R}$. 
\definecolor{gray1}{gray}{0.9}
\definecolor{gray2}{gray}{0.8}
\definecolor{gray3}{gray}{0.7}
\definecolor{gray4}{gray}{0.6}
\definecolor{gray5}{gray}{0.5}
\begin{figure}[htbp]
\vspace{-1em}
 \resizebox{0.5\textwidth}{!}{
\begin{tabular}{|c|c|c|c|c|c|c|c|}
    \hline
    \multirow{2}{*}{\textbf{$col_1$}} & \multicolumn{2}{c|}{\textbf{$col_2$}} & {\textbf{$col_3$}} & {\textbf{$col_4$}} \\ \cline{2-3}
    & \textbf{$field_{21}$} & \textbf{$field_{22}$} &   &  \\ \hline
    abc \cellcolor{gray1} & 123 & foo & 
    \begin{tabular}{|c|c|}
        \hline
        \textbf{$field_{31}$} & \textbf{$field_{32}$} \\ \hline
        s1 \cellcolor{gray4} & \cellcolor{gray4} 113.2 \\ \hline
        % s2 & 123.3 \\ \hline
    \end{tabular} &  
    NULL \\ \hline
    def \cellcolor{gray2} \cellcolor{gray2} & 243 \cellcolor{gray2}  & bar \cellcolor{gray2}  & NULL \cellcolor{gray2}  & NULL  \cellcolor{gray2} \\ \hline
    ghj \cellcolor{gray1} & 123 \cellcolor{gray3} &  bar \cellcolor{gray3} & 
    \begin{tabular}{|c|c|}
        \hline
        \textbf{$field_{31}$} & \textbf{$field_{32}$} \\ \hline
        s1 \cellcolor{gray4} & 345.2 \cellcolor{gray4} \\ \hline
        s3 & 212.0 \\ \hline
    \end{tabular} & 
    \begin{tabular}{|c|c|c}
        \hline
        \textbf{key} & \textbf{value} \\ \hline
        % \multirow{2}{*}{\textbf{key}} & \multicolumn{2}{c|}{\textbf(value)} \\ \cline{2-3}
        % & \textbf{$field_{41}$} & \textbf{$field_{42}$} \\ \hline
        k1 & \begin{tabular}{|c|c|}
            \hline
            \textbf{$field_{41}$} & \textbf{$field_{42}$} \\ \hline
            v1 & true \\ \hline
            v2 & false \cellcolor{gray5} \\ \hline
        \end{tabular} \\ \hline
        k2 & NULL \\ \hline
    \end{tabular} \\ \hline
\end{tabular}
\vspace{1em} % Space between the tables
\begin{tabular}{|c|l|}
\hline
\cellcolor{gray1} &  $\$.col_1$ \\
\hline
\cellcolor{gray2} & $\$.[?(@.col_1='def')]$ \\
\hline
\cellcolor{gray3} &  $\$.[?(@.col_1='ghj')].col_2$ \\
\hline
\cellcolor{gray4} & $\$.col_3.[item].[?(@.field_{31}='s1')]$  \\
\hline
\cellcolor{gray5} & $\$.col_4.[value].[item].[?(@.field_{41}='v2')].field_{42}$  \\
\hline
\end{tabular}
}
\caption{Example of a nested relation. Data Guard's field path expressions allow masking of data inside nested attributes such as structs, arrays and maps. 
%The shaded cells with different gray scales in the figure represent the following paths, from light to dark respectively: \\
%$\$.col_1$ \\
%$\$.[?(@.col_1='def')]$ \\
%$\$.[?(@.col_1='ghj')].col_2$ \\
%$\$.col_3.[item].[?(@.field_{31}='s1')]$ \\
%$\$.col_4.[value].[item].[?(@.field_{41}='v2')].field_{42}$ \\
}
\label{fig:relationexample}
\vspace{-1em}
\end{figure}

In order to mask data at a sub-cell granularity (e.g. $field_{31}$, $field_{42}$ shown in Figure~\ref{fig:relationexample}), we need an operator for selecting this data. There have been prior proposals~\cite{Colby90} which introduce recursive selection and projection operators, which can be used to select and mask data at sub-cell granularity. While there have been efforts to extend SQL language to support recursive operations~\cite{sql1999}, very few commercial and open-source systems support such extensions. While introducing these extensions to existing open source systems like Spark and Trino is in theory possible, it would require a significant modification of the SQL standard adopted by these systems and a non-trivial investment of effort to drive alignment across the open-source community. On the other hand, there have been numerous path DSLs such as XMLPath \cite{XPath} and JsonPath \cite{rfc9535} that have been developed and successfully adopted to support element-wise operators. Given these trade-offs, using a DSL to evaluate field paths efficiently during data access was a cost-effective alternative. Leveraging existing DSLs such as XPath and JsonPath was not an option due to several limitations that prevent their usage for structured data handling. XPath is intended exclusively for processing XML documents. JsonPath is schema-unaware and does not distinguish between types such as maps and structs. Nevertheless, we use JsonPath to guide the design of the field path DSL described next. 

\subsubsection{Field Path Expressions}
In this section, we provide a formal definition of {\em field path} that is used to select data at a sub-cell-level for non-atomic data types such as structs, arrays and maps. 
\begin{definition} \label{def:fieldpath}
A \emph{field path} is a sequence of operators $P_1, P_2, \ldots, P_n$, where each operator $P_i$ operates on the result set of $P_{i-1}$ and has one of the following types:
\begin{enumerate}
\item {\bf Root access operator}: Denoted by a special symbol $\$$, this operator is an identity operator whose result set is the input relation. The root access operator is always the first operator in the sequence of operators for any given field path.  
\item {\bf Transform operator}:  A transform operator is immediately preceded by either a root access or another transform operator and is prefixed with a $.$ symbol. There are three types of transform operators:
    \begin{enumerate}
        \item {\bf Dereference operator}: This operator is of the form $.<name>$ and projects a sub-attribute of a composite attribute. 
        \item {\bf Filter operator}: This operator selects a subset of a relation that satisfies a given condition. The operator is of the form $[?(<condition>)]$, where condition is a SQL predicate string on the input relation.
        \item {\bf Unnest operator}: a cell-wise transform operator that applies all subsequent operators on each cell of inner relations in collection-type attributes such as arrays and maps. There are three forms of unnest operators:
        \begin{enumerate}
            \item $[item]$: an operator that applies to array types and provides access to the array items, 
            \item $[key]$: an operator that applies to map types and provides access to the map keys, and 
            \item $[value]$: an operator that applies to map types and provides access to the map values.
        \end{enumerate}
    \end{enumerate}
\end{enumerate}
\end{definition}

\subsubsection{Examples}
We next provide examples of field path expressions that demonstrate their ability to select data elements from the different attributes of the relation shown in Figure~\ref{fig:relationexample}.
\begin{enumerate} 
\item A dereference operator following a root access operator can be used to select an attribute from a relation. For example, $\$.col_1$ selects $col_1$.  
\item A filter operator following a root access operator is referred to as a {\em row selector} and has the ability to select rows from a relation. For example, the field path $\$.[?(@.col_1='def')]$ selects rows from the relation satisfying $col_1 = 'def'$.  
\item A row selector followed by a dereference operator selects cell values of an attribute from the selected rows. For example, the field path $\$.[?(@.col_1='ghj')].col_2$ selects values of $col_2$ from rows satisfying the condition $col_1='ghj'$. 
\item A filter operator following an unnest operator on a collection type field (e.g. arrays and maps) filters elements of the collection. For example, the field path $\$.col_3.[item].[?(@.field_{31}='s1')]$ selects elements of $col_3$ such that each element satisfies the condition $field_{31} = 's1'$.  Similarly, the field path $\$.col_4.[value].[item].[?(@.field_{41}='v2')].field_{42}$ selects values of the field $field_{42}$ from the collection of map values satisfying $field_{41}='v2'$. 
\end{enumerate}
In summary, field path expressions allow data selection at much finer granularities than is possible using traditional selection and projection operators available in commercial databases.

In order to mask data addressed by a field path expression, we need the ability to assign policy labels to them. Thus, the field paths associated with a relation are stored with the schema of the relation in the data catalog. Some field paths are automatically extracted from the schema when a schema is ingested into the data catalog. Additional field paths such as row selectors are added to the catalog by owners of the schema. 

\subsubsection{Data-Masking Operator} \label{subsubsec:masking_operator}
We next describe how the field path expressions are used to define data masking operations on a data element. Given an input relation $\mathcal{R}$, a data access policy $p$, and a field path $f$, the data-masking operator is a relational algebra function denoted by $mask(\mathcal{R}, p, f)$ and defined as follows:
\begin{equation}
    mask(\mathcal{R}, p, f) = 
    \begin{cases}
    \sigma_{\neg\text{pred OR p.cond}}(\mathcal{R}), \text{if $f=\$.[?(pred)]$} \\
    {\tau_{f \rightarrow m(f)} (\sigma_{\neg\text{p.cond}}(\mathcal{R})))} \cup \sigma_{\text{p.cond}}(\mathcal{R}), \text{otherwise}
    \end{cases}  
\label{eq:enforcement}
\end{equation}
where $\tau$ denotes a transformation operator that masks the data element at path $f$ by applying a masking function $m$ and reassembles the transformed attribute (potentially, a nested relation) back into its original structure. 
Equation~\ref{eq:enforcement} shows that when $f$ does not contain a dereference operator, the masking operator reduces to a {\em row-level mask} and removes the matching rows from the result set. 
Otherwise, the transformation operator functions as a {\em column-level mask} applying the function $m$ to values of $f$. The masking function $m$ in Equation~\ref{eq:enforcement} has the following behavior:
\begin{itemize}
    \item If $f$ is an atomic attribute, $m(f)$ sets the value of $f$ to NULL.
    \item If $f$ is an array element, $m(f)$ removes the element from the array.
    \item If $f$ is a map key, $m(f)$ removes the key value pair from the map,
    \item If $f$ is a map value, $m(f)$ sets the map value to NULL.
\end{itemize}
In summary, field paths and data masking operators give us the capability of masking data at a very fine granularity. The exact implementation of the data-masking operator will be discussed in detail in Section~\ref{subsubsec:pc}. 

%\section{TerEffic Architecture design}
\label{sec:archi}
We design a ternary LLM accelerator architecture on FPGA to showcase the benefits of the aforementioned optimizations. We choose the MatMul-free LM model~\cite{scalable} as the representative ternary LLM for our design. Although \cite{scalable} also proposed an FPGA design, it was a simple and inefficient prototype as their main focus was on the model architecture. 

\begin{figure}[h]
    \centering
    \vspace{-3mm}
    \includegraphics[width=.9\linewidth]{figures/arch.pdf}
    \caption{TerEffic Hardware Architecture}
    \label{fig:model}
    \vspace{-6mm}
\end{figure}

\subsection{Architecture Overview}

TerEffic overall hardware architecture is shown in Fig.\ref{fig:model}, along with its two major components, the HGRN\cite{HGRN} and the GLU\cite{glu} module. HGRN is a powerful RNN-based alternative to the self-attention mechanism\cite{attention}, while GLU, widely utilized in models like Llama\cite{llama}, serves as a robust enhancement for feed-forward networks (FFNs).
In the model, addition, subtraction, and dot product (Dot) can be directly implemented using LUTs and DSPs, while the sigmoid function is approximated by piecewise linear equations~\cite{sigmoid}. Consequently, the primary implementation challenge lies in the BitLinear module consisting of a Root-Mean-Square Normalization (RMSNorm) module and a ternary MatMul module.
\subsubsection{RMSNorm module}
The RMSNorm\cite{RMSNorm} is more computationally efficient than the traditional LayerNorm\cite{attention} but maintains high accuracy, making it well-suited for FPGA. The algorithm for RMSNorm is presented below:
\vspace{-1mm}
\begin{equation}
        r = \sqrt{\frac{1}{d} \sum_{i=1}^d x_i^2 + \epsilon} \quad , \quad 
        \text{RMSNorm}(X) = \frac{X\odot W_n}{r}
\label{eq:RMSNorm}
\end{equation}
\vspace{-1mm}

where $X\in\mathbb{R}^{1 \times d}$ denotes the input activation, $W_n\in\mathbb{R}^{1 \times d}$ denotes the normalization weight, $r$ denotes the RMS result and $\epsilon$ is a small constant. As shown in the architecture in Fig. \ref{fig:RMSNorm}, the computation of r and $X\odot W_n$ can be executed in parallel. As the former has longer latency, the results of $X\odot W_n$ are temporarily stored in on-chip buffers. These results are then repeatedly retrieved from the buffers to match the multiple iterations required by the subsequent ternary MatMul modules. Moreover, as divisions incur high hardware cost and long cycle latency, we replace the divisions ($\div r$) with DSP-based multiplications ($\times \frac{1}{r}$), using r as an index to retrieve 1/r from an on-chip look-up table consisting of a small amount of SRAM. This method saves hardware resources and significantly increases maximum frequency.

\subsubsection{Ternary MatMul Module}
The Ternary MatMul Module serves as the primary computational core executing $X\times W$, where $X\in\mathbb{R}^{1 \times d}$ denotes the normalized activation and $W\in\mathbb{R}^{d \times d}$ denotes the ternary weight matrix. The module consists of submodules that perform the dot product between $X$ and one column of $W$. Each submodule is made up of TMUs for ternary MatMuls and a reduction tree to aggregate the TMU outputs. A detailed architecture is shown in Fig.\ref{fig:TMM}. To address the memory bandwidth limitations, we partition $X$ and each column of $W$ into x groups. Each submodule, which contains $\frac{d}{x}$ TMUs, computes the MatMuls for one group. Moreover, due to the limited computational capacity, the d columns of $W$ are divided into y sets and processed sequentially by $\frac{d}{y}$ submodules. The values of x and y are set based on the on-chip memory bandwidth and computing capacity, optimized by the alignment strategy discussed in Section \ref{sec:Compute-Memory Alignment}.


\begin{figure}[h]
    \vspace{-5mm}
    \centering
    \begin{subfigure}[b]{0.4\textwidth} 
        \centering
        \includegraphics[width=\textwidth]{figures/RMSNorm.pdf} 
        \caption{RMSNorm Module}
        \label{fig:RMSNorm}
    \end{subfigure} 
    \begin{subfigure}[b]{0.4\textwidth} 
        \centering
        \includegraphics[width=\textwidth]{figures/TernaryMatMulModule.pdf}  
        \caption{Ternary MatMul Module}
        \label{fig:TMM}
    \end{subfigure}
    \vspace{-2mm}
    \caption{BitLinear Module}
    \label{fig:bitlinear module}
    \vspace{-6mm}
\end{figure}


\subsection{Architecture Variants}
We propose two memory architectures--one aims to store all weights in on-chip memory, and another utilizes an HBM. Since a single card is insufficient to hold all the weights of a moderate-sized model, the fully-on-chip architecture may need to leverage multiple cards. For larger models, we switch to HBM-assisted architecture that requires only one card for weight storage. We explore different parallelism methods in both architectures to achieve higher energy efficiency. Though the methods differ from the batching method for GPU, we use terms like 'multi-batch' and 'batch size' for convenience.
\subsubsection{TerEffic On-Chip Architecture}
\label{sec:On-Chip Architecture}
The on-chip architecture corresponds to the orange line in Fig. \ref{fig:Alignment}, where we store all weights on-chip and benefit from the massive on-chip bandwidth.
The basic design is based on the 24-layer, 370M model from~\cite{scalable}. Each layer is segmented into 10 main stages, as shown in Fig. \ref{fig:stages} (Norm denotes RMSNorm and TMM denotes ternary MatMul), with the latency of each stage indicated in the red row. Through the optimizations discussed before, the latency per layer is 820 cycles, resulting in a total latency of $\approx$  20,000 cycles across all 24 layers.

After 1.6-Bit Weight Compression, the 370M model occupies $\approx$ 57.6MB of memory, exceeding the on-chip capacity of 42MB (see TABLE \ref{tab:SRAM}). Thus, a 2-card system is required, with each card holding the weights for 12 layers and processing them accordingly, as shown in Fig. \ref{fig:multi-card}. As the model size increases, more cards become necessary—for instance, the 1.3B model requires eight cards each storing three layers.

Though this design delivers the shortest latency, the power efficiency is relatively low as only a portion of the resources is utilized at any point in time. To better harness parallelism, we construct a \textbf{pipeline}. A simple 2-stage example is illustrated in Fig. \ref{fig:pipeline}, where a batch size of two is processed concurrently by different TMUs.
As shown by the purple row in Fig. \ref{fig:stages}, each pipeline stage takes 160 cycles, resulting in a single-layer latency of 1600 cycles. With each card handling up to a batch size of 10, the 2-card system can process a batch size of 20, boosting throughput by $10\times$ over the basic design.

\begin{figure}
    \centering
    \includegraphics[width=\linewidth]{figures/Stages.pdf}
    \caption{Stage latency of a layer for on-chip single-batch, on-chip multi-batch (with pipeline parallelism), and HBM-assisted multi-batch (with full-resource parallelism)}
    \label{fig:stages}
    \vspace{-4mm}
\end{figure}


\begin{figure}
    \centering
    \begin{subfigure}[b]{0.23\textwidth} 
        \centering
        \includegraphics[width=\textwidth]{figures/Multi-Card.pdf} 
        \caption{On-Chip Architecture}
        \label{fig:multi-card}
        \vspace{-1mm}
    \end{subfigure} 
    \begin{subfigure}[b]{0.23\textwidth} 
        \centering
        \includegraphics[width=\textwidth]{figures/HBM_version.pdf}  
        \caption{HBM-Assisted Architecture}
        \vspace{-1mm}
        \label{fig:HBM}
    \end{subfigure}
    \caption{TerEffic Architecture Variants}
    \label{fig:single batch}
    \vspace{-6mm}
\end{figure}
\subsubsection{TerEffic  HBM-Assisted Architecture}
\label{sec:HBM-Assisted Architecture}
For larger models, an HBM-assisted architecture (Fig.\ref{fig:HBM}) is required. Xilinx U280 FPGA features 8GB HBM with 32 channels, providing a total maximum bandwidth of 8Kb. As discussed in Section \ref{sec:Compute-Memory Alignment}, the HBM's bandwidth is much lower than on-chip memory, resulting in a longer latency of about 2,400 cycles/layer. 

For this architecture, the pipeline parallelism is not suitable, as the latency after pipelining is shorter than the data-fetch latency from HBM. Moreover, pipelining requires two layers of weights stored on-chip in a ping-pong buffer, which is infeasible for large models. Hence, we introduce \textbf{full-resource parallelism} with a simplified example (2 stages, batch size=2) presented in Fig. \ref{fig:parallelism}. Unlike in the previous pipeline, all the TMUs are utilized for processing each stage. As the batch size increases, the TMUs available for each input decrease and the latency increases. We set the batch size to 15 to accommodate 2,400 cycle HBM data-fetch latency. The detailed latency is shown in the orange row in Fig. \ref{fig:stages}, where the latency of each TMM stage is proportional to the computational workload.


\begin{figure}[t]
    \vspace{-3mm}
    \centering
    \begin{subfigure}[b]{0.15\textwidth} 
        \centering
        \includegraphics[width=\textwidth]{figures/Pipeline_Parallelism.pdf} 
        \caption{Pipeline}
        \vspace{-2mm}
        \label{fig:pipeline}
    \end{subfigure}
    \hspace{10mm}  
    \begin{subfigure}[b]{0.15\textwidth} 
        \centering
        \includegraphics[width=\textwidth]{figures/Full-Resource_Parallelism.pdf}  
        \caption{Full-Resource}
        \vspace{-2mm}
        \label{fig:Full-Resource}
    \end{subfigure}
    \caption{Parallelism Methods}
    \label{fig:parallelism}
    \vspace{-5mm}
\end{figure}

\section{Implementation}

We implement the proposed On-device Sora on iPhone 15 Pro~\cite{apple2023}, leveraging its GPU of 2.15 TFLOPS and 3.3 GB of available memory, with the two methods proposed in Sec. \ref{sec:ours1} and \ref{sec:ours2}. In addition, to execute large video generative models (\ie, T5 \cite{raffel2020exploring} and STDiT \cite{opensora}) with the limited device memory, we devise and implement Concurrent Inference with Dynamic Loading (CI-DL), which partitions the models into smaller blocks that can be loaded into the memory and executed concurrently. The details of CI-DL is described in \Cref{sec:ours3}. The model components—T5~\cite{raffel2020exploring}, STDiT~\cite{opensora}, and VAE~\cite{doersch2016tutorial}—in PyTorch~\cite{paszke2019pytorch} are converted to MLPackage, an Apple’s CoreML framework~\cite{sahin2021introduction} for machine learning apps. Since current version of CoreML \cite{apple2023} lacks support for certain diffusion-related operations in text-to-video generation, we develop custom solutions like xFormer \cite{xFormers2022} and cache-based acceleration. We implement denoising scheduling, sampling pipeline, and tensor-to-video conversion in Swift~\cite{swift} using Apple-provided libraries. To optimize models, T5~\cite{raffel2020exploring}, the largest in video generation, is quantized to int8, while others models  (STDiT~\cite{opensora} and VAE~\cite{doersch2016tutorial}) run in float32; we found that they are challenging to quantize due to sensitivity and performance degradation.%Our future implementations of On-device Sora will explore additional optimization to further enhance model efficiency.

%\jjm{In addition to the two key challenges mentioned in \Cref{sec:challenges}, there is one more additional challenge, which is a high memory requirement.}
%To execute large video generative models (i.e., T5 \cite{raffel2020exploring} and STDiT \cite{opensora}) with the limited device memory, we propose Concurrent Inference with Dynamic Loading, which partitions the models into smaller blocks that can be loaded into the memory and executed concurrently. By parallelizing model execution and block loading, it effectively accelerates iterative model inference, e.g., multiple denoising steps. Also, it improves memory utilization while minimizing the block loading overhead by retaining specific model blocks in memory dynamically based on the available runtime memory.
%\jjm{For a more detailed explanation of CI-DL, please refer to \Cref{sec:ours3}.}
% 20 20 20 -> 16 16 16
% 10 10 10 -> 27 * 7 189 210 ->189 



\begin{table*}[t]
    \centering
    \caption{Recommendation performance of different denoising methods. The highest scores are in bold, and the runner-ups are with underlines. A significant improvement over the runner-up is marked with * (i.e., two-sided t-test with $0.05 \le p < 0.1$) and ** (i.e., two-sided t-test with $p < 0.05$).}
    \resizebox{\textwidth}{!}{

\begin{tabular}{lcccccccccccc}
    \toprule
    \multicolumn{1}{c}{\multirow{3}{*}{\textbf{Model}}} & \multicolumn{4}{c}{\textbf{Gowalla} } & \multicolumn{4}{c}{\textbf{Yelp2018} } & \multicolumn{4}{c}{\textbf{MIND} }  \\
    \cmidrule(lr){2-5} \cmidrule(lr){6-9} \cmidrule(lr){10-13}
    & \multicolumn{2}{c}{\textbf{Recall} } & \multicolumn{2}{c}{\textbf{NDCG} } & \multicolumn{2}{c}{\textbf{Recall} } & \multicolumn{2}{c}{\textbf{NDCG} } & \multicolumn{2}{c}{\textbf{Recall} } & \multicolumn{2}{c}{\textbf{NDCG} } \\
    \cmidrule(lr){2-3} \cmidrule(lr){4-5} \cmidrule(lr){6-7} \cmidrule(lr){8-9} \cmidrule(lr){10-11} \cmidrule(lr){12-13}
    & \textbf{@20} & \textbf{@50} & \textbf{@20} & \textbf{@50} & \textbf{@20} & \textbf{@50} & \textbf{@20} & \textbf{@50} & \textbf{@20} & \textbf{@50} & \textbf{@20} & \textbf{@50} \\
    % \cmidrule(lr){2-5} \cmidrule(lr){6-9} \cmidrule(lr){10-13}
    % & \textbf{Recall@20} & \textbf{Recall@50} & \textbf{NDCG@20} & \textbf{NDCG@50} & \textbf{Recall@20} & \textbf{Recall@50} & \textbf{NDCG@20} & \textbf{NDCG@50} & \textbf{Recall@20} & \textbf{Recall@50} & \textbf{NDCG@20} & \textbf{NDCG@50} \\
    \midrule
    \textbf{MF}& 0.1486 & 0.2410 & 0.1073 & 0.1370 & 0.0621 & 0.1187 & 0.0483 & 0.0704 & 0.0658 & 0.1219 & 0.0430 & 0.0615 \\
    ~+\textbf{R-CE}& 0.1456 & 0.2362 & 0.1053 & 0.1343 &\underline{0.0654} &\underline{0.1239} & 0.0506 & 0.0733 &\underline{0.0716} & 0.1311 & 0.0468 & 0.0663 \\
    ~+\textbf{T-CE}& 0.1326 & 0.2197 & 0.0920 & 0.1197 & 0.0571 & 0.1113 & 0.0430 & 0.0639 & 0.0359 & 0.0812 & 0.0215 & 0.0363 \\
    ~+\textbf{DeCA}& 0.1463 & 0.2356 & 0.1068 & 0.1355 & 0.0645 & 0.1225 & 0.0502 & 0.0729 & 0.0714 &\underline{0.1312} & 0.0471 &\underline{0.0668} \\
    ~+\textbf{BOD}&\underline{0.1489} &\underline{0.2415} &\underline{0.1079} &\underline{0.1376} & 0.0654 & 0.1235 &\underline{0.0511} &\underline{0.0738} & 0.0713 & 0.1300 &\underline{0.0473} & 0.0665 \\
    ~+\textbf{DCF}& 0.1489 & 0.2413 & 0.1073 & 0.1367 & 0.0635 & 0.1208 & 0.0493 & 0.0715 & 0.0710 & 0.1297 & 0.0472 & 0.0665 \\
    \cmidrule(lr){2-13}
    ~+\textbf{PLD (ours)}&\textbf{0.1520**} &\textbf{0.2475**} &\textbf{0.1097} &\textbf{0.1404*} &\textbf{0.0677**} &\textbf{0.1264**} &\textbf{0.0527**} &\textbf{0.0755**} &\textbf{0.0769**} &\textbf{0.1379**} &\textbf{0.0513*} &\textbf{0.0713**} \\
    \multicolumn{1}{c}{Gain}& +2.04\% $\uparrow$& +2.47\% $\uparrow$& +1.71\% $\uparrow$& +1.98\% $\uparrow$& +3.46\% $\uparrow$& +2.02\% $\uparrow$& +3.09\% $\uparrow$& +2.33\% $\uparrow$& +7.36\% $\uparrow$& +5.09\% $\uparrow$& +8.46\% $\uparrow$& +6.83\% $\uparrow$\\
    \midrule
    \textbf{LightGCN}& 0.1553 & 0.2509 & 0.1142 & 0.1449 & 0.0665 & 0.1270 & 0.0516 & 0.0750 &\underline{0.0817} &\underline{0.1485} &\underline{0.0538} &\underline{0.0757} \\
    ~+\textbf{R-CE}& 0.1536 & 0.2481 & 0.1131 & 0.1434 & 0.0554 & 0.1042 & 0.0428 & 0.0617 & 0.0723 & 0.1315 & 0.0478 & 0.0670 \\
    ~+\textbf{T-CE}& 0.1146 & 0.1859 & 0.0859 & 0.1088 & 0.0532 & 0.1004 & 0.0412 & 0.0595 & 0.0674 & 0.1222 & 0.0447 & 0.0626 \\
    ~+\textbf{DeCA}& 0.1540 & 0.2495 & 0.1133 & 0.1440 &\underline{0.0678} &\underline{0.1298} &\underline{0.0526} &\underline{0.0766} & 0.0812 & 0.1480 & 0.0532 & 0.0751 \\
    ~+\textbf{BOD}&\underline{0.1560} &\underline{0.2519} &\underline{0.1154} &\underline{0.1461} & 0.0672 & 0.1280 & 0.0523 & 0.0758 & 0.0809 & 0.1475 & 0.0532 & 0.0750 \\
    ~+\textbf{DCF}& 0.1276 & 0.2072 & 0.0948 & 0.1203 & 0.0619 & 0.1180 & 0.0482 & 0.0699 & 0.0734 & 0.1342 & 0.0483 & 0.0681 \\
    \cmidrule(lr){2-13}
    ~+\textbf{PLD (ours)}&\textbf{0.1580**} &\textbf{0.2558**} &\textbf{0.1157} &\textbf{0.1472*} &\textbf{0.0693**} &\textbf{0.1325**} &\textbf{0.0538**} &\textbf{0.0783**} &\textbf{0.0837**} &\textbf{0.1516**} &\textbf{0.0551**} &\textbf{0.0774**} \\
    \multicolumn{1}{c}{Gain}& +1.23\% $\uparrow$& +1.53\% $\uparrow$& +0.32\% $\uparrow$& +0.71\% $\uparrow$& +2.31\% $\uparrow$& +2.02\% $\uparrow$& +2.44\% $\uparrow$& +2.22\% $\uparrow$& +2.43\% $\uparrow$& +2.06\% $\uparrow$& +2.47\% $\uparrow$& +2.33\% $\uparrow$\\
    \bottomrule
\end{tabular}
    }
\label{tab:performance}%
\end{table*}
\section{Practical Considerations} \label{sec:considerations}
In this section, we discuss practical issues that needed to be considered when deploying Data Guard at-scale in LinkedIn. \\
1. {\bf View Schema Evolution: } \
In a large warehouse, tables often undergo changes to schema and new field additions (either top-level columns or nested fields) to tables are very common. As Data Guard data-masking views are schema-preserving,
the schema of the view at creation time may be different from the schema of the view at the consumption time due to newly added fields. This means that newly added fields may accidentally be consumed in user queries, 
resulting in non-compliant accesses. To prevent this data leakage, we dynamically rewrite the data-masking view at the view consumption time via the ViewShift $getView()$ API to set all newly added field values to $NULL$. \\ 
%The list of these non-compliant fields are computed by comparing schema of the underlying table with the compliant schema of the view, which was registered to the catalog at the view creation time.\\
2. {\bf Handling non-nullable fields: } \
The data-masking function described thus far replaces fields to be masked with $NULL$ values and thus, assumes that fields are nullable. 
In large warehouses, it is common to define schemas containing non-nullable fields. 
To ensure compliance, our approach when rolling out Data Guard at LinkedIn, has been to over-filter data and filter out the entire row from the result set when a non-nullable field needs to be masked. It is possible to extend this behavior to provide non-NULL replacement values which are defined system-wide. This would however require changes by data consumers to handle such values in their applications. 
An area of ongoing work is to enhance Data Guard's APIs to provide data consumer-specified replacement values for non-nullable fields in order to avoid over-filtering data and maximize data utility. \\
3. {\bf Enforcement Verification: } \ 
In addition to proactive enforcement of policies at the time of data access, Data Guard also monitors access logs for all data accesses to the warehouse data. 
Data Guard's monitoring sub-system relies on raw access logs from underlying storage system (i.e. HDFS in our case) along with query execution logs from data processing engines 
like Spark and Trino to detect any inadvertent access to warehouse data that bypasses Data Guard enforcement. 
Data Guard also detects accesses to stale versions of Data Guard views from the access logs. 
In each scenario, Data Guard notifies consumers about potential access violations and provides them with steps to remediate their accesses. \\
4. {\bf Result Caching: }\ 
Commercial warehouses commonly support result set caching, even though this functionality is not available currently in LinkedIn's warehouse. One of the conditions for result caching is that queries should not include non-deterministic SQL functions such as {\em CURRENT\_TIMESTAMP()}. Thus, the data-masking views described in this paper are not eligible for result set caching. However, caching results of data-masking views has limited value in our environment because the view evaluation depends on data subject consents which are frequently updated. In scenarios where such updates are infrequent, result set caching can indeed be effective. To enable reuse of cached results, alternatives such as defining data-masking views as parametrized views that use timestamp as a parameter passed to the view or defining the masking views as materialized views appear viable.  \\
5. {\bf Engine optimizations: }\ An important limitation of implementing the data-masking operator using UDFs is that it prevents engines from applying optimizations such as predicate pushdown or nested column pruning. Our primary focus in implementing and rolling out Data Guard at-scale inside LinkedIn has been on the APIs for enforcing policies and ensuring a friction-free developer experience. While we implemented a number of performance optimizations to ensure that the cost of adopting data-masking views remains acceptable, we recognize the opportunity for further optimizations in the future. 
Software development is increasingly conceived as a collaboration activity between developers and AIs. Indeed, IDEs already implement features to enable interactive development, with AI suggesting implementations that are reused by developers.

Although multiple studies show this interaction can be successful, there is still limited understanding of how the models must be configured and used in the context of code generation tasks. This study addresses this gap, systematically investigating the impact of several key parameters, including the repeated submission of a prompt to accommodate for the non-deterministic nature of the models.

Our study reveals several key findings about the usage of ChatGPT. In particular, we discovered how creativity, although up to a limited extent, is useful to increase the range of methods whose code can be generated correctly. A major role is played by parameter top-p, which is commonly underrated, and instead has a major impact on the correctness of the results, with lower values producing better results. Finally, prompts should be submitted multiple times, with $5$ repetitions combined with a temperature of $1.2$ resulting in an effective configuration in our experiments.  

Future work concerns two main research directions. One is about replicating this experiment with other AI assistants, to validate our findings in multiple contexts. The second research direction concerns finding strategies to deal with the need to submit the same prompt multiple times to obtain a useful result, and thus developing approaches able to select or merge multiple responses automatically. 

\begin{acks}
We would like to thank Kapil Surlaker and Raghu Hiremagalur for providing numerous critical inputs on the design of Data Guard. We would like to thank Yong Li, Manasa Subramanian, Divya Singh, Yanwen Lin, Justin Heaton, Vishwaa Patel, Bryan Ji, Qiang Fu, Steve Cao, Wenning Ding, and Carleen Li for their significant contributions to the development of Data Guard. We also thank Kiran Shivaram, Souvik Ghosh, Chris Harris, David Leung and their team members for being early adopters and providing valuable customer feedback that helped improve the product.
Finally, we would like to thank Maneesh Varshney and Chris Harris for their helpful feedback in improving the quality of the paper.
\end{acks}

%\clearpage

\bibliographystyle{ACM-Reference-Format}
\bibliography{sample}
\end{document}
\endinput
