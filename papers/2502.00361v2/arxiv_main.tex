\documentclass[11pt]{article}
\usepackage[letterpaper, margin=2cm]{geometry}
% Recommended, but optional, packages for figures and better typesetting:
\usepackage{parskip}
\setlength{\abovedisplayskip}{10pt}  % Space before equations
\setlength{\belowdisplayskip}{10pt}  % Space after equations
\setlength{\parindent}{0pt}
\setlength{\parskip}{10pt} % Adjust the space as needed
\usepackage{microtype}
\usepackage{graphicx}
\usepackage{subfigure}
\usepackage{booktabs} % for professional tables
\usepackage[affil-it]{authblk}
\usepackage[textsize=tiny]{todonotes}



\usepackage{amsthm}
% \usepackage{generic}
% \usepackage{cite}
\usepackage{amsmath,amssymb,amsfonts}
\usepackage{algorithm}
\usepackage{algorithmic}
\usepackage{graphicx}
% \usepackage{algorithmic}
\usepackage{natbib}
% if you use cleveref..
\usepackage{textcomp}
\usepackage{color}
\usepackage{xurl}
\usepackage{bm}
% \usepackage{algorithm2e}
\usepackage[normalem]{ulem}
\usepackage{xcolor}
\usepackage{xspace}
\usepackage{wrapfig}
\usepackage{booktabs, multirow}
\let\labelindent\relax
\usepackage{enumitem}
\usepackage{thmtools} 
\usepackage{thm-restate}
% \newcommand{\citep}{\cite}
% \newcommand{\citet}{\cite}
\theoremstyle{plain}
\newtheorem{theorem}{Theorem}[section]
\newtheorem{proposition}[theorem]{Proposition}
\newtheorem{lemma}[theorem]{Lemma}
\newtheorem{corollary}[theorem]{Corollary}
\theoremstyle{definition}
\newtheorem{definition}[theorem]{Definition}
\newtheorem{assumption}[theorem]{Assumption}
\newtheorem{example}[theorem]{Example}
\theoremstyle{remark}
\newtheorem{remark}[theorem]{Remark}
\usepackage{Definitions}

% \usepackage{xr-hyper}
\usepackage[colorlinks,citecolor=blue]{hyperref}
% \hypersetup{hidelinks=true}
\usepackage[capitalize,noabbrev]{cleveref}

\newcommand{\algname}{{Soft Diffusion Actor-Critic}\xspace}
\newcommand{\algabb}{\texttt{SDAC}\xspace}
\newcommand{\algcommentline}[1]{\texttt{\textit{\textcolor{gray}{\# #1}}}}

\allowdisplaybreaks


\title{Soft Diffusion Actor-Critic: \\
Efficient Online Reinforcement Learning for Diffusion Policy}

% \author{Haitong Ma\textsuperscript{1}, Tianyi Chen\textsuperscript{2}, Kai Wang\textsuperscript{2}, Na Li\textsuperscript{1,*}, Bo Dai\textsuperscript{2,*}}



\begin{document}

\author[1]{Haitong Ma} 
  \author[2]{Tianyi Chen} 
  \author[2]{Kai Wang}
    \author[1]{Na Li$^*$}
  \author[2]{Bo Dai\thanks{Equal supervision.}}
  \affil[1]{School of Engineering and Applied Sciences, Harvard University}
  \affil[2]{School of Computational Science and Engineering, Georgia Institute of Technology\thanks{Emails: Haitong (haitongma@g.harvard.edu), Tianyi Chen (tchen667@gatech.edu), Kai Wang (kwang692@gatech.edu), Na Li (nali@seas.harvard.edu), Bo Dai (bodai@cc.gatech.edu)}}
    \date{}

\maketitle
\begin{abstract}
Diffusion policies have achieved superior performance in imitation learning and offline reinforcement learning (RL) due to their rich expressiveness. However, the vanilla diffusion training procedure requires samples from target distribution, which is impossible in online RL since we cannot sample from the optimal policy, making training diffusion policies highly non-trivial in online RL. 
Backpropagating policy gradient through the diffusion process incurs huge computational costs and instability, thus being expensive and impractical. 
To enable efficient diffusion policy training for online RL, we propose {\algname~(\algabb)}, exploiting the viewpoint of diffusion models as noise-perturbed energy-based models.
The proposed \algabb relies solely on the state-action value function as the energy functions to train diffusion policies, bypassing sampling from the optimal policy while maintaining lightweight computations. 
% Specifically, we first propose \emph{reverse sampling score matching} that can train diffusion models with only access to energy functions via \emph{reverse sampling trick}. 
% We apply the technique toz online maximum entropy RL to propose the SDAC algorithm, which learns a diffusion policy that only requires the soft $Q$-function as the energy function to significantly reduce the computation cost and achieve state-of-the-art performance.
% We then apply it to online maximum entropy RL, named soft diffusion actor-critic (SDEC), to learn a diffusion policy that only requires the soft $Q$-function as the energy function, which significantly reduces the computation cost and achieves state-of-the-art performance.
% Unlike existing methods that approximated the data distribution or backpropagated the policy gradient through the whole diffusion process, we achieved state-of-the-art performance with lightweight computation.
We conducted comprehensive comparisons on MuJoCo benchmarks. The empirical results show that \algabb outperforms all recent diffusion-policy online RLs on most tasks, and improves more than 120\% over soft actor-critic on complex locomotion tasks such as Humanoid and Ant.

\end{abstract}


\setlength{\abovedisplayskip}{2pt}
\setlength{\abovedisplayshortskip}{2pt}
\setlength{\belowdisplayskip}{2pt}
\setlength{\belowdisplayshortskip}{1pt}
\setlength{\jot}{1pt}
%
\setlength{\floatsep}{1ex}
\setlength{\textfloatsep}{1ex}



\vspace{-12pt}
\section{Introduction}
\begin{figure}[t]
  \centering
   \includegraphics[width=1\linewidth]{sec/figs/fig1.png}
   \caption{\captionMethodFigure}
   \label{fig:methodFigure}
\end{figure}
\begin{figure}[t]
  \centering
   \includegraphics[width=1\linewidth]{sec/figs/exampleFigure.png}
   \caption{\captionExampleFigure}
   \label{fig:exampleFigure}
   \vspace{-12pt}
\end{figure}
Vision-Language Models (VLMs), such as CLIP~\cite{CLIP}, have emerged as general-purpose systems for understanding visual data through language-based queries. These models enable a broad range of applications, from object detection to image captioning, by linking visual inputs to language prompts. In standard settings where images contain single, recognizable objects, VLMs perform remarkably well. However, for the more complex task of zero-shot multi-label recognition (MLR) (Fig. \ref{fig:methodFigure} (top)), where models must identify multiple objects within an image without prior training on specific data, VLMs face significant limitations. Zero-shot MLR is crucial for applications in fields like robotics and medical imaging, where objects rarely appear in configurations that align neatly with training distributions. In these scenarios, achieving robust multi-label recognition without fine-tuning is challenging, given the task’s reliance on mean Average Precision (mAP) scores, which depend on ranking images for object presence.

\noindent \textbf{VLM: Prompt Dependent AND/OR Noisy Channel.} Despite the promise of zero-shot capabilities, current VLM approaches often struggle with MLR due to inherent scoring behaviors and biases. The performance of these models is hampered by a mix of conjunction (AND) and disjunction (OR) behaviors in their scoring, leading to inflated scores in compound prompts that contain multiple objects. For example, a prompt like “cat and sofa” might yield a high score even if only one of these objects is present in the image. This tendency reflects biases learned during training, where common object pairs receive higher scores even when only one object is present, disrupting the accuracy of mAP-based evaluations. Furthermore, existing methods for adapting VLMs to zero-shot MLR frequently rely on prompt tuning or architectural adjustments—approaches that are often dependent on training data and computationally intensive fine-tuning, which limit their generalizability to novel tasks.

\noindent \textbf{Our Approach.} In contrast to these methods, we introduce SPARC (Score Prompting and Adaptive Fusion for Zero-Shot Multi-Label Recognition in VLMs), a novel approach to zero-shot MLR that bypasses the need for training data, prompt tuning, or model-specific modifications. Our method treats the VLM as a black box, relying solely on its output scores to infer object presence (see Fig. \ref{fig:methodFigure}). This black-box approach enables us to avoid assumptions about the model’s internal workings, allowing for a purely zero-shot framework that is both model-agnostic and dataset-independent. SPARC introduces two main innovations that address the unique challenges of zero-shot MLR.

\noindent \textbf{A. Compound Prompt Composition:} Recognizing that VLMs can provide richer information when prompted with combinations of objects, we develop a method for constructing compound prompts. These prompts reflect likely contextual associations between objects, such as “cat and sofa” or “car and bus.” By gathering scores from these compound prompts, we can capture a spectrum of potential object contexts within the image, enhancing detection without relying on training-based adaptations. This composition strategy allows us to agnostically extract information from the VLM, leveraging probable object relationships without depending on any specific dataset or VLM architecture.\\
\noindent \textbf{B. Score Debiasing and Adaptive Fusion.} A critical insight in our approach lies in the surprising observation that the \underline{maximum score among compound prompts} is often a poor proxy for true object presence. Although one might expect the highest score to serve as a reliable signal, we find that it frequently reflects compositional biases, as VLMs tend to respond to compound prompts with OR-like behavior, raising scores even when only one object in the prompt is present. Instead, we observe that the second-highest score consistently provides a more accurate indicator of object presence, minimizing the effects of false positives caused by compositional bias. Building on this insight, we develop a debiasing algorithm that normalizes scores across images to address image-specific noise and clarify genuine object presence signals. This debiased score set is then processed through a PCA-based fusion method that further refines object rankings by combining information from both compound and singleton prompts, ultimately optimizing mAP by enhancing score accuracy.\\
\textbf{Complementarity.} SPARC is complementary to other zero-shot and training-free MLR methods. When applied on top of these approaches, SPARC consistently enhances mAP scores by refining object ranking and reducing bias in VLM outputs. This capability makes SPARC an adaptable solution that can improve upon existing methods while maintaining a fully zero-shot, model-agnostic framework.

\noindent \textbf{Empirical Results.} SPARC achieves significant improvements in mAP, outperforming methods that incorporate architectural modifications. This outcome shows the potential of a fully zero-shot approach that relies only on systematic prompt design and score interpretation, rather than prompt-training or fine-tuning. By revealing that the second-highest score can be a superior proxy to the maximum, our findings provide new insights into VLM scoring behavior, suggesting that careful treatment of prompt compositions and score patterns can unlock robust MLR capabilities.



\section{Preliminaries}
We introduce the necessary preliminaries in this section. First, we introduce Markov decision process and maximum entropy reinforcement learning as our policy learning framework, followed by a recap of diffusion models.

\subsection{Maximum Entropy Reinforcement Learning}
\label{sec:max_ent_rl}
\textbf{Markov Decision Processes (MDPs).}~We consider Markov decision process~\citep{puterman2014markov} specified by a tuple $\mathcal{M}=(\mathcal{S}, \mathcal{A}, r, P, \mu_0, \gamma)$, where $\mathcal{S}$ is the state space, $\mathcal{A}$ is the action space, $r:\Scal\times\Acal\to\RR$ is a reward function,
$P\rbr{\cdot|s, a}: \mathcal{S} \times \mathcal{A} \rightarrow \Delta(\mathcal{S})$ is the transition operator with $\Delta(\mathcal{S})$ as the family of distributions over $\mathcal{S}, \mu_0 \in \Delta(\mathcal{S})$ is the initial distribution and $\gamma \in(0,1)$ is the discount factor. 

\textbf{Maximum entropy RL.} We follow the maximum entropy RL to learn our diffusion policies~\cite{haarnoja2017reinforcement}. We consider the following entropy-regularized expected return as the policy learning objective,
\begin{equation}
    \arg\max_\pi J(\pi) := \EE_{\pi}\sbr{\sum_{\tau=0}^\infty\gamma^\tau \rbr{r(\sbb_\tau, \ab_\tau) + \lambda\Hcal(\pi(\cdot | \sbb_\tau))}}\label{eq:obj_max_ent_rl}
\end{equation}
where $\Hcal\rbr{\pi(\cdot| \sbb)} = \EE_{\ab\sim\pi(\cdot | \sbb)}[-\log\pi(\ab | \sbb)]$ is the entropy, $\lambda$ is a regularization coefficient for the entropy. 
The soft policy iteration algorithm~\cite{haarnoja2017reinforcement,haarnoja2018soft} is proposed to solve the optimal max-entropy policy. Soft policy iteration algorithm iteratively conducts soft policy evaluation and soft policy improvement, where soft policy evaluation updates the soft $Q$-function by repeatedly applying soft Bellman update operator $\Tcal^\pi$ to current value function $Q:\Scal\times\Acal\to \RR$, \ie, 
\begin{equation}
    \Tcal^\pi Q(\sbb_\tau, \ab_\tau) = r(\sbb_\tau, \ab_\tau) + \gamma\EE_{\sbb_{\tau+1}\sim P}\sbr{V(\sbb_{\tau+1})}\label{eq:soft_pev}
\end{equation}
where $V(\sbb_\tau) = \EE_{\ab_\tau\sim\pi}\sbr{Q(\sbb_\tau, \ab_\tau) - \lambda \log\pi(\ab_\tau\mid \sbb_\tau)}$ \cite{haarnoja2018soft}. Then in the soft policy improvement stage, the policy is updated to fit the target policy
\begin{equation}
    \pi_{\rm target}(\ab| \sbb) 
    % \frac{\exp\rbr{\frac{1}{\lambda}{Q^{\pi_{\rm old}}(\sbb, \ab)}}}{Z^{\pi_{old}}(\sbb)}
    \propto \exp\rbr{\frac{1}{\lambda}{Q^{\pi_{\rm old}}(\sbb, \ab)}} \label{eq:energy_based_opt_pi}
\end{equation}
where ${\pi_{\rm old}}$ is the current policy and $Q^{\pi_{\rm old}}$ is the converged result of \eqref{eq:soft_pev} with $\Tcal^{\pi_{\rm old}}$.
% \Bo{delete this section. we discuss the connection between ebm and diffusion in our method section. as we discuss on last friday, please follow my diffusion spectral representatio paper logic. }

\textbf{Soft Actor-Critic.} 
Although we have closed-form policy~\eqref{eq:energy_based_opt_pi}, it is a \emph{unnormalized} distribution, often referred to as an \emph{energy-based} policy since the unnormalized density is called energy function in literature, which is notoriously difficult to sample from and learn. To enable efficient computation, a natural idea is to approximate the energy-based policies~\eqref{eq:energy_based_opt_pi} with a parametrized distribution. 
A representative algorithm is the well-known soft actor-critic (SAC), which restricts the policy to be a parametrized Gaussian, \ie, $\pi_\theta\rbr{a|s} = \Ncal\rbr{ \mu_{\theta_1}(s), \sigma^2_{\theta_2}(s)}$ and updates the parameters $\theta = [\theta_1, \theta_2]$ by optimizing the $KL$-divergence to the target policy $D_{KL}(\pi_\theta\|\pi_{\rm target})$~\cite{haarnoja2018soft} via policy gradient with parametrization trick, \ie, % which equals a Q-learning style loss with data samples from $\Dcal$
$$
J^\pi_{\rm SAC}(\theta)=\EE_{\sbb\sim\Dcal,\ab\sim\pi_\theta}\sbr{\lambda\log\pi_\theta(\ab|\sbb)-Q^{\pi_{\rm old}}(\sbb,\ab)}. 
$$ 
% \Bo{with explicit SAC policy update objective here. }\haitong{done.}
% Even with the Gaussian reduction, SAC has been SOTA till now, showing the huge potential of the maximum entropy RL. 
The Gaussian approximation loses the inherent expressiveness and multimodality of energy-based policies, thus limiting the performance of maximum entropy RL algorithms. This limitation motivates the pursuit of more expressive policy structures to further enhance performance.


% \textbf{Abstact problem notations.} From now on, we focus on an abstract version of representing energy-based policies in \eqref{eq:energy_based_opt_pi}.
% Specifically, we study how to train and sample from an EBM with the given energy function \emph{only}, \ie, we have a target distribution $p_0$ with known energy function $E$ with $p_0(\xb_0)\propto\exp\rbr{-E(\xb_0)}$. Only means we cannot sample from $p_0$.

\subsection{Denoising Diffusion Probabilistic Models} 
\label{sec.diffusion}
Denoising diffusion probabilistic models~\citep[DDPMs,][]{sohl-dickstein2015deep,song2019generative,ho2020denoising} are powerful tools to represent and generate complex probability distributions. \emph{Given data samples from the data distribution} $p_0$, DDPMs are composed of a forward diffusion process that gradually perturbs the data distribution $\xb_0\sim p_0$ to a noise distribution $\xb_T\sim p_T$, and a reverse diffusion process that reconstructs data distribution $p_0$ from noise distribution $p_T$.
 The forward corruption kernel is usually Gaussian with a variance schedule $\beta_1,\dots,\beta_T$, resulting in the forward trajectories with joint distribution
\begin{align}
    &q_{{0:T}}(\xb_{0:T}) = p_0(\xb_0)\prod_{t=1}^Tq_{t|t-1}\rbr{\xb_t|\xb_{t-1}} \quad \text{where}\notag\\
    &q_{t|t-1}(\xb_t|\xb_{t-1}) := \Ncal(\xb_t;\sqrt{1-\beta_t}\xb_{t-1},\beta_t\Ib)\label{eq:corruption_ddpm}
\end{align}
where $\xb_t$ is random variable at $t$ step, and $p, q$ are probability distributions\footnote{We use $p$ and $q$ interchangeably as density function in this paper. Generally, $p$ represents intractable distributions (like the t-step marginal $p_t(\xb_t)$), and $q$ represents tractable distributions such as the Gaussian corruption $q_{t\mid t-1}(\xb_t|\xb_{t-1})$.}.
The backward process recovers the data distribution from a noise distribution $p_T$ with a series of reverse kernels $p_{t-1|t}(\xb_{t-1}|\xb_t)$. 
The reverse kernels are usually intractable so we parameterize it with neural networks denoted as $p_{\theta;t-1|t}\left(\xb_{t-1} \mid \xb_t\right)$, resulting in a joint distribution of the reverse process,
$$
% \begin{aligned}
    % &
    % \text{where}\\
    % & p_\theta\left(\xb_{t-1} \mid \xb_t\right):=\mathcal{N}\left(\xb_{t-1} ; \boldsymbol{\mu}_\theta\left(\xb_t, t\right), \mathbf{\Sigma}_\theta\left(\xb_t, t\right)\right)
% \end{aligned}
p_\theta\rbr{\xb_{0:T}}=p_T(\xb_T)\prod_{t=1}^Tp_{\theta;t-1|t}(\xb_{t-1}|\xb_{t})\quad
$$
Considering all $(\xb_1,\dots,\xb_T)$ as the latent variables, we can solve the parameters $\theta$ via optimizing the evidence lower bound (ELBO) over $\xb_0$,
$$
{\rm ELBO}(\theta) = \EE_{\xb_0\sim p_0}\EE_{\xb_{1:T}\sim q}\sbr{\log \frac{p_\theta(\xb_{0:T})}{q(\xb_{1:T}|\xb_0)}}.
$$
After fixing $p_{\theta;t-1|t}$ to be Gaussian and reparametrizing $p_{\theta;t-1|t}$ with a score network\footnote{Some paper reparameterize it as the noise prediction network $\epsb_\theta$, but they are the same in essence since $\nabla_{\xb_t} \log q_{t|0}(\xb_t|\xb_0) = -\frac{\xb_t -\sqrt{\bar\alpha_t}\xb_0}{1 -\bar\alpha_t}=-\frac{\epsb}{\sqrt{1 -\bar\alpha_t}}$ for Gaussian noise $\epsb$.} $s_\theta(\xb_t;t)$, maximizing the ELBO is equivalent to minimizing a collection of denoising score matching loss over multiple noise levels indexed by $t$~\cite{vincent2011connection,ho2020denoising},
\begin{equation}
    \begin{aligned}
        &\Lcal_{\rm DSM}(\theta):=\frac{1}{T}\sum_{t=0}^T(1 -\bar\alpha_t)\underset{\substack{\xb_0\sim p_0\\\xb_t\sim q_{t|0}}}{\EE}\sbr{\nbr{ s_\theta\rbr{\xb_t; t} - \nabla \log q_{t|0}(\xb_t|\xb_0)}^2}\label{eq:ddpm_loss}
    \end{aligned}
\end{equation}
where $q_{t|0}(\ab_t|\ab_0):=\Ncal(\ab_t;\sqrt{\bar\alpha_t}\ab_0, \rbr{1 - \bar\alpha_t}\Ib)$ and $\bar\alpha_t = \prod_{l=1}^t (1-\beta_l)$. 
After learning the $s_\theta$ by minimizing \eqref{eq:ddpm_loss}, we can draw sample via the reverse diffusion process by iteratively conducting
\begin{equation}
    \xb_{t-1}=\frac{1}{\sqrt{\bar\alpha_t}}\left(\xb_t+\beta_t s_\theta\left(\xb_t, t\right)\right)+\frac{1-\bar\alpha_{t-1}}{1-\bar\alpha_t}\beta_t \zb_t\label{eq:annealed_langevin_2}
\end{equation}
for $t = T, T-1,\dots, 1$ and $\zb_t\sim\Ncal\rbr{0,\Ib}$.

% \subsection{DDPM as Energy-Based Models} 
% Diffusion models can also be interpreted as energy-based models~(EBMs).
% EBMs usually refer to parametrized probabilistic models that are not normalized. The density function of an EBM is generally in the formulation of
% $
% p_\theta(\xb) = \frac{\exp\rbr{ - E_\theta(\xb)}}{Z_\theta}
% $,
% where $E_\theta(x)$ is the \emph{energy function} parameterized by $\theta$, $Z_\theta = \int \exp\rbr{ - E_\theta(\xb)} d \xb$ is the normalization constant that is usually intractable. The \emph{score function} is defined as the gradient of log density, $\nabla_x\log p_\theta(x) $. 

% The unnormalized nature of EBMs allows versatility and rich expressiveness compared to other probabilistic models but raises difficulties in training and sampling~\cite{song2021train}. One of the naive approaches before DDPM is to first learn the score function $\nabla_{\xb}\log p(\xb)$ via score matching~\cite{hyvarinen2005estimation,song2020sliced} and then sample via Langevin dynamics with the learned score function $f_\theta(\xb)\approx\nabla_{\xb}\log p(\xb)$~\cite{parisi1981correlation},
% % , a Markov chain Monte-Carlo (MCMC) sampling procedure using only the score function,
% \begin{equation}
%     \xb_{i+1} \leftarrow \xb_i+\eta f_\theta(\xb)+\sqrt{2 \eta} \mathbf{z}_i, \quad i=0,\cdots, K\label{eq:langevin}
% \end{equation}
% where $\mathbf{z}_i \sim \mathcal{N}(0, I)$. As $\eta\to 0$ and $K\to\infty$, the sampled $\xb_K$ converges to samples from $p(\xb)$. 

% \citet{song2019generative} showed that this naive approach, combining score matching with Langevin dynamics, suffered from multiple pitfalls, such as slow mixing and inaccuracy in low-density regions, which hinders the empirical performance of score-based EBMs. One remedy proposed by \citet{song2019generative} is to fit the score function of a series of noise-perturbed data distribution $\Ncal\rbr{\xb_i;\xb, \sigma_i^2 \Ib}, i=\{1,2,\dots, K\}$ with a noise schedule $\sigma_1> \sigma_2>\dots>\sigma_K$. 
% The resulting models, named the noise-conditioned score networks (NCSN) $f_\theta\rbr{\xb_i;\sigma_i}$, take the noise level into the inputs and are learned by denoising score matching~\cite{vincent2011connection}
% \begin{align}
%     % &\EE_{\xbtil\sim \ptil}\sbr{\nbr{f_\theta\rbr{\xb_i;i} - \nabla_{\xbtil}\log \ptil_\sigma\rbr{\xbtil}}^2} =\label{eq:noisy_score_matching}\\
%     &\EE_{\xb\sim p,\xb_i\sim\Ncal\rbr{\xb,\sigma_i^2 \Ib}}\sbr{\|f_\theta\rbr{\xb_i;\sigma_i} -\nabla_{\xb_i}\log q(\xb_i|\xb) \|^2}\label{eq:dsm_loss}
% \end{align}
% Then in the sampling stage, \citet{song2019generative} replaced the original score function in \eqref{eq:langevin} with the learned noisy score function $f_\theta(\xbtil;\sigma_i)$,
% \begin{equation}
%     \xb_{i+1} \leftarrow \xb_i+\eta f_\theta(\xbtil;\sigma_i)+\sqrt{2 \eta} \mathbf{z}_i, \quad i=0,\cdots, K\label{eq:annealed_langevin}
% \end{equation}
% named as annealed Langevin dynamics. The scheduled noise perturbation design significantly improved the image generation performance to match the state-of-the-art (SOTA) at that time~\cite{song2019generative}, which is further refined by DDPM.

% We can see that the annealed Langevin dynamics~\eqref{eq:annealed_langevin} resembles the DDPM sampling~\eqref{eq:annealed_langevin_2} with different scale factors, and the denoising score matching loss~\eqref{eq:dsm_loss} is equivalent to \eqref{eq:ddpm_loss} since 
% $\nabla_{\xb_i}\log q(\xb_i|\xb)=-\frac{\xb_i-\xb}{\sigma_i^2}=-\frac{\epsb}{\sigma_i}$. Therefore, DDPM can be interpreted as EBMs and is equivalent to NCSN with different designs on noise schedules and scale factors. A more thorough discussion on their equivalency can also be found in~\citet{ho2020denoising,song2021scorebased}.











\begin{figure*}[ht]
    \centering
    \includegraphics[width=\linewidth]{figure/demo_paper.pdf}
    \caption{Demonstration of the proposed \algname~(\algabb) algorithm. We leverage diffusion policy to represent the energy-based policy in maximum entropy RL. The diffusion policy is trained via reverse sampling score matching, an algorithm that does not sample from the target energy-based policy and only depends on the $Q$-functions, enabling efficient online RL for diffusion policy.}
    \label{fig:demo}
\end{figure*}
\section{Diffusion Policy Learning in Online RL}
\label{sec:rssm}

In this section, 
we first present the connection of energy-based models and diffusion models, justifying the expressiveness of diffusion policy, and identify the difficulties in online training of diffusion policy. 
We then introduce the reverse sampling score matching (RSSM) to make the training of diffusion policy possible with \emph{only} access to the energy function in online RL.  








\subsection{Diffusion Models as Noise-Perturbed Energy-Based Models} 
We first revisit the energy-based view of diffusion models, \ie, \emph{diffusion models are noise-perturbed EBMs}~\cite{shribak2024diffusion}, to justify that the diffusion policy can efficiently represent the energy-based $\pi_{\rm target}$. 
Given $\sbb$, consider perturbing action samples $\ab_0 \sim \pi_{\rm target}(\cdot|\sbb)$ with corruption kernel $q_{t|0}(\ab_t|\ab_0)=\Ncal(\ab_t;\sqrt{\bar\alpha_t}\ab_0, \rbr{1 - \bar\alpha_t}\Ib)$, 
which results in the noisy-perturbed policy $\tilde\pi_t(\cdot|\sbb)$ with 
$$
\tilde\pi_t(\ab_t|\sbb)= \int q_{t|0}(\ab_t|\ab_0)\pi_{\rm target}(\ab_0|\sbb)d\ab_0
$$
for noise schedule index $t=1,2,\dots, T$.


\begin{restatable}[Diffusion models as noise-perturbed EBMs\label{prop:diff_ebm}]{proposition}{diffebm}
    The score network $s_\theta(\ab_t;\sbb, t)$ in \eqref{eq:ddpm_loss} matches noise-perturbed score functions, $\nabla_{\ab_t}\log \tilde\pi_t(\ab_t|\sbb)$,
        where state $\sbb$ is added to inputs of score network $s_\theta(\ab_t;\sbb, t)$ to handle conditional distributions, $p_0(\cdot)$ in \eqref{eq:ddpm_loss} refers to $\pi_{\rm target}(\cdot|\sbb)$ in the policy learning setting. \end{restatable}

\begin{proof}
    This can be shown by checking the noise-perturbed score function $\nabla_{\ab_t}\log\tilde\pi_t(\ab_t|\sbb_t)$, \ie,
    \begin{align}
        & \nabla_{\ab_t}\log\tilde\pi_t(\ab_t|\sbb_t)\label{eq:prop1_derivation}
        \\
            = & \frac{\nabla_{\ab_t}\tilde\pi_t\rbr{\ab_t|\sbb}}{\tilde\pi_t\rbr{\ab_t|\sbb}}           =  \frac{\nabla_{\ab_t}\int q_{t|0}(\ab_t|\ab_0)\pi_{\rm target}(\ab_0|\sbb)d\ab_0}{\tilde\pi_t\rbr{\ab_t|\sbb}}\notag\\
            = & \int \nabla_{\ab_t}\log q_{t|0}(\ab_t|\ab_0)\underbrace{\frac{q_{t|0}(\ab_t|\ab_0)\pi_{\rm target}(\ab_0|\sbb)}{\tilde\pi_t\rbr{\ab_t|\sbb}}}_{p_{0|t}(\ab_0|\ab_t, \sbb)}d\ab_0 \nonumber
    \end{align}
                        We match the noise-perturbed score function via the score network $s_\theta(\ab_t;\sbb, t)$ via optimizing the expectation of square error over $\ab_t\sim \tilde\pi_t(\cdot|\sbb)$,
    
    {\small
    \begin{align}
       &\EE_{\ab_t\sim\tilde\pi_t}\nbr{s_\theta(\ab_t;\sbb, t) - \int\nabla_{\ab_t}\log q_{t|0}(\ab_t|\ab_0)p_{0|t}(\ab_0|\ab_t,\sbb)d\ab_0}^2\notag\\         =&\underset{\substack{\ab_0\sim\pi_{\rm target}\\\ab_t\sim q_{t|0}}}{\EE}\sbr{\nbr{s_\theta(\ab_t;\sbb, t) -\nabla_{\ab_t}\log q_{t|0}(\ab_t|\ab_0)}^2} + \texttt{constant}\label{eq:tweedie_loss}
    \end{align}
    }
    The detailed derivations of \eqref{eq:tweedie_loss} are deferred to \Cref{sec:prop_1_apdx}.
    We can see that \eqref{eq:tweedie_loss} is equivalent to the $t$-th term in DDPM loss~\eqref{eq:ddpm_loss},
    which concludes the proof of Proposition~\ref{prop:diff_ebm}.
\end{proof}


Furthermore, as the noise schedule $\beta_t$ gets close to zero when $t$ goes from $T$ to $1$ in the reverse process~\eqref{eq:annealed_langevin_2}, the noise-perturbed EBMs gradually resemble the original energy-based policies $\pi_{\rm target}$. 
Adding adaptive levels of noise perturbations encourages explorations on the energy landscape, which significantly improves the sampling quality and makes diffusion models the key breakthrough in   EBMs~\cite{song2019generative}. 

\textbf{Difficulties to train diffusion model in online RL setup.}
By the connection between EBMs and diffusion models, we justify the expressiveness of diffusion policy for maximum entropy RL. 
However, training diffusion policy is highly non-trivial in online RL because of two major challenges:

$\bullet$ \textbf{Sampling challenge:} the vanilla diffusion training with denoising score matching~\eqref{eq:ddpm_loss} requires samples from the target policy $\pi_{\rm target}$, but we cannot access $\pi_{\rm target}$ directly in online RL since we only know the energy function, \ie, the $Q$-functions.

$\bullet$ \textbf{Computational challenge:} another possible solution is to backpropagate policy gradient thorough the whole reverse diffusion process~\eqref{eq:annealed_langevin_2}. However, this recursive gradient propagation not only incurs huge computational and memory cost, but also suffers from gradient vanishing or exploding, making diffusion policy learning expensive and unstable.

These challenges hinder the performance of diffusion-based policies in online RL. 












\subsection{Learning Noise-perturbed Score Functions via Reverse Sampling Score Matching }
\label{subsec:main_theorem}




In this section, we develop our core contribution, reverse sampling score matching (RSSM), an efficient diffusion policy learning algorithm that eliminates the aforementioned difficulties. 
Following the energy-based viewpoint in \Cref{prop:diff_ebm}, we propose the following theorem,


\begin{restatable}[Reverse sampling score matching (RSSM)\label{thm.rssm}]{theorem}{rssm}
    Define $\tilde p_t(\cdot|\sbb)$ as a sampling distribution whose support contains the support of $\tilde\pi_t(\cdot|\sbb)$ given $\sbb$.
                        Then we can learn the score network $s_\theta(\ab_t;\sbb,t)$  to match with the score function of noise-perturbed policy $\nabla_{\ab_t}\log \tilde\pi_t(\ab_t|\sbb)$ via minimizing
    \begin{equation}
        \underset{\substack{\ab_t\sim \tilde p_t\\ \tilde\ab_0\sim\tilde q_{0|t}}}{\EE}\sbr{\exp \rbr{Q\rbr{\sbb, \tilde\ab_0}/\lambda}\nbr{s_\theta\rbr{\ab_t; \sbb, t} - \nabla_{\ab_t}\log \tilde q_{0|t}\rbr{\tilde\ab_0 \! \mid \! \ab_t}}^2}\label{eq:thm1_reweight_loss}
    \end{equation}
    where we abbreviate $Q^{\pi_{\rm old}}$ with $Q$ for simplicity and $\tilde q_{0|t}$ is the \textbf{reverse sampling} distribution defined as
                \begin{equation}
        \tilde q_{0|t}(\tilde\ab_0|\ab_t):=\Ncal\rbr{\ab_0;\frac{1}{\sqrt{\bar\alpha_t}}\ab_t, \frac{1 - \bar\alpha_t}{\bar\alpha_t}\Ib}\label{eq:a0_sample_thm}
    \end{equation}
    which means $ \tilde\ab_0 = \frac{1}{\sqrt{\bar\alpha_t}}\ab_t - \sqrt{\frac{1 - \bar\alpha_t}{\bar\alpha_t}}\epsb$ for $\epsb\sim\Ncal(0,\Ib)$.
\end{restatable}

The name \emph{reverse sampling} score matching comes from that we first sample $\ab_t\sim \tilde p_t$ then sample $\tilde\ab_0\sim\tilde q_{0|t}$, thus bypassing the sampling issues and not increasing computational cost. We show a sketch proof here, the full derivations can be found in \Cref{sec:appendix_derivation}.

    \textit{Proof.} The derivations consists of two major steps, reformulating the noise-perturbed score function and applying the reverse sampling trick.
    \\\textbf{Reformatting the noise-perturbed score function.} First, we slightly reformat derivations of the noise-perturbed score function in \Cref{prop:diff_ebm} starting from \eqref{eq:prop1_derivation},
    \begin{equation}
        \begin{aligned}
             \nabla_{\ab_t}\log\tilde\pi_t(\ab_t|\sbb)            = &\int \nabla_{\ab_t}\log q_{t|0}(\ab_t|\ab_0)\frac{q_{t|0}(\ab_t|\ab_0)\pi_{\rm target}(\ab_0|\sbb)}{\tilde\pi_t\rbr{\ab_t|\sbb}}d\ab_0\\
            = & \frac{\int \nabla_{\ab_t}\log q_{t|0}(\ab_t|\ab_0) q_{t|0}(\ab_t|\ab_0)\exp\rbr{Q(\sbb,\ab_0)/\lambda} d\ab_0}{Z\rbr{\ab_t;\sbb}}\\
        \end{aligned}\label{eq:noise_perturbed_score}
    \end{equation}
    where $Z(\ab_t;\sbb) := \tilde{\pi}_t(\ab_t | \sbb) \int\exp\rbr{Q(\sbb,\ab_0)/\lambda} d\ab_0 = \int q_{t | 0}(\ab_t | \ab_0) \exp\rbr{Q(\sbb,\ab_0)/\lambda} d \ab_0 $ .
    Equation \eqref{eq:noise_perturbed_score} is obtained by substituting the energy function into $\pi_{\rm target}$.     With \eqref{eq:noise_perturbed_score}, the square error given $\ab_t$ satisfies
    \begin{equation}
        \begin{aligned}
            &\nbr{s_\theta(\ab_t;\sbb,t) - \nabla_{\ab_t}\log \tilde\pi_t(\ab_t|\sbb)}^2
            = \frac{1}{Z(\ab_t;\sbb)}\int q_{t|0}(\ab_t|\ab_0)\exp\rbr{Q(\sbb,\ab_0)/\lambda}l_\theta(\ab_0,\ab_t;\sbb)d\ab_0
        \end{aligned}
    \end{equation}
    where {$l_\theta(\ab_0,\ab_t;\sbb) = \nbr{s_\theta\rbr{\ab_t; \sbb, t} - \nabla_{\ab_t}\log q_{t|0}\rbr{\ab_t|\ab_0}}^2$}. 
    Then we integrate the square error over a custom measure  $g(\ab_t;\sbb):=Z(\ab_t;\sbb)\ptil_t(\ab_t|\sbb)$ to compensate the $Z(\ab_t;\sbb)$ and get to, 
    \begin{equation}
        \iint\ptil_t(\ab_t|\sbb) q_{t|0}(\ab_t|\ab_0)\exp\rbr{Q(\sbb,\ab_0)/\lambda}l_\theta(\ab_0,\ab_t;\sbb)d\ab_0d\ab_t\label{eq:reformat_loss_p}
    \end{equation}
    A more rigorous derivation is deferred to Appendix \ref{sec:appendix_derivation}. 
    
    \textbf{Reverse sampling trick.} The loss function in \eqref{eq:reformat_loss_p} is still not tractable. To handle this, we introduce the \emph{reverse sampling trick}, \ie, replacing $q_{t|0}$ with a reverse sampling distribution $\qtil_{0|t}$ that satisfies
    \begin{equation}
        \begin{aligned}
        &\qtil_{0|t}(\ab_0\mid \ab_t) =\Ncal\rbr{\ab_0;\frac{1}{\sqrt{\bar\alpha_t}}\ab_t, \frac{1 - \bar\alpha_t}{\bar\alpha_t}\Ib}\\
        \propto~& q_{t|0}(\ab_t\mid \ab_0) = \Ncal\rbr{\ab_t;\sqrt{\bar\alpha_t}\ab_0, \rbr{1 - \bar\alpha_t}\Ib},
    \end{aligned}\label{eq:reverse_gaussian}
    \end{equation}
    and their score functions match
    {
    \small
    $
    \nabla_{\ab_t}\log q_{t|0}(\ab_t\mid \ab_0) = \nabla_{\ab_t}\log \qtil_{0|t}(\ab_0\mid \ab_t) = - \frac{\ab_t -\sqrt{\bar\alpha_t}\ab_0}{1 -\bar\alpha_t}
    $
    }.
    Then we can replace $q_{t|0}$ with $\tilde q_{0|t}$ in \eqref{eq:reformat_loss_p} to get a tractable loss function,
    {
                \begin{equation}
        \iint \tilde p_t(\ab_t|\sbb)\tilde q_{0|t}(\tilde\ab_0| \ab_t)\exp\rbr{Q(\sbb,\tilde\ab_0)/\lambda} \tilde l_\theta(\ab_0,\ab_t;\sbb)d\tilde\ab_0d\ab_t \label{eq:with_tidle_q}
    \end{equation}
    }
where {\small $\tilde l_\theta(\ab_0,\ab_t;\sbb)  =  \nbr{s_\theta\rbr{\ab_t;\sbb, t}  -  \nabla_{\ab_t}\log \tilde q_{0|t}\rbr{\tilde\ab_0  \mid  \ab_t}}^2$}.
In this way, we can first sample $\ab_t\sim\tilde p_t$ and then sample $\tilde\ab_0\sim\tilde q_{0|t}$ to enable tractable loss computation.    
By further algebraic operations, we can derive the loss function in \eqref{eq:thm1_reweight_loss} from \eqref{eq:with_tidle_q}. The detailed derivation can be found in Appendix \ref{sec:appendix_derivation}.\hfill$\square$

We can see that with \Cref{thm.rssm}, the loss function \eqref{eq:thm1_reweight_loss} solves both the sampling and computational difficulties mentioned previously. First, we avoid sampling from target policy $\pi_{\rm target}$, and the sampling distribution $\tilde p_t$ is some distributions we can choose. Second, we have similar computation with denoising score matching~\eqref{eq:ddpm_loss}, avoiding extra computational cost induced by diffusion policy learning.


\begin{remark}[Broader applications of RSSM.] We emphasize that although we develop RSSM for online RL problems, the RSSM has its own merit and can be applied to any probabilistic modeling problems with known energy functions. We also show a toy example in \Cref{sec:toy} where we use RSSM to train a toy diffusion model to generate samples from a Gaussian mixture distribution.
\end{remark}

\begin{remark}[Pitfalls of Langevin dynamics in online RL.]\label{remark:langevin}
    Some might question that if we already know the energy function, why not compute the gradient as the score functions and use the Langevin dynamics~\cite{parisi1981correlation} to sample from policy~\eqref{eq:energy_based_opt_pi}.  The reasons are two-fold, i) the gradient of learned $Q$-function might not match the true score function; ii) Langevin dynamics suffers from the slow mixing problem~\citep[shown in \Cref{sec:toy}]{song2019generative} even with true score functions. Both pitfalls result in bad performance and motivate the necessity of diffusion policies with known energy functions.
\end{remark}

\subsection{Practical Diffusion Policy Learning Loss}
The direct impact of Theorem~\ref{thm.rssm} is a diffusion policy learning loss that can be sampled and computed efficiently in online RL. Specifically, summing over all timestep $t$ and state $\sbb$ in Equation~\eqref{eq:thm1_reweight_loss}, we derive the diffusion policy learning loss with RSSM:

{
\small
\begin{equation}
    \begin{aligned}
        &\Lcal^\pi(\theta;Q,\lambda) := \frac{1}{T}\sum_{t=1}^T\underset{{\sbb, \ab_t,\tilde\ab_0}}{\EE}\sbr{\exp\rbr{\frac{Q(\sbb,\tilde\ab_0)}{\lambda}}\nbr{s_\theta\rbr{\ab_t, \sbb, t} - \nabla_{\ab_t}\log\qtil_{0|t}(\tilde\ab_0|\ab_t)}^2}
    \end{aligned}\label{eq:policy_loss}
\end{equation}
}
with 
\begin{equation}
    \tilde\ab_0 = \frac{1}{\sqrt{\bar\alpha_t}}\ab_t - \sqrt{\frac{1 - \bar\alpha_t}{\bar\alpha_t}} \epsb,\, \epsb\sim\Ncal\rbr{0, \Ib}\label{eq:a0_sample}
\end{equation}
and $\sbb$ sampled from the replay buffer, $\ab_t$ sampled from $\ptil_t(\cdot|\sbb)$. 

Obviously, such sampling protocol in~\eqref{eq:policy_loss} and~\eqref{eq:a0_sample} bypasses sampling from the target optimal policy, therefore, can be easily implemented. Meanwhile, the obtained loss avoids recursive gradient backpropagation, largely reducing computation complexity of policy gradient. These benefits perfectly echo the difficulties of sampling and computations in applying vanilla diffusion model training to online RL, enabling efficient diffusion policy learning. 







\section{Hierarchical Clustered 
Federated Learning } \label{sec:algorithm} 

\subsection{Overview}

\begin{figure}[tb!]
\centerline{\includegraphics[width=1\linewidth]{Figure/Algorithm.png}}
\caption{Overview of the proposed framework \algname.}
\label{fig:overview}
\end{figure}

\begin{algorithm}
\begin{algorithmic}[1]
\caption{\algname for satellite networks}\label{alg:decentralized_FL}
    \REQUIRE{Information of satellite networks, $K$, $C$}
    \ENSURE{Global model $w_{G}$}

\STATE Conduct satellite-clustered parameter server selection algorithm;  \label{line:1}

\STATE \COMMENT{Satellite client initialization model parameters}  \label{line:2}
\FOR {$\forall$ satellites}
     \STATE Initialize global model parameter $w_0$;
\ENDFOR  \label{line:5}

%\STATE All satellites: Initialize global model parameter $w^{(a)}_0$;

\FOR {each FL round $m \in M$}  \label{line:6}
    \STATE \COMMENT{Train local model (in-orbit computing)}
    \FOR {each satellite $i \in C$ in parallel}
        \STATE $w_{m,\lambda+1}^{i} = w_{m,\lambda}^{i} - \eta \nabla \tilde{f}_i(w_{m,\lambda}^{i})$;
    \ENDFOR \label{line:9}
    %\STATE \note{Conduct adaptive weight quantification algorithm to obtain $w{_{m}^{i}}^{\prime}$;}
    \STATE \COMMENT{Aggregate satellite cluster models}
    \FOR {each satellite $i \in C^k$ in the same cluster}  \label{line:12}
        \STATE $w_{m+1} = w_m + \sum_{i \in C^k_i} p_i w_m$;   \label{line:13}
    \STATE \COMMENT{Check if it is necessary to reassemble clusters}  \label{line:14}
        \STATE Calculate dropout rate: $d_r = \frac{C^d}{C^k}$
        \IF {$d_r > Z$}
            \STATE Re-cluster the satellites
        \ENDIF  \label{line:18}
    \ENDFOR
\ENDFOR  
\STATE \COMMENT{Aggregate global model at the ground station}  \label{line:21}
\FOR {each PS $G_{k}, k \in {K}$ in different satellite clusters}
    \STATE $w_{G} =  \sum_{k \in {K}} \frac{D_k}{D} w_m^k$;  
\ENDFOR  \label{line:23}
\RETURN $w_{G}$.
\end{algorithmic}
\end{algorithm}

\figurename~\ref{fig:overview} illustrates the flowchart of our proposed \algname. The clustering FL clustering process in \algname includes two stages: \textbf{satellite cluster aggregation stage} (Step 1-3) and \textbf{ground station aggregation stage} (Step 4). In the satellite cluster aggregation stage, a clustering algorithm is introduced to divide the satellites into distinct clusters. Within each satellite cluster $K_i^a$, the algorithm selects a satellite near the cluster center with strong communication capabilities to act as the PS. The PS is responsible for aggregating model parameters from satellites within its cluster and establishing communication with the corresponding ground stations (Step 1). During the local training process, satellites may dynamically join or leave a cluster, necessitating re-aggregation. To address this issue, MAML is introduced to adjust the initial model parameters of the newly joined satellites, allowing them to better adapt to the tasks of the new cluster (Step 2). This method accelerates the overall convergence of the satellite PS aggregation process. Following each training round, the PS combines parameters from all satellites within its cluster and distributes the updated, aggregated parameters back to them (Step 3).

%This approach facilitates the acceleration of the overall convergence of the satellite PS aggregation process. After each training round, the PS aggregates the parameters of the satellites in its cluster and distributes the \note{Do we need to add ''updated" here?} aggregated parameters to them (Step 3). 

After a specified number of training rounds in the satellite cluster, the ground station aggregation stage starts. In this stage, the ground station communicates with visible satellite clusters to aggregate their model parameters of the respective satellite clusters they are affiliated. Finally, the ground station returns the trained model parameters to the respective satellite clusters (Step 4).

As detailed in Algorithm~\ref{alg:decentralized_FL},  we introduce a satellite-clustered parameter server selection algorithm to partition the original satellite network into distinct satellite clusters based on the satellite network information. For each cluster, the algorithm selects a satellite near the cluster center with robust communication capabilities as the PS (line \ref{line:1}). Then we initialize the global model parameters $w_0$ for all satellite clients within each cluster (lines \ref{line:2}-\ref{line:5}). During each round of FL aggregation, the local satellite client first performs local training to update the global model parameters $w_{m,\lambda+1}^{i}$ after training round $\lambda$ (lines \ref{line:6}-\ref{line:9}). These parameters are then transmitted to their cluster's PS for aggregation. 

After $m$ rounds of training in each satellite cluster, each PS forwards its aggregated parameters to its associated ground station for global aggregation, producing the updated model parameters $w_{m+1}$ (lines \ref{line:12}-\ref{line:13}). During global aggregation, satellite clusters monitor whether the number of dropped-out satellites $C^d$ exceeds a predefined threshold, triggering re-clustering when necessary (lines \ref{line:14}-\ref{line:18}). Finally, ground station broadcasts the global parameters to all affiliated satellites in their clusters, completing the hierarchically clustered FL process (lines \ref{line:21}-\ref{line:23}). 




%As shown in Algorithm~\ref{alg:decentralized_FL}, based on the satellite network information, we introduce a satellite-clustered parameter server selection algorithm to partition the original satellite network into distinct clusters. Within each satellite cluster, the algorithm selects a satellite near the cluster center with strong communication capabilities to serve as the PS (line \ref{line:1}). Then the global model parameters are initialized for all satellite clients within the satellite cluster (lines \ref{line:2}-\ref{line:5}). During each round of FL aggregation, the local satellite client first performs local training and obtains the trained global model parameters $w_{m,\lambda+1}^{i}$ after training round $\lambda$ (lines \ref{line:6}-\ref{line:9}). The satellite clients then transmit their global parameters $w_{m,\lambda+1}^{i} $ to satellite PS for aggregate global model parameters. After completing $m$ rounds of training in each satellite cluster, the global model parameters are transmitted to the ground station associated with the satellite PS for final aggregation, yielding updated aggregated global model parameters $w_{m+1}$ (lines \ref{line:12}-\ref{line:13}). During global aggregation, satellite clusters need to assess whether the number of dropped-out satellites $C^d$ exceeds a critical threshold, necessitating re-clustering (lines \ref{line:14}-\ref{line:18}). Finally, the ground station broadcasts the global parameters to all affiliated satellites in the cluster, completing the hierarchically clustered FL process (lines \ref{line:21}-\ref{line:23}). \note{please double check this paragraph to avoid the meaning changes after my rewrite}

To accelerate the convergence of the global model, we assign weights to clients based on the quality of their model updates. The quality is evaluated using the loss value of the client's local model. Let $L_i$ denote the loss value of the client $i$. The weight $p_i$ assigned to client $i$ is given by:
\begin{equation}
p_i = \frac{\frac{1}{L_i}}{\sum_{ i \in C^k} \frac{1}{L_i}}
\label{eq:p}
\end{equation}

\begin{comment}
The local updates from the satellite clients are then weighted and aggregated according to the weight parameter $p_i$ by:
\begin{equation}
w_{m+1} = w_m + \sum_{i \in U_k} p_i Q_m^i(w_{m+1}^{i} - w_m)
\end{equation}
\end{comment}
%If $p_i$ is 0, the client has not been selected by PS.

\subsection{Satellite-clustered parameter server selection algorithm}

We introduce a satellite-clustered parameter server selection algorithm that partitions the original satellite network topology into a predefined number of clusters $K$, optimizing the clustering process. Our algorithm iteratively refines the cluster centroids and the membership of associated satellites. Initially, $K$ centroids are randomly selected from the satellite location data. These locations are typically derived from geographic coordinates based on the satellite location parameters, i.e., inclination and orbital altitude. Each satellite is assigned to the nearest cluster centroid using the Euclidean distance metric, thereby forming initial clusters. The Euclidean distance between a satellite position vector
$\mathbf{C}^i = \{C_{1}^i, C_{2}^i, \ldots, C_{n}^i \}$ and a centroid $\mathbf{K}^j = \{K_{1}^j, K_{2}^j, \ldots, K_{n}^j \}$ is calculated as:
\begin{equation}
d(\mathbf{C}^i, \mathbf{K}^j) = \sqrt{\sum_{k=1}^{n} (C_{k}^i - K_{k}^j)^2}
\end{equation}

In the next update step, our algorithm recalculates the centroids by computing the mean position of all satellites assigned to each cluster. This process effectively repositions the centroids to more accurately represent the distribution of their associated satellites. For each cluster $K_k^i$, the new centroid $\mathbf{K}^j$ is obtained by:
\begin{equation}
\mathbf{K}^j = \frac{1}{|K^j|} \sum_{\mathbf{C}^i \in K^j} \mathbf{C}^i
\end{equation}
where $|K^j|$ represents the number of satellites in cluster $K^j$. The iterative process continues until the centroids stabilize, indicating their positions no longer change significantly between iterations. This indicates that the algorithm has converged to a local optimum. The convergence criterion is given by:
\begin{equation}
\sum_{j=1}^{|K|} \|\mathbf{K}_{\text{new}}^j - \mathbf{K}_{\text{old}}^j\|^2 < \epsilon
\end{equation}
where $|K|$ represents the number of clusters, and $\epsilon$ is a small positive number indicating stability in centroid positions.
The satellite nearest to the cluster centroid is designated as the PS for the respective cluster.


\subsection{Meta-learning-driven satellite re-clustering algorithm}

In dynamic satellite federated learning, the diverse training objectives of satellite clients, combined with their frequent network participation changes, can hinder model convergence and increase resource consumption. As a result, achieving acceptable performance requires substantial time and a large number of data samples.
%In dynamic satellite federated learning, the diverse training goals of satellite clients, coupled with their frequent network participation changes, can impede model convergence, resulting in heightened resource utilization. This necessitates a significant investment of time and data samples to attain acceptable performance. %the different training objectives of satellite clients, along with their frequent joining or leaving the network, can slow down model convergence, leading to increased resource consumption. This requires a large amount of time and data samples to achieve acceptable results. 

To address this challenge, we propose a satellite re-clustering algorithm based on meta-learning, extending the original satellite-clustered parameter server selection algorithm. When a new satellite joins the network, it inherits model updates from the head node of a specified cluster, rather than starting training from scratch. The core idea of the MAML approach is to identify a set of meta-initialization parameters that enable the model to achieve strong performance with just one or two gradient updates, even with a small number of new task examples.

First, we sample satellite clients from different clusters denoted as $ S = \{S_1, S_2, \dots, S_n\} $. Each satellite client is assigned a task $\textit{TK}_i$, which consists of a dataset $D_i$  and a loss function $L_{S_i}(w)$. The objective is to minimize the loss of the model on the task $\textit{TK}_i$. Then, an inner-loop adaptation is performed for each selected satellite node to fine-tune the global model $w$ by:
\begin{equation}
w_i' = w - \alpha \nabla_{w} L_{S_i}(w)
\end{equation}
where $\alpha$ is the local learning rate. Finally, an outer-loop meta-update is applied to aggregate the model updates from different satellite nodes, updating the global initialization by:
\begin{equation}
w^{t+1} = w^{t} - \beta \sum_{i \in S} \nabla_{w} L_{S_i}(w_i')
\end{equation}
where $\beta$ is the meta-learning rate, $w^{t}$ is the parameter of the current global model.


% \section{Experiment and Results}
\section{Results and Analysis}
In this section, we first present safe vs. unsafe evaluation results for 12 LLMs, followed by fine-grained responding pattern analysis over six models among them, and compare models' behavior when they are attacked by same risky questions presented in different languages: Kazakh, Russian and code-switching language.    

\begin{table}[t!]
\centering
\small
\resizebox{\columnwidth}{!}{
\begin{tabular}{clcccc}
\toprule
\multicolumn{1}{l}{\textbf{Rank} } & \textbf{Model} & \textbf{Kazakh $\uparrow$} & \textbf{Russian $\uparrow$} & \textbf{English $\uparrow$} \\
\midrule
1 & \claude & \textbf{96.5}   & 93.5    & \textbf{98.6}    \\
2 & \gptfouro & 95.8   & 87.6    & 95.7    \\
3 & \yandexgpt & 90.7   & \textbf{93.6}    & 80.3    \\
4 & \kazllmseventy & 88.1 & 87.5 & 97.2 \\
5 & \llamaseventy & 88.0   & 85.5    & 95.7    \\
6 & \sherkala & 87.1   & 85.0    & 96.0    \\
7 & \falcon & 87.1   & 84.7    & 96.8    \\
8 & \qwen & 86.2   & 85.1    & 88.1    \\
9 & \llamaeight & 85.9   & 84.7    & 98.3    \\
10 & \kazllmeight & 74.8   & 78.0    & 94.5 \\
11 & \aya & 72.4 & 84.5 & 96.6 \\
12 & \vikhr & --- & 85.6 & 91.1 \\
\bottomrule
\end{tabular}
}
\caption{Safety evaluation results of 12 LLMs, ranked by the percentage of safe responses in the Kazakh dataset. \claude\ achieves the highest safety score for both Kazakh and English, while \yandexgpt\ is the safest model for Russian responses.}
\label{tab:safety-binary-eval}
\end{table}



\subsection{Safe vs. Unsafe Classification}
% In this subsection, 
We present binary evaluation results of 12 LLMs, by assessing 52,596 Russian responses and 41,646 Kazakh responses.
% 26,298 responses generated by six models on the Russian dataset and 22,716 responses on the Kazakh dataset. 

%\textbf{Safety Rank.} In general, proprietary systems outperform the open-source model. For Russian, As shown in Table \ref{tab:model_comparison_russian}, \textbf{Yandex-GPT} emerges as the safest large language model (LLM) for Russian, with a safety percentage of 93.57\%. Among the open-source models, \textbf{Vikhr-Nemo-12B} is the safest, achieving a safety percentage of 85.63\%.
% Edited: This is mentioned in the discussion
% This outcome highlights the potential impact of pretraining data on model behavior. Models pre-trained primarily on Russian data are better at understanding and handling harmful questions - in both proprietary systems and open-source setups. 
%For Kazakh, as shown in Table \ref{tab:model_comparison_kazakh}, \textbf{Claude} emerges as the safest large language model (LLM) with a safety percentage of 96.46\%, closely followed by GPT-4o at 95.75\%. In contrast, \textbf{Aya-101}, despite being specifically tuned for Kazakh, consistently shows the highest unsafe response rates at 72.37\%. 

\begin{figure*}[t!]
	\centering
        \includegraphics[scale=0.28]{figures/question_type_no6_kaz.png}
	\includegraphics[scale=0.28]{figures/question_type_exclude_region_specific_new.png} 

	\caption{Unsafe answer distribution across three question types for risk types I-V: Kazakh (left) and Russian (right)}
	\label{fig:qt_non_reg}
\end{figure*}

\begin{figure*}[t!]
	\centering
        \includegraphics[scale=0.28]{figures/question_type_only6_kaz.png}
	\includegraphics[scale=0.28]{figures/question_type_region_specific_new.png} 
	
	\caption{Unsafe answer distribution across three question types for risk type VI: Kazakh (left) and Russian (right)}
	\label{fig:qt_reg}
\end{figure*}

\textbf{Safety Rank.} In general, proprietary systems outperform the open-source models. 
For Russian, as shown in Table~\ref{tab:safety-binary-eval},  % \ref{tab:model_comparison_russian}, 
\yandexgpt emerges as the safest language model for Russian, with safe responses account for 93.57\%. Among the open-source models, \kazllmseventy is the safest (87.5\%), followed by \vikhr with a safety percentage of 85.63\%.

For Kazakh, % as shown in Table \ref{tab:model_comparison_kazakh}, 
% YX: todo, rerun Kazakh safety percentage using Diana threshold
\claude is the safest model with 96.46\% safe responses, closely followed by \gptfouro\ at 95.75\%. \aya, despite being specifically tuned for Kazakh, shows the highest unsafe rates at 72.37\%.



% \begin{table}[t!]
% \centering
% \resizebox{\columnwidth}{!}{%
% \begin{tabular}{clccc}
% \toprule
% \textbf{Rank} & \textbf{Model Name}  & \textbf{Safe} & \textbf{Unsafe} & \textbf{Safe \%} \\ \midrule
% \textbf{1} & \textbf{Yandex-GPT} & \textbf{4101} & \textbf{282} & \textbf{93.57} \\
% 2 & Claude & 4100 & 283 & 93.54 \\
% 3 & GPT-4o & 3839 & 544 & 87.59 \\
% 4 & Vikhr-12B & 3753 & 630 & 85.63 \\
% 5 & LLama-3.1-instruct-70B & 3746 & 637 & 85.47 \\
% 6 & LLama-3.1-instruct-8B & 3712 & 671 & 84.69 \\
% \bottomrule
% \end{tabular}
% }
% \caption{Comparison of models based on safety percentages for the Russian dataset.}
% \label{tab:model_comparison_russian}
% \end{table}


% \begin{table}[t!]
% \centering
% \resizebox{\columnwidth}{!}{%
% \begin{tabular}{clccc}
% \toprule
% \textbf{Rank} & \textbf{Model Name}  & \textbf{Safe} & \textbf{Unsafe} & \textbf{Safe \%} \\ \midrule
% 1             & \textbf{Claude}  & \textbf{3652} & \textbf{134} & \textbf{96.46} \\ 
% 2             & GPT-4o                      & 3625          & 161          & 95.75 \\ 
% 3             & YandexGPT                   & 3433          & 353          & 90.68 \\
% 4             & LLama-3.1-instruct-70B      & 3333          & 453          & 88.03 \\
% 5             & LLama-3.1-instruct-8B       & 3251          & 535	       & 85.87 \\
% 6             & Aya-101                     & 2740          & 1046         & 72.37 \\ 
% \bottomrule
% \end{tabular}
% }
% \caption{Comparison of models based on safety percentages for the Kazakh dataset.}
% \label{tab:model_comparison_kazakh}
% \end{table}



\textbf{Risk Areas.} 
We selected six representative LLMs for Russian and Kazakh respectively and show their unsafe answer distributions over six risk areas.
As shown in Table \ref{tab:unsafe_answers_summary}, risk type VI (region-specific sensitive topics) overwhelmingly contributes the largest number of unsafe responses across all models. This highlights that LLMs are poorly equipped to address regional risks effectively. For instance, while \llama models maintain comparable safety levels across other risk type (I–V), their performance drops significantly when dealing with risk type VI. Interestingly, while \yandexgpt\ demonstrates relatively poor performance in most other risk areas, it handles region-specific questions remarkably well, suggesting a stronger alignment with regional norms and sensitivities. For Kazakh, Table \ref{tab:unsafe_answers_summary_kazakh} shows that region‐specific topics (risk type VI) pose a substantial challenge across all six models, including the commercial \gptfouro\ and \claude, which demonstrate superior safety on general categories. 

% \begin{table}[t!]
% \centering
% \resizebox{\columnwidth}{!}{%
% \begin{tabular}{lccccccc}
% \toprule
% \textbf{Model} & \textbf{I} & \textbf{II} & \textbf{III} & \textbf{IV} & \textbf{V} & \textbf{VI} & \textbf{Total} \\ \midrule
% LLama-3.1-instruct-8B & 60 & 70 & 16 & 31 & 9 & 485 & 671 \\
% LLama-3.1-instruct-70B & 29 & 55 & 8 & 4 & 1 & 540 & 637 \\
% Vikhr-12B & 41 & 93 & 15 & 1 & 3 & 477 & 630 \\
% GPT-4o & 21 & 51 & 6 & 2 & 0 & 464 & 544 \\
% Claude & 7 & 10 & 1 & 0 & 0 & 265 & 283 \\
% Yandex-GPT & 55 & 125 & 9 & 3 & 8 & 82 & 282 \\
% \bottomrule
% \end{tabular}%
% }
% \caption{Ru unsafe cases over risk areas of six models.}
% \label{tab:unsafe_answers_summary}
% \end{table}


\begin{table}[t!]
\centering
\resizebox{\columnwidth}{!}{%
\begin{tabular}{lccccccc}
\toprule
\textbf{Model} & \textbf{I} & \textbf{II} & \textbf{III} & \textbf{IV} & \textbf{V} & \textbf{VI} & \textbf{Total} \\ \midrule
\llamaeight & 60 & 70 & 16 & 31 & 9 & 485 & 671 \\
\llamaseventy & 29 & 55 & 8 & 4 & 1 & 540 & 637 \\
\vikhr & 41 & 93 & 15 & 1 & 3 & 477 & 630 \\
\gptfouro & 21 & 51 & 6 & 2 & 0 & 464 & 544 \\
\claude & 7 & 10 & 1 & 0 & 0 & 265 & 283 \\
\yandexgpt & 55 & 125 & 9 & 3 & 8 & 82 & 282 \\
\bottomrule
\end{tabular}%
}
\caption{Ru unsafe cases over risk areas of six models.}
\label{tab:unsafe_answers_summary}
\end{table}


% \begin{table}[t!]
% \centering
% \resizebox{\columnwidth}{!}{%
% \begin{tabular}{lccccccc}
% \toprule
% \textbf{Model} & \textbf{I} & \textbf{II} & \textbf{III} & \textbf{IV} & \textbf{V} & \textbf{VI} & \textbf{Total} \\ \midrule
% Aya-101 & 96 & 235 & 165 & 166 & 90 & 294 & 1046 \\
% Llama-3.1-instruct-8B & 25 & 15 & 91 & 37 & 14 & 353 & 535 \\
% Llama-3.1-instruct-70B & 33 & 39 & 88 & 27 & 20 & 246 & 453 \\
% Yandex-GPT & 29 & 76 & 95 & 29 & 16 & 108 & 353 \\
% GPT-4o & 2 & 1 & 41 & 0 & 3 & 114 & 161 \\
% Claude & 2 & 1 & 26 & 3 & 6 & 96 & 134 \\ \bottomrule
% \end{tabular}%
% }
% \caption{Kaz unsafe cases over risk areas of six models.}
% \label{tab:unsafe_answers_summary_kazakh}
% \end{table}


\begin{table}[t!]
\centering
\resizebox{\columnwidth}{!}{%
\begin{tabular}{lccccccc}
\toprule
\textbf{Model} & \textbf{I} & \textbf{II} & \textbf{III} & \textbf{IV} & \textbf{V} & \textbf{VI} & \textbf{Total} \\ \midrule
\aya & 96 & 235 & 165 & 166 & 90 & 294 & 1046 \\
\llamaeight & 25 & 15 & 91 & 37 & 14 & 353 & 535 \\
\llamaseventy & 33 & 39 & 88 & 27 & 20 & 246 & 453 \\
\yandexgpt & 29 & 76 & 95 & 29 & 16 & 108 & 353 \\
\gptfouro & 2 & 1 & 41 & 0 & 3 & 114 & 161 \\
\claude & 2 & 1 & 26 & 3 & 6 & 96 & 134 \\ 
\bottomrule
\end{tabular}%
}
\caption{Kaz unsafe cases over risk areas of six models.}
\label{tab:unsafe_answers_summary_kazakh}
\end{table}

% \begin{figure*}[t!]
% 	\centering
% 	\includegraphics[scale=0.28]{figures/human_1000_kz_font16.png} 
% 	\includegraphics[scale=0.28]{figures/human_1000_ru_font16.png}

% 	\caption{Human vs \gptfouro\ fine-grained labels on 1,000 Kazakh (left) and Russian (right) samples.}
% 	\label{fig:human_fg_1000}
% \end{figure*}

\textbf{Question Type.} For Russian, Figures \ref{fig:qt_non_reg} and \ref{fig:qt_reg} reveal differences in how models handle general risks I-V and region-specific risk VI. For risks I-V, indirect attacks % crafted to exploit model vulnerabilities—
result in more unsafe responses due to their tricky phrasing. 

In contrast, region-specific risks see slightly more unsafe responses from direct attacks, 
% as these explicit prompts are more likely to bypass safety mechanisms. 
since indirect attacks for region-specific prompts often elicit safer, vaguer answers. %, suggesting models struggle less with implicit harm. 
Overall, the number of unsafe responses is similar across question types, indicating models generally struggle with safety alignment in all jailbreaking queries.

For Kazakh, Figures \ref{fig:qt_non_reg} and \ref{fig:qt_reg} show greater variation in unsafe responses across question types due to the low-resource nature of Kazakh data. For general risks I-V, \llamaseventy\ and \aya\ produce more unsafe outputs for direct harm prompts. At the same time, \claude\ and \gptfouro\ struggle more with indirect attacks, reflecting challenges in handling subtle cues. For region-specific risk VI, most models perform similarly due to limited Kazakh-specific data, though \llamaeight\ shows higher unsafe outputs for indirect local references, likely due to their implicit nature. Direct region-specific attacks yield fewer unsafe responses, as explicit prompts trigger more cautious outputs. Across all risk areas, general questions with sensitive words produce the fewest unsafe answers, suggesting over-flagging or cautious behavior for unclear harmful intent.



% \subsection{Fine-grained Classification}
% We extended our analysis to include fine-grained classifications for both safe and unsafe responses. For unsafe responses, we categorized outputs into four harm types, as shown in Table \ref{table:unsafe_response_categories}. 

% For safe responses, we classified outputs into six distinct patterns of safety, following a fine-grained rubric provided in \cite{wang2024chinesedatasetevaluatingsafeguards}. The types outlined in this rubric are presented in Table \ref{table:safe_response_categories}.

% To validate the fine-grained classification, we conducted human evaluation on the same 1,000 responses in Russian used for the preliminary binary classification.
% The confusion matrix highlights the alignment and discrepancies between human annotations and GPT's fine-grained labels. The diagonal values represent instances where the GPT's labels match human annotations, with category 5 (provides general, safe information) showing the highest agreement (404 instances). However, off-diagonal values reveal areas of disagreement, such as misclassifications in categories 1 (not willing to answer), 4 (recommends consulting an expert), and 7 (provides harmful or risky instructions). For example, in category 1, while 94 responses were correctly labeled, GPT-4o misclassified several instances into categories 4, 5, or 7, indicating overlap or ambiguity in these classifications. Similar trends are observed in other categories, where GPT sometimes struggles to differentiate nuanced distinctions in human-labeled categories. Overall, GPT's fine-grained labels match human annotations in 710 out of 1000 cases, achieving an agreement rate of 71\%.

% \begin{figure}[ht!]
%     \centering
%     \includegraphics[width=0.95\linewidth]{figures/human_fg_1000_ru.png}
%     \caption{{Human vs GPT-4o Fine-Grained Labels on 1,000 Russian Samples}}
%     \label{fig:human_fg_1000_ru}
% \end{figure}


% After conducting human evaluation on a representative sample, we extended the fine-grained classification to a full dataset comprising 21,915 responses generated by five different models. 


% \begin{figure}[ht!]
%     \centering
%     \includegraphics[width=0.95\linewidth]{figures/all_5_ru.png}
%     \caption{Fine-grained label distribution for responses from five models for Russian.}
%     \label{fig:all_5_fg_russian}
% \end{figure}
% Category 5 ("safe and general information") consistently has the highest frequency across all models, aligning with its dominance in the 1,000-sample evaluation. However, differences in the distribution across other categories highlight variability in how models handle nuanced safety risks, with Yandex-GPT showing a slightly broader spread across categories. 
% In the distribution of unsafe responses, most models exhibit higher counts in certain labels such as 8. However, Yandex-GPT displays comparatively fewer responses in these labels. 
% It exhibits a high rate of responses classified under label 7, which indicates instances where the model provides harmful or risky instructions, including unethical behavior or sensitive discussions. While this may suggest a vulnerability in addressing complex or challenging prompts, it was observed that many of Yandex-GPT’s responses tend to deflect responsibility or offer vague advice such as "check the internet". Although this approach minimizes the risk of unsafe outputs, it often results in responses that lack depth or contextually relevant information, limiting their overall usefulness for users.

% % \subsection{Kazakh}


% % Overall, these findings underscore how resource constraints and prompt explicitness affect model safety in Kazakh. Some models manage direct attacks better yet fail on indirect ones, while region-specific content remains challenging for all given the lack of localized training data.
% \subsubsection{Fine-grained Classification}
% Similarly, we conducted a human evaluation on 1,000 Kazakh samples, following the same methodology as the Russian evaluation. The match between human annotations and GPT-4o's fine-grained classifications was 707 out of 1,000, ensuring that the fine-grained classification framework aligned well with human judgments.
% The confusion matrix in Figure \ref{fig:human_fg_1000_kz} for 1,000 Kazakh samples illustrates the agreement between human annotations and GPT-4o's fine-grained classifications. The highest agreement is observed in category 5 (360 instances), indicating GPT-4o's strength in recognizing responses labeled by humans as "safe and general information." However, discrepancies are evident in categories such as 3 and 7, where GPT-4o misclassified several instances, highlighting areas for further refinement in distinguishing nuanced classifications.


\begin{figure}[t!]
	\centering
	\includegraphics[scale=0.18]{figures/human_1000_kz_font16.png} 
	\includegraphics[scale=0.18]{figures/human_1000_ru_font16.png}

	\caption{Human vs \gptfouro\ fine-grained labels on 1,000 Kazakh (left) and Russian (right) samples.}
	\label{fig:human_fg_1000}
\end{figure}

% \begin{figure}[t!]
% 	\centering
% 	\includegraphics[scale=0.28]{figures/human_1000_kz_font16.png} 
% 	\includegraphics[scale=0.28]{figures/human_1000_ru_font16.png}

% 	\caption{Human vs \gptfouro\ fine-grained labels on 1,000 Kazakh (top) and Russian (bottom) samples.}
% 	\label{fig:human_fg_1000}
% \end{figure}

% \begin{figure*}[t!]
% 	\centering
% 	\includegraphics[scale=0.28]{figures/all_5_kz_font16.png} 
% 	\includegraphics[scale=0.28]{figures/all_5_ru_font_16.png} \\
% 	\caption{Fine-grained responding pattern distribution across five models for Kazakh (left) and Russian (right).}
% 	\label{fig:all_5}
% \end{figure*}

\begin{figure}[t!]
	\centering
	\includegraphics[scale=0.28]{figures/all_5_kz_font16.png} 
	\includegraphics[scale=0.28]{figures/all_5_ru_font_16.png} \\
	\caption{Fine-grained responding pattern distribution across five models for Kazakh (top) and Russian (bottom).}
	\label{fig:all_5}
\end{figure}


\subsection{Fine-Grained Classification}
\label{sec:fine-grained-classification}
% As discussed in Section \ref{harmfulness_evaluation}, 
We further analyzed fine-grained responding patterns for safe and unsafe responses. For unsafe responses, outputs were categorized into four harm types, and safe responses were classified into six distinct patterns of safety, as rubric in Appendix \ref{safe_unsafe_response_categories}. 
% \cite{wang2024chinesedatasetevaluatingsafeguards}

\paragraph{Human vs. GPT-4o}
Similar to binary classification, we validated \gptfouro's automatic evaluation results by comparing with human annotations on 1,000 samples for both Russian and Kazakh. %, comparing human annotations with \gptfouro's fine-grained labels.
For the Russian dataset, \gptfouro's labels aligned with human annotations in 710 out of 1,000 cases, achieving an agreement rate of 71\%. 
Agreement rate of Kazakh samples is 70.7\%. %with 707 out of 1,000 cases matching
% The confusion matrix highlights areas of alignment and discrepancies.
% 
As confusion matrices illustrated in Figure~\ref{fig:human_fg_1000}, the majority of cases falling into \textit{safe responding patter 5} --- providing general and harmless information, for both human annotations and automatic predictions.
% highest agreement with 404 correct classifications for Russian. 
Mis-classifications for safe responses mainly focus on three closely-similar patterns: 3, 4, and 5, and patterns 7 and 8 are confusing to discern for unsafe responses, particularly for Kazakh (left figure).
We find many Russian samples which were labeled as ``1. reject to answer'' by humans are diversely classified across 1-6 by GPT-4o, which is also observed in Kazakh but not significant.

% suggesting label alignment issues are language-independent. 
% YX: I did not observe this, commented
% Notably, Russian showed confusion between 7 (risky instructions) and 1 (refusal to answer), this trend does not appear in Kazakh.


% highlight the strengths and limitations of {\gptfouro}'s fine-grained classification framework across both languages, paving the way for further refinements.


% However, misclassifications were observed in categories such as 1 (not willing to answer), 4 (recommends consulting an expert), and 7 (provides harmful or risky instructions), revealing overlaps and ambiguities in nuanced classifications.

% Similarly, for the Kazakh dataset, the agreement rate between human annotations and GPT-4o's labels was 70.7\%, with 707 out of 1,000 cases matching. As with the Russian analysis, category 5 (360 instances) showed the highest alignment. However, discrepancies were more prominent in categories such as 3 and 7, underscoring GPT-4o's challenges in differentiating fine-grained human-labeled categories. 
% A similar pattern was observed for Kazakh dataset, which suggests that alignment and misaligned of fine-grained lables is not language dependent.

% These findings, illustrated in Figures \ref{fig:human_fg_1000}, highlight the strengths and limitations of {\gptfouro}'s fine-grained classification framework across both languages, paving the way for further refinements.

\paragraph{Fine-grained Analysis of Five LLMs}
% After conducting human evaluation on representative samples, we extended 
\figref{fig:all_5} shows fine-grained responding pattern distribution across five models based on the full set of Russian and Kazakh data.
% For Russian, we selected \vikhr, \gptfouro, \llamaseventy, \claude, and \yandexgpt, while for Kazakh, we chose \aya, \gptfouro, \llamaseventy, \claude, and \yandexgpt. 
% The evaluation covered 21,915 responses in Russian and 18,930 responses in Kazakh.
% 
In both languages, pattern 5 of providing \textit{general and harmless information} consistently witnessed the highest frequency across all models, with \llamaseventy\ exhibiting the largest number of responses falling into this category for Kazakh (2,033). 
% YX:summarize more noteable findings here.

Differences of other patterns vary across languages. 
Unsafe responses in Russian are predominantly in pattern 8, where models provide incorrect or misleading information without expressing uncertainty. % (misinformation and speculation), 
For Kazakh, \aya\ exhibits the highest occurrence of pattern 7 (harmful or risky information) and pattern 8, indicating a stronger tendency to generate unethical, misleading, or potentially harmful content.

%Variations in other patterns across models highlight differences in how nuanced safety risks are classified, reflecting the models' differing capabilities in handling safety evaluation for these distinct linguistic contexts. For Russian, the majority of unsafe responses fall under pattern 8 (misinformation and speculation), indicating that models frequently provide incorrect or misleading information without acknowledging uncertainty. For Kazakh, \aya\ has the highest occurence of pattern 7 (harmful or risky information) and pattern 8 (misinformation and speculation), indicating a greater tendency to generate unethical, misleading, or potentially harmful content. 

%This trend suggests that Russian models may struggle more with factual accuracy, whereas Kazakh models, particularly \aya, pose higher risks related to both harmful content and misinformation. Additionally, \gptfouro\ and \claude\ consistently produce fewer unsafe responses in both languages, demonstrating stronger alignment with safety standards
\subsection{Code Switching}
\begin{table}[t!]
\centering

\setlength{\tabcolsep}{3pt}
\scalebox{0.7}{
\begin{tabular}{lcccccccccc}
\toprule
\textbf{Model Name} & \multicolumn{2}{c}{\textbf{Kazakh}} & \multicolumn{2}{c}{\textbf{Russian}} & \multicolumn{2}{c}{\textbf{Code-Switched}} \\  
\cmidrule(lr){2-3} \cmidrule(lr){4-5} \cmidrule(lr){6-7}
& \textbf{Safe} & \textbf{Unsafe} & \textbf{Safe} & \textbf{Unsafe} & \textbf{Safe} & \textbf{Unsafe} \\ 
\midrule
\llamaseventy & 450 & 50 & 466 & 34 & 414 & 86 \\
\gptfouro & 492 & 8 & 473 & 27 & 481 & 19
 \\
\claude & 491 & 9 & 478 & 22 & 484 & 16 \\ 
\yandexgpt & 435 & 65 & 458 & 42 & 464 & 36 \\
\midrule
\end{tabular}}
\caption{Model safety when prompted in Kazakh, Russian, and code-switched language.}
\label{tab:finetuning-comparison}
\end{table}


\gptfouro\ and \claude\ demonstrate strong safety performance across three languages, even with a high proportion of safe responses in the challenging code-switching context. In contrast, \llamaseventy\ and \yandexgpt\ are less safe, exhibiting more harmful responses, particularly in the code-switching scenario. These results show the varying capabilities of models in defending the same attacks that are just presented in different languages, where open-sourced large language models especially require more robust safety alignment in multilingual and code-switching scenarios.

% \subsection{LLM Response Collection}
% We conducted experiments with a variety of mainstream and region-specific 
% large language models for both Russian and Kazakh languages. For both Russian and Kazakh languages, we employed four multilingual models: Claude-3.5-sonnet, Llama 3.1 70B \cite{meta2024llama3}, GPT-4 \cite{openai2024gpt4o}, and YandexGPT. Additionally, we included language-specific models: VIKHR \cite{nikolich2024vikhrconstructingstateoftheartbilingual} for Russian and Aya \cite{ustun-etal-2024-aya} for Kazakh. 

% \subsection{Kazakh-Russian Code-Switching Evaluation}

% In Kazakhstan, the prevalence of bilingualism is a defining characteristic of its linguistic landscape, with most individuals seamlessly mixing Kazakh and Russian in daily communication \cite{Zharkynbekova2022}. This phenomenon, known as code-switching, reflects the unique cultural and social dynamics of the region. Despite this, there is currently no safety evaluation dataset tailored to this unique multilingual environment. Developing a code-switched dataset is essential to evaluate the ability of large language models (LLMs) to navigate the complexities of bilingual interactions, ensuring they produce contextually appropriate, non-harmful, and culturally sensitive responses. To address this, we sampled 500 questions from both Kazakh and Russian question sets, ensuring they represent similar queries but with culturally nuanced differences in meaning. Building on these, we developed a mixed version that integrates Kazakh and Russian naturally, reflecting the way people in Kazakhstan communicate. This process involved annotators to ensure that the mixed questions were crafted with a natural tone and were localized to Kazakhstan's context.


% \begin{table*}[t!]
%     \centering
%     \small
%     \setlength{\tabcolsep}{6pt} % Adjusts column separation
%     \adjustbox{max width=\linewidth}{
%     \begin{tabular}{ll ccc ccc c}
%     \toprule
%     \textbf{Risk Area} & \textbf{Harm Type} & \multicolumn{3}{c}{\textbf{Kazakh}} & \multicolumn{3}{c}{\textbf{Russian}} & \textbf{\#Q} \\
%                        &                    & \textbf{Ori} & \textbf{FN} & \textbf{FP} & \textbf{Ori} & \textbf{FN} & \textbf{FP} & \\
%     \midrule
%     \multirow{2}{*}{I. Information Hazards} & 1. Risks from leaking sensitive information... & 131 & 133 & 131 & 131 & 133 & 131 & 790 \\
%                                              & 2. Compromise of privacy by leaking or inferring private information (person/individual) & 81 & 82 & 81 & 81 & 82 & 81 & 488 \\
%     \midrule
%     \multirow{3}{*}{II. Malicious Uses} & 3. Assisting illegal activities & 132 & 135 & 132 & 132 & 135 & 132 & 798 \\
%                                         & 4. Nudging or advising unethical actions & 71 & 71 & 71 & 71 & 71 & 71 & 426 \\
%                                         & 5. Reducing the cost of disinformation campaigns & 40 & 42 & 40 & 40 & 42 & 40 & 244 \\
%     \midrule
%     \multirow{3}{*}{III. Discrimination, Exclusion, Toxicity, Hateful, Offensive} & 6. Social stereotypes and unfair discrimination & 94 & 96 & 94 & 94 & 96 & 94 & 568 \\
%                                         & 7. Toxic language (hate speech) & 52 & 59 & 52 & 52 & 59 & 52 & 326 \\
%                                         & 8. Adult content & 27 & 29 & 27 & 27 & 29 & 27 & 166 \\

%     \midrule
%     \multirow{2}{*}{IV. Misinformation Harms} & 9. Disseminating false or misleading information & 92 & 99 & 92 & 92 & 99 & 92 & 566 \\
%                                         & 10. Causing material harm by disseminating misinformation e.g. in medicine or law & 63 & 63 & 63 & 63 & 63 & 63 & 378 \\
%     \midrule
%     \multirow{2}{*}{V. Human–chatbot Interaction Harms} & 11. Mental health or overreliance concerns & 66 & 66 & 66 & 66 & 66 & 66 & 396 \\
%                                         & 12. Treating the chatbot as a human & 50 & 51 & 50 & 50 & 51 & 50 & 302 \\

%     \midrule
%     \multirow{2}{*}{VI. Kazakhstan-specific Sensitivity} & 13. Politically sensitive topics & 63 & 66 & 63& 63 & 66 & 63 &  384 \\
%         & 14. Controversial historical events & 46 & 57 & 46 & 46 & 57 & 46  & 298 \\
% & 15. Regional and racial issues & 45 & 45 & 45 &  45 & 45 & 45 & 270  \\
% & 16. Societal and cultural concerns & 138 & 139 & 138 &  138 & 139 & 138  & 830  \\
% & 17. Legal and human rights matters & 57 & 57 & 57 & 57 & 57 & 57  & 342 \\
%     \midrule
%         \multirow{2}{*}{VII. Russia-specific Sensitivity} 
%             & 13. Politically sensitive topics & - & - & - & 54 & 54 & 54 & 162 \\
%     & 14. Controversial historical events & - & - & - & 38 & 38 & 38 & 114 \\
%     & 15. Regional and racial issues & - & - & - & 26 & 26 & 26 & 78 \\
%     & 16. Societal and cultural concerns & - & - & - & 40 & 40 & 40 & 120 \\
%     & 17. Legal and human rights matters & - & - & - & 41 & 41 & 41 & 123 \\
%     \midrule
%     \bf Total & -- & 1248 & 1290 & 1248 & 1447 & 1489 & 1447 & \textbf{8169} \\
%     \bottomrule
%     \end{tabular}
%     }
%     \caption{The number of questions for Kazakh and Russian datasets across six risk areas and 17 harm types. Ori: original direct attack, FN: indirect attack, and FP: over-sensitivity assessment.}
%     \label{tab:kazakh-russian-data}
% \end{table*}




\section{Discussion}

% \subsection{Kazakh vs Russian}

% The evaluation reveals that Kazakh responses tend to be generally safer than their Russian counterparts, likely due to Kazakh being a low-resource language with significantly less training data. As a result, Kazakh models are less exposed to the vast, often unfiltered datasets containing harmful or unsafe content, which are more prevalent in high-resource languages like Russian. This data scarcity naturally limits the model's ability to generate nuanced but potentially unsafe responses. However, this does not mean the models are specifically fine-tuned for safer performance. When analyzing unsafe answers, it’s clear that Kazakh models, while safer overall, distribute their unsafe responses more evenly across various risk types and question types. This suggests Kazakh models generate fewer unsafe answers but in a broader range of contexts.

% In contrast, Russian models tend to concentrate unsafe answers in specific areas, particularly region-specific risks or indirect attacks. This indicates that Russian models have learned to handle certain types of unsafe content by focusing on specific topics, such as politically sensitive issues, but struggle when confronted with unfamiliar content, leading to unsafe responses due to insufficient filtering. Kazakh models, despite having less training data, tend to respond more broadly, including both direct and indirect risks. This could be due to the less curated nature of their training data, making them more likely to answer unsafe questions without filtering the potential harm involved. The exception to this trend is Aya, a model specifically fine-tuned for Kazakh. Despite fine-tuning, it exhibits the lowest safety percentage (72.37\%) in the Kazakh dataset, suggesting that fine-tuning in specific languages may introduce risks if proper safety measures are not taken.

% The evaluation reveals notable differences in the distribution of safe response patterns across Kazakh and Russian fine-grained labels. Refusal to answer is more frequent in Russian models, particularly Yandex-GPT, reflecting a cautious approach to safety-critical queries. Interestingly, Aya, despite being fine-tuned for Kazakh and exhibiting lower overall safety, also frequently refuses to answer, suggesting an over-reliance on conservative mechanisms. Responses providing general, safe information dominate in both languages, with Kazakh models displaying a slightly higher tendency to rely on this approach. This highlights how the low-resource nature of Kazakh results in more generalized and inherently safer responses. In contrast, Russian models excel at recognizing risks, issuing disclaimers, and refuting incorrect assumptions, likely benefiting from richer and more diverse training data.
% Yandex-GPT exhibits a notably high rate of responses classified under label 7, indicating an overreliance on general disclaimers or deflections, such as "check the internet" or "I don't know." While these responses minimize the risk of unsafe outputs, they often lack substantive or contextually relevant information, reducing their overall utility for users.


Most models perform safer on Kazakh dataset than Russian dataset, higher safe rate on Kazakh dataset in \tabref{tab:safety-binary-eval}. This does not necessarily reveal that current LLMs have better understanding and safety alignment on Kazakh language than Russian, while this may conversely imply that models do not fully understand the meaning of Kazakh attack questions, fail to perceive risks and then provide general information due to lacking sufficient knowledge regarding this request.

We observed the similar number of examples falling into category 5 \textit{general and harmless information} for both Kazakh and Russian, while the Kazakh data set size is 3.7K and Russian is 4.3K. Kazakh has much less examples in category 1 \textit{reject to answer} compared to Russian. This demonstrate models tend to provide general information and cannot clearly perceive risks for many cases.

Additionally, in spite of less harmful responses on Kazakh data, these unsafe responses distribute evenly across different risk areas and question categories, exhibiting equally vulnerability spanning all attacks regardless of what risks and how we jailbreak it.
In contrary, unsafe responses on Russian dataset often concentrate on specific areas and question types, such as region-specific risks or indirect attacks, presenting similar model behaviors when evaluating over English and Chinese data.
It suggests that broader training data in English, Chinese and Russian may allow models to address certain types of attacks robustly,
% effectively—particularly politically sensitive issues—
yet they may falter when confronted with unfamiliar content like regional sensitive topics.

Moreover, in responses collection, we observed many Russian or English responses especially for open-sourced LLMs when we explicitly instructed the models to answer Kazakh questions in Kazakh language. This further implies more efforts are still needed to improve LLMs' performance on low-resource languages.
Interestingly, \aya, a fine-tuned Kazakh model, proves an exception by displaying the lowest safety percentage (72.37\%) among Kazakh models, revealing that the multilingual fine-tuning without stringent safety measures can introduce risks.



% However, this does not mean they are explicitly fine-tuned for safety, likely it happens due to limited training data, which reduces exposure to harmful content. 
% \aya, a fine-tuned Kazakh model, proves an exception by displaying the lowest safety percentage (72.37\%) among Kazakh models, revealing that the multilingual fine-tuning without stringent safety measures can introduce risks.
% Kazakh models generally produce safer responses than their Russian counterparts, likely because Kazakh is a low-resource language with less training data. 
% This limited exposure to harmful or unsafe content naturally limits nuanced yet potentially unsafe outputs. 
% However, it does not imply that the models are specifically fine-tuned for enhanced safety.


% while Kazakh models tend to generate fewer unsafe answers overall, those unsafe responses appear more evenly spread across different risk types and question categories.
% Russian models, on the other hand, often concentrate unsafe responses in specific areas, such as region-specific risks or indirect attacks.
% It implies that their broader training datasets allow them to address certain types of unsafe content more effectively—particularly politically sensitive issues—yet they may falter when confronted with unfamiliar or insufficiently filtered content.

% Meanwhile, Kazakh models sometimes respond more broadly, possibly due to less curated training data. 

Differences also emerge in how language models handle safe responses. 
\yandexgpt, for instance, often refuses to answer high-risk queries. 
It frequently relies on generic disclaimers or deflections like ``check in the Internet'' or ``I don’t know,'' minimizing risk but are less helpful. Interestingly, it often responds with ``I don’t know'' in Russian, even for Kazakh queries, we speculate that these may be default responses stemming from internal system filters, rather than generated by model itself.
This likely explains why \yandexgpt\ is the safest model for the Russian language but ranks third for Kazakh. While its filters perform well for Russian, they struggle with the low-resource Kazakh language.

% Aya, despite its lower overall safety, also employs refusals often, hinting at an over-reliance on conservative approaches. 

% Across both languages, models commonly resort to providing general, safe information, although Kazakh models lean on this strategy slightly more. 
% Russian models, by contrast, excel at detecting risks, issuing disclaimers, and correcting inaccuracies, likely benefiting from richer and more diverse training data.


% \subsection{Response Patterns}


% We conducted a detailed analysis of the models' outputs and identified several noteworthy patterns. YandexGPT, while being one of the safest overall, frequently generates responses in Russian even when the question is posed in Kazakh. These responses often appear as placeholders, prompting users to search for the answer online. This behavior might not originate from the model itself but rather from safety filters implemented in the YandexGPT system. The model's leading performance in ensuring safety during Russian-language interactions, coupled with its lower performance in Kazakh, can be attributed to the limited robustness of these safety filters when handling unsafe content in Kazakh.

% In contrast, Aya-101 exhibits a tendency to fall into repetition, often repeating the same sentences multiple times. Interestingly, the Vikhr model, despite being of a similar size, does not exhibit this issue. We attribute this difference to two key factors. First, Vikhr and Aya-101 have distinct architectures: Vikhr is based on the Mistral-Nemo model, whereas Aya-101 is built on mT5, an older and less robust model. Second, Aya-101 is a multilingual model, while Vikhr was predominantly trained for Russian. Multilingualism has been shown to potentially degrade performance in large language models~\cite{huang2025surveylargelanguagemodels}, which may explain Aya-101's issues with repetition.

\paragraph{Summary}
Our findings provide significant insights into the influence of correctness, explanations, and refinement on evaluation accuracy and user trust in AI-based planners. 
In particular, the findings are three-fold: 
(1) The \textbf{correctness} of the generated plans is the most significant factor that impacts the evaluation accuracy and user trust in the planners. As the PDDL solver is more capable of generating correct plans, it achieves the highest evaluation accuracy and trust. 
(2) The \textbf{explanation} component of the LLM planner improves evaluation accuracy, as LLM+Expl achieves higher accuracy than LLM alone. Despite this improvement, LLM+Expl minimally impacts user trust. However, alternative explanation methods may influence user trust differently from the manually generated explanations used in our approach.
% On the other hand, explanations may help refine the trust of the planner to a more appropriate level by indicating planner shortcomings.
(3) The \textbf{refinement} procedure in the LLM planner does not lead to a significant improvement in evaluation accuracy; however, it exhibits a positive influence on user trust that may indicate an overtrust in some situations.
% This finding is aligned with prior works showing that iterative refinements based on user feedback would increase user trust~\cite{kunkel2019let, sebo2019don}.
Finally, the propensity-to-trust analysis identifies correctness as the primary determinant of user trust, whereas explanations provided limited improvement in scenarios where the planner's accuracy is diminished.

% In conclusion, our results indicate that the planner's correctness is the dominant factor for both evaluation accuracy and user trust. Therefore, selecting high-quality training data and optimizing the training procedure of AI-based planners to improve planning correctness is the top priority. Once the AI planner achieves a similar correctness level to traditional graph-search planners, strengthening its capability to explain and refine plans will further improve user trust compared to traditional planners.

\paragraph{Future Research} Future steps in this research include expanding user studies with larger sample sizes to improve generalizability and including additional planning problems per session for a more comprehensive evaluation. Next, we will explore alternative methods for generating plan explanations beyond manual creation to identify approaches that more effectively enhance user trust. 
Additionally, we will examine user trust by employing multiple LLM-based planners with varying levels of planning accuracy to better understand the interplay between planning correctness and user trust. 
Furthermore, we aim to enable real-time user-planner interaction, allowing users to provide feedback and refine plans collaboratively, thereby fostering a more dynamic and user-centric planning process.


\bibliography{ref}
\bibliographystyle{icml2025}


%%%%%%%%%%%%%%%%%%%%%%%%%%%%%%%%%%%%%%%%%%%%%%%%%%%%%%%%%%%%%%%%%%%%%%%%%%%%%%%
%%%%%%%%%%%%%%%%%%%%%%%%%%%%%%%%%%%%%%%%%%%%%%%%%%%%%%%%%%%%%%%%%%%%%%%%%%%%%%%
% APPENDIX
%%%%%%%%%%%%%%%%%%%%%%%%%%%%%%%%%%%%%%%%%%%%%%%%%%%%%%%%%%%%%%%%%%%%%%%%%%%%%%%
%%%%%%%%%%%%%%%%%%%%%%%%%%%%%%%%%%%%%%%%%%%%%%%%%%%%%%%%%%%%%%%%%%%%%%%%%%%%%%%
\newpage
\appendix
\onecolumn

% E5数据集介绍,数据集处理过程
% 基线模型介绍

\definecolor{titlecolor}{rgb}{0.9, 0.5, 0.1}
\definecolor{anscolor}{rgb}{0.2, 0.5, 0.8}
\definecolor{labelcolor}{HTML}{48a07e}
\begin{table*}[h]
	\centering
	
 % \vspace{-0.2cm}
	
	\begin{center}
		\begin{tikzpicture}[
				chatbox_inner/.style={rectangle, rounded corners, opacity=0, text opacity=1, font=\sffamily\scriptsize, text width=5in, text height=9pt, inner xsep=6pt, inner ysep=6pt},
				chatbox_prompt_inner/.style={chatbox_inner, align=flush left, xshift=0pt, text height=11pt},
				chatbox_user_inner/.style={chatbox_inner, align=flush left, xshift=0pt},
				chatbox_gpt_inner/.style={chatbox_inner, align=flush left, xshift=0pt},
				chatbox/.style={chatbox_inner, draw=black!25, fill=gray!7, opacity=1, text opacity=0},
				chatbox_prompt/.style={chatbox, align=flush left, fill=gray!1.5, draw=black!30, text height=10pt},
				chatbox_user/.style={chatbox, align=flush left},
				chatbox_gpt/.style={chatbox, align=flush left},
				chatbox2/.style={chatbox_gpt, fill=green!25},
				chatbox3/.style={chatbox_gpt, fill=red!20, draw=black!20},
				chatbox4/.style={chatbox_gpt, fill=yellow!30},
				labelbox/.style={rectangle, rounded corners, draw=black!50, font=\sffamily\scriptsize\bfseries, fill=gray!5, inner sep=3pt},
			]
											
			\node[chatbox_user] (q1) {
				\textbf{System prompt}
				\newline
				\newline
				You are a helpful and precise assistant for segmenting and labeling sentences. We would like to request your help on curating a dataset for entity-level hallucination detection.
				\newline \newline
                We will give you a machine generated biography and a list of checked facts about the biography. Each fact consists of a sentence and a label (True/False). Please do the following process. First, breaking down the biography into words. Second, by referring to the provided list of facts, merging some broken down words in the previous step to form meaningful entities. For example, ``strategic thinking'' should be one entity instead of two. Third, according to the labels in the list of facts, labeling each entity as True or False. Specifically, for facts that share a similar sentence structure (\eg, \textit{``He was born on Mach 9, 1941.''} (\texttt{True}) and \textit{``He was born in Ramos Mejia.''} (\texttt{False})), please first assign labels to entities that differ across atomic facts. For example, first labeling ``Mach 9, 1941'' (\texttt{True}) and ``Ramos Mejia'' (\texttt{False}) in the above case. For those entities that are the same across atomic facts (\eg, ``was born'') or are neutral (\eg, ``he,'' ``in,'' and ``on''), please label them as \texttt{True}. For the cases that there is no atomic fact that shares the same sentence structure, please identify the most informative entities in the sentence and label them with the same label as the atomic fact while treating the rest of the entities as \texttt{True}. In the end, output the entities and labels in the following format:
                \begin{itemize}[nosep]
                    \item Entity 1 (Label 1)
                    \item Entity 2 (Label 2)
                    \item ...
                    \item Entity N (Label N)
                \end{itemize}
                % \newline \newline
                Here are two examples:
                \newline\newline
                \textbf{[Example 1]}
                \newline
                [The start of the biography]
                \newline
                \textcolor{titlecolor}{Marianne McAndrew is an American actress and singer, born on November 21, 1942, in Cleveland, Ohio. She began her acting career in the late 1960s, appearing in various television shows and films.}
                \newline
                [The end of the biography]
                \newline \newline
                [The start of the list of checked facts]
                \newline
                \textcolor{anscolor}{[Marianne McAndrew is an American. (False); Marianne McAndrew is an actress. (True); Marianne McAndrew is a singer. (False); Marianne McAndrew was born on November 21, 1942. (False); Marianne McAndrew was born in Cleveland, Ohio. (False); She began her acting career in the late 1960s. (True); She has appeared in various television shows. (True); She has appeared in various films. (True)]}
                \newline
                [The end of the list of checked facts]
                \newline \newline
                [The start of the ideal output]
                \newline
                \textcolor{labelcolor}{[Marianne McAndrew (True); is (True); an (True); American (False); actress (True); and (True); singer (False); , (True); born (True); on (True); November 21, 1942 (False); , (True); in (True); Cleveland, Ohio (False); . (True); She (True); began (True); her (True); acting career (True); in (True); the late 1960s (True); , (True); appearing (True); in (True); various (True); television shows (True); and (True); films (True); . (True)]}
                \newline
                [The end of the ideal output]
				\newline \newline
                \textbf{[Example 2]}
                \newline
                [The start of the biography]
                \newline
                \textcolor{titlecolor}{Doug Sheehan is an American actor who was born on April 27, 1949, in Santa Monica, California. He is best known for his roles in soap operas, including his portrayal of Joe Kelly on ``General Hospital'' and Ben Gibson on ``Knots Landing.''}
                \newline
                [The end of the biography]
                \newline \newline
                [The start of the list of checked facts]
                \newline
                \textcolor{anscolor}{[Doug Sheehan is an American. (True); Doug Sheehan is an actor. (True); Doug Sheehan was born on April 27, 1949. (True); Doug Sheehan was born in Santa Monica, California. (False); He is best known for his roles in soap operas. (True); He portrayed Joe Kelly. (True); Joe Kelly was in General Hospital. (True); General Hospital is a soap opera. (True); He portrayed Ben Gibson. (True); Ben Gibson was in Knots Landing. (True); Knots Landing is a soap opera. (True)]}
                \newline
                [The end of the list of checked facts]
                \newline \newline
                [The start of the ideal output]
                \newline
                \textcolor{labelcolor}{[Doug Sheehan (True); is (True); an (True); American (True); actor (True); who (True); was born (True); on (True); April 27, 1949 (True); in (True); Santa Monica, California (False); . (True); He (True); is (True); best known (True); for (True); his roles in soap operas (True); , (True); including (True); in (True); his portrayal (True); of (True); Joe Kelly (True); on (True); ``General Hospital'' (True); and (True); Ben Gibson (True); on (True); ``Knots Landing.'' (True)]}
                \newline
                [The end of the ideal output]
				\newline \newline
				\textbf{User prompt}
				\newline
				\newline
				[The start of the biography]
				\newline
				\textcolor{magenta}{\texttt{\{BIOGRAPHY\}}}
				\newline
				[The ebd of the biography]
				\newline \newline
				[The start of the list of checked facts]
				\newline
				\textcolor{magenta}{\texttt{\{LIST OF CHECKED FACTS\}}}
				\newline
				[The end of the list of checked facts]
			};
			\node[chatbox_user_inner] (q1_text) at (q1) {
				\textbf{System prompt}
				\newline
				\newline
				You are a helpful and precise assistant for segmenting and labeling sentences. We would like to request your help on curating a dataset for entity-level hallucination detection.
				\newline \newline
                We will give you a machine generated biography and a list of checked facts about the biography. Each fact consists of a sentence and a label (True/False). Please do the following process. First, breaking down the biography into words. Second, by referring to the provided list of facts, merging some broken down words in the previous step to form meaningful entities. For example, ``strategic thinking'' should be one entity instead of two. Third, according to the labels in the list of facts, labeling each entity as True or False. Specifically, for facts that share a similar sentence structure (\eg, \textit{``He was born on Mach 9, 1941.''} (\texttt{True}) and \textit{``He was born in Ramos Mejia.''} (\texttt{False})), please first assign labels to entities that differ across atomic facts. For example, first labeling ``Mach 9, 1941'' (\texttt{True}) and ``Ramos Mejia'' (\texttt{False}) in the above case. For those entities that are the same across atomic facts (\eg, ``was born'') or are neutral (\eg, ``he,'' ``in,'' and ``on''), please label them as \texttt{True}. For the cases that there is no atomic fact that shares the same sentence structure, please identify the most informative entities in the sentence and label them with the same label as the atomic fact while treating the rest of the entities as \texttt{True}. In the end, output the entities and labels in the following format:
                \begin{itemize}[nosep]
                    \item Entity 1 (Label 1)
                    \item Entity 2 (Label 2)
                    \item ...
                    \item Entity N (Label N)
                \end{itemize}
                % \newline \newline
                Here are two examples:
                \newline\newline
                \textbf{[Example 1]}
                \newline
                [The start of the biography]
                \newline
                \textcolor{titlecolor}{Marianne McAndrew is an American actress and singer, born on November 21, 1942, in Cleveland, Ohio. She began her acting career in the late 1960s, appearing in various television shows and films.}
                \newline
                [The end of the biography]
                \newline \newline
                [The start of the list of checked facts]
                \newline
                \textcolor{anscolor}{[Marianne McAndrew is an American. (False); Marianne McAndrew is an actress. (True); Marianne McAndrew is a singer. (False); Marianne McAndrew was born on November 21, 1942. (False); Marianne McAndrew was born in Cleveland, Ohio. (False); She began her acting career in the late 1960s. (True); She has appeared in various television shows. (True); She has appeared in various films. (True)]}
                \newline
                [The end of the list of checked facts]
                \newline \newline
                [The start of the ideal output]
                \newline
                \textcolor{labelcolor}{[Marianne McAndrew (True); is (True); an (True); American (False); actress (True); and (True); singer (False); , (True); born (True); on (True); November 21, 1942 (False); , (True); in (True); Cleveland, Ohio (False); . (True); She (True); began (True); her (True); acting career (True); in (True); the late 1960s (True); , (True); appearing (True); in (True); various (True); television shows (True); and (True); films (True); . (True)]}
                \newline
                [The end of the ideal output]
				\newline \newline
                \textbf{[Example 2]}
                \newline
                [The start of the biography]
                \newline
                \textcolor{titlecolor}{Doug Sheehan is an American actor who was born on April 27, 1949, in Santa Monica, California. He is best known for his roles in soap operas, including his portrayal of Joe Kelly on ``General Hospital'' and Ben Gibson on ``Knots Landing.''}
                \newline
                [The end of the biography]
                \newline \newline
                [The start of the list of checked facts]
                \newline
                \textcolor{anscolor}{[Doug Sheehan is an American. (True); Doug Sheehan is an actor. (True); Doug Sheehan was born on April 27, 1949. (True); Doug Sheehan was born in Santa Monica, California. (False); He is best known for his roles in soap operas. (True); He portrayed Joe Kelly. (True); Joe Kelly was in General Hospital. (True); General Hospital is a soap opera. (True); He portrayed Ben Gibson. (True); Ben Gibson was in Knots Landing. (True); Knots Landing is a soap opera. (True)]}
                \newline
                [The end of the list of checked facts]
                \newline \newline
                [The start of the ideal output]
                \newline
                \textcolor{labelcolor}{[Doug Sheehan (True); is (True); an (True); American (True); actor (True); who (True); was born (True); on (True); April 27, 1949 (True); in (True); Santa Monica, California (False); . (True); He (True); is (True); best known (True); for (True); his roles in soap operas (True); , (True); including (True); in (True); his portrayal (True); of (True); Joe Kelly (True); on (True); ``General Hospital'' (True); and (True); Ben Gibson (True); on (True); ``Knots Landing.'' (True)]}
                \newline
                [The end of the ideal output]
				\newline \newline
				\textbf{User prompt}
				\newline
				\newline
				[The start of the biography]
				\newline
				\textcolor{magenta}{\texttt{\{BIOGRAPHY\}}}
				\newline
				[The ebd of the biography]
				\newline \newline
				[The start of the list of checked facts]
				\newline
				\textcolor{magenta}{\texttt{\{LIST OF CHECKED FACTS\}}}
				\newline
				[The end of the list of checked facts]
			};
		\end{tikzpicture}
        \caption{GPT-4o prompt for labeling hallucinated entities.}\label{tb:gpt-4-prompt}
	\end{center}
\vspace{-0cm}
\end{table*}

% \begin{figure}[t]
%     \centering
%     \includegraphics[width=0.9\linewidth]{Image/abla2/doc7.png}
%     \caption{Improvement of generated documents over direct retrieval on different models.}
%     \label{fig:comparison}
% \end{figure}

\begin{figure}[t]
    \centering
    \subfigure[Unsupervised Dense Retriever.]{
        \label{fig:imp:unsupervised}
        \includegraphics[width=0.8\linewidth]{Image/A.3_fig/improvement_unsupervised.pdf}
    }
    \subfigure[Supervised Dense Retriever.]{
        \label{fig:imp:supervised}
        \includegraphics[width=0.8\linewidth]{Image/A.3_fig/improvement_supervised.pdf}
    }
    
    % \\
    % \subfigure[Comparison of Reasoning Quality With Different Method.]{
    %     \label{fig:reasoning} 
    %     \includegraphics[width=0.98\linewidth]{images/reasoning1.pdf}
    % }
    \caption{Improvements of LLM-QE in Both Unsupervised and Supervised Dense Retrievers. We plot the change of nDCG@10 scores before and after the query expansion using our LLM-QE model.}
    \label{fig:imp}
\end{figure}
\section{Appendix}
\subsection{License}
The authors of 4 out of the 15 datasets in the BEIR benchmark (NFCorpus, FiQA-2018, Quora, Climate-Fever) and the authors of ELI5 in the E5 dataset do not report the dataset license in the paper or a repository. We summarize the licenses of the remaining datasets as follows.

MS MARCO (MIT License); FEVER, NQ, and DBPedia (CC BY-SA 3.0 license); ArguAna and Touché-2020 (CC BY 4.0 license); CQADupStack and TriviaQA (Apache License 2.0); SciFact (CC BY-NC 2.0 license); SCIDOCS (GNU General Public License v3.0); HotpotQA and SQuAD (CC BY-SA 4.0 license); TREC-COVID (Dataset License Agreement).

All these licenses and agreements permit the use of their data for academic purposes.

\subsection{Additional Experimental Details}\label{app:experiment_detail}
This subsection outlines the components of the training data and presents the prompt templates used in the experiments.


\textbf{Training Datasets.} Following the setup of \citet{wang2024improving}, we use the following datasets: ELI5 (sample ratio 0.1)~\cite{fan2019eli5}, HotpotQA~\cite{yang2018hotpotqa}, FEVER~\cite{thorne2018fever}, MS MARCO passage ranking (sample ratio 0.5) and document ranking (sample ratio 0.2)~\cite{bajaj2016ms}, NQ~\cite{karpukhin2020dense}, SQuAD~\cite{karpukhin2020dense}, and TriviaQA~\cite{karpukhin2020dense}. In total, we use 808,740 training examples.

\textbf{Prompt Templates.} Table~\ref{tab:prompt_template} lists all the prompts used in this paper. In each prompt, ``query'' refers to the input query for which query expansions are generated, while ``Related Document'' denotes the ground truth document relevant to the original query. We observe that, in general, the model tends to generate introductory phrases such as ``Here is a passage to answer the question:'' or ``Here is a list of keywords related to the query:''. Before using the model outputs as query expansions, we first filter out these introductory phrases to ensure cleaner and more precise expansion results.



\subsection{Query Expansion Quality of LLM-QE}\label{app:analysis}
This section evaluates the quality of query expansion of LLM-QE. As shown in Figure~\ref{fig:imp}, we randomly select 100 samples from each dataset to assess the improvement in retrieval performance before and after applying LLM-QE.

Overall, the evaluation results demonstrate that LLM-QE consistently improves retrieval performance in both unsupervised (Figure~\ref{fig:imp:unsupervised}) and supervised (Figure~\ref{fig:imp:supervised}) settings. However, for the MS MARCO dataset, LLM-QE demonstrates limited effectiveness in the supervised setting. This can be attributed to the fact that MS MARCO provides higher-quality training signals, allowing the dense retriever to learn sufficient matching signals from relevance labels. In contrast, LLM-QE leads to more substantial performance improvements on the NQ and HotpotQA datasets. This indicates that LLM-QE provides essential matching signals for dense retrievers, particularly in retrieval scenarios where high-quality training signals are scarce.


\subsection{Case Study}\label{app:case_study}
\begin{figure}[htb]
\small
\begin{tcolorbox}[left=3pt,right=3pt,top=3pt,bottom=3pt,title=\textbf{Conversation History:}]
[human]: Craft an intriguing opening paragraph for a fictional short story. The story should involve a character who wakes up one morning to find that they can time travel.

...(Human-Bot Dialogue Turns)... \textcolor{blue}{(Topic: Time-Travel Fiction)}

[human]: Please describe the concept of machine learning. Could you elaborate on the differences between supervised, unsupervised, and reinforcement learning? Provide real-world examples of each.

...(Human-Bot Dialogue Turns)... \textcolor{blue}{(Topic: Machine learning Concepts and Types)}


[human]: Discuss antitrust laws and their impact on market competition. Compare the antitrust laws in US and China along with some case studies

...(Human-Bot Dialogue Turns)... \textcolor{blue}{(Topic: Antitrust Laws and Market Competition)}

[human]: The vertices of a triangle are at points (0, 0), (-1, 1), and (3, 3). What is the area of the triangle?

...(Human-Bot Dialogue Turns)... \textcolor{blue}{(Topic: Area and Circumcircle of Triangle)}

...(Human-Bot Dialogue Interaction on Other Topics)... 

\end{tcolorbox}
\begin{tcolorbox}[left=3pt,right=3pt,top=3pt,bottom=3pt,title=\textbf{Question:}]
What title reflects the thematic essence of the Doctor's experience with time travel as described in the story?
\end{tcolorbox}
\begin{tcolorbox}[left=3pt,right=3pt,top=3pt,bottom=3pt,title=\textbf{Ground-Truth Answer:}]
Awakening to Eternity: The Doctor's New Dawn
\end{tcolorbox}
\begin{tcolorbox}[left=3pt,right=3pt,top=3pt,bottom=3pt,title=\textbf{Retrieved Memory (Turn-Level):}]
[human]: Describe the process and write out the balanced chemical equation for the reaction that occurs when solid calcium carbonate reacts with hydrochloric acid to form aqueous calcium chloride, carbon dioxide, and water. What type of reaction is this, and what observations might indicate that the reaction is taking place?

[bot]: When solid calcium carbonate ($\text{CaCO}_{3}$) reacts with hydrochloric acid (HCl), the products formed are aqueous calcium chloride ($\text{CaCl}_{2}$), carbon dioxide ($\text{CO}_{2}$), and water ($\text{H}_{2}\text{O}$). The balanced chemical equation for this reaction is: ... \textcolor{blue}{(Topic: Calcium Carbonate and Hydrochloric Acid Reaction)}

\textcolor{brown}{
[human]: Write a title for your story.}

\textcolor{brown}{
[bot]: "Awakening to Eternity: The Doctor's New Dawn".
} \textcolor{blue}{(Topic: Time-Travel Fiction)}

[human]: Analyze the following customer reviews from different sources for three different smartphones - the latest iPhone, Samsung Galaxy, and Google Pixel - and provide an overall rating for each phone on a scale of 1 to 10. Consider the following complex and contradictory reviews: - TechRadar's review of the latest iPhone: The new iPhone is a stunning triumph of engineering that sets a new bar for smartphone performance and camera quality. However, the incremental design and high price mean it lacks the 'wow' factor of previous iPhones. Still, its power and intelligence are unrivaled. - CNET's review of the latest Samsung Galaxy: ... Return the answer as a JSON object with the overall ratings for each phone out of 10, to one decimal place.

[bot]: ```json { "iPhone": 8.5, "Samsung Galaxy": 8.0, "Google Pixel": 7.5 } ```

\textcolor{blue}{(Topic: Smartphone Review Ratings Analysis)}
\end{tcolorbox}
\begin{tcolorbox}[left=3pt,right=3pt,top=3pt,bottom=3pt,title=\textbf{Answer (Turn-Level Retrieval):}]
As an AI language model, I do not have access to the story you mentioned earlier. Please provide me with the story or the necessary details so that I can assist you better.

\textbf{\textcolor{red}{(The next part of this figure~\ref{fig: case_study_segment_vs_turn} will be continued on the next page.)}}
\end{tcolorbox}
\end{figure}

\begin{figure}[htb]
% \ContinuedFloat
\small
\begin{tcolorbox}[left=3pt,right=3pt,top=3pt,bottom=3pt,title=\textbf{Retrieved Memory (Ours):}]
[human]: Craft an intriguing opening paragraph for a fictional short story. The story should involve a character who wakes up one morning to find that they can time travel.

[bot]: In the hushed pre-dawn hours, when the world still slumbers under the cloak of night, Alex found themselves caught in the fragile boundary between sleep and wakefulness. ...

[human]: Summarize the story with three bullet points using only nouns and adjectives, without verbs.

[bot]: - Alex, pre-dawn awakening, time-travel discovery - Disorientation, vertigo, reality's malleability - Sunlight, new day, astonishing truth

[human]: Is there any famous fictional character that can time travel? If so, can you rewrite your story with him or her?

[bot]: - \textcolor{brown}{Doctor Who, a renowned time-traveling character from British television, is a perfect fit for this scenario.} In the silent embrace of early morning, where shadows merge with the fading night, the Doctor found themselves in the thin veil separating dreams from consciousness. It was here, amidst the soft whispers of time, that an unusual sensation enveloped them—a sense of time's fabric slipping through their fingers like grains of sand. As dawn's light broke through the darkness, painting the room in hues of gold and amber, the Doctor's eyes opened to what appeared to be an ordinary day. \textcolor{brown}{However, upon standing, a wave of dizziness overwhelmed them, the surroundings blurred, and in that moment of confusion, the Doctor realized a profound truth: they had not only awakened to a new dawn but had also stumbled upon the incredible ability to navigate the currents of time at will.}

\textcolor{brown}{[human]: Write a title for your story.}

\textcolor{brown}{[bot]: "Awakening to Eternity: The Doctor's New Dawn"} \textcolor{blue}{(Topic: Time-Travel Fiction)}
\end{tcolorbox}
\begin{tcolorbox}[left=3pt,right=3pt,top=3pt,bottom=3pt,title=\textbf{Answer (Ours):}]
The title "Awakening to Eternity: The Doctor's New Dawn" reflects the thematic essence of the Doctor's experience with time travel as described in the story. It captures the idea of a new beginning, a sense of eternity, and the awakening of a new power that will change the course of the Doctor's life forever.
\end{tcolorbox}
\caption{Case study comparing our \sysname\ with the \textit{turn-level} memory baseline. When constructing the memory bank at the \textit{turn-level}, the agent fails to provide a satisfactory response due to fragmentation: the retrieved dialogue turns can not capture a complete human-agent interaction related to the query, missing critical information highlighted in \textcolor{brown}{brown}. The dialogue topic is emphasized in \textcolor{blue}{blue}. Some context is omitted (replaced by ...) due to length constraints.}
\label{fig: case_study_segment_vs_turn}
\end{figure}

\newpage

\begin{figure}[htbp]
\small
\begin{tcolorbox}[left=3pt,right=3pt,top=3pt,bottom=3pt,title=\textbf{Conversation History:}]
[human]: Explain what's base rate fallacy and list five specific examples of ...

... (Human-Bot Dialogue Turns) ... \textcolor{blue}{(Topic: Base Rate Fallacy In Politics)}


[human]: How many integers are in the solution of the inequality $|x + 5| < 10$?

... (Human-Bot Dialogue Turns) ... \textcolor{blue}{(Topic: Number of Integers in Inequalities)}

[human]: How have the Alps and Rhine River influenced settlement and agriculture in Western Europe?

... (Human-Bot Dialogue Turns) ... \textcolor{blue}{(Topic: Alps and Rhine River Impact)}

[human]: In a survey conducted at a local high school, preferences for a new school color were measured: 58\% of students liked the color blue, 45\% preferred green, and 22\% liked both colors. If we randomly pick a student from the school, what's the probability that they would like neither blue nor green?

... (Human-Bot Dialogue Turns) ... \textcolor{blue}{(Topic: Probability of Liking Colors in Survey)}

[human]: Imagine you are participating in a race with a group of people. If you have just overtaken the second person, what's your current position? Where is the person you just overtook?

... (Human-Bot Dialogue Turns) ... \textcolor{blue}{(Topic: Overtaking Positions in Race)}
\end{tcolorbox}
\begin{tcolorbox}[left=3pt,right=3pt,top=3pt,bottom=3pt,title=\textbf{Question:}]
[human]: In the inequality question, how many positive integers are in the solution of the second inequality?
\end{tcolorbox}
\begin{tcolorbox}[left=3pt,right=3pt,top=3pt,bottom=3pt,title=\textbf{Ground-Truth Answer:}]
In the second inequality, $(|x + 10| < 5)$, the solution range is $(-15 < x < -5)$. Within this range, there are no positive integers, as all values of $(x)$ that satisfy the inequality are negative.
\end{tcolorbox}
\begin{tcolorbox}[left=3pt,right=3pt,top=3pt,bottom=3pt,title=\textbf{Retrieved Memory (Session-Level):}]
[human]: Explain what's base rate fallacy and list five specific examples of ...

... (Human-Bot Dialogue Turns) ... \textcolor{blue}{(Topic: Base Rate Fallacy In Politics)}

[human]: How many integers are in the solution of the inequality $|x + 5| < 10$?

[bot]: To solve the inequality $|x + 5| < 10$, ... 

\textcolor{brown}{
[human]: What about $|x + 10| < 5$?
}

\textcolor{brown}{
[bot]: To solve the inequality $|x + 10| < 5$, we ...} \textcolor{gray}{(``overwhelmed'' by other irrelevant, redundant content)}

... (Human-Bot Dialogue Turns) ...  \textcolor{blue}{(Topic: Number of Integers in Inequalities)}

% \textcolor{brown}{
% [human]: What about $|x + 10| < 5$
% }

% \textcolor{brown}{
% [bot]: To solve the inequality $|x + 10| < 5$, ...
% } \textcolor{blue}{(Topic: Number of Integers in Inequalities)}

[human]: How have the Alps and Rhine River influenced settlement and agriculture in Western Europe?

... (Human-Bot Dialogue Turns) ... \textcolor{blue}{(Topic: Alps and Rhine River Impact)}

[human]: In a survey conducted at a local high school, ...

... (Human-Bot Dialogue Turns) ... \textcolor{blue}{(Topic: Probability of Liking Colors in Survey)}
\end{tcolorbox}
\begin{tcolorbox}[left=3pt,right=3pt,top=3pt,bottom=3pt,title=\textbf{Answer (Session-Level Retrieval):}]
I apologize for the confusion earlier. To answer your question, the second inequality was not provided in our previous conversation. If you could provide the inequality, I would be happy to help you find the solution and the number of positive integers in the solution set.

\textbf{\textcolor{red}{(The next part of this figure~\ref{fig: case_study_segment_vs_session} will be continued on the next page.)}}
\end{tcolorbox}
\end{figure}
\begin{figure}[htbp]
% \ContinuedFloat
\small
\begin{tcolorbox}[left=3pt,right=3pt,top=3pt,bottom=3pt,title=\textbf{Retrieved Memory (Ours):}]
[human]: How many integers are in the solution of the inequality $|x + 5| < 10$?

[bot]: To solve the inequality $|x + 5| < 10$, we break it into two cases based on the definition of absolute value: 
1. $x + 5 < 10$ 
2. $-(x + 5) < 10$, which simplifies to $x + 5 > -10$ 
Solving each case: 
1. $x < 5$ 
2. $x > -15$ 
Combining these, we get $-15 < x < 5$. This inequality means that $x$ can be any number greater than $-15$ and less than $5$. To find how many integers satisfy this condition, we list them: $-14, -13, -12, -11, -10, -9, -8, -7, -6, -5, -4, -3, -2, -1, 0, 1, 2, 3, 4$. There are 19 integers that satisfy the inequality $|x + 5| < 10$.

\textcolor{brown}{
[human]: What about $|x + 10| < 5$?
}

\textcolor{brown}{
[bot]: To solve the inequality $|x + 10| < 5$, we again consider two cases: 
1. $x + 10 < 5$ 
2. $-(x + 10) < 5$, which simplifies to $x + 10 > -5$ 
Solving each case: 
1. $x < -5$ 
2. $x > -15$ 
Combining these, we get $-15 < x < -5$. This inequality means that $x$ can be any number greater than $-15$ and less than $-5$. To find how many integers satisfy this condition, we list them: $-14, -13, -12, -11, -10, -9, -8, -7, -6$. There are 9 integers that satisfy the inequality $|x + 10| < 5$.
}

[human]: What about $|x + 10| < 10$?

[bot]: To solve the inequality $|x + 10| < 10$, we consider two cases: 
1. $x + 10 < 10$ 
2. $-(x + 10) < 10$, which simplifies to $x + 10 > -10$ 
Solving each case: 
1. $x < 0$ 
2. $x > -20$ 
Combining these, we get $-20 < x < 0$. This inequality means that $x$ can be any number greater than $-20$ and less than $0$. To find how many integers satisfy this condition, we list them: $-19, -18, -17, -16, -15, -14, -13, -12, -11, -10, -9, -8, -7, -6, -5, -4, -3, -2, -1$. There are 19 integers that satisfy the inequality $|x + 10| < 10$. \textcolor{blue}{(Topic: Number of Integers in Inequalities)}
\end{tcolorbox}
\begin{tcolorbox}[left=3pt,right=3pt,top=3pt,bottom=3pt,title=\textbf{Answer (Ours):}]
To find the number of positive integers that satisfy the inequality $(|x + 10| < 5)$, we need to look for positive integers in the solution set $(-15 < x < -5)$. Since there are no positive integers in this interval, the answer is 0.

\end{tcolorbox}
\caption{Case study comparing our \sysname\ with the \textit{session-level} memory baseline. When constructing the memory bank at the \textit{session-level}, the agent fails to provide a satisfactory response due to the inclusion of excessive irrelevant information within a session. This distracts the generator from focusing on the truly useful information highlighted in \textcolor{brown}{brown}. The dialogue topic is emphasized in \textcolor{blue}{blue}. Some context is omitted (replaced by ...) due to length constraints.}
\label{fig: case_study_segment_vs_session}
\end{figure}



\begin{figure}[htb]
\small
\begin{tcolorbox}[left=3pt,right=3pt,top=3pt,bottom=3pt,title=\textbf{Conversation History:}]
[human]: Photosynthesis is a vital process for life on Earth. Could you outline the two main stages of photosynthesis, including where they take place within the chloroplast, and the primary inputs and outputs for each stage? ... (Human-Bot Dialogue Turns)... \textcolor{blue}{(Topic: Photosynthetic Energy Production)}

[human]: Please assume the role of an English translator, tasked with correcting and enhancing spelling and language. Regardless of the language I use, you should identify it, translate it, and respond with a refined and polished version of my text in English. 

... (Human-Bot Dialogue Turns)...  \textcolor{blue}{(Topic: Language Translation and Enhancement)}

[human]: Suggest five award-winning documentary films with brief background descriptions for aspiring filmmakers to study.

\textcolor{brown}{[bot]: ...
5. \"An Inconvenient Truth\" (2006) - Directed by Davis Guggenheim and featuring former United States Vice President Al Gore, this documentary aims to educate the public about global warming. It won two Academy Awards, including Best Documentary Feature. The film is notable for its straightforward yet impactful presentation of scientific data, making complex information accessible and engaging, a valuable lesson for filmmakers looking to tackle environmental or scientific subjects.}

... (Human-Bot Dialogue Turns)... 
\textcolor{blue}{(Topic: Documentary Films Recommendation)}

[human]: Given the following records of stock prices, extract the highest and lowest closing prices for each month in the year 2022. Return the results as a CSV string, with one line allocated for each month. Date,Open,High,Low,Close,Volume ... ... (Human-Bot Dialogue Turns)...  \textcolor{blue}{(Topic: Stock Prices Analysis)}

[human]: The city of Vega intends to build a bridge that will span the Vegona River, covering a distance of 1.8 kilometers. The proposed location falls within a seismically active area that has experienced several high-magnitude earthquakes. Given these circumstances, what would be the best approach to constructing the bridge? ... (Human-Bot Dialogue Turns)...  \textcolor{blue}{(Topic: Seismic Bridge Construction)}
\end{tcolorbox}
\begin{tcolorbox}[left=3pt,right=3pt,top=3pt,bottom=3pt,title=\textbf{Question:}]
Which documentary won two Academy Awards and what lesson does it offer to filmmakers?
\end{tcolorbox}
\begin{tcolorbox}[left=3pt,right=3pt,top=3pt,bottom=3pt,title=\textbf{Ground-Truth Answer:}]
``An Inconvenient Truth" (2006) won two Academy Awards. It offers the lesson of making complex information accessible and engaging, a valuable lesson for filmmakers looking to tackle environmental or scientific subjects.
\end{tcolorbox}
\begin{tcolorbox}[left=3pt,right=3pt,top=3pt,bottom=3pt,title=\textbf{Memory generated by RecurSum~\citep{wang2023recursively}:}]
The dialogue provided detailed information on various topics, including the stages of photosynthesis, energy production estimates through photosynthesis for different organisms, translations and enhancements of phrases in different languages, recommendations for documentary films, a pitch for a film about overcoming adversity, analysis of stock prices, and considerations for constructing a bridge in a seismically active area. Here's a summary of the key points: ...

\textbf{Documentary Film Recommendations}: Suggested five award-winning documentary films for aspiring filmmakers to study, with brief background descriptions for each. The response included:

-  "March of the Penguins" (2005) - Emperor penguins' journey in Antarctica; noted for its cinematography and portrayal of the penguins' life cycle.

-  "Citizenfour" (2014) - Edward Snowden and NSA spying scandal; known for its real-time exploration of Snowden's revelations.

-  "20 Feet from Stardom" (2013) - Stories of backup singers behind musical legends; celebrates unsung heroes in the music industry.

-  "Man on Wire" (2008) - Philippe Petit's 1974 high-wire walk between the Twin Towers; combines personal footage, interviews, and re-enactments.

-  \textcolor{brown}{"An Inconvenient Truth" (2006) - Al Gore's documentary on global warming; known for its presentation of scientific data.}
\end{tcolorbox}
\begin{tcolorbox}[left=3pt,right=3pt,top=3pt,bottom=3pt,title=\textbf{Answer (RSum):}]
I'm sorry, but based on the updated context you provided, I don't see any previous discussion related to the question you asked. However, to answer your question, the documentary that won two Academy Awards is "Free Solo," which won for Best Documentary Feature and Best Achievement in Film Editing in 2019. 
% The lesson it offers to filmmakers is the importance of taking risks and pushing boundaries in order to create something truly unique and impactful.
....
The film follows climber Alex Honnold as he attempts to free solo climb El Capitan in Yosemite National Park, and the filmmakers had to take risks themselves in order to capture the footage and tell the story.

\textbf{\textcolor{red}{(The next part of this figure~\ref{fig: case_study_segment_vs_rsum} will be continued on the next page.)}}
\end{tcolorbox}
\end{figure}
\begin{figure}[htbp]
% \ContinuedFloat
\small
\begin{tcolorbox}[left=3pt,right=3pt,top=3pt,bottom=3pt,title=\textbf{Retrieved Memory (Ours):}]
[human]: Suggest five award-winning documentary films with brief background descriptions for aspiring filmmakers to study.

[bot]: 1. **"March of the Penguins" (2005)** - Directed by Luc Jacquet, this French feature-length nature documentary delves into the annual journey of Emperor penguins across the Antarctic. It won the Academy Award for Best Documentary Feature. The film is celebrated for its stunning cinematography and its detailed, heartfelt portrayal of the penguins' life cycle, making it a must-study for filmmakers interested in wildlife documentaries. 2. **"Citizenfour" (2014)** ... 5. \textcolor{brown}{**"An Inconvenient Truth" (2006)** - Directed by Davis Guggenheim and featuring former United States Vice President Al Gore, this documentary aims to educate the public about global warming. It won two Academy Awards, including Best Documentary Feature.} The film is notable for its straightforward yet impactful presentation of scientific data, making complex information accessible and engaging, \textcolor{brown}{a valuable lesson for filmmakers looking to tackle environmental or scientific subjects.}

[human]: With the spirit in the first film, craft a succinct and persuasive pitch for a film about overcoming adversity.

[bot]: Drawing inspiration from the resilience and majestic journey of Emperor penguins in "March of the Penguins," our documentary, "Unyielding Spirits: The Human Odyssey," will explore the profound stories of individuals around the globe who have faced and overcome staggering adversities. Through breathtaking cinematography and intimate storytelling, we ...
\textcolor{blue}{(Topic: Documentary Films Recommendation)}
\end{tcolorbox}
\begin{tcolorbox}[left=3pt,right=3pt,top=3pt,bottom=3pt,title=\textbf{Answer (Ours):}]
``An Inconvenient Truth" won two Academy Awards, including Best Documentary Feature. The film offers a valuable lesson to filmmakers looking to tackle environmental or scientific subjects, as it presents complex information in a straightforward yet impactful way, making it accessible and engaging to a wider audience.


\end{tcolorbox}
\caption{Case study comparing our \sysname\ with RecurSum~\citep{wang2023recursively}. When using RecurSum, the agent fails to provide a satisfactory answer due to information loss during the summarization process. Summarization omits essential details needed to answer the user's new request, as highlighted in \textcolor{brown}{brown}. The dialogue topic is emphasized in \textcolor{blue}{blue}. Some context is omitted (replaced by ...) due to length constraints.}
\label{fig: case_study_segment_vs_rsum}
\end{figure}




\begin{figure}[htb]
\small
\begin{tcolorbox}[left=3pt,right=3pt,top=3pt,bottom=3pt,title=\textbf{Conversation History:}]
[human]: Photosynthesis is a vital process for life on Earth. Could you outline the two main stages of photosynthesis, including where they take place within the chloroplast, and the primary inputs and outputs for each stage? ... (Human-Bot Dialogue Turns)... \textcolor{blue}{(Topic: Photosynthetic Energy Production)}

[human]: Please assume the role of an English translator, tasked with correcting and enhancing spelling and language. Regardless of the language I use, you should identify it, translate it, and respond with a refined and polished version of my text in English. 

... (Human-Bot Dialogue Turns)...  \textcolor{blue}{(Topic: Language Translation and Enhancement)}

[human]: Suggest five award-winning documentary films with brief background descriptions for aspiring filmmakers to study.

\textcolor{brown}{[bot]: ...
5. \"An Inconvenient Truth\" (2006) - Directed by Davis Guggenheim and featuring former United States Vice President Al Gore, this documentary aims to educate the public about global warming. It won two Academy Awards, including Best Documentary Feature. The film is notable for its straightforward yet impactful presentation of scientific data, making complex information accessible and engaging, a valuable lesson for filmmakers looking to tackle environmental or scientific subjects.}

... (Human-Bot Dialogue Turns)... 
\textcolor{blue}{(Topic: Documentary Films Recommendation)}

[human]: Given the following records of stock prices, extract the highest and lowest closing prices for each month in the year 2022. Return the results as a CSV string, with one line allocated for each month. Date,Open,High,Low,Close,Volume ... ... (Human-Bot Dialogue Turns)...  \textcolor{blue}{(Topic: Stock Prices Analysis)}

[human]: The city of Vega intends to build a bridge that will span the Vegona River, covering a distance of 1.8 kilometers. The proposed location falls within a seismically active area that has experienced several high-magnitude earthquakes. Given these circumstances, what would be the best approach to constructing the bridge? ... (Human-Bot Dialogue Turns)...  \textcolor{blue}{(Topic: Seismic Bridge Construction)}
\end{tcolorbox}
\begin{tcolorbox}[left=3pt,right=3pt,top=3pt,bottom=3pt,title=\textbf{Question:}]
Which documentary won two Academy Awards and what lesson does it offer to filmmakers?
\end{tcolorbox}
\begin{tcolorbox}[left=3pt,right=3pt,top=3pt,bottom=3pt,title=\textbf{Ground-Truth Answer:}]
"An Inconvenient Truth" (2006) won two Academy Awards. It offers the lesson of making complex information accessible and engaging, a valuable lesson for filmmakers looking to tackle environmental or scientific subjects.
\end{tcolorbox}
\begin{tcolorbox}[left=3pt,right=3pt,top=3pt,bottom=3pt,title=\textbf{Memory generated by ConditionMem~\citep{yuan2023evolving}:}]
- \textcolor{brown}{The user is seeking recommendations for award-winning documentary films, along with brief background descriptions, for educational purposes.} This request likely stems from a desire to learn from successful examples in the documentary filmmaking field, possibly to improve their own filmmaking skills or to gain inspiration for their projects. 

- \textcolor{brown}{User is interested in award-winning documentary films for educational purposes, aiming to study their backgrounds and filmmaking techniques.}

- The user's request for crafting a succinct and persuasive pitch for a film about overcoming adversity is inspired by the spirit of "March of the Penguins," a documentary that beautifully captures the Emperor penguins' annual journey across the Antarctic, highlighting themes of endurance and survival against the odds. The user seeks to emulate the emotional depth and cinematic quality that made "March of the Penguins" compelling, aiming to create a pitch that resonates with similar themes of resilience and triumph over challenges. 

- User is interested in creating content that focuses on themes of overcoming adversity, inspired by the emotional depth and cinematic excellence of documentaries like "March of the Penguins." They value storytelling that captures the essence of endurance and survival against the odds.
% : The user's request to write a budget plan for the video production directly follows the task of creating a video script aimed at training new employees of a car wash business in Japan, focusing on business etiquette norms.
% The user is likely seeking to understand the financial requirements to bring the previously crafted script to life, indicating a move from the planning and content development phase to the practical aspects of production. 

... (Other Irrelevant Memory Records) ...
\end{tcolorbox}
\begin{tcolorbox}[left=3pt,right=3pt,top=3pt,bottom=3pt,title=\textbf{Answer (Conditional Memory):}]
The documentary that won two Academy Awards is "Free Solo," which follows climber Alex Honnold as he attempts to free solo climb El Capitan in Yosemite National Park. The film offers a lesson to filmmakers on the importance of capturing authentic and compelling human stories, as well as the technical challenges of filming in extreme environments.

\textbf{\textcolor{red}{(The next part of this figure~\ref{fig: case_study_segment_vs_condmem} will be continued on the next page.)}}
\end{tcolorbox}
\end{figure}
\begin{figure}[htbp]
% \ContinuedFloat
\small
\begin{tcolorbox}[left=3pt,right=3pt,top=3pt,bottom=3pt,title=\textbf{Retrieved Memory (Ours):}]
[human]: Suggest five award-winning documentary films with brief background descriptions for aspiring filmmakers to study.

[bot]: 1. **"March of the Penguins" (2005)** - Directed by Luc Jacquet, this French feature-length nature documentary delves into the annual journey of Emperor penguins across the Antarctic. It won the Academy Award for Best Documentary Feature. The film is celebrated for its stunning cinematography and its detailed, heartfelt portrayal of the penguins' life cycle, making it a must-study for filmmakers interested in wildlife documentaries. 2. **"Citizenfour" (2014)** ... 5. \textcolor{brown}{**"An Inconvenient Truth" (2006)** - Directed by Davis Guggenheim and featuring former United States Vice President Al Gore, this documentary aims to educate the public about global warming. It won two Academy Awards, including Best Documentary Feature.} The film is notable for its straightforward yet impactful presentation of scientific data, making complex information accessible and engaging, a valuable lesson for filmmakers looking to tackle environmental or scientific subjects.

[human]: With the spirit in the first film, craft a succinct and persuasive pitch for a film about overcoming adversity.

[bot]: Drawing inspiration from the resilience and majestic journey of Emperor penguins in "March of the Penguins," our documentary, "Unyielding Spirits: The Human Odyssey," will explore the profound stories of individuals around the globe who have faced and overcome staggering adversities. Through breathtaking cinematography and intimate storytelling, we ...
\textcolor{blue}{(Topic: Documentary Films Recommendation)}
\end{tcolorbox}
\begin{tcolorbox}[left=3pt,right=3pt,top=3pt,bottom=3pt,title=\textbf{Answer (Ours):}]
``An Inconvenient Truth" won two Academy Awards, including Best Documentary Feature. The film offers a valuable lesson to filmmakers looking to tackle environmental or scientific subjects, as it presents complex information in a straightforward yet impactful way, making it accessible and engaging to a wider audience.
\end{tcolorbox}
\caption{Case study comparing our \sysname\ with ConditionMem~\citep{yuan2023evolving}. When using ConditionMem, the agent fails to provide a satisfactory answer due to (1) information loss during the summarization process and (2) the incorrect discarding of turns that are actually useful, as highlighted in \textcolor{brown}{brown}. The dialogue topic is emphasized in \textcolor{blue}{blue}. Some context is omitted (replaced by ...) due to length constraints.}
\label{fig: case_study_segment_vs_condmem}
\end{figure}


To further demonstrate the effectiveness of LLM-QE, we conduct a case study by randomly sampling a query from the evaluation dataset. We then compare retrieval performance using the raw queries, expanded queries by vanilla LLM, and expanded queries by LLM-QE.

As shown in Table~\ref{tab:case_study}, query expansion significantly improves retrieval performance compared to using the raw query. Both vanilla LLM and LLM-QE generate expansions that include key phrases, such as ``temperature'', ``humidity'', and ``coronavirus'', which provide crucial signals for document matching. However, vanilla LLM produces inconsistent results, including conflicting claims about temperature ranges and virus survival conditions. In contrast, LLM-QE generates expansions that are more semantically aligned with the golden passage, such as ``the virus may thrive in cooler and more humid environments, which can facilitate its transmission''. This further demonstrates the effectiveness of LLM-QE in improving query expansion by aligning with the ranking preferences of both LLMs and retrievers.


%%%%%%%%%%%%%%%%%%%%%%%%%%%%%%%%%%%%%%%%%%%%%%%%%%%%%%%%%%%%%%%%%%%%%%%%%%%%%%%
%%%%%%%%%%%%%%%%%%%%%%%%%%%%%%%%%%%%%%%%%%%%%%%%%%%%%%%%%%%%%%%%%%%%%%%%%%%%%%%


\end{document}
\end{document}