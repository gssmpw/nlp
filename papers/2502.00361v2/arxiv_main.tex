\documentclass[11pt]{article}
\usepackage[letterpaper, margin=2cm]{geometry}
% Recommended, but optional, packages for figures and better typesetting:
\usepackage{parskip}
\setlength{\abovedisplayskip}{10pt}  % Space before equations
\setlength{\belowdisplayskip}{10pt}  % Space after equations
\setlength{\parindent}{0pt}
\setlength{\parskip}{10pt} % Adjust the space as needed
\usepackage{microtype}
\usepackage{graphicx}
\usepackage{subfigure}
\usepackage{booktabs} % for professional tables
\usepackage[affil-it]{authblk}
\usepackage[textsize=tiny]{todonotes}



\usepackage{amsthm}
% \usepackage{generic}
% \usepackage{cite}
\usepackage{amsmath,amssymb,amsfonts}
\usepackage{algorithm}
\usepackage{algorithmic}
\usepackage{graphicx}
% \usepackage{algorithmic}
\usepackage{natbib}
% if you use cleveref..
\usepackage{textcomp}
\usepackage{color}
\usepackage{xurl}
\usepackage{bm}
% \usepackage{algorithm2e}
\usepackage[normalem]{ulem}
\usepackage{xcolor}
\usepackage{xspace}
\usepackage{wrapfig}
\usepackage{booktabs, multirow}
\let\labelindent\relax
\usepackage{enumitem}
\usepackage{thmtools} 
\usepackage{thm-restate}
% \newcommand{\citep}{\cite}
% \newcommand{\citet}{\cite}
\theoremstyle{plain}
\newtheorem{theorem}{Theorem}[section]
\newtheorem{proposition}[theorem]{Proposition}
\newtheorem{lemma}[theorem]{Lemma}
\newtheorem{corollary}[theorem]{Corollary}
\theoremstyle{definition}
\newtheorem{definition}[theorem]{Definition}
\newtheorem{assumption}[theorem]{Assumption}
\newtheorem{example}[theorem]{Example}
\theoremstyle{remark}
\newtheorem{remark}[theorem]{Remark}
\usepackage{Definitions}

% \usepackage{xr-hyper}
\usepackage[colorlinks,citecolor=blue]{hyperref}
% \hypersetup{hidelinks=true}
\usepackage[capitalize,noabbrev]{cleveref}

\newcommand{\algname}{{Soft Diffusion Actor-Critic}\xspace}
\newcommand{\algabb}{\texttt{SDAC}\xspace}
\newcommand{\algcommentline}[1]{\texttt{\textit{\textcolor{gray}{\# #1}}}}

\allowdisplaybreaks


\title{Soft Diffusion Actor-Critic: \\
Efficient Online Reinforcement Learning for Diffusion Policy}

% \author{Haitong Ma\textsuperscript{1}, Tianyi Chen\textsuperscript{2}, Kai Wang\textsuperscript{2}, Na Li\textsuperscript{1,*}, Bo Dai\textsuperscript{2,*}}



\begin{document}

\author[1]{Haitong Ma} 
  \author[2]{Tianyi Chen} 
  \author[2]{Kai Wang}
    \author[1]{Na Li$^*$}
  \author[2]{Bo Dai\thanks{Equal supervision.}}
  \affil[1]{School of Engineering and Applied Sciences, Harvard University}
  \affil[2]{School of Computational Science and Engineering, Georgia Institute of Technology\thanks{Emails: Haitong (haitongma@g.harvard.edu), Tianyi Chen (tchen667@gatech.edu), Kai Wang (kwang692@gatech.edu), Na Li (nali@seas.harvard.edu), Bo Dai (bodai@cc.gatech.edu)}}
    \date{}

\maketitle
\begin{abstract}
Diffusion policies have achieved superior performance in imitation learning and offline reinforcement learning (RL) due to their rich expressiveness. However, the vanilla diffusion training procedure requires samples from target distribution, which is impossible in online RL since we cannot sample from the optimal policy, making training diffusion policies highly non-trivial in online RL. 
Backpropagating policy gradient through the diffusion process incurs huge computational costs and instability, thus being expensive and impractical. 
To enable efficient diffusion policy training for online RL, we propose {\algname~(\algabb)}, exploiting the viewpoint of diffusion models as noise-perturbed energy-based models.
The proposed \algabb relies solely on the state-action value function as the energy functions to train diffusion policies, bypassing sampling from the optimal policy while maintaining lightweight computations. 
% Specifically, we first propose \emph{reverse sampling score matching} that can train diffusion models with only access to energy functions via \emph{reverse sampling trick}. 
% We apply the technique toz online maximum entropy RL to propose the SDAC algorithm, which learns a diffusion policy that only requires the soft $Q$-function as the energy function to significantly reduce the computation cost and achieve state-of-the-art performance.
% We then apply it to online maximum entropy RL, named soft diffusion actor-critic (SDEC), to learn a diffusion policy that only requires the soft $Q$-function as the energy function, which significantly reduces the computation cost and achieves state-of-the-art performance.
% Unlike existing methods that approximated the data distribution or backpropagated the policy gradient through the whole diffusion process, we achieved state-of-the-art performance with lightweight computation.
We conducted comprehensive comparisons on MuJoCo benchmarks. The empirical results show that \algabb outperforms all recent diffusion-policy online RLs on most tasks, and improves more than 120\% over soft actor-critic on complex locomotion tasks such as Humanoid and Ant.

\end{abstract}


\setlength{\abovedisplayskip}{2pt}
\setlength{\abovedisplayshortskip}{2pt}
\setlength{\belowdisplayskip}{2pt}
\setlength{\belowdisplayshortskip}{1pt}
\setlength{\jot}{1pt}
%
\setlength{\floatsep}{1ex}
\setlength{\textfloatsep}{1ex}



\vspace{-12pt}
\section{Introduction}
\begin{figure}[t]
  \centering
   \includegraphics[width=1\linewidth]{sec/figs/fig1.png}
   \caption{\captionMethodFigure}
   \label{fig:methodFigure}
\end{figure}
\begin{figure}[t]
  \centering
   \includegraphics[width=1\linewidth]{sec/figs/exampleFigure.png}
   \caption{\captionExampleFigure}
   \label{fig:exampleFigure}
   \vspace{-12pt}
\end{figure}
Vision-Language Models (VLMs), such as CLIP~\cite{CLIP}, have emerged as general-purpose systems for understanding visual data through language-based queries. These models enable a broad range of applications, from object detection to image captioning, by linking visual inputs to language prompts. In standard settings where images contain single, recognizable objects, VLMs perform remarkably well. However, for the more complex task of zero-shot multi-label recognition (MLR) (Fig. \ref{fig:methodFigure} (top)), where models must identify multiple objects within an image without prior training on specific data, VLMs face significant limitations. Zero-shot MLR is crucial for applications in fields like robotics and medical imaging, where objects rarely appear in configurations that align neatly with training distributions. In these scenarios, achieving robust multi-label recognition without fine-tuning is challenging, given the task’s reliance on mean Average Precision (mAP) scores, which depend on ranking images for object presence.

\noindent \textbf{VLM: Prompt Dependent AND/OR Noisy Channel.} Despite the promise of zero-shot capabilities, current VLM approaches often struggle with MLR due to inherent scoring behaviors and biases. The performance of these models is hampered by a mix of conjunction (AND) and disjunction (OR) behaviors in their scoring, leading to inflated scores in compound prompts that contain multiple objects. For example, a prompt like “cat and sofa” might yield a high score even if only one of these objects is present in the image. This tendency reflects biases learned during training, where common object pairs receive higher scores even when only one object is present, disrupting the accuracy of mAP-based evaluations. Furthermore, existing methods for adapting VLMs to zero-shot MLR frequently rely on prompt tuning or architectural adjustments—approaches that are often dependent on training data and computationally intensive fine-tuning, which limit their generalizability to novel tasks.

\noindent \textbf{Our Approach.} In contrast to these methods, we introduce SPARC (Score Prompting and Adaptive Fusion for Zero-Shot Multi-Label Recognition in VLMs), a novel approach to zero-shot MLR that bypasses the need for training data, prompt tuning, or model-specific modifications. Our method treats the VLM as a black box, relying solely on its output scores to infer object presence (see Fig. \ref{fig:methodFigure}). This black-box approach enables us to avoid assumptions about the model’s internal workings, allowing for a purely zero-shot framework that is both model-agnostic and dataset-independent. SPARC introduces two main innovations that address the unique challenges of zero-shot MLR.

\noindent \textbf{A. Compound Prompt Composition:} Recognizing that VLMs can provide richer information when prompted with combinations of objects, we develop a method for constructing compound prompts. These prompts reflect likely contextual associations between objects, such as “cat and sofa” or “car and bus.” By gathering scores from these compound prompts, we can capture a spectrum of potential object contexts within the image, enhancing detection without relying on training-based adaptations. This composition strategy allows us to agnostically extract information from the VLM, leveraging probable object relationships without depending on any specific dataset or VLM architecture.\\
\noindent \textbf{B. Score Debiasing and Adaptive Fusion.} A critical insight in our approach lies in the surprising observation that the \underline{maximum score among compound prompts} is often a poor proxy for true object presence. Although one might expect the highest score to serve as a reliable signal, we find that it frequently reflects compositional biases, as VLMs tend to respond to compound prompts with OR-like behavior, raising scores even when only one object in the prompt is present. Instead, we observe that the second-highest score consistently provides a more accurate indicator of object presence, minimizing the effects of false positives caused by compositional bias. Building on this insight, we develop a debiasing algorithm that normalizes scores across images to address image-specific noise and clarify genuine object presence signals. This debiased score set is then processed through a PCA-based fusion method that further refines object rankings by combining information from both compound and singleton prompts, ultimately optimizing mAP by enhancing score accuracy.\\
\textbf{Complementarity.} SPARC is complementary to other zero-shot and training-free MLR methods. When applied on top of these approaches, SPARC consistently enhances mAP scores by refining object ranking and reducing bias in VLM outputs. This capability makes SPARC an adaptable solution that can improve upon existing methods while maintaining a fully zero-shot, model-agnostic framework.

\noindent \textbf{Empirical Results.} SPARC achieves significant improvements in mAP, outperforming methods that incorporate architectural modifications. This outcome shows the potential of a fully zero-shot approach that relies only on systematic prompt design and score interpretation, rather than prompt-training or fine-tuning. By revealing that the second-highest score can be a superior proxy to the maximum, our findings provide new insights into VLM scoring behavior, suggesting that careful treatment of prompt compositions and score patterns can unlock robust MLR capabilities.



\section{Preliminaries}
We introduce the necessary preliminaries in this section. First, we introduce Markov decision process and maximum entropy reinforcement learning as our policy learning framework, followed by a recap of diffusion models.

\subsection{Maximum Entropy Reinforcement Learning}
\label{sec:max_ent_rl}
\textbf{Markov Decision Processes (MDPs).}~We consider Markov decision process~\citep{puterman2014markov} specified by a tuple $\mathcal{M}=(\mathcal{S}, \mathcal{A}, r, P, \mu_0, \gamma)$, where $\mathcal{S}$ is the state space, $\mathcal{A}$ is the action space, $r:\Scal\times\Acal\to\RR$ is a reward function,
$P\rbr{\cdot|s, a}: \mathcal{S} \times \mathcal{A} \rightarrow \Delta(\mathcal{S})$ is the transition operator with $\Delta(\mathcal{S})$ as the family of distributions over $\mathcal{S}, \mu_0 \in \Delta(\mathcal{S})$ is the initial distribution and $\gamma \in(0,1)$ is the discount factor. 

\textbf{Maximum entropy RL.} We follow the maximum entropy RL to learn our diffusion policies~\cite{haarnoja2017reinforcement}. We consider the following entropy-regularized expected return as the policy learning objective,
\begin{equation}
    \arg\max_\pi J(\pi) := \EE_{\pi}\sbr{\sum_{\tau=0}^\infty\gamma^\tau \rbr{r(\sbb_\tau, \ab_\tau) + \lambda\Hcal(\pi(\cdot | \sbb_\tau))}}\label{eq:obj_max_ent_rl}
\end{equation}
where $\Hcal\rbr{\pi(\cdot| \sbb)} = \EE_{\ab\sim\pi(\cdot | \sbb)}[-\log\pi(\ab | \sbb)]$ is the entropy, $\lambda$ is a regularization coefficient for the entropy. 
The soft policy iteration algorithm~\cite{haarnoja2017reinforcement,haarnoja2018soft} is proposed to solve the optimal max-entropy policy. Soft policy iteration algorithm iteratively conducts soft policy evaluation and soft policy improvement, where soft policy evaluation updates the soft $Q$-function by repeatedly applying soft Bellman update operator $\Tcal^\pi$ to current value function $Q:\Scal\times\Acal\to \RR$, \ie, 
\begin{equation}
    \Tcal^\pi Q(\sbb_\tau, \ab_\tau) = r(\sbb_\tau, \ab_\tau) + \gamma\EE_{\sbb_{\tau+1}\sim P}\sbr{V(\sbb_{\tau+1})}\label{eq:soft_pev}
\end{equation}
where $V(\sbb_\tau) = \EE_{\ab_\tau\sim\pi}\sbr{Q(\sbb_\tau, \ab_\tau) - \lambda \log\pi(\ab_\tau\mid \sbb_\tau)}$ \cite{haarnoja2018soft}. Then in the soft policy improvement stage, the policy is updated to fit the target policy
\begin{equation}
    \pi_{\rm target}(\ab| \sbb) 
    % \frac{\exp\rbr{\frac{1}{\lambda}{Q^{\pi_{\rm old}}(\sbb, \ab)}}}{Z^{\pi_{old}}(\sbb)}
    \propto \exp\rbr{\frac{1}{\lambda}{Q^{\pi_{\rm old}}(\sbb, \ab)}} \label{eq:energy_based_opt_pi}
\end{equation}
where ${\pi_{\rm old}}$ is the current policy and $Q^{\pi_{\rm old}}$ is the converged result of \eqref{eq:soft_pev} with $\Tcal^{\pi_{\rm old}}$.
% \Bo{delete this section. we discuss the connection between ebm and diffusion in our method section. as we discuss on last friday, please follow my diffusion spectral representatio paper logic. }

\textbf{Soft Actor-Critic.} 
Although we have closed-form policy~\eqref{eq:energy_based_opt_pi}, it is a \emph{unnormalized} distribution, often referred to as an \emph{energy-based} policy since the unnormalized density is called energy function in literature, which is notoriously difficult to sample from and learn. To enable efficient computation, a natural idea is to approximate the energy-based policies~\eqref{eq:energy_based_opt_pi} with a parametrized distribution. 
A representative algorithm is the well-known soft actor-critic (SAC), which restricts the policy to be a parametrized Gaussian, \ie, $\pi_\theta\rbr{a|s} = \Ncal\rbr{ \mu_{\theta_1}(s), \sigma^2_{\theta_2}(s)}$ and updates the parameters $\theta = [\theta_1, \theta_2]$ by optimizing the $KL$-divergence to the target policy $D_{KL}(\pi_\theta\|\pi_{\rm target})$~\cite{haarnoja2018soft} via policy gradient with parametrization trick, \ie, % which equals a Q-learning style loss with data samples from $\Dcal$
$$
J^\pi_{\rm SAC}(\theta)=\EE_{\sbb\sim\Dcal,\ab\sim\pi_\theta}\sbr{\lambda\log\pi_\theta(\ab|\sbb)-Q^{\pi_{\rm old}}(\sbb,\ab)}. 
$$ 
% \Bo{with explicit SAC policy update objective here. }\haitong{done.}
% Even with the Gaussian reduction, SAC has been SOTA till now, showing the huge potential of the maximum entropy RL. 
The Gaussian approximation loses the inherent expressiveness and multimodality of energy-based policies, thus limiting the performance of maximum entropy RL algorithms. This limitation motivates the pursuit of more expressive policy structures to further enhance performance.


% \textbf{Abstact problem notations.} From now on, we focus on an abstract version of representing energy-based policies in \eqref{eq:energy_based_opt_pi}.
% Specifically, we study how to train and sample from an EBM with the given energy function \emph{only}, \ie, we have a target distribution $p_0$ with known energy function $E$ with $p_0(\xb_0)\propto\exp\rbr{-E(\xb_0)}$. Only means we cannot sample from $p_0$.

\subsection{Denoising Diffusion Probabilistic Models} 
\label{sec.diffusion}
Denoising diffusion probabilistic models~\citep[DDPMs,][]{sohl-dickstein2015deep,song2019generative,ho2020denoising} are powerful tools to represent and generate complex probability distributions. \emph{Given data samples from the data distribution} $p_0$, DDPMs are composed of a forward diffusion process that gradually perturbs the data distribution $\xb_0\sim p_0$ to a noise distribution $\xb_T\sim p_T$, and a reverse diffusion process that reconstructs data distribution $p_0$ from noise distribution $p_T$.
 The forward corruption kernel is usually Gaussian with a variance schedule $\beta_1,\dots,\beta_T$, resulting in the forward trajectories with joint distribution
\begin{align}
    &q_{{0:T}}(\xb_{0:T}) = p_0(\xb_0)\prod_{t=1}^Tq_{t|t-1}\rbr{\xb_t|\xb_{t-1}} \quad \text{where}\notag\\
    &q_{t|t-1}(\xb_t|\xb_{t-1}) := \Ncal(\xb_t;\sqrt{1-\beta_t}\xb_{t-1},\beta_t\Ib)\label{eq:corruption_ddpm}
\end{align}
where $\xb_t$ is random variable at $t$ step, and $p, q$ are probability distributions\footnote{We use $p$ and $q$ interchangeably as density function in this paper. Generally, $p$ represents intractable distributions (like the t-step marginal $p_t(\xb_t)$), and $q$ represents tractable distributions such as the Gaussian corruption $q_{t\mid t-1}(\xb_t|\xb_{t-1})$.}.
The backward process recovers the data distribution from a noise distribution $p_T$ with a series of reverse kernels $p_{t-1|t}(\xb_{t-1}|\xb_t)$. 
The reverse kernels are usually intractable so we parameterize it with neural networks denoted as $p_{\theta;t-1|t}\left(\xb_{t-1} \mid \xb_t\right)$, resulting in a joint distribution of the reverse process,
$$
% \begin{aligned}
    % &
    % \text{where}\\
    % & p_\theta\left(\xb_{t-1} \mid \xb_t\right):=\mathcal{N}\left(\xb_{t-1} ; \boldsymbol{\mu}_\theta\left(\xb_t, t\right), \mathbf{\Sigma}_\theta\left(\xb_t, t\right)\right)
% \end{aligned}
p_\theta\rbr{\xb_{0:T}}=p_T(\xb_T)\prod_{t=1}^Tp_{\theta;t-1|t}(\xb_{t-1}|\xb_{t})\quad
$$
Considering all $(\xb_1,\dots,\xb_T)$ as the latent variables, we can solve the parameters $\theta$ via optimizing the evidence lower bound (ELBO) over $\xb_0$,
$$
{\rm ELBO}(\theta) = \EE_{\xb_0\sim p_0}\EE_{\xb_{1:T}\sim q}\sbr{\log \frac{p_\theta(\xb_{0:T})}{q(\xb_{1:T}|\xb_0)}}.
$$
After fixing $p_{\theta;t-1|t}$ to be Gaussian and reparametrizing $p_{\theta;t-1|t}$ with a score network\footnote{Some paper reparameterize it as the noise prediction network $\epsb_\theta$, but they are the same in essence since $\nabla_{\xb_t} \log q_{t|0}(\xb_t|\xb_0) = -\frac{\xb_t -\sqrt{\bar\alpha_t}\xb_0}{1 -\bar\alpha_t}=-\frac{\epsb}{\sqrt{1 -\bar\alpha_t}}$ for Gaussian noise $\epsb$.} $s_\theta(\xb_t;t)$, maximizing the ELBO is equivalent to minimizing a collection of denoising score matching loss over multiple noise levels indexed by $t$~\cite{vincent2011connection,ho2020denoising},
\begin{equation}
    \begin{aligned}
        &\Lcal_{\rm DSM}(\theta):=\frac{1}{T}\sum_{t=0}^T(1 -\bar\alpha_t)\underset{\substack{\xb_0\sim p_0\\\xb_t\sim q_{t|0}}}{\EE}\sbr{\nbr{ s_\theta\rbr{\xb_t; t} - \nabla \log q_{t|0}(\xb_t|\xb_0)}^2}\label{eq:ddpm_loss}
    \end{aligned}
\end{equation}
where $q_{t|0}(\ab_t|\ab_0):=\Ncal(\ab_t;\sqrt{\bar\alpha_t}\ab_0, \rbr{1 - \bar\alpha_t}\Ib)$ and $\bar\alpha_t = \prod_{l=1}^t (1-\beta_l)$. 
After learning the $s_\theta$ by minimizing \eqref{eq:ddpm_loss}, we can draw sample via the reverse diffusion process by iteratively conducting
\begin{equation}
    \xb_{t-1}=\frac{1}{\sqrt{\bar\alpha_t}}\left(\xb_t+\beta_t s_\theta\left(\xb_t, t\right)\right)+\frac{1-\bar\alpha_{t-1}}{1-\bar\alpha_t}\beta_t \zb_t\label{eq:annealed_langevin_2}
\end{equation}
for $t = T, T-1,\dots, 1$ and $\zb_t\sim\Ncal\rbr{0,\Ib}$.

% \subsection{DDPM as Energy-Based Models} 
% Diffusion models can also be interpreted as energy-based models~(EBMs).
% EBMs usually refer to parametrized probabilistic models that are not normalized. The density function of an EBM is generally in the formulation of
% $
% p_\theta(\xb) = \frac{\exp\rbr{ - E_\theta(\xb)}}{Z_\theta}
% $,
% where $E_\theta(x)$ is the \emph{energy function} parameterized by $\theta$, $Z_\theta = \int \exp\rbr{ - E_\theta(\xb)} d \xb$ is the normalization constant that is usually intractable. The \emph{score function} is defined as the gradient of log density, $\nabla_x\log p_\theta(x) $. 

% The unnormalized nature of EBMs allows versatility and rich expressiveness compared to other probabilistic models but raises difficulties in training and sampling~\cite{song2021train}. One of the naive approaches before DDPM is to first learn the score function $\nabla_{\xb}\log p(\xb)$ via score matching~\cite{hyvarinen2005estimation,song2020sliced} and then sample via Langevin dynamics with the learned score function $f_\theta(\xb)\approx\nabla_{\xb}\log p(\xb)$~\cite{parisi1981correlation},
% % , a Markov chain Monte-Carlo (MCMC) sampling procedure using only the score function,
% \begin{equation}
%     \xb_{i+1} \leftarrow \xb_i+\eta f_\theta(\xb)+\sqrt{2 \eta} \mathbf{z}_i, \quad i=0,\cdots, K\label{eq:langevin}
% \end{equation}
% where $\mathbf{z}_i \sim \mathcal{N}(0, I)$. As $\eta\to 0$ and $K\to\infty$, the sampled $\xb_K$ converges to samples from $p(\xb)$. 

% \citet{song2019generative} showed that this naive approach, combining score matching with Langevin dynamics, suffered from multiple pitfalls, such as slow mixing and inaccuracy in low-density regions, which hinders the empirical performance of score-based EBMs. One remedy proposed by \citet{song2019generative} is to fit the score function of a series of noise-perturbed data distribution $\Ncal\rbr{\xb_i;\xb, \sigma_i^2 \Ib}, i=\{1,2,\dots, K\}$ with a noise schedule $\sigma_1> \sigma_2>\dots>\sigma_K$. 
% The resulting models, named the noise-conditioned score networks (NCSN) $f_\theta\rbr{\xb_i;\sigma_i}$, take the noise level into the inputs and are learned by denoising score matching~\cite{vincent2011connection}
% \begin{align}
%     % &\EE_{\xbtil\sim \ptil}\sbr{\nbr{f_\theta\rbr{\xb_i;i} - \nabla_{\xbtil}\log \ptil_\sigma\rbr{\xbtil}}^2} =\label{eq:noisy_score_matching}\\
%     &\EE_{\xb\sim p,\xb_i\sim\Ncal\rbr{\xb,\sigma_i^2 \Ib}}\sbr{\|f_\theta\rbr{\xb_i;\sigma_i} -\nabla_{\xb_i}\log q(\xb_i|\xb) \|^2}\label{eq:dsm_loss}
% \end{align}
% Then in the sampling stage, \citet{song2019generative} replaced the original score function in \eqref{eq:langevin} with the learned noisy score function $f_\theta(\xbtil;\sigma_i)$,
% \begin{equation}
%     \xb_{i+1} \leftarrow \xb_i+\eta f_\theta(\xbtil;\sigma_i)+\sqrt{2 \eta} \mathbf{z}_i, \quad i=0,\cdots, K\label{eq:annealed_langevin}
% \end{equation}
% named as annealed Langevin dynamics. The scheduled noise perturbation design significantly improved the image generation performance to match the state-of-the-art (SOTA) at that time~\cite{song2019generative}, which is further refined by DDPM.

% We can see that the annealed Langevin dynamics~\eqref{eq:annealed_langevin} resembles the DDPM sampling~\eqref{eq:annealed_langevin_2} with different scale factors, and the denoising score matching loss~\eqref{eq:dsm_loss} is equivalent to \eqref{eq:ddpm_loss} since 
% $\nabla_{\xb_i}\log q(\xb_i|\xb)=-\frac{\xb_i-\xb}{\sigma_i^2}=-\frac{\epsb}{\sigma_i}$. Therefore, DDPM can be interpreted as EBMs and is equivalent to NCSN with different designs on noise schedules and scale factors. A more thorough discussion on their equivalency can also be found in~\citet{ho2020denoising,song2021scorebased}.











\begin{figure*}[ht]
    \centering
    \includegraphics[width=\linewidth]{figure/demo_paper.pdf}
    \caption{Demonstration of the proposed \algname~(\algabb) algorithm. We leverage diffusion policy to represent the energy-based policy in maximum entropy RL. The diffusion policy is trained via reverse sampling score matching, an algorithm that does not sample from the target energy-based policy and only depends on the $Q$-functions, enabling efficient online RL for diffusion policy.}
    \label{fig:demo}
\end{figure*}
\section{Diffusion Policy Learning in Online RL}
\label{sec:rssm}

In this section, 
we first present the connection of energy-based models and diffusion models, justifying the expressiveness of diffusion policy, and identify the difficulties in online training of diffusion policy. 
We then introduce the reverse sampling score matching (RSSM) to make the training of diffusion policy possible with \emph{only} access to the energy function in online RL.  








\subsection{Diffusion Models as Noise-Perturbed Energy-Based Models} 
We first revisit the energy-based view of diffusion models, \ie, \emph{diffusion models are noise-perturbed EBMs}~\cite{shribak2024diffusion}, to justify that the diffusion policy can efficiently represent the energy-based $\pi_{\rm target}$. 
Given $\sbb$, consider perturbing action samples $\ab_0 \sim \pi_{\rm target}(\cdot|\sbb)$ with corruption kernel $q_{t|0}(\ab_t|\ab_0)=\Ncal(\ab_t;\sqrt{\bar\alpha_t}\ab_0, \rbr{1 - \bar\alpha_t}\Ib)$, 
which results in the noisy-perturbed policy $\tilde\pi_t(\cdot|\sbb)$ with 
$$
\tilde\pi_t(\ab_t|\sbb)= \int q_{t|0}(\ab_t|\ab_0)\pi_{\rm target}(\ab_0|\sbb)d\ab_0
$$
for noise schedule index $t=1,2,\dots, T$.


\begin{restatable}[Diffusion models as noise-perturbed EBMs\label{prop:diff_ebm}]{proposition}{diffebm}
    The score network $s_\theta(\ab_t;\sbb, t)$ in \eqref{eq:ddpm_loss} matches noise-perturbed score functions, $\nabla_{\ab_t}\log \tilde\pi_t(\ab_t|\sbb)$,
        where state $\sbb$ is added to inputs of score network $s_\theta(\ab_t;\sbb, t)$ to handle conditional distributions, $p_0(\cdot)$ in \eqref{eq:ddpm_loss} refers to $\pi_{\rm target}(\cdot|\sbb)$ in the policy learning setting. \end{restatable}

\begin{proof}
    This can be shown by checking the noise-perturbed score function $\nabla_{\ab_t}\log\tilde\pi_t(\ab_t|\sbb_t)$, \ie,
    \begin{align}
        & \nabla_{\ab_t}\log\tilde\pi_t(\ab_t|\sbb_t)\label{eq:prop1_derivation}
        \\
            = & \frac{\nabla_{\ab_t}\tilde\pi_t\rbr{\ab_t|\sbb}}{\tilde\pi_t\rbr{\ab_t|\sbb}}           =  \frac{\nabla_{\ab_t}\int q_{t|0}(\ab_t|\ab_0)\pi_{\rm target}(\ab_0|\sbb)d\ab_0}{\tilde\pi_t\rbr{\ab_t|\sbb}}\notag\\
            = & \int \nabla_{\ab_t}\log q_{t|0}(\ab_t|\ab_0)\underbrace{\frac{q_{t|0}(\ab_t|\ab_0)\pi_{\rm target}(\ab_0|\sbb)}{\tilde\pi_t\rbr{\ab_t|\sbb}}}_{p_{0|t}(\ab_0|\ab_t, \sbb)}d\ab_0 \nonumber
    \end{align}
                        We match the noise-perturbed score function via the score network $s_\theta(\ab_t;\sbb, t)$ via optimizing the expectation of square error over $\ab_t\sim \tilde\pi_t(\cdot|\sbb)$,
    
    {\small
    \begin{align}
       &\EE_{\ab_t\sim\tilde\pi_t}\nbr{s_\theta(\ab_t;\sbb, t) - \int\nabla_{\ab_t}\log q_{t|0}(\ab_t|\ab_0)p_{0|t}(\ab_0|\ab_t,\sbb)d\ab_0}^2\notag\\         =&\underset{\substack{\ab_0\sim\pi_{\rm target}\\\ab_t\sim q_{t|0}}}{\EE}\sbr{\nbr{s_\theta(\ab_t;\sbb, t) -\nabla_{\ab_t}\log q_{t|0}(\ab_t|\ab_0)}^2} + \texttt{constant}\label{eq:tweedie_loss}
    \end{align}
    }
    The detailed derivations of \eqref{eq:tweedie_loss} are deferred to \Cref{sec:prop_1_apdx}.
    We can see that \eqref{eq:tweedie_loss} is equivalent to the $t$-th term in DDPM loss~\eqref{eq:ddpm_loss},
    which concludes the proof of Proposition~\ref{prop:diff_ebm}.
\end{proof}


Furthermore, as the noise schedule $\beta_t$ gets close to zero when $t$ goes from $T$ to $1$ in the reverse process~\eqref{eq:annealed_langevin_2}, the noise-perturbed EBMs gradually resemble the original energy-based policies $\pi_{\rm target}$. 
Adding adaptive levels of noise perturbations encourages explorations on the energy landscape, which significantly improves the sampling quality and makes diffusion models the key breakthrough in   EBMs~\cite{song2019generative}. 

\textbf{Difficulties to train diffusion model in online RL setup.}
By the connection between EBMs and diffusion models, we justify the expressiveness of diffusion policy for maximum entropy RL. 
However, training diffusion policy is highly non-trivial in online RL because of two major challenges:

$\bullet$ \textbf{Sampling challenge:} the vanilla diffusion training with denoising score matching~\eqref{eq:ddpm_loss} requires samples from the target policy $\pi_{\rm target}$, but we cannot access $\pi_{\rm target}$ directly in online RL since we only know the energy function, \ie, the $Q$-functions.

$\bullet$ \textbf{Computational challenge:} another possible solution is to backpropagate policy gradient thorough the whole reverse diffusion process~\eqref{eq:annealed_langevin_2}. However, this recursive gradient propagation not only incurs huge computational and memory cost, but also suffers from gradient vanishing or exploding, making diffusion policy learning expensive and unstable.

These challenges hinder the performance of diffusion-based policies in online RL. 












\subsection{Learning Noise-perturbed Score Functions via Reverse Sampling Score Matching }
\label{subsec:main_theorem}




In this section, we develop our core contribution, reverse sampling score matching (RSSM), an efficient diffusion policy learning algorithm that eliminates the aforementioned difficulties. 
Following the energy-based viewpoint in \Cref{prop:diff_ebm}, we propose the following theorem,


\begin{restatable}[Reverse sampling score matching (RSSM)\label{thm.rssm}]{theorem}{rssm}
    Define $\tilde p_t(\cdot|\sbb)$ as a sampling distribution whose support contains the support of $\tilde\pi_t(\cdot|\sbb)$ given $\sbb$.
                        Then we can learn the score network $s_\theta(\ab_t;\sbb,t)$  to match with the score function of noise-perturbed policy $\nabla_{\ab_t}\log \tilde\pi_t(\ab_t|\sbb)$ via minimizing
    \begin{equation}
        \underset{\substack{\ab_t\sim \tilde p_t\\ \tilde\ab_0\sim\tilde q_{0|t}}}{\EE}\sbr{\exp \rbr{Q\rbr{\sbb, \tilde\ab_0}/\lambda}\nbr{s_\theta\rbr{\ab_t; \sbb, t} - \nabla_{\ab_t}\log \tilde q_{0|t}\rbr{\tilde\ab_0 \! \mid \! \ab_t}}^2}\label{eq:thm1_reweight_loss}
    \end{equation}
    where we abbreviate $Q^{\pi_{\rm old}}$ with $Q$ for simplicity and $\tilde q_{0|t}$ is the \textbf{reverse sampling} distribution defined as
                \begin{equation}
        \tilde q_{0|t}(\tilde\ab_0|\ab_t):=\Ncal\rbr{\ab_0;\frac{1}{\sqrt{\bar\alpha_t}}\ab_t, \frac{1 - \bar\alpha_t}{\bar\alpha_t}\Ib}\label{eq:a0_sample_thm}
    \end{equation}
    which means $ \tilde\ab_0 = \frac{1}{\sqrt{\bar\alpha_t}}\ab_t - \sqrt{\frac{1 - \bar\alpha_t}{\bar\alpha_t}}\epsb$ for $\epsb\sim\Ncal(0,\Ib)$.
\end{restatable}

The name \emph{reverse sampling} score matching comes from that we first sample $\ab_t\sim \tilde p_t$ then sample $\tilde\ab_0\sim\tilde q_{0|t}$, thus bypassing the sampling issues and not increasing computational cost. We show a sketch proof here, the full derivations can be found in \Cref{sec:appendix_derivation}.

    \textit{Proof.} The derivations consists of two major steps, reformulating the noise-perturbed score function and applying the reverse sampling trick.
    \\\textbf{Reformatting the noise-perturbed score function.} First, we slightly reformat derivations of the noise-perturbed score function in \Cref{prop:diff_ebm} starting from \eqref{eq:prop1_derivation},
    \begin{equation}
        \begin{aligned}
             \nabla_{\ab_t}\log\tilde\pi_t(\ab_t|\sbb)            = &\int \nabla_{\ab_t}\log q_{t|0}(\ab_t|\ab_0)\frac{q_{t|0}(\ab_t|\ab_0)\pi_{\rm target}(\ab_0|\sbb)}{\tilde\pi_t\rbr{\ab_t|\sbb}}d\ab_0\\
            = & \frac{\int \nabla_{\ab_t}\log q_{t|0}(\ab_t|\ab_0) q_{t|0}(\ab_t|\ab_0)\exp\rbr{Q(\sbb,\ab_0)/\lambda} d\ab_0}{Z\rbr{\ab_t;\sbb}}\\
        \end{aligned}\label{eq:noise_perturbed_score}
    \end{equation}
    where $Z(\ab_t;\sbb) := \tilde{\pi}_t(\ab_t | \sbb) \int\exp\rbr{Q(\sbb,\ab_0)/\lambda} d\ab_0 = \int q_{t | 0}(\ab_t | \ab_0) \exp\rbr{Q(\sbb,\ab_0)/\lambda} d \ab_0 $ .
    Equation \eqref{eq:noise_perturbed_score} is obtained by substituting the energy function into $\pi_{\rm target}$.     With \eqref{eq:noise_perturbed_score}, the square error given $\ab_t$ satisfies
    \begin{equation}
        \begin{aligned}
            &\nbr{s_\theta(\ab_t;\sbb,t) - \nabla_{\ab_t}\log \tilde\pi_t(\ab_t|\sbb)}^2
            = \frac{1}{Z(\ab_t;\sbb)}\int q_{t|0}(\ab_t|\ab_0)\exp\rbr{Q(\sbb,\ab_0)/\lambda}l_\theta(\ab_0,\ab_t;\sbb)d\ab_0
        \end{aligned}
    \end{equation}
    where {$l_\theta(\ab_0,\ab_t;\sbb) = \nbr{s_\theta\rbr{\ab_t; \sbb, t} - \nabla_{\ab_t}\log q_{t|0}\rbr{\ab_t|\ab_0}}^2$}. 
    Then we integrate the square error over a custom measure  $g(\ab_t;\sbb):=Z(\ab_t;\sbb)\ptil_t(\ab_t|\sbb)$ to compensate the $Z(\ab_t;\sbb)$ and get to, 
    \begin{equation}
        \iint\ptil_t(\ab_t|\sbb) q_{t|0}(\ab_t|\ab_0)\exp\rbr{Q(\sbb,\ab_0)/\lambda}l_\theta(\ab_0,\ab_t;\sbb)d\ab_0d\ab_t\label{eq:reformat_loss_p}
    \end{equation}
    A more rigorous derivation is deferred to Appendix \ref{sec:appendix_derivation}. 
    
    \textbf{Reverse sampling trick.} The loss function in \eqref{eq:reformat_loss_p} is still not tractable. To handle this, we introduce the \emph{reverse sampling trick}, \ie, replacing $q_{t|0}$ with a reverse sampling distribution $\qtil_{0|t}$ that satisfies
    \begin{equation}
        \begin{aligned}
        &\qtil_{0|t}(\ab_0\mid \ab_t) =\Ncal\rbr{\ab_0;\frac{1}{\sqrt{\bar\alpha_t}}\ab_t, \frac{1 - \bar\alpha_t}{\bar\alpha_t}\Ib}\\
        \propto~& q_{t|0}(\ab_t\mid \ab_0) = \Ncal\rbr{\ab_t;\sqrt{\bar\alpha_t}\ab_0, \rbr{1 - \bar\alpha_t}\Ib},
    \end{aligned}\label{eq:reverse_gaussian}
    \end{equation}
    and their score functions match
    {
    \small
    $
    \nabla_{\ab_t}\log q_{t|0}(\ab_t\mid \ab_0) = \nabla_{\ab_t}\log \qtil_{0|t}(\ab_0\mid \ab_t) = - \frac{\ab_t -\sqrt{\bar\alpha_t}\ab_0}{1 -\bar\alpha_t}
    $
    }.
    Then we can replace $q_{t|0}$ with $\tilde q_{0|t}$ in \eqref{eq:reformat_loss_p} to get a tractable loss function,
    {
                \begin{equation}
        \iint \tilde p_t(\ab_t|\sbb)\tilde q_{0|t}(\tilde\ab_0| \ab_t)\exp\rbr{Q(\sbb,\tilde\ab_0)/\lambda} \tilde l_\theta(\ab_0,\ab_t;\sbb)d\tilde\ab_0d\ab_t \label{eq:with_tidle_q}
    \end{equation}
    }
where {\small $\tilde l_\theta(\ab_0,\ab_t;\sbb)  =  \nbr{s_\theta\rbr{\ab_t;\sbb, t}  -  \nabla_{\ab_t}\log \tilde q_{0|t}\rbr{\tilde\ab_0  \mid  \ab_t}}^2$}.
In this way, we can first sample $\ab_t\sim\tilde p_t$ and then sample $\tilde\ab_0\sim\tilde q_{0|t}$ to enable tractable loss computation.    
By further algebraic operations, we can derive the loss function in \eqref{eq:thm1_reweight_loss} from \eqref{eq:with_tidle_q}. The detailed derivation can be found in Appendix \ref{sec:appendix_derivation}.\hfill$\square$

We can see that with \Cref{thm.rssm}, the loss function \eqref{eq:thm1_reweight_loss} solves both the sampling and computational difficulties mentioned previously. First, we avoid sampling from target policy $\pi_{\rm target}$, and the sampling distribution $\tilde p_t$ is some distributions we can choose. Second, we have similar computation with denoising score matching~\eqref{eq:ddpm_loss}, avoiding extra computational cost induced by diffusion policy learning.


\begin{remark}[Broader applications of RSSM.] We emphasize that although we develop RSSM for online RL problems, the RSSM has its own merit and can be applied to any probabilistic modeling problems with known energy functions. We also show a toy example in \Cref{sec:toy} where we use RSSM to train a toy diffusion model to generate samples from a Gaussian mixture distribution.
\end{remark}

\begin{remark}[Pitfalls of Langevin dynamics in online RL.]\label{remark:langevin}
    Some might question that if we already know the energy function, why not compute the gradient as the score functions and use the Langevin dynamics~\cite{parisi1981correlation} to sample from policy~\eqref{eq:energy_based_opt_pi}.  The reasons are two-fold, i) the gradient of learned $Q$-function might not match the true score function; ii) Langevin dynamics suffers from the slow mixing problem~\citep[shown in \Cref{sec:toy}]{song2019generative} even with true score functions. Both pitfalls result in bad performance and motivate the necessity of diffusion policies with known energy functions.
\end{remark}

\subsection{Practical Diffusion Policy Learning Loss}
The direct impact of Theorem~\ref{thm.rssm} is a diffusion policy learning loss that can be sampled and computed efficiently in online RL. Specifically, summing over all timestep $t$ and state $\sbb$ in Equation~\eqref{eq:thm1_reweight_loss}, we derive the diffusion policy learning loss with RSSM:

{
\small
\begin{equation}
    \begin{aligned}
        &\Lcal^\pi(\theta;Q,\lambda) := \frac{1}{T}\sum_{t=1}^T\underset{{\sbb, \ab_t,\tilde\ab_0}}{\EE}\sbr{\exp\rbr{\frac{Q(\sbb,\tilde\ab_0)}{\lambda}}\nbr{s_\theta\rbr{\ab_t, \sbb, t} - \nabla_{\ab_t}\log\qtil_{0|t}(\tilde\ab_0|\ab_t)}^2}
    \end{aligned}\label{eq:policy_loss}
\end{equation}
}
with 
\begin{equation}
    \tilde\ab_0 = \frac{1}{\sqrt{\bar\alpha_t}}\ab_t - \sqrt{\frac{1 - \bar\alpha_t}{\bar\alpha_t}} \epsb,\, \epsb\sim\Ncal\rbr{0, \Ib}\label{eq:a0_sample}
\end{equation}
and $\sbb$ sampled from the replay buffer, $\ab_t$ sampled from $\ptil_t(\cdot|\sbb)$. 

Obviously, such sampling protocol in~\eqref{eq:policy_loss} and~\eqref{eq:a0_sample} bypasses sampling from the target optimal policy, therefore, can be easily implemented. Meanwhile, the obtained loss avoids recursive gradient backpropagation, largely reducing computation complexity of policy gradient. These benefits perfectly echo the difficulties of sampling and computations in applying vanilla diffusion model training to online RL, enabling efficient diffusion policy learning. 







\section{Hierarchical Clustered 
Federated Learning } \label{sec:algorithm} 

\subsection{Overview}

\begin{figure}[tb!]
\centerline{\includegraphics[width=1\linewidth]{Figure/Algorithm.png}}
\caption{Overview of the proposed framework \algname.}
\label{fig:overview}
\end{figure}

\begin{algorithm}
\begin{algorithmic}[1]
\caption{\algname for satellite networks}\label{alg:decentralized_FL}
    \REQUIRE{Information of satellite networks, $K$, $C$}
    \ENSURE{Global model $w_{G}$}

\STATE Conduct satellite-clustered parameter server selection algorithm;  \label{line:1}

\STATE \COMMENT{Satellite client initialization model parameters}  \label{line:2}
\FOR {$\forall$ satellites}
     \STATE Initialize global model parameter $w_0$;
\ENDFOR  \label{line:5}

%\STATE All satellites: Initialize global model parameter $w^{(a)}_0$;

\FOR {each FL round $m \in M$}  \label{line:6}
    \STATE \COMMENT{Train local model (in-orbit computing)}
    \FOR {each satellite $i \in C$ in parallel}
        \STATE $w_{m,\lambda+1}^{i} = w_{m,\lambda}^{i} - \eta \nabla \tilde{f}_i(w_{m,\lambda}^{i})$;
    \ENDFOR \label{line:9}
    %\STATE \note{Conduct adaptive weight quantification algorithm to obtain $w{_{m}^{i}}^{\prime}$;}
    \STATE \COMMENT{Aggregate satellite cluster models}
    \FOR {each satellite $i \in C^k$ in the same cluster}  \label{line:12}
        \STATE $w_{m+1} = w_m + \sum_{i \in C^k_i} p_i w_m$;   \label{line:13}
    \STATE \COMMENT{Check if it is necessary to reassemble clusters}  \label{line:14}
        \STATE Calculate dropout rate: $d_r = \frac{C^d}{C^k}$
        \IF {$d_r > Z$}
            \STATE Re-cluster the satellites
        \ENDIF  \label{line:18}
    \ENDFOR
\ENDFOR  
\STATE \COMMENT{Aggregate global model at the ground station}  \label{line:21}
\FOR {each PS $G_{k}, k \in {K}$ in different satellite clusters}
    \STATE $w_{G} =  \sum_{k \in {K}} \frac{D_k}{D} w_m^k$;  
\ENDFOR  \label{line:23}
\RETURN $w_{G}$.
\end{algorithmic}
\end{algorithm}

\figurename~\ref{fig:overview} illustrates the flowchart of our proposed \algname. The clustering FL clustering process in \algname includes two stages: \textbf{satellite cluster aggregation stage} (Step 1-3) and \textbf{ground station aggregation stage} (Step 4). In the satellite cluster aggregation stage, a clustering algorithm is introduced to divide the satellites into distinct clusters. Within each satellite cluster $K_i^a$, the algorithm selects a satellite near the cluster center with strong communication capabilities to act as the PS. The PS is responsible for aggregating model parameters from satellites within its cluster and establishing communication with the corresponding ground stations (Step 1). During the local training process, satellites may dynamically join or leave a cluster, necessitating re-aggregation. To address this issue, MAML is introduced to adjust the initial model parameters of the newly joined satellites, allowing them to better adapt to the tasks of the new cluster (Step 2). This method accelerates the overall convergence of the satellite PS aggregation process. Following each training round, the PS combines parameters from all satellites within its cluster and distributes the updated, aggregated parameters back to them (Step 3).

%This approach facilitates the acceleration of the overall convergence of the satellite PS aggregation process. After each training round, the PS aggregates the parameters of the satellites in its cluster and distributes the \note{Do we need to add ''updated" here?} aggregated parameters to them (Step 3). 

After a specified number of training rounds in the satellite cluster, the ground station aggregation stage starts. In this stage, the ground station communicates with visible satellite clusters to aggregate their model parameters of the respective satellite clusters they are affiliated. Finally, the ground station returns the trained model parameters to the respective satellite clusters (Step 4).

As detailed in Algorithm~\ref{alg:decentralized_FL},  we introduce a satellite-clustered parameter server selection algorithm to partition the original satellite network into distinct satellite clusters based on the satellite network information. For each cluster, the algorithm selects a satellite near the cluster center with robust communication capabilities as the PS (line \ref{line:1}). Then we initialize the global model parameters $w_0$ for all satellite clients within each cluster (lines \ref{line:2}-\ref{line:5}). During each round of FL aggregation, the local satellite client first performs local training to update the global model parameters $w_{m,\lambda+1}^{i}$ after training round $\lambda$ (lines \ref{line:6}-\ref{line:9}). These parameters are then transmitted to their cluster's PS for aggregation. 

After $m$ rounds of training in each satellite cluster, each PS forwards its aggregated parameters to its associated ground station for global aggregation, producing the updated model parameters $w_{m+1}$ (lines \ref{line:12}-\ref{line:13}). During global aggregation, satellite clusters monitor whether the number of dropped-out satellites $C^d$ exceeds a predefined threshold, triggering re-clustering when necessary (lines \ref{line:14}-\ref{line:18}). Finally, ground station broadcasts the global parameters to all affiliated satellites in their clusters, completing the hierarchically clustered FL process (lines \ref{line:21}-\ref{line:23}). 




%As shown in Algorithm~\ref{alg:decentralized_FL}, based on the satellite network information, we introduce a satellite-clustered parameter server selection algorithm to partition the original satellite network into distinct clusters. Within each satellite cluster, the algorithm selects a satellite near the cluster center with strong communication capabilities to serve as the PS (line \ref{line:1}). Then the global model parameters are initialized for all satellite clients within the satellite cluster (lines \ref{line:2}-\ref{line:5}). During each round of FL aggregation, the local satellite client first performs local training and obtains the trained global model parameters $w_{m,\lambda+1}^{i}$ after training round $\lambda$ (lines \ref{line:6}-\ref{line:9}). The satellite clients then transmit their global parameters $w_{m,\lambda+1}^{i} $ to satellite PS for aggregate global model parameters. After completing $m$ rounds of training in each satellite cluster, the global model parameters are transmitted to the ground station associated with the satellite PS for final aggregation, yielding updated aggregated global model parameters $w_{m+1}$ (lines \ref{line:12}-\ref{line:13}). During global aggregation, satellite clusters need to assess whether the number of dropped-out satellites $C^d$ exceeds a critical threshold, necessitating re-clustering (lines \ref{line:14}-\ref{line:18}). Finally, the ground station broadcasts the global parameters to all affiliated satellites in the cluster, completing the hierarchically clustered FL process (lines \ref{line:21}-\ref{line:23}). \note{please double check this paragraph to avoid the meaning changes after my rewrite}

To accelerate the convergence of the global model, we assign weights to clients based on the quality of their model updates. The quality is evaluated using the loss value of the client's local model. Let $L_i$ denote the loss value of the client $i$. The weight $p_i$ assigned to client $i$ is given by:
\begin{equation}
p_i = \frac{\frac{1}{L_i}}{\sum_{ i \in C^k} \frac{1}{L_i}}
\label{eq:p}
\end{equation}

\begin{comment}
The local updates from the satellite clients are then weighted and aggregated according to the weight parameter $p_i$ by:
\begin{equation}
w_{m+1} = w_m + \sum_{i \in U_k} p_i Q_m^i(w_{m+1}^{i} - w_m)
\end{equation}
\end{comment}
%If $p_i$ is 0, the client has not been selected by PS.

\subsection{Satellite-clustered parameter server selection algorithm}

We introduce a satellite-clustered parameter server selection algorithm that partitions the original satellite network topology into a predefined number of clusters $K$, optimizing the clustering process. Our algorithm iteratively refines the cluster centroids and the membership of associated satellites. Initially, $K$ centroids are randomly selected from the satellite location data. These locations are typically derived from geographic coordinates based on the satellite location parameters, i.e., inclination and orbital altitude. Each satellite is assigned to the nearest cluster centroid using the Euclidean distance metric, thereby forming initial clusters. The Euclidean distance between a satellite position vector
$\mathbf{C}^i = \{C_{1}^i, C_{2}^i, \ldots, C_{n}^i \}$ and a centroid $\mathbf{K}^j = \{K_{1}^j, K_{2}^j, \ldots, K_{n}^j \}$ is calculated as:
\begin{equation}
d(\mathbf{C}^i, \mathbf{K}^j) = \sqrt{\sum_{k=1}^{n} (C_{k}^i - K_{k}^j)^2}
\end{equation}

In the next update step, our algorithm recalculates the centroids by computing the mean position of all satellites assigned to each cluster. This process effectively repositions the centroids to more accurately represent the distribution of their associated satellites. For each cluster $K_k^i$, the new centroid $\mathbf{K}^j$ is obtained by:
\begin{equation}
\mathbf{K}^j = \frac{1}{|K^j|} \sum_{\mathbf{C}^i \in K^j} \mathbf{C}^i
\end{equation}
where $|K^j|$ represents the number of satellites in cluster $K^j$. The iterative process continues until the centroids stabilize, indicating their positions no longer change significantly between iterations. This indicates that the algorithm has converged to a local optimum. The convergence criterion is given by:
\begin{equation}
\sum_{j=1}^{|K|} \|\mathbf{K}_{\text{new}}^j - \mathbf{K}_{\text{old}}^j\|^2 < \epsilon
\end{equation}
where $|K|$ represents the number of clusters, and $\epsilon$ is a small positive number indicating stability in centroid positions.
The satellite nearest to the cluster centroid is designated as the PS for the respective cluster.


\subsection{Meta-learning-driven satellite re-clustering algorithm}

In dynamic satellite federated learning, the diverse training objectives of satellite clients, combined with their frequent network participation changes, can hinder model convergence and increase resource consumption. As a result, achieving acceptable performance requires substantial time and a large number of data samples.
%In dynamic satellite federated learning, the diverse training goals of satellite clients, coupled with their frequent network participation changes, can impede model convergence, resulting in heightened resource utilization. This necessitates a significant investment of time and data samples to attain acceptable performance. %the different training objectives of satellite clients, along with their frequent joining or leaving the network, can slow down model convergence, leading to increased resource consumption. This requires a large amount of time and data samples to achieve acceptable results. 

To address this challenge, we propose a satellite re-clustering algorithm based on meta-learning, extending the original satellite-clustered parameter server selection algorithm. When a new satellite joins the network, it inherits model updates from the head node of a specified cluster, rather than starting training from scratch. The core idea of the MAML approach is to identify a set of meta-initialization parameters that enable the model to achieve strong performance with just one or two gradient updates, even with a small number of new task examples.

First, we sample satellite clients from different clusters denoted as $ S = \{S_1, S_2, \dots, S_n\} $. Each satellite client is assigned a task $\textit{TK}_i$, which consists of a dataset $D_i$  and a loss function $L_{S_i}(w)$. The objective is to minimize the loss of the model on the task $\textit{TK}_i$. Then, an inner-loop adaptation is performed for each selected satellite node to fine-tune the global model $w$ by:
\begin{equation}
w_i' = w - \alpha \nabla_{w} L_{S_i}(w)
\end{equation}
where $\alpha$ is the local learning rate. Finally, an outer-loop meta-update is applied to aggregate the model updates from different satellite nodes, updating the global initialization by:
\begin{equation}
w^{t+1} = w^{t} - \beta \sum_{i \in S} \nabla_{w} L_{S_i}(w_i')
\end{equation}
where $\beta$ is the meta-learning rate, $w^{t}$ is the parameter of the current global model.




\section{Experimental Results}
\begin{table*}[t]
\centering
\caption{Total Variation Distance on CIFAR-10-LT ($N_l = 500$, $M_l = 4000$) with different class imbalance ratios $\gamma_l$ and $\gamma_u$ under five different unlabeled class distributions.}
\label{tab:cifar10-tv}
\resizebox{\textwidth}{!}{
\begin{tabular}{lccccccccccc}
\toprule
& & \multicolumn{2}{c}{consistent} & \multicolumn{2}{c}{uniform} & \multicolumn{2}{c}{reversed} & \multicolumn{2}{c}{middle} & \multicolumn{2}{c}{head-tail} \\
\cmidrule(lr){3-4} \cmidrule(lr){5-6} \cmidrule(lr){7-8} \cmidrule(lr){9-10} \cmidrule(lr){11-12}
& & $\gamma_l = 150$ & $\gamma_l = 100$ & $\gamma_l = 150$ & $\gamma_l = 100$ & $\gamma_l = 150$ & $\gamma_l = 100$ & $\gamma_l = 150$ & $\gamma_l = 100$ & $\gamma_l = 150$ & $\gamma_l = 100$ \\
Model & Estimator & $\gamma_u = 150$ & $\gamma_u = 100$ & $\gamma_u = 1$ & $\gamma_u = 1$ & $\gamma_u = 1/150$ & $\gamma_u = 1/100$ & $\gamma_u = 150$ & $\gamma_u = 100$ & $\gamma_u = 150$ & $\gamma_u = 100$ \\
\midrule
Supervised & MLLS & 0.269 ± 0.252 & 0.038 ± 0.006 & 0.251 ± 0.046 & 0.255 ± 0.060 & 0.429 ± 0.028 & 0.493 ± 0.050 & 0.333 ± 0.042 & 0.320 ± 0.009 & 0.457 ± 0.034 & 0.444 ± 0.043 \\
Supervised & RLLS & 0.043 ± 0.001 & 0.044 ± 0.010 & 0.348 ± 0.034 & 0.305 ± 0.068 & 0.769 ± 0.016 & 0.678 ± 0.028 & 0.430 ± 0.008 & 0.368 ± 0.013 & 0.539 ± 0.018 & 0.503 ± 0.020 \\
\midrule
MLE & IPW & 0.027 ± 0.001 & 0.027 ± 0.000 & 0.319 ± 0.072 & 0.243 ± 0.010 & 0.674 ± 0.020 & 0.646 ± 0.041 & 0.438 ± 0.020 & 0.454 ± 0.026 & 0.547 ± 0.049 & 0.491 ± 0.059 \\
MLE & OR & 0.045 ± 0.004 & 0.042 ± 0.000 & 0.215 ± 0.026 & 0.203 ± 0.032 & 0.433 ± 0.017 & 0.395 ± 0.033 & 0.193 ± 0.006 & 0.209 ± 0.037 & 0.307 ± 0.147 & 0.249 ± 0.130 \\
MLE & DR & 0.090 ± 0.002 & 0.079 ± 0.000 & 0.407 ± 0.027 & 0.360 ± 0.007 & 0.425 ± 0.007 & 0.421 ± 0.029 & 0.256 ± 0.001 & 0.286 ± 0.031 & 0.435 ± 0.136 & 0.362 ± 0.122 \\
\midrule
EM & IPW & 0.035 ± 0.002 & 0.040 ± 0.001 & 0.021 ± 0.001 & 0.029 ± 0.015 & 0.303 ± 0.187 & 0.091 ± 0.010 & 0.119 ± 0.011 & 0.105 ± 0.022 & 0.104 ± 0.026 & 0.104 ± 0.051 \\
EM & OR & 0.037 ± 0.003 & 0.042 ± 0.002 & 0.016 ± 0.001 & 0.024 ± 0.012 & 0.269 ± 0.183 & 0.090 ± 0.008 & 0.122 ± 0.012 & 0.103 ± 0.022 & 0.072 ± 0.012 & 0.073 ± 0.024 \\
EM & DR & 0.034 ± 0.004 & 0.037 ± 0.001 & 0.014 ± 0.001 & 0.027 ± 0.020 & 0.264 ± 0.191 & 0.092 ± 0.005 & 0.111 ± 0.019 & 0.097 ± 0.026 & 0.077 ± 0.016 & 0.073 ± 0.028 \\
\midrule
SimPro & IPW & 0.070 ± 0.011 & 0.058 ± 0.000 & 0.046 ± 0.001 & 0.049 ± 0.005 & 0.254 ± 0.074 & 0.223 ± 0.098 & 0.097 ± 0.025 & 0.067 ± 0.002 & 0.105 ± 0.066 & 0.110 ± 0.079 \\
SimPro & OR & 0.071 ± 0.012 & 0.058 ± 0.000 & 0.045 ± 0.001 & 0.049 ± 0.006 & 0.040 ± 0.003 & 0.059 ± 0.017 & 0.074 ± 0.006 & 0.075 ± 0.002 & 0.033 ± 0.003 & 0.033 ± 0.003 \\
SimPro & DR & 0.017 ± 0.004 & 0.026 ± 0.001 & 0.019 ± 0.002 & 0.018 ± 0.003 & 0.039 ± 0.003 & 0.058 ± 0.025 & 0.091 ± 0.007 & 0.031 ± 0.001 & 0.015 ± 0.003 & 0.019 ± 0.007 \\
\bottomrule
\end{tabular}
}
\end{table*}


\begin{table*}[t]
\centering
\caption{Total Variation Distance on CIFAR-100-LT ($N_l = 50$, $M_l = 400$) with different class imbalance ratios $\gamma_l$ and $\gamma_u$ under five different unlabeled class distributions.}
\label{tab:cifar100-tv}
\resizebox{\textwidth}{!}{
\begin{tabular}{lccccccccccc}
\toprule
& & \multicolumn{2}{c}{consistent} & \multicolumn{2}{c}{uniform} & \multicolumn{2}{c}{reversed} & \multicolumn{2}{c}{middle} & \multicolumn{2}{c}{head-tail} \\
\cmidrule(lr){3-4} \cmidrule(lr){5-6} \cmidrule(lr){7-8} \cmidrule(lr){9-10} \cmidrule(lr){11-12}
& & $\gamma_l = 20$ & $\gamma_l = 10$ & $\gamma_l = 20$ & $\gamma_l = 10$ & $\gamma_l = 20$ & $\gamma_l = 10$ & $\gamma_l = 20$ & $\gamma_l = 10$ & $\gamma_l = 20$ & $\gamma_l = 10$ \\
Model & Estimator & $\gamma_u = 20$ & $\gamma_u = 10$ & $\gamma_u = 1$ & $\gamma_u = 1$ & $\gamma_u = 1/20$ & $\gamma_u = 1/10$ & $\gamma_u = 20$ & $\gamma_u = 10$ & $\gamma_u = 20$ & $\gamma_u = 10$ \\
\midrule
Supervised & MLLS & 0.707 ± 0.016 & 0.313 ± 0.100 & 0.445 ± 0.172 & 0.309 ± 0.119 & 0.383 ± 0.075 & 0.397 ± 0.006 & 0.570 ± 0.001 & 0.373 ± 0.107 & 0.543 ± 0.009 & 0.231 ± 0.057 \\
Supervised & RLLS & 0.520 ± 0.007 & 0.133 ± 0.003 & 0.337 ± 0.125 & 0.253 ± 0.082 & 0.424 ± 0.060 & 0.463 ± 0.003 & 0.454 ± 0.021 & 0.306 ± 0.074 & 0.460 ± 0.028 & 0.241 ± 0.040 \\
\midrule
MLE & IPW & 0.075 ± 0.000 & 0.071 ± 0.001 & 0.229 ± 0.001 & 0.167 ± 0.002 & 0.565 ± 0.005 & 0.443 ± 0.007 & 0.415 ± 0.000 & 0.311 ± 0.005 & 0.343 ± 0.000 & 0.280 ± 0.001 \\
MLE & OR & 0.065 ± 0.002 & 0.061 ± 0.001 & 0.200 ± 0.007 & 0.143 ± 0.001 & 0.526 ± 0.011 & 0.399 ± 0.023 & 0.360 ± 0.003 & 0.256 ± 0.012 & 0.328 ± 0.003 & 0.266 ± 0.005 \\
MLE & DR & 0.149 ± 0.019 & 0.145 ± 0.010 & 0.243 ± 0.004 & 0.214 ± 0.019 & 0.568 ± 0.005 & 0.464 ± 0.014 & 0.403 ± 0.014 & 0.309 ± 0.012 & 0.365 ± 0.007 & 0.320 ± 0.004 \\
\midrule
EM & IPW & 0.097 ± 0.008 & 0.092 ± 0.004 & 0.239 ± 0.007 & 0.179 ± 0.003 & 0.478 ± 0.012 & 0.329 ± 0.020 & 0.262 ± 0.016 & 0.202 ± 0.003 & 0.312 ± 0.002 & 0.227 ± 0.001 \\
EM & OR & 0.121 ± 0.007 & 0.108 ± 0.005 & 0.261 ± 0.007 & 0.189 ± 0.004 & 0.489 ± 0.013 & 0.335 ± 0.020 & 0.274 ± 0.016 & 0.211 ± 0.004 & 0.336 ± 0.003 & 0.235 ± 0.001 \\
EM & DR & 0.125 ± 0.005 & 0.111 ± 0.004 & 0.269 ± 0.007 & 0.194 ± 0.005 & 0.497 ± 0.010 & 0.336 ± 0.024 & 0.281 ± 0.019 & 0.219 ± 0.008 & 0.336 ± 0.007 & 0.233 ± 0.004 \\
\midrule
SimPro & IPW & 0.125 ± 0.001 & 0.100 ± 0.005 & 0.166 ± 0.007 & 0.141 ± 0.009 & 0.353 ± 0.023 & 0.261 ± 0.008 & 0.202 ± 0.003 & 0.158 ± 0.005 & 0.277 ± 0.009 & 0.197 ± 0.003 \\
SimPro & OR & 0.133 ± 0.005 & 0.100 ± 0.004 & 0.160 ± 0.007 & 0.138 ± 0.010 & 0.322 ± 0.014 & 0.253 ± 0.008 & 0.202 ± 0.003 & 0.156 ± 0.005 & 0.269 ± 0.006 & 0.191 ± 0.004 \\
SimPro & DR & 0.122 ± 0.003 & 0.106 ± 0.006 & 0.188 ± 0.009 & 0.149 ± 0.006 & 0.343 ± 0.023 & 0.257 ± 0.007 & 0.219 ± 0.010 & 0.172 ± 0.002 & 0.279 ± 0.007 & 0.198 ± 0.004 \\
\bottomrule
\end{tabular}
}
\end{table*}
\begin{table*}[t]
\centering
\caption{Top-1 accuracy (\%) on CIFAR-10-LT ($N_l = 500$, $M_l = 4000$) with different class imbalance ratios $\gamma_l$ and $\gamma_u$ under five different unlabeled class distributions. In most settings, our two stage algorithm improves SimPro (9 / 10) and BOAT (8 / 10). We use {\green green} to indicate when our plug-in improves and {\red red} when it degrades the base model.}
\label{tab:cifar10-acc}
\resizebox{\textwidth}{!}{
\begin{tabular}{lcccccccccc}
\toprule

& \multicolumn{2}{c}{consistent} & \multicolumn{2}{c}{uniform} & \multicolumn{2}{c}{reversed} & \multicolumn{2}{c}{middle} & \multicolumn{2}{c}{head-tail} \\
\cmidrule(lr){2-3} \cmidrule(lr){4-5} \cmidrule(lr){6-7} \cmidrule(lr){8-9} \cmidrule(lr){10-11}

& $\gamma_l = 150$ & $\gamma_l = 100$ & $\gamma_l = 150$ & $\gamma_l = 100$ & $\gamma_l = 150$ & $\gamma_l = 100$ & $\gamma_l = 150$ & $\gamma_l = 100$ & $\gamma_l = 150$ & $\gamma_l = 100$ \\
& $\gamma_u = 150$ & $\gamma_u = 100$ & $\gamma_u = 1$ & $\gamma_u = 1$ & $\gamma_u = 1/150$ & $\gamma_u = 1/100$ & $\gamma_u = 150$ & $\gamma_u = 100$ & $\gamma_u = 150$ & $\gamma_u = 100$ \\

\midrule

FixMatch & 62.9 $\pm$ 0.36 & 67.8 $\pm$ 1.13 & 67.6 $\pm$ 2.56 & 73.0 $\pm$ 3.81 & 59.9 $\pm$ 0.82 & 62.5 $\pm$ 0.94 & 64.3 $\pm$ 0.63 & 71.7 $\pm$ 0.46 & 58.3 $\pm$ 1.46 & 66.6 $\pm$ 0.87 \\
CReST+ & 67.5 $\pm$ 0.45 & 76.3 $\pm$ 0.86 & 74.9 $\pm$ 0.90 & 82.2 $\pm$ 1.53 & 62.0 $\pm$ 1.18 & 62.9 $\pm$ 1.39 & 58.5 $\pm$ 0.68 & 71.4 $\pm$ 0.60 & 59.3 $\pm$ 0.72 & 67.2 $\pm$ 0.48 \\
DASO & 70.1 $\pm$ 1.81 & 76.0 $\pm$ 0.37 & 83.1 $\pm$ 0.47 & 86.6 $\pm$ 0.84 & 64.0 $\pm$ 0.11 & 71.0 $\pm$ 0.95 & 69.0 $\pm$ 0.31 & 73.1 $\pm$ 0.68 & 70.5 $\pm$ 0.59 & 71.1 $\pm$ 0.32 \\
% w/ ACR$\dagger$ (Wei \& Gan, 2023) & 70.9 $\pm$ 0.37 & 76.1 $\pm$ 0.42 & 91.9 $\pm$ 0.02 & 92.5 $\pm$ 0.19 & 83.2 $\pm$ 0.39 & 85.2 $\pm$ 0.12 & 77.6 $\pm$ 0.20 & 79.3 $\pm$ 0.30 & 73.8 $\pm$ 0.83 & 79.3 $\pm$ 0.48 \\
% w/ SimPro & 74.2 $\pm$ 0.90 & 80.7 $\pm$ 0.30 & 93.6 $\pm$ 0.08 & 93.8 $\pm$ 0.10 & 83.5 $\pm$ 0.95 & 85.8 $\pm$ 0.48 & 82.6 $\pm$ 0.38 & 84.8 $\pm$ 0.54 & 81.0 $\pm$ 0.27 & 83.0 $\pm$ 0.36 \\
Supervised & 63.2 $\pm$ 0.14 & 66.0 $\pm$ 0.27 & 63.3 $\pm$ 0.28 & 65.8 $\pm$ 0.19 & 63.1 $\pm$ 0.19 & 65.9 $\pm$ 0.51 & 63.5 $\pm$ 0.22 & 65.8 $\pm$ 0.03 & 63.0 $\pm$ 0.18 & 66.4 $\pm$ 0.07 \\
\midrule
EM & 69.1 $\pm$ 1.29 & 73.8 $\pm$ 0.71 & 94.0 $\pm$ 0.08 & 93.2 $\pm$ 0.94 & 76.6 $\pm$ 2.72 & 82.2 $\pm$ 0.24 & 79.5 $\pm$ 0.35 & 81.6 $\pm$ 0.58 & 79.2 $\pm$ 0.50 & 79.8 $\pm$ 0.17 \\
\midrule
SimPro & 74.4 $\pm$ 0.71 & 79.7 $\pm$ 0.45 & 93.3 $\pm$ 0.10 & 93.3 $\pm$ 0.47 & 83.8 $\pm$ 0.80 & 84.1 $\pm$ 0.24 & 78.7 $\pm$ 0.30 & 84.2 $\pm$ 0.26 & 81.2 $\pm$ 0.20 & 82.0 $\pm$ 1.07 \\
% \midrule
SimPro+ & \green 77.8 $\pm$ 1.50 & \green 81.2 $\pm$ 0.39 & \green 93.7 $\pm$ 0.07 & \green 93.7 $\pm$ 0.24 & \red 83.3 $\pm$ 0.38 & \green 84.7 $\pm$ 0.78 & \green 79.2 $\pm$ 0.70 & \green 85.4 $\pm$ 0.66 & \green 81.3 $\pm$ 0.27 & \green 82.5 $\pm$ 0.56 \\
\midrule
BOAT & 80.5 $\pm$ 0.39 & 83.3 $\pm$ 0.27 & 93.9 $\pm$ 0.03 & 94.1 $\pm$ 0.10 & 79.7 $\pm$ 0.25 & 81.1 $\pm$ 0.15 & 79.7 $\pm$ 1.15 & 81.6 $\pm$ 0.09 & 79.4 $\pm$ 0.44 & 80.9 $\pm$ 0.16 \\
% \midrule
BOAT+ & \green 81.6 $\pm$ 0.15 & \green 83.8 $\pm$ 0.04 & \red 93.7 $\pm$ 0.23 & 94.1 $\pm$ 0.17 & \green 80.4 $\pm$ 0.71 & \green 81.7 $\pm$ 0.38 & \green 80.3 $\pm$ 0.28 & \green 83.1 $\pm$ 0.45 & \green 79.7 $\pm$ 0.29 & \green 81.0 $\pm$ 0.36 \\
\bottomrule
\end{tabular}
}
\end{table*}

\begin{table*}[t]
\centering
\caption{Top-1 accuracy (\%) on CIFAR-100-LT ($N_l = 50$, $M_l = 400$) with different class imbalance ratios $\gamma_l$ and $\gamma_u$ under five different unlabeled class distributions. Despite poor estimation in stage 1, our approach does not degrade the accuracy for most of the settings. We use {\green green} to indicate when our plug-in improves and {\red red} when it degrades the base method.}
\label{tab:cifar100-acc}
\resizebox{\textwidth}{!}{
\begin{tabular}{lccccccccccc}
\toprule

& \multicolumn{2}{c}{consistent} & \multicolumn{2}{c}{uniform} & \multicolumn{2}{c}{reversed} & \multicolumn{2}{c}{middle} & \multicolumn{2}{c}{head-tail} \\
\cmidrule(lr){2-3} \cmidrule(lr){4-5} \cmidrule(lr){6-7} \cmidrule(lr){8-9} \cmidrule(lr){10-11}

& $\gamma_l = 20$ & $\gamma_l = 10$ & $\gamma_l = 20$ & $\gamma_l = 10$ & $\gamma_l = 20$ & $\gamma_l = 10$ & $\gamma_l = 20$ & $\gamma_l = 10$ & $\gamma_l = 20$ & $\gamma_l = 10$ \\
& $\gamma_u = 20$ & $\gamma_u = 10$ & $\gamma_u = 1$ & $\gamma_u = 1$ & $\gamma_u = 1/20$ & $\gamma_u = 1/10$ & $\gamma_u = 20$ & $\gamma_u = 10$ & $\gamma_u = 20$ & $\gamma_u = 10$ \\

\midrule
% FixMatch & 40.0 $\pm$ 0.96 & 45.2 $\pm$ 0.55 & 39.6 $\pm$ 1.16 & \\
% CReST+ & 40.1 $\pm$ 1.28 & 44.5 $\pm$ 0.94 & 37.6 $\pm$ 0.88 & \\
% DASO & 43.0 $\pm$ 0.15 & 49.8 $\pm$ 0.24 & 49.4 $\pm$ 0.93 & \\
Supervised & 32.4 $\pm$ 0.40 & 38.4 $\pm$ 0.18 & 32.7 $\pm$ 0.25 & 38.0 $\pm$ 0.22 & 32.5 $\pm$ 0.51 & 38.4 $\pm$ 0.43 & 32.3 $\pm$ 0.08 & 37.9 $\pm$ 0.43 & 32.1 $\pm$ 0.33 & 38.2 $\pm$ 0.38 \\
% \midrule
EM & 42.4 $\pm$ 0.43 & 49.6 $\pm$ 0.30 & 50.9 $\pm$ 0.27 & 58.0 $\pm$ 0.35 & 42.1 $\pm$ 0.16 & 49.8 $\pm$ 0.47 & 42.8 $\pm$ 0.41 & 49.6 $\pm$ 0.36 & 41.5 $\pm$ 1.26 & 49.5 $\pm$ 0.18 \\
\midrule
SimPro & 42.5 $\pm$ 0.58 & 49.6 $\pm$ 0.22 & 51.7 $\pm$ 0.22 & 58.1 $\pm$ 0.53 & 44.9 $\pm$ 0.21 & 51.8 $\pm$ 0.42 & 42.7 $\pm$ 0.06 & 49.8 $\pm$ 0.45 & 43.3 $\pm$ 0.76 & 50.9 $\pm$ 0.19 \\
% \midrule
SimPro+ & \green 42.8 $\pm$ 0.49 & \green 50.1 $\pm$ 0.33 & \red 51.6 $\pm$ 0.63 & \red 57.8 $\pm$ 0.38 & \red 44.7 $\pm$ 0.51 & \red 51.4 $\pm$ 0.88 & \green 43.4 $\pm$ 0.58 & \green 50.4 $\pm$ 0.28 & \green 43.8 $\pm$ 0.50 & \red 50.7 $\pm$ 0.76 \\
\midrule
BOAT & 43.7 $\pm$ 0.16 & 51.4 $\pm$ 0.32 & 55.1 $\pm$ 0.95 & 60.5 $\pm$ 0.15 & 43.1 $\pm$ 0.49 & 52.7 $\pm$ 0.23 & 43.6 $\pm$ 0.19 & 51.4 $\pm$ 0.39 & 43.9 $\pm$ 0.42 & 51.4 $\pm$ 0.14 \\
% \midrule
BOAT+ & \green 44.8 $\pm$ 0.13 & 51.4 $\pm$ 0.51 & \red 53.8 $\pm$ 0.32 & 60.5 $\pm$ 0.69 & \green 43.4 $\pm$ 0.56 & \red 52.4 $\pm$ 0.36 & \green 43.9 $\pm$ 0.59 & \red 50.8 $\pm$ 0.09 & \red 43.6 $\pm$ 0.50 & \green 51.9 $\pm$ 0.49 \\
\bottomrule
\end{tabular}
}
\end{table*}

We perform experiments for each stage of our algorithm. In the first stage, we compare among various methods to estimate the unlabeled class distribution $P(Y|A=0)$, showing that SimPro + DR performs well. In the second stage, we freeze the unlabeled class distribution, using our best estimator  SimPro + DR, and plug it into 2 SOTA semi-supervised learning algorithms, SimPro and BOAT~\cite{boat}. We show that this simple procedure improves the existing methods, and is even capable of improving the original SimPro when used for both stages.


% \textbf{Datasets} We adopt 4 standard benchmarks used frequently in other semi-supervised learning work: CIFAR-10, CIFAR-100~\cite{cifar}, STL-10~\cite{stl10} and Imagenet-127~\cite{cossl}. To match our RTSSL setting, we create long-tailed labeled and unlabeled sets from CIFAR-10 and CIFAR-100. Specifically, we use $\gamma_l$ and $n_1$ to denote the imbalance ratio and the head class's number of samples of the labeled data, the remaining class's size is computed as $n_c = n_1 \times \gamma_l^{-\frac{c-1}{C-1}}$ and likewise, $\gamma_u$ and $m_1$ of the unlabeled data. For CIFAR-10, we fix $n_1=500$ and $m_1=4000$. We test 2 different configurations $\gamma_l=\gamma_c=150$ and $\gamma_l=\gamma_c=100$. We further permute classes the unlabeled sets in 5 ways: consistent, uniform, reversed, middle and headtail, similar to \cite{simpro} and visualized in figure~\ref{fig:distribution}, which results in 10 different datasets in total. Similarly for CIFAR-100, we fix $n_1=500$ and $m_1=4000$, use 2 configurations $\gamma_l=\gamma_c=20$ and $\gamma_l=\gamma_c=10$, and permute the classes in 5 ways, resulting in 10 datasets as well. For STL-10, the unlabeled set has no ground truth labels, therefore we use all samples in the head class and set the imbalance ratio $\gamma_l$ to $10$ or $20$. Imagenet-127 is a naturally long-tailed dataset with 127 classes. We train on 32x32 and 64x64 image resolutions following ~\cite{cossl}.


\textbf{Datasets} We evaluate our method on four standard semi-supervised learning benchmarks: CIFAR-10, CIFAR-100~\cite{cifar}, STL-10~\cite{stl10}, and Imagenet-127~\cite{cossl}. To simulate RTSSL, we construct long-tailed labeled and unlabeled sets for CIFAR-10 and CIFAR-100. The labeled data follows an imbalance ratio $\gamma_l$ with head class size $n_1$, while the remaining class sizes are computed as $n_c = n_1 \times \gamma_l^{-\frac{c-1}{C-1}}$. The unlabeled data follows a similar setup with $\gamma_u$ and $m_1$.  

For CIFAR-10, we set $n_1 = 500$, $m_1 = 4000$, and test two configurations: $\gamma_l = \gamma_u = 150$ and $\gamma_l = \gamma_u = 100$. We generate 10 datasets by permuting the unlabeled class distributions in five ways: \textit{consistent, uniform, reversed, middle}, and \textit{head-tail}, as in~\cite{simpro}. CIFAR-100 follows the same setup with $n_1 = 50$, $m_1 = 400$, and $\gamma_l, \gamma_u$ values of 20 and 10.  

For STL-10, where unlabeled data lacks ground-truth labels, we use all head-class samples and set $\gamma_l$ to 10 or 20. Imagenet-127 is naturally long-tailed with 127 classes, and we train on 32$\times$32 and 64$\times$64 resolutions as in~\cite{cossl}.


\paragraph{Training.} We follow the implementation and hyperparameter settings of \cite{simpro}. We defer these details in \cref{subsec:training-setting}. One important exception is that for Imagenet-127, we use the smaller Wide ResNet-28-2 in stage 1 and the larger ResNet-50 for stage 2, to demonstrate that a smaller model is sufficient for distribution estimation.


\begin{table}[t]
\small
\centering
\caption{Top-1 Accuracy (\%) on STL-10. Our two-stage algorithms improves both SimPro and BOAT for both settings.}
\label{tab:stl10-acc}
% \resizebox{\linewidth}{!}{
\begin{tabular}{lcc}
\toprule
Method & $\gamma_l=10$ & $\gamma_l=20$ \\ \hline
Supervised & 73.9 $\pm$ 0.57 & 70.4 $\pm$ 0.95 \\
\midrule
MLE & 67.6 $\pm$ 0.57 & 58.9 $\pm$ 4.05 \\
\midrule
EM & 84.9 $\pm$ 0.14 & 83.6 $\pm$ 0.25 \\
\midrule
SimPro & 82.4 $\pm$ 1.57 & 80.5 $\pm$ 0.96 \\
SimPro+ & \green 83.9 $\pm$ 0.76 & \green 82.7 $\pm$ 0.86 \\
\midrule
BOAT & 83.8 $\pm$ 0.20 & 82.0 $\pm$ 0.34 \\
BOAT+ & \green 84.1 $\pm$ 0.38 & \green 82.4 $\pm$ 0.10 \\
\bottomrule
\end{tabular}
\end{table}















\begin{table}[t]
% \setlength{\tabcolsep}{3.5pt}
\small
\centering
\caption{Top-1 Accuracy (\%) on Imagenet-127. Our two-stage approach improves both SimPro and BOAT for both resolutions.}
\label{tab:imagenet-127-acc}
% \resizebox{\linewidth}{!}{
\begin{tabular}{lcc}
\toprule
Method & $32 \times 32$ & $64 \times 64$ \\ \hline
SimPro & 54.8 & 63.7 \\
SimPro+ & \green 55.1 & \green 64.2 \\
\midrule
BOAT & 51.6 & 58.7 \\
BOAT+ & \green 52.0 & \green 59.2 \\

\bottomrule
\end{tabular}
% }
\end{table}


\begin{table}[t]
% \setlength{\tabcolsep}{3.5pt}
\small\centering
\caption{Total Variation Distance on Imagenet-127}
\label{tab:imagenet-127-tv}
% \resizebox{\linewidth}{!}{
\begin{tabular}{cccc}
\toprule
Method & Estimator & $32 \times 32$ & $64 \times 64$ \\ \hline
MLE & IPW  & 0.103 $\pm$ 0.034 & 0.051 $\pm$ 0.000 \\
MLE & OR  & 0.153 $\pm$ 0.052 & 0.041 $\pm$ 0.000 \\
MLE & DR  & \green 0.100 $\pm$ 0.029 & \green 0.075 $\pm$ 0.003 \\
\midrule
EM & IPW  & 0.141 $\pm$ 0.006 & 0.163 $\pm$ 0.010 \\
EM & OR  & 0.205 $\pm$ 0.006 & 0.236 $\pm$ 0.011 \\
EM & DR  & \green 0.024 $\pm$ 0.001 & \green 0.042 $\pm$ 0.004 \\
\midrule
SimPro & IPW  & 0.041 $\pm$ 0.012 & 0.224 $\pm$ 0.040 \\
SimPro & OR  & 0.036 $\pm$ 0.014 & 0.291 $\pm$ 0.079 \\
SimPro & DR  & \green 0.017 $\pm$ 0.000 & \green 0.037 $\pm$ 0.004 \\
\bottomrule
\end{tabular}
% }
\end{table}

\subsection{Better results on label distribution} 
\label{subsec:label}
We have mentioned various ways throughout the papers to estimate the unlabeled class distribution. In what follows, method consists of a model, which is how the learning is done, and an estimator, which is how the final distribution is estimated using parameters learned from the model.

%\begin{enumerate}
%\item 
\noindent
\textbf{Supervised}. The model is trained on the labeled set only and used to estimate the unlabeled class distribution \cite{unifiedlabelshift}. 2 successful estimators for this setting are \textbf{RLLS} \cite{rlls} and \textbf{MLLS} \cite{mlls}. 

%\item 
\noindent\textbf{MLE}. The model is trained by directly maximizing the likelihood \cref{eq:likelihood}. We also use the decomposition $P(Y|X)$ and $P(A|Y)$, and write the unlabeled term as $P(A=0, X) = \sum_{c} P(Y=c|X) P(A=0|Y=c)$, which enables gradient descent training on these parameters. This is also the MLE method to estimate $P(A|Y)$ in \cite{arelabelsinformative}.

%\item 
\noindent\textbf{EM}. We further test the EM algorithm in \cref{subsec:em}. In particular we also use strong and weak augmentations similar to FixMatch, but not the pseudo labeling operator. Confidence thresholding removes the soft predictions of the non-dominant classes, which may be better to keep since our target of the first stage is the global class statistics. We also try 3 estimators on this model.

%\item 
\noindent\textbf{SimPro} \cite{simpro} can be seen as our previous EM but also with FixMatch's confidence thresholding and logit adjustment loss in \cref{subsec:simpro}. Confidence thresholding is a powerful regularization technique that encodes the assumption that classes are well separated \cite{entropyminimization}, but can introduce bias to the estimation, which justifies the use of DR.
%\end{enumerate}

% For semi-supervised methods MLE, EM and SimPro, as we now have additional information on the missingness mechanism, we can use 3 estimators OR, IPW and DR presented in \cref{subsec:2-stage}


Results on \cref{tab:cifar10-tv} presents the performance of various models and estimators on CIFAR-10. We can see that SimPro + DR performs best. In contrast, SimPro + OR, SimPro's original way of estimating $P(Y|A=0)$, and SimPro + IPW tend to underperform EM on the consistent and uniform datasets. The consistent setting is worth noting, since it arises when data is sampled uniformly at random for labeling,  representative of a large number of real world situations. EM is competitive to SimPro as well even without pseudo labeling, but overall we found this regularization to offer significant gains in the reversed, middle and head-tail settings. Finally, Supervised with either MLLS or RLLS estimators performs much worse than the semi-supervise methods.

\cref{tab:imagenet-127-tv} aligns with the observations  made in \cref{tab:cifar10-tv}. In particular, SimPro + DR is the best method for class distribution estimation of the much larger Imagenet-127. We also found that a small neural network and a small image resolution is sufficient for the distribution estimation of the much larger dataset Imagenet-127. This matches our intuition that the finite-dimensional parameter is easier to learn.

\cref{tab:cifar100-tv} shows that most methods understandably struggle to estimate the class distributions in CIFAR-100. This is because there are few samples in each class, the head class has 10 times less samples while the number of classes multiplies 10 times compared to CIFAR-10. We see here that SimPro + DR is not the best method, but the relative gap between estimators are small.

% Among the models, the supervised baseline do not perform well even in the consistent setting, showing that when unlabeled data is available during training, learning from them can be valuable for class distribution estimation, especially in the cases with little labeled data like ours. Both the MLE and supervised models perform badly on the reversed, middle and head-tail settings

% Among the estimators, we see that DR boosts the performance of SimPro and EM in CIFAR-10, and of all semi-supervised models in Imagenet-127. It does not improve MLE on CIFAR-10, and it does not improve on CIFAR-100. However, for most of the time, the decrease is not much. In constrast, IPW estimators can be significantly worse, for example in the reversed setting of CIFAR-10, where the distance is $0.254$ for $\gamma_l=150$ and $0.233$ for $\gamma_l=100$, compared to OR's 0.040 and 0.059. 

% Both the MLE and supervised models perform badly on the reversed, middle and head-tail settings. EM does a decent job, though not as well as SimPro, on all 5 distribution settings of CIFAR-10. However, on Imagenet-127, EM without DR performs worse than MLE.

% We note that the performance on DR is similar to OR in these cases, showing that DR has a double robustness property. While IPW only relies on the finite-dimensional $P(A|Y)$, which intuitively is easy to estimate, we found that the inverse probability weight can nevertheless be unstable when some probabilities are small, and this is where DR shows its strength by combining both IPW and OR.



\subsection{Two-stage algorithm improves accuracy}

In the second stage of our algorithm, we freeze our estimation and plug it in SimPro and BOAT. We denote SimPro+ and BOAT+ for algorithms that use our first stage estimate.



\cref{tab:cifar10-acc} shows that for CIFAR-10 SimPro+ and BOAT+ improve over their original versions across most settings, leading to large improvements in both the consistent and middle class distribution settings. In particular, our two-stage approach improves SimPro in 9 / 10 settings and BOAT in 8 / 10 settings.
We also observe consistent improvements ove both base algorithms, SimPro and BOAT, for several other datasets. \cref{tab:stl10-acc} demonstrates improvements for 2 / 2 class imbalance ratios in STL-10 and \cref{tab:imagenet-127-acc} for 2 / 2  different resolutions of ImageNet-127. 


We also evaluate on CIFAR-100 for multiple unlabeled  class distribution settings and with mediocre class label distribution estimates in stage 1, demonstrate no degradation in accuracy in stage 2. As shown in \cref{tab:cifar100-acc}, the two stage algorithm with a mediocre stage 1 estimation leads to parity with the baseline. Stage 2 provides small improvements in 5 / 10 settings for SimPro and in 4 / 10 (with 2 ties) for BOAT.


\subsection{Ablation Study: Alternative implementations.}
\label{subsec:ablation-1}
In this section, we ablate on our 2-stage choice. Specifically, we consider 2 alternative implementations:
\paragraph{\textbf{Doubly-robust risk}}  
This approach is \cite{arelabelsinformative, onnonrandommissinglabels}, as discussed in \cref{sec:background}. we consider the doubly-robust risk as our training loss. We use the missingness mechanism estimation from stage-1 of SimPro+ for fair comparison. \cref{eq:dr-risk} is used for training in which the pseudo-labeling operators can be applied straightforwardly. More detail in \cref{subsec:dr-risk}
\paragraph{\textbf{Batch-update doubly-robust $P(Y|A)$}} Different from SimPro+, here we remove the first stage and instead update our doubly robust estimation of the unlabeled class distribution using a moving average of the batch statistics.

\cref{tab:cifar10-ablation-1} shows that the batch-update version of SimPro+ is significantly worse on the consistent and uniform settings, while the doubly-robust risk is worst overall, especially in the reversed setting where $P(A|Y)$ is very small for the labeled tail classes, causing instability issues during training. In conclusion, our 2-stage approach is the best.


\begin{table}[t]
\small
\centering
\caption{Top-1 Accuracy (\%) on CIFAR-10. We compare our 2-stage SimPro+ with 1) an 1-stage alternative that updates and uses the doubly-robust estimation on-the-fly and 2) SimPro with doubly-robust risk. We use $\gamma_l=150$. {\green green} color indicates that our method performs best.}
\label{tab:cifar10-ablation-1}
\resizebox{\linewidth}{!}{
\begin{tabular}{lccccc}
\toprule
Method & consistent & uniform & reversed & middle & headtail\\ \hline
SimPro+ & \green 77.8 & \green 93.7 & \green 83.3 & \green 79.2 & \green 81.3 \\
batch-update & 71.9 & 91.4 & 82.6 & 78.6 & 81.2 \\
DR-risk & 72.1 & 89.8 & 67.1 & 75.6 & 79.5 \\
\bottomrule
\end{tabular}
}
\end{table}
\section{Concluding Remarks}
In this paper, we proposed a novel approach utilizing multimodal LLMs to generate gesture-aware speech recognition transcripts for patients with language disorders. Our framework integrates verbal speech and iconic gestures, enabling the generation of enriched transcripts that capture the latent meaning conveyed through both modalities. Through extensive experimentation, we demonstrated that the proposed method effectively contextualizes incomplete or disfluent speech by incorporating gesture information, leading to more accurate and meaningful representations of the speaker's intent. These findings highlight the potential of our approach to significantly contribute to the field of speech and language therapy, offering innovative tools that can enhance the quality of life for individuals with language disorders by facilitating better communication and assessment methods.

\subsection{Ethical Statement} 
Our dataset was obtained from AphasiaBank with the approval of the Institutional Review Board (IRB) and adheres to the data sharing guidelines set by TalkBank\footnote{https://talkbank.org/share/ethics.html}. This includes complying with the Ground Rules for all TalkBank databases, which are based on the American Psychological Association Code of Ethics~\cite{american2002ethical}.

\subsection{Limitation \& Future Work} 
%This study represents a preliminary investigation into using multimodal LLMs to generate gesture-aware speech recognition transcripts. 
While the results are promising, we recognize several limitations and outline our plans to extend this work further.

One primary limitation is the absence of a definitive ground truth for quantitative evaluation. Since our model generates transcripts by synthesizing speech and gesture data from scratch, traditional benchmarks, such as comparisons with standard speech recognition outputs, are insufficient. Moreover, existing original transcripts lack gesture annotations, making direct comparisons challenging. In future work, we aim to address this gap by collaborating with certified pathologists to conduct qualitative assessments, such as A-B preference tests, to evaluate the effectiveness of gesture-enriched transcripts in accurately conveying the speaker's intentions.

To support quantitative evaluations, we plan to develop novel metrics that assess transcript quality, including grammar accuracy, semantic consistency, and the integration of multimodal information. Such metrics will provide a more objective basis for assessing our model's performance and facilitate comparisons with other multimodal and unimodal approaches.

Another limitation of this study is its focus on structured gestures from a specific task, the Peanut Butter Sandwich Task. While this task offers a controlled context for testing our approach, it does not encompass the diversity of gestures and communication patterns seen in everyday scenarios. As part of our future work, we plan to expand the scope of our model to include tasks such as the Cinderella Story Recall Task~\cite{bird1996cinderella}, which involves unstructured and complex narrative gestures. This expansion will allow us to evaluate the adaptability and robustness of our model in handling varied linguistic and gestural contexts.

In summary, while this study establishes a strong foundation for gesture-aware speech recognition, we aim to refine and extend our methods through collaborative qualitative evaluations, the development of robust quantitative metrics, and broader task applications. These efforts will ensure that our approach continues to evolve, ultimately contributing to more effective communication tools and interventions for individuals with language disorders.





\bibliography{ref}
\bibliographystyle{icml2025}


%%%%%%%%%%%%%%%%%%%%%%%%%%%%%%%%%%%%%%%%%%%%%%%%%%%%%%%%%%%%%%%%%%%%%%%%%%%%%%%
%%%%%%%%%%%%%%%%%%%%%%%%%%%%%%%%%%%%%%%%%%%%%%%%%%%%%%%%%%%%%%%%%%%%%%%%%%%%%%%
% APPENDIX
%%%%%%%%%%%%%%%%%%%%%%%%%%%%%%%%%%%%%%%%%%%%%%%%%%%%%%%%%%%%%%%%%%%%%%%%%%%%%%%
%%%%%%%%%%%%%%%%%%%%%%%%%%%%%%%%%%%%%%%%%%%%%%%%%%%%%%%%%%%%%%%%%%%%%%%%%%%%%%%
\newpage
\appendix
\onecolumn

\appendix

\section{Appendix: Prompt}
\label{sec:appendix}
``Here is a sketch of an image. 
$\{input\_color\_mask\}$, while the rest of the white space is the background. 
I need you to infer details of the image based on the given sketch.
The details should include the possible background likely to be present with the $\{input\_color\_mask\}$, the attribute of each object (like wearing, texture, color etc.), the state (including action, posture, etc.) of each object, the direction of each object and the relationships between objects.

You should first analyze the mask carefully, considering the size, location, and relative position of each object mask. Ensure that specific actions are analyzed based on the mask, and infer each aspect with a reasoning process before providing the final output.
The final output format should be: $\{format\_example\}$, and you should refer to the example: $\{few\_shot\}$. You are going to complete the "" in each item, you need to complete them in multiple short phrases based on your above reasoning.

The state and relationship should be as detailed as possible while ensuring they align with the mask, formatted as: objectA action/spatial relation objectB, with both objectA and objectB included.
You should properly refer to some examples of attributes of object $\{attributes\}$ and relationships $\{relationships\}$.
Do not include words like `or', `possibly' in your final output, there should no ambiguity in your output.
Make sure all aspects of given mask is filled.''
%%%%%%%%%%%%%%%%%%%%%%%%%%%%%%%%%%%%%%%%%%%%%%%%%%%%%%%%%%%%%%%%%%%%%%%%%%%%%%%
%%%%%%%%%%%%%%%%%%%%%%%%%%%%%%%%%%%%%%%%%%%%%%%%%%%%%%%%%%%%%%%%%%%%%%%%%%%%%%%


\end{document}
\end{document}