\newpage
%======================
\section{Prompting Templates}\label{sect:app:prompting}
\subsection{Providing Contextual Information}
\begin{table}[H]
    \centering\footnotesize
    \caption{A summary of the prompting templates used in this work.
    }
    \label{tab:context:appendix}
\begin{tabularx}{\linewidth}{l|X|X}
  \toprule
  \bf Abbreviation & \bf Description &\bf Prompting Template\\ \midrule
\verb|continue|&Simple continuation based on a short prefix (no prompting). & None\\
\midrule
\verb|cot|&Ask the model to generate an outline first before generating the article. A short prefix is provided. & \footnotesize
\ttfamily{Extend the following text. First write an outline. Then insert **Extended Text:** After that write the text using a minimum of 15 paragraphs containing a minimum of 1000 words, in plain text only without titles or headings. The text you need to extend is: <prefix>}\\
\midrule
\verb|short keywords|&A few, unordered list of keywords. & \footnotesize
\ttfamily{Using these keywords: <keywords>, write an article, in a minimum of 15 paragraphs containing a minimum of 1000 words, in plain text only without titles or headings.}\\
\midrule
\verb|keywords|&Many unordered keywords. &\footnotesize
\ttfamily{Using these keywords: <keywords>, write an article, in a minimum of 15 paragraphs containing a minimum of 1000 words, in plain text only without titles or headings.}\\
\midrule
\verb|summary|&A summary of the entire article. &\footnotesize
\ttfamily{Write about the following in a minimum of 15 paragraphs containing a minimum of 1000 words, in plain text only without titles or headings: <summary>.}\\
\midrule
\verb|summary+keywords|&Both a summary and an unordered list of many keywords. &\footnotesize
\ttfamily{Write about the following in a minimum of 15 paragraphs containing a minimum of 1000 words, in plain text only without titles or headings: <summary>. Using these keywords: <keywords>.}\\
\midrule
\verb|excerpt|&An ordered list of long excerpts from the original article. &\footnotesize
\ttfamily{Write about the following in a minimum of 15 paragraphs containing a minimum of 1000 words, in plain text only without titles or headings: <ordered excerpts>.}\\
  \bottomrule
  \end{tabularx}
\end{table}

\subsection{Generating Contextual Information}
To generate the keywords and summary, we use the instruction-tuned Gemini 1.0 Pro model with the following prompts: 
\begin{itemize}
    \item \textbf{Keywords}: 
    \begin{lstlisting}
Generate keywords for the following text, but only respond with the keywords in plain text separated by commas. The text is: 
<article>
    \end{lstlisting}
    \item \textbf{Summary}: 
    \begin{lstlisting}
Summarize the following text in few sentences, but only respond with the text summary in plain text. The text is: 
<article>
    \end{lstlisting}
\end{itemize}


\subsection{Quality of Texts}
As discussed in Section~\ref{sect:results}/Q6, we use the instruction-tuned Gemini 1.0 Pro model to estimate the quality of texts and relate those to fractal parameters in Figure~\ref{fig:quality}. The prompt we use to estimate the quality of an article is provided below:

\begin{lstlisting}
First, explain briefly what is good and what is bad about the quality of the following document. Then, rate it on a scale from '1' to '5', where '1' is poorest and '5' is best. The last character in your response must be a single digit that corresponds to your rating. The document is:\n <article>
\end{lstlisting}

Examples of how the model evaluates the quality of articles is provided in Appendix~\ref{sect:app:rho_doc_sample}.


%=====================
\newpage
\section{Prompts as Do-Queries Example}\label{sect:app:doquery_example}
To illustrate the differences between Equations~\ref{eq:do_query} and~\ref{eq:cond}, suppose hypothetically that all of the world's conversations are about two topics only: fairy tales or the weather. These two will be our possible contexts. Let us also assume that 75\% of the time, people talk about the weather. 

In addition, let us suppose that only three prefixes are always used with equal probability as shown in the table below:
\begin{table}[H]
    \centering
    \begin{tabular}{l|c|c}
         \bf Prefix & $p(\mathrm{weather} | \mathrm{prefix})$ & $p(\mathrm{fairy tale} | \mathrm{prefix})$  \\ \hline
         ``It is a sunny day'' &  50\% & 50\% \\
         ``It is a lovely sunny day & 85\% & 15\%\\
         ``It is raining'' & 90\% & 10\%
    \end{tabular}
\end{table}

If a language model is trained on this data and we prompt it with the phrase ``\emph{It is a sunny day},'' what should the model predict next? There are two ways to interpret this question. One option is that the model should give a prediction based on \emph{actual} historical data, using the conditional distribution $p\left(\mathrm{suffix}\;|\;\mathrm{prefix}\right)$. In this case, there is 50\% probability that the context is about the weather, so the model will continue the prefix by describing the weather 50\% of time and continue the prefix by describing a fairy tale in 50\% of the time.

Another interpretation, however, is that by prompting the model, we are asking the model to act as if there was an \emph{intervention} that forced all conversations to start with the prefix ``\emph{It is a sunny day}'' and  predict how the historical distribution would have looked like. In that case, Equation~\ref{eq:do_query} implies that the model should sample a context first from its marginal distribution, which would be ``weather'' with probability 75\% and fairy tales with probability 25\%. Once the context is selected, it can use the historical data to predict the continuation of the prefix ``\emph{It is a sunny day}'' after conditioning on that particular context. The intuition here from the causal graph in Figure~\ref{fig:main_a} is that it is the context that dictates or causes the conversations, not the other way around, so the distribution of underlying contexts would remain the same.

%======================
\newpage
\section{Data Card}\label{sect:app:data_card}
\begin{table}[H]
    \centering\footnotesize
    \caption{Generated And Grounded Language Examples (GAGLE) data card.
    }
    \label{tab:context:appendix}
\begin{tabularx}{\linewidth}{l|X}
  \toprule
  \bf Description &  GAGLE comprises of LLM-generated articles. All articles are generated using the public checkpoints of the open-source models Mistral-7B~\citep{jiang2023mistral7b} and Gemma-2B~\citep{gemmateam2024gemmaopenmodelsbased}. 
  
  The seed for all articles are sourced from five academic datasets: (1) WIKIPEDIA articles~\citep{wikidump}, (2) BIGPATENT, consisting of over one million records of U.S. patent documents~\citep{sharma2019bigpatent}, (3) NEWSROOM, containing over one million news articles~\citep{Grusky_2018}, (4) SCIENTIFIC, a collection of research papers obtained from ArXiv and PubMed OpenAccess repositories~\citep{Cohan_2018}, and (5) BILLSUM, containing US Congressional and California state bills~\citep{kornilova2019billsum}. So, all articles are of encyclopedic, news, legal, scientific or patents nature.  The dataset contains, at most, 1,000 articles from each domain.
  
  The articles are generated via the following prompting strategies:
  \begin{enumerate}
      \item \verb|continue (pt)|: Simple continuation based on a short prefix (no prompting) using pretrained model.
      \item \verb|continue (it)|: Simple continuation based on a short prefix (no prompting) using instruction-tuned model.
\item \verb|cot|: Ask the model to generate an summary first before generating the article. A short prefix is provided.
\item \verb|short keywords|: A few, unordered list of keywords.
\item \verb|keywords|: Many unordered keywords.
\item \verb|summary|: A summary of the entire article.
\item \verb|summary + keywords|: Both a summary and an unordered list of many keywords.
\item \verb|excerpt|: An ordered list of long excerpts from the original article.
  \end{enumerate}
With the exception of \verb|continue (pt)|, all other prompts are used in instruction-tuned models. 
 \\ \midrule
  \bf Primary Data Modality & Text.\\ \midrule
  \bf Data Fields & \begin{enumerate}
      \item \textbf{ID}: a unique ID identifying the original ground-truth article.
      \item \textbf{Model}: one of Mistral-7B and Gemma-2B.
      \item \textbf{Domain}: one of WIKIPEDIA, BIGPATENT, NEWSROOM, SCIENTIFIC, and BILLSUM.
      \item \textbf{Prompt}: one of the specified prompting methods.
      \item \textbf{Temperature}: numeric. Either 0.0 (greedy decoding), 0.5, or 1.0 (pretraining temperature). 
      \item \textbf{Prefix}: Prefix used to generate the article.
      \item\textbf{Quality}: description of the quality of the article along with a rating from 1 (poorest) to 5 (best).
      \item\textbf{Text}: Actual LLM-generated text.
      \item\textbf{Log-Perplexity Scores}: The scores generated by one of Mistral-7B or Gemma-2B pretrained models.
  \end{enumerate} \\ \midrule
  \bf Intended Use Case & Facilitate research in domains related to detecting and analyzing LLM-generated tests.  \\ \midrule
  \bf Access Type& Unrestricted. \\ \midrule 
  \bf CC-by-4.0 \\ \midrule 
  \bf Sensitive Human Attributes & N/A\\ 
  \bottomrule
  \end{tabularx}
\end{table}

\newpage
\section{Contextual Information in the Prompt: Full Figures}\label{sect:app:info_density_full_figures}
\begin{figure}[H]
    \centering
    \includegraphics[width=0.85\columnwidth]{figures/3/point3_1_S_gemini_m_0.pdf}
    \includegraphics[width=0.85\columnwidth]{figures/3/point3_1_S_gemini_m_0.5.pdf}
    \includegraphics[width=0.85\columnwidth]{figures/3/point3_1_S_gemini_m_1.0.pdf}
    \caption{H\"older Exponent. Generating model = Gemini 1.0 Pro}
    \label{fig:append_c_s_gemini_m}
\end{figure}

\begin{figure}[H]
    \centering
    \includegraphics[width=0.85\columnwidth]{figures/3/point3_1_S_mistral_7b_0.pdf}
    \includegraphics[width=0.85\columnwidth]{figures/3/point3_1_S_mistral_7b_0.5.pdf}
    \includegraphics[width=0.85\columnwidth]{figures/3/point3_1_S_mistral_7b_1.0.pdf}
    \caption{H\"older Exponent. Generating model = Mistral-7B}
    \label{fig:append_c_s_mistral}
\end{figure}
\begin{figure}[H]
    \centering
    \includegraphics[width=0.85\columnwidth]{figures/3/point3_1_S_gemma_2b_0.pdf}
    \includegraphics[width=0.85\columnwidth]{figures/3/point3_1_S_gemma_2b_0.5.pdf}
    \includegraphics[width=0.85\columnwidth]{figures/3/point3_1_S_gemma_2b_1.0.pdf}
    \caption{H\"older Exponent. Generating model = Gemma-2B}
    \label{fig:append_c_s_gemma}
\end{figure}

\begin{figure}[H]
    \centering
    \includegraphics[width=0.85\columnwidth]{figures/3/point3_1_H_gemini_m_0.pdf}
    \includegraphics[width=0.85\columnwidth]{figures/3/point3_1_H_gemini_m_0.5.pdf}
    \includegraphics[width=0.85\columnwidth]{figures/3/point3_1_H_gemini_m_1.0.pdf}
    \caption{Hurst Exponent. Generating model = Gemini 1.0 Pro}
    \label{fig:append_c_h_gemini_m}
\end{figure}

\begin{figure}[H]
    \centering
    \includegraphics[width=0.85\columnwidth]{figures/3/point3_1_H_mistral_7b_0.pdf}
    \includegraphics[width=0.85\columnwidth]{figures/3/point3_1_H_mistral_7b_0.5.pdf}
    \includegraphics[width=0.85\columnwidth]{figures/3/point3_1_H_mistral_7b_1.0.pdf}
    \caption{Hurst Exponent. Generating model = Mistral-7B}
    \label{fig:append_c_h_mistral}
\end{figure}
\begin{figure}[H]
    \centering
    \includegraphics[width=0.85\columnwidth]{figures/3/point3_1_H_gemma_2b_0.pdf}
    \includegraphics[width=0.85\columnwidth]{figures/3/point3_1_H_gemma_2b_0.5.pdf}
    \includegraphics[width=0.85\columnwidth]{figures/3/point3_1_H_gemma_2b_1.0.pdf}
    \caption{Hurst Exponent. Generating model = Gemma-2B}
    \label{fig:append_c_h_gemma}
\end{figure}

\newpage

\section{Example Documents of Generating Processes with Large and Small Values of $\HHH$ }\label{sect:app:rho_doc_sample}

\subsection{Large Values of $\HHH$}
\begin{itemize}
    \item\textbf{Data domain}: Newsroom.
    \item\textbf{Generating}: 
        \begin{itemize}
            \item\emph{Model Name:} Mistral-7B (pretrained)
            \item\emph{Decoding Temperature}: 0.5
            \item\emph{Prompting Method}: \verb|continue|
        \end{itemize}
    \item\textbf{Fractal Parameters}: 
        \begin{itemize}
            \item\emph{Scoring Model Name:} Gemini 1.0 Pro.
            \item\emph{H\"older Exponent}: $\SSS=0.630$
            \item\emph{Hurst Exponent}:  $\HHH=0.808$
        \end{itemize}
\end{itemize}
\subsubsection{Document I}
{\bf\ttfamily Ground-truth Record ID:}
\begin{lstlisting}[basicstyle=\footnotesize\ttfamily]
"newsroom-00001-00064-6"
\end{lstlisting}

{\bf\ttfamily Prefix: }

{\ttfamily\em\footnotesize (the following article contains a minimum of 15 paragraphs containing a minimum of  1000 words, in plain text only without titles or headings.)

A year ago, the 25-year-old Serb swept into Paris after amassing one of the most spectacular five months in men's tennis history, winning every tournament he'd entered. Now, far from a perfect season, the tournament's top seed arrives at the French Open,
}

{\bf\ttfamily Generated: }

\begin{lstlisting}
the last major of the year, having lost his last two matches, his ranking and his No. 1 ranking.

In the past year, Djokovic has been beaten by a 17-year-old American, lost the last seven sets he's played, and been beaten by a player ranked 111th in the world.

The Serb, who won the Australian Open and Wimbledon in the first two months of the year, was beaten by 17-year-old American Jared Donaldson in the final of the Stuttgart Open last month.

He then lost to Dominic Thiem in the Rome Masters final, before being beaten by Gael Monfils in the quarterfinals of the Madrid Open.

Djokovic, who has been ranked No. 1 for the last seven years, is now ranked 11th in the world.

The Serb will be hoping to bounce back in Paris, where he has won three of the last four French Open titles.

He will face 23-year-old Russian Andrey Rublev in the first round of the tournament.

Rublev is ranked 22nd in the world, but has been on a roll in the last few months, winning the St. Petersburg Open and the ATP 500 event in Rome.

The Serb has been in fine form in the last few months, winning the Australian Open and Wimbledon in the first two months of the year.

Djokovic has been in fine form in the last few months, winning the Australian Open and Wimbledon in the first two months of the year.

The Serb has been in fine form in the last few months, winning the Australian Open and Wimbledon in the first two months of the year.

The Serb has been in fine form in the last few months, winning the Australian Open and Wimbledon in the first two months of the year.

The Serb has been in fine form in the last few months, winning the Australian Open and Wimbledon in the first two months of the year.

The Serb has been in fine form in the last few months, winning the Australian Open and Wimbledon in the first two months of the year.

The Serb has been in fine form in the last few months, winning the Australian Open and Wimbledon in the first two months of the year.

The Serb has been in fine form in the last few months, winning the Australian Open and Wimbledon in the first two months of the year.

The Serb has been in fine form in the last few months, winning the Australian Open and Wimbledon in the first two months of the year.

The Serb has been in fine form in the last few months, winning the Australian Open and Wimbledon in the first two months of the year.

The Serb has been in fine form in the last few months, winning the Australian Open and Wimbledon in the first two months of the year.

The Serb has been in fine form in the last few months, winning the Australian Open and Wimbledon in the first two months of the year.

The Serb has been in fine form in the last few months, winning the Australian Open and Wimbledon in the first two months of the year.

The Serb has been in fine form in the last few months, winning the Australian Open and Wimbledon in the first two months of the year.

The Serb has been in fine form in the last few months, winning the Australian Open and Wimbledon in the first two months of the year.

The Serb has been in fine form in the last few months, winning the Australian Open and Wimbledon in the first two months of the year.

The Serb has been in fine form in the last few months, winning the Australian Open and Wimbledon in the first two months of the year.

The Serb has been in fine form in the last few months, winning the Australian Open and Wimbledon in the first two months of the year.

The Serb has been in fine form in the last few months, winning the Australian Open and Wimbledon in the first two months of the year.

The Serb has been in fine form in the last few months, winning the Australian Open and Wimbledon in the first two months of the year.

The Serb has been in fine form in the last few months, winning the Australian Open and Wimbledon in the first two months of the year.


The Serb has been in fine form in the last
\end{lstlisting}

{\bf\ttfamily Gemini 1.0 Pro Evaluation of Quality of Generated Text:}
\begin{lstlisting}
**Good:** The document provides a detailed account of Novak Djokovic's recent performances and rankings. It includes specific instances of his losses, such as being beaten by a 17-year-old American and a player ranked 111th in the world. The document also highlights Djokovic's past achievements, such as winning the Australian Open and Wimbledon earlier in the year.

**Bad:** The document repeats the sentence "The Serb has been in fine form in the last few months, winning the Australian Open and Wimbledon in the first two months of the year" 17 times, which makes the content repetitive and unnecessarily long.

Rating: 3/5
\end{lstlisting}

{\bf\ttfamily Ground-truth: }
\begin{lstlisting}
A year ago, the 25-year-old Serb swept into Paris after amassing one of the most spectacular five months in men's tennis history, winning every tournament he'd entered.

Now, far from a perfect season, the tournament's top seed arrives at the French Open, which starts Sunday in Paris, with fissures in his impenetrable facade.

Djokovic has not won a clay title in 2012 and dropped his last two finals on dirt to defending champ Rafael Nadal- after beating him the previous seven times.

He has also shown flashes of anger and signs of frustration, emotions that, in the past, have undermined his performance.

"I am not comparing last year and this one," Djokovic said Monday following his 7-5, 6-3 loss to Nadal in the rain-delayed final in Rome. "I feel good on the court and I need to make a few adjustment before Paris, but I'll be in top form."

If Nadal's resurgent spring and 45-1 record in Paris make him the favorite, Djokovic still will be dealing with heightened expectations.

A Paris title would place him in rarefied company - one of just seven men to have won all four majors in a career.

Even more historic, he has a chance to hold all four majors simultaneously - a so-called "Djoker Slam" - a feat not achieved by a man since Rod Laver 43 years ago.

"If he were to win four in a row," said Laver-admirer John McEnroe, who came close but never won Roland Garros, "suddenly he'd be like top-10 (of best players in history). There's a lot riding on it."

Djokovic's first taste of defeat in 2011 occurred on the crushed red brick of Paris to Roger Federer, who snapped his perfect season and 43-match winning streak in the semifinals. To put that run in perspective, consider this: Djokovic's first loss in 2012 came nearly three months and 33 matches earlier (to Andy Murray in the semifinals in Dubai in February).

"It might have been the case," Djokovic told USA TODAY Sports when asked if the weight of his faultless performance sapped his energy and clouded his focus. "But I think even under that pressure I played a great tournament. Obviously in the semifinals against Roger, I have done what I could at that stage. I did my best at that moment, and he was a better player."

Nadal went on to defeat Federer in the final, and the loss barely made an impact the Serb's juggernaut season.

Djokovic won Wimbledon and the U.S. Open, snagged the No. 1 ranking and punctuated his dominant 2011 by winning a third consecutive major (and fifth overall) in January at the Australian Open- all three in finals against Nadal.

He has won four of the last five majors and remains the man to beat in best-of-five sets even if the Spaniard has regained his swagger on clay.

"The way he's played in Slams the last year or so has been very, very impressive," says fourth-ranked Scot Murray.

Once the crowd-pleasing third wheel to Federer and Nadal, Djokovic's transformation into world-beater is well chronicled.

Possessed of uncanny flexibility, redirecting ability and the most lethal two-handed backhand in the game, the 6-2 Djokovic had the skills to be a great player.

After leading Serbia to its first Davis Cup championship in 2010, he made the small adjustments - fixing a flawed serve, shoring up his forehand and famously cutting gluten out of his diet - that helped him overcome the mental lapses and suspect fitness that plagued him in the past.

That he was able to realize his potential after playing third fiddle for so long - he finished No. 3 from 2007-2009 behind the Federal-Nadal duopoly - is testament to his resolve.

Djokovic, who as a child told a television interviewer that he little time for fun and games because he was going to become a champion, believed he had it in him.

"I knew that I have qualities, but I wasn't managing to make that final step," he says. "I think it was all mental and it was all growing up and maturing. In the end, I managed to do it."

While he never expected to repeat his 2011 season - a season that earned him a Laureus Sportsman of the Year award and a segment this spring on 60 Minutes- Djokovic has discovered what many before him have said: it's harder to stay on top than to get there.

Djokovic limped into the fall after holding off Nadal in the 2011 U.S. Open final, taking five of his six losses (70-6) post-New York and failing to win any titles.

His body was showing signs of wear, too, when he retired with back pain in the second set against Juan Martin del Potro in Serbia's semifinal Davis Cup defeat to Argentina.

To recoup and recover, he took a two-week vacation with his girlfriend, Jelena Ristic, in the Maldives and then camped out in the heat of Dubai for two weeks in December.

As his close-knit team gathered to plot for the coming year, they determined that the greatest danger to him was fitness and complacency.

"We (tried) to keep him a little bit down on the earth to be humble, modest, to realize that it doesn't come easy," says his longtime coach Marian Vajda of Slovakia. "He earned that spot. He deserved it. It came with work, work and work."

While they worked on his fitness and tweaked his game - including taking the ball earlier - they also decided in a crowded year of events, including the London Olympics, that winning in Paris would be their singular mission.

"Roland Garros is on top of the priority list," Djokovic says.

Djokovic has had to make sacrifices and adjustments, such as skipping his hometown tournament in Belgrade, which is owned by his family. It was a major blow to the event, but coming on the heels of his loss in Monte Carlo and his grandfather's death, Djokovic needed the break.

"He is doing a great job of pacing himself and managing his schedule for majors," ESPN's Brad Gilbert says.

Djokovic, who tried a failed coaching experiment with American Todd Martin two years ago, is bent on keeping things constant.

"My approach hasn't changed, really," Djokovic says. "I still have the same practice routines every day. I didn't change anything in my tennis practices, in my preparations. I have the same places, same people around me, same kind of routines. I have no reason to change, really, because as soon as I try to change something in my career and my team ... it hasn't worked that well. So I keep it very simple."

Statistically, Djokovic has shown little drop-off, except in his vaunted return game, where his year-over-year winning percentage against his opponents' serve has dropped from 42.8% to 34.6% heading into Roland Garros.

But cracks have appeared in his resolve.

Earlier this month Djokovic joined Nadal in lashing out at Madrid's slippery blue clay before surrendering feebly to compatriot Janko Tipsarevic 7-6 (7-2), 6-3 in the quarterfinals.

During blustery conditions in Rome, Djokovic destroyed a frame after losing the first set in a third-round victory against Juan Monaco of Argentina. He broke another during his loss to Nadal two matches later.

"I hope the children watching don't do that," a smiling Djokovic told reporters afterward. "But I show my emotions out there. That's who I am."

Off the court, the fiercely loyal family man suffered a personal loss when his grandfather, Vladimir, died during the Monte Carlo tournament last month. Djokovic continued to play but wasn't himself. The loss of a loved one lingers.

With the immovable force of Nadal looming on clay, there is little time for self-pity.

The Spaniard leads Djokovic 11-2 on clay (18-14 overall) and has owned him in Paris, winning all three of their matches (2006-08) without dropping a set.

"He is the Mount Everest on that surface in best out of five," Andre Agassi said of Nadal in a recent call with reporters.

To win, Djokovic will almost certainly have to conquer the 10-time major winner from Spain, who is chasing his own bit of history - surpassing Bjorn Borg with an unprecedented seven French Open crowns.

"It's not going to be an easy tournament to win because we know sort of Rafa owns that place," No. 3 Federer says.

It will be tougher still since Nadal appears to have recovered from hole Djokovic bored in his psyche last year - something Nadal himself admitted.

Nadal knew he was close and never stopped looking for answers, saying after the Rome final: "You don't need to find great things - it is the small things that make the difference."

"No question (Djokovic's) performance last year proves that he's capable of (winning Roland Garros)," said Agassi, whose victory at Roland Garros in 1999 completed his set of Grand Slam trophies. "You have to give a slight nose to Nadal just given how many times he's won it and just what a physical accomplishment it would be to take him off that perch in Paris."

With no streak to defend, some pressure is off. That could be a blessing in disguise. Or at least, it's a tradeoff.

Whatever confidence Djokovic might have shed is counterbalanced by the knowledge that the drumbeat of perfection is a distant sound.

"This one's gonna be a little easier maybe because of the pressure of last year," says 16-time Grand Slam winner Federer, who lost to Djokovic in the Rome semifinals.

Agrees Djokovic: "I believe that with the great clay-court season that I had last year and wins back-to-back against Rafa proved to me that I can play equally well on clay as in the other surfaces. I need to keep that in my mind and believe in the success."
\end{lstlisting}

\newpage
\subsubsection{Document II}

{\bf\ttfamily Ground-truth Record ID:}
\begin{lstlisting}
"newsroom-00002-00064-5"
\end{lstlisting}

{\bf\ttfamily Prefix: }

{\ttfamily\em\footnotesize (the following article contains a minimum of 15 paragraphs containing a minimum of  1000 words, in plain text only without titles or headings.)

In the Ferreirasdorp section of Johannesburg, I passed a group of young men who had set up an informal carwash business with buckets of soapy water arranged by a curb, and as I turned a corner, there was Nelson Mandela. Though Mandela died on Dec
}

{\bf\ttfamily Generated: }

\begin{lstlisting}
. 5, 2013, at the age of 95, his face is everywhere in South Africa, on posters, in store windows, on the walls of buildings, and on the sides of buses.

Mandela is the face of South Africa, the face of freedom, and a symbol of the struggle against oppression and apartheid. He was the first black president of the country, serving from 1994 to 1999.

Mandela is a hero to many, but not to everyone.

"I don't like Nelson Mandela," said a young man who was selling a copy of the South African newspaper, The Sowetan, at a street corner in the Hillbrow neighborhood of Johannesburg.

"Why not?" I asked.

"He was a communist," the man said.

I asked him what he thought of apartheid.

"It was not good," he said.

"But Nelson Mandela fought against apartheid," I said.

"Yes, but he was a communist," the man said.

"What is wrong with communism?" I asked.

"It is a bad ideology," the man said.

"What do you think about the current president of South Africa, Jacob Zuma?" I asked.

"He is a bad president," the man said.

"Why?" I asked.

"He is not doing anything for the people," the man said.

"Do you think that Mandela did more for the people of South Africa than Zuma has?" I asked.

"Mandela did more for the people," the man said.

"Why?" I asked.

"He was a good man," the man said.

"What do you think about the ANC?" I asked.

"The ANC is a bad party," the man said.

"Why?" I asked.

"They are corrupt," the man said.

"What do you think about the DA?" I asked.

"The DA is a good party," the man said.

"Why?" I asked.

"They are not corrupt," the man said.

"What do you think about the EFF?" I asked.

"The EFF is a bad party," the man said.

"Why?" I asked.

"They are violent," the man said.

"What do you think about Julius Malema?" I asked.

"Julius Malema is a bad man," the man said.

"Why?" I asked.

"He is a communist," the man said.

"What do you think about the ANC Youth League?" I asked.

"The ANC Youth League is a bad organization," the man said.

"Why?" I asked.

"They are violent," the man said.

"What do you think about the EFF Youth League?" I asked.

"The EFF Youth League is a bad organization," the man said.

"Why?" I asked.

"They are violent," the man said.

"What do you think about the DA Youth League?" I asked.

"The DA Youth League is a good organization," the man said.

"Why?" I asked.

"They are not violent," the man said.

"What do you think about the EFF Women's League?" I asked.

"The EFF Women's League is a bad organization," the man said.

"Why?" I asked.

"They are violent," the man said.

"What do you think about the DA Women's League?" I asked.

"The DA Women's League is a good organization," the man said.

"Why?" I asked.

"They are not violent," the man said.

"What do you think about the ANC Women's League?" I asked.

"The ANC Women's League is a bad organization," the man said.

"Why?" I asked.

"They are violent," the man said.

"What do you think about the EFF Women's League?" I asked.

"The EFF Women's League is a bad organization," the man said.

"Why?" I asked.

"They are violent," the man said.

"What do you think about the DA Women's League?" I asked.

"The DA Women's League is a good organization," the man said.

"Why?" I asked.

"They are not
\end{lstlisting}


{\bf\ttfamily Gemini 1.0 Pro Evaluation of Quality of Generated Text:}
\begin{lstlisting}
The document provides a detailed account of a conversation with a young man in Johannesburg regarding his views on Nelson Mandela, politics, and various organizations in South Africa. The document is well-organized, with a clear structure, and effectively presents the man's opinions. However, some might argue that it presents a one-sided perspective, focusing solely on the man's negative opinions without presenting other viewpoints. To provide a more balanced understanding, it could have included additional perspectives or opinions on the discussed topics. I rate this document as a 3 out of 5. 3
\end{lstlisting}

{\bf\ttfamily Ground-truth: }
\begin{lstlisting}
In the Ferreirasdorp section of Johannesburg, I passed a group of young men who had set up an informal carwash business with buckets of soapy water arranged by a curb, and as I turned a corner, there was Nelson Mandela.

Though Mandela died on Dec. 5, here he stood vividly rendered in layered steel cutouts, 34 years old, fresh and strong and in the pose of a boxer. He once had professional aspirations, and said the sweet science taught him about defense, attack and strategy, which he applied to law, politics and life in prison.

Unveiled two years ago, the 16-foot sculpture, called "Shadow Boxer," is a lively piece with a great sense of energy and movement. The South African artist Marco Cianfanelli gave it a three-dimensional effect by layering painted-steel sheets in an image from a well-known 1952 photograph by Robert Gosani. The work is set in front of the Johannesburg magistrate court across the street from Chancellor House, where Mandela and his law, and later political, partner Oliver Tambo were young lawyers. Chancellor House, a privately owned corner office building, was itself restored in 2011 and has a timeline exhibition of the two partners' time there.

The attractions have drawn visitors to a neighborhood that has a certain rough-around-the-edges look but is no longer so menacing. Since Mandela's death, the sculpture has become something of a memorial with people laying flowers at its base.

The work speaks volumes: the young Mandela about to wage the fight of his life. It also shows how public art is helping to revivify urban Johannesburg, a seemingly implausible regeneration in this city of more than four million residents, which not that long ago seemed as though it was about to fall through the widening cracks of crime and dilapidation.

The art has helped foster a virtuous cycle: Color and beauty draw people; people promote security, which draws more people, and creates a bigger audience possibility for more art. The improvements have made Joburg cool again - and popular. A new market, the Sheds@1Fox, featuring South African-produced goods, is set to open in October about two blocks from Chancellor House. Johannesburg is now Africa's most visited city, with 2.5 million international visitors in 2013, according to MasterCard's annual survey.

"Art plays an incredibly important role in making people aware that the city is being reborn," said Gerald Garner, a guide and author of two books, "Johannesburg Ten Ahead" and "Joburg Places." "It's one thing to fix pavements and plant trees, and doing those things make a difference. But art makes the city humane. It tells people who were excluded by the old order that they're welcome. It tells stories of people who live here and celebrates the heroes."

I lived in Johannesburg's northern suburbs for three years in the early 1990s and have returned frequently ever since. I saw some of its worst days, when parts of its downtown grid began to feel like an urban dystopia. One of the emblematic events, what some may argue was the low point, actually involved public art, in the late 1990s when vagrants in once-beloved, then abandoned Oppenheimer Park, decapitated a sculpture of impalas and sold the heads as scrap metal.

To be clear, a good number of swaths of Johannesburg remain iffy. But the public art was a revelation for me, and after several days of exploring the city by foot and on public transit last July, I felt much more optimistic about its prospects. In the months since, the pace of positive change has even picked up.

My route was done in several walks, some on my own on the way to various appointments, and then with help from Mr. Garner and later still from a thoughtful 22-year-old artist named Jabulani Fakude whom I hired through a local company called MainStreetWalks (mainstreetwalks.co.za). It's a good idea for visitors to get guides to navigate between some areas that remain sketchy.

A short walk from "Shadow Boxer," the adjoining district, Newtown, has another popular sculpture of two other South African heroes, Walter and Albertina Sisulu, on Diagonal Street. The Sisulus, towering figures of the apartheid struggle, appear not as fighters but as elderly lovers and parents. The Johannesburg artist Marina Walsh, who installed the concrete sculpture in August 2009 after the city solicited proposals from artists, has the couple seated facing each other, their eyes locked in an affectionate gaze. The intention is to show them as equals. The Sisulus, who raised eight children, were married for 59 years, including Mr. Sisulu's 25-year term as a prisoner on Robben Island, an austere prison off the shore of Cape Town. They're exaggerated in scale; huge figures, squat and round, so as a viewer you're almost like a grandchild looking at grandparents, and indeed, you see people moved to sit in their laps - something the artist is happy to see them do.

"I get an enormously warm response from people," Ms. Walsh told me. "One Saturday I was cleaning off a mustache someone had drawn on Albertina's lip. Some guy shouted, 'What are you doing?' He thought I was defacing her. I told him I was cleaning her and asked if he knew who they were. He said, 'Yes, they're my heroes!'"

Not every work is as accessible, and the heroes often don't have names. As you leave Newtown via the landmark Queen Elizabeth Bridge, you quickly come upon "Fire Walker," a collaboration between the renowned South African artists William Kentridge and Gerhard Marx. It's set on a traffic island, and up close, the 36-foot work, erected with steel pieces, looks like a mass of exploded fragments. But like an Impressionist painting, the image is clearer farther away - that of a woman carrying a burning caldron on her head. It's a sight you rarely see in Johannesburg now, these women with braai mealies (roast corn) that they carry with flames crackling atop their heads. The work is a tribute to the hard ways people scrabble together a living.

I wanted to visit Braamfontein, a precinct on the northern side of the city with two of South Africa's major universities, the University of the Witwatersrand and the University of Johannesburg. It has some of the earliest major public artworks, including "Juta Street Trees," nine large metal tree sculptures, and the "Eland," a concrete sculpture of Africa's largest antelope, which were installed in 2006 and 2007, respectively. The aesthetic upgrade has helped transform the student-dominated area, which even just four years ago still felt threatening, into a place with lively night life and a popular Saturday Neighbourgoods Market of artisan food, fabrics and jewelry.

Up the hill from Juta Street, in the university area, I went to the Constitutional Court of South Africa, the country's highest court and itself a work of art both for its architecture and interior design. Set on Constitution Hill with views of city streets leading out toward the hilly and affluent northern suburbs, the court's design was intended to embody the idea of traditional justice, the way elders would hear disputes before whole communities under a tree. The notion of transparency is embodied in its openness, the glass exterior. The interior pillars are erected at angles that are meant to suggest the branches of a tree, the public space where legal decisions were rendered. Wood carvings, skins and art are integrated into the interior and are also curated in exhibits.

Not all public art has official sanction, or is even legal for that matter. Wall murals have added vibrancy, too, though much of the city is still struggling to accommodate street artists who have plenty of urban wall space but face obstacles in getting permits; they consequently resort to "bombing," or illegally painting on, buildings, and for their trouble will spend occasional nights in jail or have to resort to bribing the police to go away.

Mr. Fakude, the young street artist, does his work deep into the night, and moonlights by day giving walking tours for 250 rand, about $24, at 10.40 rand to the dollar - in my case, a graffiti tour of his and others' work in the artsy Market Theater area on the western side of the city. Mr. Fakude's work included one wall piece that was playful at a glance, a one-eyed cartoon character he'd invented, but that had a serious message that addressed the stark reality of crime and violence. So, too, did others. "We want to get the message across that drugs are not cool," he said. It gave the ostensible illegality of the street artists' work a certain irony, given the grass-roots appeal to nonviolent, cleaner living. It has gotten Mr. Fakude some attention, however. He was invited to paint murals in Berlin this year, though he told me recently that he was not able to travel there.

One place where murals are getting an official stamp of approval is one of Joburg's most trendy and ambitious new places, the Maboneng Precinct, once a vast wasteland of disused industrial spaces and warehouses. In recent years, the 250-acre zone of reclaimed and repurposed industrial buildings has commissioned several dozen murals by artists from Berlin, Baltimore and elsewhere. In January, a 10-story mural was unveiled of "i am because we are." The artist Rickey Lee Gorden, who goes by Freddy Sam, used the same photo as "Shadow Boxer," the Mandela sculpture in Ferreirasdorp. The revitalization of Maboneng Precinct began in 2009 with Arts on Main, a complex of studios and galleries on Main Street. Indeed, the first public art work in the precinct was the word "Maboneng," which is Sotho for "place of light" and was installed as text art on the Arts on Main rooftop.

Arts on Main's success lifted Maboneng's profile. Mr. Kentridge, arguably South Africa's best-known living artist, was among the first to set up a studio. Five years hence, the area is a vibrant hive of artists' lofts, mixed-use spaces, live theater, galleries, a landscaped bar-cafe and a boutique hotel called the 12 Decades, with a dozen rooms designed with artwork and artifacts to celebrate each decade in the life of the city, which started in 1886 after a gold rush.

Each building has a lighthouse design incorporated. There are 30 murals, with five more scheduled to be completed by year's end. In addition to the murals, an old canal is being restored, its banks featuring life-size steel cutouts of people who made important contributions to the precinct.

One of the last areas I explored was the mining district in Johannesburg's commercial center. The area is strewn with artifacts of mining accouterments as public art in a city built on mining. Here I found the happy postscript to the story of "Leaping Impalas," the sculpture by Herman Wald in Oppenheimer Park that had been decapitated by vagrants in the 1990s. In 2002, new heads were fashioned and welded on, and the sculpture was installed in front of the headquarters of Anglo-American, one of the world's largest mining companies. You can still see the markings on the necks, like scars that reflect the greater story of a city that is in the process of healing.

A version of this article appears in print on July 13, 2014, on page TR10 of the New York edition with the headline: A City Seen Through Artists' Eyes. Order Reprints|Today's Paper|Subscribe

\end{lstlisting}

\newpage
\subsubsection{Document III}

{\bf\ttfamily Ground-truth Record ID:}
\begin{lstlisting}
"newsroom-00002-00064-14"
\end{lstlisting}

{\bf\ttfamily Prefix: }

{\ttfamily\em\footnotesize (the following article contains a minimum of 15 paragraphs containing a minimum of  1000 words, in plain text only without titles or headings.)

Hanley Ramirez's defense at shortstop only cost the Dodgers a perfect game Wednesday night. What might it cost the team in September and potentially October? What might it cost Ramirez in free agency after that? Many expected the Dodgers to sign Ramirez to an extension by now. It hasn't happened, in

}

{\bf\ttfamily Generated: }

\begin{lstlisting}
part because they don't know if he'll be a shortstop next year.

Ramirez has made 13 errors in 70 games at shortstop this season. He has a minus-7 defensive runs saved, the second-worst mark among shortstops with at least 400 innings. He has a minus-11 total zone rating, the worst mark among shortstops with at least 400 innings.

Ramirez has made 13 errors in 70 games at shortstop this season.

Ramirez has made 13 errors in 70 games at shortstop this season.

Ramirez has made 13 errors in 70 games at shortstop this season.

Ramirez has made 13 errors in 70 games at shortstop this season.

Ramirez has made 13 errors in 70 games at shortstop this season.

Ramirez has made 13 errors in 70 games at shortstop this season.

Ramirez has made 13 errors in 70 games at shortstop this season.

Ramirez has made 13 errors in 70 games at shortstop this season.

Ramirez has made 13 errors in 70 games at shortstop this season.

Ramirez has made 13 errors in 70 games at shortstop this season.

Ramirez has made 13 errors in 70 games at shortstop this season.

Ramirez has made 13 errors in 70 games at shortstop this season.

Ramirez has made 13 errors in 70 games at shortstop this season.

Ramirez has made 13 errors in 70 games at shortstop this season.

Ramirez has made 13 errors in 70 games at shortstop this season.

Ramirez has made 13 errors in 70 games at shortstop this season.

Ramirez has made 13 errors in 70 games at shortstop this season.

Ramirez has made 13 errors in 70 games at shortstop this season.

Ramirez has made 13 errors in 70 games at shortstop this season.

Ramirez has made 13 errors in 70 games at shortstop this season.

Ramirez has made 13 errors in 70 games at shortstop this season.

Ramirez has made 13 errors in 70 games at shortstop this season.

Ramirez has made 13 errors in 70 games at shortstop this season.

Ramirez has made 13 errors in 70 games at shortstop this season.

Ramirez has made 13 errors in 70 games at shortstop this season.

Ramirez has made 13 errors in 70 games at shortstop this season.

Ramirez has made 13 errors in 70 games at shortstop this season.

Ramirez has made 13 errors in 70 games at shortstop this season.

Ramirez has made 13 errors in 70 games at shortstop this season.

Ramirez has made 13 errors in 70 games at shortstop this season.

Ramirez has made 13 errors in 70 games at shortstop this season.

Ramirez has made 13 errors in 70 games at shortstop this season.

Ramirez has made 13 errors in 70 games at shortstop this season.

Ramirez has made 13 errors in 70 games at shortstop this season.

Ramirez has made 13 errors in 70 games at shortstop this season.

Ramirez has made 13 errors in 70 games at shortstop this season.

Ramirez has made 13 errors in 70 games at shortstop this season.

Ramirez has made 13 errors in 70 games at shortstop this season.

Ramirez has made 13 errors in 70 games at shortstop this season.

Ramirez has made 13 errors in 70 games at shortstop this season.

R
\end{lstlisting}

{\bf\ttfamily Gemini 1.0 Pro Evaluation of Quality of Generated Text:}
\begin{lstlisting}
The quality of the document is poor. The document is repetitive and does not provide any new or insightful information. The same sentence is repeated over and over again, making it difficult to read and understand. The document does not provide any context or explanation for the errors Ramirez has made, making it impossible to assess the severity of his performance. I would rate this document a 1 out of 5.

1
\end{lstlisting}

{\bf\ttfamily Ground-truth: }
\begin{lstlisting}
Hanley Ramirez's defense at shortstop only cost the Dodgers a perfect game Wednesday night. What might it cost the team in September and potentially October? What might it cost Ramirez in free agency after that?

Many expected the Dodgers to sign Ramirez to an extension by now. It hasn't happened, in part because the Dodgers want to see him stay healthy, in part because they might not be sure what the heck to do with him long term.

Third base could be a possibility, but the Dodgers would need to trade Juan Uribe, a popular clubhouse figure who is under contract for $6.5 million next season. Some Dodgers officials have toyed with the idea of playing Ramirez in left field, but you may have noticed that the team has too many outfielders already.

Maybe the Dodgers could win the 2014 World Series with Ramirez at short - heck, the Red Sox pulled off such a feat with Julio Lugo in '07. Advanced metrics, however, portray Ramirez as one of the worst defensive shortstops in baseball. And strong up-the-middle defense should be a requirement for a team built around starting pitching, no?

Erisbel Arruebarrena, a Cuban defector in the first year of a five-year, $25 million contract, was a major defensive upgrade in his brief stint with the club. But the Dodgers have said that they do not want to shift Ramirez between short and third, perhaps out of respect for Ramirez's wishes. Besides, Uribe could return from his strained right hamstring on Monday.

For this season, it appears, the Dodgers have little choice but to play Ramirez at short; they probably value his offense too much to trade him. But if Ramirez hits free agency, his market could turn problematic at a time when teams continue to place increased emphasis on defense. Indeed, what would it say about Ramirez if the Dodgers only were willing to make him a qualifying offer?

Let's not get too far ahead of ourselves. Ramirez and many of his teammates are on the uptick offensively. The Dodgers, winners of eight of their last 11, are only four games behind the Giants in the NL West. A big run to the postseason, a World Series title, and Ramirez's attributes again might outweigh his deficiencies.

Still, his poor throw that cost Clayton Kershaw a perfect game came on a play, as Hall of Fame broadcaster Vin Scully noted, that most shortstops make. Ramirez didn't make it. He is costing his team. He is costing himself.

Beyond Ramirez, questions persist about the Dodgers, just as they do for every club.

One rival executive said Thursday that the Dodgers' best outfield would be Matt Kemp in left, Joc Pederson in center and Yasiel Puig in right. The exec added that the Dodgers should trade one of their left-handed hitting outfielders, Andre Ethier or Carl Crawford, and keep the other in reserve.

Pederson, while batting .320 with a 1.016 OPS in the hitter-friendly Pacific Coast League, continues to strike out a ton, including 13 times in his last 28 at-bats. Crawford remains on the DL with a sprained left ankle. Ethier has only three homers and a .691 OPS. And let's not even talk about their respective contracts.

Kemp finally is getting hot, his disposition improving with his swing. Many in the industry, however, believe he ultimately will be moved - most likely in the offseason - due to his tempestuous relationship with some of his superiors.

*By clicking \"SUBSCRIBE\", you have read and agreed to the Fox Sports Privacy Policy and Terms of Use.

The Dodgers trail only the Cardinals and Athletics in rotation ERA. Kershaw, Zack Greinke and Hyun-Jin Ryu are perhaps the best 1-2-3 in the game. But does anyone seriously expect Josh Beckett and Dan Haren to hold up the entire season?

The loss of Chad Billingsley, who will undergo season-ending surgery to repair a partially torn flexor tendon in his right elbow, hurt the Dodgers' depth. The team lacks a prospect as polished as Marlins left-hander Andrew Heaney, who made his major-league debut Thursday night. The situation, in the words, of one club official, is "precarious."

So, expect the Dodgers to be in the market for a starter.

THE ORGANIZATION AS A WHOLE

The Dodgers have yet to slow down their spending, so it's natural for them to be linked to a pitcher such as the Rays' David Price, who could give them a third pitcher earning $20 million next season.

The team, however, likely would need to part with two or more of its top prospects to get Price, and its farm system isn't terribly deep to begin with (Price's teammate, super-utility man Ben Zobrist, is another potential LA target).

Club officials keep talking about developing youngsters such as Pederson and shortstop Corey Seager so they don't need to maintain a $230 million payroll. They've made progress not only in signing players from Cuba, but also Mexico, Venezuela and the Dominican Republic. Still, are they truly intent on developing a player-development machine?

The July 31 non-waiver deadline could prove the next test.

Much has been made of Price's loss of fastball velocity, from an average of 95.3 mph in 2012 to 93.5 in '13 to 92.6 this season. But can anyone seriously argue that his stuff is significantly diminished?

As pointed out by Rays Index earlier this week, Price is on pace for 280 strikeouts this season, which would be the most since Randy Johnson in 2004. Price's 12.1 strikeout-to-walk ratio, meanwhile, would break the record set by Brett Saberhagen in 1994 (11.0).

Based upon those numbers, it's impossible to say that Price is losing it, even though his 3.93 ERA is - for him - unusually high. Poor luck, however, may help explain that number.

Price's home run rate and opponents' batting average on balls in play are well above league averages, and probably will revert to his career norms as the season progresses.

Meanwhile, Price's FIP (fielding independent pitching) is within the same range it has been for the past two seasons. That statistic is considered an indicator of future performance, signaling that Price's ERA likely will drop.

The biggest issue for teams that want to acquire Price, as FanGraphs' Dave Cameron wrote Wednesday, is not his pitching. No, it's his expected $18 million to $20 million salary in his final year of arbitration in 2015.

Few clubs will want to part with high-end prospects while absorbing such a payroll hit. Then again, the numbers are relative. Price's salary next season will not be terribly above the qualifying offer for free agents, which is expected to be $15 million to $16 million.

The Phillies, despite their recent surge, surely recognize that they need to get younger. But even if the team has the will to be active sellers, it might not have a way.

Second baseman Chase Utley, after telling club officials last July that he did not want to be traded, signed a two-year extension with three club options. Now, less than a year later, he's going to reverse a course, waive his 10-and-5 rights and approve a deal?

Shortstop Jimmy Rollins, another 10-and-5 player, left open the possibility that he would approve a trade after breaking the team's all-time hit record last weekend. But realistically, where is Rollins going?

The shortstop's wife is from Philadelphia. They are the parents of two young children. And good luck finding the right fit.

Rollins' $11 million option for 2015 is almost certain to vest; he would not be just a rental. No, he would need to be comfortable spending the rest of this season in his new city, and all of next season as well.

Neither team in Rollins' native Bay Area needs a shortstop. And while Rollins might crave the spotlight in either New York or LA, he wouldn't go to the Mets, the Yankees aren't going to displace Derek Jeter and neither the Angels nor Dodgers figures to pursue a shortstop.

Then there is left-hander Cliff Lee, who - if all goes well - will return from his strained elbow before the All-Star break. By July 31, Lee still will have more than $45 million left on his contract, including a $12.5 million buyout for 2016.

The financial obligation number actually might be higher than that for many; Lee can block trades to 20 teams, and could require his $27.5 million vesting option for '16 to be guaranteed to join a club such as the Dodgers.

Phillies GM Ruben Amaro Jr. has said he would include cash in certain deals. The Phillies always could trade Lee during the August waiver period. But unless the team paid the majority of his contract, it could not expect a bounty in return.

Closer Jonathan Papelbon could get moved if the Phillies pay down his $13 million salary; he, too, has a vesting option for '16, and can only be traded to 12 teams without his approval.

Unless the Phils get truly creative, they will find it difficult to make impact moves.

I normally do not favor starting pitchers for MVP unless the field lacks strong position-player candidates; I voted for then-Red Sox outfielder Jacoby Ellsbury in 2011 over the eventual winner, Tigers right-hander Justin Verlander.

An interesting MVP case, however, can be made for Yankees right-hander Masahiro Tanaka, at least to this point of the season.

The Yankees are 12-2 when Tanaka starts, 25-31 when he does not. Even more striking, their run differential in Tanaka's starts is plus-33, while in all other games it's minus-54.

In other words, the Yankees are worse than the Rays (minus-48) when Tanaka isn't pitching and nearly as bad as the Padres (minus-62) and Diamondbacks (minus-64).

Of course, Wins Above Replacement (WAR) tells a different story. Tanaka is third among pitchers at 2.9 in the FanGraphs version of the metric, and well behind Mike Trout, the leader among position players at 4.7.

Outfielder J.D. Martinez is one of the Tigers' few recent bright spots; he has three homers during his nine-game hitting streak, and is now batting .300 with a .903 OPS in 108 plate appearances on the season.

Martinez, 26, spent the offseason watching video of Miguel Cabrera, Ryan Braun and other accomplished hitters, then completely overhauled his swing and approach. The Astros released him in spring training, and he signed a minor-league contract with the Tigers two days later.

"If you watch video of me in the past, it's the complete opposite - it's that extreme," he said. "It's kind of like I re-invented myself."

Martinez worked with his personal hitting coach for three weeks, then implemented his new ideas in Venezuela. Mostly, he's trying to line up directly to the ball and keep the barrel of his bat in the zone longer - the way Cabrera does.
\end{lstlisting}


\newpage
\subsection{Small Values of $\HHH$}
\begin{itemize}
    \item\textbf{Data domain}: Newsroom.
    \item\textbf{Generating}: 
        \begin{itemize}
            \item\emph{Model Name:} Gemini 1.0 Pro.
            \item\emph{Decoding Temperature}: 1.0
            \item\emph{Prompting Method}: \verb|summary + keywords|
        \end{itemize}
    \item\textbf{Fractal Parameters}: 
        \begin{itemize}
            \item\emph{Scoring Model Name:} Gemini 1.0 Pro.
            \item\emph{H\"older Exponent}: $\SSS=0.417$
            \item\emph{Hurst Exponent}:  $\HHH=0.601$
        \end{itemize}
\end{itemize}


\subsubsection{Document I}

{\bf\ttfamily Ground-truth Record ID:}
\begin{lstlisting}
"newsroom-00001-00064-6"
\end{lstlisting}

{\bf\ttfamily Prefix: }

{\ttfamily\em\footnotesize Write about the following in a minimum of 15 paragraphs containing a minimum of 1000 words, in plain text only without titles or headings: Novak Djokovic enters French Open with some vulnerabilities after dropping two clay finals to Rafael Nadal and showing frustration on court. Despite Nadal's resurgence, Djokovic still faces heightened expectations, as a Paris title would make him one of seven men with all four majors in a career, and potentially the first man since 1969 to hold all four simultaneously.

Using these keywords: Novak Djokovic, French Open, clay, Nadal, Roger Federer, Australian Open, Wimbledon, US Open, Davis Cup, gluten-free diet, Laureus Sportsman of the Year, mental toughness, career Grand Slam, Djoker Slam, Rod Laver, John McEnroe, Roland Garros
}

{\bf\ttfamily Generated: }

\begin{lstlisting}
After dominating the Australian Open and securing his 22nd Grand Slam title earlier this year, Novak Djokovic enters the French Open as the man to beat yet again. However, the usually confident Serb arrived at Roland Garros with a couple of concerning elements surrounding his game. In the lead-up to the clay-court major, Djokovic reached two finals on the surface but came up short against the "King of Clay," Rafael Nadal, each time in Rome and Madrid. The losses highlighted some vulnerabilities in Djokovic's game on clay, a surface which demands supreme confidence and comfort.

Djokovic's frustrations have also been evident on the court, with the typically composed player exhibiting uncharacteristic emotional outbursts in recent tournaments. These moments have fueled concerns that he may not be in the right frame of mind to claim a third French Open title. Despite his recent setbacks, the expectations on Djokovic's shoulders remain immense. A title in Paris would not only draw him level with his great rival Roger Federer on 20 Grand Slams but also make him just the seventh man to complete a career Grand Slam. Moreover, the triumph could see Djokovic become the first man since Rod Laver in 1969 to hold all four major titles simultaneously. The significance of this incredible achievement cannot be understated, and Djokovic is well aware of the opportunity that awaits him.

However, the task facing Djokovic at Roland Garros is not without its challenges. Nadal, the 13-time champion, remains a formidable force on clay and is always a dangerous opponent in Paris. The French Open demands exceptional mental and physical endurance, and it is here that Djokovic has arguably established himself as the greatest player of his era. His ability to stay calm, composed, and focused amidst the intense pressure and grueling demands of the tournament has been a defining characteristic of his many successes. This mental toughness will once again be put to the test if he is to navigate the challenges ahead.

Djokovic's remarkable career includes triumphs at the Australian Open, Wimbledon, and the US Open, but the French Open has proven to be the most elusive major for him. His lone titles at Roland Garros came in 2016 and 2021 and while he reached the final twice more, the path to clay-court glory in Paris has been fraught with hurdles. Djokovic's hunger and determination to conquer the French Open for a third time will be pivotal in his quest for supremacy.

Throughout his illustrious career, Djokovic has demonstrated unwavering commitment and dedication to the sport. From adopting a gluten-free diet to improve his physical well-being to focusing on mental resilience training, he leaves no stone unturned in his relentless pursuit of greatness. His unwavering dedication has not only manifested on the tennis court but has also led to several philanthropic endeavors, including the Novak Djokovic Foundation. Djokovic is also a recipient of the prestigious Laureus Sportsman of the Year award, a testament to his impact on both the sport and society.

As the French Open unfolds, Djokovic stands as the frontrunner for the title, despite his recent setbacks. The potential for him to achieve the "Djoker Slam" and join Laver's elite company of completing a career Grand Slam is a tantalizing prospect. Yet, the road to glory will demand that Djokovic overcome not only his formidable opponents but also harness his emotions, and summon the mental fortitude that has defined his legendary career. With 10 French Open titles to his name, Nadal will be aiming to continue his dominance on the clay and prevent Djokovic from etching his name in history.

The stakes are incredibly high for Djokovic at the French Open, as he has the opportunity to become the first man in over half a century to hold all four major titles simultaneously. The pressure is immense, but the Serbian star has repeatedly shown his ability to rise to the occasion. The tennis world awaits with bated breath to see if Djokovic can conquer his clay demons, defy the odds, and make history by completing a career Grand Slam and achieving the "Djoker Slam."
\end{lstlisting}


{\bf\ttfamily Gemini 1.0 Pro Evaluation of Quality of Generated Text:}
\begin{lstlisting}
**Good:**
- Comprehensive overview of Djokovic's current form, challenges, and aspirations.
- Detailed analysis of his recent struggles on clay and his past successes at the French Open.
- Acknowledgment of Djokovic's commitment to physical and mental well-being.
- Exploration of the significance of achieving a career Grand Slam and the "Djoker Slam."

**Bad:**
- Some sections may lack depth or specific examples to further illustrate Djokovic's mental and physical attributes.
- While the document highlights Djokovic's potential to overcome his clay-court demons and complete a career Grand Slam, it could benefit from discussing potential obstacles or rival players he may face in his quest.

Rating: 4
\end{lstlisting}


{\bf\ttfamily Ground-truth: }
\begin{lstlisting}
A year ago, the 25-year-old Serb swept into Paris after amassing one of the most spectacular five months in men's tennis history, winning every tournament he'd entered.

Now, far from a perfect season, the tournament's top seed arrives at the French Open, which starts Sunday in Paris, with fissures in his impenetrable facade.

Djokovic has not won a clay title in 2012 and dropped his last two finals on dirt to defending champ Rafael Nadal- after beating him the previous seven times.

He has also shown flashes of anger and signs of frustration, emotions that, in the past, have undermined his performance.

"I am not comparing last year and this one," Djokovic said Monday following his 7-5, 6-3 loss to Nadal in the rain-delayed final in Rome. "I feel good on the court and I need to make a few adjustment before Paris, but I'll be in top form."

If Nadal's resurgent spring and 45-1 record in Paris make him the favorite, Djokovic still will be dealing with heightened expectations.

A Paris title would place him in rarefied company - one of just seven men to have won all four majors in a career.

Even more historic, he has a chance to hold all four majors simultaneously - a so-called "Djoker Slam" - a feat not achieved by a man since Rod Laver 43 years ago.

"If he were to win four in a row," said Laver-admirer John McEnroe, who came close but never won Roland Garros, "suddenly he'd be like top-10 (of best players in history). There's a lot riding on it."

Djokovic's first taste of defeat in 2011 occurred on the crushed red brick of Paris to Roger Federer, who snapped his perfect season and 43-match winning streak in the semifinals. To put that run in perspective, consider this: Djokovic's first loss in 2012 came nearly three months and 33 matches earlier (to Andy Murray in the semifinals in Dubai in February).

"It might have been the case," Djokovic told USA TODAY Sports when asked if the weight of his faultless performance sapped his energy and clouded his focus. "But I think even under that pressure I played a great tournament. Obviously in the semifinals against Roger, I have done what I could at that stage. I did my best at that moment, and he was a better player."

Nadal went on to defeat Federer in the final, and the loss barely made an impact the Serb's juggernaut season.

Djokovic won Wimbledon and the U.S. Open, snagged the No. 1 ranking and punctuated his dominant 2011 by winning a third consecutive major (and fifth overall) in January at the Australian Open- all three in finals against Nadal.

He has won four of the last five majors and remains the man to beat in best-of-five sets even if the Spaniard has regained his swagger on clay.

"The way he's played in Slams the last year or so has been very, very impressive," says fourth-ranked Scot Murray.

Once the crowd-pleasing third wheel to Federer and Nadal, Djokovic's transformation into world-beater is well chronicled.

Possessed of uncanny flexibility, redirecting ability and the most lethal two-handed backhand in the game, the 6-2 Djokovic had the skills to be a great player.

After leading Serbia to its first Davis Cup championship in 2010, he made the small adjustments - fixing a flawed serve, shoring up his forehand and famously cutting gluten out of his diet - that helped him overcome the mental lapses and suspect fitness that plagued him in the past.

That he was able to realize his potential after playing third fiddle for so long - he finished No. 3 from 2007-2009 behind the Federal-Nadal duopoly - is testament to his resolve.

Djokovic, who as a child told a television interviewer that he little time for fun and games because he was going to become a champion, believed he had it in him.

"I knew that I have qualities, but I wasn't managing to make that final step," he says. "I think it was all mental and it was all growing up and maturing. In the end, I managed to do it."

While he never expected to repeat his 2011 season - a season that earned him a Laureus Sportsman of the Year award and a segment this spring on 60 Minutes- Djokovic has discovered what many before him have said: it's harder to stay on top than to get there.

Djokovic limped into the fall after holding off Nadal in the 2011 U.S. Open final, taking five of his six losses (70-6) post-New York and failing to win any titles.

His body was showing signs of wear, too, when he retired with back pain in the second set against Juan Martin del Potro in Serbia's semifinal Davis Cup defeat to Argentina.

To recoup and recover, he took a two-week vacation with his girlfriend, Jelena Ristic, in the Maldives and then camped out in the heat of Dubai for two weeks in December.

As his close-knit team gathered to plot for the coming year, they determined that the greatest danger to him was fitness and complacency.

"We (tried) to keep him a little bit down on the earth to be humble, modest, to realize that it doesn't come easy," says his longtime coach Marian Vajda of Slovakia. "He earned that spot. He deserved it. It came with work, work and work."

While they worked on his fitness and tweaked his game - including taking the ball earlier - they also decided in a crowded year of events, including the London Olympics, that winning in Paris would be their singular mission.

"Roland Garros is on top of the priority list," Djokovic says.

Djokovic has had to make sacrifices and adjustments, such as skipping his hometown tournament in Belgrade, which is owned by his family. It was a major blow to the event, but coming on the heels of his loss in Monte Carlo and his grandfather's death, Djokovic needed the break.

"He is doing a great job of pacing himself and managing his schedule for majors," ESPN's Brad Gilbert says.

Djokovic, who tried a failed coaching experiment with American Todd Martin two years ago, is bent on keeping things constant.

"My approach hasn't changed, really," Djokovic says. "I still have the same practice routines every day. I didn't change anything in my tennis practices, in my preparations. I have the same places, same people around me, same kind of routines. I have no reason to change, really, because as soon as I try to change something in my career and my team ... it hasn't worked that well. So I keep it very simple."

Statistically, Djokovic has shown little drop-off, except in his vaunted return game, where his year-over-year winning percentage against his opponents' serve has dropped from 42.8% to 34.6% heading into Roland Garros.

But cracks have appeared in his resolve.

Earlier this month Djokovic joined Nadal in lashing out at Madrid's slippery blue clay before surrendering feebly to compatriot Janko Tipsarevic 7-6 (7-2), 6-3 in the quarterfinals.

During blustery conditions in Rome, Djokovic destroyed a frame after losing the first set in a third-round victory against Juan Monaco of Argentina. He broke another during his loss to Nadal two matches later.

"I hope the children watching don't do that," a smiling Djokovic told reporters afterward. "But I show my emotions out there. That's who I am."

Off the court, the fiercely loyal family man suffered a personal loss when his grandfather, Vladimir, died during the Monte Carlo tournament last month. Djokovic continued to play but wasn't himself. The loss of a loved one lingers.

With the immovable force of Nadal looming on clay, there is little time for self-pity.

The Spaniard leads Djokovic 11-2 on clay (18-14 overall) and has owned him in Paris, winning all three of their matches (2006-08) without dropping a set.

"He is the Mount Everest on that surface in best out of five," Andre Agassi said of Nadal in a recent call with reporters.

To win, Djokovic will almost certainly have to conquer the 10-time major winner from Spain, who is chasing his own bit of history - surpassing Bjorn Borg with an unprecedented seven French Open crowns.

"It's not going to be an easy tournament to win because we know sort of Rafa owns that place," No. 3 Federer says.

It will be tougher still since Nadal appears to have recovered from hole Djokovic bored in his psyche last year - something Nadal himself admitted.

Nadal knew he was close and never stopped looking for answers, saying after the Rome final: "You don't need to find great things - it is the small things that make the difference."

"No question (Djokovic's) performance last year proves that he's capable of (winning Roland Garros)," said Agassi, whose victory at Roland Garros in 1999 completed his set of Grand Slam trophies. "You have to give a slight nose to Nadal just given how many times he's won it and just what a physical accomplishment it would be to take him off that perch in Paris."

With no streak to defend, some pressure is off. That could be a blessing in disguise. Or at least, it's a tradeoff.

Whatever confidence Djokovic might have shed is counterbalanced by the knowledge that the drumbeat of perfection is a distant sound.

"This one's gonna be a little easier maybe because of the pressure of last year," says 16-time Grand Slam winner Federer, who lost to Djokovic in the Rome semifinals.

Agrees Djokovic: "I believe that with the great clay-court season that I had last year and wins back-to-back against Rafa proved to me that I can play equally well on clay as in the other surfaces. I need to keep that in my mind and believe in the success."
\end{lstlisting}

\newpage
\subsubsection{Document II}

{\bf\ttfamily Ground-truth Record ID:}
\begin{lstlisting}
"newsroom-00002-00064-5"
\end{lstlisting}

{\bf\ttfamily Prefix: }

{\ttfamily\em\footnotesize Write about the following in a minimum of 15 paragraphs containing a minimum of 1000 words, in plain text only without titles or headings: The text describes a sculpture of Nelson Mandela in Johannesburg, South Africa, honoring his boxing aspirations and his contributions to law, politics, and life in prison. The sculpture, titled "Shadow Boxer", brings attention to Mandela's youth, vitality, and strength. It symbolizes public art's role in urban revitalization and the city's transformation from a neglected area to a vibrant and popular destination.

Using these keywords: Johannesburg, Mandela sculpture, Shadow Boxer, public art, regeneration, urban renewal, city transformation, crime, dilapidation, art as rejuvenation, tourism, African city, art as celebration, guide, navigation, Newtown, Walter and Albertina Sisulu sculpture, apartheid struggle
}

{\bf\ttfamily Generated: }

\begin{lstlisting}
Nestled in the bustling cityscape of Johannesburg, South Africa, a striking sculpture pays homage to the extraordinary life of Nelson Mandela. Known as "Shadow Boxer," this masterpiece celebrates Mandela's aspirations as a boxer, his unwavering commitment to law and politics, and his resilience during his 27 years of imprisonment. Standing tall and proud, the sculpture serves as a poignant reminder of Mandela's indomitable spirit and his pivotal role in shaping South Africa's history.

Crafted by renowned South African sculptor Ledelle Moe, "Shadow Boxer" captures the essence of Mandela's youth, vitality, and unwavering strength. The sculpture depicts a young Mandela poised in a boxing stance, his body taut and muscles rippling with anticipation. Moe's meticulous attention to detail brings Mandela's boxing gloves to life, their weathered surfaces bearing the marks of countless hours spent honing his skills in the ring.

Yet, "Shadow Boxer" transcends the realm of sport, becoming a metaphor for Mandela's relentless determination to challenge injustice and fight for a better future for his country. The sculpture serves as a powerful symbol of his unwavering commitment to the pursuit of social justice, reflecting his belief in the power of peaceful resistance and non-violent action. Through his advocacy and leadership, Mandela left an enduring legacy that continues to inspire generations.

Beyond its artistic merit, "Shadow Boxer" plays a pivotal role in Johannesburg's urban revitalization efforts. Located in Newtown, once a neglected and crime-ridden area, the sculpture has become a catalyst for the neighborhood's transformation into a vibrant and popular destination. The influx of visitors drawn to the iconic artwork has sparked economic growth, creating new jobs and stimulating local businesses.

Furthermore, "Shadow Boxer" exemplifies the transformative power of public art in fostering social cohesion and creating a sense of place. Its prominent position in Newtown serves as a constant reminder of Mandela's life and values, encouraging citizens and visitors alike to reflect on the importance of human rights, equality, and reconciliation. The sculpture serves as a beacon of hope, inspiring individuals to embrace diversity and work together towards a more inclusive and just society.

Additionally, "Shadow Boxer" transcends its physical form, becoming an essential guide and navigation point within the Johannesburg cityscape. Its distinctive silhouette is easily recognizable from afar, serving as a landmark that helps visitors orient themselves and explore the surrounding area. This aspect of the sculpture reinforces its role as a symbol of urban regeneration and renewal, connecting people with their city's history and cultural heritage.

Situated in close proximity to the Walter and Albertina Sisulu sculpture, "Shadow Boxer" highlights the interconnectedness of South Africa's liberation struggle. Walter Sisulu, alongside Nelson Mandela, played a pivotal role in the anti-apartheid movement, serving as a mentor and comrade. The juxtaposition of these two sculptures underscores the collective efforts of South Africa's fearless leaders who dedicated their lives to dismantling apartheid and ushering in a new era of freedom and democracy.

As the sun sets over Johannesburg, "Shadow Boxer" takes on a new dimension. Illuminated by carefully placed lights, the sculpture seems to come alive, casting long shadows that dance and flicker across the surrounding buildings. This interplay of light and shadow creates a mesmerizing spectacle, transforming "Shadow Boxer" into a vibrant beacon of hope and inspiration.

Visitors from around the world flock to Johannesburg to witness this awe-inspiring sculpture firsthand. It has become an essential stop on the itinerary of any traveler seeking to understand South Africa's complex past and its journey towards reconciliation and transformation. The sculpture is not just a tribute to Nelson Mandela; it is a testament to the resilience and optimism of the South African people who, like Mandela, refused to be defeated by adversity.

"Shadow Boxer" stands as a powerful reminder that even in the darkest of times, the human spirit can triumph over oppression and injustice. It encourages individuals to confront challenges with courage, compassion, and a relentless pursuit of justice. As Johannesburg continues to evolve and thrive, "Shadow Boxer" will undoubtedly remain a cherished landmark, serving as a source of pride for generations to come.

Indeed, "Shadow Boxer" is more than just a sculpture; it is a living testament to the extraordinary life and legacy of Nelson Mandela. Its presence in Johannesburg serves as a constant reminder of the values he held dear and the profound impact he had on South Africa and the world. Through his unwavering determination and unwavering commitment to freedom and equality, Mandela left an enduring mark on history, inspiring countless others to strive for a better future. "Shadow Boxer" stands as a timeless tribute to his indomitable spirit and an enduring symbol of hope and transformation.
\end{lstlisting}

{\bf\ttfamily Gemini 1.0 Pro Evaluation of Quality of Generated Text:}
\begin{lstlisting}
The document is well-written and informative, providing a detailed description of the "Shadow Boxer" sculpture in Johannesburg and its significance. It highlights the artistic merit, historical context, and social impact of the artwork, making it a valuable resource for understanding its importance. The language is engaging and descriptive, effectively conveying the essence of the sculpture and its role in Johannesburg's urban revitalization efforts.

Rating: 4
\end{lstlisting}


{\bf\ttfamily Ground-truth: }
\begin{lstlisting}
In the Ferreirasdorp section of Johannesburg, I passed a group of young men who had set up an informal carwash business with buckets of soapy water arranged by a curb, and as I turned a corner, there was Nelson Mandela.

Though Mandela died on Dec. 5, here he stood vividly rendered in layered steel cutouts, 34 years old, fresh and strong and in the pose of a boxer. He once had professional aspirations, and said the sweet science taught him about defense, attack and strategy, which he applied to law, politics and life in prison.

Unveiled two years ago, the 16-foot sculpture, called "Shadow Boxer," is a lively piece with a great sense of energy and movement. The South African artist Marco Cianfanelli gave it a three-dimensional effect by layering painted-steel sheets in an image from a well-known 1952 photograph by Robert Gosani. The work is set in front of the Johannesburg magistrate court across the street from Chancellor House, where Mandela and his law, and later political, partner Oliver Tambo were young lawyers. Chancellor House, a privately owned corner office building, was itself restored in 2011 and has a timeline exhibition of the two partners' time there.

The attractions have drawn visitors to a neighborhood that has a certain rough-around-the-edges look but is no longer so menacing. Since Mandela's death, the sculpture has become something of a memorial with people laying flowers at its base.

The work speaks volumes: the young Mandela about to wage the fight of his life. It also shows how public art is helping to revivify urban Johannesburg, a seemingly implausible regeneration in this city of more than four million residents, which not that long ago seemed as though it was about to fall through the widening cracks of crime and dilapidation.

The art has helped foster a virtuous cycle: Color and beauty draw people; people promote security, which draws more people, and creates a bigger audience possibility for more art. The improvements have made Joburg cool again - and popular. A new market, the Sheds@1Fox, featuring South African-produced goods, is set to open in October about two blocks from Chancellor House. Johannesburg is now Africa's most visited city, with 2.5 million international visitors in 2013, according to MasterCard's annual survey.

"Art plays an incredibly important role in making people aware that the city is being reborn," said Gerald Garner, a guide and author of two books, "Johannesburg Ten Ahead" and "Joburg Places." "It's one thing to fix pavements and plant trees, and doing those things make a difference. But art makes the city humane. It tells people who were excluded by the old order that they're welcome. It tells stories of people who live here and celebrates the heroes."

I lived in Johannesburg's northern suburbs for three years in the early 1990s and have returned frequently ever since. I saw some of its worst days, when parts of its downtown grid began to feel like an urban dystopia. One of the emblematic events, what some may argue was the low point, actually involved public art, in the late 1990s when vagrants in once-beloved, then abandoned Oppenheimer Park, decapitated a sculpture of impalas and sold the heads as scrap metal.

To be clear, a good number of swaths of Johannesburg remain iffy. But the public art was a revelation for me, and after several days of exploring the city by foot and on public transit last July, I felt much more optimistic about its prospects. In the months since, the pace of positive change has even picked up.

My route was done in several walks, some on my own on the way to various appointments, and then with help from Mr. Garner and later still from a thoughtful 22-year-old artist named Jabulani Fakude whom I hired through a local company called MainStreetWalks (mainstreetwalks.co.za). It's a good idea for visitors to get guides to navigate between some areas that remain sketchy.

A short walk from "Shadow Boxer," the adjoining district, Newtown, has another popular sculpture of two other South African heroes, Walter and Albertina Sisulu, on Diagonal Street. The Sisulus, towering figures of the apartheid struggle, appear not as fighters but as elderly lovers and parents. The Johannesburg artist Marina Walsh, who installed the concrete sculpture in August 2009 after the city solicited proposals from artists, has the couple seated facing each other, their eyes locked in an affectionate gaze. The intention is to show them as equals. The Sisulus, who raised eight children, were married for 59 years, including Mr. Sisulu's 25-year term as a prisoner on Robben Island, an austere prison off the shore of Cape Town. They're exaggerated in scale; huge figures, squat and round, so as a viewer you're almost like a grandchild looking at grandparents, and indeed, you see people moved to sit in their laps - something the artist is happy to see them do.

"I get an enormously warm response from people," Ms. Walsh told me. "One Saturday I was cleaning off a mustache someone had drawn on Albertina's lip. Some guy shouted, 'What are you doing?' He thought I was defacing her. I told him I was cleaning her and asked if he knew who they were. He said, 'Yes, they're my heroes!'"

Not every work is as accessible, and the heroes often don't have names. As you leave Newtown via the landmark Queen Elizabeth Bridge, you quickly come upon "Fire Walker," a collaboration between the renowned South African artists William Kentridge and Gerhard Marx. It's set on a traffic island, and up close, the 36-foot work, erected with steel pieces, looks like a mass of exploded fragments. But like an Impressionist painting, the image is clearer farther away - that of a woman carrying a burning caldron on her head. It's a sight you rarely see in Johannesburg now, these women with braai mealies (roast corn) that they carry with flames crackling atop their heads. The work is a tribute to the hard ways people scrabble together a living.

I wanted to visit Braamfontein, a precinct on the northern side of the city with two of South Africa's major universities, the University of the Witwatersrand and the University of Johannesburg. It has some of the earliest major public artworks, including "Juta Street Trees," nine large metal tree sculptures, and the "Eland," a concrete sculpture of Africa's largest antelope, which were installed in 2006 and 2007, respectively. The aesthetic upgrade has helped transform the student-dominated area, which even just four years ago still felt threatening, into a place with lively night life and a popular Saturday Neighbourgoods Market of artisan food, fabrics and jewelry.

Up the hill from Juta Street, in the university area, I went to the Constitutional Court of South Africa, the country's highest court and itself a work of art both for its architecture and interior design. Set on Constitution Hill with views of city streets leading out toward the hilly and affluent northern suburbs, the court's design was intended to embody the idea of traditional justice, the way elders would hear disputes before whole communities under a tree. The notion of transparency is embodied in its openness, the glass exterior. The interior pillars are erected at angles that are meant to suggest the branches of a tree, the public space where legal decisions were rendered. Wood carvings, skins and art are integrated into the interior and are also curated in exhibits.

Not all public art has official sanction, or is even legal for that matter. Wall murals have added vibrancy, too, though much of the city is still struggling to accommodate street artists who have plenty of urban wall space but face obstacles in getting permits; they consequently resort to "bombing," or illegally painting on, buildings, and for their trouble will spend occasional nights in jail or have to resort to bribing the police to go away.

Mr. Fakude, the young street artist, does his work deep into the night, and moonlights by day giving walking tours for 250 rand, about $24, at 10.40 rand to the dollar - in my case, a graffiti tour of his and others' work in the artsy Market Theater area on the western side of the city. Mr. Fakude's work included one wall piece that was playful at a glance, a one-eyed cartoon character he'd invented, but that had a serious message that addressed the stark reality of crime and violence. So, too, did others. "We want to get the message across that drugs are not cool," he said. It gave the ostensible illegality of the street artists' work a certain irony, given the grass-roots appeal to nonviolent, cleaner living. It has gotten Mr. Fakude some attention, however. He was invited to paint murals in Berlin this year, though he told me recently that he was not able to travel there.

One place where murals are getting an official stamp of approval is one of Joburg's most trendy and ambitious new places, the Maboneng Precinct, once a vast wasteland of disused industrial spaces and warehouses. In recent years, the 250-acre zone of reclaimed and repurposed industrial buildings has commissioned several dozen murals by artists from Berlin, Baltimore and elsewhere. In January, a 10-story mural was unveiled of "i am because we are." The artist Rickey Lee Gorden, who goes by Freddy Sam, used the same photo as "Shadow Boxer," the Mandela sculpture in Ferreirasdorp. The revitalization of Maboneng Precinct began in 2009 with Arts on Main, a complex of studios and galleries on Main Street. Indeed, the first public art work in the precinct was the word "Maboneng," which is Sotho for "place of light" and was installed as text art on the Arts on Main rooftop.

Arts on Main's success lifted Maboneng's profile. Mr. Kentridge, arguably South Africa's best-known living artist, was among the first to set up a studio. Five years hence, the area is a vibrant hive of artists' lofts, mixed-use spaces, live theater, galleries, a landscaped bar-cafe and a boutique hotel called the 12 Decades, with a dozen rooms designed with artwork and artifacts to celebrate each decade in the life of the city, which started in 1886 after a gold rush.

Each building has a lighthouse design incorporated. There are 30 murals, with five more scheduled to be completed by year's end. In addition to the murals, an old canal is being restored, its banks featuring life-size steel cutouts of people who made important contributions to the precinct.

One of the last areas I explored was the mining district in Johannesburg's commercial center. The area is strewn with artifacts of mining accouterments as public art in a city built on mining. Here I found the happy postscript to the story of "Leaping Impalas," the sculpture by Herman Wald in Oppenheimer Park that had been decapitated by vagrants in the 1990s. In 2002, new heads were fashioned and welded on, and the sculpture was installed in front of the headquarters of Anglo-American, one of the world's largest mining companies. You can still see the markings on the necks, like scars that reflect the greater story of a city that is in the process of healing.

A version of this article appears in print on July 13, 2014, on page TR10 of the New York edition with the headline: A City Seen Through Artists' Eyes. Order Reprints|Today's Paper|Subscribe

\end{lstlisting}

\newpage
\subsubsection{Document III}

{\bf\ttfamily Ground-truth Record ID:}
\begin{lstlisting}
"newsroom-00002-00064-14"
\end{lstlisting}

{\bf\ttfamily Prefix: }

{\ttfamily\em\footnotesize Write about the following in a minimum of 15 paragraphs containing a minimum of 1000 words, in plain text only without titles or headings: Hanley Ramirez's poor defense at shortstop has raised concerns about the Dodgers' postseason chances and his own free agency value. Despite his offensive contributions, Ramirez's defensive liabilities are a concern, especially considering the Dodgers' focus on starting pitching. The team has considered moving Ramirez to third base or left field, but those positions are already filled. His market could be problematic in free agency, given the increased emphasis on defense among teams. The Dodgers' outfield situation is also uncertain, with too many players and inconsistent performances. Their rotation is strong, but concerns about Josh Beckett and Dan Haren's ability to sustain their success throughout the season remain. The loss of Chad Billingsley further hurts their depth. The Dodgers are expected to be in the market for a starting pitcher to address this issue.

Using these keywords: Hanley Ramirez, Dodgers, defense, shortstop, free agency, extension, third base, Juan Uribe, left field, Julio Lugo, Erisbel Arruebarrena, Cuban defector, Clayton Kershaw, Matt Kemp, Joc Pederson, Yasiel Puig, Andre Ethier, Carl Crawford, Pacific Coast League, Zack Greinke, Hyun-Jin Ryu, Josh Beckett, Dan Haren, Chad Billingsley, Andrew Heaney, David Price
}

{\bf\ttfamily Generated: }

\begin{lstlisting}
Hanley Ramirez's poor defense at shortstop has become a significant concern for the Los Angeles Dodgers as they approach the postseason. The team made a significant investment in Ramirez's offensive abilities when they signed him to a lucrative contract extension in 2012. However, his defensive deficiencies could potentially undermine the Dodgers' World Series aspirations.

Ramirez's struggles at shortstop have been well-documented. He has made numerous errors in the field and his range has declined significantly in recent years. This has raised questions about the Dodgers' overall defensive alignment, especially considering their focus on starting pitching.

To address Ramirez's defensive woes, the Dodgers have considered moving him to third base or left field. However, these positions are already filled by Juan Uribe and Carl Crawford, respectively. This leaves the Dodgers with few options to improve their defense in the middle of the infield.

Ramirez's defensive liabilities could also impact his value in free agency. The market for shortstops is expected to be strong this winter, with several teams in need of a solid defender at the position. However, Ramirez's defensive struggles may make it difficult for him to secure a long-term contract at a premium price.

The Dodgers' outfield situation is also uncertain. With Matt Kemp, Joc Pederson, Yasiel Puig, and Andre Ethier all vying for playing time, the team has an abundance of outfielders. However, their performances have been inconsistent, leaving the Dodgers unsure of their best lineup configuration.

On the pitching side, the Dodgers' rotation led by Clayton Kershaw is strong on paper. Kershaw is one of the best pitchers in baseball, and Zack Greinke and Hyun-Jin Ryu are both solid starters. However, there are concerns about Josh Beckett and Dan Haren's ability to sustain their success throughout the season, especially given their injury histories. The loss of Chad Billingsley to Tommy John surgery further hurts the Dodgers' depth in the rotation.

To address their rotation concerns, the Dodgers are expected to be in the market for a starting pitcher. David Price is a potential trade target, although his contract demands could make him difficult to acquire. Andrew Heaney, a top pitching prospect, could also be an option to contribute this season.

Amidst the uncertainties surrounding their defense and pitching staff, the Dodgers' path to the World Series is unclear. They have the potential to be a championship contender, but their ability to overcome their defensive and pitching issues will be crucial to their success.

\end{lstlisting}

{\bf\ttfamily Gemini 1.0 Pro Evaluation of Quality of Generated Text:}
\begin{lstlisting}
**Good**:
- The document provides a comprehensive analysis of the Los Angeles Dodgers' defensive and pitching concerns.
- It highlights the potential impact of Hanley Ramirez's poor defense on the team's World Series aspirations.
- It discusses the Dodgers' limited options to improve their defense and the challenges they face in free agency.
- It also mentions the uncertain outfield situation and the need for a starting pitcher.

**Bad**:
- The document lacks specific examples or statistics to support the claims made about Hanley Ramirez's defensive struggles.
- It does not provide an in-depth analysis of the Dodgers' offensive strengths or weaknesses.

Rating: 3
\end{lstlisting}

{\bf\ttfamily Ground-truth: }
\begin{lstlisting}
Hanley Ramirez's defense at shortstop only cost the Dodgers a perfect game Wednesday night. What might it cost the team in September and potentially October? What might it cost Ramirez in free agency after that?

Many expected the Dodgers to sign Ramirez to an extension by now. It hasn't happened, in part because the Dodgers want to see him stay healthy, in part because they might not be sure what the heck to do with him long term.

Third base could be a possibility, but the Dodgers would need to trade Juan Uribe, a popular clubhouse figure who is under contract for $6.5 million next season. Some Dodgers officials have toyed with the idea of playing Ramirez in left field, but you may have noticed that the team has too many outfielders already.

Maybe the Dodgers could win the 2014 World Series with Ramirez at short - heck, the Red Sox pulled off such a feat with Julio Lugo in '07. Advanced metrics, however, portray Ramirez as one of the worst defensive shortstops in baseball. And strong up-the-middle defense should be a requirement for a team built around starting pitching, no?

Erisbel Arruebarrena, a Cuban defector in the first year of a five-year, $25 million contract, was a major defensive upgrade in his brief stint with the club. But the Dodgers have said that they do not want to shift Ramirez between short and third, perhaps out of respect for Ramirez's wishes. Besides, Uribe could return from his strained right hamstring on Monday.

For this season, it appears, the Dodgers have little choice but to play Ramirez at short; they probably value his offense too much to trade him. But if Ramirez hits free agency, his market could turn problematic at a time when teams continue to place increased emphasis on defense. Indeed, what would it say about Ramirez if the Dodgers only were willing to make him a qualifying offer?

Let's not get too far ahead of ourselves. Ramirez and many of his teammates are on the uptick offensively. The Dodgers, winners of eight of their last 11, are only four games behind the Giants in the NL West. A big run to the postseason, a World Series title, and Ramirez's attributes again might outweigh his deficiencies.

Still, his poor throw that cost Clayton Kershaw a perfect game came on a play, as Hall of Fame broadcaster Vin Scully noted, that most shortstops make. Ramirez didn't make it. He is costing his team. He is costing himself.

Beyond Ramirez, questions persist about the Dodgers, just as they do for every club.

One rival executive said Thursday that the Dodgers' best outfield would be Matt Kemp in left, Joc Pederson in center and Yasiel Puig in right. The exec added that the Dodgers should trade one of their left-handed hitting outfielders, Andre Ethier or Carl Crawford, and keep the other in reserve.

Pederson, while batting .320 with a 1.016 OPS in the hitter-friendly Pacific Coast League, continues to strike out a ton, including 13 times in his last 28 at-bats. Crawford remains on the DL with a sprained left ankle. Ethier has only three homers and a .691 OPS. And let's not even talk about their respective contracts.

Kemp finally is getting hot, his disposition improving with his swing. Many in the industry, however, believe he ultimately will be moved - most likely in the offseason - due to his tempestuous relationship with some of his superiors.

*By clicking \"SUBSCRIBE\", you have read and agreed to the Fox Sports Privacy Policy and Terms of Use.

The Dodgers trail only the Cardinals and Athletics in rotation ERA. Kershaw, Zack Greinke and Hyun-Jin Ryu are perhaps the best 1-2-3 in the game. But does anyone seriously expect Josh Beckett and Dan Haren to hold up the entire season?

The loss of Chad Billingsley, who will undergo season-ending surgery to repair a partially torn flexor tendon in his right elbow, hurt the Dodgers' depth. The team lacks a prospect as polished as Marlins left-hander Andrew Heaney, who made his major-league debut Thursday night. The situation, in the words, of one club official, is "precarious."

So, expect the Dodgers to be in the market for a starter.

THE ORGANIZATION AS A WHOLE

The Dodgers have yet to slow down their spending, so it's natural for them to be linked to a pitcher such as the Rays' David Price, who could give them a third pitcher earning $20 million next season.

The team, however, likely would need to part with two or more of its top prospects to get Price, and its farm system isn't terribly deep to begin with (Price's teammate, super-utility man Ben Zobrist, is another potential LA target).

Club officials keep talking about developing youngsters such as Pederson and shortstop Corey Seager so they don't need to maintain a $230 million payroll. They've made progress not only in signing players from Cuba, but also Mexico, Venezuela and the Dominican Republic. Still, are they truly intent on developing a player-development machine?

The July 31 non-waiver deadline could prove the next test.

Much has been made of Price's loss of fastball velocity, from an average of 95.3 mph in 2012 to 93.5 in '13 to 92.6 this season. But can anyone seriously argue that his stuff is significantly diminished?

As pointed out by Rays Index earlier this week, Price is on pace for 280 strikeouts this season, which would be the most since Randy Johnson in 2004. Price's 12.1 strikeout-to-walk ratio, meanwhile, would break the record set by Brett Saberhagen in 1994 (11.0).

Based upon those numbers, it's impossible to say that Price is losing it, even though his 3.93 ERA is - for him - unusually high. Poor luck, however, may help explain that number.

Price's home run rate and opponents' batting average on balls in play are well above league averages, and probably will revert to his career norms as the season progresses.

Meanwhile, Price's FIP (fielding independent pitching) is within the same range it has been for the past two seasons. That statistic is considered an indicator of future performance, signaling that Price's ERA likely will drop.

The biggest issue for teams that want to acquire Price, as FanGraphs' Dave Cameron wrote Wednesday, is not his pitching. No, it's his expected $18 million to $20 million salary in his final year of arbitration in 2015.

Few clubs will want to part with high-end prospects while absorbing such a payroll hit. Then again, the numbers are relative. Price's salary next season will not be terribly above the qualifying offer for free agents, which is expected to be $15 million to $16 million.

The Phillies, despite their recent surge, surely recognize that they need to get younger. But even if the team has the will to be active sellers, it might not have a way.

Second baseman Chase Utley, after telling club officials last July that he did not want to be traded, signed a two-year extension with three club options. Now, less than a year later, he's going to reverse a course, waive his 10-and-5 rights and approve a deal?

Shortstop Jimmy Rollins, another 10-and-5 player, left open the possibility that he would approve a trade after breaking the team's all-time hit record last weekend. But realistically, where is Rollins going?

The shortstop's wife is from Philadelphia. They are the parents of two young children. And good luck finding the right fit.

Rollins' $11 million option for 2015 is almost certain to vest; he would not be just a rental. No, he would need to be comfortable spending the rest of this season in his new city, and all of next season as well.

Neither team in Rollins' native Bay Area needs a shortstop. And while Rollins might crave the spotlight in either New York or LA, he wouldn't go to the Mets, the Yankees aren't going to displace Derek Jeter and neither the Angels nor Dodgers figures to pursue a shortstop.

Then there is left-hander Cliff Lee, who - if all goes well - will return from his strained elbow before the All-Star break. By July 31, Lee still will have more than $45 million left on his contract, including a $12.5 million buyout for 2016.

The financial obligation number actually might be higher than that for many; Lee can block trades to 20 teams, and could require his $27.5 million vesting option for '16 to be guaranteed to join a club such as the Dodgers.

Phillies GM Ruben Amaro Jr. has said he would include cash in certain deals. The Phillies always could trade Lee during the August waiver period. But unless the team paid the majority of his contract, it could not expect a bounty in return.

Closer Jonathan Papelbon could get moved if the Phillies pay down his $13 million salary; he, too, has a vesting option for '16, and can only be traded to 12 teams without his approval.

Unless the Phils get truly creative, they will find it difficult to make impact moves.

I normally do not favor starting pitchers for MVP unless the field lacks strong position-player candidates; I voted for then-Red Sox outfielder Jacoby Ellsbury in 2011 over the eventual winner, Tigers right-hander Justin Verlander.

An interesting MVP case, however, can be made for Yankees right-hander Masahiro Tanaka, at least to this point of the season.

The Yankees are 12-2 when Tanaka starts, 25-31 when he does not. Even more striking, their run differential in Tanaka's starts is plus-33, while in all other games it's minus-54.

In other words, the Yankees are worse than the Rays (minus-48) when Tanaka isn't pitching and nearly as bad as the Padres (minus-62) and Diamondbacks (minus-64).

Of course, Wins Above Replacement (WAR) tells a different story. Tanaka is third among pitchers at 2.9 in the FanGraphs version of the metric, and well behind Mike Trout, the leader among position players at 4.7.

Outfielder J.D. Martinez is one of the Tigers' few recent bright spots; he has three homers during his nine-game hitting streak, and is now batting .300 with a .903 OPS in 108 plate appearances on the season.

Martinez, 26, spent the offseason watching video of Miguel Cabrera, Ryan Braun and other accomplished hitters, then completely overhauled his swing and approach. The Astros released him in spring training, and he signed a minor-league contract with the Tigers two days later.

"If you watch video of me in the past, it's the complete opposite - it's that extreme," he said. "It's kind of like I re-invented myself."

Martinez worked with his personal hitting coach for three weeks, then implemented his new ideas in Venezuela. Mostly, he's trying to line up directly to the ball and keep the barrel of his bat in the zone longer - the way Cabrera does.
\end{lstlisting}

\newpage
\section{Full Figures}\label{sect:app:full_figures}
The title of each figure is of the format: "\textbf{Dataset / Scoring Model / Generating Model}". The $x$-axis is the temperature and hues correspond to different mixing ratio of human-generated and LLM-generated texts. From left to right, the proportion of LLM-generated texts is: (0\%, 25\%, 50\%, 75\%, 100\%). The purpose of including these is to examine how sensitive fractal parameters are when LLM-generated texts are mixed with natural language. We observe that all fractal parameter indeed vary smoothly. 

\begin{figure}[H]
    \centering
    \includegraphics[width=\columnwidth]{figures/full/full_plots_BigP_Gemini1.0Pro_Gemini1.0Pro.pdf}
    \includegraphics[width=\columnwidth]{figures/full/full_plots_BigP_Gemini1.0Pro_Mistral_7B.pdf}
    \includegraphics[width=\columnwidth]{figures/full/full_plots_BigP_Gemini1.0Pro_Gemma_2B.pdf}
    % \caption{Gemini scoring, BigP}
    \label{fig:full_gemini_bigp}
\end{figure}

\begin{figure}[H]
    \centering
    \includegraphics[width=\columnwidth]{figures/full/full_plots_Bills_Gemini1.0Pro_Gemini1.0Pro.pdf}
    \includegraphics[width=\columnwidth]{figures/full/full_plots_Bills_Gemini1.0Pro_Mistral_7B.pdf}
    \includegraphics[width=\columnwidth]{figures/full/full_plots_Bills_Gemini1.0Pro_Gemma_2B.pdf}
    % \caption{Gemini scoring, Bills}
    \label{fig:full_gemini_bills}
\end{figure}

\begin{figure}[H]
    \centering
    \includegraphics[width=\columnwidth]{figures/full/full_plots_News_Gemini1.0Pro_Gemini1.0Pro.pdf}
    \includegraphics[width=\columnwidth]{figures/full/full_plots_News_Gemini1.0Pro_Mistral_7B.pdf}
    \includegraphics[width=\columnwidth]{figures/full/full_plots_News_Gemini1.0Pro_Gemma_2B.pdf}
    % \caption{Gemini scoring, News}
    \label{fig:full_gemini_news}
\end{figure}

\begin{figure}[H]
    \centering
    \includegraphics[width=\columnwidth]{figures/full/full_plots_Science_Gemini1.0Pro_Gemini1.0Pro.pdf}
    \includegraphics[width=\columnwidth]{figures/full/full_plots_Science_Gemini1.0Pro_Mistral_7B.pdf}
    \includegraphics[width=\columnwidth]{figures/full/full_plots_Science_Gemini1.0Pro_Gemma_2B.pdf}
    % \caption{Gemini scoring, Science}
    \label{fig:full_gemini_science}
\end{figure}

\begin{figure}[H]
    \centering
    \includegraphics[width=\columnwidth]{figures/full/full_plots_Wiki_Gemini1.0Pro_Gemini1.0Pro.pdf}
    \includegraphics[width=\columnwidth]{figures/full/full_plots_Wiki_Gemini1.0Pro_Mistral_7B.pdf}
    \includegraphics[width=\columnwidth]{figures/full/full_plots_Wiki_Gemini1.0Pro_Gemma_2B.pdf}
    % \caption{Gemini scoring, Wiki}
    \label{fig:full_gemini_wiki}
\end{figure}


\begin{figure}[H]
    \centering
    \includegraphics[width=\columnwidth]{figures/full/full_plots_BigP_Mistral_7B_Gemini1.0Pro.pdf}
    \includegraphics[width=\columnwidth]{figures/full/full_plots_BigP_Mistral_7B_Mistral_7B.pdf}
    \includegraphics[width=\columnwidth]{figures/full/full_plots_BigP_Mistral_7B_Gemma_2B.pdf}
    % \caption{MISTRAL_7B scoring, BigP}
    \label{fig:full_mistral_bigp}
\end{figure}

\begin{figure}[H]
    \centering
    \includegraphics[width=\columnwidth]{figures/full/full_plots_Bills_Mistral_7B_Gemini1.0Pro.pdf}
    \includegraphics[width=\columnwidth]{figures/full/full_plots_Bills_Mistral_7B_Mistral_7B.pdf}
    \includegraphics[width=\columnwidth]{figures/full/full_plots_Bills_Mistral_7B_Gemma_2B.pdf}
    % \caption{MISTRAL_7B scoring, Bills}
    \label{fig:full_mistral_bills}
\end{figure}

\begin{figure}[H]
    \centering
    \includegraphics[width=\columnwidth]{figures/full/full_plots_News_Mistral_7B_Gemini1.0Pro.pdf}
    \includegraphics[width=\columnwidth]{figures/full/full_plots_News_Mistral_7B_Mistral_7B.pdf}
    \includegraphics[width=\columnwidth]{figures/full/full_plots_News_Mistral_7B_Gemma_2B.pdf}
    % \caption{MISTRAL_7B scoring, News}
    \label{fig:full_mistral_news}
\end{figure}

\begin{figure}[H]
    \centering
    \includegraphics[width=\columnwidth]{figures/full/full_plots_Science_Mistral_7B_Gemini1.0Pro.pdf}
    \includegraphics[width=\columnwidth]{figures/full/full_plots_Science_Mistral_7B_Mistral_7B.pdf}
    \includegraphics[width=\columnwidth]{figures/full/full_plots_Science_Mistral_7B_Gemma_2B.pdf}
    % \caption{MISTRAL_7B scoring, Science}
    \label{fig:full_mistral_science}
\end{figure}

\begin{figure}[H]
    \centering
    \includegraphics[width=\columnwidth]{figures/full/full_plots_Wiki_Mistral_7B_Gemini1.0Pro.pdf}
    \includegraphics[width=\columnwidth]{figures/full/full_plots_Wiki_Mistral_7B_Mistral_7B.pdf}
    \includegraphics[width=\columnwidth]{figures/full/full_plots_Wiki_Mistral_7B_Gemma_2B.pdf}
    % \caption{MISTRAL_7B scoring, Wiki}
    \label{fig:full_mistral_wiki}
\end{figure}



\begin{figure}[H]
    \centering
    \includegraphics[width=\columnwidth]{figures/full/full_plots_BigP_Gemma_2B_Gemini1.0Pro.pdf}
    \includegraphics[width=\columnwidth]{figures/full/full_plots_BigP_Gemma_2B_Mistral_7B.pdf}
    \includegraphics[width=\columnwidth]{figures/full/full_plots_BigP_Gemma_2B_Gemma_2B.pdf}
    % \caption{GEMMA_2B scoring, BigP}
    \label{fig:full_gemma_bigp}
\end{figure}

\begin{figure}[H]
    \centering
    \includegraphics[width=\columnwidth]{figures/full/full_plots_Bills_Gemma_2B_Gemini1.0Pro.pdf}
    \includegraphics[width=\columnwidth]{figures/full/full_plots_Bills_Gemma_2B_Mistral_7B.pdf}
    \includegraphics[width=\columnwidth]{figures/full/full_plots_Bills_Gemma_2B_Gemma_2B.pdf}
    % \caption{GEMMA_2B scoring, Bills}
    \label{fig:full_gemma_bills}
\end{figure}

\begin{figure}[H]
    \centering
    \includegraphics[width=\columnwidth]{figures/full/full_plots_News_Gemma_2B_Gemini1.0Pro.pdf}
    \includegraphics[width=\columnwidth]{figures/full/full_plots_News_Gemma_2B_Mistral_7B.pdf}
    \includegraphics[width=\columnwidth]{figures/full/full_plots_News_Gemma_2B_Gemma_2B.pdf}
    % \caption{GEMMA_2B scoring, News}
    \label{fig:full_gemma_news}
\end{figure}

\begin{figure}[H]
    \centering
    \includegraphics[width=\columnwidth]{figures/full/full_plots_Science_Gemma_2B_Gemini1.0Pro.pdf}
    \includegraphics[width=\columnwidth]{figures/full/full_plots_Science_Gemma_2B_Mistral_7B.pdf}
    \includegraphics[width=\columnwidth]{figures/full/full_plots_Science_Gemma_2B_Gemma_2B.pdf}
    % \caption{GEMMA_2B scoring, Science}
    \label{fig:full_gemma_science}
\end{figure}

\begin{figure}[H]
    \centering
    \includegraphics[width=\columnwidth]{figures/full/full_plots_Wiki_Gemma_2B_Gemini1.0Pro.pdf}
    \includegraphics[width=\columnwidth]{figures/full/full_plots_Wiki_Gemma_2B_Mistral_7B.pdf}
    \includegraphics[width=\columnwidth]{figures/full/full_plots_Wiki_Gemma_2B_Gemma_2B.pdf}
    % \caption{GEMMA-2B scoring, Wiki}
    \label{fig:full_gemma_wiki}
\end{figure}