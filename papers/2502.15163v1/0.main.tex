\documentclass[lettersize,journal]{IEEEtran}
\usepackage{amsmath,amsfonts}
\usepackage{algorithmic}
\usepackage{algorithm}
\usepackage{array}
\usepackage[caption=false,font=normalsize,labelfont=sf,textfont=sf]{subfig}
\usepackage{textcomp}
\usepackage{stfloats}
\usepackage{url}
\usepackage{verbatim}
\usepackage{graphicx}
\usepackage{cite}
\usepackage{hyperref}
\usepackage{makecell}
\usepackage{multirow}
\usepackage{colortbl}
\usepackage{rotating}
\usepackage{ragged2e}
\usepackage{tablefootnote}
\usepackage{hhline}
\usepackage[switch]{lineno}
\newtheorem{lemma}{\bf Lemma}
\newtheorem{theorem}{\bf Theorem}
\hyphenation{op-tical net-works semi-conduc-tor IEEE-Xplore}
% updated with editorial comments 8/9/2021

\begin{document}
% \linenumbers

\title{HOpenCls: Training Hyperspectral Image Open-Set Classifiers in Their Living Environments}

\author{Hengwei~Zhao, Xinyu~Wang, Zhuo~Zheng, Jingtao~Li, Yanfei~Zhong% <-this % stops a space

\thanks{This work was supported by the National Natural Science Foundation of China under Grant No. 42325105.}
% \textit{(Corresponding author: Yanfei Zhong, Xinyu Wang)}}% <-this % stops a space
\thanks{Hengwei Zhao, Jingtao Li and Yanfei Zhong are with the State Key Laboratory of Information Engineering in Surveying, Mapping and Remote Sensing, Wuhan University, China (e-mail:whu\_zhaohw@whu.edu.cn; jingtaoli@whu.edu.cn; zhongyanfei@whu.edu.cn).}
\thanks{Zhuo Zheng is with the Department of Computer Science, Stanford University, United States (e-mail: zhuozheng@cs.stanford.edu).}
\thanks{Xinyu Wang is with the School of Remote Sensing and Information Engineering, Wuhan University, China (e-mail: wangxinyu@whu.edu.cn).}
}

% The paper headers
\markboth{Journal of \LaTeX\ Class Files,~Vol.~14, No.~8, August~2021}%
{Shell \MakeLowercase{\textit{et al.}}: A Sample Article Using IEEEtran.cls for IEEE Journals}

% \IEEEpubid{0000--0000/00\$00.00~\copyright~2021 IEEE}
% Remember, if you use this you must call \IEEEpubidadjcol in the second
% column for its text to clear the IEEEpubid mark.

\maketitle

\begin{abstract}
Hyperspectral image (HSI) open-set classification is critical for HSI classification models deployed in real-world environments, where classifiers must simultaneously classify known classes and reject unknown classes. Recent methods utilize auxiliary unknown classes data to improve classification performance. However, the auxiliary unknown classes data is strongly assumed to be completely separable from known classes and requires labor-intensive annotation. To address this limitation, this paper proposes a novel framework, \textit{HOpenCls}, to leverage the unlabeled wild data---that is the mixture of known and unknown classes. Such wild data is abundant and can be collected freely during deploying classifiers in their \textit{living environments}. The key insight is reformulating the open-set HSI classification with unlabeled wild data as a positive-unlabeled (PU) learning problem. Specifically, the multi-label strategy is introduced to bridge the PU learning and open-set HSI classification, and then the proposed gradient contraction and gradient expansion module to make this PU learning problem tractable from the observation of abnormal gradient weights associated with wild data. Extensive experiment results demonstrate that incorporating wild data has the potential to significantly enhance open-set HSI classification in complex real-world scenarios.
\end{abstract}

\begin{IEEEkeywords}
Hyperspectral image classification, Open-set classification, Positive-unlabeled learning.
\end{IEEEkeywords}

\section{Introduction}
\IEEEPARstart{H}{yperspectral} image (HSI) can record the spectral characteristics of ground objects~\cite{6555921}. As a key technology of HSI processing, HSI classification is aimed at assigning a unique category label to each pixel based on the spectral and spatial characteristics of this pixel~\cite{FPGA,10078841,10696913,10167502}, which is widely used in agriculture~\cite{WHU-Hi}, forest~\cite{ITreeDet}, city~\cite{WANG2022113058}, ocean~\cite{WHU-Hi} studies and so on.

The existing HSI classifiers~\cite{FPGA,10325566,10047983,9573256} typically assume the closed-set setting, where all HSI pixels are presumed to belong to one of the \textit{known} classes. However, due to the practical limitations of field investigations across wide geographical areas and the high annotation costs associated with the limited availability of domain experts, it is inevitable to have outliers in the vast study area~\cite{MDL4OW,Fang_OpenSet,Kang_OpenSet}. These outliers do not belong to any known classes and will be referred as \textit{unknown} classes hereafter. A classifier based on closed-set assumption will misclassify the unknown class as one of the known classes. For example, in the University of Pavia HSI dataset (Fig.~\ref{fig:open_set_example}), objects such as vehicles, buildings with red roofs, carports, and swimming pools are ignored from the original annotations~\cite{MDL4OW}. These objects are misclassified as one of predefined known classes.

\begin{figure}[!t]
    \centering
    \includegraphics[width=0.98\columnwidth]{example_paviau.png}
    \caption{Comparison of classification results between closed-set based classifier and open-set based classifier for the University of Pavia dataset. The dataset originally contains nine \textit{known} land cover classes, however, significant misclassifications occur in the \textit{unknown} classes in closed-set based results. For instance, these unknown buildings with red roofs are misclassified as Bare S., Meadows, and other known materials by closed-set based classifier~\cite{FPGA}. Note that there is a significant overlap in the distribution of spectral curves between known and unknown classes in HSI datasets, which poses a major problem to open-set HSI classification.}
    \label{fig:open_set_example}
\end{figure}

Open-set classification (Fig.~\ref{fig:open_set_example}), as a critical task for safely deploying models in real-world scenarios, addresses the above problem by accurately classifying known class samples and rejecting unknown class outliers~\cite{OpenMax,MDL4OW,Fang_OpenSet}. Moreover, the recent advanced researches have explored training with an auxiliary unknown classes dataset to regularize the classifiers to produce lower confidence~\cite{Entropy,WOODS} or higher energies~\cite{Energy} on these unknown classes samples.

Despite its promise, there are some limitations when open-set classification meets HSI. First, the limited number of training samples, combined with significant spectral overlap between known and unknown classes (see Fig.~\ref{fig:open_set_example}), causes the classifier to overfit on the training samples. Second, the distribution of the auxiliary unknown classes dataset may not align well with the distribution of real-world unknown classes, potentially leading to the misclassification of the test-time data. Finally, it is labor-intensive to ensure the collected extra unknown classes dataset does not overlap with the known classes.

To mitigate these limitations, this paper leverages unlabeled ``in-the-wild'' hyperspectral data (referred to as ``wild data''), which can be collected \textit{freely} during deploying HSI classifiers in the open real-world environments, and has been largely neglected for open-set HSI classification purposes. Such data is abundant, has a better match to the test-time distribution than the collected auxiliary unknown classes dataset, and does not require any annotation workloads. Moreover, the information about unknown classes stored in the wild data can be leveraged to promote the rejection of unknown classes in the case of spectral overlap. While leveraging wild data naturally suits open-set HSI classification, it also poses a unique challenge: wild data is not pure and consists of both known and unknown classes. This challenge originates from the marginal distribution of wild data, which can be modeled by the Huber contamination model~\cite{Huber}:
\begin{equation}
    \mathbb{P}_{wild}=\pi\mathbb{P}_{k}+(1-\pi)\mathbb{P}_{u},
    \label{eq:huber_contamination_model}
\end{equation}
where $\mathbb{P}_{k}$ and $\mathbb{P}_{u}$ represent the distributions of known and unknown classes, respectively. Here, $\pi=\pi_{1}+\dots+\pi_{C}$, and $\pi_{c}$ refers to the probability (or class prior~\cite{DistPU}) of the known class $c \in [1,C]$ in $\mathbb{P}_{wild}$.
The known component of wild data acts as noise, potentially disrupting the training process (further analysis can be found in Section~\ref{sec:Methodology}). 

\begin{center}
    \fbox{\begin{minipage}{23em}
        This paper aims to propose a novel framework---\textit{HOpenCls}---to effectively leverage wild data for open-set HSI classification. Wild data is easily available as it's naturally generated during classifier deployment in real-world environments. This framework can be regarded as training open-set HSI classifiers in their \textit{living environments}.
    \end{minipage}}
\end{center}

To handle the lack of ``clean'' unknown classes datasets, the insight of this paper is to formulate a positive-unlabeled (PU) learning problem~\cite{DistPU,T-HOneCls} in the rejection of unknown classes: learning a binary classifier to classify positive (known) and negative (unknown) classes only from positive and unlabeled (wild) data. What's more, the high intra-class variance of positive class and the high class prior of positive class are potential factors that limit the ability of PU learning methods to address unknown class rejection task. To overcome these limitations, the multi-label strategy is introduced to the \textit{HOpenCls} to decouple the original unknown classes rejection task into multiple sub-PU learning tasks, where the $c$-th sub-PU learning task is responsible for classifying the known class $c$ against all other classes. Compared to the original unknown classes rejection task, each sub-PU learning task exhibits reduced intra-class variance and class prior in the positive class.

Beyond the mathematical reformulation, a key contribution of this paper is a novel PU learning method inspired by the abnormal gradient weights found in wild data. First of all, when the auxiliary unknown classes dataset is replaced by the wild data, this paper demonstrates that the adverse effects impeding the rejection of unknown classes originate from the larger gradient weights associated with the component of known classes in the wild data. Therefore, a gradient contraction (Grad-C) module is designed to reduce the gradient weights associated with all training wild data, and then, the gradient weights of wild unknown samples are recovered by the gradient expansion (Grad-E) module to enhance the fitting capability of the classifier. Compared to other PU learning methods~\cite{nnPU,DistPU,PUET,HOneCls}, the combination of Grad-C and Grad-E modules provides the capability to reject unknown classes in a class prior-free manner. Given the spectral overlap characteristics in HSI, estimating class priors for each known class is highly challenging~\cite{T-HOneCls}, and the class prior-free PU learning method is more suitable for open-set HSI classification.

Extensive experiments have been conducted to evaluate the proposed \textit{HOpenCls}. For thorough comparison, two groups of methods are compared: (1) trained with only $\mathbb{P}_{k}$ data, and (2) trained with both $\mathbb{P}_{k}$ data and an additional dataset. The experimental results demonstrate that the proposed framework substantially enhances the classifier's ability to reject unknown classes, leading to a marked improvement in open-set HSI classification performance. Taking the challenging WHU-Hi-HongHu dataset as an example, \textit{HOpenCls} boosts the overall accuracy in open-set classification (Open OA) by 8.20\% compared to the strongest baseline, with significantly improving the metric of unknown classes rejection (F1\textsuperscript{U}) by 38.91\%. The key contributions of this paper can be summarized as follows:
\begin{itemize}
    \item[1)] This paper proposes a novel framework, \textit{HOpenCls}, for open-set HSI classification, designed to effectively leverage wild data. To the best of our knowledge, this paper pioneers the exploration of PU learning for open-set HSI classification.
    \item[2)] The multi-PU head is designed to incorporate the multi-label strategy into \textit{HOpenCls}, decoupling the original unknown classes rejection task into multiple sub-PU learning tasks. As demonstrated in the experimental section, the multi-PU strategy is crucial for bridging PU learning with open-set HSI classification.
    \item[3)] The Grad-C and Grad-E modules, derived from the theoretical analysis of abnormal gradient weights, are proposed for the rejection of unknown classes. The combination of these modules forms a novel class prior-free PU learning method.
    \item[4)] Extensive comparisons and ablations are conducted across: (1) a diverse range of datasets, and (2) varying assumptions about the relationship between the auxiliary dataset distribution and the test-time distribution. The proposed \textit{HOpenCls} achieves state-of-the-art performance, demonstrating significant improvements over existing methods.
\end{itemize}

\section{Related Works}
\subsection{Deep Learning Based HSI Classification}

According to the different learning paradigms, deep learning-based HSI classification approaches can be categorized into patch-based methods and patch-free methods~\cite{FPGA}. Patch-based methods focus on modeling local spectral-spatial information mapping functions:
\begin{equation}
    P_{pb}:R^{S{\times}S}{\rightarrow}R,
    \label{eq:patch_based_framework}
\end{equation}
where a neural network is trained with patches of size $S{\times}S$ extracted from the HSI imagery~\cite{10400402,10050427,9785505}. In contrast, patch-free methods aim to capture global spectral-spatial information mapping functions:
\begin{equation}
    P_{pf}:R^{H{\times}W{\times}C}{\rightarrow}R^{H{\times}W},
    \label{eq:patch_free_framework}
\end{equation}
where $H{\times}W$ represents the spatial size of the HSI imagery and $C$ denotes the number of spectral bands~\cite{8737729,HU2022147,9347487}.
Due to avoiding the utilization of patches, compared with patch-based methods, the GFLOPs of patch-free methods significantly decrease, and the inference time of patch-free methods has significantly improved hundreds of times~\cite{FPGA}.

Most of the existing HSI classifiers are based on the closed-set assumption, which is designed to classify known classes. In contrast, this work focuses on open-set HSI classification, which extends the capability of closed-set based HSI classifiers to not only classify known classes but also reject unknown classes.

\subsection{Open-Set Classification}

The objective of open-set classification is to simultaneously classify known classes and reject unknown classes~\cite{9040673}. Compared to closed-set classification, open-set classification is more challenging due to the incomplete supervision available for rejecting unknown classes~\cite{9857485}. Therefore, this section reviews open-set classification methods from the perspective of unknown classes rejection.

A rich line of method focuses on designing scoring functions for detecting unknown classes, such as the maximum predicted softmax probability (MSP)~\cite{MSP}, OpenMax~\cite{OpenMax}, ODIN score~\cite{ODIN}, Energy score~\cite{Energy}, Entropy score~\cite{Entropy}, MaxLogit score~\cite{KLMatching}, and KL Matching~\cite{KLMatching}. Recent researches have demonstrated that reconstruction loss~\cite{8953952,MDL4OW,Fang_OpenSet} and prototype distance~\cite{Kang_OpenSet,9296325,CACLoss,10415443} can also serve as metrics for rejecting unknown classes. However, these methods are typically trained with known data, and this paper demonstrates that a more robust open-set classifier can be achieved by incorporating naturally occurring wild data from the real-world environments, which can be collected freely.

Another line of research tries to reject unknown classes by using regularization during training~\cite{OE,EOS,Energy,Entropy,WOODS}. These methods typically require an auxiliary unknown classes dataset that is disjoint from $\mathbb{P}_{k}$. For example, models are encouraged to produce lower confidence~\cite{OE,Entropy} or higher energy scores~\cite{Energy} for these auxiliary unknown samples. Similar to this paper, WOODS~\cite{WOODS} also tries to leverage wild data from $\mathbb{P}_{wild}$ by formulating a constrained optimization problem. However, the performance of WOODS is limited by the scarcity of training samples and the significant spectral overlap in HSI.

Different from the abovementioned, this work pioneers the exploration of addressing the open-set HSI classification problem from the perspective of PU learning. Extensive comparisons and ablations demonstrate the clear superiority of the proposed framework.

\subsection{PU Learning}

Early PU learning methods rely on the two-step approach~\cite{FOODY20061,Gong_Wang_Ye_Xu_Lin_2018}, which first extracts reliable negative samples from unlabeled data, and then trains a supervised binary classifier by these positive and selected negative samples. However, the performance of these two-step classifiers is constrained by the reliability of the selected negative samples.

Recent research has shifted towards addressing PU learning using one-step approaches, such as cost-sensitive learning~\cite{9201373,LU2021112584}, label disambiguation~\cite{ijcai2019p590}, and density ratio estimation~\cite{kato2018learning}. What's more, the risk estimation-based methods have proven to be some of the most theoretically and practically effective~\cite{nnPU,ITreeDet,LI2022102947,HOneCls,PUET,DistPU}. However, most of these methods assume that the class prior is known beforehand, which is actually difficult to estimate due to the spectral overlap characteristic in HSI~\cite{T-HOneCls}.

Several researches are striving towards the PU learning without class prior. The generator in the generative adversarial network is replaced by a classifier to learn from PU data in PAN~\cite{PAN}. vPU~\cite{vPU} and T-HOneCls~\cite{T-HOneCls} formalize the PU learning task as the variational problem. P3Mix~\cite{p3mix} proposes a heuristic mixup approach to select partners for positive samples from unlabeled data.

In contrast to these methods, this paper focuses on the challenge of open-set classification. Additionally, in the aspect of class prior-free PU learning, this paper originates from a nontrivial perspective: the abnormal gradient weights associated with wild data.

\section{Methodology}\label{sec:Methodology}
\section{Pivot-based Single Model Ensemble}
\label{sec:Pivot-based Single Model Ensemble}



In this section, we first introduce the overview of \ours framework (\S\ref{sec:overview}).
Then, we describe the candidate generation process through pivot translation (\S\ref{sec:pivot-based candidate generation}) and the aggregation process (\S\ref{sec:candidate aggregation}).


\subsection{Overview}
\label{sec:overview}

Our objective is the same as that of conventional translation tasks: converting the given source language sentence $x$ into the target language sentence $\hat{y}$.
\ours consists of two steps: candidate generation and candidate aggregation.
Figure~\ref{fig:overall} illustrates an overview of the proposed ensemble framework.


As the first step, we input $x$ to generate candidates through a single multilingual NMT model.
One translation path could be directly translating from the source to the target through the source$\rightarrow$target path.
Alternatively, pivot translations can be achieved by employing high-resource pivot languages, enabling translation paths from source$\rightarrow$pivot and pivot$\rightarrow$target.
During the pivot process, leveraging abundant parallel data enables knowledge transfer from high-resource pivot languages, thereby facilitating the generation of diverse and more accurate translations.
Through these $n$ paths, we can obtain a candidate pool $C = \{c_1, ..., c_n\}$ composed of $n$ candidates in the target language, employing only a single model.

As the second step, a ranking process is first conducted within the candidate pool $C$ since not all candidates contribute to the ensemble.
Using the estimated quality of each candidate, we select the top-$\textit{k}$ candidates.
We then generate the final output $\hat{y}$ using the selected high-quality candidates.
This generation-based approach facilitates the production of outputs superior to existing candidates.


\subsection{Pivot-based Candidate Generation}
\label{sec:pivot-based candidate generation}


In the first step, \ours takes a source sentence $x$ as input and generates $n$ candidates.
Direct translation yields only one candidate, whereas pivot translation enables the generation of multiple candidates from a single source sentence using a single model.
Generating candidates through pivot translation has two major advantages: diversity and quality.


First, we can obtain diverse candidates that can act complementarily.
One of the key principles for the ensemble is that the participants must be sufficiently diverse to provide various inductive biases.
In \ours, each source sentence is translated diversely by passing through multiple translation paths.
Diverse translation paths enhance the likelihood of providing expressions that convey the accurate meaning of the source sentence.
Pivot-based candidate generation shares a similar goal with a previous study that generates paraphrases through round-trip translation, aiming to generate diverse translations~\cite{thompson-post-2020-paraphrase}.


Second, by utilizing a parallel corpus of high-resource pivot languages, pivoting enables more accurate translations.
For low-resource language pairs, more appropriate translations can be achieved through two-step decoding through a pivot language~\cite{he-etal-2022-tencent}.
Moreover, leveraging pivot languages with abundant parallel data, not limited to English, allows us to obtain better translations~\cite{paul2009importance, dabre-etal-2015-leveraging}.


In addition, pivot translation with a single model offers practical benefits over employing multiple models. 
Firstly, it can reduce the costs of operating multiple models including LLMs. 
Secondly, the substantial performance disparities among models mean that using the top-performing single model for candidate generation often leads to higher-quality outcomes. 
Lastly, it reduces inference latency by using a single model for two batched inferences, while multi-model ensembles require up to 11, causing significant overhead and limiting real-time response capability.
Given that pivot translation with a single model allows for the creation of diverse and more accurate translations, we utilize an MNMT model to generate the candidates.


\minisection{Selecting pivot languages}
For each language pair, we carefully select pivot languages based on the assumption that pivot language with abundant mutual knowledge would allow us to obtain higher-quality candidates.
We select $n$ top-performing paths for our study based on BLEU scores on the FLORES-200 benchmark~\cite{nllb}.
We evaluate the outputs for each path, including direct translation and through various pivot translations.
\nllb~\cite{nllb} is used to generate candidates, and results on the FLORES-200 for selecting translation paths are in Appendix~\ref{sec:apdx_top4 pivot langauges}.
If pivot languages are selected based on BLEU scores, high-resource languages are predominantly chosen, rather than low-resource ones.
The experiments detailed in Appendix~\ref{apdx:resource level of pivot languages} demonstrate that overly prioritizing diversity by employing low-resource pivot languages, at the expense of candidate quality, does not result in improvements in the final translation.
The experiments comparing metrics for selecting translation paths are in Appendix~\ref{apdx: Metric for Selecting Translation Paths}.
As a result, we compose the candidate pool using the 4 paths.


\subsection{Candidate Aggregation}
\label{sec:candidate aggregation}


In the aggregation step, we take the candidate pool $C$ as input and output the merged final translation $\hat{y}$.
The post-hoc aggregation process encompasses two stages: selecting and merging.
In the first stage, we select candidates by ranking method.
There are two approaches for selecting candidates.
One approach evaluates each translation path and selects the best paths for all source sentences.
The other approach involves selecting the best top-$\textit{k}$ candidates for each source sentence.
After selecting $\textit{k}$ candidates, we generate the final translation $\hat{y}$ using the merging module.
This process enables the creation of better outputs beyond the quality of existing candidates.


\minisection{Selecting the top-$\textit{k}$ candidates}
The pivot language that generates the highest-quality candidate varies for each source sentence.
The best output is not guaranteed from one translation path alone, as it can vary depending on factors such as the size of the parallel corpus and the relationship between languages.
First, \ours uses QE to rank all $n$ candidates from candidate pool $C = \{c_1, ..., c_n\}$.
Afterward, we select top-$\textit{k}$ candidates among $n$ candidate pool.
Selecting the top-$\textit{k}$ candidates ensures the quality of the output by filtering out low-quality candidates while also efficiently reducing the cost during the merging process.
We use the reference-free COMETkiwi (\textit{wmt22-COMETkiwi-da}) \cite{rei2022cometkiwi} for ranking candidates.


\minisection{Generating the final translation}
To generate the final translation $\hat{y}$ by merging the top-$\textit{k}$ candidates, we explore methods from two categories: encoder-decoder ensemble architectures and LLM-based approach.
Employing encoder-decoder architectures during the merging process offers the advantage of relatively low training costs.
We conduct experiments using Fusion-in-Decoder (FiD)~\cite{fid} and TRICE~\cite{trice} architectures.
The former method involves passing 
\texttt{Translate source into <target language>}
\texttt{referring <target language> candidate.}
\texttt{source: <$x$>}
\texttt{candidate: <$c_{k}$>}
through the encoder, representations are concatenated and merged in the decoder.
The latter approach involves concatenating \texttt{<$x$></$s$><$l_{s}$>;<$c_{1}$></$s$><$l_{t}$>;...;<$c_{k}$></$s$><$l_{t}$>} with language token \texttt{<$l_{lang}$>} and providing it as input.
Encoder-decoder ensemble architectures are further described in detail in Appendix~\ref{apdx:Fid/TRICE illustration}.


On the other hand, the LLM-based ensemble implicitly leverages their translation capabilities during ensemble, as the source sentence is also provided.
We conduct merging experiments with \textsc{GenFuser}~\cite{llm-blender}, Llama-3~\cite{llama3modelcard}, and GPT models~\cite{gpt3.5, gpt4, gpt4o}. 
When employing \textsc{GenFuser}, we construct the input by concatenating top-\textit{$k$} candidates to the prompt, as presented in \citet{llm-blender}.
For merging with Llama-3 and GPT, we use the prompt template in Appendix~\ref{sec:apdx_prompt_templates}.
By leveraging a variety of candidates, each with different strengths, the aggregation process can effectively mitigate errors in a complementary manner.

\section{Experimental Results}
\subsection{Experimental Settings}
\noindent \textbf{Datasets:}
To evaluate the performance of different methods, four HSI datasets are utilized: the University of Pavia dataset (Fig.~\ref{fig:open_set_example}) and the WHU-Hi series datasets~\cite{WHU-Hi} (Fig.~\ref{fig:datasets}): WHU-Hi-HongHu, WHU-Hi-LongKou, and WHU-Hi-Hanchuan. Notably, the WHU-Hi series datasets present significant spectral overlap~\cite{WHU-Hi}, making open-set HSI classification particularly challenging.

In the University of Pavia dataset, the original 9 classes are treated as known classes, while pixels that differ significantly from these known classes, such as buildings with notable reflective differences, are labeled as unknown classes, following the settings in~\cite{MDL4OW}. For the WHU-Hi series datasets, certain crop categories are considered as known classes, while other categories are labeled as unknown. This aligns with practical crop mapping scenarios, where the labeling workload is substantially reduced by focusing on crop categories, minimizing the need for annotating non-crop classes.

To simulate limited training sample conditions, only 100 samples per class are selected from each dataset for model training, with an additional 4000 wild samples randomly drawn from the imagery. Detailed information about the datasets is provided in Table~\ref{tab:datasets}.

\begin{figure*}[!t]
    \centering
    \includegraphics[width=0.98\textwidth]{dataset.png}
    \caption{The WHU-Hi series HSI datasets: WHU-Hi-HongHu, WHU-Hi-LongKou, and WHU-Hi-HanChuan.}
    \label{fig:datasets}
\end{figure*}

\begin{table*}[ht]
    \footnotesize
    \centering
    \renewcommand{\arraystretch}{1.1} % Adjusts the row spacing
    \resizebox{16cm}{!} 
    { 
    \begin{tblr}{hline{1,2,Z} = 0.8pt, hline{3-Y} = 0.2pt,
                 colspec = {Q[l,m, 13em] Q[l,m, 6em] Q[c,m, 8em] Q[c,m, 5em] Q[l,m, 14em]},
                 colsep  = 4pt,
                 row{1}  = {0.4cm, font=\bfseries, bg=gray!30},
                 row{2-Z} = {0.2cm},
                 }
\textbf{Dataset}       & \textbf{Table Source} & \textbf{\# Tables / Statements} & \textbf{\# Words / Statement} & \textbf{Explicit Control}\\ 
\SetCell[c=5]{c} \textit{Single-sentence Table-to-Text}\\
ToTTo \cite{parikh2020tottocontrolledtabletotextgeneration}   & Wikipedia        & 83,141 / 83,141                  & 17.4                          & Table region      \\
LOGICNLG \cite{chen2020logicalnaturallanguagegeneration} & Wikipedia        & 7,392 / 36,960                  & 14.2                          & Table regions      \\ 
HiTab \cite{cheng-etal-2022-hitab}   & Statistics web   & 3,597 / 10,672                  & 16.4                          & Table regions \& reasoning operator \\ 
\SetCell[c=5]{c} \textit{Generic Table Summarization}\\
ROTOWIRE \cite{wiseman2017challengesdatatodocumentgeneration} & NBA games      & 4,953 / 4,953                   & 337.1                         & \textbf{\textit{X}}                   \\
SciGen \cite{moosavi2021scigen} & Sci-Paper      & 1,338 / 1,338                   & 116.0                         & \textbf{\textit{X}}                   \\
NumericNLG \cite{suadaa-etal-2021-towards} & Sci-Paper   & 1,355 / 1,355                   & 94.2                          & \textbf{\textit{X}}                    \\
\SetCell[c=5]{c} \textit{Table Question Answering}\\
FeTaQA \cite{nan2021fetaqafreeformtablequestion}     & Wikipedia      & 10,330 / 10,330                 & 18.9                          & Queries rewritten from ToTTo \\
\SetCell[c=5]{c} \textit{Query-Focused Table Summarization}\\
QTSumm \cite{zhao2023qtsummqueryfocusedsummarizationtabular}                        & Wikipedia      & 2,934 / 7,111                   & 68.0                          & Queries from real-world scenarios\\ 
\textbf{eC-Tab2Text} (\textit{ours})                           & e-Commerce products      & 1,452 / 3,354                   & 56.61                          & Queries from e-commerce products\\
    \end{tblr}
    }
\caption{Comparison between \textbf{eC-Tab2Text} (\textit{ours}) and existing table-to-text generation datasets. Statements and queries are used interchangeably. Our dataset specifically comprises tables from the e-commerce domain.}
\label{tab:datasets}
\end{table*}

\noindent \textbf{Compared Methods:}
Two groups of methods are compared for thorough comparison: (1) trained with only $\mathbb{P}_{k}$ data: OpenMax~\cite{OpenMax}, CAC Loss~\cite{CACLoss}, MDL4OW~\cite{MDL4OW}, KL Matching~\cite{KLMatching}, MSP~\cite{MSP}, Energy~\cite{Energy} and Entropy~\cite{Entropy}; (2) trained with both $\mathbb{P}_{k}$ data and an additional dataset: DS3L~\cite{DS3L}, WOODS~\cite{WOODS}, OE~\cite{OE} and EOS~\cite{EOS}. Specifically, in addition to wild data, pure unknown classes data from $\mathbb{P}_{u}$ and wild data excluding unknown classes from $\mathbb{P}_{wild}-\mathbb{P}_{u}$ are used as the auxiliary data to demonstrate the robustness of \textit{HOpenCls} with respect to varying auxiliary data sources. To ensure a fair comparison, an equal amount of data (4000) from both distributions in the auxiliary dataset is randomly selected for model training.

Moreover, \textit{HOpenCls} is also compared with recently proposed PU learning methods (HOneCls~\cite{HOneCls} and T-HOneCls~\cite{T-HOneCls}) to demonstrate its effectiveness in PU learning tasks.

\noindent \textbf{Training Details:}
For patch-free methods, FreeNet~\cite{FPGA} is used as the spectral-spatial feature extractor. In the patch-based approaches, the encoder of FreeNet is used for the same purpose, with patch sizes increased to 9 to counteract the underutilization of spatial information caused by smaller spatial blocks. The \textit{HOpenCls} model was trained by 130 epoches (lr=$3{\times}10^{-4}$, momentum=0.9, weight\_decay=$10^{-4}$) with a ``CosineAnnealingLR'' learning rate policy. The default values for the weight $\beta$ and the Taylor expansion order $o$ were set to $1$ and $2$, respectively.

\noindent \textbf{Metrics:}
Following standard practice, Overall Accuracy (OA) is used to evaluate classifier performance. OA with and without unknown classes is referred to as Open OA and Closed OA, respectively. To evaluate the model’s ability to reject unknown classes, the F1 score (F1\textsuperscript{u}) and the Area Under the Curve (AUC\textsuperscript{u}) are also reported. All experiments were repeated 5 times, and the mean and standard deviation values are reported.

\subsection{Main Results}

\begin{figure*}[!t]
    \centering
    \includegraphics[width=0.98\textwidth]{results_HH.png}
    \caption{Open-set classification maps of WHU-Hi-HongHu dataset.}
    \label{fig:result_HH}
\end{figure*}

% Please add the following required packages to your document preamble:

% Beamer presentation requires \usepackage{colortbl} instead of \usepackage[table,xcdraw]{xcolor}
\begin{table*}[t]
\centering
\caption{Main Results. Eurus-2-7B-PRIME demonstrates the best reasoning ability.}
\label{tab:main_results}
\resizebox{\textwidth}{!}{
\begin{tabular}{lcccccc}
\toprule
\textbf{Model}                     & \textbf{AIME 2024}                           & \textbf{MATH-500} & \textbf{AMC}          & \textbf{Minerva Math} & \textbf{OlympiadBench} & \textbf{Avg.}          \\ \midrule
\textbf{GPT-4o}                    & 9.3                                          & 76.4              & 45.8                  & 36.8                  & \textbf{43.3}          & 43.3                   \\
\textbf{Llama-3.1-70B-Instruct}    & 16.7                                         & 64.6              & 30.1                  & 35.3                  & 31.9                   & 35.7                   \\
\textbf{Qwen-2.5-Math-7B-Instruct} & 13.3                                         & \textbf{79.8}     & 50.6                  & 34.6                  & 40.7                   & 43.8                   \\
\textbf{Eurus-2-7B-SFT}            & 3.3                                          & 65.1              & 30.1                  & 32.7                  & 29.8                   & 32.2                   \\
\textbf{Eurus-2-7B-PRIME}          & \textbf{26.7 {\color[HTML]{009901} (+23.3)}} & 79.2 {\color[HTML]{009901}(+14.1)}      & \textbf{57.8 {\color[HTML]{009901}(+27.7)}} & \textbf{38.6 {\color[HTML]{009901}(+5.9)}}  & 42.1 {\color[HTML]{009901}(+12.3) }          & \textbf{48.9 {\color[HTML]{009901}(+ 16.7)}} \\ \bottomrule
\end{tabular}
}
\end{table*}

\begin{table*}[!t]
    \centering
    \caption{Closed OA for different methods. ±$x$ denotes the standard error.}
    \label{tab:closed_OA}
    \resizebox{\linewidth}{!}{%
    \begin{tabular}{l|ccccccc|ccccc} 
    \hline
    \multicolumn{1}{c|}{\multirow{3}{*}{Datasets}} & \multicolumn{7}{c|}{{\cellcolor[rgb]{0.753,0.753,0.753}}With known classes data from~$\mathbb{P}_{k}$only}                                                                                                                              & \multicolumn{5}{c}{{\cellcolor[rgb]{0.753,0.753,0.753}}With known classes data from~$\mathbb{P}_{k}$~and wild data from~$\mathbb{P}_{wild}$}                                                         \\
    \multicolumn{1}{c|}{}                          & OpenMax                     & CAC Loss                    & MDL4OW                      & KL Matching                 & MSP                         & Energy                      & Entropy                     & DS3L                        & WOODS                       & OE                          & EOS                         & \textit{HOpenCls} (ours)                              \\
    \multicolumn{1}{c|}{}                          & $P_{pb}$                    & $P_{pb}$                    & $P_{pb}$                    & $P_{pb}$                    & $P_{pf}$                    & $P_{pf}$                    & $P_{pf}$                    & $P_{pb}$                    & $P_{pb}$                    & $P_{pf}$                    & $P_{pf}$                    & $P_{pf}$                              \\ 
    \hline
    WHU-Hi-HongHu                                  & 93.78\textsuperscript{±0.6} & 94.64\textsuperscript{±0.1} & 95.93\textsuperscript{±0.1} & 93.78\textsuperscript{±0.6} & 98.23\textsuperscript{±0.1} & 98.23\textsuperscript{±0.1} & 98.23\textsuperscript{±0.1} & 93.93\textsuperscript{±0.2} & 90.98\textsuperscript{±1.6} & 96.68\textsuperscript{±0.2} & 96.79\textsuperscript{±0.3} & \textbf{98.57}\textsuperscript{±0.2}  \\
    WHU-Hi-LongKou                                 & 96.97\textsuperscript{±0.4} & 96.99\textsuperscript{±0.3} & 98.64\textsuperscript{±0.1} & 96.97\textsuperscript{±0.4} & 99.37\textsuperscript{±0.1} & 99.37\textsuperscript{±0.1} & 99.37\textsuperscript{±0.1} & 97.17\textsuperscript{±0.4} & 94.89\textsuperscript{±1.9} & 94.46\textsuperscript{±0.7} & 95.14\textsuperscript{±1.1} & \textbf{99.75}\textsuperscript{±0.0}  \\
    WHU-Hi-HanChuan                                & 94.37\textsuperscript{±0.4} & 95.42\textsuperscript{±0.5} & 96.74\textsuperscript{±0.3} & 94.37\textsuperscript{±0.4} & 98.50\textsuperscript{±0.2} & 98.50\textsuperscript{±0.2} & 98.50\textsuperscript{±0.2} & 95.03\textsuperscript{±0.4} & 93.59\textsuperscript{±2.0} & 98.28\textsuperscript{±0.2} & 98.17\textsuperscript{±0.7} & \textbf{99.77}\textsuperscript{±0.1}  \\
    University of Pavia                            & 96.70\textsuperscript{±0.6} & 98.21\textsuperscript{±0.2} & 98.72\textsuperscript{±0.3} & 96.70\textsuperscript{±0.6} & 99.65\textsuperscript{±0.1} & 99.65\textsuperscript{±0.1} & 99.65\textsuperscript{±0.1} & 97.85\textsuperscript{±0.2} & 96.31\textsuperscript{±1.4} & 99.25\textsuperscript{±0.2} & 99.27\textsuperscript{±0.1} & \textbf{99.71}\textsuperscript{±0.1}  \\
    \hline
    \end{tabular}
    }
\end{table*}

\noindent \textbf{\textit{HOpenCls} Achieves Superior Performance:}
\textit{HOpenCls} achieves the highest scores across all metrics and datasets. The results for Open OA, F1\textsuperscript{u}, and AUC\textsuperscript{u} are detailed in Table~\ref{tab:main_results}, while the closed OA results are presented in Table~\ref{tab:closed_OA}. Classification maps generated by different methods can be found in Fig.~\ref{fig:result_HH}-Fig.~\ref{fig:result_PU}. Compared to the second-best method, \textit{HOpenCls} shows improvements in Open OA of 8.20, 3.46, 4.99, and 3.05 on the WHU-Hi-HongHu, WHU-Hi-LongKou, WHU-Hi-HanChuan, and University of Pavia, respectively. Notably, for unknown class rejection tasks, \textit{HOpenCls} exhibits significant improvements in F1\textsuperscript{u}, with gains of 38.91, 3.66, 3.56, and 20.93 on the same datasets, respectively.

\begin{figure*}[!t]
    \centering
    \includegraphics[width=0.98\textwidth]{results_LK.png}
    \caption{Open-set classification maps of WHU-Hi-LongKou dataset.}
    \label{fig:result_LK}
\end{figure*}

\begin{figure*}[!t]
    \centering
    \includegraphics[width=0.98\textwidth]{results_HC.png}
    \caption{Open-set classification maps of WHU-Hi-HanChuan dataset.}
    \label{fig:result_HC}
\end{figure*}

\begin{table*}[!t]
    \caption{Results of classifiers with extra data from different distributions. ↑ indicates larger values are better. ±$x$ denotes the standard error.}
    \label{tab:additional_data}
    \resizebox{\linewidth}{!}{%
    \begin{tabular}{lccccccccccccc} 
    \hline
    \multicolumn{1}{c}{\multirow{2}{*}{Methods}} & \multirow{2}{*}{$P$}         & \multicolumn{3}{c}{WHU-Hi-HongHu}                                                                                   & \multicolumn{3}{c}{WHU-Hi-LongKou}                                                                                  & \multicolumn{3}{c}{WHU-Hi-HanChuan}                                                                                & \multicolumn{3}{c}{University of Pavia}                                                                              \\
    \multicolumn{1}{c}{}                         &                              & Open OA↑                             & F1\textsuperscript{u}↑               & AUC\textsuperscript{u}↑               & Open OA↑                             & F1\textsuperscript{u}↑               & AUC\textsuperscript{u}↑               & Open OA↑                             & F1\textsuperscript{u}↑               & AUC\textsuperscript{u}↑              & Open OA↑                             & F1\textsuperscript{u}↑               & AUC\textsuperscript{u}↑                \\ 
    \hline
                                                 & \multicolumn{1}{l}{}         & \multicolumn{12}{c}{{\cellcolor[rgb]{0.753,0.753,0.753}}With known classes data from~$\mathbb{P}_{k}$ and pure unknown classes data from~$\mathbb{P}_{u}$}                                                                                                                                                                                                                                                                                                                                                                                   \\
    DS3L~\cite{DS3L}                                         & \multicolumn{1}{l}{$P_{pb}$} & 89.08\textsuperscript{±0.7}          & 71.64\textsuperscript{±2.1}          & 96.73\textsuperscript{±0.6}           & 94.56\textsuperscript{±0.5}          & 94.05\textsuperscript{±0.6}          & 99.25\textsuperscript{±0.1}           & 73.64\textsuperscript{±2.1}          & 76.79\textsuperscript{±2.4}          & 93.32\textsuperscript{±0.6}          & 95.50\textsuperscript{±0.5}          & 84.92\textsuperscript{±1.4}          & 99.54\textsuperscript{±0.2}            \\
    WOODS~\cite{WOODS}                                        & $P_{pb}$                     & 74.48\textsuperscript{±11.4}         & 55.31\textsuperscript{±11.0}         & 99.28\textsuperscript{±0.4}           & 92.70\textsuperscript{±1.2}          & 91.51\textsuperscript{±1.3}          & 99.86\textsuperscript{±0.1}           & 86.71\textsuperscript{±4.8}          & 90.50\textsuperscript{±4.2}          & 95.00\textsuperscript{±3.4}          & 89.47\textsuperscript{±3.9}          & 74.72\textsuperscript{±9.0}          & 99.95\textsuperscript{±0.1}            \\
    OE~\cite{OE}                                           & $P_{pf}$                     & 95.23\textsuperscript{±0.6}          & 86.49\textsuperscript{±1.8}          & 99.99\textsuperscript{±0.0}           & 98.83\textsuperscript{±0.2}          & 98.66\textsuperscript{±0.3}          & 99.99\textsuperscript{±0.0}           & \textbf{98.85}\textsuperscript{±0.2} & \textbf{99.15}\textsuperscript{±0.1} & 99.94\textsuperscript{±0.0}          & 98.11\textsuperscript{±0.4}          & 92.59\textsuperscript{±1.4}          & \textbf{100.00}\textsuperscript{±0.0}  \\
    EOS~\cite{EOS}                                          & $P_{pf}$                     & 95.45\textsuperscript{±0.3}          & 88.28\textsuperscript{±1.2}          & 99.98\textsuperscript{±0.0}           & 98.44\textsuperscript{±0.2}          & 98.19\textsuperscript{±0.2}          & \textbf{100.00}\textsuperscript{±0.0} & 98.63\textsuperscript{±0.1}          & 98.99\textsuperscript{±0.1}          & \textbf{99.94}\textsuperscript{±0.0} & 97.49\textsuperscript{±0.7}          & 90.36\textsuperscript{±2.7}          & \textbf{100.00}\textsuperscript{±0.0}  \\
    \textit{HOpenCls} (ours)                                     & $P_{pf}$                     & \textbf{96.95}\textsuperscript{±0.2} & \textbf{98.98}\textsuperscript{±0.4} & \textbf{100.00}\textsuperscript{±0.0} & \textbf{99.10}\textsuperscript{±0.2} & \textbf{99.91}\textsuperscript{±0.0} & \textbf{100.00}\textsuperscript{±0.0} & 98.15\textsuperscript{±0.3}          & 98.73\textsuperscript{±0.2}          & 99.73\textsuperscript{±0.1}          & \textbf{99.34}\textsuperscript{±0.1} & \textbf{99.69}\textsuperscript{±0.0} & \textbf{100.00}\textsuperscript{±0.0}  \\ 
    \hline
                                                 &                              & \multicolumn{12}{c}{{\cellcolor[rgb]{0.753,0.753,0.753}}With known classes data from~$\mathbb{P}_{k}$ and wild data excluding unknown classes from~$\mathbb{P}_{wild}-\mathbb{P}_{u}$}                                                                                                                                                                                                                                                                                                                                                                 \\
    DS3L~\cite{DS3L}                                          & $P_{pb}$                     & 82.26\textsuperscript{±0.7}          & 41.83\textsuperscript{±3.0}          & 71.59\textsuperscript{±4.2}           & 60.37\textsuperscript{±0.6}          & 14.46\textsuperscript{±1.5}          & 15.24\textsuperscript{±1.7}           & 54.87\textsuperscript{±8.7}          & 50.76\textsuperscript{±14.0}         & 81.89\textsuperscript{±1.8}          & 91.21\textsuperscript{±1.3}          & 60.07\textsuperscript{±8.7}          & 93.54\textsuperscript{±4.0}            \\
    WOODS~\cite{WOODS}                                        & $P_{pb}$                     & 31.87\textsuperscript{±9.2}          & 16.42\textsuperscript{±3.6}          & 34.81\textsuperscript{±11.6}          & 53.61\textsuperscript{±23.6}         & 47.23\textsuperscript{±29.7}         & 52.47\textsuperscript{±26.8}          & 77.14\textsuperscript{±5.5}          & 84.32\textsuperscript{±3.3}          & 87.45\textsuperscript{±3.4}          & 44.12\textsuperscript{±6.5}          & 26.29\textsuperscript{±3.6}          & 75.66\textsuperscript{±7.5}            \\
    OE~\cite{OE}                                           & $P_{pf}$                     & 36.16\textsuperscript{±0.6}          & 27.29\textsuperscript{±0.7}          & 71.16\textsuperscript{±4.6}           & 65.40\textsuperscript{±3.2}          & 69.09\textsuperscript{±2.0}          & 93.36\textsuperscript{±2.1}           & 82.68\textsuperscript{±1.3}          & 87.80\textsuperscript{±0.9}          & 92.02\textsuperscript{±1.1}          & 78.86\textsuperscript{±2.8}          & 49.79\textsuperscript{±2.9}          & 95.53\textsuperscript{±0.5}            \\
    EOS~\cite{EOS}                                          & $P_{pf}$                     & 30.83\textsuperscript{±0.7}          & 25.53\textsuperscript{±1.2}          & 53.76\textsuperscript{±3.2}           & 49.96\textsuperscript{±0.4}          & 60.74\textsuperscript{±0.2}          & 94.06\textsuperscript{±1.8}           & 73.07\textsuperscript{±0.4}          & 82.82\textsuperscript{±0.3}          & 91.44\textsuperscript{±1.0}          & 54.38\textsuperscript{±2.6}          & 32.22\textsuperscript{±1.3}          & 95.22\textsuperscript{±1.3}            \\
    \textit{HOpenCls} (ours)                                     & $P_{pf}$                     & \textbf{92.77}\textsuperscript{±0.8} & \textbf{71.58}\textsuperscript{±4.2} & \textbf{95.30}\textsuperscript{±1.5}  & \textbf{98.77}\textsuperscript{±0.2} & \textbf{98.50}\textsuperscript{±0.2} & \textbf{99.16}\textsuperscript{±0.3}  & \textbf{87.37}\textsuperscript{±5.4} & \textbf{89.29}\textsuperscript{±5.2} & \textbf{97.91}\textsuperscript{±0.6} & \textbf{97.09}\textsuperscript{±0.4} & \textbf{88.48}\textsuperscript{±1.4} & \textbf{99.69}\textsuperscript{±0.1}   \\
    \hline
    \end{tabular}
    }
\end{table*}

\begin{figure*}[!t]
    \centering
    \includegraphics[width=0.98\textwidth]{results_PU.png}
    \caption{Open-set classification maps of University of Pavia dataset.}
    \label{fig:result_PU}
\end{figure*}

Several observations should be highlighted: (1) A significant bottleneck in the development of HSI classifiers is the challenge of rejecting unknown classes. Different classifiers show substantial variation in their performance on this task, take the WHU-Hi-HongHu HSI dataset as an example, with some suffering from under-recognition (e.g., KL Matching, MSP, Entropy, and DS3L) and others from over-recognition (e.g., Energy, WOODS, OE, and EOS). The proposed \textit{HOpenCls} effectively balances these issues, achieving a 38.91 and 20.93 improvement in F1\textsuperscript{u} on the WHU-Hi-HongHu and University of Pavia datasets, respectively. (2) \textit{HOpenCls} not only excels at unknown classes rejection but also improves classification performance on known classes. The task of unknown classes rejection has the ability to improve the classification performance of known classes in the proposed framework, which exhibits the effectiveness of the multi-task architecture. (3) The use of additional data can negatively affect classifier performance. For example, on the WHU-Hi-HongHu dataset, WOODS underperforms compared to OpenMax and CAC Loss, both of which do not incorporate additional data. Additionally, classifiers like OE and EOS experience significant degradation due to the influence of known class components in wild data. This suggests that while wild data is easily accessible, it introduces greater challenges for algorithm design.

\noindent \textbf{\textit{HOpenCls} is Robust to the Sources of Auxiliary Dataset:}
Three types of extra datasets are used to evaluate the robustness of the proposed \textit{HOpenCls} concerning the sources of the extra dataset. Wild data, which is the focus of this paper, can be obtained from the living environments of the classifiers with almost no cost, the results are shown in Table~\ref{tab:main_results}. Pure unknown classes data represents the ideal extra data but is labor-intensive to obtain, originating from the distribution of $\mathbb{P}_{u}$. Wild data without unknown classes poses the greatest challenge, providing no information about unknown classes and coming from the distribution of $\mathbb{P}_{wild}-\mathbb{P}_{u}$. The results of the models trained with pure unknown data or wild data excluding unknown classes are shown in Table~\ref{tab:additional_data}.

Some conclusions can be drawn from the results: (1) \textit{HOpenCls} demonstrates robustness to the source of extra data, and the performance of the proposed model does not significantly decrease compared to other methods when wild data excluding unknown classes is used as extra data. (2) The use of pure unknown data significantly enhances performance, indicating that it is the most effective form of extra data. However, acquiring unknown data is labor-intensive, and the collected unknown data is usually difficult to cover all the unknown classes. (3) Wild data is the potential extra data that can be almost freely collected, but the wild data propose higher demands on the algorithm design due to the wild data is the mixture of known and unknown classes. Due to the influence of known classes in the wild data, both OE and EOS suffer from significant performance degradation. (4) The proposed \textit{HOpenCls} achieves the best metrics across all datasets when using wild data excluding unknown classes, suggesting that it excels at discovering novel classes because this dataset does not contain any information about unknown classes.

\noindent \textbf{The combination of Grad-C and Grad-E Modules is a Powerful Classifier for PU Learning:}
Compared to other HSI PU learning methods, the combination of Grad-C and Grad-E modules, referred to as \textit{HOpenCls(PU)}, achieves better performance in PU learning tasks (Table~\ref{tab:pu_results}). Due to the limitations of inaccurate class priors, the performance of HOneCls significantly decreases. Compared to the class prior-free method (T-HOneCls), the proposed method gets better performance, particularly in high class prior scenarios, such as for Cotton and Broad-leaf soybean.

\begin{table}[!t]
    \centering
    \caption{F1 scores for different PU learning methods. ±$x$ denotes the standard error.}
    \label{tab:pu_results}
    \resizebox{\linewidth}{!}{%
    \begin{tabular}{cccccc} 
    \hline
    \multirow{3}{*}{Methods\tablefootnote{The experimental settings are consistent with those outlined in~\cite{T-HOneCls}.}} & \multicolumn{5}{c}{Positive class (class prior)}                                                                                                                                                                                    \\
                                                                                                                              & Cotton\tablefootnote{The Cotton class from the WHU-Hi-HongHu dataset.} & Broad-leaf soybean                   & Corn                                 & Rape                                 & Cowpea                                \\
                                                                                                                              & (0.3769)                                                               & (0.2873)                             & (0.1569)                             & (0.1317)                             & (0.0617)                              \\ 
    \hline
    HOneCls~\cite{HOneCls}                                                                                                                   & \textbf{99.44}\textsuperscript{±0.2}                                   & 88.02\textsuperscript{±0.3}          & 99.67\textsuperscript{±0.1}          & 81.81\textsuperscript{±1.2}          & 58.97\textsuperscript{±3.6}           \\
    T-HOneCls~\cite{T-HOneCls}                                                                                                                 & 98.15\textsuperscript{±0.3}                                            & 92.64\textsuperscript{±0.9}          & 99.70\textsuperscript{±0.1}          & 97.81\textsuperscript{±0.2}          & 90.31\textsuperscript{±1.1}           \\
    \textit{HOpenCls(PU)}                                                                                                              & 99.33\textsuperscript{±0.2}                                            & \textbf{94.98}\textsuperscript{±1.2} & \textbf{99.72}\textsuperscript{±0.0} & \textbf{98.40}\textsuperscript{±0.2} & \textbf{90.67}\textsuperscript{±2.1}  \\
    \hline
    \end{tabular}
    }
\end{table}

Furthermore, the performance of other loss function for the HSI PU learning task is also analyzed within the \textit{HOpenCls} framework, the proposed $\mathcal{L}_{tbce}$ get the best performance (Table~\ref{tab:diff_loss_results}).

\begin{table}[!t]
    \centering
    \caption{Results of different loss function of PU learning in the HOpenCls framework for unknown classes rejection. ↑ indicates larger values are better. ±$x$ denotes the standard error.}
    \label{tab:diff_loss_results}
    \resizebox{\linewidth}{!}{%
    \begin{tabular}{ccccccc} 
    \hline
    \multirow{2}{*}{$\mathcal{L}_{u}$}     & \multicolumn{2}{c}{WHU-Hi-HongHu}                                           & \multicolumn{2}{c}{WHU-Hi-LongKou}                                          & \multicolumn{2}{c}{University of  Pavia}                                     \\
                                       & Open OA↑                             & F1\textsuperscript{u}↑               & Open OA↑                             & F1\textsuperscript{u}↑               & Open OA↑                             & F1\textsuperscript{u}↑                \\ 
    \hline
    $\mathcal{L}_{Tar}$~\cite{T-HOneCls}      & 89.53\textsuperscript{±1.1}          & 71.80\textsuperscript{±2.1}          & 91.87\textsuperscript{±1.1}          & 90.59\textsuperscript{±1.2}          & 92.30\textsuperscript{±0.0}          & 74.02\textsuperscript{±0.0}           \\
    $\mathcal{L}_{tbce}^{w}$ & \textbf{96.03}\textsuperscript{±0.4} & \textbf{87.61}\textsuperscript{±1.2} & \textbf{99.15}\textsuperscript{±0.1} & \textbf{99.03}\textsuperscript{±0.1} & \textbf{97.17}\textsuperscript{±0.2} & \textbf{88.76}\textsuperscript{±0.7}  \\
    \hline
    \end{tabular}}
\end{table}

\subsection{Ablation Experiments Analysis}

Ablation experiments are conducted to evaluate the effectiveness of each module in the proposed \textit{HOpenCls}. The results are presented in Table~\ref{tab:ablation_experiments}. More detailed analysis can be found in the following:

\noindent \textbf{Multi-PU Head:}
The comparison between exp.1 and exp.2 highlights the significant role of the multi-PU head in the proposed \textit{HOpenCls}. For example, on the WHU-Hi-HongHu dataset, the Open OA, F1\textsuperscript{u}, and AUC\textsuperscript{u} scores improved by 27.66, 11.82, and 27.65, respectively. Similar improvements can also be observed in the comparison between exp.3 and exp.4.

\noindent \textbf{Grad-C Module:}
The comparison betweent exp.2 and exp.4 demonstrates that the Grad-C module is an effective solution for addressing the rejection of unknown classes in HSI. According to the Theorem~\ref{theorem}, the proposed $\mathcal{L}_{tbce}$ achieves better performance when the probability of the positive class is lower. Therefore, based on the combined experimental results of exp.1 to exp.4, greater improvements are observed when combining the multi-PU head with $\mathcal{L}_{tbce}$.

\noindent \textbf{Grad-E Module:}
Building upon $\mathcal{L}_{tbce}$, better open-set HSI classification results can be achieved by selectively restoring the gradient for wild unknown data, as demonstrated by the comparison between exp.4 and exp.5. Additionally, ensuring consistency between the two networks (exp.4 and exp.6) and applying the $\mathcal{L}_{tbce}^{w}$ (exp.6 and exp.7) further improves the performance of \textit{HOpenCls}.

\begin{table*}[!t]
    \centering
    \caption{Ablation experiments for different modules. ↑ indicates larger values are better. ±$x$ denotes the standard error.}
    \label{tab:ablation_experiments}
    \resizebox{\linewidth}{!}{%
    \begin{tabular}{ccccccccccccc} 
    \hline
    \multirow{2}{*}{Exp.} & \multicolumn{4}{c}{Ablation Experiments}                                                                                                  & \multicolumn{4}{c}{WHU-Hi-HongHu}                                                                                                                         & \multicolumn{4}{c}{WHU-Hi-LongKou}                                                                                                                         \\ 
    \cline{2-5}
                          & Grad-C Module\tablefootnote{A single network is employed in exp.1-exp.4.} & Grad-E Module                  & KL Loss      & Multi-PU Head & Open OA↑                             & F1\textsuperscript{u}↑               & AUC\textsuperscript{u}↑              & Close OA↑                            & Open OA↑                             & F1\textsuperscript{u}↑               & AUC\textsuperscript{u}↑              & Close OA↑                             \\ 
    \hline
    1                     & $\mathcal{L}_{bce}$                                                       &                                &              &               & 29.65\textsuperscript{±0.3}          & 26.74\textsuperscript{±0.1}          & 67.58\textsuperscript{±0.8}          & 95.01\textsuperscript{±0.8}          & 47.94\textsuperscript{±0.1}          & 59.88\textsuperscript{±0.1}          & 74.03\textsuperscript{±2.9}          & 99.48\textsuperscript{±0.1}           \\
    2                     & $\mathcal{L}_{bce}$                                                       &                                &              & $\checkmark$  & 57.31\textsuperscript{±11.1}         & 38.56\textsuperscript{±7.6}          & 95.23\textsuperscript{±2.9}          & 97.88\textsuperscript{±0.2}          & 68.78\textsuperscript{±1.6}          & 71.36\textsuperscript{±1.0}          & 98.59\textsuperscript{±0.5}          & 99.60\textsuperscript{±0.1}           \\
    3                     & $\mathcal{L}_{tbce}$                                                      &                                &              &               & 35.74\textsuperscript{±0.8}          & 28.95\textsuperscript{±0.3}          & 75.51\textsuperscript{±3.4}          & 92.63\textsuperscript{±1.3}          & 53.66\textsuperscript{±0.8}          & 62.64\textsuperscript{±0.4}          & 73.70\textsuperscript{±3.9}          & 98.99\textsuperscript{±0.5}           \\
    4                     & $\mathcal{L}_{tbce}$                                                      &                                &              & $\checkmark$  & 86.00\textsuperscript{±4.1}          & 66.40\textsuperscript{±7.6}          & 97.95\textsuperscript{±0.7}          & 97.34\textsuperscript{±0.6}          & 91.06\textsuperscript{±0.8}          & 89.76\textsuperscript{±0.8}          & 99.39\textsuperscript{±0.3}          & 99.44\textsuperscript{±0.1}           \\ 
    \cline{2-13}
    5                     & $\mathcal{L}_{tbce}$                                                      & $\mathcal{L}_{bce}^{w}$+MixPro &              & $\checkmark$  & 94.44\textsuperscript{±0.8}          & 83.90\textsuperscript{±2.6}          & 99.21\textsuperscript{±0.1}          & 98.15\textsuperscript{±0.6}          & 97.28\textsuperscript{±0.6}          & 96.74\textsuperscript{±0.7}          & 99.96\textsuperscript{±0.0}          & 99.53\textsuperscript{±0.1}           \\
    6                     & $\mathcal{L}_{tbce}$                                                      & $\mathcal{L}_{bce}^{w}$+MixPro & $\checkmark$ & $\checkmark$  & 94.59\textsuperscript{±0.5}          & 83.60\textsuperscript{±1.4}          & \textbf{99.28}\textsuperscript{±0.1} & 98.38\textsuperscript{±0.2}          & 97.92\textsuperscript{±0.1}          & 97.49\textsuperscript{±0.2}          & 99.98\textsuperscript{±0.0}          & 99.62\textsuperscript{±0.1}           \\
    7                     & $\mathcal{L}_{tbce}^{w}$+MixPro                                           & $\mathcal{L}_{bce}^{w}$+MixPro & $\checkmark$ & $\checkmark$  & \textbf{96.03}\textsuperscript{±0.4} & \textbf{87.61}\textsuperscript{±1.2} & 99.22\textsuperscript{±0.2}          & \textbf{98.57}\textsuperscript{±0.2} & \textbf{99.15}\textsuperscript{±0.1} & \textbf{99.03}\textsuperscript{±0.1} & \textbf{99.99}\textsuperscript{±0.0} & \textbf{99.75}\textsuperscript{±0.0}  \\
    \hline
    \end{tabular}
    }
    \end{table*}

\begin{figure*}[!t]
    \centering
    \subfloat[\small{Analysis of the order of Taylor series}]{
    \label{fig:taylor_series_analysis}
    \includegraphics[width=0.33\textwidth]{taylor_series.png}}
    \subfloat[\small{Analysis of the $\beta$}]{
    \label{fig:beta_analysis}
    \includegraphics[width=0.33\textwidth]{beta.png}}
    \subfloat[\small{Analysis of the $\tau$}]{
    \label{fig:tao_analysis}
    \includegraphics[width=0.33\textwidth]{tao.png}}
    \caption{Parametric Analysis of the HOpenCls framework.}
    \label{fig:parametric_analysis}
\end{figure*}

\noindent \textbf{Confidence Score Updating:}
Additional experiments were conducted to analyze the confidence scores updating strategies in the Grad-C and Grad-E modules, with results shown in Table~\ref{tab:confidence_update_stargeies}. The results demonstrate that discrete updating is more suited to the Grad-E module, while continuous updating works better for the Grad-C module. This suggests that less ambiguous information is more beneficial for the Grad-E module.

\begin{table}[!t]
    \centering
    \caption{Results of different confidence scores updating strategies. ↑ indicates larger values are better. ±$x$ denotes the standard error.}
    \label{tab:confidence_update_stargeies}
    \resizebox{\linewidth}{!}{%
    \begin{tabular}{cccccc} 
    \hline
    \multirow{2}{*}{$w_{c}$} & \multirow{2}{*}{$w_{e}$} & \multicolumn{2}{c}{WHU-Hi-HongHu}                                           & \multicolumn{2}{c}{WHU-Hi-LongKou}                                           \\
                             &                          & Open OA↑                             & F1\textsuperscript{u}↑               & Open OA↑                             & F1\textsuperscript{u}↑                \\ 
    \hline
    Continuous               & Continuous               & 94.43\textsuperscript{±0.5}          & 82.72\textsuperscript{±1.5}          & 97.03\textsuperscript{±0.3}          & 96.53\textsuperscript{±0.1}           \\
    Discrete                 & Discrete                 & 94.82\textsuperscript{±0.2}          & 81.46\textsuperscript{±1.1}          & 99.10\textsuperscript{±0.2}          & 98.96\textsuperscript{±0.3}           \\
    Discrete                 & Continuous               & 95.16\textsuperscript{±0.3}          & 84.64\textsuperscript{±1.0}          & 98.23\textsuperscript{±0.2}          & 97.83\textsuperscript{±0.2}           \\
    Continuous               & Discrete                 & \textbf{96.03}\textsuperscript{±0.4} & \textbf{87.61}\textsuperscript{±1.2} & \textbf{99.15}\textsuperscript{±0.1} & \textbf{99.03}\textsuperscript{±0.1}  \\
    \hline
    \end{tabular}
    }
\end{table}

\noindent \textbf{MixPro:}
Compared to the straightforward approach (Pro), MixPro leverages the consistency between the known classes classifier and multiple sub-PU classifiers. This strategy integrates the classification capability of the known classes classifier into the unknown classes rejection task, further enhancing the performance of the proposed \textit{HOpenCls}. (Table~\ref{tab:probability_mixture}).

\begin{table}[!t]
    \tiny
    \centering
    \caption{Analysis of the MixPro. ↑ indicates larger values are better. ±$x$ denotes the standard error.}
    \label{tab:probability_mixture}
    \resizebox{\linewidth}{!}{%
    \begin{tabular}{ccccc} 
    \hline
    \multirow{2}{*}{Pro/MixPro} & \multicolumn{2}{c}{WHU-Hi-HongHu}                                                             & \multicolumn{2}{c}{WHU-Hi-LongKou}                                                             \\
                                & Open OA↑                                      & F1\textsuperscript{u}↑                        & Open OA↑                                      & F1\textsuperscript{u}↑                         \\ 
    \hline
    Pro                         & 95.75\textsuperscript{±0.3}                   & 87.07\textsuperscript{±1.4}                   & 99.03\textsuperscript{±0.2}                   & 98.88\textsuperscript{±0.2}                    \\
    ProMix                      & \textbf{\textbf{96.03}}\textsuperscript{±0.4} & \textbf{\textbf{87.61}}\textsuperscript{±1.2} & \textbf{\textbf{99.15}}\textsuperscript{±0.1} & \textbf{\textbf{99.03}}\textsuperscript{±0.1}  \\
    \hline
    \end{tabular}
    }
\end{table}

\begin{table}[!t]
    \centering
    \caption{Results of different number of networks. ↑ indicates larger values are better. ±$x$ denotes the standard error.}
    \label{tab:number_of_networks}
    \resizebox{\linewidth}{!}{%
    \begin{tabular}{ccccc} 
    \hline
    \multirow{2}{*}{Number of Networks} & \multicolumn{2}{c}{WHU-Hi-HongHu}                                                             & \multicolumn{2}{c}{WHU-Hi-LongKou}                                                             \\
                                        & Open OA↑                                      & F1\textsuperscript{u}↑                        & Open OA↑                                      & F1\textsuperscript{u}↑                         \\ 
    \hline
    Single                              & 94.74\textsuperscript{±0.3}                   & 83.87\textsuperscript{±0.7}                   & 97.66\textsuperscript{±0.5}                   & 97.15\textsuperscript{±0.6}                    \\
    Two                                 & \textbf{\textbf{96.03}}\textsuperscript{±0.4} & \textbf{\textbf{87.61}}\textsuperscript{±1.2} & \textbf{\textbf{99.15}}\textsuperscript{±0.1} & \textbf{\textbf{99.03}}\textsuperscript{±0.1}  \\
    \hline
    \end{tabular}
    }
\end{table}

\noindent \textbf{Two Networks Cooperative Optimization:}
In the proposed \textit{HOpenCls}, two networks are optimized cooperatively to mitigate discrepancies (such as low-probability predictions) between $\mathcal{L}_{tbce}^{w}$ and $\mathcal{L}_{bce}^{w}$. As shown in Table~\ref{tab:number_of_networks}, optimizing a single network yields worse performance compared to the proposed \textit{HOpenCls}.

\noindent \textbf{Parametric Analysis}
Experiments were conducted to analyze the parametric sensitivity of \textit{HOpenCls}, with the results presented in Fig.~\ref{fig:parametric_analysis}. Several key observations can be made from the results: (1) The proposed \textit{HOpenCls} is robust to variations in $o$, $\beta$, and $\tau$; (2) Better unknown classes rejection results (F1\textsuperscript{u}) are achieved with lower-order Taylor series expansions, consistent with Theorem~\ref{theorem}; (3) Increasing $\tau$ leads to higher Open OA and F1\textsuperscript{u}, suggesting that the Grad-E module benefits from more accurate unknown classes data.

\section{Conclusion}
This paper proposes a novel open-set HSI classification framework utilizing wild data. Wild data holds significant promise due to its abundance, ease of collection, and better alignment with real-world data distributions. However, the mixed margin distribution of wild data, comprising both $\mathbb{P}_{k}$ and $\mathbb{P}_{u}$, presents challenges for its effective utilization. To overcome this challenge, this paper formulates it as a PU learning problem and specifically proposes a multi-PU head, Grad-C module, and Grad-E module to make it tractable. The results demonstrate that wild data can dramatically promote open-set HSI classification in practice, thereby helping to expedite the deployment of trustworthy models in complex real-world scenarios.

% \section*{Acknowledge}
% The authors would like to thank Dr. Wenzhe Jiao for his valuable comments and suggestions, which greatly enhanced the manuscript.

\appendix
\section*{Proof of the Theorem.\ref{theorem}}

Considering that the $\mathcal{L}_{tbce}(f(\boldsymbol{x}_{wild}),0)$ is bounded:
\begin{equation}
    0 \leq \mathcal{L}_{tbce}(f(\boldsymbol{x}_{wild}),0) \leq \mathcal{N}_{t},
\end{equation}
The effectiveness of the $\mathcal{L}_{tbce}$ for rejecting unknown classes can be proven as follows.

\begin{IEEEproof}[Proof of Theorem \ref{theorem}]
    From Eqn.~\ref{eq:huber_contamination_model}, we have
    \begin{equation}\nonumber
        \begin{aligned}
            {\mathcal{R}_{pu}(f)} &= \frac{1}{2}\left({\mathcal{R}_{k}^{+}(f)}+{\mathcal{R}_{wild}^{-}(f)}\right)\\
                                  &= \frac{1}{2}\left({\mathcal{R}_{k}^{+}(f)}+{\pi}{\mathcal{R}_{k}^{-}(f)}+{(1-\pi)}{\mathcal{R}_{u}^{-}(f)}\right)\\
                                  &= {R_{u}(f)}+\frac{1}{2}\left({\pi}{\mathcal{R}_{k}^{-}(f)}-{\pi}{\mathcal{R}_{u}^{-}(f)}\right),
        \end{aligned}
    \end{equation}
    where $\mathcal{R}^{-}_{k}(f)=\mathbb{E}_{(\boldsymbol{x}_{k},0){\sim}{\mathbb{P}_{k}}}\left[\mathcal{L}_{u}(f(\boldsymbol{x}),0)\right]$. Considered that $\mathcal{L}_{tbce}(f(\boldsymbol{x}_{wild}),0)$ is bounded:
    \begin{equation}\nonumber
        \begin{aligned}
            {\mathcal{R}_{pu}(f)} &\leq {\mathcal{R}_{u}(f)}+\frac{1}{2}{\pi}{\mathcal{R}_{k}^{-}(f)}\\
                                  &\leq {\mathcal{R}_{u}(f)}+\frac{1}{2}{\pi}{\mathcal{N}_t}
        \end{aligned}
    \end{equation}

    \begin{equation}\nonumber
        \begin{aligned}
            {\mathcal{R}_{pu}(f)} &\geq {\mathcal{R}_{u}(f)}-\frac{1}{2}{\pi}{\mathcal{R}_{u}^{-}(f)}\\
                                  &\geq {\mathcal{R}_{u}(f)}-\frac{1}{2}{{\pi}\mathcal{N}_{t}}.
        \end{aligned}
    \end{equation}
    Then we can ontain:
    \begin{equation}
        {\mathcal{R}_{pu}(f)}-{\frac{1}{2}\pi}{\mathcal{N}_t} \leq {\mathcal{R}_{u}(f)} \leq {\mathcal{R}_{pu}(f)}+{\frac{1}{2}{\pi}\mathcal{N}_{t}}
    \label{eq:bearing}
    \end{equation}
    The Theorem~\ref{theorem} can be proved as follows:
    \begin{equation}\nonumber
        \begin{aligned}
            {\mathcal{R}_{u}(\hat{f})}-{\mathcal{R}_{u}(f^{*})} \leq {\mathcal{R}_{pu}(\hat{f})}-{\mathcal{R}_{pu}(f^{*})}+{\pi}\mathcal{N}_{t} \leq {\pi}{\mathcal{N}_{t}}
        \end{aligned}
    \end{equation}
    Considered that $f^{*}$ is the global minimizers of the $\mathcal{R}_{u}(f)$:
    \begin{equation}\nonumber
        0 \leq {\mathcal{R}_{u}(\hat{f})}-{\mathcal{R}_{u}(f^{*})} \leq {{\pi}\mathcal{N}_{t}}.
    \end{equation}
    From Eqn.~\ref{eq:bearing}, we can obtain:
    \begin{equation}\nonumber
        {\mathcal{R}_{u}(f)}-{\frac{1}{2}{\pi}\mathcal{N}_{t}} \leq {\mathcal{R}_{pu}(f)} \leq {\mathcal{R}_{u}(f)}+{\frac{1}{2}{\pi}\mathcal{N}_{t}}.
    \end{equation}
    Then:
    \begin{equation}\nonumber
        \begin{aligned}
            {\mathcal{R}_{pu}(f^{*})}-{\mathcal{R}_{pu}(\hat{f})} \leq {\mathcal{R}_{u}(f^{*})}-{\mathcal{R}_{u}(\hat{f})}+{{\pi}\mathcal{N}_{t}} \leq {{\pi}\mathcal{N}_{t}}.
        \end{aligned}
    \end{equation}
    Considered that $\hat{f}$ is the global minimizers of the $\mathcal{R}_{pu}(f)$:
    \begin{equation}\nonumber
        0 \leq {\mathcal{R}_{pu}(f^{*})}-{\mathcal{R}_{pu}(\hat{f})} \leq {{\pi}\mathcal{N}_{t}}.
    \end{equation}
\end{IEEEproof}

{\small
% \bibliographystyle{ieee_fullname}
\bibliographystyle{IEEEtran}
\bibliography{HOpenCls_ref}
}

\end{document}