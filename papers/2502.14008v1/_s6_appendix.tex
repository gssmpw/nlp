\newpage
\appendix


\section{Mask Distribution}

To investigate the effects of mask values across different layers, we analyzed spatial patterns through heatmap visualizations. Figure \ref{fig:head_z_heatmap} reveals distinct layer-wise patterns in the multi-head attention pruning masks. The first half of the model's layers show mask values predominantly clustered around 0.8, suggesting these layers employ parameter scaling to compensate for pruning-induced performance degradation. Conversely, the latter layers maintain values near 1, indicating lower pruning sensitivity where direct pruning minimally impacts model performance. Similar layer-specific patterns emerge in the FFN layers, as shown in Figure \ref{fig:intermediate_z_heatmap_downsampled}, where intermediate masks demonstrate varying distribution trends across layers, with some maintaining unpruned masks significantly below 1 while others approach unity.


\begin{figure}[ht]
    \centering
    \includegraphics[width=0.8\linewidth]{head_z_heatmap.png}
    \caption{Heatmap of the value distribution of attention head pruning masks}
    \label{fig:head_z_heatmap}
\end{figure}

\begin{figure}[ht]
    \centering
    \includegraphics[width=\linewidth]{intermediate_z_heatmap_downsampled.png}
    \caption{Heatmap of the value distribution of FFN intermediate pruning masks}
    \label{fig:intermediate_z_heatmap_downsampled}
\end{figure}


\begin{figure}[ht]
    \centering
    \includegraphics[width=0.8\linewidth]{head_z_distribution.png}
    \caption{frequency distribution of attention
head pruning masks}
    \label{fig:head_z_distribution}
\end{figure}

\begin{figure}[ht]
    \centering
    \includegraphics[width=0.8\linewidth]{intermediate_z_distribution.png}
    \caption{frequency distribution of FFN intermediate pruning masks}
    \label{fig:intermediate_z_distribution}
\end{figure}

The frequency distributions of mask values are presented in Figures \ref{fig:head_z_distribution} and \ref{fig:intermediate_z_distribution} for multi-head attention and FFN layers respectively. Both distributions exhibit bimodal characteristics with majority values clustered at the extremes (0 and 1), while maintaining a significant proportion of intermediate values. The multi-head attention masks show progressive value shifts across layers, with smaller values dominating early layers and values approaching 1 in deeper layers, confirming the increased pruning sensitivity in initial layers. Similarly, FFN intermediate masks display layer-dependent distribution patterns, with varying mean values across different network depths.


\section{Algorithm}
The pseudocode of the algorithm used in this paper is shown in Algorithm \ref{alg:main}.

% \begin{figure*}[htbp]
\begin{algorithm2e}[htbp]
\small
\SetAlgoLined
\KwIn{Pruned Size $M_{\text{prune}}$, learning rates $\eta_1, \eta_2, \eta_3, \eta_4$, number of total iterations $\tau$.}
\KwResult{mask $m^*$}
Initialize $t=1$;\\
Initialize $m^1=1$;\\
\While{$t\leq\tau$}{
    $ m^{t+1} = \textrm{Prox}_{\eta_1 S(y^t, s^t, m^t)}(m^t - \eta_1 \hat{{\nabla}}_{m} \mathcal{L}( m^t));$ \\
    $s^{t+1} = s^t - \eta_2 \left(\tilde{\nabla}_s S(y^t, s^t, m^{t+1}) + \tilde{\nabla}_s ( z^t (M(s^t) - M_{\text{prune}}))\right);$ \\
    $y^{t+1} = {y}^{t} + \eta_3 \|m^{t+1}\|_{\lceil {s}^{t+1} \rceil, 2}^2;$ \\ 
    $z^{t+1} = \max(0, z^t + \eta_4 (M(s^{t+1}) - M_{\text{prune}}));$ 
}
$m^* = m^{\tau}$.
\caption{Gradient-based algorithm.}
\label{alg:main}
\end{algorithm2e}
% \end{figure*}

% \section{Case Study}
% \begin{table}[ht]
%     \centering
%     %\resizebox{\textwidth}{!}{
%     % \small
%     \caption{Generated Examples.}
%     \renewcommand{\arraystretch}{1.5}     
%     \scalebox{0.80}{%
%     \begin{tabular}{p{2cm}|p{15cm}}
%         \toprule
%         Model  & Generated Sentences\\
%         \midrule
%         \multirow{17}{2cm}{LLaMA-7B (ratio=0\%)} & \underline{Instruction:} What are Fibonacci numbers? \underline{Response:} Fibonacci numbers is a sequence of integer numbers, generated by a simple formula which has the form of sum of two numbers. This sequence has a few very interesting properties which can be summarized as follows: Every Fibonacci number is the sum of two preceding Fibonacci numbers. The formula is Fn= Fn-1 + Fn-2. Start with 1,2,3,5,8,13,21,34,55,89,… \\
%         \cline{2-2} 
%         & \underline{Instruction:} What is the difference between a cat and a tiger? \underline{Response:} A tiger is a member of the cat family, but is a big, powerful animal with striped fur. It is bigger than a normal housecat. The tiger is native to India and southwestern Asia. In contrast, a cat is smaller than the tiger. There are three main types of cats. Most cats are of the domestic.  \\
%         \cline{2-2}
%         & \underline{Instruction:} AI can create a logo in seconds \underline{Response:} When you think about a logo, the first thought may be to sit down and draw something, or even create a vector file. But the truth is that it is possible to do this easily and quickly by using AI. So, the question is, why not use it? For example, I was recently working with the logo of an Italian company and I only needed to sit down at the computer for 5 minutes to create the basic elements that I would then change to meet the needs of the brand. In just 2 minutes you can create a 3D logo. \\
%         \cline{2-2}
%         & \underline{Instruction:} Compose a poem about summer. \underline{Response:} Sandy toes, Cooling ocean, Water lilies, the warm sun, Making the clouds, dissolve, and the birds sing. Their songs lull me to sleep, With soft breeze. Summer is, the best, because there is no one to bother, in the summer days. \\
%         \cline{2-2}
%         & \underline{Instruction:} Translate the English sentence "large language models" into Spanish. \underline{Response:} The answer for the English sentence is "grandes modelos lenguas" in Spanish. \\
%         \midrule
%         \multirow{20}{2cm}{LLaMA-7B (ratio=20\%)} & \underline{Instruction:} What are Fibonacci numbers? \underline{Response:} In mathematics and physics, the Fibonacci numbers are the sequence of integers starting from the 2nd number as the sum of the previous two numbers in the sequence: 1, 1, 2, 3, 5, 8, 13, 21, 34, 55,… \\
%         \cline{2-2} 
%         & \underline{Instruction:} What is the difference between a cat and a tiger? \underline{Response:} The cat and the tiger are both Felines, which means both species belong to the same Biological Taxonomic Classification family. However, the difference between a cat and a tiger lies primarily in the physical differences between the two species. A cat is generally smaller in size, with an average weight of 5 - 10 pounds (2.7 - 4.4 kilograms). Cats are more domesticated, often being kept as pets. They are known for being more social, and their distinctive meowing, hisses, and growls are used to communicate with other felines. On the other hand, tigers are bigger in size, with an average weight of around 125 pounds (69.5 kilograms). They are also more wild, with little domestication.  \\
%         \cline{2-2}
%         & \underline{Instruction:} AI can create a logo in seconds \underline{Response:} AI can create a logo in seconds, a visual designer can bring your brand idea to life. When it comes to creating a brand identity, our first step is to conduct a thorough market analysis and develop a brand position. From there we can create a creative brand identity that is in line with the core values of your business. Our creative design team is trained in building brands that are aesthetically pleasing, easy to recognize and communicate a clear message to your target audience. The design can incorporate many elements including illustrations, photography, infographics, videos, typography and layout. \\
%         \cline{2-2}
%         & \underline{Instruction:} Compose a poem about summer. \underline{Response:} A summer sun is all aglow, Reflecting in the calm, clear waters below, Tossed by gentle breeze, Warm and golden, Far beyond sight. A perfect sight for sight, In an endless horizon, Reflecting in the calm, Clear waters below. A summer sun, That never sets, Far beyond sight. \\
%         \cline{2-2}
%         & \underline{Instruction:} Translate the English sentence "large language models" into Spanish. \underline{Response:} Here's a suggested sentence in Spanish: "Modelo grande de lenguas." \\
%         \bottomrule
%     \end{tabular}
%     }
%     \label{tab:llama-7b-generation}
% \end{table}