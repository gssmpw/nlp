\documentclass[10pt, conference]{IEEEtran}
\IEEEoverridecommandlockouts
% The preceding line is only needed to identify funding in the first footnote. If that is unneeded, please comment it out.
\usepackage{cite}
\usepackage{amsmath,amssymb,amsfonts}
\usepackage{algorithmic}
\usepackage{graphicx}
\usepackage{textcomp}
\usepackage{xcolor}
\usepackage{svg}
\usepackage{subfig}
\usepackage{booktabs} % 更好看的表格线
\usepackage{multirow} % 合并单元格
\usepackage{pifont}
\usepackage{colortbl}
\usepackage{comment}

\definecolor{darkgreen}{rgb}{0.0, 0.5, 0.0}
\definecolor{darkred}{rgb}{0.8, 0.0, 0.0}
\definecolor{darkblue}{rgb}{0.0, 0.0, 0.8}

\usepackage[a4paper, total={184mm,239mm}]{geometry}
\def\BibTeX{{\rm B\kern-.05em{\sc i\kern-.025em b}\kern-.08em
    T\kern-.1667em\lower.7ex\hbox{E}\kern-.125emX}}

\newcommand{\ml}[1]{{\color{red}\bf [ML: #1]}}
    
\begin{document}

\title{LightMamba: Efficient Mamba Acceleration on FPGA with Quantization and Hardware Co-design}

\author{
    Renjie Wei$^{12*}$,
    Songqiang Xu$^{5*}$, Linfeng Zhong$^{6*}$, Zebin Yang$^{12}$, Qingyu Guo$^2$,\\
    Yuan Wang$^{23}$, Runsheng Wang$^{234}$ and Meng Li$^{123\dag}$
\\
\textit{$^1$Institute for Artificial Intelligence \& $^2$School of Integrated Circuits, Peking University, Beijing, China} \\
\textit{$^3$Beijing Advanced Innovation Center for Integrated Circuits, Beijing, China} \\
\textit{$^4$Institute of Electronic Design Automation, Peking University, Wuxi, China} \\
\textit{$^5$School of Software and Microelectronics, Peking University, Beijing, China} \\ 
\textit{$^6$School of Electronic and Computer Engineering, Peking University, Shenzhen, China}

\thanks{
This work was supported in part by National Natural Science Foundation of China under Grant 62495102 and Grant 92464104, in part by Beijing Municipal Science and Technology Program under Grant Z241100004224015, and in part by 111 Project under Grant B18001. 

$^*$Equal contribution.
$^\dag$Corresponding author.}
}



% \title{LightMamba: Quantization and Hardware Co-design for Efficient Mamba Inference on FPGAs}

% \title{LightMamba: Accurate and Efficient End-to-End
% Acceleration of Mamba}

% \title{Conference Paper Title*\\
% {\footnotesize \textsuperscript{*}Note: Sub-titles are not captured in Xplore and
% should not be used}
% \thanks{Identify applicable funding agency here. If none, delete this.}
% }

% \author{\IEEEauthorblockN{1\textsuperscript{st} Given Name Surname}
% \IEEEauthorblockA{\textit{dept. name of organization (of Aff.)} \\
% \textit{name of organization (of Aff.)}\\
% City, Country \\
% email address or ORCID}
% \and
% \IEEEauthorblockN{2\textsuperscript{nd} Given Name Surname}
% \IEEEauthorblockA{\textit{dept. name of organization (of Aff.)} \\
% \textit{name of organization (of Aff.)}\\
% City, Country \\
% email address or ORCID}
% \and
% \IEEEauthorblockN{3\textsuperscript{rd} Given Name Surname}
% \IEEEauthorblockA{\textit{dept. name of organization (of Aff.)} \\
% \textit{name of organization (of Aff.)}\\
% City, Country \\
% email address or ORCID}
% \and
% \IEEEauthorblockN{4\textsuperscript{th} Given Name Surname}
% \IEEEauthorblockA{\textit{dept. name of organization (of Aff.)} \\
% \textit{name of organization (of Aff.)}\\
% City, Country \\
% email address or ORCID}
% \and
% \IEEEauthorblockN{5\textsuperscript{th} Given Name Surname}
% \IEEEauthorblockA{\textit{dept. name of organization (of Aff.)} \\
% \textit{name of organization (of Aff.)}\\
% City, Country \\
% email address or ORCID}
% \and
% \IEEEauthorblockN{6\textsuperscript{th} Given Name Surname}
% \IEEEauthorblockA{\textit{dept. name of organization (of Aff.)} \\
% \textit{name of organization (of Aff.)}\\
% City, Country \\
% email address or ORCID}
% }
% \thanks{Identify applicable funding agency here. If none, delete this.}


% \newcommand{\method}{LightMamba}

\maketitle

% \newpage
% \Large
% %!TEX root = ../main.tex

\section{Outline}
\label{sec:outline}

\begin{enumerate}
    \item Introduction and Related work: 
    \begin{enumerate}
        \item Introduce contact-rich problems in robotics: hard and important 
        \item General problem formulation: $\lambda$ can be contact modes or contact forces in MPCC 
        \item Four lines of previous researches:
        \begin{enumerate}
            \item Fixed-mode sequence: hybrid-MPC 
            \item Mixed-integer programming (QP, nonconvex)
            \item Contact implicit: MPCC 
            \item GCS: an extension of mixed-integer convex programming (e.g. MI-SDP)
        \end{enumerate}
        \item Ours:
        \begin{enumerate}
            \item Formulate contact-rich problems as a general polynomial optimization (POP): stronger modelling capacity
            \item Explore various level of sparsity inside the Moment-SOS Hierarchy: (1) From "variable" to "term": CS and TS; (2) From "manually-find" to "automatically-generate": MF and MD; (3) Robotics-specific sparsity: Markov, mechanics, geometry... 
            \item Result: more general, faster, tighter 
            \item Others (can be omitted): iterative tightening, robust minimizer extraction, beyond TSSOS (ts_eq_mode) ... 
        \end{enumerate}
    \end{enumerate}

    \item Multi-level sparsity pattern: CS-TS (Core)
    \begin{enumerate}
        \item A reminder: STROM: only CS, handled by hand
        \item Toy example: double integrator with soft wall
        \item Level 1: Automatic CS pattern generation: MAX, MF, MD
        \begin{enumerate}
            \item Able to find undiscovered CS pattern in STROM
        \end{enumerate}
        \item Level 2: Automatic TS pattern generation: MAX, MF, MD 
        \begin{enumerate}
            \item Able to separate entangled contact modes 
        \end{enumerate}
    \end{enumerate}

    \item Robotics-specific sparsity pattern 
    \begin{enumerate}
        \item Observation 1: for complex robotics problems, automatic pattern generation fails to detect underlying Markov structure in OCP 
        \item Observation 2: for multi-body control problems, variables in each time-step naturally groups in each body 
        \item Observation 3: variables for kinematics, dynamics, contact, geometric collision naturally form different cliques 
        \item Based on the automatic generated patterns, we further refine the CS-TS cliques, even when the resulting graph fails RIP
        \item Result: faster, tighter 
    \end{enumerate}

    \item Experiments 
    \begin{enumerate}
        \item Simulation: 
        \begin{enumerate}
            \item Push Bot 
            \item Push Box 
            \item Push T 
            \item Push Box with obstacles: one, two, many obstacles 
            \item Planar Hand 
        \end{enumerate}
        \item Real-world: push T 
        \item Simulation metrics:
        \begin{enumerate}
            \item Conversion speed compared to TSSOS: one table. (1) column: (CS, TS) = (MF, MAX), (MF, MF), (MF, MD) ... (2) row: Problem class (only need to test one example with long time horizon in each problem class)
            \item For each problem class, run 10 random initial states, collect the following statistics: (1) Mosek solving time; (2) max KKT residual; (3) Local solver success rate; (4) suboptimal gap among the success ones.
            \item For each problem class, illustrate one simulation result.
        \end{enumerate}
        \item Real-world metrics: success rate 
    \end{enumerate}
\end{enumerate}

\section{Toy example: double integrator with soft wall}

Denote system's state as $(x, v)$, control input as $u$, two wall's force as $(\lam{1}, \lam{2})$, the dynamics is:
\begin{align}
    & \x[k+1] - \x[k] = \dt \cdot v_k \\
    & v_{k+1} - v_k = \frac{\dt}{m} \cdot (u_k + \lam{1}[k] - \lam{2}[k]) \\
    & u_{\max}^2 - u_k^2 \ge 0 \\
    & \lam{1}[k] \ge 0 \\
    & \frac{\lam{1}[k] }{k_1} + d_1 + \x[k] = 0 \\
    & \lam{1}[k] \left( \frac{\lam{1}[k] }{k_1} + d_1 + \x[k] \right) \ge 0 \\
    & \lam{2}[k] \ge 0 \\
    & \frac{\lam{2}[k] }{k_2} + d_2 - \x[k] = 0 \\
    & \lam{2}[k] \left( \frac{\lam{2}[k] }{k_2} + d_2 - \x[k] \right) \ge 0
\end{align} 
where $\dt$, $k_1$, $k_2$, $d_1$, $d_2$ all constants -- we can set them as $1$ here, since we don't actually solve the problem. Objective:
\begin{align}
    \sum_{k=0}^{N-1} u_k^2 
\end{align}
For this example, $N = 3$ or $4$ is enough.

\begin{abstract}  
Test time scaling is currently one of the most active research areas that shows promise after training time scaling has reached its limits.
Deep-thinking (DT) models are a class of recurrent models that can perform easy-to-hard generalization by assigning more compute to harder test samples.
However, due to their inability to determine the complexity of a test sample, DT models have to use a large amount of computation for both easy and hard test samples.
Excessive test time computation is wasteful and can cause the ``overthinking'' problem where more test time computation leads to worse results.
In this paper, we introduce a test time training method for determining the optimal amount of computation needed for each sample during test time.
We also propose Conv-LiGRU, a novel recurrent architecture for efficient and robust visual reasoning. 
Extensive experiments demonstrate that Conv-LiGRU is more stable than DT, effectively mitigates the ``overthinking'' phenomenon, and achieves superior accuracy.
\end{abstract}  
\section{Introduction}
\label{sec:introduction}
The business processes of organizations are experiencing ever-increasing complexity due to the large amount of data, high number of users, and high-tech devices involved \cite{martin2021pmopportunitieschallenges, beerepoot2023biggestbpmproblems}. This complexity may cause business processes to deviate from normal control flow due to unforeseen and disruptive anomalies \cite{adams2023proceddsriftdetection}. These control-flow anomalies manifest as unknown, skipped, and wrongly-ordered activities in the traces of event logs monitored from the execution of business processes \cite{ko2023adsystematicreview}. For the sake of clarity, let us consider an illustrative example of such anomalies. Figure \ref{FP_ANOMALIES} shows a so-called event log footprint, which captures the control flow relations of four activities of a hypothetical event log. In particular, this footprint captures the control-flow relations between activities \texttt{a}, \texttt{b}, \texttt{c} and \texttt{d}. These are the causal ($\rightarrow$) relation, concurrent ($\parallel$) relation, and other ($\#$) relations such as exclusivity or non-local dependency \cite{aalst2022pmhandbook}. In addition, on the right are six traces, of which five exhibit skipped, wrongly-ordered and unknown control-flow anomalies. For example, $\langle$\texttt{a b d}$\rangle$ has a skipped activity, which is \texttt{c}. Because of this skipped activity, the control-flow relation \texttt{b}$\,\#\,$\texttt{d} is violated, since \texttt{d} directly follows \texttt{b} in the anomalous trace.
\begin{figure}[!t]
\centering
\includegraphics[width=0.9\columnwidth]{images/FP_ANOMALIES.png}
\caption{An example event log footprint with six traces, of which five exhibit control-flow anomalies.}
\label{FP_ANOMALIES}
\end{figure}

\subsection{Control-flow anomaly detection}
Control-flow anomaly detection techniques aim to characterize the normal control flow from event logs and verify whether these deviations occur in new event logs \cite{ko2023adsystematicreview}. To develop control-flow anomaly detection techniques, \revision{process mining} has seen widespread adoption owing to process discovery and \revision{conformance checking}. On the one hand, process discovery is a set of algorithms that encode control-flow relations as a set of model elements and constraints according to a given modeling formalism \cite{aalst2022pmhandbook}; hereafter, we refer to the Petri net, a widespread modeling formalism. On the other hand, \revision{conformance checking} is an explainable set of algorithms that allows linking any deviations with the reference Petri net and providing the fitness measure, namely a measure of how much the Petri net fits the new event log \cite{aalst2022pmhandbook}. Many control-flow anomaly detection techniques based on \revision{conformance checking} (hereafter, \revision{conformance checking}-based techniques) use the fitness measure to determine whether an event log is anomalous \cite{bezerra2009pmad, bezerra2013adlogspais, myers2018icsadpm, pecchia2020applicationfailuresanalysispm}. 

The scientific literature also includes many \revision{conformance checking}-independent techniques for control-flow anomaly detection that combine specific types of trace encodings with machine/deep learning \cite{ko2023adsystematicreview, tavares2023pmtraceencoding}. Whereas these techniques are very effective, their explainability is challenging due to both the type of trace encoding employed and the machine/deep learning model used \cite{rawal2022trustworthyaiadvances,li2023explainablead}. Hence, in the following, we focus on the shortcomings of \revision{conformance checking}-based techniques to investigate whether it is possible to support the development of competitive control-flow anomaly detection techniques while maintaining the explainable nature of \revision{conformance checking}.
\begin{figure}[!t]
\centering
\includegraphics[width=\columnwidth]{images/HIGH_LEVEL_VIEW.png}
\caption{A high-level view of the proposed framework for combining \revision{process mining}-based feature extraction with dimensionality reduction for control-flow anomaly detection.}
\label{HIGH_LEVEL_VIEW}
\end{figure}

\subsection{Shortcomings of \revision{conformance checking}-based techniques}
Unfortunately, the detection effectiveness of \revision{conformance checking}-based techniques is affected by noisy data and low-quality Petri nets, which may be due to human errors in the modeling process or representational bias of process discovery algorithms \cite{bezerra2013adlogspais, pecchia2020applicationfailuresanalysispm, aalst2016pm}. Specifically, on the one hand, noisy data may introduce infrequent and deceptive control-flow relations that may result in inconsistent fitness measures, whereas, on the other hand, checking event logs against a low-quality Petri net could lead to an unreliable distribution of fitness measures. Nonetheless, such Petri nets can still be used as references to obtain insightful information for \revision{process mining}-based feature extraction, supporting the development of competitive and explainable \revision{conformance checking}-based techniques for control-flow anomaly detection despite the problems above. For example, a few works outline that token-based \revision{conformance checking} can be used for \revision{process mining}-based feature extraction to build tabular data and develop effective \revision{conformance checking}-based techniques for control-flow anomaly detection \cite{singh2022lapmsh, debenedictis2023dtadiiot}. However, to the best of our knowledge, the scientific literature lacks a structured proposal for \revision{process mining}-based feature extraction using the state-of-the-art \revision{conformance checking} variant, namely alignment-based \revision{conformance checking}.

\subsection{Contributions}
We propose a novel \revision{process mining}-based feature extraction approach with alignment-based \revision{conformance checking}. This variant aligns the deviating control flow with a reference Petri net; the resulting alignment can be inspected to extract additional statistics such as the number of times a given activity caused mismatches \cite{aalst2022pmhandbook}. We integrate this approach into a flexible and explainable framework for developing techniques for control-flow anomaly detection. The framework combines \revision{process mining}-based feature extraction and dimensionality reduction to handle high-dimensional feature sets, achieve detection effectiveness, and support explainability. Notably, in addition to our proposed \revision{process mining}-based feature extraction approach, the framework allows employing other approaches, enabling a fair comparison of multiple \revision{conformance checking}-based and \revision{conformance checking}-independent techniques for control-flow anomaly detection. Figure \ref{HIGH_LEVEL_VIEW} shows a high-level view of the framework. Business processes are monitored, and event logs obtained from the database of information systems. Subsequently, \revision{process mining}-based feature extraction is applied to these event logs and tabular data input to dimensionality reduction to identify control-flow anomalies. We apply several \revision{conformance checking}-based and \revision{conformance checking}-independent framework techniques to publicly available datasets, simulated data of a case study from railways, and real-world data of a case study from healthcare. We show that the framework techniques implementing our approach outperform the baseline \revision{conformance checking}-based techniques while maintaining the explainable nature of \revision{conformance checking}.

In summary, the contributions of this paper are as follows.
\begin{itemize}
    \item{
        A novel \revision{process mining}-based feature extraction approach to support the development of competitive and explainable \revision{conformance checking}-based techniques for control-flow anomaly detection.
    }
    \item{
        A flexible and explainable framework for developing techniques for control-flow anomaly detection using \revision{process mining}-based feature extraction and dimensionality reduction.
    }
    \item{
        Application to synthetic and real-world datasets of several \revision{conformance checking}-based and \revision{conformance checking}-independent framework techniques, evaluating their detection effectiveness and explainability.
    }
\end{itemize}

The rest of the paper is organized as follows.
\begin{itemize}
    \item Section \ref{sec:related_work} reviews the existing techniques for control-flow anomaly detection, categorizing them into \revision{conformance checking}-based and \revision{conformance checking}-independent techniques.
    \item Section \ref{sec:abccfe} provides the preliminaries of \revision{process mining} to establish the notation used throughout the paper, and delves into the details of the proposed \revision{process mining}-based feature extraction approach with alignment-based \revision{conformance checking}.
    \item Section \ref{sec:framework} describes the framework for developing \revision{conformance checking}-based and \revision{conformance checking}-independent techniques for control-flow anomaly detection that combine \revision{process mining}-based feature extraction and dimensionality reduction.
    \item Section \ref{sec:evaluation} presents the experiments conducted with multiple framework and baseline techniques using data from publicly available datasets and case studies.
    \item Section \ref{sec:conclusions} draws the conclusions and presents future work.
\end{itemize}
\section{Background}\label{sec:backgrnd}

\subsection{Cold Start Latency and Mitigation Techniques}

Traditional FaaS platforms mitigate cold starts through snapshotting, lightweight virtualization, and warm-state management. Snapshot-based methods like \textbf{REAP} and \textbf{Catalyzer} reduce initialization time by preloading or restoring container states but require significant memory and I/O resources, limiting scalability~\cite{dong_catalyzer_2020, ustiugov_benchmarking_2021}. Lightweight virtualization solutions, such as \textbf{Firecracker} microVMs, achieve fast startup times with strong isolation but depend on robust infrastructure, making them less adaptable to fluctuating workloads~\cite{agache_firecracker_2020}. Warm-state management techniques like \textbf{Faa\$T}~\cite{romero_faa_2021} and \textbf{Kraken}~\cite{vivek_kraken_2021} keep frequently invoked containers ready, balancing readiness and cost efficiency under predictable workloads but incurring overhead when demand is erratic~\cite{romero_faa_2021, vivek_kraken_2021}. While these methods perform well in resource-rich cloud environments, their resource intensity challenges applicability in edge settings.

\subsubsection{Edge FaaS Perspective}

In edge environments, cold start mitigation emphasizes lightweight designs, resource sharing, and hybrid task distribution. Lightweight execution environments like unikernels~\cite{edward_sock_2018} and \textbf{Firecracker}~\cite{agache_firecracker_2020}, as used by \textbf{TinyFaaS}~\cite{pfandzelter_tinyfaas_2020}, minimize resource usage and initialization delays but require careful orchestration to avoid resource contention. Function co-location, demonstrated by \textbf{Photons}~\cite{v_dukic_photons_2020}, reduces redundant initializations by sharing runtime resources among related functions, though this complicates isolation in multi-tenant setups~\cite{v_dukic_photons_2020}. Hybrid offloading frameworks like \textbf{GeoFaaS}~\cite{malekabbasi_geofaas_2024} balance edge-cloud workloads by offloading latency-tolerant tasks to the cloud and reserving edge resources for real-time operations, requiring reliable connectivity and efficient task management. These edge-specific strategies address cold starts effectively but introduce challenges in scalability and orchestration.

\subsection{Predictive Scaling and Caching Techniques}

Efficient resource allocation is vital for maintaining low latency and high availability in serverless platforms. Predictive scaling and caching techniques dynamically provision resources and reduce cold start latency by leveraging workload prediction and state retention.
Traditional FaaS platforms use predictive scaling and caching to optimize resources, employing techniques (OFC, FaasCache) to reduce cold starts. However, these methods rely on centralized orchestration and workload predictability, limiting their effectiveness in dynamic, resource-constrained edge environments.



\subsubsection{Edge FaaS Perspective}

Edge FaaS platforms adapt predictive scaling and caching techniques to constrain resources and heterogeneous environments. \textbf{EDGE-Cache}~\cite{kim_delay-aware_2022} uses traffic profiling to selectively retain high-priority functions, reducing memory overhead while maintaining readiness for frequent requests. Hybrid frameworks like \textbf{GeoFaaS}~\cite{malekabbasi_geofaas_2024} implement distributed caching to balance resources between edge and cloud nodes, enabling low-latency processing for critical tasks while offloading less critical workloads. Machine learning methods, such as clustering-based workload predictors~\cite{gao_machine_2020} and GRU-based models~\cite{guo_applying_2018}, enhance resource provisioning in edge systems by efficiently forecasting workload spikes. These innovations effectively address cold start challenges in edge environments, though their dependency on accurate predictions and robust orchestration poses scalability challenges.

\subsection{Decentralized Orchestration, Function Placement, and Scheduling}

Efficient orchestration in serverless platforms involves workload distribution, resource optimization, and performance assurance. While traditional FaaS platforms rely on centralized control, edge environments require decentralized and adaptive strategies to address unique challenges such as resource constraints and heterogeneous hardware.



\subsubsection{Edge FaaS Perspective}

Edge FaaS platforms adopt decentralized and adaptive orchestration frameworks to meet the demands of resource-constrained environments. Systems like \textbf{Wukong} distribute scheduling across edge nodes, enhancing data locality and scalability while reducing network latency. Lightweight frameworks such as \textbf{OpenWhisk Lite}~\cite{kravchenko_kpavelopenwhisk-light_2024} optimize resource allocation by decentralizing scheduling policies, minimizing cold starts and latency in edge setups~\cite{benjamin_wukong_2020}. Hybrid solutions like \textbf{OpenFaaS}~\cite{noauthor_openfaasfaas_2024} and \textbf{EdgeMatrix}~\cite{shen_edgematrix_2023} combine edge-cloud orchestration to balance resource utilization, retaining latency-sensitive functions at the edge while offloading non-critical workloads to the cloud. While these approaches improve flexibility, they face challenges in maintaining coordination and ensuring consistent performance across distributed nodes.


\section{Bellman Error Centering}

Centering operator $\mathcal{C}$ for a variable $x(s)$ is defined as follows:
\begin{equation}
\mathcal{C}x(s)\dot{=} x(s)-\mathbb{E}[x(s)]=x(s)-\sum_s{d_{s}x(s)},
\end{equation} 
where $d_s$ is the probability of $s$.
In vector form,
\begin{equation}
\begin{split}
\mathcal{C}\bm{x} &= \bm{x}-\mathbb{E}[x]\bm{1}\\
&=\bm{x}-\bm{x}^{\top}\bm{d}\bm{1},
\end{split}
\end{equation} 
where $\bm{1}$ is an all-ones vector.
For any vector $\bm{x}$ and $\bm{y}$ with a same distribution $\bm{d}$,
we have
\begin{equation}
\begin{split}
\mathcal{C}(\bm{x}+\bm{y})&=(\bm{x}+\bm{y})-(\bm{x}+\bm{y})^{\top}\bm{d}\bm{1}\\
&=\bm{x}-\bm{x}^{\top}\bm{d}\bm{1}+\bm{y}-\bm{y}^{\top}\bm{d}\bm{1}\\
&=\mathcal{C}\bm{x}+\mathcal{C}\bm{y}.
\end{split}
\end{equation}
\subsection{Revisit Reward Centering}


The update (\ref{src3}) is an unbiased estimate of the average reward
with  appropriate learning rate $\beta_t$ conditions.
\begin{equation}
\bar{r}_{t}\approx \lim_{n\rightarrow\infty}\frac{1}{n}\sum_{t=1}^n\mathbb{E}_{\pi}[r_t].
\end{equation}
That is 
\begin{equation}
r_t-\bar{r}_{t}\approx r_t-\lim_{n\rightarrow\infty}\frac{1}{n}\sum_{t=1}^n\mathbb{E}_{\pi}[r_t]= \mathcal{C}r_t.
\end{equation}
Then, the simple reward centering can be rewrited as:
\begin{equation}
V_{t+1}(s_t)=V_{t}(s_t)+\alpha_t [\mathcal{C}r_{t+1}+\gamma V_{t}(s_{t+1})-V_t(s_t)].
\end{equation}
Therefore, the simple reward centering is, in a strict sense, reward centering.

By definition of $\bar{\delta}_t=\delta_t-\bar{r}_{t}$,
let rewrite the update rule of the value-based reward centering as follows:
\begin{equation}
V_{t+1}(s_t)=V_{t}(s_t)+\alpha_t \rho_t (\delta_t-\bar{r}_{t}),
\end{equation}
where $\bar{r}_{t}$ is updated as:
\begin{equation}
\bar{r}_{t+1}=\bar{r}_{t}+\beta_t \rho_t(\delta_t-\bar{r}_{t}).
\label{vrc3}
\end{equation}
The update (\ref{vrc3}) is an unbiased estimate of the TD error
with  appropriate learning rate $\beta_t$ conditions.
\begin{equation}
\bar{r}_{t}\approx \mathbb{E}_{\pi}[\delta_t].
\end{equation}
That is 
\begin{equation}
\delta_t-\bar{r}_{t}\approx \mathcal{C}\delta_t.
\end{equation}
Then, the value-based reward centering can be rewrited as:
\begin{equation}
V_{t+1}(s_t)=V_{t}(s_t)+\alpha_t \rho_t \mathcal{C}\delta_t.
\label{tdcentering}
\end{equation}
Therefore, the value-based reward centering is no more,
 in a strict sense, reward centering.
It is, in a strict sense, \textbf{Bellman error centering}.

It is worth noting that this understanding is crucial, 
as designing new algorithms requires leveraging this concept.


\subsection{On the Fixpoint Solution}

The update rule (\ref{tdcentering}) is a stochastic approximation
of the following update:
\begin{equation}
\begin{split}
V_{t+1}&=V_{t}+\alpha_t [\bm{\mathcal{T}}^{\pi}\bm{V}-\bm{V}-\mathbb{E}[\delta]\bm{1}]\\
&=V_{t}+\alpha_t [\bm{\mathcal{T}}^{\pi}\bm{V}-\bm{V}-(\bm{\mathcal{T}}^{\pi}\bm{V}-\bm{V})^{\top}\bm{d}_{\pi}\bm{1}]\\
&=V_{t}+\alpha_t [\mathcal{C}(\bm{\mathcal{T}}^{\pi}\bm{V}-\bm{V})].
\end{split}
\label{tdcenteringVector}
\end{equation}
If update rule (\ref{tdcenteringVector}) converges, it is expected that
$\mathcal{C}(\mathcal{T}^{\pi}V-V)=\bm{0}$.
That is 
\begin{equation}
    \begin{split}
    \mathcal{C}\bm{V} &= \mathcal{C}\bm{\mathcal{T}}^{\pi}\bm{V} \\
    &= \mathcal{C}(\bm{R}^{\pi} + \gamma \mathbb{P}^{\pi} \bm{V}) \\
    &= \mathcal{C}\bm{R}^{\pi} + \gamma \mathcal{C}\mathbb{P}^{\pi} \bm{V} \\
    &= \mathcal{C}\bm{R}^{\pi} + \gamma (\mathbb{P}^{\pi} \bm{V} - (\mathbb{P}^{\pi} \bm{V})^{\top} \bm{d_{\pi}} \bm{1}) \\
    &= \mathcal{C}\bm{R}^{\pi} + \gamma (\mathbb{P}^{\pi} \bm{V} - \bm{V}^{\top} (\mathbb{P}^{\pi})^{\top} \bm{d_{\pi}} \bm{1}) \\  % 修正双重上标
    &= \mathcal{C}\bm{R}^{\pi} + \gamma (\mathbb{P}^{\pi} \bm{V} - \bm{V}^{\top} \bm{d_{\pi}} \bm{1}) \\
    &= \mathcal{C}\bm{R}^{\pi} + \gamma (\mathbb{P}^{\pi} \bm{V} - \bm{V}^{\top} \bm{d_{\pi}} \mathbb{P}^{\pi} \bm{1}) \\
    &= \mathcal{C}\bm{R}^{\pi} + \gamma (\mathbb{P}^{\pi} \bm{V} - \mathbb{P}^{\pi} \bm{V}^{\top} \bm{d_{\pi}} \bm{1}) \\
    &= \mathcal{C}\bm{R}^{\pi} + \gamma \mathbb{P}^{\pi} (\bm{V} - \bm{V}^{\top} \bm{d_{\pi}} \bm{1}) \\
    &= \mathcal{C}\bm{R}^{\pi} + \gamma \mathbb{P}^{\pi} \mathcal{C}\bm{V} \\
    &\dot{=} \bm{\mathcal{T}}_c^{\pi} \mathcal{C}\bm{V},
    \end{split}
    \label{centeredfixpoint}
    \end{equation}
where we defined $\bm{\mathcal{T}}_c^{\pi}$ as a centered Bellman operator.
We call equation (\ref{centeredfixpoint}) as centered Bellman equation.
And it is \textbf{centered fixpoint}.

For linear value function approximation, let define
\begin{equation}
\mathcal{C}\bm{V}_{\bm{\theta}}=\bm{\Pi}\bm{\mathcal{T}}_c^{\pi}\mathcal{C}\bm{V}_{\bm{\theta}}.
\label{centeredTDfixpoint}
\end{equation}
We call equation (\ref{centeredTDfixpoint}) as \textbf{centered TD fixpoint}.

\subsection{On-policy and Off-policy Centered TD Algorithms
with Linear Value Function Approximation}
Given the above centered TD fixpoint,
 mean squared centered Bellman error (MSCBE), is proposed as follows:
\begin{align*}
    \label{argminMSBEC}
 &\arg \min_{{\bm{\theta}}}\text{MSCBE}({\bm{\theta}}) \\
 &= \arg \min_{{\bm{\theta}}} \|\bm{\mathcal{T}}_c^{\pi}\mathcal{C}\bm{V}_{\bm{{\bm{\theta}}}}-\mathcal{C}\bm{V}_{\bm{{\bm{\theta}}}}\|_{\bm{D}}^2\notag\\
 &=\arg \min_{{\bm{\theta}}} \|\bm{\mathcal{T}}^{\pi}\bm{V}_{\bm{{\bm{\theta}}}} - \bm{V}_{\bm{{\bm{\theta}}}}-(\bm{\mathcal{T}}^{\pi}\bm{V}_{\bm{{\bm{\theta}}}} - \bm{V}_{\bm{{\bm{\theta}}}})^{\top}\bm{d}\bm{1}\|_{\bm{D}}^2\notag\\
 &=\arg \min_{{\bm{\theta}},\omega} \| \bm{\mathcal{T}}^{\pi}\bm{V}_{\bm{{\bm{\theta}}}} - \bm{V}_{\bm{{\bm{\theta}}}}-\omega\bm{1} \|_{\bm{D}}^2\notag,
\end{align*}
where $\omega$ is is used to estimate the expected value of the Bellman error.
% where $\omega$ is used to estimate $\mathbb{E}[\delta]$, $\omega \doteq \mathbb{E}[\mathbb{E}[\delta_t|S_t]]=\mathbb{E}[\delta]$ and $\delta_t$ is the TD error as follows:
% \begin{equation}
% \delta_t = r_{t+1}+\gamma
% {\bm{\theta}}_t^{\top}\bm{{\bm{\phi}}}_{t+1}-{\bm{\theta}}_t^{\top}\bm{{\bm{\phi}}}_t.
% \label{delta}
% \end{equation}
% $\mathbb{E}[\delta_t|S_t]$ is the Bellman error, and $\mathbb{E}[\mathbb{E}[\delta_t|S_t]]$ represents the expected value of the Bellman error.
% If $X$ is a random variable and $\mathbb{E}[X]$ is its expected value, then $X-\mathbb{E}[X]$ represents the centered form of $X$. 
% Therefore, we refer to $\mathbb{E}[\delta_t|S_t]-\mathbb{E}[\mathbb{E}[\delta_t|S_t]]$ as Bellman error centering and 
% $\mathbb{E}[(\mathbb{E}[\delta_t|S_t]-\mathbb{E}[\mathbb{E}[\delta_t|S_t]])^2]$ represents the the mean squared centered Bellman error, namely MSCBE.
% The meaning of (\ref{argminMSBEC}) is to minimize the mean squared centered Bellman error.
%The derivation of CTD is as follows.

First, the parameter  $\omega$ is derived directly based on
stochastic gradient descent:
\begin{equation}
\omega_{t+1}= \omega_{t}+\beta_t(\delta_t-\omega_t).
\label{omega}
\end{equation}

Then, based on stochastic semi-gradient descent, the update of 
the parameter ${\bm{\theta}}$ is as follows:
\begin{equation}
{\bm{\theta}}_{t+1}=
{\bm{\theta}}_{t}+\alpha_t(\delta_t-\omega_t)\bm{{\bm{\phi}}}_t.
\label{theta}
\end{equation}

We call (\ref{omega}) and (\ref{theta}) the on-policy centered
TD (CTD) algorithm. The convergence analysis with be given in
the following section.

In off-policy learning, we can simply multiply by the importance sampling
 $\rho$.
\begin{equation}
    \omega_{t+1}=\omega_{t}+\beta_t\rho_t(\delta_t-\omega_t),
    \label{omegawithrho}
\end{equation}
\begin{equation}
    {\bm{\theta}}_{t+1}=
    {\bm{\theta}}_{t}+\alpha_t\rho_t(\delta_t-\omega_t)\bm{{\bm{\phi}}}_t.
    \label{thetawithrho}
\end{equation}

We call (\ref{omegawithrho}) and (\ref{thetawithrho}) the off-policy centered
TD (CTD) algorithm.

% By substituting $\delta_t$ into Equations (\ref{omegawithrho}) and (\ref{thetawithrho}), 
% we can see that Equations (\ref{thetawithrho}) and (\ref{omegawithrho}) are formally identical 
% to the linear expressions of Equations (\ref{rewardcentering1}) and (\ref{rewardcentering2}), respectively. However, the meanings 
% of the corresponding parameters are entirely different.
% ${\bm{\theta}}_t$ is for approximating the discounted value function.
% $\bar{r_t}$ is an estimate of the average reward, while $\omega_t$ 
% is an estimate of the expected value of the Bellman error.
% $\bar{\delta_t}$ is the TD error for value-based reward centering, 
% whereas $\delta_t$ is the traditional TD error.

% This study posits that the CTD is equivalent to value-based reward 
% centering. However, CTD can be unified under a single framework 
% through an objective function, MSCBE, which also lays the 
% foundation for proving the algorithm's convergence. 
% Section 4 demonstrates that the CTD algorithm guarantees 
% convergence in the on-policy setting.

\subsection{Off-policy Centered TDC Algorithm with Linear Value Function Approximation}
The convergence of the  off-policy centered TD algorithm
may not be guaranteed.

To deal with this problem, we propose another new objective function, 
called mean squared projected centered Bellman error (MSPCBE), 
and derive Centered TDC algorithm (CTDC).

% We first establish some relationships between
%  the vector-matrix quantities and the relevant statistical expectation terms:
% \begin{align*}
%     &\mathbb{E}[(\delta({\bm{\theta}})-\mathbb{E}[\delta({\bm{\theta}})]){\bm{\phi}}] \\
%     &= \sum_s \mu(s) {\bm{\phi}}(s) \big( R(s) + \gamma \sum_{s'} P_{ss'} V_{\bm{\theta}}(s') - V_{\bm{\theta}}(s)  \\
%     &\quad \quad-\sum_s \mu(s)(R(s) + \gamma \sum_{s'} P_{ss'} V_{\bm{\theta}}(s') - V_{\bm{\theta}}(s))\big)\\
%     &= \bm{\Phi}^\top \mathbf{D} (\bm{TV}_{\bm{{\bm{\theta}}}} - \bm{V}_{\bm{{\bm{\theta}}}}-\omega\bm{1}),
% \end{align*}
% where $\omega$ is the expected value of the Bellman error and $\bm{1}$ is all-ones vector.

The specific expression of the objective function 
MSPCBE is as follows:
\begin{align}
    \label{MSPBECwithomega}
    &\arg \min_{{\bm{\theta}}}\text{MSPCBE}({\bm{\theta}})\notag\\ 
    % &= \arg \min_{{\bm{\theta}}}\big(\mathbb{E}[(\delta({\bm{\theta}}) - \mathbb{E}[\delta({\bm{\theta}})]) \bm{{\bm{\phi}}}]^\top \notag\\
    % &\quad \quad \quad\mathbb{E}[\bm{{\bm{\phi}}} \bm{{\bm{\phi}}}^\top]^{-1} \mathbb{E}[(\delta({\bm{\theta}}) - \mathbb{E}[\delta({\bm{\theta}})]) \bm{{\bm{\phi}}}]\big) \notag\\
    % &=\arg \min_{{\bm{\theta}},\omega}\mathbb{E}[(\delta({\bm{\theta}})-\omega) \bm{\bm{{\bm{\phi}}}}]^{\top} \mathbb{E}[\bm{\bm{{\bm{\phi}}}} \bm{\bm{{\bm{\phi}}}}^{\top}]^{-1}\mathbb{E}[(\delta({\bm{\theta}}) -\omega)\bm{\bm{{\bm{\phi}}}}]\\
    % &= \big(\bm{\Phi}^\top \mathbf{D} (\bm{TV}_{\bm{{\bm{\theta}}}} - \bm{V}_{\bm{{\bm{\theta}}}}-\omega\bm{1})\big)^\top (\bm{\Phi}^\top \mathbf{D} \bm{\Phi})^{-1} \notag\\
    % & \quad \quad \quad \bm{\Phi}^\top \mathbf{D} (\bm{TV}_{\bm{{\bm{\theta}}}} - \bm{V}_{\bm{{\bm{\theta}}}}-\omega\bm{1}) \notag\\
    % &= (\bm{TV}_{\bm{{\bm{\theta}}}} - \bm{V}_{\bm{{\bm{\theta}}}}-\omega\bm{1})^\top \mathbf{D} \bm{\Phi} (\bm{\Phi}^\top \mathbf{D} \bm{\Phi})^{-1} \notag\\
    % &\quad \quad \quad \bm{\Phi}^\top \mathbf{D} (\bm{TV}_{\bm{{\bm{\theta}}}} - \bm{V}_{\bm{{\bm{\theta}}}}-\omega\bm{1})\notag\\
    % &= (\bm{TV}_{\bm{{\bm{\theta}}}} - \bm{V}_{\bm{{\bm{\theta}}}}-\omega\bm{1})^\top {\bm{\Pi}}^\top \mathbf{D} {\bm{\Pi}} (\bm{TV}_{\bm{{\bm{\theta}}}} - \bm{V}_{\bm{{\bm{\theta}}}}-\omega\bm{1}) \notag\\
    &= \arg \min_{{\bm{\theta}}} \|\bm{\Pi}\bm{\mathcal{T}}_c^{\pi}\mathcal{C}\bm{V}_{\bm{{\bm{\theta}}}}-\mathcal{C}\bm{V}_{\bm{{\bm{\theta}}}}\|_{\bm{D}}^2\notag\\
    &= \arg \min_{{\bm{\theta}}} \|\bm{\Pi}(\bm{\mathcal{T}}_c^{\pi}\mathcal{C}\bm{V}_{\bm{{\bm{\theta}}}}-\mathcal{C}\bm{V}_{\bm{{\bm{\theta}}}})\|_{\bm{D}}^2\notag\\
    &= \arg \min_{{\bm{\theta}},\omega}\| {\bm{\Pi}} (\bm{\mathcal{T}}^{\pi}\bm{V}_{\bm{{\bm{\theta}}}} - \bm{V}_{\bm{{\bm{\theta}}}}-\omega\bm{1}) \|_{\bm{D}}^2\notag.
\end{align}
In the process of computing the gradient of the MSPCBE with respect to ${\bm{\theta}}$, 
$\omega$ is treated as a constant.
So, the derivation process of CTDC is the same 
as for the TDC algorithm \cite{sutton2009fast}, the only difference is that the original $\delta$ is replaced by $\delta-\omega$.
Therefore, the updated formulas of the centered TDC  algorithm are as follows:
\begin{equation}
 \bm{{\bm{\theta}}}_{k+1}=\bm{{\bm{\theta}}}_{k}+\alpha_{k}[(\delta_{k}- \omega_k) \bm{\bm{{\bm{\phi}}}}_k\\
 - \gamma\bm{\bm{{\bm{\phi}}}}_{k+1}(\bm{\bm{{\bm{\phi}}}}^{\top}_k \bm{u}_{k})],
\label{thetavmtdc}
\end{equation}
\begin{equation}
 \bm{u}_{k+1}= \bm{u}_{k}+\zeta_{k}[\delta_{k}-\omega_k - \bm{\bm{{\bm{\phi}}}}^{\top}_k \bm{u}_{k}]\bm{\bm{{\bm{\phi}}}}_k,
\label{uvmtdc}
\end{equation}
and
\begin{equation}
 \omega_{k+1}= \omega_{k}+\beta_k (\delta_k- \omega_k).
 \label{omegavmtdc}
\end{equation}
This algorithm is derived to work 
with a given set of sub-samples—in the form of 
triples $(S_k, R_k, S'_k)$ that match transitions 
from both the behavior and target policies. 

% \subsection{Variance Minimization ETD Learning: VMETD}
% Based on the off-policy TD algorithm, a scalar, $F$,  
% is introduced to obtain the ETD algorithm, 
% which ensures convergence under off-policy 
% conditions. This paper further introduces a scalar, 
% $\omega$, based on the ETD algorithm to obtain VMETD.
% VMETD by the following update:
% \begin{equation}
% \label{fvmetd}
%  F_t \leftarrow \gamma \rho_{t-1}F_{t-1}+1,
% \end{equation}
% \begin{equation}
%  \label{thetavmetd}
%  {{\bm{\theta}}}_{t+1}\leftarrow {{\bm{\theta}}}_t+\alpha_t (F_t \rho_t\delta_t - \omega_{t}){\bm{{\bm{\phi}}}}_t,
% \end{equation}
% \begin{equation}
%  \label{omegavmetd}
%  \omega_{t+1} \leftarrow \omega_t+\beta_t(F_t  \rho_t \delta_t - \omega_t),
% \end{equation}
% where $\rho_t =\frac{\pi(A_t | S_t)}{\mu(A_t | S_t)}$ and $\omega$ is used to estimate $\mathbb{E}[F \rho\delta]$, i.e., $\omega \doteq \mathbb{E}[F \rho\delta]$.

% (\ref{thetavmetd}) can be rewritten as
% \begin{equation*}
%  \begin{array}{ccl}
%  {{\bm{\theta}}}_{t+1}&\leftarrow& {{\bm{\theta}}}_t+\alpha_t (F_t \rho_t\delta_t - \omega_t){\bm{{\bm{\phi}}}}_t -\alpha_t \omega_{t+1}{\bm{{\bm{\phi}}}}_t\\
%   &=&{{\bm{\theta}}}_{t}+\alpha_t(F_t\rho_t\delta_t-\mathbb{E}_{\mu}[F_t\rho_t\delta_t|{{\bm{\theta}}}_t]){\bm{{\bm{\phi}}}}_t\\
%  &=&{{\bm{\theta}}}_t+\alpha_t F_t \rho_t (r_{t+1}+\gamma {{\bm{\theta}}}_t^{\top}{\bm{{\bm{\phi}}}}_{t+1}-{{\bm{\theta}}}_t^{\top}{\bm{{\bm{\phi}}}}_t){\bm{{\bm{\phi}}}}_t\\
%  & & \hspace{2em} -\alpha_t \mathbb{E}_{\mu}[F_t \rho_t \delta_t]{\bm{{\bm{\phi}}}}_t\\
%  &=& {{\bm{\theta}}}_t+\alpha_t \{\underbrace{(F_t\rho_tr_{t+1}-\mathbb{E}_{\mu}[F_t\rho_t r_{t+1}]){\bm{{\bm{\phi}}}}_t}_{{b}_{\text{VMETD},t}}\\
%  &&\hspace{-7em}- \underbrace{(F_t\rho_t{\bm{{\bm{\phi}}}}_t({\bm{{\bm{\phi}}}}_t-\gamma{\bm{{\bm{\phi}}}}_{t+1})^{\top}-{\bm{{\bm{\phi}}}}_t\mathbb{E}_{\mu}[F_t\rho_t ({\bm{{\bm{\phi}}}}_t-\gamma{\bm{{\bm{\phi}}}}_{t+1})]^{\top})}_{\textbf{A}_{\text{VMETD},t}}{{\bm{\theta}}}_t\}.
%  \end{array}
% \end{equation*}
% Therefore, 
% \begin{equation*}
%  \begin{array}{ccl}
%   &&\textbf{A}_{\text{VMETD}}\\
%   &=&\lim_{t \rightarrow \infty} \mathbb{E}[\textbf{A}_{\text{VMETD},t}]\\
%   &=& \lim_{t \rightarrow \infty} \mathbb{E}_{\mu}[F_t \rho_t {\bm{{\bm{\phi}}}}_t ({\bm{{\bm{\phi}}}}_t - \gamma {\bm{{\bm{\phi}}}}_{t+1})^{\top}]\\  
%   &&\hspace{1em}- \lim_{t\rightarrow \infty} \mathbb{E}_{\mu}[  {\bm{{\bm{\phi}}}}_t]\mathbb{E}_{\mu}[F_t \rho_t ({\bm{{\bm{\phi}}}}_t - \gamma {\bm{{\bm{\phi}}}}_{t+1})]^{\top}\\
%   &=& \lim_{t \rightarrow \infty} \mathbb{E}_{\mu}[{\bm{{\bm{\phi}}}}_tF_t \rho_t ({\bm{{\bm{\phi}}}}_t - \gamma {\bm{{\bm{\phi}}}}_{t+1})^{\top}]\\   
%   &&\hspace{1em}-\lim_{t \rightarrow \infty} \mathbb{E}_{\mu}[ {\bm{{\bm{\phi}}}}_t]\lim_{t \rightarrow \infty}\mathbb{E}_{\mu}[F_t \rho_t ({\bm{{\bm{\phi}}}}_t - \gamma {\bm{{\bm{\phi}}}}_{t+1})]^{\top}\\
%   && \hspace{-2em}=\sum_{s} d_{\mu}(s)\lim_{t \rightarrow \infty}\mathbb{E}_{\mu}[F_t|S_t = s]\mathbb{E}_{\mu}[\rho_t\bm{{\bm{\phi}}}_t(\bm{{\bm{\phi}}}_t - \gamma \bm{{\bm{\phi}}}_{t+1})^{\top}|S_t= s]\\   
%   &&\hspace{1em}-\sum_{s} d_{\mu}(s)\bm{{\bm{\phi}}}(s)\sum_{s} d_{\mu}(s)\lim_{t \rightarrow \infty}\mathbb{E}_{\mu}[F_t|S_t = s]\\
%   &&\hspace{7em}\mathbb{E}_{\mu}[\rho_t(\bm{{\bm{\phi}}}_t - \gamma \bm{{\bm{\phi}}}_{t+1})^{\top}|S_t = s]\\
%   &=& \sum_{s} f(s)\mathbb{E}_{\pi}[\bm{{\bm{\phi}}}_t(\bm{{\bm{\phi}}}_t- \gamma \bm{{\bm{\phi}}}_{t+1})^{\top}|S_t = s]\\   
%   &&\hspace{1em}-\sum_{s} d_{\mu}(s)\bm{{\bm{\phi}}}(s)\sum_{s} f(s)\mathbb{E}_{\pi}[(\bm{{\bm{\phi}}}_t- \gamma \bm{{\bm{\phi}}}_{t+1})^{\top}|S_t = s]\\
%   &=&\sum_{s} f(s) \bm{\bm{{\bm{\phi}}}}(s)(\bm{\bm{{\bm{\phi}}}}(s) - \gamma \sum_{s'}[\textbf{P}_{\pi}]_{ss'}\bm{\bm{{\bm{\phi}}}}(s'))^{\top}  \\
%   &&-\sum_{s} d_{\mu}(s) {\bm{{\bm{\phi}}}}(s) * \sum_{s} f(s)({\bm{{\bm{\phi}}}}(s) - \gamma \sum_{s'}[\textbf{P}_{\pi}]_{ss'}{\bm{{\bm{\phi}}}}(s'))^{\top}\\
%   &=&{\bm{\bm{\Phi}}}^{\top} \textbf{F} (\textbf{I} - \gamma \textbf{P}_{\pi}) \bm{\bm{\Phi}} - {\bm{\bm{\Phi}}}^{\top} {d}_{\mu} {f}^{\top} (\textbf{I} - \gamma \textbf{P}_{\pi}) \bm{\bm{\Phi}}  \\
%   &=&{\bm{\bm{\Phi}}}^{\top} (\textbf{F} - {d}_{\mu} {f}^{\top}) (\textbf{I} - \gamma \textbf{P}_{\pi}){\bm{\bm{\Phi}}} \\
%   &=&{\bm{\bm{\Phi}}}^{\top} (\textbf{F} (\textbf{I} - \gamma \textbf{P}_{\pi})-{d}_{\mu} {f}^{\top} (\textbf{I} - \gamma \textbf{P}_{\pi})){\bm{\bm{\Phi}}} \\
%   &=&{\bm{\bm{\Phi}}}^{\top} (\textbf{F} (\textbf{I} - \gamma \textbf{P}_{\pi})-{d}_{\mu} {d}_{\mu}^{\top} ){\bm{\bm{\Phi}}},
%  \end{array}
% \end{equation*}
% \begin{equation*}
%  \begin{array}{ccl}
%   &&{b}_{\text{VMETD}}\\
%   &=&\lim_{t \rightarrow \infty} \mathbb{E}[{b}_{\text{VMETD},t}]\\
%   &=& \lim_{t \rightarrow \infty} \mathbb{E}_{\mu}[F_t\rho_tR_{t+1}{\bm{{\bm{\phi}}}}_t]\\
%   &&\hspace{2em} - \lim_{t\rightarrow \infty} \mathbb{E}_{\mu}[{\bm{{\bm{\phi}}}}_t]\mathbb{E}_{\mu}[F_t\rho_kR_{k+1}]\\  
%   &=& \lim_{t \rightarrow \infty} \mathbb{E}_{\mu}[{\bm{{\bm{\phi}}}}_tF_t\rho_tr_{t+1}]\\
%   &&\hspace{2em} - \lim_{t\rightarrow \infty} \mathbb{E}_{\mu}[  {\bm{{\bm{\phi}}}}_t]\mathbb{E}_{\mu}[{\bm{{\bm{\phi}}}}_t]\mathbb{E}_{\mu}[F_t\rho_tr_{t+1}]\\ 
%   &=& \lim_{t \rightarrow \infty} \mathbb{E}_{\mu}[{\bm{{\bm{\phi}}}}_tF_t\rho_tr_{t+1}]\\
%   &&\hspace{2em} - \lim_{t \rightarrow \infty} \mathbb{E}_{\mu}[ {\bm{{\bm{\phi}}}}_t]\lim_{t \rightarrow \infty}\mathbb{E}_{\mu}[F_t\rho_tr_{t+1}]\\  
%   &=&\sum_{s} f(s) {\bm{{\bm{\phi}}}}(s)r_{\pi} - \sum_{s} d_{\mu}(s) {\bm{{\bm{\phi}}}}(s) * \sum_{s} f(s)r_{\pi}  \\
%   &=&\bm{\bm{\bm{\Phi}}}^{\top}(\textbf{F}-{d}_{\mu} {f}^{\top}){r}_{\pi}.
%  \end{array}
% \end{equation*}


\section{Quantization Framework}
\label{section:quantization}
\begin{comment}
Our quantization framework for accelerating large language model (LLM) inference introduces a novel, selective approach focused on reducing critical delay by prioritizing low-critical-delay weights. The design is optimized to maintain model accuracy while improving inference speed through adaptive quantization at multiple levels, including per-tile, per-channel, and layer sensitivity granularity. This methodology addresses computational efficiency challenges in high-dimensional LLMs by targeting only the weights that most impact inference delays.

\subsection{Uniform and Non-Uniform Quantization} 
% Previous studies have demonstrated that the primary goal of quantization techniques is to minimize the Mean Square Error (MSE) between the original and the quantized model.
Quantization is a powerful technique for reducing the memory footprint of neural network models by decreasing the bitwidth of their parameters. The literature on quantization can be broadly categorized into two main approaches: uniform and non-uniform methods.

\noindent \textbf{Uniform Quantization} Uniform quantization maps a continuous range of values to discrete levels that are evenly spaced. In a $b$-bit quantization scheme, the data $x$ is mapped to $2^b$ distinct levels. The quantization step $s$, also known as the scaling factor, is defined as:

\begin{equation}
s = \frac{x_{\text{max}} - x_{\text{min}}}{2^b}
\end{equation}

The quantization function $q(x)$ maps the data $x$ to the nearest quantization level, given by:
\begin{equation}
q(x) = x_{\text{min}} + is \quad \text{for} \quad i = 0, 1, 2, \ldots, 2^b-1
\end{equation}

This approach ensures that each quantized level is equidistant from its neighbors, resulting in a fixed-length representation for each interval. 

\noindent {\textbf{Non-uniform Quantization}} Non-uniform quantization, in contrast, allows for variable-sized intervals. Let $T = \{t_0 = -\infty, t_1, t_2, \ldots, t_{M-1}, t_M = +\infty\}$ be the set of thresholds that define the boundaries between quantization intervals. These thresholds partition the source range $X$ into $M+1$ disjoint regions $R_k = [t_{k-1}, t_k)$, for $k = 1, 2, \ldots, M+1$. 

The quantization function $q(x)$ assigns an input signal $x$ to the closest representation level $y_k$ based on which interval $R_k$ it falls into:
\begin{equation}
q(x) = y_k \quad \text{if} \quad x \in R_k
\end{equation}

The key difference is that the decision thresholds $(T)$ are not necessarily equally spaced, allowing for a more flexible allocation of quantization levels based on the data distribution within the source range $X$.
\end{comment}

Our quantization framework for LLM inference introduces a timing-aware strategy, detailed in Algorithm~\ref{quantization}, which prioritizes weights with low critical-path delays to minimize latency while preserving model fidelity. The adaptive method operates across levels and layer sensitivity, optimizing performance by focusing on weights most critical to efficiency. The framework comprises three key components: \circled{1} sensitivity-aware uniform quantization to identify and preserve critical weights \textit{(Lines 1-3)}, \circled{2} critical-path delay aware non-uniform quantization to optimize weight patterns for hardware efficiency \textit{(Lines 4-10)}, and \circled{3} adaptive DVFS to maximize performance by matching quantization levels with optimal operating frequencies.

\subsection{Sensitivity-aware Uniform Quantization}
\begin{figure}[t!]
	\scriptsize
	\centering
	\section{Sensitivity Analyses of WLS}

While the multi-factor weighted least squares (WLS) model provides a systematic approach to control for multiple confounders, datasets often exhibit \emph{imbalanced subgroup distributions} or heterogeneity that can affect statistical inferences.  To ensure the robustness of our parameter estimates, we perform a bootstrap-based sensitivity analysis.

\subsection{Bootstrap-Based Parameter Estimation}

Parameter estimates can be sensitive to random fluctuations in the data.  To assess this sensitivity, we use bootstrapping. We create many "new" datasets by resampling with replacement from the original dataset (keeping the same overall size).  We fit the WLS model on each bootstrap sample and aggregate the resulting estimates. This approach provides a distribution for each parameter. We report the mean and standard deviation of these bootstrap estimates, along with the 95\% confidence interval (CI) based on the 2.5th and 97.5th percentiles of the bootstrap distribution. We then check whether the original parameter estimate falls within this bootstrap CI ("Coverage"). If the original estimate lies within the CI, it provides evidence that the estimate is stable to sampling variability, and thus robust to the specific composition of the sample.

\subsection{Summary of Sensitivity Findings}

The bootstrap results, presented in Tables \ref{tab:bootstrap_comparison_binoculars} through \ref{tab:radar_bootstrap}, show that for all detectors and all parameters, the original coefficient estimates lie within the 95\% confidence intervals derived from the bootstrap resampling.  This indicates strong stability of the parameter estimates. The relatively narrow confidence intervals and consistent "WITHIN CI" coverage across all parameters and detectors provide substantial evidence that our main WLS findings are robust to sampling variability. This strengthens our confidence in the reported effects of CEFR level, sex, academic genre, and language environment on detector accuracy.

\begin{table*}[htbp]
  \centering
  \resizebox{\textwidth}{!}{%
    \begin{tabular}{lcccccc}
      \hline
      \textbf{Parameter} & \textbf{Original Value} & \textbf{Bootstrap Mean} & \textbf{Bootstrap Std} & \textbf{CI Lower} & \textbf{CI Upper} & \textbf{Coverage} \\
      \hline
      Intercept & 0.9482 & 0.9480 & 0.0079 & 0.9318 & 0.9625 & WITHIN CI \\
      C(cefr)[T.B1\_1] & -0.0039 & -0.0040 & 0.0085 & -0.0204 & 0.0132 & WITHIN CI \\
      C(cefr)[T.B1\_2] & -0.0007 & -0.0006 & 0.0075 & -0.0149 & 0.0150 & WITHIN CI \\
      C(cefr)[T.B2\_0] & 0.0025 & 0.0025 & 0.0078 & -0.0125 & 0.0172 & WITHIN CI \\
      C(cefr)[T.XX\_0] & -0.0501 & -0.0499 & 0.0049 & -0.0592 & -0.0400 & WITHIN CI \\
      C(Sex)[T.M] & 0.0010 & 0.0011 & 0.0054 & -0.0096 & 0.0114 & WITHIN CI \\
      C(academic\_genre)[T.Life Sciences] & -0.0075 & -0.0073 & 0.0082 & -0.0224 & 0.0086 & WITHIN CI \\
      C(academic\_genre)[T.Sciences \& Technology] & 0.0016 & 0.0018 & 0.0072 & -0.0126 & 0.0156 & WITHIN CI \\
      C(academic\_genre)[T.Social Sciences] & 0.0016 & 0.0018 & 0.0068 & -0.0110 & 0.0149 & WITHIN CI \\
      C(language\_env)[T.ESL] & -0.0144 & -0.0143 & 0.0056 & -0.0247 & -0.0038 & WITHIN CI \\
      C(language\_env)[T.NS] & -0.0501 & -0.0499 & 0.0049 & -0.0592 & -0.0400 & WITHIN CI \\
      \hline
    \end{tabular}}
   \caption{Bootstrap Sensitivity Analysis for detector: \texttt{binoculars}}
  \label{tab:bootstrap_comparison_binoculars}
\end{table*}






\begin{table*}[ht]
\centering
\resizebox{\textwidth}{!}{%
\begin{tabular}{lcccccc}
\toprule
\textbf{Parameter} & \textbf{Original Value} & \textbf{Bootstrap Mean} & \textbf{Bootstrap Std} & \textbf{CI Lower} & \textbf{CI Upper} & \textbf{Coverage} \\
\midrule
Intercept                                   & 0.7480 & 0.7480 & 0.0157 & 0.7179 & 0.7790 & WITHIN CI \\
C(Sex)[T.M]                                & 0.0035 & 0.0032 & 0.0103 & --0.0176 & 0.0226 & WITHIN CI \\
C(cefr)[T.B1\_1]                           & --0.0073 & --0.0074 & 0.0180 & --0.0403 & 0.0294 & WITHIN CI \\
C(cefr)[T.B1\_2]                           & --0.0103 & --0.0102 & 0.0158 & --0.0413 & 0.0214 & WITHIN CI \\
C(cefr)[T.B2\_0]                           & --0.0435 & --0.0432 & 0.0155 & --0.0728 & --0.0114 & WITHIN CI \\
C(cefr)[T.XX\_0]                           & --0.0071 & --0.0071 & 0.0084 & --0.0245 & 0.0100 & WITHIN CI \\
C(academic\_genre)[T.Life Sciences]         & 0.0286 & 0.0287 & 0.0143 & 0.0007 & 0.0557 & WITHIN CI \\
C(academic\_genre)[T.Sciences \& Technology]  & 0.0084 & 0.0088 & 0.0137 & --0.0179 & 0.0350 & WITHIN CI \\
C(academic\_genre)[T.Social Sciences]        & 0.0083 & 0.0085 & 0.0120 & --0.0167 & 0.0312 & WITHIN CI \\
C(language\_env)[T.ESL]                      & --0.0014 & --0.0012 & 0.0094 & --0.0201 & 0.0185 & WITHIN CI \\
C(language\_env)[T.NS]                       & --0.0071 & --0.0071 & 0.0084 & --0.0245 & 0.0100 & WITHIN CI \\
\bottomrule
\end{tabular}
}
\caption{Bootstrap Sensitivity Analysis for detector: \texttt{chatgpt-roberta}}
\label{tab:bootstrap_chatgpt-roberta}
\end{table*}











\begin{table*}[ht]
\centering
\resizebox{\textwidth}{!}{%
\begin{tabular}{lcccccc}
\toprule
\textbf{Parameter} & \textbf{Original Value} & \textbf{Bootstrap Mean} & \textbf{Bootstrap Std} & \textbf{CI Lower} & \textbf{CI Upper} & \textbf{Coverage} \\
\midrule
Intercept                                   & 0.7944    & 0.7948   & 0.0098  & 0.7754  & 0.8133 & WITHIN CI \\
C(academic\_genre)[T.Life Sciences]         & --0.0397  & --0.0398 & 0.0108  & --0.0611 & --0.0187 & WITHIN CI \\
C(academic\_genre)[T.Sciences \& Technology]  & --0.0185  & --0.0189 & 0.0093  & --0.0370 & --0.0010 & WITHIN CI \\
C(academic\_genre)[T.Social Sciences]        & --0.0060  & --0.0063 & 0.0087  & --0.0233 & 0.0101 & WITHIN CI \\
C(cefr)[T.B1\_1]                             & 0.0127    & 0.0129   & 0.0104  & --0.0074 & 0.0347 & WITHIN CI \\
C(cefr)[T.B1\_2]                             & 0.0284    & 0.0282   & 0.0093  & 0.0103  & 0.0468 & WITHIN CI \\
C(cefr)[T.B2\_0]                             & 0.0345    & 0.0339   & 0.0106  & 0.0132  & 0.0554 & WITHIN CI \\
C(cefr)[T.XX\_0]                             & --0.0223  & --0.0222 & 0.0059  & --0.0338 & --0.0104 & WITHIN CI \\
C(Sex)[T.M]                                 & --0.0068  & --0.0069 & 0.0075  & --0.0210 & 0.0079 & WITHIN CI \\
C(language\_env)[T.ESL]                       & --0.0077  & --0.0075 & 0.0071  & --0.0208 & 0.0065 & WITHIN CI \\
C(language\_env)[T.NS]                        & --0.0223  & --0.0222 & 0.0059  & --0.0338 & --0.0104 & WITHIN CI \\
\bottomrule
\end{tabular}
}
\caption{Bootstrap Sensitivity Analysis for detector: \texttt{detectgpt}}
\label{tab:bootstrap_detectgpt}
\end{table*}













\begin{table*}[ht]
\centering
\resizebox{\textwidth}{!}{%
\begin{tabular}{lcccccc}
\toprule
\textbf{Parameter} & \textbf{Original Value} & \textbf{Bootstrap Mean} & \textbf{Bootstrap Std} & \textbf{CI Lower} & \textbf{CI Upper} & \textbf{Coverage} \\
\midrule
Intercept                                   & 0.8963    & 0.8963   & 0.0084  & 0.8796  & 0.9124 & WITHIN CI \\
C(cefr)[T.B1\_1]                            & -0.0076   & -0.0071  & 0.0093  & -0.0241 & 0.0105 & WITHIN CI \\
C(cefr)[T.B1\_2]                            & 0.0060    & 0.0059   & 0.0082  & -0.0107 & 0.0222 & WITHIN CI \\
C(cefr)[T.B2\_0]                            & 0.0180    & 0.0181   & 0.0083  & 0.0023  & 0.0346 & WITHIN CI \\
C(cefr)[T.XX\_0]                            & -0.0214   & -0.0214  & 0.0049  & -0.0309 & -0.0119 & WITHIN CI \\
C(Sex)[T.M]                                & -0.0003   & -0.0005  & 0.0062  & -0.0130 & 0.0113 & WITHIN CI \\
C(academic\_genre)[T.Life Sciences]          & 0.0127    & 0.0123   & 0.0077  & -0.0025 & 0.0281 & WITHIN CI \\
C(academic\_genre)[T.Sciences \& Technology] & -0.0012   & -0.0013  & 0.0080  & -0.0174 & 0.0144 & WITHIN CI \\
C(academic\_genre)[T.Social Sciences]        & 0.0070    & 0.0073   & 0.0071  & -0.0068 & 0.0216 & WITHIN CI \\
C(language\_env)[T.ESL]                      & -0.0021   & -0.0021  & 0.0060  & -0.0141 & 0.0102 & WITHIN CI \\
C(language\_env)[T.NS]                       & -0.0214   & -0.0214  & 0.0049  & -0.0309 & -0.0119 & WITHIN CI \\
\bottomrule
\end{tabular}
}
\caption{Bootstrap Sensitivity Analysis for detector: \texttt{fastdetectgpt}}
\label{tab:bootstrap_fastdetectgpt}
\end{table*}









\begin{table*}[ht]
\centering
\resizebox{\textwidth}{!}{%
\begin{tabular}{lcccccc}
\toprule
\textbf{Parameter} & \textbf{Original Value} & \textbf{Bootstrap Mean} & \textbf{Bootstrap Std} & \textbf{CI Lower} & \textbf{CI Upper} & \textbf{Coverage} \\
\midrule
Intercept                                   & 0.4873    & 0.4873   & 0.0026  & 0.4824  & 0.4926 & WITHIN CI \\
C(cefr)[T.B1\_1]                            & -0.0031   & -0.0031  & 0.0025  & -0.0081 & 0.0017 & WITHIN CI \\
C(cefr)[T.B1\_2]                            & -0.0087   & -0.0087  & 0.0025  & -0.0137 & -0.0039 & WITHIN CI \\
C(cefr)[T.B2\_0]                            & -0.0045   & -0.0045  & 0.0028  & -0.0102 & 0.0008 & WITHIN CI \\
C(cefr)[T.XX\_0]                            & 0.0102    & 0.0102   & 0.0013  & 0.0077  & 0.0127 & WITHIN CI \\
C(Sex)[T.M]                                & 0.0018    & 0.0019   & 0.0016  & -0.0012 & 0.0049 & WITHIN CI \\
C(academic\_genre)[T.Life Sciences]          & -0.0141   & -0.0143  & 0.0040  & -0.0220 & -0.0066 & WITHIN CI \\
C(academic\_genre)[T.Sciences \& Technology] & -0.0060   & -0.0060  & 0.0020  & -0.0100 & -0.0021 & WITHIN CI \\
C(academic\_genre)[T.Social Sciences]        & -0.0017   & -0.0017  & 0.0020  & -0.0057 & 0.0022 & WITHIN CI \\
C(language\_env)[T.ESL]                      & -0.0035   & -0.0035  & 0.0017  & -0.0069 & -0.0001 & WITHIN CI \\
C(language\_env)[T.NS]                       & 0.0102    & 0.0102   & 0.0013  & 0.0077  & 0.0127 & WITHIN CI \\
\bottomrule
\end{tabular}
}
\caption{Bootstrap Sensitivity Analysis for detector: \texttt{fastdetectllm}}
\label{tab:bootstrap_fastdetectllm}
\end{table*}







\begin{table*}[ht]
\centering
\resizebox{\textwidth}{!}{%
\begin{tabular}{lcccccc}
\toprule
\textbf{Parameter} & \textbf{Original Value} & \textbf{Bootstrap Mean} & \textbf{Bootstrap Std} & \textbf{CI Lower} & \textbf{CI Upper} & \textbf{Coverage} \\
\midrule
Intercept                                    & 0.8366  & 0.8369  & 0.0175  & 0.8039  & 0.8712 & WITHIN CI \\
C(cefr)[T.B1\_1]                             & -0.0061 & -0.0064 & 0.0189  & -0.0445 & 0.0289 & WITHIN CI \\
C(cefr)[T.B1\_2]                             & -0.0160 & -0.0158 & 0.0169  & -0.0481 & 0.0179 & WITHIN CI \\
C(cefr)[T.B2\_0]                             & -0.0399 & -0.0405 & 0.0175  & -0.0755 & -0.0064 & WITHIN CI \\
C(cefr)[T.XX\_0]                             & -0.1186 & -0.1183 & 0.0107  & -0.1383 & -0.0979 & WITHIN CI \\
C(Sex)[T.M]                                  & -0.0078 & -0.0078 & 0.0130  & -0.0324 & 0.0178 & WITHIN CI \\
C(academic\_genre)[T.Life Sciences]           & 0.0065  & 0.0069  & 0.0185  & -0.0291 & 0.0427 & WITHIN CI \\
C(academic\_genre)[T.Sciences \& Technology]  & 0.0068  & 0.0068  & 0.0172  & -0.0253 & 0.0405 & WITHIN CI \\
C(academic\_genre)[T.Social Sciences]         & -0.0031 & -0.0033 & 0.0154  & -0.0333 & 0.0257 & WITHIN CI \\
C(language\_env)[T.ESL]                       & -0.0020 & -0.0021 & 0.0133  & -0.0273 & 0.0241 & WITHIN CI \\
C(language\_env)[T.NS]                        & -0.1186 & -0.1183 & 0.0107  & -0.1383 & -0.0979 & WITHIN CI \\
\bottomrule
\end{tabular}
}
\caption{Bootstrap Sensitivity Analysis for detector \texttt{gltr}}
\label{tab:gltr_bootstrap}
\end{table*}









\begin{table*}[ht]
\centering
\resizebox{\textwidth}{!}{%
\begin{tabular}{lcccccc}
\toprule
\textbf{Parameter} & \textbf{Original Value} & \textbf{Bootstrap Mean} & \textbf{Bootstrap Std} & \textbf{CI Lower} & \textbf{CI Upper} & \textbf{Coverage} \\
\midrule
Intercept                                    & 0.6493  & 0.6499  & 0.0129  & 0.6247  & 0.6755 & WITHIN CI \\
C(cefr)[T.B1\_1]                             & -0.0094 & -0.0096 & 0.0138  & -0.0373 & 0.0179 & WITHIN CI \\
C(cefr)[T.B1\_2]                             & -0.0612 & -0.0611 & 0.0127  & -0.0849 & -0.0366 & WITHIN CI \\
C(cefr)[T.B2\_0]                             & -0.0870 & -0.0872 & 0.0129  & -0.1118 & -0.0619 & WITHIN CI \\
C(cefr)[T.XX\_0]                             & -0.1011 & -0.1013 & 0.0070  & -0.1146 & -0.0873 & WITHIN CI \\
C(Sex)[T.M]                                  & -0.0043 & -0.0044 & 0.0088  & -0.0213 & 0.0126 & WITHIN CI \\
C(academic\_genre)[T.Life Sciences]          & 0.0248  & 0.0245  & 0.0126  & -0.0016 & 0.0490 & WITHIN CI \\
C(academic\_genre)[T.Sciences \& Technology] & 0.0373  & 0.0369  & 0.0111  & 0.0153  & 0.0597 & WITHIN CI \\
C(academic\_genre)[T.Social Sciences]        & -0.0084 & -0.0090 & 0.0103  & -0.0288 & 0.0115 & WITHIN CI \\
C(language\_env)[T.ESL]                       & 0.0011  & 0.0009  & 0.0084  & -0.0150 & 0.0176 & WITHIN CI \\
C(language\_env)[T.NS]                        & -0.1011 & -0.1013 & 0.0070  & -0.1146 & -0.0873 & WITHIN CI \\
\bottomrule
\end{tabular}
}
\caption{Bootstrap Sensitivity Analysis for detector \texttt{gpt2-base}}
\label{tab:gpt2base_bootstrap}
\end{table*}








\begin{table*}[ht]
\centering
\small
\resizebox{\textwidth}{!}{%
\begin{tabular}{lcccccc}
\toprule
\textbf{Parameter} & \textbf{Original Value} & \textbf{Bootstrap Mean} & \textbf{Bootstrap Std} & \textbf{CI Lower} & \textbf{CI Upper} & \textbf{Coverage} \\
\midrule
Intercept                                   & 0.5402   & 0.5412   & 0.0149   & 0.5127   & 0.5703 & WITHIN CI \\
C(cefr)[T.B1\_1]                            & 0.0274   & 0.0270   & 0.0159   & -0.0050  & 0.0583 & WITHIN CI \\
C(cefr)[T.B1\_2]                            & 0.0008   & 0.0003   & 0.0149   & -0.0266  & 0.0294 & WITHIN CI \\
C(cefr)[T.B2\_0]                            & 0.0060   & 0.0053   & 0.0159   & -0.0255  & 0.0359 & WITHIN CI \\
C(cefr)[T.XX\_0]                            & -0.0240  & -0.0244  & 0.0080   & -0.0393  & -0.0093 & WITHIN CI \\
C(Sex)[T.M]                                & -0.0026  & -0.0030  & 0.0097   & -0.0214  & 0.0150 & WITHIN CI \\
C(academic\_genre)[T.Life Sciences]         & 0.0082   & 0.0078   & 0.0145   & -0.0199  & 0.0352 & WITHIN CI \\
C(academic\_genre)[T.Sciences \& Technology] & 0.0303   & 0.0297   & 0.0133   & 0.0037   & 0.0555 & WITHIN CI \\
C(academic\_genre)[T.Social Sciences]       & -0.0165  & -0.0170  & 0.0125   & -0.0420  & 0.0066 & WITHIN CI \\
C(language\_env)[T.ESL]                      & 0.0119   & 0.0118   & 0.0102   & -0.0083  & 0.0321 & WITHIN CI \\
C(language\_env)[T.NS]                       & -0.0240  & -0.0244  & 0.0080   & -0.0393  & -0.0093 & WITHIN CI \\
\bottomrule
\end{tabular}
}
\caption{Bootstrap Sensitivity Analysis for detector \texttt{gpt2-large}}
\label{tab:gpt2large_bootstrap}
\end{table*}












\begin{table*}[ht]
\centering
\small
\resizebox{\textwidth}{!}{%
\begin{tabular}{lcccccc}
\toprule
\textbf{Parameter} & \textbf{Original Value} & \textbf{Bootstrap Mean} & \textbf{Bootstrap Std} & \textbf{CI Lower} & \textbf{CI Upper} & \textbf{Coverage} \\
\midrule
Intercept                                   & 0.5402  & 0.5412  & 0.0039  & 0.4986  & 0.5139 & WITHIN CI \\
C(cefr)[T.B1\_1]                            & $-0.0102$ & $-0.0103$ & 0.0046  & $-0.0197$ & $-0.0019$ & WITHIN CI \\
C(cefr)[T.B1\_2]                            & $-0.0251$ & $-0.0250$ & 0.0039  & $-0.0331$ & $-0.0178$ & WITHIN CI \\
C(cefr)[T.B2\_0]                            & $-0.0482$ & $-0.0482$ & 0.0044  & $-0.0570$ & $-0.0398$ & WITHIN CI \\
C(cefr)[T.XX\_0]                            & $-0.0026$ & $-0.0026$ & 0.0019  & $-0.0064$ & 0.0009 & WITHIN CI \\
C(Sex)[T.M]                                 & 0.0033  & 0.0033  & 0.0026  & $-0.0018$ & 0.0083 & WITHIN CI \\
C(academic\_genre)[T.Life Sciences]          & 0.0202  & 0.0201  & 0.0052  & 0.0101  & 0.0306 & WITHIN CI \\
C(academic\_genre)[T.Sciences \& Technology] & 0.0046  & 0.0047  & 0.0031  & $-0.0015$ & 0.0106 & WITHIN CI \\
C(academic\_genre)[T.Social Sciences]        & 0.0034  & 0.0034  & 0.0029  & $-0.0022$ & 0.0090 & WITHIN CI \\
C(language\_env)[T.ESL]                       & 0.0187  & 0.0186  & 0.0021  & 0.0144  & 0.0226 & WITHIN CI \\
C(language\_env)[T.NS]                        & $-0.0026$ & $-0.0026$ & 0.0019  & $-0.0064$ & 0.0009 & WITHIN CI \\
\bottomrule
\end{tabular}
}
\caption{Bootstrap Sensitivity Analysis for detector \texttt{llmdet}}
\label{tab:llmdet_bootstrap}
\end{table*}





\begin{table*}[ht]
\centering
\small
\resizebox{\textwidth}{!}{%
\begin{tabular}{lcccccc}
\toprule
\textbf{Parameter} & \textbf{Original Value} & \textbf{Bootstrap Mean} & \textbf{Bootstrap Std} & \textbf{CI Lower} & \textbf{CI Upper} & \textbf{Coverage} \\
\midrule
Intercept & 0.6886 & 0.6890   & 0.0148   & 0.6611   & 0.7175 & WITHIN CI \\
C(cefr)[T.B1\_1] & 0.0381 & 0.0380   & 0.0159   & 0.0065   & 0.0681 & WITHIN CI \\
C(cefr)[T.B1\_2] & 0.0324 & 0.0324   & 0.0143   & 0.0027   & 0.0604 & WITHIN CI \\
C(cefr)[T.B2\_0] & 0.0134 & 0.0129   & 0.0144   & $-0.0165$ & 0.0403 & WITHIN CI \\
C(cefr)[T.XX\_0] & -0.0200 & $-0.0201$& 0.0077   & $-0.0343$& $-0.0051$ & WITHIN CI \\
C(Sex)[T.M] & -0.0020 & $-0.0018$& 0.0105   & $-0.0222$& 0.0183 & WITHIN CI \\
C(academic\_genre)[T.Life Sciences] & -0.0409 & $-0.0409$& 0.0153   & $-0.0699$& $-0.0112$ & WITHIN CI \\
C(academic\_genre)[T.Sciences \& Technology] & 0.0021 & 0.0021 & 0.0136 & $-0.0253$& 0.0298 & WITHIN CI \\
C(academic\_genre)[T.Social Sciences] & -0.0040 & $-0.0043$& 0.0126   & $-0.0290$& 0.0194 & WITHIN CI \\
C(language\_env)[T.ESL] & -0.0215 & $-0.0215$& 0.0105   & $-0.0430$& $-0.0014$ & WITHIN CI \\
C(language\_env)[T.NS] & -0.0200 & $-0.0201$& 0.0077   & $-0.0343$& $-0.0051$ & WITHIN CI \\
\bottomrule
\end{tabular}
}
\caption{Bootstrap Sensitivity Analysis for detector \texttt{radar}}
\label{tab:radar_bootstrap}
\end{table*}



	\caption{Distribution highlighting sensitive weights important for accuracy.} 
	\label{fig:weight_distribution}
\end{figure}

The framework begins with a sensitivity analysis of model weights, identifying weights values that can tolerate quantization without significantly affecting accuracy, as outlined in Algorithm 1. The framework initially separates \textit{outliers} (outside the blue lines) and \textit{salient weights} (in red) from normal values, as shown in Fig.\ref{fig:weight_distribution}.

\noindent \textbf{Outliers \& Salient Weights:} We incorporate outlier removal to manage extreme weight values based on inter-quartile range scaling. To compute outliers in the weight distribution, we employ the 3$\sigma$ rule~\cite{olive}. Outliers are identified as values lying beyond three standard deviations from the mean. 

% This helps detect and manage extreme values that may affect model performance. 

% \noindent \textbf{Extremely-Salient Weights:} 
From the normal values obtained after this distribution, we rely on Taylor series expansion to estimate the most salient weights in the model. Following~\cite{squeezellm}, we use an approximation to the Hessian \(H \) based on the Fisher information matrix \( F \), which can be calculated over a sample dataset \( D \) as
\begin{equation}
F = \frac{1}{|D|} \sum_{d \in D} g_d g_d^{\top},
\label{eq:fisher}
\end{equation}
where \( g \) is the gradient and \( H \approx F \). This only requires computing the gradient for a set of samples. For each weight tensor \( W \), the weight sensitivity is computed as \(\Lambda_{W} = F \). Weights with higher \( \Lambda_{W} \) values are considered more salient due to their significant impact on the model's output. We preserve the top 0.05\% of the weights based on this criterion. Cumulatively, both outliers and extremely salient weight values correspond to less than 0.5\% of the total weight values. For this reason, we handle these weight values separately and apply per-channel quantization for this set of weight values, isolating them to maintain model precision.

\begin{algorithm}
\caption{Quantization Framework}
\label{quantization}
{\footnotesize
\begin{algorithmic}[1]
%\begin{footnotesize}
\Require calibration dataset $X$, pre-trained weight matrix $W$, gradient $G$
\Require number of bits $n$, tile size $t$, quantile threshold $k$, target frequencies $f_1$, $f_2$
\Ensure Quantized weight matrix $W_q$

\State $W_s, S \leftarrow \text{ExtractSalientValues}(W, G)$ \Comment{Isolate values with high saliency}
\State $W_o, O \leftarrow \text{ExtractOutliers}(W_s)$ \Comment{Separate outlier weights}
\State $W_{s,o}^{q} \leftarrow \text{Quantize}(W_{s} + W_{o})$  \Comment{Quantize outliers and salient weights}

\State $W_t \leftarrow 
\text{ReshapeIntoTiles}(\text{PadMatrix}(W_o, t), t)$ \Comment{Tile reshaping}

\State $\Lambda_{T_k} \leftarrow \text{CalculateTileSensitivities}(G)$ \Comment{Compute sensitivity for each tile}
\State $M_l, M_h \leftarrow \text{CreateMasks}(M, \text{ComputeAdaptiveK}(\Lambda_{T_k}, k))$ \Comment{Classify tiles as low or high sensitivity}
\State $W_{l,i}, W_{h,i} \leftarrow W_{t,i} \odot M_{l,i}, W_{t,i} \odot M_{h,i}$ \Comment{Apply masks}
\State $W_{l,i} \leftarrow \text{Quantize}(W_{l,i}, f_1)$ \Comment{Quantize low-sensitivity tiles}
\State $W_{h,i} \leftarrow \text{Quantize}(W_{h,i}, f_2)$  \Comment{Quantize high-sensitivity tiles}
\State $W_q \leftarrow W_{l,i}, W_{h,i}, W_{s,o}$
\State \Return $W_q$
%\end{footnotesize}
\end{algorithmic}
}
\end{algorithm}
%\vspace{-.5cm}

\subsection{Critical-path delay aware Non-Uniform Quantization} 
\label{section:crit_quant}
Uniform Quantization discretizes continuous values into \(2^b\) evenly spaced levels. On the other hand, non-uniform quantization adapts to the data distribution using variable interval sizes defined by thresholds \( T \), which partition the input range into regions \( R_k = [t_{k-1}, t_k) \). Each region \( R_k \) is assigned a representation level \( y_k \), where \( y_k \) is the quantized value corresponding to data points within \( R_k \). In this work, we leverage non-uniform quantization to more efficiently map the distribution of weights to specific values that reduce the critical-path delays (as discussed in Sec.\ref{section:motivation}), thereby optimizing frequency and energy consumption.

% To do so, we target weights impacting bottleneck layers are selectively quantized using e.g., attention and projection layers) are optimized for efficient inference.

\noindent \textbf{Tile-Based Sensitivity Analysis:} To optimize the model for efficient inference on hardware, the weight tensors are divided into fixed-size tiles (\(128 \times 128\) by default). Specifically, the sensitivity of each tile is evaluated 
as the sum of the absolute values of the gradients for each tile, normalized by the size of the tile, based on Eq.\ref{eq:fisher}. For a given $k$th tile \( T_k \), we compute a \textit{per-tile sensitivity score} \( \Lambda_{T_k} \) using a diagonal approximation of the Fisher information matrix:
\begin{equation}
\Lambda_{T_k} = \frac{\sum_{i,j} g_{k,i,j}^2}{\text{tile\_rows} \times \text{tile\_cols}}
\end{equation}

where \( g_{k,i,j} \) denotes the gradient of the loss with respect to each weight in the \( k \)-th tile, and \( \text{tile\_rows} \times \text{tile\_cols} \) represents the total number of elements within the tile. This score captures the average Fisher information across all weights in the tile, providing a quantitative measure of the tile's sensitivity in relation to its influence on the model's output.

\noindent \textbf{Tile Sensitivity Mapping}: 
To balance hardware efficiency and model accuracy, tiles in each layer are classified as \underline{low-sensitive} or \underline{high-sensitive} based on their relative importance. Determining a fixed sensitivity threshold for each layer is challenging, as weight distributions vary significantly across layers. To address this, we employ a dynamic tile sensitivity mapping strategy that adapts to the cumulative sensitivity distribution of each layer.

The process starts by computing the sensitivity of all tiles in a given layer, derived as the normalized sum of absolute gradient magnitudes within each tile. Sensitivities are then sorted in descending order to rank tiles by importance. A cumulative sum of these sorted sensitivities is calculated and normalized against the total layer sensitivity, generating a cumulative distribution curve from 0 to 1.

The mapping threshold \(k \) is derived from this curve and represents the fraction of tiles classified as low-sensitive, ensuring a specified percentage of total sensitivity (e.g., 95\%) is retained. Tiles contributing most to overall sensitivity are marked high-sensitive, while the rest are classified as low-sensitive. Mathematically, \(k \) is the ratio of the index where cumulative sensitivity exceeds the threshold to the total number of tiles, defaulting to 1.0 if no such index exists.

Once \(k \) is determined, boolean masks separate tiles into low- and high-sensitivity categories. Low-sensitivity tiles are quantized more aggressively, while high-sensitivity tiles retain higher precision to preserve performance. The adaptive quantization and computation flow based on the DVFS characteristics are described in detail in Sec.\ref{section:execution}.

% \subsection{Algorithm Design}


% \subsection{Tile-Based Weight Mapping and Distribution}

% To optimize hardware deployment, the framework reshapes weight tensors based on tile sizes suited for GPU and TPU hardware. Each weight matrix is partitioned into fixed-size tiles (128 $\times$ 128 by default) and then quantized based on their sensitivity classification. This enables efficient batching of low-importance tiles on high-utilization GPUs, reducing computational load and minimizing memory transfers.
% \begin{itemize}

    % \item \textbf{Hardware Implementation:}     
    % Low-sensitivity tiles are mapped onto under-utilized GPUs in a multi-GPU setup, redistributing computational loads based on the memory and processing capacity of each device. This enhances model throughput and scales efficiently across GPUs, essential for real-time deployment.
% \end{itemize}

% \subsection{Model Optimization and Integration}


\section{Experimental Results}

Hardware
\section{Experiments}
\label{sec:experiments}
The experiments are designed to address two key research questions.
First, \textbf{RQ1} evaluates whether the average $L_2$-norm of the counterfactual perturbation vectors ($\overline{||\perturb||}$) decreases as the model overfits the data, thereby providing further empirical validation for our hypothesis.
Second, \textbf{RQ2} evaluates the ability of the proposed counterfactual regularized loss, as defined in (\ref{eq:regularized_loss2}), to mitigate overfitting when compared to existing regularization techniques.

% The experiments are designed to address three key research questions. First, \textbf{RQ1} investigates whether the mean perturbation vector norm decreases as the model overfits the data, aiming to further validate our intuition. Second, \textbf{RQ2} explores whether the mean perturbation vector norm can be effectively leveraged as a regularization term during training, offering insights into its potential role in mitigating overfitting. Finally, \textbf{RQ3} examines whether our counterfactual regularizer enables the model to achieve superior performance compared to existing regularization methods, thus highlighting its practical advantage.

\subsection{Experimental Setup}
\textbf{\textit{Datasets, Models, and Tasks.}}
The experiments are conducted on three datasets: \textit{Water Potability}~\cite{kadiwal2020waterpotability}, \textit{Phomene}~\cite{phomene}, and \textit{CIFAR-10}~\cite{krizhevsky2009learning}. For \textit{Water Potability} and \textit{Phomene}, we randomly select $80\%$ of the samples for the training set, and the remaining $20\%$ for the test set, \textit{CIFAR-10} comes already split. Furthermore, we consider the following models: Logistic Regression, Multi-Layer Perceptron (MLP) with 100 and 30 neurons on each hidden layer, and PreactResNet-18~\cite{he2016cvecvv} as a Convolutional Neural Network (CNN) architecture.
We focus on binary classification tasks and leave the extension to multiclass scenarios for future work. However, for datasets that are inherently multiclass, we transform the problem into a binary classification task by selecting two classes, aligning with our assumption.

\smallskip
\noindent\textbf{\textit{Evaluation Measures.}} To characterize the degree of overfitting, we use the test loss, as it serves as a reliable indicator of the model's generalization capability to unseen data. Additionally, we evaluate the predictive performance of each model using the test accuracy.

\smallskip
\noindent\textbf{\textit{Baselines.}} We compare CF-Reg with the following regularization techniques: L1 (``Lasso''), L2 (``Ridge''), and Dropout.

\smallskip
\noindent\textbf{\textit{Configurations.}}
For each model, we adopt specific configurations as follows.
\begin{itemize}
\item \textit{Logistic Regression:} To induce overfitting in the model, we artificially increase the dimensionality of the data beyond the number of training samples by applying a polynomial feature expansion. This approach ensures that the model has enough capacity to overfit the training data, allowing us to analyze the impact of our counterfactual regularizer. The degree of the polynomial is chosen as the smallest degree that makes the number of features greater than the number of data.
\item \textit{Neural Networks (MLP and CNN):} To take advantage of the closed-form solution for computing the optimal perturbation vector as defined in (\ref{eq:opt-delta}), we use a local linear approximation of the neural network models. Hence, given an instance $\inst_i$, we consider the (optimal) counterfactual not with respect to $\model$ but with respect to:
\begin{equation}
\label{eq:taylor}
    \model^{lin}(\inst) = \model(\inst_i) + \nabla_{\inst}\model(\inst_i)(\inst - \inst_i),
\end{equation}
where $\model^{lin}$ represents the first-order Taylor approximation of $\model$ at $\inst_i$.
Note that this step is unnecessary for Logistic Regression, as it is inherently a linear model.
\end{itemize}

\smallskip
\noindent \textbf{\textit{Implementation Details.}} We run all experiments on a machine equipped with an AMD Ryzen 9 7900 12-Core Processor and an NVIDIA GeForce RTX 4090 GPU. Our implementation is based on the PyTorch Lightning framework. We use stochastic gradient descent as the optimizer with a learning rate of $\eta = 0.001$ and no weight decay. We use a batch size of $128$. The training and test steps are conducted for $6000$ epochs on the \textit{Water Potability} and \textit{Phoneme} datasets, while for the \textit{CIFAR-10} dataset, they are performed for $200$ epochs.
Finally, the contribution $w_i^{\varepsilon}$ of each training point $\inst_i$ is uniformly set as $w_i^{\varepsilon} = 1~\forall i\in \{1,\ldots,m\}$.

The source code implementation for our experiments is available at the following GitHub repository: \url{https://anonymous.4open.science/r/COCE-80B4/README.md} 

\subsection{RQ1: Counterfactual Perturbation vs. Overfitting}
To address \textbf{RQ1}, we analyze the relationship between the test loss and the average $L_2$-norm of the counterfactual perturbation vectors ($\overline{||\perturb||}$) over training epochs.

In particular, Figure~\ref{fig:delta_loss_epochs} depicts the evolution of $\overline{||\perturb||}$ alongside the test loss for an MLP trained \textit{without} regularization on the \textit{Water Potability} dataset. 
\begin{figure}[ht]
    \centering
    \includegraphics[width=0.85\linewidth]{img/delta_loss_epochs.png}
    \caption{The average counterfactual perturbation vector $\overline{||\perturb||}$ (left $y$-axis) and the cross-entropy test loss (right $y$-axis) over training epochs ($x$-axis) for an MLP trained on the \textit{Water Potability} dataset \textit{without} regularization.}
    \label{fig:delta_loss_epochs}
\end{figure}

The plot shows a clear trend as the model starts to overfit the data (evidenced by an increase in test loss). 
Notably, $\overline{||\perturb||}$ begins to decrease, which aligns with the hypothesis that the average distance to the optimal counterfactual example gets smaller as the model's decision boundary becomes increasingly adherent to the training data.

It is worth noting that this trend is heavily influenced by the choice of the counterfactual generator model. In particular, the relationship between $\overline{||\perturb||}$ and the degree of overfitting may become even more pronounced when leveraging more accurate counterfactual generators. However, these models often come at the cost of higher computational complexity, and their exploration is left to future work.

Nonetheless, we expect that $\overline{||\perturb||}$ will eventually stabilize at a plateau, as the average $L_2$-norm of the optimal counterfactual perturbations cannot vanish to zero.

% Additionally, the choice of employing the score-based counterfactual explanation framework to generate counterfactuals was driven to promote computational efficiency.

% Future enhancements to the framework may involve adopting models capable of generating more precise counterfactuals. While such approaches may yield to performance improvements, they are likely to come at the cost of increased computational complexity.


\subsection{RQ2: Counterfactual Regularization Performance}
To answer \textbf{RQ2}, we evaluate the effectiveness of the proposed counterfactual regularization (CF-Reg) by comparing its performance against existing baselines: unregularized training loss (No-Reg), L1 regularization (L1-Reg), L2 regularization (L2-Reg), and Dropout.
Specifically, for each model and dataset combination, Table~\ref{tab:regularization_comparison} presents the mean value and standard deviation of test accuracy achieved by each method across 5 random initialization. 

The table illustrates that our regularization technique consistently delivers better results than existing methods across all evaluated scenarios, except for one case -- i.e., Logistic Regression on the \textit{Phomene} dataset. 
However, this setting exhibits an unusual pattern, as the highest model accuracy is achieved without any regularization. Even in this case, CF-Reg still surpasses other regularization baselines.

From the results above, we derive the following key insights. First, CF-Reg proves to be effective across various model types, ranging from simple linear models (Logistic Regression) to deep architectures like MLPs and CNNs, and across diverse datasets, including both tabular and image data. 
Second, CF-Reg's strong performance on the \textit{Water} dataset with Logistic Regression suggests that its benefits may be more pronounced when applied to simpler models. However, the unexpected outcome on the \textit{Phoneme} dataset calls for further investigation into this phenomenon.


\begin{table*}[h!]
    \centering
    \caption{Mean value and standard deviation of test accuracy across 5 random initializations for different model, dataset, and regularization method. The best results are highlighted in \textbf{bold}.}
    \label{tab:regularization_comparison}
    \begin{tabular}{|c|c|c|c|c|c|c|}
        \hline
        \textbf{Model} & \textbf{Dataset} & \textbf{No-Reg} & \textbf{L1-Reg} & \textbf{L2-Reg} & \textbf{Dropout} & \textbf{CF-Reg (ours)} \\ \hline
        Logistic Regression   & \textit{Water}   & $0.6595 \pm 0.0038$   & $0.6729 \pm 0.0056$   & $0.6756 \pm 0.0046$  & N/A    & $\mathbf{0.6918 \pm 0.0036}$                     \\ \hline
        MLP   & \textit{Water}   & $0.6756 \pm 0.0042$   & $0.6790 \pm 0.0058$   & $0.6790 \pm 0.0023$  & $0.6750 \pm 0.0036$    & $\mathbf{0.6802 \pm 0.0046}$                    \\ \hline
%        MLP   & \textit{Adult}   & $0.8404 \pm 0.0010$   & $\mathbf{0.8495 \pm 0.0007}$   & $0.8489 \pm 0.0014$  & $\mathbf{0.8495 \pm 0.0016}$     & $0.8449 \pm 0.0019$                    \\ \hline
        Logistic Regression   & \textit{Phomene}   & $\mathbf{0.8148 \pm 0.0020}$   & $0.8041 \pm 0.0028$   & $0.7835 \pm 0.0176$  & N/A    & $0.8098 \pm 0.0055$                     \\ \hline
        MLP   & \textit{Phomene}   & $0.8677 \pm 0.0033$   & $0.8374 \pm 0.0080$   & $0.8673 \pm 0.0045$  & $0.8672 \pm 0.0042$     & $\mathbf{0.8718 \pm 0.0040}$                    \\ \hline
        CNN   & \textit{CIFAR-10} & $0.6670 \pm 0.0233$   & $0.6229 \pm 0.0850$   & $0.7348 \pm 0.0365$   & N/A    & $\mathbf{0.7427 \pm 0.0571}$                     \\ \hline
    \end{tabular}
\end{table*}

\begin{table*}[htb!]
    \centering
    \caption{Hyperparameter configurations utilized for the generation of Table \ref{tab:regularization_comparison}. For our regularization the hyperparameters are reported as $\mathbf{\alpha/\beta}$.}
    \label{tab:performance_parameters}
    \begin{tabular}{|c|c|c|c|c|c|c|}
        \hline
        \textbf{Model} & \textbf{Dataset} & \textbf{No-Reg} & \textbf{L1-Reg} & \textbf{L2-Reg} & \textbf{Dropout} & \textbf{CF-Reg (ours)} \\ \hline
        Logistic Regression   & \textit{Water}   & N/A   & $0.0093$   & $0.6927$  & N/A    & $0.3791/1.0355$                     \\ \hline
        MLP   & \textit{Water}   & N/A   & $0.0007$   & $0.0022$  & $0.0002$    & $0.2567/1.9775$                    \\ \hline
        Logistic Regression   &
        \textit{Phomene}   & N/A   & $0.0097$   & $0.7979$  & N/A    & $0.0571/1.8516$                     \\ \hline
        MLP   & \textit{Phomene}   & N/A   & $0.0007$   & $4.24\cdot10^{-5}$  & $0.0015$    & $0.0516/2.2700$                    \\ \hline
       % MLP   & \textit{Adult}   & N/A   & $0.0018$   & $0.0018$  & $0.0601$     & $0.0764/2.2068$                    \\ \hline
        CNN   & \textit{CIFAR-10} & N/A   & $0.0050$   & $0.0864$ & N/A    & $0.3018/
        2.1502$                     \\ \hline
    \end{tabular}
\end{table*}

\begin{table*}[htb!]
    \centering
    \caption{Mean value and standard deviation of training time across 5 different runs. The reported time (in seconds) corresponds to the generation of each entry in Table \ref{tab:regularization_comparison}. Times are }
    \label{tab:times}
    \begin{tabular}{|c|c|c|c|c|c|c|}
        \hline
        \textbf{Model} & \textbf{Dataset} & \textbf{No-Reg} & \textbf{L1-Reg} & \textbf{L2-Reg} & \textbf{Dropout} & \textbf{CF-Reg (ours)} \\ \hline
        Logistic Regression   & \textit{Water}   & $222.98 \pm 1.07$   & $239.94 \pm 2.59$   & $241.60 \pm 1.88$  & N/A    & $251.50 \pm 1.93$                     \\ \hline
        MLP   & \textit{Water}   & $225.71 \pm 3.85$   & $250.13 \pm 4.44$   & $255.78 \pm 2.38$  & $237.83 \pm 3.45$    & $266.48 \pm 3.46$                    \\ \hline
        Logistic Regression   & \textit{Phomene}   & $266.39 \pm 0.82$ & $367.52 \pm 6.85$   & $361.69 \pm 4.04$  & N/A   & $310.48 \pm 0.76$                    \\ \hline
        MLP   &
        \textit{Phomene} & $335.62 \pm 1.77$   & $390.86 \pm 2.11$   & $393.96 \pm 1.95$ & $363.51 \pm 5.07$    & $403.14 \pm 1.92$                     \\ \hline
       % MLP   & \textit{Adult}   & N/A   & $0.0018$   & $0.0018$  & $0.0601$     & $0.0764/2.2068$                    \\ \hline
        CNN   & \textit{CIFAR-10} & $370.09 \pm 0.18$   & $395.71 \pm 0.55$   & $401.38 \pm 0.16$ & N/A    & $1287.8 \pm 0.26$                     \\ \hline
    \end{tabular}
\end{table*}

\subsection{Feasibility of our Method}
A crucial requirement for any regularization technique is that it should impose minimal impact on the overall training process.
In this respect, CF-Reg introduces an overhead that depends on the time required to find the optimal counterfactual example for each training instance. 
As such, the more sophisticated the counterfactual generator model probed during training the higher would be the time required. However, a more advanced counterfactual generator might provide a more effective regularization. We discuss this trade-off in more details in Section~\ref{sec:discussion}.

Table~\ref{tab:times} presents the average training time ($\pm$ standard deviation) for each model and dataset combination listed in Table~\ref{tab:regularization_comparison}.
We can observe that the higher accuracy achieved by CF-Reg using the score-based counterfactual generator comes with only minimal overhead. However, when applied to deep neural networks with many hidden layers, such as \textit{PreactResNet-18}, the forward derivative computation required for the linearization of the network introduces a more noticeable computational cost, explaining the longer training times in the table.

\subsection{Hyperparameter Sensitivity Analysis}
The proposed counterfactual regularization technique relies on two key hyperparameters: $\alpha$ and $\beta$. The former is intrinsic to the loss formulation defined in (\ref{eq:cf-train}), while the latter is closely tied to the choice of the score-based counterfactual explanation method used.

Figure~\ref{fig:test_alpha_beta} illustrates how the test accuracy of an MLP trained on the \textit{Water Potability} dataset changes for different combinations of $\alpha$ and $\beta$.

\begin{figure}[ht]
    \centering
    \includegraphics[width=0.85\linewidth]{img/test_acc_alpha_beta.png}
    \caption{The test accuracy of an MLP trained on the \textit{Water Potability} dataset, evaluated while varying the weight of our counterfactual regularizer ($\alpha$) for different values of $\beta$.}
    \label{fig:test_alpha_beta}
\end{figure}

We observe that, for a fixed $\beta$, increasing the weight of our counterfactual regularizer ($\alpha$) can slightly improve test accuracy until a sudden drop is noticed for $\alpha > 0.1$.
This behavior was expected, as the impact of our penalty, like any regularization term, can be disruptive if not properly controlled.

Moreover, this finding further demonstrates that our regularization method, CF-Reg, is inherently data-driven. Therefore, it requires specific fine-tuning based on the combination of the model and dataset at hand.
\section{Conclusion}
In this work, we propose a simple yet effective approach, called SMILE, for graph few-shot learning with fewer tasks. Specifically, we introduce a novel dual-level mixup strategy, including within-task and across-task mixup, for enriching the diversity of nodes within each task and the diversity of tasks. Also, we incorporate the degree-based prior information to learn expressive node embeddings. Theoretically, we prove that SMILE effectively enhances the model's generalization performance. Empirically, we conduct extensive experiments on multiple benchmarks and the results suggest that SMILE significantly outperforms other baselines, including both in-domain and cross-domain few-shot settings.

\bibliographystyle{IEEEtran}%\vspace{-0.05in}
\bibliography{quantization,hardware}

% \begin{abstract}
% This document is a model and instructions for \LaTeX.
% This and the IEEEtran.cls file define the components of your paper [title, text, heads, etc.]. *CRITICAL: Do Not Use Symbols, Special Characters, Footnotes, 
% or Math in Paper Title or Abstract.
% \end{abstract}

% \begin{IEEEkeywords}
% component, formatting, style, styling, insert
% \end{IEEEkeywords}

% \section{Introduction}
% This document is a model and instructions for \LaTeX.
% Please observe the conference page limits. 

% \section{Ease of Use}

% \subsection{Maintaining the Integrity of the Specifications}

% The IEEEtran class file is used to format your paper and style the text. All margins, 
% column widths, line spaces, and text fonts are prescribed; please do not 
% alter them. You may note peculiarities. For example, the head margin
% measures proportionately more than is customary. This measurement 
% and others are deliberate, using specifications that anticipate your paper 
% as one part of the entire proceedings, and not as an independent document. 
% Please do not revise any of the current designations.

% \section{Prepare Your Paper Before Styling}
% Before you begin to format your paper, first write and save the content as a 
% separate text file. Complete all content and organizational editing before 
% formatting. Please note sections \ref{AA}--\ref{SCM} below for more information on 
% proofreading, spelling and grammar.

% Keep your text and graphic files separate until after the text has been 
% formatted and styled. Do not number text heads---{\LaTeX} will do that 
% for you.

% \subsection{Abbreviations and Acronyms}\label{AA}
% Define abbreviations and acronyms the first time they are used in the text, 
% even after they have been defined in the abstract. Abbreviations such as 
% IEEE, SI, MKS, CGS, ac, dc, and rms do not have to be defined. Do not use 
% abbreviations in the title or heads unless they are unavoidable.

% \subsection{Units}
% \begin{itemize}
% \item Use either SI (MKS) or CGS as primary units. (SI units are encouraged.) English units may be used as secondary units (in parentheses). An exception would be the use of English units as identifiers in trade, such as ``3.5-inch disk drive''.
% \item Avoid combining SI and CGS units, such as current in amperes and magnetic field in oersteds. This often leads to confusion because equations do not balance dimensionally. If you must use mixed units, clearly state the units for each quantity that you use in an equation.
% \item Do not mix complete spellings and abbreviations of units: ``Wb/m\textsuperscript{2}'' or ``webers per square meter'', not ``webers/m\textsuperscript{2}''. Spell out units when they appear in text: ``. . . a few henries'', not ``. . . a few H''.
% \item Use a zero before decimal points: ``0.25'', not ``.25''. Use ``cm\textsuperscript{3}'', not ``cc''.)
% \end{itemize}

% \subsection{Equations}
% Number equations consecutively. To make your 
% equations more compact, you may use the solidus (~/~), the exp function, or 
% appropriate exponents. Italicize Roman symbols for quantities and variables, 
% but not Greek symbols. Use a long dash rather than a hyphen for a minus 
% sign. Punctuate equations with commas or periods when they are part of a 
% sentence, as in:
% \begin{equation}
% a+b=\gamma\label{eq}
% \end{equation}

% Be sure that the 
% symbols in your equation have been defined before or immediately following 
% the equation. Use ``\eqref{eq}'', not ``Eq.~\eqref{eq}'' or ``equation \eqref{eq}'', except at 
% the beginning of a sentence: ``Equation \eqref{eq} is . . .''

% \subsection{\LaTeX-Specific Advice}

% Please use ``soft'' (e.g., \verb|\eqref{Eq}|) cross references instead
% of ``hard'' references (e.g., \verb|(1)|). That will make it possible
% to combine sections, add equations, or change the order of figures or
% citations without having to go through the file line by line.

% Please don't use the \verb|{eqnarray}| equation environment. Use
% \verb|{align}| or \verb|{IEEEeqnarray}| instead. The \verb|{eqnarray}|
% environment leaves unsightly spaces around relation symbols.

% Please note that the \verb|{subequations}| environment in {\LaTeX}
% will increment the main equation counter even when there are no
% equation numbers displayed. If you forget that, you might write an
% article in which the equation numbers skip from (17) to (20), causing
% the copy editors to wonder if you've discovered a new method of
% counting.

% {\BibTeX} does not work by magic. It doesn't get the bibliographic
% data from thin air but from .bib files. If you use {\BibTeX} to produce a
% bibliography you must send the .bib files. 

% {\LaTeX} can't read your mind. If you assign the same label to a
% subsubsection and a table, you might find that Table I has been cross
% referenced as Table IV-B3. 

% {\LaTeX} does not have precognitive abilities. If you put a
% \verb|\label| command before the command that updates the counter it's
% supposed to be using, the label will pick up the last counter to be
% cross referenced instead. In particular, a \verb|\label| command
% should not go before the caption of a figure or a table.

% Do not use \verb|\nonumber| inside the \verb|{array}| environment. It
% will not stop equation numbers inside \verb|{array}| (there won't be
% any anyway) and it might stop a wanted equation number in the
% surrounding equation.

% \subsection{Some Common Mistakes}\label{SCM}
% \begin{itemize}
% \item The word ``data'' is plural, not singular.
% \item The subscript for the permeability of vacuum $\mu_{0}$, and other common scientific constants, is zero with subscript formatting, not a lowercase letter ``o''.
% \item In American English, commas, semicolons, periods, question and exclamation marks are located within quotation marks only when a complete thought or name is cited, such as a title or full quotation. When quotation marks are used, instead of a bold or italic typeface, to highlight a word or phrase, punctuation should appear outside of the quotation marks. A parenthetical phrase or statement at the end of a sentence is punctuated outside of the closing parenthesis (like this). (A parenthetical sentence is punctuated within the parentheses.)
% \item A graph within a graph is an ``inset'', not an ``insert''. The word alternatively is preferred to the word ``alternately'' (unless you really mean something that alternates).
% \item Do not use the word ``essentially'' to mean ``approximately'' or ``effectively''.
% \item In your paper title, if the words ``that uses'' can accurately replace the word ``using'', capitalize the ``u''; if not, keep using lower-cased.
% \item Be aware of the different meanings of the homophones ``affect'' and ``effect'', ``complement'' and ``compliment'', ``discreet'' and ``discrete'', ``principal'' and ``principle''.
% \item Do not confuse ``imply'' and ``infer''.
% \item The prefix ``non'' is not a word; it should be joined to the word it modifies, usually without a hyphen.
% \item There is no period after the ``et'' in the Latin abbreviation ``et al.''.
% \item The abbreviation ``i.e.'' means ``that is'', and the abbreviation ``e.g.'' means ``for example''.
% \end{itemize}
% An excellent style manual for science writers is \cite{b7}.

% \subsection{Authors and Affiliations}
% \textbf{The class file is designed for, but not limited to, six authors.} A 
% minimum of one author is required for all conference articles. Author names 
% should be listed starting from left to right and then moving down to the 
% next line. This is the author sequence that will be used in future citations 
% and by indexing services. Names should not be listed in columns nor group by 
% affiliation. Please keep your affiliations as succinct as possible (for 
% example, do not differentiate among departments of the same organization).

% \subsection{Identify the Headings}
% Headings, or heads, are organizational devices that guide the reader through 
% your paper. There are two types: component heads and text heads.

% Component heads identify the different components of your paper and are not 
% topically subordinate to each other. Examples include Acknowledgments and 
% References and, for these, the correct style to use is ``Heading 5''. Use 
% ``figure caption'' for your Figure captions, and ``table head'' for your 
% table title. Run-in heads, such as ``Abstract'', will require you to apply a 
% style (in this case, italic) in addition to the style provided by the drop 
% down menu to differentiate the head from the text.

% Text heads organize the topics on a relational, hierarchical basis. For 
% example, the paper title is the primary text head because all subsequent 
% material relates and elaborates on this one topic. If there are two or more 
% sub-topics, the next level head (uppercase Roman numerals) should be used 
% and, conversely, if there are not at least two sub-topics, then no subheads 
% should be introduced.

% \subsection{Figures and Tables}
% \paragraph{Positioning Figures and Tables} Place figures and tables at the top and 
% bottom of columns. Avoid placing them in the middle of columns. Large 
% figures and tables may span across both columns. Figure captions should be 
% below the figures; table heads should appear above the tables. Insert 
% figures and tables after they are cited in the text. Use the abbreviation 
% ``Fig.~\ref{fig}'', even at the beginning of a sentence.

% \begin{table}[htbp]
% \caption{Table Type Styles}
% \begin{center}
% \begin{tabular}{|c|c|c|c|}
% \hline
% \textbf{Table}&\multicolumn{3}{|c|}{\textbf{Table Column Head}} \\
% \cline{2-4} 
% \textbf{Head} & \textbf{\textit{Table column subhead}}& \textbf{\textit{Subhead}}& \textbf{\textit{Subhead}} \\
% \hline
% copy& More table copy$^{\mathrm{a}}$& &  \\
% \hline
% \multicolumn{4}{l}{$^{\mathrm{a}}$Sample of a Table footnote.}
% \end{tabular}
% \label{tab1}
% \end{center}
% \end{table}

% \begin{figure}[htbp]
% \centerline{\includegraphics{fig1.png}}
% \caption{Example of a figure caption.}
% \label{fig}
% \end{figure}

% Figure Labels: Use 8 point Times New Roman for Figure labels. Use words 
% rather than symbols or abbreviations when writing Figure axis labels to 
% avoid confusing the reader. As an example, write the quantity 
% ``Magnetization'', or ``Magnetization, M'', not just ``M''. If including 
% units in the label, present them within parentheses. Do not label axes only 
% with units. In the example, write ``Magnetization (A/m)'' or ``Magnetization 
% \{A[m(1)]\}'', not just ``A/m''. Do not label axes with a ratio of 
% quantities and units. For example, write ``Temperature (K)'', not 
% ``Temperature/K''.

% \section*{Acknowledgment}

% The preferred spelling of the word ``acknowledgment'' in America is without 
% an ``e'' after the ``g''. Avoid the stilted expression ``one of us (R. B. 
% G.) thanks $\ldots$''. Instead, try ``R. B. G. thanks$\ldots$''. Put sponsor 
% acknowledgments in the unnumbered footnote on the first page.

% \section*{References}

% Please number citations consecutively within brackets \cite{b1}. The 
% sentence punctuation follows the bracket \cite{b2}. Refer simply to the reference 
% number, as in \cite{b3}---do not use ``Ref. \cite{b3}'' or ``reference \cite{b3}'' except at 
% the beginning of a sentence: ``Reference \cite{b3} was the first $\ldots$''

% Number footnotes separately in superscripts. Place the actual footnote at 
% the bottom of the column in which it was cited. Do not put footnotes in the 
% abstract or reference list. Use letters for table footnotes.

% Unless there are six authors or more give all authors' names; do not use 
% ``et al.''. Papers that have not been published, even if they have been 
% submitted for publication, should be cited as ``unpublished'' \cite{b4}. Papers 
% that have been accepted for publication should be cited as ``in press'' \cite{b5}. 
% Capitalize only the first word in a paper title, except for proper nouns and 
% element symbols.

% For papers published in translation journals, please give the English 
% citation first, followed by the original foreign-language citation \cite{b6}.

% \begin{thebibliography}{00}
% \bibitem{b1} G. Eason, B. Noble, and I. N. Sneddon, ``On certain integrals of Lipschitz-Hankel type involving products of Bessel functions,'' Phil. Trans. Roy. Soc. London, vol. A247, pp. 529--551, April 1955.
% \bibitem{b2} J. Clerk Maxwell, A Treatise on Electricity and Magnetism, 3rd ed., vol. 2. Oxford: Clarendon, 1892, pp.68--73.
% \bibitem{b3} I. S. Jacobs and C. P. Bean, ``Fine particles, thin films and exchange anisotropy,'' in Magnetism, vol. III, G. T. Rado and H. Suhl, Eds. New York: Academic, 1963, pp. 271--350.
% \bibitem{b4} K. Elissa, ``Title of paper if known,'' unpublished.
% \bibitem{b5} R. Nicole, ``Title of paper with only first word capitalized,'' J. Name Stand. Abbrev., in press.
% \bibitem{b6} Y. Yorozu, M. Hirano, K. Oka, and Y. Tagawa, ``Electron spectroscopy studies on magneto-optical media and plastic substrate interface,'' IEEE Transl. J. Magn. Japan, vol. 2, pp. 740--741, August 1987 [Digests 9th Annual Conf. Magnetics Japan, p. 301, 1982].
% \bibitem{b7} M. Young, The Technical Writer's Handbook. Mill Valley, CA: University Science, 1989.
% \end{thebibliography}
% \vspace{12pt}
% \color{red}
% IEEE conference templates contain guidance text for composing and formatting conference papers. Please ensure that all template text is removed from your conference paper prior to submission to the conference. Failure to remove the template text from your paper may result in your paper not being published.

\end{document}
