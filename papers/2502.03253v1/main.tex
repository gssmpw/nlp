% 
% Annual Cognitive Science Conference
% Sample LaTeX Paper -- Proceedings Format
% 

% Original : Ashwin Ram (ashwin@cc.gatech.edu)       04/01/1994
% Modified : Johanna Moore (jmoore@cs.pitt.edu)      03/17/1995
% Modified : David Noelle (noelle@ucsd.edu)          03/15/1996
% Modified : Pat Langley (langley@cs.stanford.edu)   01/26/1997
% Latex2e corrections by Ramin Charles Nakisa        01/28/1997 
% Modified : Tina Eliassi-Rad (eliassi@cs.wisc.edu)  01/31/1998
% Modified : Trisha Yannuzzi (trisha@ircs.upenn.edu) 12/28/1999 (in process)
% Modified : Mary Ellen Foster (M.E.Foster@ed.ac.uk) 12/11/2000
% Modified : Ken Forbus                              01/23/2004
% Modified : Eli M. Silk (esilk@pitt.edu)            05/24/2005
% Modified : Niels Taatgen (taatgen@cmu.edu)         10/24/2006
% Modified : David Noelle (dnoelle@ucmerced.edu)     11/19/2014
% Modified : Roger Levy (rplevy@mit.edu)     12/31/2018



%% Change "letterpaper" in the following line to "a4paper" if you must.

\documentclass[10pt,letterpaper]{article}

\usepackage{cogsci}

\cogscifinalcopy % Uncomment this line for the final submission 


\usepackage{pslatex}
\usepackage{apacite}
\usepackage{float} % Roger Levy added this and changed figure/table
                   % placement to [H] for conformity to Word template,
                   % though floating tables and figures to top is
                   % still generally recommended!

%\usepackage[none]{hyphenat} % Sometimes it can be useful to turn off
%hyphenation for purposes such as spell checking of the resulting
%PDF.  Uncomment this block to turn off hyphenation.


%\setlength\titlebox{4.5cm}
% You can expand the titlebox if you need extra space
% to show all the authors. Please do not make the titlebox
% smaller than 4.5cm (the original size).
%%If you do, we reserve the right to require you to change it back in
%%the camera-ready version, which could interfere with the timely
%%appearance of your paper in the Proceedings.

% user defined imports
\usepackage{tabularx}
\usepackage{booktabs}
\usepackage{siunitx}
\usepackage{longtable}
\usepackage{multirow}
\usepackage{svg}
\usepackage{makecell}
\usepackage{caption}
\usepackage{amssymb}
\usepackage{subfig}
\usepackage{graphicx}
\usepackage{natbib}
\usepackage{svg}
\usepackage{mdframed}
\usepackage{listings}
\usepackage[symbol]{footmisc}

\newcommand{\TODO}[1]{\color{red}{\textbf{TODO: #1}}\color{black}}

\setcounter{footnote}{0}
\title{How do Humans and Language Models Reason About Creativity? A Comparative Analysis}
\renewcommand{\thefootnote}{\fnsymbol{footnote}}
\author{{\large \bf Antonio Laverghetta Jr.\textsuperscript{1}\thanks{Corresponding author: $<$aml7990@psu.edu$>$} Tuhin Chakrabarty\textsuperscript{2} Tom Hope\textsuperscript{3}
Jimmy Pronchick\textsuperscript{1} Krupa Bhawsar\textsuperscript{1} Roger E. Beaty\textsuperscript{1}} \\
  \textsuperscript{1}Pennsylvania State University \\
  \textsuperscript{2}Stony Brook University\\
  \textsuperscript{3}Hebrew University of Jerusalem \\
  % Madison, WI 53706 USA
  % Department of Educational Psychology, 1025 W. Johnson Street
  % Madison, WI 53706 USA}
}


\begin{document}

\maketitle
\renewcommand*{\thefootnote}{\arabic{footnote}}
\setcounter{footnote}{0}


\begin{abstract}
Creativity assessment in science and engineering is increasingly based on both human and AI judgment, but the cognitive processes and biases behind these evaluations remain poorly understood. We conducted two experiments examining how including example solutions with ratings impact creativity evaluation, using a finegrained annotation protocol where raters were tasked with explaining their originality scores and rating for the facets of remoteness (whether the response is ``far'' from everyday ideas), uncommonness (whether the response is rare), and cleverness. In Study 1, we analyzed creativity ratings from 72 experts with formal science or engineering training, comparing those who received example solutions with ratings (example) to those who did not (no example). Computational text analysis revealed that, compared to experts with examples, no-example experts used more comparative language (e.g., ``better/worse'') and emphasized solution uncommonness, suggesting they may have relied more on memory retrieval for comparisons. In Study 2, parallel analyses with state-of-the-art LLMs revealed that models prioritized uncommonness and remoteness of ideas when rating originality, suggesting an evaluative process rooted around the semantic similarity of ideas. In the example condition, while LLM accuracy in predicting the true originality scores improved, the correlations of remoteness, uncommonness, and cleverness with originality also increased substantially --- to upwards of $0.99$ --- suggesting a homogenization in the LLMs evaluation of the individual facets. These findings highlight important implications for how humans and AI reason about creativity and suggest diverging preferences for what different populations prioritize when rating. 

% In contrast, experts with examples used less comparative language and focused more on direct assessments of cleverness.

%TC Great expert

\textbf{Keywords:} 
creativity; large language models; text analysis; STEM
\end{abstract}

\section{Introduction}
\section{Introduction}


\begin{figure}[t]
\centering
\includegraphics[width=0.6\columnwidth]{figures/evaluation_desiderata_V5.pdf}
\vspace{-0.5cm}
\caption{\systemName is a platform for conducting realistic evaluations of code LLMs, collecting human preferences of coding models with real users, real tasks, and in realistic environments, aimed at addressing the limitations of existing evaluations.
}
\label{fig:motivation}
\end{figure}

\begin{figure*}[t]
\centering
\includegraphics[width=\textwidth]{figures/system_design_v2.png}
\caption{We introduce \systemName, a VSCode extension to collect human preferences of code directly in a developer's IDE. \systemName enables developers to use code completions from various models. The system comprises a) the interface in the user's IDE which presents paired completions to users (left), b) a sampling strategy that picks model pairs to reduce latency (right, top), and c) a prompting scheme that allows diverse LLMs to perform code completions with high fidelity.
Users can select between the top completion (green box) using \texttt{tab} or the bottom completion (blue box) using \texttt{shift+tab}.}
\label{fig:overview}
\end{figure*}

As model capabilities improve, large language models (LLMs) are increasingly integrated into user environments and workflows.
For example, software developers code with AI in integrated developer environments (IDEs)~\citep{peng2023impact}, doctors rely on notes generated through ambient listening~\citep{oberst2024science}, and lawyers consider case evidence identified by electronic discovery systems~\citep{yang2024beyond}.
Increasing deployment of models in productivity tools demands evaluation that more closely reflects real-world circumstances~\citep{hutchinson2022evaluation, saxon2024benchmarks, kapoor2024ai}.
While newer benchmarks and live platforms incorporate human feedback to capture real-world usage, they almost exclusively focus on evaluating LLMs in chat conversations~\citep{zheng2023judging,dubois2023alpacafarm,chiang2024chatbot, kirk2024the}.
Model evaluation must move beyond chat-based interactions and into specialized user environments.



 

In this work, we focus on evaluating LLM-based coding assistants. 
Despite the popularity of these tools---millions of developers use Github Copilot~\citep{Copilot}---existing
evaluations of the coding capabilities of new models exhibit multiple limitations (Figure~\ref{fig:motivation}, bottom).
Traditional ML benchmarks evaluate LLM capabilities by measuring how well a model can complete static, interview-style coding tasks~\citep{chen2021evaluating,austin2021program,jain2024livecodebench, white2024livebench} and lack \emph{real users}. 
User studies recruit real users to evaluate the effectiveness of LLMs as coding assistants, but are often limited to simple programming tasks as opposed to \emph{real tasks}~\citep{vaithilingam2022expectation,ross2023programmer, mozannar2024realhumaneval}.
Recent efforts to collect human feedback such as Chatbot Arena~\citep{chiang2024chatbot} are still removed from a \emph{realistic environment}, resulting in users and data that deviate from typical software development processes.
We introduce \systemName to address these limitations (Figure~\ref{fig:motivation}, top), and we describe our three main contributions below.


\textbf{We deploy \systemName in-the-wild to collect human preferences on code.} 
\systemName is a Visual Studio Code extension, collecting preferences directly in a developer's IDE within their actual workflow (Figure~\ref{fig:overview}).
\systemName provides developers with code completions, akin to the type of support provided by Github Copilot~\citep{Copilot}. 
Over the past 3 months, \systemName has served over~\completions suggestions from 10 state-of-the-art LLMs, 
gathering \sampleCount~votes from \userCount~users.
To collect user preferences,
\systemName presents a novel interface that shows users paired code completions from two different LLMs, which are determined based on a sampling strategy that aims to 
mitigate latency while preserving coverage across model comparisons.
Additionally, we devise a prompting scheme that allows a diverse set of models to perform code completions with high fidelity.
See Section~\ref{sec:system} and Section~\ref{sec:deployment} for details about system design and deployment respectively.



\textbf{We construct a leaderboard of user preferences and find notable differences from existing static benchmarks and human preference leaderboards.}
In general, we observe that smaller models seem to overperform in static benchmarks compared to our leaderboard, while performance among larger models is mixed (Section~\ref{sec:leaderboard_calculation}).
We attribute these differences to the fact that \systemName is exposed to users and tasks that differ drastically from code evaluations in the past. 
Our data spans 103 programming languages and 24 natural languages as well as a variety of real-world applications and code structures, while static benchmarks tend to focus on a specific programming and natural language and task (e.g. coding competition problems).
Additionally, while all of \systemName interactions contain code contexts and the majority involve infilling tasks, a much smaller fraction of Chatbot Arena's coding tasks contain code context, with infilling tasks appearing even more rarely. 
We analyze our data in depth in Section~\ref{subsec:comparison}.



\textbf{We derive new insights into user preferences of code by analyzing \systemName's diverse and distinct data distribution.}
We compare user preferences across different stratifications of input data (e.g., common versus rare languages) and observe which affect observed preferences most (Section~\ref{sec:analysis}).
For example, while user preferences stay relatively consistent across various programming languages, they differ drastically between different task categories (e.g. frontend/backend versus algorithm design).
We also observe variations in user preference due to different features related to code structure 
(e.g., context length and completion patterns).
We open-source \systemName and release a curated subset of code contexts.
Altogether, our results highlight the necessity of model evaluation in realistic and domain-specific settings.









% \section{Background}
\section{Study 1: Human Creativity Evaluations}
Our first study sought to understand the key factors underlying human expert evaluation of the creativity of solutions to design problems (DPT) items. A participant in this task is given a scientific or engineering problem (e.g., increasing the use of renewable energy) and is instructed to come up with as many novel solutions to the problem as they can think of. Similar to expert-level science, the best solutions are both original and feasible, though unlike other STEM assessments the DPT benefits from but is not contingent on expertise to come up with creative ideas. The greater complexity of DPT responses compared to those from other creativity tests and its relationship to scientific creativity more broadly makes it a strong choice for our analysis. Unlike prior studies, which often have experts rate only the originality or quality of products, we instead ask our raters to provide fine-grained assessments of cleverness (whether the solution is insightful or witty), remoteness (whether the solution is ``far'' from everyday ideas), and uncommonness (whether the solution is rare, given by few people) in addition to originality, each of which is thought to influence ratings of creativity \citep{silvia2008assessing}. These assessments are performed both with and without the presence of example creativity ratings to DPT items, enabling us to examine how added context affects the evaluation process. Finally, we ask experts to briefly explain their originality scores, enabling us to employ methods from computational text analysis to probe the cognitive processes experts employ when rating and how such processes may be modulated by added context.

\subsection{Methods}
We use the data from \citet{Patterson2025}, who obtained more than 7000 responses to DPT items from undergraduate STEM majors. Each response was rated for originality using a five-point Likert scale by at least three expert raters with formal training in engineering. We drop items that did not obtain at least one rating from every point of the scale (certain items never had a response that received a five). We convert Likert scores into factor scores, as this has been shown to provide more accurate creativity ratings \citep{silvia2008another}, and we treat these factor scores as the true originality scores of each response.

We recruit 80 participants on Prolific to provide finegrained creativity ratings to DPT responses, requiring that they have a bachelor's degree or higher in a STEM field and are fluent in English. We split participants into two conditions: a \textit{no example} condition where participants are given responses to rate without any additional context, and an \textit{example} condition where participants are first shown example solutions with originality scores for responses to the same prompt being rated. We pull three example solutions from the same dataset while ensuring that participants never rate them. We include a solution with a score of one, one with a score of three, and one with a score of five, to avoid biasing participants towards either end of the scale. We first have each participant rate for originality following the same procedure, instructions, and facet definitions as \citet{Patterson2025}. After rating originality, participants in both groups then provide 1-2 sentences explaining their rating process \citep{orwig2024creative}, and they finish by rating the uncommonness, remoteness, and cleverness of the response using a five-point Likert scale for each. We instruct participants to be specific in their explanations, to draw on their domain expertise as holders of a STEM degree, and to avoid overly simplistic explanations (e.g., ``it's not original'' or ``it's an obvious answer''). We define a good explanation as being at least one sentence long and including specific details from the participant's prior experience, the response, or the examples (if applicable). We also provide definitions of uncommonness, remoteness, and cleverness for the final rating task, emphasizing that each facet is related while being distinct from originality. We include educational background and AI use checks at the end of the survey.

We administer each participant 15 DPT responses at random. To encourage high-quality explanations, we offer \$20 per hour to complete a 30-minute study. We exclude participants with an approval rating of less than 90\%, who report using AI to complete the task, or who report an education level lower than the minimum specified on Prolific. We also exclude participants who were exceptionally slow or fast (with a completion time further than three standard deviations from the mean), who gave the same rating for every response, or who did not follow our instructions for formatting explanations (as checked by a research assistant). This resulted in a final sample size of 37 participants and 481 ratings in the example condition and 35 participants and 455 responses for the no example.

% out of the full archival set

When examining the participants' explanations, we employ an analysis plan similar to \citet{orwig2024creative}, who used LIWC to analyze explanations of originality scores for AUTs. However, recent work has found that LLMs can predict psycholinguistic features of text more strongly than LIWC, even zero shot \citep{rathje2024gpt}. Therefore, we use LLMs to automatically rate linguistic markers in the explanations. We instruct LLMs to rate for the following variables: 

\begin{itemize}
    \item \textit{Past/future expressions}: Is the explanation past-focused or future-focused in its evaluation of the response?
    \item \textit{Perceptual details}: Does the explanation focus on the process of perceiving (``observe'', ``seen'', ``heard'', ``feel'', etc.)?
    \item \textit{Causal/analytical}: Does the explanation involve a structured evaluation of the response, evidencing an analytical process, or is the explanation more intuitive in its justifications?
    \item \textit{Comparative}: Does the explanation make explicit references to standards or examples or compare the response to other ideas?
    \item \textit{Cleverness}: Does the explanation refer to the cleverness, wittiness, shrewdness, or ingenuity (or lack thereof) of the response?
\end{itemize}

Both past/future language use and perceptual details have been explored to assess cognitive strategies employed on other creativity tests \citep{orwig2024creative}. We elect to use causal/analytical, comparative, and cleverness linguistic markers to aid in assessing whether participants employed a more structured process --- which might be evidenced by causal/analytical or comparative language use --- or a more intuitive process, as evidenced by language indicating sensory experiences or other ``gut reactions'' (e.g, ``it feels like a clever idea''). These linguistic markers also map onto the finegrained facets participants were asked to rate, with cleverness language mapping onto cleverness and comparative language mapping onto remoteness and uncommonness (as both remoteness and uncommonness often require making references to prior solutions). We use both \textsc{claude-3.5-sonnet}\footnote{https://www.anthropic.com/news/claude-3-5-sonnet} and \textsc{gpt-4o}\footnote{https://openai.com/index/hello-gpt-4o/} to check for reliability in ratings and avoid biases specific to a single LLM, though due to space constraints we mainly report results from \textsc{gpt-4o} as this is the model \citet{rathje2024gpt} validated. To encourage deterministic output, we set the temperature for both models to $0$ and top P to $1$. We instruct LLMs to rate each facet and provide a binary evaluation of whether the explanation does or does not contain the feature. Prompts are provided in the supplementary materials.

% We drop any explanations for which the model fails to follow this instruction.



% Specifically, we have each participant first rate for originality following the same procedure as \TODO{cite the paper that obtained the gold scores}, after which they are instructed to provide 1-2 sentences explaining their originality rating. We instruct participants to draw on their domain expertise while responding, asking them to consider if they have observed similar solutions to the problem in the past, to help ensure that the explanation is grounded in the originality of the solution and not merely its quality. Finally, participants rate the uncommonness, remoteness, and cleverness of the solution using separate five-point Likert scales, to disentangle how each facets contributes to the final creativity score. Figure \TODO{make it} shows the experimental interface, we collect data using Qualtrics and give 30 minutes to complete the task. Participants are split into two conditions: a \textit{no oracle} conditions where participants are shown DPT solutions without any additional context (equivalent to how creativity ratings are typically solicited), and an \textit{oracle} condition where participants are first shown example solutions with originality scores for responses to the same prompt being rated. We pull these solutions from the same dataset, while ensuring that participants never rate them. At the end, participants complete a demographic questionnaire and a check for use of generative AI during the study.

% The design problems task (DPT) is a test of domain-general creative problem solving and divergent thinking in science and engineering \TODO{cite}.   Further, creativity evaluation also hinges on weighing multiple competing factors: a highly uncommon solution may still receive a poor creativity score if it is not especially clever or could not be feasibly put into practice. This makes the DPT a strong testbed for our experiments: evaluation is more complex than purely theoretical creativity tests like the alternative uses task \TODO{cite}, yet it is not so challenging as to require domain experts to obtain meaningful creativity scores, enabling us to recruit a larger pool of participants.

% The task targets STEM undergraduate students; a general understanding of science and engineering is beneficial but not necessary for the task.

% Our goal is to obtain \textit{finegrained} creativity assessments for these responses, to better understand how each facet of creativity influences a raters final score, and to provide explanations for why responses are assigned a particular rating. 

% \TODO{define the research questions and null hypotheses}

\subsection{Results}
We begin by examining inter-correlations among all facets (cleverness, remoteness, uncommonness) and correlations between each facet and originality for both conditions. Results are in Figure \ref{fig:experiment_1_correlations}. As expected, each facet is moderately correlated with originality as well as each other, with Pearson r in the range 0.45--0.67 (all correlations are significant).\footnote{Results from all correlational analysis in both studies were similar using Spearman $\rho$.} Comparing the example to no example conditions, we see an increase in correlation between originality and cleverness and a decrease in correlation between originality and both remoteness and uncommonness. Changes in correlation across conditions were significant for cleverness-remoteness (Fisher's z = 2.83, p $<$ 0.01), remoteness-uncommonness (z = -4.61, p $<$ 0.001), and remoteness-originality (z = -2.96, p $<$ 0.01), but were insignificant for all other comparisons. Notably, the presence of the examples did not make experts significantly more accurate in their evaluations of originality, with correlations in the moderate range for both conditions (no example r = 0.44, example r = 0.47).

% We report descriptive statistics for all Likert evaluations in Table \TODO{make it}, broken down by condition.

% Examining the distribution of cleverness more closely (the only facet to become more strongly related to originality in the oracle condition), we plot the distribution of cleverness scores for both conditions in Figure \ref{fig:experiment_1_cleverness}. 

% Participants in the oracle condition appear to be stricter judges of cleverness, giving more 1 or 2 rating than their no oracle counterparts, though this difference was only marginally significant (Mann Whitney U test = 102948.5, p $<$ 0.1).

\begin{figure}[htb]
    \centering
    \footnotesize
    \includegraphics[width=0.8\linewidth]{Figures/Correlations_human.eps}
    % \includesvg[width=0.85\linewidth]{Figures/Correlations_human.svg}
    \caption{Pearson correlations among pairwise Likert ratings for both conditions. o = originality, c = cleverness, u = uncommonness, r = remoteness.}
    \label{fig:experiment_1_correlations}
\end{figure}

% \begin{figure}[htb]
%     \centering
%     \includegraphics[width=1\linewidth]{Figures/cleverness.png}
%     \caption{Distribution of cleverness ratings for both conditions.}
%     \label{fig:experiment_1_cleverness}
% \end{figure}

Turning to participant explanations, \textsc{gpt-4o}'s ratings did not reveal significant differences per condition for perceptual details, past/future language use, or cleverness, but differences are significant for both causal/analytical language (Mann-Whitney U = 78039.5, p $<$ 0.05) and comparative language (U = 75627.5, p $<$ 0.01) with the example condition using less comparative and causal/analytical language than the no examples. Distributions for linguistic markers are shown in Figure \ref{fig:liwc_analysis}. \textsc{claude-3.5-sonnet}'s ratings generally agreed with \textsc{gpt-4o} (Cramer's V in the range 0.549--0.798) with the only notable departure being that \textsc{claude-3.5-sonnet} found no significant difference in causal/analytical language between the conditions (U = 75076.5, p $<$ 0.5). We report additional linguistic marker analysis in the supplementary materials.

% We report model agreement statistics and linguistic marker analyses in the supplementary files.

\subsection{Discussion}
As expected, the facet ratings for cleverness, remoteness, and uncommonness did not perfectly correlate with each other nor with originality, implying that participants do not weigh each facet equally when assessing originality. Further, correlations changed by a significant degree when including example ratings, with both remoteness and uncommonness becoming weaker predictors of originality and cleverness becoming a stronger one. Given that participants in the no example condition needed to actively retrieve example solutions from memory when evaluating, a possible explanation is that this retrieval process biased them towards placing stronger emphasis on the remoteness and uncommonness of the response in relation to solutions they had seen in the past, while example participants would not need to focus as much effort on thinking of prior solutions and could instead focus on the cleverness of the idea. Notably, participants in both groups did not differ significantly in terms of education, making it unlikely this effect could be explained as a skill confound. The idea that participants in the example condition were biased toward cleverness rather than the other facets was also partially supported by their explanations, as no example participants used significantly more comparative language than example participants. Given that assessing remoteness or uncommonness often requires making direct comparisons to prior solutions, it makes sense that an evaluation rooted around these facets would contain more comparisons than an evaluation rooted around cleverness, which is more readily evaluated in isolation (e.g., whether the idea is resource efficient, not immediately obvious, etc.).

\section{Study 2: LLM Creativity Evaluations}
\section{Experiment 1: Few-shot Semi-supervised Medical Image Segmentation (FS-Semi)}
\label{sec:task2}
We implement our GEMINI learning on few-shot semi-supervised (FS-Semi) medical image segmentation (GEMINI-Semi) providing a variant on the situation that labels are very few. Three public-available tasks are enrolled in our experiments for a very complete evaluation.
\subsection{Experiments configurations}
\label{sec:configurations2}
\subsubsection{Variant design} The variant of our GEMINI-Semi learns a segmentation head $Seg_{\kappa}$ on the extracted dense features $f^{A},f^{B}$. Therefore, except the optimization for deformable homeomorphism learning $\mathcal{L}_{DHL}$, the GEMINI-Semi also has an additional optimization for segmentation $\mathcal{L}_{Seg}$:
\begin{equation}\label{equ:variant2}
\underset{\xi,\theta,\kappa}{\arg\min}\ (\mathcal{L}_{DHL}(\theta,\xi,\mathcal{S}_{ul})+\mathcal{L}_{Seg}(\theta,\kappa,\mathcal{S}_{l})),
\end{equation}
where the $\mathcal{S}_{ul}$ and the $\mathcal{S}_{l}$ are the unlabeled dataset and the labeled dataset. In our experiment, we utilize the sum of Dice loss and cross-entropy loss \cite{ma2021loss} to train segmentation objective $\mathcal{L}_{Seg}$. The other compared DCRL methods (Sec.\ref{sec:comparison2}) also use the same setting as this variant which adds the $\mathcal{L}_{Seg}$ in the training to learn segmentation.
\begin{table}
  \centering
  \caption{Total seven publicly available datasets are involved in this paper for the experiments of our GEMINI's variants, achieving great reproducibility.}\label{dataset}
\resizebox{\linewidth}{!}{
  \begin{tabular}{lccccccccc}
  \toprule
  \textbf{Dataset}                       &\textbf{Type}    &\textbf{Num}  &\textbf{FS-Semi} &\textbf{SS-MIP}\\
  \midrule
  %\midrule
  ASOCA \cite{gharleghi2022automated}    &3D cardiac CT    &60            &$\surd$          &\\
  CAT08 \cite{schaap2009standardized}    &3D cardiac CT    &32            &$\surd$          &\\
  WHS-CT \cite{zhuang2019evaluation}     &3D cardiac CT    &60            &$\surd$          &\\
  CANDI \cite{kennedy2012candishare}     &3D brain MRI     &103           &$\surd$          &$\surd$\\
  SCR \cite{van2006segmentation}         &2D chest X-ray   &247           &$\surd$          &$\surd$\\
  KiPA22 \cite{he2021meta}               &3D kidney CT     &130           &                 &$\surd$\\
  %CARDIAC               &3D cardiac CT              &302                 &                 &$\surd$\\
  ChestX-ray8 \cite{wang2017chestx}      &2D chest X-ray   &112,120       &                 &$\surd$\\
  \bottomrule
  \end{tabular}}
\end{table}

\subsubsection{Datasets} We evaluate GEMINI on three public tasks in 2D and 3D dimensions, showcasing its powerful representation ability in semi-supervised tasks \cite{you2024mine,you2024rethinking} with minimal labels (Tab.\ref{dataset}). \textbf{Task 1: FS-Semi cardiac structure segmentation (3D)} targets seven cardiac structures on 3D CT images, combining WHS-CT \cite{zhuang2019evaluation} (20 labeled, 40 unlabeled), ASOCA \cite{gharleghi2022automated} (60 unlabeled), and CAT08 \cite{schaap2009standardized} (32 labeled from\footnote{\url{http://www.sdspeople.fudan.edu.cn/zhuangxiahai/0/mmwhs/}}). Images are cropped and resampled to $144\times144\times128$, with a five-shot evaluation (5, 100, and 47 images as labeled training, unlabeled training, and testing sets). \textbf{Task 2: FS-Semi brain tissue segmentation (3D)} involves 27 brain tissues on 3D T1 MR images from the CANDI dataset \cite{kennedy2012candishare} (103 labeled). Cropped volumes of $160\times160\times128$ undergo five-shot evaluation (5, 78, and 20 images as labeled training, unlabeled training, and testing sets). \textbf{Task 3: FS-Semi chest structure segmentation (2D)} focuses on three chest-related structures in 2D chest X-rays using the SCR dataset \cite{van2006segmentation} (247 labeled) whose images are from the JSRT database \cite{shiraishi2000development}, split into 5 labeled, 142 unlabeled, and 100 testing images for five-shot evaluation. All tasks use rotation [$-20^\circ$, $20^\circ$] and scaling [0.75, 1.25] for data augmentation.

\subsubsection{Comparison setting} \label{sec:comparison2}
We compare GEMINI-Semi with 19 widely-used methods and our GVSL \cite{He_2023_CVPR} (CVPR 2023) to demonstrate its superiority. \textbf{1)} We train a U-Net \cite{ronneberger2015u} to establish upper and lower bounds using 5 labeled images (Five) and all labeled training data (Full). \textbf{2) Semi-supervised methods} without homeomorphism prior (UA-MT \cite{yu2019uncertainty}, MASSL \cite{chen2019multi}, DPA-DBN \cite{he2020dense}, CPS \cite{chen2021semi}) highlight the significance of prior knowledge for semi-supervised learning with limited labels. \textbf{3) Atlas-based methods} with homeomorphism prior (VM \cite{ba2018un}, LC-VM \cite{BalakrishnanVoxelMorph(u)}, LT-Net \cite{wang2020lt}) illustrate the limitation caused by the inefficient correspondence learning. \textbf{4) Learning registration to learn segmentation methods} with homeomorphism prior (DeepAtlas \cite{xu2019deepatlas}, DataAug \cite{zhao2019data}, DeepRS \cite{he2020deep}, PC-Reg-RT \cite{he2021few}, BRBS \cite{he2022learning}) show gains from improved correspondence but are limited by pseudo-labels from unreliable GVS. \textbf{5) Dense contrastive representation learning methods} without homeomorphism prior (VADeR \cite{o2020unsupervised}, GLCL \cite{chaitanya2020contrastive}, DSC-PM \cite{li2021dense}, PixPro \cite{xie2021propagate}, DenseCL \cite{wang2022densecl}, SetSim \cite{wang2022exploring}) reveal FP\&N problem in DCRL. For fairness, all methods use 2D/3D U-Net \cite{ronneberger2015u} with group normalization \cite{wu2018group} as the backbone.

\subsubsection{Implementation and evaluation metrics} In this task, our GEMINI-Semi is implemented by PyTorch \cite{paszke2019pytorch} on NVIDIA GeForce RTX 3090 GPUs with 24 GB memory. We take Adam whose learning rate is $1\times10^{-4}$ to optimize our framework for fast convergence. For task 1 and task 2, we sample two unlabeled images and one labeled image randomly in each iteration to save the memory for large 3D images, and for task 3, we sample 10 unlabeled images and 5 labeled images randomly in each iteration for 2D images. Following \cite{he2022learning}, we perform an affine transformation on these images via AntsPy\footnote{\url{https://github.com/ANTsX/ANTsPy}} to normalize the spatial system. We utilize the DSC [\%] to evaluate the area-based overlap index and the average Hausdorf distances (AVD) to evaluate the coincidence of the surface \cite{taha2015metrics}.

\subsection{Results and Analysis}
\label{sec:results2}
\begin{table*}
\centering
\caption{The quantitative evaluation demonstrates our powerful representation ability in FS-Semi tasks. Our GEMINI-Semi achieves the best performance on CT, MR, and X-ray images compared with 19 popular methods and the GVSL. The ``unable" means that the extremely poor results make the AVD unable to be calculated. The ``-" means that the setting is unable to be implemented. The ``HP" means these methods have or do not have homeomorphism prior. ``T1", ``T2", ``T3" are the task 1, task 2, task 3. The red and blue values are the highest and the second-highest values in the columns.}
\resizebox{\textwidth}{!}{
\begin{tabular}{clccccccccccccccc}
  \toprule
  \multirow{2}{*}{\textbf{Type}}
  &\multirow{2}{*}{\textbf{Method}}
  &\multirow{2}{*}{\textbf{HP}}
  &\multicolumn{2}{c}{\textbf{T1: 3D cardiac structures}}
  &\multicolumn{2}{c}{\textbf{T2: 3D brain tissues}}
  &\multicolumn{2}{c}{\textbf{T3: 2D chest structures}}
  &\textbf{AVG}\\ \cmidrule(r){4-5}\cmidrule(r){6-7}\cmidrule(r){8-9}\cmidrule(r){10-10}
  &
  &
  &DSC$_{\pm std}\uparrow$
  &AVD$_{\pm std}\downarrow$
  &DSC$_{\pm std}\uparrow$
  &AVD$_{\pm std}\downarrow$
  &DSC$_{\pm std}\uparrow$
  &AVD$_{\pm std}\downarrow$
  &DSC$_{\pm std}\uparrow$
  \\
  \midrule
  Full
  &U-Net \cite{ronneberger2015u}
  &$\times$
  &-
  &-
  &88.7$_{\pm1.2}$
  &0.31$_{\pm0.04}$
  &96.1$_{\pm1.4}$
  &2.28$_{\pm1.00}$
  &-
  \\
  Five
  &U-Net \cite{ronneberger2015u}
  &$\times$
  &84.3$_{\pm9.6}$
  &2.43$_{\pm2.14}$
  &69.5$_{\pm8.8}$
  &1.59$_{\pm0.84}$
  &83.4$_{\pm6.9}$
  &10.34$_{\pm4.80}$
  &79.1$_{\pm8.4}$
  \\
  \cdashline{1-10}[0.8pt/2pt]
  Semi
  &UA-MT \cite{yu2019uncertainty}
  &$\times$
  &66.4$_{\pm16.2}$
  &4.69$_{\pm2.27}$
  &75.5$_{\pm3.4}$
  &1.31$_{\pm0.95}$
  &83.9$_{\pm6.2}$
  &9.52$_{\pm4.03}$
  &75.3$_{\pm8.6}$
  \\
  &CPS \cite{chen2021semi}
  &$\times$
  &87.4$_{\pm5.4}$
  &1.40$_{\pm0.76}$
  &37.1$_{\pm1.8}$
  &unable
  &63.2$_{\pm1.4}$
  &19.57$_{\pm5.67}$
  &62.6$_{\pm2.9}$
  \\
  &MASSL \cite{chen2019multi}
  &$\times$
  &77.4$_{\pm8.7}$
  &9.07$_{\pm3.11}$
  &80.5$_{\pm3.1}$
  &0.92$_{\pm0.43}$
  &81.9$_{\pm7.0}$
  &10.99$_{\pm4.58}$
  &79.9$_{\pm6.3}$
  \\
  &DPA-DBN \cite{he2020dense}
  &$\times$
  &68.0$_{\pm14.5}$
  &5.75$_{\pm3.89}$
  &68.7$_{\pm8.2}$
  &3.90$_{\pm2.39}$
  &67.4$_{\pm8.7}$
  &24.05$_{\pm6.75}$
  &68.0$_{\pm10.5}$
  \\
  %\midrule
  Atlas
  &VM \cite{ba2018un}
  &$\surd$
  &81.0$_{\pm6.1}$
  &2.13$_{\pm0.78}$
  &83.1$_{\pm1.8}$
  &0.56$_{\pm0.08}$
  &59.9$_{\pm5.0}$
  &15.36$_{\pm4.34}$
  &74.7$_{\pm4.3}$
  \\
  &LC-VM \cite{BalakrishnanVoxelMorph(u)}
  &$\surd$
  &81.7$_{\pm6.0}$
  &2.04$_{\pm0.77}$
  &83.0$_{\pm1.8}$
  &0.56$_{\pm0.07}$
  &60.2$_{\pm7.4}$
  &14.72$_{\pm4.89}$
  &74.9$_{\pm5.1}$
  \\
  &LT-Net \cite{wang2020lt}
  &$\surd$
  &77.8$_{\pm7.8}$
  &2.25$_{\pm0.95}$
  &82.6$_{\pm1.2}$
  &0.57$_{\pm0.05}$
  &60.4$_{\pm7.4}$
  &14.62$_{\pm4.84}$
  &73.6$_{\pm5.5}$
  \\
  %\hline
  LRLS
  &DeepAtlas \cite{xu2019deepatlas}
  &$\surd$
  &87.9$_{\pm4.3}$
  &1.30$_{\pm0.57}$
  &79.3$_{\pm2.6}$
  &0.74$_{\pm0.12}$
  &64.8$_{\pm9.6}$
  &12.87$_{\pm3.56}$
  &77.3$_{\pm5.5}$
  \\
  &DataAug \cite{zhao2019data}
  &$\surd$
  &82.2$_{\pm5.2}$
  &2.04$_{\pm0.73}$
  &83.9$_{\pm1.2}$
  &0.55$_{\pm0.06}$
  &22.2$_{\pm2.8}$
  &unable
  &62.8$_{\pm3.1}$
  \\
  &DeepRS \cite{he2020deep}
  &$\surd$
  &87.0$_{\pm5.0}$
  &1.60$_{\pm0.90}$
  &73.0$_{\pm5.9}$
  &0.93$_{\pm0.25}$
  &86.0$_{\pm5.6}$
  &8.55$_{\pm3.98}$
  &82.0$_{\pm5.5}$
  \\
  &PC-Reg-RT \cite{he2021few}
  &$\surd$
  &88.5$_{\pm4.9}$
  &1.23$_{\pm0.72}$
  &73.1$_{\pm3.1}$
  &1.09$_{\pm0.17}$
  &59.1$_{\pm3.6}$
  &20.71$_{\pm5.21}$
  &73.6$_{\pm3.9}$
  \\
  &BRBS \cite{he2022learning}
  &$\surd$
  &\color{blue}91.1$_{\pm3.9}$
  &\color{red}\textbf{0.93$_{\pm0.57}$}
  &\color{blue}87.2$_{\pm1.0}$
  &0.43$_{\pm0.05}$
  &71.5$_{\pm6.4}$
  &10.85$_{\pm2.99}$
  &83.3$_{\pm3.8}$
  \\
  %\hline
  DCRL
  &VADeR \cite{o2020unsupervised}
  &$\times$
  &85.4$_{\pm4.7}$
  &1.69$_{\pm0.77}$
  &81.2$_{\pm3.2}$
  &0.59$_{\pm0.13}$
  &79.9$_{\pm5.8}$
  &8.95$_{\pm3.37}$
  &82.2$_{\pm4.6}$
  \\
  &DenseCL \cite{wang2022densecl}
  &$\times$
  &87.3$_{\pm4.3}$
  &1.52$_{\pm0.79}$
  &83.9$_{\pm1.9}$
  &0.48$_{\pm0.09}$
  &77.1$_{\pm8.8}$
  &12.11$_{\pm6.51}$
  &82.8$_{\pm5.0}$
  \\
  &SetSim \cite{wang2022exploring}
  &$\times$
  &87.0$_{\pm4.5}$
  &1.60$_{\pm0.84}$
  &81.2$_{\pm3.0}$
  &0.58$_{\pm0.13}$
  &79.0$_{\pm7.3}$
  &11.72$_{\pm5.03}$
  &82.4$_{\pm4.9}$
  \\
  &DSC-PM \cite{li2021dense}
  &$\times$
  &87.0$_{\pm4.6}$
  &1.60$_{\pm0.86}$
  &82.6$_{\pm2.1}$
  &0.53$_{\pm0.09}$
  &85.7$_{\pm6.2}$
  &7.33$_{\pm3.32}$
  &85.1$_{\pm4.3}$
  \\
  &PixPro \cite{xie2021propagate}
  &$\times$
  &89.5$_{\pm3.9}$
  &1.31$_{\pm0.75}$
  &86.3$_{\pm1.2}$
  &\color{blue}0.38$_{\pm0.04}$
  &83.3$_{\pm8.7}$
  &8.73$_{\pm4.55}$
  &\color{blue}86.4$_{\pm4.6}$
  \\
  &GLCL\cite{chaitanya2020contrastive}
  &$\times$
  &84.5$_{\pm7.0}$
  &1.82$_{\pm1.09}$
  &83.0$_{\pm2.7}$
  &0.52$_{\pm0.11}$
  &85.5$_{\pm8.9}$
  &8.65$_{\pm5.18}$
  &84.3$_{\pm6.2}$
  \\
  %\hline
  \cdashline{1-10}[0.8pt/2pt]
  \textbf{DCRL}
  &\textbf{GVSL-Semi (CVPR)} \cite{He_2023_CVPR}
  &$\surd$
  &90.0$_{\pm3.7}$
  &1.21$_{\pm0.81}$
  &82.3$_{\pm5.9}$
  &0.55$_{\pm0.27}$
  &\color{blue}86.3$_{\pm5.5}$
  &\color{blue}7.18$_{\pm4.01}$
  &86.2$_{\pm5.0}$
  \\
  \textbf{(Ours)}
  &\textbf{GEMINI-Semi}
  &$\surd$
  &\color{red}\textbf{91.2$_{\pm3.6}$}
  &\color{blue}0.97$_{\pm0.56}$
  &\color{red}\textbf{87.3$_{\pm1.0}$}
  &\color{red}\textbf{0.35$_{\pm0.03}$}
  &\color{red}\textbf{87.7$_{\pm5.2}$}
  &\color{red}\textbf{7.14$_{\pm3.63}$}
  &\color{red}\textbf{88.7$_{\pm3.3}$}
  \\
  \bottomrule
\end{tabular}
}
\label{tab:metrics2}
\end{table*}
\begin{figure}
  \centering
  \includegraphics[width=\linewidth]{./picture/results2.pdf}
  \caption{Our GEMINI-Semi has significant visual superiority on three FS-Semi medical image segmentation tasks.}\label{Fig:results2}
\end{figure}
\subsubsection{Quantitative evaluation shows metric superiority}
As shown in Tab.\ref{tab:metrics2}, 19 compared methods demonstrate that the DCRL will greatly improve the representability, and the homeomorphism prior (``HP") further improves the reliability of the representation learning. There are three interesting observations in Tab.\ref{tab:metrics2}: \textbf{1)} The semi-supervised methods are limited by the extremely few labels. They utilize the pseudo-label generation (UA-MT, CPS) or multi-task learning (MASSL, DPA-DBN) to improve the representation, but the extremely few labels have no enough ability to give them reliable optimization directions to overcome the noise in pseudo labels or multiple tasks. As a result, the UA-MT, MASSL, and DPA-DBN have worse performance than U-Net on task 1, and the CPS is worse on task 2 and 3. \textbf{2)} With the ``HP", the Atlas and LRLS methods achieve robust performance in task 1 and task 2, but are limited in task 3. The ``HP" brings an alignment between labeled and unlabeled images for numerous reliable pseudo labels. Therefore, they have achieved significant improvement on task 1 and task 2 compared with the semi-supervised methods. However, the X-ray images in task 3 have low contrast and their appearances are varied caused by the 2D projection of 3D human body, this makes inefficient GVS brings large misalignment between images, thus interfering with the learning and reducing the performance. \textbf{3)} The DCRL methods have robust performance in all three tasks compared with the LRLS methods, although the VADeR, DenseCL, SetSim, DSC-PM, PixPro and GLCL have no homeomorphism prior. Because their feature-level learning reduce the direct interference caused by misalignment in LRLS's pseudo labels and the supervision from the few labels bring basic representability which will promote their correspondence discovery. However, the FP\&N problem is still a problem in the learning and their performance on task 3 is poor without ``HP" like the semi-supervised methods.

Compared with the LRLS, other DCRL methods, and our previous GVSL-Semi, our GEMINI-Semi achieves the best performance on three tasks with four observations: \textbf{1)} Compared with the LRLS methods which have ``HP", our method has better performance on all tasks. Although the BRBS has similar performance as our GEMINI-Semi on task 1 and task 2, our method achieves 16.2\% DSC and 3.71 AVD higher and lower than it on task 3. This is because our GEMINI-Semi utilizes our GSS for alignment measurement and shares the representation between the segmentation and deformation learning, bringing more efficient and robust learning for alignment. It has a great ability to construct positive feature pairs even with varied appearances. The gradient from our DHL also trains the soft negative feature pairs to drive the learning of distinct representations for potentially different semantics in shared backbones, bringing a regularization for potential mispaired positive pairs. \textbf{2)} Compared with the other DCRL methods which have no ``HP", our GEMINI-Semi shows great improvements in all three tasks. It achieves more than 1.7\%, 1.0\%, and 2.0\% DSC improvements on task 1, 2, and 3 compared with the best DCRL models without ``HP" (PixPro in task 1 and 2, DSC-PM in task 3). Because the ``HP" in our GEMINI-Semi constructs a more reliable correspondence discovery process which reduces the production risk of the FP\&N pairs, bringing comprehensive improvement for the DCRL. \textbf{3)} Compared to our CVPR vision (GVSL-Semi), we find even though the GVSL utilizes the visual similarity like the BRBS, it also achieves great performance in task 3, demonstrating the superiority of the DCRL paradigm. The GVSL-semi avoids the interference of pseudo labels like BRBS reducing the noisy information from the extremely mis-alignment, so that it takes the advantage of DCRL and our homeomorphism prior and achieves good performance in all three tasks. Our GEMINI-Semi promotes the GVSL and utilizes the GSS for a more powerful dense representation learning, thus achieving the highest 88.7\% average DSC in this experiment. \textbf{4)} Compared with the fully supervised setting (``Full") in task 2 (83 labeled images), our GEMINI-Semi achieves a similar performance only with 5 labeled images demonstrating our great potential in reducing of annotation costs. In the task 3, our framework is lower than the upper bound (96.1\%) only with five annotations, but it still achieves significant improvement (4.3\%) compared with the model directly trained on five labeled images.

\subsubsection{Qualitative evaluation shows visual superiority}
As shown in Fig.\ref{Fig:results2}, we show typical cases on the three tasks in this experiment and our framework has higher accuracy on thin regions and fewer outliers. In the task 1, the segmentation result of our method has better integrity, and the different semantic structures have good adjacency without outliers. However, the other four DCRL methods have discontinuous mis-segmentation which destroys the heart topology. This is because the pairing strategies in the DCRL methods are unable to make the pairs under the condition of topology consistency, so the large-scale mispaired features interrupt the learning and make numerous outliers. The same as the task 3, there are also serious outlier problems in the four typical DCRL methods and the GVSL, and our GEMINI-Semi has fine segmentation. In the task 2, our GEMINI and GVSL show finer segmentation on the thin brain structures which is sensitive and will be interrupted by the noise in the semi-supervised learning process. In some prominent and gully regions of task 2 (enlarged part), the compared four DCRL methods have numerous distortions due to their unreliable correspondence discovery, showing their fragility.





\section{Conclusion}
\section{Conclusion}
In this work, we propose a simple yet effective approach, called SMILE, for graph few-shot learning with fewer tasks. Specifically, we introduce a novel dual-level mixup strategy, including within-task and across-task mixup, for enriching the diversity of nodes within each task and the diversity of tasks. Also, we incorporate the degree-based prior information to learn expressive node embeddings. Theoretically, we prove that SMILE effectively enhances the model's generalization performance. Empirically, we conduct extensive experiments on multiple benchmarks and the results suggest that SMILE significantly outperforms other baselines, including both in-domain and cross-domain few-shot settings.


% \section{General Formatting Instructions}

% The entire content of a paper (including figures, references, and anything else) can be no longer than six pages in the \textbf{initial submission}. In the \textbf{final submission}, the text of the paper, including an author line, must fit on six pages. Up to one additional page can be used for acknowledgements and references.

% The text of the paper should be formatted in two columns with an
% overall width of 7 inches (17.8 cm) and length of 9.25 inches (23.5
% cm), with 0.25 inches between the columns. Leave two line spaces
% between the last author listed and the text of the paper; the text of
% the paper (starting with the abstract) should begin no less than 2.75 inches below the top of the
% page. The left margin should be 0.75 inches and the top margin should
% be 1 inch.  \textbf{The right and bottom margins will depend on
%   whether you use U.S. letter or A4 paper, so you must be sure to
%   measure the width of the printed text.} Use 10~point Times Roman
% with 12~point vertical spacing, unless otherwise specified.

% The title should be in 14~point bold font, centered. The title should
% be formatted with initial caps (the first letter of content words
% capitalized and the rest lower case). In the initial submission, the
% phrase ``Anonymous CogSci submission'' should appear below the title,
% centered, in 11~point bold font.  In the final submission, each
% author's name should appear on a separate line, 11~point bold, and
% centered, with the author's email address in parentheses. Under each
% author's name list the author's affiliation and postal address in
% ordinary 10~point type.

% Indent the first line of each paragraph by 1/8~inch (except for the
% first paragraph of a new section). Do not add extra vertical space
% between paragraphs.


% \section{First Level Headings}

% First level headings should be in 12~point, initial caps, bold and
% centered. Leave one line space above the heading and 1/4~line space
% below the heading.


% \subsection{Second Level Headings}

% Second level headings should be 11~point, initial caps, bold, and
% flush left. Leave one line space above the heading and 1/4~line
% space below the heading.


% \subsubsection{Third Level Headings}

% Third level headings should be 10~point, initial caps, bold, and flush
% left. Leave one line space above the heading, but no space after the
% heading.


% \section{Formalities, Footnotes, and Floats}

% Use standard APA citation format. Citations within the text should
% include the author's last name and year. If the authors' names are
% included in the sentence, place only the year in parentheses, as in
% \citeA{NewellSimon1972a}, but otherwise place the entire reference in
% parentheses with the authors and year separated by a comma
% \cite{NewellSimon1972a}. List multiple references alphabetically and
% separate them by semicolons
% \cite{ChalnickBillman1988a,NewellSimon1972a}. Use the
% ``et~al.'' construction only after listing all the authors to a
% publication in an earlier reference and for citations with four or
% more authors.


% \subsection{Footnotes}

% Indicate footnotes with a number\footnote{Sample of the first
% footnote.} in the text. Place the footnotes in 9~point font at the
% bottom of the column on which they appear. Precede the footnote block
% with a horizontal rule.\footnote{Sample of the second footnote.}


% \subsection{Tables}

% Number tables consecutively. Place the table number and title (in
% 10~point) above the table with one line space above the caption and
% one line space below it, as in Table~\ref{sample-table}. You may float
% tables to the top or bottom of a column, and you may set wide tables across
% both columns.

% \begin{table}[H]
% \begin{center} 
% \caption{Sample table title.} 
% \label{sample-table} 
% \vskip 0.12in
% \begin{tabular}{ll} 
% \hline
% Error type    &  Example \\
% \hline
% Take smaller        &   63 - 44 = 21 \\
% Always borrow~~~~   &   96 - 42 = 34 \\
% 0 - N = N           &   70 - 47 = 37 \\
% 0 - N = 0           &   70 - 47 = 30 \\
% \hline
% \end{tabular} 
% \end{center} 
% \end{table}


% \subsection{Figures}

% All artwork must be very dark for purposes of reproduction and should
% not be hand drawn. Number figures sequentially, placing the figure
% number and caption, in 10~point, after the figure with one line space
% above the caption and one line space below it, as in
% Figure~\ref{sample-figure}. If necessary, leave extra white space at
% the bottom of the page to avoid splitting the figure and figure
% caption. You may float figures to the top or bottom of a column, and
% you may set wide figures across both columns.

% \begin{figure}[H]
% \begin{center}
% \fbox{CoGNiTiVe ScIeNcE}
% \end{center}
% \caption{This is a figure.} 
% \label{sample-figure}
% \end{figure}


% \section{Acknowledgments}

% Place acknowledgments (including funding information) in a section at
% the end of the paper.


% \section{References Instructions}

% Follow the APA Publication Manual for citation format, both within the
% text and in the reference list, with the following exceptions: (a) do
% not cite the page numbers of any book, including chapters in edited
% volumes; (b) use the same format for unpublished references as for
% published ones. Alphabetize references by the surnames of the authors,
% with single author entries preceding multiple author entries. Order
% references by the same authors by the year of publication, with the
% earliest first.

% Use a first level section heading, ``{\bf References}'', as shown
% below. Use a hanging indent style, with the first line of the
% reference flush against the left margin and subsequent lines indented
% by 1/8~inch. Below are example references for a conference paper, book
% chapter, journal article, dissertation, book, technical report, and
% edited volume, respectively.

% \nocite{ChalnickBillman1988a}
% \nocite{Feigenbaum1963a}
% \nocite{Hill1983a}
% \nocite{OhlssonLangley1985a}
% \nocite{Matlock2001}
% \nocite{NewellSimon1972a}
% \nocite{ShragerLangley1990a}


\bibliographystyle{apacite}

\setlength{\bibleftmargin}{.125in}
\setlength{\bibindent}{-\bibleftmargin}

\bibliography{antonio}



% \clearpage
% \appendix
% \subsection{Lloyd-Max Algorithm}
\label{subsec:Lloyd-Max}
For a given quantization bitwidth $B$ and an operand $\bm{X}$, the Lloyd-Max algorithm finds $2^B$ quantization levels $\{\hat{x}_i\}_{i=1}^{2^B}$ such that quantizing $\bm{X}$ by rounding each scalar in $\bm{X}$ to the nearest quantization level minimizes the quantization MSE. 

The algorithm starts with an initial guess of quantization levels and then iteratively computes quantization thresholds $\{\tau_i\}_{i=1}^{2^B-1}$ and updates quantization levels $\{\hat{x}_i\}_{i=1}^{2^B}$. Specifically, at iteration $n$, thresholds are set to the midpoints of the previous iteration's levels:
\begin{align*}
    \tau_i^{(n)}=\frac{\hat{x}_i^{(n-1)}+\hat{x}_{i+1}^{(n-1)}}2 \text{ for } i=1\ldots 2^B-1
\end{align*}
Subsequently, the quantization levels are re-computed as conditional means of the data regions defined by the new thresholds:
\begin{align*}
    \hat{x}_i^{(n)}=\mathbb{E}\left[ \bm{X} \big| \bm{X}\in [\tau_{i-1}^{(n)},\tau_i^{(n)}] \right] \text{ for } i=1\ldots 2^B
\end{align*}
where to satisfy boundary conditions we have $\tau_0=-\infty$ and $\tau_{2^B}=\infty$. The algorithm iterates the above steps until convergence.

Figure \ref{fig:lm_quant} compares the quantization levels of a $7$-bit floating point (E3M3) quantizer (left) to a $7$-bit Lloyd-Max quantizer (right) when quantizing a layer of weights from the GPT3-126M model at a per-tensor granularity. As shown, the Lloyd-Max quantizer achieves substantially lower quantization MSE. Further, Table \ref{tab:FP7_vs_LM7} shows the superior perplexity achieved by Lloyd-Max quantizers for bitwidths of $7$, $6$ and $5$. The difference between the quantizers is clear at 5 bits, where per-tensor FP quantization incurs a drastic and unacceptable increase in perplexity, while Lloyd-Max quantization incurs a much smaller increase. Nevertheless, we note that even the optimal Lloyd-Max quantizer incurs a notable ($\sim 1.5$) increase in perplexity due to the coarse granularity of quantization. 

\begin{figure}[h]
  \centering
  \includegraphics[width=0.7\linewidth]{sections/figures/LM7_FP7.pdf}
  \caption{\small Quantization levels and the corresponding quantization MSE of Floating Point (left) vs Lloyd-Max (right) Quantizers for a layer of weights in the GPT3-126M model.}
  \label{fig:lm_quant}
\end{figure}

\begin{table}[h]\scriptsize
\begin{center}
\caption{\label{tab:FP7_vs_LM7} \small Comparing perplexity (lower is better) achieved by floating point quantizers and Lloyd-Max quantizers on a GPT3-126M model for the Wikitext-103 dataset.}
\begin{tabular}{c|cc|c}
\hline
 \multirow{2}{*}{\textbf{Bitwidth}} & \multicolumn{2}{|c|}{\textbf{Floating-Point Quantizer}} & \textbf{Lloyd-Max Quantizer} \\
 & Best Format & Wikitext-103 Perplexity & Wikitext-103 Perplexity \\
\hline
7 & E3M3 & 18.32 & 18.27 \\
6 & E3M2 & 19.07 & 18.51 \\
5 & E4M0 & 43.89 & 19.71 \\
\hline
\end{tabular}
\end{center}
\end{table}

\subsection{Proof of Local Optimality of LO-BCQ}
\label{subsec:lobcq_opt_proof}
For a given block $\bm{b}_j$, the quantization MSE during LO-BCQ can be empirically evaluated as $\frac{1}{L_b}\lVert \bm{b}_j- \bm{\hat{b}}_j\rVert^2_2$ where $\bm{\hat{b}}_j$ is computed from equation (\ref{eq:clustered_quantization_definition}) as $C_{f(\bm{b}_j)}(\bm{b}_j)$. Further, for a given block cluster $\mathcal{B}_i$, we compute the quantization MSE as $\frac{1}{|\mathcal{B}_{i}|}\sum_{\bm{b} \in \mathcal{B}_{i}} \frac{1}{L_b}\lVert \bm{b}- C_i^{(n)}(\bm{b})\rVert^2_2$. Therefore, at the end of iteration $n$, we evaluate the overall quantization MSE $J^{(n)}$ for a given operand $\bm{X}$ composed of $N_c$ block clusters as:
\begin{align*}
    \label{eq:mse_iter_n}
    J^{(n)} = \frac{1}{N_c} \sum_{i=1}^{N_c} \frac{1}{|\mathcal{B}_{i}^{(n)}|}\sum_{\bm{v} \in \mathcal{B}_{i}^{(n)}} \frac{1}{L_b}\lVert \bm{b}- B_i^{(n)}(\bm{b})\rVert^2_2
\end{align*}

At the end of iteration $n$, the codebooks are updated from $\mathcal{C}^{(n-1)}$ to $\mathcal{C}^{(n)}$. However, the mapping of a given vector $\bm{b}_j$ to quantizers $\mathcal{C}^{(n)}$ remains as  $f^{(n)}(\bm{b}_j)$. At the next iteration, during the vector clustering step, $f^{(n+1)}(\bm{b}_j)$ finds new mapping of $\bm{b}_j$ to updated codebooks $\mathcal{C}^{(n)}$ such that the quantization MSE over the candidate codebooks is minimized. Therefore, we obtain the following result for $\bm{b}_j$:
\begin{align*}
\frac{1}{L_b}\lVert \bm{b}_j - C_{f^{(n+1)}(\bm{b}_j)}^{(n)}(\bm{b}_j)\rVert^2_2 \le \frac{1}{L_b}\lVert \bm{b}_j - C_{f^{(n)}(\bm{b}_j)}^{(n)}(\bm{b}_j)\rVert^2_2
\end{align*}

That is, quantizing $\bm{b}_j$ at the end of the block clustering step of iteration $n+1$ results in lower quantization MSE compared to quantizing at the end of iteration $n$. Since this is true for all $\bm{b} \in \bm{X}$, we assert the following:
\begin{equation}
\begin{split}
\label{eq:mse_ineq_1}
    \tilde{J}^{(n+1)} &= \frac{1}{N_c} \sum_{i=1}^{N_c} \frac{1}{|\mathcal{B}_{i}^{(n+1)}|}\sum_{\bm{b} \in \mathcal{B}_{i}^{(n+1)}} \frac{1}{L_b}\lVert \bm{b} - C_i^{(n)}(b)\rVert^2_2 \le J^{(n)}
\end{split}
\end{equation}
where $\tilde{J}^{(n+1)}$ is the the quantization MSE after the vector clustering step at iteration $n+1$.

Next, during the codebook update step (\ref{eq:quantizers_update}) at iteration $n+1$, the per-cluster codebooks $\mathcal{C}^{(n)}$ are updated to $\mathcal{C}^{(n+1)}$ by invoking the Lloyd-Max algorithm \citep{Lloyd}. We know that for any given value distribution, the Lloyd-Max algorithm minimizes the quantization MSE. Therefore, for a given vector cluster $\mathcal{B}_i$ we obtain the following result:

\begin{equation}
    \frac{1}{|\mathcal{B}_{i}^{(n+1)}|}\sum_{\bm{b} \in \mathcal{B}_{i}^{(n+1)}} \frac{1}{L_b}\lVert \bm{b}- C_i^{(n+1)}(\bm{b})\rVert^2_2 \le \frac{1}{|\mathcal{B}_{i}^{(n+1)}|}\sum_{\bm{b} \in \mathcal{B}_{i}^{(n+1)}} \frac{1}{L_b}\lVert \bm{b}- C_i^{(n)}(\bm{b})\rVert^2_2
\end{equation}

The above equation states that quantizing the given block cluster $\mathcal{B}_i$ after updating the associated codebook from $C_i^{(n)}$ to $C_i^{(n+1)}$ results in lower quantization MSE. Since this is true for all the block clusters, we derive the following result: 
\begin{equation}
\begin{split}
\label{eq:mse_ineq_2}
     J^{(n+1)} &= \frac{1}{N_c} \sum_{i=1}^{N_c} \frac{1}{|\mathcal{B}_{i}^{(n+1)}|}\sum_{\bm{b} \in \mathcal{B}_{i}^{(n+1)}} \frac{1}{L_b}\lVert \bm{b}- C_i^{(n+1)}(\bm{b})\rVert^2_2  \le \tilde{J}^{(n+1)}   
\end{split}
\end{equation}

Following (\ref{eq:mse_ineq_1}) and (\ref{eq:mse_ineq_2}), we find that the quantization MSE is non-increasing for each iteration, that is, $J^{(1)} \ge J^{(2)} \ge J^{(3)} \ge \ldots \ge J^{(M)}$ where $M$ is the maximum number of iterations. 
%Therefore, we can say that if the algorithm converges, then it must be that it has converged to a local minimum. 
\hfill $\blacksquare$


\begin{figure}
    \begin{center}
    \includegraphics[width=0.5\textwidth]{sections//figures/mse_vs_iter.pdf}
    \end{center}
    \caption{\small NMSE vs iterations during LO-BCQ compared to other block quantization proposals}
    \label{fig:nmse_vs_iter}
\end{figure}

Figure \ref{fig:nmse_vs_iter} shows the empirical convergence of LO-BCQ across several block lengths and number of codebooks. Also, the MSE achieved by LO-BCQ is compared to baselines such as MXFP and VSQ. As shown, LO-BCQ converges to a lower MSE than the baselines. Further, we achieve better convergence for larger number of codebooks ($N_c$) and for a smaller block length ($L_b$), both of which increase the bitwidth of BCQ (see Eq \ref{eq:bitwidth_bcq}).


\subsection{Additional Accuracy Results}
%Table \ref{tab:lobcq_config} lists the various LOBCQ configurations and their corresponding bitwidths.
\begin{table}
\setlength{\tabcolsep}{4.75pt}
\begin{center}
\caption{\label{tab:lobcq_config} Various LO-BCQ configurations and their bitwidths.}
\begin{tabular}{|c||c|c|c|c||c|c||c|} 
\hline
 & \multicolumn{4}{|c||}{$L_b=8$} & \multicolumn{2}{|c||}{$L_b=4$} & $L_b=2$ \\
 \hline
 \backslashbox{$L_A$\kern-1em}{\kern-1em$N_c$} & 2 & 4 & 8 & 16 & 2 & 4 & 2 \\
 \hline
 64 & 4.25 & 4.375 & 4.5 & 4.625 & 4.375 & 4.625 & 4.625\\
 \hline
 32 & 4.375 & 4.5 & 4.625& 4.75 & 4.5 & 4.75 & 4.75 \\
 \hline
 16 & 4.625 & 4.75& 4.875 & 5 & 4.75 & 5 & 5 \\
 \hline
\end{tabular}
\end{center}
\end{table}

%\subsection{Perplexity achieved by various LO-BCQ configurations on Wikitext-103 dataset}

\begin{table} \centering
\begin{tabular}{|c||c|c|c|c||c|c||c|} 
\hline
 $L_b \rightarrow$& \multicolumn{4}{c||}{8} & \multicolumn{2}{c||}{4} & 2\\
 \hline
 \backslashbox{$L_A$\kern-1em}{\kern-1em$N_c$} & 2 & 4 & 8 & 16 & 2 & 4 & 2  \\
 %$N_c \rightarrow$ & 2 & 4 & 8 & 16 & 2 & 4 & 2 \\
 \hline
 \hline
 \multicolumn{8}{c}{GPT3-1.3B (FP32 PPL = 9.98)} \\ 
 \hline
 \hline
 64 & 10.40 & 10.23 & 10.17 & 10.15 &  10.28 & 10.18 & 10.19 \\
 \hline
 32 & 10.25 & 10.20 & 10.15 & 10.12 &  10.23 & 10.17 & 10.17 \\
 \hline
 16 & 10.22 & 10.16 & 10.10 & 10.09 &  10.21 & 10.14 & 10.16 \\
 \hline
  \hline
 \multicolumn{8}{c}{GPT3-8B (FP32 PPL = 7.38)} \\ 
 \hline
 \hline
 64 & 7.61 & 7.52 & 7.48 &  7.47 &  7.55 &  7.49 & 7.50 \\
 \hline
 32 & 7.52 & 7.50 & 7.46 &  7.45 &  7.52 &  7.48 & 7.48  \\
 \hline
 16 & 7.51 & 7.48 & 7.44 &  7.44 &  7.51 &  7.49 & 7.47  \\
 \hline
\end{tabular}
\caption{\label{tab:ppl_gpt3_abalation} Wikitext-103 perplexity across GPT3-1.3B and 8B models.}
\end{table}

\begin{table} \centering
\begin{tabular}{|c||c|c|c|c||} 
\hline
 $L_b \rightarrow$& \multicolumn{4}{c||}{8}\\
 \hline
 \backslashbox{$L_A$\kern-1em}{\kern-1em$N_c$} & 2 & 4 & 8 & 16 \\
 %$N_c \rightarrow$ & 2 & 4 & 8 & 16 & 2 & 4 & 2 \\
 \hline
 \hline
 \multicolumn{5}{|c|}{Llama2-7B (FP32 PPL = 5.06)} \\ 
 \hline
 \hline
 64 & 5.31 & 5.26 & 5.19 & 5.18  \\
 \hline
 32 & 5.23 & 5.25 & 5.18 & 5.15  \\
 \hline
 16 & 5.23 & 5.19 & 5.16 & 5.14  \\
 \hline
 \multicolumn{5}{|c|}{Nemotron4-15B (FP32 PPL = 5.87)} \\ 
 \hline
 \hline
 64  & 6.3 & 6.20 & 6.13 & 6.08  \\
 \hline
 32  & 6.24 & 6.12 & 6.07 & 6.03  \\
 \hline
 16  & 6.12 & 6.14 & 6.04 & 6.02  \\
 \hline
 \multicolumn{5}{|c|}{Nemotron4-340B (FP32 PPL = 3.48)} \\ 
 \hline
 \hline
 64 & 3.67 & 3.62 & 3.60 & 3.59 \\
 \hline
 32 & 3.63 & 3.61 & 3.59 & 3.56 \\
 \hline
 16 & 3.61 & 3.58 & 3.57 & 3.55 \\
 \hline
\end{tabular}
\caption{\label{tab:ppl_llama7B_nemo15B} Wikitext-103 perplexity compared to FP32 baseline in Llama2-7B and Nemotron4-15B, 340B models}
\end{table}

%\subsection{Perplexity achieved by various LO-BCQ configurations on MMLU dataset}


\begin{table} \centering
\begin{tabular}{|c||c|c|c|c||c|c|c|c|} 
\hline
 $L_b \rightarrow$& \multicolumn{4}{c||}{8} & \multicolumn{4}{c||}{8}\\
 \hline
 \backslashbox{$L_A$\kern-1em}{\kern-1em$N_c$} & 2 & 4 & 8 & 16 & 2 & 4 & 8 & 16  \\
 %$N_c \rightarrow$ & 2 & 4 & 8 & 16 & 2 & 4 & 2 \\
 \hline
 \hline
 \multicolumn{5}{|c|}{Llama2-7B (FP32 Accuracy = 45.8\%)} & \multicolumn{4}{|c|}{Llama2-70B (FP32 Accuracy = 69.12\%)} \\ 
 \hline
 \hline
 64 & 43.9 & 43.4 & 43.9 & 44.9 & 68.07 & 68.27 & 68.17 & 68.75 \\
 \hline
 32 & 44.5 & 43.8 & 44.9 & 44.5 & 68.37 & 68.51 & 68.35 & 68.27  \\
 \hline
 16 & 43.9 & 42.7 & 44.9 & 45 & 68.12 & 68.77 & 68.31 & 68.59  \\
 \hline
 \hline
 \multicolumn{5}{|c|}{GPT3-22B (FP32 Accuracy = 38.75\%)} & \multicolumn{4}{|c|}{Nemotron4-15B (FP32 Accuracy = 64.3\%)} \\ 
 \hline
 \hline
 64 & 36.71 & 38.85 & 38.13 & 38.92 & 63.17 & 62.36 & 63.72 & 64.09 \\
 \hline
 32 & 37.95 & 38.69 & 39.45 & 38.34 & 64.05 & 62.30 & 63.8 & 64.33  \\
 \hline
 16 & 38.88 & 38.80 & 38.31 & 38.92 & 63.22 & 63.51 & 63.93 & 64.43  \\
 \hline
\end{tabular}
\caption{\label{tab:mmlu_abalation} Accuracy on MMLU dataset across GPT3-22B, Llama2-7B, 70B and Nemotron4-15B models.}
\end{table}


%\subsection{Perplexity achieved by various LO-BCQ configurations on LM evaluation harness}

\begin{table} \centering
\begin{tabular}{|c||c|c|c|c||c|c|c|c|} 
\hline
 $L_b \rightarrow$& \multicolumn{4}{c||}{8} & \multicolumn{4}{c||}{8}\\
 \hline
 \backslashbox{$L_A$\kern-1em}{\kern-1em$N_c$} & 2 & 4 & 8 & 16 & 2 & 4 & 8 & 16  \\
 %$N_c \rightarrow$ & 2 & 4 & 8 & 16 & 2 & 4 & 2 \\
 \hline
 \hline
 \multicolumn{5}{|c|}{Race (FP32 Accuracy = 37.51\%)} & \multicolumn{4}{|c|}{Boolq (FP32 Accuracy = 64.62\%)} \\ 
 \hline
 \hline
 64 & 36.94 & 37.13 & 36.27 & 37.13 & 63.73 & 62.26 & 63.49 & 63.36 \\
 \hline
 32 & 37.03 & 36.36 & 36.08 & 37.03 & 62.54 & 63.51 & 63.49 & 63.55  \\
 \hline
 16 & 37.03 & 37.03 & 36.46 & 37.03 & 61.1 & 63.79 & 63.58 & 63.33  \\
 \hline
 \hline
 \multicolumn{5}{|c|}{Winogrande (FP32 Accuracy = 58.01\%)} & \multicolumn{4}{|c|}{Piqa (FP32 Accuracy = 74.21\%)} \\ 
 \hline
 \hline
 64 & 58.17 & 57.22 & 57.85 & 58.33 & 73.01 & 73.07 & 73.07 & 72.80 \\
 \hline
 32 & 59.12 & 58.09 & 57.85 & 58.41 & 73.01 & 73.94 & 72.74 & 73.18  \\
 \hline
 16 & 57.93 & 58.88 & 57.93 & 58.56 & 73.94 & 72.80 & 73.01 & 73.94  \\
 \hline
\end{tabular}
\caption{\label{tab:mmlu_abalation} Accuracy on LM evaluation harness tasks on GPT3-1.3B model.}
\end{table}

\begin{table} \centering
\begin{tabular}{|c||c|c|c|c||c|c|c|c|} 
\hline
 $L_b \rightarrow$& \multicolumn{4}{c||}{8} & \multicolumn{4}{c||}{8}\\
 \hline
 \backslashbox{$L_A$\kern-1em}{\kern-1em$N_c$} & 2 & 4 & 8 & 16 & 2 & 4 & 8 & 16  \\
 %$N_c \rightarrow$ & 2 & 4 & 8 & 16 & 2 & 4 & 2 \\
 \hline
 \hline
 \multicolumn{5}{|c|}{Race (FP32 Accuracy = 41.34\%)} & \multicolumn{4}{|c|}{Boolq (FP32 Accuracy = 68.32\%)} \\ 
 \hline
 \hline
 64 & 40.48 & 40.10 & 39.43 & 39.90 & 69.20 & 68.41 & 69.45 & 68.56 \\
 \hline
 32 & 39.52 & 39.52 & 40.77 & 39.62 & 68.32 & 67.43 & 68.17 & 69.30  \\
 \hline
 16 & 39.81 & 39.71 & 39.90 & 40.38 & 68.10 & 66.33 & 69.51 & 69.42  \\
 \hline
 \hline
 \multicolumn{5}{|c|}{Winogrande (FP32 Accuracy = 67.88\%)} & \multicolumn{4}{|c|}{Piqa (FP32 Accuracy = 78.78\%)} \\ 
 \hline
 \hline
 64 & 66.85 & 66.61 & 67.72 & 67.88 & 77.31 & 77.42 & 77.75 & 77.64 \\
 \hline
 32 & 67.25 & 67.72 & 67.72 & 67.00 & 77.31 & 77.04 & 77.80 & 77.37  \\
 \hline
 16 & 68.11 & 68.90 & 67.88 & 67.48 & 77.37 & 78.13 & 78.13 & 77.69  \\
 \hline
\end{tabular}
\caption{\label{tab:mmlu_abalation} Accuracy on LM evaluation harness tasks on GPT3-8B model.}
\end{table}

\begin{table} \centering
\begin{tabular}{|c||c|c|c|c||c|c|c|c|} 
\hline
 $L_b \rightarrow$& \multicolumn{4}{c||}{8} & \multicolumn{4}{c||}{8}\\
 \hline
 \backslashbox{$L_A$\kern-1em}{\kern-1em$N_c$} & 2 & 4 & 8 & 16 & 2 & 4 & 8 & 16  \\
 %$N_c \rightarrow$ & 2 & 4 & 8 & 16 & 2 & 4 & 2 \\
 \hline
 \hline
 \multicolumn{5}{|c|}{Race (FP32 Accuracy = 40.67\%)} & \multicolumn{4}{|c|}{Boolq (FP32 Accuracy = 76.54\%)} \\ 
 \hline
 \hline
 64 & 40.48 & 40.10 & 39.43 & 39.90 & 75.41 & 75.11 & 77.09 & 75.66 \\
 \hline
 32 & 39.52 & 39.52 & 40.77 & 39.62 & 76.02 & 76.02 & 75.96 & 75.35  \\
 \hline
 16 & 39.81 & 39.71 & 39.90 & 40.38 & 75.05 & 73.82 & 75.72 & 76.09  \\
 \hline
 \hline
 \multicolumn{5}{|c|}{Winogrande (FP32 Accuracy = 70.64\%)} & \multicolumn{4}{|c|}{Piqa (FP32 Accuracy = 79.16\%)} \\ 
 \hline
 \hline
 64 & 69.14 & 70.17 & 70.17 & 70.56 & 78.24 & 79.00 & 78.62 & 78.73 \\
 \hline
 32 & 70.96 & 69.69 & 71.27 & 69.30 & 78.56 & 79.49 & 79.16 & 78.89  \\
 \hline
 16 & 71.03 & 69.53 & 69.69 & 70.40 & 78.13 & 79.16 & 79.00 & 79.00  \\
 \hline
\end{tabular}
\caption{\label{tab:mmlu_abalation} Accuracy on LM evaluation harness tasks on GPT3-22B model.}
\end{table}

\begin{table} \centering
\begin{tabular}{|c||c|c|c|c||c|c|c|c|} 
\hline
 $L_b \rightarrow$& \multicolumn{4}{c||}{8} & \multicolumn{4}{c||}{8}\\
 \hline
 \backslashbox{$L_A$\kern-1em}{\kern-1em$N_c$} & 2 & 4 & 8 & 16 & 2 & 4 & 8 & 16  \\
 %$N_c \rightarrow$ & 2 & 4 & 8 & 16 & 2 & 4 & 2 \\
 \hline
 \hline
 \multicolumn{5}{|c|}{Race (FP32 Accuracy = 44.4\%)} & \multicolumn{4}{|c|}{Boolq (FP32 Accuracy = 79.29\%)} \\ 
 \hline
 \hline
 64 & 42.49 & 42.51 & 42.58 & 43.45 & 77.58 & 77.37 & 77.43 & 78.1 \\
 \hline
 32 & 43.35 & 42.49 & 43.64 & 43.73 & 77.86 & 75.32 & 77.28 & 77.86  \\
 \hline
 16 & 44.21 & 44.21 & 43.64 & 42.97 & 78.65 & 77 & 76.94 & 77.98  \\
 \hline
 \hline
 \multicolumn{5}{|c|}{Winogrande (FP32 Accuracy = 69.38\%)} & \multicolumn{4}{|c|}{Piqa (FP32 Accuracy = 78.07\%)} \\ 
 \hline
 \hline
 64 & 68.9 & 68.43 & 69.77 & 68.19 & 77.09 & 76.82 & 77.09 & 77.86 \\
 \hline
 32 & 69.38 & 68.51 & 68.82 & 68.90 & 78.07 & 76.71 & 78.07 & 77.86  \\
 \hline
 16 & 69.53 & 67.09 & 69.38 & 68.90 & 77.37 & 77.8 & 77.91 & 77.69  \\
 \hline
\end{tabular}
\caption{\label{tab:mmlu_abalation} Accuracy on LM evaluation harness tasks on Llama2-7B model.}
\end{table}

\begin{table} \centering
\begin{tabular}{|c||c|c|c|c||c|c|c|c|} 
\hline
 $L_b \rightarrow$& \multicolumn{4}{c||}{8} & \multicolumn{4}{c||}{8}\\
 \hline
 \backslashbox{$L_A$\kern-1em}{\kern-1em$N_c$} & 2 & 4 & 8 & 16 & 2 & 4 & 8 & 16  \\
 %$N_c \rightarrow$ & 2 & 4 & 8 & 16 & 2 & 4 & 2 \\
 \hline
 \hline
 \multicolumn{5}{|c|}{Race (FP32 Accuracy = 48.8\%)} & \multicolumn{4}{|c|}{Boolq (FP32 Accuracy = 85.23\%)} \\ 
 \hline
 \hline
 64 & 49.00 & 49.00 & 49.28 & 48.71 & 82.82 & 84.28 & 84.03 & 84.25 \\
 \hline
 32 & 49.57 & 48.52 & 48.33 & 49.28 & 83.85 & 84.46 & 84.31 & 84.93  \\
 \hline
 16 & 49.85 & 49.09 & 49.28 & 48.99 & 85.11 & 84.46 & 84.61 & 83.94  \\
 \hline
 \hline
 \multicolumn{5}{|c|}{Winogrande (FP32 Accuracy = 79.95\%)} & \multicolumn{4}{|c|}{Piqa (FP32 Accuracy = 81.56\%)} \\ 
 \hline
 \hline
 64 & 78.77 & 78.45 & 78.37 & 79.16 & 81.45 & 80.69 & 81.45 & 81.5 \\
 \hline
 32 & 78.45 & 79.01 & 78.69 & 80.66 & 81.56 & 80.58 & 81.18 & 81.34  \\
 \hline
 16 & 79.95 & 79.56 & 79.79 & 79.72 & 81.28 & 81.66 & 81.28 & 80.96  \\
 \hline
\end{tabular}
\caption{\label{tab:mmlu_abalation} Accuracy on LM evaluation harness tasks on Llama2-70B model.}
\end{table}

%\section{MSE Studies}
%\textcolor{red}{TODO}


\subsection{Number Formats and Quantization Method}
\label{subsec:numFormats_quantMethod}
\subsubsection{Integer Format}
An $n$-bit signed integer (INT) is typically represented with a 2s-complement format \citep{yao2022zeroquant,xiao2023smoothquant,dai2021vsq}, where the most significant bit denotes the sign.

\subsubsection{Floating Point Format}
An $n$-bit signed floating point (FP) number $x$ comprises of a 1-bit sign ($x_{\mathrm{sign}}$), $B_m$-bit mantissa ($x_{\mathrm{mant}}$) and $B_e$-bit exponent ($x_{\mathrm{exp}}$) such that $B_m+B_e=n-1$. The associated constant exponent bias ($E_{\mathrm{bias}}$) is computed as $(2^{{B_e}-1}-1)$. We denote this format as $E_{B_e}M_{B_m}$.  

\subsubsection{Quantization Scheme}
\label{subsec:quant_method}
A quantization scheme dictates how a given unquantized tensor is converted to its quantized representation. We consider FP formats for the purpose of illustration. Given an unquantized tensor $\bm{X}$ and an FP format $E_{B_e}M_{B_m}$, we first, we compute the quantization scale factor $s_X$ that maps the maximum absolute value of $\bm{X}$ to the maximum quantization level of the $E_{B_e}M_{B_m}$ format as follows:
\begin{align}
\label{eq:sf}
    s_X = \frac{\mathrm{max}(|\bm{X}|)}{\mathrm{max}(E_{B_e}M_{B_m})}
\end{align}
In the above equation, $|\cdot|$ denotes the absolute value function.

Next, we scale $\bm{X}$ by $s_X$ and quantize it to $\hat{\bm{X}}$ by rounding it to the nearest quantization level of $E_{B_e}M_{B_m}$ as:

\begin{align}
\label{eq:tensor_quant}
    \hat{\bm{X}} = \text{round-to-nearest}\left(\frac{\bm{X}}{s_X}, E_{B_e}M_{B_m}\right)
\end{align}

We perform dynamic max-scaled quantization \citep{wu2020integer}, where the scale factor $s$ for activations is dynamically computed during runtime.

\subsection{Vector Scaled Quantization}
\begin{wrapfigure}{r}{0.35\linewidth}
  \centering
  \includegraphics[width=\linewidth]{sections/figures/vsquant.jpg}
  \caption{\small Vectorwise decomposition for per-vector scaled quantization (VSQ \citep{dai2021vsq}).}
  \label{fig:vsquant}
\end{wrapfigure}
During VSQ \citep{dai2021vsq}, the operand tensors are decomposed into 1D vectors in a hardware friendly manner as shown in Figure \ref{fig:vsquant}. Since the decomposed tensors are used as operands in matrix multiplications during inference, it is beneficial to perform this decomposition along the reduction dimension of the multiplication. The vectorwise quantization is performed similar to tensorwise quantization described in Equations \ref{eq:sf} and \ref{eq:tensor_quant}, where a scale factor $s_v$ is required for each vector $\bm{v}$ that maps the maximum absolute value of that vector to the maximum quantization level. While smaller vector lengths can lead to larger accuracy gains, the associated memory and computational overheads due to the per-vector scale factors increases. To alleviate these overheads, VSQ \citep{dai2021vsq} proposed a second level quantization of the per-vector scale factors to unsigned integers, while MX \citep{rouhani2023shared} quantizes them to integer powers of 2 (denoted as $2^{INT}$).

\subsubsection{MX Format}
The MX format proposed in \citep{rouhani2023microscaling} introduces the concept of sub-block shifting. For every two scalar elements of $b$-bits each, there is a shared exponent bit. The value of this exponent bit is determined through an empirical analysis that targets minimizing quantization MSE. We note that the FP format $E_{1}M_{b}$ is strictly better than MX from an accuracy perspective since it allocates a dedicated exponent bit to each scalar as opposed to sharing it across two scalars. Therefore, we conservatively bound the accuracy of a $b+2$-bit signed MX format with that of a $E_{1}M_{b}$ format in our comparisons. For instance, we use E1M2 format as a proxy for MX4.

\begin{figure}
    \centering
    \includegraphics[width=1\linewidth]{sections//figures/BlockFormats.pdf}
    \caption{\small Comparing LO-BCQ to MX format.}
    \label{fig:block_formats}
\end{figure}

Figure \ref{fig:block_formats} compares our $4$-bit LO-BCQ block format to MX \citep{rouhani2023microscaling}. As shown, both LO-BCQ and MX decompose a given operand tensor into block arrays and each block array into blocks. Similar to MX, we find that per-block quantization ($L_b < L_A$) leads to better accuracy due to increased flexibility. While MX achieves this through per-block $1$-bit micro-scales, we associate a dedicated codebook to each block through a per-block codebook selector. Further, MX quantizes the per-block array scale-factor to E8M0 format without per-tensor scaling. In contrast during LO-BCQ, we find that per-tensor scaling combined with quantization of per-block array scale-factor to E4M3 format results in superior inference accuracy across models. 



\end{document}