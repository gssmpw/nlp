\section{Related Work}

The principles of self-adaptation in MLS have been explored in various contexts. Recent advancements, such as Tundo et al. \cite{b28} observed, these systems often rely on predefined configurations, limiting their potential to dynamically adapt to unforeseen runtime conditions. Similarly, while Convolutional Neural Networks (CNNs) have revolutionized object detection \cite{b29}, their application often emphasizes optimizing individual models \cite{b30, b31, b32, b33, b34} rather than system-wide adaptability. 

Notably, a recent survey \cite{b22} highlighted the underutilization of unsupervised learning in self-adaptive systems, echoing the need for scalable, robust architectures for MLS. Despite these advancements, practical applications of self-adaptive systems to real-time scenarios, such as resource-constrained edge-based object detection, remain limited, leaving critical gap in the field. 

The emergence of Green AI has significantly reshaped the landscape of Machine Learning (ML) research, emphasizing energy efficiency and sustainability. Verdecchia et al. \cite{b25} provided a comprehensive review of 98 studies, underscoring a predominant focus on energy efficiency mechanisms within ML systems. However, the translation of these theoretical advancements into practical solutions, particularly in runtime contexts, remains sparse. While early works, such as IBM's autonomic computing vision \cite{b26}, laid the groundwork for self-adaptive systems, these concepts have evolved to encompass ML-enabled systems, introducing the possibility of dynamic adaptability to changing operational conditions.

Multiple studies have explored sustainable development in AI through energy-efficient methods. For instance, Jarvenpaa et al. \cite{b27} identified 12 architectural tactics for sustainability in ML-enabled system, bridging the gap between academic research and practical implementations. Similar studies such as Dagoberto et al. \cite{b11} demonstrated a remarkable 28.7\% reduction in CO2e emissions through hyperparameter optimization, aligning with the Green AI framework. These findings align with broader themes of integrating sustainability-focused tactics at the design stage to achieve computational efficiency and reduce ecological footprints. 

Efforts like those by Dagoberto et al. \cite{b12} studied AutoML systems, optimizing energy consumption without comprising accuracy. These studies contribute to the ongoing development of eco-conscious ML processes by introducing metrics and practices that prioritize energy efficiency during training and inference phases. However, while these approaches address sustainability at the design level, runtime energy efficiency, particularly for real-time applications, remains under-explored.

Departing from the aforementioned approaches, our work focuses on real-time object detection at the edge, leveraging runtime adaptability. It is based on a model selection approach grounded in Epsilon-Greedy exploration to optimize real-time object detection at the edge. Further as demonstrated by our results, this approach balances computational efficiency, accuracy, and resource fairness by dynamically adapting to runtime conditions. 