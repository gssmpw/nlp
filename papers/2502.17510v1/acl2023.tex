% This must be in the first 5 lines to tell arXiv to use pdfLaTeX, which is strongly recommended.
\pdfoutput=1
% In particular, the hyperref package requires pdfLaTeX in order to break URLs across lines.

\documentclass[11pt]{article}

% Remove the "review" option to generate the final version.
\usepackage[]{ACL2023}

% Standard package includes
\usepackage{times}
\usepackage{latexsym}

% For proper rendering and hyphenation of words containing Latin characters (including in bib files)
\usepackage[T1]{fontenc}
% For Vietnamese characters
% \usepackage[T5]{fontenc}
% See https://www.latex-project.org/help/documentation/encguide.pdf for other character sets

% This assumes your files are encoded as UTF8
\usepackage[utf8]{inputenc}

% This is not strictly necessary, and may be commented out.
% However, it will improve the layout of the manuscript,
% and will typically save some space.
\usepackage{microtype}

% This is also not strictly necessary, and may be commented out.
% However, it will improve the aesthetics of text in
% the typewriter font.

% new
\usepackage{graphicx}
\usepackage{subfigure}
\usepackage{tabularx,booktabs}
\usepackage{enumitem}
\usepackage{amsmath}
\usepackage{inconsolata}
\usepackage{amssymb}
\usepackage{multirow}
\usepackage{algorithm,algorithmic}
\usepackage{pifont}
\newcommand{\xmark}{\ding{55}}%
\newcommand{\cmark}{\ding{51}}%
\usepackage{inconsolata}
\usepackage{colortbl}
\newcommand{\ouralg}{Recurrent-KIF} 

\captionsetup[subfigure]{skip=0pt}

% If the title and author information does not fit in the area allocated, uncomment the following
%
%\setlength\titlebox{<dim>}
%
% and set <dim> to something 5cm or larger.

\title{Recurrent Knowledge Identification and Fusion for Language Model Continual Learning}
%\title{Fast-Slow Bi-level Optimization: Knowledge Identification and Fusion for Language Model Continual Learning}


% Author information can be set in various styles:
% For several authors from the same institution:
% \author{Author 1 \and ... \and Author n \\
%         Address line \\ ... \\ Address line}
% if the names do not fit well on one line use
%         Author 1 \\ {\bf Author 2} \\ ... \\ {\bf Author n} \\
% For authors from different institutions:
% \author{Author 1 \\ Address line \\  ... \\ Address line
%         \And  ... \And
%         Author n \\ Address line \\ ... \\ Address line}
% To start a seperate ``row'' of authors use \AND, as in
% \author{Author 1 \\ Address line \\  ... \\ Address line
%         \AND
%         Author 2 \\ Address line \\ ... \\ Address line \And
%         Author 3 \\ Address line \\ ... \\ Address line} dagger

\author{Yujie Feng$^{1}$\thanks{~ Equal contribution.}\enspace, Xujia Wang$^{2}$\footnotemark[1]\enspace, Zexin Lu$^{4}$, Shenghong Fu$^{1}$, Guangyuan Shi$^{1}$ \\ \textbf{Yongxin Xu}$^{3}$\textbf{,} \textbf{Yasha Wang}$^{3}$\textbf{,} \textbf{Philip S. Yu}$^{5}$\textbf{,} \textbf{Xu Chu}$^{3}$\thanks{ ~ Corresponding author.}\enspace\textbf{,} \textbf{Xiao-Ming Wu}$^{1}$\footnotemark[2] \\
$^1$The Hong Kong Polytechnic University
$^2$Tsinghua University 
$^3$Peking University \\
$^4$Huawei Hong Kong Research Center 
$^5$University of Illinois at Chicago \\
 yujie.feng@connect.polyu.hk, xiao-ming.wu@polyu.edu.hk 
}

\begin{document}
\maketitle
\begin{abstract}
\begin{abstract} 
The integration of Large Language Models (LLMs) into software development has revolutionized the field, particularly through the use of Retrieval-Augmented Code Generation (RACG) systems that enhance code generation with information from external knowledge bases. However, the security implications of RACG systems, particularly the risks posed by vulnerable code examples in the knowledge base, remain largely unexplored. This risk is notably concerning given that public code repositories, which often serve as the sources for knowledge base collection in RACG systems, are usually accessible to anyone in the community. Malicious attackers can exploit this accessibility to inject vulnerable code into the knowledge base, making it toxic. 
Once these poisoned samples are retrieved and incorporated into the generated code, they can propagate security vulnerabilities into the final product. This paper presents the first comprehensive study on the security risks associated with RACG systems, focusing on how vulnerable code in the knowledge base compromises the security of generated code. We investigate the LLM-generated code security across different settings through extensive experiments using four major LLMs, two retrievers, and two poisoning scenarios. Our findings highlight the significant threat of knowledge base poisoning, where even a single poisoned code example can compromise up to 48\% of the generated code. 
Our findings provide crucial insights into vulnerability introduction in RACG systems and offer practical mitigation recommendations, thereby helping improve the security of LLM-generated code in future works.
\end{abstract}


\end{abstract}

\section{Introduction}
\section{Introduction}

\begin{figure}[h]
    \centering
    \begin{overpic}[trim=0cm 0cm 0cm 0cm,clip,angle=0,origin=c,width=.4\linewidth]{images/teaser_absolute.png}
        %  trim={<left> <lower> <right> <upper>}
        %  \put(horiz, vert)
        %  \put(horiz, vert){\rotatebox{90}{Text}}
        %
        \put(107, 32){$\mathbf{\to}$}
    \end{overpic}\hspace{1cm}
    \begin{overpic}[trim=0cm 0cm 0cm 0cm,clip,angle=0,origin=c,width=.4\linewidth]{images/teaser_translated_yellow.png}
        %  trim={<left> <lower> <right> <upper>}
        %  \put(horiz, vert)
        %  \put(horiz, vert){\rotatebox{90}{Text}}
        %
    \end{overpic}
    \caption{Using translation methods, a controller trained on an environment with a given visual variation \textit{(left)} can be reused without any training or fine-tuning on a different environment (\textit{right}) with comparable performance. In red we see the trajectory of a car driven by the same controller when connected to two different encoders, one for each visual variation.
    }
    \label{fig:teaser}
\end{figure}

Deep Reinforcement Learning (RL) has enabled agents to achieve remarkable performance in complex decision-making tasks, from robotic manipulation to high-dimensional games (Mnih et al., 2015; Silver et al., 2017). 
Although recent RL techniques achieved strong improvements over sample efficiency \citep{yarats2021drqv2, kostrikov2020image}, training new agents remains a costly process, both in computational and temporal terms.
Despite these advances, most methods still require at least partial retraining when dealing with domain shifts such as visual appearance, reward functions, or action spaces \citep{pmlr-v97-cobbe19a, zhang2020learning}. These domain changes typically require expensive retraining, which can be prohibitive for real-world settings that require millions of interactions.

A variety of approaches have been proposed to address these shifting conditions. Domain randomization \citep{tobin2017domain, sadeghi2016cad2rl} trains agents across diverse visual styles or physics settings, promoting invariant features but demanding broader coverage of possible variations. Multi-task RL \citep{parisotto2015actor, teh2017distral} attempts to learn shared representations across multiple tasks.

In the supervised setting, recent representation learning techniques \citep{Moschella2022-yf,maiorca2023latent, norelli2022b, cannistraci2023bricks}, show that it is possible to zero-shot recombine encoders and decoders to perform new tasks across different modalities (images, text..) and tasks (classification, reconstruction) and even architectures.
In RL, methods adopting the relative representation framework \citep{Moschella2022-yf} have shown promising results in adapting encoders to different controllers with zero or few-shots adaptation, for robotic control from proprioceptive states \citep{jian2021adversarial} or for playing games in the Gymnasium suite \citep{towers2024gymnasium} from pixels \citep{ricciardi2025r3lrelativerepresentationsreinforcement}.
These methods, however, still require training models to use the new relative representations.

By contrast, \cite{maiorca2023latent} suggest that modules from independently trained neural networks can be connected via a simple linear or affine transformation, with no training constraint or fine-tuning required, if such transformations can be reliably estimated from a small set of “anchor” samples, pairs of states or observations deemed semantically equivalent.

Our main contribution is the implementation of a RL method based on semantic alignment to map between latent spaces of different neural models, so that their encoders and controllers can be stitched with the goal of creating new agents that can act on visual-task combinations never seen together in training. This includes the use of the transformations to map modules from different networks, and the collection of anchor samples used to estimate these transformations. We call our method Semantic Alignment for Policy Stitching (\textbf{SAPS}).
We perform analyses and empirical tests on the CarRacing and LunarLander environments to show the performance of new agents created via zero-shot stitching of encoders and controllers trained on different visual-task variations, demonstrating significant gains compared to existing zero-shot methods.

\section{Related Work}
\subsection{Continual Learning for LLMs}
Continual learning (CL) \cite{zhou2024continual} focuses on developing algorithms that accumulate knowledge from non-stationary data. In the LLM era, model mixture-based methods using PEFT have become dominant \cite{wang2023rehearsal, huang2024mitigating, wang2024inscl}, typically divided into model ensemble and merging approaches.

Model ensemble methods isolate parameters by assigning independent PEFT blocks to each task \cite{feng2023towards, pham2023continual, ke2023sub, li2024revisiting, he2024seekr, wang2024self}. For example, O-LoRA \cite{wang2023orthogonal} enforces orthogonality among LoRA adapters, while SAPT \cite{zhao2024sapt} uses a selection module to combine blocks based on task correlations. 
%Although these methods effectively preserve task-specific knowledge, they limit inter-task knowledge transfer and incur high memory overhead as the number of tasks increases, thus hindering scalability.
While preserving task-specific knowledge, they hinder inter-task transfer and incur high memory overhead as the number of tasks increases, limiting their scalability.
%Model ensemble methods adopt the concept of parameter isolation, learning tasks in a pipeline manner where each task is assigned an independent PEFT block \cite{wang2024rehearsal, he2024seekr, wang2024self}. For instance, \citet{wang2023orthogonal} proposed O-LoRA that constrains the learning of PEFT blocks to maintain orthogonality. Similarly, \citet{zhao2024sapt} introduced SAPT, which employs a learnable selection module to combine PEFT blocks based on task correlations, thereby enhancing KT. While such approaches can effectively prevent forgetting of old knowledge, they inherently restrict knowledge transfer across tasks. Moreover, as the number of tasks increases, the number of required PEFT blocks grows, resulting in significant memory storage demands and limiting their scalability for handling long task sequences. Model ensemble methods effectively isolate task-specific knowledge but impose high memory requirements as tasks accumulate. 

In contrast, model merging methods combine multiple models into a single model \cite{cheng2024dam, alexandrov2024mitigating, ren2024analyzing}, alleviating memory constraints.
%by eliminating the need for separate task blocks.
For example, global model merging approaches \cite{wortsman2022model, ilharco2023editing} perform a weighted fusion of models before and after training, typically assuming that all model weights contribute equally to each task.
However, determining which and how to merge parameters remains an open problem.
In this paper, we propose {\ouralg}, a novel framework that leverages the dynamic importance of parameters across different tasks by employing knowledge identification and fusion techniques to mitigate CF and promote KT.



\subsection{Parameter Importance Identification}
Identifying important parameters or knowledge regions within LLMs has gained significant attention in the NLP community \cite{zhao2023does, liu2023good, feng2024tasl2, xu2024parenting, shi2024understanding}. This research improves our understanding of LLMs and enhances their performance across a variety of tasks, including model editing \cite{wang2024editing}, compression \cite{zhang2023adalora}.

In the context of CL, \citet{du2024unlocking} use the gradient magnitudes to selectively update parameters. \citet{feng2024tasl} employ gradient-based metrics to compare the parameter importance distributions of current and historical tasks, merging task-shared regions to promote KT and retaining task-specific regions to prevent CF.
However, these approaches are limited by their reliance on static importance estimations for previous tasks, which become outdated as the model evolves. 
%Static importance estimations fail to capture the dynamic nature of knowledge acquisition and retention, leading to decreased robustness and accuracy in knowledge localization over time.

To address this limitation, \citet{wu2024meta} introduce VR-MCL, a replay-based method that dynamically updates importance information while reducing variance from random sampling. 
Although VR-MCL achieves dynamic importance estimation for historical tasks, it mainly focuses on preserving task-specific knowledge and does not update task-shared regions, thus limiting KT across tasks.
%As the closest related work, VR-MCL also achieves dynamic estimation of importance for historical tasks. However, this approach primarily focuses on preserving task-specific knowledge and neglects updates to task-shared regions, thereby limiting KT across tasks.
In contrast, inspired by the CLS theory, we propose a dynamic importance estimation method that iteratively updates parameter importance through inner and outer loops.
%we propose a dynamic importance estimation method that continuously updates parameter importance distributions through the iterative execution of inner and outer loops. 
Our approach performs multi-round knowledge fusion, adaptively adjusting the integration of new and historical knowledge based on the latest model state. This method outperforms traditional post-training fusion by enhancing robustness and enabling smoother optimization.
%Additionally, our multi-round knowledge fusion approach adaptively adjusts fusion weights for new and historical knowledge based on the latest model state, providing significant advantages over traditional post-training fusion by enhancing robustness and achieving smoother optimization.

%Unlike VR-MCL, which requires hyper-gradient updates and additional computational overhead, our method leverages residuals from the inner and outer loops, eliminating the need for extra gradient computations. This results in improved efficiency and ensures effective knowledge integration for CL.

%To address this, we propose a novel bi-level knowledge identification and fusion framework that enables dynamic importance estimations, improving robustness and accuracy in knowledge localization.
%To address these challenges, this paper introduces a dynamic importance estimation technique to continuously capture up-to-date importance distributions. By recalculating importance distributions multiple times, our method not only enhances robustness against biases but also improves the accuracy of knowledge localization.


% \subsection{Bi-level Optimization}
% Bi-level optimization models, which represent nested decision-making structures \cite{vicente1994bilevel}, has gained significant attention in CL \cite{pham2023continual, hao2024bilevel}. These studies aim to mitigate CF by introducing additional learning components \cite{qiang2024bilora, zhang2024blo} or memory units \cite{ren2024analyzing}. For example, \citet{pham2023continual} proposed DualNets, which maintain two separate systems with distinct supervised and unsupervised loss constraints. 
% However, such methods often prove inefficient for LLMs due to their high resource demands.

% The closest related work is \citet{wu2024meta}, which introduced VR-MCL, a replay-based method for updating importance information while reducing variance from random sampling. However, this approach mainly focuses on preserving task-specific knowledge and neglects updates to task-shared regions, which limits KT across tasks.

% In contrast, inspired by the CLS theory from neuroscience, we propose a novel bi-level optimization paradigm that integrates knowledge identification and fusion based on parameter importance. Unlike VR-MCL, which requires hyper-gradient updates and additional computational overhead, our method leverages residuals from the inner and outer loops, eliminating the need for extra gradient computations. This results in improved efficiency and ensures effective knowledge integration for CL.


%In contrast, inspired by the CLS theory in neuroscience, we propose a new bi-level optimization paradigm that integrates knowledge identification and fusion based on parameter importance. Unlike VR-MCL, which relies on hyper-gradient updates requiring additional computation, our approach directly utilizes residuals from the inner and outer loops, eliminating extra gradient computations and improving efficiency, while ensuring effective knowledge integration for CL.

%By leveraging parameter importance from both loops, our method enables more efficient and precise knowledge acquisition, facilitating continual learning.




%Furthermore, our fine-grained bi-level model training strategy dynamically and continuously captures up-to-date parameter importance distributions for both current and historical tasks multiple times during training, offering a more robust and adaptive solution compared to previous methods based on static importance estimation.








%useless


%However, these methods are hindered by static importance estimations, which are biased by random training data and become outdated as the model evolves. 
%Although these methods show promise in mitigating CF, they are limited by static importance estimations caused by two factors: bias introduced by the randomness in training data when importance is estimated only once, and outdated importance distributions as the model state evolves during training.
%For instance, in continual learning, knowledge localization techniques have proven effective for understanding task-specific and shared regions in LLMs. Researchers such as xxx and xxx have leveraged gradient-based metrics to estimate parameter importance, enabling effective parameter merging. These methods have shown promising results in mitigating catastrophic forgetting. However, a common limitation of these approaches is the issue of static importance estimations. This arises due to two key factors: 1.	The randomness inherent in training data can lead to biased importance estimations when conducted only once. 2.	As the model's state evolves during training, the previously estimated importance distribution may no longer remain accurate.

% \section{Setting and Notations \ Preliminary}
% 
\begin{figure*}[t]
  \centering
  \includegraphics[width=0.95\linewidth]{imgs/method5.pdf}
  \caption{\textbf{Iterative update process of {\ouralg} for the $b$-th iteration.} 
  The notation $\epsilon_{k}^q$ represents training samples drawn from $\mathcal{D}_k$, while $\phi_{b}$ refers to samples drawn from $\mathcal{M}_{<k}$.
  \textbf{Inner Learner (Step 1):} Performs $Q$ iterations to rapidly adapt to the new task while identifying the parameter importance distribution.
    \textbf{Outer Learner (Step 2):} Retrieves historical task information using memory data and performs knowledge fusion, guided by the importance distributions of both current and historical tasks. 
    \textbf{Recurrent Updates (Step 3):} This inner-outer loop cycle is repeated, ensuring that each fusion knowledge step is based on up-to-date importance distributions.
    %Utilizes the memory buffer to retrieve historical task information and performs knowledge fusion guided by the importance distributions of current and historical tasks. This learning cycle is repeated iteratively, ensuring that each knowledge fusion step is based on up-to-date parameter importance distributions.
  }
  \label{fig:method}
\end{figure*}




\paragraph{Problem Formulation}
%\subsection{Continual Learning Setup}
Continual learning aims to progressively accumulate knowledge from a sequence of tasks $\{\mathcal{T}_1, \ldots, \mathcal{T}_K\}$. Each task $\mathcal{T}_k$ includes a distinct dataset $\mathcal{D}_k = \left\{ \left( x_i^k, y_i^k \right) \right\}_{i=1}^{N_k}$ of size $N_k$, where $x_i^k \in \mathcal{X}_k$ and $y_i^k \in \mathcal{Y}_k$.
The model, parameterized by $\Theta$, is trained sequentially on these tasks to minimize the following objective:
% \begin{equation}
% \max_{\Theta} \sum_{k=1}^{K} \sum_{x,y \in \mathcal{D}_k} \log p_{\Theta}(y \mid x)
% \end{equation}
\begin{equation}
\mathcal{L} = \mathbb{E}_{(x, y) \sim \bigcup_{k=1}^K \mathcal{D}_k} \left[ -\log p_\Theta(y \mid x) \right]
\end{equation}

In this work, we consider a practical scenario where a small portion of data from previous tasks is stored in a memory buffer to facilitate the CL process. 
Specifically, we randomly store $\left| \mathcal{M} \right|$ samples from each task $\mathcal{T}_i$ in memory $\mathcal{M}_i$. During training, the model is jointly optimized on the new task data $\mathcal{D}_k$ and the memory buffer $\mathcal{M}_{<k}$.
%which contains data from all preceding tasks.



\paragraph{Notation}
We consider a pre-trained model $\theta \in \mathbb{R}^n$ with $n$ parameters.
After training on task $\mathcal{T}_{k-1}$, the model are denoted as $\theta^{k-1}$.
Fine-tuning on a new task $\mathcal{T}_k$ produces updated parameters $\theta^k$.
The difference $\tau^k = \theta^k - \theta^{k-1}$, referred to as the \textit{task vector} or \textit{training residual} \cite{ilharco2023editing}, represents task-specific parameter updates.
In the {\ouralg} framework, we obtain transient training residuals through each iteration of the inner and outer loops. Specifically, two task vectors are employed to capture and quantify the new knowledge learned in the inner loop and the historical knowledge retrieved in the outer loop. %These represent residuals at different levels, thereby facilitating iterative knowledge fusion.


%In our {\ouralg} framework, two task vectors are employed to capture and quantify the new knowledge learned in the inner loop and the historical knowledge retrieved in the outer loop, thereby representing training residuals at different levels.

%In our {\ouralg} framework, task vectors are used to capture and quantify the knowledge learned by the model across different tasks. Specifically, two task vectors, $\tau^{\text{in}}$ and $\tau^{\text{out}}$, are generated by the inner and outer loops, respectively, to represent parameter residuals at different levels. These vectors are then employed in the knowledge fusion process to effectively merge task-specific and task-shared knowledge based on parameter importance.


\section{Proposed Method: {\ouralg}}

\begin{figure*}[t]
  \centering
  \includegraphics[width=0.95\linewidth]{imgs/method5.pdf}
  \caption{\textbf{Iterative update process of {\ouralg} for the $b$-th iteration.} 
  The notation $\epsilon_{k}^q$ represents training samples drawn from $\mathcal{D}_k$, while $\phi_{b}$ refers to samples drawn from $\mathcal{M}_{<k}$.
  \textbf{Inner Learner (Step 1):} Performs $Q$ iterations to rapidly adapt to the new task while identifying the parameter importance distribution.
    \textbf{Outer Learner (Step 2):} Retrieves historical task information using memory data and performs knowledge fusion, guided by the importance distributions of both current and historical tasks. 
    \textbf{Recurrent Updates (Step 3):} This inner-outer loop cycle is repeated, ensuring that each fusion knowledge step is based on up-to-date importance distributions.
    %Utilizes the memory buffer to retrieve historical task information and performs knowledge fusion guided by the importance distributions of current and historical tasks. This learning cycle is repeated iteratively, ensuring that each knowledge fusion step is based on up-to-date parameter importance distributions.
  }
  \label{fig:method}
\end{figure*}




\paragraph{Problem Formulation}
%\subsection{Continual Learning Setup}
Continual learning aims to progressively accumulate knowledge from a sequence of tasks $\{\mathcal{T}_1, \ldots, \mathcal{T}_K\}$. Each task $\mathcal{T}_k$ includes a distinct dataset $\mathcal{D}_k = \left\{ \left( x_i^k, y_i^k \right) \right\}_{i=1}^{N_k}$ of size $N_k$, where $x_i^k \in \mathcal{X}_k$ and $y_i^k \in \mathcal{Y}_k$.
The model, parameterized by $\Theta$, is trained sequentially on these tasks to minimize the following objective:
% \begin{equation}
% \max_{\Theta} \sum_{k=1}^{K} \sum_{x,y \in \mathcal{D}_k} \log p_{\Theta}(y \mid x)
% \end{equation}
\begin{equation}
\mathcal{L} = \mathbb{E}_{(x, y) \sim \bigcup_{k=1}^K \mathcal{D}_k} \left[ -\log p_\Theta(y \mid x) \right]
\end{equation}

In this work, we consider a practical scenario where a small portion of data from previous tasks is stored in a memory buffer to facilitate the CL process. 
Specifically, we randomly store $\left| \mathcal{M} \right|$ samples from each task $\mathcal{T}_i$ in memory $\mathcal{M}_i$. During training, the model is jointly optimized on the new task data $\mathcal{D}_k$ and the memory buffer $\mathcal{M}_{<k}$.
%which contains data from all preceding tasks.



\paragraph{Notation}
We consider a pre-trained model $\theta \in \mathbb{R}^n$ with $n$ parameters.
After training on task $\mathcal{T}_{k-1}$, the model are denoted as $\theta^{k-1}$.
Fine-tuning on a new task $\mathcal{T}_k$ produces updated parameters $\theta^k$.
The difference $\tau^k = \theta^k - \theta^{k-1}$, referred to as the \textit{task vector} or \textit{training residual} \cite{ilharco2023editing}, represents task-specific parameter updates.
In the {\ouralg} framework, we obtain transient training residuals through each iteration of the inner and outer loops. Specifically, two task vectors are employed to capture and quantify the new knowledge learned in the inner loop and the historical knowledge retrieved in the outer loop. %These represent residuals at different levels, thereby facilitating iterative knowledge fusion.


%In our {\ouralg} framework, two task vectors are employed to capture and quantify the new knowledge learned in the inner loop and the historical knowledge retrieved in the outer loop, thereby representing training residuals at different levels.

%In our {\ouralg} framework, task vectors are used to capture and quantify the knowledge learned by the model across different tasks. Specifically, two task vectors, $\tau^{\text{in}}$ and $\tau^{\text{out}}$, are generated by the inner and outer loops, respectively, to represent parameter residuals at different levels. These vectors are then employed in the knowledge fusion process to effectively merge task-specific and task-shared knowledge based on parameter importance.



\paragraph{Overview}
{\ouralg} restructures the training process into multiple iterative learning cycles, each comprising two key components as illustrated in Figure \ref{fig:method}:
(i) \textit{\textbf{Inner Learner with Knowledge Identification:}} rapidly acquires new task knowledge while estimating the corresponding parameter importance, and
(ii) \textit{\textbf{Outer Learner with Knowledge Fusion:}} utilizes a memory buffer to retrieve historical task information. 
By leveraging the importance distributions of both current and historical tasks, it provides global control for effective knowledge transfer through redundant knowledge pruning and key knowledge merging.


%By comparing importance distributions of current and historical tasks, it provides global control over knowledge retention and transfer.
%Figure \ref{fig:method} illustrates the {\ouralg} framework, with the following subsections detailing each component.
%Figure \ref{fig:method} provides a comprehensive overview of {\ouralg}, and the following subsections elaborate on each component in detail.



\subsection{Inner Learner with Knowledge Identification}  
Assume the current task is \(\mathcal{T}_k\), and the iterative update for the model parameters $\theta^{k-1}$ at the $b$-th iteration are denoted by $\theta_b^{k-1}$ \footnote{For simplicity, we omit the superscripts $k-1$ in subsequent descriptions.}.
In the inner loop, the model initializes with $\theta_{b(0)} = \theta_b$ and is rapidly updated over \(Q\) gradient steps using batch data $\epsilon_{k}^q$ sampled from \(\mathcal{D}_k\) at the $q$-th step. 
%without any constraint.
After obtaining $\theta_{b(Q)}$ the task-specific updates are encapsulated in the task vector $\tau_b^{in} \in \mathbb{R}^n$:
\begin{equation}
\tau_b^{\text{in}} = \theta_{b(Q)} - \theta_{b(0)}
\end{equation}

This task vector captures the knowledge acquired for the current task. However, $\tau^{in}$ often contains redundant information, and directly merging it into the model may compromise historical knowledge, leading to catastrophic forgetting.
To address this, we propose a knowledge identification technique to identify the key parameters which storing critical knowledge within the task vector.
%enabling precise global control in the outer loop.

We use a commonly adopted importance metric in model pruning \cite{konishi2023spg}, defined as the magnitude of the gradient-weight product:
%Following commonly used importance metrics from the model pruning community \cite{konishi2023spg}, we define parameter importance as:
%we define parameter importance based on the magnitude of the gradient-weight product:  
\begin{equation}
\bar{I}\left(w_{i j}\right)=\left|w_{i j} \nabla_{w_{i j}} \mathcal{L}\right| \label{eq:1}
\end{equation}
where \(w_{ij}\) represents trainable parameters.  

Due to stochastic batch sampling and training dynamics, the metric in Eq. (\ref{eq:1}) may be unreliable, introducing variability \cite{zhang2022platon}. To mitigate this, we apply an exponential moving average \cite{zhang2023adalora} to smooth the trajectory gradients over $Q$ inner loop iterations:  
\begin{equation}
\begin{split}
I_{b(q)}   =\alpha_{1} I_{b(q-1)} + \left(1-\alpha_{1}\right) \bar{I}_{b(q)} \label{eq:I}
\end{split}
\end{equation}
where $\alpha_{1}$ is the smoothing factor, $q \in \left\{ 1, 2, ..., Q \right\}$ is the iteration number in the inner loop, and $I_{b(q)}$ represents smoothed importance.
The inner task vector \(\tau_b^{\text{in}}\) and its associated parameter importance \(I_b^{\text{in}}\) are then passed to the outer learner.


% Due to stochastic batch sampling and training dynamics, the metric in Eq. (\ref{eq:1}) may be unreliable, introducing variability and uncertainty \cite{zhang2022platon}. To mitigate this, we apply an exponential moving average \cite{zhang2023adalora} to smooth the trajectory gradients over $Q$ inner loop iterations:  
% \begin{equation}
% \begin{split}
% \bar{I}^{(q)}\left(w_{i j}\right)  =\alpha_{1} \bar{I}^{(q-1)}\left(w_{i j}\right)+ \left(1-\alpha_{1}\right) I^{(q)}\left(w_{i j}\right) \label{eq:I}
% \end{split}
% \end{equation}
% \begin{equation}
% \begin{split}
% \bar{U}^{(q)}\left(w_{i j}\right)  =\alpha_{2} \bar{U}^{(q-1)}\left(w_{i j}\right)+ \\ \left(1-\alpha_{2}\right)\left|I^{(q)}\left(w_{i j}\right)-\bar{I}^{(q)}\left(w_{i j}\right)\right| \label{eq:U}
% \end{split}
% \end{equation}
% where $\alpha_{1}$ and $\alpha_{2}$ are smoothing factors,
% $\bar{I}^{(q)}$ represents smoothed sensitivity and $\bar{U}^{(q)}$ quantifies uncertainty using the variation between $I^{(q)}$ and $\bar{I}^{(q)}$.
% Importance is then computed as:
% \begin{equation}
% I^{(q)}\left(w_{i j}\right)=\bar{I}^{(q)}\left(w_{i j}\right) \cdot \bar{U}^{(q)}\left(w_{i j}\right) \label{eq:2}
% \end{equation}

% The inner task vector \(\tau_b^{\text{in}}\) and its parameter importance \(I_b^{\text{in}}\) are then passed to the outer learner.



\subsection{Outer Learner with Knowledge Fusion}
The outer loop manages the global merging of knowledge, guided by parameter importance.
%From the inner loop, we obtain the current task vector $\tau_b^{\text{in}}$ and its parameter importance distribution \(I_b^{\text{in}}\).
To access historical knowledge, after acquiring $\theta_{b(Q)}$, the outer loop samples data $\phi_{b}$ from the memory buffer $\mathcal{M}_{<k}$. It then performs a single training iteration, updating the parameters to $\theta_{b(M)}$. Then the outer task vector $\tau_b^{\text{out}} \in \mathbb{R}^n$, capturing historical task information, is defined as:
\begin{equation}
\tau_b^{\text{out}} = \theta_{b(M)} - \theta_{b(Q)}
\end{equation}

\paragraph{Dynamic Update of Historical Importance Distribution.}
While obtaining the outer task vector, we calculate the historical task importance distribution based on the latest model state $\theta_{b(Q)}$, using Eq. (\ref{eq:1}).
The update process is then expressed as:
\begin{equation}
\bar{I}_b^{\text{out}} = \mathbb{P}(\bar{I}_b^{\text{out}} \mid \theta_{b(Q)})
\end{equation} 

This update, based on conditional probability, enables the computation of the historical importance distribution $I_b^{\text{out}}$ using the current model state.
This distinguishes it from traditional static importance estimation methods and ensures more accurate knowledge identification. 
However, the limited sample size from the memory buffer can introduce significant variance in the importance estimates.
%However, the limited sample size from the memory buffer and the use of a single outer loop iteration can introduce significant variance.
To address this, we also apply exponential smoothing to the previous outer loop distribution $I_{b-1}^{\text{out}}$:
\begin{equation}
I_b^{\text{out}} = \alpha_2 \bar{I}_b^{\text{out}} + (1 - \alpha_2) I_{b-1}^{\text{out}} \label{eq:out}
\end{equation} 
where $\alpha_2$ is the smoothing factor, enhancing stability and robustness in importance estimation.

% Similarly, the parameter importance distribution within $\tau_b^{\text{out}}$ is estimated using Eq. (\ref{eq:1}) and denoted as \(\bar{I}_b^{\text{out}}\). However, the limited sample size from the memory buffer and single outer loop iteration can introduce significant variance.
% To address this, we apply exponential smoothing using the previous outer loop distribution $I_{b-1}^{\text{out}}$:
% \begin{equation}
% I_b^{\text{out}} = \alpha_2 \bar{I}_b^{\text{out}} + (1 - \alpha_2) I_{b-1}^{\text{out}} \label{eq:out}
% \end{equation} 
% where $\alpha_2$ is a smoothing factor, enhancing stability and accuracy in importance estimation.
%This approach reduces uncertainty caused by single-sample mini-batch data and enhances the stability and accuracy of importance estimation.

\paragraph{Knowledge Fusion via Importance-based Binary Mask.}
Knowledge fusion is guided by the importance distributions $I_b^{\text{in}}$ and $I_b^{\text{out}}$.
To binarize the importance distributions, a quantile-based threshold $\delta$ is applied to select the top 20\% of parameters from both $I_b^{\text{in}}$ and $I_b^{\text{out}}$. This generates binary masks $m_b^{in} \in \mathbb{R}^n$ and $m_b^{out} \in \mathbb{R}^n$, defined as:
\begin{equation}
m_b^{\text{in}} = \mathbb{I}(I_b^{\text{in}} \geq \delta_b^{in}), m_b^{\text{out}} = \mathbb{I}(I_b^{\text{out}} \geq \delta_b^{out}) \label{eq:mask}
\end{equation} 
where $\mathbb{I}(\cdot)$ is the indicator function that outputs 1 if the condition is met and 0 otherwise.
Knowledge fusion is then performed as follows:
%Knowledge fusion is then performed as:
\begin{equation}
\theta_{b+1} = \theta_b + (m_b^{\text{in}} \odot \tau_b^{\text{in}} + m_b^{\text{out}} \odot \tau_b^{\text{out}}) \label{eq:fusion}
\end{equation} 
where $\odot$ denotes element-wise multiplication.

This knowledge fusion mechanism provides precise global control, effectively tackling key challenges in CL.
First, redundant information in the task vectors $\tau^{\text{in}}$ and $\tau^{\text{out}}$ is filtered out via the mask operation. Second, task-shared knowledge is effectively merged to facilitate knowledge transfer. Lastly, task-specific knowledge is preserved to prevent catastrophic forgetting.

The inner and outer loops operate iteratively, enabling multi-round fusion of knowledge. This iterative process facilitates the capture and absorption of useful information generated during training, providing smoother optimization compared to traditional post-training fusion methods.
Detailed implementation of {\ouralg} algorithm is provided in the Appendix (Algorithm~\ref{alg:my_algorithm}).


% Furthermore, our bi-level framework dynamically updates importance distributions of historical tasks during each outer loop iteration based on the latest model state. This ensures accurate and robust knowledge , addressing limitations of previous static importance estimations-based methods.
% Detailed implementation of {\ouralg} algorithm is provided in the Appendix (Algorithm~\ref{alg:my_algorithm}).

\section{Experiments and Analysis}\label{sec:exp}
\section{Experiments}\label{sec:exp}

\begin{table}[t]
\centering
\caption{\textbf{Quantitative results on OpenCompass~\cite{2023opencompass} multimodal leaderboard.}
$^{\ddag}$ denotes closed-source models. Hall denotes HallusionBench.
}
\label{tab:exp_it_oc}
\setlength{\tabcolsep}{1pt}
\begin{tabular}{l|c|c|cccccccc}
\toprule
Models   & Params & Avg. & MM- & MM- & MM- & Math- & Hall & AI2D  & OCR- & MMVet \\
   &  &  & Bench & Star & MU & Vista &  &  & Bench & \\
\midrule
Step-1o$^{\ddag}$   & N/A   & \textbf{77.7}  & 87.3  & 69.3  & 69.9 & 74.7  & 55.8 & 89.1 & 926 & \textbf{82.8}  \\
SenseNova$^{\ddag}$  & N/A   & 77.4  & 85.7  & \textbf{72.7}  & 69.6 & \textbf{78.4}  & 57.4 & 87.8 & 894 & 78.2  \\
InternVL2.5-78B-MPO~\cite{wang2024mpo}  & 78B  & 77.0   & 87.7  & 72.1  & 68.2  & 76.6  & 58.1  & 89.2 & 909 & 73.5  \\
Qwen2.5-VL-72B~\cite{bai2025qwen25vltechnicalreport}   & 73.4B  & 76.2  & \textbf{87.8}  & 71.1  & 67.9  & 70.8  & 58.8  & 88.2  & 881  & 76.7  \\
TeleMM$^{\ddag}$   & N/A   & 75.9  & 79.9 & 70.8 & 66.6 & 75.7  & \textbf{60.6}  & 88.5 & 891 & 75.7  \\
InternVL2.5-38B-MPO~\cite{wang2024mpo}  & 38B  & 75.3  & 85.4  & 70.1 & 63.8 & 73.6 & 59.7 & 87.9 & 894 & 72.6  \\
InternVL2.5-78B~\cite{chen2024expanding}  & 78B  & 75.2 & 87.5  & 69.5 & 70 & 71.4 & 57.4 & 89.1 & 853 & 71.8   \\
Qwen2-VL-72B~\cite{qwen2-vl_2024}   & 73.4B  & 74.8  & 85.9  & 68.6  & 64.3  & 69.7  & 58.7  & 88.3  & 888  & 73.9  \\
InternVL2.5-38B~\cite{chen2024expanding}  & 38B  & 73.5  & 85.4  & 68.5  & 64.6  & 72.4  & 57.9  & 87.6  & 841  & 67.2  \\
JT-VL-Chat-V3.0$^{\ddag}$  & N/A   & 73.4  & 81.7  & 67.5  & 59.3  & 71.9  & 53.9  & 87.2  & \textbf{967}  & 69.2  \\
Taiyi$^{\ddag}$  & N/A   & 73.0  & 84.8  & 69  & 60.4  & 72.3  & 56.8  & \textbf{90.8}  & 820  & 67.9  \\
Step-1.5V$^{\ddag}$  & N/A   & 72.5 & 82.0  & 65.1  & 61.2  & 69.7  & 54.3  & 87.5  & 886  & 71.3  \\
Gemini-1.5-Pro-002$^{\ddag}$~\cite{geminiteam2024gemini15unlockingmultimodal}   & N/A   & 72.1 & 82.8  & 67.1  & 68.6  & 67.8  & 55.9  & 83.3  & 770  & 74.6  \\
InternVL2.5-26B-MPO~\cite{wang2024mpo}  & 26B  & 72.1  & 84.2  & 67.7  & 56.4  & 71.5  & 52.4  & 86.2  & 905  & 68.1  \\
GPT-4o-20241120$^{\ddag}$~\cite{openai2024gpt4ocard}  & NA   & 72.0   & 84.3  & 65.1  & \textbf{70.7}  & 59.9  & 56.2  & 84.9  & 806  & 74.5  \\
LLaVA-OneVision-72B~\cite{li2024llavaonevision}  & 73B  & 68.0  & 84.5  & 65.8  & 56.6  & 68.4  & 47.9  & 86.2  & 741  & 60.6  \\
NVLM-D-72B~\cite{nvlm2024}   & 79.4B  & 67.6  & 78.5  & 63.7  & 60.8  & 63.9  & 49.7  & 80.1  & 849  & 58.9  \\
Molmo-72B~\cite{deitke2024molmo}  & 73.3B  & 64.1  & 79.5  & 63.3  & 52.8  & 55.8  & 46.6  & 83.4  & 701  & 61.1  \\
\rowcolor{Gray} \textbf{\method-72B}   & 71.8B  & 75.1  & 86.3  & 70.7  & 57.6  & 73.3  & 56.4  & 87.6  & 889   & 79.8  \\
\midrule
\multicolumn{11}{l}{\textit{Models smaller than 20B}} \\
\midrule
Ola-7b~\cite{ola_2025}   & 8.88B   & \textbf{72.6}  & \textbf{84.3}  & \textbf{70.8}  & \textbf{57.0}  & 68.4  & \textbf{53.5}  & \textbf{86.1}  & 822  & \textbf{78.6}  \\
Qwen2.5-VL-7B~\cite{bai2025qwen25vltechnicalreport}   & 8.29B   & 70.4  & 82.6  & 64.1  & 56.2  & 65.8  & 56.3  & 84.1  & 877  & 66.6  \\
InternVL2.5-8B-MPO~\cite{wang2024mpo}   & 8B   & 70.3  & 82  & 65.2  & 54.8  & 67.9  & 51.7  & 84.5  & \textbf{882}  & 68.1  \\
MiniCPM-o-2.6~\cite{yao2024minicpm}   & 8.67B   & 70.2  & 80.6  & 63.3  & 50.9  & \textbf{73.3}  & 51.1  & 86.1  & 889  & 67.2  \\
Ovis1.6-Gemma2-9B~\cite{lu2024ovis}  & 10.2B  & 68.8  & 80.5  & 62.9  & 55.0  & 67.2  & 52.2  & 84.4  & 830  & 65.0  \\
InternVL2.5-8B~\cite{chen2024expanding}   & 8B   & 68.1  & 82.5  & 63.2  & 56.2  & 64.5  & 49.0  & 84.6  & 821  & 62.8  \\
POINTS1.5-Qwen2.5-7B~\cite{points1.5_2024} & 8.3B   & 67.4  & 80.7  & 61.1  & 53.8  & 66.4  & 50.0  & 81.4  & 832  & 62.2  \\
Valley-Eagle$^{\ddag}$   & 8.9B   & 67.4  & 80.7  & 60.9  & \textbf{57.0}  & 64.6  & 48.0  & 82.5  & 842  & 61.3  \\
Qwen2-VL-7B~\cite{qwen2-vl_2024}  & 8B   & 67.0  & 81.0 & 60.7 & 53.7 & 61.4  & 50.4 & 83 & 843 & 61.8 \\
DeepSeek-VL2~\cite{wu2024deepseekvl2}   & 16.1B  & 66.4  & 81.2  & 61.0  & 50.7  & 59.4  & 51.5  & 84.5  & 825  & 60.0  \\
VITA-1.5~\cite{fu2025vita}   & 8.3B   & 63.3  & 76.8  & 60.2  & 52.6  & 66.2  & 44.6  & 79.2  & 741  & 52.7  \\
Baichuan-Omni~\cite{baichuan-omni}   & 7B   & -  & 75.6  & -  & 47.3  & 51.9  & 47.8  & -  & 700  & 65.4  \\
LLaVA-OneVision-7B~\cite{li2024llavaonevision}   & 8B   & 61.2  & 76.8  & 56.7  & 46.8  & 58.5  & 47.5  & 82.8  & 697  & 50.6  \\
Molmo-7B-D~\cite{deitke2024molmo}   & 8B   & 58.9  & 70.9  & 54.4  & 48.7  & 47.3  & 47.7  & 79.6  & 694  & 53.3  \\
% MiniCPM-o 2.6~\cite{yao2024minicpm}   & 8B   & 70.2  & 80.5  & 64.0  & 50.4  & 71.9  & 51.9  & 85.8  & 897  & 67.5  \\
\rowcolor{Gray} \textbf{\method-9B}  & 8.8B   & 69.7  & 80.7  & 60.5  & 51.2  & 68.3  & 51.8  & 84.5  & 883 & 72.3 \\
\bottomrule
\end{tabular}
\end{table}

\begin{table}[t]
  \caption{\textbf{Performance comparison on video and Interleave benchmarks} compared with existing approaches. $^*$ indicates officially released checkpoints evaluated by us. Best performance is marked \textbf{bold}. }
  \label{tab: video_n_interleave}
  \centering
  \setlength{\tabcolsep}{7.5pt}
  \begin{tabular}{lccccc}
    \toprule
       & \multicolumn{2}{c}{\textbf{VideoMME}} & \multicolumn{1}{c}{\textbf{MVBench}} & \multicolumn{2}{c}{\textbf{Llava-Interleave}}\\
    \cmidrule(r){2-3} \cmidrule(r){4-4} \cmidrule(r){5-6}
    Model & w/o subs & w subs & avg & in-domain & out-domain \\
    \midrule
     MiniCPM-V-2.6~\cite{yao2024minicpm} &  60.9 &  63.6 &  - &  - &  - \\
     LLaVA-OneVision-7B~\cite{li2024llavaonevision} &  58.2 &  - &  - &  - &  - \\
     Qwen2-VL-7B~\cite{qwen2-vl_2024} &  63.3 &  69.0 &  67.0 &  49.5$^*$ &  51.0$^*$ \\
     InternVL2-8B~\cite{chen2024far} &  56.3 & 59.3 &  65.8 &  - &  - \\
     VITA-1.5~\cite{fu2025vita} &  56.1 & 58.7 &  55.4 &  - &  - \\
     Baichuan-Omni~\cite{baichuan-omni} &  58.2 & - &  60.9 &  - &  - \\
     MiniCPM-o-2.6~\cite{yao2024minicpm} & 63.0$^*$ & 65.3$^*$ & 58.1$^*$ &  43.5$^*$ &  36.8$^*$ \\
     \rowcolor{Gray} \textbf{\method-9B} &  60.4  & 65.0 &  66.3 &  59.8 &  87.8 \\
    \midrule
    VideoLLaMA2-72B~\cite{cheng2024videollama2} & 61.4 & 63.1 & 62.0 & - & - \\
    LLaVA-OneVision-72B~\cite{li2024llavaonevision} &  66.2 &  69.5 &  59.4 &  - &  - \\
    Qwen2-VL-72B~\cite{qwen2-vl_2024} &  71.2 &  77.8 &  \textbf{73.6} &  - &  - \\
    InternVL2-Llama3-76B~\cite{chen2024far} &  64.7  & 67.8 &  69.6 &  - &  - \\
    \rowcolor{Gray} \textbf{\method-72B} &  65.2  & 67.7 &  69.6 &  \textbf{63.5} &  \textbf{89.9} \\
    \midrule
    GPT-4v~\cite{GPT4VisionSystemCard} & 59.9 & 63.3 & 43.7 & 39.2 & 57.78 \\
    GPT-4o-20240513~\cite{openai2024gpt4ocard} & 71.9 & 77.2 & - & - & - \\
    Gemini-1.5-Pro~\cite{geminiteam2024gemini15unlockingmultimodal} & \textbf{75.0} & \textbf{81.3} & - & - & - \\
    \bottomrule
\end{tabular}
\end{table}


In this section, we present a comprehensive evaluation of our \method model, comprising both quantitative and qualitative analyses of its performance. Furthermore, we conduct ablation studies to analyze the contributions of several key design components to the performance of our \method model, providing insights into their distinct impacts.

% In this section, we first evaluate the model’s performance on a variety of mainstream benchmarks, demonstrating the advantages of \method.
% Then, a series of qualitative results are presented to show the model’s specific capabilities, including multimodal understanding and free-form image generation.
% Finally, we conduct an ablation study to analyze several key components in \method.

\subsection{Quantitative Results}\label{subsec:exp_quantitative_results}

\subsubsection{Image-Text Understanding}
To evaluate the effectiveness of our \method in image-text understanding, we benchmark it against state-of-the-art MLLMs on the OpenCompass~\cite{2023opencompass} multimodal leaderboard, a widely recognized platform for multimodal evaluation. This leaderboard contains 8 different multimodal benchmarks, including complex VQA (MMBench~\cite{liu2025mmbench}, MMStar~\cite{chen2024we}, MMMU~\cite{yue2023mmmu}, AI2D~\cite{kembhavi2016diagram}, and MMVet~\cite{yu2024mm}), multimodal reasoning (MathVista~\cite{lu2024mathvista}), hallucination evaluation (Hallusionbench~\cite{Guan_2024_hallusionbench}), and OCR (OCRBench~\cite{Liu_2024}).
\cref{tab:exp_it_oc} shows the overall results. Our \method-72B model achieves top-tier performance on most benchmarks, surpassing closed-source models like GPT-4o and Gemini-1.5-Pro. Furthermore, our \method-9B model exhibits competitive performance among models of similar size, showcasing its robust capabilities in image-text understanding tasks.

% In this section, we compare our \method with leading MLLMs on the mainstream OpenCompass~\cite{2023opencompass} multimodal leaderboard to demonstrate its advancement on image-text understanding.
% This leaderboard contains 8 different multimodal benchmarks, including complex VQA (MMBench~\cite{liu2025mmbench}, MMStar~\cite{chen2024we}, MMMU~\cite{yue2023mmmu}, AI2D~\cite{kembhavi2016diagram}, and MMVet~\cite{yu2024mm}), multimodal reasoning (MathVista~\cite{lu2024mathvista}), hallucination evaluation (Hallusionbench~\cite{Guan_2024_hallusionbench}), and OCR (OCRBench~\cite{Liu_2024}).
% \cref{tab:exp_it_oc} shows the overall results.
% \method exhibits competitive performance compared with other MLLMs.
% Our \method-71B model achieves top-tier performance on most benchmarks.
% It outperforms closed-source models such as GPT-4o and Gemini-1.5-Pro.
% The \method-9B model also achieves competitive performance among other vision-language specific MLLM models smaller than 20B.
% Notably, it achieves excellent performance on MathVista, AI2D, and MMVet, demonstrating its comprehensive ability on multimodal reasoning and complex VQA.




\subsubsection{Video \& Interleaved Image-Text Understanding}

We evaluate our model's video and interleaved image-text understanding abilities on three mainstream benchmarks.

\textbf{Video-MME}~\cite{fu2024video}: Video-MME is a benchmark designed to evaluate MLLMs in full-spectrum video analysis. It encompasses a wide variety of video types across multiple domains and durations, featuring multimodal inputs such as video, subtitles, and audio. For this benchmark, testing is conducted with under 96 frames, and results are reported for both "with subtitles" and "without subtitles" settings.

\textbf{MVBench}~\cite{li2024mvbench}: MVBench serves as a video understanding benchmark aimed at thoroughly evaluating the temporal awareness of MLLMs in an open-world context. It includes 20 challenging video tasks that range from perception to cognition, which cannot be adequately addressed using a single frame. Testing for this benchmark utilizes dynamic sampling frames.

\textbf{LLaVA-Interleave Bench}~\cite{llava-next_2024}: LLaVA-Interleave Bench comprises a comprehensive suite of multi-image benchmarks collected from public datasets or generated via the GPT-4V API. It is created to assess the interleaved multi-image reasoning capabilities of MLLMs, with reported results for both "in-domain" and "out-domain" subsets.

As shown in Table~\ref{tab: video_n_interleave}, \method-9B achieves the second-best results across VideoMME and MVBench (outperformed only by Qwen2-VL-7B but requiring significantly fewer frames). However, the performance gains do not scale up to \method-72B due to limitations in the quantity of instruction-tuned video data. Moreover, both our \method-9B and \method-72B greatly surpass all other baselines in multi-image benchmarks, both in-domain and out-of-domain, highlighting their potential as strong competitors for complex tasks.


\subsubsection{Audio Understanding}

We evaluate our M2-omni model's audio understanding abilities on four mainstream benchmarks.

\textbf{Multilingual LibriSpeech (MLS)}~\cite{MLS_English}: The Multilingual LibriSpeech dataset is an extensive collection of read audiobooks sourced from Librivox, available in eight different languages. We utilize the English test set from this dataset to assess the model's speech comprehension capabilities. The latest version of this corpus comprises approximately 50,000 hours.

\textbf{Librispeech}~\cite{Librispeech}: The Librispeech corpus comprises approximately 1,000 hours of transcribed speech audio data derived from read English audiobooks. The entire dataset is categorized into three training sets (100 hours of clean, 360 hours of clean, and 500 hours of other), two validation sets (clean and other), and two test sets (clean and other). In this study, we assess our model's audio comprehension capabilities using both the clean and other testsets.

\textbf{Aishell1}~\cite{AISHELL1}:  The Aishell1 dataset comprises 178 hours of speech data, recorded by 400 speakers from various accent regions across China. It is organized into three subsets: a training set consisting of 340 speakers, a validation set with 40 speakers, and a test set featuring 20 speakers.

\textbf{AudioCaps}~\cite{AudioCaps}: AudioCaps is a comprehensive dataset featuring audio event descriptions specifically curated for the purpose of audio captioning. The sounds within this collection are derived from the AudioSet dataset. We utilize this dataset to assess the audio captioning capabilities of our \method.
 % To facilitate accurate captioning, annotators were supplied with audio tracks and corresponding categorical hints, with additional video hints provided as necessary.

The results are presented in Table~\ref{tab:exp_audio_understand}, and our \method-9B demonstrates competitive performance in speech recognition and audio captioning tasks. 
Specifically, our \method-9B is comparable to GPT-4o-Realtime~\cite{openai2024gpt4ocard}.
In addition, \method-9B significantly outperforms all other baselines on AudioCaps benchmarks, while achieving the second-best results for the MLS English, Librispeech other, Librispeech-clean and Aishell1 benchmarks.

\begin{table}[]
\centering
\caption{\textbf{Quantitative results on speech recognition and audio captioning.}
 $^*$ indicates results from \cite{yao2024minicpm}.
}
\label{tab:exp_audio_understand}
\setlength{\tabcolsep}{7pt}
\begin{tabular}{l|cccccccccc}
\toprule
Models   & MLS- & Librispeech- & Librispeech- & Aishell1 & AudioCaps \\
                & English & other & clean &  & \\
                & WER$\downarrow$ & WER$\downarrow$ & WER$\downarrow$ & WER$\downarrow$ & CIDER$\uparrow$ \\
\midrule
UIO2-L-1.1B~\cite{lu2023uio2}   & - & - & - & - & 45.7   \\
UIO2-XL-3.2B~\cite{lu2023uio2}  & - & - & - & - & 45.7   \\
UIO2-XXL-6.8B~\cite{lu2023uio2} & - & - & - & - & 48.9  \\
Whisper-large-v2~\cite{Whisper}  & \textbf{6.83} & \textbf{5.16} & 2.87 & - & - \\
Paraformer-cn~\cite{gao2022paraformer} & - & - & - & 2.12 & - \\
VITA-1.5~\cite{VITA_1.5} & - & 7.5 & 3.4 & 2.2 & - \\
Mini-Omini2~\cite{mini_omni2} & - & 9.8 & 4.8 & - & - \\
Freeze-Omini~\cite{Freeze_Omni} & - & 10.5 & 4.1 & 2.8 & - \\
MiniCPM-o-2.6~\cite{yao2024minicpm} & - & - & \textbf{1.7} & \textbf{1.6} & - \\
GPT-4o-Realtime~\cite{openai2024gpt4ocard} & - & - & 2.6$^*$ & 7.3$^*$ & - \\
\rowcolor{Gray} \textbf{\method-9B}   & 7.19 & 5.29 & 2.07 & 1.99 & \textbf{49.2} \\
\bottomrule
\end{tabular}
\end{table}

\begin{table}[t]
\centering
\caption{\textbf{Quantitative results on language benchmarks.} $^*$ indicates officially released checkpoints evaluated using the tools provided by OpenCompass~\cite{2023opencompass}.
}
\label{tab:exp_language}
\setlength{\tabcolsep}{5pt}
\begin{tabular}{cccccccc}
\hline
Tasks & MMLU & AGIEVAL & ARC-C & GPQA & MATH & HellaSwag & \begin{tabular}[c]{@{}l@{}}Avg.\\ Accuracy\end{tabular} \\ \hline
LLama3.1-8B & 69.4 & 41.2$^*$ & 83.4 & 30.4 & 51.9 & 75.1$^*$ & 58.6  \\
\rowcolor{Gray} \textbf{\method-9B} & 68.5 & 43.7 & 78.7 & 32.3 & 51.8 & 80.1 & 59.2  \\ \hline
\end{tabular}
\end{table}

\subsubsection{Audio Generation}
In this section, we also evaluated our model on the commonly-used test set: SEED-TTS test-zh. \textbf{SEED-TTS}~\cite{SEED_TTS} serves as an out-of-domain evaluation test set, comprising diverse input texts and reference speeches from various domains. We present the experimental results for \method-9B and the baseline models in Table~\ref{tab:exp_audio_generation}. As shown in Table~\ref{tab:exp_audio_generation}, our model outperforms MiniCPM-o-2.6~\cite{yao2024minicpm} in speech generation capability, achieving significant improvements in both evaluation metrics. However, our \method-9B still lags behind traditional vertical speech generation models, highlighting the need for further research and development to bridge this gap.


\subsubsection{Text-only Performance}
In this section, we assess the performance of our proposed \method-9B model and its initial counterpart, Llama3.1-8B~\cite{llama3_2024}. To evaluate the models' knowledge and examination capabilities, we employ a range of benchmarks, including AGIEVAL~\cite{zhong2023agievalhumancentricbenchmarkevaluating} and MMLU~\cite{hendrycks2021measuringmassivemultitasklanguage}. Furthermore, we utilize a diverse set of benchmarks to evaluate the models' multi-step problem-solving capabilities, including MATH~\cite{hendrycks2021measuringmathematicalproblemsolving} for mathematical derivation, HellaSwag~\cite{zellers2019hellaswagmachinereallyfinish} for commonsense reasoning in real-world contexts, ARC-C~\cite{allenai:arc} for scientific logical chains, and GPQA~\cite{rein2023gpqagraduatelevelgoogleproofqa} for critical analysis in expert-level domains. For all evaluation datasets, we adopt a generation-based assessment approach with greedy decoding.

Our experimental results, presented in \cref{tab:exp_language}, demonstrate that the performance of our proposed \method-9B model outperforms its initial counterpart, Llama3.1-8B across most evaluation datasets,   which is attributed to our multi-stage language preservation strategy and the high-quality instruction tuning data used in our training process.

% In this section, we evaluate the performance of our \method-9B and its initial Llama3.1~\cite{llama3_2024} models. To assess the models' knowledge and examination capabilities, we utilize the AGIEVAL~\cite{zhong2023agievalhumancentricbenchmarkevaluating},  MMLU~\cite{hendrycks2021measuringmassivemultitasklanguage} benchmarks. Additionally, we employ  MATH~\cite{hendrycks2021measuringmathematicalproblemsolving}, HellaSwag~\cite{zellers2019hellaswagmachinereallyfinish}, ARC-C~\cite{allenai:arc} and GPQA~\cite{rein2023gpqagraduatelevelgoogleproofqa} to evaluate the models' multi-step problem-solving ability, including mathematical derivation, commonsense reasoning in real-world contexts, scientific logical chains, and critical analysis in expert-level domains. For all evaluation datasets, we adopt a generation-based assessment approach with greedy decoding. The overall results are in \cref{tab:exp_language}.It can be observed that in most of the evaluation datasets, the performance of our \method-9B and Llama3.1~\cite{llama3_2024} models is comparable, maintaining their linguistic capabilities. Furthermore, in some rankings, our models exhibit superior performance in certain aspects compared to their text-only baseline models. This improvement is attributed to our multi-stage language preservation strategy and the high-quality instruction tuning data used in our training process.

\begin{table}[t]
\centering
\caption{
\textbf{Free-form dialogue generation evaluation results.}
}
% \vspace{3pt}
\setlength{\tabcolsep}{8pt}
\begin{tabular}{c|c|c|c}
\toprule
Model & Relevance & Fluency & Informativeness\\
\midrule
TextBind~\cite{li2023textbind} & 3.85 & 4.30 & 3.25\\
\rowcolor{Gray} \textbf{\method-9B} & 4.60 & 4.80 & 3.80\\
\bottomrule
\end{tabular}
\label{tab-model_freeform_results}
% \vspace{-12pt}
\end{table}




% For a evaluation of open-world multi-turn multimodal instruction following, we collect a test set comprising 50 conversations from realistic scenarios and utilize \method-9B to generate arbitrarily interleaved text and images in proper conversation contexts. For quantitative results, we ask GPT-4o~\cite{openai2024gpt4ocard} to rate each conversation ranging from 0 to 5 considering relevance, fluency and informativeness. We carry out our quantitative results against recent work TextBind~\cite{li2023textbind}. As shown in \cref{tab-model_freeform_results}, \method-9B exhibits overall better understanding and generating ability of multi-turn multimodal conversations. More qualitative cases can be found in \cref{fig-IT-Freeform-Result}.



\begin{table}[t]
  \caption{\textbf{Quantitative results on audio generation.} $^*$ indicates officially released checkpoints evaluated by us.}
  \label{tab:exp_audio_generation}
  \centering
  \setlength{\tabcolsep}{14pt}
  \begin{tabular}{lccccc}
    \toprule
       & \multicolumn{2}{c}{\textbf{SEED test-zh}}\\
    \cmidrule(r){2-3}
    Model & CER(\%)$\downarrow$ & SS$\uparrow$  \\
    \midrule

     Human & 1.26 &0.755 \\
     Vocoder Resyn. & 1.27 & 0.720 \\
     \midrule
     Seed-TTS~\cite{SEED_TTS} & 1.12 & 0.796 \\
     FireRedTTS~\cite{FireRedTTS} & 1.51 &0.635 \\
     MaskGCT~\cite{MaskGCT} & 2.27 & 0.774 \\
     E2-TTS(32 NFE)~\cite{E2_TTS} & 1.97 & 0.730 \\
     F5-TTS(32 NFE)~\cite{F5_TTS} & 1.56 & 0.741 \\
     CosyVoice~\cite{CosyVoice} &3.63 &0.723 \\
     CosyVoice2~\cite{CosyVoice2} &1.45 &0.748 \\
     CosyVoice2-S~\cite{CosyVoice2} &1.45 &0.753 \\
     CosyVoice2-S~\cite{CosyVoice2} &1.45 &0.753 \\
     \midrule
     MiniCPM-o-2.6~\cite{yao2024minicpm} &8.03$^*$ &0.474$^*$ \\
     \rowcolor{Gray} \textbf{\method-9B} &  6.36  & 0.604 \\
    \bottomrule
\end{tabular}
\end{table}


\subsubsection{User Experience Evaluation}\label{sec:human_evaluation}
\textbf{Evaluation Metric}:
Current benchmarks such as MMBench~\cite{liu2025mmbench}, MMStar~\cite{chen2024we}, and MMMU~\cite{yue2023mmmu} primarily focus on assessment through judgment-style questions. However, this assessment does not align with the users' actual interactive experience with MLLMs. To address this limitation, drawing inspiration from SuperclueV~\cite{supercluev}, we develop evaluation criteria specifically for assessing the models' performance on user experience, which contains four key dimensions: relevance, fluency, informativeness, and format rationality. \textit{Relevance} assesses the extent to which the model's responses align with both the provided prompts and the multimodal inputs.
\textit{Fluency} evaluates the naturalness, smoothness, clarity, comprehensibility, and anthropomorphic quality of the model's responses.
\textit{Informativeness} measures the extent to which the model's responses provide relevant information, knowledge, and analytical reasoning, enhancing their utility, detail, depth, and innovation.
\textit{Format rationality} examines the model's ability to adaptively generate appropriately structured and clear formats, for presenting results based on varying prompt types.



% Current benchmarks such as MMBench~\cite{liu2025mmbench}, MMStar~\cite{chen2024we}, and MMMU~\cite{yue2023mmmu} primarily focus on assessment through judgment-style questions. However, this assessment does not align with the users' actual interactive experience with MLLMs. Drawing inspiration from SuperclueV~\cite{supercluev}, we develop evaluation criteria specifically for assessing the models' experience performance, which contains four key dimensions: relevance, fluency, content richness, and format rationality. \textbf{Relevance} assesses the extent to which the model's responses align with both the provided prompts and the multi-modal inputs.
% \textbf{Fluency} evaluates the naturalness, smoothness, clarity, comprehensibility, and anthropomorphic quality of the model's responses.
% \textbf{Content richness} gauges the degree to which the model's responses are enriched with supplementary information, knowledge, and analytical reasoning, enhancing their utility, detail, depth, and innovation.
% \textbf{Format rationality} examines the model's ability to adaptively generate appropriately structured and clear formats for presenting results based on varying prompt types.


\begin{table}[t]
\centering
\caption{
\textbf{Detailed model experience evaluation standards.}
}
% \vspace{3pt}
\setlength{\tabcolsep}{4pt}
\begin{tabular}{c|c}
\toprule
Score & Description\\
\midrule
1 & Totally unsatisfied, totally unacceptable \\
2 & Basically not satisfied, with many obvious problems \\
3 & Generally satisfied, with a few obvious problems \\
4 & Basically satisfied, minor flaws allowed \\
5 & Completely satisfied, almost perfect \\
\bottomrule
\end{tabular}
\label{tab-model_expr_standards}
% \vspace{-12pt}
\end{table}

\textbf{Evaluation Dataset}: We collect chat samples from the actual users' multi-turn interaction dialogues, which cover a variety of tasks, including visual question answering (VQA), conversational interactions, chart interpretation, mathematical problem-solving, optical character recognition (OCR), and other related tasks. GPT-4o~\cite{openai2024gpt4ocard} is instructed to follow the evaluation criteria to generate initial reference answers for these collected samples. To ensure accuracy, human annotators refine the initial responses generated by GPT-4o. This process yields an evaluation dataset with nearly 300 samples, each with a corresponding ground truth.

We utilize GPT-4o to evaluate the model's responses against the ground truth, adhering to the standards outlined in  \cref{tab-model_expr_standards}.  As shown in \cref{tab-user_experience},  our M2-omni model, after undergoing  alignment tuning,  demonstrates an average increase of 5.7\%-23.4\% in user experience performance, which is further validated by human annotations on selected cases. Meanwhile, our model's performance on the OC benchmark across other modalities remains relatively consistent, thereby demonstrating the effectiveness of our unified training strategy, which integrates DPO and instruction tuning in the alignment tuning stage.

% We employ GPT-4o to score the models' responses compared with ground truth according to the standards of \cref{tab-model_expr_standards}.  \cref{tab-user_experience} shows the model after alignment tuning demonstrates an average increase of 5.7\% in performance. This enhancement is corroborated by human annotations on selected cases. Simultaneously, the general capabilities on OC benchmark across other modalities remain nearly the same, with a decrease in average evaluation scores of less than 1\%. This demostrates the effectiveness of our unified training strategy that integrates DPO and
% instruction tuning in alignment tuning stage.


\subsubsection{Free-Form Dialogue Generation}
To evaluate the open-world multi-turn multimodal instruction following capabilities of our model, we create a test set consisting of 50 conversations derived from realistic scenarios. We utilize \method-9B to generate arbitrarily interleaved text and images in proper conversation contexts.
For quantitative results, following our user experience evaluation metric, we employ GPT-4o to rate each conversation on a scale of 0 to 5 across three evaluation dimensions: relevance, fluency, and informativeness.
We carry out our quantitative results against recent work TextBind~\cite{li2023textbind}. As shown in \cref{tab-model_freeform_results}, \method-9B exhibits overall better understanding and generating ability of multi-turn multimodal conversations. More qualitative cases can be found in \cref{fig-IT-Freeform-Result}.





\begin{table}[t]
\centering\footnotesize
\caption{
\textbf{Detailed evaluation on user experience benchmark and OC benchmark. OC is short for the OpenCompass image-text understanding benchmark.}
}
% \vspace{3pt}
\setlength{\tabcolsep}{3pt}
\begin{tabular}{c|c|c|c|c|c|c}
\toprule
Model & Relevance & Fluency & Informativeness & Format Rationality & Expr. Avg($\Delta$\%) & OC Avg($\Delta$)\\
\midrule
\method-9B & 4.556 & 4.036 & 2.742 & 3.573 & 3.726 & -\\
\rowcolor{Gray} \method-9B-Align & 4.893 & 4.735 & 4.118 & 4.644 & 4.598(+23.4\%) & -0.3\\
\method-72B & 4.942 & 4.689 & 3.267 & 4.265 & 4.351 & -\\
\rowcolor{Gray} \method-72B-Align & 4.946 & 4.875 & 3.961 & 4.615 & 4.598(+5.7\%) & -0.2\\
InternVL2-26B~\cite{internvl_2024} & 4.886 & 4.76 & 4.15 & 4.52 & 4.577 & -\\
GPT-4o~\cite{openai2024gpt4ocard} & 5 & 4.878 & 3.854 & 4.831 & 4.64 & -\\
\bottomrule
\end{tabular}
\label{tab-user_experience}
% \vspace{-12pt}
\end{table}



\subsection{Qualitative Results}\label{subsec:exp_qualitative_results}

In this section, we qualitatively assess the capabilities of our \method, presenting examples of each modality and different tasks.

We show multimodal understanding abilities of our \method in \cref{fig-exp_case_all}. \method demonstrates promising capabilities in processing cross-modal problems, encompassing image understanding, video understanding, interleaved image-text understanding, and image-audio understanding. More examples can be found in the appendix, provided in \cref{subsec:appendix_cases}.

\cref{fig-IT-Freeform-Result} illustrates the model's ability to generate free-form dialogue, where our \method can create images based on the conversation context without explicit user input, useful for explaining ideas to users.




\begin{figure}[t]
    \centering
    \includegraphics[width=0.9\linewidth]{figures/case_exp.pdf}
    \caption{
    \textbf{Cases for multimodal understanding.}
    \method shows great potential to solve various multimodal problems.
    }
    \label{fig-exp_case_all}
\end{figure}




\begin{figure}[t]
    \centering
    \includegraphics[width=0.9\linewidth]{figures/free_form_gen.pdf}
    \caption{
    \textbf{Cases for Free-Form Dialogue Generation.}
    }
    \label{fig-IT-Freeform-Result}
\end{figure}


\subsection{Ablation Study}\label{subsec:exp_ablation}

\begin{table}[t]
\centering
\caption{\textbf{Ablation studies on step balancing strategy.} The loss weight setting [1,1,1] corresponds to the uniform weighting of the loss functions for image-text pairs, interleaved image-text, and video datasets.  * and \# represent the loss weight settings. * is obtained through experimental trials and parameter tuning. \# is obtained by normalizing the loss weights using the inverse of the loss at convergence, as described in Section \cref{subsubsec-Step Balancing Strategy}. We evaluate the few-shot performance on VQA tasks and the zero-shot performance on the captioning task of our pre-trained model.}
\label{tab:ablation_step_balance_pretrain}
\setlength{\tabcolsep}{4pt}
\begin{tabular}{c|c|ccc}
\toprule
\multicolumn{1}{l|}{Data Sample Balance} & Loss Weight Balance & \multicolumn{1}{l}{OK-VQA(4-shot)} & \multicolumn{1}{l}{VQAv2(4-shot)} & \multicolumn{1}{l}{Flickr30k(0-shot)} \\ \hline
Random Sample                        & {[}1,1,1{]}          & 40.5                             & 54.3                             & 87.0                                 \\
Round-robin                          & {[}1,1,1{]}          & 41.6                             & 54.4                             & 88.1                                 \\
Accumulation                         & {[}1,1,1{]}          & 41.7                             & 54.6                             & 88.2                                 \\
Accumulation                         & ${[}0.2,1.0,0.03{]}^{*}$   & 39.7                             & 52.5                             & 87.1                                 \\
Accumulation                         & ${[}0.45,0.36,1.09{]}^{\#}$ & \textbf{42.1}                             & \textbf{55.4}                             & \textbf{88.2}                                 \\
\bottomrule
\end{tabular}
\end{table}


In this section, we conduct ablation studies to investigate the effectiveness of our step balance strategy and dynamic adaptive balance strategy in our M2-omni model. These experiments aim to provide insights into the impact of these key components on our M2-omni’s performance.

\subsubsection{Step Balance Strategy}\label{subsubsec:step_balance_ablaton}

As described in \cref{subsubsec-Step Balancing Strategy} ,  we investigate the impact of various data sample balancing strategies and loss weight balancing schemes on the multimodal joint training stage of pre-training. We evaluate the performance of candidate strategies on two VQA benchmarks, OK-VQA~\cite{marino2019ok} and VQAv2~\cite{goyal2017making}, and assess its image captioning performance using the Flickr30k~\cite{young2014image} benchmark.

For pretrained models lacking in instruction following ability, to assess the effectiveness of our approach, we evaluate the performance of these models on VQA tasks using a few-shot approach and on image caption tasks using a zero-shot approach. \cref{tab:ablation_step_balance_pretrain} presents the results of our M2-omni pretrained models, which demonstrate the effectiveness of our step balance strategy.

% Besides, three task weighting manner are compared: [1,1,1], which means all data shares the same optimization step size; [0.2,1.0,0.03], which is consistent with that proposed in \cite{alayrac2022flamingo}; [0.45,0.36,1.09], the inverse of the loss at convergence state, as \cref{subsubsec-Step Balancing Strategy} described. Note that the three values in the ratio correspond to image-text pairs, interleaved image-text and video datasets.

%  We directly evaluate the pre-trained model's performance on VQA tasks using a few-shot approach and on image caption tasks using a zero-shot approach. For VQA tasks, we use two benchmarks: OK-VQA~\cite{marino2019ok} and VQAv2~\cite{goyal2017making}, while for image captioning, we use the Flickr30k~\cite{young2014image} benchmark. \cref{tab:ablation_step_balance_pretrain} shows the results of training models on the combined datasets using three different merging regimes. It can be observed that the accumulation strategies and setting the task weights to the inverse of the loss achieve the best performance.







\begin{table}[t]
\centering
\caption{
\textbf{Ablation results of the dynamic adaptive balance strategy}. Results for unimodal baselines are derived from the following single-modal models: \textsuperscript{$\dagger$} Image-Text Model, \textsuperscript{$\ddagger$} Video-Text Model, and \textsuperscript{
$\mathsection$} Audio-Text Model. The best result for each benchmark is \textbf{bolded}, while the best result for each model across all epochs is \underline{underlined}.
}
% \vspace{3pt}
\setlength{\tabcolsep}{3pt}
\begin{tabular}{l|l|ccccc|cc|cc}
\toprule
Models &  & MM- & OK- & VQAv2 & Text- & GQA & MSVD- & MSRVTT & Audio & MLS- \\
& & Bench & VQA &&VQA&& QA & QA & Caps & English($\downarrow$) \\
\midrule
\multirow{3}{*}{\makecell[l]{Single-modal\\Baselines}} & ep1 & 68.0\textsuperscript{$\dagger$} & 56.4\textsuperscript{$\dagger$} & 74.8\textsuperscript{$\dagger$} & \underline{70.4}\textsuperscript{$\dagger$} & 58.4\textsuperscript{$\dagger$} & 72.3\textsuperscript{$\ddagger$} & 59.3\textsuperscript{$\ddagger$} & 29.0\textsuperscript{$\mathsection$} & 11.4\textsuperscript{$\mathsection$} \\
& ep2 & \underline{\textbf{77.8}}\textsuperscript{$\dagger$} & \underline{59.8}\textsuperscript{$\dagger$} & \underline{76.9}\textsuperscript{$\dagger$} & 69.8\textsuperscript{$\dagger$} & 60.6\textsuperscript{$\dagger$} & \underline{\textbf{76.5}}\textsuperscript{$\ddagger$} & \underline{60.1}\textsuperscript{$\ddagger$} & \underline{39.9}\textsuperscript{$\mathsection$} & 9.33\textsuperscript{$\mathsection$} \\
& ep3 & 77.3\textsuperscript{$\dagger$} & 58.0\textsuperscript{$\dagger$} & 76.8\textsuperscript{$\dagger$} & 69.1\textsuperscript{$\dagger$} & \underline{60.8}\textsuperscript{$\dagger$} & 74.4\textsuperscript{$\ddagger$} & 58.6\textsuperscript{$\ddagger$} & 39.5\textsuperscript{$\mathsection$} & \underline{8.96}\textsuperscript{$\mathsection$} \\
\midrule
\multirow{3}{*}{\makecell[l]{Mixture \\w/o MM-Bal.}}
& ep1 & 70.5 & 55.9 & 75.7 & 70.2 & 57.7 & \underline{75.1} & \underline{59.6} & 27.5 & 12.1 \\
& ep2 & \underline{75.8} & \underline{58.8} & \underline{77.0} & \underline{70.5} & \underline{\textbf{61.1}} & 73.4 & 58.5 & 33.5 & 9.45 \\
& ep3 & 75.6 & 58.4 & 76.5 & 69.5 & 60.1 & 70.2 & 56.9 & \underline{39.6} & \underline{8.98} \\
\midrule
\multirow{3}{*}{\makecell[l]{Mixture \\w/ MM-Bal.}}
& ep1 & 74.7 & 59.6 & 76.0 & 71.2 & 59.0 & 73.1 & 58.7 & 35.5 & 9.27 \\
& ep2 & \underline{\textbf{77.8}} & \underline{\textbf{61.7}} & \underline{\textbf{77.2}} & \underline{\textbf{71.8}} & 60.5 & \underline{74.8} & 58.5 & 41.2 & 8.31 \\
& ep3 & 77.1 & 60.5 & 77.0 & 69.8 & \underline{60.7} & 74.6 & \underline{\textbf{60.2}} & \underline{\textbf{44.1}} & \underline{\textbf{8.04}} \\

\bottomrule
\end{tabular}
\label{tab-multi_task_balanced_ablation}
% \vspace{-12pt}
\end{table}

\subsubsection{Dynamic Adaptive Balance Strategy}

We conducted a evaluation of our dynamic adaptive balance strategy across text-image, video, and audio modalities using constrained datasets. The evaluation was conducted on benchmark datasets specific to each modality: for text-image tasks, MMbench~\cite{liu2025mmbench}, OK-VQA~\cite{marino2019ok}, VQAv2~\cite{goyal2017making}, TextVQA~\cite{singh2019towards}, and GQA~\cite{hudson2019gqa} were employed; for video, MSVD-QA~\cite{xu2017video} and MSRVTT-QA~\cite{xu2017video} benchmarks were utilized; and for audio, we assessed performance on the AudioCaps~\cite{kim2019audiocaps} (AAC) and MLS~\cite{Pratap2020MLSAL}-English (ASR) tasks. The experimental outcomes are detailed in Table~\ref{tab-multi_task_balanced_ablation}.

In contrast to actual training pipeline, our evaluation involved instruction tuning starting from pre-trained models. Specifically, for each modality, we initially trained single-modality baseline models (the 'Sinle-modal Baselines' in Table~\ref{tab-multi_task_balanced_ablation}) individually over three epochs to establish the maximum achievable performance per modality. The results indicate that optimal performance was predominantly observed by the second epoch. However, the ASR task, due to its more complex patterns, had not fully converged even by the third epoch. Subsequently, we combined data from all three modalities to train a unified model (the 'Mixture w/o MM-Bal.' in Table~\ref{tab-multi_task_balanced_ablation}). Under this multimodal training regimen, the image-text modality reached its optimal performance at the second epoch, while the video modality achieved peak performance as early as the first epoch and with performance consistently decreasing in subsequent epochs. In contrast, the audio modality demonstrated continuous improvement, attaining its best performance by the third epoch. These observations underscore the imbalance in training progress among different modalities when engaged in multimodal training.

To address this imbalance, we introduced the dynamic adaptive balance strategy within our M2-omni training framework. This strategy dynamically adjusts the loss weights for each modality based on their respective training progress. In the context of this evaluation, it accelerates the training of the audio modality while appropriately reducing the learning weights for the image-text and video modalities to prevent overfitting. The evaluation results for this balanced training approach are denoted as 'Mixture w/ MM-Bal.' in Table~\ref{tab-multi_task_balanced_ablation}. The results demonstrate that, although some degree of imbalance among modalities persists, the balanced training strategy significantly alleviates the issues observed with simple mixed training: optimal performances across benchmarks are now concentrated around the second and third epochs, and performance across all modalities has been markedly enhanced. Moreover, under the balanced training strategy, the model achieved single-modality optimal performance in 7 out of 9 benchmarks. The best-performing model (at epoch 2) surpassed the optimal performance of each single-modality baseline in 6 out of 9 benchmarks (MMBench, OK-VQA, VQAv2, TextVQA, AudioCaps, MLS-English). Additionally, for the audio modality, the model at epoch 3 outperformed the single-modality baselines in 5 out of 9 benchmarks (OK-VQA, VQAv2, MSRVTT-QA, AudioCaps, MLS-English), with significant improvements in audio performance. These experimental results highlight the effectiveness of our dynamic adaptive balance strategy.


\section{Conclusion}
\begin{figure*}[h]
    \centering
    \includegraphics[width=14cm]{figures/visualized_kitti5.jpg}
    \caption[Qualitative Results on KITTI \textit{val.} set]{\textbf{Qualitative results on the KITTI \textit{val} set for the car class.} The proposed method (green) and ground truth (red).
    } \label{fig:KITTI visualized}
\end{figure*}

\begin{figure*}[t]
    \centering
     \includegraphics[width=14cm]{figures/custom_result_monodetr_monoground3.jpg}
    \caption{\textbf{Qualitative results on the custom dataset.} Comparison of detection results between the proposed model (blue), the state-of-the-art models (green), and ground truth (red) in ego-view (left) and bird's-eye view (right); MonoDETR (left) and MonoGround (right).}
    \label{fig:custom_result_visualized}
\end{figure*}

\section{Conclusion}
\label{sec:conclusion}
This paper presents a novel approach to monocular 3D object detection by integrating a Vision Foundation Model as the backbone with the DETR architecture, enabling enhanced depth estimation and feature extraction within a single-stage, real-time framework. By incorporating a Hierarchical Feature Fusion Block for multi-scale detection and 6D Dynamic Anchor Boxes for iterative bounding box refinement, the proposed model achieves improved performance without relying on additional data sources, such as LiDAR. Future work will focus on extending the model's capabilities to detect 3D bounding boxes while accounting for rolling and pitching angles.

\section*{Limitations}


We acknowledge two limitations in this work.
Firstly, {\ouralg} is a rehearsal-based method. The outer loop relies on memory data to retrieve and dynamically update the parameter importance distributions of historical tasks.
This reliance may limit its applicability in scenarios where privacy concerns or data retention restrictions are present. Generative replay techniques could provide a solution by simulating the distribution of previous tasks without direct access to historical data.


Secondly, the time complexity of {\ouralg} increases with larger backbone models, primarily due to element-wise operations and multi-round fusion. For element-wise operations, global merging strategies have proven suboptimal, highlighting the need for balanced fusion granularity. Future work could explore focusing on specific important layers or adopting modular approaches to enhance efficiency. 
For multi-round fusion, we could further investigate how fusion frequency impacts performance and analyze the semantic knowledge learned at different stages of the training process. This could help minimize unnecessary iterations, while still preserving the benefits of iterative integration.




% Secondly, the parameter-wise operation encounters challenges in time complexity. As the size of the base model increases, both localization and fusion become more time-consuming. 
% While PEFT techniques have been employed to mitigate this issue, there is still room for optimization. Future work could explore focusing on specific important layers or adopting modular approaches to streamline these processes and reduce computational overhead.



%Second, the current parameter-wise merging strategy presents two challenges. The first challenge is time complexity: as the base model size increases, both parameter identification and merging become more time-consuming. While we have leveraged PEFT techniques to reduce time complexity, further optimization is possible, such as focusing on key layers or model components. The second challenge is that parameter-wise operations overlook inter-parameter dependencies. As seen in the model pruning literature, parameters with low individual importance may still play crucial roles in larger model components. Hence, developing more effective modular operations remains an area for future exploration.


%Secondly, the current parameter-wise merging strategy poses two challenges. The first is time complexity: as the size of the base model increases, both parameter identification and merging require more time. While we have leveraged PEFT techniques to reduce time complexity, there is still room for optimization, such as by focusing on specific important layers or key model structures. The second challenge is that parameter-wise operations do not account for dependencies between parameters. As seen in model pruning community, individual parameters may have low importance, but from a broader perspective, they could play a crucial role in specific model components. Therefore, designing more effective modular operations remains an area for future exploration.




%\section*{Ethics Statement}


%\section*{Acknowledgements}


% Entries for the entire Anthology, followed by custom entries
\bibliography{acl2023}
\bibliographystyle{acl_natbib}

\appendix
\label{sec:appendix}

%\clearpage

\appendix
\section*{Appendix}
\begin{table*}[h!]
\caption{The basic information of grid-based spatio-temporal data.}
\label{tbl:append_data}
\begin{threeparttable}
\resizebox{1.9\columnwidth}{!}{
\begin{tabular}{cccccccc}
\toprule
Dataset & City & Type & Temporal Period & Spatial partition & Interval & Mean & Std \\
\hline
TaxiBJ & Beijing & Taxi flow&  2014/03/01 - 2014/06/30 & $32 \times 32$ & Half an hour & 111.5 & 139.3 \\
BikeDC & Washington, D.C. & Bike flow&  2010/09/20 - 2010/10/20 & $20 \times 20$ & Half an hour & 0.924 & 4.88 \\



CellularSH & Shanghai & Cellular traffic &  2014/08/01 - 2014/08/21 & $32\times28$ & One hour & 0.175 & 0.212 \\
CellularNJ & Nanjing & Cellular traffic &  2021/02/02 - 2021/02/22 & $20\times28$ & One hour & 0.842 & 1.30 \\
CrowdBJ & Beijing & Crowd flow &  2018/01/01 - 2018/01/31 & $1010$ & One hour & 7.07 & 11.1 \\
CrowdBM & Baltimore & Crowd flow &  2019/01/01 - 2019/05/31 & $403$ & One hour & 14.4 & 29.3 \\
Los-Speed & Los Angeles & Traffic speed&  2012/03/01 - 2012/03/07 & $207$ & Five minutes & 59.0 & 12.5 \\

\bottomrule
\end{tabular}}
\end{threeparttable}
\end{table*}

% \begin{table*}[t!]
% \caption{The basic information of Graph-based spatio-temporal data.}
% \label{tbl:append_data_graph}
% \begin{threeparttable}
% \resizebox{1.8\columnwidth}{!}{
% \begin{tabular}{ccccccccc}
% \toprule
% Dataset & City & Type & Temporal Period & Interval & \#Nodes & \#Edges & Mean & Std \\
% \hline

% TrafficBJ & Beijing & Traffic speed & 2022/03/05 - 2022/04/05 & 15min& 13675& 24444& 6.837&  3.412\\
% TrafficSH & Shanghai & Traffic speed & 2022/01/27 - 2022/02/27 & 15min & 21099& 39065& 7.815&  4.044\\
% TrafficNJ & Nanjing & Traffic speed  & 2022/03/05 - 2022/04/05 & 15min & 13419& 25100& 6.699&  4.253\\

% \bottomrule
% \end{tabular}}
% \end{threeparttable}
% \end{table*}
\begin{table*}[h!]
\caption{Short-term prediction results on two additional datasets in terms of both deterministic and probabilistic metrics. \textbf{Bold} indicates the best performance, while \underline{underlining} denotes the second-best.}
\label{tbl:short1-app}
\begin{threeparttable}
% \resizebox{2.0\columnwidth}

\resizebox{1.5\columnwidth}{!}{
\begin{tabular}{ccccccccccc}
\toprule
\multirow{2}{*}{\textbf{Model}}
& \multicolumn{5}{c}{\textbf{CellularNJ}} & \multicolumn{5}{c}{\textbf{CrowdBM}}   \\
\cmidrule(lr){2-6} \cmidrule(lr){7-11} 
 &\textbf{MAE} & \textbf{RMSE}  &\textbf{CRPS} & \textbf{QICE} & \textbf{IS} & 
\textbf{MAE} & \textbf{RMSE} & \textbf{CRPS} & \textbf{QICE} & \textbf{IS} \\


\midrule
D3VAE& 0.580 & 1.135   &	0.565&	0.096&6.03	&	11.0&24.7	&	0.593& 0.110	&136.4\\


DiffSTG& 0.317& 0.649&	0.291&	0.071&3.11	&	8.88&21.3	&0.453&0.047	&68.5\\

TimeGrad& 0.340&  0.357  &	0.432&	0.162&5.87	&10.1	& \underline{12.4}	&\textbf{0.240}&\underline{0.085}	&\underline{46.9}\\


CSDI&0.129 & 0.237   &	\underline{0.111}&	 \underline{0.039}&	\underline{0.80}&	7.31& 19.3&	0.390& 0.054&61.1\\

NPDiff& \underline{0.123}&  \underline{0.175}  &0.128	&	0.133&2.22	&	\underline{5.42}& 13.7	&0.331&0.119	&91.2\\

DyDiffusion&0.222&   0.357 &0.196	&0.080	&	1.80&-	&-	&-	&-&-\\




\cmidrule(lr){1-1} \cmidrule(lr){2-6} \cmidrule(lr){7-11}
\textbf{CoST}&\textbf{0.102} &\textbf{0.172}    &\textbf{0.090}	&\textbf{0.037}	&\textbf{0.682}	&\textbf{5.04}	&\textbf{12.1}	&\underline{0.256}	&\textbf{0.027}& \textbf{37.8}\\
%\textbf{Reduction}& &    &	&	&	&	&	&	&\\
\bottomrule
\end{tabular}}
\end{threeparttable}
\end{table*}


% \input{Tables/short-term2-append}



\end{document}
