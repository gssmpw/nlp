% This must be in the first 5 lines to tell arXiv to use pdfLaTeX, which is strongly recommended.
\pdfoutput=1
% In particular, the hyperref package requires pdfLaTeX in order to break URLs across lines.

\documentclass[11pt]{article}

% Change "review" to "final" to generate the final (sometimes called camera-ready) version.
% Change to "preprint" to generate a non-anonymous version with page numbers.
% \usepackage[review]{acl}
\usepackage{acl}

% Standard package includes
\usepackage{times}
\usepackage{latexsym}

% For proper rendering and hyphenation of words containing Latin characters (including in bib files)
\usepackage[T1]{fontenc}
% For Vietnamese characters
% \usepackage[T5]{fontenc}
% See https://www.latex-project.org/help/documentation/encguide.pdf for other character sets

% This assumes your files are encoded as UTF8
\usepackage[utf8]{inputenc}

% This is not strictly necessary, and may be commented out,
% but it will improve the layout of the manuscript,
% and will typically save some space.
\usepackage{microtype}

% This is also not strictly necessary, and may be commented out.
% However, it will improve the aesthetics of text in
% the typewriter font.
\usepackage{inconsolata}

%Including images in your LaTeX document requires adding
%additional package(s)
\usepackage{graphicx}
\usepackage{amsmath}
\usepackage{times}
\usepackage{framed}
\usepackage{ragged2e}
\usepackage{bm}
\usepackage{soul}
\usepackage{latexsym}
\usepackage{subfigure}
\usepackage{amsthm}
\usepackage{graphicx}
\usepackage{float}
\usepackage{longtable}
\usepackage{booktabs} 
\usepackage{multirow}
\usepackage{multicol}
%\usepackage{hhline}
\usepackage{amssymb}
\usepackage{dashbox}%
\usepackage{multirow}
\usepackage{textcomp}
\usepackage{tcolorbox}
\usepackage{bbding}
\usepackage{bbm}
\usepackage[ruled, vlined, linesnumbered]{algorithm2e}
\usepackage{mathtools}
%\usepackage{algorithm}
%\usepackage{algorithmic}
%\usepackage{algorithm2e}
\usepackage{adjustbox}
\usepackage[ruled,vlined]{algorithm2e}
\usepackage{color, soul}
\usepackage{pifont}
% \definecolor{decentgrey}{RGB}{232,232,232}
% \definecolor{mypattern}{RGB}{218,239,248}
\usepackage{array}
\newcolumntype{P}[1]{>{\centering\arraybackslash}p{#1}}
\newcolumntype{M}[1]{>{\centering\arraybackslash}m{#1}}
\usepackage{cleveref}
\crefname{section}{§}{§§}
\Crefname{section}{§}{§§}
\crefname{figure}{Figure}{Figure}
\Crefname{figure}{Figure}{Figure}
\crefname{table}{Table}{Table}
\Crefname{table}{Table}{Table}
\usepackage{tabularx}
\newcommand\ourmethod{\textsc{RewardAgent}\xspace}
\newcommand\ourmethodmini{\textsc{RewardAgent}\textsubscript{\textsc{mini}}\xspace}
\newcommand\ourmethodllama{\textsc{RewardAgent}\textsubscript{\textsc{Llama}}\xspace}
\newcommand\ourdataset{\textsc{IFBench}\xspace}


\newcommand{\Mark}[1]{{\color{blue} #1}}
\usepackage{colortbl}


% If the title and author information does not fit in the area allocated, uncomment the following
%
%\setlength\titlebox{<dim>}
%
% and set <dim> to something 5cm or larger.

\title{Agentic Reward Modeling: Integrating Human Preferences with \\ Verifiable Correctness Signals for Reliable Reward Systems}

% Author information can be set in various styles:
% For several authors from the same institution:
% \author{Author 1 \and ... \and Author n \\
%         Address line \\ ... \\ Address line}
% if the names do not fit well on one line use
%         Author 1 \\ {\bf Author 2} \\ ... \\ {\bf Author n} \\
% For authors from different institutions:
% \author{Author 1 \\ Address line \\  ... \\ Address line
%         \And  ... \And
%         Author n \\ Address line \\ ... \\ Address line}
% To start a separate ``row'' of authors use \AND, as in
% \author{Author 1 \\ Address line \\  ... \\ Address line
%         \AND
%         Author 2 \\ Address line \\ ... \\ Address line \And
%         Author 3 \\ Address line \\ ... \\ Address line}

% \author{First Author \\
%   Affiliation / Address line 1 \\
%   Affiliation / Address line 2 \\
%   Affiliation / Address line 3 \\
%   \texttt{email@domain} \\\And
%   Second Author \\
%   Affiliation / Address line 1 \\
%   Affiliation / Address line 2 \\
%   Affiliation / Address line 3 \\
%   \texttt{email@domain} \\}

\author{Hao Peng\thanks{\quad Equal contribution.}, Yunjia Qi$^{*}$, Xiaozhi Wang, Zijun Yao, Bin Xu, Lei Hou, Juanzi Li\\
Department of Computer Science and Technology, Tsinghua University \\
\texttt{\{peng-h24\}@mails.tsinghua.edu.cn}}


%\author{
%  \textbf{First Author\textsuperscript{1}},
%  \textbf{Second Author\textsuperscript{1,2}},
%  \textbf{Third T. Author\textsuperscript{1}},
%  \textbf{Fourth Author\textsuperscript{1}},
%\\
%  \textbf{Fifth Author\textsuperscript{1,2}},
%  \textbf{Sixth Author\textsuperscript{1}},
%  \textbf{Seventh Author\textsuperscript{1}},
%  \textbf{Eighth Author \textsuperscript{1,2,3,4}},
%\\
%  \textbf{Ninth Author\textsuperscript{1}},
%  \textbf{Tenth Author\textsuperscript{1}},
%  \textbf{Eleventh E. Author\textsuperscript{1,2,3,4,5}},
%  \textbf{Twelfth Author\textsuperscript{1}},
%\\
%  \textbf{Thirteenth Author\textsuperscript{3}},
%  \textbf{Fourteenth F. Author\textsuperscript{2,4}},
%  \textbf{Fifteenth Author\textsuperscript{1}},
%  \textbf{Sixteenth Author\textsuperscript{1}},
%\\
%  \textbf{Seventeenth S. Author\textsuperscript{4,5}},
%  \textbf{Eighteenth Author\textsuperscript{3,4}},
%  \textbf{Nineteenth N. Author\textsuperscript{2,5}},
%  \textbf{Twentieth Author\textsuperscript{1}}
%\\
%\\
%  \textsuperscript{1}Affiliation 1,
%  \textsuperscript{2}Affiliation 2,
%  \textsuperscript{3}Affiliation 3,
%  \textsuperscript{4}Affiliation 4,
%  \textsuperscript{5}Affiliation 5
%\\
%  \small{
%    \textbf{Correspondence:} \href{mailto:email@domain}{email@domain}
%  }
%}

\begin{document}
\maketitle
\begin{abstract}

Reward models (RMs) are crucial for the training and inference-time scaling up of large language models (LLMs). 
However, existing reward models primarily focus on human preferences, neglecting verifiable correctness signals which have shown strong potential in training LLMs. In this paper, we propose \textit{agentic reward modeling},  a reward system that combines reward models with verifiable correctness signals from different aspects to provide reliable rewards. We empirically implement a reward agent, named \ourmethod, that combines human preference rewards with two verifiable signals: factuality and instruction following, to provide more reliable rewards. We conduct comprehensive experiments on existing reward model benchmarks and inference time best-of-n searches on real-world downstream tasks. \ourmethod significantly outperforms vanilla reward models, demonstrating its effectiveness.
We further construct training preference pairs using \ourmethod and train an LLM with the DPO objective, achieving superior performance on various NLP benchmarks compared to conventional reward models. Our codes are publicly released to facilitate further research\footnote{\url{https://github.com/THU-KEG/Agentic-Reward-Modeling}}.
\end{abstract}

\section{Introduction}
% \textsc{Sol-Ver}


% This seems to be a better flow: synthetic data by the same size model is not good -> data has a lot of fales positive (fales positive is high cite paper: alphaCode) ->  data need to be filtered/checked -> we build unit test verifier for filtering. -> iteratively improve both

Large language models (LLMs) have demonstrated impressive ability in code generation, significantly enhancing the programming efficiency and productivity of human developers~\cite{li2022competition,roziere2023code,codealpaca}.
The ability to code is largely driven by high-quality online resources where coding problems, human-written solutions, and corresponding unit tests are freely available. However, as these free resources being depleted, the momentum of LLMs' improvement diminishes.
% Building on these successes, researchers are exploring additional methods to optimize LLM performance across more diverse coding scenarios.

To address the scarcity of supervised data for code generation, recent studies have adopted synthetic data generation techniques such as {\sc Self-Instruct}~\cite{wang2023self} to augmenting LLM training sets. 
% leveraging the robust code generation abilities of teacher models, enabling the distillation of synthetic data that further enhances LLMs.
% the capabilities of LLMs in diverse programming contexts.
Specifically, previous work collects and designs code instructions and generates corresponding responses using a high-capacity teacher LLM. This generated data is then employed to fine-tune a student LLM, thereby enhancing its code generation abilities. 
Although synthetic code data produced in this manner has demonstrated success, it relies on the availability of a strong teacher model, presumably with a larger parameter size and higher computation costs.
Additionally, existing work has shown that training a model on data generated by itself is ineffective because errors introduced during generation tend to accumulate over iterations~\cite{dubey2024llama}. As a result, there is a critical need for effective methods to verify the generated data.

Despite this pressing need, evaluating the correctness of the generated code is not trivial and often demands substantial programming expertise, even for human annotators.
Recently, some research has explored the use of LLM-as-a-judge~\cite{mcaleese2024llm,alshahwan2024automated,dong2024self}, which can automatically provide feedback on the code it has generated, for example by running the code against unit tests it produces. However, as we will show in this work, the current capability of LLMs to generate unit tests, herein referred to as LLM-as-a-verifier, is substantially worse than their ability to produce code solutions, herein referred to as LLM-as-a-solver (Section~\ref{sec:base-performance}). This is because of the lack of high-quality data specifically for unit test generation during the post-training phase, i.e., most fine-tuning data focuses on code generation, while only a small portion targets unit test generation.
%(both data quantity and quality)

% To address this limitation, one effective approach is to utilize synthetic data generated by LLMs, known as \textit{self-instruct} data~\cite{wang2023self}.
% This method has shown promising results when the model is trained on data generated by larger, more competent models. However, previous research has revealed that training models on their own generated data may not be beneficial (Llama 3.1 report). This may be due to the model accumulating its own mistakes during fine-tuning. Therefore, it is crucial to check the correctness of the generated solutions. 

In this work, we propose \textsc{Sol-Ver}, a self-play solver-verifier framework to iteratively train a model for both code and test generation. The main idea is to let the LLM-as-a-solver and LLM-as-a-verifier help each other. Specifically, we ask the model to generate code solutions and unit tests for the same set of coding problems. By  executing the generated test against the generated code, we obtain feedback for training, involving two steps: (1) SFT training: we take the passed examples for fine-tuning the model, and (2) DPO training: we take both passed and failed examples as preference pairs to further train the model aligning with the preference. These training steps are for both code generation and unit test generation, and they can be repeated in an iterative manner. 

The experimental results on Llama 3.1 8B model show that we can successfully improve the model’s performance on both code and test generation without relying on human-annotated data or larger models. Specifically, on MBPP and LiveCodeBench, we achieve an average of 19.63\% and 17.49\% relative improvement for code and test generation respectively.

\begin{figure*}[h]
    \centering
    \includegraphics[width=0.9\textwidth]{figs/icml-overview.pdf}
    \caption{An overview of the {\sc Sol-Ver} framework.
We train an LLM to both generate coding solutions (solver) and unit tests (verifier) in an iterative self-play framework, whereby synthetic preference pairs are constructed at each iteration depending on whether the code passes the generated tests or not.
We show that this approach enables the model to self-improve in both capabilities (see \autoref{tab:full-performance}).
    }
\if 0
insimultan
    We thus propose SOL-VER, a
self-play solver-verifier framework that jointly im-
proves a single model’s code and test generation
capacity. By iteratively refining code (LLM-as-
a-solver) and tests (LLM-as-a-verifier) together,
we boost both capabilities without relying on hu-
man annotations or larger teacher models
\fi 
    \label{fig:enter-label}
\end{figure*}

In summary, our work makes the following contributions:
\begin{itemize}[leftmargin=*]
    \item \textbf{Identification of a {\em critical gap}:} We analyze and highlight the significant gap in LLMs' abilities between code generation and unit test generation.
    \item \textbf{Novel \textit{Self-Play} Framework:} We propose a novel iterative framework where the model simultaneously functions as a code solver and a verifier. This methodology effectively self-aligns the model's outputs with desired performance criteria without relying on external annotations or teacher models.
    \item \textbf{High-Quality Synthetic Data Generation:} We contribute a generalizable method for creating high-quality synthetic data for both code and unit test generation. This data augmentation approach can be extended to various model training scenarios in the coding domain.
\end{itemize}



\section{Preliminaries}
In the LLM domain, a reward model is typically a regression model that takes an instruction and a response as input and outputs a reward score~\citep{ouyang2022training}, which can be formulated as $r_{\text{RM}}(x,y)$, where $x$ denotes an instruction and $y$ represents a response. Reward models are typically trained on a large set of preference pairs based on the Bradley-Terry (BT) model~\citep{bradley1952rank}.


However, due to the subjectivity and complexity of human preferences and the capacity limitations of the BT model~\citep{munos2023nash, swamy2024minimaximalist, sun2024rethinking}, reward models often exhibit subjective bias, such as favoring longer and detailed outputs~\citep{saito2023verbosity}, while neglecting verifiable correctness signals like factuality~\citep{liu2024rm, tan2024judgebench}. On the other hand, training LLMs with verifiable correctness signals has shown strong potential~\citep{lambert2024t, guo2025deepseek}. Based on these considerations, we propose \textit{agentic reward modeling}, a reward system that integrates reward models with verifiable correctness signals from different aspects to provide more reliable rewards. Agentic reward modeling can be formulated as follows:
\begin{equation}
% \small
r(x, y) = \underbrace{\lambda \cdot r_{\text{RM}}(x, y)}_{\text{base reward}} + \sum_{i \in A_x} \underbrace{w_i \cdot a_i(x, y)}_{\text{correctness signals}}
\label{eq:eq1}
\end{equation}
$\lambda$ denotes the weight of the base reward model. $a_i$ denotes a specific verification agent that provides verifiable correctness signals, such as rule-based rewards~\citep{mu2024rule}. $w_i$ denotes the corresponding weight for each verification agent, which can be set as a hyper-parameter or adaptive to the instruction. $A_x$ is an index subset of the complete set of verification agents $A$ and is determined based on the instruction $x$. Equation~\ref{eq:eq1} provides the fundamental concept of agentic reward modeling, which can be implemented in various ways to construct a reward agent and our implementation is in \cref{sec:method}.


\section{SELF-SELECTION FRAMEWORK}
In this section, we elaborate on the proposed \framework~framework for enhanced Retrieval-Augmented Generation (RAG). 
Before explaining our method, we first revisit two preliminary concepts, i.e. Large Language Model (LLM) and Retrieval-Augmented Generation (RAG).  
Then, we present the formulation of our \framework~framework with detailed notations.
In order to strengthen the capabilities of the LLM in accurately generating and selecting responses, we develop a novel \approach~method, which is essentially fining tuning the LLM over a newly built Retrieval-Generation Preference (RGP) dataset. 
%定义 

\subsection{Preliminaries}
%1.1
\subsubsection{\textbf{Large Language Model (LLM)}}
For an LLM represented by $\mathcal{M}$, given a prompt $\bar{p}$ and a query $q$ as inputs, it returns a textual answer $\bar{a}$ as the output, which is formally expressed as 
\begin{equation}\label{eq:llm-ans}
\bar{a} = \mathcal{M}(\bar{p}, q).
\end{equation}
\subsubsection{\textbf{Retrieval-Augmented Generation (RAG)}} 
An RAG system employs a retriever to enhance the capability of the LLM by enabling it to access external knowledge beyond its internal parametric knowledge~\cite{lee2019LatentRetrieval, Guu2020REALM}. 
Given a query $q$, the retriever $\mathcal{R}$ searches for the relevant knowledge (e.g., passages) $C$ from an external knowledge base or corpus. 
%Let $C$ denote the retrieved passages. 
A common approach for RAG is to include the retrieved passages $C$ in the input to the LLM to improve the response quality.
Formally,
\begin{equation}
C = \mathcal{R}(q)\label{eq:retrieval},
\end{equation}
\begin{equation}\label{eq:rag-ans}
\hat{a} = \mathcal{M}(\hat{p}, q, C),
\end{equation}
where $\hat{p}$ represents the prompt used in RAG and $\hat{a}$ denotes the answer predicted by the LLM taking into account the retrieved passages $C$.


\subsection{Task Formulation}

%\header{Task Definition}
In this part we present the formulation of our \framework~framework.
An illustration is provided in \Figref{sample}. 
Given a query $q$, we first prompt an LLM $\mathcal{M}$ with $\bar{p}$ denoting the prompt to output the answer $\bar{a}$ with its explanation $\bar{e}$, where we refer to $\bar{a}$ as the \emph{LLM Answer} and $\bar{e}$ as the \emph{LLM Explanation}.
Next, we use the retriever $\mathcal{R}$ to gather relevant passages $C$ (Eq. (\ref{eq:retrieval})) to the same query $q$.
Then, we prompt the LLM $\mathcal{M}$ with $\hat{p}$ while providing $q$ and $C$ in the input, to generate $\hat{a}$ with its explanation $\hat{e}$, where we refer to  $\hat{a}$ and $\hat{e}$  as the \emph{RAG Answer} and the \emph{RAG Explanation}.
Finally, we prompt the LLM $M$ with a prompt $p$ by taking the query $q$, the \emph{LLM Answer} $\bar{a}$ with its \emph{LLM Explanation} $\bar{e}$, the \emph{RAG Answer} $\hat{a}$ with its \emph{RAG Explanation} $\hat{e}$ as inputs to select one as the final answer $a$ and final explanation $e$.
Formally,
\begin{equation}\label{eq:llm-ans-e}
(\bar{a}, \bar{e}) = \mathcal{M}(\bar{p}, q);
\end{equation}
\begin{equation}\label{eq:rag-ans-e}
(\hat{a}, \hat{e}) = \mathcal{M}(\hat{p}, q, C);
\end{equation}
\begin{equation}\label{eq:final-ans-e}
(a, e) = \mathcal{M}(p, q, (\bar{a}, \bar{e}), (\hat{a}, \hat{e})).
\end{equation}
%We refer to this new RAG flow as a \framework~framework, as illustrated in \Figref{sample}. 



% \begin{equation}\label{eq:llm-ans-e}
% (\bar{a}, \bar{e}) = \mathcal{M}(\bar{p}, q);
% \end{equation}
% \begin{equation}\label{eq:rag-ans-e}
% (\hat{a}, \hat{e}) = \mathcal{M}(\hat{p}, q, C);
% \end{equation}
% \begin{equation}\label{eq:final-ans-e}
% (a, e) = \mathcal{M}(p, q, (\bar{a}, \bar{e}), (\hat{a}, \hat{e})).
% \end{equation}
% The whole pipeline of \framework~framework is illustrated in \Figref{sample}. 

% \begin{equation}\label{eq:sel}
% a = \mathcal{M}(p, q, \bar{a}, \hat{a}).
% \end{equation}


%\subsection{Self-Selection Framework} 
% Our goal is to empower an LLM in an RAG system to effectively harness both its internal parametric knowledge and retrieved external knowledge to produce the answer to a query.
% In our proposed \framework~framework, to better leverage the powerful reasoning capability of LLMs, we further instruct the LLM to provide the corresponding explanation (i.e., reasoning steps) to support the predicted answer.
% First, an LLM is adopted to infer the \emph{LLM Answer} with the corresponding explanation relying sorely on its internal knowledge.
% Meanwhile, the LLM generates the \emph{RAG Answer} with the corresponding explanation depending on the external retrieved knowledge $C$.
% After holistically evaluating both candidates, the LLM selects the most appropriate answer and provides an explanation, thereby providing a knowledge fusion process.
% In \framework~framework, the same LLM is responsible not only for generating candidate answers but also for selecting the most accurate one among them.


% Formally, given a query $q$, the LLM $\mathcal{M}$ is requested to output the \emph{LLM Answer} $\bar{a}$ with its explanation $\bar{e}$ and the \emph{RAG Answer} $\hat{a}$ with its explanation $\hat{e}$. 
% Next, the LLM is instructed to produce the explanation $e$ to the final answer $a$,  
% \begin{equation}\label{eq:llm-ans-e}
% (\bar{a}, \bar{e}) = \mathcal{M}(\bar{p}, q);
% \end{equation}
% \begin{equation}\label{eq:rag-ans-e}
% (\hat{a}, \hat{e}) = \mathcal{M}(\hat{p}, q, C);
% \end{equation}
% \begin{equation}\label{eq:final-ans-e}
% (a, e) = \mathcal{M}(p, q, (\bar{a}, \bar{e}), (\hat{a}, \hat{e})).
% \end{equation}
% The whole pipeline of \framework~framework is illustrated in \Figref{sample}. 





\subsection{Self-Selection-RGP}

\subsubsection{\textbf{Motivation}}
We evaluate the performance of the proposed \framework~framework on two widely used QA datasets, Natural Question (NQ)~\cite{kwiatkowski2019natural} and TriviaQA~\cite{joshi2017triviaqa}, using existing open-source models, including Mistral 7B~\cite{jiang2023mistral7b} and Llama2-13B-Chat~\cite{touvron2023llama2openfoundation} without any model parameter updates. 
We report the experimental results in \tabref{tab:main} of Section \ref{sec:experiments}.
We find that our \framework~framework is promising in enhancing LLMs' answer generation by fusing internal knowledge with external knowledge, but directly applying such knowledge fusion does not always bring enhancements. 
For instance, simply equipping Mistral-7B with a retriever outperforms applying our \framework~to Mistral-7B with the same retriever on the NQ dataset. 
One assumption is that LLMs struggle to reliably discern the correct answer between two candidates generated from different knowledge sources.
In essence, this knowledge selection process is consistent with the goal of preference alignment in LLMs, i.e. generating the desired (positive) sample while rejecting the undesired (negative) one from a pair of preference data.
To address this challenge, we explore tuning LLMs through preference alignment techniques to enhance their ability to discern and select the correct answer from two candidates generated by different knowledge sources.
To achieve this goal, we develop a novel \approach~method to enhance LLMs' capabilities in identifying and generating correct answers, as shown in \Figref{method}.
We first build a preference dataset, then employ a simple yet effective augmentation technique to expand it, and finally apply the augmented preference dataset to train open-sourced LLMs with Direct Preference Optimization (DPO)~\cite{Rafailov2023DPO}. 

\subsubsection{\textbf{Retrieval-Generation Preference Dataset}}
Here we explain how we build the Retrieval-Generation Preference (RGP) dataset used for fine-tuning LLMs in~\approach.

\header{Preference Candidate Generation}
We first employ an LLM to produce two sets of responses for each query $q$: (i) an \emph{LLM Answer} $\bar{a}$ with its \emph{LLM Explanation} $\bar{e}$, derived from the model’s internal parametric knowledge; and (ii) an \emph{RAG Answer} $\hat{a}$ with its \emph{RAG Explanation} $\hat{e}$, relying on the externally retrieved information.
Specifically, we randomly select a subset of QA pairs from three existing open-domain QA datasets, including WebQuestions~\cite{berant2013webq}, SQuAD2.0~\cite{rajpurkar2018squad}, and SciQ~\cite{welbl2017sciq}.
Let $\mathcal{D}$ denote the obtained set of QA pairs. Formally,
\begin{equation} \label{eq:gold}
    \mathcal{D} = \bigl\{q^{(i)}, a_g^{(i)}\bigr\}_{i=1}^N
\end{equation}
where $a_g$ is the golden answer to the query $q$, $N$ is the number of QA pairs and  $i$ is the $i$-th QA pair in $\mathcal{D}$.
For each query $q$ in $\mathcal{D}$, we utilize a retriever $\mathcal{R}$ to retrieve the top-K passages $C$ from a corpus (Eq. (\ref{eq:retrieval})).
To ensure the quality of the constructed preference dataset, we employ GPT-3.5~\cite{Ouyang2022Training} as the model $\mathcal{M}$ for candidate answer and explanation generation given a query.
According to Eq. (\ref{eq:llm-ans-e}) and Eq. (\ref{eq:rag-ans-e}), we generate the answers and explanations ($\bar{a}$, $\bar{e}$) and ($\hat{a}$, $\hat{e}$). 
Finally, we obtain a collection of preference candidates for constructing the RGP datasets.
Formally, the $D$ is expanded as
\begin{equation} \label{eq:gold}
    \mathcal{D} = \bigl\{q^{(i)}, a_g^{(i)}, \bar{a}^{(i)},\bar{e}^{(i)}, \hat{a}^{(i)}, \hat{e}^{(i)}\bigr\}_{i=1}^N.
\end{equation}

\header{Preference Data Filtering}
In the RGP dataset, each instance should include both a desired (positive) answer and an undesired (negative) answer.
We filter these required instances from the collection $\mathcal{D}$.
For each instance in $D$, we first employ GPT-3.5 to assess whether the \emph{LLM Answer} $\bar{a}$ and the \emph{RAG Answer} $\hat{a}$ are correct by comparing each to the golden answer $a_g$.
After that, we only retain the instances that contain one right answer and one wrong answer, where (i) $\bar{a}$ is correct but $\hat{a}$ is incorrect; or (ii) $\hat{a}$ is correct but $\bar{a}$ is incorrect.
Based on this strategy, we gather all appropriate instances in $\mathcal{D}$ to build our RGP dataset $\mathbb{D}$. Formally,
\begin{equation} \label{eq:gold}
    \mathbb{D} = \bigl\{q^{(j)}, a_g^{(j)}, (a_p^{(j)}, e_p^{(j)}), (a_n^{(j)}, e_n^{(j)})\bigr\}_{j=1}^M
\end{equation}
where $a_p$ and $e_p$ represent the positive answer and its explanation, $a_n$ and $e_n$ represent the negative answer and its explanation, $M$ denotes the number of instances and $j$ denotes the $j$-th instance in $\mathbb{D}$.
Finally, we retain $3,756$ preference instances in the RGP dataset.
We promise to release it for facilitating future reseach.


\subsubsection{\textbf{Retrieval-Generation Preference Alignment}}
With the constructed RGP dataset, we train open-source LLMs to enhance their ability to distinguish the positive answer from the negative counterpart.

\header{RGP Dataset Augmentation}
To improve LLMs' preference alignment, we first augment the RGP dataset through a simple yet effective approach to produce more preference instances. 
In particular, given a query $q$ in RGP, we search for the top-K similar queries in the RGP datasets and we regard all answers to these $K$ queries as negative answers to the query $q$.
Formally, for each query $q$ in the RGP dataset~$\mathbb{D}$,  we denote the obtained most similar queries and their responses in RGP as $\mathbb{G}$:  
\begin{equation} \label{eq:group}
    \small
    \mathbb{G}^{(i)} = \{ q^{(j)}, y_w^{(j)}, y_l^{(j)}~|~\underset{\text{top-}\mathrm{K}}{\mathrm{argmax}}\ S\left(q^{(i)}, q^{(j)}\right)\}~~~\forall q^{(i)} \in \mathbb{D}, i\neq j
\end{equation}
where $S(q^{(i)}, q^{(j)})$ represents the similarity score between $q^{(i)}$ and $q^{(j)}$, and $y_w^{(j)}$ and $y_l^{(j)}$ represent the corresponding positive and negative response (i.e., answer with its explanation) for $q^{(j)}$ in RGP.
For one query $q^{(i)}$, $y_w^{(i)}$ and $y_l^{(i)}$ are the original positive and negative response in RGP.
Then, we regard all $y_w^{(j)}$ and $y_l^{(j)}$ in the obtained set $\mathbb{G}^{(i)}$ as the additional negative responses to $q^{(i)}$.
For each query, now we have $1$ positive response and $2K+1$ negative responses, which can be used to form $2K+1$ pairs of preference instances. 
Let $\mathbb{D}_{aug}$ denote the augmented RGP dataset. Formally,
\begin{equation}
\mathbb{D}_{aug}^{(i)} = \left\{ \left( q^{(i)}, y_w^{(i)}, y_{lj}^{(i)} \right) \right\}_{i=1}^M, \, j=1,2,\dots,2K+1,
\end{equation}
where $y_{lj}^{(i)}$ is the $j$-th negative response to the query $q^{(i)}$. 

\header{Retrieval-Generation Preference Training} 
With the augmented preference dataset, our goal is to train open-sourced LLMs to enhance their capabilities in distinguishing the correct answers from the incorrect ones.
During the preference alignment phase of LLM training, each instance in the augmented dataset $\mathbb{D}_{aug}$ comprises three key elements: an input $x$, a desired response $y_w$, and an undesired response $y_l$,  which is denoted as $y_w \succ y_l \mid x$.
Specifically, the input $x$ consists of a query $q$, the desired response $y_w$, the undesired response $y_l$, and a prompt $p$ designed to instruct the LLM $M$ to choose between $y_w$ and $y_l$ (see Eq. (\ref{eq:final-ans-e})).
The desired response $y_w$ includes the correct answer along with its explanation, while the undesired response $y_l$ contains an incorrect answer and its explanation.
To enhance the robustness of the trained model, we randomly alternate the order of $y_w$ and $y_l$ within the input $x$.


We adopt Direct Preference Optimization (DPO)~\cite{Rafailov2023DPO} to train LLMs.
It enables preference data to be directly associated with the optimal policy, eliminating the need for any additional reward model.
DPO formulates a maximum likelihood objective as follows:
\begin{equation}\label{eq:optimum_model}
    \tiny
    \mathcal{L}_\text{DPO}(\pi_{\theta}; \pi_{ref}) = -\mathbb{E}_{(x, y_w, y_l)\sim \mathcal{D}}\left[\log \sigma \left(\beta \log \frac{\pi_{\theta}(y_w\mid x)}{\pi_{ref}(y_w\mid x)} - \beta \log \frac{\pi_{\theta}(y_l\mid x)}{\pi_{ref}(y_l\mid x)}\right)\right]
\end{equation}
where  $\beta$ represents the deviation of the policy $\pi_{\theta}$ from the reference model $\pi_{ref}$.
Our proposed optimization method aims to enhance LLMs' answer selection capability, enabling them to holistically evaluate multiple responses generated from diverse knowledge sources and identify the most accurate one among them.
In addition, we also hope this optimization method can further improve the inherent ability of LLMs in answer generation (more analysis is provided in Section \ref{sec:llm-improve}).





\section{Experiments}
This section presents experiments on several reward model benchmarks, including experimental setup (\cref{sec:exp_setup}), results (\cref{sec:exp_result}), and analyses (\cref{sec:exp_analysis}).


\begin{table*}
    \centering
    \small
    % \resizebox{\linewidth}{!}{
    \begin{tabular}{lccccccc}
    \toprule
    \multirow{2}{*}{Model} & \multicolumn{2}{c}{RM-Bench} & \multirow{2}{*}{JudgeBench} & \multicolumn{3}{c}{IFBench} & \multirow{2}{*}{Overall} \\
    \cmidrule{2-3} \cmidrule{5-7}
    & Normal & Hard & & Simple & Normal & Hard & \\
    \midrule
ArmoRM-Llama3-8B-v0.1 &$76.7$&$34.6$&$51.9$&$72.3$&$66.2$&$59.5$&$56.5$\\
INF-ORM-Llama3.1-70B &$77.5$&$25.1$&$59.1$&$78.7$&$69.2$&$53.8$&$55.7$\\
Skywork-Reward-Llama-3.1-8B-v0.2 &$78.0$&$31.8$&$57.8$&$78.7$&$69.2$&$59.8$&$58.1$\\
Skywork-Reward-Gemma-2-27B &$82.7$&$35.1$&$55.8$&$\boldsymbol{87.2}$&$68.4$&$56.1$&$59.2$\\
internlm2-7b-reward &$72.6$&$19.9$&$56.2$&$74.5$&$61.7$&$55.7$&$52.0$\\
internlm2-20b-reward &$74.4$&$26.1$&$61.7$&$74.5$&$68.4$&$58.7$&$56.4$\\
\midrule
GPT-4o &$71.4$&$27.9$&$64.6$&$\underline{85.1}$&$66.2$&$54.4$&$56.3$\\
GPT-4o mini &$60.5$&$15.0$&$51.9$&$70.2$&$59.4$&$51.9$&$45.9$\\
o3-mini &$76.0$&$38.6$&$66.6$&$81.9$&$\underline{76.3}$&$64.6$&$62.8$\\
Llama3-8B Instruct &$\phantom{0}9.3$&$20.2$&$\phantom{0}2.6$&$12.8$&$12.8$&$13.6$&$11.3$\\
DeepSeek-R1 &$83.7$&$50.1$&$\boldsymbol{74.4}$&$72.3$&$74.4$&$64.0$&$69.1$\\
DeepSeek-R1-Distill-Llama-8B &$42.1$&$56.8$&$47.7$&$53.2$&$55.6$&$54.2$&$50.3$\\
\midrule
\ourmethodllama &$79.3$&$53.5$&$52.9$&$70.2$&$63.9$&$67.8$&$63.2$\\
\quad w/ search engine &$76.0$&$49.9$&$55.2$&$74.5$&$69.2$&$67.8$&$62.5$\\
\ourmethodmini &$\boldsymbol{86.0}$&$\boldsymbol{60.2}$&$\underline{68.2}$&$78.7$&$69.2$&$\boldsymbol{78.0}$&$\boldsymbol{72.5}$\\
\quad w/ search engine &$\underline{84.2}$&$\underline{59.7}$&$60.7$&$68.1$&$\boldsymbol{80.5}$&$\underline{76.1}$&$\underline{70.3}$\\
    \bottomrule
    \end{tabular}
    % }
    \caption{Experimental results (\%) of all investigated baselines and \ourmethod. The overall score is the average of RM-Bench, JudgeBench, and the micro-averaged score of three subsets of IFBench. By default, \ourmethod relies on its parametric knowledge, and ``w/ search engine'' denotes using Google API as an external source.}
    \label{tab:main_exp}
\end{table*}


\subsection{Experimental Setup}
\label{sec:exp_setup}

\paragraph{\ourmethod Implementation}
We adopt the advanced and lightweight ArmoRM~\citep{wang2024interpretable} as the base reward model to compute human preference scores. As \ourmethod is agnostic to reward models, one can also adopt other advanced reward models. 
We use GPT-4o mini~\citep{OpenAI2024} as the LLM backbone for implementing all modules and developing \ourmethodmini. We also employ the open-source LLM Llama3-8B Instruct~\citep{dubey2024llama} as the backbone and develop \ourmethodllama, except for the instruction-following verification agent, which requires strong coding capabilities and is instead powered by Qwen2.5-Coder 7B~\citep{hui2024qwen2}.
We adopt two knowledge sources for the factuality verification agent: an external search engine using Google API and the LLM’s parametric parameters. More details are placed in appendix~\ref{sec:app_method}.



\paragraph{Evaluation Benchmarks}
Reward model benchmarks typically involve an instruction and a response pair and require selecting the better response as the chosen one.
We use RM-Bench~\citep{liu2024rm}, JudgeBench~\citep{tan2024judgebench}, and a new benchmark \ourdataset as evaluation benchmarks, as both RM-Bench and JudgeBench include response pairs involving factual correctness. We select the chat subset of RM-Bench as the evaluation set, using both the normal and hard settings. For JudgeBench, we use the knowledge subset as the evaluation set.
We further construct a new benchmark \ourdataset to evaluate reward models on selecting responses that better follow constraints in instructions as there is no existing relevant benchmark.
Specifically, we first construct instructions with several implicit constraints, integrating the constraint information with the primary task objective through paraphrasing. The constraints include both hard constraints, such as length, format, and keywords, as well as soft constraints, such as content and style. We then use GPT-4o to generate $8$ responses for each instruction with a sampling temperature of $1.0$. For each instruction, we create a response pair, selecting the one that satisfies all constraints as the chosen response and otherwise rejected. Based on the number of unsatisfied constraints (UC) in the rejected response, we split \ourdataset instances into three subsets: simple (\#UC$\geq$3), normal (\#UC$=$2), and hard (\#UC$=$1),  containing $47$, $133$, and $264$ instances respectively. 
We report the micro-averaged accuracy across the three subsets as the final metric for \ourdataset. More evaluation details on these benchmarks are provided in appendix~\ref{sec:app_exp}.



\paragraph{Baselines}
\looseness = -1
We mainly investigate two categories of baselines: (1) typical reward models, which are specifically trained for reward modeling and typically implemented as regression models to score each response and select the one with the highest reward score as the chosen response. We investigate several advanced and representative reward models, including ArmoRM~\citep{wang2024interpretable}, INF-ORM-Llama3.1-70B~\citep{infly2024inf}, Skywork-Reward~\citep{liu2024skywork}, internlm2 reward~\citep{cai2024internlm2}. (2) LLMs as generative reward models, where large language models serve as generative reward models to score responses or perform pairwise comparisons to select the best response~\citep{lambert2024rewardbench}. We evaluate proprietary models, including GPT-4o~\citep{OpenAI20244o}, GPT-4o mini~\citep{OpenAI2024}, o3-mini~\citep{openai2025o3mini}, and open-source LLMs, including Llama3-8B Instruct~\citep{dubey2024llama}, 
% {\color{red}Qwen2.5-Coder 7B~\citep{hui2024qwen2}}, 
DeepSeek-R1, and R1 distilled Llama3-8B model~\citep{guo2025deepseek}. 
We evaluate all the baselines using the code repository provided by \citet{lambert2024rewardbench}.


\subsection{Experimental Results}
\label{sec:exp_result}



Table~\ref{tab:main_exp} presents the experimental results, and we can observe that:
(1) Existing reward models fall short in selecting more factual responses or better adhering to hard constraints in instructions, which may limit their reliability in real-world applications.
(2) \ourmethod significantly outperforms the base reward model AromRM and the corresponding LLM backbone GPT-4o mini and Llama3-8B Instruct. It demonstrates that designing an appropriate reward agentic workflow can effectively enhance reward model performance.
(3) Even when using Llama3-8B Instruct as the LLM backbone, \ourmethodllama outperforms reward models with much more parameters and more advanced proprietary LLMs such as GPT-4o, which suggests that \ourmethod is more cost-efficient without requiring additional reward modeling training data or more parameters to achieve advanced performance.
(4) Using a search engine as an external knowledge source for factuality slightly reduces performance in RM-Bench and JudgeBench. One possible reason is that the retrieved information may contain noise or irrelevant information~\citep{chen2024benchmarking}. We leave the detailed analysis and design of retrieval-augmented agents for future work.
(5) \ourmethod achieves significant improvements on IFBench, particularly in the hard subset. It suggests that while not perfectly solved, existing LLMs can effectively analyze hard constraints and generate verification code, which can help the training of advanced LLMs~\citep{lambert2024t}.

In conclusion, incorporating additional verification agents for specific scenarios~\cite{mu2024rule, lambert2024t}, particularly those with verifiable correctness, can develop more reliable and advanced reward systems, presenting a promising direction for future reward model development.



\subsection{Analysis}
\label{sec:exp_analysis}


\begin{table}
    \centering
    \small
    \resizebox{\linewidth}{!}{
    \setlength{\tabcolsep}{3pt}
    \begin{tabular}{lccc}
    \toprule
    Model & RM-Bench & JudgeBench & IFBench \\
    \midrule
    \ourmethodmini &$73.1$&$68.2$&$75.5$\\
    \hspace{2mm}\textit{-- factuality verifier} &$54.0$&$52.9$&$73.6$\\
    \hspace{2mm}\textit{-- if verifier} &$74.7$&$66.2$&$60.4$\\
    \hspace{2mm}\textit{-- both} &$55.4$&$58.8$&$58.8$\\
    \midrule[0.1pt]
    Oracle setting &$76.7$&$70.1$&$77.5$\\
    \midrule
    \ourmethodllama &$66.4$&$52.9$&$66.9$\\
    \hspace{2mm}\textit{-- factuality verifier} &$51.9$&$51.6$&$65.8$\\
    \hspace{2mm}\textit{-- if verifier} &$58.0$&$57.5$&$57.2$\\
    \hspace{2mm}\textit{-- both} &$44.8$&$55.5$&$57.2$\\
    \midrule[0.1pt]
    Oracle setting &$79.5$&$73.1$&$68.5$\\
    \bottomrule
    \end{tabular}
    }
    \caption{Experimental results (\%) of ablation study and the oracle setting. \textit{-- factuality verifier} and \textit{-- if verifier} refer to the reduction of the corresponding verification agent into a single LLM scorer.
    The results are the micro-averaged scores of all the corresponding subsets.}
    \label{tab:analysis}
\end{table}

We first conduct an ablation study on the verification agents in \ourmethod. Specifically, we investigate three settings: \textit{-- factuality verifier}, \textit{-- if verifier}, and \textit{-- both}, where the corresponding verification agents are reduced to \textbf{a single step}: using an additional LLM backbone to directly score the response, which is equivalent to the simple ensemble of the reward model ArmoRM with the corresponding LLM as a generative reward model~\citep{costereward2024}.
The ablation results are shown in Table~\ref{tab:analysis}. We can observe that removing the well-designed verification agent leads to a significant performance decrease. It demonstrates the importance of well-designed verification agents, and we encourage the community to develop more advanced verification agents for a more reliable \ourmethod.



We also observe the oracle setting of \ourmethod that invokes the most appropriate verification agents, that is, invoking the factuality agent on RM-Bench and JudgeBench, and the instruction-following verification agent on IFBench. The experimental results are shown in Table~\ref{tab:analysis}, and we observe that both \ourmethodmini and \ourmethodllama perform significantly better in the oracle setting. This further demonstrates the effectiveness of the verification agents and suggests that the planner in \ourmethod still has a large room for improvement and we leave developing a more advanced planner for future work. This also suggests that in some specific and well-defined scenarios, one can adopt the corresponding verification agent alone to achieve better results.

\begin{figure*}
    \centering
    \includegraphics[width=0.98\linewidth]{figures/best_of_n.pdf}
    \caption{Best-of-n results (\%) on TriviaQA, IFEval, and CELLO using the base reward model ArmoRM and \ourmethod to search. ``+Oracle'' denotes using the oracle setting of \ourmethod as mentioned in \cref{sec:exp_analysis}.}
    \label{fig:enter-label}
\end{figure*}


\begin{table*}
    \centering
    \small
    \begin{tabular}{lccccccc}
    \toprule
    DPO Training Data & MMLU & MMLU-Pro & TriviaQA & TruthfulQA & IFEval & CELLO & MT-Bench\\
    \midrule
    -- & $58.9$ & $28.8$ & $54.8$ & $39.5$ & $43.3$ & $51.5$ & $5.2$ \\
    \midrule
    Original UF & $58.7$ & $29.3$ & $54.0$ & $42.0$ & $56.8$ & $62.0$  & $6.0$ \\
    ArmoRM-UF & $58.1$ & $29.9$ & $52.5$ & $45.0$ & $58.6$ & $60.8$ & $6.0$ \\
    \ourmethodllama-UF & $59.1$ & $30.5$ & $55.1$ & $44.1$ & $\mathbf{59.4}$ & $60.1$ & $5.8$ \\
    \midrule
    ArmoRM-OP & $58.4$ & $30.4$ & $51.6$ & $44.4$ & $52.7$ & $58.1$ & $6.0$ \\
    \ourmethodllama-OP & $\mathbf{59.5}$ & $\mathbf{31.3}$ & $\mathbf{55.3}$ & $\mathbf{48.5}$ & $58.2$ & $\mathbf{65.7}$ & $\mathbf{6.1}$ \\
    \bottomrule
    \end{tabular}
    \caption{Experimental results (\%) of LLMs trained with DPO on different training data. ``ArmoRM-UF'' denotes using ArmoRM to construct preference pairs from UltraFeedback. ``UF'' and ``OP'' are short for UltraFeedback and on-policy data, respectively. ``Original UF'' refers to using the original GPT-4 annotated preference pairs from UltraFeedback to train the LLM. ``--'' denotes the original LLM zephyr-7b-sft-full without further DPO training.}
    \label{tab:dpo_results}
\end{table*}



















\section{Applications}
This section explores applying \ourmethod to inference-time search (\cref{sec:best_of_n}) and the training of LLMs (\cref{sec:dpo_train}) to further validate its effectiveness.


\subsection{Best-of-N Search}
\label{sec:best_of_n}

One important application of reward models is to conduct the inference-time search to find a better response~\citep{brown2024large,zhang2024generative}, which unleashes the inference-time scaling laws of LLMs~\citep{snell2024scaling, wu2024inference}. 
Therefore, we explore applying \ourmethod to the best-of-n search on downstream tasks. Specifically, we evaluate the best-of-n performance searched by \ourmethod on factuality question answering and constrained instruction following tasks.

\paragraph{Experimental Setup}
We conduct the best-of-n experiments on the factuality question answering dataset TriviaQA~\citep{joshi2017triviaqa}, and the instruction-following datasets IFEval~\citep{zhou2023instruction} and CELLO~\citep{he2024can}. We use Llama3-8B Instruct and GPT-4o as the policy models to generate $32$ responses for each instruction with $1.0$ sampling temperature. We perform best-of-n search using the base reward model ArmoRM~\citep{wang2024interpretable}, \ourmethodmini, and the oracle setting of \ourmethodmini.  The oracle setting refers to invoking the factuality verification agent on TriviaQA, and the instruction-following verification agent on IFEval and CELLO.



\paragraph{Experimental Results}
The results of the best-of-n experiments using Llama3-8B Instruct as the policy model are shown in Figure~\ref{fig:enter-label}. We can observe that \ourmethod significantly improves the best-of-n performance compared to using the base reward model ArmoRM, and the oracle setting further improves the results. 
It further validates the effectiveness of \ourmethod. 
The results using GPT-4o as the policy model are provided in appendix~\ref{sec:app_exp}, demonstrating the same trends and conclusions. We encourage the community to design more verification agents to unleash the inference scaling laws of LLMs across different scenarios.





\subsection{DPO Training}
\label{sec:dpo_train}
Reward models are primarily used to train LLMs using RL~\citep{ouyang2022training} or DPO~\citep{rafailov2024direct}. 
Considering RL training is resource-intensive, we explore employing \ourmethod to construct preference pairs for DPO training to validate its effectiveness in real-world applications.

\paragraph{Experimental Setup}
We construct two training datasets based on: (1) UltraFeedback~\citep{cui2024ultrafeedback}, where each instruction contains $4$ responses sampled from various LLMs. (2) on-policy, which contains $20,000$ instructions sampled from UltraFeedback and
each instruction contains $8$ responses sampled from the policy model itself with $1.0$ sampling temperature. We use reward models to score each response, taking the highest-scored response as the chosen one and the lowest as the rejected one to construct training pairs.
We adopt the zephyr-7b-sft-full~\citep{tunstall2023zephyr} model as the policy model to conduct DPO training because it is trained only using SFT~\citep{ouyang2022training}. We evaluate the DPO-trained LLMs on various NLP benchmarks, including MMLU~\citep{hendrycksmeasuring}, MMLU-Pro~\citep{wang2024mmlu}, TriviaQA~\citep{joshi2017triviaqa}, TruthfulQA~\citep{lin2022truthfulqa}, IFEval~\citep{zhou2023instruction}, CELLO~\citep{he2024can}, and MT-Bench~\citep{zheng2023judging}. More experimental details are provided in appendix~\ref{sec:app_exp}.



\paragraph{Experimental Results}

The experimental results are shown in Table~\ref{tab:dpo_results}. We can observe that LLMs trained with data constructed by \ourmethod generally outperform those trained with ArmoRM, especially on the factuality question answering and instruction-following datasets. The improvement is more significant in on-policy data.
Furthermore, models trained with \ourmethod-annotated data consistently outperform those trained on original UltraFeedback that is constructed with GPT-4. Notably, \ourmethodllama uses open-source Llama3-8B Instruct and Qwen2.5-Coder 7B as the LLM backbones, at a much lower cost than GPT-4.
The results further validate the effectiveness and applicability of \ourmethod. We believe using a more powerful LLM backbone in \ourmethod can achieve more advanced results and encourage the community to explore more advanced reward agents for better performance and reliability.

\section{Related Work}
Reward models are typically employed to score responses and are crucial to the success of modern LLMs. Since the emergence of RLHF~\citep{ouyang2022training}, numerous studies have focused on developing more advanced reward models to help train LLMs. The approaches mainly include designing model architectures~\citep{wang2024interpretable,dorka2024quantile,chen2025LDLRewardGemma} and utilizing more high-quality data or new training objectives~\citep{infly2024inf,yuan2024advancing,park2024offsetbias,liu2024skywork,cai2024internlm2,cao2024compass,lou2024uncertainty,litool2024,wang2024helpsteer2}. There are also various studies exploring using LLMs as generative reward models~\citep{zheng2023judging,mahan2024generative,skyworkcritic2024,cao2024compass,tan2024judgebench,yu2024self,alexandru2025atlaseleneminigeneral}.
Reward models are typically used for inference-time scaling laws~\citep{irvine2023rewarding,wu2024inference,snell2024scaling,brown2024large,xin2024deepseek} or for training, such as RL\citep{ouyang2022training} or DPO~\citep{rafailov2024direct}.

Despite the success of reward models, they primarily focus on human preferences, which may be susceptible to subjective biases or reward hacking~\citep{saito2023verbosity,singhal2023long,gao2023scaling,zhang2024lists,chen2024odin}. A notable limitation is \textit{verbosity bias}~\citep{saito2023verbosity}, where reward models tend to favor longer responses~\citep{singhal2023long, liu2024rm}. Additionally, some studies have shown that reward models may overlook correctness signals, such as factuality~\citep{lin2024flame, liu2024rm, tan2024judgebench}. These limitations affect the reliability of reward models, thereby impacting the performance of the trained LLMs~\citep{singhal2023long}.


Recently, several studies have shown that rule-based reward models or verifiable reward signals achieve impressive results in specific domains such as math~\citep{guo2025deepseek}, safety~\citep{mu2024rule}, instruction-following~\citep{lambert2024t}, medical~\citep{chen2024huatuogpt}, and finance~\citep{qian2025fino1}. The simplicity and advanced performance of rule-based reward models demonstrate significant potential for training LLMs, but it is still non-trivial to generalize to general domains.
In this paper, we explore combining human preferences from reward models with verifiable correctness signals to develop more reliable reward systems.
We believe that combining human preferences with verifiable correctness signals is a promising direction and encourage further research efforts in this area.

\section{Conclusion}

In this paper, we propose \textit{agentic reward modeling}, a reward system that integrates the human preferences from conventional reward models with verifiable correctness signals to provide more reliable rewards. We empirically implement a reward agent, named \ourmethod, which consists of a router, well-designed verification agents for factuality and instruction-following, and a judger. We conduct extensive experiments on reward modeling benchmarks, best-of-n search, and DPO training. \ourmethod significantly outperforms other reward models and LLMs as generative reward models. We encourage more research efforts to develop more advanced and reliable reward systems.
\section*{Limitations}

The main limitations of this work lie in the implementation of \ourmethod: (1) The verification agents are far from providing perfect rewards, as the average score on reward modeling benchmarks only reaches $72.5\%$. This suggests that achieving perfect rewards is challenging and requires further research efforts. (2) We only implement verification agents for factuality and instruction-following, which we believe are current weaknesses in reward models~\citep{liu2024rm} and important factors affecting LLM applications and user experiences. We encourage the community to explore more verifiable correctness signals. In conclusion, we believe the contribution of \textit{agentic reward modeling} concept is substantial, and we look forward to developing more advanced reward systems in the future.

\section*{Ethical Considerations}
We discuss the ethical considerations here:
(1) Intellectual property. 
We have strictly adhered to the licenses of all utilized artifacts, including datasets, models, and code repositories. We will open-source \ourmethod, code, and IFBench under the MIT license\footnote{\url{https://opensource.org/license/mit}}.
(2) Intended use and potential risk control.
We propose \textit{agentic reward modeling}, a reward system that integrates human preferences with correctness signals. We implement a reward agent named \ourmethod to provide more reliable rewards. We believe that all data used is well anonymized. Our model does not introduce additional ethical concerns but may provide incorrect rewards due to performance limitations. Users should not conduct reward hacking~\citep{skalse2022defining} and should carefully check important information.
(3) AI assistance. 
We have used ChatGPT to refine some sentences.

% Bibliography entries for the entire Anthology, followed by custom entries
%\bibliography{anthology,custom}
% Custom bibliography entries only
\bibliography{custom}

\newpage
\clearpage
\appendix
\section*{Appendices}
\section{\ourmethod Details}
\label{sec:app_method}

\looseness=-1
Tables~\ref{tab:planner} to~\ref{tab:if_agent} present the LLM prompts used for the implementation of \ourmethod. We employed Serper\footnote{\url{https://serper.dev/}} to implement our external search engine and we utilize the \texttt{gpt-4o-mini-2024-07-18} model in the \ourmethodmini version.
%Prompts,可以更换为example

% \begin{table*}
%     \centering
%     \small
%     \begin{adjustbox}{max width=1\linewidth}
%     {
%     \begin{tabular}{p{\linewidth}}
%     \toprule
%     % \textbf{Prompt For Router} \\
%     % \midrule
%    Given the following instruction, determine whether the following check in needed. \\
%     \\
%         \text{[Instruction]} \\
%         \{instruction\} \\
%     \\
%         \text{[Checks]} \\
%        \{ 
%             ``name'': ``constraint check'', 
%             ``desp'': ``A `constraint check' is required if the instruction contains any additional constraints or requirements on the output, such as length, keywords, format, number of sections, frequency, order, etc.'', 
%             ``identifier'': ``[[A]]'' 
%         \}, 
%         \{  
%             ``name'': ``factuality check'', 
%             ``desp'': ``A `factuality check' is required if the generated response to the instruction potentially contains claims about factual information or world knowledge.'', 
%             ``identifier'': ``[[B]]'' 
%         \} \\
%         \\
%         If the instruction requires some checks, please output the corresponding identifiers (such as [[A]], [[B]]). \\
%         Please do not output other identifiers if the corresponding checkers not needed. \\
%     \bottomrule
%     \end{tabular}
%     }
%     \end{adjustbox}
%     \caption{Our prompt for the router, where the \{instruction\} part varies based on the input. }
%     \label{tab:planner}
% \end{table*}

% \begin{table*}
%     \centering
%     \small
%     \begin{adjustbox}{max width=1\linewidth}
%     {
%     \begin{tabular}{p{\linewidth}}
%     \toprule
%     % \textbf{Prompt For Difference Proposal} \\
%     % \midrule
%     \textbf{Prompt For Difference Proposal} \\
%         \text{[Answers]} \\
%         \{formatted\_answers\} \\
%         \\
%         \text{[Your Task]} \\
%         Given the above responses, please identify and summarize one key points of contradiction or inconsistency between the claims. \\
%         \\
%         \text{[Requirements]} \\
%         1. Return a Python list containing only the most significant differences between the two answers. \\
%         2. Do not include any additional explanations, only output the list. \\
%         3. If there are no inconsistencies, return an empty list. \\
%     \midrule
%     \textbf{Prompt For Query Generation} \\
%     \text{[Original question that caused the inconsistency]} \\
%         \{instruction\} \\
% \\
%         \text{[Inconsistencies]} \\
%         \{inconsistencies\} \\
%         \\
%         \text{[Your Task]} \\
%         To resolve the inconsistencies, We need to query search engine. For each contradiction, please generate a corresponding query that can be used to retrieve knowledge to resolve the contradiction.  \\
%         \\
%         \text{[Requirements]} \\
%         1. Each query should be specific and targeted, aiming to verify or disprove the conflicting points.  \\
%         2. Provide the queries in a clear and concise manner, returning a Python list of queries corrresponding to the inconsistencies. \\
%         3. Do not provide any additional explanations, only output the list. \\
%         \midrule
%     \textbf{Prompt For Verification} \\
%     Evaluate which of the two answers is more factual based on the supporting information. \\
%         \text{[Support knowledge sources]}: \\
%         \{supports\} \\
%         \\
%         \text{[Original Answers]}: \\
%         \{formatted\_answers\} \\
%         \\
%         \text{[Remeber]} \\
%         For each answer, provide a score between 1 and 10, where 10 represents the highest factual accuracy. Your output should only consist of the following: \\
%         Answer A: [[score]] (Wrap the score of A with [[ and ]]) \\
%         Answer B: <<score>> (Wrap the score of B with << and >>) \\
%         Please also provide a compact explanation. \\
%     \bottomrule
%     \end{tabular}
%     }
%     \end{adjustbox}
%     \caption{Our prompt for assessing factuality in verification agents, with the \{formatted\_answers\}, \{supports\}, \{inconsistencies\}, \{instruction\} and \{supports\} parts varying based on the input. }
%     \label{tab:factuality_agent}
% \end{table*}


% \begin{table*}
%     \centering
%     \small
%     \begin{adjustbox}{max width=1\linewidth}
%     {
%     \begin{tabular}{p{\linewidth}}
%     \toprule
%     % \textbf{Prompt For Difference Proposal} \\
%     % \midrule
%     \textbf{Prompt For Constraint Parsing} \\
%        You are an expert in natural language processing and constraint checking. Your task is to analyze a given instruction and identify which constraints need to be checked. \\
%         \\
%         The `instruction' contains a specific task query along with several explicitly stated constraints. Based on the instructions, you need to return a list of checker names that should be applied to the constraints. \\
%         \\
%         Task Example: \\  
%         Instruction: Write a 300+ word summary of the Wikipedia page ``https://en.wikipedia.org/wiki/Raymond\_III,\_Count\_of\_Tripol''. Do not use any commas and highlight at least 3 sections that have titles in markdown format, for example, *highlighted section part 1*, *highlighted section part 2*, *highlighted section part 3*.\\
%         Response: \\
%         NumberOfWordsChecker: 300+ word \\
%         HighlightSectionChecker: highlight at least 3 sections that have titles in markdown format\\
%         ForbiddenWordsChecker: Do not use any commas \\
%         \\
%         Task Instruction: \\
%         \{instruction\} \\
%         \\
%         \#\#\# Your task: \\
%         - Generate the appropriate checker names with corresponding descriptions from the original instruction description. \\
%         - Return the checker names with their descriptions separated by `\textbackslash n'  \\
%         - Focus only on the constraints explicitly mentioned in the instruction (e.g., length, format, specific exclusions).  \\
%         - Do **not** generate checkers for the task query itself or its quality. \\
%         - Do **not** infer or output constraints that are implicitly included in the instruction (e.g., general style or unstated rules). \\
%         - Each checker should be responsible for checking only one constraint. \\
%     \midrule
%     \textbf{Prompt For Code Generation} \\
%     You are tasked with implementing a Python function `check\_following' that determines whether a given `response' satisfies a constraint defined by a checker. The function should return `True' if the constraint is satisfied, and `False' otherwise. \\
% \\
%         \text{[Instruction to check]}: \\
%         \{instruction\} \\
% \\
%         \text{[Specific Checker and Description]}: \\
%         \{checker\_name\} \\
% \\
%         Requirements: \\
%         - The function accepts only one parameter: `response' which is a Python string. \\
%         - The function must return a boolean value (`True' or `False') based on whether the `response' adheres to the constraint described by the checker. \\
%         - The function must not include any I/O operations, such as `input()' or `ArgumentParser'. \\
%         - The Python code for each checker should be designed to be generalizable, e.g., using regular expressions or other suitable techniques. \\
%         - Only return the exact Python code, with no additional explanations. \\
%     \bottomrule
%     \end{tabular}
%     }
%     \end{adjustbox}
%     \caption{Our prompt for assessing instruction-following in verification agents, with the \{instruction\} and \{checker\_name\} parts varying based on the input. }
%     \label{tab:if_agent}
% \end{table*}
\section{Experimental Details}
\label{sec:app_exp}
In this section, we provide a detailed description of the evaluation process, divided into three parts: the construction and distribution details of \ourdataset~\ref{sec:app_exp_ifbench}, the evaluation dataset settings~\ref{sec:app_exp_evaluation}, and additional experimental results~\ref{sec:app_exp_more_res}.

\subsection{\ourdataset Details}
\label{sec:app_exp_ifbench}

\ourdataset is a benchmark designed to evaluate reward models for multi-constraint instruction-following. The dataset comprises $444$ carefully curated instances, each containing: an instruction with $3$ to $5$ multi-constraints, a chosen response satisfying all constraints, and a rejected response violating specific constraints. All instances were constructed using \texttt{gpt-4o-2024-11-20} version through the following systematic pipeline.

\looseness=-1
\paragraph{Instruction Construction} We sampled $500$ initial instructions from the Open Assistant~\cite{kopf2023openassistant}. To ensure clarity and simplicity, we constrained the initial instruction length to $5$ to $20$ words. Subsequently, we employed GPT-4o to generate five distinct categories of constraints for each initial instruction. It then autonomously selected $3$ to $5$ constraints and paraphrased them into $1$ to $2$ sentences. The paraphrased constraints were integrated into the initial instruction. Finally, we use GPT-4o to evaluate the final instructions and filter out those with internal contradictions, resulting in a final set of $444$ instructions.

\looseness=-1
\begin{itemize}
    \item {\bf Content Constraints: } Specify conditions governing response, including topic focus, detail depth, and content scope limitations.
    \item {\bf Style Constraints: } Control linguistic characteristics such as tone, sentiment polarity, empathetic expression, and humor.
    \item {\bf Length Constraints: } Dictate structural requirements including word counts, paragraph composition, and specific opening phrases.
    \item {\bf Keyword Constraints: } Enforce lexical constraints through keyword inclusion, prohibited terms, or character-level specifications.
    \item {\bf Format Constraints: } Define presentation standards that include specific formats such as JSON, Markdown, or Python, along with section organization and punctuation rules.
\end{itemize}


\paragraph{Response Construction} For each instruction, we generated $8$ candidate responses using GPT-4o with temperature $1.0$ to maximize diversity. The chosen response was selected as the unique candidate satisfying all constraints through automated verification. Rejected responses were systematically selected to ensure balanced distributions of unsatisfied constraint (UC) categories and counts. As shown in Figure~\ref{fig:IFbench}, instances are stratified by difficulty: simple (\#UC$\geq$3), normal (\#UC$=$2), and hard (\#UC$=$1), with detailed information of UC category distributions. Specifically, (a) shows the distribution by the number of unsatisfied constraints in the rejected responses, where the sum of all parts equals the total number of instances. (b) presents the distribution by the categories of all unsatisfied constraints, where the sum of all parts equals the total number of unsatisfied constraints.

\begin{figure}[!ht]
    \centering
    \subfigure[]{
    \includegraphics[width=0.45\linewidth]{figures/IFBench_F1.pdf} }
    \subfigure[]{
    \includegraphics[width=0.45\linewidth]{figures/IFBench_F2.pdf} }
    \caption{Proportion (\%) of data in \ourdataset based on the number of unsatisfied constraints per instance and the categories of all unsatisfied constraints. }
    \label{fig:IFbench}
\end{figure}

\begin{figure*}
    \centering
    \includegraphics[width=0.98\linewidth]{figures/gpt4o_best_of_n.pdf}
    \caption{Best-of-n results (\%) on TriviaQA, IFEval, and CELLO using the base reward model ArmoRM and \ourmethod to search. ``+Oracle'' denotes using the oracle setting of \ourmethod as mentioned in \cref{sec:exp_analysis}.}
    \label{fig:gpt4o_best_of_n}
\end{figure*}


\subsection{Evaluation Details}
\label{sec:app_exp_evaluation}

% 表3中各个数据集的evaluation setting,metric 
\paragraph{Best-of-N} For the TriviaQA, we sample $500$ instances from the validation split in \texttt{rc.nocontext} version. The model is prompted to generate direct answers, and we report the exact match accuracies. For the IFEval, we report the average accuracy across the strict prompt, strict instruction, loose prompt, and loose instruction settings. For the CELLO, we report the average score based on the official evaluation script. All three tasks are conducted under a zero-shot setting.

\paragraph{DPO Training}
For MT-Bench and CELLO, we employ FastChat\footnote{\url{https://github.com/lm-sys/FastChat/tree/main/fastchat/llm_judge}} and the official evaluation script respectively, to conduct the evaluations and report the average scores.
For the other tasks, we use the \texttt{lm-evaluation-harness}\footnote{\url{https://github.com/EleutherAI/lm-evaluation-harness}} for evaluation. Specifically, we adopt a 5-shot setting for the MMLU and MMLU-Pro tasks, while using a zero-shot setting for TriviaQA and TruthfulQA. Notably, for TruthfulQA, we use the \texttt{truthfulqa\_gen} setting. 

\subsection{More Results on Best-of-N}
\label{sec:app_exp_more_res}
We conduct best-of-n search experiments using \texttt{gpt-4o-2024-11-20} as the policy model, with the results presented in Figure~\ref{fig:gpt4o_best_of_n}. The results demonstrate that \ourmethod significantly improves best-of-n performance compared to the base reward model ArmoRM, even when applied to a more powerful policy model than \ourmethod.




\begin{table*}
    \centering
    \small
    \begin{adjustbox}{max width=1\linewidth}
    {
    \begin{tabular}{p{\linewidth}}
    \toprule
    % \textbf{Prompt For Router} \\
    % \midrule
   Given the following instruction, determine whether the following check in needed. \\
    \\
        \text{[Instruction]} \\
        \{instruction\} \\
    \\
        \text{[Checks]} \\
       \{ 
            ``name'': ``constraint check'', 
            ``desp'': ``A `constraint check' is required if the instruction contains any additional constraints or requirements on the output, such as length, keywords, format, number of sections, frequency, order, etc.'', 
            ``identifier'': ``[[A]]'' 
        \}, 
        \{  
            ``name'': ``factuality check'', 
            ``desp'': ``A `factuality check' is required if the generated response to the instruction potentially contains claims about factual information or world knowledge.'', 
            ``identifier'': ``[[B]]'' 
        \} \\
        \\
        If the instruction requires some checks, please output the corresponding identifiers (such as [[A]], [[B]]). \\
        Please do not output other identifiers if the corresponding checkers not needed. \\
    \bottomrule
    \end{tabular}
    }
    \end{adjustbox}
    \caption{Our prompt for the router, where the \{instruction\} part varies based on the input. }
    \label{tab:planner}
\end{table*}

\begin{table*}
    \centering
    \small
    \begin{adjustbox}{max width=1\linewidth}
    {
    \begin{tabular}{p{\linewidth}}
    \toprule
    % \textbf{Prompt For Difference Proposal} \\
    % \midrule
    \textbf{Prompt For Difference Proposal} \\
        \text{[Answers]} \\
        \{formatted\_answers\} \\
        \\
        \text{[Your Task]} \\
        Given the above responses, please identify and summarize one key points of contradiction or inconsistency between the claims. \\
        \\
        \text{[Requirements]} \\
        1. Return a Python list containing only the most significant differences between the two answers. \\
        2. Do not include any additional explanations, only output the list. \\
        3. If there are no inconsistencies, return an empty list. \\
    \midrule
    \textbf{Prompt For Query Generation} \\
    \text{[Original question that caused the inconsistency]} \\
        \{instruction\} \\
\\
        \text{[Inconsistencies]} \\
        \{inconsistencies\} \\
        \\
        \text{[Your Task]} \\
        To resolve the inconsistencies, We need to query search engine. For each contradiction, please generate a corresponding query that can be used to retrieve knowledge to resolve the contradiction.  \\
        \\
        \text{[Requirements]} \\
        1. Each query should be specific and targeted, aiming to verify or disprove the conflicting points.  \\
        2. Provide the queries in a clear and concise manner, returning a Python list of queries corrresponding to the inconsistencies. \\
        3. Do not provide any additional explanations, only output the list. \\
        \midrule
    \textbf{Prompt For Verification} \\
    Evaluate which of the two answers is more factual based on the supporting information. \\
        \text{[Support knowledge sources]}: \\
        \{supports\} \\
        \\
        \text{[Original Answers]}: \\
        \{formatted\_answers\} \\
        \\
        \text{[Remeber]} \\
        For each answer, provide a score between 1 and 10, where 10 represents the highest factual accuracy. Your output should only consist of the following: \\
        Answer A: [[score]] (Wrap the score of A with [[ and ]]) \\
        Answer B: <<score>> (Wrap the score of B with << and >>) \\
        Please also provide a compact explanation. \\
    \bottomrule
    \end{tabular}
    }
    \end{adjustbox}
    \caption{Our prompt for assessing factuality in verification agents, with the \{formatted\_answers\}, \{supports\}, \{inconsistencies\}, \{instruction\} and \{supports\} parts varying based on the input. }
    \label{tab:factuality_agent}
\end{table*}


\begin{table*}
    \centering
    \small
    \begin{adjustbox}{max width=1\linewidth}
    {
    \begin{tabular}{p{\linewidth}}
    \toprule
    % \textbf{Prompt For Difference Proposal} \\
    % \midrule
    \textbf{Prompt For Constraint Parsing} \\
       You are an expert in natural language processing and constraint checking. Your task is to analyze a given instruction and identify which constraints need to be checked. \\
        \\
        The `instruction' contains a specific task query along with several explicitly stated constraints. Based on the instructions, you need to return a list of checker names that should be applied to the constraints. \\
        \\
        Task Example: \\  
        Instruction: Write a 300+ word summary of the Wikipedia page ``https://en.wikipedia.org/wiki/Raymond\_III,\_Count\_of\_Tripol''. Do not use any commas and highlight at least 3 sections that have titles in markdown format, for example, *highlighted section part 1*, *highlighted section part 2*, *highlighted section part 3*.\\
        Response: \\
        NumberOfWordsChecker: 300+ word \\
        HighlightSectionChecker: highlight at least 3 sections that have titles in markdown format\\
        ForbiddenWordsChecker: Do not use any commas \\
        \\
        Task Instruction: \\
        \{instruction\} \\
        \\
        \#\#\# Your task: \\
        - Generate the appropriate checker names with corresponding descriptions from the original instruction description. \\
        - Return the checker names with their descriptions separated by `\textbackslash n'  \\
        - Focus only on the constraints explicitly mentioned in the instruction (e.g., length, format, specific exclusions).  \\
        - Do **not** generate checkers for the task query itself or its quality. \\
        - Do **not** infer or output constraints that are implicitly included in the instruction (e.g., general style or unstated rules). \\
        - Each checker should be responsible for checking only one constraint. \\
    \midrule
    \textbf{Prompt For Code Generation} \\
    You are tasked with implementing a Python function `check\_following' that determines whether a given `response' satisfies a constraint defined by a checker. The function should return `True' if the constraint is satisfied, and `False' otherwise. \\
\\
        \text{[Instruction to check]}: \\
        \{instruction\} \\
\\
        \text{[Specific Checker and Description]}: \\
        \{checker\_name\} \\
\\
        Requirements: \\
        - The function accepts only one parameter: `response' which is a Python string. \\
        - The function must return a boolean value (`True' or `False') based on whether the `response' adheres to the constraint described by the checker. \\
        - The function must not include any I/O operations, such as `input()' or `ArgumentParser'. \\
        - The Python code for each checker should be designed to be generalizable, e.g., using regular expressions or other suitable techniques. \\
        - Only return the exact Python code, with no additional explanations. \\
    \bottomrule
    \end{tabular}
    }
    \end{adjustbox}
    \caption{Our prompt for assessing instruction-following in verification agents, with the \{instruction\} and \{checker\_name\} parts varying based on the input. }
    \label{tab:if_agent}
\end{table*}

\end{document}
