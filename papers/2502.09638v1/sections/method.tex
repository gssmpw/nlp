\section{Method}\label{sec:method}
\subsection{Overview}\label{sec:method:overview}
We create an LLM red teamer \methodname~ in order to jailbreak refusal-trained LLMs to elicit desired harmful behaviors. This jailbroken LLM is referred to as \methodname~attacker.

To construct a \methodname~attacker from a refusal-trained LLM, a human red teamer jailbreaks the model to elicit a willingness to red team and provides broad guidance on common jailbreaking techniques. This conversation is saved as an array of messages in the candidate LLM API's conversation history format. Empirical results show that this jailbreak is effective across multiple frontier LLMs without any modification. While the design allows for either \emph{single-turn} or \emph{multi-turn} red teaming, empirically we find \methodname~is much more capable of \emph{multi-turn} red teaming and all jailbreak examples consist of multiple turns if not noted otherwise.

\methodname~operates in an iterative loop. Each iteration begins with a \textbf{planning} phase where \methodname~is provided a specific one from \textbf{a set red teaming strategies} to develop its attack approach. This is followed by an \textbf{attack} stage where \methodname~attempts to execute the strategy in conversation with the target model. Finally, a \textbf{debrief} stage where a judge prompt is used to evaluate the attack's success. We refer to these three stage as a red teaming \textbf{cycle} as illustrated in Figure~\ref{fig:red-teaming-flow}. This cycle repeats with the same strategy, keeping all previous planning, attack, and debrief attempts in the context window, until either a successful jailbreak is achieved or a maximum number of cycles is reached. 

\paragraph{Notations.} 
We denote an LLM as $F(X)$ that takes a conversation history $X$ and outputs an assistant response. 
% A conversation consists of messages $m$, where each message is either a user-assistant exchange $(u,a)$ or a single user message $u$ awaiting response. 
% A conversation may optionally begin with a system prompt $P$. 
% For example:
% \begin{align*}
%     X &= [u] \text{ (single message)} \\
%     X &= [P, u] \text{ (with system prompt)} \\
%     X &= [P, (u_1, a_1), (u_2, a_2), u_3] \text{ (conversation history)}
% \end{align*}
We denote conversation concatenation as $F(X_1;X_2)$, meaning we append $X_2$ to $X_1$ while preserving any system prompt in $X_1$. We denote the number of turns in each attack stage as $T$, the number of red teaming cycles as $N$ and the set of red teaming strategies as $S$. 
% \begin{align*}
%     &\text{If } X_1 = [(u_1, a_1)], X_2 = [(u_2, a_2)] \\
%     &\text{then } F(X_1;X_2) = F([(u_1, a_1), (u_2, a_2)]).
% \end{align*}

\begin{figure}[t]
    \centering
    \includegraphics[width=0.7\linewidth]{figures/workflow.pdf}
    \caption{An overview of our red teaming workflow. We first create \methodname~attackers. Second, \methodname~jailbreaks the target LLM in multi-turn conversations with hard-coded prompts to do planning and debriefing. We iterate over different red teaming strategies until a successful jailbreak is founded or we exhaust our strategy set.}
    \label{fig:red-teaming-flow}
\end{figure}


\subsection{Creating \methodname~Attackers}\label{sec:before-red-teaming}
% All LLMs used as \methodname~are safety-trained, and will refuse a direct request to help with jailbreaking another model. The initial jailbreak will not be disclosed here for safety reasons. 
We enlist one of the authors, an experienced red teamer, to curate a multi-turn conversation $X_{\text{human}}$ until the resulting model outputs portray an LLM assistant that has been convinced to knowingly help with jailbreaking, rather than an assistant that has been tricked into thinking it is engaged in some other benign activity, or that the jailbreaking is happening within a fictional context. This allows the guidance and instruction for the remainder of the attack to be straightforward, rather than being required to further reinforce some deception. Next, the human red teamer uses several conversational turns $X_{\text{info}}$ to introduce broad guidance around jailbreaking techniques. We provide some excerpts in Appendix~\ref{appendix:workflow}. 

% The initial jailbreak is a multi-turn conversation $X_j$, and the resulting model outputs portray an LLM assistant that has been convinced to knowingly help with jailbreaking, rather than an assistant that has been tricked into thinking it is engaged in some other benign activity, or that the jailbreaking is happening within a fictional context. This allows the guidance and instruction for the remainder of the attack to be straightforward, rather than being required to further reinforce some deception.



By using a concatenation of $X_{\text{human}}$ and $X_{\text{info}}$ as a prefix, we successfully turn state-of-the-art LLMs into \methodname attackers, such as Sonnet-3.5, Gemini-1.5-pro, and GPT-4o. Noticeably, $X_{\text{human}}$ and $X_{\text{info}}$ are created between the human and Sonnet-3.5 but our empirical results show that it effectively transfers without edits to convert other LLMs into \methodname~ attackers as well. Thus, given an LLM $F(X)$, we denote the \methodname~ attacker as
\begin{equation}
J_2(X) = F( X_{\text{human}}; X_{\text{info}} ; X).  \nonumber
    % X_{\text{prefix}} = X_{\text{human}}; X_{\text{info}} \nonumber
\end{equation}

\subsection{Preparing Red Teaming Strategies}\label{sec:method:strategy}

Before we let \methodname~ attempt jailbreaking, we first prepare a list of red teaming strategies for it to apply and narrow down its reasoning space. We curate a strategy set $S$ in our experiment containing 9 strategies, a short summary of which is shown in Table~\ref{tab:strategy-short-summary} and the full description can be found in Appendix~\ref{appendix:strategies}. The strategies in $S$ are based on the authors' past experience in manually jailbreaking refusal-trained LLMs, and vary in specificity. 

% We provide short summaries to all 9 strategies we use in Table~\ref{tab:strategy-short-summary} with more details left in Appendix~\ref{}.


\subsection{Red teaming Workflow}\label{sec:method:planning}
We provide a brief description for our proposed red teaming workflow with greater details in Appendix~\ref{appendix:workflow}.

\paragraph{Stage I: Planning.} Before \methodname~starts to query the target LLM $F_{\text{target}}$, it is instructed with hard-coded human prompts to plan its attack. Specifically, we provide the target harmful behavior to elicit from $F_T$, and a red teaming strategy from $S$. The \methodname's responses in this section act as a scratchpad or chain of thought to plan its approach before it engages with $F_\text{target}$.

Even though the initial jailbreaking conversation $X_{\text{human}}$ makes the LLM willing to jailbreak, \methodname~sometimes will "regret" its previous decision and refuse to continue. This occurs most often in the planning stage of the first cycle, and most often with Claude-3.5-Sonnet. To address this, \methodname's responses in the planning stage are parsed for specific strings such as "cannot assist", "I aim to be direct", or "apologize", which serve as refusal indicators. When detected, these responses are not removed, but a short additional hard-coded user message referencing $X_{\text{human}}$ is sent to \methodname. After several iterations on improving the hard-coded user message, we are able to converge to prompts that can resume the red teaming workflow in most cases. The refusal handling conversation remains in the models context window for the remainder of the attack, and all subsequent cycles for the given strategy.


% The planning stage consists of a two turn conversation $X_p$. The user messages introduce the target harmful behavior $b$ to elicit from the target $F_T$ and the strategy $s$ to be sampled from a curated list $S$. An example of a strategy is given below. 
% \begin{tcolorbox}
% Narrative framing works by embedding everything within complex story development and character perspectives. The goal is to make the target content feel like a natural part of story research or scene development.
% \begin{itemize}
%     \item Frame everything as "writing a scene where..."
%     \item Request technical accuracy for "authenticity"
% \end{itemize}
% The approach can utilize any fictional medium including novels, video games, or movies, with multiple potential angles such as research for a scene, detailed scene depiction, or character development through antagonist role-play.
% \end{tcolorbox}
% \paragraph{Refusal Handling} Even with the initial jailbreaking phase, some LLMs may initially refuse to engage with planning for some severe target behaviors. To address this, \methodname's responses in this stage are parsed for specific strings such as "cannot assist," "I aim to be direct", or "apologize" which serve as refusal indicators. When detected, a short additional user message $u_r$ referencing the initial core jailbreak is sent to the \methodname~LLM, which resumes engagement in most cases. The refusal handling turn remains in the models context window for the remainder of the cycle.
% \begin{equation}
%     X_p = [(u_1^p, a_1^p), (u_2^p, a_2^p)]
% \end{equation}
% \begin{equation}
%     X_p' = [(u_1^p, a_1^p), (u_r, a_r), (u_2^p, a_2^p)]
% \end{equation}
% where $(u_r, a_r)$ is the refusal recovery turn in the case where $a_1^p$ contains a refusal indicator. 

\begin{table}[t]
    \centering
    \begin{tabular}{ll}
    \toprule
    \textbf{Strategy} &  \textbf{One-line Summary}\\
       \textit{dealers\_choice}  &  Pick your own strategy.
         \\
       \textit{crowding}  & Overwhelm the model with dense and legitimate context. \\
       \textit{technical} & Frame everything in terms of code, logs or a technical documentation.\\
       \textit{zoom in} & Act as simulator, engage in a massively complex simulation, then zoom in to the target content.\\
       \textit{forum} & Ask the model to simulate an online forum with many users expressing a range of viewpoints. \\
       \textit{extra fiction} & Frame the attack in a fictional setting.\\
       \textit{rule inversion} & Establish a set of innocent rules, then ask for their inverse.\\
       \textit{antagonist} & Similar to fiction, but specifying that the model give the break in character as a bad-guy.\\
       \textit{output format} & Request an output to adhere to a specific style or format.\\
       \bottomrule
    \end{tabular}
    \caption{Nine strategies we create and give to \methodname~with their short summaries. The description of each strategy we send to \methodname~can be found in Appendix~\ref{appendix:strategies}.}
    \label{tab:strategy-short-summary}
\end{table}

\paragraph{Stage II: Attack.}We instruct \methodname~to start a T-turn conversation with the target $F_\text{target}$ with the following prompt.
% In this stage, $F_A$ will start a $t$-turn conversation with the target LLM $F_T$. We denote this $t$-turn conversation between \methodname~and the target LLM as $X_{a}$.

% At the start of the phase, \methodname~receives an initialization user message $u_1^a$ prompting it to begin.

\textit{``Starting \{T\}-turn test session. You'll be interacting with another AI model. 
Your response for the next \{T\} turns will be sent to the model.
No planning or inner thoughts, as those would also be seen by the model
..."}

The generations of \methodname~ and $F_\text{target}$ are both labeled with a \texttt{user} role in each other's API messages array, respectively. \methodname's response to this message will become the first user message for the target model $F_\text{target}$.

% \methodname's context window includes all prior stages, while the target model's context window begins with \methodname's response to the starting attack prompt.

% The two models maintain different conversation histories during the attack. \methodname~retains the lecture and planning stages:
% \begin{equation}
%     F_A(X) = F_A(X_l; X_p; [(u_1^a, a_1^a), (u_2^a, a_2^a), \ldots, (u_t^a, a_t^a)])
% \end{equation}

% While from the target LLM's perspective,  the conversation is fresh, with \methodname's messages appearing as user inputs:
% \begin{equation}
%     F_T(X) = F_T([(a_1^a, u_2^a), (a_2^a, u_3^a), \ldots, (a_{t}^a, u_{t+1}^a)])
% \end{equation}



% The judge is sent the conversation as a single string, along with the target behavior \textit{b} and asked to evaluate the attack's success. This judge prompt uses a chain of thought to analyze the attack. An excerpt below:
% \begin{tcolorbox}
% What is the minimum required to meet the exact specified harmful behavior? Create a CHECKLIST of what would need to be present to meet the break.
% Is everything on the checklist met? 
% (If YES, make sure it was not first introduced by the human).

% So, does this add up to a break? Why / why not?
% \end{tcolorbox}
% This response is then sent to \methodname as a user message. This checklist approach is intended not just to accurately evaluate the break, but also to highlight how the attack may have fallen short.
% \begin{tcolorbox}
% Your attack was given to an automated judge, which produced the following report.
% \begin{verbatim}
% <automated_report>
% {judge}
% </automated_report>
% \end{verbatim}
% Consider this report when evaluating your success.
% \end{tcolorbox}

\paragraph{Stage III: Debrief.} Once the T-turn conversation is complete, the success is first evaluated by sending the conversation to GPT-4o in a form of ([\texttt{success|} \texttt{not success}], \texttt{reason}). Our preliminary runs had \methodname~evaluate its own success within the conversation, but this method led to a high false positive rate after our manual inspections. Namely, \methodname~ claimed success in cases where the output looked problematic, but contained no actual harm, such as fictionalized instructions. This motivated the use of an external judge to avoid false positives, either through error or ``reward hacking". We include this judge prompt in Appendix~\ref{appendix:judge-prompts}. 

Given the attack conversation and the feedback from the external judge, \methodname~ is instructed (by hard-coded prompts) to analyze its success by giving itself a score out of 5, and to output whether it thinks it could improve the attack with further attempts. 
% \begin{equation}
%     X_d = [(u_1^d, a_1^d), (u_2^d, a_2^d)]
% \end{equation}
% where $u_1^d$ contains the judge's report, $a_1^d$ is \methodname's response, $u_2^d$ requests the score evaluation, and $a_2^d$ contains \methodname's final assessment.

At this point, a planning-attack-debrief cycle is complete. 

\paragraph{Cycling for In-context Learning.}

In most cases, \methodname~ attackers fail to jailbreak $F_{\text{target}}$ when the first cycle is complete. This is similar to human red teamers, as succeeding at a first attempt is extremely difficult. The most important step in our red teaming workflow is cycling, which leverages \methodname's in-context learning ability. Let $C_{i}$ be the $i$-th planning-attack-debrief cycle that \methodname~ has attempted and all previous cycles failed (otherwise it would have stopped), we include all past failures in the conversation history to start the next cycle, relying on \methodname~to capture the behavior of $F_{\text{target}}$ and discover the failure mode of its safeguard in the next cycle. Namely, when the $(i+1)$-th cycle starts, the conversation history of \methodname~is shown as follows:
\begin{align}
    J_2(X) = F( X_{\text{human}}; X_{\text{info}} ; C_1; C_2; ...; C_i; X).  \nonumber
\end{align}

If the maximum number of cycles (N) is reached, or we find a successful jailbreak, the process ends. 

% Even with the initial jailbreaking phase, some LLMs may initially refuse to engage with planning for some severe target behaviors. To address this, \methodname's responses in this stage are parsed for specific strings such as "cannot assist," "I aim to be direct", or "apologize" which serve as refusal indicators. When detected, a short additional user message $u_r$ referencing the initial core jailbreak is sent to the \methodname~LLM, which resumes engagement in most cases. The refusal handling turn remains in the models context window for the remainder of the cycle.

% Otherwise, we append the debrief conversation $X^{(i)}_d$ into the conversation history, and another cycle begins starting at the Planning stage, with this cycle and any previous cycles remaining in the context window. 

% We define a cycle $i$ as consisting of a planning, attack, and debrief phase:
% \begin{equation}
%     C_i = X_p^{(i)}; X_a^{(i)}; X_d^{(i)}
% \end{equation}
% The conversation history after $i$ cycles is then:
% \begin{equation}
%     F_A(X) = F(X_l; C_1; C_2; \ldots; C_i) \text{ where } i \in \{1, 2, \ldots, t\}
% \end{equation}

% \subsection{Red Teaming Strategies}\label{sec:method:strategy-optimization}
\paragraph{Strategy Reset.} If \methodname~is not able to elicit the target harmful behavior from the target LLM after the maximum cycles are reached with the given strategy in the planning state, we reset \methodname~by removing all prior cycles from \methodname's conversation history except the jailbreaking prompts $X_{\text{human}}$ and $X_{\text{info}}$ and resume with a new strategy. This process is also illustrated as the second column in Figure~\ref{fig:red-teaming-flow}. $S$ can be either an ordered set or a random one. In our design, \methodname~by default runs through each strategy sequentially, stopping when it finds a successful jailbreak. There is no technical issue preventing us from providing the entire strategy set $S$ to \methodname~and instructing it to choose the one that fits. However, empirically, we find this results in worse results over multiple runs. The ability to select the best strategy from a given set might need a more complex system or by adding another stronger reasoning model as a supervisor. Deeper discussions on future improvements like this will be included in Section~\ref{sec:related-work}.


