
\section{FORMATIVE STUDY}
We conducted a formative study to gain deeper insights into current concept designers' workflows and the challenges they face using traditional and AI-based ideation tools.
% In this work, we decided to focus on environment concept design as it involves complex spatial and visual elements that demand high levels of creativity, consistency, and attention to detail. 
\subsection{Participants}
We recruited 22 professional environment concept designers (15 males, 7 females; ages 23 to 45) across three studies. Each participant was assigned a unique ID. Participants were recruited through personal referrals and directly contacting studios by email to request collaboration. Detailed participant information, including their participation in each study, is provided in Table ~\ref{tab:demographics}. We will highlight the relevant details of the participants in each study.

In the formative study, we worked with 12 environment concept designers (P1-P12) from various industries, including Animation (P1, P4, P6, P8-P9), Game (P5, P10-P11), Art Outsourcing (P3, P12), and Freelancing (P2), with 3 to 15 years of experience (mean = 7.7, SD = 4.6). Participation in the study was voluntary, and uncompensated.

\begin{table*}[h!]
\centering
\small
\begin{tabular}{|c|c|c|c|c|c|}
\hline
\textbf{ID} & \textbf{Years of Experience} & \textbf{Industry} & \textbf{Formative} & \textbf{Summative} & \textbf{Field Study} \\ \hline
1 & 3 & Animation, Films, TV shows & \checkmark & \checkmark & \checkmark \\ \hline
2 & 4 & Freelancing & \checkmark & \checkmark & \\ \hline
3 & 7 & Art Outsourcing & \checkmark & \checkmark & \checkmark \\ \hline
4 & 3 & Animation, Films, TV shows & \checkmark & \checkmark & \\ \hline
5 & 3 & Game & \checkmark & \checkmark & \checkmark \\ \hline
6 & 5 & Animation, Films, TV shows & \checkmark & \checkmark & \\ \hline
7 & 3 & Game & \checkmark & & \\ \hline
8 & 8 & Animation, Films, TV shows & \checkmark & & \\ \hline
9 & 12 & Animation, Films, TV shows & \checkmark & & \\ \hline
10 & 15 & Game & \checkmark & & \\ \hline
11 & 13 & Game & \checkmark & & \\ \hline
12 & 14 & Art Outsourcing & \checkmark & & \\ \hline
13 & 5 & Art Outsourcing & & \checkmark & \checkmark \\ \hline
14 & 8 & Game & & \checkmark & \\ \hline
15 & 12 & Game & & \checkmark & \\ \hline
16 & 3 & Animation, Films, TV shows & & \checkmark & \checkmark \\ \hline
17 & 2 & Animation, Films, TV shows & & \checkmark & \checkmark \\ \hline
18 & 2 & Animation, Films, TV shows & & \checkmark & \checkmark \\ \hline
19 & 1 & Animation, Films, TV shows & & \checkmark & \\ \hline
20 & 1 & Animation, Films, TV shows & & \checkmark & \\ \hline
21 & 5 & Freelancing & & \checkmark & \\ \hline
22 & 11 & Game & & \checkmark & \checkmark \\ \hline
\end{tabular}
\caption{Demographic Details of Participants}
\Description{This table provides detailed demographic information about the 22 professional environment concept designers who participated in the study. Participants were identified by a unique ID and categorized based on their years of experience, industry affiliation, and participation in the formative study, summative study, and field study phases. The participants had a wide range of professional experience, from early-career designers with 1 year of experience to seasoned professionals with up to 15 years in the field. Detailed participant breakdown: ID 1: 3 years of experience, works in Animation, Films, and TV Shows, participated in the formative study. ID 2: 4 years of experience, works as a Freelancer, participated in both formative and summative studies. ID 3: 7 years of experience, works in Art Outsourcing, participated in formative and summative studies. ID 4: 3 years of experience, works in Animation, Films, and TV Shows, participated in the formative study. ID 5: 3 years of experience, works in Game Design, participated in the formative study. ID 6: 5 years of experience, works in Animation, Films, and TV Shows, participated in the summative study. ID 7: 3 years of experience, works in Game Design, participated in the summative study. ID 8: 8 years of experience, works in Animation, Films, and TV Shows, participated in formative and summative studies. ID 9: 12 years of experience, works in Animation, Films, and TV Shows, participated in formative and summative studies. ID 10: 15 years of experience, works in Game Design, participated in the summative study. ID 11: 13 years of experience, works in Game Design, participated in formative and summative studies. ID 12: 14 years of experience, works in Art Outsourcing, participated in the summative and field studies. ID 13: 5 years of experience, works in Art Outsourcing, participated in formative, summative, and field studies. ID 14: 8 years of experience, works in Game Design, participated in formative and summative studies. ID 15: 12 years of experience, works in Game Design, participated in formative and summative studies. ID 16: 3 years of experience, works in Animation, Films, and TV Shows, participated in the summative and field studies. ID 17: 2 years of experience, works in Animation, Films, and TV Shows, participated in the formative study. ID 18: 2 years of experience, works in Animation, Films, and TV Shows, participated in the formative study. ID 19: 1 year of experience, works in Animation, Films, and TV Shows, participated in the formative study. ID 20: 1 year of experience, works in Animation, Films, and TV Shows, participated in the formative study. ID 21: 5 years of experience, works as a Freelancer, participated in formative and field studies. ID 22: 11 years of experience, works in Game Design, participated in formative and field studies.}
\label{tab:demographics}
\end{table*}



\subsection{Study Procedure}
Each participant took part in a 1-2 hour interview covering three main topics: 1) Their typical design workflow, 2) Past design projects, and 3) Current AI tool usage. We asked participants to prepare three specific projects: their most recent project, a typical project, and the most challenging project in their work experience. For each project, we explored the design task, the procedures they followed, and their overall approach. This included discussing the use of design tools, methods for research and brainstorming, reference materials collected for various design elements, presentation of sketches to directors or clients, and the challenges encountered throughout the process.

\subsection{Findings}
To analyze the data, we organized and summarized the transcribed interview recordings, and one of the authors, with prior experience as a professional concept designer, developed a coding framework to identify key themes for thematic analysis. Two art directors reviewed the coding framework from an animation studio and an art-outsourcing studio, each managing 15 and 40 concept designers, respectively.
Thematic analysis was discussed collaboratively among a team of three people to ensure consensus and validity. This process revealed patterns in concept designers’ research and brainstorming workflows, the purposes of the references they gathered, and the challenges they faced with traditional and AI design tools. 



\subsubsection{Challenges during researching}

% RR
Our participants employed a variety of tools during their research. They used search engines like Google\footnote{Google, www.google.com} to gather information and chatbots like ChatGPT\footnote{ChatGPT, https://chatgpt.com/} to explore topics in depth. For initial visual exploration and reference gathering, they relied on online platforms such as Pinterest\footnote{Pinterest, www.pinterest.com}, portfolio websites like Artstation\footnote{Artstation, https://www.artstation.com/}, and image databases like Shutterstock\footnote{Shutterstock, https://www.shutterstock.com/}. Most participants mentioned that this process is straightforward when the briefing is clear, and the themes are familiar, like “\textit{cyberpunk streets}” (P1) or “\textit{Japanese shrines}” (P3). However, when specifications were vague, or the topic was less common—a frequent challenge in environment concept design—they reported greater difficulty in finding relevant information and references. These observations align with findings from previous literature~\cite{80lv2020, bigbadWorld2015, interview2020, iterationandreference2023}. 

This challenge usually arises from two main issues. First, designers often struggle to find search queries and references that align with their design intentions. “\textit{The client asked me to design an internet world for a celebrity, showcasing her popularity. I spent half a day trying keywords like ‘digital world,’ ‘internet world,’ and ‘matrix world’ on Pinterest, but found nothing suitable}” (P1). “\textit{Often, I remember having seen a similar reference before, but now I don't know how to find it}” (P2). Second, traditional search tools often fail to provide sufficient references for unique design topics. One participant noted, “\textit{I was asked to design a Grand Mayan market and a Mayan ballcourt, 80\% based on history. I couldn’t find any relevant design work, and the references on Pinterest were fragmented and lacked useful information}” (P4). Another added, “\textit{We frequently get tasks that require blending different styles and themes, but it's hard to find similar concept art online}” (P3).

\subsubsection{Challenges during brainstorming}
Concept designers often need to create 3–5 design variations per environment, a task that becomes challenging for uncommon designs \cite{bigbadWorld2015}. Most participants noted they typically have only “\textit{half}” (P6)  to “\textit{one}” (P12) day for idea exploration, leaving little time for deeper creative development. “\textit{I need to reserve the entire afternoon for sketching, leaving only the morning for research and exploring different possibilities}” (P4). Designers rely on visual exploration for inspiration, but time constraints and the challenges outlined in the previous section often limit their access to diverse references, restricting creative ideas and exploration. For example, “\textit{I was asked to design an Aztec village with three variations, but the architectural references I found on Pinterest all looked quite similar. With a tight schedule, the final designs I proposed ended up being somewhat alike}” (P9).

Designers spend significant effort in creating design variations, especially for complex design specifications. “\textit{A recent project involved designing a Chinese Steampunk world with realistic and plausible designs. With no existing references, I spent an entire day just sketching one building}” (P12). Generating innovative designs or integrating unique elements into a cohesive vision is another common challenge. “\textit{I often spend a great deal of time contemplating what elements to add to enhance the richness of the scene}” (P5). Also, they often have to try multiple combinations of design elements from references to create a suitable outcome. As one participant explained, “\textit{I often fill an entire A3 canvas with sketches to explore various possible design combinations}” (P2). Furthermore, within the same project, designers are usually tasked with creating multiple scenes within a shared setting, such as “\textit{creating various architectures and their interiors within the same game environment}”(P3). 


\subsubsection{Problems with current AI design tools}
To address these challenges in traditional workflows, many studios and designers have started integrating GenAI into their processes~\cite{boucher2024resistance, ko2023large, filmhandbook2024, vimpari2023adapt}. In our study, all participants had experience using AI design tools, with 9 of them already integrating these tools into their workflow. The AI tools used included Midjourney\footnote{MidJourney, https://www.midjourney.com/}, Stable Diffusion\footnote{StableDiffusion, https://stablediffusionweb.com/}, DALL-E\footnote{Dall-E, https://openai.com/index/dall-e-3/}, and more advanced systems like Comfy UI~\cite{comfyui}. However, we found no consistent usage patterns. 6 participants (P1, P3-P4, P8, P11-12) primarily used these tools for image generation, formulating prompts based on concrete ideas and modifying them if the results did not align with their vision. Only 3 participants (P5, P9, P10) used the tools for ideation, providing simple inputs to explore topics. We identified several reasons why current AI design tools are not yet effective ideation tools for concept designers.

Most AI design tools, like Stable Diffusion and Midjourney, rely on text-based prompts that often require complex inputs, such as multiple keywords or lengthy descriptions ~\cite{mahdavi2024ai}. This contrasts with designers' typical workflow, which starts with simple keyword searches on platforms like Pinterest and progresses to image-based exploration. “\textit{As a concept designer, I don’t want to spend time crafting precise prompts}” (P2). Additionally, crafting a suitable prompt often requires a clear idea in advance, making it difficult to use during the initial ideation. “\textit{We usually use this tool to generate images only when we already have a clear idea in mind}” (P3). Many participants noted they struggled to create prompts that generated the desired outcomes. “\textit{I tried modifying the prompt in MidJourney several times, but I still couldn’t get what I wanted}” (P6). Furthermore, most image-generation AI tools struggle to produce diverse results from similar input, limiting their usefulness for breadth idea exploration. “\textit{I have to re-craft the prompt to get something noticeably different}” (P9). “\textit{I feel like everything the AI generates looks pretty much the same every time, similar compositions, styles, and often stereotypical elements}” (P1).

Concept designers require grounded and accurate information to support their designs. However, AI hallucinations pose a significant barrier, discouraging designers from adopting AI tools. “\textit{I usually avoid using AI-generated images as reference pictures because relying on incorrect content could lead to even worse outcomes}” (P6). Additionally, AI design tools often do not provide enough information in the generated outputs. A common issue is the lack of detail, particularly in the structure of objects, making it difficult for designers to identify visual elements for further reference. As one participant noted, “\textit{The content generated by AI is usually only useful to me as a mood reference because the details are often a complete mess}” (P3). Although some tools offer detailed prompts based on simple inputs, they can be challenging to interpret, such as “\textit{AI-generated images often include some interesting elements, but I don’t know what they are. As a result, I can’t incorporate them into my design}” (P1). Moreover, the generated images often do not align accurately with the prompts, “\textit{AI-generated images often include additional elements that are not specified in the prompt}” (P10).

The iterative process is key to achieving a great design for concept designers~\cite{iterationandreference2023}. However, AI design tools often lack the control and flexibility needed to refine output after generation. As one participant noted, “\textit{I only wanted to change the style of one building, but the entire image ended up changing}” (P12). Another shared, “\textit{The generated results often make me question how my changes to the prompt are actually affecting the outcome}” (P3). Moreover, the lack of detailed information accompanying AI-generated images hinders further ideation, “\textit{The AI-generated images contain many visual elements, but without information about them, I don’t know how to modify or adjust them}” (P5).

These factors combined make current AI image-generation tools difficult to use for visual idea exploration and challenging to integrate into a concept designer's workflow.

% \subsubsection{Challenges during researching}
% After receiving a briefing, designers typically begin by researching to gain a clear understanding of the topic \cite{bigbadWorld2015}. They extract key terms from the briefing and use online image databases like Pinterest\footnote{Pinterest, www.pinterest.com} and search engines for initial visual exploration and deeper knowledge, which is crucial for setting a design direction. This process is straightforward when the briefing is clear, and the themes are familiar, like cyberpunk streets or Japanese shrines. However, when specifications are vague, or the topic is uncommon—common in environment concept design—designers face greater difficulty in finding relevant information and references~\cite{bigbadWorld2015}. This challenge usually arises from two main issues:
% \begin{enumerate}
%     \item Designers often struggle to find search queries and references that align with their design intentions: “The client asked me to design an internet world for a celebrity, showcasing her popularity. I spent half a day trying keywords like ‘digital world,’ ‘internet world,’ and ‘matrix world’ on Pinterest, but found nothing suitable.” (P1) “Often, I remember having seen a reference before, but now I don't know how to find it.” (P2)
%     \item Traditional search tools often struggle to provide sufficient references for unique design topics: One participant noted, “I was asked to design a Grand Mayan market and a Mayan ballcourt, 80\% based on history. I couldn’t find any relevant design work, and the references on Pinterest were fragmented and lacked useful information.” (P4) Another added, “We frequently get tasks that require blending different styles and themes, but it's hard to find similar concept art online.” (P3)
% \end{enumerate}


% Concept designers are often required to provide three to five design variations for each environment, but generating diverse ideas can be challenging, particularly for uncommon design tasks \cite{bigbadWorld2015}. In our study, we conclude this difficulty arises from several factors:
% \begin{enumerate}
%     \item \textbf{Challenge for Creative Solutions:} Most participants noted they typically have only 0.5 to 1 day for idea exploration (P1-P7, P4-P3, P2). With such tight schedules, they rely on quickly found references, leaving little time for deeper creative development. Additionally, it is often difficult to come up with innovative designs or integrate unique elements into the overall design. Many participants mentioned that they usually spent a lot of time thinking about what objects should be added to the scene. (P1, P6, P7, P10, P3, P2)
%     \item \textbf{Insufficient Diverse References and Information:} Designers rely heavily on visual exploration for inspiration but often struggle to find enough diverse references to support multiple distinct designs, as discussed in Section 3.3.2.  This lack of diverse references can limit their ability to generate unique ideas and explore different creative directions.  “I was asked to design an Aztec village with three variations, but the architectural references I found on Pinterest all looked quite similar. With a tight schedule, the final designs I proposed ended up being somewhat alike.” (P9)
%     \item \textbf{Significant effort to create one design variation:} Some complex design specifications demand much more effort. “A recent project involved designing a Chinese Steampunk world with realistic and plausible designs. With no existing references, I spent an entire day just sketching one building.” (P12) Participants also noted they often have to try multiple combinations of design elements from references to create a suitable outcome. (P1, P4, P3). Additionally, they frequently need to design multiple scenes within the same setting (P1, P9-6, P3-11), such as creating multiple architectures and their interiors within the same game environment. (P3)
% \end{enumerate}

% \subsubsection{Problem of Current AI Design Tools}
% To address these challenges in traditional workflows, many studios and designers have started integrating GenAI into their processes.~\cite{boucher2024resistance, ko2023large, filmhandbook2024, vimpari2023adapt}. In our study, all participants had experience using AI design tools, with 9 of them already integrating these tools into their workflow. The AI tools used included Midjourney, Stable Diffusion, DALL-E, and more advanced systems like Comfy UI. However, we found no consistent usage patterns. 6 participants (P1, P8, P4, P11, P3-P12) primarily used these tools for image generation, formulating prompts based on concrete ideas and modifying them if the results didn’t align with their vision. Only 3 participants (P9, P10, P5) used the tools for ideation, providing simple inputs to explore topics. We identified several reasons why current AI design tools are not yet effective ideation tools for concept designers.
% \begin{enumerate}
%     \item \textbf{Difficulty Formulating Prompts:} Most AI design tools rely on text-based prompts, often requiring complex inputs—such as dozens of keywords in Stable Diffusion or lengthy descriptions in Midjourney—to produce desired results. This differs significantly from designers' usual workflow, which begins with simple keyword searches on platforms like Pinterest, followed by image-based exploration. Additionally, crafting a suitable prompt often requires a clear idea upfront, making it challenging to use during early ideation. Many participants noted they struggled to create prompts that generated the desired outcomes. (P1-P6, P8-P10, P5-P3, P2)
%     \item \textbf{Little Variation with Similar Prompt:} Most AI design tools struggle to produce diverse results from the same input, limiting their usefulness for breadth idea exploration. Four designers noted they had to frequently modify the prompt to see more variations. (P1, P9, P5, P3)
%     \item \textbf{Lacking Clear Detail Information:} AI design tools often lack sufficient detail in the generated images, particularly in the structure of objects, making it difficult for designers to identify visual elements needed for further reference searches. While the tools provide detailed prompts, they are often hard to interpret, and the generated image may not match the prompt accurately. Four participants noted that while AI often generates interesting visual elements that could enhance their designs, the lack of detailed information makes it difficult to find real-world references to support those elements. (P1, P10, P11, P3)
%     \item \textbf{Low Flexibility to Modify the Result:} AI design tools offer limited control and flexibility after image generation. Users can adjust the prompt or input an image for the system to modify, but fine-tuning specific elements is often not possible. Additionally, with minimal information or guidance provided, users struggle to determine how to proceed, hindering the depth of idea exploration. “The generated results often make me question how my changes to the prompt are actually affecting the outcome.” (P3) “The AI-generated images contain many visual elements, but without information about them, I don’t know how to modify or adjust them.” (P5)
% \end{enumerate}
% These factors, taken together, make current AI image-generation tools less ideal for efficient visual idea exploration.

\begin{figure*}
    \centering
    \includegraphics[width=1\linewidth]{figures/04_Type_of_references.png}
    \caption{Based on our formative study, concept designers categorize references into three types: (a) Hero (or Main) References: These align closely with the designer's creative vision, conveying the overall story, mood, or design, and are crucial for guiding the project. (b) Detailed Supporting References: These provide specific details, like structure or texture, helping designers accurately implement finer aspects of the design. (c) Miscellaneous References: These cover a range of purposes, including lighting, atmosphere, and color palette, supporting various design elements.}
    \Description{The figure categorizes design references into three types. (a) Main/Hero References depict key visual inspirations that establish the story, mood, or overall design direction, represented by historical Egyptian-themed illustrations. (b) Detailed Supporting References provide specific elements, including (b1) desert backgrounds with landscapes and oases, (b2) props and religious artifacts such as statues and clothing, and (b3) stone and building materials with examples of ancient architecture and textures. (c) Miscellaneous References cover aspects like lighting, atmosphere, and composition, illustrated with cinematic and artistic depictions of grand historical scenes. This classification helps designers structure their reference gathering process effectively.}
    \label{fig:type}
\end{figure*}

\subsubsection{Type of references collected for environment concept design}
Environment concept designers gather diverse reference sets tailored to specific tasks, each serving different purposes. Designers also have unique ways of sourcing and organizing references. To understand these patterns, we analyzed reference frequency and collaborated with designers, identifying the following categories:
\begin{itemize}
    \item \textbf{Hero (or Main) Reference:} These references closely reflect the designer’s creative intent, aligning with the design theme and serving as a guide for establishing the overall mood, shapes, and composition of the design (Figure \ref{fig:type}-a). 
    \item \textbf{Detailed Supporting Reference:} These references provide specific detailed contents that support the design of the project. Typically, photographs provide specific details, such as mechanical structures or architectural features, offering accuracy and reliability for refining intricate design aspects  (Figure \ref{fig:type}-b).
    \item \textbf{Miscellaneous Reference:} Designers often collect references like lighting, atmosphere, art style, color palette, composition, and shot angle to enhance their designs based on project needs. These references, guided by the project's goals or the designer's vision, are not always essential and are categorized as Miscellaneous References. (Figure \ref{fig:type}-c).
\end{itemize}

\subsection{Design Goals}
Based on our findings, we proposed three design goals to better support concept designers during the ideation stage:
\begin{itemize}
    \item \textbf{DG1: Breadth exploration:} To help designers efficiently explore a wide range of ideas and gain a comprehensive understanding of the design topic, the system should support the brainstorming of various ideas using input methods that align with their workflow. This could include allowing users to input natural language instructions, such as task specifications, or directly upload relevant references.
\end{itemize}
\begin{itemize}
    \item \textbf{DG2: Depth exploration:} The system should offer detailed information and references to help designers refine and expand their design solutions while deepening their understanding of both the generated ideas and the design task. Moreover, the provided information should align with the designers' usual reference-gathering practices.
\end{itemize}
\begin{itemize}
    \item \textbf{DG3: Flexible iterative exploration:} The system should allow users to refine design ideas while maintaining control easily. It should support the efficient exploration of variations on the same theme to ensure consistency and creative flow. Designers should be able to narrow or expand the design space as needed, enhancing the creative process.
\end{itemize}

