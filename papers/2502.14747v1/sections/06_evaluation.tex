\section{SUMMATIVE STUDY}
% 一開始就要講清楚high level的關聯性
% Ideation結果要講清楚是搜集Reference過程

Our summative study examines the effectiveness of a creativity support tool in enhancing designers' early ideation workflows and fostering their creative processes. Rather than evaluating the quality of final design outputs, the study focuses on how AIdeation supports the workflow compared to designers' existing processes. We conducted a within-subject comparative study with 16 professional environment concept designers, focusing on three key aspects: 


% To evaluate whether AIdeation enhances designers' early ideation workflow and supports their creative process, we conducted a within-subject comparative study with 16 professional environment concept designers, focusing on three key aspects: 
\begin{itemize}
    \item \textbf{A1:} Support for the Ideation Process
    \item \textbf{A2:} Quality and Efficiency of Ideation 
    \item \textbf{A3:} Workflow Support at Each Stage
\end{itemize}
Given the diversity of concept designers' workflows, we set the baseline to each participant's preferred existing workflow. Participants were free to use any of their current methods, such as image databases, search engines, and AI design tools like Midjourney or DALL-E 3, or a combination of these. For those with no prior experience using AI design tools, we provided access to ChatGPT-4 with DALL-E 3 and a brief tutorial, as AIdeation is also based on DALL-E 3. Participants could then decide whether to incorporate this into their process.

% 設計的時候就寫說設計原因
% 論述強度要加強

\subsection{Study Design}
\subsubsection{Procedure}
The study lasted 2 to 2.5 hours, beginning with a 10-minute briefing. Participants completed a 30-minute design task under each condition, with each task preceded by a 10-minute practice session. They also received a 10-minute tutorial on AIdeation. To minimize bias from prior experience, participants were briefed on key differences between AIdeation and other AI tools, such as its use of natural language inputs instead of prompts and its more limited styling options. After each task, participants took a 10-minute break. Both the conditions and design topics were counterbalanced. After both tasks, participants completed a questionnaire and a 20-30 minute post-study interview. They were compensated approximately 34 USD.


\subsubsection{Task overview}
Participants completed a design task that replicated their ideation workflow using both the baseline method and AIdeation. For each condition, participants were assigned an environment concept design topic and instructed to gather at least three sets of references for interior and exterior designs using the widely used PureRef reference board ~\footnote{Pureref, https://www.pureref.com/}. The design topics were: 1) a Mayan Observatory and Planetarium with an observation hall featuring a dome, and 2) a Tibetan Meditation Research Center with a main research hall. Both tasks required blending traditional and contemporary architectural styles, a common challenge in real-world projects. Participants were asked to follow their ideation process, including research and brainstorming, ensuring that the selected references would suit future designs and client presentations, thereby simulating real-world constraints. Each reference set is needed to represent a distinct ideation result, including at least one main reference supported by detailed references.

\subsubsection{Pilot study and refinements}
Initially, we designed a 40-minute task, which also included a sketching stage after the research and brainstorming stages. We piloted this design with two professional concept designers, but neither was able to complete the tasks, reporting feeling "extremely stressed" as such tasks typically require a full day. Additionally, sketching diverted their focus from exploring the AIdeation system, despite it being the primary focus of the study. Based on these findings, we excluded sketching from the final study design. The revised tasks and design topics were reviewed and validated by three art directors from animation, game, and art outsourcing studios, who confirmed that using references alone to communicate with clients is a common practice, especially under time constraints.

% Initially, in our pilot study, two professional concept designers were asked to complete the research, brainstorming, and rough sketching tasks within 40 minutes for each condition. However, 


% \subsubsection{Evaluation approach}
% We adopted a self-report approach, aligning with prior research in the HCI and creativity communities~\cite{lubos2024llm,satyanarayan2019critical,palani2022don, son2024genquery}. Participants provided feedback based on their experiences, evaluating AIdeation across key dimensions as a creativity support tool. For the ideation results, participants self-assessed which method provided better support for idea quality and efficiency during the tasks. Given the difficulty of directly comparing outputs between two conditions (collected references), external expert evaluation was not conducted. Instead, participants were asked during interviews to explain their questionnaire ratings in detail, providing qualitative insights into their experiences. To complement this, we later conducted a field study to assess AIdeation's impact on real-world projects, focusing on idea quality, quantity, and creativity.

%Due to the challenges in recruiting professional concept designers, .

% We recruited 16 professional environment concept designers (11 males, 5 females; ages 23 to 45) from various industries and five studios, including animation (P1–P8), game design (P9–P12), art outsourcing (P13, P14), and freelancing (P15, P16). Participants had between 1 and 12 years of experience, with an average of 4.63 years and a standard deviation of 3.24 years. 

% \subsection{Study Procedure}
% The study lasted 2 to 2.5 hours, starting with a 10-minute briefing. For each design task, participants began with a 10-minute practice session, followed by a 30-minute task. They also received a 10-minute tutorial on AIdeation. To reduce bias from prior experience, participants were briefed on key differences between AIdeation and other AI tools, such as its use of natural language inputs instead of prompts and its more limited styling options. After each task, participants took a 10-minute break. Both the conditions and design topics were counterbalanced. After both tasks, participants completed a questionnaire and a 20-30 minute post-study interview. They were compensated around 34 USD.

% > 1. The study uses a one-sample Wilcoxon signed rank analysis method. I know that the one-sample Wilcoxon signed rank used to test the median in the data was significantly different from the set value (people set the value first).  Therefore, in this case, I am not sure is it ok to used it in this study, especially, the 7 point Likert scale from strong prefer baseline to strong prefer AIdeation. Maybe other reviewers could provide more information about it.

\subsubsection{Measurements}
The questionnaire focused on comparing the support provided by each condition for the ideation process and different workflow stages. Participants rated their preferences across various aspects. For the ideation process, they evaluated breadth and depth of exploration support, flexibility in idea exploration, efficiency in generating diverse ideas, quality of ideas, and creativity enhancement. For example, we asked the participants about their preferences using questions such as, “\textit{Which system allows you to generate a variety of design ideas more efficiently?}” For workflow support, they compared the systems on information collection efficiency, reference-gathering efficiency, the usefulness of collected references, and which system better supported the visual presentation of ideas. Additionally, participants provided feedback on their overall satisfaction and enjoyment while also comparing task difficulty and efficiency between the two conditions. The full questionnaire can be found in Appendix \ref{AppendixA}. All responses were measured using a 7-point Likert scale, where 7 indicated a strong preference for AIdeation, and 1 indicated a preference for their original workflow. For questions related to efficiency, participants were asked to disregard image generation time. This measurement approach aligns with previous findings on preference elicitation, emphasizing the importance of task sensitivity and granularity in capturing meaningful differences between options~\cite{evangelidis2024task}. Specifically, using comparative questions enhances sensitivity to utility differences, while choosing a 7-point scale balances granularity and interpretability for moderate differences in preferences. A one-sample Wilcoxon signed-rank test was performed to evaluate whether responses differed significantly from the neutral midpoint (4). This non-parametric test is appropriate for analyzing the ordinal data collected through the 7-point Likert scale questionnaire, as it does not assume a normal distribution and is well-suited for assessing central tendency differences in ordinal data~\cite{conover1999practical}. By testing whether the median response significantly deviates from the neutral point, this approach effectively determines whether participants exhibited a systematic preference for one condition over the other. This methodology is supported by previous research on the suitability of nonparametric tests for ordinal data and preference-based measures~\cite{roberson1995analysis, capanu2006testing, taheri2013generalization} and is consistent with previous studies employing similar analytical frameworks~\cite{chen2023aircharge}.

% A one-sample Wilcoxon signed-rank test was performed to evaluate whether responses significantly differed from the neutral midpoint (4). This non-parametric test is suitable for analyzing the ordinal data collected through the 7-point Likert scale questionnaire, which evaluates preferences between two conditions. This approach is supported by prior research~\cite{roberson1995analysis, capanu2006testing, taheri2013generalization} and is consistent with previous studies employing similar methodologies~\cite{chen2023aircharge}.

% In the in-depth interview, we first explored participants' attitudes toward AI-generated images, prior experience with AI design tools, typical ideation strategies, and how they approached the task using AIdeation. We then focused on how their experience with AIdeation differed from their original workflow, particularly in terms of the ideation process and overall workflow. Participants also explained the reasoning behind their questionnaire ratings. The interview concluded with discussions on their favorite AIdeation features, suggestions for improvements, and how they might integrate AIdeation into their workflow. 
%RR
In the in-depth interview, we first explored participants' attitudes toward AI-generated images, prior experience with AI design tools, typical ideation strategies, and how they approached the task using AIdeation. We then focused on how their experience with AIdeation differed from their original workflow, particularly in terms of the ideation process and overall workflow. For example, we asked participants to compare their experiences when searching for references using two different approaches. Additionally, we requested that they explain the reasoning behind their questionnaire ratings. For instance, they were asked to elaborate on why they preferred AIdeation for better efficiency. The interview concluded with discussions on their favorite AIdeation features, suggestions for improvements, and how they might integrate AIdeation into their workflow. The detailed interview questions can be found in Appendix \ref{AppendixB}. The interview data were analyzed similarly to the formative study. Three researchers summarized the transcribed recordings, and a former concept designer on the team identified key themes for thematic analysis. The findings were then reviewed and discussed among the researchers to ensure consensus.


\subsubsection{Evaluation approach}
We adopted a self-report approach, aligning with prior research in the HCI and creativity communities~\cite{lubos2024llm,satyanarayan2019critical,palani2022don, son2024genquery}. Participants provided feedback based on their experiences, evaluating AIdeation across key dimensions as a creativity support tool. For the ideation results, participants self-assessed which method provided better support for idea quality and efficiency during the tasks. Given the difficulty of directly comparing outputs between two conditions (collected references), external expert evaluation was not conducted. Instead, participants were asked during interviews to explain their questionnaire ratings in detail, providing qualitative insights into their experiences. To complement this, we later conducted a field study to assess AIdeation's impact on real-world projects, focusing on idea quality, quantity, and creativity.



% > 2. Creativity or Quality and Quantity of Ideation Results: normally, we measure the quality and quantity of ideation results by two experts to evaluate all results. I know evaluating the designers' ideas from several dimensions, such as novel, depth, and breadth, is also acceptable. However, the measurement of ideation results and creativity used in this study is a bit questionable.



% Participants were asked to complete a 30-minute design task replicating their ideation workflow using both the baseline and AIdeation. Each participant was assigned an environment concept design topic for each condition and instructed to gather at least 3 sets of references for both the interior and exterior designs using the widely-used reference board, PureRef. The design topics were: 1) a Mayan Observatory and Planetarium with an observation hall featuring a dome, and 2) a Tibetan Meditation Research Center with a main research hall. Both tasks required blending traditional and contemporary architectural styles, a common challenge in real-world projects. Participants were informed that the references should be suitable for future designs and client presentations, mirroring real-world scenarios. Each set of references had to represent a distinct ideation process, including one main reference with supporting details. 

% We adopted a self-report approach, as proposed in prior research within the HCI and creativity communities~\cite{lubos2024llm,satyanarayan2019critical,palani2022don, son2024genquery}. Participants provided feedback based on their own experiences, evaluating AIdeation across key value dimensions as a new system. For the ideation results, participants are asked to self-assess which method helps them achieve better idea quality or quantity while performing two tasks. The external expert evaluation was not conducted, as directly comparing outputs from AIdeation and traditional methods, which are collected references, would be difficult. Acknowledging this limitation, we asked each participant during the interview to explain in detail the reasoning behind their ratings for each questionnaire item. Furthermore, we conducted a complementary field study to evaluate AIdeation's real-world impact on project outcomes, focusing on its influence on idea quality, quantity, and creativity. This approach offers practical insights to supplement our findings.

% % RR
% % 我們在這個實驗中沒有做後面的Sketching的原因  然後再講pilot study
% In our pilot study, we asked two professional concept designers to complete a rough sketch alongside reference gathering within 40 minutes for each condition. However, neither participant was able to finish the tasks in either condition. Both reported feeling "extremely stressed", as such design tasks typically require an entire day to complete. Additionally, they spent minimal time on AIdeation system, as their focus was diverted to sketching, despite sketching not being the primary focus of our study. As a result, we decided to exclude the sketching component from the final study design. The entire task procedure and both design topics were reviewed and confirmed with three art directors from animation, game, and art outsourcing studios, who noted that using references alone to communicate with clients is a common practice when time is limited. 

\subsubsection{Participants}
% RR
We recruited 16 professional environment concept designers from various industries and five studios, including animation (P1, P4, P6, P16-P20), game (P5, P14-P15, P22), art outsourcing (P3, P13), and freelancing (P2, P21). 6 participants (P1-P6) had also participated in the earlier formative study. Participants had between 1 and 12 years of professional concept design experience (M = 4.6, SD = 3.2).