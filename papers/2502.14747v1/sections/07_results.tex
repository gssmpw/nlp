
\begin{figure*}
    \centering
    \includegraphics[width=1\linewidth]{figures/07_Ideation_result.png}
    \caption{The distribution of user preference for baseline vs. AIdeation: (a) Preference rating on a 7-point Likert scale for idea Exploration; (b) Overall preference for Satisfaction, Enjoyment, Task Difficulty, and Task Efficiency}
    \Description{The figure presents the distribution of user preferences comparing a baseline method with AIdeation. (a) Idea Exploration Preference is rated on a 7-point Likert scale across four aspects: breadth, depth, flexibility, and creativity. AIdeation is preferred in all categories, with significant differences indicated by asterisks. (b) Overall Preference evaluates user satisfaction, enjoyment, task difficulty, and task efficiency. AIdeation scores higher in satisfaction and enjoyment with significant differences, while task difficulty and efficiency also favor AIdeation. The scale ranges from strongly preferring the baseline to strongly preferring AIdeation, showing a clear trend toward AIdeation's advantages in ideation tasks.}
    \label{fig:ideation result}
\end{figure*}

\section{RESULTS \& FINDINGS}
\subsection{A1: Supporting Ideation Process}
In this section, we first address our key aspects using the information collected from the study. Following that, we will report additional notable qualitative findings based on our observations.
\subsubsection{Breadth, depth, and flexibility of idea exploration
}
Figure \ref{fig:ideation result}-a shows that participants preferred AIdeation for breadth (Mean = 5.19, \textit{p} = 0.014), depth (Mean = 5.00, \textit{p} = 0.033), and flexibility (Mean = 4.93, \textit{p} = 0.046) in idea exploration, with 69\% expressing a preference for AIdeation in terms of both breadth and depth. Participants mentioned: “\textit{The randomization provided by AIdeation offers a lot of possibilities}” (P19). and “\textit{The additional references, combined with my existing ideas, really expanded my design space}” (P21). Most participants found AIdeation offered better diversity than their original tools with the same input (P1, P3, P5, P13, P15-P16, P19, P21-P22). “\textit{Compared to MidJourney, AIdeation provided much more diversity, and I didn't even need to think of a prompt}” (P16). “\textit{Each iteration offered significant variation, helping me break out of my usual direction and explore new ideas}” (P4). However, some participants noted limitations in atmospheric or stylistic diversity (P2, P6, P20). “\textit{I feel like when I input 'Mayan Architecture,' the system often gave me stereotypical results}” (P2). “\textit{I always get a similar style from AI images}” (P6).

Regarding the depth of exploration, five participants noted that they could efficiently narrow down their design focus with AIdeation (P3, P18, P20-P22). “\textit{Compared to the design ideas I find on Artstation or Pinterest, which cannot be modified, I can use AIdeation to refine the idea}” (P21). Three participants also mentioned that AIdeation accurately provided detailed information and references that helped in further design work (P3, P5, P18). “\textit{AIdeation offers so much information that would normally take a lot of time to gather from different platforms}” (P18). However, two participants mentioned that AIdeation lacked image-to-image search functionality, which prevented them from obtaining similar images and restricted deeper exploration (P1, P6).

Finally, 56\% of the participants preferred the flexibility of exploration with AIdeation. Most suggested that AIdeation was easy to use and its functionality helped them access diverse design elements (P1, P14-P15, P16-P18, P22). However, three users found it challenging to achieve their desired results with AIdeation due to the lack of detailed control (P1, P2, P6). “\textit{The layout kept changing when I combined references or refined by instructions, but I wanted to keep that layout}” (P2). “\textit{I just wanted to adjust the atmosphere of the image, but I couldn't do that with this system}” (P6).


\subsubsection{Creativity}
Compared to their original workflow, participants significantly preferred AIdeation to enhance creativity (Figure \ref{fig:ideation result} -a, Mean = 5.56, \textit{p} = 0.001), with 81\% expressing a preference for it. Many users noted that AIdeation provided unexpected brainstorming results (P1, P13-P18, P21-P22). Several participants appreciated the keywords provided by AIdeation, mentioning that they could simply refer to the detail information bar for inspiration (P1, P4-P5, P13-P14). Furthermore, combining their original design ideas with the additional reference introduced different ways of thinking (P1, P4, P14, P17, P21-P22). As one participant mentioned, “\textit{The system combined different styles and content in various ways, which gave me a lot of inspiration}” (P21). Another added, “\textit{Combining diverse or uncommon elements sparked new ideas}” (P1).

\subsubsection{Overall satisfaction, task efficiency, and difficulty}
Figure \ref{fig:ideation result}-b shows the distribution of the overall satisfaction, enjoyment, and preferences of participants regarding task difficulty and efficiency. The participants significantly preferred AIdeation in terms of overall satisfaction (Mean = 5.19, \textit{p} = 0.005) and enjoyment (Mean = 5.19, \textit{p} = 0.005), with 75\% and 69\% expressing a preference, respectively. “\textit{The system is really easy to use, compared with other AI tools I used before}” (P16). “\textit{I really enjoy seeing the result of combination, each time I was surprised by unexpected results}” (P1). Results also suggest that participants significantly preferred AIdeation for reducing task difficulty (Mean = 5.19, \textit{p} = 0.005). Participants explained that AIdeation allowed them to work without thinking too much (P3, P16, P18) and provided well-organized and relevant information that would normally take more time to gather using their original workflow (P1, P4-P5, P19), such as “\textit{Those keywords truly helped me quickly find a variety of reference materials}” (P4).

However, the results for task efficiency were mixed. Despite being asked to ignore image generation time, some participants who preferred their baseline workflow noted that waiting for results in AIdeation took longer compared to browsing visuals continuously on Pinterest (P1, P17-P19). Additionally, three participants barely completed their tasks because they spent most of their time trying to achieve an exact match to their idea but were unsuccessful (P2,P6, P18). One participant who preferred the baseline for both satisfaction and efficiency mentioned, “\textit{I kept modifying the prompt but couldn’t get the result I wanted, which was frustrating}” (P6).

In summary, compared to their original workflow, AIdeation improved the creativity of participants and provided better support for both breadth and depth of exploration with improved flexibility. In general, the participants reported greater satisfaction and enjoyment with AIdeation and significantly reduced the difficulty of the task.

\begin{figure*}
    \centering
    \includegraphics[width=1\linewidth]{figures/08_Design_idea_result.png}
    \caption{The distribution of user preference for baseline vs. AIdeation:  Preference rating on a 7-point Likert scale for Quality and Efficiency of Ideation}
    \Description{The figure presents user preferences comparing a baseline method with AIdeation on a 7-point Likert scale. Design Ideas Efficiency shows a strong preference for AIdeation, with a significant difference indicated by an asterisk. Design Ideas Quality is more evenly distributed, though AIdeation is still preferred. The scale ranges from strongly preferring the baseline to strongly preferring AIdeation, demonstrating that AIdeation enhances efficiency and is generally favored for idea quality}
    \label{fig:design idea result}
\end{figure*}

\subsection{A2: Quality and Efficiency of Ideation}
Figure \ref{fig:design idea result} shows the distribution of the participants' preferences for various aspects of the ideation process. The results indicate that the participants significantly preferred AIdeation to efficiently generate a variety of design ideas (Mean = 5.44, \textit{p} = 0.003) while maintaining a quality similar to their original workflow (Mean = 4.31, \textit{p} = 0.41). 75\% of participants preferred AIdeation for its efficiency in assisting with design idea generation. 

Most participants noted that AIdeation quickly offered a wide range of diverse design elements that aligned with the topic (P1, P13-P16, P18, P21- P22). “\textit{In my original workflow, I need to first think of keywords to search on Pinterest. Even if I find an interesting design, it's hard to extract key information from the image. AIdeation, however, provides plenty of ideas with well-organized information}” (P22). “\textit{In MidJourney, I have to come up with the idea first, but AIdeation already presents many ideas}” (P14).

Half of the participants noted that AIdeation significantly reduced the time needed to combine two design elements (P1, P3-P5, P13, P18, P21-P22). “\textit{Before, I had to manually combine two design elements, and if the result wasn’t ideal, the time was wasted. With AIdeation, I can instantly combine elements with some level of control, saving me a lot of time}” (P13).
Additionally, most users took advantage of AIdeation's "explore more" functionality to quickly generate interior design ideas based on their selected exterior design ideas. Most suggested that this feature significantly helped them obtain diverse interior results without starting the process from scratch (P1, P3, P15-P16, P18, P20, P22). “\textit{The ability to generate images in this narrative-driven way is truly a groundbreaking innovation}” (P18).  “\textit{This will be very useful if I need to design multiple rooms within the same building}” (P13).

Regarding ideation quality, some participants noted that their original workflow with Pinterest often yielded more diverse and unexpected design elements, leading to fresh ideas. In contrast, AIdeation tended to offer more relevant design elements (P2, P6). Other participants who preferred the baseline or remained neutral noted that they could achieve similar or better quality using their original workflow (P6, P13, P18, P20). “\textit{The AI-generated images are very different from what I have in mind. If it were a real photo, the quality would be better, and the details would be more accurate}” (P6).

In summary, the participants considered AIdeation to have helped them generate more diverse design ideas while maintaining quality similar to their original workflow. 

\begin{figure*}
    \centering
    \includegraphics[width=1\linewidth]{figures/09_Workflow_result.png}
    \caption{The distribution of user preference for baseline vs. AIdeation:  Preference rating on a 7-point Likert scale for Workflow Support}
    \Description{The figure compares user preferences for baseline vs. AIdeation on a 7-point Likert scale. AIdeation is significantly preferred for information support and visual presentation. Reference gathering efficiency and usefulness of references also favor AIdeation but with a more balanced distribution. The scale ranges from strongly preferring the baseline to strongly preferring AIdeation, showing AIdeation improves reference gathering and workflow support.}
    \label{fig:workflow result}
\end{figure*}

\subsection{A3: Workflow Support at Each Stage}
\subsubsection{Support for research, reference gathering and visual presentation}
As shown in Figure \ref{fig:workflow result}, participants significantly preferred AIdeation for efficiently gathering information for the design task (Mean = 5.25, \textit{p} = 0.009), with 81\% expressing a preference. Many found that AIdeation helped them quickly understand the design topic (P1-P3, P15-P17, P19, P21-P22). One participant noted, “\textit{Normally, we spend hours researching complex topics without existing references, but AIdeation provided a whole package of concepts and information instantly, saving a lot of time}” (P3). Participants mentioned that the keywords and detailed supporting references provided by AIdeation for each design idea helped them quickly obtain the information needed to develop their designs. “\textit{Even if the AI-generated image didn't fit my needs, I could use the keywords provided by the system to find a lot of useful information}” (P5). “\textit{I don’t need to come up with keywords or read through a lot of text when using AIdeation}” (P13).

The participants also significantly preferred AIdeation to better visually present the design idea (Figure \ref{fig:workflow result}, Mean = 5.31, \textit{p} = 0.004), with 69\% expressing preference. “\textit{Usually, I have to sketch multiple design versions because no reference fits my needs. But with AIdeation, I can select elements from several generated images and directly present my idea to the client}” (P3).
However, there was no significant preference regarding the efficiency of gathering references or the usefulness of references with AIdeation. Participants who preferred the baseline stated that “\textit{Even though AIdeation provides accurate references, the overall quantity is much less compared to my original workflow (Pinterest)}” (P1).
% “\textit{I usually prefer photorealistic references, but the generated images always have an ‘AI style’.}” (P6)


\subsubsection{Integrating AIdeation into workflow}
At the end of the study, we asked the participants if and how they would use AIdeation in their real-world projects. Most of the participants indicated that they would use AIdeation immediately after receiving design specifications to explore different concepts (P1-P6, P13-P16, P18-P22). Several noted that AIdeation is more efficient and user-friendly then other AI tools (P1, P13-P14, P18, P21-P22). Three participants found it particularly useful for initiating new designs based on existing concepts (P3, P5, P15). Many suggested using AIdeation's output to communicate more effectively with directors or clients, thereby significantly increasing efficiency (P3-P5, P13, P16-P17, P21-P22). Additionally, two users mentioned potential applications for photobashing with AIdeation's outputs (P15, P20).

In summary, the participants found AIdeation to be more efficient in collecting relevant information and visually presenting the ideation results. While some preferred traditional methods for sourcing references, most considered AIdeation's suggestions valuable for ideation. Additionally, it showed the potential to streamline workflows and enhance client communication.

\subsection{Qualitative Findings on AIdeation Usage}
\subsubsection{Impact of intuitive vs. Analytical usage on AIdeation}
Participants who intuitively engaged in AIdeation and freely experimented achieved better results than those who overanalyzed the process. Those who quickly iterated without perfecting the prompts generated more diverse and creative outputs, while participants who spent excessive time refining the inputs produced fewer results and found the tool less efficient.
For example, P5 embraced a simple and iterative approach, generating 16 hero references in 4 ideation cycles with 10 refinements, aligning well with her creative vision. In contrast, P2 spent considerable time crafting inputs and struggled with unsatisfactory outputs, completing 8 cycles with 4 refinements but only 6 hero references, ultimately perceiving the tool as less effective.


\subsubsection{Controllability of AIdeation}
AIdeation received polarized opinions regarding its controllability. On the positive side, many participants were impressed with the degree to which the tool understood their intentions, especially when combining references and refining with instruction. (P3-P5, P13, P15, P20) “\textit{For me, ease of use is the most important factor for an AI tool. AIdeation met that goal and was able to capture the key points I wanted}” (P15). “\textit{Combining references allows me to control specific parts and choose what to merge, which is extremely helpful}” (P19). However, some participants had contrasting experiences. “\textit{I wanted to keep the material but change the layout from square to round, and after trying several times, it still didn’t work}” (P18). Interestingly, some participants appreciated the lower level of controllability. "\textit{Each generated idea had significant variation, and I could extract different elements from them}" (P16).

% move from discussion, need to be shorten
\subsubsection{Expectations toward AIdeation and their effect}
Users' expectations toward AIdeation strongly influenced their experience. When seen mainly as an image generation tool—a common perception in our study—users often produced less diverse results, focusing on precise prompts and modifications (see Section 7.4.1). This mindset, common among users familiar with other AI design tools, emphasizes control and exact results. As one participant noted, “\textit{With AI, I expect a complete design. Unlike Pinterest, where I look for elements, AI images seem finished, so I feel the need to tweak them for clients}” (P18). This approach can limit opportunities, as users may overlook valuable design elements. In contrast, treating AIdeation like browsing Pinterest encourages the discovery and integration of new ideas, enriching the creative process.