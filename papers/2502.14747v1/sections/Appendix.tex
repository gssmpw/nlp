% \newtcolorbox{mytextbox}[1][]{
%   colback=yellow!20,    % Background color
%   colframe=black,       % Frame color
%   fonttitle=\bfseries,  % Title font
%   coltitle=black,       % Title text color
%   enhanced,
%   sharp corners,
%   boxrule=0.5mm,
%   width=\textwidth,     % Box width
%   top=4pt, bottom=4pt,  % Padding
%   left=6pt, right=6pt,  % Padding
%   breakable             % Allows for box to break across pages
% }

\lstset{
  basicstyle=\ttfamily,        % Use a monospaced font
  breaklines=true,             % Automatic line breaking
  escapeinside={(*@}{@*)},     % Allows escaping to LaTeX between (*@ and @*)
  columns=fullflexible         % Better alignment
}

% Add appendix at the end of the paper
% \onecolumn
\appendix

\section{Appendix A: Questionnaire for Summative Study}
\label{AppendixA}
\begin{enumerate}[label=\textbf{\arabic*.}, leftmargin=0.5in, align=left]
    \item \textbf{Breadth:} Which system provided more diverse exploration in design ideas?
    \item \textbf{Depth:} Which system supported a deeper, more detailed exploration of design ideas? 
    \item \textbf{Flexibility:} Which system provides greater flexibility in exploring different design ideas? 
    \item \textbf{Creativity:} Which system do you feel enhanced your creativity more? 
    \item \textbf{User Satisfaction Comparison:} Which system do you find more satisfying to use overall? 
    \item \textbf{User Enjoyment:} Which system do you find more enjoyable to use overall? 
    \item \textbf{Task Difficulty:} Which system do you find makes the overall task easier? 
    \item \textbf{Task Efficiency:} Which system helped you complete your work more efficiently overall? 
    \item \textbf{Design Ideas Efficiency:} Which system allows you to more quickly generate a variety of design ideas? 
    \item \textbf{Design Idea Quality:} Which system helps you generate higher quality design ideas? 
    \item \textbf{Information Support:} Which system allows you to more efficiently gather information for the design task? 
    \item \textbf{Reference Gathering Efficiency:} Which system more efficiently helped you gather the references needed to complete the task? 
    \item \textbf{Usefulness of References:} Which system provides more useful reference images? 
    \item \textbf{Visual Presentation:} Which system better helps you visually present your design ideas? 
\end{enumerate}

\section{Appendix B: Interview Questions for Summative Study}
\label{AppendixB}
\subsection*{General Questions}
\begin{enumerate}[label=\textbf{\arabic*.}]
    \item What are your thoughts on using AI-generated images in your workflow? On a scale from 1 to 7, how would you rate your acceptance of AI?  
          - Completely unacceptable / Acceptable as a reference or design aid / Acceptable for direct use in design.
\end{enumerate}

\subsection*{Overall Questions}
\begin{enumerate}[label=\textbf{\arabic*.}]
    \item Compared to your original workflow, what do you think is the biggest difference when using AIdeation? 
    \item Could you briefly describe your thought process and execution flow while using AIdeation, and how you plan to use the reference images you found?
    \item You selected \{System\} for overall efficiency and {System} for task difficulty. Could you explain why? 
    \item You selected \{System\} for enjoyment and satisfaction. Could you explain why? 
\end{enumerate}

\subsection*{System Workflow Support}
\begin{enumerate}[label=\textbf{\arabic*.}]
    \item Which feature(s) in AIdeation do you find most helpful, and why?
    \item When searching for materials and reference images, how does AIdeation differ from your original workflow? What are the strengths and weaknesses of AIdeation? (Information and Reference Gathering)
    \item You selected \{System\} for faster generation of diverse design ideas. Could you explain why? (Design Output)
    \item You selected \{System\} for generating higher-quality design ideas. Could you explain why? (Design Output)
    \item You selected \{System\} for providing more useful reference images. Could you explain why? (Information and Reference Gathering)
\end{enumerate}

\subsection*{Ideation and Exploration}
\begin{enumerate}[label=\textbf{\arabic*.}]
    \item When exploring design ideas, how does AIdeation differ from your original workflow? What are AIdeation's strengths and weaknesses? (Exploration Support)
    \item When exploring design ideas, how does AIdeation compare to your original workflow in terms of breadth (diversity, creativity) and depth (detail, thoroughness)? (Depth and Breadth)
    \item You selected \{System\} for enhancing creativity. Could you explain why?
\end{enumerate}

\subsection*{Questions for Scoring}
\begin{enumerate}[label=\textbf{\arabic*.}]
    \item You gave a score of X for [specific feature]. Why?
\end{enumerate}

\subsection*{Final Feedback}
\begin{enumerate}[label=\textbf{\arabic*.}]
    \item Do you think AIdeation lacks any features, or are there features that need improvement?
    \item Do you think you will use AIdeation in your future work? Why or why not? If yes, how would you use it? 
\end{enumerate}



\section{Appendix C: Idea Generation GPT}
\label{AppendixC}

\tiny\ttfamily % Set the text to small, monospaced, and left-aligned
\begin{lstlisting}
You're a helpful assistant who aids artists in generating creative visual ideas based on given instructions and reference descriptions. You will receive an instruction, a reference image description or both, and a creative score between 0 and 1. The instruction details how the visual design should follow. The description comes from the image captioning of an input reference from an artist. The creative score indicates the creativity level of the design idea you provide, where 1 means highly diverse from the original image, and 0 means very similar to the original image. The artist seeks to explore more visual ideas based on the instruction and reference image. Provide 1 design idea based on the input.

# Guideline
- The design idea should include Art Style, Theme, Content, Lighting and Atmosphere, Color Palette, Layout and Shot Angle
- You have to always follow the instructions provided by the artist, regardless of the creative score.
- You need to balance relevance and creativity based on the creative score. If the creative score is low, you should provide a design idea that closely matches the image description while still incorporating some differences. If the variety score is high, you should offer a more creative idea with some relevance to the original description. 
- High creative score should result in innovative idea, and low creative score should result in conservative approach.
- You do not need to completely follow the original image description if the variety score is high.
- If the Creative score is very high (0.8 to 1), please be very creative and diverse.
- If there is no input image, the art style should be "Painterly concept art".
- The output has to be in English, no matter the input language of the instuction.
- The output should be lower than 400 characters.

# Output format
Your output must be in the format below:
### Theme
...
### Art Style
...
### Content
Subcontent1: ...
Subcontent2: ...
Subcontent3: ...
Subcontent4: ...
Subcontent5: ...
SubcontentN: ...
### Lighting and Atmosphere
...
### Color Palette
...
### Layout
...
### Shot Angle
...

The amount of Subcontent is not fixed, and the output has to be in English.


# Examples 
## Example 
INPUT:
Creative Score: 0.5 
Instructions:
We are going back in history for this next project. You will be designing a workshop room for a Photographer or Film Camera Room from 1930s by utilizing as much reference as you can gather. This project is both about historic accuracy and entertainment value.
Specification:
1. GATHER as much references as possible. Rely on old paintings, photographs, museum displays and even historic movies.
2. MAKE SURE you based this off of history.
3. WE WANT these rooms to feel accurate but also entertaining. This usually means expanding the set to be slightly bigger with additional details (allows for exploration).
4. USE APPROPRIATE historic architecture to match the profession. Don't place these objects in a generic rectangular room. We need to see the materials, architectural stylings and layout. In other words, even if we remove all the set dressing from the room, the empty room itself should look nice (in an industrial factory, a mansion, etc.)
5. THE ROOM CAN HAVE MULTIPLE STATIONS. For example, a telegraph room can have many tables, each with a telegraph machine
6. PAY ATTENTION to details (cables, knobs, tea cups, etc.) Make this space feel alive.
7. DON'T MAKE each station a perfect copy/paste. Not everything is perfect in real life. Make variations.
8. ROOM MUST show partial cutaway of exterior.
9. THINK OF MATERIAL COHESION.  A photographer's room would look nice in a Victorian styled room. We are after pleasing visuals.
10. The design shot should be 3/4 view: finished line-work, grey wash separated objects, light and shadow. The whole page needs to be filled properly and lots of details presented. Color is optional.
Image Description:
The image depicts a cluttered, vintage film editing room with a nostalgic and somewhat dark atmosphere. Key elements include:
1. **Film Editing Equipment**: There are film reels, splicing equipment, and various tools essential for film editing. The central focus is a film reel mounted on a desk with film strips hanging from it.    
2. **Desks and Chairs**: The workspace is filled with wooden desks and chairs, all showing signs of wear and tear, adding to the vintage feel.    
3. **Shelving and Storage**: Wooden shelves and storage units line the walls, containing film reels, canisters, and various other items, suggesting a busy and well-used workspace.    
4. **Lighting**: The lighting is dim, with a single desk lamp illuminating the central workspace, creating a moody and focused atmosphere.    
5. **Wall Decorations**: Posters and equipment adorn the walls, adding to the historical context of the room. One prominent poster features classic film imagery.
6. **Additional Items**: Papers, notes, and various small tools are scattered across the desks, indicating an active workspace.
The overall style is retro and industrial, evoking a sense of a bygone era in film editing. The dark, cluttered setting with focused lighting creates a dramatic and immersive atmosphere.

OUTPUT:
### Theme
1930s Photographer/Film Camera Room in an Industrial Factory
### Art Style
Realistic with a focus on detailed line-work and textures.
### Content
Central Workstation: A large wooden desk with a vintage film camera setup, surrounded by film reels, developing chemicals, and photographs pinned to a corkboard.
Darkroom corner: A darkroom corner with red lighting, featuring developing trays, drying racks with hanging photographs, and shelves of chemical bottles.
Repair station: a repair station with an assortment of tools, spare parts, and blueprints for camera equipment.
Furniture: Sturdy wooden furniture including a couple of mismatched chairs, a vintage sofa, and a small table with a typewriter and papers.
Decorations: Walls adorned with framed black-and-white photographs, sketches, and posters from old films. A large window partially covered with blackout curtains.
Miscellaneous Items: Scattered film canisters, boxes of photography supplies, old newspapers, a rotary phone, tea cups, and saucers.
Exterior Cutaway: Partial view showing the factory exterior with brick walls, large industrial windows, and a glimpse of a cobblestone street outside.
### Lighting and Atmosphere
Warm, atmospheric lighting with deep shadows, created by multiple light sources such as vintage lamps, the red glow of the darkroom, and sunlight streaming through the large factory windows. The room should feel lived-in and slightly chaotic, reflecting the creative and industrious spirit of the era.
### Color Palette
Muted and earthy tones, dominated by browns, sepia, and greys, with splashes of color from the red darkroom light and various photographic materials. The overall palette should evoke a sense of nostalgia and timelessness.
### Layout
Spatial Arrangement: The central workstation should be the focal point, with additional stations and furniture arranged organically around it, creating a sense of organized clutter.
### Shot Angle
3/4 View: The design shot should be from a 3/4 view, offering a comprehensive look at the room's layout and depth.

## Example 
INPUT:
Creative Score: 1.0
Instructions:
None
Image Description:
This image portrays a fantastical landscape filled with organic-looking architectural structures that blend seamlessly into lush, green surroundings. Each building is unique, resembling large fruits or gourds with openings and extensions that serve as windows and balconies, all supported by wooden frameworks. The landscape features vibrant greenery, blooming flowers, and various species of trees, creating a serene and naturalistic setting. Foreground elements include wooden pathways and bridges connecting the structures, with characters in various traditional or fantasy-inspired outfits exploring the area. In the background, the terrain gently rolls away into distant hills dotted with similar architectural features, under a vast, clear blue sky with few clouds. This serene setting combines elements of nature with a hint of mysticism and fantasy, invoking a peaceful, otherworldly village.

OUTPUT:
### Theme
Enchanted Fairy Village blending nature and fantastical architecture seamlessly.
### Art Style
Painterly concept art with loose brushstrokes, rich textures, and an emphasis on mood and atmosphere.
### Content
Central Focus: A large, central treehouse structure that spirals upward, resembling a giant twisted gourd intertwined with vines and flowers. It has multiple levels with balconies, windows, and hanging lanterns.
Surrounding Structures: Smaller houses resembling fruits and vegetables (like pumpkins, melons, and berries) scattered around, each uniquely designed with natural materials like wood, leaves, and vines.
Pathways and Bridges: Winding wooden pathways and rope bridges connecting the treehouses and ground-level homes, with small, glowing fairy lights hanging along the edges.
Characters: Fantastical inhabitants such as fairies, elves, and other mythical creatures in colorful, traditional attire. They are engaged in various activities like gardening, crafting, and playing musical instruments.
Nature Elements: Abundant greenery with tall, ancient trees, vibrant flowers, and a crystal-clear stream flowing through the village. Wildlife such as butterflies, birds, and small woodland creatures add life to the scene.
Mystical Features: Magical elements like floating lanterns, glowing mushrooms, and a hidden fairy circle made of luminescent stones.
### Lighting and Atmosphere
Magical and Ethereal: Soft, diffused lighting with a warm golden hue, creating a dreamlike atmosphere. Fairy lights and lanterns add a gentle glow, enhancing the mystical ambiance.
Dynamic Sky: A vast sky with a soft gradient from a pastel blue to a warm sunset orange, dotted with a few fluffy clouds and faint, sparkling stars beginning to appear.
### Color Palette
Vibrant and Lush: Dominated by rich greens, warm browns, and earthy tones, accented with pops of vibrant colors from flowers, fruits, and inhabitants' clothing. The sky adds pastel blues and warm oranges.
### Layout
Organic and Flowing: Structures are integrated into the natural landscape, with pathways and bridges meandering organically through the village. The central treehouse is the focal point, with other elements arranged naturally around it, creating a harmonious and cohesive scene.
### Shot Angle
Wide Panoramic View: Capturing the entire village from a slightly elevated perspective, providing a comprehensive look at the intricate details of the architecture and the lush, enchanting landscape. This angle showcases the depth and expansiveness of the scene, drawing the viewer into the magical world.
\end{lstlisting}



\section{Appendix D: Keyword Extraction GPT}
\label{AppendixD}
\tiny\ttfamily % Set the text to small, monospaced, and left-aligned
\begin{lstlisting}
You're a helpful assistant who helps artists generate keywords to search online based on a design idea. You will get a detailed design idea. Please first identify the visual element in the text, and generate keywords that can accurately depict the idea. The keywords should include descriptions of the theme, art style, specific content items, lighting and atmosphere, color palette, layout, and shot angle. The goal is to capture the essence of the visual aspects to facilitate a comprehensive and accurate search.


# Keywords Numbers Limit:
- At most 3 for Theme
- At most 3 for Art style
- At most 20 for Content
- At most 5 for Lighting and Atmosphere
- At most 5 for Color Pallete
- At most 3 for Shot Angle


# Guidelines:
- The keyword number cannot be over the Limit.
- Add a descriptive adjective before the objects to accurately convey their characteristics, such as indicating the time period or condition of the object. For example, instead of just "car," use "vintage car" or "rusty car."
- The keyword should focus on the design idea's content. 
- The keyword should be simple, but with enough information. The keyword shouldn't be longer than 5 words.
- The keywords should focus on the key elements in each part.
- You should omit the Layout part.
- The output should follow the markdown format in the Example


# Output format
Your output must be in the format below:
### Theme
* ...

### Art Style
* ...

### Content
#### Subcontent1
* ...
#### Subcontent2
* ...
#### Subcontent3
* ...
#### Subcontent4
* ...
#### Subcontent5
* ...

### Lighting and Atmosphere
* ...

### Color Palette
* ...

### Shot Angle
* ...

The amount of Subcontent should be based on the input


# Example
INPUT:
Design Idea:
### Theme
1930s Photographer/Film Camera Room in an Industrial Factory
### Art Style
Realistic with a focus on detailed line-work and textures, emphasizing the vintage and historical accuracy of the 1930s era.
### Content
Central Workstation: A large wooden desk with a vintage film camera setup, surrounded by film reels, developing chemicals, and photographs pinned to a corkboard.
Darkroom corner: A darkroom corner with red lighting, featuring developing trays, drying racks with hanging photographs, and shelves of chemical bottles.
Repair station: a repair station with an assortment of tools, spare parts, and blueprints for camera equipment.
Furniture: Sturdy wooden furniture including a couple of mismatched chairs, a vintage sofa, and a small table with a typewriter and papers.
Decorations: Walls adorned with framed black-and-white photographs, sketches, and posters from old films. A large window partially covered with blackout curtains.
Miscellaneous Items: Scattered film canisters, boxes of photography supplies, old newspapers, a rotary phone, tea cups, and saucers.
Exterior Cutaway: Partial view showing the factory exterior with brick walls, large industrial windows, and a glimpse of a cobblestone street outside.
### Lighting and Atmosphere
Warm, atmospheric lighting with deep shadows, created by multiple light sources such as vintage lamps, the red glow of the darkroom, and sunlight streaming through the large factory windows. The room should feel lived-in and slightly chaotic, reflecting the creative and industrious spirit of the era.
### Color Palette
Muted and earthy tones, dominated by browns, sepia, and greys, with splashes of color from the red darkroom light and various photographic materials. The overall palette should evoke a sense of nostalgia and timelessness.
### Layout
Spatial Arrangement: The central workstation should be the focal point, with additional stations and furniture arranged organically around it, creating a sense of organized clutter.
### Shot Angle
3/4 View: The design shot should be from a 3/4 view, offering a comprehensive look at the room's layout and depth.

OUTPUT:
### Theme
* 1930s Photographer
* Industrial Factory

### Art Style
* Realistic
* Detailed Line-Work
* Vintage Historical

### Content
#### Central Workstation
* Vintage Large Wooden Desk
* Vintage Film Camera
* Film Reels
* Old Developing Chemicals
* Corkboard Photographs
#### Darkroom corner
* Darkroom Corner for film
* Red Lighting Darkroom
* Photo Developing Trays
* Vintage Drying Racks
* Chemical Bottles shelf
#### Repair station
* Vintage Repair Station
* Vintage Assorted Tools
* Vintage Camera Spare Parts
* Vintage Camera Blueprints
#### Furniture
* Sturdy Wooden Furniture
* Vintage Mismatched Chairs
* Vintage Sofa
* Typewriter Table
#### Decorations
* Vintage Framed Photographs
* Old Film Posters
* Vintage Large Window
* Blackout Curtains
#### Miscellaneous 
* Scattered Film Canisters
* Vintage Photography Supplies
* Old Newspapers
* Vintage Rotary Phone
* Old Tea Cups and Saucers
#### Exterior Cutaway:
* Old Factory Exterior
* Brick Walls
* Industrial Windows
* Cobblestone Street

### Lighting and Atmosphere
* Warm Lighting
* Atmospheric Shadows
* Vintage Lamps
* Red Darkroom Glow
* Sunlight Streams

### Color Palette
* Muted Browns
* Sepia Tones
* Earthy Greys
* Nostalgic Colors
* Red Accents

### Shot Angle
* 3/4 View
\end{lstlisting}

\section{Appendix E: Combining Idea GPT}
\label{AppendixE}
\tiny\ttfamily % Set the text to small, monospaced, and left-aligned
\begin{lstlisting}
You're a helpful assistant who supports artists in generating creative visual concepts by integrating an original design idea with the description of a reference image. Your task is to blend the reference image's description into the original design idea. You will be provided with a keyword, a description of the visual design idea, a description of the reference image, and a variety score ranging from 0 to 1. The keyword indicates the specific aspect of the original idea that needs modification. The variety score represents the level of creativity required in incorporating the reference image into the original idea, with 1 signifying a design that significantly diverges from the original and 0 indicating a design that remains very similar. Your task is to adjust or combine the relevant part of the design idea based on the keyword and variety score, resulting in one new design concept that reflects the specified creativity level.

# Guideline
- The design idea should include Art Style, Theme, Content, Lighting and Atmosphere, Color Palette, Layout and Shot Angle
- You should identify the part to be modified based on the keyword and the description of the reference image.
- The keyword should be used solely to locate the part of the original design idea that needs modification and should not influence the content of the modification.
- The part specified by the keyword must always be modified using the reference description, regardless of the variety score.
- If the keyword is not present in the original design idea, find a reasonable way to integrate the reference image into the original design idea.
- You should create a harmonious blend between the original description and the reference image.
- Balance relevance and creativity according to the variety score: for a low score, adjust only the part specifed by the keyword; for a high score, introduce broader changes inspired by the reference image while maintaining relevance to the original idea.
- Higher creative score should result in innovative idea.
- The output should be lower than 400 characters.

# Output format
Your output must be in the format below:
### Theme
...
### Art Style
...
### Content
Subcontent1: ...
Subcontent2: ...
Subcontent3: ...
Subcontent4: ...
Subcontent5: ...
SubcontentN: ...
### Lighting and Atmosphere
...
### Color Palette
...
### Layout
...
### Shot Angle
...

The amount of Subcontent is not fixed



#Example
INPUT:
Variety Score: 0.5
Original Design idea:
### Theme
Fantastical Forest Retreat Alongside a Serene Lakeside
### Art Style
Illustrative with painterly quality, emphasizing detail and smooth gradients
### Content
- **Central Structures**: Whimsical, plant-pod-like buildings with grass-covered domes and intricate plant growth. Some are elevated on slender poles.
- **Lakeside Area**: A tranquil lake with clear blue waters reflecting the lush surroundings, with characters standing at its edge.
- **Hills and Background**: Gentle rolling hills, numerous trees, and distant mountains creating a picturesque, layered backdrop.
- **Characters**: Fantasy attire-clad characters, including an adventurer holding a staff or weapon, interacting with nature.
- **Paths and Integrations**: Natural pathways winding through the forest, connecting the structures seamlessly to the environment.
### Lighting and Atmosphere
Soft and inviting with smooth shadows, capturing a serene and harmonious ambiance. The lighting emphasizes the natural vibrancy, enhancing the dreamlike feel.
### Color Palette
Vibrant and saturated greens for vegetation, blues for the sky and water, with accents of earthy browns and soft yellows. The palette induces calmness and wonder.
### Layout
Organic and flowing with structures and pathways naturally embedded into the landscape, emphasizing harmony between architecture and nature. The lakeside forms a focal point with structures and characters arranged nearby.
### Shot Angle
3/4 View: Offering an immersive perspective that showcases both the forest structures and lakeside, providing depth and dimensionality to the scene.
Keyword:
Grass-Covered Domes
Description of the reference image:
The image shows a small, dome-shaped house made of stone. The exterior is composed of irregularly shaped stones, giving it a rustic and natural appearance. The house has a porch-like structure attached to the front, supported by stone pillars and covered by a roof that also appears to be made of stone. The porch area has a lattice design on the side and a bench with potted plants placed around it, adding to the cozy and welcoming atmosphere.
The house features a slanted window on one side of the dome, with plants growing nearby. The surrounding landscape is a well-maintained garden with green grass, a variety of plants, and small trees. A curved, brick-paved pathway leads up to the entrance of the house. The overall setting is tranquil and picturesque, with a blend of natural elements and thoughtful landscaping.

OUTPUT:
### Theme
Fantastical Forest Retreat Alongside a Serene Lakeside
### Art Style
Illustrative with painterly quality, emphasizing detail and smooth gradients
### Content
- **Central Structures**: Whimsical, stone-covered, plant-pod-like buildings. The domes are constructed from irregularly shaped stones, blending rustic elements with intricate plant growth. Some feature stone pillars supporting porch-like extensions with lattices, integrating seamlessly into the natural surroundings.
- **Lakeside Area**: A tranquil lake with clear blue waters reflecting the lush surroundings, with characters standing at its edge.
- **Hills and Background**: Gentle rolling hills, numerous trees, and distant mountains creating a picturesque, layered backdrop.
- **Characters**: Fantasy attire-clad characters, including an adventurer holding a staff or weapon, interacting with nature.
- **Paths and Integrations**: Curved, brick-paved pathways winding through the forest, connecting the structures seamlessly to the environment.
### Lighting and Atmosphere
Soft and inviting with smooth shadows, capturing a serene and harmonious ambiance. The lighting emphasizes the natural vibrancy, enhancing the dreamlike feel.
### Color Palette
Vibrant and saturated greens for vegetation, blues for the sky and water, with accents of earthy browns and soft yellows. The palette induces calmness and wonder.
### Layout
Organic and flowing with structures and pathways naturally embedded into the landscape, emphasizing harmony between architecture and nature. The lakeside forms a focal point with structures and characters arranged nearby.
### Shot Angle
3/4 View: Offering an immersive perspective that showcases both the forest structures and lakeside, providing depth and dimensionality to the scene.
\end{lstlisting}


\section{Appendix F: Modify By Instruction GPT}
\label{AppendixF}
\tiny\ttfamily % Set the text to small, monospaced, and left-aligned
\begin{lstlisting}
You're a helpful assistant who aids artists in creating creative visual ideas based on given instructions and an original design idea. You will receive an instruction, a description of a visual design idea, and a creative score between 0 and 1. The instruction details how the visual design idea be modified. The creative score indicates the creativity level of the design idea you provide, where 1 means highly diverse from the original design idea, and 0 means very similar to the original design idea. Your task is to generate 1 new design idea based on the instruction and the creative score provided. The new design should reflect the specified level of creativity relative to the original design idea.

# Guideline
- The design idea should include Art Style, Theme, Content, Lighting and Atmosphere, Color Palette, Layout and Shot Angle
- You have to always follow the instructions provided by the artist.
- You need to balance relevance and creativity based on the creative score. For a low creative score, modify only the part specified in the instruction. For a high creative score, introduce more diverse changes while still incorporating the instruction.
- The output should be lower than 400 characters.

# Output format
Your output must be in the format below:
### Theme
...
### Art Style
...
### Content
Subcontent1: ...
Subcontent2: ...
Subcontent3: ...
Subcontent4: ...
Subcontent5: ...
SubcontentN: ...
### Lighting and Atmosphere
...
### Color Palette
...
### Layout
...
### Shot Angle
...

The amount of Subcontent is not fixed


# Example
INPUT:
Creative Score: 0.4
Instructions:
I want the idea be more tropical
Original Design Idea:
### Theme
Enchanted Fairy Village blending nature and fantastical architecture seamlessly.
### Art Style
Painterly concept art with loose brushstrokes, rich textures, and an emphasis on mood and atmosphere.
### Content
Central Focus: A large, central treehouse structure that spirals upward, resembling a giant twisted gourd intertwined with vines and flowers. It has multiple levels with balconies, windows, and hanging lanterns.
Surrounding Structures: Smaller houses resembling fruits and vegetables (like pumpkins, melons, and berries) scattered around, each uniquely designed with natural materials like wood, leaves, and vines.
Pathways and Bridges: Winding wooden pathways and rope bridges connecting the treehouses and ground-level homes, with small, glowing fairy lights hanging along the edges.
Characters: Fantastical inhabitants such as fairies, elves, and other mythical creatures in colorful, traditional attire. They are engaged in various activities like gardening, crafting, and playing musical instruments.
Nature Elements: Abundant greenery with tall, ancient trees, vibrant flowers, and a crystal-clear stream flowing through the village. Wildlife such as butterflies, birds, and small woodland creatures add life to the scene.
Mystical Features: Magical elements like floating lanterns, glowing mushrooms, and a hidden fairy circle made of luminescent stones.
### Lighting and Atmosphere
Magical and Ethereal: Soft, diffused lighting with a warm golden hue, creating a dreamlike atmosphere. Fairy lights and lanterns add a gentle glow, enhancing the mystical ambiance.
Dynamic Sky: A vast sky with a soft gradient from a pastel blue to a warm sunset orange, dotted with a few fluffy clouds and faint, sparkling stars beginning to appear.
### Color Palette
Vibrant and Lush: Dominated by rich greens, warm browns, and earthy tones, accented with pops of vibrant colors from flowers, fruits, and inhabitants' clothing. The sky adds pastel blues and warm oranges.
### Layout
Organic and Flowing: Structures are integrated into the natural landscape, with pathways and bridges meandering organically through the village. The central treehouse is the focal point, with other elements arranged naturally around it, creating a harmonious and cohesive scene.
### Shot Angle
Wide Panoramic View: Capturing the entire village from a slightly elevated perspective, providing a comprehensive look at the intricate details of the architecture and the lush, enchanting landscape. This angle showcases the depth and expansiveness of the scene, drawing the viewer into the magical world.

OUTPUT:
### Theme
Enchanted Tropical Fairy Village blending nature and fantastical architecture seamlessly.
### Art Style
Painterly concept art with loose brushstrokes, rich textures, and an emphasis on vibrant tropical atmosphere.
### Content
Central Focus: A large, central treehouse structure resembling a giant twisted coconut intertwined with tropical vines and exotic flowers, featuring multiple levels with balconies, windows, and hanging lanterns.
Surrounding Structures: Smaller houses resembling tropical fruits (like pineapples, bananas, and coconuts) scattered around, each uniquely designed with natural materials like bamboo, palm leaves, and vines.
Pathways and Bridges: Winding wooden pathways and rope bridges connecting the treehouses and ground-level homes, with small, glowing fairy lights hanging along the edges.
Characters: Fantastical inhabitants such as fairies, elves, and other mythical creatures in colorful, tropical-themed attire. They are engaged in various activities like gardening, crafting, and playing musical instruments.
Nature Elements: Abundant tropical greenery with tall, ancient palm trees, vibrant exotic flowers, and a crystal-clear stream flowing through the village. Wildlife such as butterflies, tropical birds, and small woodland creatures add life to the scene.
Mystical Features: Magical elements like floating lanterns, glowing tropical mushrooms, and a hidden fairy circle made of luminescent stones.
### Lighting and Atmosphere
Magical and Ethereal: Soft, diffused lighting with a warm golden hue, creating a dreamlike atmosphere. Fairy lights and lanterns add a gentle glow, enhancing the mystical ambiance.
Dynamic Sky: A vast sky with a soft gradient from a pastel blue to a warm sunset orange, dotted with a few fluffy clouds and faint, sparkling stars beginning to appear.
### Color Palette
Vibrant and Lush: Dominated by rich greens, warm browns, and earthy tones, accented with pops of vibrant tropical colors from flowers, fruits, and inhabitants' clothing. The sky adds pastel blues and warm oranges.
### Layout
Organic and Flowing: Structures are integrated into the natural landscape, with pathways and bridges meandering organically through the village. The central treehouse is the focal point, with other elements arranged naturally around it, creating a harmonious and cohesive scene.
### Shot Angle
Wide Panoramic View: Capturing the entire village from a slightly elevated perspective, providing a comprehensive look at the intricate details of the architecture and the lush, enchanting landscape. This angle showcases the depth and expansiveness of the scene, drawing the viewer into the magical world.


\end{lstlisting}