\section{BACKGROUND: WORKFLOW OF THE ENTERTAINMENT INDUSTRY AND CONCEPT DESIGNERS}
The entertainment industry's production process, whether for films, TV shows, or video games, transforms creative ideas into final products through a series of four stages: 1) \textit{development}, where the initial concept and creative direction are set; 2) \textit{pre-production}, involving detailed planning and preparation; 3) \textit{production}, where the main content is created; and 4) \textit{post-production}, which includes editing, enhancing, and polishing of the final product~\cite{gameworkshop2018, bigbadWorld2015, directing2020, Singh2023Artificial, filmmaker2019}.

Concept designers are pivotal across the first three stages, particularly in the \textit{pre-production} stage. During \textit{development}, concept designers collaborate with art directors/clients to visualize core ideas through initial sketches and designs to define the project's aesthetic and visual tone~\cite{artofgame2008}. During \textit{pre-production}, they design scenes, characters, environments, and props to provide blueprints for computer graphics (CG) and set construction teams~\cite{gameworkshop2018, levelup2014}. During \textit{production}, their work ensures consistency as concepts are translated into tangible assets~\cite{bigbadWorld2015, randomguidebook2023}. Figure \ref{fig:importance} shows actual examples of concept designs that led to their final products in several well-known movies and games.

\begin{figure*}
    \centering
    \includegraphics[width=1\linewidth]{figures/02_Importance_of_concept_design.png}
    \caption{The figure showcases designs from concept to final product, including four well-known projects: (a) a scene from Star Wars, (b) characters from DC Comics (Harley Quinn, the Joker, and the Penguin), (c) a prop from Mad Max: Fury Road, and (d) a creature from Genshin Impact. This demonstrates the critical role of concept designers in shaping the creative vision from the earliest production stages to the final product}
    \Description{The figure illustrates the transformation from concept design to final production through four examples. A vertical arrow labeled "from Concept Design to Final Product" connects the two stages. (a) A scene from Star Wars shows an early illustration of a hangar with a spaceship, followed by a final film still of the same scene. (b) DC Comics characters, including Harley Quinn and the Joker, are depicted in initial sketches and concept art, transitioning to their final movie appearances. (c) A prop from Mad Max: Fury Road is shown as a vehicle design sketch, with the final version appearing in a film still. (d) A creature from Genshin Impact progresses from multiple concept sketches to a fully rendered in-game model. The figure highlights how concept designers shape creative vision from early ideas to production.}
    \label{fig:importance}
\end{figure*}


Concept designers undertake the majority of their work in \textit{pre-production} stage, with the workflow consisting of the following two phases~\cite{conceptart2018, bigbadWorld2015, randomguidebook2023, iterationandreference2023}:
\begin{enumerate}
    \item \textbf{Early ideation (or blue sky) phase:} 
    % After receiving a design specification from the art director/client (Figure~\ref{fig:workflow}-a), the designer research and gather information, brainstorm design ideas (Figure 3-b), explores visuals, gathering reference images from sources such as Pinterest\footnote{Pinterest, https://www.pinterest.com/}, search engines, or portfolio websites like Artstation\footnote{Artstation, https://www.artstation.com/} (Figure 3-c) that match their creative vision and generating preliminary sketches that explore various options, maintaining consistency across different settings while presenting ample variations~\cite{skillful2005}. These early concepts and references are shown to the director/client for feedback (Figure 3-d).

    This phase focuses on brainstorming and exploring initial ideas. Designers research the topic, perform visual searches, brainstorm ideas, and create preliminary sketches to propose creative options for feedback from art directors or clients. If they are not satisfied with the results, designers iterate the process until the direction of the concept is approved.
    %designer researches and explores visuals (Figure 3-b), gathering reference images (Figure 3-c) aligned with their creative vision and creating rough sketches exploring different possibilities, ensuring consistency across various environments while offering sufficient variation. These initial concepts and references are presented to the director or client for feedback (Figure 3-d).

    \item \textbf{Final concept phase:} Once initial concepts are approved, designers refine the sketches into detailed and polished designs. They enhance chosen concepts with depth, texture, and fine details to align with the project's vision. Approved final designs serve as comprehensive guides for the \textit{production} teams, which are realized through 3D modeling or set construction. The designer may provide ongoing support to ensure consistency throughout production~\cite{levelup2014}.
\end{enumerate}

This work focuses on the early ideation phase, establishing the project's creative vision and shaping its direction, style, and coherence~\cite{80lv2020,adobe2020,randomguidebook2023}. This stage demands intensive creativity and is often seen as the most exciting part of the workflow~\cite{iterationandreference2023,rassa2018concept}. 

% In early ideation phase, concept designers receive a briefing of design specification from art director/client, such as a description of what the project, the scene of the keyframe is about, and a set of references that fit their intention ~\cite{80lv2020}. Designers then analyse and disassemble the brief, into the core of the assignment, and what directions they should think of taking it in. Based on this, designers get into the \textit{ideation cycle}. They start to \textit{research} the topics extracted from the brief. During this step, they read about the subject, gather information, visual searching for a lot of image references to be knowledgable about the topic~\cite{iterationandreference2023}. When the designers consider having enough information and references gathered, they start to \textit{brainstorm} the idea. They write down the design elements they want to includes into their design, and then rough sketch multiple variation of the design ~\cite{skillful2005}. The research and brainstorming steps are often 交錯的 and iterative, the designer may think of the design idea when gathering reference, and they may want to gathering more information when they brainstorm some new idea. After multiple ideation cycle, the designer may be satisfied with few ideas (normally one to five) they come up with. They then 稍微 refine those idea對應到的sketches, and arrange the reference the sketch 參考的, as a set of design idea, to make those design idea 可以被清楚理解. and present to the director/client~\cite{80lv2020}. 

\begin{figure*}
    \centering
    \includegraphics[width=1\linewidth]{figures/03_Workflow.png}
    \caption{A typical workflow for an environment concept designer begins with receiving the design specification from the art director or client. The designer then (a) determines a potential design direction and enters the iterative ideation cycle, which includes (b) researching based on the task, and (c) brainstorming innovative ideas. Once some suitable design ideas are formed, (d) both sketches and references are presented to art directors or clients for feedback. Upon approval, (e) they refine the sketch into a polished, detailed design, which is then shared with other teams, such as the CG team.}
    \Description{The figure illustrates the iterative workflow of an environment concept designer, progressing from initial specifications to a final concept design. (a) The process begins with a design direction, defining a Victorian-era train station with key elements like Gothic infrastructure, steam trains, and a large mechanical clock. A design specification lists detailed requirements, including interior and exterior views and an angled perspective. (b) Research follows, with online searches for reference images to develop a comprehensive understanding. (c) Brainstorming organizes design elements, considering aspects like lighting, materials, and architectural details, with initial sketches. (d) Design ideas, including sketches and references, are reviewed for approval. (e) Once approved, the final concept design is refined with detailed rendering, transitioning from an outlined sketch to a fully developed environment ready for collaboration with other teams. The workflow is structured as iterative ideation cycles, emphasizing research, feedback, and refinement.}
    \label{fig:workflow}
\end{figure*}

In the early ideation phase, concept designers receive a design specification briefing from the art director or client, which includes a project description, keyframe scene details, and a set of relevant references~\cite{80lv2020}. Designers analyze the brief to identify its core elements and potential design directions (Figure \ref{fig:workflow}-a), then begin the \textit{ideation cycle}. The cycle starts with \textit{research}, where designers study the subject, gather information, explore visuals, and collect image references to develop a comprehensive understanding (Figure \ref{fig:workflow}-b)~\cite{iterationandreference2023}. This step ensures that future designs are coherent, such as maintaining historical accuracy, aligning with the period's style, or achieving mechanical and structural feasibility. With sufficient references and information, designers move to \textit{brainstorming}, where they list design elements and create rough sketches with multiple variations (Figure \ref{fig:workflow}-c)~\cite{skillful2005}. Research and brainstorming often intertwine as designers refine ideas while gathering references or seeking new material when generating fresh concepts. This iterative process helps designers gradually develop and refine their designs. After several ideation cycles, designers complete a small set of ideas they find most suitable (typically one to five), polishing the sketches and organizing the corresponding references into cohesive design ideas (Figure \ref{fig:workflow}-d). These finalized ideas are then presented to the art directors or clients for feedback~\cite{80lv2020}. Designers may repeat multiple ideation cycles until the art directors or clients are satisfied with the direction of the concept. Once approved, the process transitions to the final concept phase (Figure \ref{fig:workflow}-e).



% 細節的Workflow Stage在這邊講清楚
% Design spec generate很多idea, user可以做出更多detail research
% 以前是要分開去research再去combine,現在一次做完。
% 符合Sepc 做完research,然後去Design
% 現在的workflow是這樣: ---