

% > Line 865: should it say, “The system is really [easy] to use…”?
% > It would be useful for the authors to clarify whether the field study took place after the ideation study and whether any further development or refinement to the AIdeation system took place in-between. The field study set-up lacks some important details, such as how the participants were briefed or understood how to use the system (some of which depends on whether they were same participants from the previous activities or not?) nor does it describe what or how data was captured (e.g., interviews, surveys, diary study?).
% > 
% > The findings from the field study are brief and not currently well organised, e.g., S1 and S2 presented as distinct sections, then S3 and S4 folded into one section that also covers some detail from S1 and S2. It is not clear whether the field study offers anything new to the paper as a whole.



\section{FIELD STUDY}
After the summative study, we conducted a week-long field study across four studios to evaluate how AIdeation supported production projects in terms of quality, creativity, and efficiency, particularly with external assessment and validation by directors and clients. 
We also examined how AIdeation integrated into the design process, influenced the creative workflow, and addressed challenges or limitations encountered in real-world project settings.
% To evaluate AIdeation's practical application in concept design, we conducted a week-long field study across 4 studios. This approach allowed us to observe how AIdeation integrates into real-world workflows and assess its impact on the creative process. Our primary focus was on examining how AIdeation enhances the design process, facilitates the generation of diverse, high-quality solutions, and identifies the challenges or limitations it encounters.

\subsection{Participants: Studios and Designers}
We recruited 8 participants across 4 studios (S1-S4) from the summative study, as their familiarity with AIdeation made them well-suited to integrate AIdeation into their commercial projects. To control for selection bias, these participants had an average satisfaction score of 5.25, closely aligned with the overall mean score of 5.19 from the summative study.

S1 (P1, P16-P18) is a visual effects (VFX) studio focusing on designing environments for animation, films, TV shows, and advertisements; S2 (P3, P13) is an art outsourcing studio; S3 (P22) is an AAA game studio; and S4 (P5) is a game studio known for creating side-scrolling Metroidvania games, a subgenre of action-adventure and platformer games that are known for their non-linear exploration and progression. The professional concept design experience of the participants ranged from 2 to 11 years (mean = 4.5, SD = 2.9), similar to that experienced with an average of 4.6 years from the summative study. 
%Since participants from each studio completed one project, and we collected data from four studios, we refer to these studios as S1 through S4 in this study.

\begin{table*}[t]
\centering
\footnotesize
\begin{tabular}{|c|c|c|c|c|c|c|c|c|}
\hline
Studio ID  & Field & Task Type & AIdeation Usage \% & Other Tools Used & Env. & Ideas Gen. & Cycles & Ideas Used\\
\hline
1  & Animation, Films, TV shows &Environment Design & 40\% & Pinterest, Midjourney & 3 & 93 & 14 & 14\\
\hline
2  & Art Outsourcing &Matte Painting & 100\% & None & 1 & 105 & 10 & 5\\
\hline
3 & Game & Visual Development & 80\% & Pinterest & 3 & 652 & 45 & 28 \\
\hline
4  & Game & Environment Design & 90\% & Pinterest & 5 & 242 & 29 & 23 \\
\hline
\end{tabular}
\caption{Summary of the 4 studios' usage of AIdeation: studio's field, project tasks, usage of AIdeation among all tools used, number of environment design tasks, total ideas generated, total ideation cycles, and the number of ideas generated by AIdeation that were used in the final output of the designers.}
\Description{This table summarizes the usage of AIdeation across four studios, providing details about their fields, project tasks, and tool utilization. The table highlights the percentage of AIdeation usage among all tools, other tools used, the number of environments designed, total ideas generated, ideation cycles completed, and the number of ideas selected for final outputs.
 Studio 1:  Working in the animation, film, and TV show industry, this studio focused on environment design tasks. They used AIdeation for 40\% of their workflow alongside Pinterest and Midjourney, generating 93 ideas across 14 cycles and finalizing 14 ideas.  Studio 2:  Specializing in matte painting under art outsourcing, this studio relied exclusively on AIdeation, achieving 100\% usage. They generated 105 ideas in 10 cycles, with 5 ideas selected for their final output.  Studio 3:  A game studio focusing on visual development tasks, this studio utilized AIdeation for 80\% of its workflow alongside Pinterest. They completed 45 cycles, generating 652 ideas, and selected 28 for the final output.  Studio 4:  Another game studio working on environment design, this studio used AIdeation for 90\% of its workflow, also integrating Pinterest. They generated 242 ideas across 29 cycles, with 23 ideas selected for their final designs. Overall, the four studios collectively generated 1,092 ideas across 98 cycles, selecting 60 ideas that contributed to the creation of 12 environments. The data highlights the varying degrees of AIdeation adoption and its integration with other tools across different industries and tasks.}
\label{tab:studio_production}
\end{table*}

\subsection{Study Procedure and Evaluation}
We deployed the same AIdeation system from the summative study on AWS (Amazon Web Services) EC2, assigning each studio its own EC2 instance. 
We asked designers to incorporate AIdeation into their current projects, emphasizing its use during early ideation alongside other design tools.
We used a combination of diary studies and interviews to explore participants’ experiences with AIdeation. Participants documented their workflows, including how they used the tool, the ideas they selected, their iteration processes, and the progression from initial input to final results.

We then conducted 30-minute online interviews in which participants assessed how AIdeation impacted their design outcomes and efficiency. The participants estimated the time typically required for the project based on previous experience and compared it with the actual time spent using AIdeation. Participants also reported on directors' and clients' feedback on the results. Additionally, they shared how AIdeation supported their projects, highlighted new insights gained during its use in real-world projects, and identified areas for improvement. 
% The detailed questions are provided in Appendix X. 
Although the studios shared the final production results with the authors as part of the field study, they withheld permission for us to publish them due to NDAs and unreleased games.

\begin{table*}[t]
\centering
\footnotesize
\begin{tabular}{|c|c|c|c|c|}
\hline
Studio ID & Actual Time (AIdeation) & Estimated Time (Original) & Time Difference & Time Difference (\%) \\
\hline
1 & 1.5 Working Days & 2 Working Days & -0.5 Working Days & -25\% \\
\hline
2 & 5 Hours & 4 Hours & +1 Hour & +25\% \\
\hline
3 & 2 Working Days & 5 Working Days & -3 Working Days & -60\% \\
\hline
4 & 6 Working Days & 14 Working Days & -8 Working Days & -57\% \\
\hline
\end{tabular}

\caption{Comparison of concept design time using AIdeation (actual) vs. the same designs using existing workflow (estimated).}
\Description{Comparison of Estimated Time Using Original Workflow vs Actual Time Using AIdeation. This table provides a comparison of the time efficiency between traditional workflows and workflows incorporating AIdeation across four different studios. The table highlights both the original time estimates (without AIdeation) and the actual time taken using AIdeation for completing their respective design tasks. The table also calculates the time saved (or extended) and expresses the difference as a percentage. Studio ID: A unique identifier for each studio, linking it to the projects discussed in the study. Estimated Time (Original): The time each studio estimated they would spend on the design task using their traditional workflow (without AIdeation). Time is expressed either in working days or hours, depending on the scope of the task. Actual Time (AIdeation): The actual time spent on the same task using AIdeation. This column shows how long the process took with the AI tool, also expressed in working days or hours. Time Difference: This column indicates the difference in time between the estimated time using the traditional workflow and the actual time spent using AIdeation. A negative value indicates time saved, while a positive value shows time overrun. Time Difference (\%): The percentage change in time compared to the original workflow. A negative percentage represents a reduction in time (i.e., improved efficiency), while a positive percentage represents an increase in time (i.e., decreased efficiency). Breakdown of Studio Performance: Studio 1: Estimated Time: 2 working days Actual Time with AIdeation: 1.5 working days Time Difference: -0.5 working days Time Difference (\%): -25\% Result: Studio 1 saved 0.5 working days, resulting in a 25\% improvement in efficiency when using AIdeation. Studio 2: Estimated Time: 4 hours Actual Time with AIdeation: 5 hours Time Difference: +1 hour Time Difference (\%): +25\% Result: Studio 2 experienced a slight decrease in efficiency, taking 1 extra hour (25\% longer) to complete their task using AIdeation. Studio 3: Estimated Time: 5 working days Actual Time with AIdeation: 2 working days Time Difference: -3 working days Time Difference (\%): -60\% Result: Studio 3 saw a significant improvement, reducing their working time by 3 days, equating to a 60\% improvement in efficiency. Studio 4: Estimated Time: 14 working days Actual Time with AIdeation: 6 working days Time Difference: -8 working days Time Difference (\%): -57\% Result: Studio 4 also reported a major time reduction, saving 8 working days, which translates to a 57\% improvement in efficiency. Summary of Findings: Studio 1 and Studio 3 reported moderate and significant improvements in time efficiency, respectively, reducing their work time by 25\% and 60\% through the use of AIdeation. Studio 4 experienced the largest time savings, cutting down their work by 8 days, which resulted in a 57\% reduction. However, Studio 2 experienced a slight decrease in efficiency, taking 1 additional hour (25\% longer) than estimated, likely due to the challenges mentioned in their case study.}
\label{tab:time_comparison}
\end{table*}

\subsection{Results and Findings}
% 從Overview開始做
% 先講accross 4個的全面的結果
% 做了幾個場景,用了多久,生出多少Idea
% 可以摘要Table 1, 給個Summary
% 要講一下重點
% 這裡是給Quantitative Overview

% Table 2 的也要做一個Summary
% 還有report client / director的feedback
% \subsubsection{Integration of AIdeation into Workflow}
Table \ref{tab:studio_production} presents background information on the project each studio was working on and how they utilized AIdeation in their current project (where cycles are defined as starting a new input or using the "explore more" feature). The four studios collectively generated 1,092 ideas across 98 cycles, ultimately selecting 60 ideas, which contributed to the design of 12 environments in total.

% The following sections offer an in-depth look at two case studies (Figure 10), followed by a broader analysis of AIdeation’s performance, its support for workflow, and its impact on ideation efficiency and quality.

\begin{figure*}
    \centering
    \includegraphics[width=1\linewidth]{figures/11_Design_tools_comparison.png}
    \caption{A comparison between the initial outputs from (a) AIdeation and (b) DALL-E 3 on ChatGPT, using the same input provided by Field Study Participant S3, revealed notable differences. The participant observed that AIdeation produced designs with significantly greater diversity and richness compared to those generated by DALL-E 3 on ChatGPT.}
    \Description{The figure compares design outputs generated by (a) AIdeation and (b) DALL·E 3 on ChatGPT using the same input from a field study participant. AIdeation produces more diverse and visually rich designs, incorporating intricate structures and varied compositions. In contrast, DALL·E 3 on ChatGPT generates more uniform designs with consistent lighting and patterns. The participant noted AIdeation’s greater variety in architectural complexity and artistic expression.}
    \label{fig:design tool}
\end{figure*}


\subsubsection{Design efficiency, quality, and creativity}
Table \ref{tab:time_comparison} compares the estimated time for their original workflows with the actual time spent using AIdeation. Both S3 and S4 reported significant efficiency improvements, with time spent reduced from 5 days to 2 days and 14 days to 6 days, respectively. Both participants emphasized that AIdeation helped them identify a design direction, particularly when they were unsure how to begin working from the provided design specifications. “\textit{(AIdeation) Can quickly provide multiple directions for our team to explore and develop}”(S4). However, S2 experienced a slight decrease in efficiency, they stated that “\textit{The client didn't seem to favor the artistic style generated by AIdeation, and the image generation process was somewhat time-consuming}.”

S1, S3, and S4 reported that AIdeation significantly enhanced the quality of their final designs. All three noted that AIdeation enriched their designs with a variety of elements. Figure \ref{fig:design tool} presents a comparison between the results generated by AIdeation and DALL-E 3 in ChatGPT, using the same input provided by a participant in the field study (S3). According to the participant, AIdeation results demonstrated significantly greater diversity and richness in design. The S1 project leader stated: “\textit{The generated images might not always have the exact level of detail I needed, but I was able to extract many useful design elements}.” The participant from S4 shared that when directors saw the scenes she created using AIdeation, they were “\textit{strikingly impressed}.”

All the studios agreed that AIdeation boosted the creativity of their final design. “\textit{I couldn’t stop exploring new concepts. Every iteration sparked an eagerness to try something I hadn’t thought of before}” (S3). “\textit{There were often some great unexpected outcomes that we ended up incorporating into our designs}” (S2).

Some participants also discovered uses beyond our expectations. For instance, participants from S1 used AIdeation to generate unique patterns, which they found difficult to source online, and incorporated them into their designs. The participant from S4 suggested that sometimes, simply uploading an image without providing any prompts could still generate quite good ideas.

\subsubsection{Challenges}
Participants also reported some issues. Users from S1 and S2 mentioned the styling and aesthetics problems. 
“\textit{The generated scenes were consistently symmetrical. This limited their ability to present the outputs effectively to clients, as the composition and layout lacked variation}” (S1).

Participants desired greater controllability, noting that AIdeation lacked a gradual generation feature. “\textit{Sometimes I just wanted to remove one element from the idea or adjust the composition, but the whole image changed}” (S4). “\textit{AIdeation tended to generate overly complex designs in the initial cycle, I needed to include instructions to simplify the designs}” (S3). A user from S1 suggested adding the Inpainting functionality for more detailed control.



% In summary, AIdeation presented both benefits and challenges in creative workflows. It boosted efficiency and creativity, especially in complex projects, but may also generated overly complex designs and had limitations in composition and styling. Users appreciated its ability to quickly generate diverse ideas but also emphasized the need for more control and faster generation times. Despite these issues, AIdeation proved helpful in early-stage ideation, with potential for improvement in control and customization to enhance its effectiveness across different projects.

\subsection{Case Studies}
We selected two cases to discuss both negative and positive feedback in depth: S2 showed limitations of AIdeation and opporunities for improvement, while S4 demonstrated significant improvements in quality, efficiency, and creativity.

\subsubsection{Case study \#1 - Large mountainous environment scenes (S2)}
Two environment concept designers were tasked with creating a mountain scene featuring a narrow path winding upward to a massive rocky summit. The final design would be used as a poster and web page background. For this project, the designers had previously used MidJourney's output to communicate with their clients and switched to using AIdeation exclusively for the field study. Starting with the client’s specifications and styling keywords, they first picked 25 ideas from 10 cycles and ultimately selected 5 images for the client to confirm the design direction. The client chose 2 images as the main reference points, which the designers then used to draft the final result. Figure \ref{fig:field study}-a shows examples of generated images and selected ones. 

% 遇到的問題是畫面客戶比較難想像成品 主要是畫面排列不甚自然 
% 在景深表現上空間感比較難拉出遙遠的空間感

While the designers reported that AIdeation improved the ideation process for them, their client strongly preferred the aesthetic qualities and depth rendering of MidJourney over Dall-E, which was used by AIdeation. 
In terms of working time, the designers exceeded their initial 4-hour estimate by one hour, unlike other projects that reported significant time savings. A key factor was that this project had clear and detailed design specifications from the client, making ideation a smaller part of the task. Most of the time was spent generating images that accurately aligned with the client’s design requirements and aesthetic preferences.
%The client also struggled to visualize the final product from AIdeation's outputs, which lacked strong spatial representation. 
%AIdeation's longer generation time also contributed to them.
%The designers attributed the inefficiency to two factors: 1) the project focused more on matte painting, emphasizing aesthetics over design complexity, and 2) their lack of familiarity with AIdeation. In this case, the client's design specifications were already clear and detailed, so the designers primarily needed the tool to generate images that accurately aligned with the design requirements. 


\subsubsection{Case study \#2 - Metroidvania game environment design (S4)}
The designer used AIdeation to create three key game scenes and the backgrounds for two secondary scenes, which will be directly incorporated into the final game. We selected the most interesting case from the key scenes: a steel bridge scene. The designer started with a rough concept from another game scene and was tasked with designing a steel bridge in the same style. The bridge's structure needed to be both complex and coherent. Before using AIdeation, she had spent two days experimenting with various approaches but could not create a satisfactory design.

With AIdeation, she input the reference concept design with brief instructions. After the initial generation and only two iterations of "refining by instruction", she obtained the design elements she needed to realize her creative vision (Figure \ref{fig:field study}-b). For other tasks, she followed the same process, completing her designs and the art director was “\textit{strikingly impressed}.” The designer reported: “\textit{With a tight deadline for an upcoming game update and complex design tasks, I was prepared to work overtime for two weeks. Thanks to AIdeation providing multiple design solutions and many design elements, I managed to work overtime for just one week}” (S4).

In this case, the designer was tasked with creating complex scenes featuring intricate structures while adhering to a specific style, a process that typically demands extensive research, brainstorming, and finding suitable references. %AIdeation directly provided diverse design solutions and references, supporting the logical coherence of the designs. This enabled the designer to quickly go through multiple ideation cycles. Additionally, AIdeation's iterative exploration feature effectively helped the designer refine their ideas precisely.


% 這邊再放一個延續的work
% Studio 4 後面用了 1.5個月, 用AIdeation跑了69個cycle, 生了590個Idea, 做完了22個場景。因為這次死線是固定的,所以估計不出節省的時間,但品質是大幅提升。在使用AIdeation的這1.5個月,設計出來的場景讓主管覺得很震撼。品質超過沒有使用AIdeation前的作品。

% We are happy to share that after the field study, studio 1 and 4 keep using AIdeation. studio 1 後面用了 0.5個月,用AIdeation跑了35個cycle,生了221個Idea,做了6個場景。 用AIdeation大概節省了40 \% 的時間。其中用AIdeation生了很多特殊圖樣和雕紋大幅提升效率(原本要自己手繪)和作品品質. Studio 4 後面用了 1.5個月, 用AIdeation跑了69個cycle, 生了590個Idea, 做完了22個場景。因為這次死線是固定的,所以估計不出節省的時間,但品質是大幅提升。在使用AIdeation的這1.5個月,設計出來的場景讓主管覺得很震撼。品質超過沒有使用AIdeation前的作品。

\subsection{Continued Usage in Production beyond the Field Study}
We are excited to share that after the end of the field study, Studios 1 and 4 have continued to use AIdeation in production to date. For example, in one of Studio 1's projects, it iterated 35 cycles and 221 ideas to create 6 scenes in 2 weeks, saving approximately 40\% of the time while significantly enhancing quality, especially with intricate patterns and textures. In one of Studio 4's projects, it iterated 69 cycles and 590 ideas to create 22 scenes in 6 weeks. While time savings couldn't be measured due to a fixed deadline, the quality of their outputs greatly exceeded previous work, “\textit{The team leader was amazed by the quality of my work over the past 1.5 months}.” remarked by the participant from S4.


%In summary, AIdeation presented both benefits and challenges in creative workflows. It boosted efficiency and creativity, especially in complex projects, but may also generate overly complex designs with limitations in composition and styling. Users appreciated its ability to quickly generate diverse ideas but also emphasized the need for more control and faster generation times. Despite these issues, AIdeation proved helpful in early-stage ideation, with potential for improvement in control and customization to enhance its effectiveness across different projects.


\begin{figure*}
    \centering
    \includegraphics[width=1\linewidth]{figures/10_Field_study.png}
    \caption{The workflow and results of using AIdeation on real-world projects from two designers in two studios (S4 and S2) are as follows: (a) The designer (S4) was tasked with creating a functional steel bridge based on an existing concept design. After just two idea refinements, the designer achieved the desired result, supporting the detailed structure design that she had already spent two days conceptualizing; (b) The designer was tasked with creating a mountain scene matte painting for a webpage background. After 10 cycles of ideation, they selected five results to present to the client. However, the entire process took an hour longer than their original workflow estimate.}
    \Description{The figure presents two case studies of designers using AIdeation in professional workflows. (a) In Studio 4, a designer was tasked with creating a functional steel bridge based on an existing concept. After two refinement cycles modifying structure and style, they finalized a design that complemented their previous two days of work. (b) In Studio 2, a designer iteratively refined a mountain scene matte painting for a webpage background. After 10 ideation cycles, they selected five images to present to the client. While AIdeation supported detailed exploration, the process took an hour longer than their original workflow estimate.}
    \label{fig:field study}
\end{figure*}

% \subsubsection{Integration of AIdeation into Workflow}
% Among the four studios using AIdeation in the early ideation phase of their recent projects, S3 and S4 reported a significant boost in efficiency, S1 observed a slight improvement, while S2 experienced a slight decrease in efficiency. Table 2 compares the estimated time for their original workflows with the actual time spent using AIdeation.

% Participants from S1, tasked with designing a sci-fi arena for eSports, noted that “the generated images may not always have the exact level of detail I need,” but they were able to extract many useful design elements. They also found the keywords suggested by AIdeation helpful in identifying the right references for their project. 

% The participant from S3, an environment concept designer with 12 years of experience, was working on an alien dungeon-like setting for a video game still in development. She used AIdeation to explore visual directions, generating 652 design ideas across 45 cycles during testing. She reported that AIdeation provided excellent early-stage ideas and offered a diverse range of design directions far more efficiently than traditional methods or other AI tools. Each cycle brought surprising new ideas, significantly boosting her creativity. She remarked, “I couldn’t stop exploring more ideas and starting another cycle with AIdeation. Every iteration made me excited about the generated result. Even though not all the ideas were exactly what I wanted, the system sparked an eagerness to explore new concepts that I’ve never experienced before.” 

% Participants also reported some issues with AIdeation. Users from S1 mentioned that the generated images often contained odd floating UI elements in their sci-fi settings, and the scenes were consistently symmetrical. This limited their ability to present the outputs effectively to clients, as the composition and layout lacked variation.

% Participants from S3 and S4 expressed a desire for greater controllability, noting that AIdeation lacks a gradual generation feature. "Sometimes I just want to remove one element from the idea, but the whole image changes." (S4)  One participant also noted that AIdeation tended to generate overly complex designs in the initial cycle, requiring them to include instructions to simplify the designs. (S3)

% In summary, AIdeation presented both benefits and challenges in creative workflows. It boosted efficiency and creativity, especially in complex projects, but may also generated overly complex designs and had limitations in composition and styling. Users appreciated its ability to quickly generate diverse ideas but also emphasized the need for more control and faster generation times. Despite these issues, AIdeation proved helpful in early-stage ideation, with potential for improvement in control and customization to enhance its effectiveness across different projects.