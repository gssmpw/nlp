\section{INTRODUCTION}
\label{sec:introduction}

% The entertainment industry's production process, whether for films, TV shows, or video games, transforms creative ideas into finished products through a series of four stages: 1) \textit{development}, where the initial concept and creative direction are set; 2) \textit{pre-production}, involving detailed planning and preparation; 3) \textit{production}, where the main content is created; and 4) \textit{post-production}, which includes editing, enhancing, and polishing the final product~\cite{gameworkshop2018, bigbadWorld2015, directing2020, Singh2023Artificial, filmmaker2019}.

% Concept designers are pivotal across the first three stages, particularly the \textit{pre-production} stage. During \textit{development}, concept designers collaborate with art directors/clients to visualize core ideas through initial sketches and designs to define the project's aesthetic and visual tone~\cite{artofgame2008}. During \textit{pre-production}, they design sets, characters, environments, and props to provide blueprints for computer graphics (CG) and set construction teams~\cite{gameworkshop2018, levelup2014}. During \textit{production}, their work ensures consistency as concepts are translated into tangible assets~\cite{bigbadWorld2015, randomguidebook2023}. Figure 2 shows actual examples of concept designs that led to their final production in several well-known movies and games.

% \begin{figure*}
%     \centering
%     \includegraphics[width=1\linewidth]{figures/02 Importance of Concept Design.png}
%     \caption{The figure showcases designs from concept to final product, including four well-known projects: (a) a scene from Star Wars, (b) characters from DC Comics (Harley Quinn, the Joker, and the Penguin), (c) a prop from Mad Max: Fury Road, and (d) a creature from Genshin Impact. This demonstrate the critical role of concept designers in shaping the creative vision from the earliest stages of design to the final production}
%     \label{fig:enter-label}
% \end{figure*}


% Concept designers undertake the majority of their work in \textit{pre-production} stage, with workflow consisting of the following two phases~\cite{conceptart2018, bigbadWorld2015, randomguidebook2023}:
% \begin{enumerate}
%     \item \textbf{Blue Sky or Early Ideation (Research, Brainstorming, Sketching)} 
%     After receiving a design specification from the art director/client (Figure 3-a), the designer brainstorm design ideas (Figure 3-b), explores visuals, gathering reference images from sources such as Pinterest\footnote{Pinterest, https://www.pinterest.com/}, search engines, or portfolio websites like Artstation\footnote{Artstation, https://www.artstation.com/} (Figure 3-c) that match their creative vision and generating preliminary sketches that explore various options, maintaining consistency across different settings while presenting ample variations. These early concepts and references are shown to the director/client for feedback (Figure 3-d).
%     %designer researches and explores visuals (Figure 3-b), gathering reference images (Figure 3-c) aligned with their creative vision and creating rough sketches exploring different possibilities, ensuring consistency across various environments while offering sufficient variation. These initial concepts and references are presented to the director or client for feedback (Figure 3-d).

%     \item \textbf{Refinement and Final Concept} Once initial concepts are approved, the designer refines the sketches into detailed, polished designs. They enhance chosen concepts with depth, texture, and fine details to align with the project's vision. (Figure 3-e) Approved final designs serve as comprehensive guides for the \textit{production} teams, to be realized through 3D modeling or set construction. The designer may provide ongoing support to ensure consistency throughout production~\cite{levelup2014}.
% \end{enumerate}

% 11/15
Concept design is the initial step in visual development within the entertainment industry, including films, TV shows, and video games~\cite{adobe2020, artofgame2008, bigbadWorld2015, randomguidebook2023, gameworkshop2018, levelup2014}.
Concept designers collaborate with art directors to visualize core ideas through initial sketches to
define the aesthetic and visual tone of the projects~\cite{artofgame2008}, along with detailed character, environment, and prop design to provide blueprints for computer graphics (CG) and set construction teams~\cite{gameworkshop2018, levelup2014}.
For fantasy and sci-fi settings, everything needs to be designed and created by concept designers~\cite{adobe2020}. 

The concept design workflow consists of two main phases~\cite{bigbadWorld2015, conceptart2018, randomguidebook2023}: 1) \textit{early ideation (or blue sky) phase}, where raw ideas are researched, brainstormed, and explored, followed by the 2) \textit{final concept phase}, where approved concepts are refined with detailed specifications for use by the production teams.
Literature has described several challenges during the early ideation phase~\cite{bigbadWorld2015, 80lv2020, artstation2021, conceptart2018}, including difficulties in finding references that align with their creative vision~\cite{son2024genquery, li2022analyzing, bigbadWorld2015} combined with extreme time constraints with designers being expected to deliver multiple designs daily~\cite{bigbadWorld2015, 80lv2020, artstation2021}, resulting in limited depth of research~\cite{bigbadWorld2015, conceptart2018} and limited unique designs incorporating diverse elements explored (i.e. limited breadth)~\cite{bigbadWorld2015, 80lv2020, interview2020}. 
%Futhermore, Inefficient
%communication between directors and designers, whether due to vague instructions or the challenge of translating
%complex ideas into visual formats, often results in multiple revisions, further complicating the process
%translating complex ideas into visual formats that meet the , often require in multiple revisions, and ~\cite{80lv2020, bigbadWorld2015, interview2020}.

With the rapid advancement of generative AI (GenAI), concept designers and studios have adopted them into their workflows~\cite{boucher2024resistance, ko2023large, filmhandbook2024, vimpari2023adapt, qin2023does}. Recent studies highlight challenges faced by artists and designers in integrating GenAI into their workflows. Crafting effective prompts often requires significant trial and error, as users struggle with optimal structures~\cite{mahdavi2024ai}. Additionally, many tools lack intuitive design, posing barriers for non-technical users~\cite{shi2023understanding}. GenAI's single-step generation contrasts with the iterative, reflective practices of human creativity~\cite{zhang2024confrontation}, and outputs often misalign with users’ visions, necessitating extensive fine-tuning and post-editing~\cite{mahdavi2024ai, vimpari2023adapt}. However, recent research in HCI shows promise in better integrating GenAI into traditional creative workflows, such as graphic design~\cite{choi2024creativeconnect, son2024genquery}, animation ~\cite{Tseng2024KeyframerEA}, fashion design~\cite{davis2024fashioning, rw29}, and interior design~\cite{wang2024roomdreaming, hou2024c2ideas}. Motivated by the success of these advancements, this work aims to develop a flexible system that leverages concept designers' domain knowledge, focusing on facilitating iterative ideation and enhancing their workflows.

We structure this work into four parts: 1) a formative study to understand the workflow, ideation processes, and needs of concept designers using both traditional methods and AI tools; 2) the design and implementation of a human-AI collaborative ideation system; 3) a summative study focusing on the ideation process; and 4) a field study in real-world commercial projects to assess its quality and efficiency of the final design outcomes.

We conducted a formative study with 12 professional environment concept designers. Among character, prop, and environment concept designs, we focused on environment concept design for the scope of this paper because it typically requires the most concept design resources and involves designing complex spatial and visual elements across both large (macro) and detailed (micro) scales. Through in-depth interviews and workflow analyses, we examined participants' design processes, reference-gathering strategies, and their use of GenAI tools.
Designers often struggle to gather diverse, relevant references, especially for unique or poorly defined topics. Traditional search tools often do not align with the creative intentions of designers or provide sufficient material to blend different styles and themes.
Additionally, generating multiple unique design variations under tight deadlines is a major obstacle, with participants citing limited time, insufficient reference diversity, and the extensive effort required to create complex designs. While GenAI tools offer potential, participants noted critical limitations, including difficulties in formulating prompts, low diversity in outputs, AI hallucinations, lack of detailed information, and limited flexibility for refining results. These findings highlight the need for tools that better support concept design workflows' iterative and exploratory nature.

% before RR
% %to reduce the time required to create final
% %visual presentations for clients and directors. 
% However, most designers reported that current AI tools are only being used in the later phase of their workflow after they have completed the research and brainstorming phase because most AI design tools focus only on generating images based on user specifications, 
% %overlooking the specific needs and complexity of their unique ideation workflow. 
% lacking support for concept designers' unique ideation workflow that requires rapid iterations of research and brainstorming. 

%are not designed to fit seamlessly into a concept designer’s workflow and differ significantly from their
%usual working habits. Moreover, most AI design tools focus only
%on generating images based on user specifications, overlooking the specific needs and complexity of their unique ideation workflow.


% Formative Study
%In response to these challenges, we conducted a formative study
%with 12 professional environment concept designers(Working experience range from 1-11 years) to delve deeper into their creative workflows and ideation processes, aiming to identify the obstacles they encounter both in their traditional workflow and when using AI design tools. We chose to focus on environment concept design because it involves complex spatial and visual elements that require high creativity and attention to detail. This exploration was carried out through workflow observations, case studies, and interviews. Here is the summary of our findings:
% 1. In early ideation stage, traditional search tools often do not efficiently assist designers in finding the right design direction or creating diverse design alternatives, particularly when the topic is uncommon. Several factors contribute to this. First, designers may struggle to find images that align with their creative intentions, either due to difficulties in formulating appropriate search queries or simply because the desired references are unavailable. Second, these tools often provide limited diversity or irrelevant results, making it challenging to generate innovative ideas. Lastly, creating visual designs from various design elements requires significant effort, further complicating the process. 
% 2. There are no consistent usage patterns for AI tools among designers. 6 designers only utilised AI tools when they had a concrete idea, relying on the tools to help visualise their designs. 3 designers focused on modifying prompts to generate design ideas and explore visual elements. Meanwhile, 3 designers opted not to use the tools at all. 
% We identified that several challenges make it difficult to integrate AI design tools into concept designers' workflows. First, most current tools are text-prompt-based, while designers typically operate within an image-based workflow, making it challenging to create prompts that produce the desired results. Second, the generated images often lack clear information on specific elements, making it hard to extract visual components as keywords for finding related references. Third, the results generated by these tools tend to show little variation with similar prompts. Lastly, the low controllability and flexibility of these tools make it difficult to manipulate the generated images effectively. These factors collectively make current AI image-generation tools less suitable for efficient visual idea exploration.


% System Design
Based on our observations, we designed AIdeation to bridge the gap between GenAI and concept design, enhancing the early ideation phase of concept designers. The key components of AIdeation include: 
\begin{enumerate}
    \item \textbf{Brainstorming: Supporting Breadth Exploration:} AIdeation generates a wide variety of diverse design ideas based on user input, which can be in the form of natural language, images, or both (Figure \ref{fig:hero image}-a). These design ideas are presented visually, providing an overview that helps designers quickly grasp different directions while offering high-diversity variations for further exploration. 
    \item \textbf{Research: Supporting Depth Exploration:} AIdeation extracts key design elements from the generated ideas and presents them as keywords to help designers explore further into the visual elements. These keywords also link to corresponding search results, supporting the reference-gathering process (Figure \ref{fig:hero image}-b).
    \item \textbf{Refining Idea: Supporting Flexible Iterative Exploration:} AIdeation allows designers to refine their ideas through an iterative process (Figure \ref{fig:hero image}-c). Users can refine their designs by combining them with additional references or issuing natural language instructions to adjust specific elements. This flexibility helps designers experiment with both broad and focused refinements, aligning design ideas with their creative intent while maintaining design diversity.
\end{enumerate}

% Evaluation and Result
To evaluate AIdeation, we conducted a summative study focusing on the ideation process and a field study to examine its impact in real-world settings and the final design outcomes. The summative study employed a within-subjects design with 16 professional environment concept designers, using their original workflow as the baseline. The study simulated real-world tasks where designers were assigned topics involving both exterior and interior scenes. Findings showed that participants significantly preferred AIdeation for enhancing creativity (\textit{p} = 0.001), found it more efficient for generating diverse ideas (\textit{p} = 0.003) while maintaining comparable quality, and reported higher satisfaction (\textit{p} = 0.005) and enjoyment (\textit{p} = 0.005) with AIdeation.


%Designers were asked to explore visuals and gather references based on the topic, using either their original workflow or AIdeation. There were no restrictions for the baseline; designers could use any preferred tools, including traditional or AI-based tools. They were required to collect reference sets for both interior and exterior scenes to present to the client and support future design work. Findings from our study showed that compared to their original workflows, participants significantly preferred AIdeation for enhancing breadth (\textit{p} = 0.014) and depth (\textit{p} = 0.033) of idea exploration, flexibility (\textit{p} = 0.046), and creativity (\textit{p} = 0.001). They also found AIdeation more efficient for generating diverse ideas (\textit{p} = 0.003) while maintaining comparable quality. Furthermore, participants reported higher satisfaction (\textit{p} = 0.005), greater enjoyment (\textit{p} = 0.005), and reduced task difficulty (\textit{p} = 0.005) with AIdeation, as well as improved support for gathering design information (\textit{p} = 0.009) and visually presenting ideas (\textit{p} = 0.004). While some participants highlighted limitations, such as restricted control over specific design modifications or slower image generation, AIdeation demonstrated the potential to improve ideation workflows and support creative processes effectively.

% field Study
For the field study, we collaborated with 4 design studios and 8 professional environment concept designers, who used AIdeation as part of their ongoing commercial projects for one week. 
All studios reported improved creativity, with 3 reporting improved efficiency and quality.
%Two studios reported significant efficiency gains: a Metroidvania game designer reduced their workload from 14 to 6 days, while a AAA studio designer completed a task in 2 days instead of 5. 
After the completion of the field study, 2 studios have continued using AIdeation for commercial projects to date.
%, with one studio reporting a significant quality boost based on feedback from their director.






%Participants from two studios reported a significant boost in efficiency; for instance, a designer working on a Metroidvania game reduced their workload from 14 days to 6 days, while another designer at a AAA game studio completed a task in 2 days instead of 5. In other cases, such as a matte painting project, AIdeation showed limitations, slightly increasing the time required due to challenges with aesthetics and spatial representation. Despite this, the tool demonstrated potential for early ideation in real-world scenarios, with designers expressing satisfaction and an interest in incorporating it into future projects.

% Contribution
In summary, our key contributions are as follows:
\begin{itemize}
    \item An in-depth understanding of concept designers' workflows in the early ideation stage and how AI design tools are currently used in practice. 
    \item The design and implementation of a system, AIdeation, that allows concept artists to rapidly explore creative ideas through a flexible, iterative approach. Designed with a human-centered AI process, it addresses key barriers to GenAI adoption, including AI factual accuracy and lack of transparency/creative control.
    \item Empirical evidence that AIdeation improves creativity, satisfaction, and efficiency of concept artists' workflow through: 1) a comparative study with 16 professional concept designers; 2) a field deployment to production use in 4 studios; and 3) continued usage to date by two studios after the completion of the field study.
\end{itemize}
%We d
%AIdeation bridges the gap between concept designers and AI design tools by incorporating domain knowledge and offering an intuitive design interface, making it easy to use and enhancing the designer's workflow in real-world projects.


