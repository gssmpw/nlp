%% Nice command to hyperlink email addresses
% \newcommand{\email}[1]{\href{mailto:#1}{\nolinkurl{#1}}}

%% References
\newcommand{\Sec}[1]{\hyperref[sec:#1]{\S\ref*{sec:#1}}} %section
\newcommand{\Section}[1]{\hyperref[sec:#1]{Section~\ref*{sec:#1}}} %section
\newcommand{\App}[1]{\autoref{sec:#1}} %appendix
\newcommand{\AppFull}[1]{\hyperref[sec:#1]{Appendix~\ref*{sec:#1}}} %appendix
\newcommand{\Eqn}[1]{\hyperref[eq:#1]{(\ref*{eq:#1})}} %equation
\newcommand{\Fig}[1]{\hyperref[fig:#1]{Figure~\ref*{fig:#1}}} %figure
\newcommand{\Tab}[1]{\hyperref[tab:#1]{Table~\ref*{tab:#1}}} %table
\newcommand{\Thm}[1]{\hyperref[thm:#1]{Theorem~\ref*{thm:#1}}} %theorem
\newcommand{\Cor}[1]{\hyperref[cor:#1]{Corollary~\ref*{cor:#1}}} %corollary
\newcommand{\Alg}[1]{\hyperref[alg:#1]{Algorithm~\ref*{alg:#1}}} %algorithm
\newcommand{\Def}[1]{\hyperref[def:#1]{Definition~\ref*{def:#1}}} %definition

%% Black board fonts
\newcommand{\Real}{{\mathbb R}}
\newcommand{\Complex}{{\mathbb C}}

%% Quick way to switch from displaymath to an inline mode
\newenvironment{inlinemath}{$}{$}

%% Comments to each other, etc.
\newcommand{\Note}[1]{{\color{red}\sf#1}}
\newcommand{\jc}[1]{{\color{orange}\sf JC: #1}}
\newcommand{\jcmod}[2]{{\color{orange}\sout{#1} #2}}
\newcommand{\revisit}[1]{{\color{blue}\sf #1}}


%% Misc things
\newcommand{\Pval}{P} % symbol for R times the sum of the sizes of a tensor
\newcommand{\Qval}{Q} % symbol for the product of the sizes


%%% Local Variables:
%%% mode: latex
%%% TeX-master: "paper"
%%% TeX-command-default: "PDFLaTeX"
%%% LaTeX-command-style: (("." "pdflatex")) 
%%% End:
