\documentclass[journal,twoside,web]{IEEEtran}





\IEEEoverridecommandlockouts  % This command is only needed if 
                                                      % you want to use the \thanks command

%\overrideIEEEmargins % Needed to meet printer Prequirements.

\usepackage{ulem} %临时用一下,投稿前删除


\usepackage{cite}
\usepackage{amssymb,mathtools,amsfonts,amsthm,amsmath,}

\usepackage{enumerate} 
\usepackage{enumitem}


\newcommand{\Rom}[1]{\MakeUppercase{\romannumeral #1}}

\usepackage{algorithm}
\usepackage[noend]{algpseudocode}
\renewcommand{\algorithmicrequire}{\textbf{Input:}}
\renewcommand{\algorithmicensure}{\textbf{Output:}}





\usepackage{xcolor}
\usepackage{graphicx}
\usepackage{subfigure}
\graphicspath{{./Figure/}}



\theoremstyle{remark}
\newtheorem{remark}{Remark}
\newtheorem{lemma}{Lemma}
\newtheorem{theorem}{Theorem}
\newtheorem{proposition}{Proposition}
\newtheorem{definition}{Definition}
\newtheorem{example}{Example}



\title{%Property Verification under Decentralized Processing of Partially Ordered Observation Sequences
%\\State Estimation and Event Inference under Decentralized Processing of Partially Ordered Observation Sequences\\
%Decentralized Processing and Property Verificaition of Partially Ordered Observation Sequences}
Decentralized State Estimation and Opacity Verification Based on Partially Ordered Observation Sequences}


\author{Dajiang Sun, Christoforos N. Hadjicostis, \IEEEmembership{Fellow, IEEE} and Zhiwu Li, \IEEEmembership{Fellow, IEEE}
	\thanks{Dajiang Sun is with the School of Electro-Mechanical Engineering, Xidian University, Xi'an, 710071, China (e-mail: djsun@stu.xidian.edu.cn).}%
	\thanks{Christoforos N. Hadjicostis is with the Department of Electrical and Computer Engineering, University of Cyprus, Nicosia, Cyprus (e-mail: hadjicostis.christoforos@ucy.ac.cy).}
	\thanks{Zhiwu Li is with the Institute of Systems Engineering, Macau University of Science and Technology, Taipa 999078, China, and also with the School of Electro-Mechanical Engineering, Xidian University, Xi'an, 710071, China (e-mail: zhwli@xidian.edu.cn).}
%  \thanks{\textcolor{red}{V-1, February 15, 2024--February 19, 2024, Restart on April 8, 2024}}
%  \thanks{\textcolor{red}{V-2, June 12, 2024--, 2024}}
	}


\begin{document}

%\thispagestyle{empty}
%\pagestyle{empty}
\pagenumbering{arabic}
\maketitle

\begin{abstract}
In this paper, we investigate state estimation and opacity verification problems within a decentralized observation architecture.
Specifically, we consider a discrete event system whose behavior is recorded by a set of observation sites. 
These sites transmit the partially ordered sequences of observations that they record to a coordinator whenever a \textit{synchronization} occurs.
To properly analyze the system behavior from the coordinator's viewpoint, we first introduce the notion of a \textit{Complete Synchronizing Sequence structure} (CSS structure), which concisely captures the state evolution of each system state upon different information provided by the observation sites.
Based on the CSS structure, we then construct corresponding current-state and initial-state estimators for offline state estimation at the coordinator.
When used to verify state-isolation properties under this decentralized architecture, the use of CSS structure demonstrates a significant reduction in complexity compared with existing approaches in the literature.
In particular, we discuss how to verify initial-state opacity at the coordinator, as well as a novel opacity notion, namely current-state-at-synchronization opacity.
%two \textcolor{red}{novel opacity notions, namely initial-state opacity} at the coordinator and current-state-at-synchronization opacity.
%\textcolor{red}{\sout{In particular, we show that the CSS structure construction leads to a polynomial-time approach for verifying diagnosability, which has not been demonstrated in the literature. Additionally, due to the nature of information-flow in this decentralized architecture, we propose a new notion, called current-synchronization opacity, to characterize whether, under certain circumstances, an outside observer can determine that the system will inevitably pass through a secret state before the next synchronization step.}}
%\textcolor{blue}{Finally, we discuss that the proposed method can be used to verify diagnosability in the decentralized observation architecture.}
\end{abstract}
\begin{IEEEkeywords}
Decentralized state estimation, discrete event system, finite state automaton, opacity, synchronization.
\end{IEEEkeywords}


\section{Introduction}
\IEEEPARstart{W}{ith} the proliferation of information technology, network connectivity, and sensing capabilities, there is an increasing utilization of highly complex systems composed of multiple components that collaborate to fulfill intricate tasks, such as communication networks, autonomous vehicles, and other advanced cyber and cyber-physical systems.
These systems are often mathematically captured using discrete event systems (DESs), which are typically characterized by discrete states and event-triggered dynamics.
DESs are widely used in modeling and analyzing the higher logical behavior of these complex systems.
Due to the limited sensing capability of a plant, we may not have full information about the system’s internal state, which is needed to solve an array of control problems and/or to verify certain properties of the system. 
Therefore, the state estimation problem \cite{Ramadge1986} is of great importance in practical applications. 

Depending on the time instant at which state estimates are needed, different notions have been proposed and studied, such as current-state estimation, delayed-state estimation and initial-state estimation \cite{Hadjicostis2020, WangLafortuneLin2007, LinWangHanShe2020, HanWangLiChenChen2023}. 
Dealing with these estimation tasks depends on the observed system behavior. 
Normally, an observation architecture consists of a system itself, an observation site that observes the system, and a computational unit that analyzes the data provided by the observation site. 
However, emerging complex and large-scale systems have given rise to decentralized observation structures \cite{DeboukLafortuneTeneketzis2000}, where there may not exist a monolithic observation site that is able to record all accessible information generated by the system. 
In this paper, we concentrate on initial- and current-state estimation (and corresponding property verification problems) under a decentralized observation structure, where the information received by the coordinator is a set of partially ordered sequences of observations recorded at local sites.


State estimation serves as the basis for various so-called state isolation properties, such as diagnosability \cite{Sampath1995, YooLafortune2002,JiangHuang2001}, detectability \cite{ShuLinYing2007,YinLiWang2018}, and opacity \cite{SabooriHadjicostis2007,JacobLesageFaure2016}. 
In the absence of a monolithic observer, two main architectures---decentralized and distributed---are used to describe property verification.
Debouk \textit{et al}. \cite{DeboukLafortuneTeneketzis2000} focus on the case of two observation sites and propose three protocols for coordinated decentralized diagnosis, with the third protocol being directly linked to  \textit{co-diagnosability}. 
A polynomial time approach for the verification of co-diagnosability is proposed in \cite{QiuKumar2006}.
Furthermore, the co-diagnosability of a networked DES, considering communication delays and intermittent losses of observations, is addressed in \cite{NunesMoreiraAlvesCarvalhoBasilio2018}.
The work in \cite{TakaiUshio2012} studies a decentralized fault diagnosis problem modeled by Mealy automata under state-dependence and nondeterministic output functions.
Additionally, co-diagnosability and $K$-co-diagnosability are explored within the framework of Petri nets \cite{RanSuGiuaSeatzu2018}.
The problems of decentralized diagnosis under disjunctive and conjunctive decision-making processes are analyzed in \cite{TomolaCabralCarvalhoMoreira2017, TaKaiKumar2017}.
A general distributed protocol for fault diagnosis and state estimation, called DiSIR, is studied in \cite{KeroglouHadjicostis2018} in the absence of a coordinator.
In \cite{OliveiraCabralMoreira2022}, the authors address the problem of robust decentralized diagnosis of DES against transient failures in the communication of the observations to the local diagnosers.
Li \textit{et. al.} \cite{LiHadjicostisWu2021}  deal with decentralized fault diagnosis when the information received by the coordinator involves state estimates, 
possible corrupted due to attacks on the communication channels.
For more details, readers are referred to survey papers \cite{ZaytoonLafortune2013,LafortuneLinHadjicostis2018,BasilioHadjicostisSu2021}  and book \cite{Hadjicostis2020} on this topic.
 
 
 
The notion of opacity introduced in \cite{Mazare2004, BryansKoutnyMazarRyan2008} for transition systems ensures that an intruder cannot determine whether a predicate representing secret information is true.
Since then, various notions related to opacity (e.g., initial-state, current-state, delayed step, and infinite-step opacity), depending on different security requirements, have drawn considerable attention in the context of DESs modeled by automata \cite{SabooriHadjicostis2007,SabooriHadjicostis2008ini,SabooriHadjicostis2011,WuLafortune2013, YinLafortune2017}. 
The verification of these notions in systems modeled by Petri nets has also been explored; see, e.g., \cite{YongLiSeatzuGiua2017, TongLanSeatzu2022,CongFantiManginiL2018, CongFantiManginiLi2019}.
The works in \cite{DongWuLi2024,YangDengQiuJiang2021} extend the notion of opacity to networked discrete event systems, considering communication losses and delays.
Recently, the concept of strong opacity is established in \cite{FalconeMarchand2015,MaYinLi2021,HanZhangZhangLiChen2023}, which focuses not only on whether a system is in a secret state at a given instant but also on whether the past visit to a secret state can be determined by an outside observer.
In \cite{PaoliLin2012}, decentralized opacity is investigated by considering whether there is a coordinator among local sites or not. In \cite{WuLafortune2013}, the authors define several notions of joint opacity based on a decentralized structure where the information sent to the coordinator from local intruders consists of  state estimates.
Zhu \textit{et. al.} in \cite{ZhuLiWu2022} introduce $K$-step opacity in centralized and decentralized structures within the framework of Petri nets.








In general, decentralized information processing depends on three components: 
1) the kind of information that a local site sends to the coordinator; 
2) the instants at which synchronizations are initiated (i.e., when the local sites send local information or decisions to the coordinator); 
3) the rules that the coordinator follows to calculate global information. The first ingredient could be a sequence of observations, a set of state estimates, or a local decision (depending on the local site’s own observation). 
The second ingredient is the synchronization strategy, whereas the last ingredient correlates with the goal that we wish to perform (e.g., fault diagnosis). 
The key feature in a decentralized information setting is that no information is sent back to or communicated among local sites, which is significantly different from a distributed observation setting where such feedback is allowed.




In our previous work \cite{SunHadjicostisLi2023}, we investigated state estimation at a single synchronization step under decentralized observation-based information processing (DO-based protocol), where the information sent to the coordinator by each observation site (OS) is a sequence of observations.
However, this method is insufficient for a comprehensive understanding of the system's overall behavior, as is needed, for instance, to verify certain properties under this protocol.
In this paper, under a DO-based protocol with a given synchronization strategy, we develop a systematic methodology for analyzing properties of interest, as induced by the specific synchronization strategy, by defining a \textit{Complete Synchronizing Sequence structure} (CSS structure).
We will show that under the framework of the DO-based protocol, any synchronization strategy can be interpreted as a CSS structure.
The CSS structure essentially encompasses all information that may be received by the coordinator, as well as the corresponding state evolution of each state upon receiving this information.
Consequently, once the CSS structure has been constructed, the problem of state estimation can be addressed by simply taking unions of states without a complex process at each synchronization step.
Moreover, based on the CSS structure, we present the notions of current-state and initial-state estimator under the DO-based protocol.

Subsequently, we propose and verify state isolation properties, such as %\textcolor{red}{\sout{diagnosability,}} 
initial-state opacity and current-state-at-synchronization opacity under the DO-based protocol.
%\textcolor{red}{Compared with the approach for verifying this kind of diagnosability in \cite{Hadjicostis2020}, the method presented here has lower complexity. Additionally, we present an approach to verify this type of diagnosability, which is of polynomial complexity thanks to the CSS structure. Specifically, due to the nature of the synchronization process, several observations may occur between two consecutive synchronizations.}
Current-state-at-synchronization opacity is a new notion, proposed to capture the situation where an outside observer can never be certain whether the system is in a secret state immediately after a synchronization.
%\textcolor{red}{\sout{To the best of our knowledge, the notion of current-synchronization opacity is introduced for the first time in this paper.}}
The remainder of this paper is organized as follows.
In Section \Rom{2}, we review some background on language and automata theory, and formulate the DO-based protocol under a specific synchronization strategy. 
The main contributions are developed as follows.
\begin{itemize}
	\item In Section \Rom{3}, the notion of CSS structure and its properties/complexity are presented; we also construct the feasible CSS structure, from which the current-state and initial-state estimators can be obtained.
	%\item \textcolor{red}{\sout{In Section \Rom{4}, DO-based diagnosability is formally defined and the DO-diagnoser, constructed based on the current-state estimator, is used to detect the occurrence of a fault event. Moreover, a DO-verifier with polynomial time complexity is introduced to verify this property.}}
	\item In Section \Rom{4}, the verification of DO-based initial-state opacity is proposed using two approaches, one with doubly-exponential complexity and the other with exponential complexity. 
    Additionally, the concept of DO-based current-state-synchronization opacity is defined and its verification is developed.
    We also delve into the discussion on how to verify diagnosability under the DO-based protocol.
\end{itemize}

Section \Rom{5} concludes this article. 
Note that there is no overlap in contributions between this paper and our work in~\cite{SunHadjicostisLi2023} that focuses on online state estimation; this paper assumes instead a fixed synchronization strategy and focuses on the verification of properties of interest.


\section{Preliminaries}\label{sec-2}

\subsection{System Model}

Let $\Sigma$ be a finite set of symbols (events). 
A string over $\Sigma$ is a sequence of $n$ events, i.e., $s=\alpha_1\alpha_2\dots\alpha_i\dots\alpha_n$, $\alpha_i\in\Sigma$, $i\in\{1,2,\dots,n\}$.
The length of $s$ is the number of events in the sequence, denoted by $|s|$. 
We denote by $\Sigma^*$ the set of all finite-length strings over $\Sigma$, including the empty string $\epsilon$ with $|\epsilon|=0$. 
A language $L\subseteq\Sigma^*$ is a set of strings \cite{Hadjicostis2020, CassandrasLafortune2008}. 
Given strings $s$, $t$ $\in \Sigma^*$, the concatenation of $s$ and $t$ is the string $st$ (also denoted as $s\cdot t$), i.e., the sequence of symbols in $s$ followed by that in $t$. 
For any $\sigma\in\Sigma$, $s\in\Sigma^*$, we use $\sigma\in s$ to denote that $\sigma$ occurs in $s$, i.e., $s=u\sigma v$ for some strings $u,v\in\Sigma^*$.
%\textcolor{blue}{We also write $t\in s$ if string $t$ is a sub-string of $s$, i.e., $s=utv$ for some strings $u,v\in\Sigma^*$.}\footnote{\textcolor{blue}{Useful?}}
Let $\bar{s}$ be the prefix-closure of $s$, i.e., $\bar{s}=\{t\in\Sigma^*|\exists t'\in\Sigma^*: tt'=s\}$, and $s/t$ be the symbol sequence after $t$ in $s$, i.e., for $t\in\overline{s}$, we have $t\cdot(s/t)=s$ (note that $s/s=\epsilon$ and $s/\epsilon=s$).

A DES considered in this paper is modeled as a nondeterministic finite automaton (NFA) $G=(X, \Sigma, \delta, X_0)$, where $X$ is the finite set of states, $\Sigma$ is the finite set of events, $\delta:X\times\Sigma\rightarrow 2^X$ is the next-state transition function, and $X_0$ is the set of possible initial states. 
For a set $X'\subseteq X$ and $\sigma\in\Sigma$, we define $\delta(X',\sigma)=\bigcup_{x'\in X'}\delta(x',\sigma)$;
with this notation at hand, the transition function $\delta$ can be extended recursively to $\delta^*$ (denoted by $\delta$ for the sake of brevity) whose domain is $X\times\Sigma^*$ instead of $X\times\Sigma$:
$\delta(x,\epsilon)=x$ and $\delta(x,\sigma s)=\delta(\delta(x,\sigma),s)$ for any $x\in X$, $\sigma\in\Sigma$, and $s\in\Sigma^*$. 
The system execution (or behavior) of $G$ starting from state $x$ is captured by $L(G,x)=\{s\in\Sigma^*|\delta(x, s)\neq\emptyset\}$.
For a set $X'\subseteq X$, we define $L(G,X')=\bigcup_{x'\in X'}L(G,x')$.
The system behavior generated by $G$ is described by $L(G)=L(G,X_0)$.
%\textcolor{blue}{If there exists a set of marked states $X_m$, $X_m\subseteq X$, the system model is extended to $G=(X, \Sigma, \delta, X_0, X_m)$ and the marked behavior of $G$ is $L_m(G)=\{s\in\Sigma^*|\exists x_0\in X_0,\delta(x_0,s)\cap X_m\neq\emptyset\}$.}\footnote{\textcolor{blue}{Useful?}}


\subsection{Decentralized Observation-Based Information Processing}

We assume that there are $m$ observation sites $O_i$ and let $\mathcal{I}=\{1,2,\ldots,m\}$ be an index set. 
For $i\in \mathcal{I}$, we denote by $\Sigma_{o_i}$ the set of events that can be observed by OS $O_i$. 
The natural projection function $P_{\Sigma_{o_i}}:\Sigma^*\rightarrow\Sigma^*_{o_i}$ maps any system behavior to the sequence of observations associated with it, defined recursively as 
\begin{align*}
	P_{\Sigma_{o_i}}(\epsilon)=\epsilon 
	\quad 
	\text{and}
	\quad
	P_{\Sigma_{o_i}}(s\sigma)=
	\begin{cases}
		P_{\Sigma_{o_i}}(s)\sigma, &\text{if} \quad  \sigma\in \Sigma_{o_i},\\
		P_{\Sigma_{o_i}}(s),& \text{if} \quad \sigma\notin \Sigma_{o_i}.
	\end{cases}
\end{align*}

For the sake of brevity, $P_{\Sigma_{o_i}}$ and $\Sigma_{o_i}$ will be denoted by $P_i$ and $\Sigma_i$, respectively. 
For any language $L\subseteq \Sigma^*$, define $P_i(L)=\{\omega\in\Sigma_i^*|\exists s\in L: P_i(s)=\omega\}$.
%The domain of $P_i$ is also extended to $2^{\Sigma^*}$, i.e., $P_i(L)=\{\omega\in\Sigma_i^*|\exists s\in L, P_i(s)=\omega\}$.
For each event $\sigma\in\Sigma$, we use $I(\sigma)=\{i\in\mathcal{I}|\sigma\in\Sigma_i\}$ to denote the index set of OSs that can observe $\sigma$. 

%For an event $\sigma\in\bigcup_{i\in\mathcal{I}}\Sigma_i$, we define for $i\in I(\sigma)$, $\Omega_i(\sigma)=\{\sigma\omega\in\Sigma_i^*|\exists x\in X, \sigma\omega\in P_i(L(G,x))\}$ to be the set of sequences of observations that can be observed by $O_i$ and start with event $\sigma$.

\begin{figure}[tbp]
	\centering
	\includegraphics[scale=0.4]{architecture.pdf}
	\caption{Decentralized observation architecture.}
	\label{architecture}
\end{figure}

The decentralized observation-based information processing, abbreviated as DO-based protocol, adopted in this work is depicted in Fig. \ref{architecture}.
We assume that OSs have no knowledge of the system model, while the coordinator knows the system model and the DO-based protocol introduced below.
%, the specific synchronization strategy utilized, and the observable event sets $\Sigma_i$ for each $i\in\mathcal{I}$.
In this context, we use $\Sigma_{\mathcal{I}}=\bigcup_{i\in\mathcal{I}}\Sigma_i$ to denote the set of observable events from the coordinator's perspective.
The DO-based protocol discussed in this paper consists of three main components:

1) The information that $O_i$ sends to the coordinator: Each $O_i$ is associated with a projection function $\mathcal{P}_i$ and a sync-activation function $\mathcal{C}_i$:
\begin{align*}
	\mathcal{P}_i:L(G)\rightarrow\Sigma^*_i
	\quad 
	\text{and}
	\quad
	\mathcal{C}_i:\mathcal{P}_i(L(G))\rightarrow \{0,1\}.
\end{align*}
Function $\mathcal{P}_i$, which records the observation sequence prepared to be sent to the coordinator at the next synchronization, is defined as follows: for any $s\in L(G)$, $\mathcal{P}_i(s)=P_i(s')$,  where $s'$ is the suffix of $s$ starting with the event immediately following the last synchronization. 
At each synchronization step, the information recorded at an OS undergoes two stages: the sequence of observations before the synchronization and empty string $\epsilon$ after the synchronization.
Function $\mathcal{C}_i$ decides whether (decision ``1'') or not (decision ``0'') $O_i$ signals the coordinator to initiate a synchronization.\footnote{The coordinator inherently knows which specific OS signals the synchronization, which will be analyzed later.}


%\footnote{In order to avoid dense notation, we assume that the coordinator lacks knowledge of the specific OSs that signal the synchronization. Later, we will also briefly introduce the method to address this situation, enhancing our approach with more detailed handling.}

Function $\mathcal{C}_i$ is defined based on system behaviors, which may require infinite memory for realization. 
In real-world applications, this mechanism is typically implemented using a practical strategy, which is usually described by the observations recorded at the OSs rather than the system behavior.
To this end, this function can be represented as $\mathcal{C}_i:\Sigma^*_i\rightarrow \{0,1\}$.
Readers can refer to the work in \cite{Hadjicostis2020} for further details. 
The function $\mathcal{C}_i$ utilized in this paper will be elaborated upon later.




2) Synchronization strategy: 
The process is governed by a function
$f:\mathcal{C}_1(\mathcal{P}_1(L(G)))\times\mathcal{C}_2(\mathcal{P}_2(L(G)))\times\dots\times\mathcal{C}_m(\mathcal{P}_m\allowbreak(L(G)))\rightarrow\{0,1\}$, determining the need for synchronization for all OSs, defined as follows: for $s\in L(G)$,
$f(\prod_{i\in\mathcal{I}}\mathcal{C}_i\allowbreak(\mathcal{P}_i(s)))=\mathcal{C}_1(\mathcal{P}_1(s))\vee\mathcal{C}_2(\mathcal{P}_2(s))\vee\dots\vee\mathcal{C}_m(\mathcal{P}_m(s))\in\{0,1\}$,
%$f(\prod_{i\in\mathcal{I}}\mathcal{C}_i(P_i(s)))=\bigvee_{i\in\mathcal{I}}\mathcal{C}_i(P_i(s))$,
where decision ``1'' implies that at least one $O_i$ signals the coordinator to trigger a \textit{synchronization}. 
In such case, the coordinator requests information from all OSs, each sending its preserved sequence $\mathcal{P}_i(s)$ to the coordinator; decision ``0'' means no such process is initiated. 
%The process is governed by a function $f:\prod_{i\in\mathcal{I}}\mathcal{C}_i(P_i(L(G)))\rightarrow\{0,1\}$, determining the need for synchronization for all OSs, defined as follows: for $s\in L(G)$, $f(\prod_{i\in\mathcal{I}}\mathcal{C}_i(P_i(s)))=\bigvee_{i\in\mathcal{I}}\mathcal{C}_i(P_i(s))$, where decision ``1'' implies that at least one $O_i$ signals the coordinator to trigger a \textit{synchronization}. 
%In such case, the coordinator requests information from all OSs, each sending its preserved sequence $\mathcal{P}_i(s)$ to the coordinator; decision ``0'' means no such process is initiated. 


The overall synchronization process can then be described by a function $f_{\mathcal{I}}:L(G)\rightarrow\{0,1\}$ such that for any $s\in L(G)$, $f_{\mathcal{I}}(s)=f(\prod_{i\in\mathcal{I}}\mathcal{C}_i(\mathcal{P}_i(s)))$.
With this notation at hand, the synchronization strategy  for the DO-based protocol is defined as $\Upsilon=(\mathcal{P},f_{\mathcal{I}})$, where $\mathcal{P}$ is the abstraction for $\{\mathcal{P}_{i}\}_{i\in\mathcal{I}}$.

3) The synchronization function: 
The coordinator receives partially ordered sequences of observations provided by OSs and utilizes this information for state estimation or decision making. 
Specifically, when designated to perform state estimation, this task can be formalized as the synchronization function below:

\begin{align*}
	\mathcal{S}:(\prod_{i\in\mathcal{I}}\Sigma_i^*)^*\times 2^X\rightarrow 2^X.
\end{align*}
Depending on the time instant at which the set of possible states needs to be determined, the synchronization function will be defined later (see beginning of Section \Rom{3}), and will result in DO-based current- and initial-state estimation.

%At each synchronization, the coor
%\textcolor{blue}{We refer to the observation sequences provided by OSs at each synchronization as partially ordered sequences of observations (PO-sequences) \cite{Hadjicostis2020}}\footnote{\textcolor{blue}{Necessary? Note the difference between SI-states and PO-sequence. I should not let reader confuse these two name. Maybe let them be one thing? In oder to classify this and SI-state, may i delete this sentence here and use originate name here after- partially ordered sequences of observations}. }

Given a system $G=(X,\Sigma,\delta,X_0)$ under the DO-based protocol\footnote{We use synchronization strategy $\Upsilon$ to represent the entire DO-based protocol.} $\Upsilon$, let us denote the system behavior $s\in L(G)$ of subsequences/synchronizations by  $\vec{s}=s_1{\scriptstyle\sim} s_2 {\scriptstyle\sim}\dots s_j {\scriptstyle\sim}s_{j+1}$, where $s=s_1 s_2 \ldots s_{j+1}$ and ``${\scriptstyle\sim}$'' represents the synchronization process, such that for any $k\in\{1,\dots,j\}$, $f_{\mathcal{I}}(s_1s_2\dots s_k)=1$. 
The system behavior with synchronizations is called a \textit{run} in a system.
We use the notation $\tilde{s}$ to represent the prefix of $s$ such that the last synchronization \textit{can} occur immediately after $\tilde{s}$, i.e., for any $s\in L(G)$, $\tilde{s}\in \overline{s}$, $f_{\mathcal{I}}(\tilde{s})=1$, and for all $s'\in \overline{s}\setminus\overline{\tilde{s}}$, $f_{\mathcal{I}}(s')=0$.
Additionally, we let $\tilde{s}=\epsilon$ if no synchronization has occurred or can occur within the sequence $s$, i.e., for all $s'\in\overline{s},f_{\mathcal{I}}(s')=0$.
Therefore, in this representation, we have $\tilde{s}=s_1s_2\dots s_j$.


Given a DO-based protocol $\Upsilon$, we define the corresponding decentralized observation synchronization-based projection (DO-projection) as follows (we also provide an example at the end of this section to clarify notation).

\begin{definition}\label{def-projection}
	Given a system $G$ under DO-based protocol $\Upsilon=(\mathcal{P}, f_{\mathcal{I}})$, the DO-projection function $P_{\Upsilon}:L(G)\rightarrow (\prod_{i\in\mathcal{I}}\allowbreak\Sigma_i^*)^*$ is defined recursively as follows: for any $s\in L(G)$,
	\begin{itemize}
%\item $P_{\Upsilon}(s)=(\epsilon,\dots,\epsilon)$ if $\tilde{s}=\epsilon$;
%\item $P_{\Upsilon}(s)=P_{\Upsilon}(\tilde{s})$ if $\tilde{s}\neq\epsilon$, where $P_{\Upsilon}(\tilde{s})=P_{\Upsilon}(s')\allowbreak(P_1(s''),\dots,P_m(s''))$ and $\tilde{s}=s's''\land f_{\mathcal{I}}(s')=1\land(\forall t\in\overline{\tilde{s}}\setminus(\overline{s'}\cup\{\tilde{s}\}):f_{\mathcal{I}}(t)=0)$.
\item $P_{\Upsilon}(s)=(\epsilon,\dots,\epsilon)$ if $\tilde{s}=\epsilon$;
\item $P_{\Upsilon}(s)=P_{\Upsilon}(\tilde{s})$ if $\tilde{s}\neq\epsilon$, where 
\begin{itemize}
\item $P_{\Upsilon}(\tilde{s})=(P_1(\tilde{s}),\dots,P_m(\tilde{s}))$ if $\nexists s'\in\overline{\tilde{s}}\backslash\{\tilde{s}\}: f_{\mathcal{I}}(s')=1$;
\item $P_{\Upsilon}(\tilde{s})=P_{\Upsilon}(s')\allowbreak(P_1(s''),\dots,P_m(s''))$ if  $\exists s'\in\overline{\tilde{s}}\backslash\{\tilde{s}\}:f_{\mathcal{I}}(s')=1\land(\forall t\in\overline{\tilde{s}}\setminus(\overline{s'}\cup\{\tilde{s}\}):f_{\mathcal{I}}(t)=0)\land s''=s/ s'$.
%$\land f_{\mathcal{I}}(s')=1\land(\forall t\in\overline{\tilde{s}}\setminus(\overline{s'}\cup\{\tilde{s}\}):f_{\mathcal{I}}(t)=0)$.
\end{itemize}

%\begin{multline*}
%P_{\Upsilon}(\tilde{s})=\\
%\begin{cases}
%(P_1(\tilde{s}),\dots,P_m(\tilde{s})) & \text{if}\:\nexists s'\in\overline{\tilde{s}}\backslash\{\tilde{s}\}):f_{\mathcal{I}}(s')=1\\
%P_{\Upsilon}(s')\allowbreak(P_1(s''),\dots,P_m(s''))
%\end{cases}
%\end{multline*}
%
%
%$P_{\Upsilon}(\tilde{s})=P_{\Upsilon}(s')\allowbreak(P_1(s''),\dots,P_m(s''))$ and $\tilde{s}=s's''\land f_{\mathcal{I}}(s')=1\land(\forall t\in\overline{\tilde{s}}\setminus(\overline{s'}\cup\{\tilde{s}\}):f_{\mathcal{I}}(t)=0)$.
\end{itemize}
\end{definition}

%\textcolor{red}{Do I need to prove this projection is well-defined? i.e., for any $s\in L(G)$, there is only one sequence of SI-states.}

Intuitively, at each synchronization step, the coordinator can only obtain the system's partially ordered behavior following the last synchronization.
This fundamental property is captured by the DO-projection $P_{\Upsilon}(s)$. 
For example,  given a run $\vec{s}=s_1{\scriptstyle\sim}s_2{\scriptstyle\sim}\dots s_j{\scriptstyle\sim}s_{j+1}$, $P_{\Upsilon}(s)=(P_1(s_1),\dots,P_m(s_1))\dots(P_1(s_j),\dots,P_m(s_j))$ represents a sequence of sets of partially ordered sequences of observations received by the coordinator. 

The synchronization information state (SI-state), denoted as $SI(s)=(\mathcal{P}_1(s),\dots,\mathcal{P}_m(s))$,  captures the current (i.e., after the last synchronization step) observation sequences recorded at the $m$ OSs.\footnote{Since at each synchronization step, the information recorded at each OS undergoes two stages, in the rest of this paper, unless otherwise specified, the content of an SI-state $SI(s)$ refers to the information recorded prior to the occurrence of the synchronization.}
%\vspace{1cm}
%To streamline this representation, we rewrite the definition of the function $\mathcal{P}_i$ as $\mathcal{P}_i(s)=P_i(s/\tilde{s})$, focusing solely on the second state of each synchronization step.
%
%The synchronization information state (SI-state), denoted as $SI(s)=(\mathcal{P}_1(s),\dots,\mathcal{P}_m(s))=(P_1(s/\tilde{s}),\dots,P_m(s/\tilde{s}))$, captures the current observation sequences recorded at the $m$ OSs.
%\vspace{1cm}
That is, the SI-state describes the information recorded at OSs between two consecutive synchronizations and the SI-state becomes $(\epsilon,\dots,\epsilon)$ after each synchronization.
In order to simplify the notation, we use $\tau=(\tau^{(1)},\dots,\tau^{(m)})\in\prod_{i\in\mathcal{I}}\Sigma_i^*$ to denote an SI-state.
Specifically, we denote the SI-state $(\epsilon,\dots,\epsilon)$ as $\tau_0$ and refer to it as the initial SI-state after each synchronization.

We also assume that each OS  possesses finite memory, i.e., $O_i$ can remember at most $\kappa_i$, $\kappa_i\in\mathbb{N}$, events. Therefore, in this paper, we implement the synchronization strategy $\Upsilon_{\mathbb{N}}=(\mathcal{P},f_{\mathcal{I}})$, characterized by the following condition:
\begin{align}\label{SS}
	\forall i\in\mathcal{I}, \forall s\in L(G): \mathcal{C}_i(P_i(s))=1\Leftrightarrow|\mathcal{P}_i(s)|=\kappa_i.
\end{align}


\begin{remark}
	The synchronization strategy introduced and utilized herein is chosen for its conciseness, facilitating a straightforward exposition of the mechanisms and conclusions presented throughout this paper.
	However, the methodologies described hereafter are not limited to this specific strategy; indeed, any synchronization strategy that falls within the framework of a DO-based protocol can be applied, provided that each OS  possesses finite memory and the function $\mathcal{C}_i$ is defined based on the observation sequences recorded at each OS rather than the system behavior.
\end{remark}


We use notation $P_{\mathcal{I}}$ to denote the natural projection function with respect to (w.r.t.) $\Sigma_{\mathcal{I}}$.
Then, from the viewpoint of the coordinator, the set of possible states reachable from a state in $X'$, with $X'\subseteq X$, upon the occurrence of $\sigma\in\Sigma_{\mathcal{I}}\cup \{\epsilon\}$ is given by\footnote{We also write $\operatorname{R}_{\sigma}(x)$ when $X'=\{x\}$, i.e., when $|X'|=1$.}
\begin{align*}
	\operatorname{R}_{\sigma}(X')=\{x\in X|\exists u\in\Sigma^*: x\in\delta(X',u)\land P_{\mathcal{I}}(u)=\sigma\},
\end{align*}
whereas the unobservable reach of $X'\subseteq X$ is defined as $\operatorname{UR}(X')=\operatorname{R}_{\epsilon}(X')$. 



\begin{figure}[tbp]
	\centering
	\includegraphics[scale=1.0]{example-1.pdf}
	\caption{An NFA model $G$ under DO-based protocol, where $\Sigma_1=\{\alpha_{12}, \beta_{13}\}$, $\Sigma_2=\{\alpha_{12},\gamma_{2}\}$, and $\Sigma_3=\{\beta_{13},\gamma_{3}\}$.}
	\label{fig-example-1}
\end{figure}



\begin{example}\label{example-1}

Consider the system $G$ shown in Fig.~\ref{fig-example-1} where there exist three OSs, i.e., $\mathcal{I}=\{1,2,3\}$.
Let $\Upsilon_{\mathbb{N}}=(\mathcal{P}, f_{\mathcal{I}})$ be its synchronization strategy so that for all $i\in\mathcal{I}$, $\kappa_i=2$, indicating that each OS can record a sequence with at most 2 events before it signals the coordinator to initiate a synchronization.



Initially, the SI-state is $\tau_0=(\epsilon,\epsilon,\epsilon)$, indicating that no events in $\Sigma_{\mathcal{I}}$ have been executed within the system.
Now, consider the sequence $s_1=\alpha_{12}\upsilon_2\gamma_3\alpha_{12}$, which occurs in the system, starting from state $q_0$ and ending in state $q_4$.
We analyze this process event by event:



1) When the first event of $s_1$, i.e., $\alpha_{12}$, occurs, since $\alpha_{12}\in\Sigma_1$ and $\alpha_{12}\in\Sigma_2$, both $O_1$ and $O_2$ record its occurrence. 
The resulting SI-state is $SI(\alpha_{12})=(\alpha_{12},\alpha_{12},\epsilon)$;



2) Next, when the second event of $s_1$, i.e., $\upsilon_2$, occurs, since $\upsilon_2\notin \Sigma_{\mathcal{I}}$, none of OSs records its occurrence; so the SI-state remains $SI(\alpha_{12}\upsilon_2)=(\alpha_{12},\alpha_{12},\epsilon)$;



3) By analogy, when the third event of $s_1$, i.e., $\gamma_3$, occurs, the SI-state is $SI(\alpha_{12}\upsilon_2\gamma_3)=(\alpha_{12},\alpha_{12},\gamma_3)$;



4) Finally, when the last event of $s_1$, i.e., $\alpha_{12}$, occurs, $SI(\alpha_{12}\upsilon_2\gamma_3\alpha_{12})=(\alpha_{12}\alpha_{12},\alpha_{12}\alpha_{12},\gamma_3)$.


At this point, note that $|\alpha_{12}\alpha_{12}|=2=\kappa_1=\kappa_2$, which implies that $\mathcal{C}_1(\alpha_{12}\alpha_{12})=\mathcal{C}_2(\alpha_{12}\alpha_{12})=1$.
This causes $O_1$ and $O_2$ to signal the coordinator to initiate a \textit{synchronization}.
Consequently, function $f_{\mathcal{I}}(s_1)=1$.
As a result, the coordinator requests information from all OSs and subsequently receives SI-state $(\alpha_{12}\alpha_{12},\alpha_{12}\alpha_{12},\gamma_3)$.
Specifically, each OS sends its partially ordered sequence of observations to the coordinator: $O_1$ sends $\alpha_{12}\alpha_{12}$, $O_2$ sends $\alpha_{12}\alpha_{12}$, and $O_3$ sends $\gamma_3$.
After this transmission, the SI-state resets to $\tau_0=(\epsilon,\epsilon,\epsilon)$, indicating that no events have been observed since the last transmission of information to the coordinator.



Meanwhile, the coordinator estimates the system states based on the received SI-state.
If the coordinator knows the system starts from state $q_0$, then its set of state estimates before receiving the SI-state is $\operatorname{UR}(q_0)=\{q_0,q_1\}$.
Therefore, at this synchronization, the coordinator needs to update its estimates based on $\{q_0,q_1\}$ and SI-state $\tau=(\alpha_{12}\alpha_{12},\alpha_{12}\alpha_{12},\gamma_3)$.
This can be done by examining the possible executions in the system that match the received SI-state.
For example, the possible system behaviors starting from $q_0$ or $q_1$ could be 
$t_1=\alpha_{12}\upsilon_{2}\gamma_{3}\alpha_{12}$, $t_2=\alpha_{12}\upsilon_{2}\gamma_{3}\upsilon_1\alpha_{12}$, or $t_3=\alpha_{12}\upsilon_{2}\gamma_{3}\alpha_{12}\upsilon_2$, %(since $\upsilon_2$ is unobservable to any of OSs, as will be explained in the next section). 
since for any $i\in\mathcal{I}=\{1,2,3\}$ $P_i(t_1)=P_i(t_2)=P_i(t_3)=\tau^{(i)}$.
Then, the state estimates obtained by the coordinator are $\delta(\{q_0,q_1\},t_1)\cup\delta(\{q_0,q_1\},t_2)\cup\delta(\{q_0,q_1\},t_3)=\{q_2,q_3,q_4\}$



Following this, assume that the sequence $s_2=\beta_{13}\gamma_{2}\gamma_{3}\alpha_{12}$ is subsequently executed in the system. 
A synchronization happens after $\beta_{13}\gamma_{2}\gamma_{3}$ thanks to $SI(s_1\beta_{13}\gamma_{2}\gamma_{3})=(\beta_{13},\gamma_{2},\beta_{13}\allowbreak\gamma_{3})$ with $|\beta_{13}\gamma_{3}|=2=\kappa_3$. 
Note that according to our notation, $SI(s_1s_2)=(\alpha_{12},\alpha_{12},\epsilon)$.
In this case, the corresponding run can be represented as $\vec{s}=s_1{\scriptstyle\sim} s_2 {\scriptstyle\sim}\alpha_{12}$ and $\tilde{s}=s_1s_2$.
Based on Definition~\ref{def-projection}, we have $P_{\Upsilon_{\mathbb{N}}}(s)=SI(s_1)SI(\beta_{13}\gamma_{2}\gamma_{3})=(\alpha_{12}\alpha_{12},\alpha_{12}\alpha_{12},\gamma_3)(\beta_{13},\gamma_{2},\beta_{13}\gamma_{3})$.~\hfill\rule{1ex}{1ex}



Example \ref{example-1} illustrates the process of system execution involving synchronization under the DO-based protocol.
However, it does not provide an efficient method to systematically track all possible executions for all possible system states.
This issue will be addressed in the next section, where a more comprehensive approach will be introduced, proving valuable for property verification purposes.
\end{example}
%\textcolor{red}{The coordinator knows OR not know which OS apply for the synchronization
%	\begin{enumerate}
%		\item The coordinator know which makes all definition and procedure dense;
%		\item The coordinator does not know and I will use 1/4 pages (one small section) to explain the method briefly.
%\end{enumerate}
%}




\section{State Estimation and Observer Construction Under DO-based Protocol}
In this section, we first introduce and analyze the notion of \textit{Complete Synchronizing Sequence structure} (CSS structure), which essentially encompasses all SI-states and the state evolution of each state upon any SI-state. 
Following this, we demonstrate how the CSS structure can be utilized to construct the DO-based state estimator.

In \cite{SunHadjicostisLi2023}, the notions of DO-based current-state estimation (DO-CSE) and DO-based initial-state estimation (DO-ISE) are proposed to capture the current-state estimate and initial-state estimate after the coordinator receives an SI-state $\tau$ based on the latest state estimates. 
Due to the more general scenarios analyzed in this paper,  these two notions are redefined as follows.
Given a system $G=(X,\Sigma,\delta,X_0)$ under DO-based protocol $\Upsilon_{\mathbb{N}}$, suppose that a run $\vec{s}$ is generated by the system.
\begin{enumerate}
	\item DO-CSE aims to determine the set of states that the system possibly resides in immediately after the last synchronization, including the states that may be reached via unobservable events, i.e.,
	\begin{multline*}
		\mathcal{E}^c(P_{\Upsilon_{\mathbb{N}}}(s),X_0)=\{x\in X|\exists x_0\in X_0,\exists u\in L(G), \\ P_{\Upsilon_{\mathbb{N}}}(s)=P_{\Upsilon_{\mathbb{N}}}(u)\wedge x\in\operatorname{UR}(\delta(x_0,\tilde{u}))\},
	\end{multline*} where $\tilde{u}$ is the prefix of $u$ that is immediately followed by the last synchronization.
	
	\item DO-ISE aims to determine the set of initial states after several synchronizations, i.e.,
	\begin{multline*}
		\mathcal{E}^{i}(P_{\Upsilon_{\mathbb{N}}}(s),X_0)=\{x_0\in X_0|\exists u\in L(G,x_0):\\
		P_{\Upsilon_{\mathbb{N}}}(s)=P_{\Upsilon_{\mathbb{N}}}(u)\}. 
	\end{multline*}
\end{enumerate}
Based on the definition of synchronization, for any string $u\in L(G)$, if $\tilde{u}\neq\epsilon$, then $f_{\mathcal{I}}(\tilde{u})=1$. 
The last event in $\tilde{u}$ is always observable to at least one of the OSs. 
Thus, we use operator $\operatorname{UR}(\cdot)$ to extend the state estimates to include the unobservable reach of $\delta(x_0,\tilde{u})$ (w.r.t. the set of events that are not observable to \textit{any} OS).
This definition allows for a more comprehensive coverage of the system's possible states after a sequence of events.

\begin{remark}\label{remark-aftersync}
	Within the framework of DO-CSE, $\mathcal{E}^c(P_{\Upsilon_{\mathbb{N}}}(s),X_0)$ denotes the state estimate right after $\tilde{s}$ (note that $P_{\Upsilon_{\mathbb{N}}}(s)=P_{\Upsilon_{\mathbb{N}}}(\tilde{s})$) including the unobservable reach w.r.t. the coordinator.
	In this context, $\mathcal{E}^c(P_{\Upsilon_{\mathbb{N}}}(s),X_0)$ is equivalent to $\mathcal{E}^c(P_{\Upsilon_{\mathbb{N}}}(\tilde{s}),X_0)$, even if events in $\Sigma_{\mathcal{I}}$ occur after $\tilde{s}$.
	Consequently, the estimate $\mathcal{E}^c(P_{\Upsilon_{\mathbb{N}}}(s),X_0)$ remains valid only immediately following $\tilde{s}$. 
	The coordinator will only be able to update this estimate at the next synchronization. 
	If desirable, the coordinator could presumably attempt to predict system behaviors and potential states between synchronizations based on its knowledge of the synchronization strategy, which could be a valuable topic for future research.
\end{remark}

\subsection{Construction of CSS structure and its Properties}

The concept of \textit{S-builder} was introduced in \cite{SunHadjicostisLi2023} to provide the set of all possible system observation sequences matching a given SI-state $\tau$; state estimation is subsequently performed based on this set.
This method is applicable in online scenarios and requires no preprocessing of the synchronization strategy. 
In other words, the coordinator relies solely on $\tau$, $I(\sigma)$ for all $\sigma\in\Sigma_{\mathcal{I}}$, and the state estimates derived from the most recent synchronization.
In the sequel, we detail a preprocessing approach that can be applied to the synchronization strategy, and facilitates state estimation and observer construction, which can be important when trying to verify properties of interest, as discussed later.
This preprocessing approach can also be used for online estimation, e.g., in terms of eliminating the need for (on-line) construction of \textit{S-builders}. 


We first introduce the absorbing transition function $\hat{h}$ as a mapping $\hat{h}:\prod_{i\in\mathcal{I}}\Sigma_i^*\times\Sigma_{\mathcal{I}}\rightarrow \prod_{i\in\mathcal{I}}\Sigma_i^*$, defined as follows:  for any $\tau=(\tau^{(1)},\dots,\tau^{(m)})\in \prod_{i\in\mathcal{I}}\Sigma_i^*$, $\tau'=(\tau'^{(1)},\dots,\tau'^{(m)})\in \prod_{i\in\mathcal{I}}\Sigma_i^*$, and $\sigma\in\Sigma_{\mathcal{I}}$, it holds 
%\begin{multline*}
%	\hat{h}(\tau,\sigma)=\tau'\Rightarrow \forall i\in I(\sigma),\forall j\in\mathcal{I}/I(\sigma):\\
%	\tau\models\Upsilon_{\mathbb{N}}\land\tau'^{(i)}=\tau^{(i)}\sigma\wedge\tau'^{(j)}=\tau^{(j)}.
%\end{multline*}\emph{}
\begin{multline*}
	\hat{h}(\tau,\sigma)=\tau'\Rightarrow \tau\models\Upsilon_{\mathbb{N}}\land (\forall i\in I(\sigma),\\
	\forall j\in\mathcal{I}/I(\sigma):(\tau'^{(i)}=\tau^{(i)}\sigma\wedge\tau'^{(j)}=\tau^{(j)})),
\end{multline*}
where the notation ``$\models$'' is explained below.
The function $\hat{h}$ is defined to capture the process by which all OSs record events during the evolution of the system between two consecutive synchronizations. 
In this context, $\tau'$ can be described as the outcome of $\tau$ when $\sigma$ occurs.
Due to the existence of the synchronization strategy $\Upsilon_{\mathbb{N}}$, the first component of the domain of $\hat{h}$ is constrained to the set of SI-states that cannot cause a synchronization, i.e., the set $\{\tau|\tau\in\prod_{i\in\mathcal{I}}\Sigma_i^*\land(\forall i\in\mathcal{I}$: $|\tau^{(i)}|< \kappa_i)\}$, which is denoted as $\tau\models\Upsilon_{\mathbb{N}}$.
%$\tau\in\prod_{i\in\mathcal{I}}\Sigma_i^*$ that cannot cause a synchronization,  i.e., for all $i\in\mathcal{I}$, $|\tau^{(i)}|< \kappa_i$.
Given that synchronizations always occur as long as the events in $\Sigma_{\mathcal{I}}$ continue to be generated in the system, we refer to SI-states that are immediately followed by a synchronization and are received at the coordinator, as critical SI-states (CSI-states). For any CSI-state $\tau$, $\hat{h}(\tau,\sigma)$ is \textit{not} defined.




We now present the \textit{Complete Synchronizing Sequence structure} (CSS structure),  which serves as a fundamental component of this paper.
To clearly define CSS structure,  we first introduce a structure called \textit{Synchronizing Sequence structure} (SS structure) for a given state $x\in X$.

An SS structure $\mathcal{T}_x$ w.r.t. a state $x\in X$ in system $G$ under DO-based protocol $\Upsilon_{\mathbb{N}}$ is defined as $\mathcal{T}_x=(\mathcal{X},T, h_a,h_r,\Sigma_{\mathcal{I}}\cup\{\epsilon\},T_0, \mathcal{X}_0,T_c)$, where
	\begin{itemize}
		\item $\mathcal{X}\subseteq X\times L$ is the set of system states augmented with layer set $L=\{0,1,\dots,l_u\}\subset\mathbb{N}$;
		\item $T\subseteq \prod_{i\in\mathcal{I}}\Sigma_i^*$ is the set of SI-states during the evolution of the system $G$;
	    \item $h_a:\mathcal{X}\times \Sigma_{\mathcal{I}}\rightarrow 2^T$ is the transition function, which is defined as follows: for any $(x',l')\in \mathcal{X}$, $\sigma\in\Sigma_{\mathcal{I}}$, and $\tau\in T$, it holds
	\begin{multline*}
		((x',l'),\sigma,\tau)\in h_{a}\Leftrightarrow \exists\tau'\in T,\exists \sigma'\in\Sigma_{\mathcal{I}}\cup\{\epsilon\}:\\ 
		(\tau',\sigma',(x',l'))\in h_r\land \hat{h}(\tau',\sigma)=\tau\land\operatorname{R}_{\sigma}(x')\neq\emptyset;
	\end{multline*}
	\item $h_r:T\times (\Sigma_{\mathcal{I}}\cup\{\epsilon\})\rightarrow 2^{\mathcal{X}}$ is the transition function, which is defined as follows: for any $\tau\in T$, $(x',l')\in \mathcal{X}$, and $\sigma\in\Sigma_{\mathcal{I}}\cup\{\epsilon\}$, it holds
	\begin{multline*}
		(\tau,\sigma,(x',l'))\in h_r\Leftrightarrow \mathcal{CD}_1\lor\mathcal{CD}_2,\\
		\shoveleft{\text{where}\ \mathcal{CD}_1\Leftrightarrow  \tau=\tau_0\land\sigma=\epsilon\land (x',l')=\mathcal{X}_0}\\
		\shoveleft{\text{and}\ \mathcal{CD}_2\Leftrightarrow\sigma\in\Sigma_{\mathcal{I}}\land(\exists (x'', l'')\in \mathcal{X}:}\\ 
		((x'', l''),\sigma,\tau)\in h_a\land x'\in\operatorname{R}_{\sigma}(x'')\land l'=l''+1);
	\end{multline*}
	\item $\Sigma_{\mathcal{I}}$ is the set of events recognized by the coordinator;
	\item $T_0=\tau_0\in T$ is the initial SI-state;
	\item $\mathcal{X}_0=(x,0)$ is the state $x$ with layer ``0'';
	\item $T_c\subseteq T$ is the set of CSI-states.
	\end{itemize}


We next present a simple example to illustrate the construction of SS structure, serving as a guide for the reader through the subsequent discussion and analysis and offering a comprehensive explanation of the structure.
\begin{figure}[htbp]
	\centering
	\includegraphics[scale=0.9]{T_x_of_x_2.pdf}
	\caption{$\mathcal{T}_{q_2}$ of the system $G$ in Example  \ref{example-1}.}
	\label{fig-SS-state2}
\end{figure}
\begin{example}\label{example-t_x}
	Fig. \ref{fig-SS-state2} shows an SS structure w.r.t. system state ``$q_2$'' in Example  \ref{example-1}.
	The oval states, containing the sets of sequences of events, are SI-states, whereas the square states, containing the pairs of system states and numbers, form the set $\mathcal{X}$.
    Note that $\mathcal{X}_0=(q_2,0)$ and $(\beta_{13},\epsilon,\beta_{13}\gamma_3)$ is the CSI-state.\hfill\rule{1ex}{1ex}
\end{example}


An SS structure $\mathcal{T}_x$ is a bipartite ordered graph that reflects the evolution of system state $x$ through various possible SI-states.
This structure is characterized by two distinct types of alternating layers—system state layers and SI-state layers—where transitions connect the states between these two layers.
System state layers contain system states, i.e., the set $\mathcal{X}$, where each state is associated with an integer that indicates its layer, distinguishing identical states across different system state layers.
SI-state layers consist of SI-states, i.e., those in set $T$, which naturally distinguish themselves across layers by sequentially absorbing possible events.
For example, in Fig. \ref{fig-SS-state2}, the system state layers are $\{(q_2,0)\}$, $\{(q_3,1)\}$, and $\{(q_0,2),(q_1,2)\}$, where the values ``0'', ``1'', ``2'' represent the respective layers.
The SI-states layers are $\{(\epsilon,\epsilon,\epsilon)\}$, $\{(\beta_{13},\epsilon,\beta_{13})\}$, and $\{(\beta_{13},\epsilon,\beta_{13}\gamma_3)\}$.



%the second component in each state denotes the corresponding layer.

%The SI-state layers appear before each system state layer, illustrating the sequential structure.




Functions $h_a$ and $h_r$ facilitate the transitions from system states to SI-states and from SI-states to system states, respectively.
Specifically, the SI-states, as mapped by $h_a$ from different system states augmented with the same layer, are grouped into the same SI-state layer.
Conversely, $h_r$ maps an SI-state to a subset of system states, each augmented with the same layer. 
Together, these two functions form the CSS structure into consecutive alternating layers, where system states and SI-states are sequentially connected. 
In detail,

1) In the definition of $h_{a}$: 
	$\tau'$ represents an SI-state that transitions (without requirement for a specific $\sigma'$, suggesting that any $\sigma'\in\Sigma_{\mathcal{I}}$ is acceptable) to $(x',l')$, i.e., $(\tau',\sigma',(x',l'))\in h_r$. 
	Symbol $\sigma$ is the observable event that might occur from system state $x'$.
	With these elements, a mapping from $(x',l')$ to $\tau$ is defined such that $\tau$ is the outcome of $\tau'$ when absorbing $\sigma$.
	This process signifies that the resulting SI-state, $\tau$, is determined based on the current system state being $x'$ and the occurrence of event $\sigma$, with the preceding SI-state being $\tau'$.
	For the example in Fig.~\ref{fig-SS-state2}, note that (i) there exists a transition relation $(\tau_0,\epsilon, (q_2,0))\in h_{r}$, since $\operatorname{R}_{\beta_{13}}(q_2)\neq\emptyset$ (which indicates that event $\beta_{13}$ can be the observable event occurring from $q_2$); (ii) $\hat{h}(\tau_0,\beta_{13})=(\beta_{13},\epsilon,\beta_{13})$. 
	Using (i) and (ii), we know $((q_2,0),\beta_{13},(\beta_{13},\epsilon,\beta_{13}))\in h_a$.

2) In the definition of $h_{r}$: 
    Let us first focus on Condition $\mathcal{CD}_2$ (Condition $\mathcal{CD}_1$ will be discussed later): 
	$((x'',l''),\sigma,\tau)\in h_a$ indicates that $(x'',l'')$ in the preceding system state layer (where $l'=l''+1$) transitions to SI-state $\tau$ with observable event $\sigma$.
	The set $\operatorname{R}_{\sigma}(x'')$ represents the state estimates following the occurrence of event $\sigma$ when the system state was $x''$. 
	With these elements, a mapping from $\tau$ to $(x',l')$ is defined if and only if the system could evolve from $x''$ to $x'$ upon the occurrence of observable event $\sigma$ and the SI-state $\tau$ is the outcome of absorbing $\sigma$.
	For the example in Fig.~\ref{fig-SS-state2}, we know that (i) there exists transition relation $((q_2,0),\beta_{13},(\beta_{13},\epsilon,\beta_{13}))\in h_a$, since $\operatorname{R}_{\beta_{13}}(q_2)=\{q_3\}$; (ii) $1=0+1$.
	Using (i) and (ii), we deduce that $((\beta_{13},\epsilon,\beta_{13}),\beta_{13}, (q_3,1))\in h_r$.

The condition $\mathcal{CD}_1$ within the definition of $h_r$ serves as the initial mapping from initial SI-state $\tau_0$ to $\mathcal{X}_0=(x,0)$, i.e., $\mathcal{X}_0=h_r(\tau_0,\epsilon)$. 
This setup is the foundation from which the entire SS structure is constructed.
Then, the SS structure is developed through the interdependence of the two functions, $h_a$ and $h_r$, where the output of one function forms the input of the other. 
Furthermore, the SI-state continuously absorbs events until it transitions into a CSI-state. 
This mechanism ensures that the size of the SS structure is finite, as it is constrained by the predicate $\hat{h}(\tau',\sigma)=\tau$ specified in the definition of $h_a$.
For example, in Fig.~\ref{fig-SS-state2}, we initiate a mapping as $(\tau_0,\epsilon, (q_2,0))$ in $h_r$. 
The rest of $\mathcal{T}_{q_2}$ is constructed using the function described above.
In the end, the SI-state layer terminates in the CSI-state $\{(\beta_{13},\epsilon, \beta_{13}\gamma_3)\}$ and the system state layer terminates in the set of states $\{(q_0,2), (q_1,2)\}$.

The above concept allows the systematic analysis of the evolution of system state $x$ upon any possible SI-state.
We are now ready to define the CSS structure.
For the sake of notational simplicity, we will continue to use the notations from the SS structure when defining the CSS structure.\footnote{Henceforth, unless otherwise specified, all related notations will refer to the CSS structure.}
%Given a system $G$ under DO-based protocol $\Upsilon_{\mathbb{N}}$, we formally define CSS structure next. 


\begin{definition}\label{def-ASS}
	A \textit{Complete Synchronizing Sequence structure} (CSS structure) $\mathcal{T}$ w.r.t.  system $G$ under DO-based protocol $\Upsilon_{\mathbb{N}}$ is an eight-tuple structure 
	\begin{align*}
		\mathcal{T}=(\mathcal{X},T, h_a,h_r,\Sigma_{\mathcal{I}}\cup\{\epsilon\},T_0, \mathcal{X}_0,T_c)
	\end{align*}
where
\begin{itemize}
	\item $T$, $\Sigma_{\mathcal{I}}$, $T_0$, and $T_c$ are the same as those in SS structure;
	\item $\mathcal{X}\subseteq X\times X\times L$ is the set of state pairs augmented with layer set $L=\{0,1,\dots,l_u\}\subset\mathbb{N}$ and $(\rho_0,\rho_1)\in X\times X$ and $\rho_d\in L$ denote the state pair component and layer component of a state $\rho=(\rho_0,\rho_1,\rho_d)\in\mathcal{X}$;
	%\item $T\subset \prod_{i\in\mathcal{I}}\Sigma_i^*$ is the set of SI-states during the evolution of the system $G$;
	\item $h_a:\mathcal{X}\times \Sigma_{\mathcal{I}}\rightarrow 2^T$ is the transition function, which is defined as follows: for any $\rho=(\rho_0,\rho_1, \rho_d)\in \mathcal{X}$, $\sigma\in\Sigma_{\mathcal{I}}$, and $\tau\in T$, it holds
	\begin{multline*}
		(\rho,\sigma,\tau)\in h_{a}\Leftrightarrow \exists\tau'\in T,\exists \sigma'\in\Sigma_{\mathcal{I}}\cup\{\epsilon\}:\\ 
		(\tau',\sigma',\rho)\in h_r\land \hat{h}(\tau',\sigma)=\tau\land\operatorname{R}_{\sigma}(\rho_1)\neq\emptyset;
	\end{multline*}
    \item $h_r:T\times (\Sigma_{\mathcal{I}}\cup\{\epsilon\})\rightarrow 2^{\mathcal{X}}$ is the transition function, which is defined as follows: for any $\tau\in T$, $\rho=(\rho_0,\rho_1, \rho_d)\in \mathcal{X}$, and $\sigma\in\Sigma_{\mathcal{I}}\cup\{\epsilon\}$, it holds
    \begin{multline*}
    	(\tau,\sigma,\rho)\in h_r\Leftrightarrow \mathcal{CD}_1\lor\mathcal{CD}_2,\\
    	\shoveleft{\text{where}\ \mathcal{CD}_1\Leftrightarrow \sigma=\epsilon\land \tau=\tau_0\land \rho\in\mathcal{X}_0}\\
    	\shoveleft{\text{and}\ \mathcal{CD}_2\Leftrightarrow\sigma\in\Sigma_{\mathcal{I}}\land(\exists \rho'=(\rho_0,\rho_1', \rho_d')\in \mathcal{X}:}\\ 
    	(\rho',\sigma,\tau)\in h_a\land \rho_1\in\operatorname{R}_{\sigma}(\rho_1')\land \rho_d=\rho_d'+1);
    \end{multline*}
%\item $\Sigma_{\mathcal{I}}$ is the set of events recognized by the coordinator;
%\item $T_0=\tau_0\in T$ is the initial SI-state;
\item $\mathcal{X}_0=\{(x,x,0)|x\in X\}$ is the set of initial state pairs with layer ``0''.
%\item $T_c\subset T$ is the set of CSI-states.
\end{itemize}
\end{definition}


Compared with the SS structure, the CSS structure comprises \textit{all} system states in a structure and exhibits the following differences.
The set $\mathcal{X}$ is now defined as the set of state pairs augmented with a layer set.
This modification is necessary to retain the corresponding initial system states at each system state layer.
Certain details in the functions $h_a$ and $h_r$ have been adjusted accordingly, although the main idea remains the same.
The primary effect of the changes to the functions is that, apart from $\mathcal{X}_0$, which is initialized first, the second component of each $\rho\in\mathcal{X}$ indicates whether it can be reached from the first component via specific system behaviors, with the corresponding SI-states connecting to the state $\rho$.
Consequently, the first element of $\rho$ in the definition of $h_r$ remains unchanged.
Overall, a CSS structure is a bipartite ordered graph that reflects the evolution of all system states upon any possible SI-state. 

\begin{figure*}[htbp]
	\centering
	\includegraphics[scale=0.8]{ASS-1.pdf}
	\caption{The CSS structure of system $G$ under DO-based protocol $\Upsilon_{\mathbb{N}}$. %States containing state pairs correspond to states in $\mathcal{X}$, while states containing sequences of events correspond to SI-states. 
		For clarity, the sets of state pairs that share the same layer are enclosed in a dotted rectangle, with the corresponding layer listed below each rectangle. The SI-states with grey shading indicate that they are CSI-states of $T_c$, labeled from $\tau_1$ to $\tau_{11}$.}
	\label{fig-ASS-1}
\end{figure*}

\begin{example}\label{example-2}
Consider again the system $G$ and the DO-based protocol $\Upsilon_{\mathbb{N}}$ shown in Fig.~\ref{fig-example-1}. 
The corresponding CSS structure is illustrated in Fig.~\ref{fig-ASS-1}.
It is clear that by adding the component ``$q_0$'' to the first position of each state in $\mathcal{X}$ of $\mathcal{T}_{q_2}$ in Fig. \ref{example-t_x}, $\mathcal{T}_{q_2}$ becomes part of this structure.

From the initial SI-state $\tau_0=(\epsilon,\epsilon,\epsilon)$, based on the condition $\mathcal{CD}_1$ in Definition~\ref{def-ASS}, there are transition relations in $h_r$ that connect $\tau_0$ to each state pair of $\mathcal{X}_0$ via $\epsilon$, all labeled with layer 0, i.e., $(\tau_0,\epsilon, (q,q,0))\in h_r$ for all $q\in\{q_0,q_1,q_2,q_3,\allowbreak q_4\}$.
Similar to the way that $\mathcal{T}_{q_2}$ in Example~\ref{example-t_x} is obtained, given a state pair $\rho=(q_0,q_0,0)$, by
 $\operatorname{R}_{\alpha_{12}}(q_0)\neq\emptyset$ and $(\tau_0,\epsilon, \allowbreak\rho)\in h_r$, 
the event $\alpha_{12}$ can be absorbed by $\tau_0$ 
such that $\hat{h}(\tau_0,\allowbreak\alpha_{12})=(\alpha_{12},\alpha_{12},\epsilon)$, which leads to $(\rho,\alpha_{12},(\alpha_{12},\alpha_{12},\epsilon))\in h_a$.
As in the case of condition $\mathcal{CD}_2$ in Definition~\ref{def-ASS}, due to $(\rho,\alpha_{12}, (\alpha_{12},\alpha_{12},\epsilon))\in h_a$ and $\operatorname{R}_{\alpha_{12}}(q_0)\allowbreak=\{q_2,q_3,q_4\}$, the following transition relation in $h_r$ holds: $\{(q_0,q_2,1),(q_0,q_3,1),(q_0,q_4,1)\}\in h_r((\alpha_{12},\alpha_{12},\epsilon),\alpha_{12})$.
Furthermore, given that $((q_1,q_1),\alpha_{12},\allowbreak(\alpha_{12},\alpha_{12},\epsilon))\in h_a$ and $\operatorname{R}_{\alpha_{12}}(q_1)=\{q_2\}$, we also have 
$\{(q_1,q_2,1)\}\in h_r((\alpha_{12},\alpha_{12},\allowbreak\epsilon),\alpha_{12})$.
Combining these relations, we find $h_r((\alpha_{12},\alpha_{12},\epsilon),\allowbreak\alpha_{12})=\{(q_0,q_2,1),(q_0,q_3,1), (q_0,q_4,1), (q_1,q_2,1)\}$.

Let us now focus on the state pair $(q_0,q_2,1)$.
It holds that $((q_0,q_2,1), \beta_{13},\tau_1)\in h_a$ due to $((\alpha_{12},\alpha_{12},\epsilon),\alpha_{12},\allowbreak(q_0,q_2,1))\in h_r$, $\operatorname{R}_{\beta_{13}}(q_2)\neq\emptyset$, and 
$\hat{h}((\alpha_{12},\alpha_{12},\epsilon),\beta_{13})=\tau_1$.
Additionally, $(\tau_1,\beta_{13},(q_0,q_3,2))\in h_r$ because of $((q_0,q_2,\allowbreak1),\beta_{13},\tau_1)\in h_a$ and  $\operatorname{R}_{\beta_{13}}(q_2)=\{q_3\}$.
The rest of CSS structure can be constructed in a similar way. 
Note that, for any state pair transitioned to exclusively by CSI-states, such as $(q_1,q_3,2)$, there is no subsequent transition relation in $h_a$, based on the definition of $\hat{h}$.\hfill\rule{1ex}{1ex}
\end{example}




Since the coordinator knows the details of the synchronization strategy, it inherently knows which OSs signal the synchronization.
According to the description of the DO-based protocol, each synchronization step is triggered by the OS (or OSs) that records the last event of a system execution.
Then, a CSS structure satisfies the following property.

%\textcolor{red}{These OSs are referred to as critical OSs. 
%Obviously, the sequences recorded by other OSs do not qualify to initiate a synchronization.
%Let $\mathcal{I}_{\tau}^c$ denote the set of indices of critical OSs when the CSI-state is $\tau$.
%Then, the last event can be expressed by $\tau^{(i)}_{fi}$, where $\tau^{(i)}_{fi}$ is the last event in $\tau^{(i)}$ and $i\in\mathcal{I}_{\tau}^c$.
%Note that for any $i,j\in\mathcal{I}_{\tau}^c$, $\tau^{(i)}_{fi}=\tau^{(j)}_{fi}$.}
\begin{lemma}\label{Lemma-ASS-lastEvent}
Let $\mathcal{T}=(\mathcal{X},T, h_a,h_r,\Sigma_{\mathcal{I}}\cup\{\epsilon\},T_0, \mathcal{X}_0,T_c)$ be a CSS structure. 
Then, it holds%$\forall \tau\in T_c, \forall (\tau,\sigma_1,\rho),(\tau,\sigma_2,\rho')\in h_r: \sigma_1=\sigma_2.$
		\begin{align*}
			\forall \tau\in T_c, \forall (\tau,\sigma,\rho),(\tau,\sigma',\rho')\in h_r: \sigma=\sigma'.
		\end{align*}
\end{lemma}
\begin{proof}
	We prove this by contradiction.
	Assume there exist $\tau\in T_c$ and $(\tau,\sigma,\rho),(\tau,\sigma',\rho')\in h_r$, such that $\sigma\neq\sigma'$.
	Then, there exist two SI-states $\tau_1,\tau_2\in T$, such that $\hat{h}(\tau_1,\sigma)=\tau$ and $\hat{h}(\tau_2,\sigma')=\tau$.
	If $I(\sigma)=I(\sigma')$, then for any $i\in I(\sigma)$, it holds $\tau_1^{(i)}\sigma=\tau^{(i)}=\tau_2^{(i)}\sigma'$ which contradicts the assumption that $\sigma\neq\sigma'$.
	If $I(\sigma)\neq I(\sigma')$, then either $I(\sigma)\backslash I(\sigma')\neq \emptyset$ or $I(\sigma')\backslash I(\sigma)\neq \emptyset$ holds.
	Let us assume $I(\sigma)\backslash I(\sigma')\neq \emptyset$.
	For any $j\in I(\sigma)\setminus I(\sigma')$, we have $\tau^{(j)}_1\sigma=\tau^{(j)}=\tau^{(j)}_2$, which indicates $|\tau^{(j)}_1\sigma|=|\tau^{(j)}_2|$.
    Since the synchronization is triggered by the OS that records the last event, i.e., event $\sigma$ or $\sigma'$, then $|\tau^{(j)}_1\sigma|=\kappa_j=|\tau^{(j)}_2|$, which contradicts the fact that $\tau_2\models\Upsilon_{\mathbb{N}}$, concluding that $\sigma=\sigma'$.
\end{proof}

Lemma~\ref{Lemma-ASS-lastEvent} indicates that the transitions originating from a CSI-state in a CSS structure share the same event label.
Let $M(\tau)$ denote the set of state pairs which are transitioned to by $\tau$ in a CSS structure, i.e., for any event $\sigma\in\Sigma_{\mathcal{I}}$ and for any state $\rho$ in $\mathcal{X}$, we have $M(\tau)=\{(\rho_0,\rho_1)|(\tau,\sigma,\rho)\in h_r\}$.

As seen earlier, the CSS structure is used to interpret the given DO-based protocol, allowing us to explore the state evolution of each state upon any SI-state. Therefore, given a system, the state evolution is decoded in its corresponding CSS structure, as concluded below.

\begin{proposition}\label{pro-current-string}
	Let $\mathcal{T}=(\mathcal{X},T, h_a,h_r,\Sigma_{\mathcal{I}}\cup\{\epsilon\},T_0, \mathcal{X}_0,T_c)$ be the CSS structure w.r.t. system $G$ under DO-based protocol $\Upsilon_{\mathbb{N}}$. The following hold:
	%	\begin{align*}
		%		SI(s)\in T\quad \text{and} \quad SI(\tilde{s})=\tau_0\lor SI(\tilde{s})\in T_c
		%	\end{align*}
	\begin{enumerate}
		\item $\forall \tau\in T\setminus\{\tau_0\},\forall x,x'\in X:((x,x')\in M(\tau)\Leftrightarrow (\exists t\in L(G,x):x'\in\delta(x,t)\land (\forall i\in\mathcal{I}:P_i(t)=\tau^{(i)})))$;
		\item $\forall s\in L(G):SI(s)\in T \land SI(\tilde{s})\in T_c\cup\{\tau_0\}$. %i.e., $T$ contains all SI-states that could appear in the system evolution;
		%\item $\forall \tau\in T,\forall (x,x')\in M(\tau),\exists t\in L(G,x):x'\in \delta(x,t)\land (\forall i\in \mathcal{I}:P_i(t)=\tau^{(i)})$
		%\item $\forall x\in X, \forall t\in L(G,x),\exists \tau\in T\setminus\{\tau_0\}: \tilde{t}=\epsilon\land\tau^{(i)}=P_i(t)\Rightarrow \delta(x,t)=\{x'|(x,x')\in M(\tau)\}$
		%\forall (x,x')\in M(\tau),\exists t\in L(G,x):x'\in \delta(x,t)\land (\forall i\in \mathcal{I}:P_i(t)=\tau^{(i)})$
	\end{enumerate}
\end{proposition}
\begin{proof}
	1) ($\Rightarrow$)
	If $(x,x')\in M(\tau)$, according to the definition of CSS structure, there exist a sequence of events $\omega=\sigma_1\sigma_2\dots\sigma_n$, a sequence of SI-states $\tau_0\tau_1\dots\tau_n$, and a sequence of states $x_0x_1x_2\dots x_n$, such that $\tau_0=(\epsilon,\dots,\epsilon)$, $\tau_n=\tau$, $x_0=x$, $x_n=x'$ and for any $j\in\{1,\dots,n\}$, we have $((x,x_{j-1},j-1),\sigma_j,\tau_j)\in h_a)$ and $(\tau_j,\sigma_j,(x,x_j,j))\in h_r$ where $\tau_j=\hat{h}(\tau_{j-1},\sigma_j)$ and $x_j\in\operatorname{R}_{\sigma_j}(x_{j-1})$.
	Since $\tau\in T\setminus\{\tau_0\}$, it follows that $n>0$ in this context.
	Then, there exists $t\in L(G,x)$ such that for all $i\in\mathcal{I}$, $P_i(t)=\tau^{(i)}$. 
	In accordance with the definition of the operator $\operatorname{R}(\cdot)$, it follows that $x'\in\delta(x,t)$ which appropriately signifies the ``$\Rightarrow$'' direction.
	
	The direction of ``$\Leftarrow$'' can be proven similarly based on the definition of CSS structure, which completes the proof of 1) in this proposition.
	
	2) To prove the truth of the predicate ``$\forall s\in L(G):SI(s)\in T$'', we proceed as follows.
	Given a string $s\in L(G)$, we know $SI(s)=(\mathcal{P}_1(s),\dots,\mathcal{P}_m(s))=(P_1(s\backslash\tilde{s}),\dots,P_m(s\backslash\tilde{s}))$.
	Then, there exist $x,x'\in X$ such that $x\in \delta(X_0,\tilde{s})$ and $x'\in\delta(x,s\backslash\tilde{s})$.
	By $(x,x,0)\in\mathcal{X}_0$, 
	similarly to the proof of 1) in this proposition, there exists an SI-state $\tau\in T$ such that for any $i\in\mathcal{I}$, $P_i(s\backslash\tilde{s})=\tau^{(i)}$.
    
    If there is no synchronization within the sequence $s$, it follows that $SI(\tilde{s})=\tau_0$;
    otherwise, $\tilde{s}\neq\epsilon$ and $SI(\tilde{s})\in T$.
    Since the string $\tilde{s}$ is immediately followed by a synchronization, $SI(\tilde{s})$ is a CSI-state, i.e., there does not exist an event  $\sigma\in\Sigma_{\mathcal{I}}$ such that $\hat{h}(SI(\tilde{s}),\sigma)$ is defined; we conclude that $SI(\tilde{s})\in T_c$.
\end{proof}


%\begin{proposition}\label{pro-current-string}
%	Let $\mathcal{T}=(\mathcal{X},T, h_a,h_r,\Sigma_{\mathcal{I}}\cup\{\epsilon\},T_0, \mathcal{X}_0,T_c)$ be the CSS structure w.r.t. system $G$ under DO-based protocol $\Upsilon_{\mathbb{N}}$. It holds:
%%	\begin{align*}
%%		SI(s)\in T\quad \text{and} \quad SI(\tilde{s})=\tau_0\lor SI(\tilde{s})\in T_c
%%	\end{align*}
%	\begin{enumerate}
%		\item $\forall s\in L(G):SI(s)\in T \land (SI(\tilde{s})=\tau_0\lor SI(\tilde{s})\in T_c)$; %i.e., $T$ contains all SI-states that could appear in the system evolution;
%		%\item $\forall \tau\in T,\forall (x,x')\in M(\tau),\exists t\in L(G,x):x'\in \delta(x,t)\land (\forall i\in \mathcal{I}:P_i(t)=\tau^{(i)})$
%		%\item $\forall x\in X, \forall t\in L(G,x),\exists \tau\in T\setminus\{\tau_0\}: \tilde{t}=\epsilon\land\tau^{(i)}=P_i(t)\Rightarrow \delta(x,t)=\{x'|(x,x')\in M(\tau)\}$
%		%\forall (x,x')\in M(\tau),\exists t\in L(G,x):x'\in \delta(x,t)\land (\forall i\in \mathcal{I}:P_i(t)=\tau^{(i)})$
%		\item $\forall \tau\in T\setminus\{\tau_0\},\exists x,x'\in X:((x,x')\in M(\tau)\Leftrightarrow (\exists t\in L(G,x):x\in\delta(x,t)\land (\forall i\in\mathcal{I}:P_i(t)=\tau^{(i)})))$
%	\end{enumerate}
%\end{proposition}
%\begin{proof}
%1) Let us prove ``$\forall s\in L(G):SI(s)\in T$'' first.
%For any $s\in L(G)$, we know $SI(s)=(\mathcal{P}_1(s),\dots,\mathcal{P}_m(s))=(P_1(s\backslash\tilde{s}),\dots,P_m(s\backslash\tilde{s}))$.
%Let $\omega_k=P_{\mathcal{I}}(s\backslash\tilde{s})$ and $P_i(s\backslash\tilde{s})=P_i(\omega_k)$, $i\in\mathcal{I}$, where $k$ denotes the length of the string $\omega_k$, then we prove this by induction on $|\omega_k|$.
%
%Induction basis: $|\omega_0|=0$, then $SI(s)=\tau_0\in T$.
%
%Induction step: Assume that the induction hypothesis is true when $|\omega_n|=n$. 
%
%
%
%\vspace{1cm}
%
%We consider the case when $|\omega_{n+1}|=n+1$ and let $\omega_{n+1}=\omega_{n}\sigma_{n+1}$, $\sigma_{n+1}\in\Sigma_{\mathcal{I}}$.
%Since $s\in L(G)$, we know there exist $x,x_1\in X$, $s_1\in L(G,x)$, such that $x\in\delta(X_0,\tilde{s})$, $x_1\in\delta(x,s_1)$, $P_{\mathcal{I}}(s_1)=\omega_n$, and $\operatorname{R}_{\sigma_{n+1}}(x_1)\neq\emptyset$.
%Let $\tau_1=(P_1(\omega_n),\dots,P_m(\omega_n))$.
%According to the induction hypothesis and the definition of CSS structure, we know that $\tau_1\in T$ and $(x,x_1)\in M(\tau_1)$.
%Since $\omega_{n+1}=\omega_{n}\sigma_{n+1}$, we know that $\hat{h}(\tau_1,\sigma_{n+1})=(P_1(\omega_{n+1}),\dots,P_m(\omega_{n+1}))$.
%According to the definition of $h_a$ in Definition \ref{def-ASS}, $(P_1(\omega_{n+1}),\dots,P_m(\omega_{n+1}))\in T$, thereby completing the proof.
%
%
%\vspace{1cm}
%When $\tilde{s}=\epsilon$, we know that $SI(\tilde{s})=\tau_0$ according to the definition of $\tilde{s}$.
%When $\tilde{s}\neq\epsilon$, based on the 1) in this proposition, we know $SI(\tilde{s})\in T$.
%Since string $\tilde{s}$ is immediately followed by a synchronization, we know $SI(\tilde{s})$ is a CSI-state, i.e., there does not exist an event  $\sigma\in\Sigma_{\mathcal{I}}$ such that $\hat{h}(SI(\tilde{s}),\sigma)$ is defined.
%Therefore, $SI(\tilde{s})\in T_c$.
%
%2) 
%($\Rightarrow$)
%If $(x,x')\in M(\tau)$, then according to the definition of CSS structure, there exist a sequence of events $\omega=\sigma_1\sigma_2\dots\sigma_n$, a sequence of SI-states $\tau_0\tau_1\dots\tau_n$, and a sequence of states $x_0x_1x_2\dots x_n$, such that $\tau_0=(\epsilon,\dots,\epsilon)$, $\tau_n=\tau$, $x_0=x$, $x_n=x'$ and for any $j\in\{1,\dots,n\}$, we have $((x,x_{j-1}),\sigma_j,\tau_j)\in h_a)$ and $(\tau_j,\sigma_j,(x,x_j))\in h_r$ where $\tau_j=\hat{h}(\tau_{j-1},\sigma_j)$ and $x_j\in\operatorname{R}_{\sigma_j}(x_{j-1})$.
%Note that, since $\tau\in T\setminus\{\tau_0\}$, $n>0$ in this situation.
%Then, there exists $t\in L(G,x)$ such that for all $i\in\mathcal{I}$, $P_i(t)=\tau^{(i)}$. 
%In accordance with the definition of the operater $\operatorname{R}_{\cdot}(\cdot)$, it follows that $x\in\delta(x,t)$ which approaciately signifies the ``$\Rightarrow$'' relation.
%
%($\Leftarrow$)
%Let $P_{\mathcal{I}}(t)=\omega$.
%Then, this can be proved by induction on $|\omega|$ with the induction basis $|\omega|=1$, following a procedure analogue to that described in 1) of this proposition.
%\end{proof}

Proposition \ref{pro-current-string}.2) indicates that the set $T$ in the system's CSS structure $\mathcal{T}$ contains all possible SI-states, including CSI-states, during the system evolution.
Combined with the conclusion in Proposition \ref{pro-current-string}.1), the CSS structure can be used to perform DO-based current-state estimation at one synchronization step, based on the assumption that the system is known to be in a subset of states $X'$.
This process is described in the following result.


\begin{proposition}\label{pro-current-entend}
Let $\mathcal{T}=(\mathcal{X},T, h_a,h_r,\Sigma_{\mathcal{I}}\cup\{\epsilon\},T_0, \mathcal{X}_0,T_c)$ be the CSS structure of a system $G$ under DO-based protocol $\Upsilon_{\mathbb{N}}$. 
%Assume that a string $s$ is generated by the system from a set of possible states $X'$ such that only one synchronization happens with $P_{\Upsilon_{\mathbb{N}}}(s)=\tau$ where $\tau\in T_c$; it holds: $\mathcal{E}^c(\tau,X')=\{x\in X|\exists x'\in X', (x',x)\in M(\tau)\}$.
Given $s\in L(G,X')$ such that $P_{\Upsilon_{\mathbb{N}}}(s)=\tau$ where $\tau\in T_c$, i.e., only one synchronization happens, it holds
\begin{align*}
\mathcal{E}^c(\tau,X')=\{x\in X|\exists x'\in X', (x',x)\in M(\tau)\}
\end{align*}
	%	\begin{enumerate}
		%		\item $\mathcal{E}^c(\tau,x)=\{x'\in X|\exists (x,x',l)\in\mathcal{X},\exists\sigma\in\Sigma_{\mathcal{I}}\cup\{\epsilon\}:(\tau,\sigma,(x,x',l))\in h_r\}$;
		%		\item $\mathcal{E}^{ec}(\tau,x)=\{x''\in X|\exists x'\in\mathcal{E}^c(\tau,x),\exists \tau'\in T\setminus T_c, \exists \sigma\in\Sigma_{\mathcal{I}}:(\tau',\sigma,(x',x'',l))\in h_r\}$.
		%	\end{enumerate}
\end{proposition}
\begin{proof}
The proof follows directly from Lemma \ref{Lemma-ASS-lastEvent} and Proposition \ref{pro-current-string}.
\end{proof}

The following proposition presents the method to perform DO-based initial-state estimation at one synchronization step.


\begin{proposition}\label{pro-ini}
Let $\mathcal{T}=(\mathcal{X},T, h_a,h_r,\Sigma_{\mathcal{I}}\cup\{\epsilon\},T_0, \mathcal{X}_0,T_c)$ be a CSS structure of a system $G$ under DO-based protocol $\Upsilon_{\mathbb{N}}$. 
%Assume that a string $s$ is generated by the system such that only one synchronization happens with $P_{\Upsilon_{\mathbb{N}}}(s)=\tau$ where $\tau\in T_c$. Then,  it holds: $
%	     \mathcal{E}^i(\tau,X_0)=\{x\in X_0|\exists x'\in X, (x,x')\in M(\tau)\}.
%$
Given $s\in L(G,X_0)$ such that $P_{\Upsilon_{\mathbb{N}}}(s)=\tau$ where $\tau\in T_c$, i.e., only one synchronization happens, it holds
\begin{align*}
	     \mathcal{E}^i(\tau,X_0)=\{x\in X_0|\exists x'\in X, (x,x')\in M(\tau)\}.
\end{align*}


\end{proposition}
\begin{proof}
According to the definition of $h_r$ in Definition \ref{def-ASS}, the first element of $\rho$ remains unchanged.
With this understanding, this proof follows directly from Proposition \ref{pro-current-string}.
\end{proof}

\begin{example}
	Let us consider the system in Example \ref{example-2}. 
	When the sequence $s_1=\alpha_{12}\upsilon_2\gamma_3\alpha_{12}$ is generated from state $q_0$, the SI-states are as follows, in chronological order: $(\alpha_{12},\alpha_{12},\epsilon)$, $(\alpha_{12},\alpha_{12},\gamma_3)$, $(\alpha_{12}\alpha_{12},\alpha_{12}\alpha_{12},\gamma_3)$. 
	Synchronization occurs after the sequence $s_1$, and the coordinator receives $\tau_5=(\alpha_{12}\alpha_{12},\alpha_{12}\alpha_{12},\gamma_3)$. 
	Based on the CSS structure shown in Fig.~\ref{fig-ASS-1}: 1) If we aim to estimate the current-state of the system, the corresponding state estimates are $\{q_2,q_3,q_4\}$, since $(q_0,q_2,3)$, $(q_0,q_3,3)$, and $(q_0,q_4,3)$ are transitioned to by $\tau_5$.
	2) If we aim to estimate the initial-state of the system solely based on the sequence $s_1$, the corresponding initial-state estimate is $\{q_0\}$, since the first state component of the state pairs that are transitioned to by $\tau_5$ is $q_0$.\hfill\rule{1ex}{1ex}
	
\end{example}


As the CSS structure $\mathcal{T}$ is constructed from any state $x\in X$, we can obtain the state estimates from any system state upon any possible CSI-state.
However, this may increase the complexity of building such a structure, since it is not necessary to identify all possible CSI-states from every system state.
In other words, some CSI-states in $\mathcal{T}$ may not be received by the coordinator during the system evolution.
In Algorithm~\ref{algo1-ASS}, we demonstrate how to construct this structure such that only the CSI-states that can actually be received by the coordinator during the system evolution are computed. 
We denote this structure as $\mathcal{T}_{fe}$ (feasible CSS structure).


In Algorithm~\ref{algo1-ASS}, the set $\mathcal{X}_0$ is assigned the set of pairs of identical system initial states.
From the states in $\mathcal{X}_0$, the structure is expanded using a breadth-first search that alternates between SI-state layers and state pair layers.
The variable $l_{u}$ denotes the depth of state pair layers within the breadth-first level.
The procedures $\Call{SI-State}{\tau}$ and $\Call{State-SI}{\rho}$ are employed to explore the next state pairs and SI-states, respectively, building the relations of transition functions $h_r$ and $h_a$.




In lines \ref{algo1-add to ini1}--\ref{algo1-add to ini2}, %the second element in the pair of $(\rho_0,x,\rho_d+1)$ 
state $x$ represents potential state estimates under the CSI-state $\tau$ and $(x,x,0)$ is added to the set $\mathcal{X}_0$ to explore the possible CSI-states in the next synchronization.
This mechanism, as described, can be effectively adapted to build the feasible CSS structure, particularly when the set $\mathcal{X}_0$ is initialized with ``$\{(x_0,x_0,0)|x_0\in X_0\}$'' to avoid unnecessary calculations.
\begin{algorithm}[t]
	\caption{Construction of feasible CSS structure.}\label{algo1-ASS}
	\begin{algorithmic}[1]
		\Require System $G$ and DO-based protocol $\Upsilon_{\mathbb{N}}$.
		\Ensure  A feasible CSS structure $\mathcal{T}_{fe}=(\mathcal{X},T,h_a,h_r,\Sigma_{\mathcal{I}}\cup\{\epsilon\},T_0, \allowbreak \mathcal{X}_0,T_c)$.
		\State $\mathcal{X}_0\gets \{(x_0,x_0,0)|x_0\in X_0\}$, $T\gets T_0$, $\mathcal{X}\gets\mathcal{X}_0$, 
		\State $T_0\gets \{\tau_0\}$, $h_a\gets\emptyset$, $h_r\gets\emptyset$, $T_{in}\gets\{\tau_0\}$;
		\For{$\rho\in\mathcal{X}_0$ that has not been examined} \label{algo1-examine-ini}
		\State $l_u\gets 0$ and $h_r\gets h_r\cup\{(\tau_0, \epsilon,\rho)\}$;
		\EndFor
		\While{$T_{in}\neq\emptyset$}
		\State $T_{in}\gets\emptyset$;
		\For{$\rho\in\mathcal{X}$, s.t. $\rho_d=l_u$, that has not been examined}
		\State $\Call{State-SI}{\rho}$, $T_{in}\gets T_{in}\cup\{\tau|(\rho,\sigma,\tau)\in h_a\}$;
		\EndFor
		\For{$\tau\in T_{in}$}
		\State  $\Call{SI-State}{\tau}$;
		\EndFor
		\State $l_u\gets l_u+1$;
		\EndWhile
		\If{$\exists\rho\in\mathcal{X}_0$ that has not been examined}
		\State Go to line \ref{algo1-examine-ini};
		\EndIf
		\State Return $\mathcal{T}_{fe}$; %投稿前删
		\Procedure{State-SI}{$\rho$}
		\For{$(\tau',\sigma',\rho)\in h_r$ that has not been examined}
		\State $h_a\gets \{(\rho,\sigma,\tau)|\exists\sigma\in\Sigma_{\mathcal{I}}$: $\hat{h}(\tau',\sigma)=\tau\land\operatorname{R}_{\sigma}(\rho_1)\neq\emptyset\}$, $T\gets T\cup\{\tau\}$;
		\If{$\tau$ is a CSI-state}
		\State $T_c\gets T_c\cup\{\tau\}$;
		\EndIf
		\EndFor
		\EndProcedure
		\Procedure{SI-State}{$\tau$}
		\For{$(\rho,\sigma,\tau)\in h_a$ that has not been examined}
		\State $h_r\gets h_r\cup\{(\tau, \sigma, (\rho_0,x,\rho_d+1))|x\in\operatorname{R}_{\sigma}(\rho_1)\}$, $\mathcal{X}\gets \mathcal{X}\cup\{(\rho_0,x,\rho_d+1)|x\in\operatorname{R}_{\sigma}(\rho_1)\}$, $L\gets L\cup\{\rho_d+1\}$;\label{algo1-add to ini1}
		\If{$\tau\in T_c$}
		\State $\mathcal{X}_0\gets\mathcal{X}_0\cup\{(x,x,0)|x\in\operatorname{R}_{\sigma}(\rho_1)\}$; \label{algo1-add to ini2}
		\EndIf
		\EndFor
		\EndProcedure
	\end{algorithmic}
\end{algorithm}





\begin{proposition}
	Let $\mathcal{T}_{fe}=(\mathcal{X},T, h_a,h_r,\Sigma_{\mathcal{I}}\cup\{\epsilon\},T_0, \mathcal{X}_0,T_c)$ be the feasible CSS structure of a system $G$ under DO-based protocol $\Upsilon_{\mathbb{N}}$. The following hold:
	\begin{enumerate}
		\item $\bigcup_{s\in L(G)}SI(s)=T$; %i.e., $T$ contains all SI-states that could appear in the system evolution;
		\item $\bigcup_{s\in L(G)}SI(\tilde{s})=\{\tau_0\}\cup T_c$.%, i.e., $T_c$ is the set of all CSI-states.
	\end{enumerate}
\end{proposition}
\begin{proof}
This follows directly from Algorithm~\ref{algo1-ASS}, by induction on the length of $P_{\mathcal{I}}(s)$.
\end{proof}



\begin{remark}
Different synchronization strategies lead to different CSS structure structures, characterized by the function $\hat{h}$ in the definition of $h_a$. 
At first glance, the computation complexity of both DO-based current-state estimation and DO-based initial-state estimation (simply taking the union of states in precomputed feasible CSS structure) at one synchronization step appears lower than the online approach described in \cite{SunHadjicostisLi2023}. 
However, this does not negate the necessity of the online method. 
Note that the CSS structure must be constructed beforehand when employing state estimation using the method in this paper. 
This introduces additional computational costs compared with the one-time online method.  
The online method offers more flexibility in handling situations, where the synchronization strategy is not fixed or not predetermined. 
For example, in one synchronization step,  $|\mathcal{P}_i(s)|\leq 2$ for all $i\in\mathcal{I}$ in $(\ref{SS})$ in Section \ref{sec-2}.B, while in the next synchronization step, it could change to $|\mathcal{P}_i(s)|\leq 8$ for all $i\in\mathcal{I}$ to minimize the costs associated with communication.
However, as analyzed later, when a DO-based protocol employs a fixed synchronization strategy, the CSS structure can be effectively utilized to construct state-estimators (observers).
\end{remark}



\subsection{Current-State and Initial-State Estimator Construction}

Since a CSS structure reflects the system evolution from any system state to possible system states through potential CSI-states, it can be employed to recursively solve both current and initial state estimation problems upon a sequence of CSI-states in accordance with Propositions \ref{pro-current-entend} and \ref{pro-ini}, respectively.
%\textcolor{blue}{Furthermore, after the final recursive step, extended state estimation is computed using the conclusions in Propositions \ref{pro-current-entend}.2). }
Furthermore, in many applications, such as diagnosability and opacity, it is essential to have detailed knowledge of all state estimates prior to the system's operation, particularly for verification purposes.
In light of this, we introduce the notions of DO-based current-state estimator (DO-observer) and DO-based initial-state estimator below.




\begin{definition}\label{DO-observer}
Consider system $G=(X,\Sigma, \delta, X_0)$ under DO-based protocol $\Upsilon_{\mathbb{N}}$, and let $\mathcal{T}_{fe}=(\mathcal{X},T,h_a,h_r,\Sigma_{\mathcal{I}}\cup\{\epsilon\},\allowbreak T_0, \mathcal{X}_0,T_c)$ be its feasible CSS structure. 
The DO-observer of $G$ is a deterministic finite automaton (DFA) $G^{\Upsilon_{\mathbb{N}}}_{obs}=Ac(X_{obs},T_c,f_{obs},\allowbreak X_{obs,0})$\footnote{$Ac(\cdot)$ is the accessible part of $\cdot$, i.e., the part that can be reached from the initial state.} where
	\begin{itemize}
		\item $X_{obs}\subseteq 2^X\setminus\{\emptyset\}$ is the set of states;
		\item $T_c$ is the set of CSI-states which serves as the set of transition events;
		\item $X_{0,obs}=\operatorname{UR}(X_0)\in X_{obs}$ is the initial state;
		\item $f_{obs}:X_{obs}\times T_c\rightarrow X_{obs}$ is the transition function defined as %by: for any $q\in X_{obs}$, $\tau\in T_c$, we have
		\begin{align*}
			%f_{obs}(q,\tau)=\{x|\exists x'\in q: x\in \mathcal{E}^c(\tau,x')\}.
			f_{obs}(q,\tau)= \mathcal{E}^c(\tau,q), \forall q\in X_{obs}, \tau\in T_c.
		\end{align*}
	\end{itemize}
\end{definition}


Similar to a standard observer \cite{Hadjicostis2020}, the DO-observer contains all possible state estimates following any sequence of CSI-states starting from the initial state set $X_{0,obs}$. 
The key difference, however, lies in the driving events: in the DO-observer, the transitions from one state to another are driven by elements in $T_c$, rather than (observable) events in $\Sigma_{\mathcal{I}}$. 
This arrangement is logical, as the coordinator can only update its state estimate upon receiving synchronization information, i.e., a CSI-state.

Note that the initial state of the DO-observer is assigned as the unobservable reach of the system initial states. 
Actually, every subsequent state is assigned as the unobservable reach of the set of state estimates immediately following the last synchronization. 
%\textcolor{blue}{Therefore, the predicate ``$x\in \mathcal{E}^c(\tau,x')$'' is employed in the definition of $f_{obs}(q,\tau)$.}
These settings are implemented since the coordinator updates its state estimate based on the state estimates prior to the occurrence of a CSI-state.

Function $f_{obs}$ operates as follows: based on the current state $q$, it explores any $\tau$ in $T_c$, considered as its synchronization information, such that $\mathcal{E}^c(\tau,q)$ is not empty. 
The result is then adopted as the next state of the estimator.
The domain of function $f_{obs}$ can be extended to $X_{obs}\times T_c^*$ in the standard recursive manner: $f_{obs}(q,\lambda\tau)=f_{obs}(f_{obs}(q,\lambda),\tau)$ for $q\in X_{obs}$, $\lambda\in T_c^*$, and $\tau\in T_c$ ($f_{obs}(q,\lambda\tau)$ is undefined if $f_{obs}(q,\lambda)$ is undefined).
\begin{proposition}\label{pro-obs}
	Given a system $G$ and its DO-observer $G^{\Upsilon_{\mathbb{N}}}_{obs}$, we have (i) $\bigcup_{s\in L(G)}{P_{\Upsilon_{\mathbb{N}}}(s)}= L(G^{\Upsilon_{\mathbb{N}}}_{obs})$, and (ii) for any $s\in L(G)$, $\mathcal{E}^c(P_{\Upsilon_{\mathbb{N}}}(s),X_0)=f_{obs}(X_{0,obs},P_{\Upsilon_{\mathbb{N}}}(s))$.
%	\begin{align*}
%		\mathcal{E}^c(P_{\Upsilon_{\mathbb{N}}}(s),X_0)=f_{obs}(X_{0,obs},P_{\Upsilon_{\mathbb{N}}}(s)).
%	\end{align*}
\end{proposition}
\begin{proof}
We first prove (i).%``$\bigcup_{s\in L(G)}{P_{\Upsilon_{\mathbb{N}}}(s)}= L(G^{\Upsilon_{\mathbb{N}}}_{obs})$''.
	
($\subseteq$) 
Given a string $s\in L(G)$, its corresponding run is $\vec{s}=s_1{\scriptstyle\sim}s_2{\scriptstyle\sim} \dots {\scriptstyle\sim}s_j{\scriptstyle\sim} \dots {\scriptstyle\sim} s_n{\scriptstyle\sim}s_{n+1}$, where $s_j=s_{j1}s_{j2}\dots s_{jk_{j}}$, for any $j\in\{1,2,\dots,n\}$ and $|s_j|=k_j$.
Then, there exist $x_0\in X_0$, $x_j\in X$,  such that $x_j\in\delta(x_{j-1},s_j)$.
According to the definition of function $f_{obs}$, there exist $\iota\in T^*_c$, and $X_j\in 2^X$, for any $j\in\{1,2,\dots,n\}$, such that $\iota=(P_1(s_1),\dots,P_m(s_1))\dots(P_1(s_j),\dots,P_m(s_j))\dots(P_1(s_n),\\\dots, P_m(s_n))$, $f_{obs}(X_{j-1},(P_1(s_j), \dots,P_m(s_j)))=X_j$, and $x_j\in X_j$.
Together, these indicate $\iota\in L(G^{\Upsilon_{\mathbb{N}}}_{obs})$, thereby completing the proof of the ($\subseteq$)  relation.
	
The reverse ($\supseteq$) can be proven similarly, which completes the proof of (i).
	
We prove (ii) by induction on the length of $|P_{\Upsilon_{\mathbb{N}}}(s)|$, i.e.,  the number of CSI-states received by the coordinator.
Clearly, the basic induction step holds: when $|P_{\Upsilon_{\mathbb{N}}}(s)|=0$ (i.e., $\tilde{s}=\epsilon$), according to the definition of DO-CES, $\mathcal{E}^c(\epsilon,X_0)=\operatorname{UR}(X_0)=X_{0,obs}$, i.e., the initial state of $G^{\Upsilon_{\mathbb{N}}}_{obs}$.
	
For the induction step, we assume that the induction hypothesis is true when $|P_{\Upsilon_{\mathbb{N}}}(s)|=k$, $k\in\mathbb{N}$.
Then, let us consider the case when $|P_{\Upsilon_{\mathbb{N}}}(s)|=k+1$.
Suppose that $s=s_1 {\scriptstyle\sim} s_2{\scriptstyle\sim}  \dots \allowbreak{\scriptstyle\sim} s_j {\scriptstyle\sim} \dots {\scriptstyle\sim} s_k{\scriptstyle\sim}s_{k+1}{\scriptstyle\sim}s_{k+2}$ and $t=s_1 {\scriptstyle\sim} s_2 \dots {\scriptstyle\sim} s_j {\scriptstyle\sim} \dots {\scriptstyle\sim} s_k$.	
According to the inductive assumption,  
$\mathcal{E}^c(P_{\Upsilon_{\mathbb{N}}}(t),X_0)=f_{obs}\allowbreak(X_{0,obs},P_{\Upsilon_{\mathbb{N}}}(t))$.
Let $\tau=(P_1(s_{k+1}),\dots,P_m(s_{k+1}))$.
Based on Proposition~\ref{pro-current-entend}  and the function $f_{obs}$, we know
$\mathcal{E}^c(\tau,\mathcal{E}^c(P_{\Upsilon_{\mathbb{N}}}(t), X_0))=\allowbreak f_{obs}(f_{obs}(X_{0,obs},P_{\Upsilon_{\mathbb{N}}}(t)),\tau)$, thereby completing the proof.
\end{proof}




%\vspace{2cm}
%
%	DO-ECES aims to determine the set of states that the system possibly resides in before the next synchronization, i.e.,\footnote{\textcolor{red}{Or I define this definition before property?}}
%	\begin{multline*}
%		\mathcal{E}^{ec}(P_{\Upsilon}(s),X_0)=\{x\in X|\exists x_0\in X_0,\exists u\in L(G), \\ P_{\Upsilon}(s)=P_{\Upsilon}(t)\wedge x\in\delta(q,u)\}. 
%	\end{multline*}
%	
%Specifically, $\mathcal{E}^c(P_{\Upsilon}(s),X_0)$ captures the unobservable reach of state estimates immediately following the last synchronization in a run $\vec{s}$. In contrast, $\mathcal{E}^{ec}(P_{\Upsilon}(s),X_0)$ extends beyond $\mathcal{E}^c(P_{\Upsilon}(s),X_0)$ by also considering the possible states of the system prior to the next synchronization. This definition allows for a more comprehensive understanding of the system's dynamic states through the sequence of events.





\begin{figure}[htbp]
	\centering
	\includegraphics[scale=0.6]{DO-based_observer.pdf}
	\caption{DO-based observer for $G$, where the transitions labeled as ``$\tau$'' correspond to the labels shown in Fig.~\ref{fig-ASS-1}.}
	\label{DO-based observer}
\end{figure}
\begin{example}
		Let us revisit system $G$ in Example \ref{example-2}. 
		Since its feasible CSS structure has the same structure as in Fig.~\ref{fig-ASS-1}, we will refer to it as the feasible CSS structure from now on. 
		Suppose the initial state set is $\{q_0,q_1\}$. 
		The DO-observer for $G$ is shown in Fig.~\ref{DO-based observer}, with the initial state being $\{q_0,q_1\}$.
		By receiving the sequence of CSI-states $\tau_5\tau_3$, the state estimate of the coordinator is $f_{obs}(\{q_0,q_1\},\tau_5\tau_3)=\{q_2\}$.\hfill\rule{1ex}{1ex}
\end{example}
\begin{remark}
Given that our focus is on interpreting the synchronization strategy, specifically the construction of a CSS structure and the verification of relevant properties from the coordinator’s perspective, the DO-based current-state estimation is defined to concentrate on the instant when synchronization occurs, even if multiple observations are generated between synchronizations.
This notion, along with the DO-observer, can be clearly extended to account for what happens between synchronizations based on different purposes, which correlates with the discussion in Remark \ref{remark-aftersync}.

%The DO-based current-state estimation concentrates on the instant when a synchronization occurs(despite the fact that multiple observations are generated between synchronizations), since the focus is on interpreting the synchronization strategy, specifically the construction of an CSS structure and the verification of some relevant properties from the point of view of the coordinator.	
%Clearly, the notions of DO-based current-state estimation and DO-observer can be extended to account for what happens between synchronizations based on different purposes, which correlates with the discussion in Remark \ref{remark-aftersync}.
\end{remark}
Next, we define the notion of a DO-based initial-state estimator, which aligns with the trellis-based approach discussed in the literature \cite{SabooriHadjicostis2008ini,Hadjicostis2020}.
\begin{definition}
	Consider system $G=(X,\Sigma, \delta, X_0)$ under DO-based protocol $\Upsilon_{\mathbb{N}}$, and let $\mathcal{T}_{fe}=(\mathcal{X},T, h_a,h_r,\Sigma_{\mathcal{I}}\cup\{\epsilon\},T_0, \mathcal{X}_0,T_c)$ be its feasible CSS structure. 
	The DO-based initial-state estimator of $G$ is a DFA $G^{\Upsilon_{\mathbb{N}}}_{i,obs}=Ac(X^{i}_{obs},T_c,f^i_{obs},X^{i}_{0,obs})$ where 
	\begin{itemize}
		\item $X^i_{obs}\subseteq 2^{X\times X}\backslash\{\emptyset\}$ is the set of states;
		\item $T_c$ is the set of CSI-states which serves as the set of transition events;
		\item $X^{i}_{0,obs}=\{(x_0,x_0)|x_0\in X_0\}$ is the initial state of $G$;
		\item $f^i_{obs}:X^i_{obs}\times T_c\rightarrow X^i_{obs}$ is the transition function defined as: for any $m\in X^i_{obs}$ and $\tau\in T_c$, it holds
%		\begin{align*}
%			f^i_{obs}(m,\tau)=\{(x_1,x_3)|\exists (x_1,x_2)\in m,x_3\in\mathcal{E}^c(\tau,\{x_2\})\}
%		\end{align*}
		\begin{align*}
			&\hspace{-2mm}f^i_{obs}(m,\tau)=\{(x_1,x_3)|\exists (x_1,x_2)\in m,(x_2,x_3)\in M(\tau)\}.
		\end{align*}
	\end{itemize}
\end{definition}
The initial state $X^{i}_{0,obs}$ is the set of pairs of identical initial system states.
Each element in any other state $m\in X^i_{obs}\backslash \{X^{i}_{0,obs}\}$ contains the initial state component and the corresponding reachable current state component upon a sequence of CSI-states so far.
This mechanism is implemented by function $f^i_{obs}$.
Like function $f_{obs}$, the domain of $f^i_{obs}$ can be extended to $X^i_{obs}\times T_c^*$ in a similar manner. For any state $m\in X^i_{obs}$, we use $F(m)$ to denote the set of first components in $m$, i.e., $F(m)=\{x|\exists x'\in X:(x,x')\in m\}$. 
Then, we have the following result.
\begin{proposition}\label{prop-ini}
	For any $s\in L(G)$, $\bigcup_{s\in L(G)}{P_{\Upsilon_{\mathbb{N}}}(s)}=L(G^{\Upsilon_{\mathbb{N}}}_{i,obs})$ and $		\mathcal{E}^{i}(P_{\Upsilon_{\mathbb{N}}}(s),X_0)=F(f^i_{obs}(X^i_{0,obs},P_{\Upsilon_{\mathbb{N}}}(s)))$ hold.
%	\begin{align*}
%		\mathcal{E}^{i}(P_{\Upsilon_{\mathbb{N}}}(s),X_0)=F(f^i_{obs}(X^i_{0,obs},P_{\Upsilon_{\mathbb{N}}}(s))).
%	\end{align*}
\end{proposition}
\begin{proof}
The proof of this proposition employs the same strategy as that utilized in the proof of Proposition \ref{pro-obs}, and is therefore omitted here.
\end{proof}
\subsection{Complexity Analysis}
Given a system $G=(X,\Sigma,\delta,X_0)$ under DO-based protocol $\Upsilon_{\mathbb{N}}$, in the worst case, the maximum number of sequences of observations observed by $O_i$ is $\Delta_i:=|\Sigma_i|^0+|\Sigma_i|+\dots+|\Sigma_i|^j+\dots+|\Sigma_i|^{\kappa_i}$, where $j\in\{0,1,\dots,\kappa_i\}$ and $|\Sigma_i|^j$ is number of sequences of length $j$.
The number of sequences that cannot (can) initiate a synchronization is $\Delta_i-|\Sigma_i|^{\kappa_i}=\frac{|\Sigma_i|^{\kappa_i } - 1}{|\Sigma_i| - 1}\approx|\Sigma_i|^{\kappa_i-1}$ ($|\Sigma_i|^{\kappa_i}$).
Therefore, the number of SI-states that cannot initiate a synchronization is bounded by $\prod^m_{i=1}|\Sigma_i|^{\kappa_i-1}$.
If we set $O_j$ to be the OS that initiates the synchronization, the number of CSI-states is at most $|\Sigma_j|^{\kappa_j}\prod_{i\in\mathcal{I}\setminus\{j\}}|\Sigma_i|^{\kappa_i-1}=|\Sigma_j|\cdot|\Sigma_j|^{\kappa_j-1}\cdot\prod_{i\in\mathcal{I}\setminus\{j\}}|\Sigma_i|^{\kappa_i-1}=|\Sigma_j|\cdot\prod^m_{i=1}|\Sigma_i|^{\kappa_i-1}$.
Thus, the total number of CSI-states can be bounded by $\sum_{j=1}^m(|\Sigma_j|\cdot\prod^m_{i=1}|\Sigma_i|^{\kappa_i-1})=\prod^m_{i=1}|\Sigma_i|^{\kappa_i-1}\cdot\sum_{j=1}^m|\Sigma_j|$.\footnote{Note that the symbol $\sum_{j=1}^m$ represents the summation operator, while $\Sigma_i$ denotes a set of events.}
Note that this is the worst case (e.g., when no observable events are shared among the OSs, which means $\sum_{j=1}^m|\Sigma_j|=|\Sigma_{\mathcal{I}}|$).
Since the number of CSI-states is bounded by $\Delta_c:=|\Sigma_{\mathcal{I}}|\cdot\prod^m_{i=1}|\Sigma_i|^{\kappa_i-1}$, the number of SI-states is bounded by $\Delta:=\Delta_c+\prod^m_{i=1}|\Sigma_i|^{\kappa_i-1}=(|\Sigma_{\mathcal{I}}|+1)\cdot\prod^m_{i=1}|\Sigma_i|^{\kappa_i-1}$.

Let $i_{u}$ be one of the indices of OSs that can record the most events, i.e., $i_{u}\in\arg \max_{i\in\mathcal{I}}\kappa_i$.
The length of the longest sequence of observable events that can be generated by the system and can initiate a synchronization is $l_u:=\kappa_{i_{u}}+\Sigma_{i\in\mathcal{I}\setminus\{i_{u}\}}(\kappa_i-1)$.

With the above notation at hand, the number of states in a CSS structure is at most  $l_u\cdot |X|^2+|X|+\Delta$ (where there are at most $l_u$ layers, each with at most $|X|^2$ states), the number of state pairs with layer ``0'' is $|X|$, and that of SI-states is $\Delta$.
Therefore, the state complexity of constructing a CSS structure is $O(l_u\cdot |X|^2+\Delta)$.
If there is only one initial state pair in a CSS structure (like an SS), each SI-state is transitioned to by at most $|X|\cdot m$ transitions, and there are at most $m$ state estimation processes at each SI-state, resulting in complexity of $m\cdot|X|^2$.
Therefore, the computation complexity of constructing a CSS structure is $O((|X|\cdot m+m\cdot|X|^2)\cdot\Delta\cdot|X|)$, i.e., $O(\Delta\cdot m\cdot|X|^2(1+|X|))$.

Compared with the approach for DO-CSE in \cite{SunHadjicostisLi2023}, which has computational complexity of $O(|X|^2\cdot\prod_{i=1}^{m}(\kappa_i+1)\cdot m)$, assuming that a CSS structure is precomputed, the complexity at a synchronization step is reduced to $O(|X|^2)$.
There are at most $2^{|X|}$ states and $2^{|X|}\cdot\Delta_c$ transitions in a DO-observer. 
Therefore, the worst-case computational complexity of constructing a DO-observer is $O(2^{|X|}\cdot\Delta_c)$.
Similarly, the computational complexity of constructing the DO-based initial-state estimator is $O(2^{|X|^2}\cdot\Delta_c)$.


\section{Applications of DO-based State Estimator}

Opacity is an important notion that characterizes the security and privacy of information flow from a system to the public.
The secret of a system is  typically represented by a subset of states $X_S\subseteq X$.
In the setting considered in this paper, the system is observed by multiple observation sites that follow a DO-based protocol; thus, we consider the possibility that there exists a malicious outside observer (eavesdropper) that can track all partially ordered sequences of observations transmitted to the coordinator at each synchronization.
This consideration is natural if we treat the information flow from the OSs to the coordinator as insecure.
In this section, we define and analyze two kinds of opacity, called DO-based initial-state opacity (DO-ISO) and DO-based current-state-at-synchronization opacity (DO-CSSO).
In the end, we also provide some brief remarks on the verification of diagnosability under the DO-based protocol.





\subsection{Verification of DO-based Initial-State Opacity}
\begin{definition}
	(DO-ISO) A system $G=(X,\Sigma,\delta,X_0)$ under DO-based protocol $\Upsilon_{\mathbb{N}}$ and a set of secret initial states $X_S\subseteq X_0$ is said to be \textit{DO-based initial-state opaque} (w.r.t. $\Upsilon_{\mathbb{N}}$ and $X_S$)~if 
		\begin{align*}
		\forall s\in L(G,X_S), \exists s'\in L(G,X_0\backslash X_S):P_{\Upsilon_{\mathbb{N}}}(s)=P_{\Upsilon_{\mathbb{N}}}(s').
	\end{align*}
\end{definition}
Intuitively, DO-ISO\footnote{We use DO-ISO (DO-CSSO) as the acronym for both ``DO-based initial-state opaque/opacity'' (``DO-based current-state-at-synchronization opaque/opacity'').} requires that the intruder should never know with certainty if a system starts from a secret state after arbitrary number of synchronization steps.
With this understanding, the problem of the verification of DO-ISO can be reduced to the initial-state estimation problem, i.e., determining whether the initial-state estimates always contain a non-secret state.
Therefore, we have the following outcome.
\begin{theorem}\label{theorem-ini-1}
Given a system $G=(X,\Sigma,\delta,X_0)$ under DO-based protocol $\Upsilon_{\mathbb{N}}$ and  a set of secret initial states $X_S\subseteq X_0$, let $G^{\Upsilon_{\mathbb{N}}}_{i,obs}=(X^{i}_{obs},T_c,f^i_{obs},X^{i}_{0,obs})$ be its DO-based initial-state estimator. System $G$ is DO-ISO w.r.t. $\Upsilon_{\mathbb{N}}$ and $X_S$ if and only if for all $m\in X^{i}_{obs}$, $F(m)\nsubseteq X_S$ holds.
\end{theorem}
\begin{proof}
($\Rightarrow$) Suppose that system $G$ is DO-ISO. 
Then, for any $s\in L(G,X_S)$, there exists $s'\in L(G,X_0\backslash X_S)$ such that $P_{\Upsilon_{\mathbb{N}}}(s)=P_{\Upsilon_{\mathbb{N}}}(s')$.
	%Since $P_{\Upsilon_{\mathbb{N}}}(s)=P_{\Upsilon_{\mathbb{N}}}(s')$, 
Then, we know $\mathcal{E}^{i}(P_{\Upsilon_{\mathbb{N}}}(s),X_0\backslash X_S)\neq\emptyset$.
It is easy to verify that $\mathcal{E}^{i}(P_{\Upsilon_{\mathbb{N}}}(s),X_0)=\mathcal{E}^{i}(P_{\Upsilon_{\mathbb{N}}}(s),X_S)\cup\allowbreak \mathcal{E}^{i}(P_{\Upsilon_{\mathbb{N}}}(s),X_0\backslash X_S)$.
By Proposition~\ref{prop-ini}, $F(f^i_{0,obs}(X^i_{0,obs}, \allowbreak P_{\Upsilon_{\mathbb{N}}}(s)))\cap (X_0\backslash X_S)\neq\emptyset$. 
Then, $f^i_{0,obs}(X^i_{0,obs},P_{\Upsilon_{\mathbb{N}}}(s))\in X^{i}_{obs}$ and $F(f^i_{0,obs}(X^i_{0,obs},P_{\Upsilon_{\mathbb{N}}}(s)))\nsubseteq X_S$ hold, which completes the proof in this direction.
    
(``$\Leftarrow$'') Suppose that for any $m\in X^{i}_{obs}$, $F(m)\nsubseteq X_S$. 
According to Proposition~\ref{prop-ini}, there exists $s\in L(G,X_0)$ such that $m=f^i_{0,obs}(X^i_{0,obs},P_{\Upsilon_{\mathbb{N}}}(s))$ and $\mathcal{E}^{i}(P_{\Upsilon_{\mathbb{N}}}(s),X_0)=F(m)$.
Then, it gives $\mathcal{E}^{i}(P_{\Upsilon_{\mathbb{N}}}(s),X_0)\nsubseteq X_S$, indicating that there exists $s'\in L(G,X_0\setminus X_S)$ such that  $P_{\Upsilon_{\mathbb{N}}}(s)=P_{\Upsilon_{\mathbb{N}}}(s')$, i.e., $G$ is DO-ISO.
\end{proof}

In \cite{WuLafortune2013}, the notion of reversed automaton is introduced to verify centralized initial-state opacity. 
However, this method cannot be implemented directly to verify DO-ISO here, as the coordinator receives a set of partially ordered sequences at a time, rather than a single observation.
Nevertheless, inspired by this method, we present an exponential time approach to verify DO-ISO, compared with the one in Theorem \ref{theorem-ini-1}, which has doubly-exponential complexity. 
Instead of introducing the reversed automaton of the original system, we directly define the notion of a synchronization-reversed observer as follows.
\begin{definition}\label{def-sro}
	The synchronization-reversed observer of a system $G$ is a deterministic finite-state automaton $G^{\Upsilon_{\mathbb{N}}}_{R,obs}=Ac(X_{R,obs},T_c, f_{R,obs}, X)$, where
	\begin{itemize}
		\item $X_{R,obs}\subseteq 2^X\setminus\{\emptyset\}$ is the set of states;
		\item $T_c$ is the set of CSI-states of system $G$'s feasible CSS structure;
		\item $X\in X_{R,obs}$ is the initial state;
		\item $f_{R,obs}: X_{R,obs}\times T_c\rightarrow X_{R,obs}$ is the transition function defined as: for any $q\in X_{R,obs}$, $\tau\in T_c$,
		\begin{align*}
			f_{R,obs}(q,\tau)=\{x\in X|\exists x'\in q,(x,x')\in M(\tau)\}.
		\end{align*}
	\end{itemize}
\end{definition}

Intuitively, the differences between DO-observer and synchronization-reversed observer lie in the transition function and the initial state.
Function $f_{R,obs}$ computes the next state by reversing the state pairs in $M(\tau)$, i.e., it takes the union of the first components in $M(\tau)$ whose corresponding second components belong to $q$.
The initial state $X$ is the entire state space of system $G$.
Like function $f_{obs}$, the domain of $f_{R,obs}$ can be extended to $X_{R,obs}\times T^*_c$ in the same manner.
Given a sequence of CSI-states $\iota\in T^*_c$, $\iota_R$ is the reversed sequence of $\iota$.
We arrive at the following conclusion based on $G^{\Upsilon_{\mathbb{N}}}_{R,obs}$.

\begin{lemma}\label{lemma-reverse}
Let $G^{\Upsilon_{\mathbb{N}}}_{R,obs}$ be the synchronization-reversed observer of a system $G$ and for any $s\in L(G)$, let $\iota=P_{\Upsilon_{\mathbb{N}}}(s)$ be the sequence of CSI-states received by the coordinator. 
Then, it holds $\mathcal{E}^{i}(\iota,X_0)=f_{R,obs}(X,\iota_R)\cap X_0$.
\end{lemma}
\begin{proof}
	By using the extended definition of $f_{R,obs}$, for any sequence of CSI-states $\iota=\tau_n\tau_{n-1}\dots\tau_1\in L(G^{\Upsilon_{\mathbb{N}}}_{obs})$, we find $f_{R,obs}(X,\iota_R)=\{x_n\in X|\exists x_0, x_1, x_2, \dots, x_n\in X,(x_n,x_{n-1})\in M(\tau_n),(x_{n-1},x_{n-2})\in M(\tau_{n-1}),\dots,\\(x_{1},x_{0})\in M(\tau_{1})\}$. 
	Then, $f_{R,obs}(X,\iota_R)\cap X_0=\{x_n\in X_0|\exists x_0, x_1, x_2, \dots, x_{n-1}\in X,x_n\in X_0,(x_n,x_{n-1})\in M(\tau_n),(x_{n-1},x_{n-2})\in M(\tau_{n-1}),\dots,(x_{1},x_{0})\in M(\tau_{1})\}$. 
	Combining Proposition \ref{pro-current-string}.1) with the definition of $\mathcal{E}^{i}(\iota,X_0)$, we have $f_{R,obs}(X,\iota_R)\cap X_0=\mathcal{E}^{i}(\iota,X_0)$, which completes the proof.
\end{proof}


\begin{theorem}
	Given a system $G=(X,\Sigma,\delta,X_0)$ under the DO-based protocol $\Upsilon_{\mathbb{N}}$ and  a set of secret states $X_S\subseteq X_0$, let $G^{\Upsilon_{\mathbb{N}}}_{R,obs}=Ac(X_{R,obs},T_c, f_{R,obs}, X)$ be its synchronization-reversed observer. System $G$ is DO-ISO w.r.t. $\Upsilon_{\mathbb{N}}$ and $X_S$ if and only if 
\begin{align*}
\forall q\in X_{R,obs}:q\cap X_0\neq\emptyset\Rightarrow (q\cap X_0)\nsubseteq X_S.
\end{align*}
\end{theorem}
\begin{proof}
According to Theorem \ref{theorem-ini-1}, the system is DO-ISO if and only if its initial state estimates should never be a subset of $X_S$. 
We proceed by contrapositive.

($\Rightarrow$)
Assume that there exists $q\in X_{R,obs}$ such that $q\cap X_0\allowbreak\neq\emptyset$ and $ (q\cap X_0)\subseteq X_S$.
By the definition of $G^{\Upsilon_{\mathbb{N}}}_{R,obs}$ and Lemma \ref{lemma-reverse}, there exists $s\in L(G)$ such that $\mathcal{E}^{i}(\iota,X_0)=q\cap X_0$ where $\iota=P_{\Upsilon_{\mathbb{N}}}(s)$ and $q=f_{R,obs}(X,\iota_R)$.
Therefore, $\mathcal{E}^{i}(\iota,X_0)\subseteq X_S$, indicating that the system is not DO-ISO.

($\Leftarrow$)
If the system is not DO-ISO, there exists $s\in L(G)$ such that $\mathcal{E}^{i}(\iota,X_0)\subseteq X_S$ where $\iota=P_{\Upsilon_{\mathbb{N}}}(s)$.
Due to Lemma \ref{lemma-reverse}, we have $\mathcal{E}^{i}(\iota,X_0)=f_{R,obs}(X,\iota_R)\cap X_0$, suggesting that there exists $q=f_{R,obs}(X,\iota_R)\in X_{R,obs}$ such that $q\cap X_0\neq\emptyset$ and $ (q\cap X_0)\subseteq X_S$.
This completes the proof.

%	For any $q\in X_{R,obs}$, there exists $\iota_R\in L(G^{\Upsilon_{\mathbb{N}}}_{R,obs})$ such that $q=f_{R,obs}(X,\iota_R)$. 
%	By using the extended definition of $f_{R,obs}$ (as in the proof of Lemma~\ref{lemma-reverse}), $\iota\in L(G,q)$ holds.
%	Therefore, for any $q'\in X_{R,obs}$ such that $q'\cap X_0\neq\emptyset$, there exist $\iota_R'$ and $\iota'$ such that $q'=f_{R,obs}(X,\iota_R')$ and $\iota'\in L(G,X_0)$.
%	Combining this with Lemma~\ref{lemma-reverse}, for any $\iota=P_{\Upsilon_{\mathbb{N}}}(s)\in L(G^{\Upsilon_{\mathbb{N}}}_{obs})$, the system's set of initial state estimates is $f_{R,obs}(X,\iota_R)\cap X_0$, suggesting that the system's initial state estimates will never be a subset of $X_S$ if and only if  for all $q\in X_{R,obs}$, we have ``$q\cap X_0\neq\emptyset\Rightarrow (q\cap X_0)\nsubseteq X_S$'', thereby completing the proof.
\end{proof}


\begin{remark}
	Both approaches analyzed above are based on the structure of the system's CSS structure.
	Assuming that the CSS structure is computed beforehand, the computational complexity of constructing a synchronization-reversed observer is equivalent to that of computing a DO-observer, i.e., $O(\Delta_c\cdot 2^{|X|})$.
	The approach utilizing the DO-based initial-state estimator is of high complexity, as it stores more information, namely, the sets of pairs of possible initial and current states. 
	When verifying DO-ISE, one can choose the appropriate method based on the underlying requirements.
\end{remark}

\begin{figure}[htbp]
	\centering
	\includegraphics[scale=0.7]{synchronization-reversed_observer.pdf}
	\caption{The synchronization-reversed observer $G^{\Upsilon_{\mathbb{N}}}_{R,obs}$ for $G$. The transitions labeled as ``$\tau$'' correspond to the labels shown in Fig.~\ref{fig-ASS-1}.}
	\label{fig-sro}
\end{figure}

\begin{example}
	Consider again the system $G$ in Fig.~\ref{fig-example-1}, where $X_0=X$ and $X_S=\{q_0\}$.
	The feasible CSS structure of $G$ is shown in Fig.~\ref{fig-ASS-1}.
	Based on Definition \ref{def-sro}, the synchronization-reversed observer is constructed and shown in Fig.~\ref{fig-ASS-1}.
    For example, since $(q_4,q_2)\in M(\tau_3)$, there exists a transition relation $(\{q_2\},\tau_3,\{q_4\})\in f_{R,obs}$ in $G^{\Upsilon_{\mathbb{N}}}_{R,obs}$.
    By taking $\iota_R=\tau_6\tau_1$, after observing the sequence of CSI-states $\iota=\tau_1\tau_6$, the outside observer will know for sure that the system initial state was $q_0$.
    Therefore, this system is not DO-ISO w.r.t. $\Upsilon_{\mathbb{N}}$ and $X_S$.\hfill\rule{1ex}{1ex}
\end{example}

\subsection{Verification of DO-based Current-State-at-Synchronization Opacity}
The notion of centralized current-state opacity requires that the system should possibly reside in a non-secret state at any given instant.
However, this property cannot be directly applied to a system under the DO-based protocol, where several events, unknown to the coordinator (and the outside observer) until the next synchronization, will be executed in the system after each synchronization.
To address this, a new privacy requirement is introduced, focusing solely on the timing immediately after each synchronization.
Since the synchronization instant represents the most clearly defined and predictable moment in the system's state, it is crucial to ensure that the system's state information at these instants remains confidential.
% and the observable events executed before the next synchronization are not considered in this context.
As noted in Remark~\ref{remark-aftersync}, addressing the privacy/security considerations between synchronizations is left for future research due to the complexity of this topic, which cannot be fully covered here within the allowed space.
	\begin{definition}
		(DO-CSSO) A system $G=(X,\Sigma,\delta,X_0)$ under DO-based protocol $\Upsilon_{\mathbb{N}}$ and a set of secret states $X_S\subseteq X$ is said to be \textit{DO-based current-state-at-synchronization opaque}  (w.r.t. $\Upsilon_{\mathbb{N}}$ and $X_S$) if the intruder can never know with certainty if the system resides in a secret state immediately following a synchronization, i.e.,
			\begin{multline*}
				\forall x_0\in X_0, \forall s\in L(G,x_0): \operatorname{UR}(\delta(x_0,\tilde{s}))\cap X_S\neq\emptyset
				\Rightarrow\\
				\exists x_0'\in X_0, \exists s'\in L(G,x_0'): P_{\Upsilon_{\mathbb{N}}}(s)=P_{\Upsilon_{\mathbb{N}}}(s')\land\\
				\operatorname{UR}(\delta(x_0',\tilde{s'}))\cap (X\backslash X_S)\neq\emptyset.
			\end{multline*}
	\end{definition}
	%The key distinction between standard current-state opacity and DO-CSSO lies in the specific instants at which privacy requirements are described and needed.
	In DO-CSSO, for any string $s\in L(G,x_0)$, only the string $\tilde{s}$ is used to describe the privacy requirements.
	If the system may reside in a secret state (i.e., $\operatorname{UR}(\delta(x_0,\tilde{s}))\cap X_S\neq\emptyset$), there should be another string $s'\in L(G,x_0')$ that has the same DO-projection as $s$ (i.e., $P_{\Upsilon_{\mathbb{N}}}(s)=P_{\Upsilon_{\mathbb{N}}}(s')$) and leads the system to a non-secret state (i.e., $\operatorname{UR}(\delta(x_0',\tilde{s'}))\cap (X\backslash X_S)\neq\emptyset$).
	In this situation, the DO-observer can be used to verify this property by simply checking if there are non-secret states within each state of the DO-observer.
\begin{theorem}\label{theorem-do-CSSO}
	Given a system $G=(X,\Sigma,\delta,X_0)$ under DO-based protocol $\Upsilon_{\mathbb{N}}$ and  a set of secret states $X_S\subseteq X$, let $G^{\Upsilon_{\mathbb{N}}}_{obs}=Ac(X_{obs},T_c,f_{obs},X_{obs,0})$ be its DO-observer. System $G$ is DO-CSSO w.r.t. $\Upsilon_{\mathbb{N}}$ and $X_S$ if and only if for any $q\in X_{obs}$, $q\nsubseteq X_S$.
\end{theorem}
\begin{proof}
According to the definition of DO-CSE, it is trivial to deduce that system $G$ is DO-CSSO if for all $s\in L(G)$, $\mathcal{E}^c(P_{\Upsilon_{\mathbb{N}}}(s),X_0)\cap (X\backslash X_S)\neq\emptyset$.
We proceed with the contrapositive.

($\Rightarrow$) Assume that there exists $q\in X_{obs}$, $q\subseteq X_S$.
According to the definition of DO-observer and Proposition~\ref{pro-obs}, there exists $s\in L(G)$, $q=\mathcal{E}^c(P_{\Upsilon_{\mathbb{N}}}(s),X_0)$.
Thus, it holds $\mathcal{E}^c(P_{\Upsilon_{\mathbb{N}}}(s),X_0)\cap (X\backslash X_S)=\emptyset$.
The system is not DO-CSSO.

($\Leftarrow$) If the system is not DO-CSSO, there exists $s\in L(G)$, $\mathcal{E}^c(P_{\Upsilon_{\mathbb{N}}}(s),X_0)\cap (X\backslash X_S)=\emptyset$.
According to the definition of DO-observer and Proposition~\ref{pro-obs}, there exists $q\in X_{obs}$, $q\subseteq X_S$.
This completes the proof.
\end{proof}
\begin{example}
Let us consider the system in Example~\ref{example-2} where the set of initial states is $\{q_0,q_1\}$ and the set of secret states is $X_S=\{q_2\}$.
Its DO-observer is shown in Fig.~\ref{DO-based observer}.
According to Theorem~\ref{theorem-do-CSSO}, system $G$ is not DO-CSSO w.r.t. $\Upsilon_{\mathbb{N}}$ and $X_S=\{q_2\}$.
If the sequence of CSI-states $\tau_5\tau_3$ is sent to the coordinator, the outside observer can determine that the system is in secret state $q_2$.
\end{example}

	

	
\subsection{Discussion of Diagnosability Under the DO-based Protocol}
It is shown in \cite{Hadjicostis2020} that the standard fault diagnosis problem of a DES can be reduced to a state isolation/estimation problem.
By properly refining the state space of the original system to incorporate the occurrence of fault events, similar to the procedure in \cite{Sampath1995}, the refined system is constructed.
The states in the refined system  are classified into two categories: those augmented with the symbol ``$F$'', indicating that the system has encountered one or more fault events before reaching these states, and those not augmented with ``$F$'', implying the system has not executed any fault events up to that point.
%The states in the diagnoser are classified into two categories: those augmented with the symbol 'F,' indicating that the system has encountered one or more fault events before reaching these states, and those not augmented with 'F,' implying the system has not executed any fault events up to that point.
%those that have been reached by previously executing one or more fault events and those that have been reached without previously executing fault events.
Consequently, this problem is reduced to finding so-called $F$-indeterminate cycles \cite{Sampath1995} in the observer of the refined system.
Furthermore, the results in \cite{YooLafortune2002, JiangHuang2001} show that the verification of diagnosability can be tested in polynomial time by constructing a parallel composition structure called a verifier.


The verification of diagnosability under the DO-based protocol has been preliminarily done in \cite{Hadjicostis2020}, based on a partial-order-based estimation protocol introduced accordingly in \cite{Hadjicostis2020}.
However, with the notion of CSS structure introduced in this paper, we can use the method and technique in the standard fault diagnosis problem to more effectively
analyze and verify the diagnosability under the DO-based protocol.
As seen, although the system behavior is observed by several OSs in a decentralized observation setting, unlike other decentralized information processing schemes where the coordinator receives local state estimates or local decisions, the coordinator in the DO-based protocol receives partially ordered system behaviors (CSI-states).
Furthermore, based on Proposition~\ref{pro-current-string}, the CSS structure reflects the evolution of all system states upon any possible CSI-state, where each CSI-state is related to some system behaviors.
As a result, the coordinator obtains the state estimates of the system based on synchronization information, where a \textit{monolithic} observer can be constructed.
Although there exist observable events which cannot be received by the coordinator until the next synchronization, the state estimates of the coordinator are updated once these events are received.
In this sense, the DO-based protocol exhibits both centralized and decentralized characteristics.
Therefore, the method to verify diagnosability under the DO-based protocol is analogous to the centralized approach.
Due to the redundancy of the methods and concepts, and to preserve the focus established in the previous sections, we refrain from presenting the full formal details of this approach in this paper.



\section{Conclusions}
In this paper, we present a relatively efficient approach to the problem of state estimation for a DES under a decentralized observation architecture where observation sites send to the coordinator the sequences of observations they recorded.
We also discuss how to verify state-isolation DO-based properties using this approach.
To this end, we first provide the formal description of the DO-based protocol.
Then, we interpret a given DO-based strategy into a structure called \textit{Complete Synchronizing Sequence structure} (CSS structure), which includes the relations between states under any SI-state.
Based on the CSS structure, the DO-observer and DO-based initial state estimator are defined and constructed, with the processes simply involving the union of states without exploring possible system behavior at each synchronization step.
It is shown that, using the CSS structure, DO-based initial-state opacity and DO-based current-state-at-synchronization opacity can be verified similarly to the standard centralized observation architecture.
%\textcolor{red}{\sout{Lastly, we introduce and verify a new notion called DO-based current-synchronization opacity, which aims to capture privacy requirements during the synchronization process.}}

In the future, we plan to utilize the CSS structure to develop methods for detecting fault events in the presence of local and global errors, i.e., when the synchronization information received by the coordinator is tampered.
It is also worth exploring the privacy/security considerations between synchronizations.



%extend the application of CSS structure to the context of supervisory control under the DO-based protocol.





\begin{thebibliography}{99}
	
	\bibitem{Ramadge1986} P. Ramadge, ``Observability of discrete event systems,'' in \textit{Proceedings of 25th IEEE Conference on Decision and Control (CDC)}, 1986, pp.~1108--1112.
	
	\bibitem{Hadjicostis2020} C. N. Hadjicostis, \textit{Estimation and Inference in Discrete Event Systems}. Springer, 2020.
	
	\bibitem{WangLafortuneLin2007} W. L. Wang, S. Lafortune, and F. Lin, ``An algorithm for calculating indistinguishable states and clusters in finite-state automata with partially observable transitions,'' \textit{Systems \& Control Letters}, vol.~56, no.~9--10, pp.~656–661, 2007.
	
	\bibitem{LinWangHanShe2020} F. Lin, W. L. Wang, L. T. Han, and B. She, ``State estimation of multichannel networked discrete event systems,'' \textit{IEEE Transactions on Control of Network Systems}, vol.~7, no.~1, pp.~53--63, 2020.
	
	\bibitem{HanWangLiChenChen2023} X. G. Han, J. L. Wang, Z. W. Li, X. Y. Chen, and Z. Q. Chen, ``Revisiting state estimation and weak detectability of discrete-event systems,'' \textit{IEEE Transactions on Automation Science and Engineering}, vol.~20, no.~1, pp.~662--674, 2023.
	
	
	\bibitem{DeboukLafortuneTeneketzis2000} R. Debouk, S. Lafortune, and D. Teneketzis, ``Coordinated decentralized protocols for failure diagnosis of discrete event systems,'' \textit{Discrete Event Dynamic Systems}, vol.~10, no.~1, pp.~33--86, 2000.
	
	\bibitem{Sampath1995} M. Sampath, R. Sengupta, S. Lafortune, K. Sinnamohideen, and D. Teneketzis, ``Diagnosability of discrete-event systems,'' \textit{IEEE Transactions on Automatic Control}, vol.~40, no.~9, pp.~1555--1575, 1995.
	
	\bibitem{YooLafortune2002} T.-S. Yoo and S. Lafortune, ``Polynomial-time verification of diagnosability of partially observed discrete-event systems,'' \textit{IEEE Transactions on Automatic Control}, vol.~47, no.~9, pp.~1491--1495, 2002.
	
	\bibitem{JiangHuang2001} S. B. Jiang, Z. D. Huang, V. Chandra, and R. Kumar, ``A polynomial algorithm for testing diagnosability of discrete-event systems,'' \textit{IEEE Transactions on Automatic Control}, vol.~46, no.~8, pp. 1318--1321, 2001.
	
	
	
	\bibitem{ShuLinYing2007}  S. L. Shu, F. Lin, and H. Ying, ``Detectability of discrete event systems,'' \textit{IEEE Transactions on Automatic Control}, vol.~52, no.~12, pp.~2356--2359, 2007.
	
	\bibitem{YinLiWang2018} X. Yin, Z. J. Li, and W. L. Wang, ``Trajectory detectability of discrete-event systems,'' \textit{Systems \& Control Letters}, vol.~119, pp.~101–107, 2018.
	
	\bibitem{SabooriHadjicostis2007} A. Saboori and C. N. Hadjicostis, ``Notions of security and opacity in discrete event systems,'' in \textit{Proceedings of 46th IEEE Conference on Decision and Control (CDC)}, 2007, pp.~5056--5061.
	
	\bibitem{JacobLesageFaure2016} R. Jacob, J. Lesage, and J. Faure, ``Overview of discrete event systems opacity: Models, validation, and quantification,'' \textit{Annual Reviews in Control}, vol.~41, pp.~135--146, 2016.
	
	
	\bibitem{QiuKumar2006} W. B. Qiu and R. Kumar, ``Decentralized failure diagnosis of discrete event systems,'' \textit{IEEE Transactions on Systems, Man, Cybernetics-Part A: Systems and Humans}, vol.~36, no.~2, pp.~384--395, 2006.
	
	
	\bibitem{NunesMoreiraAlvesCarvalhoBasilio2018} C. E. Nunes, M. V. Moreira, M. V. Alves, L. K. Carvalho, and J. C. Basilio, ``Codiagnosability of networked discrete event systems subject to communication delays and intermittent loss of observation,'' \textit{Discrete Event Dynamic Systems}, vol.~28, pp.~215--246, 2018.
	%In Nunes et al. (2018), the decentralized diagnosis of networked DES subject to communication delays and intermittent loss of observations is proposed.In this paper, we address the problem of failure diagnosis of networked DES with the decentralized diagnosis structure proposed in Protocol 3 of Debouk et al. (2000), i.e.: (i) there is no communication between local diagnosers; (ii) each local diagnoser infers the occurrence of the failure event based on its own observations; (iii) the failure event is diag- nosed when at least one of the local diagnosers identifies its occurrence. We also consider that the observation of event occurrences is distributed in the plant, i.e., the plant has sev- eral measurement sites, and each site has exclusive communication channels to send the information regarding event occurrences to local networked diagnosers, as shown in Fig. 1. In addition, we assume the existence of communication delays between measurement sites and local networked diagnosers, which may result in an observation order different from the actual order of event occurrences in the plant.
	
	
	\bibitem{TakaiUshio2012} S. Takai and T. Ushio, ``Verification of codiagnosability for discrete event systems modeled by Mealy automata with nondeterministic output functions,'' \textit{IEEE Transactions on Automatic Control}, vol.~57, no.~3, pp.~798--804, 2012.
	
	\bibitem{RanSuGiuaSeatzu2018} N. Ran, H. Y. Su, A. Giua, and C. Seatzu, ``Codiagnosability analysis of bounded Petri nets,'' \textit{IEEE Transactions on Automatic Control}, vol.~63, no.~4, pp.~1192--1199, 2018.
	
	
	\bibitem{TomolaCabralCarvalhoMoreira2017} J. H. A. Tomola, F. G. Cabral, L. K. Carvalho, and M. V. Moreira, ``Robust disjunctive-codiagnosability of discrete-event systems against permanent loss of observations,'' \textit{IEEE Transactions on Automatic Control}, vol.~62, no.~11, pp.~5808--5815, 2017.
	
	\bibitem{TaKaiKumar2017} S. Takai and R. Kumar, ``A generalized framework for inference-based diagnosis of discrete event systems capturing both disjunctive and conjunctive decision-making,'' \textit{IEEE Transactions on Automatic Control}, vol.~62, no.~6, pp.~2778--2793, 2017.
	
	
	\bibitem{KeroglouHadjicostis2018} C. Keroglou and C. N. Hadjicostis, ``Distributed fault diagnosis in discrete event systems via set intersection refinements,'' \textit{IEEE Transactions on Automatic Control}, vol.~63, no.~10, pp.~3601--3607, 2018.
	
	\bibitem{OliveiraCabralMoreira2022} V. Oliveira, F. G. Cabral, and M. V. Moreira, ``$K$-loss robust codiagnosability of discrete-event systems,'' \textit{Automatica}, vol.~140, p. 110222, 2022.
	
	\bibitem{LiHadjicostisWu2021} Y. T. Li, C. N. Hadjicostis, and N. Q. Wu, ``Error- and tamper-tolerant decentralized diagnosability of discrete event systems under cost constraints,'' in \textit{Proceedings of European Control Conference (ECC)}, pp.~42--47, 2021. 
	
	\bibitem{ZaytoonLafortune2013} J. Zaytoon and S. Lafortune, ``Overview of fault diagnosis methods for discrete event systems,'' \textit{Annual Reviews in Control}, vol.~37, no.~2, pp.~308–320, 2013.
	
	\bibitem{LafortuneLinHadjicostis2018} S. Lafortune, F. Lin, and C. N. Hadjicostis, ``On the history of diagnosability and opacity in discrete event systems,'' \textit{Annual Reviews in Control}, vol.~45, pp.~257--266, 2018.
	
	
	\bibitem{BasilioHadjicostisSu2021} J. Basilio, C. N. Hadjicostis, and R. Su, ``Analysis and control for resilience of discrete event systems: Fault diagnosis, opacity and cyber security,'' \textit{Foundations and Trends in Systems and Control}, vol.~8, no.~4, pp.~285--443, 2021.
	
	
	\bibitem{Mazare2004} L. Mazar\'e, ``Using unification for opacity properties,'' in \textit{Proceedings of the Workshop on Issues in the Theory of Security}, 2004, pp.~165--176.
	
	\bibitem{BryansKoutnyMazarRyan2008} J. W. Bryans, M. Koutny, L. Mazar\'e, and P. Y. Ryan, ``Opacity generalised to transition systems,'' \textit{International Journal of Information Security}, vol.~7, no.~6, pp.~421–435, 2008.
	
	
	\bibitem{SabooriHadjicostis2008ini} A. Saboori and C. N. Hadjicostis, ``Verification of initial-state opacity in security applications of DES,'' in \textit{Proceedings of 9th International Workshop on Discrete Event Systems (WODES)}, 2008, pp.~328--333.
	
	\bibitem{SabooriHadjicostis2011} A. Saboori and C. N. Hadjicostis, ``Verification of $K$-step opacity and analysis of its complexity,'' \textit{IEEE Transactions on Automation Science and Engineering}, vol.~8, no. 3, pp.~549--559, 2011.
	
	\bibitem{WuLafortune2013} Y. Wu and S. Lafortune, ``Comparative analysis of related notions of opacity in centralized and coordinated architectures,'' \textit{Discrete Event Dynamic Systems}, vol.~23, no.~3, pp.~307--339, 2013.
	
	\bibitem{YinLafortune2017} X. Yin and S. Lafortune, ``A new approach for the verification of infinite-step and $K$-step opacity using two-way observers,'' \textit{Automatica}, vol.~80, pp.~162--171, 2017.
	
	
	
	\bibitem{YongLiSeatzuGiua2017} Y. Tong, Z. W. Li, C. Seatzu, and A. Giua, ``Verification of state-based opacity using Petri nets,'' \textit{IEEE Transactions on Automatic Control}, vol.~62, no.~6, pp.~2823--2837, 2017.
	
	
	\bibitem{TongLanSeatzu2022} Y. Tong, H. Lan, and C. Seatzu, ``Verification of $K$-step and infinite- step opacity of bounded labeled Petri nets,'' \textit{Automatica}, vol.~140, p.~110221, 2022.
	
	
	\bibitem{CongFantiManginiL2018} X. Y. Cong, M. P. Fanti, A. M. Mangini, and Z. W. Li, ``On-line verification of current-state opacity by Petri nets and integer linear programming,'' \textit{Automatica}, vol.~94, pp.~205--213, 2018.
	
	\bibitem{CongFantiManginiLi2019} X. Y. Cong, M. P. Fanti, A. M. Mangini, and Z. W. Li, ``On-line verification of initial-state opacity by Petri nets and integer linear programming,'' \textit{ISA Transactions}, vol.~93, pp.~108--114, 2019.
	
	
	\bibitem{DongWuLi2024} Y. F. Dong, N. Q. Wu, and Z. W. Li, ``State-based opacity verification of networked discrete event systems using labeled Petri nets,'' \textit{IEEE/CAA Journal of Automatica Sinica}, vol.~11, no.~5, pp.~1274--1291, 2024.
	
	\bibitem{YangDengQiuJiang2021} J. Yang, W. Deng, D. Qiu, and C. Jiang, “Opacity of networked discrete event systems,” \textit{Information Sciences}, vol.~543, pp.~328–344, 2021.
	
	
	\bibitem{FalconeMarchand2015} Y. Falcone and H. Marchand, ``Enforcement and validation (at runtime) of various notions of opacity,” \textit{Discrete Event Dynamic Systems}, vol.~25, no.~4, pp.~531–570, 2015.
	
	\bibitem{MaYinLi2021} Z. Y. Ma, X. Yin, and Z. W. Li, ``Verification and enforcement of strong infinite- and $k$-step opacity  using state recognizers,'' \textit{Automatica}, vol.~133, p. 109838, 2021.
	
	\bibitem{HanZhangZhangLiChen2023} X. G. Han, K. Z. Zhang, J. H. Zhang, Z. W. Li, and Z. Q. Chen, ``Strong current-state and initial-state opacity of discrete-event systems'', \textit{Automatica}, vol.~148, p. 110756, 2023.
	
	\bibitem{PaoliLin2012} A. Paoli and F. Lin, ``Decentralized opacity of discrete event systems,'' in \textit{Proceedings of 2012 American Control Conference (ACC)}, pp.~6083--6088, 2012.
	
	\bibitem{ZhuLiWu2022} G. H. Zhu, Z. W. Li, and N. Q. Wu, ``Online verification of $K$-step opacity by Petri nets in centralized and decentralized structures,'' \textit{Automatica}, vol.~145, p. 110528, 2022.
	
	
	
	\bibitem{BasilioLafortune2009} J. C. Basilio and S. Lafortune, ``Robust codiagnosability of discrete event systems,'' in \textit{Proceedings of American Control Conference (ACC)}, 2009, pp.~2202--2209.
	
	\bibitem{VianaAlvesBasilio2022} G. S. Viana, M. V. S. Alves, and J. C. Basilio, ``Codiagnosability of networked discrete event systems with timing structure,'' \textit{IEEE Transactions on Automatic Control}, vol.~67, no.~8, pp.~3933--3948, 2022.
	
	
	%\bibitem{RamadgeWonham1987} P. J. Ramadge and W. M. Wonham, ``Supervisory control of a class of discrete event processes,'' \textit{SIAM Journal on Control and Optimization}, vol.~25, no.~1, pp.~206--230, 1987.
	
	%\bibitem{CieslakDesclauxFawazVaraiya1988} R. Cieslak, C. Desclaux, A. S. Fawaz and P. Varaiya, ``Supervisory control of discrete-event processes with partial observations,'' \textit{IEEE Transactions on Automatic Control}, vol.~33, no.~3, pp.~249--260, 1988.
	
	
	
	
	
	%\bibitem{Hadjicostis2016Partial} C. N. Hadjicostis and C. Seatzu, ``Decentralized state estimation in discrete event systems under partially ordered observation sequences,'' in \textit{Proceedings of 13th International Workshop on Discrete Event Systems (WODES)}, 2016, pp.~367--372.
	
	\bibitem{CassandrasLafortune2008} C. G. Cassandras and S. Lafortune, \textit{Introduction to Discrete Event Systems}. Springer, 2008.
	
	
	\bibitem{SunHadjicostisLi2023} D. J. Sun, C. N. Hadjicostis, and Z. W. Li, ``Decentralized state estimation via breadth-first search through partially ordered observation sequences,'' in \textit{Proceedings of 62nd IEEE Conference on Decision and Control (CDC)}, 2023, pp.~6917--6922.
	
	
\end{thebibliography}





\end{document}
