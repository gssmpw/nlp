\section{RELATED WORKS}
While CBFs have demonstrated their robustness and scalability \cite{Li_2021, Lopez_2021, Lindemann_2024} as a safety method, there is limited literature on their application in Unmanned Aerial Vehicles (UAVs) and to the best of available knowledge, almost none specifically for TUAVs. The application of CBFs in non-tethered UAVs offers valuable insights for managing stability, safety, and constraint adherence in tethered drone systems. In \cite{Singletary_2022}, an implicit CBF for high-speed safety provides a foundation for adapting spatial and safety constraints to confined environments. Studies by \cite{Panja_2023} and \cite{Tayal_2024} showcase collision avoidance techniques that could aid in managing tether proximity risks. \cite{Zheng_2023} shows an adaptive CBF for path-following under disturbances and \cite{Wang_2018, Wang_2023, Liu_2024} illustrates adaptive safe stabilization and event-triggered control approaches that further address environmental constraints and efficient control adaptations, vital for tethered UAVs. 

Other safety filters such as Model Predictive Control (MPC) \cite{Korsarnovsky_2020, Singh_2001, Lindqvist_2020}, Explicit Reference Governor (ERG) \cite{Hermand_2018}, reachability analysis \cite{Ankit_2024, Zhou_2015, Ding_2012} have been applied to UAVs. However, they lack the real-time adaptability and formal safety guarantees provided by Control Barrier Functions (CBFs). CBFs have been shown to be more computationally efficient \cite{Li_2021} and less conservative in enforcing safety constraints \cite{Tayal_2024}. Most recent research on TUAV safety filters is largely based on Model Predictive Control (MPC) \cite{Bolognini_2022, Valerio_2022}; however, MPC can be computationally expensive and may not always guarantee real-time performance, particularly for systems with fast dynamics or stringent computational constraints such as the TUAV. CBFs offer a flexible and adaptable framework that allows for dynamic adjustment of safety boundaries as conditions change, which is crucial in TUAV systems that must account for both tether length and external environmental factors. 

Motivated by the societal need for safety in TUAV operations, this paper explores the integration of Control Barrier Function Quadratic Programs into the control architecture of tethered drones to enhance robot safety and operational efficiency. The specific contributions of this paper are as follows: the development of a nonlinear backstepping control approach for the precise control of tethered unmanned aerial vehicles (TUAVs) in three-dimensional space; the explicit integration of backstepping control with Control Barrier Functions (CBFs) to enforce safety constraints dynamically; the utilization of Quadratic Programming (QP) to ensure constraint satisfaction and optimize control inputs in real time; and the establishment of a flexible control framework that can be extended to other tethered systems or safety-critical robotic applications.