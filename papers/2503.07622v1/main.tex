\documentclass[conference]{IEEEtran}
\IEEEoverridecommandlockouts
% The preceding line is only needed to identify funding in the first footnote. If that is unneeded, please comment it out.
\usepackage{cite}
\usepackage{amsmath,amssymb,amsfonts}
\usepackage{algorithmic}
\usepackage{graphicx}
\usepackage{textcomp}
\usepackage{xcolor}
\usepackage{comment}
\usepackage{hyperref}
\usepackage{threeparttable}
\usepackage[font=footnotesize]{caption}
\usepackage{afterpage}
\usepackage{float}
\usepackage{subcaption} % Modern replacement for subfigure

\usepackage{balance}
%\usepackage[moderate,tracking=normal]{savetrees}

\def\BibTeX{{\rm B\kern-.05em{\sc i\kern-.025em b}\kern-.08em
    T\kern-.1667em\lower.7ex\hbox{E}\kern-.125emX}}
\begin{document}

%\title{Investigating Human Gaze Dynamics in Response to Robotic Failure in Human-Robot Collaboration
%}
\title{Real-Time Detection of Robot Failures Using Gaze Dynamics in Collaborative Tasks
}


    
\author{\IEEEauthorblockN{Ramtin Tabatabaei}
\IEEEauthorblockA{\textit{The University of Melbourne} \\
Melbourne, Australia \\
stabatabaeim@student.unimelb.edu.au}
\and

\IEEEauthorblockN{Vassilis Kostakos}
\IEEEauthorblockA{\textit{The University of Melbourne} \\
Melbourne, Australia \\
vassilis.kostakos@unimelb.edu.au}
\and

\IEEEauthorblockN{Wafa Johal}
\IEEEauthorblockA{\textit{The University of Melbourne} \\
Melbourne, Australia \\
wafa.johal@unimelb.edu.au}}



% \author{\IEEEauthorblockN{Anonymous Author(s)}
% }



\maketitle

\begin{abstract}

Detecting robot failures during collaborative tasks is crucial for maintaining trust in human-robot interactions. This study investigates user gaze behaviour as an indicator of robot failures, utilising machine learning models to distinguish between non-failure and two types of failures: executional and decisional. Eye-tracking data were collected from 26 participants collaborating with a robot on Tangram puzzle-solving tasks. Gaze metrics, such as average gaze shift rates and the probability of gazing at specific areas of interest, were used to train machine learning classifiers, including Random Forest, AdaBoost, XGBoost, SVM, and CatBoost.
The results show that Random Forest achieved 90\% accuracy for detecting executional failures and 80\% for decisional failures using the first 5 seconds of failure data. Real-time failure detection was evaluated by segmenting gaze data into intervals of 3, 5, and 10 seconds. These findings highlight the potential of gaze dynamics for real-time error detection in human-robot collaboration.





%Results show that Random Forest achieved ~90% accuracy for execution failures and ~80% for decisional failures using the first 5 seconds of failure data. Real-time detection was evaluated by segmenting gaze data into intervals of 3, 5, and 10 seconds. The findings highlight the potential of gaze dynamics for real-time error detection in human-robot collaboration.

\end{abstract}


\begin{IEEEkeywords}
Robot Failures, Gaze Dynamics, Human-Robot Collaboration, Machine Learning Classifiers
\end{IEEEkeywords}


\section{Introduction}

% \textcolor{red}{Still on working}

% \textcolor{red}{add label for each section}


Robot learning relies on diverse and high-quality data to learn complex behaviors \cite{aldaco2024aloha, wang2024dexcap}.
Recent studies highlight that models trained on datasets with greater complexity and variation in the domain tend to generalize more effectively across broader scenarios \cite{mann2020language, radford2021learning, gao2024efficient}.
% However, creating such diverse datasets in the real world presents significant challenges.
% Modifying physical environments and adjusting robot hardware settings require considerable time, effort, and financial resources.
% In contrast, simulation environments offer a flexible and efficient alternative.
% Simulations allow for the creation and modification of digital environments with a wide range of object shapes, weights, materials, lighting, textures, friction coefficients, and so on to incorporate domain randomization,
% which helps improve the robustness of models when deployed in real-world conditions.
% These environments can be easily adjusted and reset, enabling faster iterations and data collection.
% Additionally, simulations provide the ability to consistently reproduce scenarios, which is essential for benchmarking and model evaluation.
% Another advantage of simulations is their flexibility in sensor integration. Sensors such as cameras, LiDARs, and tactile sensors can be added or repositioned without the physical limitations present in real-world setups. Simulations also eliminate the risk of damaging expensive hardware during edge-case experiments, making them an ideal platform for testing rare or dangerous scenarios that are impractical to explore in real life.
By leveraging immersive perspectives and interactions, Extended Reality\footnote{Extended Reality is an umbrella term to refer to Augmented Reality, Mixed Reality, and Virtual Reality \cite{wikipediaExtendedReality}}
(XR)
is a promising candidate for efficient and intuitive large scale data collection \cite{jiang2024comprehensive, arcade}
% With the demand for collecting data, XR provides a promising approach for humans to teach robots by offering users an immersive experience.
in simulation \cite{jiang2024comprehensive, arcade, dexhub-park} and real-world scenarios \cite{openteach, opentelevision}.
However, reusing and reproducing current XR approaches for robot data collection for new settings and scenarios is complicated and requires significant effort.
% are difficult to reuse and reproduce system makes it hard to reuse and reproduce in another data collection pipeline.
This bottleneck arises from three main limitations of current XR data collection and interaction frameworks: \textit{asset limitation}, \textit{simulator limitation}, and \textit{device limitation}.
% \textcolor{red}{ASSIGN THESE CITATION PROPERLY:}
% \textcolor{red}{list them by time order???}
% of collecting data by using XR have three main limitations.
Current approaches suffering from \textit{asset limitation} \cite{arclfd, jiang2024comprehensive, arcade, george2025openvr, vicarios}
% Firstly, recent works \cite{jiang2024comprehensive, arcade, dexhub-park}
can only use predefined robot models and task scenes. Configuring new tasks requires significant effort, since each new object or model must be specifically integrated into the XR application.
% and it takes too much effort to configure new tasks in their systems since they cannot spawn arbitrary models in the XR application.
The vast majority of application are developed for specific simulators or real-world scenarios. This \textit{simulator limitation} \cite{mosbach2022accelerating, lipton2017baxter, dexhub-park, arcade}
% Secondly, existing systems are limited to a single simulation platform or real-world scenarios.
significantly reduces reusability and makes adaptation to new simulation platforms challenging.
Additionally, most current XR frameworks are designed for a specific version of a single XR headset, leading to a \textit{device limitation} 
\cite{lipton2017baxter, armada, openteach, meng2023virtual}.
% and there is no work working on the extendability of transferring to a new headsets as far as we know.
To the best of our knowledge, no existing work has explored the extensibility or transferability of their framework to different headsets.
These limitations hamper reproducibility and broader contributions of XR based data collection and interaction to the research community.
% as each research group typically has its own data collection pipeline.
% In addition to these main limitations, existing XR systems are not well suited for managing multiple robot systems,
% as they are often designed for single-operator use.

In addition to these main limitations, existing XR systems are often designed for single-operator use, prohibiting collaborative data collection.
At the same time, controlling multiple robots at once can be very difficult for a single operator,
making data collection in multi-robot scenarios particularly challenging \cite{orun2019effect}.
Although there are some works using collaborative data collection in the context of tele-operation \cite{tung2021learning, Qin2023AnyTeleopAG},
there is no XR-based data collection system supporting collaborative data collection.
This limitation highlights the need for more advanced XR solutions that can better support multi-robot and multi-user scenarios.
% \textcolor{red}{more papers about collaborative data collection}

To address all of these issues, we propose \textbf{IRIS},
an \textbf{I}mmersive \textbf{R}obot \textbf{I}nteraction \textbf{S}ystem.
This general system supports various simulators, benchmarks and real-world scenarios.
It is easily extensible to new simulators and XR headsets.
IRIS achieves generalization across six dimensions:
% \begin{itemize}
%     \item \textit{Cross-scene} : diverse object models;
%     \item \textit{Cross-embodiment}: diverse robot models;
%     \item \textit{Cross-simulator}: 
%     \item \textit{Cross-reality}: fd
%     \item \textit{Cross-platform}: fd
%     \item \textit{Cross-users}: fd
% \end{itemize}
\textbf{Cross-Scene}, \textbf{Cross-Embodiment}, \textbf{Cross-Simulator}, \textbf{Cross-Reality}, \textbf{Cross-Platform}, and \textbf{Cross-User}.

\textbf{Cross-Scene} and \textbf{Cross-Embodiment} allow the system to handle arbitrary objects and robots in the simulation,
eliminating restrictions about predefined models in XR applications.
IRIS achieves these generalizations by introducing a unified scene specification, representing all objects,
including robots, as data structures with meshes, materials, and textures.
The unified scene specification is transmitted to the XR application to create and visualize an identical scene.
By treating robots as standard objects, the system simplifies XR integration,
allowing researchers to work with various robots without special robot-specific configurations.
\textbf{Cross-Simulator} ensures compatibility with various simulation engines.
IRIS simplifies adaptation by parsing simulated scenes into the unified scene specification, eliminating the need for XR application modifications when switching simulators.
New simulators can be integrated by creating a parser to convert their scenes into the unified format.
This flexibility is demonstrated by IRIS’ support for Mujoco \cite{todorov2012mujoco}, IsaacSim \cite{mittal2023orbit}, CoppeliaSim \cite{coppeliaSim}, and even the recent Genesis \cite{Genesis} simulator.
\textbf{Cross-Reality} enables the system to function seamlessly in both virtual simulations and real-world applications.
IRIS enables real-world data collection through camera-based point cloud visualization.
\textbf{Cross-Platform} allows for compatibility across various XR devices.
Since XR device APIs differ significantly, making a single codebase impractical, IRIS XR application decouples its modules to maximize code reuse.
This application, developed by Unity \cite{unity3dUnityManual}, separates scene visualization and interaction, allowing developers to integrate new headsets by reusing the visualization code and only implementing input handling for hand, head, and motion controller tracking.
IRIS provides an implementation of the XR application in the Unity framework, allowing for a straightforward deployment to any device that supports Unity. 
So far, IRIS was successfully deployed to the Meta Quest 3 and HoloLens 2.
Finally, the \textbf{Cross-User} ability allows multiple users to interact within a shared scene.
IRIS achieves this ability by introducing a protocol to establish the communication between multiple XR headsets and the simulation or real-world scenarios.
Additionally, IRIS leverages spatial anchors to support the alignment of virtual scenes from all deployed XR headsets.
% To make an seamless user experience for robot learning data collection,
% IRIS also tested in three different robot control interface
% Furthermore, to demonstrate the extensibility of our approach, we have implemented a robot-world pipeline for real robot data collection, ensuring that the system can be used in both simulated and real-world environments.
The Immersive Robot Interaction System makes the following contributions\\
\textbf{(1) A unified scene specification} that is compatible with multiple robot simulators. It enables various XR headsets to visualize and interact with simulated objects and robots, providing an immersive experience while ensuring straightforward reusability and reproducibility.\\
\textbf{(2) A collaborative data collection framework} designed for XR environments. The framework facilitates enhanced robot data acquisition.\\
\textbf{(3) A user study} demonstrating that IRIS significantly improves data collection efficiency and intuitiveness compared to the LIBERO baseline.

% \begin{table*}[t]
%     \centering
%     \begin{tabular}{lccccccc}
%         \toprule
%         & \makecell{Physical\\Interaction}
%         & \makecell{XR\\Enabled}
%         & \makecell{Free\\View}
%         & \makecell{Multiple\\Robots}
%         & \makecell{Robot\\Control}
%         % Force Feedback???
%         & \makecell{Soft Object\\Supported}
%         & \makecell{Collaborative\\Data} \\
%         \midrule
%         ARC-LfD \cite{arclfd}                              & Real        & \cmark & \xmark & \xmark & Joint              & \xmark & \xmark \\
%         DART \cite{dexhub-park}                            & Sim         & \cmark & \cmark & \cmark & Cartesian          & \xmark & \xmark \\
%         \citet{jiang2024comprehensive}                     & Sim         & \cmark & \xmark & \xmark & Joint \& Cartesian & \xmark & \xmark \\
%         \citet{mosbach2022accelerating}                    & Sim         & \cmark & \cmark & \xmark & Cartesian          & \xmark & \xmark \\
%         ARCADE \cite{arcade}                               & Real        & \cmark & \cmark & \xmark & Cartesian          & \xmark & \xmark \\
%         Holo-Dex \cite{holodex}                            & Real        & \cmark & \xmark & \cmark & Cartesian          & \cmark & \xmark \\
%         ARMADA \cite{armada}                               & Real        & \cmark & \xmark & \cmark & Cartesian          & \cmark & \xmark \\
%         Open-TeleVision \cite{opentelevision}              & Real        & \cmark & \cmark & \cmark & Cartesian          & \cmark & \xmark \\
%         OPEN TEACH \cite{openteach}                        & Real        & \cmark & \xmark & \cmark & Cartesian          & \cmark & \cmark \\
%         GELLO \cite{wu2023gello}                           & Real        & \xmark & \cmark & \cmark & Joint              & \cmark & \xmark \\
%         DexCap \cite{wang2024dexcap}                       & Real        & \xmark & \cmark & \xmark & Cartesian          & \cmark & \xmark \\
%         AnyTeleop \cite{Qin2023AnyTeleopAG}                & Real        & \xmark & \xmark & \cmark & Cartesian          & \cmark & \cmark \\
%         Vicarios \cite{vicarios}                           & Real        & \cmark & \xmark & \xmark & Cartesian          & \cmark & \xmark \\     
%         Augmented Visual Cues \cite{augmentedvisualcues}   & Real        & \cmark & \cmark & \xmark & Cartesian          & \xmark & \xmark \\ 
%         \citet{wang2024robotic}                            & Real        & \cmark & \cmark & \xmark & Cartesian          & \cmark & \xmark \\
%         Bunny-VisionPro \cite{bunnyvisionpro}              & Real        & \cmark & \cmark & \cmark & Cartesian          & \cmark & \xmark \\
%         IMMERTWIN \cite{immertwin}                         & Real        & \cmark & \cmark & \cmark & Cartesian          & \xmark & \xmark \\
%         \citet{meng2023virtual}                            & Sim \& Real & \cmark & \cmark & \xmark & Cartesian          & \xmark & \xmark \\
%         Shared Control Framework \cite{sharedctlframework} & Real        & \cmark & \cmark & \cmark & Cartesian          & \xmark & \xmark \\
%         OpenVR \cite{openvr}                               & Real        & \cmark & \cmark & \xmark & Cartesian          & \xmark & \xmark \\
%         \citet{digitaltwinmr}                              & Real        & \cmark & \cmark & \xmark & Cartesian          & \cmark & \xmark \\
        
%         \midrule
%         \textbf{Ours} & Sim \& Real & \cmark & \cmark & \cmark & Joint \& Cartesian  & \cmark & \cmark \\
%         \bottomrule
%     \end{tabular}
%     \caption{This is a cross-column table with automatic line breaking.}
%     \label{tab:cross-column}
% \end{table*}

% \begin{table*}[t]
%     \centering
%     \begin{tabular}{lccccccc}
%         \toprule
%         & \makecell{Cross-Embodiment}
%         & \makecell{Cross-Scene}
%         & \makecell{Cross-Simulator}
%         & \makecell{Cross-Reality}
%         & \makecell{Cross-Platform}
%         & \makecell{Cross-User} \\
%         \midrule
%         ARC-LfD \cite{arclfd}                              & \xmark & \xmark & \xmark & \xmark & \xmark & \xmark \\
%         DART \cite{dexhub-park}                            & \cmark & \cmark & \xmark & \xmark & \xmark & \xmark \\
%         \citet{jiang2024comprehensive}                     & \xmark & \cmark & \xmark & \xmark & \xmark & \xmark \\
%         \citet{mosbach2022accelerating}                    & \xmark & \cmark & \xmark & \xmark & \xmark & \xmark \\
%         ARCADE \cite{arcade}                               & \xmark & \xmark & \xmark & \xmark & \xmark & \xmark \\
%         Holo-Dex \cite{holodex}                            & \cmark & \xmark & \xmark & \xmark & \xmark & \xmark \\
%         ARMADA \cite{armada}                               & \cmark & \xmark & \xmark & \xmark & \xmark & \xmark \\
%         Open-TeleVision \cite{opentelevision}              & \cmark & \xmark & \xmark & \xmark & \cmark & \xmark \\
%         OPEN TEACH \cite{openteach}                        & \cmark & \xmark & \xmark & \xmark & \xmark & \cmark \\
%         GELLO \cite{wu2023gello}                           & \cmark & \xmark & \xmark & \xmark & \xmark & \xmark \\
%         DexCap \cite{wang2024dexcap}                       & \xmark & \xmark & \xmark & \xmark & \xmark & \xmark \\
%         AnyTeleop \cite{Qin2023AnyTeleopAG}                & \cmark & \cmark & \cmark & \cmark & \xmark & \cmark \\
%         Vicarios \cite{vicarios}                           & \xmark & \xmark & \xmark & \xmark & \xmark & \xmark \\     
%         Augmented Visual Cues \cite{augmentedvisualcues}   & \xmark & \xmark & \xmark & \xmark & \xmark & \xmark \\ 
%         \citet{wang2024robotic}                            & \xmark & \xmark & \xmark & \xmark & \xmark & \xmark \\
%         Bunny-VisionPro \cite{bunnyvisionpro}              & \cmark & \xmark & \xmark & \xmark & \xmark & \xmark \\
%         IMMERTWIN \cite{immertwin}                         & \cmark & \xmark & \xmark & \xmark & \xmark & \xmark \\
%         \citet{meng2023virtual}                            & \xmark & \cmark & \xmark & \cmark & \xmark & \xmark \\
%         \citet{sharedctlframework}                         & \cmark & \xmark & \xmark & \xmark & \xmark & \xmark \\
%         OpenVR \cite{george2025openvr}                               & \xmark & \xmark & \xmark & \xmark & \xmark & \xmark \\
%         \citet{digitaltwinmr}                              & \xmark & \xmark & \xmark & \xmark & \xmark & \xmark \\
        
%         \midrule
%         \textbf{Ours} & \cmark & \cmark & \cmark & \cmark & \cmark & \cmark \\
%         \bottomrule
%     \end{tabular}
%     \caption{This is a cross-column table with automatic line breaking.}
% \end{table*}

% \begin{table*}[t]
%     \centering
%     \begin{tabular}{lccccccc}
%         \toprule
%         & \makecell{Cross-Scene}
%         & \makecell{Cross-Embodiment}
%         & \makecell{Cross-Simulator}
%         & \makecell{Cross-Reality}
%         & \makecell{Cross-Platform}
%         & \makecell{Cross-User}
%         & \makecell{Control Space} \\
%         \midrule
%         % Vicarios \cite{vicarios}                           & \xmark & \xmark & \xmark & \xmark & \xmark & \xmark \\     
%         % Augmented Visual Cues \cite{augmentedvisualcues}   & \xmark & \xmark & \xmark & \xmark & \xmark & \xmark \\ 
%         % OpenVR \cite{george2025openvr}                     & \xmark & \xmark & \xmark & \xmark & \xmark & \xmark \\
%         \citet{digitaltwinmr}                              & \xmark & \xmark & \xmark & \xmark & \xmark & \xmark &  \\
%         ARC-LfD \cite{arclfd}                              & \xmark & \xmark & \xmark & \xmark & \xmark & \xmark &  \\
%         \citet{sharedctlframework}                         & \cmark & \xmark & \xmark & \xmark & \xmark & \xmark &  \\
%         \citet{jiang2024comprehensive}                     & \cmark & \xmark & \xmark & \xmark & \xmark & \xmark &  \\
%         \citet{mosbach2022accelerating}                    & \cmark & \xmark & \xmark & \xmark & \xmark & \xmark & \\
%         Holo-Dex \cite{holodex}                            & \cmark & \xmark & \xmark & \xmark & \xmark & \xmark & \\
%         ARCADE \cite{arcade}                               & \cmark & \cmark & \xmark & \xmark & \xmark & \xmark & \\
%         DART \cite{dexhub-park}                            & Limited & Limited & Mujoco & Sim & Vision Pro & \xmark &  Cartesian\\
%         ARMADA \cite{armada}                               & \cmark & \cmark & \xmark & \xmark & \xmark & \xmark & \\
%         \citet{meng2023virtual}                            & \cmark & \cmark & \xmark & \cmark & \xmark & \xmark & \\
%         % GELLO \cite{wu2023gello}                           & \cmark & \xmark & \xmark & \xmark & \xmark & \xmark \\
%         % DexCap \cite{wang2024dexcap}                       & \xmark & \xmark & \xmark & \xmark & \xmark & \xmark \\
%         % AnyTeleop \cite{Qin2023AnyTeleopAG}                & \cmark & \cmark & \cmark & \cmark & \xmark & \cmark \\
%         % \citet{wang2024robotic}                            & \xmark & \xmark & \xmark & \xmark & \xmark & \xmark \\
%         Bunny-VisionPro \cite{bunnyvisionpro}              & \cmark & \cmark & \xmark & \xmark & \xmark & \xmark & \\
%         IMMERTWIN \cite{immertwin}                         & \cmark & \cmark & \xmark & \xmark & \xmark & \xmark & \\
%         Open-TeleVision \cite{opentelevision}              & \cmark & \cmark & \xmark & \xmark & \cmark & \xmark & \\
%         \citet{szczurek2023multimodal}                     & \xmark & \xmark & \xmark & Real & \xmark & \cmark & \\
%         OPEN TEACH \cite{openteach}                        & \cmark & \cmark & \xmark & \xmark & \xmark & \cmark & \\
%         \midrule
%         \textbf{Ours} & \cmark & \cmark & \cmark & \cmark & \cmark & \cmark \\
%         \bottomrule
%     \end{tabular}
%     \caption{TODO, Bruce: this table can be further optimized.}
% \end{table*}

\definecolor{goodgreen}{HTML}{228833}
\definecolor{goodred}{HTML}{EE6677}
\definecolor{goodgray}{HTML}{BBBBBB}

\begin{table*}[t]
    \centering
    \begin{adjustbox}{max width=\textwidth}
    \renewcommand{\arraystretch}{1.2}    
    \begin{tabular}{lccccccc}
        \toprule
        & \makecell{Cross-Scene}
        & \makecell{Cross-Embodiment}
        & \makecell{Cross-Simulator}
        & \makecell{Cross-Reality}
        & \makecell{Cross-Platform}
        & \makecell{Cross-User}
        & \makecell{Control Space} \\
        \midrule
        % Vicarios \cite{vicarios}                           & \xmark & \xmark & \xmark & \xmark & \xmark & \xmark \\     
        % Augmented Visual Cues \cite{augmentedvisualcues}   & \xmark & \xmark & \xmark & \xmark & \xmark & \xmark \\ 
        % OpenVR \cite{george2025openvr}                     & \xmark & \xmark & \xmark & \xmark & \xmark & \xmark \\
        \citet{digitaltwinmr}                              & \textcolor{goodred}{Limited}     & \textcolor{goodred}{Single Robot} & \textcolor{goodred}{Unity}    & \textcolor{goodred}{Real}          & \textcolor{goodred}{Meta Quest 2} & \textcolor{goodgray}{N/A} & \textcolor{goodred}{Cartesian} \\
        ARC-LfD \cite{arclfd}                              & \textcolor{goodgray}{N/A}        & \textcolor{goodred}{Single Robot} & \textcolor{goodgray}{N/A}     & \textcolor{goodred}{Real}          & \textcolor{goodred}{HoloLens}     & \textcolor{goodgray}{N/A} & \textcolor{goodred}{Cartesian} \\
        \citet{sharedctlframework}                         & \textcolor{goodred}{Limited}     & \textcolor{goodred}{Single Robot} & \textcolor{goodgray}{N/A}     & \textcolor{goodred}{Real}          & \textcolor{goodred}{HTC Vive Pro} & \textcolor{goodgray}{N/A} & \textcolor{goodred}{Cartesian} \\
        \citet{jiang2024comprehensive}                     & \textcolor{goodred}{Limited}     & \textcolor{goodred}{Single Robot} & \textcolor{goodgray}{N/A}     & \textcolor{goodred}{Real}          & \textcolor{goodred}{HoloLens 2}   & \textcolor{goodgray}{N/A} & \textcolor{goodgreen}{Joint \& Cartesian} \\
        \citet{mosbach2022accelerating}                    & \textcolor{goodgreen}{Available} & \textcolor{goodred}{Single Robot} & \textcolor{goodred}{IsaacGym} & \textcolor{goodred}{Sim}           & \textcolor{goodred}{Vive}         & \textcolor{goodgray}{N/A} & \textcolor{goodgreen}{Joint \& Cartesian} \\
        Holo-Dex \cite{holodex}                            & \textcolor{goodgray}{N/A}        & \textcolor{goodred}{Single Robot} & \textcolor{goodgray}{N/A}     & \textcolor{goodred}{Real}          & \textcolor{goodred}{Meta Quest 2} & \textcolor{goodgray}{N/A} & \textcolor{goodred}{Joint} \\
        ARCADE \cite{arcade}                               & \textcolor{goodgray}{N/A}        & \textcolor{goodred}{Single Robot} & \textcolor{goodgray}{N/A}     & \textcolor{goodred}{Real}          & \textcolor{goodred}{HoloLens 2}   & \textcolor{goodgray}{N/A} & \textcolor{goodred}{Cartesian} \\
        DART \cite{dexhub-park}                            & \textcolor{goodred}{Limited}     & \textcolor{goodred}{Limited}      & \textcolor{goodred}{Mujoco}   & \textcolor{goodred}{Sim}           & \textcolor{goodred}{Vision Pro}   & \textcolor{goodgray}{N/A} & \textcolor{goodred}{Cartesian} \\
        ARMADA \cite{armada}                               & \textcolor{goodgray}{N/A}        & \textcolor{goodred}{Limited}      & \textcolor{goodgray}{N/A}     & \textcolor{goodred}{Real}          & \textcolor{goodred}{Vision Pro}   & \textcolor{goodgray}{N/A} & \textcolor{goodred}{Cartesian} \\
        \citet{meng2023virtual}                            & \textcolor{goodred}{Limited}     & \textcolor{goodred}{Single Robot} & \textcolor{goodred}{PhysX}   & \textcolor{goodgreen}{Sim \& Real} & \textcolor{goodred}{HoloLens 2}   & \textcolor{goodgray}{N/A} & \textcolor{goodred}{Cartesian} \\
        % GELLO \cite{wu2023gello}                           & \cmark & \xmark & \xmark & \xmark & \xmark & \xmark \\
        % DexCap \cite{wang2024dexcap}                       & \xmark & \xmark & \xmark & \xmark & \xmark & \xmark \\
        % AnyTeleop \cite{Qin2023AnyTeleopAG}                & \cmark & \cmark & \cmark & \cmark & \xmark & \cmark \\
        % \citet{wang2024robotic}                            & \xmark & \xmark & \xmark & \xmark & \xmark & \xmark \\
        Bunny-VisionPro \cite{bunnyvisionpro}              & \textcolor{goodgray}{N/A}        & \textcolor{goodred}{Single Robot} & \textcolor{goodgray}{N/A}     & \textcolor{goodred}{Real}          & \textcolor{goodred}{Vision Pro}   & \textcolor{goodgray}{N/A} & \textcolor{goodred}{Cartesian} \\
        IMMERTWIN \cite{immertwin}                         & \textcolor{goodgray}{N/A}        & \textcolor{goodred}{Limited}      & \textcolor{goodgray}{N/A}     & \textcolor{goodred}{Real}          & \textcolor{goodred}{HTC Vive}     & \textcolor{goodgray}{N/A} & \textcolor{goodred}{Cartesian} \\
        Open-TeleVision \cite{opentelevision}              & \textcolor{goodgray}{N/A}        & \textcolor{goodred}{Limited}      & \textcolor{goodgray}{N/A}     & \textcolor{goodred}{Real}          & \textcolor{goodgreen}{Meta Quest, Vision Pro} & \textcolor{goodgray}{N/A} & \textcolor{goodred}{Cartesian} \\
        \citet{szczurek2023multimodal}                     & \textcolor{goodgray}{N/A}        & \textcolor{goodred}{Limited}      & \textcolor{goodgray}{N/A}     & \textcolor{goodred}{Real}          & \textcolor{goodred}{HoloLens 2}   & \textcolor{goodgreen}{Available} & \textcolor{goodred}{Joint \& Cartesian} \\
        OPEN TEACH \cite{openteach}                        & \textcolor{goodgray}{N/A}        & \textcolor{goodgreen}{Available}  & \textcolor{goodgray}{N/A}     & \textcolor{goodred}{Real}          & \textcolor{goodred}{Meta Quest 3} & \textcolor{goodred}{N/A} & \textcolor{goodgreen}{Joint \& Cartesian} \\
        \midrule
        \textbf{Ours}                                      & \textcolor{goodgreen}{Available} & \textcolor{goodgreen}{Available}  & \textcolor{goodgreen}{Mujoco, CoppeliaSim, IsaacSim} & \textcolor{goodgreen}{Sim \& Real} & \textcolor{goodgreen}{Meta Quest 3, HoloLens 2} & \textcolor{goodgreen}{Available} & \textcolor{goodgreen}{Joint \& Cartesian} \\
        \bottomrule
        \end{tabular}
    \end{adjustbox}
    \caption{Comparison of XR-based system for robots. IRIS is compared with related works in different dimensions.}
\end{table*}


\section{RELATED WORKS}

Research has shown that users display common instinctive social signals during robot errors, distinguishing these situations from error-free scenarios. These signals include gaze behaviour \cite{kontogiorgos_embodiment_2020, peacock_gaze_2022, kontogiorgos_systematic_2021, ramtin_hri_2025}, facial expressions \cite{mirnig_impact_2015, stiber_modeling_2022, kontogiorgos_systematic_2021, wachowiak_time_2024}, verbalisation \cite{kontogiorgos_embodiment_2020, mirnig_impact_2015, kontogiorgos_systematic_2021}, and body movements \cite{mirnig_impact_2015, trung_head_2017, wachowiak_time_2024}. For example,  Peacock et al. \cite{peacock_gaze_2022} observed that gaze initially increases in motion during failures and then stabilises as users address the issue. Stiber et al. \cite{stiber_using_2023} identified heightened activity in facial muscles, such as smiling and brow lowering, during robot errors. Similarly, Kontogiorgos et al. \cite{kontogiorgos_systematic_2021, kontogiorgos_embodiment_2020} reported increased spoken words, longer utterances, and more gaze shifts toward the robot, reflecting greater user engagement during failures.

Several studies have explored machine-learning approaches to detect failures in human-robot interactions using various behavioural and physiological cues. For example, Peacock et al. \cite{peacock_gaze_2022} trained logistic regression models on gaze dynamics to detect failures, achieving accurate detection a few seconds after the errors occurred. Similarly, Kontogiorgos et al. \cite{kontogiorgos_systematic_2021, kontogiorgos_behavioural_2020} developed machine learning models, including XGBoost and Random Forest, that utilised multimodal behaviours—such as linguistic, facial, and acoustic features—to achieve high accuracy in distinguishing failure scenarios from non-failure scenarios in a verbal guidance scenario. Separately, Stiber et al. \cite{stiber_modeling_2022} trained Multi-Layer Perceptron (MLP) models using action units (AUs) derived from facial reactions, showing that AUs are effective if users provide timely and observable responses to robot errors. Since not all users exhibit clear facial or verbal reactions to failures, this highlights a gap in the literature. This research aims to address this gap by designing classifier models based on user gaze during a collaborative task, evaluating their performance relative to the time elapsed after a robot failure, and assessing their effectiveness in real-time settings.


\section{Methodology}
\label{sec:methodology}

We begin with reformulating~\eqref{obj:original_sparse_problem_perspective_formulation_convex_relaxation} as the following \textit{unconstrained} optimization problem
\begin{align}
    \label{obj:original_sparse_problem_convex_composite_reformulation}
    \min_{\bbeta} f(\bX \bbeta, \by) + 2 \lambda_2 \, g(\bbeta),
\end{align}
where the implicit function $g: \R^p \to \R \cup \{ \infty \}$ is defined~as
\begin{align}
    \label{eq:function_g_definition}
    g(\bbeta) = \left\{ 
    \begin{array}{cl}
        \min\limits_{\bz \in \R^p} & \frac{1}{2} \sum_{j \in [p]} \beta_j^2 / z_j \\[1ex]
        \st & \bz \in [0, 1]^p, \, \bm 1^\top \bz \leq k, \\
        & -M z_j \leq \beta_j \leq M z_j ~ \forall j \in [p].
    \end{array} 
    \right.
\end{align}
Here, we follow the standard convention that an infeasible minimization problem is assigned a value of $+\infty$. 
Note that $g$ is convex as convexity is preserved under partial minimization over a convex set~\citep[Theorem~5,3]{rockafellar1970convex}. 
Furthermore, as $f$ is assumed to be Lipschitz smooth and $g$ is non-smooth, problem~\eqref{obj:original_sparse_problem_convex_composite_reformulation} is an unconstrained optimization problem with a convex composite objective function. As such, it is amenable to be solved using the FISTA algorithm proposed in \citep{beck2009fast}. 

In the following, we first analyze the conjugate of $g$. We then propose an efficient numerical method to compute the proximal operator of $g^*$. This, in turn, enables us to compute the proximal operator of $g$, leading to an efficient implementation of the FISTA algorithm. To further enhance the performance of FISTA, we present an efficient approach to solve the minimization problem~\eqref{eq:function_g_definition}, which guides us in developing an effective restart procedure. Finally, we conclude this section by providing efficient lower bounds for each step of the BnB framework.

\subsection{Conjugate function $g^*$}
Recall that the conjugate of $g$ is defined as
\begin{align*}
    g^*(\bm \alpha) = \sup_{\bm \beta \in \R^p} ~ \bm \alpha^\top \bm \beta - g(\bm \beta).
\end{align*}
The following lemma gives a closed-form expression for $g^*$, where $\TopSum_k(\cdot)$ denotes the sum of the top $k$ largest elements, and $H_M: \R \to \R$ is the Huber loss function defined as
\begin{equation}
    H_M(\alpha_j) := \begin{cases}
        \frac{1}{2} \alpha_j^2 & \text{if } \vert{\alpha_j} \leq M \\
        M \vert{\alpha_j} - \frac{1}{2} M^2 & \text{if } \vert{\alpha_j} > M
    \end{cases}.
\end{equation}
For notational simplicity, we use the shorthand notation ${\bf H}_M(\balpha)$ to denote ${\bf H}_M(\balpha) = (H_M(\alpha_1), \dots, H_M(\alpha_p))$.

\begin{lemma}
    \label{lemma:fenchel_conjugate_of_g_closed_form_expression}
    The conjugate of $g$ is given by
    \begin{equation}
        g^*(\balpha) = \TopSum_k({\bf H}_M(\balpha)).
    \end{equation}
\end{lemma}
This closed-form expression enables us to compute the proximal of $g^*$. Note that while the proximal operators of both $\TopSum_k$ and ${\bf H}_M$ functions are known (see, for example, \citep[Examples~6.50 \& 6.54]{beck2017first}), the conjugate function $g^*$ is defined as the composition of these two functions.
Alas, there is no general formula to derive the proximal operator of a composition of two functions based on the proximal operators of the individual functions. In the next section, we will see how to bypass this compositional difficulty.


\subsection{Proximal operator of $g^*$}
Recall that the proximal operator of $g^*$ with weight parameter $\rho > 0$ is defined as
\begin{align}
    \label{eq:prox:g*}
    \prox_{\rho g^*}(\bm \mu) = \argmin_{\bm \alpha \in \R^p} ~ \frac{1}{2} \Vert{\bm \alpha - \bm \mu}_2^2 + \rho g^*(\bm \alpha).
\end{align}
The evaluation of $\prox_{\rho g^*}$ involves a minimization problem that can be reformulated as a convex SOCP problem. 
Generic solvers based on IPM and ADMM require solving systems of linear equations. This results in cubic time complexity per iteration, making them computationally expensive, particularly for large-scale problems. 
These methods also cannot return \emph{exact} solutions. 
The lack of exactness can affect the stability and reliability of the proximal operator, which is crucial for the convergence of the FISTA algorithm. 
Inspired by~\citep{busing2022monotone}, we present Algorithm~\ref{alg:PAVA_algorithm}, a customized pooled adjacent violators algorithm that provides an exact evaluation of $\prox_{\rho g^*}$ in linear time after performing a simple 1D sorting step.

\begin{theorem}
    \label{theorem:pava_algorithm_linear_time_complexity_and_exact_solution}
    For any $\bmu \in \R^p$, Algorithm~\ref{alg:PAVA_algorithm} returns the \textit{exact} evaluation of $\prox_{\rho g^*}(\bmu)$ in $\tilde {\mathcal O}(p)$.
\end{theorem}


\begin{algorithm}[tb]
\caption{Customized PAVA to solve $\text{prox}_{\rho g^*}(\bmu)$}
\label{alg:PAVA_algorithm}
\begin{flushleft}
\textbf{Input:} vector $\bmu$, scalar multiplier $\rho$, and threshold $M$ of the Huber loss function $H_M(\cdot)$\\
% \textbf{Output:} vector $\balpha^*=\text{prox}_{\rho g^*}(\bmu)$.
\end{flushleft}
\begin{algorithmic}[1]
    \STATE Initialize $\brho \in \mathbb{R}^n$ with $\rho_j = \rho$ if $j \in \{1, 2, ..., k\}$ and $\rho_j = 0$ otherwise.
    \STATE \COMMENT{Sort $\bmu$ such that $\vert{\mu_1} \geq \vert{\mu_2} \geq ... \geq \vert{\mu_p}$; $\bpi$ is the sorting order.}
    \STATE $\bmu, \bpi = \text{SpecialSort} (\bmu)$
    
    \STATE \COMMENT{STEP 1: Initialize a pool of $p$ blocks with start and end indices; each block initially has length equal to $1$}
    \STATE $\calJ = \{[1, 1], [2, 2], ..., [p, p]\}$

    \STATE \COMMENT{STEP 2: Initialize $\hat{\nu}_j$ in each block by ignoring the isotonic constraints}
    \FOR{$j=1, 2, \dots, p$} 
        \STATE $\hat{\nu}_j = \argmin_{\nu} \frac{1}{2} (\nu - \vert{\mu_j})^2 + \rho_j H_M(\nu)$ \label{alg_line:PAVA_algorithm_initialization_step}
    \ENDFOR
    
    \STATE \COMMENT{STEP 3: Whenever there is an isotonic constraint violation between two adjacent blocks, merge the two blocks by setting all values to be the minimizer of the objective function restricted to this merged block; use \textcolor{red}{Algorithm~\ref{alg:up_and_down_block_algorithm_for_merging_in_PAVA}} in~\ref{appendix_sec:proofs}}
    \WHILE{$\exists [a_1, a_2], [a_2+1, a_3] \in \calJ \text{ s.t. } \hat{\nu}_{a_1} < \hat{\nu}_{a_3}$}
        \STATE $\calJ = \calJ \setminus \{[a_1, a_2]\} \setminus \{[a_2+1, a_3]\} \cup \{[a_1, a_3]\}$
        \STATE $\hat{\nu}_{[a_1:a_3]} = \argmin\limits_{\nu} \sum\limits_{j=a_1}^{a_3} \left[ \frac{1}{2} (\nu - \vert{\mu_j})^2 + \rho_j H_M(\nu) \right]$ \label{alg_line:PAVA_algorithm_pooling_step}
    \ENDWHILE

    \STATE \COMMENT{Return $\hat{\bnu}$ with the inverse sorting order}
    \STATE \textbf{Return} $\text{sgn}(\bmu) \odot \bpi^{-1}(\hat{\bnu})$
    
\end{algorithmic}
\end{algorithm}

The proof relies on several auxiliary lemmas.
We start with the following lemma, which uncovers a close connection between the proximal operator of $g^*$ and the generalized isotonic regression problems.

\begin{lemma}
    \label{lemma:equivalence_between_proximal_operator_and_huber_isotonic_regression}
    For any $\bmu \in \R^p$, we have 
    $$\prox_{\rho g^*}(\bmu) = \sgn(\bmu) \odot \bnu^\star, $$ 
    where $\odot$ denotes the Hadamard (element-wise) product, $\bnu^\star$ is the unique solution of the following optimization problem
    \begin{align}
        \label{obj:KyFan_Huber_isotonic_regression}
        \begin{array}{cl}
            \min\limits_{\bnu \in \R^p} & \frac{1}{2} \sum_{j \in [p]} (\nu_j - \vert{\mu_j})^2 + \rho \sum_{j \in \calJ} H_M (\nu_j) \\[2ex]
            \st & \quad \nu_j \geq \nu_l \; \text{ if } \; \vert{\mu_j} \geq \vert{\mu_l} ~~ \forall j, l \in [p],
        \end{array} 
    \end{align}
    and $\calJ$ is the set of indices of the top $k$ largest elements of~$ \vert{\mu_j}, j \in [p]$. 
\end{lemma}
Problem~\eqref{obj:KyFan_Huber_isotonic_regression} replaces the $\TopSum_k$ in~\eqref{eq:prox:g*} from the conjugate function $g^*$ (as shown in Lemma~\ref{lemma:fenchel_conjugate_of_g_closed_form_expression}) with linear constraints.
%These constraints ensure that the elements of $\bnu$ maintain a non-decreasing order in magnitude.
While this may appear computationally complex, it actually converts the problem into an instance of isotonic regression~\citep{best1990active}. Such problems can be solved exactly in linear time after performing a simple sorting step.
The procedure is known as PAVA~\citep{busing2022monotone}. Specifically, Algorithm~\ref{alg:PAVA_algorithm} implements a customized PAVA variant designed to compute $\prox_{\rho g^*}$ exactly.
The following lemma shows that the vector generated by Algorithm~\ref{alg:PAVA_algorithm} is an exact solution to~\eqref{obj:KyFan_Huber_isotonic_regression}.
Intuitively, Algorithm~\ref{alg:PAVA_algorithm} iteratively merges adjacent blocks until no isotonic constraint violations remain, at which point the resulting vector is guaranteed to be the optimal solution to~\eqref{obj:KyFan_Huber_isotonic_regression}.

\begin{lemma}
    \label{lemma:PAVA_algorithm_exact_solution}
    The vector $\hat \bnu$ in Algorithm~\ref{alg:PAVA_algorithm} solves~\eqref{obj:KyFan_Huber_isotonic_regression} exactly.
\end{lemma}

Finally, the merging process in Algorithm~\ref{alg:PAVA_algorithm} can be executed efficiently. Intuitively, each element of $\bmu$ is visited at most twice; once during its initial processing and once when it is included in a merged block. This ensures that the process achieves a linear time complexity.

\begin{lemma}
    \label{lemma:PAVA_merging_linear_time_complexity}
    The merging step (lines 11-14) in Algorithm~\ref{alg:PAVA_algorithm} can be performed in linear time complexity $\mathcal O(p)$.
\end{lemma}
% \begin{proof}
%     The merging process in the PAVA algorithm involves iterating through the elements of the input vector $\bmu$ and merging adjacent blocks whenever an isotonic constraint violation is detected. Each element is visited at most twice: once when it is initially processed and once when it is part of a merged block. Therefore, the total number of operations is proportional to the number of elements, resulting in a linear time complexity, $O(n)$.
% \end{proof}

Armed with these lemmas, one can easily prove Theorem~\ref{theorem:pava_algorithm_linear_time_complexity_and_exact_solution}. Details are provided in~\ref{appendix_sec:proofs}.

\vspace{-2mm}
\subsection{FISTA algorithm with restart}
A critical computational step in FISTA is the efficient evaluation of the proximal operator of $g$. By the extended Moreau decomposition theorem~\citep[Theorem~6.45]{beck2017first}, for any weight parameter $\rho > 0$ and any point $\bmu \in \R^p$, the proximal operators of $g$ and $g^*$ satisfies
\begin{align}
    \label{eq:Moreaus_identity}
    \prox_{\rho^{-1} g}(\bmu) = \bmu - \rho^{-1} \, \prox_{\rho g^*} \left( \rho \bmu \right).
\end{align}
Hence, together with Theorem~\ref{theorem:pava_algorithm_linear_time_complexity_and_exact_solution}, we can compute exactly $\prox_{\rho^{-1} g}$ using Algorithm~\ref{alg:PAVA_algorithm} with log-linear time complexity. This enables an efficient implementation of the FISTA algorithm. 
Alas, the vanilla FISTA algorithm is prone to oscillatory behavior, which results in a sub-linear convergence rate of $\mathcal O(1/T^2)$ after $T$ iterations. 
In the following, we further accelerate the empirical convergence performance of the FISTA algorithm by incorporating a simple restart strategy based on the function value, originally proposed in~\citep{o2015adaptive}.

In simple terms, the restart strategy operates as follows: if the objective function increases during the iterative process, the momentum coefficient is reset to its initial value.
The effectiveness of the restart strategy hinges on the efficient computation of the loss function. This task essentially reduces to evaluating the implicit function $g$ defined in~\eqref{eq:function_g_definition}, which would involve solving a SOCP problem.
However, the value of $g$ can be computed efficiently by leveraging the majorization technique~\citep{kim2022convexification}, as shown in Algorithm~\ref{alg:compute_g_value_algorithm}.

\begin{algorithm}[tb]
    \caption{Algorithm to compute $g(\bbeta)$}
    \label{alg:compute_g_value_algorithm}
    \begin{flushleft}
    \textbf{Input:} vectors $\bbeta \in \R^p$ \\ %from Step 2 Line 8 in Algorithm~\ref{alg:main_algorithm}. 
    %\textbf{Output:} The value of $g(\bbeta)$
    \end{flushleft}
    \begin{algorithmic}[1]
        \STATE Initialize: $\bpsi = \boldsymbol{0} \in \mathbb{R}^k$
        \STATE Sort $\bbeta$ partially such that \\ \vspace{0.3em}\hspace*{2em}
        $\vert{\beta_1} \geq \vert{\beta_2} \geq ... \geq \vert{\beta_k} \geq \max\limits_{k+1, ..., p} \{ \vert{\beta_j} \}$
        % Let $\bs$ be the cumulative sum of the absolute values of sorted $\bbeta$.
        \STATE $s = \sum_{j=1}^p \vert{\beta_j}$
        \FOR{$j=1, 2, \dots, k$}
            \STATE $\overline{s} = s / (k - j + 1)$
            \STATE \textbf{if} $\overline{s} \geq \vert{\beta_j}$ \textbf{then} $\psi_{j:k} = \overline{s}$; break \textbf{else} $\psi_j = \vert{\beta_j}$
            \STATE $s = s - \vert{\beta_j}$
        \ENDFOR
        \STATE \textbf{return} $\sum_{j=1}^k \psi_j^2$
    \end{algorithmic}
\end{algorithm}


\begin{theorem}
    \label{theorem:compute_g_value_algorithm_correctness}
        For any $\bbeta \in \R^p$, Algorithm~\ref{alg:compute_g_value_algorithm} computes the exact value of $g(\bbeta)$, defined in~\eqref{eq:function_g_definition}, in $\mathcal O(p + p \log k)$.
\vspace{-3mm}
\end{theorem}
Theorem~\ref{theorem:compute_g_value_algorithm_correctness} guarantees that Algorithm~\ref{alg:compute_g_value_algorithm} can efficiently compute the value of $g(\bbeta)$, which is crucial for our value-based restart strategy to be effective in practice.
Empirically, we observe that the function value-based restart strategy can accelerate FISTA from the sub-linear convergence rate of \(O(1/T^2)\) to a linear convergence rate in many empirical results.
To the best of our knowledge, \textit{this is the first linear convergence result of using a first-order method in the MIP context} when calculating the lower bounds in the BnB tree.
The FISTA algorithm is summarized in Algorithm~\ref{alg:main_algorithm}.

\begin{algorithm}[!b]
    \caption{Main algorithm to solve problem \eqref{obj:original_sparse_problem_perspective_formulation_convex_relaxation}}
    \label{alg:main_algorithm}
    \begin{flushleft}
    \textbf{Input:} number of iterations $T$, coefficient $\lambda_2$ for the $\ell_2$ regularization, and step size $L$ (Lipschitz-continuity parameter of $\nabla F(\bbeta)$)  
    \end{flushleft}
    \begin{algorithmic}[1]
        \STATE Initialize: $\bbeta^0 = \mathbf{0}$, $\bbeta^1 = \mathbf{0}$, $\phi = 1$
        \STATE Let: $\rho = L / (2\lambda_2)$, $\calL^1 = f(\mathbf{\bbeta^1}, \by)$
        \FOR{$t=1, 2, 3, ..., T$}
            \STATE \COMMENT{Step 1: momentum acceleration} 
            \STATE $\bgamma^t = \bbeta^t + \frac{t}{t+3} (\bbeta^t - \bbeta^{t-1})$ \vspace{1.5mm}
            \STATE \COMMENT{Step 2: proximal gradient descent; use \textcolor{red}{Algorithm~\ref{alg:PAVA_algorithm}}} 
            \STATE $\bgamma^t = \bgamma^t - \frac{1}{L} \nabla F(\bgamma^t)$ \label{alg_line:gradient_descent} 
            \STATE $\bbeta^{t+1} = \bgamma^t - \rho^{-1} \text{prox}_{\rho g^*} (\rho \bgamma^t)$ \label{alg_line:proximal_step} \vspace{1.5mm} 
            \STATE \COMMENT{Step 3: restart; use \textcolor{red}{Algorithm~\ref{alg:compute_g_value_algorithm}}}
            \STATE $\calL^{t+1} = f(\bX \bbeta^{t+1}, \by) + 2 \lambda_2 g(\bbeta^{t+1})$
            \STATE \textbf{if} $\calL^{t+1} \geq \calL^{t}$ \textbf{then} $\phi = 1$ \textbf{else} $\phi = \phi + 1$
        \ENDFOR
        \STATE \textbf{return} $\bbeta^{T+1}$
    \end{algorithmic}
\end{algorithm}


\vspace{-3mm}
\subsection{Safe Lower Bounds for GLMs}
\label{subsec:safe_lower_bounds_for_glms}
We conclude this section by commenting on how to use Algorithm~\ref{alg:main_algorithm} in the BnB tree to prune nodes.
As an iterative algorithm, FISTA yields only an approximate solution $\hat{\bbeta}$ to~\eqref{obj:original_sparse_problem_perspective_formulation_convex_relaxation}. Consequently, while we can calculate the objective function for $\hat{\bbeta}$ efficiently, this value is not necessarily a lower bound of the original problem--only the optimal value of the relaxed problem~\eqref{obj:original_sparse_problem_perspective_formulation_convex_relaxation} serves as a guaranteed lower bound.
To get a safe lower bound, we rely on the weak duality theorem, in which for any proper, lower semi-continuous, and convex functions $F: \R^n \to \R \cup \{\infty\}$ and $G: \R^p \to \R \cup \{\infty\}$, we have
\begin{align*}
    \inf_{\bbeta \in \R^p} F(\bX \bbeta) + G(\bbeta) 
    &\geq \sup_{\bzeta \in \R^n} - F^*(-\bzeta) - G^*(\bX^\top \bzeta) \\ 
    &\geq - F^*(-\bzeta) - G^*(\bX^\top \bzeta) \,~ \forall \bzeta \in \R^n\!\!,
\end{align*}
where $F^*$ and $G^*$ denote the conjugates of $F$ and $G$, respectively, while the second inequality follows from the definition of the supremum operator. Letting $F(\bX \bbeta) = f(\bm X \bm \beta, \bm y)$, $G(\bbeta) = 2 \lambda_2 g(\bbeta)$ and $\bzeta = \nabla F(\bX \hat \bbeta)$, where $\hat \bbeta$ is the output of the FISTA Algorithm~\ref{alg:main_algorithm}, we arrive at the safe lower bound
\begin{align}
    \label{eq:fenchel_duality_theorem_F_y(Ax)+G(x)}
    P_{\text{MIP}}^\star \geq - F^*(-\hat{\bzeta}) - G^*(\bX^\top \hat{\bzeta}),
\end{align}
where the inequality follows from the relaxation bound $P_{\text{MIP}}^\star \geq P_{\text{conv}}^\star$ and the weak duality theorem.
For convenience, we provide a list of $F^*(\cdot)$ for different GLM loss functions in~\ref{appendix_sec:convex_conjugate_for_GLM_loss_functions}.
The readers are also referred to~\ref{appendix_sec:safe_lower_bound_more_discussions},  where we derive the safe lower bound for the linear regression problem with eigen-perspective relaxation as an example.



% \input{sections/04_2_Expriment}
% \input{sections/04_3_Measures}


\section{Results}
\label{sec:results}
% 
Figures~\ref{fig:LLM}-\ref{fig:DC} present results from applying counterfactual cross-validation (Algorithm~\ref{alg:C-CV}) across six benchmark scenarios detailed in \S\ref{sec:Benchmark_Toolbox}. Below, we outline our implementation approach and key findings.

For the LLM-based social network model, we conducted 10 distinct runs, constrained by OpenAI API limitations; the current simulation, with $N=1000$ and $T=30$, required approximately 100,000 GPT-3.5 API calls to generate experimental and ground truth results. Each run employs a unique treatment allocation following a staggered rollout design across three stages with $\Vec{\expr} = (0.2, 0.5, 0.8)$, each spanning 10 periods. This design implies that on average 20\% of units received the intervention in the initial 10 periods, followed by an additional 30\% in the subsequent 10 periods, and so forth.

For the remaining five experiments, we conducted 100 independent runs for each setting, utilizing a fresh treatment allocation for each run through a staggered rollout design. The design comprises four equal-length stages with treatment probabilities $\Vec{\expr} = (0, 0.2, 0.4, 0.6)$. In each figure's leftmost panel, we display the temporal evolution of outcomes through their mean and standard deviation, along with the 95th percentile across runs.

\begin{figure}
    \centering
    \includegraphics[width=1\linewidth]{plots/LLM.pdf}
    \caption{LLM-based social network with $N=1,000$ agents.}
    \label{fig:LLM}
\end{figure}

\begin{figure}
    \centering
    \includegraphics[width=1\linewidth]{plots/BAM1.pdf}
    \caption{Belief adoption model with Krupina network with $N=3,366$ users.}
    \label{fig:BAM1}
\end{figure}

\begin{figure}
    \centering
    \includegraphics[width=1\linewidth]{plots/BAM2.pdf}
    \caption{Belief adoption model with Topolcany network with $N=18,246$ users.}
    \label{fig:BAM2}
\end{figure}

\begin{figure}
    \centering
    \includegraphics[width=1\linewidth]{plots/BAM3.pdf}
    \caption{Belief adoption model with Zilina network with $N=42,971$ users.}
    \label{fig:BAM3}
\end{figure}

\begin{figure}
    \centering
    \includegraphics[width=1\linewidth]{plots/Auction.pdf}
    \caption{Ascending auction model with $N=500$ objects.}
    \label{fig:auction}
\end{figure}

The second panels of Figures~\ref{fig:LLM}-\ref{fig:DC} display the box plot of the average total treatment effect (TTE) across multiple time periods. The TTE contrasts the counterfactual of all units under treatment against all units under control:
% 
\begin{align}
    \label{eq:TTE}
    \text{TTE} :=
    \frac{1}{LN} \sum_{t=T-L+1}^T \sum_{i=1}^\UN 
    \left[
    \outcomeDW{\mathbf{1}}{i}{t}
    -
    \outcomeDW{\mathbf{0}}{i}{t}
    \right].
\end{align}
% 
In each setting, we carefully select $L$ so that the TTE in \eqref{eq:TTE} covers all periods with nonzero treatment probability, ensuring our benchmark estimators remain meaningful. The results compare ground truth (GT) values against estimates obtained from three methods: our proposed causal message-passing approach (CMP), the difference-in-means estimator (DM), and the Horvitz-Thompson estimator (HT)%
% 
\footnote{Difference-in-means (DM) and Horvitz-Thompson (HT) are expressed as:
\begin{align*}
%
\DIME :=  \frac{1}{L} \sum_{t=T-L+1}^T 
\Big(
\frac{\sum_{i=1}^N\outcomeD{}{i}{t}\treatment{i}{t}}{\sum_{i=1}^N\treatment{i}{t}} - \frac{\sum_{i=1}^N\outcomeD{}{i}{t}(1-\treatment{i}{t})}{\sum_{i=1}^N(1-\treatment{i}{t})} \Big),
%
\quad\quad
% 
\HTE :=  \frac{1}{LN} \sum_{t=T-L+1}^T \sum_{i=1}^N \left( \frac{\outcomeD{}{i}{t} \treatment{i}{t}}{\E[\treatment{i}{t}]} - \frac{\outcomeD{}{i}{t} (1 - \treatment{i}{t})}{\E[1 - \treatment{i}{t}]} \right).
%
\end{align*}}
%
\citep{savje2021average}. Finally, the rightmost two panels in each figure display the CFE under the ground truth and CMP estimates for all-control and all-treatment conditions, along with their respective 95th percentiles.

In implementing Algorithm~\ref{alg:C-CV}, we employ five validation batches ($b_v=5$). 
To select candidate estimators, we begin with a base model where each outcome is expressed as a linear function of two components: the sample mean of outcomes from the previous round and the current treatment allocation means. We then systematically modify this model by incorporating additional first-order and higher-order terms. The configurations also span batch counts from 200 to 2000 and batch sizes ranging from 0.1 to 20 percent of the population size. To estimate parameters, we employ Ridge regression with penalty parameters logarithmically spaced from $10^{-4}$ to $10^{4}$. These parameters are comprehensively combined to generate a diverse set of potential estimators, with time blocks aligned to experimental stages. For example, when $T=40$ and the design $\Vec{\expr} = (0, 0.2, 0.4, 0.6)$ with equal length blocks is used, 
$\tblockList$ has four elements, one corresponding to each block with a fixed treatment probability.
Then, the selection process incorporates both domain knowledge and observed data characteristics. For instance, the pronounced temporal patterns evident in the left panels of Figures~\ref{fig:NYC_taxi}-\ref{fig:DC}, observed in the New York City taxi model, exercise encouragement program, and data center model, necessitate estimators with detrending steps (see Remark~\ref{rem:proprocessing}). Computational efficiency is maintained by constraining the estimator search space based on the experimental context.

Overall, our framework demonstrates robust performance across all six scenarios, successfully estimating counterfactual evolutions despite strong seasonality patterns and without requiring information about the underlying interference network. The proposed method achieves significantly better performance than both DM and HT estimators, even in settings with subtle treatment effects. As illustrated in Figures~\ref{fig:LLM}-\ref{fig:NYC_taxi}, CMP yields estimates with both smaller bias and variance in different scenarios. The effectiveness of our method is particularly evident in the challenging scenarios presented in Figures~\ref{fig:BAM1}-\ref{fig:auction} and \ref{fig:DC}, where conventional estimators struggle to reliably determine even the direction of treatment effects. These comprehensive results establish our framework's capability to deliver precise estimates of counterfactual evolutions and treatment effects across diverse experimental settings.

\begin{figure}
    \centering
    \includegraphics[width=1\linewidth]{plots/NYC_taxi.pdf}
    \caption{New York City Taxi model with $N=18,768$ Routes.}
    \label{fig:NYC_taxi}
\end{figure}

\begin{figure}
    \centering
    \includegraphics[width=1\linewidth]{plots/EEP.pdf}
    \caption{Exercise encouragement program with $N=30,162$ users.}
    \label{fig:EP}
\end{figure}

\begin{figure}
    \centering
    \includegraphics[width=1\linewidth]{plots/DC_N1k.pdf}
    \caption{Data Center model with $N=1,000$ servers.}
    \label{fig:DC}
\end{figure}

\begin{remark}
    \label{rem:batch_generation}
    % 
    Selecting a predetermined number of batches for a given batch size $n^\batch$ presents a significant computational challenge, particularly in large-scale problems with time-varying treatment allocations across units. For staggered rollout designs, we implement a heuristic approach while deferring comprehensive analysis to future research. Our heuristic consists of three steps. First, we order units by their treatment duration, defined as the number of time periods under treatment. Second, we select two blocks of size $n^\batch$—one that slides through the ordered list to cover all treatment durations, and another chosen randomly to ensure sufficient between-batch variation. Third, we select individual units from these merged blocks with equal probability to generate batches with average size $n^\batch$. This procedure maintains computational efficiency while ensuring batches with diverse treatment allocations with high probability.
\end{remark}

\section{DISCUSSION}  

% In this work, we compared behavioural responses to robot failures, examining how individuals reacted to these failures and their subsequent perceptions of the robot. The failures in our experiment varied in type, timing, and the robot's acknowledgement of the failure. Our findings indicate differences in both user gaze and perception of the robot, specifically in feelings of anxiety, and perceptions of the robot being skilled and sensible. Based on these results, we will discuss the impact of two failure types, two timings, and two levels of the robot's acknowledgement of the failure on user gaze and perception.

This study compared behavioural responses to robot failures, focusing on how individuals reacted and perceived the robot. Failures varied by type, timing, and acknowledgement. The findings revealed that robot failures affect user gaze and perceptions. These findings are discussed further in the following section.

\subsection{Behavioural Response}

To address the first research question, we analysed user gaze behaviour in multiple ways: the number of gaze shifts, gaze distribution during puzzle-solving, and gaze entropy based on transition matrices. These measures allowed us to examine how the type and timing of failures, as well as whether the robot acknowledged its failure, influenced user gaze patterns and whether gaze behaviour varied across different failure scenarios. Our results showed that user gaze is a reliable indicator of robot failures. When the robot made a failure, participants exhibited more frequent gaze shifts between different AoIs, likely due to confusion and an attempt to understand what was happening. This finding is similar to the results of Kontogiorgos et al. \cite{kontogiorgos_embodiment_2020}, who found that people tend to gaze more at the robot when it makes a mistake. The literature suggests that different types of failures influence user perceptions of the robot \cite{morales_interaction_2019}, and our findings support this by showing that users exhibit distinct gaze behaviours in response to various failure types. For example, when the failure was executional, the number of gaze shifts towards the robot was significantly higher compared to when the failure was decisional. Moreover, during executional failures, the proportion of time spent looking at the robot was much higher compared to decisional failures. It is crucial for the robot to recognize the type of failure it has made so that it can determine the appropriate strategy for recovery and regain the user's trust.


The timing of the failure is also crucial for the robot, as it requires different approaches for recovery and repair. In our research, while the timing of the failure—whether at the start or end of the interaction—did not significantly affect gaze shifts, it did influence gaze transition matrices, and gaze distribution across AoIs. Failures at the beginning of the interaction led to higher median gaze transition values, indicating more randomness early on. Additionally, participants' focus on the Tangram figure was more when the failure occurred at the beginning of the interaction compared to later ones, while their focus on the robot's body or end effector was more during late failures than early ones. 


% \subsubsection{Failure Acknowledgement}

In our research, after committing a failure, the robot could either acknowledge the failure and then continue its action, or proceed without acknowledgement. We could not find significant differences in users' gaze behaviour when the robot acknowledged its failure and when it did not. As the literature suggests \cite{esterwood_you_2021, karli_what_2023, wachowiak_when_2024}, there are other verbal approaches to failure recovery, such as promises and technical explanations, which might influence users' gaze differently. Verbal failure recovery is important for robots, as it demonstrates an awareness of mistakes. This, in turn, can make the robot appear more intelligent and encourage users to engage with it more.


% \subsubsection{Anticipatory Gaze Behavior}
Our study also explored changes in users' anticipatory gaze behaviour during the task and its potential role in assisting the robot to recover from failures. Participants frequently anticipated the placement of the object before the robot executed the action, even when the robot made an error. This anticipatory gaze behaviour could serve as a valuable cue for the robot to detect its failures and initiate appropriate recovery strategies. However, we observed a decrease in participants' anticipatory gaze behaviour as the number of tasks increased. This decline may indicate reduced engagement over time, with participants being more actively collaborative at the beginning of the interaction. It also suggests that users' gaze behaviour might change throughout the interaction. These findings highlight the dynamic nature of gaze behaviour throughout the interaction.


\subsection{Subjective Measures}

To address the second research question, we examined user perceptions of the robot in three areas: perceived intelligence, sense of safety, and trust during failures. The analysis revealed how these measures varied with the type and timing of failure and whether the robot acknowledged its mistake.

The results of the subjective evaluation revealed that users' perceptions of the robot's intelligence and safety were not significantly influenced by the type of failure. However, users exhibited higher levels of trust in the robot during executional failures compared to decisional failures, suggesting that placing an object in an incorrect location reduces trust more than making an incorrect decision. Additionally, we observed interesting findings regarding the timing of the robot's failures. When failures occurred early in the interaction, users rated the robot as more intelligent and trustworthy compared to failures that occurred later. For the measure of "Sensible," this difference was statistically significant. These findings are consistent with previous research by Morales et al. \cite{morales_interaction_2019} and Lucas et al. \cite{lucas_getting_2018}. Interestingly, users reported feeling more relaxed when failures occurred later in the interaction, aligning with results from Desai et al. \cite{desai_impact_2013} and Rossi et al. \cite{rossi_how_2017}.

When the robot acknowledged its failures, users perceived it as slightly more intelligent and trustworthy but also experienced increased anxiety. This finding may be explained by the robot’s consistent physical repair actions a few seconds after each failure. When the robot did not explicitly acknowledge its failures, users might not have interpreted these actions as errors, reducing their perception of failure events.

\subsection{Limitations and Future Work}
 There were instances where participants were preoccupied with determining the placement of their next piece, which occasionally led them to overlook the robot's movements. However, these occurrences were minimal. Another limitation is the restriction to only two types of failure and whether the robot acknowledges its failure or not. The effect size in our study was medium; however, to obtain more robust results, a larger sample size would be beneficial.
 %Additionally, focusing solely on participants' gaze behaviour may not provide a comprehensive measure of failure detection. Incorporating other non-verbal cues, such as gestures or facial expressions, alongside gaze, could improve the accuracy of failure detection. 
 Furthermore, for safety reasons, the robot's arm movement was slowed and the experimenter was in the room, which may have influenced participants' perceptions. 
 Future research could address these limitations by exploring a broader range of failure types and incorporating explanatory feedback from the robot.





% %
In this work, we presented counterfactual situation testing (CST), a new actionable and meaningful framework for detecting individual discrimination in a dataset of classifier decisions.
We studied both single and multidimensional discrimination, focusing on the indirect setting.
For the latter kind, we compared its multiple and intersectional forms and provided the first evidence for the need to recognize intersectional discrimination as separate from multiple discrimination under non-discrimination law.
Compared to other methods, such as situation testing (ST) and counterfactual fairness (CF), CST uncovered more cases even when the classifier was counterfactually fair and after accounting for statistical significance.
For CF, in particular, we showed how CST equips it with confidence intervals, extending how we understand the robustness of this popular causal fairness definition. 

The decision-making settings tackled in this work are intended to showcase the CST framework and, importantly, to illustrate why it is necessary to draw a distinction between idealized and fairness given the difference comparisons when testing for individual discrimination. 
We hope the results motivate the adoption of the \textit{mutatis mutandis} manipulation over the \textit{ceteris paribus} manipulation.
We are aware that the experimental setting could be pushed further by considering higher dimensions or more complex causal structures. 
We leave this for future work.
%
Further,
extensions of CST should consider the impact of using different distance functions for measuring individual similarity \parencite{WilsonM97_HeteroDistanceFunctions}, and should explore a purely data-driven setup in which the running parameters and auxiliary causal knowledge are derived from the dataset \parencite{Cohen2013StatisticalPower, Peters2017_CausalInference}.
%
Furthermore,
extensions of CST should study settings in which the protected attribute goes beyond the binary, such as a high-cardinality categorical or an ordinal protected attribute \parencite{DBLP:journals/tkde/CerdaV22}. 
The setting in which the protected attribute is continuous is also of interest, though, in that case we could discretize it \parencite{DBLP:journals/tkde/GarciaLSLH13} and treat it as binary (the current setting) or as a high-cardinality categorical attribute.

Multidimensional discrimination testing is largely understudied \parencite{DBLP:conf/fat/0001HN23, WangRR22}. 
% Here, 
We have set a foundation for exploring the tension between multiple and intersectional discrimination, but future work should further study the problem of dealing with multiple protected attributes and their intersection.
It is of interest, for instance, formalizing the case in which one protected attribute dominates the others and the case in which the impact of each protected attribute varies based on individual characteristics.
% Formalizing the case in which one protected attribute dominates over the others as well as the case in which the effect of each protected attribute varies by individual characteristics are of interest.
While interaction terms and heterogeneous effects are understudied within SCM, both topics enjoy a well established literature in fields like economics \parencite{Wooldridge2015IntroductoryEconometrics}, which should enable future work.
% 
We hope these extensions and, overall, the fairness given the difference powering the CST framework motivate new work on algorithmic discrimination testing.

%
% EOS
%





\section*{Acknowledgement}
This research is partially supported by the Australian Research Council Discovery Early Career Research Award (Grant No. DE210100858)

\newpage

\bibliographystyle{ieeetr}
\balance
\bibliography{references,hri2025}



\end{document}
