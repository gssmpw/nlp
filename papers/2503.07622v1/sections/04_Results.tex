\section{RESULTS}

% \begin{figure*}[h!]
%   \centering
%   % First Row
%   \begin{subfigure}[b]{0.24\linewidth}
%     \centering
%     \includegraphics[width=\linewidth]{Photos/Accuracy_sectosec_EF.pdf}
%     \caption{ }
%   \end{subfigure}
%   \begin{subfigure}[b]{0.24\linewidth}
%     \centering
%     \includegraphics[width=\linewidth]{Photos/Accuracy_sectosec_DF.pdf}
%     \caption{ }
%   \end{subfigure}
%   \begin{subfigure}[b]{0.24\linewidth}
%     \centering
%     \includegraphics[width=\linewidth]{Photos/WholeOnSeconds_Accuracy_CF.pdf}
%     \caption{ }
%   \end{subfigure}
%   \begin{subfigure}[b]{0.24\linewidth}
%     \centering
%     \includegraphics[width=\linewidth]{Photos/WholeOnSeconds_Accuracy_PF.pdf}
%     \caption{ }
%   \end{subfigure}
  
%   % Second Row
%   \begin{subfigure}[b]{0.24\linewidth}
%     \centering
%     \includegraphics[width=\linewidth]{Photos/Failure detection percentage_CF3.pdf}
%     \caption{ }
%   \end{subfigure}
%   \begin{subfigure}[b]{0.24\linewidth}
%     \centering
%     \includegraphics[width=\linewidth]{Photos/Failure detection percentage_PF3.pdf}
%     \caption{ }
%   \end{subfigure}
%   \begin{subfigure}[b]{0.24\linewidth}
%     \centering
%     \includegraphics[width=\linewidth]{Photos/Failure detection percentage_CF5.pdf}
%     \caption{ }
%   \end{subfigure}
%   \begin{subfigure}[b]{0.24\linewidth}
%     \centering
%     \includegraphics[width=\linewidth]{Photos/Failure detection percentage_PF5.pdf}
%     \caption{ }
%   \end{subfigure}
  
%   \caption{The transition matrices for three different interaction scenarios—NF, EF, and DF—are presented, both for cases where the robot acknowledges its failure and where it does not. The vertical axis represents the current states, while the horizontal axis represents the next states. 'EE' stands for End Effector, 'T' for Tangram figure, 'RB' for Robot Body, 'Exp.' for Experimenter, 'PR' for Pieces (Robot), and 'PP' for Pieces (Participant). The transition matrices are displayed as heat maps.}
%   \label{Transition_matrices}
% \end{figure*}






\subsection{Evaluating Classification Performance with Varying Failure Times}
% \begin{figure}[H]
%   \centering
%   % First Row
%   \begin{subfigure}[b]{0.7\linewidth}
%     \centering
%     \includegraphics[width=\linewidth]{Photos/Accuracy_sectosec_EF.pdf}
%     \caption{ }
%   \end{subfigure}\\
%   \begin{subfigure}[b]{0.7\linewidth}
%     \centering
%     \includegraphics[width=\linewidth]{Photos/Accuracy_sectosec_DF.pdf}
%     \caption{ }
%   \end{subfigure}
  
%   \caption{Classifier accuracy assessed using the first \textit{n} seconds of the failure period: a) distinguishing between non-failure and executional failure, and b) distinguishing between non-failure and decisional failure.}
%   \label{fig2}
% \end{figure}





\begin{figure}[H]
  \centering
  \begin{subfigure}[t]{0.48\linewidth}
    \centering
    \includegraphics[width=\linewidth]{Photos/Accuracy_sectosec_EF.pdf}

    \label{GazeShiftsAllAoIs}
    \caption{ }
  \end{subfigure}
  \hfill
  \begin{subfigure}[t]{0.48\linewidth}
    \centering
    \includegraphics[width=\linewidth]{Photos/Accuracy_sectosec_DF.pdf}

    \label{GazeShiftsRobotBody}
    \caption{ }
  \end{subfigure}
  \caption{Classifier accuracy assessed using the first \textit{n} seconds of the failure period: a) distinguishing between non-failure and executional failure, and b) distinguishing between non-failure and decisional failure.}
  \label{fig2}
\end{figure}






To address the first research question, we evaluated the trained models (Random Forest, AdaBoost, and XGBoost), which were trained on the entire duration of non-failure and failure periods. We assessed their performance using only the first \textit{n} seconds of the failure period. Figure \ref{fig2} illustrates the average accuracy of the models in distinguishing between non-failure (NF) and executional failure (EF), as well as between non-failure and decisional failure (DF). As \textit{n} increases to 5 seconds, the accuracy stabilizes. For distinguishing EF from NF, the accuracy remains around 90\%, with a Recall of Failure of approximately 94\% across all classifiers. Similarly, for distinguishing DF from NF, the accuracy stabilises around 80\%, with a Recall of Failure of approximately 90\% for all classifiers.



\subsection{Evaluating Classification Performance for Real-Time Failure Detection}

\begin{figure*}[h!]
  \centering
  
% Second Row
  \begin{subfigure}[b]{0.3\linewidth}
    \centering
    \includegraphics[width=\linewidth]{Photos/WholeOnSeconds_Accuracy_EF.pdf}
    \caption{ }
    \label{fig3a}
  \end{subfigure}
  \begin{subfigure}[b]{0.3\linewidth}
    \centering
    \includegraphics[width=\linewidth]{Photos/WholeOnSeconds_Accuracy_DF.pdf}
    \caption{ }
    \label{fig3b}
  \end{subfigure}\\
  \begin{subfigure}[b]{0.3\linewidth}
    \centering
    \includegraphics[width=\linewidth]{Photos/WholeOnSeconds_Recall_EF.pdf}
    \caption{ }
    \label{fig3c}
  \end{subfigure}
  \begin{subfigure}[b]{0.3\linewidth}
    \centering
    \includegraphics[width=\linewidth]{Photos/WholeOnSeconds_Recall_DF.pdf}
    \caption{ }
    \label{fig3d}
  \end{subfigure}
    
  \caption{Classifier performance evaluated using eye-tracking metrics segmented into intervals of 3, 5, and 10 seconds with a 1-second sliding window: a) Accuracy in distinguishing between non-failure and executional failure, b) Accuracy in distinguishing between non-failure and decisional failure, c) Recall in detecting executional failure, and d) Recall in detecting decisional failure.}
  \label{fig3}
\end{figure*}




To enable the robot to detect its mistakes in real-life scenarios, it needs to repeatedly check at regular intervals whether something has gone wrong. In this section, we aim to address both research questions. In addition to the machine learning models used in the previous section, we also include SVM and CatBoost here. 


The models were trained on the entire duration of non-failure and failure periods and evaluated by segmenting the eye-tracking metrics data into intervals of 3, 5, and 10 seconds, using a sliding window of 1 second. Figures \ref{fig3a}, \ref{fig3b}, \ref{fig3c}, and \ref{fig3d} illustrate the performance of the models in distinguishing between NF and EF and between NF and DF.

For both NF vs. EF and NF vs. DF, Random Forest achieved the highest accuracy compared to other classifiers, with an accuracy of approximately 60\%. In contrast, SVM had the lowest accuracy, below 50\%. However, SVM was the most effective in detecting the highest number of failures for both failure types. 


% \begin{figure*}[h]
%   \centering
  
%   % Fourth Row
%   \begin{subfigure}[a]{0.32\linewidth}
%     \centering
%     \includegraphics[width=\linewidth]{Photos/Failure detection percentage_EF5.pdf}
%     \caption{ }
%     \label{fig3e}
%   \end{subfigure}
%   \begin{subfigure}[b]{0.32\linewidth}
%     \centering
%     \includegraphics[width=\linewidth]{Photos/Failure detection percentage_PF5.pdf}
%     \caption{ }
%     \label{fig3f}
%   \end{subfigure}
  
%   \caption{Percentage of users for whom the model successfully detected failures during each 5-second interval of the failure phase: a) Executional Failure, and b) Decisional Failure.}
%   \label{fig3}
% \end{figure*}


\begin{figure*}[h!]
  \centering
  
% Second Row
  \begin{subfigure}[b]{0.3\linewidth}
    \centering
    \includegraphics[width=\linewidth]{Photos/Failuredetectionpercentage_EF5.pdf}
    \caption{ }
    \label{fig3e}
  \end{subfigure}
  \begin{subfigure}[b]{0.3\linewidth}
    \centering
    \includegraphics[width=\linewidth]{Photos/Failuredetectionpercentage_DF5.pdf}
    \caption{ }
    \label{fig3f}
  \end{subfigure}
    
  \caption{Percentage of users for whom the model successfully detected failures during each 5-second interval of the failure phase: a) Executional Failure, and b) Decisional Failure.}
  \label{fig3}
\end{figure*}



% \begin{figure}[H]
%   \centering
%   \begin{subfigure}[t]{0.48\linewidth}
%     \centering
%     \includegraphics[width=\linewidth]{Photos/Failure detection percentage_EF5.pdf}
%    \label{fig3e}
%   \end{subfigure}
%   \hfill
%   \begin{subfigure}[t]{0.48\linewidth}
%     \centering
%     \includegraphics[width=\linewidth]{Photos/Failure detection percentage_PF5.pdf}

%     \label{fig3f}
%   \end{subfigure}
%   \caption{Percentage of users for whom the model successfully detected failures during each 5-second interval of the failure phase: a) Executional Failure, and b) Decisional Failure.}
%   \label{fig3}
% \end{figure}



Additionally, we calculated the percentage of users, during each 3-, 5-, and 10-second interval of the failure phase, for whom the model successfully detected the failure. Figure \ref{fig3e} shows these percentages for EF at the 5-second interval, and Figure \ref{fig3f} shows them for DF at the same interval.


The results showed that, for EF, the models were most accurate at detecting user reactions during the period from 4 to 7 seconds for the 3-second interval, and from 3 to 8 seconds for the 5-second interval after a failure began. Similarly, for DF, the optimal detection period occurred from 2 to 5 seconds for the 3-second interval, and from 1 to 6 seconds for the 5-second interval after a failure began.

