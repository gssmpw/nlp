Demographic bias in face recognition (FR) has emerged as a critical area of
research, given its impact on fairness, equity, and reliability across diverse
applications. As FR technologies are increasingly deployed globally, disparities
in performance across demographic groups-- such as race, ethnicity, and gender--
have garnered significant attention. These biases not only compromise the
credibility of FR systems but also raise ethical concerns, especially when these
technologies are employed in sensitive domains. This review consolidates
extensive research efforts providing a comprehensive overview of the
multifaceted aspects of demographic bias in FR.

We systematically examine the primary causes, datasets, assessment metrics, and
mitigation approaches associated with demographic disparities in FR. By
categorizing key contributions in these areas, this work provides a structured
approach to understanding and addressing the complexity of this issue. Finally,
we highlight current advancements and identify emerging challenges that need
further investigation. This article aims to provide researchers with a unified
perspective on the state-of-the-art while emphasizing the critical need for
equitable and trustworthy FR systems. 


