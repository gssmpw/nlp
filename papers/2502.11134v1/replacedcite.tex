\section{Related Work}
%在这部分添加引用 再次强调和引用的问题相比 我们问题的特点
Resource scheduling and optimization problems, pervasive and fundamental issues in complex system, have been extensively studied from both theoretical and empirical perspectives ____. There are numerous studies demonstrating effective use of DRL in several real-world resource scheduling scenarios, such as smart manufacturing ____, city-wide firefighting ____, steel production ____, resource provisioning in Internet of Things ecosystem ____, logistics and retail ____. Existing studies demonstrate that RL exhibits effectiveness in terms of the solution quality, and can achieve substantial time savings compared to the classical heuristic approaches ____. Therefore, DRL is extensively investigated as an effective approach for controlling complex systems.


Despite the clear need, there is an absence of research undertaken in the area of intelligent astronomical observation resource allocation for telescope array. Nowadays, resource allocation and management methods in most astronomical survey observation projects can be divided into two types, based on ILP algorithms ____ and human-generated heuristics ____. Nevertheless, with the increase of the number of telescopes and observation targets, especially in the environment of telescope array observation, fine-scale observation strategy optimization requires extensive manual intervention, which exceeds the ability of conventional planning algorithms and classical solvers. For the observation environment using a distributed telescope array, a flexible multilevel global scheduling model is proposed for a generic telescope array scheduling problem by Zhang et al. ____. While their algorithm produces long-term scheduling solutions in survey observation mode, it does not undertake precise resource coordination for follow-up observations of ToOs. Jia et al. implements a telescope array observation simulator and applies DRL into a space debris observation scenario ____. But since publication, there is currently no established general approach for resource management in online follow-up astronomical observation using an array of multiple telescopes.

The follow-up observation scheduling problem in astronomy can be seen as a special resource-constrained project scheduling problems (RCPSP ____). In order to achieve the robust scheduling ____, researchers propose multiple heuristic and meta-heuristic procedures to allocate time buffers in a given schedule while ensuring adherence to a predefined project due date ____. By contrast, inserting time buffers in a proactive way to deal with scheduling uncertainties is not in line with the principle of telescopic observation, because the telescope is expensive and has limited life, observation time is a very valuable resource. Li et al. develop efficient approximate dynamic programming (ADP) algorithms for RCPSP with uncertain task duration, using constraint programming and a hybrid ADP framework to enhance performance and efficiency ____. Brvcic et al. address the issue of inflexibility in proactive–reactive scheduling by introducing threshold-based cost functions for deviation penalties in projects with stochastic task duration ____. While Xie et al. focus on RCPSP with uncertain resource availability ____, they use a new Markov decision process model and a rollout-based ADP algorithm, significantly improving performance over heuristic methods. Compared with these traditional problems, the problem of RCPSP in time-domain survey to be solved in this paper focuses more on fast processing of special targets to ensure their observation quality, rather than maintaining the stability of the original tasks. 


With the development of deep learning technology in recent years, one research ____ presents an example solution that transforms the problem of packing tasks with diverse resource demands into a learning problem. The resource allocation strategies are directly learned from experience. However, it only considers a single-cluster situation, and factors such as the dependency between jobs have not been investigated. Another research ____ relies on the graph neural network to address RCPSP of varying sizes, including in presence of uncertain task duration. Cai et al. further solves the RCPSP with resource disruptions, and uses proximal policy optimization (PPO) to train the model in an end-to-end way for performance optimization ____. Their work is relevant for us, but for the follow-up observation scenario in astronomical domain, scheduling strategies should consider real locations of telescopes, distributions of observation targets, and filter requirements. The constrained observation conditions are closely related to these time-varying factors. Therefore, the current industrial scheduling methods are difficult to be directly applied in the field of astronomical observation.

In other recent work, the local rewriting is proposed for combinatorial optimization ____, the performance is assessed across three distinct domains: online job scheduling, expression simplification, and vehicle routing. It has shown better performance than heuristics using multiple metrics in solving complex problems where generating an entire solution directly is challenging. Given its effectiveness in capturing hierarchical and sequential structures, and order constraints ____, the Child-Sum Tree-LSTM architecture is well-suited for the dynamic and complex nature of resource scheduling in astronomical observations. So it is clear that despite a lack of exploration into a general intelligent resource management approach in the astronomical observation domain, the existence of mature and extensive research supports the exploration as a feasible approach for tackling the application challenges.