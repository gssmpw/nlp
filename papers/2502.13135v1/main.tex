% This must be in the first 5 lines to tell arXiv to use pdfLaTeX, which is strongly recommended.
\pdfoutput=1
% In particular, the hyperref package requires pdfLaTeX in order to break URLs across lines.

\documentclass[11pt]{article}

% Change "review" to "final" to generate the final (sometimes called camera-ready) version.
% Change to "preprint" to generate a non-anonymous version with page numbers.
\usepackage[preprint]{acl}

% Standard package includes
\usepackage{times}
\usepackage{latexsym}

% For proper rendering and hyphenation of words containing Latin characters (including in bib files)
\usepackage[T1]{fontenc}
% For Vietnamese characters
% \usepackage[T5]{fontenc}
% See https://www.latex-project.org/help/documentation/encguide.pdf for other character sets

% This assumes your files are encoded as UTF8
\usepackage[utf8]{inputenc}

% This is not strictly necessary, and may be commented out,
% but it will improve the layout of the manuscript,
% and will typically save some space.
\usepackage{microtype}

% This is also not strictly necessary, and may be commented out.
% However, it will improve the aesthetics of text in
% the typewriter font.
\usepackage{inconsolata}

%Including images in your LaTeX document requires adding
%additional package(s)
\usepackage{graphicx}

% Better reference
\usepackage[capitalize,nameinlink,noabbrev]{cleveref}

% Better enumerate
\usepackage{enumitem}

% Appendix style
\usepackage[title]{appendix}

% If the title and author information does not fit in the area allocated, uncomment the following
%
%\setlength\titlebox{<dim>}
%
% and set <dim> to something 5cm or larger.


%% Colors %%
% ------------------------------------------------------- [ Colour Definitions ]

\definecolor{cb-black}      {RGB}{  0,   0,   0}
\definecolor{cb-blue-green} {RGB}{  0,  073,  073}
\definecolor{cb-green-sea}  {RGB}{  0, 146, 146}
\definecolor{cb-rose}       {RGB}{255, 109, 182}
\definecolor{cb-salmon-pink}{RGB}{255, 182, 119}
\definecolor{cb-purple}     {RGB}{ 73,   0, 146}
\definecolor{cb-blue}       {RGB}{ 0, 109, 219}
\definecolor{cb-lilac}      {RGB}{182, 109, 255}
\definecolor{cb-blue-sky}   {RGB}{109, 182, 255}
\definecolor{cb-blue-light} {RGB}{182, 219, 255}
\definecolor{cb-burgundy}   {RGB}{146,   0,   0}
\definecolor{cb-brown}      {RGB}{146,  73,   0}
\definecolor{cb-clay}       {RGB}{219, 209,   0}
\definecolor{cb-green-lime} {RGB}{ 36, 255,  36}
\definecolor{cb-yellow}     {RGB}{255, 255, 109}

% -------------------------------------------------------------- [ Usage Macro ]

\newcommand{\cbBlack}      [1]{\textcolor{cb-black}      {#1}}
\newcommand{\cbBlueGreen}  [1]{\textcolor{cb-blue-green} {#1}}
\newcommand{\cbGreenSea}   [1]{\textcolor{cb-green-sea}  {#1}}
\newcommand{\cbRose}       [1]{\textcolor{cb-rose}       {#1}}
\newcommand{\cbSalmonPink} [1]{\textcolor{cb-salmon-pink}{#1}}
\newcommand{\cbPurple}     [1]{\textcolor{cb-purple}     {#1}}
\newcommand{\cbBlue}       [1]{\textcolor{cb-blue}       {#1}}
\newcommand{\cbLilac}      [1]{\textcolor{cb-lilac}      {#1}}
\newcommand{\cbBlueSky}    [1]{\textcolor{cb-blue-sky}   {#1}}
\newcommand{\cbBlueLight}  [1]{\textcolor{cb-blue-light} {#1}}
\newcommand{\cbBurgundy}   [1]{\textcolor{cb-burgundy}   {#1}}
\newcommand{\cbBrown}      [1]{\textcolor{cb-brown}      {#1}}
\newcommand{\cbClay}       [1]{\textcolor{cb-clay}       {#1}}
\newcommand{\cbGreenLime}  [1]{\textcolor{cb-green-lime} {#1}}
\newcommand{\cbYellow}     [1]{\textcolor{cb-yellow}     {#1}}
%%%%%%%%%%%%

% Make the title box bigger to fit all authors
\setlength\titlebox{13\baselineskip}

\title{Sleepless Nights, Sugary Days:\\Creating Synthetic Users with Health Conditions\\for Realistic Coaching Agent Interactions}

% Author information can be set in various styles:
% For several authors from the same institution:
% \author{Author 1 \and ... \and Author n \\
%         Address line \\ ... \\ Address line}
% if the names do not fit well on one line use
%         Author 1 \\ {\bf Author 2} \\ ... \\ {\bf Author n} \\
% For authors from different institutions:
% \author{Author 1 \\ Address line \\  ... \\ Address line
%         \And  ... \And
%         Author n \\ Address line \\ ... \\ Address line}
% To start a separate ``row'' of authors use \AND, as in
% \author{Author 1 \\ Address line \\  ... \\ Address line
%         \AND
%         Author 2 \\ Address line \\ ... \\ Address line \And
%         Author 3 \\ Address line \\ ... \\ Address line}

% \author{First Author \\
%   Affiliation / Address line 1 \\
%   Affiliation / Address line 2 \\
%   Affiliation / Address line 3 \\
%   \texttt{email@domain} \\\And
%   Second Author \\
%   Affiliation / Address line 1 \\
%   Affiliation / Address line 2 \\
%   Affiliation / Address line 3 \\
%   \texttt{email@domain} \\}

\author{
\textbf{
 Taedong Yun\textsuperscript{1,\textdagger},
 Eric Yang\textsuperscript{2},
 Mustafa Safdari\textsuperscript{1},
 Jong Ha Lee\textsuperscript{2},
}
\\
\textbf{
 Vaishnavi Vinod Kumar\textsuperscript{3},
 S. Sara Mahdavi\textsuperscript{1},
 Jonathan Amar\textsuperscript{2},
 Derek Peyton\textsuperscript{3},
}
\\
\textbf{
 Reut Aharony\textsuperscript{1},
 Andreas Michaelides\textsuperscript{3},
 Logan Schneider\textsuperscript{3},
 Isaac Galatzer-Levy\textsuperscript{3},
}
\\
\textbf{
 Yugang Jia\textsuperscript{2},
 John Canny\textsuperscript{1},
 Arthur Gretton\textsuperscript{1,*,\textdagger},
 Maja Matari\'c\textsuperscript{1,*,\textdagger}
}
\\
 \textsuperscript{1}Google DeepMind \phantom{ }
 \textsuperscript{2}Verily Life Sciences \phantom{ }
 \textsuperscript{3}Google
\\
 \small{\textsuperscript{*}Co-last authors (listed alphabetically)}
\\
 \small{
 \textsuperscript{\textdagger}\texttt{\{tedyun,gretton,majamataric\}@google.com}
 }
}

\begin{document}
\maketitle

%% Custom macros by TY.
% General comments for all. ON/OFF.
\newcommand{\cmt}[1]{\textcolor{red}{[#1]}}  % ON
% \newcommand\cmt[1]{}  % OFF


\begin{abstract}
We present an end-to-end framework for generating synthetic users for evaluating interactive agents designed to encourage positive behavior changes, such as in health and lifestyle coaching. The synthetic users are grounded in health and lifestyle conditions, specifically sleep and diabetes management in this study, to ensure realistic interactions with the health coaching agent. Synthetic users are created in two stages: first, structured data are generated grounded in real-world health and lifestyle factors in addition to basic demographics and behavioral attributes; second, full profiles of the synthetic users are developed conditioned on the structured data. Interactions between synthetic users and the coaching agent are simulated using generative agent-based models such as Concordia, or directly by prompting a language model. Using two independently-developed agents for sleep and diabetes coaching as case studies, the validity of this framework is demonstrated by analyzing the coaching agent's understanding of the synthetic users' needs and challenges. Finally, through multiple blinded evaluations of user-coach interactions by human experts, we demonstrate that our synthetic users with health and behavioral attributes more accurately portray real human users with the same attributes, compared to generic synthetic users not grounded in such attributes. The proposed framework lays the foundation for efficient development of conversational agents through extensive, realistic, and grounded simulated interactions.
\end{abstract}

\section{Introduction}
\label{sec:intro}


Personalization is a key capability for any interactive agents aiming to encourage positive behavior changes, for instance in health and lifestyle \citep{christakopoulou2024agentsthinkingfastslow,yang2024barrierstacticsbehavioralscienceinformed}. The efficacy of such agents must be assessed via interactions with users; however, collecting and evaluating diverse, long-term human interactions with agents is costly and time-consuming, and the development of such agents can be significantly accelerated with realistic simulation. We address the challenge of realistic user simulation in this context by proposing an end-to-end framework for generating cohorts of synthetic users grounded in health, lifestyle, and behavior, and having them interact with coaching agents. The framework is generally applicable across a range of health and lifestyle domains; we validate it in the domains of sleep and diabetes coaching.

Synthetic users generated with large language models (LLMs) have been widely studied in recent years \citep{kapania2024simulacrumstoriesexamininglarge,wang2025opencharactertrainingcustomizableroleplaying,moon2024virtual,park2024generativeagentsimulations1000,joonsung_simulacra}, often for general purpose alignment and personalization \citep{castricato2024personareproducibletestbedpluralistic}, and have been designed to reflect demographic information about the population at large. 
We focus on a more targeted application, where the synthetic users are employed for interaction with foundation models or agents specialized in health, as done in recent work  \cite{tu2024conversationaldiagnosticai,yu2024aipatientsimulatingpatientsehrs,Johri2025-kp}. We provide an overview of such work in \cref{sec:background}.

We demonstrate an end-to-end-framework for designing a cohort of synthetic users exhibiting health conditions consistent with their specific demographics and behavioral factors, in order for their interaction with a coaching agent to realistically reflect their needs and challenges. We construct the synthetic users by first generating a sample of natural language "vignettes", which are in turn generated based on real demographics, health, and behavioral data  \citep[such as the Big Five markers;][]{goldberg1992development,serapiogarcia2023personalitytraitslargelanguage}. Our framework also allows for optionally including additional information:  domain-specific rubrics such as ``barriers'' and ``goals'' as used by human coaches; challenges to action derived from the COM-B behavioural model \citep{Michie2011-gp}; or rich backstories for the synthetic users, generated consistently with the structured attributes from real data.
%For the case of sleep improvement, various studies (e.g., on railroad workers, pregnant women)  provide sleep data for user demographics which is grounded in the scientific literature on sleep disorders, and which can be weighted with specific objectives in mind (for instance, inclusion of under-represented groups). 
Finally, based on this cohort of vignettes, realistic user-coach simulation can be performed by direct LLM calls or by using generative agent-based models such as the Concordia \citep{Vezhnevets2023-vq} open-source system. More details will be discussed in \cref{sec:methods}.

We demonstrate the efficacy of our methodology for generating health-grounded synthetic personas in two separate domains, sleep coaching (\cref{sec:sleepCoaching}) and diabetes coaching (\cref{sec:diabetesCoaching}), using two independently developed sleep coaching and diabetes coaching agents. We verify that our synthetic users indeed communicate behaviors, health conditions, and barriers consistent with their assigned attributes, by inspecting the internal state-of-user model of the coaching agent for each synthetic user, by a holistic evaluation of the user-coach interactions by trained human experts, and by a comparative evaluation of our health-grounded synthetic users against generic synthetic users, again by trained human experts.
\iffalse 
Finally, we demonstrate the utility of the synthetic users generated by our method in evaluating coaching agent performance, by deliberately engineering shortcomings in agent behavior (e.g., suggesting irrelevant courses of action, repeating questions). The synthetic users successfully reveal these shortcomings via feedback in user experience questionnaires.

%Can we do: the interactions were evaluated by surveying the patient personas, and by expert annotation of the dialogues.
%In simulation: does the coach build trust and engagement? Do the synthetic users express that they feel supported? 

%Control and consistency of patients, ability to zero in on particular health issues.

%We want to ensure that the addition of personality traits does not compromise or alter the sleep disorder asigned to the persona, nor its demographic attributes.


%To say: feedback can be achieved both by querying the persona after the interaction, and by external evaluation of the agent-coach dialogue, in order to assess.... % knowledgebase validity, QA accuracy, readability, robustness, and stability

On personalization: %We may also distinguish between zero-shot prompting, which elicits the models world knowledge, against few-shot prompting, which biases model response. 



\fi 



% The work of \cite{yang2024barrierstacticsbehavioralscienceinformed} addresses the task of providing scalable and personalized nutrition coaching for cardiometabolic patients. The agent achieves personalization by first identifying underlying {\em barriers} to a healthy diet, and then delivering tailored {\em tactics} addressing these causes. A total of 28 barriers were identified, as drawn from the scientific literature on cardiometabolic conditions, and the tactics were drawn from the relevant behavioural science literature (including the COM-B model, the BCT Taxonomy, and the EAST framework). Gemini 1.5 Pro is used in the design of the diet coach.

\iffalse
The work of \cite{yang2024barrierstacticsbehavioralscienceinformed} employed a patient simulator to assess the behaviour of a nutrition coach. For each of 28 barriers, an LLM-generated patient vignette was created based on the lifestyles and medical histories of the 16 users who participated in a user study. The quality of the nutrition coach was evaluated by human experts and OpenAI’s GPT-4o. A total of 153 high quality vignettes were obtained.

This study focused on the nutrition coach's ability to support the 28 barriers. With respect to coverage of demographics, clinical traits and personality types, however, the study was more limited. The simulations were derived from a 16-user study, and as such had limited demographic and clinical scope. Due to the small sample size, the distributions and joint relationships between the demographic, clinical traits and barriers were unable to be derived in a representative manner.  Finally, the evaluation focused on the effectiveness of the nutrition coach while the patient simulator's performance was not rigorously assessed.
\fi


\section{Background}
\label{sec:background}

In this section, we briefly survey the development and goals of LLM-based synthetic personas in general, and then provide further details on prior literature on the design and application of synthetic personas for health. We next describe software systems for simulating agent interactions. Finally, we discuss some challenges in using LLM-derived synthetic users for agent development and beyond.

\subsection{LLMs as Synthetic Personas}

The time and expense required in human evaluation of LLM agents naturally evokes the need for auto-evaluation. This has led to the development of ``reinforcement learning from AI feedback (RLAIF)'', where LLMs serve to evaluate the output of other LLMs, in  a now  well-established strategy for ensuring alignment, factuality, helpfulness, and other desiderata  \citep{bai2022constitutionalaiharmlessnessai,lee2024rlaifvsrlhfscaling}.

One recent approach to RLAIF uses  LLMs to construct synthetic personas and to judge their interaction with the model being evaluated.
\citet{castricato2024personareproducibletestbedpluralistic} generated these synthetic personas 
from the US census data, and employed them to build an evaluation dataset of prompt and feedback pairs obtained solely from synthetic users. A strength of the approach is that results for specific users can be modeled, rather than general user categories. This approach also motivates the work of \citet{moon2024virtual}.

In addition to demographic information, \citet{serapiogarcia2023personalitytraitslargelanguage} showed that LLMs can be designed to display specific personality traits, such as those along the Big Five dimensions \citep{goldberg1992development}.
Effective modeling and control of personality in LLMs is important for two reasons: first
since personality is a key factor in determining effective communication; second, because it is important to model interaction of synthetic users over the broad range of personality profiles to be encountered in practice. \citet{serapiogarcia2023personalitytraitslargelanguage} found that instruction-tuned models displayed more reliable and externally valid personality types 
than pre-trained variants (perhaps due to superior instruction following), and that
larger models are better equipped to express complex traits, and to simulate social behaviours. 

%Tailored self-play has been used in improving performance in LLMs \citep{fu2023improvinglanguagemodelnegotiation}.

%LLM generation suffers from a number of challenges. One is failure to enforce consistency across long stories \cite{yang-etal-2023-doc}.

\subsection{LLMs and Synthetic Personas in Health}

Several works have proposed the use of LLMs to provide advice in health or wellness settings.

\citet{tu2024conversationaldiagnosticai} introduced Articulate Medical Intelligence Explorer (AMIE),  an LLM agent designed for conversational medical diagnosis. The AMIE agent achieved remarkable success, with greater diagnostic accuracy compared to primary care physicians in a randomized, double-blind crossover study.
%; efficient acquisition of patient symptoms through multi-round questioning, and communications rated by users as more empathetic and better structured than those of clinicians. 
AMIE was trained using self-play with simulated doctor-patient dialogues and was tested with patients simulated first by searching for medical conditions from across three databases (Health QA, MalaCards Human Disease, MedicineNet Diseases \& Conditions), and then by performing internet searches on each medical condition in order to retrieve passages on demographics, symptoms, and management, which were screened for relevance. PaLM 2 \citep{anil2023palm2technicalreport} was used as the base model for all agents, including the patients and doctors, which were given role-specific prompts.
Instruction fine tuning of both patient and doctor took place, first using static datasets of real doctor-patient interactions, and subsequently using dialogues generated through self-play. A moderator agent is used to manage the dialogue between doctor and patient, and to decide on when the conversation has ended. 

There were a number of limitations to the synthetic user (patient) design in the AMIE study. First, the retrieval of symptoms and demographics from online discussion might create bias towards demographics of individuals most likely to engage in online discussion of their condition. Second, online discussion might misattribute irrelevant symptoms to conditions, and be misinformed as to mitigation strategies. Third, the model did not explicitly control for personality traits, relying on the distribution of personality traits to be found in online discussion of the medical conditions, which might not be representative of target demographics. Related to this issue, PaLM 2 has undergone a fine-tuning process which may make the patient agents more obliging, sympathetic, and communicative than real patients.

\iffalse
% my further notes

%AG: I remove this limitation since we now use BOTH concordia AND a general-purpose LLM!

 On the other hand, our work uses Concordia \citep{Vezhnevets2023-vq} to manage the interaction between the health coach and the synthetic persona, rather than prompting a general-purpose LLM for this role.  

Datasets for medical conditions: p. 6
Health QA dataset [12] which contained 613 common medical conditions.
• MalaCards Human Disease Database1 which contained 18,455 less common disease conditions. • MedicineNet Diseases & Conditions Index2 which contained 4,617 less common conditions.

Four agents: doctor, patient, moderator, critic. Amie plays all of these.  Moderator role played by Concordia. We don't use same structure for doctor as for patient.

\fi

% \cmt{DUPLICATED: the next 2 paragraphs are also mentioned in section 1 Intro} The work of \cite{yang2024barrierstacticsbehavioralscienceinformed}  employed a patient simulator to assess the behaviour of a nutrition coach. For each of 28 barriers, an LLM-generated patient vignette was created based on the lifestyles and medical histories of the 16 users who participated in a user study. The quality of the vignette in respect of its representation of the barrier was evaluated by OpenAI’s GPT-4o. A total of 153 high quality vignettes were obtained.

% This study focused on the nutrition coach's ability to support the 28 barriers. With respect of coverage of demographics, clinical traits and personality types, however, the study was more limited. The simulations were derived from a 16-user study, and as such had limited demographic and clinical scope. Due to the small sample size, the distributions and joint relationships between the demographic, clinical traits and barriers were unable to be derived in a representative manner.  Finally, the expert evaluation focused on the effectiveness of the nutrition coach while the patient simulator's performance was not rigorously assessed.


\citet{yu2024aipatientsimulatingpatientsehrs} also described an application of LLMs to simulate user (patient) populations, based on particular medical histories from 1495 patients in the MIMIC-III database \citep{Johnson2016-eg}. The approach represented each patient's medical and demographic status as a knowledge graph (with 15,000 possible nodes and over 26,000 possible edges), then retrieved patient information from this graph in response to doctor queries. The retrieved answers were then expressed in natural language, with tone adapted in accordance with the Big Five personality traits. The key finding was that ``patient LLMs'' that retrieve from a structured representation of patient traits more accurately communicate symptoms and history than simply using unstructured clinical notes in the prompt. 

Our work differs in a number of important ways. First, the representation of each individual as a knowledge graph is suited to clinical settings, where unstructured clinical notes must be organized as a pre-processing step. Accordingly, much of the focus of the paper is on the construction of the graph, and accurate retrieval from it. The graph representation may be less suited to wellness settings, however, where the user's needs and challenges require less structure and granularity.  Second, and more importantly, the generation process can only traverse the graph of existing patients, and cannot generate new individuals. This may lead to limitations with respect to both privacy and scalability. 


\iffalse
%AG: my further notes
This works by using a Reasoning Retrieval-Augmented Generation (Reasoning RAG) workflow and a knowledge graph derived from the  (MIMIC)-III database.
This works as follows:
First develop a knowledge graph from 1495 patient records
Very large number of nodes (over 15,000) and edges
containing various medical entities such as symptoms, medical history, vitals, allergies, social history, and family medical history, extracted from discharge summaries
Six agents are used in accessing information from this knowledge graph.
Two agents query the graph
Two agents check that the query results align with the original question
Rewrite agent takes into account big 5 personality traits.

without reasoning RAG and the knowledge graph, performance dropped significantly.
\fi

%Aims: 
% make sure to give a faithful account of persona properties
% give a realistic dialogue: don't just output a list of the goals, blockers, etc.
%  --- relating to the above: in this case the sleeper agent also needs to ask clarifying questions.


\citet{Johri2025-kp} described the use of synthetic personas in evaluating an LLM system for patient history collection in medicine. The patient agent was instructed explicitly not to display medical knowledge or generate new symptoms beyond what was given in the ``vignette''.
A total of 2,000 case vignettes were considered: 1,800 questions from MedQA-United States Medical Licensing Examination covering 12 medical specialties, and 200 questions on skin disease from the Derm-Public and Derm-Private question banks.
While representing a broad range of conditions, the focus of that study was primarily on evaluating information gathering and diagnosis abilities of LLMs, rather than coaching. Unlike our work, \citeauthor{Johri2025-kp} did not attempt to ensure that patient profiles were representative of the population at large, nor were personality types modeled when generating patient dialogues.

\begin{figure*}[t]
  \centering
  \includegraphics[width=\textwidth]{figs/method_overview.pdf}
  \caption{An overview of generating synthetic users grounded in real demographics, health \& lifestyle, and behavioral \& psychological characteristics for automated evaluation of coaching agent interactions. GABM: generative agent-based model.}
  \label{fig:overview}
  \vspace{-1.4em}
\end{figure*}

\subsection{Generative Agent-Based Models}
\label{background:concordia}
Generative agent-based models, such as the Concordia system which we used \citep{Vezhnevets2023-vq}, are directly relevant to user persona simulation. Concordia offers a framework for generating agent interactions, particularly dialog-based ones, given a sufficient backstory. Its design incorporates several key features beneficial for research in generative agent-based simulation. First, Concordia employs associative memory \cite{joonsung_simulacra} and chain-of-thought reasoning \cite{wei_reasoning2022} to ensure that agent utterances are grounded in the provided backstory. Second, it utilizes metadata tags within prompts to clearly delineate different contextual elements, including agent memory, observations, prior conversation turns, and overarching goals. Furthermore, these contextual components can be assigned varying levels of importance, guiding the underlying LLM to generate subsequent dialog using appropriate chain-of-thought reasoning. Third, Concordia provides extensive logging capabilities, enabling detailed examination of the pipeline's internal processes, such as the constructed prompts sent to the LLM, the chain-of-thought reasoning process, post-processing steps, and the final output. Fourth, its modular architecture allows for flexible configuration, enabling researchers to customize the agent pipeline by swapping components.

A crucial requirement for generative agent-based simulation frameworks is the ability to integrate and manage multiple, mutable states within the interaction environment, as these states can significantly influence an agent's subsequent actions and utterances.  Critically, the mutation of these states, often lacking a true real-world simulation, must also be driven by chain-of-thought reasoning. Concordia addresses this need by providing code abstractions that facilitate the integration and management of these dynamic environmental states and their reasoned evolution.


\subsection{Challenges of Using LLM-Derived Synthetic Personas}


As discussed by \citet{kapania2024simulacrumstoriesexamininglarge}, there are a number of challenges in using LLMs as synthetic personas (users). LLMs are a consolidation over human experience, and hence they combine multiple viewpoints, failing to represent the diversity of human experience or correlations between different aspects of individuals. They can also lack depth: for instance, they can make references to particular phenomena (e.g., sleep difficulties in our test domain), but these phenomena do not represent situated knowledge, grounded in lived experience and history. This makes it difficult to model human users, since the relevant reasons for the health conditions might be missing. We mitigate this issue with the option of adding backstories, which have been shown to provide grounding to LLM-generated output \citep{moon2024virtual}. 

Training data may also be biased as they were drawn largely from English-speaking cultures and individuals with strong online presence. To demonstrate the issue in the domain of sleep disorders,  a survey of 27,000 US adults revealed that difficulty in falling and in staying asleep occurs more commonly among residents in non-metropolitan areas, and for those on lower incomes \citep{Adjaye2022sleepDifficulties}. We address this bias by specifying persona statistics drawn from relevant demographic health studies  \citep[e.g.,][]{Yfantidou2022-ay,Arges2020-zh}, rather than relying on generation by the LLM. 

%, and may not be well suited to non-western cultures, or populations with less online engagement (such as the elderly). 

%These challenges are particularly acute in the medical and wellness setting, where the incidence of medical issues in online discussions might not be representative of the population at large. For example, according to the US National Health Interview Survey of 27,000 adults in 2020, difficulty in falling and in staying asleep occurs more commonly among residents in nonmetropolitan areas, and those on lower incomes \citep{Adjaye2022sleepDifficulties}.

A further subtle risk of using LLMs in simulating synthetic users can be understood from a causal inference perspective, as discussed in \citet{Gui_2023}. Providing particular advice to an LLM (e.g., a recommendation regarding better sleep practices) can inadvertently cause unintended variation in other factors  (e.g., the LLM's beliefs about the age and lifestyle factors of the synthetic persona). We mitigate this effect through explicit control over demographic and personality factors that that the LLM might otherwise incorrectly impute, a practice recommended by \citet{Gui_2023}, although this remains an ongoing research challenge. 

\section{Methods}\label{sec:methods}

While LLMs trained on a large data corpus (e.g., the entire Internet) partially represent the general human population, the distributions of subpopulations and health conditions represented in a single model are skewed and poorly understood \citep{kapania2024simulacrumstoriesexamininglarge}. Furthermore, the instruction fine-tuning step (e.g., reinforcement learning with human feedback) typically employed by high-performance LLMs further impacts the distribution of generated outputs. In this work we propose an end-to-end framework for generating synthetic users grounded in reality, by starting from human user data including demographics, health \& lifestyle, and behavioral \& psychological characteristics, to accurately represent a desired population (\cref{fig:overview}). This real cohort can be  sampled either uniformly, or with oversampling or undersampling explicitly designed by researchers (e.g., to better surface underrepresented subgroups or conditions). Given this sampled cohort, we optionally add additional health conditions conditioned on already specified attributes of the synthetic users, either using the LLM or via a rule-based algorithm, and also optionally add rich backstories of the synthetic user, conditioned on existing attributes \citep{moon2024virtual}.

This combination of background information grounded in real data constitutes a ``vignette'' of the synthetic user, all based in natural language. In the following sections we explore how these vignettes can be used to generate interactions between synthetic users and coaching agents, in two separate  health coaching scenarios: sleep coaching and diabetes coaching, with two independently developed coaching agents.

\section{Experiments: Sleep Coaching}\label{sec:sleepCoaching}

Sleep is vital for health and well-being; insufficient sleep and untreated sleep disorders can have a detrimental effect on cognitive function, mood, mental health, and cardiovascular, cerebrovascular, and metabolic health \citep{Ramar2021-ql}.
Sleep disorders are highly prevalent: according to the 2020 National Health Interview Survey (NHIS) \cite{Adjaye2022sleepDifficulties},  14.5\% of US adults had trouble falling asleep most days over the 30-day study period, and over a quarter of adults do not meet the minimum recommended sleep duration per night.
% Lack of sleep can arise from medically diagnosed disorders, but it is far more prevalent than that caused by medical factors.

\subsection{Synthetic Users with Sleep Conditions}

To generate synthetic users grounded in real sleep conditions, we utilized the publicly available LifeSnaps dataset, a multi-modal, longitudinal, geographically-distributed dataset containing participant demographics, smartwatch measurements including sleep data, health, behavioral, and psychological trait surveys, and ecological momentary assessments \citep{Yfantidou2022-ay}. Among the fields available in this dataset, we focused on basic demographics (age, gender), basic health \& sleep attributes (body mass index, sleep duration \& efficiency), and five personality markers from the International Personality Item Pool (IPIP) version of the Big Five \citep{goldberg1992development}. Importantly, given  longitudinal sleep data over four months in LifeSnaps, we included both the average sleep duration and the variability of sleep duration, both important health factors. See \cref{appx:lifesnaps_detail} for more details.

As outlined in \cref{fig:overview}, we generated a synthetic ``sleep profile'' for each individuals in LifeSnaps ($N$=68) conditioned on real demographics, sleep duration \& efficiency, and behavioral characteristics. The generated sleep profile consists of the following four key attributes: ``primary sleep concern'', ``sleep goals'', ``reasons for goals'', and ``barriers''. The full prompt used to generate these attributes is available in \cref{appx:sleep_profile}. This sleep profile and all aforementioned structured attributes from LifeSnaps formed the ``vigenettes'' for our synthetic users with sleep conditions.

Finally, we instantiated each synthetic user in Concordia \citep{Vezhnevets2023-vq}, specifically as a ``SimpleLLMAgent'' to prioritize clarity and reproducibility, where the entire conversation history and the synthetic user's backstory are included within the prompt context for generating each subsequent utterance (\cref{appx:sleep_prompt}).

\begin{figure}[t]
  \centering
  \includegraphics[width=\columnwidth]{figs/true_vs_reasoner_sleep_profile.pdf}
  \caption{\textbf{Extracting synthetic users' sleep profile from interaction for automated evaluation of coaching agent}. Our synthetic users ($N$=68) with sleep conditions interacted with our sleep agent for 10 turns. Recall and precision were computed by comparing the internal state-of-user model in the coaching agent to the true sleep profile of the synthetic user. ``Primary sleep concern'' is a single natural language statement, while ``barriers'' and ``sleep goals'' include multiple items, all in natural language.}
  \label{fig:reasoner_sleep_profile}
  \vspace{-1.4em}
\end{figure}

\subsection{The Sleep Coaching Agent}

We used a sleep coaching agent based on \citet{christakopoulou2024agentsthinkingfastslow}, which proposed a two-agent system consisting of a ``Talker'' and a ``Reasoner'' agent. Inspired by \citet{kahneman2011thinking}, the Reasoner (``System 2'') is responsible for generating an internal model of the use based on conversation history, in addition to planning and calling tools, while the Talker  (``System 1'') is responsible for generating conversation based on the structured state generated by the Reasoner. We used Gemini 1.5 Pro \citep{geminiteam2024gemini15unlockingmultimodal} as the LLM engine for both constituent agents.

\subsection{Evaluating Agent-Synthetic User Interactions in Sleep Coaching}
\label{subsec:sleep_eval}

Explicit modeling of the user in the sleep coaching agent enabled us to compare the internal user state to the ``true'' sleep profile we  explicitly assigned to the synthetic user. After a 10-turn interaction between a synthetic user and the coaching agent, the coaching agent was able to identify the synthetic user's primary sleep concern with 89.7\% accuracy (\cref{fig:reasoner_sleep_profile}). For barriers and sleep goals, which are also explicitly modelled by the coach agent and consist of multiple items (natural language sentences) per user, we computed recall (sensitivity) and precision, as measured by the proportion of the true attributes also captured by the coaching agent's model, and the proportion of the attributes in the coaching agent's model that were present in the true attributes, respectively. When matching sentences, we used fuzzy-matching by a high-performance, instruction-tuned LLM (Gemini 1.5 Pro; \citet{geminiteam2024gemini15unlockingmultimodal}), to account for paraphrasing (e.g. ``inconsistent sleep duration'' and ``variable sleep duration''; \cref{appx:sleep_profile_fuzzy_matching}), and manually checked a random subset  (10 individuals) to make sure the answers were consistent with human answers (an easy task for modern LLMs). We obtained 71.4\% mean recall and 72.5\% mean precision for the barriers, and 66.4\% mean recall and 84.2\% mean precision for the sleep goals (\cref{fig:reasoner_sleep_profile}).

\begin{figure}[t]
  \centering
  \includegraphics[width=0.8\columnwidth]{figs/sleep_human_eval.pdf}
  \caption{\textbf{Expert evaluation of grounded synthetic users}. Human expert evaluation of interactions from two sets of synthetic users, one from the full synthetic user pipeline and another from demographics-only data without sleep conditions or behavioral/psychological characteristics, with a fixed sleep coaching agent. Five human experts annotated each case independently.}
  \label{fig:sleep_human_eval}
  \vspace{-1.4em}
\end{figure}

Finally, to validate our use of synthetic users with health and behavioral/psychological attributes in addition to basic demographic information, we conducted a human expert evaluation of interactions from two synthetic users: one from our full synthetic user pipeline and another from using demographics-only data without health conditions, with the same sleep coaching agent. We asked human evaluators, with training and quality control conducted by a clinical psychologist, to annotate their preference between two interactions with each synthetic user, with 5$\times$ coverage of each question. The evaluators overwhelmingly favored our full synthetic users over the baseline and the differences were highly statistically significant ($p\textrm{-value} = 3.7\times 10^{-12}$ by one-tailed test under binomial null distribution), with high inter-rater reliability (64\% of the cases achieving the complete 5/5 agreement and 91\% of the cases achieving 4/5 agreement or more) (\cref{fig:sleep_human_eval}).

\section{Experiments: Diabetes Coaching}\label{sec:diabetesCoaching}

Diabetes mellitus, commonly referred to as diabetes, is a highly prevalent chronic disease, affecting 15\% of all US adults in 2021 \citep{CDC2024-io}. Understanding lifestyle barriers is crucial for effectively addressing the challenges this condition presents \citep{Deslippe2023-oy}. The barriers significantly impact an individual's ability to adhere to recommended behaviors and achieve optimal health outcomes. Many barriers are intricately driven by demographic, socioeconomic, and clinical circumstances, collectively manifesting into specific challenges. By accurately representing these interconnections, we can create synthetic user profiles that faithfully reflect the real-world complexities encountered by individuals with cardiometabolic conditions, thereby enhancing the realism and applicability of our synthetic users interacting with a diabetes coaching agent (\cref{fig:overview}).

\subsection{Synthetic Users with Diabetes}

To obtain a representative distribution of cardiometabolic barriers, we leveraged  insights from 100 peer-reviewed articles in \citet{yang2024barrierstacticsbehavioralscienceinformed}, which comprehensively identified 246 challenges encountered by cardiometabolic patients, in consultation with behavioral experts. The challenges were classified into six distinct sub-categories within the COM-B (capability, opportunity, motivation - behavior) model, a widely recognized framework for understanding behavior in healthcare \citep{Michie2011-gp}. The challenges were again categorized into 21 distinct barrier concepts by behavioral experts, each belonging to a sub-category within the COM-B model, as shown in \cref{appx:comb}. Through this process, we ensured that the synthetic user profiles developed for simulated coaching agent interactions were representative, reflecting the appropriate distribution of challenges.

To derive our synthetic users with diabetes, we utilized Project Baseline Health Study (PBHS), a longitudinal cohort with diverse backgrounds representative of the entire health spectrum \citep{Arges2020-zh}.\footnote{The data use was formally approved by our institution's internal review committee, ensuring ethical compliance.} After careful preprocessing (\cref{appx:pbhs_diabetes_preprocessing}), we obtained a cohort of 345 Type 2 diabetic individuals with diverse demographic, social, medical, and health attributes (\cref{appx:pbhs_diabetes_data}).

To construct realistic vignettes of synthetic users from this cohort, we first sampled a COM-B category under its original distribution in PBHS (\cref{appx:comb_dist}), and then uniformly sampled a specific barrier within the selected COM-B category. Using this identified barrier, we randomly selected a corresponding individual from PBHS with relevant symptoms. The selected individual's demographic, socioeconomic, clinical, and behavioral survey data formed the foundation of our natural language vignette (\cref{fig:overview}). To enhance the narrative quality, realism, and depth of these vignettes, a coherent backstory was generated by an LLM, incorporating the identified barrier (\cref{appx:diabetes_nutrition_vignette_prompt}). Finally, we generated a communication style field (e.g. tone, verbosity, and confidence) to compose the final vignette. Notably, the specific technical term for the selected barrier was not included in the vignette, since human users are unlikely to label themselves under it. In total, 200 vignettes for synthetic users were generated using the commercially available Gemini 2.0 Flash model chosen for its reasoning and conversation capabilities \citep{Pichai_2024}.

\subsection{The Diabetes Coaching Agent}
The synthetic users generated from the vignettes interacted with a diabetes coaching agent, building upon methodologies developed by \citet{yang2024barrierstacticsbehavioralscienceinformed}. This coaching agent was constructed to assist users to set health goals and overcome identified barriers. After each simulated interaction, the coaching agent was asked to identify one of the 21 barriers in its modeling of the user. Both the synthetic users and the coaching agent were generated using Gemini 1.5 Pro \citep{geminiteam2024gemini15unlockingmultimodal}.

\subsection{Evaluating Agent-Synthetic User Interactions in Diabetes Coaching}
\label{subsec:diabetes_eval}

\begin{figure}[t]
  \centering
  \includegraphics[width=0.9\columnwidth]{figs/expert_assess_plot.pdf}
  \caption{\textbf{Assessing synthetic users' performance from interaction with the diabetes coaching agent}. Three human annotators reviewed 25 randomly-selected simulated interactions. Wilson's confidence interval was used for these binary annotation tasks.}
  \label{fig:cardio_exp_labels}
  \vspace{-1.5em}
\end{figure}

To assess the fidelity and reliability of the synthetic users, we solicited evaluations from a panel of three experts, comprising behavioral scientists and patient care practitioners (\cref{appx:diabetes_eval}). Each expert received  25 randomly-selected simulated interactions (out of 200) for assessment. Experts agreed that the synthetic users were highly consistent (92\%) and effectively demonstrated the barriers they were designed to portray (100\%) (\cref{fig:cardio_exp_labels}; \cref{appx:diabetes_eval_results}).

Similarly to our sleep coaching evaluation experiments (\cref{subsec:sleep_eval}), we also conducted a secondary experiment comparing our synthetic users to baseline synthetic users generated from general demographic data only, without the PBHS data and the COM-B barriers that our methodology uniquely provides, in their interactions with the same coaching agent. After randomly selecting an individual from PBHS, the baseline synthetic user was generated from their standard demographic information, while our full synthetic user had the same standard demographics, richer health data, and the individual's original barrier (\cref{appx:diabetes_nutrition_patient_agent_prompt}). Both synthetic users were asked to portray an individual with cardiometabolic health challenges, and we ensured both possessed the same standard demographic data, communication style, verbosity, and confidence, for a fair comparison.

Human annotators were assigned three evaluation questions (\cref{appx:diabetes_eval_comparison_qs}) to annotate randomly selected interactions without actual vignettes, testing implicit demonstration of the vignettes through conversations. 75 pairs of interactions (full synthetic users and baseline synthetic users) were evaluated (\cref{fig:diabetes_comparison}).

For consistent representation of a single barrier (correct or not), annotators preferred the interactions from our full synthetic users twice as much as the baseline (32\%:15\%). Moreover, the annotators overwhelmingly preferred the full synthetic user (70\%) for demonstrating the \emph{correct} original barrier. We observed mixed results for informativeness, with the baseline being slightly preferred and more than half of the interaction pairs annotated as ``Similar''. Overall, this analysis demonstrated that our synthetic users can more accurately portray the original barrier consistently in simulated interactions (\cref{fig:diabetes_comparison}), which has an important downstream effects. This implies our synthetic user pipeline can be used to generate training data for efficiently fine-tuning the coaching agent for a specific barrier for incremental improvement. On the other hand, simulated interactions with the baseline synthetic users would not reliably represent the specific barrier one seeks to better capture.

\begin{figure}[t]
  \centering
  \includegraphics[width=\columnwidth]{figs/diabetes_comparison_plot.pdf}
  \caption{\textbf{Comparing grounded synthetic users to a baseline on simulated diabetes coaching conversations.} Human annotators reviewed 75 pairs of conversations, and labeled which conversation demonstrated the dimensions represented in each row. ``Test'' represents our grounded synthetic users and ``baseline'' represents synthetic users created from basic demographics only.}
  \label{fig:diabetes_comparison}
  \vspace{-1.5em}
\end{figure}

Finally, we investigated the difference in distribution of barriers that the baseline synthetic users (using only demographic information) created, compared to the reference distribution observed in the literature (\cref{appx:barrier_butterfly}). As the baseline synthetic users only express barriers through conversation, we used the coaching agent's internal state (``diagnosis'') of the user's barrier established through interaction, which was evaluated as credible by human experts (\cref{fig:cardio_exp_labels}). Interestingly, the baseline synthetic users significantly over-sampled certain barriers (physical capability, reflective motivation), while under-sampling others (social opportunity, physical opportunity), highlighting the importance of grounding synthetic users in behavioral data from real human users whom the coaching agent is designed to serve.


\section{Conclusion}

We designed an end-to-end framework for generating synthetic users for evaluating coaching agents grounded in health, lifestyle, behavioral, and psychological attributes complementing basic demographics from human data. The additional sampling process was employed to ensure that the distribution of these attributes could be explicitly controlled (i.e., to ensure applicability to less represented sub-populations). Realistic backstories and additional health conditions (where not available in human user data) were conditionally generated from the grounded data. The final vignettes generated by this framework were used in a generative agent-based model framework to effectively simulate interactions between the synthetic users and the given coaching agent. We evaluated our methods in two independent, highly prevalent health coaching use cases of sleep coaching and diabetes management.

Efficient development of autonomous agents is significantly accelerated by evaluation metrics that can be computed \emph{without} complete dependence on human users, since human evaluation is costly and time-consuming.  While there is a large body of literature on the design of general-purpose synthetic users in this setting (see \cref{sec:background}), we have highlighted the importance of grounding synthetic users' health and behavioral attributes on a real human dataset, with concrete demonstrations and evaluations of our end-to-end pipeline in two independently developed coaching agents. While some of our demonstrations relied on access to the coaching agents' internal user model, the simulated interactions generated by our framework can be evaluated in a way that is agnostic to the design of the agent, so the framework can be applied to any conversational agents.

The extensive, realistic, and grounded simulated interactions developed from our proposed framework lay the foundation for efficient development of coaching agents, potentially through a fine-tuning or reinforcement learning loop, as will be pursued in future research.

\clearpage
\section*{Limitations}

This work introduced an end-to-end framework for evaluating a coaching agent using automated synthetic users. The ultimate goal of performing this evaluation is to improve the coaching agent itself using the metrics and signals derived from this work, which is left for future research. The coaching agents we evaluated in this work may not have the ``state-of-the-art'' performance, as that was not the focus of this paper. We also did not compare multiple commercial LLMs, as that was also not the focus of this paper.

In evaluating coaching agents, a longitudinal assessment of potential behavioral change is crucial. Effectively and realistically simulating synthetic users' behavioral change is a critical open question to be addressed, including modeling non-adherence to coaching advice (e.g., human users may not follow the advice, or may \emph{claim} they changed their behavior, even if they did not).

Personalization and adaptation of coaching and other agents are very important and relevant topics that are beyond the scope of this paper. We hope that grounded, rich, and diverse personas of synthetic users we developed in this work will significantly support the process of evaluating and improving personalized agents adapting to evolving user behaviors and needs.

\section*{Acknowledgements}

We would like to thank all organizers and participants of LifeSnaps and Project Baseline Health Study. We also thank Mercy Asiedu and Taylan Cemgil for helpful discussions, Tomas Garcia, Eileen Rivera, Maudie Roberts, and Karisha Pruitt for behavioral science \& clinical expert advice, and Aleksandra Faust, Jo\"elle Barral, and Edward Grefenstette for support.

% Bibliography entries for the entire Anthology, followed by custom entries
%\bibliography{anthology,custom}
% Custom bibliography entries only
\bibliography{ref}


% Appendix
\clearpage
\onecolumn
\appendix

% Reset figure numbering & add "A" to the labels.
\renewcommand\thefigure{A\arabic{figure}}
\setcounter{figure}{0}

% \section{Appendix}
% \label{sec:appendix}

\section{LifeSnaps Dataset Preprocessing}
\label{appx:lifesnaps_detail}
The age and the gender of the participants in the LifeSnaps dataset were available in two categories: ``under 30'' and ``over 30'', and ``male'' and ``female''. Of the 71 participants, we excluded 3 individuals whose age or gender data were missing in the dataset, resulting in 68 individuals. BMI was classified into four categories following a standard practice and terms: ``underweight'' ($\textrm{BMI} < 19$), ``normal'' ($19 \leq \textrm{BMI} < 25$), ``overweight'' ($25 \leq \textrm{BMI} < 30$), and ``obese'' ($\textrm{BMI} \geq 30$).

Given longitudinal LifeSnaps data over four months, we computed the mean and the standard deviation of sleep duration. The mean sleep duration was included in the vignettes in the number of hours. Participants with whose standard deviation of duration was no more than 1.5 hours were considered to have ``consistent'' sleep duration, and the rest were considered to have ``variable'' sleep duration.

\section{Synthetic User Prompt Generation for the Sleep Domain}
\label{appx:sleep_prompt}

Prompt components: \texttt{\cbBlue{Vignette}}, \texttt{Context}, and \texttt{\cbBurgundy{Instructions}}. A \texttt{Context} consists of the synthetic user's \texttt{\cbGreenSea{Name}} and the entire \texttt{\cbRose{conversational transcript}} so far.

\begin{quote}
\texttt{\cbBlue{Nicole is female. She is more than 30 years old. She typically sleeps for 8 hours at night. Her sleep duration is highly variable. Her sleep efficiency is typically at 94\%. She is underweight. She is neither introverted nor extraverted. She is highly agreeable. She is moderately conscientious. She feels unstable. She is highly intellectual. (...)} \\
You are \cbGreenSea{Nicole}. \\
Your last observations are: \\
\cbRose{Nicole -- ``I'm struggling with getting enough sleep.  I'd like to sleep 7-8 hours a night consistently so I can improve my energy levels at work and be more present for my family, but my demanding work schedule and racing thoughts make it difficult.'' \\
COACH -- ``It's completely understandable that a demanding work schedule and racing thoughts can make sleep a challenge. Wanting better energy and to be more present for your family are fantastic goals! Is there something specific you've tried in the past to help with sleep, or something you'd like to try now?'' (...)} \\
\cbBurgundy{Given the above, generate what Nicole would say to the COACH next in this conversation. Respond in the format ``Nicole -- ...''. If the COACH has asked for input, make sure to generate specific and detailed answer to the question. Bear in mind the primary sleep concern, sleep goals, reasons for those goals, and barriers of Nicole provided above. The tone and style of the conversation should math Nicole's descriptions above.}
}
\end{quote}

\section{Sleep Profile Generation}
\label{appx:sleep_profile}

The following prompt was used to generate sleep profiles for synthetic users based on LifeSnaps, where [name] is randomly sampled and [background\_info] is generated by real data.

\begin{quote}
Here is some background information about a fictional person, called [name]. I want to create a fictional profile of [name]. The sleep profile should be in a structured JSON format, including these fields: "primary\_sleep\_concern", "sleep\_goals", "reasons\_for\_goals", "barriers". "primary\_sleep\_concern" must be a single string. "sleep\_goals" must be a list of strings. "reasons\_for\_goals" must be a list of strings. "barriers" must be a list of strings. Be creative and make up stories if you need to, but make sure the JSON sleep profile is consistent with the given background information. Your response must start with a justification of each field you create, followed by the structured JSON. It's important that you only use escaped double quotes and single quotes, as well as escaped backslashes to not break the JSON format. It's also important that you respect the JSON format.

Background information: "[background\_info]"

Response:
\end{quote}

\section{Prompts for fuzzy matching sleep profiles}
\label{appx:sleep_profile_fuzzy_matching}

For fuzzy-matching two descriptions of primary sleep concerns to account for paraphrasing, the following prompt was used for an instruction-tuned high-performance LLM.

\begin{quote}
I would like you to answer this question.

The following are two descriptions of someone's primary sleep concern, A and B:

A: "[description A]"

B: "[description B]"

Does A and B generally communicate compatible sleep concerns? They do not have to match completely, but I want to know whether they GENERALLY describes similar concerns. It is okay if some specific details are different.

Start your answer with Yes or No, followed by an explanation.
\end{quote}

For fuzzy-matching a list of sleep attributes (such as sleep goals and barriers), the following prompt was used to generate precision and recall.

\begin{quote}
I would like you to answer two questions. I have two lists of [field name] of a person. I want to know how they match well with each other. There are [length of list A] items in the list A, and there are [length of list B] items in the list B:

A: [list A]

B: [list B]

My two questions are: How many of the items in the list A are also generally represented in the list B? And how many items in the list B are also generally represented in the list A?

When finding a match, not all the details need to match. I just want to know whether each item is generally represented in the other list, while some specific details may be different.

Start your answer with two numbers separated by a comma and followed by a period, representing the answers to the two questions, for example "3, 2.". Add your explanation after this answer.
\end{quote}

\section{COM-B categories}
\label{appx:comb}

\begin{center}
\begin{tabular}{p{0.25\columnwidth}|p{0.75\columnwidth}}
    \hline
    \textbf{COM-B Category} & \textbf{Barriers} \\
    \hline
    Psychological Capability & Don’t know the Basics, Don’t know the consequences, Planning fallacy, Memory \\ 
    \hline
    Physical Capability & Decision fatigue, Physical limitations \\  
    \hline
    Social Opportunity & Lack of social support, Conflicting opinions, Impact on others, peer pressure \\
    \hline
    Physical Opportunity & Geographic limitations, Affordability or costs, Lack of equipment, Switching settings \\
    \hline
    Reflective Motivation & Poor self-efficacy, Competing priorities , Lack of desire without reasons, Boredom \\
    \hline
    Automatic Motivation & Present bias, Anchoring effect, Gut feelings \\
    \hline
\end{tabular}
\end{center}

\section{COM-B Category Distribution}
\label{appx:comb_dist}

The COM-B sub-category distribution was 25\% reflective motivation, 21\% psychological capability, 19\% physical opportunity, 15\% social opportunity, 12\% automatic motivation, 9\% physical capability. This relative distribution of the barriers were also observed in related studies \citep{MacPherson2023-ox}.


\section{PBHS Diabetes Data Preprocessing}
\label{appx:pbhs_diabetes_preprocessing}

First, we filtered the PBHS dataset to include only individuals diagnosed with type 2 diabetes, resulting in a cohort of 345 individuals. For each individual, we leveraged their comprehensive demographic, socioeconomic, clinical, and behavioral survey data. The data encompassed a range of variables, including age, sex, race, marital status, smoking status, education level, income, insurance coverage, household size and structure, social interactions (measured through gatherings, phone calls, and text messages), attendance at organizational meetings, and employment status. Additionally, clinical parameters such as diagnosed conditions, Hemoglobin A1c (HbA1c) levels, blood glucose measurements, (systolic/diastolic) blood pressure, and body mass index (BMI) were used. Furthermore, the survey data underwent manual coding to identify if any of the individuals reported experiencing any relevant symptoms of the 21 identified barriers as their primary challenges. This coding process ensured that the nuanced, self-reported experiences of the individuals were captured and integrated into the development of our simulated user profiles.

\section{PBHS Diabetes Data Attributes}
\label{appx:pbhs_diabetes_data}
The following fields from PBHS data were used as input for synthetic user vignette generation (see \cref{appx:diabetes_nutrition_vignette_prompt}. 

\begin{center}
\begin{tabular}{p{0.3\columnwidth}|p{0.7\columnwidth}}
    \hline
    \textbf{Data Category} & \textbf{Attribute} \\
    \hline
    Basic demographics & Age at enrollment \\ 
    \hline
    Basic demographics & Race \\  
    \hline
    Basic demographics & Marital status \\
    \hline
    Basic demographics & Education \\
    \hline
    Basic demographics & Income \\
    \hline
    Basic demographics & Employment status \\
    \hline
    Social environment & People living at home \\
    \hline
    Social environment & Number of people under 18 living at home \\
    \hline
    Social environment & Weekly number of friend and family gatherings \\
    \hline
    Social environment & Weekly number of phone calls with friends and family \\
    \hline
    Social environment & Weekly number of texts with friends and family \\
    \hline
    Social environment & Weekly attendance to social organization meetings \\
    \hline
    Health data & Smoking status \\ 
    \hline
    Health data & Has insurance \\
    \hline
    Health data & Diagnostic conditions \\ 
    \hline
    Medical measurements & Hemoglobin A1C (HbA1c) \\
    \hline
    Medical measurements & Blood Glucose (mg/dl) \\
    \hline
    Medical measurements & Systolic and diastolic blood pressure \\
    \hline
    Medical measurements & Body Mass Index (BMI) \\
    \hline
    Barrier & Matched barrier from PBHS data \\
    \hline
\end{tabular}
\end{center}

\section{Synthetic User Prompt for the Diabetes Domain}
\label{appx:diabetes_nutrition_patient_agent_prompt}

Prompt components: \texttt{\cbRose{Data}}, \texttt{\cbBlue{Vignette}}, and \texttt{\cbBurgundy{Instructions}}. \texttt{\cbRose{Data}} is the synthetic user's demographic, social, and medical data from PBHS. \texttt{\cbBlue{Vignette}} is the backstory of synthetic user's barriers and \texttt{\cbRose{Data}} summarized in a few sentences. Conversation history is tracked and accounted for separately through a stateful chat session to ensure synthetic user is consistent.

\begin{quote}
\texttt{\cbBurgundy{You are a patient with cardiometabolic condition using a digital chat based application, below is your specific role. Pursue the conversation and the AI Coach will work with you. Reference your patient details and bring up information that can be relevant to the conversation. ONLY write out the conversation (not the breathing nor internal thoughts).\\} \\
Patient details: \cbRose{[PBHS data (..)]\\} \\
\cbBlue{John is a 31-year-old working dad, with a busy life. He lives with his partner and three young children and is trying to juggle work and family life. He knows he needs to eat better to manage his diabetes with an HbA1c of 6.7 and a high blood glucose level of 305, and he genuinely wants to improve. He often tells himself he will eat a healthy lunch, but finds himself eating fast food because he didn't have time to pack a lunch or get the right groceries the night before. He knows he's not doing his best, and feels like he is letting himself down, but also feels overwhelmed and unsure of where to even begin.\\} \\
\cbBurgundy{Now it's your turn to speak, you are the patient. You should follow your role and continue the conversation based on the AI Coach's last message. \\
- Make sure you focus on the backstory section of the patient details - specifically the BARRIER the patient is facing.\\
- Use the other patient details as supporting information in a natural, smooth way if necessary. When using those details, make sure to talk like a regular patient (change medical jargon to common speak) and not a doctor. \\
- DO NOT hallucinate or make up any new information that are not in the patient details. Stick to the context in the backstory presenting the barrier / challenge. \\
- Make sure to follow the communication style given in the patient details. \\ 
- You should only respond with at most 2 sentences per turn. \\
}
}
\end{quote}


\section{Diabetes Synthetic User Vignette Generation}
\label{appx:diabetes_nutrition_vignette_prompt}

\subsection{Initial Vignette Generation}
Prompt components: \texttt{\cbRose{Data}}, \texttt{\cbLilac{Barrier}}, \texttt{\cbGreenSea{Few Shot Examples}}, and \texttt{\cbBurgundy{Instructions}}. \texttt{\cbRose{Data}} is the synthetic user's demographic, social, and medical data from PBHS. \texttt{\cbGreenSea{Few Shot Examples}} are example synthetic user vignettes in JSON format to guide vignette generation. \texttt{\cbLilac{Barrier}} is the barrier the synthetic user is facing randomly selected from a prior distribution. The prompt below is for our test vignette; for the baseline vignette, we exclude health data and attributes unique to PBHS.

\begin{quote}
\texttt{\cbBurgundy{You are an expert in creating realistic patient vignettes for a digital health app with an AI coach providing goals achievement support for patients with diabetes. You are tasked with creating a realistic vignette for a patient working on their health goals. Given a barrier that a patient is experiencing, fill in the the following possible parameters that are aligned to each other and to the barrier, realistic for a diabetes patient, and collectively and informatively represent the barrier.\\}
\\
Here is the barrier the patient is experiencing: \texttt{\cbLilac{[Barrier]}}\\
\\ 
Here are the possible parameters you can fill in: \texttt{\cbRose{[PBHS Data (...)]}}  \\ 
\\ 
\texttt{\cbBurgundy{Include the following additional parameters as well:\\ 
- Name: Patient's name matching the patient SEX provided.\\ 
- Tone: (formal, academic, casual, playful, agreeable, antagnoistic, resistant, depressed, apathetic)\\ 
- Verbosity: (Shares intentional, complete sentences, responds to each question asked; Responds in short sentences or phrases; Shares unrelated information / overshares)\\ 
- Confidence: (High confidence and self awareness, knows themselves and conveys accurately; Concerned for appearance, erring towards aspirational self, overly optimistic view of oneself; Low confidence, convinced they are likely doing something wrong, apologetic)\\ 
\\ 
Here are the rules for creating the vignette: \\ 
- First, fill in each field of the parameters with a realistic value that is aligned with the barrier. The values can be numerical values or short free text descriptions. \\ 
- Then, provide a short backstory paragraph that describes the patient's situation, including the filled in parameters. The paragraph should be realistic and informative to represent the barrier. \\ 
- IMPORTANT: It is critical that the specific barrier term is not used in the vignette, including the backstory. The vignette should represent the barrier without explicitly mentioning it. \\ 
\\ }}
Here is an example of a vignette with the barrier of \texttt{\cbLilac{[Barrier]}}: \\
\texttt{\cbGreenSea{[Few shot examples]}}\\ 
\\ 
\\ 
\texttt{\cbBurgundy{Your final output should be the following, in JSON format.\\ 
\{\{\\ 
    ``reasoning'': <Your reasoning behind the vignette you are generating>,\\ 
    ``vignette'': <The vignette you have created in valid python dictionary format with keys containing the relevant parameter fields with an additional key containing the overall backstory.>\\ 
\}\}\\ 
Think step by step, and validate your reasoning with your text.
}}}
\end{quote}


\subsection{Vignette Generation Improvement}
After initial generation, we developed another ``verifier LLM'' to improve the quality of the vignettes. This verifier LLM edited components of the vignette and backstory to ensure that they were realistic and consistent with each other. This idea of ``self-refinement'' was inspired by existing literature \citep{madaan2023selfrefineiterativerefinementselffeedback,yao2023reactsynergizingreasoningacting}, but we used a separate LLM with fewer iterative steps. The prompt used by our verifier LLM is shown below.

Prompt components: \texttt{\cbBlue{Vignette}}, \texttt{\cbLilac{Barrier}}, \texttt{\cbGreenSea{Few Shot Examples}}, and \texttt{\cbBurgundy{Instructions}}. \texttt{\cbBlue{Vignette}} is the initially generated synthetic user vignette to improve. \texttt{\cbGreenSea{Few Shot Examples}} are example synthetic user vignettes in JSON format to guide vignette generation. \texttt{\cbLilac{Barrier}} is the barrier the synthetic user is facing randomly selected from a prior distribution. 

\begin{quote}
\texttt{\cbBurgundy{You are an expert in realistic patient vignettes for a digital health app with an AI coach providing goals achievement support for patients with diabetes. You are tasked with improving / editing an existing realistic patient vignette provided to you.\\ \\
Use the following methods to improve the vignette and parameters:\\
1) Remove information that causes the vignette to be unrealistic or misaligned with the barrier.\\
2) Ensure the vignette is representative to ONLY the barrier provided to you, without explicitly mentioning the barrier.\\
3) Add information that makes the vignette more realistic.\\
4) Double check that the backstory is using the correct age field and not hallucinating.\\
5) Remove any unnecessary formatting (curly brackets, hallucinations, etc).\\} \\
Here is the barrier the patient is experiencing: \cbLilac{[Barrier]} \\ \\
\cbBurgundy{Your final output should be the following, in JSON format.\\
\{\{\\
    "reasoning": <Your reasoning behind how you improved the vignette>,\\
    "vignette": <The vignette you have improved in valid python dictionary format with the exact same keys as the provided vignette.>\\
\}\}\\
Think step by step, and validate your reasoning with your text.\\
}\\
Improve the following vignette: \cbBlue{[Vignette]}\\ \\
EXAMPLES:\\
\cbGreenSea{[Few Shot Examples (...)]}\\
}
\end{quote}










\section{Diabetes Evaluation}
\label{appx:diabetes_eval}

Human experts were asked to provide insights by responding to the following evaluative questions:

\begin{enumerate}
    \item The vignette informatively represents the specified barrier. (Y/N) 
    \item The information in the vignette is self-consistent. (Y/N)
    \item The traits in the vignette are realistic for a cardiometabolic patient. (Y/N)
    \item The patient simulator's communication style specified in the vignette is reflected in the conversation. (Y/N)
    \item The patient simulator’s responses are consistent with the vignette. (Y/N)
    \item The patient simulator informatively demonstrates the barrier specified in the vignette. (Y/N)
    \item The coaching agent’s diagnosis of patient barriers is reasonable given the conversation alone. (Y/N)
\end{enumerate}

\section{Diabetes Evaluation Results}
\label{appx:diabetes_eval_results}

\begin{center}
\begin{small}
\begin{tabular}{p{5.5cm}|p{1.75cm}|p{1.75cm}|p{1.75cm}|p{1.75cm}}
\hline
\textbf{Question} & \textbf{Expert 1 Yes Response Rate} & \textbf{Expert 2 Yes Response Rate} & \textbf{Expert 3 Yes Response Rate} & \textbf{All Experts Yes Rate} \\
\hline
Q1: Vignette is informative & 0.88 & 0.92 & 0.84 & 0.76 \\
\hline
Q2: Vignette is self consistent & 0.72 & 0.88 & 0.80 & 0.68 \\
\hline
Q3: Vignette is realistic & 0.88 & 0.76 & 0.96 & 0.68 \\
\hline
Q4: Simulator portrays correct communication style & 0.88 & 0.88 & 1.00 & 0.76 \\
\hline
Q5: Simulator is consistent with vignette & 0.96 & 0.96 & 1.00 & 0.92 \\
\hline
Q6: Simulator demonstrates barrier & 1.00 & 1.00 & 1.00 & 1.00 \\
\hline
Q7: Coach diagnoses reasonable barrier & 0.96 & 1.00 & 0.92 & 0.92 \\
\hline
\end{tabular}
\end{small}
\end{center}


\section{Diabetes Baseline vs. Test Comparison Questions}
\label{appx:diabetes_eval_comparison_qs}
Human experts were asked the following evaluation questions when comparing baseline vs. test set conversations:
\begin{enumerate}[topsep=0pt,itemsep=-1ex,partopsep=1ex,parsep=1ex]
    \item Which of the two portrays one consistent barrier from the patient? (Y / N / Similar)
    \item Which of the two conveys the original barrier more accurately? (Y / N / Similar)
    \item Which of the two is more informative in understanding the patient? (Y / N / Similar)
\end{enumerate}

\section{Reference Distribution vs. LLM-Generated Distribution of Barriers In Simulated Conversations}
\label{appx:barrier_butterfly}
\begin{figure}[h]
  \centering
  \includegraphics[width=\columnwidth]{figs/diabetes_butterfly.pdf}
  \caption{\textbf{Reference distribution vs LLM-generated distribution of barriers observed in simulated conversations}. The reference is the distribution of barriers observed in cardiometabolic behavioral literature. The LLM-generated distribution are extracted by the diabetes coaching agent in 200 simulated conversations.}
  \label{fig:barrier_butterfly}
\end{figure}

\end{document}
