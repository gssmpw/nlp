\section{Related Work}
____ showed the ubiquity of hubs in many different kinds of datasets. 
Hubness is a cause of concern, as it can negatively impact many common tasks in data analysis and machine learning, such as regression, classification, outlier detection and clustering. Hubness was also shown to hinder the performance of nearest-neighbour algorithms in speech recognition, recommendation and multimedia retrieval (see ____ and references therein). Problematic hubness also occurs in distributed text representations analogous to those produced by a LLM. For example ____, ____, ____, ____ and ____ studied hubness in word and text embeddings, while ____, ____ and ____ looked at hubness in multimodal language models and cross-modal retrieval. 

Given the problems posed by hubs, various hubness reduction methods have been proposed, for example Local Scaling ____, Mutual Proximity ____, Globally Corrected Rank ____, Inverted Softmax ____, Cross-domain Similarity Local Scaling ____, Hubness Nearest Neighbor Search ____, Querybank Normalisation ____, DBNorm ____, Dual Inverted Softmax ____, F-norm ____ and Nearest Neighbor Normalization ____. These methods have been systematically compared by ____ and ____, among others. 

As shown by the plethora of hubness reduction techniques, the focus has so far been on mitigating hubness, with little attention devoted to the question of whether hubness is actually always a nuisance phenomenon to be mitigated.