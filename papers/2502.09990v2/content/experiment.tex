\section{Experiments}
\subsection{Experimental Settings}
To ensure fairness in comparison and consistency in experimental settings, we also implement X-Boundary on Llama-3-8B-Instruct, Qwen2.5-7B-Chat, and Mistral-7B-Instruct-v0.2, and evaluate it using the benchmarks described in Section~\ref{sec:comparison}.
%
Additionally, to assess the effectiveness of X-Boundary across different sizes of LLMs, we implement it on Qwen2.5-14B-Chat.
%
To construct the Separate Set, we sample 500 boundary-safe prompts from OR-Bench-80K \cite{orbench}, which have been filtered to avoid data contamination with the test set of OR-Bench. 
%
Next, we use GPT-4o to generate safe and helpful responses for these prompts, thus we get boundary-safe QA pairs.
%
The retain set consists of our collected boundary-safe QA pairs, UltraChat \cite{ultrachat}, and refusal data points generated by the trained LLMs themselves.
%
The erase set includes the harmful QA pairs for single-turn defense used in \citet{circuit_breaker} and the harmful QA pairs for multi-turn defense described in Section~\ref{sec:comparison}.
%
More implementation details of X-Boundary are listed in Appendix~\ref{app:x_training}.
\begin{table*}[!ht]
\setlength{\abovecaptionskip}{0.1in} 
\caption{Comparison of existing defense methods and X-Boundary on Qwen2.5-14B-Chat.}
% \vspace{-6pt}
\label{tab:large_model_results}
\setlength{\tabcolsep}{4pt}
\centering
\resizebox{2\columnwidth}{!}{
\begin{tabular}{c|cccccccccccc}
%\Xhline{4\arrayrulewidth}
\midrule
\multirow{2.5}{*}{\textbf{Methods}} & \multicolumn{3}{c}{\textbf{Multi-Turn ASR (\%) $\downarrow$}} & \multicolumn{4}{c}{\textbf{Over-Refusal Rate (\%) $\downarrow$}} & \multicolumn{3}{c}{\textbf{General Capability (\%) $\uparrow$}} \\
\cmidrule(lr){2-4} \cmidrule(lr){5-8} \cmidrule(lr){9-11}
% Model & Single Round ASR (\%) & Multi Round ASR (\%) & 
 & ActorAttack & RedQueen & Crescendo & XSTest & OKTest & OR-Bench & PHTest & MMLU & GSM8K & HumanEval \\
\midrule
Vanilla & 71.50 & 63.50 & 36.00 & 4.00 & 10.00 & 1.33 & 4.00 & 80.06 & 82.49 & 79.88 \\
\cmidrule(lr){1-11}
SFT & 52.00 & 10.00 & 16.00 & 43.60 & 51.33 & 31.33 & 62.67 & 79.58 & 82.18 & 81.71 \\
DPO & 54.50 & 45.00 & 32.00 & 6.40 & 14.00 & \textbf{2.67} & 8.67 & 78.58  & \textbf{83.32} & 81.10  \\
CB & \textbf{23.50} & \textbf{4.50} & \textbf{8.00} & 43.60 & 51.33 & 32.00 & 64.33 & \textbf{79.64} & 82.56 & \textbf{82.93} \\
\cmidrule(lr){1-11}
X-Boundary & 25.00 & 5.00 & 12.00 & \textbf{5.20} & \textbf{13.67} & 4.00 & \textbf{8.33} & 79.52 & 82.18 & 81.10 \\
% \Xhline{4\arrayrulewidth}
\midrule
\end{tabular}}
\vspace{-6pt}
\end{table*}
%
\subsection{Main Results}
\textbf{The explicit formulation for boundary contributes to the precise distinction between harmful and safe representations.}
% assist in accurately distinguishing between harmful and safe representations.
To investigate the effect of the explicit formulation for distinction boundary, we visualize the representation distribution of X-Boundary and without X-Boundary. 
%
Fig.~\ref{figs:t-sne} shows that, without X-Boundary, the boundary-safe representations close to harmful representations are mistakenly regarded as harmful ones.
% , and the corresponding boundary-safe queries are over-refused. 
%
This demonstrates that LLMs fail to learn a boundary that exactly distinguishes safe and harmful representations, which supports our motivation of explicitly formulating the distinction boundary.
%
With X-Boundary, harmful representations and boundary-safe representations are clearly separated as shown in Fig.~\ref{figs:t-sne}, verifying that the proposed explicit formulation contributes to establishing a precise distinction boundary. 
%
Please refer to Appendix~\ref{app:repe_angle} and Appendix~\ref{app:complete_tsne} for more detailed visualization and analysis of the representation distribution.
% And most boundary-safe queries are properly answered by LLMs, verifying that separate loss contributes to establish a precise distinction boundary.
% The distinction boundaries of other baseline methods are discussed in Appendix \ref{}.}

\textbf{X-Boundary maintains the lowest over-refusal rate while achieving SOTA defense against multi-turn jailbreaks.}
%
With a precise distinction boundary, X-Boundary reduces the ASR of ActorAttack by more than 40\% while maintaining the increase in over-refusal rate on OKTest within 5\% across three LLMs, as shown in Table~\ref{tab:main_results}.
% 
Specifically, on Llama-3-8B-Instruct, CB and X-Boundary both achieve the lowest ASR against ActorAttack, but X-Boundary demonstrates an average over-refusal rate that is lower by 20.05\%. 
% 
Similarly, on Qwen2.5-7B-Chat, X-Boundary’s average over-refusal rate is 58.50\% lower than GA, which achieves the lowest ASR against Crescendo.

\textbf{X-Boundary rarely declines general capability.}
%
Table~\ref{tab:main_results} shows that the decline of general capabilities caused by X-Boundary is generally no more than 0.5\% compared to vanilla models, across the domains of general knowledge, mathematical ability, and coding ability.
%
In contrast to SFT, which causes a 7\% reduction in coding ability for Mistral-7B-Instruct-v0.2, X-Boundary achieves a lower ASR without compromising coding capability.


\textbf{X-Boundary successfully strikes a balance between robustness and usability.}
%
As a supplement to Table~\ref{tab:main_results}, Fig.~\ref{fig:trade_off} intuitively illustrates the trade-off between the ASR against multi-turn jailbreaks and the over-refusal rate.
%
As the training process advances, the ASR steadily decreases, whereas the over-refusal rate progressively increases.
% It is inappropriate to focus solely on excelling in a single metric.
%
Considering the two metrics comprehensively, X-Boundary appears in the lower-left corner of Fig.~\ref{fig:trade_off}, indicating that it achieves a better balance compared to the baseline methods.
% achieves both low ASR and low over-refusal rate in the meantime.
%
In the same way, Fig.~\ref{fig:asr_utility} demonstrates that X-Boundary also achieves a win-win outcome with robust defense and strong general capability.
%
For specific cases of the defense performance and usability preservation of X-Boundary, please refer to Appendix \ref{app:case_study}.

% Table \ref{tab:main_results} demonstrates that X-Boundary achieve the lowest or comparable ASR in multi-turn jailbreak scenarios, maintain the lowest over-refusal rate, and preserve general capabilities with minimal impact. 
% 不能只追求低ASR,over-refusal rate如果过高,ASR就几乎失去了意义

\textbf{X-Boundary is effective across different sizes of LLMs.}
%
Table~\ref{tab:large_model_results} shows that, on Qwen2.5-14B-Chat, X-Boundary relatively reduces the ASR of three multi-turn attacks by more than 60\%, while keeping the increase in over-refusal rate within 5\% compared to the vanilla model.
%
While X-Boundary and CB achieve comparable ASR, the over-refusal rate of X-Boundary is approximately 40\% lower than that of CB.
%
Compared with the performance on Qwen2.5-7B-Chat, the performance of X-Boundary on Qwen2.5-14B-Chat has not decreased.
%
% \setlength{\belowcaptionskip}{0pt}
\begin{figure}[t]
\begin{center}
\centerline{\includegraphics[width=\columnwidth]{Figs/tsne_comparison_cb_boundary.pdf}}
\setlength{\abovecaptionskip}{0.1in} 
\caption{Visualization of the representation distribution of X-Boundary and without X-Boundary.} 
\label{figs:t-sne}
\end{center}
\vskip -0.45in
\end{figure}
%
\renewcommand{\thesubfigure}{}
\begin{figure}[t]
\begin{center}
\centerline{\includegraphics[width=\columnwidth]{Figs/Final_Test.pdf}}
\setlength{\abovecaptionskip}{0.1in} 
\caption{The trade-off between ASR of multi-turn jailbreaks and over-refusal rate on Llama-3-8B-Instruct. The data points were collected by sampling and evaluating at every 100 training steps.}
\label{fig:trade_off}
\end{center}
\vskip -0.4in
\end{figure}
%
\begin{table*}[t]
% \captionsetup{position=above} % 让 caption 显示在表格上方
\setlength{\abovecaptionskip}{0.1in} % 让 caption 和表格更紧凑
\caption{Comparison of existing defense methods and X-Boundary on reasoning models.}
% \vskip 0.1in
\label{tab:reasoning_results}
\setlength{\tabcolsep}{2pt}
\centering
\resizebox{2\columnwidth}{!}{
\begin{tabular}{cc|cccccccccccc}
%\Xhline{4\arrayrulewidth}
\midrule
\multirow{2.5}{*}{\textbf{Models}} & \multirow{2.5}{*}{\textbf{Methods}} & \multicolumn{3}{c}{\textbf{Single~\&~Multi-Turn ASR (\%) $\downarrow$}} & \multicolumn{4}{c}{\textbf{Over-Refusal Rate (\%) $\downarrow$}} & \multicolumn{3}{c}{\textbf{General Capability (\%) $\uparrow$}} \\
\cmidrule(lr){3-5} \cmidrule(lr){6-9} \cmidrule(lr){10-12}
% Model & Single Round ASR (\%) & Multi Round ASR (\%) & 
~ & ~ & HarmBench & ActorAttack & RedQueen & XSTest & OKTest & OR-Bench & PHTest & AIME & GPQA & LiveCode \\
\midrule
\multirow{6}*{\makecell{\textbf{DeepSeek-}\\\textbf{R1-Distill-}\\\textbf{Llama-8B}}} & Vanilla & 55.42 & 83.50 & 58.50 & 0.40 & 1.67 & 1.00 & 4.67 & 50.00 & 50.00 & 40.00 \\
\cmidrule(lr){2-12}
~ & SFT & 16.25 & 44.50 & 28.50 & 58.40 & 34.67 & 32.67 & 65.00 & 40.00 & 44.95 & 35.10 \\
~ & DPO & 17.50 & 67.00 & 36.50 & \textbf{0.40} & \textbf{1.67} & \textbf{1.33} & \textbf{7.33} & \textbf{50.00} & 46.97 & 42.40 \\
~ & CB & 52.80 & 77.50 & 55.00 & 1.60 & 3.33 & 3.67 & 8.00 & 46.67 & 46.97 & \textbf{40.65} \\
\cmidrule(lr){2-12}
~ & X-Boundary & \textbf{13.33} & \textbf{34.00} & \textbf{27.00} & 2.40 & 3.00 & 7.33 & 12.00 & \textbf{50.00} & \textbf{50.00} & 39.86 \\
\midrule
\multirow{6}*{\makecell{\textbf{DeepSeek-}\\\textbf{R1-Distill-}\\\textbf{Qwen-7B}}} & Vanilla & 60.42 & 79.00 & 60.50 & 1.60 & 2.00 & 1.33 & 2.67 & 53.30 & 48.99 & 39.76 \\
\cmidrule(lr){2-12}
~ & SFT & 48.33 & 59.50 & 42.50 & 7.60 & 9.00 & 11.00 & 12.67 & 46.67 & 48.99 & 36.44 \\
~ & DPO & 52.92 & 74.00 & 48.50 & \textbf{1.60} & 3.67 & \textbf{3.00} & \textbf{8.67} & \textbf{53.33} & 50.00 & 40.08 \\
~ & CB & 57.92 & 54.00 & 55.50 & 5.33 & 4.00 & 5.67 & 9.00 & 46.67 & 46.97 & 39.33 \\
\cmidrule(lr){2-12}
~ & X-Boundary & \textbf{20.80} & \textbf{43.50} & \textbf{36.00} & 2.40 & \textbf{3.00} & 7.33 & 12.00 & \textbf{53.33} & \textbf{50.51} & \textbf{40.02} \\
% \Xhline{4\arrayrulewidth}
% \Xhline{4\arrayrulewidth}
\midrule
\end{tabular}}
\vspace{-2pt}
\end{table*}
%
\begin{table*}[t]
\setlength{\abovecaptionskip}{0.1in} 
\caption{Ablation study on Qwen2.5-7B-Chat. In this table, A represents single-turn defense data, B represents multi-turn defense data, C represents boundary-safe data, and D represents the separate loss $\mathcal{L}_{\texttt{s}}$.}
% \vspace{-6pt}
\label{tab:ablation}
\setlength{\tabcolsep}{2pt}
% \vskip 0.1in
\centering
\resizebox{2\columnwidth}{!}{
\begin{tabular}{c|cccc|ccccccccccccc}
%\Xhline{4\arrayrulewidth}
\midrule
\multirow{2.5}{*}{\textbf{Models}} & \multirow{2.5}{*}{A} & \multirow{2.5}{*}{B} & \multirow{2.5}{*}{C} & \multirow{2.5}{*}{D} & \multicolumn{3}{c}{\textbf{Multi-Turn ASR (\%) $\downarrow$}} & \multicolumn{4}{c}{\textbf{Over-Refusal Rate (\%) $\downarrow$}} & \multicolumn{3}{c}{\textbf{General Capability (\%) $\uparrow$}} \\
\cmidrule(lr){6-8} \cmidrule(lr){9-12} \cmidrule(lr){13-15}
% Model & Single Round ASR (\%) & Multi Round ASR (\%) & 
~ & & & & & ActorAttack & RedQueen & Crescendo & XSTest & OKTest & OR-Bench & PHTest & MMLU & GSM8K & HumanEval \\
\midrule
Vanilla & & & & & 76.00 & 39.50 & 62.00 & 6.00 & 19.33 & 1.67 & 5.60 & 74.26 & 80.67 & 81.71 \\
\cmidrule(lr){1-15}
(a) & $\checkmark$ & & & & 63.00 & 11.50 & 30.00 &9.20 & 19.00 & 6.66 & 14.66 & 74.19 & 80.14 & 82.32\\
(b) & $\checkmark$ & $\checkmark$ & & & 15.50 & 5.50 & 12.00 & 20.40 & 26.00 & 34.00 & 43.67 & 74.21 & 80.36 & 81.10\\
(c) & $\checkmark$ & $\checkmark$ & $\checkmark$ & & 15.50 & 7.00 & 16.00 & 18.00 & 28.33 & 6.33 & 25.00 & 74.20 & 80.36 & 81.71\\
\cmidrule(lr){1-15}
X-Boundary & $\checkmark$ & $\checkmark$ & $\checkmark$ & $\checkmark$ & 17.50 & 7.50 & 16.00 & 10.40 & 16.67 & 5.33 & 15.00 & 74.17 & 80.52 & 81.10\\
% \Xhline{4\arrayrulewidth}
\midrule
\end{tabular}}
\vspace{-8pt}
\end{table*}
%
\subsection{Defense Performance on Reasoning Models}
Recently, \citet{safechain,R1_assessment} have highlighted significant safety risks in large reasoning models, such as DeepSeek-R1~\cite{guo2025deepseek}.
%
Notably, the reasoning outputs of R1 models pose even greater safety concerns than their final answers.
%
In this subsection, we explore the implementation of X-Boundary to enhance the safety defenses of reasoning models.
%
As shown in Table \ref{tab:reasoning_results}, X-Boundary effectively reduces the ASR of both single-turn and multi-turn attacks, while not causing severe over-refusal issues or a decline in reasoning ability in distilled reasoning models.
% With a detecter, X-Boundary avoids and interrupts harmful output
%
Surprisingly, the CB appears to be less effective for reasoning models, possibly because of differences in their representation space compared to standard models.
%
According to the theoretical insights in Section \ref{sec:theoretical}, X-Boundary reduces the difficulty of training and accelerates the training process.
%
Therefore, X-Boundary has an advantage over CB.
%
The effect of defense methods on the reasoning ability of R1 models is discussed in Appendix \ref{app:reasoning_ability}.

\subsection{Ablation Study}
\label{sec:ablation}
We conduct ablation studies on the impact of multi-turn defense data, boundary-safe data, and separate loss. The results are illustrated in Table~\ref{tab:ablation}. More results on Llama-3-8B-Instruct and Mistral-7B-Instruct-v0.2 are shown in Appendix~\ref{app:ablation}.

\textbf{Multi-turn defense data contribute to the reduction of ASR but intensify the over-refusal problem.} 
%
With the multi-turn defense data described in Section \ref{sec:comparison} added into erase set, the ASR of ActorAttack is reduced from 63.00\% to 15.50\% on Qwen2.5-7B-Chat. 
%
However, the over-refusal rates in OR-Bench and PHTest increase about 30.00\%.
%
This highlights that over-refusal is a critical issue in multi-turn defense tasks, which cannot be overlooked and urgently requires resolution.

\textbf{Boundary-safe data can partially mitigate the over-refusal issue.}
%
Boundary-safe QA pairs added to the retain set significantly reduce the over-refusal rate on OR-Bench and PHTest but show limited effectiveness on XSTest and OKTest. 
%
This may be because the boundary-safe QA pairs are synthesized by LLMs, leading to effectiveness on OR-Bench and PHTest, which also use synthetic data for testing. 
%
In contrast, the test queries in XSTest and OKTest are manually crafted and may differ in distribution from the synthetic data, making it difficult to achieve effective generalization.

\textbf{Simply adjusting the size of boundary-safe data can not effectively balance ASR and over-refusal rate.}
% The over-refusal rate does not continuously decrease as the data size increases.}
%
Increasing the size of boundary-safe data can reduce the over-refusal rate, but it also leads to a sharp increase in ASR against multi-turn jailbreaks.
%
Please see Appendix \ref{app:set_size_effect} for more detailed results.

\textbf{Separate loss can further reduce the over-refusal rate.}
%
Unlike simply adding boundary-safe data, separate loss markedly reduces the over-refusal rate on both manually crafted and synthetically constructed benchmarks.
%
Since the boundary-safe data shares the same source as OR-Bench, simply adding data is sufficient to reduce the over-refusal rate to a very low level, leaving little room for separate loss to make a noticeable impact. 
%
However, in the other three benchmarks, separate loss further reduces the over-refusal rate by an average of 9.75\% compared to adding boundary-safe data.