\section{Related Works}
\label{sec:rw}
Researchers have explored various techniques like natural language processing, machine learning, and social media analysis for fake news detection. 
These methods are generally divided into three categories \cite{lakzaei2024disinformation}:
\begin{itemize}
	\item \textbf{Content-based methods:}
	which focus on analyzing the text or visuals to identify fake news~\cite{lakzaei2024loss}.
	\item \textbf{Context-based methods:}
	which use contextual data such as user profiles and propagation patterns~\cite{DBLP:conf/bigdataconf/AsghariCC22}.
	\item \textbf{Hybrid methods:}
	which combine both content and context to detect fake news~\cite{DAVOUDI2022116635}.
\end{itemize}
Our approach is content-based, relying solely on textual content, which is the key feature in all existing datasets. As a result, our method can be applied to any dataset.
%However, 
We note that context information may not always be available.
%-------------------------------------------------------------------

A significant portion of existing methods for fake news detection relies on supervised approaches \cite{yadav2024emotion, qu2024qmfnd, fang2024nsep}.
The primary challenge and limitation of this approach is its dependency on large, labeled datasets, which makes it time-consuming and costly to generate. 
Additionally, obtaining sufficient labeled data can be challenging, especially when it comes to handling the vast amounts of news data available.  
Therefore, semi-supervised approaches, which can work with a small subset of labeled data and classify the larger portion of unlabeled data, provide an effective solution. 
In this paper, we propose a semi-supervised approach based on Graph Neural Networks (GNNs). 
As a result, our focus is primarily on recent semi-supervised methods and those utilizing GNNs.
%, particularly the approaches that have been introduced in recent years.
%-------------------------------------------------------------------
%-------------------------------------------------------------------

DEFD-SSL \cite{al2023robust} is a semi-supervised approach for fake news detection that combines various deep learning models, data augmentations, and distribution-aware pseudo-labeling. It uses a hybrid loss function and incorporates ensemble learning along with distribution alignment to maintain balanced accuracy, especially in datasets with class imbalances.
%-------------------------------------------------------------------
Shaeri and Katanforoush \cite{shaeri2023semi} presented a semi-supervised fake news detection method that integrates LSTM with self-attention layers. They introduced a pseudo-labeling algorithm to mitigate data scarcity, refining labels iteratively to enhance model performance. Additionally, transfer learning from sentiment analysis using pre-trained RoBERTa models further boosts accuracy. Overall, their approach demonstrates effectiveness in addressing data limitations and utilizing advanced deep learning techniques for fake news detection.
%-------------------------------------------------------------------
Canh et al. \cite{canh2023fake} proposed MCFC, a fake news detection method that uses multi-view fuzzy clustering on data from various sources, extracting features like title, content, and social media engagement. By applying multi-view clustering and semi-supervised learning, the method improves accuracy by incorporating diverse perspectives. The MCFC model outperforms traditional methods but faces challenges with parameter complexity and computational efficiency, especially with multi-source data.
%-------------------------------------------------------------------
WeSTeC \cite{akdag2024early} automates labeling by using models trained on a small labeled dataset, applying them to unlabeled data with weak labels aggregated through strategies like majority voting. It includes a RoBERTa classifier fine-tuned for the target domain and tackles semi-supervision and domain adaptation by leveraging limited labeled data and data from different domains.
%-------------------------------------------------------------------
%-------------------------------------------------------------------

Nguyen and Do \cite{nguyen2023fake} proposed a fake news detection method for Vietnamese datasets that combines a knowledge graph with semi-supervised Graph Convolutional Networks (GCN). They used GloVe embeddings, Word Mover’s Distance (WMD), and KNN to construct the knowledge graph, which is then used with GCN for detection.
%-------------------------------------------------------------------
MODEL \cite{liu2024rumor} detects rumors by learning source tweet representations through user correlation and information propagation. Using Graph Neural Networks, it extracts representations from a bipartite graph of user-tweet relationships and a tree structure representing information spread. By combining these representations, MODEL achieves high accuracy in rumor classification.
%-------------------------------------------------------------------
Inan \cite{inan2022zoka} proposed ZoKa, a fake news detection method that first identifies potential fake users using a user graph and Graph Attention Network (GAT). It then creates a content graph from user interactions and applies Edge-weighted Graph Attention Network (EGAT) with pre-trained encoders to detect fake news.
%-------------------------------------------------------------------
LOSS-GAT \cite{lakzaei2024loss} is a semi-supervised one-class approach that combines GNNs with label propagation. Initially, a two-step label propagation algorithm assigns preliminary labels to news articles, categorizing them as either fake or real. The graph structure is then refined using structural augmentation techniques to enhance its expressiveness. Finally, an improved GNN incorporates randomness in node neighborhoods during aggregation to predict labels for unlabeled data.
%-------------------------------------------------------------------
The GBCA model \cite{zhang2024gbca} integrates Graph Convolution Networks (GCN) with BERT and a co-attention mechanism to address the limitations of previous fake news detection approaches. Unlike methods focusing solely on propagation patterns or semantics, the GBCA model combines both dimensions—propagation structure and semantic features—by dynamically weighting them. This fusion enhances the model's performance, making it more accurate and efficient in detecting fake news across multiple datasets.
%-------------------------------------------------------------------
%-------------------------------------------------------------------

Although the aforementioned methods leverage the power of GNNs for fake news detection, they, along with other similar approaches, face inherent challenges arising from the limitations of standard message passing techniques in these networks.
Standard message passing primarily aggregates information from local neighbors, which may result in models overlooking global structural patterns in the graph.
This limitation is particularly problematic in tasks like fake news detection, where relationships across distant nodes, such as long-range dependencies in social networks, can provide critical contextual signals.
Furthermore, with an increasing number of layers, GNNs often suffer from over-smoothing, where node embeddings from different classes become indistinguishable.
This phenomenon reduces the discriminative power of the model, especially in large and densely connected graphs.
While various methods have been proposed to address these challenges in standard message passing, recent research has introduced innovative approaches to enhance message passing mechanisms in GNNs.
Among these, techniques leveraging Gumbel-Softmax have gained attention due to their ability to improve the flexibility and efficiency of message propagation.
%-------------------------------------------------------------------
%-------------------------------------------------------------------

Acharya and Zhang \cite{acharya2020feature} extended the Gumbel-Softmax feature selection algorithm to Graph Neural Networks (GNNs). They proposed a method for selecting and ranking features to reduce the feature set while maintaining high classification accuracy. 
%-------------------------------------------------------------------
The Learning to Propagate (L2P) \cite{xiao2021learning} framework enables GNNs to learn personalized and adaptive message propagation strategies. By utilizing latent variables, the method optimizes the propagation steps for each node individually. It further employs the Variational Expectation-Maximization (VEM) approach to estimate the maximum likelihood of the GNN parameters, enhancing the flexibility and effectiveness of the propagation process.
%-------------------------------------------------------------------
Chaudhary and Singh \cite{chaudhary2023gumbel} proposed a Gumbel-SoftMax-based Graph Convolutional Network (GS-GCN) for identifying hidden communities in complex networks. Their model utilizes the Gumbel-SoftMax function for feature extraction and incorporates a two-layer GCN, leveraging degree and adjacency matrices to enhance community detection.
%-------------------------------------------------------------------
CO-GNNs \cite{DBLP:conf/icml/FinkelshteinHBC24} propose a dynamic message-passing paradigm for graph neural networks, allowing nodes to choose actions like STANDARD, LISTEN, BROADCAST, or ISOLATE instead of uniformly propagating information. This flexibility enables adaptive information flow based on node states and graph topology. The architecture integrates an action network for selecting strategies and an environment network for updating representations.
%-------------------------------------------------------------------
DHGAT \cite{lakzaei2025decision}, or Decision-based Heterogeneous Graph Attention Network, is a model designed for fake news detection in a semi-supervised setting. It overcomes the limitations of traditional GNNs by dynamically optimizing and selecting the most suitable neighborhood type for each node in every layer in a heterogeneous graph. The architecture comprises a decision network that identifies the optimal neighborhood type and a representation network that updates node embeddings based on this selection.
%-------------------------------------------------------------------
%-------------------------------------------------------------------

The aforementioned methods have demonstrated the effectiveness of utilizing Gumbel-Softmax in enhancing the flexibility and efficiency of message propagation in GNNs.
However, despite these advancements, they still suffer from a fundamental limitation inherent to GNNs: the restriction of a node's message propagation to its L-hop neighbors in an L-layer network.
In such networks, a node cannot access neighbors beyond L hops, which prevents the propagation of information from distant nodes.
To address this limitation, we propose NOL-GAT, a novel architecture that extends the reach of message passing by allowing each node at every layer to select the optimal neighborhood order for updating its embedding vector.
Unlike traditional methods that are restricted to layer-wise propagation within fixed L-hop neighborhoods, NOL-GAT employs a mechanism that dynamically determines the most informative hop distance for each node.
This approach ensures that distant but semantically important nodes contribute effectively to the representation learning process, while avoiding unnecessary noise from irrelevant or redundant neighbors.
In the following sections, we detail the architecture and methodology of NOL-GAT, demonstrating its ability to address the challenges of existing GNN-based methods while achieving superior performance in fake news detection.