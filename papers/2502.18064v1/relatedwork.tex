\section{Related Work}
\subsection{Over-Range Signal Recovery for Accelerometers}
Restoring over-range signals for accelerometers is an unexplored area within the current research landscape. Over-range signals are lost when the acceleration of the measured object exceeds the sensor's range, presenting a critical challenge in high dynamic applications such as automotive crash testing, industrial machinery monitoring, and sports science. However, few attempts are dedicated to the accelerometer over-range signal restoration problem.
Several factors contribute to the scarcity of research in this area. The nonlinearity and complexity of signal distortion when an accelerometer exceeds its measurement range make the restoration task highly challenging. Also, the recovery of over-range signals is highly context-dependent and relies on experience observing massive data, which is difficult for traditional non-data-driven models \cite{yang2024visionzip}. Finally, obtaining paired high-range and low-range data for training data-driven models is inherently tricky. The rarity of such paired datasets impedes the development of supervised learning approaches, further complicating the task of over-range signal restoration. Consequently, most existing efforts have focused on improving the dynamic range of sensors through hardware advancements rather than algorithm design \cite{wu2024design}.

\subsection{Signal Quality Enhancement for Accelerometers}
Compared to the rarely explored area of over-range signal restoration, considerable research has focused on reducing noise in accelerometer signals.
%which is critical for applications such as health monitoring, mobile device interactions, and precision engineering \cite{engelsman2022data, engelsman2023data}. 
Traditional signal denoising approaches rely on various filtering techniques, including Kalman filters, Savitzky-Golay filters \cite{8713728}, and empirical mode decomposition \cite{liu2020denoising}. While these methods have proven effective in separating the noise from the signal and improving signal quality, they usually rely on prior knowledge of the signal or noise characteristics \cite{skog2009car}, causing poor generalization ability.
In contrast, data-driven methods learn the denoising function from data without relying on the information about signal characteristics, which can adapt to different sensors and noise patterns.
%Machine learning methods such as K-nearest neighbors (K-NNs), Convolutional Neural Networks (CNNs), Recurrent Neural Networks (RNNs), etc., have been employed to learn complex patterns and correlations in time-series data, offering superior performance over traditional methods. 
In data-driven approaches, generative deep learning models directly map low-cost signals to high-cost signals, providing a more effective way to enhance signal quality. However, their superior performance typically relies on strictly paired training data, which is impractical to obtain for low-cost and high-cost accelerometers \cite{wu2019survey}.
Therefore, few studies have attempted to apply GANs to improve accelerometer signals. 
%In this paper, we apply GANs to accelerometer signal generation for the first time and propose a weakly-supervised HEROS-GAN. This framework injects honed energy into GAN to guide the generation of rich signal details. Additionally, it leverages optimal transport theory to deeply explore the similarities between low-cost and high-cost signals, thereby providing as much supervisory information as possible.



\begin{figure*}[t]
	\centering
	\includegraphics[width=0.95\textwidth,height=0.32\textheight]{Framework.pdf} % Reduce the figure size so that it is slightly narrower than the column.
	\caption{Architecture of the HEROS-GAN. MLE (orange) and OTS (green) are applied to feature interaction on both sides.}
	\label{Framework of HEROS-GAN}
\end{figure*}