\section{Case Studies}
\label{sec:case-studies}

% 3 pages

% examples
% + array
% + pointer
% + recursion

In this section, we present several case studies to demonstrate the use of our program logic.

To make the presentation clearer, we will separate the qualitative specifications and quantitative resource bounds in the QFUA/QBUA triples.
We will use the notations
\begin{itemize}
  \item $\ffjudge{P_S}{P_R}{C}{Q_S}{Q_R}$ for the QFUA triple $\fjudge{[\neg P_S]\curlyvee P_R}{C}{[\neg Q_S]\curlyvee Q_R}$,
  \item $\bbjudge{P_S}{P_R}{C}{Q_S}{Q_R}$ for the QBUA triple $\bjudge{[P_S]\curlywedge P_R}{C}{[Q_S]\curlywedge Q_R}$, and
  \item $\ddjudge{P_S}{P_R}{C}{Q_S}{Q_R}$ for the \QBUAd triple $\djudge{[P_S]\curlywedge P_R}{C}{[Q_S]\curlywedge Q_R}$,
\end{itemize}
where $P_S,Q_S:State\to\{\true,\false\}$ and $P_R,Q_R:State\to\mathbb{Z}$.
The idea is that $p\ge([\neg P_S]\curlyvee P_R)(\sigma)$ is equivalent to $P_S(\sigma)=\true$ and $p\ge P_R(\sigma)$, while $p\le([P_S]\curlywedge P_R)(\sigma)$ is equivalent to $P_S(\sigma)=\true$ and $p\le P_R(\sigma)$.
With this notation, we provide some derived rules in Figure~\ref{fig:derived-rules} to use in the following examples.
The \textdagger\ rule can be used for both QFUA and QBUA triples, while the \textdaggerdbl\ rule is valid for all three kinds of triples.

\begin{figure}[t!]
\begin{mathpar}\footnotesize
  \inferrule[(\textdagger:Assign')]{x\notin\Fv(P_R)}{\AAjudge{P_S}{P_R}{\Assign{x}{e}}{\exists x'.P_S[x'/x]\land x=e[x'/x]}{P_R}}
  \hva\and
  \inferrule[(\textdaggerdbl:IfTrue)]{\AAjudge{P_S\land B}{P_R}{C_1}{Q_S}{Q_R}}{\AAjudge{P_S}{P_R}{\Ite{B}{C_1}{C_2}}{Q_S}{Q_R}}
  \hva\and
  \inferrule[(\textdaggerdbl:IfFalse)]{\AAjudge{P_S\land\neg B}{P_R}{C_2}{Q_S}{Q_R}}{\AAjudge{P_S}{P_R}{\Ite{B}{C_1}{C_2}}{Q_S}{Q_R}}
  \hva\and
  \inferrule[(\textdagger:WhileFalse)]{}{\aajudge{P_S\land\neg B}{P_R}{\While{B}{C}}{P_S\land\neg B}{P_R}}
  \hva\and
  \inferrule[(\textdagger:WhileSubvar)]{\forall n<k-1.\aajudge{P_S(n)\land B}{P_R(n)}{C}{P_S(n+1)\land B}{P_R(n+1)} \\ \aajudge{P_S(k-1)\land B}{P_R(k-1)}{C}{P_S(k)\land\neg B]}{P_S(k)}}{\aajudge{P_S(0)\land B}{P_R(0)}{\While{B}{C}}{P_S(k)\land\neg B}{P_R(k)}}
\end{mathpar}
\caption{Selected Derived Rules}
\label{fig:derived-rules}
\end{figure}

\subsection{Warm-up: Euclidean Algorithm}
\label{sec:cs:euclid}

The Euclidean algorithm is a method for finding the greatest common divisor of two integers. In this example, we show that the algorithm will consume at least $3n-3$ ticks in the worst case if each assignment consumes one tick. The algorithm $E$ is presented as follows. We omit the $\Tick{1}$ statement after each assignment in the following discussion.
\begin{align*}
  E\coloneqq \While{b\ne 0}{(\Local{t}{(\Assign{t}{b};\Tick{1};\Assign{b}{a\bmod b};\Tick{1};\Assign{a}{t};\Tick{1})})}
\end{align*}

We will prove that $\aajudge{a=F_n\land b=F_{n-1}}{3n-3}{E}{a=1\land b=0}{0}$ for both QFUA and QBUA triples, where $F_n$ is the $n$-th Fibonacci number, defined by $F_0=0$, $F_1=1$, and $F_n=F_{n-1}+F_{n-2}$ for $n>1$.
For the case $n=1$, the proof is straightforward by applying the rule \textsc{(\textdagger:WhileFalse)}.
And suppose $n>1$, to deal with the outer while loop, we define the loop subvariant $P_S(i)\coloneqq a=F_{n-i}\land b=F_{n-i-1}$ and $P_R(i)\coloneqq -3i$.
For the three assignments inside the local variable scope, we make the following derivation chain, where each assignment together with the predicate around it forms a triple derived by the \textsc{(\textdagger:Assign')} and \textsc{(\textdagger:Tick)} rules, and they are connected by the \textsc{(\textdagger:Seq)} rule.
\begin{align*}\footnotesize
  \SR{\begin{matrix}
    a=F_{n-i}\\{}\land b=F_{n-i-1}\land b\ne 0
  \end{matrix}}{-3i}
  \Assign{t}{b}
  &\SR{\begin{matrix}
    a=F_{n-i}\land b=F_{n-i-1}\\{}\land b\ne 0\land t=b
  \end{matrix}}{-3i-1}
  \Assign{b}{a\bmod b}\\
  &\hspace{-6em}\SR{\begin{matrix}
    a=F_{n-i}\land t=F_{n-i-1}\\{}\land t\ne 0\land b=a\bmod t
  \end{matrix}}{-3i-2}
  \Assign{a}{t}
  \SR{\begin{matrix}
    t=F_{n-i-1}\land t\ne 0\\{}\land b=F_{n-i}\bmod t\land a=t
  \end{matrix}}{-3i-3}
\end{align*}
Next, using the \textsc{(Local)} rule to erase $t$, we derive
\begin{align*}\footnotesize
  \aajudge{P_S(i)}{P_R(i)}{\Local{t}{(\cdots)}}{\begin{cases}P_S(i+1)\land b\ne 0&\text{if }i<n-2\\P_S(i+1)\land\neg(b\ne 0)&\text{if }i=n-2\end{cases}}{P_R(i+1)}.
\end{align*}
Therefore, we can apply the \textsc{(\textdagger:WhileSubvar)} and \textsc{(\textdagger:Relax)} rules to derive $\aajudge{a=F_n\land b=F_{n-1}\land b\ne 0}{3n-3}{E}{a=1\land b=0}{0}$. Combining with the $n=0$ case using the \textsc{(\textdagger:Disj)} rule, we finally prove the desired triple.

\subsection{Arrays: Insertion Sort}

Insertion sort is a simple sorting algorithm that builds the final sorted array one element at a time. We will prove that the algorithm consumes at least $2n^2-1$ ticks in the worst case when sorting an array of $n$ elements, assuming each assignment consumes one tick. The algorithm $IS$ is presented in Figure~\ref{fig:is-proof} (ignoring the annotations), where all ticks are omitted and the array $a$ is indexed from $0$ to $n-1$.

%\begin{algorithmic}\small
%  \State $IS\coloneqq$ $\Localx{i}~($ $\Assign{i}{1};$
%  \Indent
%  \Indent
%  \State $\WhileB{i<n}~($
%  \Indent
%  \State $\Localx{j}~($ $\Assign{j}{i};$
%  \Indent
%  \State $\WhileB{j>0\land a[j-1]>a[j]}~($
%  \Indent
%  \State $\Local{t}{(\Assign{t}{a[j-1]};\Assign{a[j-1]}{a[j]};\Assign{a[j]}{t})};\Assign{j}{j-1}));$
%  \EndIndent
%  \EndIndent
%  \State $\Assign{i}{i+1}))$
%  \EndIndent
%  \EndIndent
%  \EndIndent
%\end{algorithmic}

\begin{figure}[t!]
\centering
\begin{algorithmic}[1]\small
  \State $\Gray{\SR{pmin([0,n-1])}{2n^2-1}}$
  \State $\Localx{i}$
  \Indent
  \State $\Gray{\SR{pmin([0,n-1])}{2n^2-1}}$
  \State $\Assign{i}{1}$
  \State $\Gray{\SR{pmin([0,n-1])\land i=1}{2n^2-2}}$
  \State $\Gray{\SR{P_S(0)\land i<n}{2n^2-2+P_R(0)}}$
  \State $\WhileB{i<n}$
  \Indent
    \State $\Gray{\SR{P_S(k)\land i<n}{P_R(k)}}$
    \State $\Gray{\SR{i=k+1<n\land pmax([0,i-1])\land pmin([i,n-1])}{-2k^2-4k}}$
    \State \Localx{j}
    \Indent
      \State $\Gray{\SR{i=k+1<n\land pmax([0,i-1])\land pmin([i,n-1])}{-2k^2-4k}}$
      \State $\Assign{j}{i}$
      \State $\Gray{\SR{i=k+1<n\land pmax([0,i-1])\land pmin([i,n-1])\land j=i}{-2k^2-4k-1}}$
      \State $\Gray{\SR{i=k+1<n\land Q_S(0)\land (j>0\land a[j-1]>a[j])}{-2k^2-4k-1+Q_R(0)}}$
      \State $\WhileB{j>0\land a[j-1]>a[j]}$
      \Indent
        \State $\Gray{\SR{Q_S(l)\land(j>0\land a[j-1]>a[j])}{Q_R(l)}}$
        \State $\Gray{\SR{j=i-l>0\land pmax([0,i]\setminus\{j\})\land pmin(\{j\}\cup[i+1,n-1])}{-4l}}$
        \State $\Localx{t}$
        \Indent
          \State $\Gray{\SR{j=i-l>0\land pmax([0,i]\setminus\{j\})\land pmin(\{j\}\cup[i+1,n-1])}{-4l}}$
          \State $\Assign{t}{a[j]}$
          \State $\Gray{\SR{j=i-l>0\land pmax([0,i]\setminus\{j\})\land pmin(\{j\}\cup[i+1,n-1])\land t=a[j]}{-4l-1}}$
          \State $\Assign{a[j]}{a[j-1]}$
          \State $\Gray{\SR{\begin{matrix}j=i-l>0\land pmax([0,i]\setminus\{j\})\land pmin([i+1,n-1])\\{}\land\bigwedge_{x\in[0,i]\setminus\{j\}}a[x]>t\land\bigwedge_{x\in[i+1,n-1]}a[x]< t\land a[j]=a[j-1]\end{matrix}}{-4l-2}}$
          \State $\Assign{a[j-1]}{t}$
          \State $\Gray{\SR{\begin{matrix}j=i-l>0\land pmax([0,i]\setminus\{j-1\})\land pmin(\{j-1\}\cup[i+1,n-1])\\{}\land t=a[j-1]\end{matrix}}{-4l-3}}$
        \EndIndent
        \State $\Gray{\SR{j=i-l>0\land pmax([0,i]\setminus\{j-1\})\land pmin(\{j-1\}\cup[i+1,n-1])}{-4l-3}}$
        \State $\Assign{j}{j-1}$
        \State $\Gray{\SR{j=i-l-1>-1\land pmax([0,i]\setminus\{j\})\land pmin(\{j\}\cup[i+1,n-1])}{-4l-4}}$
        \State $\Gray{\SR{\begin{cases}Q_S(l+1)\land(j>0\land a[j-1]>a[j])&\text{if }l<i-1\\Q_S(l+1)\land\neg(j>0\land a[j-1]>a[j])&\text{if }l=i-1\end{cases}}{Q_R(l+1)}}$
      \EndIndent
      \State $\Gray{\SR{i=k+1<n\land Q_S(i)\land\neg(j>0\land a[j-1]>a[j])}{-2k^2-4k-1+Q_R(i)}}$
      \State $\Gray{\SR{i=k+1<n\land j=0\land pmax([0,i])\land pmin([i+1,n-1])}{-2k^2-8k-5}}$
    \EndIndent
    \State $\Gray{\SR{i=k+1<n\land pmax([0,i])\land pmin([i+1,n-1])}{-2k^2-8k-5}}$
    \State $\Assign{i}{i+1}$
    \State $\Gray{\SR{i=k+2<n+1\land pmax([0,i-1])\land pmin([i,n-1])}{-2k^2-8k-6}}$
    \State $\Gray{\SR{\begin{cases}P_S(k+1)\land i<n&\text{if }k<n-2\\P_S(k+1)\land\neg(i<n)&\text{if }k=n-2\end{cases}}{P_R(k+1)}}$
  \EndIndent
  \State $\Gray{\SR{P_S(n-1)\land\neg(i<n)}{2n^2-2+P_R(n-1)}}$
  \State $\Gray{\SR{i=n\land pmax([0,i-1])}{0}}$
  \EndIndent
  \State $\Gray{\SR{pmax([0,n-1])}{0}}$
\end{algorithmic}
\caption{Proof sketch of the triple $\aajudge{pmin([0,n{-}1])}{2n^2{-}1}{IS}{pmax([0,n{-}1])}{0}$ when $n{>}1$}
\label{fig:is-proof}
\end{figure}

In this case study, we use a simple model for arrays: we treat arrays as total value mappings of type $\bbZ \to \bbZ$ and extend program states to map variables to $\bbZ \cup (\bbZ \to \bbZ)$.
%
Below is the rule for establishing $\aajudge{P_S}{P_R}{C}{Q_S}{Q_R}$ triples for array assignments.
%
The \emph{patch} notation $a\{x \mapsto b \}$ changes the value of $a$ at $x$ to $b$, while leaving the rest of $a$ unchanged.
\begin{mathpar}\small
  \Rule{\textdagger:ArrayAssign}
  {a\notin\Fv(P_R)}
  {\aajudge{P_S}{P_R}{\Assign{a[e_1]}{e_2}}{\exists a'.P_S[a'/a]\land a=a'\{e_1 \mapsto e_2\}}{P_R}}
\end{mathpar}

We first define some predicates to describe the state of the array $a$.
Suppose $S$ is a set of indices, we define $pmax(S)$ to be true if and only if all elements in $S$ are the strict prefix maximum of the array, i.e., $\forall i\in S.\forall j<i.a[j]<a[i]$.
Similarly, $pmin(S)$ is defined to be true if and only if all elements in $S$ are the strictly suffix minimum of the array, i.e., $\forall i\in S.\forall j<i.a[j]>a[i]$.
Next, we will prove that $\aajudge{pmin([0,n-1])}{2n^2-1}{IS}{pmax([0,n-1])}{0}$, which shows a $2n^2-1$ lower bound.

We divide the proof into two cases: $n=1$ and $n>1$, and finish the proof by combining them using the \textsc{(\textdagger:Disj)} rule.
The former case is straightforward, and the latter case is more interesting, whose proof is presented in Figure~\ref{fig:is-proof}.
We use the \textsc{(\textdagger:WhileSubvar)} rule to deal with two while loops (lines 14--30 and lines 6--36).
For the inner one, the loop subvariant is $Q_S(l)\coloneqq j=i-l\land pmax([0,i]\setminus\{j\})\land pmin(\{j\}\cup[i+1,n-1])$ and $Q_R(l)\coloneqq -4l$.
For the outer one, the subvariant is $P_S(k)\coloneqq pmax([0,k-1])\land pmin([k,n-1])$ and $P_R(k)\coloneqq -2k^2-4k$.

%\subsection{Extending with Pointers: Search in Linked List}
%
%The quantitative under-approximation triples can be directly extended to pointer programs, with the use of separation logic specifications.
%We use a simple linked list search algorithm as an example, where the list is represented by a pointer pointing to the head node, and each node has two fields: $\Val$ for the value and $\Next$ for the next node.
%The following algorithm $SL$ searches for a value $v$ in the list $l$.
%\begin{align*}
%  SL\coloneqq\Assign{x}{l};\While{x\ne\Null\land x.\Val\ne v}{(\Assign{x}{x.\Next};\Tick{1})}
%\end{align*}
%
%We show that the algorithm consumes at least the length of the list ticks in the worst case by proving the triple $\aajudge{notin(v,l)}{length(l)}{SL}{x\not\mapsto}{0}$, where $notin$ and $length$ are recursive defined as follows.
%\begin{align*}
%  notin(v,x)&\coloneqq x\not\mapsto\lor(\exists u,y.x\mapsto(u,y)*(v\ne u\land notin(v,y)))\\
%  length(x)&\coloneqq\begin{cases}0&\text{if }x\not\mapsto\\1+length(y)&\text{if }x\mapsto(u,y)\end{cases}
%\end{align*}
%
%Similarly to the previous examples, we use the \textsc{(\textdagger:WhileFalse)} rule when $length(l)=0$ and the \textsc{(\textdagger:WhileSubvar)} rule when $length(l)>0$. For the latter case, the loop subvariant is $P_S(i)\coloneqq x=l.\Next^i\land notin(v,x)$ and $P_R(i)\coloneqq length(x)$, where $l.\Next^i$ denotes the $i$-th node in the list starting from $l$.

\subsection{Pointers and Procedure Calls: Depth-first Search of Binary Trees}

The quantitative under-approximation triples can be directly extended to pointer programs with the use of \emph{separation-logic}~\cite{POPL:IO01,LICS:Reynolds02} specifications.
We use a simple depth-first search algorithm on a binary tree as an example, where the tree is represented by a pointer pointing to the root node, and each node has two fields: $\Lson$ for the left child and $\Rson$ for the right child.
Our goal is to use the \QBUAd logic to show that the algorithm needs a stack size at least the size of the tree in the worst case.
The following rules about separation logic are useful, where $\ast$ denotes separating conjunction and $\Assign{x}{[y]}$ denotes a value load to the variable $x$ from the pointer $y$.
\begin{mathpar}\small
  \inferrule[(\textdagger:Load)]{}{\aajudge{x=x'*y\mapsto e}{P_R}{\Assign{x}{[y]}}{x=e[x'/x]*y\mapsto e[x'/x]}{P_R}}
  \hva\and
  \inferrule[(\textdaggerdbl:Frame)]{\AAjudge{P_S}{P_R}{C}{Q_S}{Q_R}\\\\\Mod(C)\cap\Fv(R)=\emptyset}{\AAjudge{P_S*R}{P_R}{C}{Q_S*R}{Q_R}}
\end{mathpar}

Concretely, the algorithm $DFS(x)$ described in the lower-left part outlines the traversal of a binary tree rooted at $x$, where ticks are modeled as the stack usage, and we aim to prove $\ddjudge{lchain(x)}{size(x)}{DFS(x)}{lchain(x)}{size(x)}$.
Here, $lchain$ and $size$ are recursive defined in the lower-right part, where $x \not\mapsto$ denotes $x$ does not point to any heap data and $x \mapsto (y,z)$ denotes $x$ points to a heap cell containing a pair $(y,z)$.
\begin{figure}[H]\small
\begin{minipage}[t]{0.45\textwidth}
\begin{algorithmic}[1]
  \State $DFS(x)\coloneqq$
  \Indent
  \State $\Tick{1};$
  \State $\IfB{x\ne\Null}~\Localx{y}~\Localx{z}$
  \Indent
  \State $\Assign{y}{x.\Lson};\Assign{z}{x.\Rson};$
  \State $\Ite{y\ne\Null}{DFS(y)}{\Skip};$
  \State $\Ite{z\ne\Null}{DFS(z)}{\Skip}$
  \EndIndent
  \State $\mathsf{else}~\Skip;$
  \State $\Tick{(-1)}$
  \EndIndent
\end{algorithmic}
\end{minipage}
\Gray{\vrule width 0.5pt}
\hspace{-1.5em}
\begin{minipage}[t]{0.45\textwidth}
\begin{align*}
  lchain(x)&\coloneqq x\not\mapsto\lor\exists y,z.x\mapsto(y,z)*lchain(y)*z\not\mapsto\\
  size(x)&\coloneqq\begin{cases}0&\text{if }x\not\mapsto\\1+size(y)+size(z)&\text{if }x\mapsto(y,z)\end{cases}
\end{align*}
\end{minipage}
\end{figure}

We inductively derive the triple $\ddjudge{P_n}{n}{DFS(x)}{P_n}{n}$, where $P_n\coloneqq lchain(x)\land size(x)=n$.
Note that we only need to show $\ddjudge{P_n}{n-1}{\text{(lines 3--7)}}{P_n}{n-1}$, and then use the \textsc{(\Bd:SeqR)} and \textsc{(\Bd:SeqL)} rules to compose with the two tick statements at the beginning and the end of the algorithm.

For the case $n=0$, we have $P_n=x\not\mapsto$, so the triple $\ddjudge{P_0}{-1}{\text{(lines 3--7)}}{P_0}{-1}$ is derivable by the \textsc{(\Bd:Skip)} and \textsc{(\Bd:IfFalse)} rules.
When $n>0$, we have $P_n=\exists y,z.Q_n$, where $Q_n\coloneqq x\mapsto(y,z)*lchain(y)\land size(y)=n-1*z\not\mapsto$.
We can derive $\bbjudge{Q_n}{n-1}{\text{(line 4)}}{Q_n}{n-1}$ using \textsc{(B:Load)}, and $\bbjudge{Q_n}{n-1}{\text{(line 6)}}{Q_n}{n-1}$ using \textsc{(B:IfFalse)}.
The remain task is to derive $\ddjudge{Q_n}{n-1}{\text{(line 5)}}{Q_n}{n-1}$, and then we can derive $\ddjudge{P_n}{n-1}{\text{(lines 3--7)}}{P_n}{n-1}$ by the \textsc{(\Bd:Local)} rule, as desired.

For the case $n=1$, we have $Q_1$ implies $y\not\mapsto$, so $\ddjudge{Q_n}{n-1}{\text{(line 5)}}{Q_n}{n-1}$ is derivable by the \textsc{(\Bd:Skip)} and \textsc{(\Bd:IfFalse)} rules.
When $n>1$, by induction hypothesis and the substitution rule, we have
\begin{align*}
  \ddjudge{lchain(y)\land size(y)=n-1}{n-1}{DFS(y)}{lchain(y)\land size(y)=n-1}{n-1}.
\end{align*}
Therefore, $\ddjudge{Q_n}{n-1}{\text{(line 5)}}{Q_n}{n-1}$ is derivable by the \textsc{(\Bd:Frame)} and \textsc{(\Bd:IfTrue)} rules, as required.

At the end, we derive $\ddjudge{lchain(x)}{size(x)}{DFS(x)}{lchain(x)}{size(x)}$ by combining the above cases with the \textsc{(\Bd:Disj)} rule.
