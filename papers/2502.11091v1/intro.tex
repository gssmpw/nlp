\section{Introduction}
\label{sec:intro}

% 2 pages

% Three useful guides to the organization of the intro section
%
% Organization A (Simon Peyton-Jones)
%  o Describe the problem
%  o State your contributions
%  o Organization of the paper [if not able to cover point by point
%    in contributions]
%
% Organization B (Nickolai Zeldovich)
%  o Elevator story -- high-level statement of the problem
%  o The problem in more technical terms
%  o Conventional wisdom: sketch of previous techniques and their
%    shortcomings
%  o Describe the solution to the problem as a black box,
%    so that it is clear what our solution offers over previous work
%  o Technical challenges to obtaining a solution (e.g., 1 paragraph
%    for each)
%  o State how we solve the challenges (at most a few paragraphs)
%  o Experimental highlights
%  o Contributions
%  o Organization of the paper
%
% Organization C (Derek Dreyer)
%  o Context: Set the stage, motivate the general topic
%  o Gap: Explain your specific problem and why existing work does not
%    adequately solve it
%  o Innovation: State what you've done that is new, and explain how it
%    helps fill the gap

Understanding and predicting the worst-case resource usage of programs is a fundamental challenge in computer science.
%
Knowing the bounds of resource usage, whether for memory consumption, CPU cycles, or network bandwidth, is crucial for ensuring system reliability, performance, and security.
%
However, precisely determining these bounds is notoriously difficult.
%
Static methods for analyzing resource usage tend to \emph{over-}approximate worst-case scenarios, resulting in non-tight resource bounds and, consequently, false positives.
%
On the other hand, dynamic methods, such as testing, can typically produce specific inputs of a certain size to induce high resource usage---thus \emph{under-}approximating worst-case scenarios---but do not offer an asymptotic characterization of the worst-case behavior.
%
This paper aims to develop a program logic to under-approximate worst-case resource usage and offer a compositional method for identifying scenarios with high resource usage.

Previous approaches to resource analysis have mainly concentrated on over-approximation methods, including abstract interpretation and constraint-based techniques.
%
These techniques aim to compute sound upper bounds on resource usage by taking into account all possible program behaviors.
%
However, they may produce non-tight bounds and typically do not indicate which inputs would result in the worst-case resource usage.
%
Under-approximation techniques, such as fuzzing and dynamic analysis, have been used to identify specific resource-intensive execution paths.
%
However, these techniques lack compositionality and generality for comprehensive resource analysis.
%
Incorrectness logic (IL), which has been successfully applied to bug detection, offers a promising alternative by providing a formal foundation for under-approximate reasoning~\cite{POPL:OHearn20}.
%
Recent work has adapted IL to prove non-termination~\cite{OOPSLA:RVO24}, i.e., a \emph{qualitative} argument about resource usage.
%
However, adapting IL to prove \emph{quantitative} resource bounds has not been explored.

In this paper, we adapt IL to under-approximate worst-case resource usage.
%
The key idea is to use under-approximate triples to reason about the existence of execution paths that lead to high resource usage.
%
Specifically, we introduce a new form of under-approximate triple that captures the relationship between program states and resource consumption.
%
Our approach leverages IL's compositional nature to analyze a program by breaking it down into smaller, manageable components.

One of the main challenges is to formalize resource usage within the IL framework.
%
The original IL triple $\fjudge{p}{C}{ok:q}$---also called a \emph{forward under-approximate} (FUA) triple---denotes that \emph{every} post-state that satisfies $q$ is \emph{reachable} by executing $C$ from \emph{some} pre-state that satisfies $p$.
%
Raad et al.~\cite{OOPSLA:RVO24} suggested using \emph{backward under-approximate} (BUA) triples when reasoning about non-termination; that is, $\bjudge{p}{C}{ok:q}$ denotes that \emph{every} pre-state that satisfies $p$ is possible to \emph{reach} \emph{some} post-state that satisfies $q$ by executing $C$.
%
The pre-conditions $p$ and post-conditions $q$ are \emph{qualitative}, i.e., they are Boolean-valued assertions on program states; however, they do not provide a natural way to reason about resource consumption.
%
In our work, we need to extend IL to capture the relationship between program states and resource usage.
%
We adapt the idea of \emph{quantitative Hoare logic}~\cite{PLDI:CHR14,PLDI:CHS15}: instead of the Boolean $true$, a quantitative assertion returns a natural number that indicates the amount of resource that is required to safely execute the program; at the same time, the Boolean $false$ is encoded by $\infty$.
%
Intuitively, quantitative Hoare logic \emph{refines} classic Hoare logic to reason about both the functionality and the resource usage of a program.
%
In this paper, we devise \emph{quantitative} generalizations of IL:
let $P$ and $ Q$ be assertions of type $State \to \bbZ \cup \{-\infty, +\infty\}$;
the FUA triple $\fjudge{P}{C}{Q}$ denotes that every post-state with \emph{at least} $Q$ units of resource is reachable by executing $C$ from some pre-state with \emph{at least} $P$ units of resource;
and the BUA triple $\bjudge{P}{C}{Q}$ denotes that every pre-state with \emph{at most} $P$ units of resource is possible to reach some post-state with \emph{at most} $Q$ units of resource by executing $C$.
%
In addition, we adopt Carbonneaux et al.~\cite{PLDI:CHS15}'s approach to separate qualitative assertions and quantitative resource bounds in the FUA/BUA triples to simplify the reasoning in our case studies.

Another challenge is to support reasoning about \emph{high-water marks} in our quantitative IL.
%
High-water marks provide a more precise characterization of a program's resource usage when the resource is non-monotone.
%
For example, when reasoning about stack-space bounds, we want to know the highest stack watermark \emph{during} a program's execution.
%
However, an FUA/BUA triple $\ajudge{P}{C}{Q}$ only constrains the amount of resource at the pre- and post-states;
that is, although the difference between $P$ and $Q$ serves as an under-approximation of the worst-case resource consumption of $C$, we cannot say that $P$ provides an under-approximation of the high-water mark of executing $C$.
%
It is worth noting that for quantitative Hoare logic $\vdash \left\{P\right\} C \left\{Q\right\}$, $P$ is indeed an over-approximation of the high-water mark.
%
In this paper, we devise a variant of the BUA logic to under-approximate high-water marks:
the triple $\djudge{P}{C}{Q}$ means that every pre-state with at most $P$ units of resource is possible to reach some post-state with at most $Q$ units of resource by executing $C$, \emph{and the particular execution must make the resource counter non-positive at some point}.
%
In this way, $P$ can serve as an under-approximation of the high-water mark of $C$ because, if we start to execute $C$ with $P$ units of resource, there exists an execution that consumes all the $P$ units at some point.
%
Note that different from quantitative Hoare logic, our quantitative IL allows the quantitative assertions $P,Q$ to take negative values, i.e., they have type $State \to \bbZ \cup \{{-}\infty,{+}\infty\}$.

In this paper, we focus on the theoretical properties of our quantitative IL for under-approximating worst-case resource usage.
%
We prove that both the FUA and BUA variants are sound and complete with respect to a resource-aware operational semantics for integer IMP programs.
%
To demonstrate the usefulness of our quantitative IL, we present several case studies and discuss possible extensions to support arrays (via the array theory), pointers (via the separation logic), and procedure calls.
%
In the future, we plan to automate our logic by integrating with existing tools, e.g., Pulse~\cite{OOPSLA:LRV22}.

%\paragraph*{Contributions}
%
In this paper, we make the following three main contributions.
%
\begin{itemize}
  \item We devise a program logic to under-approximate worst-case resource usage of programs. Our logic is a generalization of the incorrectness logic, and we formulate three variants: forward, backward, and backward high-water mark.
  \item We prove that our program logic (both the forward and the backward variants) is sound and complete with respect to a resource-aware operational semantics.
  \item We present several case studies to demonstrate the usefulness of our program logic. In particular, we discuss possible extensions to support arrays, pointers, and procedures.
\end{itemize}
