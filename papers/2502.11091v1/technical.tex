\section{Technical Details}
\label{sec:technical}

% 4 pages

% at the end: discuss how to support automation

In this section, we present the formal systems for the QFUA, QBUA, and \QBUAd logics for under-approximating worst-case resource consumption.
%
\cref{sec:technical:lang} formulates an IMP-style programming language with integer variables and tick commands.
%
\cref{sec:technical:qual} describes the QFUA and QBUA logics and proves their soundness and completeness.
%
\cref{sec:technical:hwm} extends QBUA to develop \QBUAd, which aims to under-approximate high-water marks.

\subsection{Syntax and Semantics}
\label{sec:technical:lang}

The language syntax defines resource-aware computation through the grammar:
\begin{align*}
C ::= \Skip \mid \Assign{x}{e} \mid \Assume{B} \mid \Tick{e} 
     \mid \Seq{C}{C} \mid \Choice{C}{C} \mid \Loop{C} \mid \Local{x}{C}
\end{align*}
The basic operations include: $\Skip$ denoting no-operation, $\Assign{x}{e}$ for variable assignments, $\Assume{B}$ constraining execution to paths satisfying boolean condition $B$, and $\Tick{e}$ modeling resource consumption by deducting the value of arithmetic expression $e$ from an implicit resource counter.
Control flow is structured through $\Seq{C_1}{C_2}$ for sequencing, $\Choice{C_1}{C_2}$ for non-deterministic choice, and $\Loop{C}$ for iterative execution. Variable scoping is managed via $\Local{x}{C}$.
All program variables are integer-valued.
Derived control structures are defined as: conditional execution $\Ite{B}{C_1}{C_2} \coloneqq \Choice{\Seq{(\Assume{B}}{C_1})}{(\Seq{\Assume{(\neg B)}}{C_2})}$ and looping constructs $\While{B}{C} \coloneqq \Seq{\Loop{\left(\Seq{\Assume{B}}{C}\right)}}{\Assume{(\neg B)}}$.

The big-step semantics formalizes program behavior with explicit resource tracking.
The judgment $\bigstep{C}{\sigma}{p}{l}{\tau}{q}$ states that executing $C$ from initial state $\sigma \in State \coloneqq Var \to \bbZ$ with resource $p \in \bbZ$ terminates in state $\tau \in State$ with residual resource $q \in \bbZ$, where $l \in \bbZ$ records the minimal resource level observed during execution.
The semantics rules are described in the Appendix (Figure~\ref{fig:fullsemantics}).
We also write $\bigstepp{C}{\sigma}{p}{\tau}{q}$ to denote $\exists l.\bigstep{C}{\sigma}{p}{l}{\tau}{q}$, and $\bigstepl{C}{\sigma}{p}{\tau}{q}$ to denote $\exists l\le 0.\bigstep{C}{\sigma}{p}{l}{\tau}{q}$.

%\begin{figure}[t!]
%\centering
%\begin{mathpar}\footnotesize
%  \inferrule[(BS:Skip)]{}{\bigstep{\Skip}{\sigma}{p}{p}{\sigma}{p}}
%  \hva\and
%  \inferrule[(BS:Assign)]{}{\bigstep{\Assign{x}{e}}{\sigma}{p}{p}{\sigma[x\mapsto\eval{e}{\sigma}]}{p}}
%  \hva\and
%  \inferrule[(BS:Assume)]{\eval{B}{\sigma}=\true}{\bigstep{\Assume{B}}{\sigma}{p}{p}{\sigma}{p}}
%  \hva\and
%  \inferrule[(BS:Tick)]{}{\bigstep{\Tick{e}}{\sigma}{p}{\min\{p,p-\eval{e}{\sigma}\}}{\sigma}{p-\eval{e}{\sigma}}}
%  \hva\and
%  \inferrule[(BS:Seq)]{\bigstep{C_1}{\sigma}{p}{l_1}{\rho}{r} \\ \bigstep{C_2}{\rho}{r}{l_2}{\tau}{q}}{\bigstep{\Seq{C_1}{C_2}}{\sigma}{p}{\min\{l_1,l_2\}}{\tau}{q}}
%  \hva\and
%  \inferrule[(BS:ChoiceL)]{\bigstep{C_1}{\sigma}{p}{l}{\tau}{q}}{\bigstep{\Choice{C_1}{C_2}}{\sigma}{p}{l}{\tau}{q}}
%  \hva\and
%  \inferrule[(BS:ChoiceR)]{\bigstep{C_2}{\sigma}{p}{l}{\tau}{q}}{\bigstep{\Choice{C_1}{C_2}}{\sigma}{p}{l}{\tau}{q}}
%  \hva\and
%  \inferrule[(BS:LoopZero)]{}{\bigstep{\Loop{C}}{\sigma}{p}{p}{\sigma}{p}}
%  \hva\and
%  \inferrule[(BS:Loop)]{\bigstep{\Seq{C}{\Loop{C}}}{\sigma}{p}{l}{\tau}{q}}{\bigstep{\Loop{C}}{\sigma}{p}{l}{\tau}{q}}
%  \hva\and
%  \inferrule[(BS:Local)]{\bigstep{C}{\sigma}{p}{l}{\tau}{q}}{\bigstep{\Local{x}{C}}{\sigma[x\mapsto v]}{p}{l}{\tau[x\mapsto v]}{q}}
%\end{mathpar}
%\caption{Big-step semantics rules}
%\label{fig:semantics}
%\end{figure} % maybe in appendix

\subsection{Quantitative Under-approximate Logic}
\label{sec:technical:qual}

We generalize both forward and backward under-approximate logic to quantitative reasoning by extending assertions from Boolean predicates $State \to \{\true, \false\}$ to \emph{resource functions} $State \to \mathbb{Z} \cup \{-\infty, +\infty\}$. These functions map program states to integer-valued resource quantities (with $\pm\infty$ denoting Boolean truth and falsehood). The generalized logics are called \emph{quantitative forward under-approximate} (QFUA) logic and \emph{quantitative backward under-approximate} (QBUA) logic, whose semantics are defined in terms of resource functions.

\begin{definition}
  The semantics of QFUA and QBUA triples are defined as follows:
  \begin{itemize}
    \item QFUA: $\fJudge{P}{C}{Q}$ holds if and only if for all $\tau$ and $q$ such that $q\ge Q(\tau)$, there exists a $\sigma$ and a $p$ such that $p\ge P(\sigma)$ and $\bigstepp{C}{\sigma}{p}{\tau}{q}$. This implies that for all $\tau$ such that $Q(\tau)\in\mathbb{Z}$, there exists an execution of $C$ ending in $\tau$ that cost at least $\inf_\sigma\{P(\sigma)\}-Q(\tau)$ ticks.
    \item QBUA: $\bJudge{P}{C}{Q}$ holds if and only if for all $\sigma$ and $p$ such that $p\le P(\sigma)$, there exists a $\tau$ and a $q$ such that $q\le Q(\tau)$ and $\bigstepp{C}{\sigma}{p}{\tau}{q}$. This implies that for all $\sigma$ such that $P(\sigma)\in\mathbb{Z}$, there exists an execution of $C$ starting from $\sigma$ that cost at least $P(\sigma)-\sup_\tau\{Q(\tau)\}$ ticks.
  \end{itemize}
\end{definition}

\begin{figure}[t!]
\begin{mathpar}\footnotesize
  P\curlywedge Q\coloneqq\lambda\sigma.\min\{P(\sigma),Q(\sigma)\}
  \hva\and
  P\curlyvee Q\coloneqq\lambda\sigma.\max\{P(\sigma),Q(\sigma)\}
  \hva\and
  \Sup x.P\coloneqq\lambda\sigma.\sup_v\{P(\sigma[x\mapsto v])\}
  \hva\and
  \Inf x.P\coloneqq\lambda\sigma.\inf_v\{P(\sigma[x\mapsto v])\}
  \hva\and
  P\preceq Q\quad\text{iff}\quad\forall\sigma.P(\sigma)\leq Q(\sigma)
  \hva\and
  [B]\coloneqq\lambda\sigma.\begin{cases}+\infty&\text{if }B(\sigma)=\true\\-\infty&\text{otherwise}\end{cases}
\end{mathpar}
\caption{Resource Function Operators}
\label{fig:operators}
\end{figure}

To formulate quantitative verification rules, we define operators on resource functions as shown in Figure~\ref{fig:operators}. The $\curlyvee$ and $\curlywedge$ operators compute pointwise maximum and minimum of two functions respectively, while $\Sup x.P$ and $\Inf x.P$---adapted from Batz et al.~\cite{POPL:BKK21}'s work and Zhang and Kaminski~\cite{OOPSLA:ZK22}'s work---capture external values over variable assignments. The refinement operator $\preceq$ establishes pointwise ordering between functions, and $[B]$ converts Boolean predicates by mapping truth values to infinite bounds.

These operators extend Boolean logic to integer-valued resource analysis. QFUA and QBUA triples use different interpretations of logical operators due to their opposite inequality directions in their semantics.
To illustrate this generalization, consider the case of conjunction: in Boolean logic, $P \land Q$ means both predicates must hold. For QFUA triples, this translates to $P \curlyvee Q$, as satisfying both $p \geq P(\sigma)$ and $p \geq Q(\sigma)$ is equivalent to $p \geq \max\{P(\sigma), Q(\sigma)\}$. Conversely, QBUA triples use $P \curlywedge Q$, because $p \leq P(\sigma)$ and $p \leq Q(\sigma)$ is equivalent to $p \leq \min\{P(\sigma), Q(\sigma)\}$. 
Consider also quantitative existential quantification: in QFUA triples, $p \geq \Inf x.P(\sigma)$ guarantees $\exists v.\ p \geq P(\sigma[x \mapsto v])$, because the infimum becomes the minimum in discrete domains. The QBUA triples case follows a similar pattern.
%
Similar duality extends to other operators (see Table~\ref{fig:correspondence} in the Appendix).
Lastly, the Boolean predicate conversion $[B]$ bridges Boolean and quantitative reasoning. For QFUA triples, $[\neg B]$ assigns $-\infty$ when $B$ is true (because $p \geq -\infty$ always holds), and $+\infty$ otherwise. For QUBA triples, $[B]$ assigns $+\infty$ when $B$ is true (because $p \leq +\infty$ always holds), and $-\infty$ otherwise.

\begin{figure}[t!]
\centering
\begin{mathpar}\footnotesize
  \inferrule[(\textdagger:Skip)]{}{\ajudge{P}{\Skip}{P}}
  \hva\and
  \Blue{\inferrule[(F:Assign)]{}{\fjudge{P}{\Assign{x}{e}}{\Inf x'.P[x'/x]\curlyvee[x\ne e[x'/x]]}}}
  \hva\and
  \Red{\inferrule[(B:Assign)]{}{\bjudge{P}{\Assign{x}{e}}{\Sup x'.P[x'/x]\curlywedge[x=e[x'/x]]}}}
 \hva\and
  \Blue{\inferrule[(F:Assume)]{}{\fjudge{P\curlyvee[\neg B]}{\Assume{B}}{P\curlyvee[\neg B]}}}
  \hva\and
  \Red{\inferrule[(B:Assume)]{}{\bjudge{P\curlywedge[B]}{\Assume{B}}{P\curlywedge[B]}}}
  \hva\and
  \inferrule[(\textdagger:Tick)]{}{\ajudge{P}{\Tick{e}}{P-e}}
  \hva\and
  \inferrule[(\textdagger:Seq)]{\ajudge{P}{C_1}{R} \\ \ajudge{R}{C_2}{Q}}{\ajudge{P}{\Seq{C_1}{C_2}}{Q}}
  \hva\and
  \inferrule[(\textdagger:ChoiceL)]{\ajudge{P}{C_1}{Q}}{\ajudge{P}{\Choice{C_1}{C_2}}{Q}}
  \hva\and
  \inferrule[(\textdagger:ChoiceR)]{\ajudge{P}{C_2}{Q}}{\ajudge{P}{\Choice{C_1}{C_2}}{Q}}
  \hva\and
  \inferrule[(\textdagger:Loop)]{\forall n<k.\ajudge{P(n)}{C}{P(n+1)}}{\ajudge{P(0)}{\Loop{C}}{P(k)}}
  \hva\and
  \Blue{\inferrule[(F:Local)]{\fjudge{P}{C}{Q}}{\fjudge{\Inf x.P}{\Local{x}{C}}{\Inf x.Q}}}
  \hva\and
  \Red{\inferrule[(B:Local)]{\bjudge{P}{C}{Q}}{\bjudge{\Sup x.P}{\Local{x}{C}}{\Sup x.Q}}}
  \hva\and
  \Blue{\inferrule[(F:Disj)]{\forall i\in I.\fjudge{P_i}{C}{Q_i}}{\fjudge{\bigcurlywedge_{i\in I}P_i}{C}{\bigcurlywedge_{i\in I}Q_i}}}
  \hva\and
  \Red{\inferrule[(B:Disj)]{\forall i\in I.\bjudge{P_i}{C}{Q_i}}{\bjudge{\bigcurlyvee_{i\in I}P_i}{C}{\bigcurlyvee_{i\in I}Q_i}}}
  \hva\and
  \Blue{\inferrule[(F:Constancy)]{\fjudge{P}{C}{Q}\\\Fv(B)\cap\Mod(C)=\emptyset}{\fjudge{P\curlyvee[B]}{C}{Q\curlyvee[B]}}}
  \hva\and
  \Red{\inferrule[(B:Constancy)]{\bjudge{P}{C}{Q}\\\Fv(B)\cap\Mod(C)=\emptyset}{\bjudge{P\curlywedge[B]}{C}{Q\curlywedge[B]}}}
  \hva\and
  \inferrule[(\textdagger:Relax)]{\ajudge{P}{C}{Q}\\\Fv(F)\cap\Mod(C)=\emptyset}{\ajudge{P+F}{C}{Q+F}}
  \hva\and
  \inferrule[(\textdagger:Cons)]{P\preceq P'\\\ajudge{P'}{C}{Q'}\\Q'\preceq Q}{\ajudge{P}{C}{Q}}
  \hva\and
  \inferrule[(\textdagger:Subst)]{\ajudge{P}{C}{Q}\\y\notin\Fv(P)\cup\Fv(Q)\cup\Fv(C)}{\ajudge{P[y/x]}{C[y/x]}{Q[y/x]}}
\end{mathpar}
\caption{Proof Rules for QFUA and QBUA Triples}
\label{fig:rules}
\end{figure}

We present the proof rules for QFUA and QBUA triples in Figure~\ref{fig:rules}. Rules marked with \textdagger\ apply to both QFUA and QBUA triples, while those in \Blue{blue} are specific to QFUA triples, and those in \Red{red} are specific to QBUA triples. Except for the \textsc{(\textdagger:Tick)} and the \textsc{(\textdagger:Relax)} rule, all other rules are generalized from the corresponding Boolean cases, with the operators replaced by their quantitative counterparts.

Concretely, the \textsc{(\textdagger:Skip)} rule trivially preserves quantitative assertions.
The \textsc{(F:Assign)} and \textsc{(B:Assign)} rules for assignment statements are generalized from the standard Floyd assignment rule $\ajudge{P}{\Assign{x}{e}}{\exists x'.P[x'/x] \land x=e[x'/x]}$, which is sound for both QFUA and QBUA triples.
The \textsc{(F:Assume)} and \textsc{(B:Assume)} rules for assume statements are generalized from the assume rule $\ajudge{P\land B}{\Assume{B}}{P\land B}$, which filters out states that do not satisfy the predicate $B$.
The \textsc{(\textdagger:Tick)} rule models the effect of a tick statement by subtracting the tick cost from the resource function.

The \textsc{(\textdagger:Seq)} rule models the sequential composition of two programs, where the postcondition of the first program coincides with the precondition of the second program.
The \textsc{(\textdagger:ChoiceL)} and \textsc{(\textdagger:ChoiceR)} rules approximate non-deterministic choice by considering the behavior of one of the branches.
The \textsc{(\textdagger:Loop)} rule uses an indexed resource function $P(n)$ as a subvariant to approximate loop behavior: if $C$ transforms $P(n)$ to $P(n+1)$ for all $n < k$, then $P(0)$ can transform to $P(k)$ via $\Loop{C}$.
The \textsc{(F:Local)} and \textsc{(B:Local)} rules use the $\Inf$ and $\Sup$ quantifiers, respectively, to erase the effect of local variables, which generalizes the existential quantifier in the Boolean case.

The \textsc{(F:Disj)} and \textsc{(B:Disj)} rules merge multiple triples into a single one by the quantitative corresponding of the disjunction operator.
The \textsc{(F:Constancy)} and \textsc{(B:Constancy)} rule preserves unmodified Boolean constraints — when program $C$ doesn't modify $B$'s free variables, both pre- and post- predicates can be simultaneously conjuncted with $B$ through the quantitative operators.
The \textsc{(\textdagger:Relax)} rule allows additive adjustments to resource functions: if $F$'s variables remain unchanged by $C$, both assertion bounds can be shifted by $F$ through arithmetic addition.
The (\textdagger:Cons) rule shares identical syntax for both QFUA and QBUA. However, the implication direction is different: QFUA rules allow weakening of preconditions and strengthening of postconditions, while QBUA rules allow the opposite.

We next show that the QBUA and QFUA rules are sound and complete with respect to the corresponding semantics.

\begin{theorem}[Soundness of QFUA and QBUA triples]\label{thm:QUA-soundness}
  For all $P,Q,C$:
  \begin{itemize}
    \item If $\fjudge{P}{C}{Q}$ is derivable, then $\fJudge{P}{C}{Q}$ holds.
    \item If $\bjudge{P}{C}{Q}$ is derivable, then $\bJudge{P}{C}{Q}$ holds.
  \end{itemize}
\end{theorem}
\begin{proof}
  By induction on the derivation of the triple. See the Appendix (\cref{thm:qfua-sound,thm:qbua-sound}).
\end{proof}

\begin{theorem}[Completeness of QFUA and QBUA triples]\label{thm:QUA-completeness}
  For all $P,Q,C$:
  \begin{itemize}
    \item If $\fJudge{P}{C}{Q}$ holds, then $\fjudge{P}{C}{Q}$ is derivable.
    \item If $\bJudge{P}{C}{Q}$ holds, then $\bjudge{P}{C}{Q}$ is derivable.
  \end{itemize}
\end{theorem}
\begin{proof}
  By induction on the structure of $C$. See the Appendix (\cref{thm:qfua-complete,thm:qbua-complete}).
\end{proof}

\subsection{Under-approximating High-water Marks}
\label{sec:technical:hwm}

\begin{figure}[t!]
\centering
\begin{mathpar}\footnotesize
  \inferrule[(\Bd:Skip)]{P\preceq 0}{\djudge{P}{\Skip}{P}}
  \hva\and
  \inferrule[(\Bd:Assign)]{P\preceq 0}{\djudge{P}{\Assign{x}{e}}{\Sup x'.P[x'/x]\curlywedge[x=e[x'/x]]}}
  \hva\and
  \inferrule[(\Bd:Assume)]{P\preceq 0}{\djudge{P\curlywedge[B]}{\Assume{B}}{P\curlywedge[B]}}
  \hva\and
  \inferrule[(\Bd:Tick)]{P\curlywedge P-e\preceq 0}{\djudge{P}{\Tick{e}}{P-e}}
  \hva\and
  \inferrule[(\Bd:SeqL)]{\djudge{P}{C_1}{R} \\ \bjudge{R}{C_2}{Q}}{\djudge{P}{\Seq{C_1}{C_2}}{Q}}
  \hva\and
  \inferrule[(\Bd:SeqR)]{\bjudge{P}{C_1}{R} \\ \djudge{R}{C_2}{Q}}{\djudge{P}{\Seq{C_1}{C_2}}{Q}}
  \hva\and
%  \inferrule[(\Bd:ChoiceL)]{\djudge{P}{C_1}{Q}}{\djudge{P}{C_1+C_2}{Q}}
%  \hva\and
%  \inferrule[(\Bd:ChoiceR)]{\djudge{P}{C_2}{Q}}{\djudge{P}{C_1+C_2}{Q}}
%  \hva\and
  \inferrule[(\Bd:LoopZero)]{P\preceq 0}{\djudge{P}{C^\star}{P}}
  \hva\and
  \inferrule[(\Bd:Loop)]{\forall n<k.\bjudge{P(n)}{C}{P(n+1)} \\ \exists m<k.\djudge{P(m)}{C}{P(m+1)}}{\djudge{P(0)}{C^\star}{P(k)}}
  \hva\and
%  \inferrule[(\Bd:Local)]{\djudge{P}{C}{Q}}{\djudge{\Sup x.P}{\Local{x}{C}}{\Sup x.Q}}
  \hva\and
  \inferrule[(\Bd:Disj)]{\forall i\in I.\djudge{P_i}{C}{Q_i}}{\djudge{\bigcurlyvee_{i\in I}P_i}{C}{\bigcurlyvee_{i\in I}Q_i}}
  \hva\and
%  \inferrule[\Bd:Constancy)]{\djudge{P}{C}{Q}\\\Fv(B)\cap\Mod(C)=\emptyset}{\djudge{P\curlywedge[B]}{C}{Q\curlywedge[B]}}
%  \hva\and
  \inferrule[(\Bd:Relax)]{\djudge{P}{C}{Q}\\\Fv(F)\cap\Mod(C)=\emptyset\\F\preceq 0}{\djudge{P+F}{C}{Q+F}}
  \hva\and
%  \inferrule[(\Bd:Cons)]{P\preceq P' \\ \djudge{P'}{C}{Q'} \\ Q'\preceq Q}{\djudge{P}{C}{Q}}
%  \hva\and
%  \inferrule[(\Bd:Subst)]{\djudge{P}{C}{Q}\\y\notin\Fv(P)\cup\Fv(Q)\cup\Fv(C)}{\djudge{P[y/x]}{C[y/x]}{Q[y/x]}}
\end{mathpar}
\caption{Selected Proof Rules for \QBUAd Triples (Full Rules in the Appendix, Figure~\ref{fig:fullqbuad})}
\label{fig:QBUAd-rules}
\end{figure}

To under-approximate the worst-case high-water mark,  we refine the QBUA semantics to track minimal resource levels during execution. We introduce \QBUAd triples, whose semantics are defined as follows.

\begin{definition}
  A \QBUAd triple $\dJudge{P}{C}{Q}$ holds if and only if for all $\sigma$ and $p$ such that $p\le P(\sigma)$, there exists a $\tau$ and a $q$ such that $q\le Q(\tau)$ and $\bigstepl{C}{\sigma}{p}{\tau}{q}$.
  This implies that for all $\sigma$ such that $P(\sigma)\in\mathbb{Z}$, there exists an execution of $C$ starting from $\sigma$ that costs at least $P(\sigma)-\sup_{\tau}\{Q(\tau)\}$ ticks, and has a high-water mark of at least $P(\sigma)$.
\end{definition}

Selected proof rules for \QBUAd triples are shown in Figure~\ref{fig:QBUAd-rules}, which are similar to the QBUA rules but with additional constraints.
The \textsc{(\Bd:Skip)}, \textsc{(\Bd:Assign)}, and \textsc{(\Bd:Assume)} rules assume that the resource function $P$ is non-positive.
The \textsc{(\Bd:Tick)} rule requires $P\curlywedge P-e\preceq 0$, which ensures that either before or after the tick operation, the resource count is non-positive.
For the sequential composition, if an execution of $\Seq{C_1}{C_2}$ makes the resource counter non-positive at some point, then the point will occur either during the execution of $C_1$ or $C_2$. This is captured by the \textsc{(\Bd:SeqL)} and \textsc{(\Bd:SeqR)} rules.
The loop is similar: if at some point during the execution of $\Loop{C}$ the resource is non-positive, then either $\Loop{C}$ degenerates into a $\Skip$ and the resource is non-positive from the start, or this point will appear during the execution of $C$ at some step. The former case is captured by the \textsc{(\Bd:LoopZero)} rule, while the latter is captured by the subvariant-base \textsc{(\Bd:Loop)} rule, with assuming $\exists m<k.\djudge{P(m)}{C}{P(m+1)}$.
The \textsc{(\Bd:Disj)} rule is similar to the \textsc{(B:Disj)} rule.
Finally, the \textsc{(\Bd:Relax)} rule allows only non-positive additive adjustments to resource functions, which ensures that the point of non-positivity resource is preserved.

The \QBUAd rules are also sound and complete, as shown in the following theorem.

\begin{theorem}[Soundness of \QBUAd triples]
  For all $P,Q,C$, if $\djudge{P}{C}{Q}$ is derivable, then $\dJudge{P}{C}{Q}$ holds.
\end{theorem}

\begin{proof}
  By induction on the derivation of the triple. Details are in the Appendix (Theorem~\ref{thm:qbuad-sound}).
\end{proof}

\begin{theorem}[Completeness of \QBUAd triples]
  For all $P,Q,C$, if $\dJudge{P}{C}{Q}$ holds, then $\djudge{P}{C}{Q}$ is derivable.
\end{theorem}

\begin{proof}
  By induction on the structure of $C$. The sequential composition case is shown as follows, while the other cases are in the Appendix (Theorem~\ref{thm:qbuad-complete}).

  \textbf{Case} $C=\Seq{C_1}{C_2}$: Let $P,Q$ be arbitrary such that $\dJudge{P}{\Seq{C_1}{C_2}}{Q}$. From the semantics of the \QBUAd triple and $\Seq{C_1}{C_2}$, we have that for any $\sigma$ and $p$ satisfying $p\le P(\sigma)$, there exists $\tau,q,\rho,r$ such that one of the following holds:
  \begin{itemize}
    \item (1) $q\le Q(\tau)$, $\bigstepl{C_1}{\sigma}{p}{\rho}{r}$, and $\bigstepp{C_2}{\rho}{r}{\tau}{q}$;
    \item (2) $q\le Q(\tau)$, $\bigstepp{C_1}{\sigma}{p}{\rho}{r}$, and $\bigstepl{C_2}{\rho}{r}{\tau}{q}$.
  \end{itemize}
  So we can divide $P$ into $P=P_1\curlyvee P_2$, where $P_i(\sigma)=\sup\{p:p\le P(\sigma)\land\text{ there exists }\tau,q,\rho,r\text{ such that (}i\text{) holds}\}$ for $i\in\{1,2\}$.
  
  For the case (1), let $R_1=\lambda\rho.\sup\{r:\exists\tau,q.q\le Q(\tau)\land\bigstepp{C}{\rho}{r}{\tau}{q}\}$, then $\dJudge{P_1}{C_1}{R_1}$ and $\bJudge{R_1}{C_2}{Q}$ holds. By induction hypothesis and Theorem~\ref{thm:QUA-completeness}, we have $\djudge{P_1}{C_1}{R_1}$ and $\bjudge{R_1}{C_2}{Q}$. Thus, we can derive $\djudge{P_1}{\Seq{C_1}{C_2}}{Q}$ using \textsc{(\Bd:SeqL)}.
  
  For the case (2), let $R_2=\lambda\rho.\sup\{r:\exists\tau,q.q\le Q(\tau)\land\bigstepl{C}{\rho}{r}{\tau}{q}\}$, then $\bJudge{P_2}{C_1}{R_2}$ and $\dJudge{R_2}{C_2}{Q}$ holds. By Theorem~\ref{thm:QUA-completeness} and induction hypothesis, we have $\bjudge{P_2}{C_1}{R_2}$ and $\djudge{R_2}{C_2}{Q}$. Thus, we can derive $\djudge{P_2}{\Seq{C_1}{C_2}}{Q}$ using \textsc{(\Bd:SeqR)}.
  
  Finally, we can derive $\djudge{P}{\Seq{C_1}{C_2}}{Q}$ using \textsc{(\Bd:Disj)}.
\end{proof}
