\section{Overview}
\label{sec:overview}

% 4 pages

% Background: IL, Nonter (FUA & BUA)
% Background: C4B Quantitative Hoare Logic
% Our solution: use examples

We first review incorrectness logic and non-termination logic to discuss the concept of under-approximation (\cref{sec:overview:il}).
%
Next, we examine quantitative Hoare logic to introduce the idea of resource-aware program logic (\cref{sec:overview:qhl}).
%
Finally, we outline our approach to developing quantitative under-approximate logic to reason about worst-case resource usage (\cref{sec:overview:qual}).

\subsection{Prior Work: Incorrectness Logic and Non-termination Logic}
\label{sec:overview:il}

Incorrectness logic (IL), introduced by O'Hearn~\cite{POPL:OHearn20}, provides a formal foundation for under-approximate reasoning about program behaviors, particularly for bug detection.
%
Unlike classic Hoare logic, which focuses on proving program correctness by over-approximating program behaviors, IL under-approximates program behaviors to identify specific execution paths that lead to bugs.
%
This makes IL well-suited for detecting errors such as memory safety violations~\cite{CAV:RBD20}, concurrency bugs~\cite{POPL:RBD22}, and non-termination~\cite{OOPSLA:RVO24}, where the goal is to find concrete evidence of incorrect behavior rather than proving the absence of errors.

The key idea behind IL is the use of \emph{forward, under-approximate} (FUA) triples, written as $\fjudge{p}{C}{\epsilon : q}$, which state that starting from a set of pre-states $p$, executing program $C$ can lead to a set of post-states under-approximated by $q$ under the exit condition $\epsilon$ (e.g., normal termination $ok$ or error $er$).
%
The under-approximate nature of FUA triples ensures that any bug detected is a true positive, as it corresponds to a concrete execution path that exhibits the error.
%
Below are two rules for under-approximating the behavior of while loops in IL.
%
Different from invariant-based reasoning in classic Hoare logic, IL essentially \emph{unrolls} a loop to examine a subset of execution paths through the loop.
%
The rule \textsc{(IL:WhileFalse)} states that if the loop condition $B$ does not hold before entering the loop, the pre-states and post-states coincide because the loop is not entered.
%
The rule \textsc{(IL:WhileSubvar)} provides \emph{subvariant}-based reasoning, which generalizes bounded unrolling by asserting that there exists some $k$ satisfying that the loop body $C$ transforms $p(n) \wedge B$ to $p(n+1) \wedge B$ for $n<k$, as well as that $C$ transforms $p(k) \wedge B$ to $q \wedge \neg B$, i.e., the loop could end after $k$ iterations.
%
\begin{mathpar}\small
    \Rule{IL:WhileFalse}
    { }
    { \fjudge{p \wedge \neg B}{\While{B}{C}}{ok: p \wedge \neg B} }
    \and
    \Rule{IL:WhileSubvar}
    { \forall n < k. \fjudge{p(n) \wedge B}{C}{ok: p(n+1) \wedge B} \\\\
      \fjudge{p(k) \wedge B}{C}{\epsilon: q \wedge \neg B}
    }
    { \fjudge{p(0) \wedge B}{\While{B}{C}}{\epsilon: q \wedge \neg B} }
\end{mathpar}

Among the prior extensions of IL, the \emph{under-approximate, non-termination logic} (\textsc{UNTer}) is the most related work because it reasons---qualitatively---about the usage of a particular kind of resource, i.e., time~\cite{OOPSLA:RVO24}.
%
\textsc{UNTer} introduces divergent triples of the form $\judge{p}{C}{\infty}$, which state that starting from any state in $p$, the program $C$ has at least one divergent (i.e., non-terminating) execution.
%
A key insight in \textsc{UNTer} is the use of \emph{backward, under-approximate} (BUA) triples, written as $\bjudge{p}{C}{\epsilon: q}$, which state that starting from a set of pre-states $p$, executing program $C$ can reach some post-states in $q$ under the exit condition $\epsilon$.
%
Consider the program $\Assign{x}{x-1}$.
%
The FUA judgement $\fjudge{x>0}{\Assign{x}{x-1}}{ok: x>0}$ is valid, because the precise set of post-states is $x \ge 0$ and $x > 0$ is its subset, i.e., an under-approximation.
%
On the other hand, the BUA judgement $\bjudge{x>0}{\Assign{x}{x-1}}{ok: x > 0}$ is \emph{invalid}, because executing from the pre-state $x=1$ results in $x=0$, which is not included in the set $x>0$.
%
\textsc{UNTer} uses BUA triples to reason about the non-termination of loops.
%
For example, the rule \textsc{(UNTer:While)} shown below states that if executing the loop body $C$ keeps the states $p$ and the loop condition $B$ unchanged, we can establish that the loop can diverge.
%
If the rule used FUA triples instead, it would prove the triple $\judge{x > 0}{\While{x>0}{\Assign{x}{x-1}}}{\infty}$, which is unsound.
%
Another rule \textsc{(UNTer:WhileSubvar)} is similar to \textsc{(IL:WhileSubvar)}, providing subvariant-based reasoning about divergent loops.
%
\begin{mathpar}\small
  \Rule{UNTer:While}
  { \bjudge{p \wedge B}{C}{ok: p \wedge B} }
  { \judge{p \wedge B}{\While{B}{C}}{\infty} }
%  \and
%  \Rule{UNTer:WhileNest}
%  { \judge{p \wedge B}{C}{\infty} }
%  { \judge{p \wedge B}{\While{B}{C}}{\infty} }
  \and
  \Rule{UNTer:WhileSubvar}
  { \forall n \in \bbN. \bjudge{p(n) \wedge B}{C}{p(n+1) \wedge B} }
  { \judge{p(0) \wedge B}{\While{B}{C}}{\infty} }
\end{mathpar}

%In our work, we follow the under-approximate reasoning principle behind IL and \textsc{UNTer} to reason---\emph{quantitatively}---about the worst-case resource usage of a program.

\subsection{Prior Work: Quantitative Hoare Logic}
\label{sec:overview:qhl}

Our work is not the first to consider developing program logic to derive resource-usage bounds.
%
We take inspiration from the \emph{quantitative Hoare logic} (QHL), which extends classic Hoare logic by incorporating resource consumption into program reasoning~\cite{PLDI:CHR14,PLDI:CHS15}.

The key idea behind QHL is to augment Hoare triples with resource annotations.
%
A \emph{quantitative Hoare triple}, written as $\pjudge{P}{C}{Q}$ where $P$ and $Q$ are quantitative assertions of type $State \to \bbN \cup \{ \infty\}$, states that if $C$ is executed in a pre-state with \emph{at least} $P$ units of resource, then after its execution, \emph{at least} $Q$ units of resource will remain.
%
Intuitively, these quantitative assertions can be seen as \emph{potential functions} that map program states to non-negative potentials: the pre-potential $P$ is sufficient to pay for the resource usage of $C$ as well as the post-potential $Q$.
%
In other words, QHL essentially captures the principle of the \emph{potential method} for amortized complexity analysis~\cite{JADM:Tarjan85}.

Below are three representative rules of QHL.
%
To explicitly annotate resource consumption, people usually introduce a primitive command $\Tick{e}$ that computes $e$ to an integer $n$ and then consumes $n$ units of resource.
(If $n$ is negative, this tick command releases $-n$ units of resource.)
%
The rule \textsc{(QHL:Tick)} states that if the pre-potential is at least $P + e$, then it is safe to execute $\Tick{e}$ and end with post-potential $P$.
%
The rule \textsc{(QHL:Seq)} illustrates the \emph{compositional} nature of QHL: to reason about the resource usage of the sequencing command $\Seq{C_1}{C_2}$, one can derive quantitative triples \emph{individually} for $C_1$ and $C_2$.
%
The rule \textsc{(QHL:While)} generalizes the invariant-based reasoning of classic Hoare logic, where the predicate $\mathsf{istrue}(B)$ encodes a potential function that returns $0$ if $B$ is true and $\infty$ otherwise.
%
\begin{mathpar}\small
  \Rule{QHL:Tick}
  { P + e \ge 0 }
  { \pjudge{P + e}{\Tick{e}}{P} }
  \and
  \Rule{QHL:Seq}
  { \pjudge{P}{C_1}{R} \\ \pjudge{R}{C_2}{Q} }
  { \pjudge{P}{\Seq{C_1}{C_2}}{Q} }
  \and
  \Rule{QHL:While}
  { \pjudge{I + \mathsf{istrue}(B)}{C}{I} }
  { \pjudge{I}{\While{B}{C}}{I + \mathsf{istrue}(\neg B)} }
\end{mathpar}

Consider the program $\While{x<n}{(\Seq{\Assign{x}{x+1}}{\Tick{1}})}$.
%
Intuitively, the total resource consumption is given by $I \coloneqq \max(0, n - x)$.
%
Using $I$ as the loop invariant, we can prove the judgement $\pjudge{I}{\While{x<n}{(\Assign{x}{x+1};\Tick{1})}}{0}$.
%
By \textsc{(QHL:While)}, this amounts to prove $\pjudge{I+\mathsf{istrue}(x<n)}{\Assign{x}{x+1}; \Tick{1}}{I}$.
%
If $x \ge n$, then the pre-potential is $\infty$ and thus the triple is valid;
otherwise, we can use \textsc{(QHL:Seq)} with $\max(0,n-x) + 1$ as the intermediate assertion and conclude by $\max(0,n-(x+1))+1 = \max(0,n-x)$ when $x<n$.

It is worth noting that QHL also supports reasoning about \emph{high-water marks}.
%
In fact, the quantitative triple $\pjudge{P}{C}{Q}$ indicates that $P$ is an upper bound (i.e., an over-approximation) of the high-water mark of executing $C$.
%
Consider the non-terminating program $\While{\mathsf{true}}{(\Tick{1}; \Tick{{-1}})}$.
%
We can prove the judgement $\pjudge{1}{\While{\mathsf{true}}{(\Tick{1};\Tick{{-1}})}}{1}$, i.e., the high-water mark of the program is upper-bounded by $1$.
%
Again, by \textsc{(QHL:While)}, this amounts to prove $\pjudge{1}{\Tick{1}; \Tick{{-1}}}{1}$.
%
This can be justified by using \textsc{(QHL:Seq)} with $0$ as the intermediate assertion.
%
Note that the triple also indicates that after the program terminates, the resource consumption is upper-bounded by $1-1=0$.
%
However, because the program is non-terminating, the triple is indeed an over-approximation in terms of resource consumption.

\subsection{This Work: Quantitative Under-approximate Logic}
\label{sec:overview:qual}

In this work, we devise \emph{quantitative under-approximate logic}, which can be seen as the incorrectness-logic variant of quantitative Hoare logic.
%
Rather than deriving sound upper bounds (i.e., over-approximations) on resource consumption, we aim to develop a program logic to prove lower bounds (i.e., under-approximations) on the worst-case resource consumption.
%
Generalizing FUA and BUA triples from incorrectness logic by replacing assertions with quantitative ones, we obtain \emph{quantitative, forward/backward, under-approximate} (QFUA/QBUA) triples to reason about worst-case resource usage:
\begin{itemize}
  \item A QFUA judgement $\fjudge{P}{C}{Q}$ states that every post-potential \emph{no less than} $Q$ is reachable by executing $C$ from some pre-potential with \emph{no less than} $P$.
  %
  As a corollary, if we fix the post-potential to $Q$, then the judgement states that there exists an execution path with some pre-potential $P' \ge P$,  thus the difference between $P$ and $Q$ serves as an under-approximation of the worst-case resource consumption, which is lower-bounded by the difference between $P'$ and $Q$.
  \item A QBUA judgement $\bjudge{P}{C}{Q}$ states that every pre-potential \emph{no greater than} $P$ is possible to reach some post-potential \emph{no greater than} $Q$ by executing $C$.
  %
  Similarly, if we fix the pre-potential to $P$, then the judgement states that there exists an execution path with some post-potential $Q' \le Q$, thus the difference between $P$ and $Q$ also serves as an under-approximation of the worst-case scenario, which is lower-bounded by the difference between $P$ and $Q'$.
\end{itemize}

Different from \textsc{UNTer}, where only backward triples are meaningful for non-termination reasoning, we will prove in \cref{sec:technical} that both QFUA and QBUA logics are sound and complete for under-approximating worst-case resource usage.
%
However, in terms of identifying worst-case scenarios, we find out that QBUA triples are preferable:
as also noted by \textsc{UNTer}~\cite{OOPSLA:RVO24}, QBUA triples denote backward under-approximation but describe \emph{forward reachability}, so a QBUA triple immediately and explicitly shows what kind of inputs can result in the under-approximation.
%
Consider the program $\Ite{x>y}{\Tick{2}}{\Tick{1}}$.
%
The QFUA judgement $\fjudge{2}{\Ite{x>y}{\Tick{2}}{\Tick{1}}}{0}$ is valid, but the QBUA judgement $\bjudge{2}{\Ite{x>y}{\Tick{2}}{\Tick{1}}}{0}$ is \emph{not}, because for a pre-state satisfying $x<y$ with $2$ units of potential, executing the conditional command \emph{cannot} reach a post-state with non-negative potential.
%
A valid QBUA judgement is $\bjudge{\min(2,[x>y])}{\Ite{x>y}{\Tick{2}}{\Tick{1}}}{0}$, where predicate $[B]$ encodes a potential function that returns $+\infty$ if $B$ is true and $-\infty$ otherwise; such a judgement explicitly indicates that there exists an execution path with $2$ units of resource usage from every pre-state satisfying $x>y$.
%
Note that we allow quantitative assertions to have type $State \to \bbZ \cup \{ {-\infty},{+\infty}\}$.
%
Below are the two rules to derive the triple.
\begin{mathpar}\small
  \Rule{B:Tick}
  { }
  { \bjudge{P}{\Tick{e}}{P-e} }
  \and
  \Rule{B:IfTrue}
  { \bjudge{\min(P, [B])}{C_1}{Q} }
  { \bjudge{\min(P, [B])}{\Ite{B}{C_1}{C_2}}{Q} }  
\end{mathpar}

Recall that a triple in quantitative Hoare logic also provides an over-approximation of the high-water mark for executing a program.
%
However, this is not the case for QFUA or QBUA triples.
%
For example, we can also derive the QBUA judgement $\bjudge{\min(20,[x>y])}{\Ite{x>y}{\Tick{2}}{\Tick{1}}}{18}$, but $\min(20,[x>y])$ is obviously \emph{not} a lower bound of the worst-case high-water mark for executing the conditional command.
%
To support under-approximating high-water marks, we devise a third kind of quantitative under-approximate triples, which we write as $\djudge{P}{C}{Q}$ and call \QBUAd triples.
%
Such a triple is a refinement of $\bjudge{P}{C}{Q}$; that is, every pre-potential no greater than $P$ is possible to reach some post-potential no greater than $Q$ by executing $C$, \emph{and during the particular execution, the potential must become non-positive at some point}.
%
We prove in \cref{sec:technical} that $\djudge{P}{C}{Q}$ indicates that $P$ serves as an under-approximation of the high-water mark for executing $C$.
%
In addition, the \QBUAd logic is also complete.
%
Below are the $\Diamond$ versions of the rules shown earlier.
%
\begin{mathpar}\small
  \Rule{\Bd:Tick}
  { \min(P,P-e) \le 0 }
  { \djudge{P}{\Tick{e}}{P-e} }
  \and
  \Rule{\Bd:IfTrue}
  { \djudge{\min(P, [B])}{C_1}{Q} }
  { \djudge{\min(P, [B])}{\Ite{B}{C_1}{C_2}}{Q} }  
\end{mathpar}
%
In our formulation, the \QBUAd logic depends on the QBUA logic.
%
For example, there are two possibilities for reasoning about the sequencing command $C_1;C_2$: executing $C_1$ makes the potential non-positive, or executing $C_2$ makes the potential non-positive, as shown below.
%
\begin{mathpar}\small
  \Rule{\Bd:SeqL}
  { \djudge{P}{C_1}{R} \\ \bjudge{R}{C_2}{Q} }
  { \djudge{P}{C_1;C_2}{Q} }
  \and
  \Rule{\Bd:SeqR}
  { \bjudge{P}{C_1}{R} \\ \djudge{R}{C_2}{Q} }
  { \djudge{P}{C_1;C_2}{Q} }
\end{mathpar}
