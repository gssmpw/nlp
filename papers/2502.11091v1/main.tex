\documentclass[a4paper,USenglish,cleveref,autoref,thm-restate]{lipics-v2021}
%This is a template for producing LIPIcs articles. 
%See lipics-v2021-authors-guidelines.pdf for further information.
%for A4 paper format use option "a4paper", for US-letter use option "letterpaper"
%for british hyphenation rules use option "UKenglish", for american hyphenation rules use option "USenglish"
%for section-numbered lemmas etc., use "numberwithinsect"
%for enabling cleveref support, use "cleveref"
%for enabling autoref support, use "autoref"
%for anonymousing the authors (e.g. for double-blind review), add "anonymous"
%for enabling thm-restate support, use "thm-restate"
%for enabling a two-column layout for the author/affilation part (only applicable for > 6 authors), use "authorcolumns"
%for producing a PDF according the PDF/A standard, add "pdfa"

\pdfoutput=1 %uncomment to ensure pdflatex processing (mandatatory e.g. to submit to arXiv)
\hideLIPIcs  %uncomment to remove references to LIPIcs series (logo, DOI, ...), e.g. when preparing a pre-final version to be uploaded to arXiv or another public repository

%\graphicspath{{./graphics/}}%helpful if your graphic files are in another directory

\nolinenumbers

\bibliographystyle{plainurl}% the mandatory bibstyle

\title{A Program Logic for Under-approximating Worst-case Resource Usage} %TODO Please add

%\titlerunning{Dummy short title} %TODO optional, please use if title is longer than one line

\author{Ziyue Jin}{Peking University}{zyjin@stu.pku.edu.cn}{}{}
\author{Di Wang}{Peking University}{wangdi95@pku.edu.cn}{}{}

\authorrunning{Z. Jin and D. Wang}

%\author{Jane {Open Access}}{Dummy University Computing Laboratory, [optional: Address], Country \and My second affiliation, Country \and \url{http://www.myhomepage.edu} }{johnqpublic@dummyuni.org}{https://orcid.org/0000-0002-1825-0097}{(Optional) author-specific funding acknowledgements}%TODO mandatory, please use full name; only 1 author per \author macro; first two parameters are mandatory, other parameters can be empty. Please provide at least the name of the affiliation and the country. The full address is optional. Use additional curly braces to indicate the correct name splitting when the last name consists of multiple name parts.

%\author{Joan R. Public\footnote{Optional footnote, e.g. to mark corresponding author}}{Department of Informatics, Dummy College, [optional: Address], Country}{joanrpublic@dummycollege.org}{[orcid]}{[funding]}

%\authorrunning{J. Open Access and J.\,R. Public} %TODO mandatory. First: Use abbreviated first/middle names. Second (only in severe cases): Use first author plus 'et al.'

%\Copyright{Jane Open Access and Joan R. Public} %TODO mandatory, please use full first names. LIPIcs license is "CC-BY";  http://creativecommons.org/licenses/by/3.0/

\begin{CCSXML}
<ccs2012>
<concept>
<concept_id>10003752.10003790.10003806</concept_id>
<concept_desc>Theory of computation~Programming logic</concept_desc>
<concept_significance>500</concept_significance>
</concept>
</ccs2012>
\end{CCSXML}

\ccsdesc[500]{Theory of computation~Programming logic}
%TODO mandatory: Please choose ACM 2012 classifications from https://dl.acm.org/ccs/ccs_flat.cfm 

\keywords{Under-approximation, Incorrectness logic, Worst-case Resource Bounds} %TODO mandatory; please add comma-separated list of keywords

%\category{} %optional, e.g. invited paper

%\relatedversion{} %optional, e.g. full version hosted on arXiv, HAL, or other respository/website
%\relatedversiondetails[linktext={opt. text shown instead of the URL}, cite=DBLP:books/mk/GrayR93]{Classification (e.g. Full Version, Extended Version, Previous Version}{URL to related version} %linktext and cite are optional

%\supplement{}%optional, e.g. related research data, source code, ... hosted on a repository like zenodo, figshare, GitHub, ...
%\supplementdetails[linktext={opt. text shown instead of the URL}, cite=DBLP:books/mk/GrayR93, subcategory={Description, Subcategory}, swhid={Software Heritage Identifier}]{General Classification (e.g. Software, Dataset, Model, ...)}{URL to related version} %linktext, cite, and subcategory are optional

%\funding{(Optional) general funding statement \dots}%optional, to capture a funding statement, which applies to all authors. Please enter author specific funding statements as fifth argument of the \author macro.

%\acknowledgements{I want to thank \dots}%optional

%\nolinenumbers %uncomment to disable line numbering



%Editor-only macros:: begin (do not touch as author)%%%%%%%%%%%%%%%%%%%%%%%%%%%%%%%%%%
%\EventEditors{John Q. Open and Joan R. Access}
%\EventNoEds{2}
%\EventLongTitle{42nd Conference on Very Important Topics (CVIT 2016)}
%\EventShortTitle{CVIT 2016}
%\EventAcronym{CVIT}
%\EventYear{2016}
%\EventDate{December 24--27, 2016}
%\EventLocation{Little Whinging, United Kingdom}
%\EventLogo{}
%\SeriesVolume{42}
%\ArticleNo{23}
%%%%%%%%%%%%%%%%%%%%%%%%%%%%%%%%%%%%%%%%%%%%%%%%%%%%%%

\usepackage{amsmath}
\usepackage{mathpartir}
\usepackage{mathtools}
\usepackage{dsfont}
\usepackage{xcolor}
\usepackage{MnSymbol}
\usepackage{graphicx}
\usepackage{stmaryrd}
\usepackage{anyfontsize}
\usepackage{algorithm}
\usepackage{algpseudocode}
\usepackage{shortcuts}
\usepackage{cleveref}
\usepackage{booktabs}
\crefname{theorem}{Thm.}{Thms.}
\crefname{lemma}{Lem.}{Lemmas}
\crefname{corollary}{Cor.}{Cors.}
\crefname{figure}{Fig.}{Figs.}
\crefname{definition}{Defn.}{Defns.}
\crefname{table}{Tab.}{Tabs.}
\crefformat{section}{\S#2#1#3}
\crefmultiformat{section}{\S#2#1#3}{ and~\S#2#1#3}{, \S#2#1#3}{ and~\S#2#1#3}
\crefname{example}{Ex.}{Exs.}
\crefname{item}{item}{items}
\crefname{footnote}{footnote}{footnotes}
\crefname{observation}{Obs.}{Obs.}
\crefname{remark}{Remark}{Remarks}
\crefname{proposition}{Prop.}{Props.}
\crefname{equation}{Eqn.}{Eqns.}
\crefname{counterexample}{Counterexample}{Counterexamples}
\crefname{property}{Property}{Properties}
\crefname{algorithm}{Algorithm}{Algorithms}
\usepackage{subcaption}

\newcommand{\Blue}[1]{\textcolor{blue}{#1}}
\newcommand{\Red}[1]{\textcolor{red}{#1}}
\newcommand{\Gray}[1]{\textcolor{gray}{#1}}

\renewcommand{\dagger}{\text{\textdagger}}

\newcommand{\QBUAd}{QBUA\textsuperscript{$\Diamond$}\ }
\newcommand{\Bd}{B\textsuperscript{$\Diamond$}}

\newcommand{\pjudge}[3]{\vdash\left\{#1\right\}#2\left\{#3\right\}}

\newcommand{\Judge}[3]{\vDash\left[#1\right]#2\left[#3\right]}
\newcommand{\judge}[3]{\vdash\left[#1\right]#2\left[#3\right]}

\newcommand{\bjudge}[3]{\vdash_{\mathsf{B}}\left[#1\right]#2\left[#3\right]}
\newcommand{\fjudge}[3]{\vdash_{\mathsf{F}}\left[#1\right]#2\left[#3\right]}
\newcommand{\ajudge}[3]{\vdash_{\dagger}\left[#1\right]#2\left[#3\right]}
\newcommand{\djudge}[3]{\vdash^{\Diamond}_{\mathsf{B}}\left[#1\right]#2\left[#3\right]}
\newcommand{\bJudge}[3]{\vDash_{\mathsf{B}}\left[#1\right]#2\left[#3\right]}
\newcommand{\fJudge}[3]{\vDash_{\mathsf{F}}\left[#1\right]#2\left[#3\right]}
\newcommand{\dJudge}[3]{\vDash^{\Diamond}_{\mathsf{B}}\left[#1\right]#2\left[#3\right]}

\newcommand{\bbjudge}[5]{\vdash_{\mathsf{B}}\left[\left.#1\right|#2\right]#3\left[\left.#4\right|#5\right]}
\newcommand{\ffjudge}[5]{\vdash_{\mathsf{F}}\left[\left.#1\right|#2\right]#3\left[\left.#4\right|#5\right]}
\newcommand{\aajudge}[5]{\vdash_{\dagger}\left[\left.#1\right|#2\right]#3\left[\left.#4\right|#5\right]}
\newcommand{\ddjudge}[5]{\vdash^{\Diamond}_{\mathsf{B}}\left[\left.#1\right|#2\right]#3\left[\left.#4\right|#5\right]}
\newcommand{\AAjudge}[5]{\vdash_{\text{\textdaggerdbl}}\left[\left.#1\right|#2\right]#3\left[\left.#4\right|#5\right]}

\newcommand{\bigstep}[6]{\left\langle#1,#2,#3\right\rangle\Downarrow^{#4}\left\langle#5,#6\right\rangle}
\newcommand{\bigstepp}[5]{\left\langle#1,#2,#3\right\rangle\Downarrow\left\langle#4,#5\right\rangle}
\newcommand{\bigstepl}[5]{\left\langle#1,#2,#3\right\rangle\Downarrow^{\le 0}\left\langle#4,#5\right\rangle}

\newcommand{\eval}[2]{\left\llbracket#1\right\rrbracket#2}

\newcommand{\post}[2]{\mathrm{post}\left\llbracket#1\right\rrbracket\left(#2\right)}
\newcommand{\pre}[2]{\mathrm{pre}\left\llbracket#1\right\rrbracket\left(#2\right)}
\newcommand{\prel}[2]{\mathrm{pre}^{\le 0}\left\llbracket#1\right\rrbracket\left(#2\right)}

\newcommand{\Skip}{\textsf{skip}}
\newcommand{\Assign}[2]{#1\coloneqq#2}
\newcommand{\Assume}[1]{\textsf{assume}~#1}
\newcommand{\Tick}[1]{\textsf{tick}~#1}
\newcommand{\Seq}[2]{#1;#2}
\newcommand{\Choice}[2]{#1+#2}
\renewcommand{\Loop}[1]{#1^*}
\newcommand{\Ite}[3]{\textsf{if}~#1~\textsf{then}~#2~\textsf{else}~#3}
\newcommand{\IfB}[1]{\textsf{if}~#1~\textsf{then}}
\renewcommand{\Else}{\textsf{else}}
\renewcommand{\While}[2]{\textsf{while}~#1~\textsf{do}~#2}
\newcommand{\WhileB}[1]{\textsf{while}~#1~\textsf{do}}
\newcommand{\Local}[2]{\textsf{local}~#1~\textsf{in}~#2}
\newcommand{\Localx}[1]{\textsf{local}~#1~\textsf{in}}

\newcommand{\SR}[2]{\left[\left.#1\right|#2\right]} %Spec and Resource

\newcommand{\Fv}{\mathrm{fv}}
\newcommand{\Mod}{\mathrm{mod}}

\newcommand{\Rinf}{\mathbb{R}^{\pm\infty}}
\newcommand{\Sup}{\mathop{\scalebox{-1}[1]{\textsf{S}}}}
\newcommand{\Inf}{\mathop{\scalebox{-1}[1]{\textsf{J}}}}

\newcommand{\true}{\mathsf{true}}
\newcommand{\false}{\mathsf{false}}

\newcommand{\Null}{\mathsf{null}}
\newcommand{\Val}{\mathsf{val}}
\newcommand{\Next}{\mathsf{next}}
\newcommand{\Lson}{\mathsf{lchild}}
\newcommand{\Rson}{\mathsf{rchild}}

\newcommand{\Time}[1]{#1_{\mathrm{time}}}

\algdef{SE}[SUBALG]{Indent}{EndIndent}{}{\algorithmicend\ }%
\algtext*{Indent}
\algtext*{EndIndent}

\sloppy

\begin{document}

\maketitle

%TODO mandatory: add short abstract of the document
\begin{abstract}
% State the problem
% Say why it's an interesting problem
% Say what your solution achieves
% Say what follows from your solution
%
Understanding and predicting the worst-case resource usage is crucial for software quality; however, existing methods either over-approximate with potentially loose bounds or under-approximate without asymptotic guarantees.
%
This paper presents a program logic to under-approximate worst-case resource usage, adapting incorrectness logic (IL) to reason quantitatively about resource consumption.
%
We propose \underline{q}uantitative \underline{f}orward and \underline{b}ackward \underline{u}nder-\underline{a}pproximate (QFUA and QBUA) triples, which generalize IL to identify execution paths leading to high resource usage.
%
We also introduce a variant of QBUA that supports reasoning about high-water marks.
%
Our logic is proven sound and complete with respect to a simple IMP-like language, and we demonstrate its utility through case studies involving arrays, pointers, and procedure calls.
%
%This work offers a formal, compositional framework for deriving lower bounds on resource usage, allowing for the identification of performance bottlenecks and security vulnerabilities in programs.
\end{abstract}

\section{Introduction}
\label{sec:intro}

\begin{figure*}[tb]
    \centering
    \includegraphics[width=0.848\linewidth]{figs/circuitnn.pdf} 
    \caption{Illustration of differentiable CircuitNN. CircuitNN is designed based on differentiable NAND gates. After DAS is guided by PI and PO pairs of the truth table, CircuitNN can get the precise circuit architecture logic equivalent to the truth table.}
    \label{fig:circuitnn}
\end{figure*}

% 1. Describe the importance of logic synthesis
% 2. Existing Problems
% (a) Neural Architecture Search: Unstable, Predefined Setting, etc.
% (b) Circuit Generation: Probabilistic Model, Logic Equivalence

With the rapid advancement of technology, the scale of integrated circuits (ICs) has expanded exponentially. 
This expansion has introduced significant challenges in chip manufacturing, particularly concerning power and area metrics.
A primary objective in IC design is achieving the same circuit function with fewer transistors, thereby reducing power usage and area occupancy.

Logic synthesis~\cite{hachtel2005logicsynth}, a critical step in electronic design automation (EDA), transforms behavioral-level circuit designs into optimized gate-level circuits, ultimately yielding the final IC layout. 
The primary goal of logic synthesis is to identify the physical implementation with the fewest gates for a given circuit function. 
This task constitutes a challenging NP-hard combinatorial optimization problem. 
Current logic synthesis tools~\cite{brayton2010abc, wolf2013yosys} rely on human-designed heuristics, often leading to sub-optimal outcomes.

Differentiable architecture search (DAS) techniques~\cite{liu2018darts, chu2020darts} offer novel perspectives on addressing challenges in this problem.
Circuit functions can be represented through truth tables, which map binary inputs to their corresponding outputs. 
Truth tables provide a precise representation of input-output relationships, ensuring the design of functionally equivalent circuits.
Inspired by this, researchers~\cite{deepmind2024ai4sys, wang2024tnet} have begun exploring the application of DAS to synthesize circuits directly from truth tables.
Specifically, \citet{deepmind2024ai4sys} proposed CircuitNN, a framework that learns differentiable connection structures with logic gates, enabling the automatic generation of logic circuits from truth tables.
This approach significantly reduces the complexity of traditional circuit generation. 
Building on this, \citet{wang2024tnet} introduced T-Net, a triangle-shaped variant of CircuitNN, incorporating regularization techniques to enhance the efficiency of DAS.

Despite these advancements, several challenges remain. 
The computational complexity of DAS grows quadratically with the number of gates, posing scalability issues.
Although triangle-shaped architecture~\cite{wang2024tnet} partially mitigates this problem, redundancy persists. 
%Additionally, DAS is susceptible to converging to local optima, limiting the ability to search architectures that satisfy the given truth tables~\cite{liu2018darts}. 
%Furthermore, hyperparameters (network depth and layer width) require extensive searches, introducing complexity and prolonging the synthesis process. 
Additionally, DAS is susceptible to converging to local optima~\cite{liu2018darts} and hyperparameters (network depth and layer width) require extensive searches. 
The challenges arise from the vast search space in DAS. 
% Even with predefined settings for CircuitNN, finding a configuration that meets the truth table requires extensive trial and error during the DAS process. 
Intuitively, limiting the search space through predefined parameters (network depth, gates per layer, and connection probabilities) can significantly reduce the complexity.

Recent advances~\cite{openai2023gpt4, abramson2024alphafold3, esser2024sd3, li2024mar} in conditional generative models have demonstrated remarkable performance across language, vision, and graph generation tasks. 
Motivated by these developments, we propose a novel approach to circuit generation that generates preliminary circuit structures to guide DAS in generating refined circuits matching specified truth tables. 
Firstly, we introduce CircuitVQ, a tokenizer with a discrete codebook for circuit tokenization. 
Built upon our Circuit AutoEncoder framework~\cite{hou2022graphmae,li2023maskgae,wu2025mgvga}, CircuitVQ is trained through a circuit reconstruction task. 
Specifically, the CircuitVQ encoder encodes input circuits into discrete tokens using a learnable codebook, while the decoder reconstructs the circuit adjacency matrix based on these tokens.
Subsequently, the CircuitVQ encoder serves as a circuit tokenizer for CircuitAR pretraining, which employs a masked autoregressive modeling paradigm~\cite{chang2022maskgit, li2023mage}. 
In this process, the discrete codes function as supervision signals. 
After training, CircuitAR can generate discrete tokens progressively, which can be decoded into initial circuit structures by the decoder of the CircuitVQ. 
These prior insights can guide DAS in producing refined circuits that match the target truth tables precisely.

Our key contributions can be summarized as follows:
\begin{itemize}
\item We introduce CircuitVQ, a circuit tokenizer that facilitates graph autoregressive modeling for circuit generation, based on our Circuit AutoEncoder framework;
\item Develop CircuitAR, a model trained using masked autoregressive modeling, which generates initial circuit structures conditioned on given truth tables;
\item Propose a refinement framework that integrates differentiable architecture search to produce functionally equivalent circuits guided by target truth tables;
\item Comprehensive experiments demonstrating the scalability and capability emergence of our CircuitAR and the superior performance of the proposed circuit generation approach.
\end{itemize}

% Motivation
% (a) Diffusion (Vision, Graph), Autoregressive (Language, Vision)
% (b) Circuit Generation for Predefined Setting
% (c) Neural Architecture Search for Strict Logic Equivalence

% Contribution
% (a) Circuit Tokenizer (new transformer arch, training strategy)
% (b) CircuitAR (train and gen strategies, post-ar strategy)
% (c) Extensive Evaluation including BitD (Bit Distance) for Scalability

\begin{figure*}[t]
\begin{center}
\includegraphics[width=.85\linewidth]{fig_overview_v3.pdf}
\end{center}
\caption{
FastAtlas Overview: In each frame, we compute charts spanning fully or partially visible triangles (a), determine texture space bounding boxes for the visible portions of the view-space projections of each chart, and tightly pack these boxes into atlases (b, here $2K \times 2K$). We simultaneously bijectively parameterize and shade the charts into their atlas boxes, obtaining high quality texture space shading (c), and use this shading to render the shaded frames (d).}
\label{fig:overview}
\label{fig:alg_overview}
\end{figure*}

\section{Overview}
\label{sec:overview}
Our work has two core contributions: a real-time, GPU-based algorithm for tight packing of general parameterized charts into compact atlases; and a real-time TSS method that
utilizes this packing.  

\paragraph*{FastAtlas Packing.}
FastAtlas runs entirely on the GPU as a series of compute shaders. It takes the bounding boxes of parameterized charts as input, and packs them into an atlas (Fig~\ref{fig:overview}b, Sec.~\ref{sec:pack}). As such, the only input it requires are the dimensions of the bounding boxes.
Its outputs are deterministic; identical input charts are packed into identical atlases. This is critical for TSS and similar applications, as it ensures that consecutive frames taken from the same camera view have the same shading. Even minute shading differences across such frames can cause sampling jitter, leading to undesirable flicker \cite{baker2012rock}. 
While prior methods such as \cite{mueller2018shading,hladky2019tessellated,hladky2021snakebinning,Neff2022MSA} cap the dimensions of the charts that can be packed as-is for a given atlas size, and scale down all charts that exceed these dimensions, we scale all charts by the same factor, and do so only when strictly necessary to achieve packing success (Figs~\ref{fig:atlas},~\ref{fig:sas_issues}). 

\paragraph*{TSS using FastAtlas.}
Our end-to-end TSS atlas generation method combines the packing method above with a novel approach for computing seamless per-frame charts. 
We define our charts as the connected components of the visible surfaces in each frame (Fig.~\ref{fig:overview}a), and efficiently compute them using a parallel union-find algorithm (Sec.~\ref{sec:visible}). Since the boundaries of these charts coincide with the contours of the rendered surface, they are {\em invisible} to the viewer. This approach 
eliminates the artifacts caused by shading discontinuities along visible seams (Fig.~\ref{fig:seams}). 

\begin{parWithWrapFigure}
\begin{wrapfigure}{l}{.27\columnwidth}%
\includegraphics[width=\linewidth]{fig_inset_view_plane.pdf}%
\end{wrapfigure}
We bijectively parametrize the {\em visible portions} of our charts by projecting them to view space (inset). This maps a constant number of texels to each pixel in the final rendered output, evenly distributing residual undersampling error across all image pixels. While conceptually straightforward, efficiently parameterizing charts containing partially visible triangles using viewspace projection is non-trivial, as the visible portions may no longer be triangular (e.g. green triangle in the inset); applying naive projection to triangles with vertices behind the camera may produce ill-posed results. Clipping triangles before projection is both computationally expensive and significantly complicates downstream operations. We avoid explicit clipping by observing that all that is required for atlas packing is the dimensions of, potentially conservative, bounding boxes of these projected visible portions. We compute such bounding boxes without explicit chart clipping by adapting a conservative screen coverage estimator \shortcite{Blinn:CalculatingScreenCoverage} (Sec.~\ref{sec:box}). We then pack the computed boxes using FastAtlas. 
\end{parWithWrapFigure}

Finally, we shade the visible portion of each chart into its corresponding atlas bounding box (Fig~\ref{fig:overview}c). 
The resulting texture is then used during rasterization as a standard texture map (Fig. ~\ref{fig:overview}d). 
Our framework is compatible with all existing approaches for texture space shading, including forward shading, raytraced illumination, or deferred shading in texture space \cite{baker:2016}. In the examples shown, we use the standard forward shading based rendering pipeline included in the G3D Innovation Engine \cite{G3D17}, a commercial grade renderer.

\section{Proposed Method}
\textbf{Problem Statement: } Let $\mathcal{X}$ and $\mathcal{Y}$ denote the input and output spaces, respectively, and $D = \{(x_i, y_i)\}_{i=1}^n$ the dataset, where $x_i \in \mathcal{X}$ and $y_i \in \mathcal{Y}$ are the $i^{th}$ question-answer pair. For each $x_i$, the goal is to generate a response $\hat{y}_i$ that maximizes the overall accuracy. The goal is to achieve this using a pool  of $N$ pretrained foundational LLMs without additional training or fine-tuning.


\begin{comment}
    

Let $\mathcal{X}$ and $\mathcal{Y}$ denote input and output spaces respectively and  $\mathcal{D}= \{(x_i,y_i)\}_{i=1}^n$ denote Dataset, where $x_i \in \mathcal{X}$ and $y_i \in \mathcal{Y}$ is the $i^{th}$ question-answer(QA) pair. For each $x_i$ we want to generate a response $\hat{y}_i$  such that overall accuracy denoted by $\frac{1}{n}\sum_{i=1}^n\mathbf{I}(y_i=\hat{y}_i)$ is maximized.  We want to maximize this using ensemble of $N$ LLMs without any additional training or fine-tuning.
\end{comment}
\renewcommand{\algorithmicrequire}{\textbf{Input:}}
\renewcommand{\algorithmicensure}{\textbf{Output:}}

\begin{algorithm}[t]
    \caption{Components of UAF - SELECTOR and FUSER}
    \label{alg:selector}
    
    \begin{algorithmic}
    \REQUIRE $D_{val}$, Pool of LLMs $\mathcal{M}= \{M^j\}_{j=1}^{N}$, Uncertainty function $U_f(.,.,.)$, Ensemble size $K$, Test data point $x_{test}$
    \ENSURE Test data response $\hat{y}_{test}$
    \newline

    \STATE \textbf{procedure }SELECTOR ($D_{val}, \mathcal{M}, U_f, K$)
    \FOR{$M^j \in \mathcal{M}$}
    %\STATE $L_j, U_j = \emptyset,\emptyset$
    \FOR{$(x_i, y_i) \in D_{val}$}
    \STATE $\hat{y}_i^j = M^j(x_i)$   \quad;\quad  $s_i^j = \mathbf{1}(\hat{y}_i^j == y_i)$
    \STATE $u_i^j = U_f(M^j, x_i, \hat{y}_i^j)$
    %\STATE $L_j \leftarrow L_j \cup IsCorrect_i^j$   \quad;\quad    $U_j \leftarrow U_j \cup u_{val_i}^j$
    \ENDFOR

    \STATE $Acc_j = \frac{1}{|D_{val}|}\sum_i s_i^j$  ;\quad $SAH_j = ROC\_AUC\_score(\{s_i^j,u_i^j\}_i)$
    \STATE $Cscore_j = Acc_j \times SAH_j$
    \ENDFOR
    \STATE \textbf{return:} $\mathcal{M}^{sel} =  TopK (\{Cscore_j\}_{j=1}^N)$,  \hfill\COMMENT{//Top $K$ LLMs}
    \STATE \quad  \quad      $Acc^{sel} = \{Acc_j| j \in \mathcal{M}^{sel}\}$ \hfill\COMMENT{//Accuracy of selected $K$ LLMs}
    %\STATE $j_1, j_2, \dots, j_K = \arg\max_{j \in \{1, 2, \dots, N\}} Cscore_j$ 
    %\ENSURE $j_1, j_2, \dots, j_K$    \hfill\COMMENT{//Indices of Top K llms}
    \newline
\STATE \textbf{procedure } FUSER ($x_{test}, \mathcal{M}^{sel}, Acc^{sel}, U_f, K$)
\FOR{$M^k \in \mathcal{M}^{sel}$}
\STATE $\hat{y}_{test}^k = M^k(x_{test})$
\STATE $u_{test}^k = U_f(M^k, x_{test}, \hat{y}_{test}^k)$
\ENDFOR
\STATE $\hat{y}_{test} = \hat{y}_{test}^{k^*} \text{, where } k^* = argmax_{k \in \{1,\dots K\}} Acc_k \times (1-u_{test}^k)$
\STATE \textbf{return:} $\hat{y}_{test}$

\end{algorithmic}
\end{algorithm}

\subsection{Uncertainty Aware Fusion (UAF)}\label{sec:uaf}
Figure \ref{fig:fuser} provides an overview of our UAF framework. At a high level, UAF consists of two modules: SELECTOR and FUSER. Given a specific task, the SELECTOR selects the top $K$ LLMs from a pool of $N$ LLMs based on performance metric. FUSER then combines the outputs of these $K$ LLMs to produce the final response.

\subsubsection{SELECTOR}
Given a pool of $N$ LLMs denoted by $\mathcal{M}$, SELECTOR selects $K$ LLMs (where $K<N$) to optimize computational efficiency and enhance overall factual accuracy by pruning underperforming LLMs. Selection is based on two criteria: (1) task-specific accuracy and (2) self-assessment of hallucinations based on an given uncertainty measure. Given a specified uncertainty measure $U_f(\cdot)$, and  a validation set $D_{val}$, we prompt each LLM with input $x_i$,   obtaining response  $\hat{y}_i^j$  and corresponding uncertainty score $u_i^j$ from the $j^{th}$ LLM $M^j$.  We compute the accuracy $Acc_j$ of $M^j$ as the fraction of correct responses. We also measure the LLMs ability for  self-assessment of hallucinations $SAH_j$ as the area under the ROC curve for the binary classification of truthful vs. hallucinatory responses using uncertainty scores. We then compute a combined score $Cscore_j = Acc_j \times SAH_j$ for each LLM. The top K models with the highest combined scores are selected greedily, where K is a   hyperparameter tuned for specific tasks.

\begin{comment}
To achieve this we sample $10\%$ of examples from $D$ and denote it as  $D_{val}$. We use the rest denoted by $D_{test}$ for evaluation. For each $(xval_i,yval_i)\in D_{val}$ we prompt each of the $N$ LLMs with $xval_i$. Let $\hat{yval}_i^j$ denote the response and $u_i^j$ its corresponding uncertainty score computed with a particular uncertainty method for $j^{th}$ LLM. Accuracy of $j^{th}$ LLM, denoted by $Acc_j$ is defined as percentage of correct responses. We also measure area under the receiver operator characteristic curve  $Unc\_auroc_j$ of $j^{th}$ LLM, by measuring the performance of classifying its own correct(truthful) from incorrect(hallucinatory) responses by varying the thresholds on the set of uncertainty scores $\{u_i^j\}_{i=1}^{nval}$. For each $j^{th}$ LLM we compute a combined score $Cscore_j = Acc_j*Unc\_auroc_j$. We use greedy method to select $K$ LLMs based on top $K$ highest $Cscore$. Here $K$ is the hyperparameter which we tune using $D_{val}$. Algorithm \ref{alg:selector} presents the pseudo-code for this method.
\end{comment}

\subsubsection{FUSER}
Given the selected ensemble of $K$ models $\mathcal{M}^{sel}$, 
%with respective accuracies $\{Acc_1, \dots, Acc_K \}$,
for each unseen example $x_{test}$, we generate outputs from the $K$ LLMs denoted by $\{\hat{y}_{test}^1,\dots,\hat{y}_{test}^K\}$ along with the  corresponding instance-specific uncertainty scores.  denoted by $\{u_{test}^1,\dots,u_{test}^K\}$. 
While there can be several fusion strategies, since we are dealing with natural language responses, the simplest one is to example-specific selection from the candidate outputs, i.e., 
\[
\hat{y}_{test} = \hat{y}_{test}^{k^*}, \quad \text{where} \quad {k^*} = \arg\max_{k \in \{1, \dots, K\}} f^k.
\]
Selection criterion $f^k$ could be based on validation set accuracy alone, inverse uncertainty or some combination of both such as $\text{Acc}_k \cdot (1 - u_{test}^k)$  or $\frac{\text{Acc}_k}{u_{test}^k}$.
The first strategy essentially reduces the ensemble to a single most accurate model, while the second one elevates the most confident one. However, both of these approaches are sub-optimal compared to combined criteria, specifically 
\[
f^k = \text{Acc}_k \cdot (1 - u_{test}^k),
\]
which yields the best performance. Experiments with  other combined selection criteria shows similar behavior to the aforementioned one and hence,  we omit the results for brevity.


%There are multiple ways to choose the final answer from these candidate responses - $\hat{y}_{test} = \hat{y}_{test}^j where j = argmax_{k \in \{1,\dots K\}}f^k$. We can define $f^k$ as $Acc_k$ , $1-u_{test}^k$ ,$Acc_k*(1-u_{test}^k)$ or $Acc_k/u_{test}^k$. First strategy essentially reduces the ensemble to a single model having highest accuracy, while second one picks the final answer as the one with the least uncertainty. Both of these are suboptimal compared to  $f_k = Acc_k*(1-u_{test}^k)$ with which we experiment here in this work. Alternative strategy $f_k = Acc_k/u_{test}^k$ shows similar behaviour as above and we omit it's analysis for brevity.

%Although there are multiple ways to aggregate these candidate responses we propose a simple aggregation technique where we choose the final answer as the one with the least uncertainty ie. $\hat{y}_{test} = \hat{y}_{test}^j where j = argmin_{k \in \{1,\dots K\}}u_{test}^k$. Algorithm \ref{alg:selector} presents the pseudo-code of UAF components.


\setlength\tabcolsep{1pt}
%
\begin{table*}[!htb]
    \centering
    \tiny
    \definecolor{lightgray}{gray}{0.9} % 定义浅灰色
    \resizebox{0.95\textwidth}{!}{
    \begin{tabular}[l]{|>{\raggedright\arraybackslash}m{1.5cm}|>{\raggedright\arraybackslash}m{4cm}|>{\raggedright\arraybackslash}m{10cm}|}
    \hline
    \textbf{Stacking Pattern} & \textbf{Input} & \textbf{Output} \\ 
    \hline
    Correct 
    & The molecule is a cycloalkane composed of a three-carbon ring. It functions as an inhalation anesthetic. It is a cycloalkane and a member of the cyclopropane class. What is its predicted SMILES?
    \newline
    \newline
    \newline
    \textcolor{red}{Gold Answer: C1CC1}
    & According to the provided description, the molecule is cyclopropane, which is a cycloalkane formed by three carbon atoms in a ring, well known for its use as an inhalation anesthetic. To confirm the SMILES notation of cyclopropane, I used the ChemDFM tool.
    \newline
    \textbf{Action:} ChemDFM  
    \newline
    \textbf{Input:} What is the SMILES notation for cyclopropane (a three-carbon cycloalkane used as an inhalation anesthetic)?  
    \newline
    \textbf{Output:} CC1(C)CC1  \textcolor{red}{\ding{55}}
    \newline
    \textbf{Observation:} The returned "CC1(C)CC1" does not correspond to cyclopropane. Cyclopropane should be a simple three-carbon ring, typically represented as "C1CC1". Next, I will call the Name2SMILES tool to further correct the result.
    \newline
    \textbf{Action:} Name2SMILES  
    \newline
    \textbf{Input:} Cyclopropane  
    \newline
    \textbf{Output:} C1CC1  
    \newline
    \textbf{Observation:} Based on the known structure of cyclopropane, the result from this tool is correct. Therefore, I will provide the correct SMILES based on the understanding of the chemical structure.  
    \newline
    \textbf{Final Answer: C1CC1 \textcolor{red}{\ding{52}}}\\
    \hline
    \end{tabular}
    }
    \caption{Case analysis of \textbf{"Correct"} behavior patterns of agents during tools stacking.}
    \label{tab:case study}
\end{table*}


%该分子由三个氮原子组成的环状结构,它具有芳香性的特性和两个亚甲基。它是一种环烯烃,也是两种异构体之一。请预测它的 SMILES。
%

\section{Related Work} \label{sec:related}

% \textbf{Adversarial Attack}
\textbf{Attacks on SLAM.} 
%With the rise of machine learning, 
The robustness of computer vision systems is being actively investigated. With the emergence of adversarial images in the digital domain by adding optimized noise directly to images~\cite{szegedy2013intriguing,carlini2017towards}, researchers find that such attacks also exist physically in the real world \cite{eykholt2018robust,song2018physical,zhao2019seeing}. To fill the gap between attacks in the digital and physical worlds, recent studies have demonstrated that attacks on real-world computer vision systems are practical \cite{eykholt2018robust,li2019adversarial,man2020ghostimage,sharif2016accessorize,zhao2019seeing,zhou2018invisible}. However, attacks on traditional computer vision methods such as SLAM are relatively less explored. \cite{yoshida2022adversarial} proposes an attack against the scan matching algorithm in LiDAR-based SLAM, while most SLAMs in AR/VR devices rely on different sensors like RGB/depth cameras and IMUs. \cite{ikram2022perceptual} and \cite{chen2024adversary} mislead visual SLAM by poisoning the images with special patterns, and \cite{wang2021can} causes the camera to fail using infrared light. In our work, we demonstrate attacks on Visual-Inertial SLAM (VI-SLAM) by perturbing the IMU readings, rather than cameras, and showing its impact on XR user experience. 

\textbf{Acoustic Injection Attacks.} Among various physical attacks, acoustic injection attacks are attractive due to their low cost. Son~\etal~\cite{son2015rocking} were the first to introduce acoustic attacks on MEMS gyroscopes, demonstrating how these attacks could lead to sensor denial-of-service and result in drone crashes. WALNUT~\cite{trippel2017walnut} expanded on this by developing output biasing and control attacks that enable precise manipulation of MEMS accelerometer outputs using modulated sound waves. Wang et al.~\cite{wang2017sonic} demonstrated a sonic gun, showcasing the vulnerability of various smart devices (\eg drones and self-balancing vehicles) to acoustic attacks. Tu et al. \cite{tu2018injected} designed side-swing and switching attacks to alter the outputs of MEMS gyroscopes and accelerometers. Furthermore, Ji et al. \cite{ji2021poltergeist} fool the object detectors by applying acoustic attack to the image stabilizers commonly used in modern cameras. However, none of the existing works study the relationship between the acoustic injections and SLAM outputs on recent XR devices. 

% \zijian{Do we need one session about security in AR/VR?}
% \yicheng{TODO}
%\jiasi{cite the AIVR paper (UMass Amherst?) paper is we have not already. They add IMU perturbation but w/o SLAM, iirc} \yicheng{Cited}

\textbf{XR Security and Privacy.} 
%Security and privacy concerns in XR systems have gained significant attention. 
For single-user XR systems, researchers have demonstrated various side-channel attacks to extract sensitive information (\eg keystrokes) through video feeds~\cite{ling2019know}, head movements~\cite{nair2023unique, slocum2023going}, architectural hints~\cite{zhang2023its,shang2020arspy}, power usage~\cite{li2024dangers}, and EM side-channel leakages~\cite{al2021vr}. In multi-user XR systems, Su et al.~\cite{su2024remote} use avatar motion data to infer keystrokes in shared VR environments. Slocum et al.~\cite{slocum2024doesn} reveal vulnerabilities in the shared state frameworks of multi-user AR. Similarly, Lebeck et al.~\cite{lebeck2017securing} highlight risks like deceptive virtual objects and emphasize access control for managing shared physical and virtual spaces. Ruth et al.~\cite{ruth2019secure} further propose a secure multi-user AR framework focusing on content sharing and permissions.
Chandio et al.~\cite{chandio2024stealthy} %introduced a multi-modal spatiotemporal attack that 
simultaneously manipulated visual and inertial sensors to disrupt XR pose estimation. However, their study evaluated the attack using offline datasets and assumed the attacker's capability to manipulate IMU data streams through acoustic means, without real experiments. Ours is the first to demonstrate acoustic injection attacks on recent XR devices, like the Hololens 2, in the real world.
 


\section*{Conclusion}
This paper aims to enhance our understanding of the computational complexity of computing various Shapley value variants. We found that for various ML models --- including decision trees, regression tree ensembles, weighted automata, and linear regression --- both local and global interventional and baseline SHAP can be computed in polynomial time under HMM modeled distributions. This extends popular algorithms, such as TreeSHAP, beyond their empirical distributional scope. We also establish strict complexity gaps between the various SHAP variants (baseline, interventional, and conditional) and prove the intractability of computing SHAP for tree ensembles and neural networks in simplified scenarios. Overall, we present SHAP as a versatile framework whose complexity depends on four key factors: \begin{inparaenum}[(i)] \item model type, \item SHAP variant, \item distribution modeling approach, \item and local vs. global explanations\end{inparaenum}. We believe this perspective provides deeper insight into the computational complexity of SHAP, paving the way for future work.




%We believe that our framework provides a more intricate understanding of SHAP computation complexity across different models, distributions, and variants, paving the way for further research.

Our work opens promising directions for future research. First, expanding our computational analysis to other SHAP-related metrics, such as asymmetric SHAP~\citep{frye20} and SAGE~\citep{covert2020understanding}, would be valuable. Additionally, we aim to explore more expressive distribution classes and relaxed assumptions beyond those in Section \ref{sec:tractable} while maintaining tractable SHAP computation. Finally, when exact computation is intractable (Section \ref{sec:intractable}), investigating the approximability of SHAP metrics through approximation and parameterized complexity theory~\citep{downey2012parameterized} is an important direction.

%Our work opens several promising avenues for future research on the computational properties of explainable AI methods, with a particular focus on SHAP. First, it would be interesting to broaden the computational analysis conducted in this work to include other popular SHAP-related metrics in the literature, such as asymmetric SHAP \cite{frye20} and SAGE \cite{covert2020understanding}. Also, in the future, we aim to explore more expressive distribution classes and relaxed distributional assumptions—extending beyond those examined in Section \ref{sec:tractable} —that still yield tractable SHAP computation. Finally, when exact computation proves intractable (Section \ref{sec:intractable}), it is worthwhile to theoretically investigate the question of the approximability of computing the SHAP metrics across various configurations, through the lens of approximation and parametrized complexity theory \cite{arora2009computational}.

%This paper aims to deepen our understanding of the computational complexity involved in obtaining different Shapley value variants. We found that for a variety of ML models, including decision trees, tree ensembles for regression, weighted automata, and linear regression models — computing both local and global interventional and baseline SHAP can be done in polynomial time when distributions are modeled by HMMs. This extends the distributional scope of popular algorithms like TreeSHAP, which is limited to empirical distributions. Additionally, we demonstrate a strict complexity gap between SHAP variants, showing that interventional and baseline SHAP can be strictly easier to compute than conditional SHAP. Despite these positive results, we uncovered intractability for various SHAP variants in neural networks and tree ensembles. Finally, we provided generalized complexity relations across SHAP variants. We believe that our framework offers a deeper understanding of the complexity involved in computing SHAP across various variants, models, distributions, as well as in both local and global computations, laying the groundwork for future research.

\bibliography{db}

\appendix
\newpage

%\section{Formalization}
%
%\begin{align*}
%  C\Coloneqq&\Skip\mid\Assign{x}{e}\mid\Tick{e}\mid\Seq{C}{C}\mid\Ite{B}{C}{C}\mid\While{B}{C}\mid\Local{x}{C}
%\end{align*}
%
%% $\Judge{P}{C}{Q}$ holds iff for all $\tau$ such that $\Time{\tau}\le Q(\tau)$, there exists a $\sigma$ such that $\Time{\sigma}\le P(\sigma)$ and $(\sigma,\tau)\in\llbracket C\rrbracket$. If $\Judge{P}{C}{Q}$ holds, then the worst-case running time of $C$ is at least $\sup_{\tau}\{Q(\tau)\}-\sup_{\sigma}\{P(\sigma)\}$.
%
%\begin{mathpar}
%  P\curlywedge Q=\lambda\sigma.\min\{P(\sigma),Q(\sigma)\}
%  \hva\and
%  P\curlyvee Q=\lambda\sigma.\max\{P(\sigma),Q(\sigma)\}
%  \hva\and
%  \Sup x.P=\lambda\sigma.\sup_v\{P(\sigma[x\mapsto v])\}
%  \hva\and
%  \Inf x.P=\lambda\sigma.\inf_v\{P(\sigma[x\mapsto v])\}
%  \hva\and
%  [B]=\lambda\sigma.\begin{cases}+\infty&\text{if }B(\sigma)\text{ is true}\\-\infty&\text{otherwise}\end{cases}
%  \hva\and
%  P\preceq Q\quad\text{iff}\quad\forall\sigma\in\Sigma.P(\sigma)\leq Q(\sigma)
%\end{mathpar}

%\section{QFUA Logic}



%\section{QBUA Logic}



%\section{\QBUAd Triple}



\section{Proofs}

\begin{table}[t!]
\centering
\caption{Correspondence between Boolean and Quantitative Operators}
\label{fig:correspondence}
\begin{tabular}{lccc}
\toprule
&Boolean&QFUA&QBUA\\
\midrule
Conjunction&$P\land Q$&$P\curlyvee Q$&$P\curlywedge Q$\\
Disjunction&$P\lor Q$&$P\curlywedge Q$&$P\curlyvee Q$\\
Existential&$\exists x.P$&$\Inf x.P$&$\Sup x.P$\\
Implication&$P\Rightarrow Q$&$P\preceq Q$&$Q\preceq P$\\
Boolean Predicate&$B$&$[\neg B]$&$[B]$\\
\bottomrule
\end{tabular}
\end{table}

\begin{figure}
\begin{mathpar}
  \inferrule[(BS:Skip)]{}{\bigstep{\Skip}{\sigma}{p}{p}{\sigma}{p}}
  \hva\and
  \inferrule[(BS:Assign)]{}{\bigstep{\Assign{x}{e}}{\sigma}{p}{p}{\sigma[x\mapsto\eval{e}{\sigma}]}{p}}
  \hva\and
  \inferrule[(BS:Assume)]{\eval{B}{\sigma}=\true}{\bigstep{\Assume{B}}{\sigma}{p}{p}{\sigma}{p}}
  \hva\and
  \inferrule[(BS:Tick)]{}{\bigstep{\Tick{e}}{\sigma}{p}{\min\{p,p-\eval{e}{\sigma}\}}{\sigma}{p-\eval{e}{\sigma}}}
  \hva\and
  \inferrule[(BS:Seq)]{\bigstep{C_1}{\sigma}{p}{l_1}{\rho}{r} \\ \bigstep{C_2}{\rho}{r}{l_2}{\tau}{q}}{\bigstep{\Seq{C_1}{C_2}}{\sigma}{p}{\min\{l_1,l_2\}}{\tau}{q}}
  \hva\and
  \inferrule[(BS:ChoiceL)]{\bigstep{C_1}{\sigma}{p}{l}{\tau}{q}}{\bigstep{\Choice{C_1}{C_2}}{\sigma}{p}{l}{\tau}{q}}
  \hva\and
  \inferrule[(BS:ChoiceR)]{\bigstep{C_2}{\sigma}{p}{l}{\tau}{q}}{\bigstep{\Choice{C_1}{C_2}}{\sigma}{p}{l}{\tau}{q}}
  \hva\and
%  \inferrule[(BS:Loop)]{\bigstep{C^n}{\sigma}{p}{l}{\tau}{q}}{\bigstep{\Loop{C}}{\sigma}{p}{l}{\tau}{q}}
%  \hva\and
  \inferrule[(BS:LoopZero)]{}{\bigstep{\Loop{C}}{\sigma}{p}{p}{\sigma}{p}}
  \hva\and
  \inferrule[(BS:Loop)]{\bigstep{\Seq{C}{\Loop{C}}}{\sigma}{p}{l}{\tau}{q}}{\bigstep{\Loop{C}}{\sigma}{p}{l}{\tau}{q}}
  \hva\and
  \inferrule[(BS:Local)]{\bigstep{C}{\sigma}{p}{l}{\tau}{q}}{\bigstep{\Local{x}{C}}{\sigma[x\mapsto v]}{p}{l}{\tau[x\mapsto v]}{q}}
\end{mathpar}
\caption{Rules for the Big-step Semantics}
\label{fig:fullsemantics}
\end{figure}

\cref{fig:fullsemantics} shows the complete rules for the big-step semantics.
%
We also write $\bigstepp{C}{\sigma}{p}{\tau}{q}$ to denote $\exists l.\bigstep{C}{\sigma}{p}{l}{\tau}{q}$, and $\bigstepl{C}{\sigma}{p}{\tau}{q}$ to denote $\exists l\le 0.\bigstep{C}{\sigma}{p}{l}{\tau}{q}$.

\begin{lemma}\label{lem:relax}
  For all $C,\sigma,p,\tau,q,l,f$, if $\bigstep{C}{\sigma}{p}{l}{\tau}{q}$, then $\bigstep{C}{\sigma}{p+f}{l+f}{\tau}{q+f}$.
\end{lemma}

\begin{proof}
  By induction on the derivation.
  \begin{itemize}
    \item \textbf{Case} \textsc{(BS:Skip)}, \textsc{(BS:Assign)}, \textsc{(BS:Assume)}:\\
      Immediate by the corresponding rule.
    \item \textbf{Case} \textsc{(BS:Tick)}:\\
      By (BS:Tick), we have $\bigstep{\Tick{e}}{\sigma}{p+f}{\min\{p+f,p-\eval{e}{\sigma}+f\}}{\sigma}{p-\eval{e}{\sigma}+f}$. Since $\min\{p+f,p-\eval{e}{\sigma}+f\}=\min\{p,p-\eval{e}{\sigma}\}+f$, the conclusion holds.
    \item \textbf{Case} \textsc{(BS:Seq)}:\\
      By the induction hypothesis, we have $\bigstep{C_1}{\sigma}{p+f}{l_1+f}{\rho}{r+f}$ and $\bigstep{C_2}{\rho}{r+f}{l_2+f}{\tau}{q+f}$. Then by (BS:Seq), we have $\bigstep{\Seq{C_1}{C_2}}{\sigma}{p+f}{\min\{l_1+f,l_2+f\}}{\tau}{q+f}$. Since $\min\{l_1+f,l_2+f\}=\min\{l_1,l_2\}+f$, the conclusion holds.
    \item \textbf{Case} \textsc{(BS:ChoiceL)}:\\
      By the induction hypothesis, we have $\bigstep{C_1}{\sigma}{p+f}{l+f}{\tau}{q+f}$. Then by (BS:ChoiceL), we have $\bigstep{\Choice{C_1}{C_2}}{\sigma}{p+f}{l+f}{\tau}{q+f}$.
    \item \textbf{Case} \textsc{(BS:ChoiceR)}:\\
      This case is similar to the previous case.
    \item \textbf{Case} \textsc{(BS:LoopZero)}:\\
      Immediate by the rule (BS:LoopZero).
    \item \textbf{Case} \textsc{(BS:Loop)}:\\
      By the induction hypothesis, we have $\bigstep{\Seq{C}{\Loop{C}}}{\sigma}{p+f}{l+f}{\tau}{q+f}$. Then by (BS:Loop), we have $\bigstep{\Loop{C}}{\sigma}{p+f}{l+f}{\tau}{q+f}$.
    \item \textbf{Case} \textsc{(BS:Local)}:\\
      By the induction hypothesis, we have $\bigstep{C}{\sigma}{p+f}{l+f}{\tau}{q+f}$. Then by (BS:Local), we have $\bigstep{\Local{x}{C}}{\sigma[x\mapsto v]}{p+f}{l+f}{\tau[x\mapsto v]}{q+f}$.
  \end{itemize}
\end{proof}

\begin{definition}[Semantics]
  The semantics of QFUA triple, QBUA triple, and \QBUAd triple are defined as follows:
  \begin{itemize}
    \item $\fJudge{P}{C}{Q}$ holds iff for all $\tau$ and $q$ such that $q\ge Q(\tau)$, there exists a $\sigma$ and a $p$ such that $p\ge P(\sigma)$ and $\bigstepp{C}{\sigma}{p}{\tau}{q}$.
    \item $\bJudge{P}{C}{Q}$ holds iff for all $\sigma$ and $p$ such that $p\le P(\sigma)$, there exists a $\tau$ and a $q$ such that $q\le Q(\tau)$ and $\bigstepp{C}{\sigma}{p}{\tau}{q}$.
    \item $\dJudge{P}{C}{Q}$ holds iff for all $\sigma$ and $p$ such that $p\le P(\sigma)$, there exists a $\tau$ and a $q$ such that $q\le Q(\tau)$ and $\bigstepl{C}{\sigma}{p}{\tau}{q}$.
  \end{itemize}
\end{definition}

\begin{definition}[Predicate Transformer] The $\mathrm{pre}$ and $\mathrm{post}$ transformers are defined as follows:
  \begin{itemize}
    \item $\pre{C}{Q}\coloneqq\lambda\sigma.\sup\{p:\exists\tau,q.q\le Q(\tau)\land\bigstepp{C}{\sigma}{p}{\tau}{q}\}$.
    \item $\prel{C}{Q}\coloneqq\lambda\sigma.\sup\{p:\exists\tau,q.q\le Q(\tau)\land\bigstepl{C}{\sigma}{p}{\tau}{q}\}$.
    \item $\post{C}{P}\coloneqq\lambda\tau.\inf\{q:\exists\sigma,p.p\ge P(\sigma)\land\bigstepp{C}{\sigma}{p}{\tau}{q}\}$.
  \end{itemize}
\end{definition}

\begin{proposition}\label{prop:const}
  For all $P,C$ such that $\Fv(P)\cap\Mod(C)=\emptyset$, for all $\sigma,p,\tau,q,l$, if $\bigstep{C}{\sigma}{p}{l}{\tau}{q}$, then $P(\sigma)=P(\tau)$.
\end{proposition}

%\begin{proposition}\label{prop:porg-const}
%  For all $C,\sigma,p,\tau,q,l$ such that $\bigstep{C}{\sigma}{p}{l}{\tau}{q}$, for all $x\notin\Fv(C)$ and $v$, $\bigstep{C}{\sigma[x\mapsto v]}{p}{l}{\tau[x\mapsto v]}{q}$.
%\end{proposition}
%
%\begin{proposition}\label{prop:pred-const}
%  For all $P,\sigma$, for all $x\notin\Fv(P)$ and $v$, $P(\sigma)=P(\sigma[x\mapsto v])$.
%\end{proposition}

\begin{proposition}\label{prop:subst}
  For all $C,\sigma,p,\tau,q,l$ such that $\bigstep{C}{\sigma}{p}{l}{\tau}{q}$, for all $x,y,v$, if $y\notin\Fv(C)$, then $\bigstep{C[y/x]}{\sigma[x\mapsto v][y\mapsto\eval{x}{\sigma}]}{p}{l}{\tau[x\mapsto v][y\mapsto\eval{x}{\tau}]}{q}$.
\end{proposition}

\begin{figure}
\begin{mathpar}
  \inferrule[(F:Skip)]{}{\fjudge{P}{\Skip}{P}}
  \hva\and
  \inferrule[(F:Assign)]{}{\fjudge{P}{\Assign{x}{e}}{\Inf x'.P[x'/x]\curlyvee[x\ne e[x'/x]]}}
  \hva\and
  \inferrule[(F:Assume)]{}{\fjudge{P\curlyvee[\neg B]}{\Assume{B}}{P\curlyvee[\neg B]}}
  \hva\and
  \inferrule[(F:Tick)]{}{\fjudge{P}{\Tick{e}}{P-e}}
  \hva\and
  \inferrule[(F:Seq)]{\fjudge{P}{C_1}{R} \\ \fjudge{R}{C_2}{Q}}{\fjudge{P}{\Seq{C_1}{C_2}}{Q}}
  \hva\and
  \inferrule[(F:ChoiceL)]{\fjudge{P}{C_1}{Q}}{\fjudge{P}{\Choice{C_1}{C_2}}{Q}}
  \hva\and
  \inferrule[(F:ChoiceR)]{\fjudge{P}{C_2}{Q}}{\fjudge{P}{\Choice{C_1}{C_2}}{Q}}
  \hva\and
  \inferrule[(F:Loop)]{\forall n<k.\fjudge{P(n)}{C}{P(n+1)}}{\fjudge{P(0)}{\Loop{C}}{P(k)}}
  \hva\and
  \inferrule[(F:Local)]{\fjudge{P}{C}{Q}}{\fjudge{\Inf x.P}{\Local{x}{C}}{\Inf x.Q}}
  \hva\and
  \inferrule[(F:Disj)]{\forall i\in I.\fjudge{P_i}{C}{Q_i}}{\fjudge{\bigcurlywedge_{i\in I}P_i}{C}{\bigcurlywedge_{i\in I}Q_i}}
  \hva\and
  \inferrule[(F:Constancy)]{\fjudge{P}{C}{Q}\\\Fv(B)\cap\Mod(C)=\emptyset}{\fjudge{P\curlyvee[B]}{C}{Q\curlyvee[B]}}
  \hva\and
  \inferrule[(F:Relax)]{\fjudge{P}{C}{Q}\\\Fv(F)\cap\Mod(C)=\emptyset}{\fjudge{P+F}{C}{Q+F}}
  \hva\and
  \inferrule[(F:Cons)]{P\preceq P'\\\fjudge{P'}{C}{Q'}\\Q'\preceq Q}{\fjudge{P}{C}{Q}}
  \hva\and
  \inferrule[(F:Subst)]{\fjudge{P}{C}{Q}\\y\notin\Fv(P)\cup\Fv(Q)\cup\Fv(C)}{\fjudge{P[y/x]}{C[y/x]}{Q[y/x]}}
\end{mathpar}
\caption{Proof Rules for QFUA Triples}
\label{fig:fullqfua}
\end{figure}

\cref{fig:fullqfua} shows the complete proof rules for QFUA triples.

\begin{theorem}[QFUA Soundness]\label{thm:qfua-sound}
  For all $P$, $C$, $Q$, if $\fjudge{P}{C}{Q}$ is derivable, then $\fJudge{P}{C}{Q}$ holds.
\end{theorem}

\begin{proof}
  By induction on the derivation.
  \begin{itemize}
    \item \textbf{Case} \textsc{(F:Skip)}:\\
      Let $P$ be arbitrary such that $\fjudge{P}{\Skip}{P}$. For any $\tau$ and $q$ satisfying $q\ge P(\tau)$, take $\sigma=\tau$ and $p=q$. Then we have $p\ge P(\sigma)$ and $\bigstep{\Skip}{\sigma}{p}{q}{\tau}{q}$ by \textsc{(BS:Skip)}.
    \item \textbf{Case} \textsc{(F:Assign)}:\\
      Let $P$ be arbitrary such that $\fjudge{P}{\Assign{x}{e}}{Q}$, where $Q=\Inf x'.P[x'/x]\curlyvee[x\ne e[x'/x]]$. For any $\tau$ and $q$ satisfying $q\ge Q(\tau)$, notice that
      \begin{align*}
        Q(\tau)
        &=\inf_v\left\{(P[x'/x]\curlyvee[x\ne e[x'/x]])(\tau[x'\mapsto v])\right\}\\
        &=\inf_v\{\max\{P[x'/x](\tau[x'\mapsto v]),[x\ne e[x'/x]](\tau[x'\mapsto v])\}\}\\
        &=\inf_v\{\max\{P(\tau[x\mapsto v]),[\eval{x}{\tau}\ne\eval{e}{\tau[x\mapsto v]}]\}\}.
      \end{align*}
      Thus, there exists a $v$ such that $q\ge P(\tau[x\mapsto v])$ and $\eval{x}{\tau}=\eval{e}{\tau[x\mapsto v]}$. Take $\sigma=\tau[x\mapsto v]$ and $p=q$, then we have $p\ge P(\sigma)$ and $\tau=\sigma[x\mapsto\eval{e}{\sigma}]$. Therefore, $\bigstep{\Assign{x}{e}}{\sigma}{p}{q}{\sigma}{q}$ holds by \textsc{(BS:Assign)}.
    \item \textbf{Case} \textsc{(F:Assume)}:\\
      Let $P$ be arbitrary such that $\fjudge{P\curlyvee[\neg B]}{\Assume{B}}{P\curlyvee[\neg B]}$. For any $\tau$ and $q$ satisfying $q\ge(P\curlyvee[\neg B])(\tau)$, we have $q\ge P(\tau)$ and $\eval{B}{\tau}=\true$. Take $\sigma=\tau$ and $p=q$. Then we have $p\ge(P\curlyvee[\neg B])(\sigma)$ and $\bigstep{\Assume{B}}{\sigma}{p}{q}{\sigma}{q}$ by \textsc{(BS:Assume)}.
    \item \textbf{Case} \textsc{(F:Tick)}:\\
      Let $P$ be arbitrary such that $\fjudge{P}{\Tick{e}}{P-e}$. For any $\tau$ and $q$ satisfying $q\ge (P-e)(\tau)=P(\tau)-\eval{e}{\tau}$, take $\sigma=\tau$ and $p=q+\eval{e}{\tau}$. Then we have $p\ge P(\sigma)$ and $\bigstep{\Tick{e}}{\sigma}{p}{\min\{p,p-\eval{e}{\sigma}\}}{\tau}{q}$ by \textsc{(BS:Tick)}.
    \item \textbf{Case} \textsc{(F:Seq)}:\\
      Let $P,Q,R,C_1,C_2$ be arbitrary such that $\fjudge{P}{C_1}{R}$ and $\fjudge{R}{C_2}{Q}$. By induction hypothesis, we have $\fJudge{P}{C_1}{R}$ and $\fJudge{R}{C_2}{Q}$. For any $\tau$ and $q$ satisfying $q\ge Q(\tau)$, there exists a $\rho$ and an $r$ such that $r\ge R(\rho)$ and $\bigstep{C_2}{\rho}{r}{l_2}{\tau}{q}$ for some $l_2$. Then, there exists a $\sigma$ and a $p$ such that $p\ge P(\sigma)$ and $\bigstep{C_1}{\sigma}{p}{l_1}{\rho}{r}$ for some $l_1$. Hence, $\bigstep{\Seq{C_1}{C_2}}{\sigma}{p}{\min\{l_1,l_2\}}{\tau}{q}$ holds by \textsc{(BS:Seq)}.
    \item \textbf{Case} \textsc{(F:ChoiceL)}:\\
      Let $P,Q,C_1,C_2$ be arbitrary such that $\fjudge{P}{C_1}{Q}$. By induction hypothesis, we have $\fJudge{P}{C_1}{Q}$. For any $\tau$ and $q$ satisfying $q\ge Q(\tau)$, there exists a $\sigma$ and a $p$ such that $p\ge P(\sigma)$ and $\bigstep{C_1}{\sigma}{p}{l}{\tau}{q}$ for some $l$. Hence, $\bigstep{\Choice{C_1}{C_2}}{\sigma}{p}{l}{\tau}{q}$ holds by \textsc{(BS:ChoiceL)}.
    \item \textbf{Case} \textsc{(F:ChoiceR)}:\\
      This case is similar to the previous one.
    \item \textbf{Case} \textsc{(F:Loop)}:\\
      Let $P,C$ be arbitrary such that $\forall n<k.\fjudge{P(n)}{C}{P(n+1)}$. By induction hypothesis, we have $\forall n<k.\fJudge{P(n)}{C}{P(n+1)}$. We will inductively prove that for all $0\le i\le k$, $\fJudge{P(k-i)}{\Loop{C}}{P(k)}$ holds:
      \begin{itemize}
        \item \textbf{Base case} $i=0$:\\
          Let $\tau$ and $q$ be arbitrary such that $q\ge P(k)(\tau)$. Take $\sigma=\tau$ and $p=q$. Then we have $p\ge P(k)(\sigma)$ and $\bigstep{\Loop{C}}{\sigma}{p}{p}{\sigma}{p}$ by \textsc{(BS:LoopZero)}. It follows that $\fJudge{P(k)}{\Loop{C}}{P(k)}$.
        \item \textbf{Inductive step} $i\to i+1$:\\
          Assume that $\fJudge{P(k-i)}{\Loop{C}}{P(k)}$ holds. Let $\tau$ and $q$ be arbitrary such that $q\ge P(k)(\tau)$. By the assumption, there exists a $\rho$ and an $r$ such that $r\ge P(k-i)(\rho)$ and $\bigstep{\Loop{C}}{\rho}{r}{l_1}{\tau}{q}$ for some $l_1$. Besides, $\fJudge{P(k-(i+1))}{C}{P(k-i)}$ implies that there exists a $\sigma$ and a $p$ such that $p\ge P(k-(i+1))(\sigma)$ and $\bigstep{C}{\sigma}{p}{l_2}{\rho}{r}$ for some $l_2$. Thus, $\bigstep{\Seq{C}{\Loop{C}}}{\sigma}{p}{l}{\tau}{q}$ holds by \textsc{(BS:Seq)} for $l=\min\{l_1,l_2\}$, and then $\bigstep{\Loop{C}}{\sigma}{p}{l}{\tau}{q}$ holds by \textsc{(BS:Loop)}. It follows that $\fJudge{P(k-(i+1))}{\Loop{C}}{P(k)}$.
      \end{itemize}
      Taking $i=k$, we have $\fJudge{P(0)}{\Loop{C}}{P(k)}$.
    \item \textbf{Case} \textsc{(F:Local)}:\\
      Let $P,Q,C$ be arbitrary such that $\fjudge{P}{C}{Q}$. By induction hypothesis, we have $\fJudge{P}{C}{Q}$. For any $\tau$ and $q$ satisfying $q\ge (\Inf x.Q)(\tau)=\inf_v\{Q(\tau[x\mapsto v])\}$, there exists a $v$ such that $q\ge Q(\tau[x\mapsto v])$. Since $\fJudge{P}{C}{Q}$, there exists a $\sigma$ and a $p$ such that $p\ge P(\sigma)$ and $\bigstep{C}{\sigma}{p}{l}{\tau[x\mapsto v]}{q}$ for some $l$. This implies that $p\ge (\Inf x.P)(\sigma[x\mapsto\eval{x}{\tau}])$ and $\bigstep{\Local{x}{C}}{\sigma[x\mapsto\eval{x}{\tau}]}{p}{l}{\tau}{q}$ by \textsc{(BS:Local)}. It follows that $\fJudge{\Inf x.P}{\Local{x}{C}}{\Inf x.Q}$.
    \item \textbf{Case} \textsc{(F:Disj)}:\\
      Let $C$ be arbitrary. And for $i\in I$, let $P_i$ and $Q_i$ be arbitrary such that $\fjudge{P_i}{C}{Q_i}$. By induction hypothesis, we have $\fJudge{P_i}{C}{Q_i}$. For any $\tau$ and $q$ satisfying $q\ge\left(\bigcurlywedge_{i\in I}Q_i\right)(\tau)$, there exists an $i\in I$ such that $q\ge Q_i(\tau)$. Since $\fJudge{P_i}{C}{Q_i}$, there exists a $\sigma$ and a $p$ such that $p\ge P_i(\sigma)$ and $\bigstep{C}{\sigma}{p}{l}{\tau}{q}$ for some $l$. Therefore, we have $p\ge\left(\bigcurlywedge_{i\in I}P_i\right)(\sigma)$ and $\bigstep{C}{\sigma}{p}{l}{\tau}{q}$.
    \item \textbf{Case} \textsc{(F:Constancy)}:\\
      Let $P,Q,B,C$ be arbitrary such that $\fjudge{P}{C}{Q}$ and $\Fv(B)\cap\Mod(C)=\emptyset$. By induction hypothesis, we have $\fJudge{P}{C}{Q}$. For any $\tau$ and $q$ satisfying $q\ge (Q\curlyvee[B])(\tau)$, we have $q\ge Q(\tau)$ and $\eval{\neg B}{\tau}=\true$. Then there exists a $\sigma$ and a $p$ such that $p\ge P(\sigma)$ and $\bigstep{C}{\sigma}{p}{l}{\tau}{q}$ for some $l$. By proposition \ref{prop:const}, we have $\eval{\neg B}{\sigma}=\true$, and then $p\ge (P\curlyvee[B])(\sigma)$. It follows that $\fJudge{P\curlyvee[B]}{C}{Q\curlyvee[B]}$.
    \item \textbf{Case} \textsc{(F:Relax)}:\\
      Let $P,Q,F,C$ be arbitrary such that $\fjudge{P}{C}{Q}$ and $\Fv(F)\cap\Mod(C)=\emptyset$. By induction hypothesis, we have $\fJudge{P}{C}{Q}$. For any $\tau$ and $q$ satisfying $q\ge (Q+F)(\tau)=Q(\tau)+F(\tau)$, we have $q-F(\tau)\ge Q(\tau)$. Then there exists a $\sigma$ and a $p$ such that $p\ge P(\sigma)$ and $\bigstep{C}{\sigma}{p}{l}{\tau}{q-F(\tau)}$ for some $l$. By lemma \ref{lem:relax} and proposition \ref{prop:const}, we have $\bigstep{C}{\sigma}{p+F(\sigma)}{l+F(\sigma)}{\tau}{q}$. Since $p+F(\sigma)\ge P(\sigma)+F(\sigma)=(P+F)(\sigma)$, we have $\fJudge{P+F}{C}{Q+F}$.
    \item \textbf{Case} \textsc{(F:Cons)}:\\
      Let $P,P',Q,Q',C$ be arbitrary such that $P\preceq P'$, $Q'\preceq Q$ and $\fjudge{P'}{C}{Q'}$. By induction hypothesis, we have $\fJudge{P'}{C}{Q'}$. For any $\tau$ and $q$ satisfying $q\ge Q(\tau)$, we have $q\ge Q'(\tau)$. Then there exists a $\sigma$ and a $p$ such that $p\ge P'(\sigma)$ and $\bigstep{C}{\sigma}{p}{l}{\tau}{q}$ for some $l$. Since $P(\sigma)\le P'(\sigma)$, we have $p\ge P(\sigma)$. It follows that $\fJudge{P}{C}{Q}$.
    \item \textbf{Case} \textsc{(F:Subst)}:\\
      Let $P,Q,C$ be arbitrary such that $\fjudge{P}{C}{Q}$, and let $x,y$ by arbitrary such that $y\notin\Fv(P)\cup\Fv(Q)\cup\Fv(C)$. By induction hypothesis, we have $\fJudge{P}{C}{Q}$. For any $\tau$ and $q$ satisfying $q\ge Q[y/x](\tau)=Q(\tau[x\mapsto\eval{y}{\tau}])$, we have $q\ge Q(\tau')$ since $y\notin\Fv(Q)$, where $\tau'=\tau[x\mapsto\eval{y}{\tau}][y\mapsto\eval{x}{\tau}]$. Then there exists a $\sigma$ and a $p$ such that $p\ge P(\sigma)$ and $\bigstep{C}{\sigma}{p}{l}{\tau'}{q}$ for some $l$. By proposition \ref{prop:subst}, we have $\bigstep{C[y/x]}{\sigma'}{p}{l}{\tau}{q}$, where $\sigma'=\sigma[x\mapsto\eval{x}{\tau}][y\mapsto\eval{x}{\sigma}]$. Since $y\notin\Fv(P)$, weh have $p\ge P(\sigma)=P[y/x](\sigma')$. It follows that $\fJudge{P[y/x]}{C[y/x]}{Q[y/x]}$.
  \end{itemize}
\end{proof}

\begin{figure}
\begin{mathpar}
  \inferrule[(B:Skip)]{}{\bjudge{P}{\Skip}{P}}
  \hva\and
  \inferrule[(B:Assign)]{}{\bjudge{P}{\Assign{x}{e}}{\Sup x'.P[x'/x]\curlywedge[x=e[x'/x]]}}
  \hva\and
  \inferrule[(B:Assume)]{}{\bjudge{P\curlywedge[B]}{\Assume{B}}{P\curlywedge[B]}}
  \hva\and
  \inferrule[(B:Tick)]{}{\bjudge{P}{\Tick{e}}{P-e}}
  \hva\and
  \inferrule[(B:Seq)]{\bjudge{P}{C_1}{R} \\ \bjudge{R}{C_2}{Q}}{\bjudge{P}{\Seq{C_1}{C_2}}{Q}}
  \hva\and
  \inferrule[(B:ChoiceL)]{\bjudge{P}{C_1}{Q}}{\bjudge{P}{C_1+C_2}{Q}}
  \hva\and
  \inferrule[(B:ChoiceR)]{\bjudge{P}{C_2}{Q}}{\bjudge{P}{C_1+C_2}{Q}}
  \hva\and
%  \inferrule[(Loop0)]{}{\bjudge{P}{C^\star}{P}}
%  \hva\and
%  \inferrule[(Loop)]{\bjudge{P}{\Seq{C^\star}{C}}{Q}}{\bjudge{P}{C^\star}{Q}}
%  \hva\and
  \inferrule[(B:Loop)]{\forall n<k.\bjudge{P(n)}{C}{P(n+1)}}{\bjudge{P(0)}{C^\star}{P(k)}}
  \hva\and
  \inferrule[(B:Local)]{\bjudge{P}{C}{Q}}{\bjudge{\Sup x.P}{\Local{x}{C}}{\Sup x.Q}}
  \hva\and
  \inferrule[(B:Disj)]{\forall i\in I.\bjudge{P_i}{C}{Q_i}}{\bjudge{\bigcurlyvee_{i\in I}P_i}{C}{\bigcurlyvee_{i\in I}Q_i}}
  \hva\and
  \inferrule[(B:Constancy)]{\bjudge{P}{C}{Q}\\\Fv(B)\cap\Mod(C)=\emptyset}{\bjudge{P\curlywedge[B]}{C}{Q\curlywedge[B]}}
  \hva\and
  \inferrule[(B:Relax)]{\bjudge{P}{C}{Q}\\\Fv(F)\cap\Mod(C)=\emptyset}{\bjudge{P+F}{C}{Q+F}}
  \hva\and
  \inferrule[(B:Cons)]{P\preceq P' \\ \judge{P'}{C}{Q'} \\ Q'\preceq Q}{\judge{P}{C}{Q}}
  \hva\and
  \inferrule[(B:Subst)]{\bjudge{P}{C}{Q}\\y\notin\Fv(P)\cup\Fv(Q)\cup\Fv(C)}{\bjudge{P[y/x]}{C[y/x]}{Q[y/x]}}
\end{mathpar}
\caption{Proof Rules for QBUA Triples}
\label{fig:fullqbua}
\end{figure}

\cref{fig:fullqbua} shows the complete proof rules for QBUA triples.

\begin{theorem}[QBUA Soundness]\label{thm:qbua-sound}
  For all $P$, $C$, $Q$, if $\bjudge{P}{C}{Q}$ is derivable, then $\bJudge{P}{C}{Q}$ holds.
\end{theorem}

\begin{proof}
  By induction on the derivation.
  \begin{itemize}
    \item \textbf{Case} \textsc{(B:Skip)}:\\
      Let $P$ be arbitrary such that $\bjudge{P}{\Skip}{P}$. For any $\sigma$ and $p$ satisfying $p\le P(\sigma)$, take $\tau=\sigma$ and $q=p$. Then we have $q\le P(\tau)$ and $\bigstep{\Skip}{\sigma}{p}{p}{\tau}{q}$ by \textsc{(BS:Skip)}.
    \item \textbf{Case} \textsc{(B:Assign)}:\\
      Let $P$ be arbitrary such that $\bjudge{P}{\Assign{x}{e}}{Q}$, where $Q=\Sup x'.P[x'/x]\curlywedge[x=e[x'/x]]$. For any $\sigma$ and $p$ satisfying $p\le P(\sigma)$, take $\tau=\sigma[x\mapsto\eval{e}{\sigma}]$ and $q=p$. Notice that
      \begin{align*}
        Q(\tau)
        &=\sup_v\{(P[x'/x]\curlywedge[x=e[x'/x]])(\tau[x\mapsto v])\}\\
        &=\sup_v\{\min\{(P[x'/x])(\tau[x\mapsto v]),[x=e[x'/x]](\tau[x\mapsto v])\}\}\\
        &=\sup_v\{\min\{P(\tau[x\mapsto v]),[\eval{x}{\tau}=\eval{e}{\tau[x\mapsto v]}]\}\}\\
        &\ge\min\{P(\tau[x\mapsto\eval{x}\sigma]),[\eval{x}{\tau}=\eval{e}{\tau[x\mapsto\eval{\sigma}{x}]}]\}\\
        &=P(\sigma).
    \end{align*}
    Combining with $\bigstep{\Assign{x}{e}}{\sigma}{p}{p}{\tau}{q}$ by \textsc{(BS:Assign)}, we have $\bJudge{P}{\Assign{x}{e}}{Q}$.
    \item \textbf{Case} \textsc{(B:Assume)}:\\
      Let $P$ be arbitrary such that $\bjudge{P\curlywedge[B]}{\Assume{B}}{P\curlywedge[B]}$. For any $\sigma$ and $p$ satisfying $p\le (P\curlywedge[B])(\sigma)$, we have $p\le P(\sigma)$ and $\eval{B}{\sigma}=\true$. Take $\tau=\sigma$ and $q=p$. Then we have $q\le (P\curlywedge[B])(\tau)$ and $\bigstep{\Assume{B}}{\sigma}{p}{p}{\tau}{q}$ by \textsc{(BS:Assume)}.
    \item \textbf{Case} \textsc{(B:Tick)}:\\
      Let $P$ be arbitrary such that $\bjudge{P}{\Tick{e}}{P-e}$. For any $\sigma$ and $p$ satisfying $p\le P(\sigma)$, take $\tau=\sigma$ and $q=p-\eval{e}{\sigma}$. Then we have $q\le (P-e)(\tau)$ and $\bigstep{\Tick{e}}{\sigma}{p}{\min\{p,p-\eval{e}{\sigma}\}}{\tau}{q}$ by \textsc{(BS:Tick)}.
    \item \textbf{Case} \textsc{(B:Seq)}:\\
      Let $P,Q,R,C_1,C_2$ be arbitrary such that $\bjudge{P}{C_1}{R}$ and $\bjudge{R}{C_2}{Q}$. By induction hypothesis, we have $\bJudge{P}{C_1}{R}$ and $\bJudge{R}{C_2}{Q}$. For any $\sigma$ and $p$ satisfying $p\le P(\sigma)$, there exists a $\rho$ and an $r$ such that $r\le R(\rho)$ and $\bigstep{C_1}{\sigma}{p}{l_1}{\rho}{r}$ for some $l_1$. Then, there exists a $\tau$ and a $q$ such that $q\le Q(\tau)$ and $\bigstep{C_2}{\rho}{r}{l_2}{\tau}{q}$ for some $l_2$. Hince, $\bigstep{\Seq{C_1}{C_2}}{\sigma}{p}{\min\{l_1,l_2\}}{\tau}{q}$ holds by \textsc{(BS:Seq)}.
    \item \textbf{Case} \textsc{(B:ChoiceL)}:\\
      Let $P,Q,C_1,C_2$ be arbitrary such that $\bjudge{P}{C_1}{Q}$. By induction hypothesis, we have $\bJudge{P}{C_1}{Q}$. For any $\sigma$ and $p$ satisfying $p\le P(\sigma)$, there exists a $\tau$ and a $q$ such that $q\le Q(\tau)$ and $\bigstep{C_1}{\sigma}{p}{l}{\tau}{q}$ for some $l$. Hence, $\bigstep{\Choice{C_1}{C_2}}{\sigma}{p}{l}{\tau}{q}$ holds by \textsc{(BS:ChoiceL)}.
    \item \textbf{Case} \textsc{(B:ChoiceR)}:\\
      This case is similar to the previous one.
    \item \textbf{Case} \textsc{(B:Loop)}:\\
      Let $P,C$ be arbitrary such that $\forall n<k.\bjudge{P(n)}{C}{P(n+1)}$. By induction hypothesis, we have $\forall n<k.\bJudge{P(n)}{C}{P(n+1)}$. We will inductively prove that for all $0\le i\le k$, $\bJudge{P(k-i)}{\Loop{C}}{P(k)}$ holds:
      \begin{itemize}
        \item \textbf{Base case} $i=0$:\\
          Let $\sigma$ and $p$ be arbitrary such that $p\le P(k)(\sigma)$. Take $\tau=\sigma$ and $q=p$. Then we have $q\le P(k)(\tau)$ and $\bigstep{\Loop{C}}{\sigma}{p}{p}{\tau}{q}$ by \textsc{(BS:LoopZero)}. It follows that $\bJudge{P(k)}{\Loop{C}}{P(k)}$.
        \item \textbf{Inductive step} $i\to i+1$:\\
          Assume that $\bJudge{P(k-i)}{\Loop{C}}{P(k)}$ holds. Let $\sigma$ and $p$ be arbitrary such that $p\le P(k-(i+1))(\sigma)$. By $\bJudge{P(k-(i+1))}{C}{P(k-i)}$, there exists a $\rho$ and an $r$ such that $r\le P(k-i)(\rho)$ and $\bigstep{C}{\sigma}{p}{l_1}{\rho}{r}$ for some $l_1$. Besides, $\bJudge{P(k-i)}{\Loop{C}}{P(k)}$ implies that there exists a $\tau$ and a $q$ such that $q\le P(k)(\tau)$ and $\bigstep{\Loop{C}}{\rho}{r}{l_2}{\tau}{q}$ for some $l_2$. Thus, $\bigstep{\Seq{C}{\Loop{C}}}{\sigma}{p}{l}{\tau}{q}$ holds by \textsc{(BS:Seq)} for $l=\min\{l_1,l_2\}$, and then $\bigstep{\Loop{C}}{\sigma}{p}{l}{\tau}{q}$ holds by \textsc{(BS:Loop)}. It follows that $\bJudge{P(k-(i+1))}{\Loop{C}}{P(k)}$.
        \end{itemize}
        Taking $i=k$, we have $\bJudge{P(0)}{\Loop{C}}{P(k)}$.
    \item \textbf{Case} \textsc{(B:Local)}:\\
      Let $P,Q,C$ be arbitrary such that $\bjudge{P}{C}{Q}$. By induction hypothesis, we have $\bJudge{P}{C}{Q}$. For any $\sigma$ and $p$ satisfying $p\le (\Sup x.P)(\sigma)=\sup_v\{P(\sigma[x\mapsto v])\}$, there exists a $v$ such that $p\le P(\sigma[x\mapsto v])$. Since $\bJudge{P}{C}{Q}$, there exists a $\tau$ and a $q$ such that $q\le Q(\tau)$ and $\bigstep{C}{\sigma[x\mapsto v]}{p}{l}{\tau}{q}$ for some $l$. This implies that $q\le(\Sup x.Q)(\tau[x\mapsto\eval{x}{\sigma}])$ and $\bigstep{\Local{x}{C}}{\sigma}{p}{l}{\tau[x\mapsto\eval{x}{\sigma}]}{q}$ by \textsc{(BS:Local)}. It follows that $\bJudge{\Sup x.P}{\Local{x}{C}}{\Sup x.Q}$.
    \item \textbf{Case} \textsc{(B:Disj)}:\\
      Let $C$ be arbitrary. And for $i\in I$, let $P_i$ and $Q_i$ be arbitrary such that $\bjudge{P_i}{C}{Q_i}$. By induction hypothesis, we have $\bJudge{P_i}{C}{Q_i}$. For any $\sigma$ and $p$ satisfying $p\le\left(\bigcurlyvee_{i\in I}P_i\right)(\sigma)$, there exists an $i\in I$ such that $p\le P_i(\sigma)$. Since $\bJudge{P_i}{C}{Q_i}$, there exists a $\tau$ and a $q$ such that $q\le Q_i(\tau)$ and $\bigstep{C}{\sigma}{p}{l}{\tau}{q}$ for some $l$. Therefore, we have $q\le\left(\bigcurlyvee_{i\in I}Q_i\right)(\tau)$ and $\bigstep{C}{\sigma}{p}{l}{\tau}{q}$.
    \item \textbf{Case} \textsc{(B:Constancy)}:\\
      Let $P,Q,B,C$ be arbitrary such that $\bjudge{P}{C}{Q}$ and $\Fv(B)\cap\Mod(C)=\emptyset$. By induction hypothesis, we have $\bJudge{P}{C}{Q}$. For any $\sigma$ and $p$ satisfying $p\le (P\curlywedge[B])(\sigma)$, we have $p\le P(\sigma)$ and $\eval{B}{\sigma}=\true$. Then there exists a $\tau$ and a $q$ such that $q\le Q(\tau)$ and $\bigstep{C}{\sigma}{p}{l}{\tau}{q}$ for some $l$. By proposition \ref{prop:const}, we have $\eval{B}{\tau}=\true$, and then $q\le (Q\curlywedge[B])(\tau)$. It follows that $\bJudge{P\curlywedge[B]}{C}{Q\curlywedge[B]}$.
    \item \textbf{Case} \textsc{(B:Relax)}:\\
      Let $P,Q,F,C$ be arbitrary such that $\bjudge{P}{C}{Q}$ and $\Fv(F)\cap\Mod(C)=\emptyset$. By induction hypothesis, we have $\bJudge{P}{C}{Q}$. For any $\sigma$ and $p$ satisfying $p\le (P+F)(\sigma)=P(\sigma)+F(\sigma)$, we have $p-F(\sigma)\le P(\sigma)$. Then there exists a $\tau$ and a $q$ such that $q\le Q(\tau)$ and $\bigstep{C}{\sigma}{p-F(\sigma)}{l}{\tau}{q}$ for some $l$. By lemma \ref{lem:relax} and proposition \ref{prop:const}, we have $\bigstep{C}{\sigma}{p}{l+F(\tau)}{\tau}{q+F(\tau)}$. Since $q+F(\tau)\le Q(\tau)+F(\tau)=(Q+F)(\tau)$, we have $\bJudge{P+F}{C}{Q+F}$.
    \item \textbf{Case} \textsc{(B:Cons)}:\\
      Let $P,P',Q,Q',C$ be arbitrary such that $P\preceq P'$, $Q'\preceq Q$ and $\bjudge{P'}{C}{Q'}$. By induction hypothesis, we have $\bJudge{P'}{C}{Q'}$. For any $\sigma$ and $p$ satisfying $p\le P(\sigma)$, we have $p\le P'(\sigma)$. Then there exists a $\tau$ and a $q$ such that $q\le Q'(\tau)$ and $\bigstep{C}{\sigma}{p}{l}{\tau}{q}$ for some $l$. Since $Q'(\tau)\le Q(\tau)$, we have $q\le Q(\tau)$. It follows that $\bJudge{P}{C}{Q}$.
    \item \textbf{Case} \textsc{(B:Subst)}:\\
      Let $P,Q,C$ be arbitrary such that $\bjudge{P}{C}{Q}$, and let $x,y$ be arbitrary such that $y\notin\Fv(P)\cup\Fv(Q)\cup\Fv(C)$. By induction hypothesis, we have $\bJudge{P}{C}{Q}$. For any $\sigma$ and $p$ satisfying $p\le P[y/x](\sigma)=P(\sigma[x\mapsto\eval{y}{\sigma}])$, we have $p\le P(\sigma')$ since $y\notin\Fv(P)$, where $\sigma'=\sigma[x\mapsto\eval{y}{\sigma}][y\mapsto\eval{x}{\sigma}]$. Then there exists a $\tau$ and a $q$ such that $q\le Q(\tau)$ and $\bigstep{C}{\sigma'}{p}{l}{\tau}{q}$ for some $l$. By proposition \ref{prop:subst}, we have $\bigstep{C[y/x]}{\sigma}{p}{l}{\tau'}{q}$, where $\tau'=\tau[x\mapsto\eval{x}{\sigma}][y\mapsto\eval{x}{\tau}]$. Since $y\notin\Fv(Q)$, we have $q\le Q(\tau)=Q[y/x](\tau')$. It follows that $\bJudge{P[y/x]}{C[y/x]}{Q[y/x]}$.
  \end{itemize}
\end{proof}

\begin{figure}
\begin{mathpar}
  \inferrule[(\Bd:Skip)]{P\preceq 0}{\djudge{P}{\Skip}{P}}
  \hva\and
  \inferrule[(\Bd:Assign)]{P\preceq 0}{\djudge{P}{\Assign{x}{e}}{\Sup x'.P[x'/x]\curlywedge[x=e[x'/x]]}}
  \hva\and
  \inferrule[(\Bd:Assume)]{P\preceq 0}{\djudge{P\curlywedge[B]}{\Assume{B}}{P\curlywedge[B]}}
  \hva\and
  \inferrule[(\Bd:Tick)]{P\curlywedge P-e\preceq 0}{\djudge{P}{\Tick{e}}{P-e}}
  \hva\and
  \inferrule[(\Bd:SeqL)]{\djudge{P}{C_1}{R} \\ \bjudge{R}{C_2}{Q}}{\djudge{P}{\Seq{C_1}{C_2}}{Q}}
  \hva\and
  \inferrule[(\Bd:SeqR)]{\bjudge{P}{C_1}{R} \\ \djudge{R}{C_2}{Q}}{\djudge{P}{\Seq{C_1}{C_2}}{Q}}
  \hva\and
  \inferrule[(\Bd:ChoiceL)]{\djudge{P}{C_1}{Q}}{\djudge{P}{C_1+C_2}{Q}}
  \hva\and
  \inferrule[(\Bd:ChoiceR)]{\djudge{P}{C_2}{Q}}{\djudge{P}{C_1+C_2}{Q}}
  \hva\and
%  \inferrule[(Loop0)]{P\preceq 0}{\djudge{P}{C^\star}{P}}
%  \hva\and
%  \inferrule[(Loop)]{\djudge{P}{\Seq{C^\star}{C}}{Q}}{\djudge{P}{C^\star}{Q}}
%  \hva\and
  \inferrule[(\Bd:LoopZero)]{P\preceq 0}{\djudge{P}{C^\star}{P}}
  \hva\and
  \inferrule[(\Bd:Loop)]{\forall n<k.\bjudge{P(n)}{C}{P(n+1)} \\ \exists m<k.\djudge{P(m)}{C}{P(m+1)}}{\djudge{P(0)}{C^\star}{P(k)}}
  \hva\and
  \inferrule[(\Bd:Local)]{\djudge{P}{C}{Q}}{\djudge{\Sup x.P}{\Local{x}{C}}{\Sup x.Q}}
  \hva\and
  \inferrule[(\Bd:Disj)]{\forall i\in I.\djudge{P_i}{C}{Q_i}}{\djudge{\bigcurlyvee_{i\in I}P_i}{C}{\bigcurlyvee_{i\in I}Q_i}}
  \hva\and
  \inferrule[\Bd:Constancy)]{\djudge{P}{C}{Q}\\\Fv(B)\cap\Mod(C)=\emptyset}{\djudge{P\curlywedge[B]}{C}{Q\curlywedge[B]}}
  \hva\and
  \inferrule[(\Bd:Relax)]{\djudge{P}{C}{Q}\\\Fv(F)\cap\Mod(C)=\emptyset\\F\preceq 0}{\djudge{P+F}{C}{Q+F}}
  \hva\and
  \inferrule[(\Bd:Cons)]{P\preceq P' \\ \djudge{P'}{C}{Q'} \\ Q'\preceq Q}{\djudge{P}{C}{Q}}
  \hva\and
  \inferrule[(\Bd:Subst)]{\djudge{P}{C}{Q}\\y\notin\Fv(P)\cup\Fv(Q)\cup\Fv(C)}{\djudge{P[y/x]}{C[y/x]}{Q[y/x]}}
%  \hva\and
%  \inferrule[(\Bd:Weaken)]{\djudge{P}{C}{Q}}{\bjudge{P}{C}{Q}}
\end{mathpar}
\caption{Proof Rules for \QBUAd Triples}
\label{fig:fullqbuad}
\end{figure}

\cref{fig:fullqbuad} shows the complete proof rules for \QBUAd triples.

\begin{theorem}[\QBUAd Soundness]\label{thm:qbuad-sound}
  For all $P$, $C$, $Q$, if $\djudge{P}{C}{Q}$ is derivable, then $\dJudge{P}{C}{Q}$ holds.
\end{theorem}

\begin{proof}
  By induction on the derivation.
  \begin{itemize}
    \item \textbf{Case} \textsc{(\Bd:Skip)}:\\
      Let $P$ be arbitrary such that $P\preceq 0$. For any $\sigma$ and $p$ satisfying $p\le P(\sigma)\le 0$, take $\tau=\sigma$ and $q=p$. Then we have $q\le P(\tau)$ and $\bigstep{\Skip}{\sigma}{p}{p}{\tau}{q}$ by \textsc{(BS:Skip)} for $p\le 0$. It follows that $\dJudge{P}{\Skip}{P}$.
    \item \textbf{Case} \textsc{(\Bd:Assign)}:\\
      Let $P$ be arbitrary such that $P\preceq 0$, and let $Q=\Sup x'.P[x'/x]\curlywedge[x=e[x'/x]]$. For any $\sigma$ and $p$ satisfying $p\le P(\sigma)$, take $\tau=\sigma[x\mapsto\eval{e}{\sigma}]$ and $q=p$. Notice that
      \begin{align*}
        Q(\tau)
        &=\sup_v\{(P[x'/x]\curlywedge[x=e[x'/x]])(\tau[x\mapsto v])\}\\
        &=\sup_v\{\min\{(P[x'/x])(\tau[x\mapsto v]),[x=e[x'/x]](\tau[x\mapsto v])\}\}\\
        &=\sup_v\{\min\{P(\tau[x\mapsto v]),[\eval{x}{\tau}=\eval{e}{\tau[x\mapsto v]}]\}\}\\
        &\ge\min\{P(\tau[x\mapsto\eval{x}\sigma]),[\eval{x}{\tau}=\eval{e}{\tau[x\mapsto\eval{\sigma}{x}]}]\}\\
        &=P(\sigma).
    \end{align*}
    Combining with $\bigstep{\Assign{x}{e}}{\sigma}{p}{p}{\tau}{q}$ by \textsc{(BS:Assign)} and $p\le P(\sigma)\le 0$, we have $\dJudge{P}{\Assign{x}{e}}{Q}$.
    \item \textbf{Case} \textsc{(\Bd:Assume)}:\\
      Let $P$ be arbitrary such that $P\preceq 0$. For any $\sigma$ and $p$ satisfying $p\le (P\curlywedge[B])(\sigma)\le 0$, we have $p\le P(\sigma)\le 0$ and $\eval{B}{\sigma}=\true$. Take $\tau=\sigma$ and $q=p$. Then we have $q\le (P\curlywedge[B])(\tau)$ and $\bigstep{\Assume{B}}{\sigma}{p}{p}{\tau}{q}$ by \textsc{(BS:Assume)} for $p\le 0$. It follows that $\dJudge{P\curlywedge[B]}{\Assume{B}}{P\curlywedge[B]}$.
    \item \textbf{Case} \textsc{(\Bd:Tick)}:\\
      Let $P$ be arbitrary such that $(P\curlywedge P-e)\preceq 0$. For any $\sigma$ and $p$ satisfying $p\le P(\sigma)$, take $\tau=\sigma$ and $q=p-\eval{e}{\sigma}$. Then we have $q\le (P-e)(\tau)$ and $\bigstep{\Tick{e}}{\sigma}{p}{\min\{p,p-\eval{e}{\sigma}\}}{\tau}{q}$ by \textsc{(BS:Tick)}. Since $\min\{p,p-\eval{e}{\sigma}\}=(P\curlywedge P-e)(\sigma)\le 0$, we have $\dJudge{P}{\Tick{e}}{P-e}$.
    \item \textbf{Case} \textsc{(\Bd:SeqL)}:\\
      Let $P,Q,R,C_1,C_2$ be arbitrary such that $\djudge{P}{C_1}{R}$ and $\bjudge{R}{C_2}{Q}$. By induction hypothesis, we have $\dJudge{P}{C_1}{R}$ and by theorem \ref{thm:qbua-sound} we have $\bJudge{R}{C_2}{Q}$. For any $\sigma$ and $p$ satisfying $p\le P(\sigma)$, there exists a $\rho$ and an $r$ such that $r\le R(\rho)$ and $\bigstep{C_1}{\sigma}{p}{l_1}{\rho}{r}$ for some $l_1\le 0$. Then there exists a $\tau$ and a $q$ such that $q\le Q(\tau)$ and $\bigstep{C_2}{\rho}{r}{l_2}{\tau}{q}$ for some $l_2$. Hence, $\bigstep{\Seq{C_1}{C_2}}{\sigma}{p}{\min\{l_1,l_2\}}{\tau}{q}$ holds by \textsc{(BS:Seq)}. Since $l_1\le 0$, we have $\min\{l_1,l_2\}\le 0$. It follows that $\dJudge{P}{\Seq{C_1}{C_2}}{Q}$.
    \item \textbf{Case} \textsc{(\Bd:SeqR)}:\\
      This case is similar to the previous one.
    \item \textbf{Case} \textsc{(\Bd:ChoiceL)}:\\
      Let $P,Q,C_1,C_2$ be arbitrary such that $\djudge{P}{C_1}{Q}$. By induction hypothesis, we have $\dJudge{P}{C_1}{Q}$. For any $\sigma$ and $p$ satisfying $p\le P(\sigma)$, there exists a $\tau$ and a $q$ such that $q\le Q(\tau)$ and $\bigstep{C_1}{\sigma}{p}{l}{\tau}{q}$ for some $l\le 0$. Hence, $\bigstep{\Choice{C_1}{C_2}}{\sigma}{p}{l}{\tau}{q}$ holds by \textsc{(BS:ChoiceL)} for $l\le 0$. It follows that $\dJudge{P}{\Choice{C_1}{C_2}}{Q}$.
    \item \textbf{Case} \textsc{(\Bd:ChoiceR)}:\\
      This case is similar to the previous one.
    \item \textbf{Case} \textsc{(\Bd:LoopZero)}:\\
      Let $P$ be arbitrary such that $P\preceq 0$. For any $\sigma$ and $p$ satisfying $p\le P(\sigma)\le 0$, take $\tau=\sigma$ and $q=p$. Then we have $q\le P(\tau)$ and $\bigstep{\Skip}{\Loop{C}}{p}{p}{\tau}{q}$ by \textsc{(BS:LoopZero)} for $p\le 0$. It follows that $\dJudge{P}{\Loop{C}}{P}$.
    \item \textbf{Case} \textsc{(\Bd:Loop)}:\\
      Let $P,C$ be arbitrary such that $\bjudge{P(n)}{C}{P(n+1)}$ for all $n<k$ and $\djudge{P(m)}{C}{P(m+1)}$ for some $m<k$. By theorem \ref{thm:qbua-sound} we have $\bJudge{P(n)}{C}{P(n+1)}$ for all $n<k$ and by induction hypothesis we have $\dJudge{P(m)}{C}{P(m+1)}$. Notice that $\forall m+1\le n<k. \bjudge{P(n)}{C}{P(n+1)}$, so by \textsc{(B:Loop)} we can derive $\bjudge{P(m+1)}{\Loop{C}}{P(k)}$. Thus, we have $\bJudge{P(m+1)}{\Loop{C}}{P(k)}$ by theorem \ref{thm:qbua-sound}.\\
      Therefore, we can proof $\dJudge{P(m)}{\Loop{C}}{P(k)}$ as follows: for any $\sigma$ and $p$ satisfying $p\le P(m)(\sigma)$, by $\dJudge{P(m)}{C}{P(m+1)}$, there exists a $\rho$ and an $r$ such that $r\le P(m+1)(\rho)$ and $\bigstep{C}{\sigma}{p}{l_1}{\rho}{r}$ for some $l_1\le 0$. Then, by $\bJudge{P(m+1)}{\Loop{C}}{P(k)}$, there exists a $\tau$ and a $q$ such that $q\le P(k)(\tau)$ and $\bigstep{\Loop{C}}{\rho}{r}{l_2}{\tau}{q}$ for some $l_2$. Hence, $\bigstep{\Seq{C}{\Loop{C}}}{\sigma}{p}{\min\{l_1,l_2\}}{\tau}{q}$ holds by \textsc{(BS:Seq)}, and consequently $\bigstep{\Loop{C}}{\sigma}{p}{\min\{l_1,l_2\}}{\tau}{q}$ holds by \textsc{(BS:Loop)}. Since $l_1\le 0$, we have $\min\{l_1,l_2\}\le 0$. It follows that $\dJudge{P(m)}{\Loop{C}}{P(k)}$.\\
      The remain thing is to inductively prove that for all $0\le i\le m$, $\dJudge{P(m-i)}{\Loop{C}}{P(k)}$ holds:
      \begin{itemize}
        \item \textbf{Base case} $i=0$:\\
          $\dJudge{P(m)}{\Loop{C}}{P(k)}$ is already proved.
        \item \textbf{Inductive step} $i\to i+1$:\\
          Assume $\dJudge{P(m-i)}{\Loop{C}}{P(k)}$ holds. Let $\sigma$ and $p$ be arbitrary such that $p\le P(m-(i+1))(\sigma)$. By $\bJudge{P(m-(i+1))}{C}{P(m-i)}$, there exists a $\rho$ and an $r$ such that $r\le P(k-i)(\rho)$ and $\bigstep{C}{\sigma}{p}{l_1}{\rho}{r}$ for some $l_1$. Besides, $\dJudge{P(m-i)}{\Loop{C}}{k}$ implies that there exists a $\tau$ and a $q$ such that $q\le P(k)(\tau)$ and $\bigstep{\Loop{C}}{\rho}{r}{l_2}{\tau}{q}$ for some $l_2\le 0$. Hence, $\bigstep{\Seq{C}{\Loop{C}}}{\sigma}{p}{\min\{l_1,l_2\}}{\tau}{q}$ holds by \textsc{(BS:Seq)}, and consequently $\bigstep{\Loop{C}}{\sigma}{p}{\min\{l_1,l_2\}}{\tau}{q}$ holds by \textsc{(BS:Loop)}. Since $l_2\le 0$, we have $\min\{l_1,l_2\}\le 0$. It follows that $\dJudge{P(m-(i+1))}{\Loop{C}}{P(k)}$.
      \end{itemize}
      Taking $i=m$, we have $\dJudge{P(0)}{\Loop{C}}{P(k)}$.
    \item \textbf{Case} \textsc{(\Bd:Local)}:\\
      Let $P,Q,C$ be arbitrary such that $\djudge{P}{C}{Q}$. By induction hypothesis, we have $\dJudge{P}{C}{Q}$. For any $\sigma$ and $p$ satisfying $p\le (\Sup x.P)(\sigma)=\sup_v\{P(\sigma\mapsto v)\}$, there exists a $v$ such that $p\le P(\sigma[x\mapsto v])$. Since $\dJudge{P}{C}{Q}$, there exists a $\tau$ and a $q$ such that $q\le Q(\tau)$ and $\bigstep{C}{\sigma[x\mapsto v]}{p}{l}{\tau}{q}$ for some $l\le 0$. This implies that $q\le (\Sup x.Q)(\tau[x\mapsto\eval{x}{\sigma}]$ and $\bigstep{\Local{x}{C}}{\sigma}{p}{l}{\tau[x\mapsto\eval{x}{\sigma}]}{q}$ by \textsc{(BS:Local)} for $l\le 0$. It follows that $\dJudge{(\Sup x.P)}{\Local{x}{C}}{(\Sup x.Q)}$.
    \item \textbf{Case} \textsc{(\Bd:Constancy)}:\\
      Let $P,Q,B,C$ be arbitrary such that $\djudge{P}{C}{Q}$ and $\Fv(B)\cap\Mod(C)=\emptyset$. By induction hypothesis, we have $\dJudge{P}{C}{Q}$. For any $\sigma$ and $p$ satisfying $p\le (P\curlywedge[B])(\sigma)$, we have $p\le P(\sigma)$ and $\eval{B}{\sigma}=\true$. Then there exists a $\tau$ and a $q$ such that $q\le Q(\tau)$ and $\bigstep{C}{\sigma}{p}{l}{\tau}{q}$ for some $l\le 0$. By proposition \ref{prop:const}, we have $\eval{B}{\tau}=\true$, and then $q\le (Q\curlywedge[B])(\tau)$. It follows that $\dJudge{P\curlywedge[B]}{C}{Q\curlywedge[B]}$.
    \item \textbf{Case} \textsc{(\Bd:Relax)}:\\
      Let $P,Q,F,C$ be arbitrary such that $\djudge{P}{C}{Q}$, $\Fv(F)\cap\Mod(C)=\emptyset$ and $F\preceq 0$. By induction hypothesis, we have $\dJudge{P}{C}{Q}$. For any $\sigma$ and $p$ satisfying $p\le (P+F)(\sigma)=P(\sigma)+F(\sigma)$, we have $p-F(\sigma)\le P(\sigma)$. Then there exists a $\tau$ and a $q$ such that $q\le Q(\tau)$ and $\bigstep{C}{\sigma}{p-F(\sigma)}{l}{\tau}{q}$ for some $l\le 0$. By lemma \ref{lem:relax} and proposition \ref{prop:const}, we have $\bigstep{C}{\sigma}{p}{l+F(\tau)}{\tau}{q+F(\tau)}$. Since $l\le 0$ and $F\preceq 0$, we have $l+F(\tau)\le 0$. It follows that $\dJudge{P+F}{C}{Q}$.
    \item \textbf{Case} \textsc{(\Bd:Cons)}:\\
      Let $P,P',Q,Q',C$ be arbitrary such that $P\preceq P'$, $Q'\preceq Q$ and $\djudge{P'}{C}{Q'}$. By induction hypothesis, we have $\dJudge{P'}{C}{Q'}$. For any $\sigma$ and $p$ satisfying $p\le P(\sigma)$, we have $p\le P'(\sigma)$. Then there exists a $\tau$ and a $q$ such that $q\le Q'(\tau)$ and $\bigstep{C}{\sigma}{p}{l}{\tau}{q}$ for some $l\le 0$. Since $Q'(\tau)\le Q(\tau)$, we have $q\le Q(\tau)$. It follows that $\dJudge{P}{C}{Q}$.
    \item \textbf{Case} \textsc{(\Bd:Subst)}:\\
      Let $P,Q,C$ be arbitrary such that $\djudge{P}{C}{Q}$, and let $x,y$ be arbitrary such that $y\notin\Fv(P)\cup\Fv(Q)\cup\Fv(C)$. By induction hypothesis, we have $\dJudge{P}{C}{Q}$. For any $\sigma$ and $p$ satisfying $p\le P[y/x](\sigma)=P(\sigma[x\mapsto\eval{y}{\sigma}])$, we have $p\le P(\sigma')$ since $y\notin\Fv(P)$, where $\sigma'=\sigma[x\mapsto\eval{y}{\sigma}][y\mapsto\eval{x}{\sigma}]$. Then there exists a $\tau$ and a $q$ such that $q\le Q(\tau)$ and $\bigstep{C}{\sigma'}{p}{l}{\tau}{q}$ for some $l\le 0$. By proposition \ref{prop:subst}, we have $\bigstep{C[y/x]}{\sigma}{p}{l}{\tau'}{q}$ for $l\le 0$, where $\tau'=\tau[x\mapsto\eval{x}{\sigma}][y\mapsto\eval{x}{\tau}]$. Since $y\notin\Fv(Q)$, we have $q\le Q(\tau)=Q[y/x](\tau')$. It follows that $\dJudge{P[y/x]}{C[y/x]}{Q[y/x]}$.
  \end{itemize}
\end{proof}

\begin{lemma}\label{lem:loop-unroll}
  For all $C,\sigma,p,l,\tau,q$, if $\bigstep{\Loop{C}}{\sigma}{p}{l}{\tau}{q}$, then there exists an $n\ge 0$ such that $\bigstep{C^n}{\sigma}{p}{l}{\tau}{q}$.
\end{lemma}

\begin{proof}
  By induction on the derivation of $\bigstep{\Loop{C}}{\sigma}{p}{l}{\tau}{q}$.
  \begin{itemize}
    \item \textbf{Case} \textsc{(BS:LoopZero)}:\\
      In this case, we have $\sigma=\tau$ and $p=l=q$, so we can take $n=0$.
    \item \textbf{Case} \textsc{(BS:Loop)}:\\
      Suppose $\bigstep{\Seq{C}{\Loop{C}}}{\sigma}{p}{l}{\tau}{q}$, then there exist $\rho,r,l_1,l_2$ such that $\bigstep{C}{\sigma}{p}{l_1}{\rho}{r}$, $\bigstep{\Loop{C}}{\rho}{r}{l_2}{\tau}{q}$, and $l=\min\{l_1,l_2\}$. By induction hypothesis, there exists an $n'\ge 0$ such that $\bigstep{C^n}{\rho}{r}{l_2}{\tau}{q}$. Taking $n=n'+1$, we have $\bigstep{C^n}{\sigma}{p}{l}{\tau}{q}$.
  \end{itemize}
\end{proof}

%\begin{lemma}\label{lem:local}
%  For all $C,x,y$ such that $x\ne y$ and $y\notin\Fv(C)$, $\Local{x}{C}$ and $\Local{y}{C[y/x]}$ are semantically equivalent. That is, for all $\sigma,p,\tau,q,l$, $\bigstep{\Local{x}{C}}{\sigma}{p}{l}{\tau}{q}$ if and only if $\bigstep{\Local{y}{C[y/x]}}{\sigma}{p}{l}{\tau}{q}$.
%\end{lemma}
%
%\begin{proof}
%  We only need to prove the forward direction since $C=C[y/x][x/y]$. Let $\sigma,p,\tau,q,l$ be arbitrary such that $\bigstep{\Local{x}{C}}{\sigma}{p}{l}{\tau}{q}$. Then from the semantics of $\Local{x}{C}$, there exists $\sigma',\tau',v$ such that $\sigma=\sigma'[x\mapsto v]$, $\tau=\tau'[x\mapsto v]$, and $\bigstep{C}{\sigma'}{p}{l}{\tau'}{q}$. By proposition \ref{prop:porg-const}, we have $\bigstep{C}{\sigma'[y\mapsto v]}{p}{l}{\tau'[y\mapsto v]}{q}$, and consequently we have $\bigstep{C[y/x]}{\sigma'[x\mapsto v][y\mapsto\eval{x}{\sigma'}]}{p}{l}{\tau'[x\mapsto v][y\mapsto\eval{x}{\tau'}]}{q}$ by proposition \ref{prop:subst}. Using \textsc{(BS:Local)} and instantiating $v=\eval{y}{\sigma'}=\eval{y}{\tau'}$, we have $\bigstep{\Local{y}{C[y/x]}}{\sigma}{p}{l}{\tau}{q}$.
%\end{proof}

\begin{theorem}[QFUA Completeness]\label{thm:qfua-complete}
  For all $P,C,Q$, if $\fJudge{P}{C}{Q}$ holds, then $\fjudge{P}{C}{Q}$ is derivable.
\end{theorem}

\begin{proof}
  By induction on the structure of $C$.
  \begin{itemize}
    \item \textbf{Case} $\Skip$:\\
      Let $P,Q$ be arbitrary such that $\fJudge{P}{\Skip}{Q}$. From the semantics of the QFUA triple and $\Skip$, we have $P\preceq Q$. Then we can derive $\fjudge{P}{\Skip}{Q}$ using \textsc{(F:Skip)} and \textsc{(F:Cons)}.
    \item \textbf{Case} $\Assign{x}{e}$:\\
      Let $P,Q$ be arbitrary such that $\fJudge{P}{\Assign{x}{e}}{Q}$. From the semantics of the QFUA triple and $\Assign{x}{e}$, we have that for any $\tau$ and $q$ satisfying $q\ge Q(\tau)$, there exists a $\sigma$ such that $q\ge P(\sigma)$ and $\tau=\sigma[x\mapsto\eval{e}{\sigma}]$. Taking $x'=\eval{x}{\sigma}$, we have $q\ge P(\tau[x\mapsto x'])$ and $\eval{x}{\tau}=\eval{e}{\sigma}=\eval{e[x'/x]}{\tau}$. This implies that $q\ge Q'(\tau)$, where $Q'(\tau)=\Inf x'.P[x'/x]\curlyvee[x\ne e[x'/x]]$. Since $q$ is arbitrary, we have $Q'\preceq Q$, and consequently $\fjudge{P}{\Assign{x}{e}}{Q}$ is derivable by \textsc{(F:Assign)} and \textsc{(F:Cons)}.
    \item \textbf{Case} $\Assume{B}$:\\
      Let $P,Q$ be arbitrary such that $\fJudge{P}{\Assume{B}}{Q}$. From the semantics of the QFUA triple and $\Assume{B}$, we have $P\preceq P\curlyvee[\neg B]\preceq Q$. Then we can derive $\fjudge{P}{\Assume{B}}{Q}$ using \textsc{(F:Assume)} and \textsc{(F:Cons)}.
    \item \textbf{Case} $\Tick{e}$:\\
      Let $P,Q$ be arbitrary such that $\fJudge{P}{\Tick{e}}{Q}$. From the semantics of the QFUA triple and $\Tick{e}$, we have $P-e\preceq Q$. Then we can derive $\fjudge{P}{\Tick{e}}{Q}$ using \textsc{(F:Tick)} and \textsc{(F:Cons)}.
    \item \textbf{Case} $\Seq{C_1}{C_2}$:\\
      Let $P,Q$ be arbitrary such that $\fJudge{P}{\Seq{C_1}{C_2}}{Q}$. From the semantics of the QFUA triple and $\Seq{C_1}{C_2}$, we have that for any $\tau$ and $q$ satisfying $q\ge Q(\tau)$, there exists $\sigma,p,\rho,r$ such that $p\ge P(\sigma)$, $\bigstepp{C_1}{\sigma}{p}{\rho}{r}$, and $\bigstepp{C_2}{\rho}{r}{\tau}{q}$. We take $R=\post{C_1}{P}$, then $\fJudge{P}{C_1}{R}$ and $\fJudge{R}{C_2}{Q}$ holds. By induction hypothesis, we have $\fjudge{P}{C_1}{R}$ and $\fjudge{R}{C_2}{Q}$. Thus, we can derive $\fjudge{P}{\Seq{C_1}{C_2}}{Q}$ using \textsc{(F:Seq)}.
    \item \textbf{Case} $\Choice{C_1}{C_2}$:\\
      Let $P,Q$ be arbitrary such that $\fJudge{P}{\Choice{C_1}{C_2}}{Q}$. From the semantics of the QFUA triple and $\Choice{C_1}{C_2}$, we have that for any $\tau$ and $q$ satisfying $q\ge Q(\tau)$, there exists a $\sigma$ and a $p$ such that $p\ge P(\sigma)$ and either $\bigstepp{C_1}{\sigma}{p}{\tau}{q}$ or $\bigstepp{C_2}{\sigma}{p}{\tau}{q}$. So we can divide $Q$ into $Q=Q_1\curlywedge Q_2$, where $Q_i(\tau)=\inf\{q:q\ge Q(\tau)\land\exists\sigma,p.p\ge P(\sigma)\land\bigstepp{C_i}{\sigma}{p}{\tau}{q}\}$ for $i\in\{1,2\}$. By definition, $\fJudge{P}{C_i}{Q_i}$ holds for $i\in\{1,2\}$, which implies that $\fjudge{P}{C_i}{Q_i}$ using induction hypothesis. Then we can derive $\fjudge{P}{\Choice{C_1}{C_2}}{Q}$ using \textsc{(F:Choice)} and \textsc{(F:Disj)}.
    \item \textbf{Case} $\Loop{C}$:\\
      Let $P,Q$ be arbitrary such that $\fJudge{P}{\Loop{C}}{Q}$. From the semantics of the QFUA triple and by lemma \ref{lem:loop-unroll}, we have that for any $\tau$ and $q$ satisfying $q\ge Q(\tau)$, there exists $\sigma,p,n$ such that $p\ge P(\sigma)$ and $\bigstepp{C^n}{\sigma}{p}{\tau}{q}$. So we can divide $Q$ into $Q=\bigcurlywedge_{n\ge 0}Q_n$, where $Q_n(\tau)=\inf\{q:q\ge Q(\tau)\land\exists\sigma,p.p\ge P(\sigma)\land\bigstepp{C^n}{\sigma}{p}{\tau}{q}\}$. In this case, $\fJudge{P}{C^n}{Q_n}$ holds for all $n\ge 0$.\\
      For each $n\ge 0$, let $R_n(i)=\post{C^i}{P}$ for $0\le i<n$ and $R_n(n)=Q_n$.\footnote{By the associativity of sequential composition, $C^i$ can be read as left-associative.} Thus, we have $\fJudge{R_n(i)}C{R_n(i+1)}$ for all $0\le i<n-1$ by definition and $\fJudge{R_n(n-1)}C{R_n(n)}$ by $\fJudge{P}{C^n}{Q_n}$. From induction hypothesis, we have $\fjudge{R_n(i)}C{R_n(i+1)}$ for all $0\le i<n$. Then we can derive $\fjudge{P}{\Loop{C}}{Q_n}$ using \textsc{(F:Loop)}. Finally, we can derive $\fjudge{P}{\Loop{C}}{Q}$ using \textsc{(F:Disj)}.
    \item \textbf{Case} $\Local{x}{C}$:\\
      Let $P,Q$ be arbitrary such that $\fJudge{P}{\Local{x}{C}}{Q}$. Pick $y$ be a fresh variable such that $y\notin\Fv(P)\cup\Fv(Q)\cup\Fv(C)$, and let $Q'=\post{C[y/x]}{P}$. We have $\fJudge{P}{C[y/x]}{Q'}$.
      By induction hypothesis,\footnote{Since $C[y/x]$ is not a subterm of $\Local{x}{C}$, we should use a stronger induction hypothesis: $\fJudge{P}{C}{Q}[\overrightarrow{y}/\overrightarrow{x}]$ implies $\fjudge{P}{C}{Q}[\overrightarrow{y}/\overrightarrow{x}]$, same as incorrectness logic~\cite{POPL:OHearn20}. Similar strengthen applies to the proof of QBUA and \QBUAd completeness.} we have $\fjudge{P}{C[y/x]}{Q'}$, so we can derive $\fjudge{\Inf y.P}{\Local{x}{C}}{\Inf y.Q'}$ using \textsc{(F:Local)}.\\
      Next, we will prove that $\Inf y.Q'\preceq Q$. For any $\tau$ and $q$ satisfying $q\ge Q(\tau)$, since $\fJudge{P}{\Local{x}{C}}{Q}$ holds, there exists a $\sigma$ and a $p$ such that $p\ge P(\sigma)$ and $\bigstepp{\Local{x}{C}}{\sigma}{p}{\tau}{q}$. From the semantics of $\Local{x}{C}$, there exists $\sigma',\tau',v$ such that $\sigma=\sigma'[x\mapsto v]$, $\tau=\tau'[x\mapsto v]$, and $\bigstepp{C}{\sigma'}{p}{\tau'}{q}$. Then we have $\bigstepp{C[y/x]}{\sigma''}{p}{\tau''}{q}$ by proposition \ref{prop:subst}, where $\sigma''=\sigma[y\mapsto\eval{x}{\sigma'}]$ and $\tau''=\tau[y\mapsto\eval{x}{\tau'}]$. Since $y$ is fresh, we have $P(\sigma'')=P(\sigma)$, then we have $q\ge Q'(\tau'')$ from $\bigstepp{C[y/x]}{\sigma''}{p}{\tau''}{q}$. This implies that $q\ge\Inf y.Q'(\tau)$, and consequently $\Inf y.Q'\preceq Q$.\\
      Finally, since $\Inf y.P=P$ by the freshness of $y$, we can derive $\fjudge{P}{\Local{x}{C}}{Q}$ using \textsc{(F:Cons)}.
  \end{itemize}
\end{proof}

\begin{theorem}[QBUA Completeness]\label{thm:qbua-complete}
  For all $P,C,Q$, if $\bJudge{P}{C}{Q}$ holds, then $\bjudge{P}{C}{Q}$ is derivable.
\end{theorem}

\begin{proof}
  By induction on the structure of $C$.
  \begin{itemize}
    \item \textbf{Case} $\Skip$:\\
      Let $P,Q$ be arbitrary such that $\bJudge{P}{\Skip}{Q}$. From the semantics of the QBUA triple and $\Skip$, we have $P\preceq Q$. Then we can derive $\bjudge{P}{\Skip}{Q}$ using \textsc{(B:Skip)} and \textsc{(B:Cons)}.
    \item \textbf{Case} $\Assign{x}{e}$:\\
      Let $P,Q$ be arbitrary such that $\bJudge{P}{\Assign{x}{e}}{Q}$. Denote $Q'=\Sup x'.P[x'/x]\curlywedge[x=e[x'/x]]$. We will prove that $Q'\preceq Q$. For seek of contradiction, suppose there exists $\tau$ such that $Q'(\tau)>Q(\tau)$. Then there exists $v$ such that $\eval{x}{\tau}=\eval{e}{\tau[x\mapsto v]}$ and $P(\tau[x\mapsto v])>Q(\tau)$. Therefore, take $\sigma=\tau[x\mapsto v]$ and $p=P(\sigma)$. From the semantics of $\bJudge{P}{\Assign{x}{e}}{Q}$, we have $p\le Q(\tau)$, which leads to a contradiction. Thus, we have $Q'\preceq Q$, and consequently $\bjudge{P}{\Assign{x}{e}}{Q}$ is derivable by \textsc{(B:Assign)} and \textsc{(B:Cons)}.
    \item \textbf{Case} $\Assume{B}$:\\
      Let $P,Q$ be arbitrary such that $\bJudge{P}{\Assume{B}}{Q}$. From the semantics of the QBUA triple and $\Assume{B}$, we have $P=P\curlywedge[B]\preceq Q$. Then we can derive $\bjudge{P}{\Assume{B}}{Q}$ using \textsc{(B:Assume)} and \textsc{(B:Cons)}.
    \item \textbf{Case} $\Tick{e}$:\\
      Let $P,Q$ be arbitrary such that $\bJudge{P}{\Tick{e}}{Q}$. From the semantics of the QBUA triple and $\Tick{e}$, we have $P-e\preceq Q$. Then we can derive $\bjudge{P}{\Tick{e}}{Q}$ using \textsc{(B:Tick)} and \textsc{(B:Cons)}.
    \item \textbf{Case} $\Seq{C_1}{C_2}$:\\
      Let $P,Q$ be arbitrary such that $\bJudge{P}{\Seq{C_1}{C_2}}{Q}$. From the semantics of the QBUA triple and $\Seq{C_1}{C_2}$, we have that for any $\sigma$ and $p$ satisfying $p\le P(\sigma)$, there exists $\tau,q,\rho,r$ such that $q\le Q(\tau)$, $\bigstepp{C_1}{\sigma}{p}{\rho}{r}$, and $\bigstepp{C_2}{\rho}{r}{\tau}{q}$. We take $R=\pre{C_2}{Q}$, then $\bJudge{P}{C_1}{R}$ and $\bJudge{R}{C_2}{Q}$ holds. By induction hypothesis, we have $\bjudge{P}{C_1}{R}$ and $\bjudge{R}{C_2}{Q}$. Thus, we can derive $\bjudge{P}{\Seq{C_1}{C_2}}{Q}$ using \textsc{(B:Seq)}.
    \item \textbf{Case} $\Choice{C_1}{C_2}$:\\
      Let $P,Q$ be arbitrary such that $\bJudge{P}{\Choice{C_1}{C_2}}{Q}$. From the semantics of the QBUA triple and $\Choice{C_1}{C_2}$, we have that for any $\sigma$ and $p$ satisfying $p\le P(\sigma)$, there exists a $\tau$ and a $q$ such that $q\le Q(\tau)$ and either $\bigstepp{C_1}{\sigma}{p}{\tau}{q}$ or $\bigstepp{C_2}{\sigma}{p}{\tau}{q}$. So we can divide $P$ into $P=P_1\curlyvee P_2$, where $P_i(\sigma)=\sup\{p:p\le P(\sigma)\land\exists\tau,q.q\le Q(\tau)\land\bigstepp{C_i}{\sigma}{p}{\tau}{q}\}$ for $i\in\{1,2\}$. By definition, $\bJudge{P_i}{C_i}{Q}$ holds for $i\in\{1,2\}$, which implies that $\bjudge{P_i}{C_i}{Q}$ using induction hypothesis. Then we can derive $\bjudge{P}{\Choice{C_1}{C_2}}{Q}$ using \textsc{(B:Choice)} and \textsc{(B:Disj)}.
    \item \textbf{Case} $\Loop{C}$:\\
      Let $P,Q$ be arbitrary such that $\bJudge{P}{\Loop{C}}{Q}$. From the semantics of the QBUA triple and by lemma \ref{lem:loop-unroll}, we have that for any $\sigma$ and $p$ satisfying $p\le P(\sigma)$, there exists $\tau,q,n$ such that $q\le Q(\tau)$ and $\bigstepp{C^n}{\sigma}{p}{\tau}{q}$. So we can divide $P$ into $P=\bigcurlyvee_{n\ge 0}P_n$, where $P_n(\sigma)=\sup\{p:p\le P(\sigma)\land\exists\tau,q.q\le Q(\tau)\land\bigstepp{C^n}{\sigma}{p}{\tau}{q}\}$. In this case, $\bJudge{P_n}{C^n}{Q}$ holds for all $n\ge 0$.\\
      For each $n\ge 0$, let $R_n(i)=\pre{C^{n-i}}{Q}$ for $0<i\le n$ and $R_n(0)=P_n$. Thus, we have $\bJudge{R_n(i-1)}C{R_n(i)}$ for all $1<i\le n$ by definition and $\bJudge{R_n(0)}C{R_n(1)}$ by $\bJudge{P_n}{C^n}{Q}$. From induction hypothesis, we have $\bjudge{R_n(i-1)}C{R_n(i)}$ for all $0<i\le n$. Then we can derive $\bjudge{P_n}{\Loop{C}}{Q}$ using \textsc{(B:Loop)}. Finally, we can derive $\bjudge{P}{\Loop{C}}{Q}$ using \textsc{(B:Disj)}.
    \item \textbf{Case} $\Local{x}{C}$:\\
      Let $P,Q$ be arbitrary such that $\bJudge{P}{\Local{x}{C}}{Q}$. Pick $y$ be a fresh variable such that $y\notin\Fv(P)\cup\Fv(Q)\cup\Fv(C)$, and let $P'=\pre{C[y/x]}{Q}$. We have $\bJudge{P'}{C[y/x]}{Q}$. By induction hypothesis, we have $\bjudge{P'}{C[y/x]}{Q}$, so we can derive $\bjudge{\Sup y.P'}{\Local{x}{C}}{\Sup y.Q}$ using \textsc{(B:Local)}.\\
      Next, we will prove that $P\preceq\Sup y.P'$. For any $\sigma$ and $p$ satisfying $p\le P(\sigma)$, since $\bJudge{P}{\Local{x}{C}}{Q}$ holds, there exists a $\tau$ and a $q$ such that $q\le Q(\tau)$ and $\bigstepp{\Local{x}{C}}{\sigma}{p}{\tau}{q}$. From the semantics of $\Local{x}{C}$, there exists $\sigma',\tau',v$ such that $\sigma=\sigma'[x\mapsto v]$, $\tau=\tau'[x\mapsto v]$, and $\bigstepp{C}{\sigma'}{p}{\tau'}{q}$. Then we have $\bigstepp{C[y/x]}{\sigma''}{p}{\tau''}{q}$ by proposition \ref{prop:subst}, where $\sigma''=\sigma[y\mapsto\eval{x}{\sigma'}]$ and $\tau''=\tau[y\mapsto\eval{x}{\tau'}]$. Since $y$ is fresh, we have $Q(\tau'')=Q(\tau)$, then we have $p\le P'(\sigma'')$ from $\bigstepp{C[y/x]}{\sigma''}{p}{\tau''}{q}$. This implies that $p\le\Sup y.P'(\sigma)$, and consequently $P\preceq\Sup y.P'$.\\
      Finally, since $\Sup y.Q=Q$ by the freshness of $y$, we can derive $\bjudge{P}{\Local{x}{C}}{Q}$ using \textsc{(B:Cons)}.
  \end{itemize}
\end{proof}

\begin{theorem}[\QBUAd Completeness]\label{thm:qbuad-complete}
  For all $P,C,Q$, if $\dJudge{P}{C}{Q}$ holds, then $\djudge{P}{C}{Q}$ is derivable.
\end{theorem}

\begin{proof}
  By induction on the structure of $C$.
  \begin{itemize}
    \item \textbf{Case} $\Skip$:\\
      Let $P,Q$ be arbitrary such that $\dJudge{P}{\Skip}{Q}$. From the semantics of the \QBUAd triple and $\Skip$, we have $P\preceq Q$ and $P\preceq 0$. Then we can derive $\djudge{P}{\Skip}{Q}$ using \textsc{(\Bd:Skip)} and \textsc{(\Bd:Cons)}.
    \item \textbf{Case} $\Assign{x}{e}$:\\
      Let $P,Q$ be arbitrary such that $\dJudge{P}{\Assign{x}{e}}{Q}$. Denote $Q'=\Sup x'.P[x'/x]\curlywedge[x=e[x'/x]]$. We will prove that $Q'\preceq Q$. For seek of contradiction, suppose there exists $\tau$ such that $Q'(\tau)>Q(\tau)$. Then there exists $v$ such that $\eval{x}{\tau}=\eval{e}{\tau[x\mapsto v]}$ and $P(\tau[x\mapsto v])>Q(\tau)$. Therefore, take $\sigma=\tau[x\mapsto v]$ and $p=P(\sigma)$. From $\dJudge{P}{\Assign{x}{e}}{Q}$, we have $p\le Q(\tau)$, which leads to a contradiction. Thus, we have $Q'\preceq Q$. Besides, we also have $P\preceq 0$ from $\dJudge{P}{\Assign{x}{e}}{Q}$, so $\djudge{P}{\Assign{x}{e}}{Q}$ is derivable by \textsc{(\Bd:Assign)} and \textsc{(\Bd:Cons)}.
    \item \textbf{Case} $\Assume{B}$:\\
      Let $P,Q$ be arbitrary such that $\dJudge{P}{\Assume{B}}{Q}$. From the semantics of the \QBUAd triple and $\Assume{B}$, we have $P=P\curlywedge[B]\preceq Q$ and $P\preceq 0$. Then we can derive $\djudge{P}{\Assume{B}}{Q}$ using \textsc{(\Bd:Assume)} and \textsc{(\Bd:Cons)}.
    \item \textbf{Case} $\Tick{e}$:\\
      Let $P,Q$ be arbitrary such that $\dJudge{P}{\Tick{e}}{Q}$. From the semantics of the \QBUAd triple and $\Tick{e}$, we have $P-e\preceq Q$ and $P\curlywedge P-e\preceq 0$. Then we can derive $\djudge{P}{\Tick{e}}{Q}$ using \textsc{(\Bd:Tick)} and \textsc{(\Bd:Cons)}.
    \item \textbf{Case} $\Seq{C_1}{C_2}$:\\
      Let $P,Q$ be arbitrary such that $\dJudge{P}{\Seq{C_1}{C_2}}{Q}$. From the semantics of the \QBUAd triple and $\Seq{C_1}{C_2}$, we have that for any $\sigma$ and $p$ satisfying $p\le P(\sigma)$, there exists $\tau,q,\rho,r$ such that one of the following holds:
      \begin{itemize}
        \item[(1)] $q\le Q(\tau)$, $\bigstepl{C_1}{\sigma}{p}{\rho}{r}$, and $\bigstepp{C_2}{\rho}{r}{\tau}{q}$;
        \item[(2)] $q\le Q(\tau)$, $\bigstepp{C_1}{\sigma}{p}{\rho}{r}$, and $\bigstepl{C_2}{\rho}{r}{\tau}{q}$.
      \end{itemize}
      So we can divide $P$ into $P=P_1\curlyvee P_2$, where $P_i(\sigma)=\sup\{p:p\le P(\sigma)\land\text{ there exists }\tau,q,\rho,r\text{ such that (}i\text{) holds}\}$ for $i\in\{1,2\}$.
      \begin{itemize}
        \item For the case (1), let $R_1=\pre{C_2}{Q}$, then $\dJudge{P_1}{C_1}{R_1}$ and $\bJudge{R_1}{C_2}{Q}$ holds. By induction hypothesis and theorem \ref{thm:qbua-complete}, we have $\djudge{P_1}{C_1}{R_1}$ and $\bjudge{R_1}{C_2}{Q}$. Thus, we can derive $\djudge{P_1}{\Seq{C_1}{C_2}}{Q}$ using \textsc{(\Bd:SeqL)}.
        \item For the case (2), let $R_2=\prel{C_2}{Q}$, then $\bJudge{P_2}{C_1}{R_2}$ and $\dJudge{R_2}{C_2}{Q}$ holds. By theorem \ref{thm:qbua-complete} and induction hypothesis, we have $\bjudge{P_2}{C_1}{R_2}$ and $\djudge{R_2}{C_2}{Q}$. Thus, we can derive $\djudge{P_2}{\Seq{C_1}{C_2}}{Q}$ using \textsc{(\Bd:SeqR)}.
      \end{itemize}
      Finally, we can derive $\djudge{P}{\Seq{C_1}{C_2}}{Q}$ using \textsc{(\Bd:Disj)}.
    \item \textbf{Case} $\Choice{C_1}{C_2}$:\\
      Let $P,Q$ be arbitrary such that $\dJudge{P}{\Choice{C_1}{C_2}}{Q}$. From the semantics of the \QBUAd triple and $\Choice{C_1}{C_2}$, we have that for any $\sigma$ and $p$ satisfying $p\le P(\sigma)$, there exists a $\tau$ and a $q$ such that $q\le Q(\tau)$ and either $\bigstepl{C_1}{\sigma}{p}{\tau}{q}$ or $\bigstepl{C_2}{\sigma}{p}{\tau}{q}$. So we can divide $P$ into $P=P_1\curlyvee P_2$, where $P_i(\sigma)=\sup\{p:p\le P(\sigma)\land\exists\tau,q.q\le Q(\tau)\land\bigstepl{C_i}{\sigma}{p}{\tau}{q}\}$ for $i\in\{1,2\}$. By definition, $\dJudge{P_i}{C_i}{Q}$ holds for $i\in\{1,2\}$, which implies that $\djudge{P_i}{C_i}{Q}$ using induction hypothesis. Then we can derive $\djudge{P}{\Choice{C_1}{C_2}}{Q}$ using \textsc{(\Bd:Choice)} and \textsc{(\Bd:Disj)}.
    \item \textbf{Case} $\Loop{C}$:\\
      Let $P,Q$ be arbitrary such that $\dJudge{P}{\Loop{C}}{Q}$. From the semantics of the \QBUAd triple and $\Loop{C}$, we have that for any $\sigma$ and $p$ satisfying $p\le P(\sigma)$, there exists $\tau,q,n$ such that $q\le Q(\tau)$ and $\bigstepl{C^n}{\sigma}{p}{\tau}{q}$. So we can divide $P$ into $P=\bigcurlyvee_{n\ge 0}P_n$, where $P_n(\sigma)=\sup\{p:p\le P(\sigma)\land\exists\tau,q.q\le Q(\tau)\land\bigstepl{C^n}{\sigma}{p}{\tau}{q}\}$. In this case, $\dJudge{P_n}{C^n}{Q}$ holds for all $n\ge 0$.\\
      For $n=0$, we have $P_0\preceq 0$ and then $\djudge{P_0}{\Loop{C}}{Q}$ is derivable by \textsc{(\Bd:LoopZero)} and \textsc{(\Bd:Cons)}.
      For each $n>0$, similar to the sequential composition, we can divide $P_n$ into $P_n=\bigcurlyvee_{0\le m<n}P_{n,m}$, where $P_{n,m}(\sigma)=\sup\{p:p\le P_n(\sigma)\land\exists\tau,q,\rho_1,r_1,\rho_2,r_2.q\le Q(\tau)\land\bigstepp{C^m}{\sigma}{p}{\rho_1}{r_1}\land\bigstepl{C}{\rho_1}{r_1}{\rho_2}{r_2}\land\bigstepp{C^{n-m-1}}{\rho_2}{r_2}{\tau}{q}\}$. For each $0\le m<n$, let
      \begin{align*}
        R_{n,m}(i)=\begin{cases}
          P_{n,m} & \text{if }i=0,\\
          \pre{C^{m-i}}{\prel{C}{\pre{C^{n-m-1}}{Q}}} & \text{if }0<i\le m,\\
          \pre{C^{n-i}}{Q} & \text{if }m<i\le n.\\
        \end{cases}
      \end{align*}
      Then we have $\bJudge{R_{n,m}(i)}{C}{R_{n,m}{i+1}}$ for all $i<n$ and $\dJudge{R_{n,m}(m)}{C}{R_{n,m}(m+1)}$. By theorem \ref{thm:qbua-complete} and induction hypothesis, we have $\bjudge{R_{n,m}(i)}{C}{R_{n,m}(i+1)}$ for all $i<n$ and $\djudge{R_{n,m}(m)}{C}{R_{n,m}(m+1)}$. Thus, we can derive $\djudge{P_{n,m}}{C^m}{Q}$ using \textsc{(\Bd:Loop)}, and then derive $\djudge{P_n}{C^n}{Q}$ using \textsc{(\Bd:Disj)}.\\
      Finally, we can derive $\djudge{P}{\Loop{C}}{Q}$ using \textsc{(\Bd:Disj)}.
    \item \textbf{Case} $\Local{x}{C}$:\\
      Let $P,Q$ be arbitrary such that $\dJudge{P}{\Local{x}{C}}{Q}$. Pick $y$ be a fresh variable such that $y\notin\Fv(P)\cup\Fv(Q)\cup\Fv(C)$, and let $P'=\prel{C[y/x]}{Q}$. We have $\dJudge{P'}{C[y/x]}{Q}$. By induction hypothesis, we have $\djudge{P'}{C[y/x]}{Q}$, so we can derive $\djudge{\Sup y.P'}{\Local{x}{C}}{\Sup y.Q}$ using \textsc{(\Bd:Local)}.\\
      Next, we will prove that $P\preceq\Sup y.P'$. For any $\sigma$ and $p$ satisfying $p\le P(\sigma)$, since $\dJudge{P}{\Local{x}{C}}{Q}$ holds, there exists a $\tau$ and a $q$ such that $q\le Q(\tau)$ and $\bigstepl{\Local{x}{C}}{\sigma}{p}{\tau}{q}$. From the semantics of $\Local{x}{C}$, there exists $\sigma',\tau',v$ such that $\sigma=\sigma'[x\mapsto v]$, $\tau=\tau'[x\mapsto v]$, and $\bigstepl{C}{\sigma'}{p}{\tau'}{q}$. Then we have $\bigstepl{C[y/x]}{\sigma''}{p}{\tau''}{q}$ by proposition \ref{prop:subst}, where $\sigma''=\sigma[y\mapsto\eval{x}{\sigma'}]$ and $\tau''=\tau[y\mapsto\eval{x}{\tau'}]$. Since $y$ is fresh, we have $Q(\tau'')=Q(\tau)$, then we have $p\le P'(\sigma'')$ from $\bigstepl{C[y/x]}{\sigma''}{p}{\tau''}{q}$. This implies that $p\le\Sup y.P'(\sigma)$, and consequently $P\preceq\Sup y.P'$.\\
      Finally, since $\Sup y.Q=Q$ by the freshness of $y$, we can derive $\djudge{P}{\Local{x}{C}}{Q}$ using \textsc{(\Bd:Cons)}.
  \end{itemize}
\end{proof}

%We introduce the following derived triples to simplify the presentation of our case studies.
%\begin{mathpar}
%  \ffjudge{P_S}{P_R}{C}{Q_S}{Q_R}\quad\text{iff}\quad\fjudge{[\neg P_S]\curlyvee P_R}{C}{[\neg Q_S]\curlyvee Q_R}
%  \hva\and
%  \bbjudge{P_S}{P_R}{C}{Q_S}{Q_R}\quad\text{iff}\quad\bjudge{[P_S]\curlywedge P_R}{C}{[Q_S]\curlywedge Q_R}
%%  \\
%%  \Blue{\SR{Spec}{Res}}\coloneqq[\neg Spec]\curlyvee Res
%%  \hva\and
%%  \Red{\SR{Spec}{Res}}\coloneqq[Spec]\curlywedge Res
%\end{mathpar}
%%
%\cref{fig:fullderived} shows the complete proof rules for the derived triples.
%
%\begin{figure}
%\begin{mathpar}
%  \inferrule[(Skip)]{}{\aajudge{P_S}{P_R}{\Skip}{P_S}{P_R}}
%  \hva\and
%  \inferrule[(Assign)]{x\notin\Fv(P_R)}{\aajudge{P_S}{P_R}{\Assign{x}{e}}{\exists x'.P_S[x'/x]\land x=e[x'/x]}{P_R}}
%  \hva\and
%  \inferrule[(Assume)]{}{\aajudge{P_S\land B}{P_R}{\Assume{B}}{P_S\land B}{P_R}}
%  \hva\and
%  \inferrule[(Tick)]{}{\aajudge{P_S}{P_R}{\Tick{e}}{P_S}{P_R-e}}
%  \hva\and
%  \inferrule[(Seq)]{\aajudge{P_S}{P_R}{C_1}{Q_S}{Q_R}\\\aajudge{R_S}{R_R}{C_2}{Q_S}{Q_R}}{\aajudge{P_S}{P_R}{\Seq{C_1}{C_2}}{Q_S}{Q_R}}
%  \hva\and
%  \inferrule[(IfTrue)]{\aajudge{P_S\land B}{P_R}{C_1}{Q_S}{Q_R}}{\aajudge{P_S}{P_R}{\Ite{B}{C_1}{C_2}}{Q_S}{Q_R}}
%  \hva\and
%  \inferrule[(IfFalse)]{\aajudge{P_S\land\neg B}{P_R}{C_2}{Q_S}{Q_R}}{\aajudge{P_S}{P_R}{\Ite{B}{C_1}{C_2}}{Q_S}{Q_R}}
%  \hva\and
%  \inferrule[(WhileFalse)]{}{\aajudge{P_S\land\neg B}{P_R}{\While{B}{C}}{P_S\land\neg B}{P_R}}
%  \hva\and
%  \inferrule[(WhileSubvar)]{\forall n<k-1.\aajudge{P_S(n)\land B}{P_R(n)}{C}{P_S(n+1)\land B}{P_R(n+1)} \\ \aajudge{P_S(k-1)\land B}{P_R(k-1)}{C}{P_S(k)\land\neg B]}{P_S(k)}}{\aajudge{P_S(0)\land B}{P_R(0)}{\While{B}{C}}{P_S(k)\land\neg B}{P_R(k)}}
%  \hva\and
%  \inferrule[(Local)]{\aajudge{P_S}{P_R}{C}{Q_S}{Q_R}\\x\notin\Fv(P_S)\cup\Fv(Q_S)}{\aajudge{\exists x.P_S}{P_R}{\Local{x}{C}}{\exists x.Q_S}{Q_R}}
%  \hva\and
%  \inferrule[(Disj)]{\forall i\in I.\aajudge{P_S(i)}{P_R}{C}{Q_S(i)}{Q_R}}{\aajudge{\bigvee_{i\in I}P_S(i)}{P_R}{C}{\bigvee_{i\in I}Q_S(i)}{Q_R}}
%  \hva\and
%  \inferrule[(Constancy)]{\aajudge{P_S}{P_R}{C}{Q_S}{Q_R}\\\Fv(B)\cap\Mod(C)=\emptyset}{\aajudge{P_S\land B}{P_R}{C}{Q_S\land B}{Q_R}}
%  \hva\and
%  \inferrule[(Relax)]{\aajudge{P_S}{P_R}{C}{Q_S}{Q_R}\\\Fv(F)\cap\Mod(C)=\emptyset}{\aajudge{P_S}{P_R+F}{C}{Q_S}{Q_R+F}}
%  \hva\and
%  \inferrule[(ConsF)]{P_S'\subseteq P_S\\\ffjudge{P_S'}{P_R}{C}{Q_S'}{Q_R}\\Q_S\subseteq Q_S'}{\ffjudge{P_S}{P_R}{C}{Q_S}{Q_R}}
%  \hva\and
%  \inferrule[(ConsB)]{P_S\subseteq P_S'\\\bbjudge{P_S'}{P_R}{C}{Q_S'}{Q_R}\\Q_S'\subseteq Q_S}{\bbjudge{P_S}{P_R}{C}{Q_S}{Q_R}}
%  \hva\and
%  \inferrule[(ConsRes)]{P_R\preceq P_R'\\\aajudge{P_S}{P_R'}{C}{Q_S}{Q_R'}\\Q_R'\preceq Q_R}{\aajudge{P_S}{P_R}{C}{Q_S}{Q_R}}
%  \hva\and
%  \inferrule[(Subst)]{\aajudge{P_S}{P_R}{C}{Q_S}{Q_R}\\y\notin\Fv(P_S)\cup\Fv(P_R)\cup\Fv(C)\cup\Fv(Q_S)\cup\Fv(Q_R)}{\aajudge{P_S[y/x]}{P_R[y/x]}{C[y/x]}{Q_S[y/x]}{Q_R[y/x]}}
%\end{mathpar}
%\caption{Derived Rules}
%\label{fig:fullderived}
%\end{figure}

%\section{Example}
\label{app:example}

Table~\ref{tab:example} shows a counseling session between a client agent and CAMI, with the client's profile shown in Table~\ref{tab:example profile}. During the session, CAMI explores Health and Relationships topics in the first 6 turns.  It subsequently performs Step Into operation from Relationships to Parenting topics in Turn 7, and from Parenting to Paternal Bond topics in Turn 9, before a Switch operation to Habituation topic in Turn 11.  The full exploration path is visualized in Figure~\ref{fig:exploration path}.

\begin{table*}[ht]
\centering
\begin{tabularx}{\textwidth}{c X}
\toprule
        {\bf Turn ID} & {\bf Utterances}                     \\ \midrule
1       & Counselor: Hello. How are you?                                                                                                                                                                                                                                                                                                                     \\ \hline
2       & Client: I am good. What about you?                                                                                                                                                                                                                                                                                                                 \\ \hline
3       & Counselor: I'm doing well, thanks for asking! Can you share how you've been feeling about your health lately, especially in relation to smoking?                                                                                                                                                                                                   \\ \hline
4       & Client: I've been pretty busy with work and juggling family stuff. Sometimes I just think about how nice it is to take a break with my coworkers outside, chatting and during those moments, life feels a bit more relaxed. You know how it is—it’s nice to unwind.                                                                                \\ \hline
5       & Counselor: It sounds like smoking is a comforting routine for you amidst your busy life with work and your daughter. I wonder how it might impact the time you spend with your family.                                                                                                                                                             \\ \hline
6       & Client: I get what you mean, but I really don't see my smoking as an issue. It helps me manage stress, and I don’t think it affects my time with my daughter. We still have good moments together, and I wouldn't want to change that.                                                                                                             \\ \hline
7       & Counselor: It sounds like you find comfort in smoking to manage stress, and that feels crucial to your parenting now. I wonder, though, how you might feel if your daughter started to see smoking as a way to unwind, too. What impact would that have on your relationship with her?                                                             \\ \hline
8       & Client: That’s a tough thought. I remember starting young and how hard it was to quit later. I wouldn’t want that for her. It makes me reflect on what I’m modeling.                                                                                                   \\ \hline
9       & Counselor: It's great that you're reflecting on your modeling for your daughter. Perhaps seeing smoking as something that may weaken your bond could help reshape how you approach stress—considering healthier ways to connect can actually strengthen your relationship and set a positive example for her.                                      \\ \hline
10      & Client: I appreciate that perspective. I do want to set a strong example for her. My experience with my own family and their struggles makes me cautious. I see the patterns and don’t want her to feel that smoking is a go-to solution for stress, like I did.                                                                                   \\ \hline
11      & Counselor: It sounds like your family’s struggles have really shaped your perspective on smoking. You want to protect your daughter from those same patterns, knowing that smoking can not only impact her stress management but also her overall learning and adaptation to challenges. That’s a powerful motivation.                             \\ \hline
12      & Client: Exactly. I want her to learn healthier coping strategies instead of following in my footsteps. I realize that I need to make a change for both our sakes. It’s not just about me; it’s about creating an environment where she can thrive without those habits. It's time to find better ways to manage stress that we can share together.   
\\ \bottomrule
\end{tabularx}
\caption{Counseling Session between CAMI and a Simulated Client.}
\label{tab:example}
\end{table*}



\begin{table*}[ht]
\centering
\begin{tabularx}{\textwidth}{l X}
\toprule
{\bf Profile Item}        & {\bf Description}                     \\ \midrule
Behavioral Problem  & Smoking \\ \hline
State of Mind       & Precontemplation \\ \hline
Personas            & You smoked when you were young.", "You have a 12-year-old daughter. \newline The friends of the your daughter have tried smoking. \newline When you are at work, you all go out and smoke and chit chat and talk about things. \newline You like to have a cigarette after eating. \\ \hline
Motivation          & You are motivated because of the risk of smoking in habituation for your daughter, as you don't want your daughter to think smoking is good thing.    \\ \hline
Beliefs             & You feel smoking enjoyable and relaxing. \newline You would be left out if you would hate to not smoke. \newline You use smoking as a way to manage weight and control eating habits. \newline You have been smoking for a long time and feels it's almost impossible to quit. \\ \hline
Motivation Topics          & Interpersonal Relationships, Parenting, Habituation
\\ \bottomrule
\end{tabularx}
\caption{Client's Profile in the Example.}
\label{tab:example profile}
\end{table*}


\begin{figure*}
    \centering
    \includegraphics[width=\linewidth]{figs/example_explore_path.pdf}
    \caption{Topic Exploration Path by the Counselor in the Example.}
    \label{fig:exploration path}
\end{figure*}



\end{document}
