\section{Related Work}
\label{sec:related}

% 1.5 pages

\paragraph*{Incorrectness Logic}
%
Incorrectness logic (IL), introduced by O'Hearn____, signifies a major shift in program verification by emphasizing \emph{under-approximate} reasoning instead of the \emph{over-approximate} approach employed in classic Hoare logic.
%
The core idea of IL is the use of \emph{forward, under-approximate} (FUA) triples to ensure that any bug detected is a true positive.
%
Note that FUA triples were initially studied by de Vries and Koutavas____, and they actually discussed \emph{backward, under-approximate} (BUA) triples as \emph{total} Hoare triples.
%
BUA triples were later studied by M{\"o}ller et al.____ and Ascari et al.____.
%
As a subroutine of \textsc{UNTer} for under-approximating non-termination, Raad et al.____ studied the theory of BUA triples and realized them in practice.
%
Le et al.____ extends IL with separation logic (ISL) to find bugs in heap-manipulating programs.
%
Raad et al.____ further developed a concurrent variant of ISL to detect concurrency bugs, e.g., races and deadlocks.
%
Zilberstein et al.____ developed Outcome logic (OL) and Outcome separation logic (OSL) as a unified foundation for reasoning about both correctness and incorrectness.
%
Among the aforementioned studies, only \textsc{UNTer}____ focuses on the resource usage of programs, as it aims to prove non-termination, i.e., a \emph{qualitative} justification for a program's usage of the time resource.
%
Recently, Cousot____ studies transformational program logics for correctness and incorrectness, including termination and non-termination.
%
In this work, we aim to establish \emph{quantitative} under-approximations of the worst-case resource usage of programs. 

\paragraph*{Resource-aware Program Logic}
%
Haslbeck and Nipkow____ reviewed three variants of Hoare logic for establishing time bounds.
%
The first one is the quantitative Hoare logic (QHL)____ that we reviewed in \cref{sec:overview:qhl}.
%
Note that QHL is not restricted to time bounds; it can also handle general resources like stack bounds.
%
QHL generalizes Boolean assertions by introducing the concept of \emph{potentials} to reason about resource consumption.
%
In QHL, the difference between the pre- and the post-potential provides an upper bound (i.e., an over-approximation) on the resource consumption.
%
Furthermore, because potentials can never be negative, the pre-potential also establishes an upper bound on the high-water mark for executing a program.
%
The second one is a Hoare-like logic to prove big-O-style time bounds, developed by Nielson____.
%
Nielson's logic establishes triples of the form $\left\{p\right\} C \left\{ E \Downarrow q \right\}$, where $p$ and $q$ are Boolean assertions and $E$ is a time bound.
%
The third one is a line of work of incorporating separation logic with potential-like notions____.
%
The high-level idea is to treat resources (or potentials) as a special type of heap fragment, allowing us to use \emph{separating conjunction} to annotate program states with quantitative information, e.g., $\left\{p \ast \$3\right\} C \left\{ q \ast \$1 \right\}$ indicates that the pre-state carries $3$ units of resources and the post-state carries $1$ unit of resources.
%
All the aforementioned logics focus on \emph{over}-approximating resource usage.
%
In this work, we aim to develop a program logic to \emph{under}-approximate worst-case resource usage.

Graded Hoare logic (GHL)____ augments Hoare logic with a \emph{preordered monoidal analysis} to reason about side-effects, including resource usage, probabilistic behavior, and differential privacy-related properties.
%
Given a preordered monoid $(M,{\le},1,{\cdot})$, GHL proves judgements of the form $\vdash_m \left\{ \phi \right\} C \left\{ \psi \right\}$, where $m \in M$ is called the \emph{analysis} of program $C$ and $\phi,\psi$ are the pre- and post-conditions.
%
For example, one can use the natural-number monoid $(\bbN,{\le},0,{+})$ to instantiate GHL to reason about resource usage, e.g., $\vdash_0 \left\{\phi\right\} \Skip \left\{\phi\right\}$ and $\vdash_2 \left\{\phi\right\} \Tick{2} \left\{\phi\right\}$.
%
GHL still follows the methodology of \emph{over-approximating}.
%
Extending our work in this paper with grading is an intriguing avenue for future research.

\paragraph*{Worst-case Resource Analysis}
%
Most existing approaches for identifying reachable worst cases are based on software testing, specifically using techniques such as random testing and symbolic execution to generate concrete inputs that can lead to significant resource usage. 
%
Recent fuzzing-based methods include \textsc{KelinciWCA}____, \textsc{SlowFuzz}____, and resource-usage-aware fuzzing____; they are considered black-box methods as they do not require knowledge of the program's content.
%
On the other hand, symbolic-execution-based methods look into the concrete structures of programs; to name a few, WISE____, SPF-WCA____, \textsc{Badger}____, and resource-type-guided symbolic execution____.
%
All the aforementioned methods can generate concrete inputs, i.e., they provide \emph{concrete} under-approximations of the worst-case resource usage.
%
In this work, we develop quantitative under-approximate logic to prove \emph{symbolic} under-approximations, e.g., sufficient conditions to achieve a resource consumption that is no less than a specified bound (see \cref{sec:case-studies}).
%
Incorporating testing and verification to under-approximate resource usage presents an interesting area for future work.