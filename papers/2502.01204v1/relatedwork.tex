\section{Related Works}
\label{sec:relatedworks}

Super-resolution of Land Surface Temperature images is commonly done with statistical models based on empirical relationships between LST and land features estimated from Visible and Near InfraRed (VNIR) satellite images. This approach was initially justified by the inverse relationship between Normalized Difference Vegetation Index (NDVI) and LST~\cite{Cai2018, Govil2019}. Later, several studies showed that VNIR indices are adapted for LST super-resolution~\cite{Kumar2015,Ferreira2019,GraneroBelinchon2019}. The most commonly used statistical models for LST super-resolution are based on simple linear regressions between the LST and VNIR features. Some of these models are Distrad, TsHARP, ATPRK or AATPRK~\cite{Essa2012,  Kustas2003, TsHARP2nd, Wang2015, Wang2016}. Granero-Belinchon et al. 2019 illustrated that these models are only slightly dependent on the VNIR feature used in the regression and that ATPRK (Area-To-Point Regression Kriging) and AATPRK (Adaptive-Area-To-Point-Regression-Kriging) with a Kriging interpolation for the residuals' definition slightly outperform TsHARP and DisTrad~\cite{GraneroBelinchon2019}. Recently, more complex models such as Data Mining Sharpener (DMS) using regression trees appeared~\cite{Gao2012, Xue2020, Guzinski2019}. All these models use relationships between the LST and VNIR features that are optimized at the LST coarse resolution and applied at the high resolution of the VNIR feature. So, these models use a reduced scale training approach.


In the recent years, Deep Learning has been widely used for super-resolution applications of remote sensing images in the VNIR, where a lot of different approaches have been developed: Convolutional Neural Networks (CNN)~\cite{Gargiulo2019, Lanaras2018}, Generative Adversarial Networks~\cite{Brodu2017} or diffusion models~\cite{Xiao2023} among others. However, Deep Learning has been only scarcely used for LST super-resolution~\cite{Nguyen2022,Chen2024,Choe2017}. A CNN has been used on images from MODIS~\cite{Nguyen2022} to increase the spatial resolution by a factor of four. Multilayer perceptrons have also been used to downscale MODIS LST to Landsat spatial resolution (by a factor of ten)~\cite{Choe2017} and more recently, \cite{Chen2024} proposed a diffusion model to downscale Landsat LST by factors of four and eight. However, in the field of remote sensing, a frequent difficulty in super-resolution is the absence of a high resolution reference image to be used in the training step. This is particularly true for LST super-resolution. So, in order to apply supervised learning schemes, models are commonly trained at reduced scale, \textit{i.e.} trained at degraded lower resolutions using the initial images as a reference to later apply these trained models at higher resolutions~\cite{Nguyen2022, Chen2024, Choe2017}.

All the aforementioned statistical and deep learning models use a scale-invariance hypothesis: models optimized at coarse resolution can be used at high resolution. This hypothesis can be problematic depending on the heterogeneity of the studied surface or the range of scales studied.