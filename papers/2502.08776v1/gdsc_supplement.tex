
\begin{table}
\centering
\caption{Empirical results on semi-synthetic drug-response data. $\pm$ denotes 95\% confindence intervals. \textbf{Bolded} results on FDR column indicate that the method(s) achieved valid FDR (up to 95\% confidence intervals). \textbf{Bolded} results on valid power column indicate that the method(s) achieved the highest valid power for that setting (up to 95\% confidence intervals).}
\begin{tabular}{||c|c|c|c|c|c|c||}
\hline
& \multicolumn{3}{|c|}{FDR} & \multicolumn{3}{|c|}{Valid power} \\
\hline
Method & $\alpha=0.05$ & $\alpha=0.1$ & $\alpha=0.25$ & $\alpha=0.05$ & $\alpha=0.1$ & $\alpha=0.25$ \\
\hline
Frequentist & \textbf{ 0.03±0.006 } & \textbf{ 0.04±0.007 } & \textbf{ 0.19±0.038 } & 0.24±0.018 & 0.33±0.021 & \textbf{ 0.63±0.062 } \\
Add-C2G & 0.14±0.035 & 0.17±0.034 & \textbf{ 0.23±0.032 } & 0.0±0.0 & 0.0±0.0 & 0.42±0.06 \\
NP-C2G & \textbf{ 0.04±0.009 } & \textbf{ 0.07±0.011 } & \textbf{ 0.23±0.02 } & \textbf{ 0.33±0.025 } & \textbf{ 0.44±0.023 } & \textbf{ 0.65±0.021 } \\
\hline
\end{tabular}
\label{table:nutlin-simulations}
\end{table}
The Genomics of Drug Sensitivity in Cancer (GDSC) is a repository of cancer cell line data, both genomic and drug-response~\cite{iorio:etal:2016:gdsc-interactions,yang:etal:2013:gdsc}. We considered a subset of 832 cell lines for which both genomic data and dose-response data against the drug Nutlin-3a were recorded. Nutlin-3a is an MDM2-inhibitor, and as such it promotes p53, a tumor suppressing protein~\cite{shen:maki:2011:p53}. However, this pathway can successfully inhibit cancer growth only when the corresponding gene, TP53, is wild-type, i.e. does not contain a missense mutation.

For each cell line in our dataset, we took the covariates $X$ to be the 50-dimensional PCA projection of its RNA expression data. We took the outcome variable $Y$ to be the z-scored cell line viability of the cell line when exposed to the maximum dose of Nutlin-3a. We let $H$ be an indicator variable that is 1 when the cell line does not have a TP53 mutation. And we took $T$ to be 1 whenever $H=1$, and otherwise we set it to a Bernoulli draw with bias equal to $\text{sigmoid}(W_{\text{GAPDH}})$, where $W_{\text{GAPDH}}$ is the RNA expression of the Glyceraldehyde 3-phosphate dehydrogenase gene. We reran our simulations with 50 different random seeds.

\cref{table:nutlin-simulations} displays the empirical results on GDSC data. Note that in this setting, all methods did a reasonable job of controlling FDR, save for Add-C2G at the lower levels of $\alpha$. However, for BART, FDRreg, and Causal Forests, this is mostly due to having an unreasonably low rejection rate. The fact that Add-C2G generally does not control FDR in this setting likely indicates that the additivity assumption is grossly violated here. Finally, we see that NP-C2G controlled FDR at all levels while maintaing high valid power.

\begin{figure*}
\centering 
\includegraphics[width=1.0\textwidth]{plots/nutlin/combined.pdf}
\caption{Observed FDR (left) and valid power curves (right) for semi-synthetic drug-response data.}
\end{figure*}