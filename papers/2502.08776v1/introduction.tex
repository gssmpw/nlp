\section{Introduction}
\label{sec:introduction}

%


%

%
%

This article concerns causal inference when the treatment only affects a subset of the population. 
%
Such cases are common in medicine, where only individuals with certain characteristics experience the benefits of a prescribed intervention.
For example, a subpopulation of Hispanic patients has a germline mutation that prevents an approved CAR-T cell therapy from binding to the cell surface~\citep{seipel:etal:2023:snp-cd19-car-t-escape}. Lung tumors often acquire a somatic mutation that prevents certain small molecule inhibitors from binding~\citep{da:etal:2011:egfr-lung-cancer}. Personalized cancer mRNA vaccines are only effective if the neoantigens targeted provoke an immune response~\citep{rojas:etal:2023:personalized-mrna-pdac}. In all of these cases, biomarkers of response were unknown at the start of the study.
%
The key analysis step in refining the treatment criteria was to determine which patients likely experienced an affect on their outcomes. Since all patients experience natural variability in their outcomes, determining the affected patient set and the extent of the effects is a statistical inference problem.

In a dataset with treatment and control groups, what effects can be inferred if only some of the treated subjects receive any effect at all?
Can we identify treated individuals that saw an effect from the treatment?
Can we estimate the magnitudes of effects? What kinds of statistical guarantees can we provide?

To address these questions, we develop a general empirical Bayes framework built on top of the classic two-groups model \citep{efron:2008}. \cref{fig:graphical_models}d shows the causal two-groups (C2G) graphical model we propose here. We assume a binary treatment variable $T$, a binary latent response or effect variable $H$, continuous outcome $Y$, and observed confounders $X$. We also consider extensions that contain unmeasured confounders $U$ (see \cref{fig:confounding}). The central statistical challenge in the C2G setup is handling the confounding between the treatment ($T$), latent response ($H$), and outcome ($Y$) when conducting inference for each subject in the study.

The questions one can answer in the C2G model depend on the modeling assumptions one is willing to make. In common parametric and semi-parametric cases, the causal two-groups model can be shown to be identifiable.
In the fully nonparametric case, the C2G model and certain estimands of interest are unidentifiable.
Despite being unable to identify the nonparametric model, it turns out that it is possible to (i) develop an empirical Bayes procedure to test for individual causal effects, (ii) provide informative intervals for estimands of interest, and (iii) maintain robustness to certain kinds of latent confounding.

We propose two estimation algorithms for different instantiations of the C2G model. In the identifiable C2G regime, we develop an EM-based optimization routine for a semi-parametric model under an additive errors assumption. Given a fitted model, there is a step-down empirical Bayes approach \citep{efron:2004, efron2012} to select individuals that came from the responder distribution. We show that this approach controls the false discovery rate (FDR) and attains near-optimal power (\cref{thm:additive-fdr}).


Our second approach handles the fully nonparametric setting. Even though there are many possible prior/responder distribution pairs that could give rise to the observed treatment distribution, it can be shown that this set only enjoys a single degree of freedom, corresponding to a closed interval (\cref{lem:conservative-pi}). Moreover, the endpoints of this interval can be calculated in closed form given estimates of the observed untreated and treated distributions. Using this result, we can construct a valid conservative selection procedure that controls FDR and attains near-optimal power (\cref{thm:general-fdr}). Moreover, we can also easily derive intervals that provably contain the ground-truth latent effect (\cref{corr:np-ite}).

Beyond the latent drug response scenario, the C2G model has applications in many other areas of medicine and science. In particular, the C2G model generalizes the intention-to-treat setting, where a treatment is assigned but compliance with treatment is unmeasured. In the out-patient setting, a patient is prescribed a medication, sent home, and may or may not actually take the treatment. The only observation available to the healthcare provider is when the patient is readmitted for subsequent treatment, leaving the provider unsure whether the patient truly complied with the original treatment assignment. Similarly, social welfare programs like housing lotteries may not actively track which specific vouchers are redeemed, jobs training programs may not take attendance, and
food subsidy programs are unable to know whether foods purchased were actually consumed by the purchaser~\citep{hidrobo:etal:2014:cash-food-vouchers}. In all of these instances, the C2G model can be reinterpreted to have $T$ be the intention-to-treat and $H$ be the latent treatment compliance status. This paper shows the limits and capabilities of causal inference in such studies, and provides new tools to make causal conclusions in the face of latent treatment effects. %