%
\section{The causal two-groups model}
\label{sec:model}
Let $(x_i, y_i, t_i)_{i=1}^n$ be an observational dataset of $n$ samples, where $t_i \in \{0,1\}$ indicates whether individual $i$ was treated $(t_i=1)$ or not $(t_i=0)$, $y_i \in \R$ is the individual outcome, and $x_i \in \R^d$ are potential confounders. For instance, $t_i=1$ indicates that the patient was prescribed a certain drug, $y_i$ is the patient's survival time in days, and $x_i$ is demographic information like age, race, and sex. We drop the subscripts from here on when it is clear we are referring to a single observation.

We focus on the primary analysis goal of selecting responders $(h = 1)$; we will consider estimation of effect sizes as a secondary goal. We adapt the two-groups (2G) model from the multiple testing literature \citep{efron:2004,efron:2008}. For each treated individual ($t=1$), we suppose they are either \textit{non-responders} or \textit{responders}, with outcome distributions $f_0(y \mid x)$ or $f_1(y \mid x)$, respectively. All untreated ($t=0$) patients are assumed to have the same outcome distribution as non-responders. Each treated individual is affected by the treatment with probability $\pi(x)$,
\begin{equation}
\label{eqn:two_groups}
\begin{aligned}
(y \mid x, h) &\sim& (1-h) f_0(y \mid x) + h f_1(y \mid x) \\
(h \mid x, t=0) &=& 0 \\
(h \mid x, t=1) &\sim& \mbox{Bernoulli}(\pi(x)) \\
(t \mid x) &\sim& \mbox{Bernoulli}(\phi(x))  \, ,
\end{aligned}
\end{equation}
where $h \in \{0,1\}$ indicates whether the individual saw any effect from the treatment. Equivalently, $h$ can capture whether the individual complied with a treatment in the intent-to-treat scenario; we will use the responder terminology for the remainder of the paper. \Cref{fig:causal_2g_model} shows the causal graphical model corresponding to \cref{eqn:two_groups}. The model assumes $X$ is a confounder of $T$, $H$, and $Y$. Scenarios with no treatment bias (e.g. randomized controlled trials) are covered as well, as \cref{eqn:two_groups} is strictly a generalization. The model also assumes that treatment response is random, allowing for exogenous variables (e.g. unmeasured biomarkers) to induce uncertainty in response. 

%

%

\subsection{When is the causal two-groups model identifiable?}

%

Our first result is that, with no further assumptions, the causal two-groups model is not identifiable, even without overlap violations.
\begin{proposition}
\label{thm:unident}
The model in \cref{eqn:two_groups} is unidentifiable without further assumptions, even when restricting $0 < \pi(x) < 1$ for all $x$. Moreover, it is unidentifiable even when $x$ is a constant and the distributions $f_0, f_1$ are restricted to being differentiable log-concave probability distributions.
\end{proposition}
For space and clarity, all proofs are presented in the appendix. The key idea is to rewrite the treatment outcome distribution $f_t(y \mid x) := p(y \mid x, t=1)$ as a mixture model,
\begin{equation}
\label{eqn:treatment-mixture}
f_t(y \mid x) = (1-\pi(x)) f_0(y \mid x) + \pi(x) f_1(y \mid x).
\end{equation}
Then one can show that even when $f_0(y \mid x)$ is fixed, there is a range of $\pi$'s and $f_1$'s that give rise to exactly the same $f_t$.

The story changes if parametric forms of the conditional likelihoods $f_0, f_1$ are known. Indeed, the form of \Cref{eqn:treatment-mixture} implies that \Cref{eqn:two_groups} is identifiable at those $x$ such that $f_0(\cdot \mid x), f_1(\cdot \mid x)$ come from a family for which finite mixtures are identifiable~\citep{teicher:1963:identifiability-finite, yakowitz:spragins:1968:identifiability-finite}. As a concrete example, we give a direct proof that \Cref{eqn:two_groups} is identifiable for the class of conditional normal likelihoods.

\begin{proposition}
\label{prop:normal-identifiability}
Suppose that $f_i$ is of the form
$f_i(y \mid x) = \Ncal(y \mid \mu_i(x), \sigma_i^2(x))$
for $i=0,1$. Then \Cref{eqn:two_groups} is identifiable for all $x$ such that $f_0(\cdot \mid x) \neq f_1(\cdot \mid x)$.
\end{proposition}

%
%
In practice, we generally do not know the parametric forms of the outcome distributions a priori. Thus, we would ideally like to avoid parametric assumptions as much as possible.
As we show in \cref{sec:additive,sec:kernel_generalized}, there is some hope here: nonparametric estimation of the true model may be impossible, but semi-parametric estimation and nonparametric testing are viable. The upshot here is that we can select responders with minimal modeling assumptions while still controlling the desired error rate.

%
%
%
%
%
%
%
%
%
%
%
%
%
%
%
%
%
%
%
%



%

\subsection{Testing in the causal two-groups model}
\label{subsec:model:empirical_bayes_testing}
We formulate responder selection as a multiple hypothesis testing problem.
For each observed outcome in a treated sample, we are testing the null hypothesis that it was drawn from the non-responder distribution,
\[
H_0 \colon h = 0 \, .
\]
The multiple testing goal is to maximize power while controlling the false discovery rate at or below the target $\alpha$ level. The false discovery rate selecting $\hat{V} \subseteq \{ i \colon 1 \leq i \leq n \, , t_i = 1 \}$ is given by
\[ \text{FDR} = \E\left[ \frac{1}{|\hat{V}|} \sum_{i \in \hat{V}} \ind[h_i = 0] \right] .\]
As in other two-groups models~\citep{efron:2004,efron:2008,efron2012,scott:etal:2014:fdr-regression,tansey:etal:2018:fdr-smoothing,tansey:etal:icml:2018:bbfdr}, we take an empirical Bayes approach.
At a high level, empirical Bayes testing in the causal two-groups model proceeds as follows:
\begin{enumerate}
    \item Fit $\hat{f}_0(y \mid x)$, a model of outcomes for the untreated population. This also serves as the outcome model for non-responders by definition in \cref{eqn:two_groups}.
    \item Provide (conservative) estimates $\hat{f}_1(y \mid x)$ and $\hat{\pi}(x)$ of $f_1(y \mid x)$ and $\pi(x)$, the heterogeneous treatment effect model and the prior probability of seeing an effect, respectively, using the outcomes of the treated population.
    \item Using the estimates $\hat{f}_0$, $\hat{f}_1$, and $\hat{\pi}$, calculate the (conservative) posterior probability of each treated sample having come from the non-responder distribution,
    \[ \hat{w}_i = \frac{(1-\hat{\pi}(x_i))\hat{f}_0(y_i \mid x_i)}{(1-\hat{\pi}(x_i))\hat{f}_0(y_i \mid x_i) + \hat{\pi}(x_i)\hat{f}_1(y_i \mid x_i)}. \]
    \item Select discoveries $\hat{V} \subseteq \{ i \colon 1 \leq i \leq n \, , t_i = 1 \}$ at the $\alpha$ FDR level using the estimates $\hat{w}_i$,
            \begin{equation}
            \label{eqn:selection}
            \begin{aligned}
            \underset{V}{\text{maximize}} & \mbox{ } |V| \\
             \text{subject to} & \mbox{ } \frac{1}{|V|} \sum_{i \in V} \hat{w}_i \leq \alpha \, .
            \end{aligned}
            \end{equation}
\end{enumerate}
The details of how to model $f_0$, $f_1$, and $\pi$ are dependent on the assumptions one is willing to make about the true causal model. We will consider two such models, one simpler and the other more general.
\Cref{sec:additive} presents the simpler model that assumes additive noise while keeping the noise distribution and outcome mean functions nonparametric.
\Cref{sec:kernel_generalized} then considers arbitrary distributions for the outcomes and response probabilities. 

\subsection{Estimands of interest in the causal two-groups model}
\label{subsec:model:estimands}
Latent (non-)responders in the treatment group make interpreting causal effects more nuanced than in the traditional causal inference setup. We will consider two causal estimands that may be of interest to the analyst.

\begin{itemize}
    \item \emph{Conditional average response effect (CARE)}. This is the analogue of the classical conditional average treatment effect (CATE). Statistically,
    $$\E[Y \mid t=1, h=1, X=x] - \E[Y \mid t=0, X=x] \, .$$
    The CARE estimates the impact of the treatment if we could intervene and force response on treated samples. For instance, rather than prescribing a drug in the out-patient setting, we could instead administer the treatment in the clinic. Removing conditioning yields the average response effect (ARE), analogous to the average treatment effect (ATE). For positive effects, the (C)ATE will generally yield a biased downward estimate of the true effect captured by the (C)ARE.

    \item \emph{Expected responder population fraction (ERPF)}. This answers the question of how many people did a treatment effect. Statistically,
    $$\frac{1}{\sum_i t_i} \sum_i t_i \hat\pi(x_i) \, .$$
    In the classical setup, the ERPF is simply one. When only a fraction of patients respond, the ERPF enables us to distinguish between treatments with a strong effect for a few patients and treatments with a small but consistent effect across the majority of a population.
\end{itemize}

Our ability to estimate these quantities depends on whether the modeling assumptions provided render the corresponding estimand identifiable or at least allow for interval estimation. When only an interval is identifiable, as we shall see in \cref{sec:kernel_generalized}, we can use the lower end to provide conservative estimates.

The potential for non-response also leads to risk-reward trade-offs. For a given set of covariates, we can estimate both the expected response fraction and the expected effect size of the response. Deciding whether to prescribe a treatment then involves a delicate balance in the face of potential treatment costs and side effects. \cref{sec:case-study} shows an example of estimating the CARE and ERPF on a population of cancer patients treated with combination immunotherapy.


%
%


































