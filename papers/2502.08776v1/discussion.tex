\section{Discussion}

%
%
Modern precision medicine is a practice of gradual refinement. Treatments are delivered first to a broad group of patients, results are measured, and correlates of response are used to stratify patients into subgroups. For instance, breast cancer patients initially received chemotherapy until, in 1967, two hormone receptors were discovered as predictive biomarkers in subsets of patients~\citep{jensen:etal:1981:er-pr-breast-cancer-discovery}. The introduction of drugs blocking these hormones led to improved outcomes for hormone-positive patients~\citep{ward:etal:1973:breast-er-trial,osborne:etal:1980:value-er-pr-breast}. A third marker, the surface protein HER2, was discovered in 1987~\citep{slamon:etal:1987:her2-breast-cancer-discovery}. More recently, clinical trials in immunotherapy have led to approved treatments for so-called triple-negative breast cancers (those without any of the three previously established biomarkers) that are positive for the PD-L1 biomarker~\citep{emens:etal:2021:tnbc-nab-pac-atezo-trial}. At each step along the way, patient data was retrospectively analyzed to identify responders and stratify by a predictive marker.

The emergence of large, real-world datasets of electronic health records~\citep[e.g.][]{jee:etal:2024:msk-chord} presents both an opportunity and a complication for data analysts in precision medicine. Cohorts of tens of thousands of patients are now readily available for mining correlations of treatment response, adverse events, overall survival, and other outcomes of interest. Analysts now can detect significant associations in fine-grained subsets of patients to stratify patients to a near-personalized level of precision. The trade-off is that these are not randomized clinical trial data. They are susceptible to all the usual pitfalls and problems of causal inference in observational data. Advancing statistical analyses for the modern era requires new causal methodology for refining patient cohorts. The methods presented here represent a step in this direction.

Going forward, the causal two-groups model raises some interesting statistical questions. On the theoretical side, one might wonder how to give a precise delineation between identifiable and non-identifiable C2G models. In this work, we gave two examples of modeling assumptions that lead to identifiable C2G models: certain classes of parametric models and additive semi-parametric models. But this is far from a complete characterization. What are the conditions that need to be placed on a set of modeling assumptions to ensure that the resulting C2G model is identifiable? 


On the practical side, one might think about extending the causal two-groups model beyond binary treatments. Indeed, many types of interventions are not binary: drugs have doses, treatments have schedules, jobs training programs have multiple days of courses, and diet interventions involve multiple foods and meals. In all of these settings, however, we can still ask the question of whether or not an individual responded to their associated treatment. What then is the appropriate extension of the C2G framework that allows for different degrees or types of treatment?

One might also ask what role independence plays in the C2G model. As it turns out, the FDR control results (\cref{thm:additive-fdr} and \cref{thm:general-fdr}) do not require i.i.d. observations. They only require accurate estimates of the outcome distributions. However, our approach to acquiring those estimates makes use of assumed i.i.d. structure among individuals. In what ways can we relax these i.i.d. assumptions in the C2G model without sacrificing statistical guarantees? Is it possible to consider interference effects and other non-i.i.d. confounding?

Finally, our immunotherapy case study presents a potentially new way to analyze combination therapy studies. Rather than simply considering the difference in survival time, one can now estimate how many patients actually benefit from combination therapy. Application to datasets of different treatment regimens may reveal that certain combinations are effective only for a small group of patients who have an outsized effect, suggesting the need for better patient stratification. %



