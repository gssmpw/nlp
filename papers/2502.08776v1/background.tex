%
\section{Background}
\label{sec:background}
%
This paper draws inspiration from three distinct areas in statistics and machine learning. The first of these is multiple hypothesis testing, where there are many separate hypotheses to be tested and the primary objective is to maximize power while controlling FDR. The second area is treatment assignment and noncompliance, a rich field that has been explored extensively in the causal inference literature. And the last area is the recent body of work in machine learning on estimating treatment effects using flexible black box models. Below, we survey related work in each of these and contrast it with the causal two-groups setting considered in this paper.

\begin{figure*}[t]
\centering
\begin{subfigure}{0.23\linewidth}
\captionsetup{justification=centering}
\centering
\scalebox{0.65}{\begin{tikzpicture}

  %
  \node[obs, ]                               (x) {$X$};
  \node[latent, below=of x]                (h) {$H$};
  \node[obs, left=of h]                   (t) {$T$};
  \node[obs, right=of h]                   (y) {$Y$};

  %
  \edge {x} {h}  ; %
  %
  \edge {t} {h}  ; %
  \edge {h} {y}  ; %

\end{tikzpicture}}
\caption{\label{fig:2g_model}Two-groups with covariates~\citep{scott:etal:2014:fdr-regression}}\end{subfigure}\quad
\begin{subfigure}{0.23\linewidth}
\captionsetup{justification=centering}
\centering
\scalebox{0.65}{
\begin{tikzpicture}

  %
  \node[obs, ]                               (x) {$X$};
  \node[obs, below=of x]                (h) {$H$};
  \node[obs, left=of h]                   (t) {$T$};
  \node[obs, right=of h]                   (y) {$Y$};

  %
  \edge {x} {t} ; %
  \edge {x} {h}  ; %
  \edge {x} {y}  ; %
  \edge {t} {h}  ; %
  \edge {h} {y}  ; %

\end{tikzpicture}}
\caption{\label{fig:rct_model}Principal stratification framework~\citep{frangakis:rubin:2002:principal-stratification} }\end{subfigure}\quad
\begin{subfigure}{0.23\linewidth}
\captionsetup{justification=centering}
\centering
\scalebox{0.65}{
\begin{tikzpicture}

  %
  \node[obs, ]                               (x) {$X$};
  \node[latent, below=of x]                (h) {$H$};
  \node[obs, left=of x]                     (w) {$W$};
  \node[obs, left=of h]                   (t) {$T$};
  \node[obs, right=of h]                   (y) {$Y$};

  %
  \edge {x} {h}  ; %
  \edge {x} {y}  ; %
  \edge {t} {h}  ; %
  \edge {h} {y}  ; %
  \edge {h} {w}  ; %

\end{tikzpicture}}
\caption{\label{fig:rct_proxy}RCT with compliance proxy~\citep{boatman:etal:2017:compliance-error}}\end{subfigure}\quad
\begin{subfigure}{0.23\linewidth}
\captionsetup{justification=centering}
\centering
\scalebox{0.65}{\begin{tikzpicture}

  %
  \node[obs, ]                               (x) {$X$};
  \node[latent, below=of x]                (h) {$H$};
  \node[obs, left=of h]                   (t) {$T$};
  \node[obs, right=of h]                   (y) {$Y$};

  %
  \edge {x} {t} ; %
  \edge {x} {h}  ; %
  \edge {x} {y}  ; %
  \edge {t} {h}  ; %
  \edge {h} {y}  ; %

\end{tikzpicture}}
\caption{\label{fig:causal_2g_model}Causal two-groups \newline (this paper) \newline \mbox { }}
\end{subfigure}
\caption{\label{fig:graphical_models}
(a) The FDR regression model of \citet{scott:etal:2014:fdr-regression}.
(b) An observational dataset with observed noncompliance \citep{frangakis:rubin:2002:principal-stratification}.
(c) A randomized controlled trial with a compliance proxy \citep{boatman:etal:2017:compliance-error}.
(d) The general causal two-groups model. The treatment $T$, (unobserved) treatment effect indicator $H$, and the outcome $Y$ are all confounded by $X$. The causal two-groups model generalizes other scenarios such as those in (a) and (b).}
\end{figure*}

\subsection{Multiple hypothesis testing with covariates}
\label{subsec:background:multiple-testing}
%
Several approaches to multiple testing incorporate covariates as side information. These methods generally work under the two-groups model \citep{efron:2004,efron:2008,efron2012} where a test statistic is drawn either from the null or alternative with some prior probability. However, both frequentist \citep{ignatiadis:etal:2016:ihw,xia:etal:2017:neuralfdr,lei:fithian:2018:adapt,chen:etal:2018:functional-fdr,li:barber:2019:sabha} and Bayesian \citep{scott:etal:2014:fdr-regression,patra:bodhi:2016:2groups-jrssb,tansey:etal:2018:fdr-smoothing,tansey:etal:icml:2018:bbfdr} approaches assume that the covariates only impact the likelihood of treatment having an effect. This is often a reasonable assumption in laboratory experiments, where biological replicates can be tested repeatedly under different experimental conditions. In observational data, the assumption is unlikely to hold as covariates are often confounders that affect the probability of treatment assignment, treatment compliance, and both the null and alternative outcomes. \Cref{fig:2g_model} shows the graphical model for existing two-groups models.


In a similar vein to the latent response model considered here, \citet{duan:etal:2024:interactive-fdr-control} propose an interactive protocol to the testing problem, in which an analyst is given the covariates and outcomes but is blinded to the treatment assignments, and they must cooperate with an algorithm that knows the treatment assignments to make discoveries. In their setup, however, the blinding to treatment assignments is not intrinsic to the problem but is rather a technique to increase power without losing FDR control.

\subsection{Causal inference with noncompliance}
\label{subsec:background:compliance}
In the statistics literature, methods for handling noncompliance were originally motivated by clinical trial data. \citet{efron:feldman:1991:compliance} observed that patients receiving placebo treatments had better outcomes as a function of compliance rate to the placebo. \citet{frangakis:rubin:2002:principal-stratification} generalized this in the \textit{principal stratification} framework, with treatment assignment and compliance jointly representing potential interventions. \Cref{fig:rct_model} shows the graphical model for a randomized controlled trial (RCT) in the principal stratification setting. 
These models assume compliance is observed, leading to identifiability of the causal effects; here we consider the case where compliance is entirely latent.

Other methods consider noisy or latent compliance with side information. \citet{angrist:etal:1996:iv-noncompliance} introduced an instrumental variable approach to estimating average treatment effects with noisy compliance. \citet{boatman:etal:2017:compliance-error} consider the scenario where a noisy proxy variable is available for compliance, such as blood testing for nicotine levels when patients may be misreporting their smoking habits. Further, they assume the true parametric model is known and identifiable. \Cref{fig:rct_proxy} shows the graphical model for the proxy variable case. More recent work leverages auxiliary variables combined with conditional ignorability assumptions or partial observability of the compliance variable~\citep{jiang:ding:2021:identification-principal-strata-auxiliary,jiang:etal:2022:multiply-robust-principal-ignorability,lu:etal:2023:principal-strata-continuous}.  These models obtain identifiability by leveraging additional information to disentangle the confounding from the causal effects. By contrast, we consider the challenging case where no instrumental or proxy variables are available.


\subsection{Estimating individual treatment effects in observational data}
\label{subsec:background:causal-effects}
A large body of recent work in machine learning has considered the problem of predicting the conditional average treatment effect~\cite[e.g.][]{shi:etal:2019:dragonnet,johansson:etal:gen-bounds-causal,bica:etal:2020:causal-gan,wang:etal:2024:cauusal-optimal-transport}. In the presence of latent response or noncompliance, the conditional average treatment effect (CATE) is generally an underestimate of the true effect if one could intervene, such as requiring compliance. CATE estimates also do not directly help identify which samples may have responded to treatment, as they integrate out the probability of response in a given sample. However, methods that provide uncertainty quantification can be used to conduct hypothesis tests for whether a given treatment has any effect on a given individual. The causal forests method \citep{wager:athey:2018:causal-forests} adapts random forests \citep{breiman:2001:random-forests} to the causal setting and provides asymptotic confidence intervals for effects. Bayesian additive regression trees (BART) \citep{chipman:etal:2010:bart} have been shown to perform well for estimating causal effects without any modification \citep{hill:etal:2011:causal-bart}. The posterior credible intervals provided by BART provide the Bayesian equivalent of frequentist confident intervals for causal effect sizes. Both causal forests and BART are predictive models with no explicit notion of null effects or noncompliance. In principle, the predictive models could be integrated into our causal two-groups procedures in place of black box neural networks. Synthetic and semi-synthetic simulations in \cref{sec:more_simulations,sec:gdsc_simulations} show that confidence intervals from these models are not sufficient to control FDR in finite samples.


