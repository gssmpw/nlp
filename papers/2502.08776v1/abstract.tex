%

%
%
%
%
%
%
%
%
%
%

Scientists often need to analyze the samples in a study that responded to treatment in order to refine their hypotheses and find potential causal drivers of response. 
Natural variation in outcomes makes teasing apart responders from non-responders a statistical inference problem. 
To handle latent responses, we introduce the causal two-groups (C2G) model, a causal extension of the classical two-groups model.
The C2G model posits that treated samples may or may not experience an effect, according to some prior probability. 
We propose two empirical Bayes procedures for the causal two-groups model, one under semi-parametric conditions and another under fully nonparametric conditions. 
The semi-parametric model assumes additive treatment effects and is identifiable from observed data. 
The nonparametric model is unidentifiable, but we show it can still be used to test for response in each treated sample. 
We show empirically and theoretically that both methods for selecting responders control the false discovery rate at the target level with near-optimal power. 
We also propose two novel estimands of interest and provide a strategy for deriving estimand intervals in the unidentifiable nonparametric model. 
On a cancer immunotherapy dataset, the nonparametric C2G model recovers clinically-validated predictive biomarkers of both positive and negative outcomes.
Code is available at \url{https://github.com/tansey-lab/causal2groups}.