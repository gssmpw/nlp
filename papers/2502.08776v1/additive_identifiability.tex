\subsection{Proof of \cref{thm:additive-identify}}

For any $x$, observe that we can shift the space by $\mu_0(x)$ so that the model in \Cref{eqn:additive_errors} becomes
\begin{align*}
f_0(y \mid x) &= g(y) \\
f_1(y \mid x) &= g(y - \tau(x)).
\end{align*}
In this case, the observable distributions are the null distribution $f_0 = g$ and the treatment distribution
\[ f_t(y \mid x) = (1-\pi(x)) g(y) + \pi(x) g(y - \tau(x)).  \]
Fixing a particular $x$, the identifiability of \Cref{eqn:additive_errors} can be restated as follows.

\begin{theorem}[Restatement of \cref{thm:additive-identify}]
\label{thm:additive-identify-restate}
Let $g$ denote a probability density function over $\R$. For $\pi \in (0,1)$, $\tau \in \R$, define the density function 
\[ f(y ; \pi, \tau) = (1- \pi) g(y) + \pi g(y-\tau). \]
Then the family of densities $\{f(y ; \pi, \tau) \,  \mid \, \pi \in (0,1), \tau \in \R \setminus \{ 0\} \}$ is identifiable.
\end{theorem}
    
We begin by recalling the characteristic function of a probability distribution. If $g$ is a probability distribution over $\R$, then its characteristic function is given by
\[ \varphi_g(\omega) = \E_{X\sim g}\left[ e^{-i \omega X} \right] = \int_{-\infty}^\infty g(x) e^{-i \omega x} \, dx,  \]
where $\omega \in \R$ and the integral is only well-defined when $g$ is a probability density function. We will make use of the following elementary property of the characteristic function.

\begin{lemma}
\label{lem:charfun-property}
For any probability density function $g$, there exists a value $\omega_0 > 0$ such that $|\varphi_g(\omega)| > 0$ for all $\omega \in [0, \omega_0]$.
\end{lemma}
\begin{proof}
This follows directly from the fact that $\varphi_g$ is uniformly continuous on $\R$ and $\varphi_g(0) = 1$ \citep{resnick:2013:probability}.
\end{proof}

With this tool in hand, we turn to the proof of \Cref{thm:additive-identify-restate}.

\begin{proof}[Proof of \Cref{thm:additive-identify-restate}]
Let $\pi_1, \pi_2 \in (0,1)$ and $\tau_1, \tau_2 \in \R \setminus \{ 0 \}$ be parameters such that $f(y;\pi_1, \tau_1) = f(y;\pi_2, \tau_2)$. Our goal is to show that we must have $\pi_1 = \pi_2$ and $\tau_1 = \tau_2$. We consider two cases.


\paragraph{Case 1: $\pi_1 = \pi_2$.} In this case, we must have
\begin{align*}
0 &= f(y;\pi_1, \tau_1) - f(y;\pi_2, \tau_2) \\
&= (1- \pi_1) g(y) + \pi_1 g(y-\tau_1) - (1- \pi_2) g(y) - \pi_2 g(y-\tau_2) \\
&= \pi_1 g(y-\tau_1) - \pi_1 g(y - \tau_2)
\end{align*}
for all $y \in \R$. Multiplying through by $e^{i \omega y}/\pi_1$ and integrating over $\R$, we have
\begin{align*}
0 &= \int_{-\infty}^\infty g(y-\tau_1) e^{i \omega y} \, dy - \int_{-\infty}^\infty g(y-\tau_2) e^{i \omega y} \, dy \\
&= \int_{-\infty}^\infty g(y) e^{i \omega (y+\tau_1)} \, dy - \int_{-\infty}^\infty g(y) e^{i \omega (y+\tau_2)} \, dy \\
&= \varphi_g(\omega) e^{i\omega \tau_1} - \varphi_g(\omega) e^{i\omega \tau_2},
\end{align*}
which holds for all $\omega \in \R$. Invoking \Cref{lem:charfun-property}, there is a value a value $\omega_0 > 0$ such that $|\varphi_g(\omega)| > 0$ for all $\omega \in [0, \omega_0]$. Plugging $\omega = \min \left( \omega_0, \frac{\pi}{|\tau_1 - \tau_2|}\right)$ into the above and dividing by $\varphi_g(\omega)$, we can rearrange to observe
\[ e^{i \omega (\tau_1 - \tau_2)} = 1. \]
This can only be true if $\omega (\tau_1 - \tau_2) = 2 \pi n$ for some $n \in \Z$. By our choice of $\omega$, it must be the case that $\tau_1 = \tau_2$.

\paragraph{Case 2: $\pi_1 \neq \pi_2$.} Here we will reach a contradiction. Assume without loss of generality that $\pi_1 < \pi_2$. Then we have
\begin{align*}
0 &= f(y;\pi_1, \tau_1) - f(y;\pi_2, \tau_2) \\
&= (1- \pi_1) g(y) + \pi_1 g(y-\tau_1) - (1- \pi_2) g(y) - \pi_2 g(y-\tau_2) \\
&= (\pi_2 - \pi_1) g(y) + \pi_1 g(y-\tau_1) - \pi_2 g(y-\tau_2)
\end{align*}
for all $y \in \R$. Multiplying through by $e^{i \omega y}$ and integrating over $\R$, we have
\begin{align*}
0 &= (\pi_2 - \pi_1) \int_{-\infty}^\infty g(y) e^{i \omega y} \, dy + \pi_1\int_{-\infty}^\infty g(y-\tau_1) e^{i \omega y} \, dy - \pi_2 \int_{-\infty}^\infty g(y-\tau_2) e^{i \omega y} \, dy \\
&= (\pi_2 - \pi_1) \varphi_g(\omega) + \pi_1 \varphi_g(\omega) e^{i\omega \tau_1} - \pi_2 \varphi_g(\omega) e^{i\omega \tau_2},
\end{align*}
for all $\omega \in \R$. Invoking \Cref{lem:charfun-property} again, we have $|\varphi_g(\omega)| > 0$ for all $\omega \in [0, \omega_0]$.  Taking $\omega = \min( \omega_0, \frac{\pi}{|\tau_1|})$, we divide through by $\varphi_g(\omega) \pi_2$ and rearrange to get
\[ e^{i \omega \tau_2} = 1 - \frac{\pi_1}{\pi_2} + \frac{\pi_1}{\pi_2} e^{i \omega \tau_1} = 1 - p + p e^{i \omega \tau_1},\]
where $p \in (0, 1/2)$. Taking magnitudes of both sides, we have
\begin{align*}
1 = \left| e^{i \omega \tau_2} \right| &= \left| 1 - p + p e^{i \omega \tau_1} \right| \\
&= \left| 1 - p + p( \cos(\omega \tau_1) + i \sin(\omega \tau_1)) \right| \\
&= \sqrt{ (1 - p + p \cos(\omega \tau_1))^2 + p^2 \sin^2(\omega \tau_1) } \\
&= \sqrt{ 1 - 2p(1-p)(1-\cos(\omega \tau_1))} < 1,
\end{align*}
where the last line follows from our choice of $\omega$. Thus, we have reached a contradiction.
\end{proof}
    


