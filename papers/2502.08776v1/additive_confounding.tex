\subsection{Proof of \cref{prop:compliance-confounding-ac2g}}

The full compliance confounding additive causal two-groups model is as follows.

\begin{align*}
T |  X=x, U=u &\sim& \textnormal{Bern}(\phi(x,u))  \\
H | X=x, U=u, T=0 &=& 0  \\
H | X=x, U=u, T=1 &\sim& \textnormal{Bern}(\pi(x,u))  \\
Y | X=x, H=h &=& \mu_{h}(x) + \epsilon  \\
\epsilon &\sim& g.
\end{align*}

To show that the additive causal two-groups model is well-specified, observe that we still have
\[ p(Y = y \mid X=x, T=0 ) 
= p(Y = y \mid X=x, H=0 ) 
=  {g}(y - {\mu}_0(x)). \]
Thus, all that remains to be shown is that when we marginalize over $U$, we can write out
\[ p(Y = y \mid X=x, T=1) =  (1-\pi(x)) {g}(y - {\mu}_0(x)) + \pi(x) {g}(y - \mu_1(x)) .\]
Working it out,
\begin{align*}
p(Y = y \mid X=x, T=1) &= \sum_{h =0}^1 p(H=h \mid X=x, T=1) p(Y = y \mid X=x, H=h) \\
&= \sum_{h =0}^1 p(H=h \mid X=x, T=1) g(y - \mu_h(x)).
\end{align*}
Marginalizing over $U$, we have
\begin{align*}
p(H=1 \mid X=x, T=1) &= \int p(U = u \mid X=x, T=1) p(H = 1 \mid X=x, U=u, T=1) \, du \\
&= \int p(U = u \mid X=x, T=1) \pi(x,u) \, du \\
&=: \pi(x),
\end{align*}
where we have simply defined $\pi(x)$ in the last line. Putting it all together gives us the desired result.


\subsection{Proof of \cref{prop:latent-nonrobust-ac2g}}

Consider the following model.
\begin{align*}
H \mid X=x, T=1 &\sim& \textnormal{Bern}(\pi(x)) \\ 
Y \mid X=x, U=u, H=0 &\sim& \mu_0(x) + u + \epsilon \\ 
Y \mid X=x, U=u, H=1 &\sim& \mu_1(x) + \epsilon \\ 
U, \epsilon &\sim& \Ncal(0,1).
\end{align*}
Marginalizing over $U$ and $\epsilon$, we have
\begin{align*}
Y \mid X=x, H=0 &\sim& \Ncal(\mu_0(x),2) \\ 
Y \mid X=x, H=1 &\sim& \Ncal(\mu_1(x),1).
\end{align*}
As the above have different standard deviations, this model cannot be rewritten in the form of \cref{eqn:additive_errors}. Thus, the additive causal two-groups model is misspecified here.