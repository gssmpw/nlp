\begin{table}[htp!]
\centering
\begin{tabular}{|| l | l||} 
 \hline
 Additive & Nonadditive \\ 
 \hline\hline
 $\beta, \gamma, \theta \sim \mathcal{N}(0, I_d/\sqrt{d} )$ & $\beta, \gamma \sim \mathcal{N}(0, I_d/\sqrt{d} )$ \\ 
  & $c \sim \text{HalfNormal}(0, 2)$ \\
 $H | T=1 \sim \text{Bern}( \text{sigmoid}(\beta^T X))$ & $W = \gamma^TX$  \\
 $T \sim \text{Bern}(0.5)$ & $T \sim \text{Bern}( \text{sigmoid}(W)$ \\
 $\mu_0 = \gamma^T X $ & $H | T=1 \sim \text{Bern}( \text{sigmoid}(\beta^T X))$ \\ 
 $\mu_1 = \mu_0 + \tau \sum_i |X_i||\gamma_i| $ & $B_{ij} \sim \text{Bern}(0.1), Z_{ij} \sim \text{StudentT}(3)$ \\
 $Y | H=0 \sim \mathcal{N}(\mu_0, 1)$  & $L = \sum_{i,j} B_{i,j} Z_{i,j} X_i X_j$ \\
 $Y | H=1 \sim \mathcal{N}(\mu_1, 1)$   & $Y \sim \mathcal{N}( \log(1 + \exp(cW + \theta^T X + \tau H + L)), 1 ) $ \\
 \hline
\end{tabular}
\caption{Data generating models for additive and nonadditive synthetic simulations.}
\label{table:synthetic-data}
\end{table}

The full data generating process for both synthetic additive and nonadditive simulations is described in \cref{table:synthetic-data}.
All simulations were performed on a cluster with Intel Xeon Gold 6348 CPUs. Each simulation ran in under 2 hours with 2 CPUs allocated. In all plots, shaded regions represent 95\% confidence intervals under the normal approximation.

For additional baselines, we also compared against three methods that are valid ways of testing in the intention-to-treat setting. FDR-Regression \citep{scott:etal:2014:fdr-regression} controls the local FDR in different subsets of the experiment. BART \citep{hill:etal:2011:causal-bart} and Causal forests \citep{wager:athey:2018:causal-forests} are two nonparametric tree-based methods for estimating individual treatment effects. 

To appropriately measure power in this setting, we considered \emph{valid power}, which we define to be 
\[ \text{vp}(\alpha) =
\begin{cases}
0 & \text{ if }  \, \, \hat{\text{fdr}}(\alpha) - \text{CI}_{\text{fdr}} > \alpha \\
\frac{\# \text{ of rejected true non-nulls}}{\# \text{of true non-nulls}} & \text{ otherwise}
\end{cases},  \]
where $\hat{\text{fdr}}(\alpha)$ is the average observed FDR of the procedure at level $\alpha$ and $\text{CI}_{\text{fdr}}$ is its associated 95\% confidence interval band, over the 50 different random runs. Intuitively, valid power captures the standard notion of power in settings where FDR has been respected.

We generally observe pathological behavior for the intention-to-treat baselines. In particular, both BART and Causal forest generally reject everything at extremely low nominal FDR levels, leading to high observed FDR rates. The valid power for these methods is categorically zero until the first nominal FDR level for which the observed FDR is acceptable, at which point the valid power immediately jumps to 1. On the other hand, we observe that FDRreg's behavior changes between settings, having high FDR values on the additive data and low FDR values on the nonadditive data. In all settings, however, the valid power of FDRreg was far below the C2G methods and the frequentist baseline.

It is worth emphasizing that these results are not intended as an indictment of the baseline methods. When the assumptions of the problem match those of the methods, these baselines generally perform quite well. Rather, these results illustrate how the causal two-groups setting breaks the assumptions behind these methods.

\begin{figure*}
\centering %
\hspace{-2em}\includegraphics[width=1.03\textwidth]{plots/additive/fdr.pdf}
\caption{FDR curves on additive synthetic data.}
\end{figure*}

\begin{figure*}
\centering %
\hspace{-2em}\includegraphics[width=1.03\textwidth]{plots/additive/power.pdf}
\caption{Valid power curves on additive synthetic data.}
\end{figure*}

%
%
%
%
%


\begin{figure*}
\centering 
\hspace{-2em}\includegraphics[width=1.03\textwidth]{plots/nonadditive/fdr.pdf}
\caption{FDR curves on nonadditive synthetic data.}
\end{figure*}

\begin{figure*}
\centering %
\hspace{-2em}\includegraphics[width=1.03\textwidth]{plots/nonadditive/power.pdf}
\caption{Valid power curves on nonadditive synthetic data.}
\end{figure*}

%
%
%
%
%