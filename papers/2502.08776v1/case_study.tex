\section{Case study: Immune checkpoint inhibitors}
\label{sec:case-study}

We apply the NP-C2G model to a dataset tracking survival of cancer patients treated with immune checkpoint inhibitors~\citep{samstein:etal:2019:tmb-immunotherapy-survival}. Each patient in the study had targeted next-generation sequencing performed on their tumors, recording the mutational state of 341-468 genes. The sequencing data generates a binary vector for each patient where one indicates a somatic mutation in a gene and zero indicates the gene is not mutated. Tumors were sequenced with different gene panels, leaving missing data in the patient-by-gene matrix. To handle this, we performed Bernoulli matrix factorization on the sequencing matrix to create 5-dimensional embeddings for each patient, where the dimension was chosen using 3-fold cross-validation on held out log-likelihood. The embeddings, combined with the clinical features of sex, age, cancer type and metastatic status, form the confounders in our analysis.

The study tracked which of three treatments was administered: anti-CLT4A, anti-PD-1/PD-L1, or a combination of both. For our analysis, we consider patients receiving anti-CLT4A drugs ($n=99$) as the baseline (untreated) group and patients receiving combination treatments ($n=255$) as the treatment group. A responder in our study is therefore a patient who received an additional effect from PD-1/PD-L1 therapy on top of the baseline effect from anti-CTLA4 therapy. Determining which patients respond to combination therapy versus monotherapy is valuable because immunotherapies can trigger life-threatening complications. A stratification of patients into those likely to benefit or be harmed by combination therapy could inform future therapy decisions and potentially save lives.


\begin{figure*}
\centering %
\includegraphics[width=.95\textwidth]{plots/survival/km_zscore.pdf}
\caption{Case study survival data. \emph{Left:} Kaplan-Meier fits to the survival data from the untreated and treated groups. \emph{Right:} Histogram of Cox proportional hazard-transformed outcomes for the untreated and treated groups.}
\label{fig:case-study-data}
\end{figure*}

To transform the survival information to numerical outcomes, we first fit a Cox proportional hazards model to the monotherapy group. For each patient $i$, we let $s_i$ denote the patient's survival time if they were uncensored. If they were censored, $s_i$ is the expectation of this time under the Cox model conditioned on being at least the observed survival time. This provides a conservative estimate of the true survival time because the conditional expectation will shrink survival times to the mean. We convert survival times to $z$-scores via the inverse normal cumulative distribution function $P(\text{survival time} \leq s_i)$ under the Cox model. We use the $z$-score as the outcome variable $y_i$ for each patient. \Cref{fig:case-study-data} shows the survival curves for both groups and the resulting outcome variables.


We fit the NP-C2G model according to the methodology outlined in \cref{sec:kernel_generalized}. For a fine-grained grid of nominal FDR thresholds $\alpha \in [0, 0.25]$, we calculate the rejection set of treated patients $S_\alpha$. To detect biomarkers of response, we perform a hypothesis test for each gene in the dataset. We compute Fisher's exact test on the contingency table over treated patients where the two variables are whether or not the patient had the mutated gene and whether or not the patient was in the rejected set $S_\alpha$. We consider both possible alternatives in our test: (i) a randomly generated table has a larger number of patients that are in $S_\alpha$ and do not have the corresponding mutation, and (ii) a randomly generated table has a smaller number of patients that are in $S_\alpha$ and do not have the corresponding mutation. After calculating the p-values for all the genes under (i) or (ii), we performed Benjamini-Hochberg correction to obtain q-values. We consider a range of possible thresholds to reject and record those genes that are rejected with q-values less than 0.1. Genes rejected under alternative (i) are termed \emph{more favorable}, as it is unlikely to have more treated patients that are both in $S_\alpha$ and have the mutation. Analogously, genes rejected under alternative (ii) are termed \emph{less favorable}.

For each detected biomarker, we report the two estimands of interest from \cref{subsec:model:estimands}. For the conditional average response effect, we calculate the upper and lower bounds of the individual response effects given by \cref{corr:np-ite}, and then average these values over individuals carrying a mutation in a given gene. For the expected responder population fraction, we use the conservative prior $\pi^\star$ to generate a conservative estimate of the responder fraction. Each calculation is performed conditioned on the mutated subpopulation and thus we report the conditional expected responder population fraction (CERPF).

\cref{table:case study} displays the results for the selected genes. Among the mutations identified as more favorable, VHL mutations have the smallest rejection threshold, the smallest q-values, the largest CERPF, and a CARE interval that is entirely positive. VHL mutations have been implicated in the literature as a positive biomarker for anti-PD-1/PD-L1 therapies~\citep{meng:etal:2024:vhl-loss-antitumor-immunity}, suggesting this is likely a true positive. Among the mutations identified as less favorable, BRAF mutations have the smallest rejection threshold, the smallest q-values, and have been implicated in increased toxicity of combination anti-CTLA4/anti-PD-1 treatments~\citep{piresdasilva:etal:2022:braf-toxicity-pd1}. Among the other mutations, there is evidence suggesting SETD2 mutations lead to more favorable outcomes for immunotherapy~\citep{jee:etal:2024:msk-chord,lu:etal:2021:SETD2-immunotherapy} and TERT mutations are associated with less favorable outcomes for anti-PD-1 therapies compared with anti-CTLA4 therapies~\citep{li:etal:2020:tert-immunotherapy}. The results for ERBB4 and GRIN2A represent potential novel discoveries. Given their small CERPF, it is likely they only affect a small subset of patients and may warrant further investigation to find more fine-grained molecular markers of response.

\begin{table}
\centering
%
%
%
%
%
%
%
%
%
%
%
%
%
\begin{tabular}{|c|c|c|c|c|c|c|}
\hline
& \multicolumn{2}{|c|}{More favorable} & \multicolumn{4}{|c|}{Less favorable} \\
\hline
Mutation & SETD2 & VHL & BRAF & ERBB4 & GRIN2A & TERT \\
\hline
Smallest $\alpha$ & 0.07 & 0.029 & 0.149 & 0.239 & 0.247 & 0.201 \\
\hline
Largest -$\log_{10}(q)$ & 1.577 & 7.511 & 3.373 & 1.125 & 1.003 & 1.963 \\
\hline
CARE interval & (-4.3, 1.4) & (1.3, 2.1) & (-8.5, 0.2) & (-9.0, 1.6) & (-7.8, 1.1) & (-7.2, 1.2) \\
\hline
CERPF & 0.39 & 0.62 & 0.22 & 0.16 & 0.15 & 0.25\\
\hline
\end{tabular}
\caption{Case study results comparing outcomes from CTLA4 treatment (null) and CTLA4/PD-1 combo treatment (treatment). \emph{Smallest $\alpha$} denotes the smallest nominal FDR rate $\alpha$ of the NP-C2G procedure that results in the gene being selected. \emph{Largest $- \log_{10}(q)$} denotes the largest $q$-value observed for the selected gene. \emph{CARE interval} denotes the average of the treatment effect intervals produced by NP-C2G over the population with the given mutation. \emph{CERPF} denotes the average of the lower-bound prior probabilities produced by NP-C2G over the treated population with the given mutation.}
\label{table:case study}
\end{table}


