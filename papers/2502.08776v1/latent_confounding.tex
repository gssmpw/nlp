
\subsection{Proof of \cref{prop:canonical-confounding}}

We will make use of the following lemma.

\begin{lemma}
\label{lem:general-latent-confounding}
Let $\Dcal$ and $\Dcal'$ be two distributions. For random variables $X, Y, H, T, U$, there is a choice of distributions for $U$, $Y|X,H,U$, $T|X,U$, and $H|X,T$ such that
\begin{enumerate}
\item $X, Y, H, T, U$ obey the canonical confounding graphical model,
\item $\pr(T=1 \mid X) = 1/2$ [no treatment overlap violations],
\item $\pr(H=0 \mid X, T=0) = 1$ [all untreated are non-responders], 
\item $\pr(H=1 \mid X, T=1) = 1/2$ [no responder overlap violations], 
\item $Y \mid X, T=0 \sim \Dcal'$, and
\item $Y \mid X, H=0, T=1 \overset{d}{=} Y \mid X, H=1, T=1 \overset{d}{=} Y \mid X, T=1 \sim \Dcal$ [matching responder and non-responder distributions].
\end{enumerate}
\end{lemma}
\begin{proof}
Our conditional distributions will be independent of $X$, so we will drop $X$ going forward. We make the following choices for the conditional distributions
\begin{align*}
Y \mid  H=0, U=0 &\sim \Dcal \\
Y \mid H=0, U=1 &= \Dcal' \\
Y \mid H=1, U=0 &= \Dcal \\
Y=1 \mid H=1, U=1 &= \Dcal' \\
\pr(T=1 \mid U=0) &= 1 \\ 
\pr(T = 1 \mid U=1) &= 0 \\ 
\pr(H=1 \mid T=1) &= 1/2 \\
\pr(H=0 \mid T=0) &= 1 \\
\pr(U=1) &= 1/2.
\end{align*}
Then we can work out the following
\begin{align*}
\pr(T=1) &= \pr(U=1)\pr(T = 1 \mid U=1) + \pr(U=0)\pr(T = 1 \mid U=0) = 1/2 \\
\pr(U=1 \mid H, T=1) &= \pr(U=1 \mid T=1) \\
&= \frac{\pr(T=1 \mid U=1)\pr(U=1)}{\pr(T=1 \mid U=1)\pr(U=1) + \pr(T=1 \mid U=0)\pr(U=0)} = 0 \\
\pr(U=1 \mid H, T=0) &= \pr(U=1 \mid T=0) \\
&= \frac{\pr(T=0 \mid U=1)\pr(U=1)}{\pr(T=0 \mid U=1)\pr(U=1) + \pr(T=0 \mid U=0)\pr(U=0)} = 1 \\
\pr(U=0 \mid H, T=1) &= 1 \\
\pr(U=0 \mid H, T=0) &= 0.
\end{align*}
And so we have
\begin{align*}
\pr(Y=y | H=1, T=1) &= \pr(Y=y \mid H=1, U=0, T=1) \pr(U=0 \mid H=1, T=1) \\
&\hspace{2em} +  \pr(Y=y \mid H=1, U=1, T=1) \pr(U=1 \mid H=1, T=1) \\
&= \pr(Y=y \mid H=1, U=0) \pr(U=0 \mid H=1, T=1) \\
&\hspace{2em} +  \pr(Y=y \mid H=1, U=1) \pr(U=1 \mid H=1, T=1) \\
&= \pr(Y=y \mid H=1, U=0).
\end{align*}
Thus, $Y \mid H=1, T=1 \sim \Dcal$. Similarly, we have
\begin{align*}
\pr(Y=y | H=0, T=1) &= \pr(Y=y \mid H=0, U=0, T=1) \pr(U=0 \mid H=0, T=1) \\
&\hspace{2em} +  \pr(Y=y \mid H=0, U=1, T=1) \pr(U=1 \mid H=0, T=1) \\
&= \pr(Y=y \mid H=0, U=0) \pr(U=0 \mid H=0, T=1) \\
&\hspace{2em} +  \pr(Y=y \mid H=0, U=1) \pr(U=1 \mid H=0, T=1) \\
&= \pr(Y=y \mid H=0, U=0).
\end{align*}
Therefore, $Y \mid H=0, T=1 \sim \Dcal$. Moreover, we can work out
\begin{align*}
\pr(Y = y \mid T=0) &= \pr(Y = y \mid H=0, T=0) \\
&= \pr(Y = y \mid U=0, H=0) \pr(U=0 \mid T=0) \\
&\hspace{3em}+ \pr(Y = y \mid U=1, H=0) \pr(U=1 \mid T=0)\\
&= \pr(Y = y \mid U=1, H=0).
\end{align*}
Thus, $Y \mid T=0 \sim \Dcal'$.
\end{proof}

\begin{proof}[Proof of \cref{prop:canonical-confounding}]
To finish the proof of \cref{prop:canonical-confounding}, we can take $\Dcal$ and $\Dcal'$ to be distributions with non-intersecting supports, e.g. uniform distributions over separated intervals. Then \cref{lem:general-latent-confounding} gives us a confounded setting in which the non-treatment distribution does not match the non-response-under-treatment distribution, violating \cref{eqn:two_groups}. Thus, the nonparametric causal two-groups model is not robust against latent confounding, and therefore is also not robust against total confounding.


Moreover, we can also see that
if we were to apply the nonparametric causal two-group model anyway, we would conclude that $\pi(x) = 1$ for all $x$ satisfying $t=1$. This is because the treatment and non-treatment distributions do not share any support, and so the only valid mixture satisfying \cref{eqn:treatment-mixture} is the trivial mixture that assigns the responder distribution to be the treatment distribution itself and for $H=1$ to occur with probability 1 when $T=1$. However, as shown in \cref{prop:canonical-confounding}, the true probability of $H=1$ is 1/2 when $T=1$. 
\end{proof}


