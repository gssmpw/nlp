\section{Robustness against unmeasured confounding}
\label{sec:confounding}

\begin{figure*}[t]
\centering
\begin{subfigure}{0.22\linewidth}
\centering
\scalebox{0.7}{\begin{tikzpicture}

  %
  \node[obs, ]                               (x) {$X$};
  \node[latent, below=of x]                (h) {$H$};
  \node[obs, left=of h]                   (t) {$T$};
  \node[obs, right=of h]                   (y) {$Y$};
  \node[latent, right=of x]                 (u) {$U$};

  %
  \edge {x} {t} ; %
  \edge {x} {h}  ; %
  \edge {x} {y}  ; %
  \edge {t} {h}  ; %
  \edge {h} {y}  ; %
  \edge {u} {t}  ;
  \edge {u} {y}  ;

\end{tikzpicture}}
\caption{\centering \label{fig:confounding:canonical}Canonical\newline confounding}
\end{subfigure}\quad
\begin{subfigure}{0.22\linewidth}
\centering
\scalebox{0.7}{\begin{tikzpicture}

  %
  \node[obs, ]                               (x) {$X$};
  \node[latent, below=of x]                (h) {$H$};
  \node[obs, left=of h]                   (t) {$T$};
  \node[obs, right=of h]                   (y) {$Y$};
  \node[latent, right=of x]                 (u) {$U$};

  %
  \edge {x} {t} ; %
  \edge {x} {h}  ; %
  \edge {x} {y}  ; %
  \edge {t} {h}  ; %
  \edge {h} {y}  ; %
  \edge {u} {t}  ;
  \edge {u} {h}  ;

\end{tikzpicture}}
\caption{\centering \label{fig:confounding:compliance}Response\newline confounding}
\end{subfigure}\quad
\begin{subfigure}{0.22\linewidth}
\centering
\scalebox{0.7}{\begin{tikzpicture}

  %
  \node[obs, ]                               (x) {$X$};
  \node[latent, below=of x]                (h) {$H$};
  \node[obs, left=of h]                   (t) {$T$};
  \node[obs, right=of h]                   (y) {$Y$};
  \node[latent, right=of x]                 (u) {$U$};

  %
  \edge {x} {t} ; %
  \edge {x} {h}  ; %
  \edge {x} {y}  ; %
  \edge {t} {h}  ; %
  \edge {h} {y}  ; %
  \edge {u} {h}  ;
  \edge {u} {y}  ;

\end{tikzpicture}}
\caption{\centering \label{fig:confounding:effect}Effect\newline confounding}
\end{subfigure}\quad
\begin{subfigure}{0.22\linewidth}
\centering
\scalebox{0.7}{\begin{tikzpicture}

  %
  \node[obs, ]                               (x) {$X$};
  \node[latent, below=of x]                (h) {$H$};
  \node[obs, left=of h]                   (t) {$T$};
  \node[obs, right=of h]                   (y) {$Y$};
  \node[latent, right=of x]                 (u) {$U$};

  %
  \edge {x} {t} ; %
  \edge {x} {h}  ; %
  \edge {x} {y}  ; %
  \edge {t} {h}  ; %
  \edge {h} {y}  ; %
  \edge {u} {t}  ;
  \edge {u} {h}  ;
  \edge {u} {y}  ;

\end{tikzpicture}}
\caption{\centering \label{fig:confounding:total}Total\newline confounding}
\end{subfigure}
\caption{\label{fig:confounding}
The four types of latent confounding in the causal two-groups model.}
\end{figure*}

In this section, we consider how latent confounding impacts statistical inferences in the causal two-groups model. Latent confounding in the C2G model occurs when an unobserved variable $U$ affects two or more of the variables in $\{T, H, Y \}$.  \Cref{fig:confounding} shows the four types of possible latent confounding scenarios.

The key question is whether either of the C2G models proposed above remains \emph{well-specified} under any of these latent confounding scenarios. By well-specified, we mean the observed non-responder and treatment distributions are within the class of probability distributions specified by the modeling assumptions. A well-specified model under latent confounding ensures that integrating out the latent confounder in the untreated group yields the original non-responder distribution and in the treatment group yields the original mixture model,
\begin{equation}
\begin{aligned}
\int p(y | x, u, t=0) p(u) du & = f_0(y \mid x) = p(y | x, h=0, t=1) \, , \\
\int p(y | x, u, t=1) p(u) du &= (1-\pi(x)) f_0(y \mid x) + \pi(x) f_1(y \mid x) \, ,
\end{aligned}
\label{eqn:well-specified}
\end{equation}
where $(\pi, f_0, f_1)$ are specified by the modeling assumptions (i.e. parametric, semi-parametric, or nonparametric). If a model remains well-specified in the presence of latent confounding, statistical inferences will still be valid.

\subsection{Confounding in the additive model}
\label{subsec:confounding:additive}

The additive causal two-groups model requires that the $(f_0, f_1)$ distributions obey \cref{eqn:additive_errors}. Because \cref{eqn:additive_errors} is agnostic to the relationship between $T$ and $H$, it is straightforward to show that the additive causal two groups model remains correctly specified and applicable under response confounding.
\begin{proposition}
\label{prop:compliance-confounding-ac2g}
The additive causal two groups model is robust against response confounding.
\end{proposition}

The story changes when we allow an unobserved random variable to affect the output $Y$. Indeed, even when this variable only affects the conditional means of the non-responder and responder distributions by an additive shift while leaving the noise distribution unchanged, the model is misspecified.

\begin{proposition}
\label{prop:latent-nonrobust-ac2g}
The additive causal two-groups model is misspecified in the presence of an unobserved $U$ such that 
\begin{align*}
Y | X=x, U=u, H=h &=& \mu_{h}(x) + \rho_{h}(u) + \epsilon  \\
\epsilon &\sim& g.
\end{align*}
As a consequence, the additive causal two-groups model is not robust in the presence of canonical, effect, or total confounding.
\end{proposition}


\subsection{Confounding in the nonparametric model}
\label{subsec:confounding:nonparametric}


For the nonparametric method, there are only two modeling assumptions. First, the outcome distribution of the untreated samples must match that of the treated non-responders. Second, the outcome distribution of the treated samples can be written as a mixture model of the form given by \cref{eqn:treatment-mixture}.
The following result shows that this minimalist approach offers an added benefit over the additive model in the form of validity under effect confounding as well as response confounding.

\begin{proposition}
\label{prop:confounding-npc2g}
The nonparametric causal two-groups model is robust against response and effect confounding.
\end{proposition}

The reason why the nonparametric causal two-groups model remains well-specified in the presence of these two types of confounding is that both still allow for $Y \indep T \mid X,  H$. When this conditional independence relationship holds, the non-responder outcome distribution coincides with the untreated outcome distribution. This allows us to write the treatment distribution as the mixture in \cref{eqn:treatment-mixture}, thereby satisfying the conditions of \cref{eqn:well-specified}. When we do not have $Y \indep T \mid X,  H$, we cannot guarantee that the nonparametric causal two-groups model is well-specified, as the following result shows.

\begin{proposition}
\label{prop:canonical-confounding}
The nonparametric causal two-groups model is not robust in the presence of canonical or total confounding.
\end{proposition}

The key to proving \cref{prop:canonical-confounding} is in showing that a canonical confounder $U$ can lead to situations in which the untreated outcome distribution $p(Y \mid X, T=0)$ differs from the treated-but-non-responder outcome distribution $p(Y \mid X, H=0, T=1)$, breaking the fundamental assumptions of the nonparametric causal two-groups model.


