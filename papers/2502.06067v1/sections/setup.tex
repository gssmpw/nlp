We start by describing the available data. After briefly reviewing classic well-specified linear regression, we set up the misspecified case with different target and source data, and we establish our estimand in this case.

\textbf{Data.}
Following the covariate-shift literature \citep{bendavid_2006_analysis,pan_2010_transfer,Csurka2017}, we refer to our fully observed (training) data as the \emph{source} data; likewise, we let \emph{target} data denote the (test) locations and covariates where we do not observe the response but would like to understand the association between covariates and response. In particular, the source data consists of $N$ triplets $(S_n, X_n, Y_n)_{n=1}^N$, with spatial location $S_n \in \spatialdomain$, covariate $X_n \in \RR^P$, and response $Y_n \in \RR$. Here $P \ge 1$, and $\spatialdomain$ represents geographic space; we assume $\spatialdomain$ is a metric space with metric $d_{\spatialdomain}$. We collect the source covariates in the matrix $X \in \RR^{N \times P}$ and the source responses in the $N$-long column vector $Y$.
The target data consists of $M$ pairs $(\Sstar_m, \Xstar_m)_{m=1}^M$, with $\Sstar_m \in \spatialdomain$, $\Xstar_m \in \RR^{P}$. The corresponding responses $\Ystar_m \in \RR, 1 \leq m \leq M$ are unobserved. We collect the target covariates in $X^{*} \in \RR^{M \times P}$ and responses in column vector $Y^{*} \in \RR^{M}$.

\textbf{Review: Well-specified Linear Model.} Though we will focus on the misspecified (spatial) case, we start by reviewing the classic well-specified case for comparison purposes.

In the classic well-specified setup, we have
\begin{align}\label{eqn:ols-model}
Y_n = \theta_{\text{OLS}}^{\transpose} X_n + \epsilon_n, \quad
\Ystar_m = \theta_{\text{OLS}}^{\transpose} \Xstar_m + \epsilon^{\star}_m
\end{align}
with column-vector parameter $\theta_{\text{OLS}} \in \RR^P$ and $\epsilon_n, \epsilon^{\star}_m \stackrel{iid}{\sim} \mathcal{N}(0, \sigma^2)$ for some (unknown) $\sigma^2 > 0$.
For any fixed set of source data points (and assuming invertibility holds as needed), we can recover the parameter exactly as
\begin{align}\label{eqn:ols-estimand}
    \theta_{\text{OLS}} &= \arg\min_{\theta \in \RR^P} \EE\big[\frac{1}{N}\sum_{n=1}^N (Y_n - \theta^{\transpose}X_n)\big] \\
    &=\EE[X^{\transpose}X]^{-1}\EE[X^{\transpose}Y]
    =(X^{\transpose}X)^{-1} X^{\transpose} \EE[Y].
\end{align}
An analogous formula holds for target points; $\theta_{\text{OLS}}$ is constant across covariate values in any case.
Since the population expectation is  unknown, analysts typically estimate $\theta_{\text{OLS}}$ via maximum likelihood. The standard estimator and confidence interval at level $\alpha$ for the $p$th coefficient, $\theta_{\text{OLS}, p}$, appear below in \cref{eqn:ols-point-estimate-and-ci}.
Under correct specification, the confidence interval is valid: that is, it provides nominal coverage. E.g., a 95\% confidence interval contains the true parameter 95\% of the time under resampling. 
\begin{align}
    \hat{\theta}_{\text{OLS}, p}  
    = e_p^{\transpose}(X^{\transpose}X)^{-1}X^{\transpose}Y, \, I_{\text{OLS}, p} = \hat{\theta}_{\text{OLS}, p} \pm z_{\alpha}\rho, \label{eqn:ols-point-estimate-and-ci}
\end{align}
where $\rho^2 = \sigma^2 e_p^{\transpose}(X^{\transpose}X)^{-1}e_p$, $z_{\alpha}$ is the $\alpha$ quantile of a standard normal distribution, and $e_p$ denotes the $P$-dimensional vector with a $1$ in entry $p$ and $0$s elsewhere. The noise variance $\sigma^2$ in \cref{{eqn:ols-point-estimate-and-ci}} is typically replaced with an estimate $\hat{\sigma}^2$. % = \frac{1}{N-P}\sum_{n=1}^N r_n^2$, where $r_n$ is the residual of the OLS fit on the $n$th source data point.

\textbf{Misspecified Spatial Setup: Data-Generating Process.} In what follows, we assume the data is generated as 
\begin{align}\label{eqn:general-dgp}
    Y_n = f(X_n, S_n) + \epsilon_n, \Ystar_m = f(\Xstar_m, \Sstar_m) + \epsilon^{\star}_m,
\end{align}
for a function $f$ that need not have a parametric form and with $\epsilon_n, \epsilon^{\star}_m \stackrel{iid}{\sim} \mathcal{N}(0, \sigma^2)$ for some (unknown) $\sigma^2 > 0$.

For ease of development, we will next make the simplifying assumption that source and target covariates are fixed functions of spatial location. Recall that all of the covariates in our examples (aerosol concentrations, proximity to highways, and air pollution) can be expected to vary spatially and be measured with minimal error. We similarly expect meteorological variables such as precipitation, humidity, and temperature to be reasonably captured by this assumption.\footnote{Conversely demographic covariates may more reasonably be thought of as noisy functions of space, and further work is needed to handle the noisy case.}
\begin{assumption}\label{assum:cov-fixed-fns}
    There exists a function $\chi: \spatialdomain \to \RR^P$ such that $\Xstar_m =\chi(\Sstar_m)$ for $1 \leq m \leq M$ and $X_n = \chi(S_n)$ for $1 \leq n \leq N$.
\end{assumption}

Under \cref{assum:cov-fixed-fns}, our data-generating process simplifies.
\begin{assumption}\label{assum:test-train-dgp}
    There exists a function $f : \spatialdomain \to \RR$ such that $\forall m \in \{1,\ldots,M\}, \Ystar_m = f(\Sstar_m) + \epsilon^{\star}_m$ and $\forall n \in \{1,\ldots,N\}, Y_n = f(S_n) + \epsilon_n$, where
    $\epsilon^{\star}_m, \epsilon_n \stackrel{iid}{\sim} \mathcal{N}(0,\sigma^2)$.
\end{assumption}

\textbf{Our Estimand.}
At a high level, our goal is to capture the relationship between a covariate and the response variable at target locations using data from source locations, while taking into account that these two sets of locations may differ. From this perspective, we can define our estimand as the parameter of the best linear approximation to the response, where ``best'' is defined by minimizing squared error over the target locations.
\begin{align}
    \label{eqn:test-set-least-squares}
    \TestParamOLS &= \arg\min_{\theta \in \RR^{P}}\EE\Big[\sum_{m=1}^M(\Ystar_m - \theta^{\top}\Xstar_m)^2|\Sstar_m\Big].
\end{align}
As in \citet{white_1980_usingleastsquares,buja_models_2019}, since the data-generating process may be non-linear, $\TestParamOLS$ is no longer constant like the well-specified case, but rather we expect it to vary as a function of the target locations.

As an aside, we note that an alternative estimand that is similar in spirit would treat the source and target locations as random, each with their own respective distribution; in this case, the estimand in \cref{eqn:test-set-least-squares} would lose the conditioning on $\Sstar_m$, and covariate-shift methods could apply. First, we note that we are not aware of approaches for constructing finite-sample confidence intervals using current covariate-shift methodology, so the question of valid confidence intervals remains open in any case. That being said, we choose to fix the source and target locations in our present work because, in the spatial applications of interest here, these locations might not each plausibly be sampled i.i.d.\ from a population. For instance, if source data arises from the United States Environmental Protection Agency (EPA), we know that EPA sensors are sited with other sensors in mind and thus are known to not be i.i.d. Also particular target locations, such as certain municipalities missing data, may be of interest; especially when there may be only a handful or even just one such target location, it's not clear that there is a meaningful population representation. So we henceforth use the conditional form of \cref{eqn:test-set-least-squares}.

To solve the minimization in \cref{eqn:test-set-least-squares}, it will be convenient to assume invertibility, as for OLS.
\begin{assumption}\label{assum:invertibility}
$\Xstart\Xstar$ is invertible.
\end{assumption}
With \cref{assum:cov-fixed-fns,assum:invertibility}, we can solve the minimization in \cref{eqn:test-set-least-squares} to find
\begin{align}
    \TestParamOLS &= (\Xstart\Xstar)^{-1}\Xstart\EE[\Ystar|\Sstar]. \label{eqn:test-set-conditional-ols}
\end{align}
See \cref{app:derivation-target-cond-ols} for a  derivation. To the best of our knowledge, the target-conditional estimand in \cref{eqn:test-set-conditional-ols} has not been previously proposed or analyzed in spatial linear regression.

To estimate $\TestParamOLS$ well, we need to estimate the conditional expectation $\EE[\Ystar \mid \Sstar]$ well. To hope to do so, we need to make regularity assumptions that make accurate estimation feasible. In classical and covariate-shift settings, these take the form of i.i.d.\ and bounded-density-ratio assumptions. Since we've seen that the classic assumptions are inappropriate in this spatial setting, we instead assume $f$ is shared across source and target (\cref{assum:test-train-dgp}) and not varying so quickly in space as to be difficult to learn from limited data.
\begin{assumption}\label{assum:lipschitz}
    The conditional expectation, $f$, is $L$-Lipschitz as a function from $(\spatialdomain, d_{\spatialdomain})  \to (\RR, |\cdot|)$. That is, for any $s, s' \in \spatialdomain$,
    %\begin{equation}
        $|f(s) - f(s')| \leq L d_{\spatialdomain}(s,s').$
    %\end{equation}
\end{assumption}
As an illustrative example of how this assumption might be satisfied, consider $\chi$ an $L_1$-Lipschitz function of the spatial domain, and $f(s)=\beta^{\transpose} \chi(s) + g(s)$ with $g$ a fixed, $L_2$-Lipschitz function of $\spatialdomain$. Then $f$ is a $(\|\beta\|_2L_1 + L_2)$-Lipschitz function of the spatial domain.

