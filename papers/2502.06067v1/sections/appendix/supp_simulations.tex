In this section, we present figures to visualize the data generating process for simulation experiments. In all experiments, an intercept is included in the regression as well as the covariates described.

\subsection{Data Generation for the Single Covariate Experiment}

\begin{figure*}
    \centering
    \includegraphics[width=\linewidth]{figures/simulation/linear_regression/two_dim_shift/data_plot.pdf}
    \caption{Spatial sites for the source (blue) and target (orange) data are shown in the left most plots for different values of shift used in generating the data. More extreme values of the shift parameter lead to larger biases in parameter estimation from the training data without adjustment. The third plot from the left shows the covariate surface, while the fourth shows the expected response at each spatial location.}
    \label{fig:two-dim-shift-data}
\end{figure*}

We show example datasets used in the simulation experiment with a single covariate and the simulation experiment in which we investigated the impact of Lipschitz constant on confidence interval width and coverage in \cref{fig:two-dim-shift-data}. The left most two panels show the distribution of source (blue) and target (orange) locations for two values of the shift parameter. Large positive values of the shift parameter lead to target distributions clustered to the top right, values close to zero lead to the target locations being approximately uniformly distributed and large negative values lead to target locations clustered to the bottom left. The third panel from the left in \cref{fig:two-dim-shift-data} shows the covariate plotted as a function of spatial location,
\begin{align}
    \chi(S^{(1)},S^{(2)}) = S^{(1)}+S^{(2)}.
\end{align}
The right most panel shows the conditional expectation of the response plotted as a function of spatial location,
\begin{align}
    \EE[\Ystar|\Sstar = (S^{(1)},S^{(2)})] = S^{(1)}+ S^{(2)} + \frac{1}{2}((S^{(1)})^2+(S^{(2)})^2).
\end{align}

The gradient of this conditional expectation is,
\begin{align}
    \begin{bmatrix}
    1 + S^{(1)} \\
    1+ S^{(2)}
    \end{bmatrix}
\end{align}
which has norm
\begin{align}
    \sqrt{(1 + S^{(1)})^2 + (1+ S^{(2)})^2}. 
\end{align}
This obtains a maximum of $2\sqrt{2}$ on $[-1,1]^2$, and so this is the Lipschitz constant of $f(S) = \EE[Y|S]$.


\subsection{Data Generation for the Three Covariate Experiment}
In \cref{fig:two-dim-shift-trig-data} we show data generated for the three covariate shift experiment. Source and target locations are generated as in the one covariate simulation, but with $N=10{,}000$ ($M=100$ is still used). These are not shown in \cref{fig:two-dim-shift-trig-data}. 

The covariates are, from left to right in \cref{fig:two-dim-shift-trig-data}
\begin{align}
    X^{(1)} &= \sin(S^{(1)}) + \cos(S^{(2)}) \\
    X^{(2)} &= \cos(S^{(1)}) - \sin(S^{(2)}) \\
    X^{(3)} &= S^{(1)} + S^{(2)}.
\end{align}
The conditional expectation of the response (right most panel in \cref{fig:two-dim-shift-trig-data}) is 
\begin{align} 
    \EE[Y|S=(S^{(1)},S^{(2)}] &= X^{(1)}X^{(2)} + \frac{1}{2}((S^{(1)})^2 + (S^{(2)})^2) \\
    &= (\sin(S^{(1)}) + \cos(S^{(2)}))(\cos(S^{(1)}) - \sin(S^{(2)})) + \frac{1}{2}((S^{(1)})^2 + (S^{(2)})^2).
\end{align}
The gradient of this conditional expectation is 
\begin{align}
    \begin{bmatrix}
    \cos(2S^{(1)})-\sin(S^{(1)} + S^{(2)}) + S^{(1)} \\
    - \cos(2S^{(1)}) + \sin(S^{(1)} - S^{(2)}) + S^{(2)}
    \end{bmatrix}.
\end{align}
We see that both arguments of this gradient are less than $3$ in absolute value, and therefore the norm of this Lipschitz constant is less than or equal to $\sqrt{3^2 + 3^2} = 3\sqrt{2}$.
\begin{figure*}
    \centering
    \includegraphics[width=\linewidth]{figures/simulation/linear_regression/two_dim_shift_trig/data_plot_new.pdf}
    \caption{The first 3 plots from the left show the covariate surfaces, while the fourth shows the expected response at each spatial location for the second simulated experiment. The source and target locations (not shown) are the same as in \cref{fig:two-dim-shift-data}, though with $N=10{,}000$.}
    \label{fig:two-dim-shift-trig-data}
\end{figure*}