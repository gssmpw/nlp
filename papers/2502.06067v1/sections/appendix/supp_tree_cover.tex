In this section we provide additional details for the real data tree cover experiment, as well as figures to visualize the data and additional experimental results. 

\subsection{Tree Cover Data}

Our analysis draws on data from \citet{lu2024quantifying}, who manually labeled 983 high-resolution Google Maps satellite images for tree cover percentage. These images were selected from random locations within the 2021 USFS Tree Canopy Cover (TCC) product \citep{usfs2023treecover}. We follow \citet{lu2024quantifying} and use three covariates: 
\begin{enumerate}
    \item \textit{Global Aridity Index (1970--2000)}: Averaged at a 30 arc-seconds resolution \citep{trabucco2019global}. This index is calculated as the ratio of precipitation to evapotranspiration, with lower values indicating more arid conditions.
    \item \textit{Elevation}: Provided by NASA’s 30-meter resolution dataset \citep{nasadem2020}.
    \item \textit{Slope}: NASA’s 30 m Digital Elevation Model. Also provided by NASA’s 30-meter resolution dataset \citep{nasadem2020}.
\end{enumerate}

\cref{fig:tree-cover-data} provides a visual overview of both the tree cover and covariates.  
As a preprocessing step, we convert the (latitude, longitude) coordinates of each data point into radians. This allows us to use the Haversine formula to compute distances in kilometers for the Wasserstein-1 cost and the nearest neighbor weighting procedure.


Elevation and slope are important factors influencing tree cover worldwide. Generally, areas at lower elevations tend to have more tree cover, often due to higher temperatures \citep{mayor2017elevation}, and sloped terrains also support greater tree coverage \citep{sandel2013human}. As done in \citet{lu2024quantifying}, we focus on these three covariates and did not include additional factors that might affect tree cover. This decision was made because our primary objective is to demonstrate uncertainty quantification rather than to provide a comprehensive explanation of tree cover dynamics.

\cref{fig:tree-cover-data} provides a visual overview of the distribution for both the tree cover and covariates. 

\begin{figure*}[!ht]
    \centering
\includegraphics[width=\linewidth]{figures/tree_cover/tree_data.pdf}
    \caption{Tree cover response and covariates. The dots represent the 983 locations considered. Top left: distribution of tree cover percentage. Top right: Average Aridity Index, measured as the ratio of precipitation to evapotranspiration. Bottom left: Elevation, measured in meters. Bottom right: Slope, measured in degrees.}
    \label{fig:tree-cover-data}
\end{figure*}

\subsection{Source and Target Data Split and Spatial Preprocessing}
We define our target region to be the Western portion of the Continental United States. In particular, we consider locations that have latitudes between $25^\circ$ and $50^\circ$ and longitudes between $-125^\circ$ and $-110^\circ$. Within spatial locations in this defined region, we pick $50\%$ of all spatial locations --- totaling 133 sites --- as \textit{target} data.

To select the \textit{source} data, we perform the following steps:
\begin{enumerate}
    \item Consider all the remaining spatial locations, i.e. exclude the 133 target points from the pool of 983 spatial points. This leaves us with 850 points.
    \item Since we are interested in evaluating whether our method and the baselines achieve nominal coverage, uniformly randomly sample $20\%$ of the remaining locations across 250 different random seeds. By doing this, for each random seed we have 170 source locations.
\end{enumerate}

\cref{fig:tree-cover-split-west} visually represents the spatial distribution of the source and target locations for one representative random seed.

As a preprocessing step, we also convert the geographical coordinates (latitude and longitude) of each data point from degrees to radians. This conversion is essential because it allows us to apply the Haversine formula, which calculates the great-circle distance between two points on the Earth's surface in kilometers, to compute distances in kilometers for the Wasserstein-1 cost and the nearest neighbor weighting procedure.

\begin{figure*}
    \centering    \includegraphics[width=0.97\linewidth]{figures/tree_cover/split_plot_west.pdf}
    \caption{Split of the tree cover dataset in a target distribution in the West of the United States. Target locations are shown in orange, while source locations are shown in blue.}
    \label{fig:tree-cover-split-west}
\end{figure*}

\subsection{Estimating the Ground Truth Parameters to Evaluate Coverage}
\label{app:real-data-coverage-computation}

In order to evaluate coverage, we repeat the data subsampling process described above 250 times. Ideally, we would estimate the coverage as the proportion of these seeds in which the estimand $\TestParamOLSp$ falls inside the confidence interval we construct for each method. However, we cannot evaluate $\TestParamOLSp$ directly, even though we have access to the target responses. To account for sampling variability, we use our method and each baseline to construct a confidence interval for the difference,
\begin{align}
    \hat{\theta}_p^{\star} - \hat{\theta}_p
\end{align}
where $\hat{\theta}_p^{\star} = e_p^{\transpose}(\Xstart \Xstar)^{-1}\Xstart\Ystar$ is the estimated parameter using the target data (which our method and baselines don't have access to) and $\hat{\theta}_p$ is the estimated parameter we compute with each method using the source responses. If this confidence interval contains $0$, we count the method as having covered the true parameter, while if it doesn't we count the method as not having covered the true parameter. We estimate the variance of $\hat{\theta}_p^{\star}$ using the model-trusting standard errors $\hat{\sigma}^2 = \frac{1}{N-P} \sum_{n=1}^N r_n^2$ where $r_n$ are the residuals of the model fit on the training data. We expect this to inflate our estimate of the standard variance of the target OLS estimate if the response surface is nonlinear, as the residuals will be larger due to bias. By possibly overestimating and incorporating this sampling variability into confidence intervals, we expect the calculated coverages to overestimate the true coverages. The resulting coverages are shown in the top row of \cref{fig:tree-cover-coverage-main-west}.

In \cref{fig:tree-cover-coverage-both-west} we also report confidence intervals by calculating the proportion of times that $\hat{\theta}_p^{\star}$ is contained in each confidence interval. $\hat{\theta}_p $ is an unbiased estimate for $\TestParamOLSp$ whether or not the model is well-specified. However, we might expect that the coverages reported with this approach underestimate the actual coverage of each method's confidence intervals due to not accounting for sampling variability in $\hat{\theta}_p^{\star}$.

\begin{figure*}[!ht]
    \centering    \includegraphics[width=0.97\linewidth]{figures/tree_cover/coverages_by_dim_both_west.pdf}
    \caption{Coverages for the difference (upper), coverages for the point estimate (middle), and confidence interval widths (lower) for our method as well as 5 other methods for the West US data. Each column represents a parameter in the tree cover experiment. Only our method consistently achieves the nominal coverage.}
    \label{fig:tree-cover-coverage-both-west}
\end{figure*}

\subsection{Experiment with Varying Lipschitz Constant}
\label{app:tree-cover-varying-lipschitz}

In this section, we provide an additional experiment with these real data where the focus is to assess how varying the underlying Lipschitz constant in \cref{assum:lipschitz} changes coverage and interval width. For this experiment we consider 9 different values for the Lipschitz constant, $L \in \{0.001, 0.005, 0.01, 0.02, 0.05, 0.1, 0.2, 0.5, 1\}$. This range of values is a reasonable range of values that a practitioner with domain knowledge in the tree cover field might be willing to specify. Indeed, the smallest Lipschitz constant considered, $0.001$, corresponds to assuming that to have a $1\%$ increase in tree cover we need to move $1/0.001 = 1000$ kilometers. This is quite an extreme value, but in some parts of arid regions such as New Mexico and Arizona it can be true. On the other hand, the largest Lipschitz constant considered, $1$, corresponds to assuming that to have $1\%$ increase in tree cover we need to move $1$ kilometer. This is also extreme, but in regions in the US where elevation changes are very pronounced (e.g. in the Colorado Rockies), tree cover can vary sharply over very short distances. 

In \cref{fig:tree-cover-multiple-lipschitz}, we present the results of this experiment. We find that varying the Lipschitz constant within this range does not significantly impact coverage. Specifically, for \textit{slope} and \textit{elevation}, coverage remains consistent across all constants except $L = 0.001$. For \textit{aridity index}, coverage is low for $L \leq 0.1$ but exceeds $95\%$ for $L \geq 0.2$. Meanwhile, confidence intervals become noticeably wider for $L > 0.5$ while remaining relatively stable for smaller values.


\begin{figure*}
    \centering
    \includegraphics[width=\linewidth]{figures/tree_cover/coverages_by_dim_lipschitz_west.pdf}
    \caption{Coverage and average confidence interval widths over 250 seeds for 9 different values for the Lipschitz constant $L \in \{0.001, 0.005, 0.01, 0.02, 0.05, 0.1, 0.2, 0.5, 1\}$. The horizontal axis is shared across all plots.}
    \label{fig:tree-cover-multiple-lipschitz}
\end{figure*}

\begin{figure*}
    \centering    \includegraphics[width=0.97\linewidth]{figures/tree_cover/confidence_interval_plot_west.pdf}
    \caption{Confidence intervals for different seeds for the West US. Each row shows confidence intervals for the various methods over the three parameters for a given seed. The dashed vertical lines represent the true parameters (black is the point estimate, orange is a 95\% confidence interval). The blue dots the point estimates for the different methods, and the blue lines are the confidence intervals.}
    \label{fig:tree-cover-conf-intervals-west}
\end{figure*}


\begin{figure*}
    \centering
    \includegraphics[width=\linewidth]{figures/tree_cover/split_plot_southeast.pdf}
    \caption{Spatial sites for the source (blue) and target (orange) data. The target data are chosen from the south-eastern part of the CONUS, whereas the source data cover the whole region.}
    \label{fig:tree-cover-split-southeast}
\end{figure*}

\subsection{Additional Experiment: Target South-East US}
\label{app:tree-cover-southeast}

In this experiment, we define our target region in the Southeastern portion of CONUS at locations with latitude in the range (25, 38) and longitude in the range (-100, -75). Out of all spatial points in this region, $50\%$ --- totaling 118 sites --- are designated as target data. Next, we select the source data by taking a uniform random sample of $20\%$ of the remaining spatial locations, repeated over 250 random seeds to assess coverage performance. Each seed yields 173 source locations. \cref{fig:tree-cover-split-southeast} illustrates the spatial split between source and target data for a representative seed. As before, as a preprocessing step we convert the (latitude, longitude) coordinates of each data point into radians. 

\paragraph{Results.} We report the results for the confidence interval coverage and width in \cref{fig:tree-cover-coverage-both-southeast}. As before, our method consistently achieves or exceeds 95\% nominal coverage for all parameters. All competing methods but KDEIW here achieves $95\%$ nominal coverage (or close to $95\%$ nominal coverage) for the aridity index and slope parameter when considering the coverage for the $\hat{\theta_p^{\star}} - \hat{\theta}_p$ (top row). All competing methods fall short of the nominal threshold for the elevation parameter. The wider intervals produced by our method (last row) reflect the trade-off between achieving reliable coverage and maintaining narrower intervals.

In the middle row we show the coverage for the point estimate $\hat{\theta_p^{\star}}$. Here we see how our method is the only one that achieves nominal coverage for all the parameters. In particular, all the other methods do not achieve nominal coverage for any of the parameters.

\paragraph{Varying Lipschitz constant.} Finally, we report also for this experiment results when varying the assumed Lipschitz constant. As explained in \cref{app:tree-cover-varying-lipschitz}, we consider 9 different values for the Lipschitz constant, $L \in \{0.001, 0.005, 0.01, 0.02, 0.05, 0.1, 0.2, 0.5, 1\}$. We report the results in \cref{fig:tree-cover-multiple-lipschitz-southeast}. As before, we find that varying the Lipschitz constant within this range does not significantly impact coverage. Specifically, here we see that for \textit{aridity index} and \textit{elevation}, coverage remains consistent across all constants above $L = 0.01$. For \textit{slope}, $95\%$ nominal coverage is achieved for $L \geq 0.05$. And --- as before --- confidence intervals become noticeably wider for $L > 0.5$ while remaining relatively stable for smaller values.






\begin{figure*}
    \centering    \includegraphics[width=0.97\linewidth]{figures/tree_cover/coverages_by_dim_both_southeast.pdf}
    \caption{Coverages for the difference (upper), coverages for the point estimate (middle), and confidence interval widths (lower) for our method as well as 5 other methods for the Southeast US data. Each column represents a parameter in the tree cover experiment. Only our method consistently achieves the nominal coverage.}
    \label{fig:tree-cover-coverage-both-southeast}
\end{figure*}

\begin{figure*}
    \centering    \includegraphics[width=0.97\linewidth]{figures/tree_cover/confidence_interval_plot_southeast.pdf}
    \caption{Confidence intervals for different seeds for the South East US. Each row shows confidence intervals for the various methods over the three parameters for a given seed. The dashed vertical lines represent the true parameters (black is the point estimate, orange is a 95\% confidence interval). The blue dots the point estimates for the different methods, and the blue lines are the confidence intervals.}
    \label{fig:tree-cover-confidence-intervals-southeast}
\end{figure*}

\begin{figure*}
    \centering
    \includegraphics[width=\linewidth]{figures/tree_cover/coverages_by_dim_lipschitz_southeast.pdf}
    \caption{Coverage and average confidence interval widths over 250 seeds for 9 different values for the Lipschitz constant $L \in \{0.001, 0.005, 0.01, 0.02, 0.05, 0.1, 0.2, 0.5, 1\}$.}
    \label{fig:tree-cover-multiple-lipschitz-southeast}
\end{figure*}