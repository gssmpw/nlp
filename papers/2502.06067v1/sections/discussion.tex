We propose a new method for constructing confidence intervals for spatial linear models. We show via theory and experiments that our intervals accurately quantify uncertainty under misspecification and covariate shift. In experiments, our method is the only method that consistently achieves (or even comes close to) nominal coverage. We observe that, very commonly in spatial data analyses, covariates and responses may be observed at different locations in space. Since our method does not actually use the source covariate values in inference for $\TestParamOLS$, it can be applied in this common missing-data scenario. Though it requires additional work, we believe the ideas here will extend naturally to the widely used class of generalized linear models.

%linear regression with spatial datasets where covariates and responses are observed at different spatial locations.

% While we assume Gaussianity, we expect our approach can be generalized beyond given moment conditions on the noise. 

%the approach would be approximately valid without Gaussianity or even iid epsilon, provided some regularity (given some moment conditions and diffuse enough weights “v”)


%==========
%While we focus on inference, our approach could be adapted to provide confidence intervals for predictions. Additionally our approach can be used for any situation with covariate shift and misspecification in linear regression, even in non-spatial problems as long as the Lipschitz assumptions is reasonable. For future work, we plan to generalize our approach to consider logistic regression and other generalized linear models.
