Scientists often want to understand the sign and magnitude of an association in cases where covariates vary spatially. Examples include the relationship between aerosol concentrations and regional precipitation changes \citep{westervelt_connecting_2018}, 
the association between proximity to major highways and the prevalence of dementia \citep{li_relationships_2023}, and the link between air pollution exposure and birth weight \citep{lee_association_2022}. Here, the covariates (aerosol concentrations, proximity to highways, and air pollution) may all be thought of as a function of spatial location. Moreover, scientists often have access to data at some spatial locations but want to infer the association of interest at other locations. For instance, a country might measure birth weight and air pollution at the municipal level for some municipalities --- and be interested in understanding the association between air pollution and birth weight in unobserved municipalities.

A priori we expect birth weight as a response to have a potentially nonlinear relationship to air pollution as a covariate. So a natural idea is to fit a flexible, nonparametric model, such as a neural network, Gaussian process, or random forest. While these methods can be expected to achieve high predictive accuracy, they often lack interpretability, and it is not clear how to draw conclusions about the specific relationships between covariates and the response variable \citep{rudin2019stop,DoshiVelez2017TowardsAR}.

Another option, made precise by \citet{buja_models_2019}, is to recognize the potentially nonlinear relationship between covariates and response, but fit a linear model to the data essentially to summarize any trend. Namely, one can use the parameter estimate as a best guess of the association trend between covariates and response, and a confidence interval captures uncertainty in this trend. And indeed, linear models remain widespread to the present day despite an understanding that they will generally be misspecified. All of the applied papers above use linear regression. Moreover, \citet[Figure 1]{castro_torres_use_2024} conducted a survey of publications up to 2022 that report quantitative methods; they found that over half of all such papers in various areas (agricultural science, social sciences, and medical and health sciences) reported results from a linear model.

Providing a valid confidence interval in this setting remains, however, an open problem. Namely, we are interested in the common setting where the linear model is misspecified, and we would like to understand associations at spatial locations that may be different from those in the observed data. Under misspecification, \citet{buja_models_2019} observes that the ordinary least squares (OLS) estimator now depends on the covariate values, and classic confidence intervals are valid only at exactly the set of observed (training) covariate values. The heteroskedasticity-consistent sandwich estimator \citep{huber_behavior_1967, white_heteroskedasticity-consistent_1980,white_1980_usingleastsquares, buja_models_2019} provides valid confidence intervals under misspecification, but then we need to assume that all covariates are samples from a single population. Generalized least squares linear regression (GLS, \citealt{aitken1936iv}) is designed to handle spatial correlation in residuals \citep[pages 22-24]{cressie2015statistics}. But there is no reason to expect that GLS solves the bias issues that arise in OLS under a combination of misspecification and distribution shift. Spatial data analysts often report Bayesian credible intervals, e.g.\ with a Gaussian process model, but these are known to underestimate uncertainty in the simultaneous presence of misspecification and distribution shift \citep{walker2013bayesian, muller_2013_risk}. Finally, importance weighting methods \citep{shimodaira_improving_2000} have two drawbacks. (1) They assume one is able to de-bias the estimator, which need not be true. (2) Confidence interval validity relies on exact estimation of importance weights, which will never be perfectly achieved in practice. We discuss related work in more detail in \cref{app:related-work}.

In what follows, we show in both real and simulated experiments (\cref{sec:experiments}) that all of the confidence-interval constructions above can provide (much) less than nominal coverage. \textbf{Our principal contribution} is to provide the first confidence intervals for spatial linear regression that are guaranteed to achieve frequentist coverage at the nominal level (a) under nonparametric assumptions about the data-generating distribution and (b) when the target locations need not match the observed data. Along the way, we formalize the problem of estimation in this setting. In our experiments, our method is (by far) the only method to consistently achieve nominal coverage.

% we care about linear regression
%Linear models remain widely used in modern spatial applications due to their interpretability. For instance, in climate science, linear regression has been employed to quantify the relationship between aerosol concentrations and regional precipitation changes \citep{westervelt_connecting_2018}. Similarly, in public health research, linear models have been used to investigate the association between proximity to major highways and the prevalence of dementia \citep{li_relationships_2023}, as well as to explore the link between air pollution exposure and birth weight \citep{lee_association_2022}. \Citet[Figure 1]{castro_torres_use_2024} conducted a review finding that over half of the publications that report using quantitative methods for data analysis in agricultural science, social sciences, and medical and health sciences report results from a linear model.  


% We care about uncertainty
%In all of these applications, quantifying uncertainty is important when communicating findings. Often, this uncertainty is expressed in the form of confidence intervals. For example, when studying the link between proximity to highways and dementia, researchers might report a point estimate for the slope describing the relationship between the distance to the nearest major highway and the measured mass of specific brain matter types observed in MRIs \citep{li_relationships_2023}. Along with this point estimate, it is customary to report a confidence interval, an interval that is meant to contain the parameter we are trying to estimate with high probability. 

% Problem: Confidence intervals don't work
%However, the probabilistic guarantees accompanying commonly confidence intervals require assumptions about the data generating process that are often violated in spatial problem. In particular, typically these guarantees require that the relationship between the measured covariates and response is actually linear (i.e., the model is well-specified), or that the covariates are sampled independent and identically distributed (i.i.d.) from some population, or both. In \cref{sec:experiments}, we illustrate with simulations that these guarantees do not hold when models are both misspecified and this i.i.d.~assumption on covariates does not hold.

%Misspecification is inevitable in essentially all statistical problems: very few relationships are truly linear. Under misspecification, the classical confidence intervals for ordinary least squares (OLS) are no longer valid unless we are interested in modeling the relationship at the exact observed covariate values \citep{buja_models_2019}. The heteroskedasticity-consistent sandwich estimator \citep{huber_behavior_1967, white_heteroskedasticity-consistent_1980,white_1980_usingleastsquares, buja_models_2019} provides valid confidence intervals under misspecification if we assume that the covariates are sampled from some population, and the goal is to model the covariate-response relationship for that population.  But in spatial modeling tasks, often, data are collected from a limited set of geographical sites --- such as air quality monitors or the residences of individuals enrolled in a study --- yet these sites rarely form a random sample from the underlying population. Moreover, we might actually be interested in the association between the response and covariates at a \emph{different} set of spatial locations. The combination of misspecification and this mismatch in covariate distributions --- commonly known as covariate shift --- can introduce bias into the estimation procedure \citep{hodges2010adding, paciorek_importance_2010, dupont2023demystifying}.

% Solution 1:
%Generalized least squares linear regression (GLS, \citealt{aitken1936iv}), which allows for spatial correlations in the residuals, is often proposed as a solution to this problem \citep[pages 22-24]{cressie2015statistics}. If the model is well-specified, GLS is efficient. However, in the presence of spatial confounding, GLS is biased \citep{schnell2020mitigating, page_estimation_2017} and may no longer be efficient. Also, the GLS approach does not account for biases introduced due to distribution shift, which, as we have already argued, is an important consideration in many spatial tasks. 

% SOlution 2:
%Another method that is widely used to address the misspecification and covariate shift problem is Bayesian linear regression with an additive, spatial Gaussian process. However, this approach has similar shortcomings to GLS. Under well-specification of the likelihood the credible intervals can account for correlated error structures. And these intervals are still valid under covariate shift. However, when there is both misspecification of the likelihood and covariate shift, the resulting credible intervals do not capture the actual amount of uncertainty we should have about parameters of interest \citep{walker2013bayesian}.

% Solution 3: Importance weighting
%Importance weighting, reweighting the available training examples to make them more representative of the locations at which we want to apply the problem, can be used to attempt to debias the estimation procedure. These leads to a weighted least squares, and confidence intervals can be computed similarly to OLS. However, these confidence intervals do not account for errors in estimating the importance weights. Perhaps more troubling, they assume we can successfully debias the estimator. But in many spatial settings, where some amount of extrapolation is desired, the estimation procedure cannot be fully debiased. 

%\paragraph{Our Contribution.}
%In this work, we formalize the problem of estimation of a regression coefficient at a fixed set of spatial locations. We modify OLS to provide a point estimate for this estimand. We provide the first confidence intervals confidence intervals for OLS in spatial problems that are guaranteed to achieve frequentist coverage at the nominal level under non-parametric assumptions about the data-generating process under covariate shift. We support the claim that our method has nominal coverage with simulated and real world experiments in \cref{sec:experiments}. In particular, we show that --- when compared to relevant baselines --- our method is the only one who consistently achieves nominal coverage for all the experiments. 

% Solution 1: Generalized least squares
% Spatial regression approaches often rely on more flexible modeling assumptions, for example a Gaussian process random effect, to address latent, spatial structure in the problem. This leads to a generalized least squares approaches \citep{aitken1936iv}, which allow for spatial correlations in the residual. 
% We often would like to describe relationships between variables that exhibit spatial dependence. For example, we might be interested in  

% Or we might be interested in the association between air pollution and birth-weight .

% Often, data are collected from a limited set of geographical sites --- such as air quality monitors or the residences of individuals enrolled in a study --- yet these sites rarely form a random sample from the underlying population. 

% Moreover, we might actually be interested in the association between the response and covariates at a \emph{different} set of spatial locations. Unless the regression model is well-specified (which is essentially never the case in practice), this mismatch in covariate distributions --- commonly known as covariate shift --- can introduce bias into the estimation procedure \citep{hodges2010adding, paciorek_importance_2010, dupont2023demystifying}.


% And the models we use to describe these associations never adequately capture all aspects of the problem, and so we expect mis-specification in whatever model we use. As well as estimating the association between variables, we generally want to report our certainty about this association, through confidence or credible intervals.  

% Solutions: 
% A familiar method for describing associations between variables is ordinary least squares linear regression (OLS, \citealt{seber2012linear}). However, in spatial settings, ordinary least squares has been criticized for not accounting for spatial correlations in residuals that are not described by the observed covariates \citep[pages 20-22]{cressie2015statistics}. Generalized least squares linear regression (GLS, \citealt{aitken1936iv}), which allows for spatial correlations in the residuals, is often proposed as a solution to this problem \citep[pages 22-24]{cressie2015statistics}. Under correct specification and with independence assumption GLS is efficient. However, in the presence of spatial confounding GLS is biased \citep{schnell2020mitigating, page_estimation_2017} and may no longer be efficient. 

% More troubling, the confidence intervals reported for OLS and GLS do not account for bias introduced by a combination of mis-specification and distribution shift. This means that the confidence intervals generally due not reflect the actual level of uncertainty around the quality of estimation, and fail to achieve the nominal coverage. 

% An alternative approach is to perform Bayesian inference in a regression model that explicitly accounts for spatial structure. However, when the likelihood is mis-specified, as is almost always the case, the resulting credible intervals also fail to reflect the degree of uncertainty we should have about the association between variables \citep{walker2013bayesian}.

%\DRB{Do we need to discuss importance weighting? I'm not sure if I've seen people actually do this in spatial linear regression, but its a natural thing to try and someone likely has.}

% What We do
% We provide a modified point estimate for OLS and GLS in spatial settings, together with confidence intervals for this point estimate. Importantly, our confidence intervals account not only for variance, but also for potential bias due to mis-specification and distribution shift. We show that under the assumptions that the covariates are fixed functions of location and that the conditional expectation of the loss is a smooth function of location, our confidence intervals have frequentist coverage at the nominal level.  

% We support this claim with simulated and real world experiments in \cref{sec:experiments}. In particular, we show that --- when compared to relevant baselines --- our method is the only one who consistently achieves nominal coverage for all the experiments. 


%This introduces a form of distribuiton shift 

% Linear regression is used ubiquitously with spatial data. For example, FIND CONCRETE EXAMPLES WHERE GWR AND SPATIAL REGRESSION ARE USED. However, the direct application of ordinary least squares on spatial data is known to lead to unreliable estimates of the relationship between spatially structured covariates and response variables because of unaccounted for spatial structure in the problem CITATION, [maybe (https://arxiv.org/pdf/2308.12181)]. Spatial generalized leas squares (SGLS) (CITATION?) offers an alternative approach, modeling the response as linear in the covariates, with correlated, Gaussian process noise. The resulting estimator is consistent in setting when data is available densely in space [maybe (https://arxiv.org/pdf/2308.12181)]. However, uncertainty estimates for the coefficients generally either rely on well-specification of the model (for example, if a Bayesian approach is taken and credible intervals are computed) or on an i.i.d.~sampling mechanism for spatial locations, which is often not reasonable in spatial settings. We show that SGLS can lead to biased estimates of parameter values when both of these assumptions are violated, and that this bias isn't properly accounted for in the computed confidence (credible) intervals.

% A second approach to regression in spatial settings is to abandon the idea of describing global relationships between covariates and responses entirely, and focus on local relationships. Geographically weighted regression (GWR) is a non-parametric approach for this, in which coefficient values for the parameter are defined at each point in the spatial domain. However, global summaries of the relationship between covariates and responses can be useful in cases where decisions must be made that effect an entire region of space. And even in cases where local summaries of the relationship are desirable, GWR does not have a mechanism for accounting for bias in estimates due to extrapolating from available training data. 

% \paragraph{Our Contribution.}
% We propose a framework for inference in linear regression with spatially-correlated data where the goal is to understand the relationship between covariates and responses at a known collection of points in space that may differ in important ways from the data available for inference. We make regularity assumptions on the data generating process, that the conditional expectation of the response varies smoothly (is Lipschitz) as a function of the space, and noise is additive, homoskedastic and Gaussian. Under these assumptions, we provide finite sample valid confidence intervals for regression coefficients that account for biases due to irregular sampling and mis-specification of the linear model.  Finally, we provide a consistent estimator for the variance of this noise. 