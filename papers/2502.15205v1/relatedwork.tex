\section{Related Work}
\label{background}
\subsection{Inspiration}
One of the first figurative definitions of inspiration can be found in Oxford English Dictionary as: ``A breathing in or infusion of some idea, purpose, etc. into the mind; the suggestion, awakening, or creation of some feeling or impulse, especially of an exalted kind'' \cite[p. 1036]{simpson_oxford_1989}. Researchers in various domains have investigated the meaning and role of inspiration for several years. From a psychological perspective, Thrash and Elliot \cite{thrash_inspiration_2003} suggested a domain-agnostic conceptualization of inspiration, noting that it has three characteristics: (1) \textit{motivation} is implied within inspiration, (2) inspiration is something that is \textit{evoked} rather than made into existence by chance or act of will, and (3) inspiration involves \textit{transcendence} in the sense that it goes beyond ordinary human actions, concerns, and cognitive processes. In that sense, according to Thrash and Elliot, when a stimulus object \textit{evokes} inspiration in a person, they gain some sort of awareness that goes beyond or \textit{transcends} their regular perspective, and lastly \textit{motivates} them to realize this new awareness into something new. In subsequent work, Thrash and Elliot \cite{thrash_inspiration_2004} argued that the inspiration process has two distinct components, to which they referred as being inspired \textit{by} and being inspired \textit{to}. The more passive nature of inspired \textit{by} consists of a person appreciating the value of some sort of stimulus object (e.g., being inspired by a beautiful flower and appreciating its beauty). On the other hand, the more active nature of inspired \textit{to} consists of a person acting upon this appreciative feeling (e.g., having the desire to design a visualization based on a flower that they saw, as reported in \cite{parsons_fixation_2021}). According to their earlier conceptualization, inspired \textit{by} falls upon the evocation and transcendence characteristics of inspiration, where inspired \textit{to} falls upon the motivation characteristic of inspiration. 

From a design perspective, Gon\c{c}alves and colleagues \cite{goncalves_what_2014} make a similar observation, suggesting that inspiration processes can be more active (e.g., purposefully searching for information) or more passive (e.g., stumbling upon useful information without purposefully searching for it). For example, seeking out temporal data visualizations exemplifies an active inspiration process. In a different scenario, stumbling across a temporal visualization while browsing social media exemplifies a passive inspiration process. In their later work, Gon\c{c}alves and colleagues \cite{goncalves_inspiration_2016} made further distinctions, suggesting that there are 4 types of inspiration processes: active search with purpose, active search without purpose, passive search, and passive attention. Active search with purpose involves a purposeful search for inspiration with a specific goal in mind. For example, looking for specific visualization examples (e.g., treemap) can be an example of an active search with purpose. Active search without purpose refers to a search with similar nature to active search with purpose, but without a specific goal in mind. For example, consulting or going through a common source of visualization inspiration (e,g., visualization blog), without having a specific project task at hand, can be considered as an example of an active search without purpose. Wilson \cite{wilson_information_1997} suggests that this type of search is used to keep one's knowledge refreshed or expanded. Passive search involves an unintentional and random encounter with relevant information that is later integrated into the design problem at hand. For example, encountering an interesting visualization example while browsing social media could be considered as an example of a passive search. Lastly, passive attention involves similar scenario to passive search, but without integration into the design problem at hand. Thus, passive attention is an inspiration that is randomly encountered, but is not immediately used (but could be used in future). In our interviews, we divided these 4 types into two types initially proposed by Gon\c{c}alves and colleagues (active and passive). 

In design, inspiration has been referred to as a ``process that can integrate the use of any entity in any form that elicits the formation of creative solutions for existing problems'' \cite[p. 29]{goncalves_what_2014}. Gon\c{c}alves and colleagues \cite{goncalves_inspiration_2016} conceptualized the inspiration process in design as cyclical and iterative. Typically, the process is initiated by the designer, where afterwards designers use inspirational materials as a starting point. These materials then get adapted to the design process or discarded and this process continues in cycles until the design problem is re-framed or solved. Even though inspiration has been conceptualized and received attention from researchers in various domains, how it is used in professional design practice remains largely unexplored \cite{scolere_digital_2021}.

\subsection{Inspiration and Creativity}
Inspiration has often been discussed in relation to creativity in the design literature. Even though there is no widely agreed upon definition of creativity \cite{silvia_creativity_2010}, most researchers agree that for solutions to be considered "creative", they have to be novel \textit{and} useful \cite{feist_meta-analysis_1998, sarkar_assessing_2011}. Inspiration is essential for creative performance in practically any design profession and researchers have long noted its significance in creativity (e.g., \cite{eckert_sources_2000, goncalves_what_2014, herring_getting_2009, koronis_crafting_2021}), sometimes even using these two words as interchangeable (e.g., \cite{chamorro-premuzic_creativity_2006}). While we do not propose new definitions of inspiration and creativity, we consider these concepts to be distinct and do not use them interchangeably (as some researchers have done in the past). Besides its creativity promoting nature, designers value inspiration for a variety of other reasons. For instance, inspiration alleviates the design process \cite{mete_creative_2006}, triggers idea generation \cite{koronis_crafting_2021, petre_complexity_2006}, reduces resources spent such as time and effort \cite{cai_extended_2010}, expands designer knowledge \cite{goncalves_inspiration_2016}, and helps in communicating designer ideas to other people \cite{eckert_sources_2000, petre_complexity_2006}. 

\subsection{Sources of Inspiration}
In design literature, it is widely accepted that designs are not produced in a vacuum \cite{eckert_adaptation_2003}. Sources of inspiration play an important role in any profession requiring creativity \cite{eckert_sources_2003, yang_design_2005}---which at least some types of data visualization design certainly do \cite{parsons_fixation_2021,bako_understanding_2022,mendez_bottom-up_2017, li_data_2018, dignazio_creative_2017}. Eckert and Stacey refer to sources of inspiration as ``conscious uses of previous designs and other objects and images in a design process'' \cite[p. 524]{eckert_sources_2000}. Gon\c{c}alves and colleagues define a source of inspiration as "any stimulus retrieved from one's memory or from outside world, during (or beyond) a design process, that directly or indirectly influences the thinking process leading up to the framing of the problem or generation of a solution" \cite[p. 3]{goncalves_inspiration_2016}. In that sense, source of inspiration can be anything from a specific visualization example, to a flower or object in nature, to a colleague, friend or other person. 

While many sources of inspiration are encountered externally in the world, designers may also draw inspiration from within themselves. For example, design precedents, defined as ``a designer's store of experiences'' \cite[p. 1]{boling_nature_2021} can act a source of inspiration for a designer in the form of precedent knowledge or experience. The source within the designer can also inspire other designers, where a designer becomes a sort of a source themselves. Scolere \cite{scolere_digital_2021} defined a term ``the digital inspirational economy'' to refer to phenomena where work of one designer becomes an inspiration for another designer. 

Sources of inspiration in design have been studied in variety of disciplines, including textile design \cite{eckert_sources_2000}, fashion design \cite{mete_creative_2006}, industrial design \cite{santulli_introducing_2011}, and interaction design \cite{halskov_kinds_2010}. However, sources of inspiration---and the phenomenon of inspiration in general---have not received similar attention in the visualization community.

\subsection{The Dual Nature of Inspiration}
\label{sec:dual}
Despite the positive connotations of inspiration, many researchers have noted its potential dual nature (e.g., \cite{cai_extended_2010}). While inspiration has been shown to promote creativity and idea generation (e.g., \cite{santulli_introducing_2011, goldschmidt_inspiring_2011}), in some cases inspirational sources can be detrimental to creativity and result in less original solutions. The phenomenon of design fixation, defined as ``a blind, and sometimes counterproductive, adherence to a limited set of ideas in the design process'' \cite[p. 4]{jansson_design_1991}, has been extensively studied in the design literature and replicated across several design disciplines. Originally demonstrated empirically in a classic study by Jansson and Smith \cite{jansson_design_1991}, design fixation often occurs when designers are presented with design examples, where their eventual solutions resemble the features of the examples they saw beforehand. While initial studies showed design students as being vulnerable to instances of design fixation, future studies further demonstrated that design fixation is prevalent among professional and expert designers as well \cite{viswanathan_study_2012, crilly_fixation_2015, condoor_design_2007, kim_design_2014, linsey_study_2010, viswanathan_design_2013}. However, even though there are negative consequences of design fixation, Cross \cite{Cross2001} suggests that it is not necessarily an inherently negative phenomenon in design and that expert designers often engage in fixation-like behavior. Furthermore, all design thinking is influenced by the previous experiences and designs encountered by a designer, and there is no meaningful way to avoid inspirational sources or prevent fixation. Thus it is important to recognize the dual nature of inspirational sources, and to study their nature and use to better prepare future designers and support practitioners in their work.

\subsection{Inspiration in Visualization Design}
Although inspiration per se has not received much direct attention in the visualization literature, researchers have written about inspiration-related practices, tools, and techniques. For instance, Willett and colleagues \cite{willett_perception_2022} have argued that it may be useful to think of the perceptual and cognitive benefits of visualization in a manner akin to the superpowers of heroes. Taking a cue from this notion, they explore how the concept of ``superpower'' can be harnessed to inspire the design of effective visualization systems. Specifically, they describe how fictional superpowers from comic books and related media can act as sources of inspiration for generating new visualization systems. They discuss a variety of visual superpowers that they have collected from various media.

Owen and Roberts \cite{owen_inspire_2023} acknowledge the importance of inspiration for visualization designers and propose a tool called VisDice, aimed at stimulating creative and inspiring ideas. They found that random prompts offered by VisDice encourage creative thinking and help in overcoming mental blockages and fixation. Brehmer and colleagues \cite{brehmer_generative_2022} proposed a technique called Diatoms for generating design inspiration for glyphs. Specifically, the technique samples various elements from a palette of visual objects and generates a variety of options for glyph designs. Besides the proposed technique, the authors also discuss inspiration in visualization design in general, and talk about various sources that designers use including the Data Visualization Society's Slack workspace, social media platforms like Reddit and Twitter, and D3.js or other similar repositories. Judelman \cite{judelman_aesthetics_2004} discussed how visualization research and design is predominantly rooted in computer science and engineering, with visualizations being created largely from people with technically-oriented backgrounds. At the same time, there are multiple resources within the more artistic and design communities that also engage in information visualization, but are being underutilized in primary visualization research. Judelman then proposes and describes various disciplinary sources, such as algorithmic art, architecture, and nature to name a few, where visualization research can draw inspiration from to find new and innovative graphic and interactive techniques for visualization design. He and Adar \cite{he_vizitcards_2017} used inspiration in an educational setting via a design workshop that was conducted in a graduate level course. Specifically, He and Adar used various (including inspiration cards) physical cards, called VizItCards, which were used to facilitate the design process during the workshop with relative success.

Most visualization research broadly related to inspiration has focused on the development of frameworks or tools and techniques that may enhance creativity and inspire. Comparatively less work has taken a practice-focused approach---e.g., investigating the ways in which practitioners seek out, and make use of inspirational sources in their design processes, and the opinions they have towards its relevance in their practice. In one notable contribution, Bako and colleagues \cite{bako_understanding_2022} investigated what types of visualization examples visualization designers use and how they find them. They interviewed 15 students and 15 professional visualization designers and found that visualization designers use multiple diverse sources to find visualization examples (e.g., social media, published media) and employ a variety of techniques and activities (e.g., merging, modifying) to integrate examples into their workflow. In another study, Parsons \cite{parsons_understanding_2022} interviewed 20 visualization practitioners to investigate their design process, specifically in terms of their decision making, methods, and processes. Findings described how designers use precedent knowledge as a source of inspiration, and reported some of other sources that designers tend to use (e.g., nature). In their later work, Parsons and colleagues \cite{parsons_fixation_2021} investigated perceptions of visualization designers on design fixation, specifically factors that designers think encourage or discourage fixation. The findings described some of the inspiration strategies that designers use to avoid fixation (e.g., looking at sources that were not closely related to problem at hand) and also reported on some of the sources that visualization designers use (e.g., existing visualizations, art, nature). 

While there are several contributions in the visualization literature related to inspiration, broadly construed, we still know very little about how visualization practitioners perceive, seek out, and use inspiration sources, and how they relate to other aspects of their practice. Inspiration is essential for any design profession; designers regularly seek out sources of inspiration, and also draw on precedent designs and other forms of design knowledge that serve as conceptual and creative springboards for design activity \cite{leifer_early_2014, eckert_sources_2000, koronis_crafting_2021}. While some types of visualization design (e.g., business reporting) may be more routine and less creative (e.g., relying on pre-determined templates and visualization types), it is not unusual for visualization design to be more open-ended, requiring creative solutions that must inevitably draw inspiration in some form.