\section{Related work}
RL has emerged as a promising approach for solving locomotion tasks, offering two primary paradigms: position-based and torque-based control.

____ and ____ predict target joint positions, which are actuated using proportional-derivative (PD) controllers. These methods are widely favoured for their ease of training and robustness in high-level command tracking. 
Within this control, the torque applied to the motors is calculated using Eq. (\ref{eq:action_torque})
\begin{equation}
\tau_t = K_p (q_{t,\ target}(\theta)- q_t) + K_d ( \dot{q_{des}} - \dot{q_t} ) \label{eq:action_torque}
\end{equation}
However, they suffer from limited compliance behaviour due to fixed stiffness and damping values (\(K_p\) and \(K_d\)), requiring extensive PD gain tuning that is often task- and robot-specific. ____ 

In this control, the PD controller acts as a low-level tracking module, where the proportional gain (P-gain) is representative of the stiffness and the derivative gain (d-gain) for the damping. ____ investigated the impact of the p-gains, suggesting that large proportional gain leads to instabilities in training. In contrast, the low proportional gain has significant tracking errors and behaves like a torque controller. Other research studied the impact on the derivative gain.____ showed small defu2024novelivative gains result in learning instabilities, and large gains prevented tracking the target velocity. 

To circumvent this PD controller, ____ and ____ studied torque control as an alternative and applied it to quadruped and biped locomotion. In this control, the actions are directly applied to the motors.
Although this control showed higher achievable rewards in the long term, it must be executed at higher speeds to perform similarly to position control. It is more difficult to train initially. The higher control speeds limit the design freedom of torque-based controllers. 

____ showed for model-based control, that adapting stiffness according to the contact force led to sufficient walking for a quadruped on uneven terrains. ____ studied the impact of including joint stiffness alongside joint positions. 
The torque was then calculated by Eq. (\ref{eq:variable_stiffness}).  
\begin{equation}
    \tau_t = K_p(\theta) (q_{t,\ target}(\theta) - q_t) - K_d(\theta) \dot{q_t}  \label{eq:variable_stiffness}
\end{equation}
Overall, this work did not test their approach to locomotion but showed the superiority of this concept against torque and position control for two scenarios: a single-legged hopper and a manipulator. The single-legged hopper achieved more considerable jumping heights, and the manipulator maintained a continuous contact force during reference motions with the adjustment of the action space. 
This motivates and applies stiffness alongside joint positions into the action space and applies it to quadruped locomotion.

Building upon the previous works, we investigate the opportunity of joint stiffnesses alongside joint positions in the action space for a reinforcement learning policy applying to quadruped locomotion.