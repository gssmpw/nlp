\section{Related Work}
% This paper contributes to the field of stable matching under choice correspondences. ____ introduced the concept of \emph{substitutability} for choice correspondences. ____ and ____ applied this concept to analyze stable matchings. In this paper, we propose a definition of \emph{path-independence (PI)} for choice correspondences, which offers two main advantages: (i) the choice correspondence can be rationalized, and (ii) it enables a clear characterization of efficient and stable matchings.

%Our definition of PI is also closely related to existing concepts in choice functions. 
PI for choice functions was first introduced by ____. Under PI choice functions, it is known that stable matchings exist ____. Recently, ____ showed that PI choice functions can be characterized through the rationalization of ordinal concavity. If a choice function is rationalized by an M${}^\natural$-concave function, it satisfies both PI and LAD____. Furthermore, any choice correspondence that can be rationalized by these functions also satisfies PI and LAD.

____ and ____ pointed out the relationship between PI choice functions and (finite) \emph{convex geometries}.
A convex geometry is a combinatorial structure that generalizes the concept of convexity in Euclidean spaces to more abstract settings. 
Formally, a convex geometry consists of a finite set paired with a closure operator that satisfies the anti-exchange property____. This structure induces a PI choice function through the \emph{extreme operator}. For further details, see Chapter~5 in the book by ____. % ____
We generalize some of the results obtained in ____ and ____ for PI choice functions to PI choice correspondences. In particular,  we utilize these results to prove the rationalizability of a PI choice correspondence.

Several studies have been conducted on efficient and stable matchings under choice correspondences.
%{\color{red}____  proposed the extension of substitutability and IRC for choice correspondences. ____ showed the existence of a stable matching assuming these conditions.}
____ characterized constrained efficient matchings under responsive choice correspondences, showing that a stable matching is constrained efficient if and only if it does not admit any stable improvement cycle.
____ and ____ analyzed constrained efficient matchings under acceptant and substitutable correspondences. In that setting, constrained efficient matchings may admit cycles (PSIC), making them difficult to fully characterize.
Under our conditions, however, we can generalize the characterization provided by ____.
Our PI strengthens substitutability, whereas our LAD is weaker than acceptance. Therefore, a straightforward comparison with ____ is not possible.

Our study also relates to the efficient allocation of indivisible goods under constraints. Specifically, a constraint on allocations can be represented by a choice correspondence that returns all feasible subsets.
____ generalized the top trading cycles mechanism that preserves Pareto efficiency, individual rationality, and group strategy-proofness for any distributional constraint representable by an M-convex set on the vector of the number of students assigned to each school, given an initial endowment.
____ provided necessary and sufficient conditions for constraints to guarantee the existence of desired mechanisms.

Finally, our results have important implications for real-life applications in market design. In matching with diversity concerns, various choice functions---such as quotas ____, reserves ____ and overlapping reserves ____---have been proposed, typically assuming strict priorities. Using our framework, we can accommodate weak priorities (i.e., ties). In each case, we provide a characterization of constrained efficient matchings in terms of cycles (PSIC).