\section{Case Study: Model Size}\label{sec:case1} 

In this section, we use \SYSTEM{} to examine the environmental impact of model size on LLM serving.

% Smaller models are often assumed to be more environmentally friendly due to their lower resource demands. In this case study, we aim to explore whether this common belief holds true across different scenarios. Specifically, we examine whether larger models, despite their higher resource consumption, can sometimes be more sustainable by better balancing quality and throughput requirements. We conduct experiments using two model families with varying sizes, leveraging the \SYSTEM{} framework to evaluate and compare their environmental impact under different functional unit settings.

\subsection{Evaluation Methodology}

\begin{figure}[!t]
    \centering
    \includegraphics[width=0.42\textwidth]{llmfu-acl2025/figs/size/carbon_per_token_quality.pdf}
    \vspace{-0.15in}
    \caption{Carbon emission per FU for different model sizes across Qscores at QPS=1 req/s.}
    \label{fig:size_carbon_per_token_quality}
\end{figure}

\paragraph{Setup.} We evaluate various model sizes from two LLM families---Qwen2.5 (7B, 14B, 32B) and Llama2 (7B, 13B)---on an NVIDIA H100 GPU paired with an Intel Xeon 8480+ CPU.

\paragraph{Benchmarking configurations.} To assess how model sizes affect the environmental impact---or how ``green" \textcolor{green!60!black}{\faPagelines} each model is in terms of carbon efficiency---we evaluate a range of FUs by adjusting serving constraints. QPS is from 1 to 20 req/s. The Qscore ranges are set to [-5, 15] for Qwen and [-5, 10] for Llama, based on the Qscore distribution of each model family (\Cref{fig:size_newsqa_score_dist} in \Cref{sec:appendix_model_size_newsqa}). These ranges ensure broad coverage while providing sufficient outputs across model sizes that meet quality requirements. TTFT is at 1s and TPOT is at 200ms to align with human reading speed. 


\subsection{Evaluation Results}

\fuelq{Are smaller models always greener? }
We first investigate whether smaller models are always greener. \Cref{fig:size_carbon_per_token_quality} shows carbon emissions per FU across model sizes under different Qscore settings at QPS = 1 req/s. We choose a relatively low QPS to ensure all models generate enough tokens without violating performance constraints. The results indicate that the answer is \textbf{no}.  

For Qwen, at a low Qscore of -5, smaller models emit less carbon. However, as Qscore increases, carbon emissions increase for all model sizes, with smaller models increasing at a faster rate. When Qscore exceeds 5, the smallest 7B model becomes the highest emitter. At Qscore 15, the 32B model has the lowest emissions, while the 7B model emits over 1.8× more. A similar trend is seen in Llama, where larger models become greener as quality requirement rise. We confirm that larger models produce higher-quality outputs with higher Qscores in \Cref{fig:size_newsqa_score_dist} in \Cref{sec:appendix_model_size_newsqa}. This underscores the need to balance model size and output quality for lower environmental impact.
% \Cref{fig:size_newsqa_score_dist} in \Cref{sec:appendix_model_size_newsqa} shows the Qscore distributions of different model sizes within each model family, confirming that larger models produce higher-quality outputs with higher Qscores.
\begin{figure}[!t]
    \centering
    \includegraphics[width=0.42\textwidth]{llmfu-acl2025/figs/size/carbonsaving_newsqa_qwen_multiscore.pdf}
    \vspace{-0.15in}
    \caption{Carbon savings of Qwen 14B and 32B compared to 7B with Qscore low (-5) and high (15). Data for Qwen 32B are missing at QPS > 4 req/s, as larger models cannot serve intensive workloads.}
    \label{fig:size_newsqa_cs_qwen}
\end{figure}

\begin{figure}[!t]
    \centering
    \includegraphics[width=0.42\textwidth]{llmfu-acl2025/figs/size/carbonsaving_newsqa_llama2_multiscore.pdf}
    \vspace{-0.15in}
    \caption{Carbon savings of Llama 13B compared to 7B with Qscore low (-5) and high (10).}
    \label{fig:size_newsqa_cs_llama}
\end{figure}

\fuelq{When are larger models greener?}
To examine when larger models become greener, we set FUs with a broader QPS range and two quality requirements: low (Qscore = -5) and high (Qscore = 15 for Qwen, 10 for Llama). \Cref{fig:size_newsqa_cs_qwen} shows that for Qwen, larger models (14B and 32B) save more carbon compared to the 7B model under high Qscore, with the 32B saving over 40\%. However, under a low-quality requirement (Qscore = -5), larger models offer no advantage. A similar trend is seen for Llama, where the 13B model saves over 20\% carbon compared to the 7B model at high quality. Thus, \textbf{larger models become greener when output quality requirements are high}.

To explain the carbon savings shift with varying QPS, we analyze its impact on \emph{service level objective (SLO) attainment}, which refers to meeting TTFT and TPOT constraints. In \Cref{fig:size_slo_attn}, we observe that once QPS exceeds a certain threshold, SLO attainment drops, as the system becomes saturated. This explains why larger models can be greener at lower QPS: they meet performance constraints while producing higher-quality output.



\begin{figure}[!t]
    \centering
    \includegraphics[width=0.42\textwidth]{llmfu-acl2025/figs/size/slo_attn.pdf}
    \vspace{-0.15in}
    \caption{SLO attainment of Qwen and Llama families across QPS range.}
    \label{fig:size_slo_attn}
\end{figure}

\begin{figure}[!t]
    \centering
    \includegraphics[width=0.42\textwidth]{llmfu-acl2025/figs/size/heatmap_qwen_new.pdf}
    \vspace{-0.15in}
    \caption{Comparison of Qwen 7B, 14B, and 32B in FUEL. Tile colors indicate the model size with the lowest carbon per FU. Tile values are carbon savings (\%) of greenest model size compared to the second greenest.}
    \label{fig:size_heatmap}
\end{figure}


\fuelq{Does a universal greenest model size exist?}

\noindent The answer is \textbf{no}. \Cref{fig:size_heatmap} shows the relative carbon savings of Qwen 7B, 14B, and 32B across various QPS and Qscore values. No model size consistently has the lowest carbon emissions. At low QPS (1-4 req/s) with high Qscore, Qwen 32B can save up to 49\% in carbon emissions compared to the second greenest. However, as QPS increases, the 32B fails to meet the performance constraints, making the 14B the greenest. When the quality requirement is low (Qscore = 0), the 7B model is always the greenest, especially at high QPS.

\takeawaybox{
Larger models are greener under high-quality, low-QPS conditions. Smaller models become greener as QPS increases. No single model size is the greenest across all scenarios.
}

