\section{Case Study: Hardware}\label{sec:case3} 

In this section, we examine how hardware platform affects the environmental impact of LLM serving. Using \SYSTEM{}, we investigate whether more advanced hardware can enhance carbon efficiency while maintaining output quality.

\subsection{Evaluation Methodology}

\paragraph{Setup.} We conduct experiments on two GPU servers with different hardware configurations, one older and one newer, as detailed in~\Cref{tab:hardware_specifications}. For fair comparisons, we use a single GPU per server for all experiments. We evaluate the Qwen2.5 (7B, 14B) and Llama2 (7B, 13B).

\paragraph{Benchmarking configurations.} Same as in \Cref{sec:case1}.

\begin{table}[!t]
    \centering
    \caption{Hardware platform specifications in this paper.}
    \footnotesize
    \begin{tabular}{l|p{2.5cm}|p{2.5cm}}
        \toprule
        \textbf{Specification} & \textbf{L40 server} & \textbf{H100 server} \\
        \midrule        
        \textbf{GPU} & 4 $\times$ L40 & 8$\times$ H100 \\
        \footnotesize{TDP} & \footnotesize{300W} & \footnotesize{350W} \\
        \footnotesize{Process size} & \footnotesize{5nm} & \footnotesize{5nm} \\
        \footnotesize{Die size} & \footnotesize{609 mm$^2$} & \footnotesize{814 mm$^2$} \\
        \footnotesize{GPU memory} & \footnotesize{40GB} & \footnotesize{80GB} \\
        \footnotesize{Release Year} & \footnotesize{2022} & \footnotesize{2023} \\
        \hline
        \textbf{CPU} & AMD EPYC 7443 & Intel Xeon 8480+ \\
        \footnotesize{TDP} & \footnotesize{200W} & \footnotesize{350W} \\
        \footnotesize{Process size} & \footnotesize{7nm} & \footnotesize{10nm} \\
        \footnotesize{Die size} & \footnotesize{4$\times$81 mm$^2$} & \footnotesize{4$\times$477 mm$^2$} \\
        \footnotesize{CPU memory} & 504GB & 1031GB \\
         \footnotesize{Release Year} & \footnotesize{2021} & \footnotesize{2023} \\
        \bottomrule
    \end{tabular}
    \label{tab:hardware_specifications}
\end{table}

\subsection{Evaluation Results}

\fuelq{How does different hardware contribute to total carbon emissions?}

\noindent \Cref{fig:hw_carbon_breakdown} shows the breakdown of carbon emissions per FU for Qwen and Llama 7B models on different hardware platforms, separating operational and embodied carbon. It is worth noting that different hardware contributes to different embodied carbon per FU, due to differences in the total embodied carbon for each hardware. The L40 platform has lower total embodied carbon than the H100, with values of 26.6 and 29.92 kgCO\textsubscript{2}eq repectively. These differences are based on calculations using the ACT modeling tool~\cite{gupta2022act} and are due to hardware factors such as process and die size. The difference is even more pronounced between the AMD EPYC 7443 and Intel Xeon 8480+ CPUs, with the AMD CPU having 9.98 kgCO\textsubscript{2}eq, compared to the Intel's 42.81 kgCO\textsubscript{2}eq, over 4x higher.

\begin{figure}[!t]
    \centering
    \includegraphics[width=0.45\textwidth]{llmfu-acl2025/figs/hw/carbon_breakdown.pdf}
    \vspace{-0.15in}
    \caption{Breakdown of Carbon emission per FU for Qwen and Llama 7B models on different hardware platforms, evaluated in FUEL with Qscore=0.}
    \label{fig:hw_carbon_breakdown}
\end{figure}

Advanced hardware like the H100 offers better performance but higher embodied carbon. Extending hardware lifetime can yield more carbon savings, especially considering the large difference in embodied carbon between older and newer devices.


\fuelq{Is LLM serving on advanced hardware greener?}

\noindent \Cref{fig:hw_carbon_models} shows the carbon emissions per FU for the Qwen and Llama model families on two hardware platforms. At low QPS, the L40 server consistently has lower carbon emissions than the H100. This means that the answer is \textbf{no}: advanced hardware is not necessarily greener.

\begin{figure}[!t]
    \centering
    \includegraphics[width=0.42\textwidth]{llmfu-acl2025/figs/hw/carbon_models.pdf}
    \vspace{-0.15in}
    \caption{Carbon emission per FU of Qwen and Llama model families on different hardware platforms, evaluated in FUEL with Qscore=0.}
    \label{fig:hw_carbon_models}
\end{figure}

The main advantage of advanced hardware like the H100 is its ability to produce higher-quality outputs and meet performance constraints, as shown in~\Cref{fig:hw_slo_attn}. Although advanced hardware may not be greener, it provides better performance and supports higher-quality LLM serving at scale.

\begin{figure}[!t]
    \centering
    \includegraphics[width=0.42\textwidth]{llmfu-acl2025/figs/hw/slo_attn.pdf}
    \vspace{-0.15in}
    \caption{SLO attainment of Qwen and Llama model families on different hardware platforms.}
    \label{fig:hw_slo_attn}
\end{figure}

\begin{figure}[!t]
    \centering
    \includegraphics[width=0.48\textwidth]{llmfu-acl2025/figs/hw/hw_qwen_heatmap.png}
    \vspace{-0.2in}
    \caption{Comparison of Qwen 7B and 14B on different hardware platforms in FUEL. Tile colors indicate the hardware with the lowest carbon emission per FU. Tile values are carbon savings (\%) of the greenest hardware compared to the second greenest.}
    \label{fig:hw_heatmap_qwen_sep}
\end{figure}

\fuelq{How to choose greener hardware?}

\noindent To answer this question, we run experiments across different FUs with varying QPS and Qscores. \Cref{fig:hw_heatmap_qwen_sep} shows the relative carbon savings of L40 and H100 servers for Qwen 7B and 14B. Hardware carbon efficiency depends mainly on model size and QPS, with a minor influence from Qscore. Newer hardware is more carbon efficient at high QPS, while older hardware is better at low QPS. These findings underscore the sustainability benefits of reusing older hardware to cut carbon emissions while maintaining performance and quality.


\takeawaybox{
Advanced hardware offers higher performance but is not always greener due to higher embodied carbon. Older hardware can achieve lower carbon emissions while still meeting quality and performance constraints.
}

