\section{Related Work}\label{sec:related} 

\begin{figure*}[!t]
    \centering
    \includegraphics[width=0.9\textwidth]{llmfu-acl2025/figs/framework.pdf}
    \caption{Overview of \SYSTEM{} framework.}
    \label{fig:framework}
\end{figure*}

\noindent \textbf{Environmental impact of LLM serving.} Researchers have recognized the environmental impact of LLM serving and explored it through modeling and profiling~\cite{ding2024sustainable}. Modeling efforts include LLMCarbon~\cite{faiz2024llmcarbon} and LLMCO2~\cite{fu2024llmco2}, which provide end-to-end carbon modeling frameworks, while LLMCampass~\cite{zhang2024llmcompass} focuses on hardware evaluation for LLM workloads. Profiling studies have run various LLM serving models across different hardware and QPS settings~\cite{nguyen2024towards,li2024arenahard,patel2024characterizing}, with GreenLLM~\cite{shi2024greenllm} and Sprout~\cite{li2024sprout} optimizing carbon emissions based on their profiling. However, none of these studies take a functional unit perspective as we do in this work.

\noindent \textbf{LLM serving optimization.} Prior work on LLM serving has primarily focused on optimizing performance and energy efficiency. Performance improvements can be categorized into model-level and system-level techniques. Model-side optimizations include quantization~\cite{lin2024awq, frantar2022gptq}, sparsification~\cite{frantar2023sparsegpt}, and speculative decoding~\cite{leviathan2023fast}. System-side approaches involve memory management~\cite{kwon2023efficient}, batching~\cite{agrawal2024taming, yu2022orca}, and kernel optimizations~\cite{dao2022flashattention}. Additionally, efforts to enhance energy efficiency include solutions like Splitwise~\cite{patel2024splitwise} and DynamoLLM~\cite{stojkovic2024dynamollm}. However, they have largely overlooked quality constraints.


