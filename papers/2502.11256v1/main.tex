% This must be in the first 5 lines to tell arXiv to use pdfLaTeX, which is strongly recommended.
\pdfoutput=1
% In particular, the hyperref package requires pdfLaTeX in order to break URLs across lines.

\documentclass[11pt]{article}
\usepackage{tcolorbox}

% Change "review" to "final" to generate the final (sometimes called camera-ready) version.
% Change to "preprint" to generate a non-anonymous version with page numbers.
\usepackage[preprint]{acl}

% Standard package includes
\usepackage{times}
\usepackage{latexsym}
\usepackage{amsfonts}

% For proper rendering and hyphenation of words containing Latin characters (including in bib files)
\usepackage[T1]{fontenc}
% For Vietnamese characters
% \usepackage[T5]{fontenc}
% See https://www.latex-project.org/help/documentation/encguide.pdf for other character sets
\usepackage{amsmath}
% This assumes your files are encoded as UTF8
\usepackage[utf8]{inputenc}
\usepackage{enumitem}
% This is not strictly necessary, and may be commented out,
% but it will improve the layout of the manuscript,
% and will typically save some space.
\usepackage{microtype}

% This is also not strictly necessary, and may be commented out.
% However, it will improve the aesthetics of text in
% the typewriter font.
\usepackage{inconsolata}

%Including images in your LaTeX document requires adding
%additional package(s)
\usepackage{graphicx}
\usepackage{cleveref}
\usepackage{fontawesome}

\usepackage{multirow}
\usepackage{multicol}
\usepackage{tabularx}
\usepackage{booktabs}
\usepackage{xcolor}
\usepackage{adjustbox}
\usepackage{makecell}

\crefformat{section}{\S#2#1#3} % see manual of cleveref, section 8.2.1
\crefformat{subsection}{\S#2#1#3}
\crefformat{subsubsection}{\S#2#1#3}

\setlist{noitemsep, leftmargin=*, topsep=0pt, partopsep=0pt}

% If the title and author information does not fit in the area allocated, uncomment the following
%
%\setlength\titlebox{<dim>}
%
% and set <dim> to something 5cm or larger.

\newcommand{\yd}[1]{{\color{orange}[Yi: #1]}}
\newcommand{\todo}[1]{{\color{red}[TODO: #1]}}
\newcommand{\SYSTEM}{FUEL}


\newcommand*{\Scale}[2][4]{\scalebox{#1}{$#2$}}%

\tcbset{
  takeaway/.style={ width=\hsize,left=0pt,right=0pt,top=0pt,bottom=0pt,colback=green!10!white,boxrule=1pt,colframe=black!30!green!50!white
  },
}
\newcounter{takeawaycounter}
\newcommand{\takeawaybox}[1]{
    \stepcounter{takeawaycounter}
    \begin{tcolorbox}[takeaway]
    \textbf{Takeaway~\arabic{takeawaycounter}:}#1
    \end{tcolorbox}
}

% \usepackage[skins]{tcolorbox}

\tcbset{
  def/.style={ width=\hsize,left=1pt,right=1pt,top=1pt,bottom=1pt,colback=black!20!orange!4!white,boxrule=1pt,colframe=black!30!orange!20!white
  },
}


\title{Unveiling Environmental Impacts of Large Language Model Serving: \\ A Functional Unit View}

% Author information can be set in various styles:
% For several authors from the same institution:
% \author{Author 1 \and ... \and Author n \\
%         Address line \\ ... \\ Address line}
% if the names do not fit well on one line use
%         Author 1 \\ {\bf Author 2} \\ ... \\ {\bf Author n} \\
% For authors from different institutions:
% \author{Author 1 \\ Address line \\  ... \\ Address line
%         \And  ... \And
%         Author n \\ Address line \\ ... \\ Address line}
% To start a separate ``row'' of authors use \AND, as in
% \author{Author 1 \\ Address line \\  ... \\ Address line
%         \AND
%         Author 2 \\ Address line \\ ... \\ Address line \And
%         Author 3 \\ Address line \\ ... \\ Address line}

\author{Yanran Wu \\
  Purdue University \\
  \texttt{wu2187@purdue.edu} \\\And
  Inez Hua \\
  Purdue University \\
  \texttt{hua@purdue.edu} \\\And
  Yi Ding \\
  Purdue University \\
  \texttt{yiding@purdue.edu}   \\
  }


% Define a counter for questions that resets every section
\newcounter{question}[section]

% Define the fuelq macro
\newcommand{\fuelq}[1]{
    \stepcounter{question} % Increment the question counter
    \paragraph{Question \thequestion:} \textit{#1} % Format output with paragraph and italicized content
}

\begin{document}
	
\maketitle

\begin{abstract}  
Test time scaling is currently one of the most active research areas that shows promise after training time scaling has reached its limits.
Deep-thinking (DT) models are a class of recurrent models that can perform easy-to-hard generalization by assigning more compute to harder test samples.
However, due to their inability to determine the complexity of a test sample, DT models have to use a large amount of computation for both easy and hard test samples.
Excessive test time computation is wasteful and can cause the ``overthinking'' problem where more test time computation leads to worse results.
In this paper, we introduce a test time training method for determining the optimal amount of computation needed for each sample during test time.
We also propose Conv-LiGRU, a novel recurrent architecture for efficient and robust visual reasoning. 
Extensive experiments demonstrate that Conv-LiGRU is more stable than DT, effectively mitigates the ``overthinking'' phenomenon, and achieves superior accuracy.
\end{abstract}  
\section{Introduction}


\begin{figure}[t]
\centering
\includegraphics[width=0.6\columnwidth]{figures/evaluation_desiderata_V5.pdf}
\vspace{-0.5cm}
\caption{\systemName is a platform for conducting realistic evaluations of code LLMs, collecting human preferences of coding models with real users, real tasks, and in realistic environments, aimed at addressing the limitations of existing evaluations.
}
\label{fig:motivation}
\end{figure}

\begin{figure*}[t]
\centering
\includegraphics[width=\textwidth]{figures/system_design_v2.png}
\caption{We introduce \systemName, a VSCode extension to collect human preferences of code directly in a developer's IDE. \systemName enables developers to use code completions from various models. The system comprises a) the interface in the user's IDE which presents paired completions to users (left), b) a sampling strategy that picks model pairs to reduce latency (right, top), and c) a prompting scheme that allows diverse LLMs to perform code completions with high fidelity.
Users can select between the top completion (green box) using \texttt{tab} or the bottom completion (blue box) using \texttt{shift+tab}.}
\label{fig:overview}
\end{figure*}

As model capabilities improve, large language models (LLMs) are increasingly integrated into user environments and workflows.
For example, software developers code with AI in integrated developer environments (IDEs)~\citep{peng2023impact}, doctors rely on notes generated through ambient listening~\citep{oberst2024science}, and lawyers consider case evidence identified by electronic discovery systems~\citep{yang2024beyond}.
Increasing deployment of models in productivity tools demands evaluation that more closely reflects real-world circumstances~\citep{hutchinson2022evaluation, saxon2024benchmarks, kapoor2024ai}.
While newer benchmarks and live platforms incorporate human feedback to capture real-world usage, they almost exclusively focus on evaluating LLMs in chat conversations~\citep{zheng2023judging,dubois2023alpacafarm,chiang2024chatbot, kirk2024the}.
Model evaluation must move beyond chat-based interactions and into specialized user environments.



 

In this work, we focus on evaluating LLM-based coding assistants. 
Despite the popularity of these tools---millions of developers use Github Copilot~\citep{Copilot}---existing
evaluations of the coding capabilities of new models exhibit multiple limitations (Figure~\ref{fig:motivation}, bottom).
Traditional ML benchmarks evaluate LLM capabilities by measuring how well a model can complete static, interview-style coding tasks~\citep{chen2021evaluating,austin2021program,jain2024livecodebench, white2024livebench} and lack \emph{real users}. 
User studies recruit real users to evaluate the effectiveness of LLMs as coding assistants, but are often limited to simple programming tasks as opposed to \emph{real tasks}~\citep{vaithilingam2022expectation,ross2023programmer, mozannar2024realhumaneval}.
Recent efforts to collect human feedback such as Chatbot Arena~\citep{chiang2024chatbot} are still removed from a \emph{realistic environment}, resulting in users and data that deviate from typical software development processes.
We introduce \systemName to address these limitations (Figure~\ref{fig:motivation}, top), and we describe our three main contributions below.


\textbf{We deploy \systemName in-the-wild to collect human preferences on code.} 
\systemName is a Visual Studio Code extension, collecting preferences directly in a developer's IDE within their actual workflow (Figure~\ref{fig:overview}).
\systemName provides developers with code completions, akin to the type of support provided by Github Copilot~\citep{Copilot}. 
Over the past 3 months, \systemName has served over~\completions suggestions from 10 state-of-the-art LLMs, 
gathering \sampleCount~votes from \userCount~users.
To collect user preferences,
\systemName presents a novel interface that shows users paired code completions from two different LLMs, which are determined based on a sampling strategy that aims to 
mitigate latency while preserving coverage across model comparisons.
Additionally, we devise a prompting scheme that allows a diverse set of models to perform code completions with high fidelity.
See Section~\ref{sec:system} and Section~\ref{sec:deployment} for details about system design and deployment respectively.



\textbf{We construct a leaderboard of user preferences and find notable differences from existing static benchmarks and human preference leaderboards.}
In general, we observe that smaller models seem to overperform in static benchmarks compared to our leaderboard, while performance among larger models is mixed (Section~\ref{sec:leaderboard_calculation}).
We attribute these differences to the fact that \systemName is exposed to users and tasks that differ drastically from code evaluations in the past. 
Our data spans 103 programming languages and 24 natural languages as well as a variety of real-world applications and code structures, while static benchmarks tend to focus on a specific programming and natural language and task (e.g. coding competition problems).
Additionally, while all of \systemName interactions contain code contexts and the majority involve infilling tasks, a much smaller fraction of Chatbot Arena's coding tasks contain code context, with infilling tasks appearing even more rarely. 
We analyze our data in depth in Section~\ref{subsec:comparison}.



\textbf{We derive new insights into user preferences of code by analyzing \systemName's diverse and distinct data distribution.}
We compare user preferences across different stratifications of input data (e.g., common versus rare languages) and observe which affect observed preferences most (Section~\ref{sec:analysis}).
For example, while user preferences stay relatively consistent across various programming languages, they differ drastically between different task categories (e.g. frontend/backend versus algorithm design).
We also observe variations in user preference due to different features related to code structure 
(e.g., context length and completion patterns).
We open-source \systemName and release a curated subset of code contexts.
Altogether, our results highlight the necessity of model evaluation in realistic and domain-specific settings.





\putsec{related}{Related Work}

\noindent \textbf{Efficient Radiance Field Rendering.}
%
The introduction of Neural Radiance Fields (NeRF)~\cite{mil:sri20} has
generated significant interest in efficient 3D scene representation and
rendering for radiance fields.
%
Over the past years, there has been a large amount of research aimed at
accelerating NeRFs through algorithmic or software
optimizations~\cite{mul:eva22,fri:yu22,che:fun23,sun:sun22}, and the
development of hardware
accelerators~\cite{lee:cho23,li:li23,son:wen23,mub:kan23,fen:liu24}.
%
The state-of-the-art method, 3D Gaussian splatting~\cite{ker:kop23}, has
further fueled interest in accelerating radiance field
rendering~\cite{rad:ste24,lee:lee24,nie:stu24,lee:rho24,ham:mel24} as it
employs rasterization primitives that can be rendered much faster than NeRFs.
%
However, previous research focused on software graphics rendering on
programmable cores or building dedicated hardware accelerators. In contrast,
\name{} investigates the potential of efficient radiance field rendering while
utilizing fixed-function units in graphics hardware.
%
To our knowledge, this is the first work that assesses the performance
implications of rendering Gaussian-based radiance fields on the hardware
graphics pipeline with software and hardware optimizations.

%%%%%%%%%%%%%%%%%%%%%%%%%%%%%%%%%%%%%%%%%%%%%%%%%%%%%%%%%%%%%%%%%%%%%%%%%%
\myparagraph{Enhancing Graphics Rendering Hardware.}
%
The performance advantage of executing graphics rendering on either
programmable shader cores or fixed-function units varies depending on the
rendering methods and hardware designs.
%
Previous studies have explored the performance implication of graphics hardware
design by developing simulation infrastructures for graphics
workloads~\cite{bar:gon06,gub:aam19,tin:sax23,arn:par13}.
%
Additionally, several studies have aimed to improve the performance of
special-purpose hardware such as ray tracing units in graphics
hardware~\cite{cho:now23,liu:cha21} and proposed hardware accelerators for
graphics applications~\cite{lu:hua17,ram:gri09}.
%
In contrast to these works, which primarily evaluate traditional graphics
workloads, our work focuses on improving the performance of volume rendering
workloads, such as Gaussian splatting, which require blending a huge number of
fragments per pixel.

%%%%%%%%%%%%%%%%%%%%%%%%%%%%%%%%%%%%%%%%%%%%%%%%%%%%%%%%%%%%%%%%%%%%%%%%%%
%
In the context of multi-sample anti-aliasing, prior work proposed reducing the
amount of redundant shading by merging fragments from adjacent triangles in a
mesh at the quad granularity~\cite{fat:bou10}.
%
While both our work and quad-fragment merging (QFM)~\cite{fat:bou10} aim to
reduce operations by merging quads, our proposed technique differs from QFM in
many aspects.
%
Our method aims to blend \emph{overlapping primitives} along the depth
direction and applies to quads from any primitive. In contrast, QFM merges quad
fragments from small (e.g., pixel-sized) triangles that \emph{share} an edge
(i.e., \emph{connected}, \emph{non-overlapping} triangles).
%
As such, QFM is not applicable to the scenes consisting of a number of
unconnected transparent triangles, such as those in 3D Gaussian splatting.
%
In addition, our method computes the \emph{exact} color for each pixel by
offloading blending operations from ROPs to shader units, whereas QFM
\emph{approximates} pixel colors by using the color from one triangle when
multiple triangles are merged into a single quad.


\begin{figure*}[ht]
    \centering
    \includegraphics[width=\textwidth, trim=79 280 93 123, clip]{figures/framework_img.pdf}
    \caption{The pipeline of the \ENDow{} framework 
    %where each component is specified in a given configuration. 
    which yields a downstream task score and a WER score of the transcript set input to the task. The pipeline is executed for several severeties of noising and types of cleaning techniques. %Acoustic noising is applied at $k$ intensities, providing $k+1$ audio versions (including the non-noised version), eventually producing $k+2$ transcript versions (including the source transcript). Applying transcript cleaning reveals the effect of \textit{types} of noise. 
    Resulting scores are plotted on a graph for the analyses, as in, e.g., \autoref{fig_cleaning_graphs}.}
    %The pipeline is executed on $k+1$ intensities of acoustic noising (including the non-noised version), producing $k+2$ scores for the downstream task (including execution on the source transcripts). This process eventually describes the effect of the \textit{intensity} of transcript noise on the downstream task. The process is repeated for $m$ cleaning techniques ($m+1$ when including no cleaning), to analyze the benefit of a cleaning approach and the effect of the \textit{types} of transcript noise.}
    \label{fig_framework}
\end{figure*}
\section{Case Study: Model Size}\label{sec:case1} 

In this section, we use \SYSTEM{} to examine the environmental impact of model size on LLM serving.

% Smaller models are often assumed to be more environmentally friendly due to their lower resource demands. In this case study, we aim to explore whether this common belief holds true across different scenarios. Specifically, we examine whether larger models, despite their higher resource consumption, can sometimes be more sustainable by better balancing quality and throughput requirements. We conduct experiments using two model families with varying sizes, leveraging the \SYSTEM{} framework to evaluate and compare their environmental impact under different functional unit settings.

\subsection{Evaluation Methodology}

\begin{figure}[!t]
    \centering
    \includegraphics[width=0.42\textwidth]{llmfu-acl2025/figs/size/carbon_per_token_quality.pdf}
    \vspace{-0.15in}
    \caption{Carbon emission per FU for different model sizes across Qscores at QPS=1 req/s.}
    \label{fig:size_carbon_per_token_quality}
\end{figure}

\paragraph{Setup.} We evaluate various model sizes from two LLM families---Qwen2.5 (7B, 14B, 32B) and Llama2 (7B, 13B)---on an NVIDIA H100 GPU paired with an Intel Xeon 8480+ CPU.

\paragraph{Benchmarking configurations.} To assess how model sizes affect the environmental impact---or how ``green" \textcolor{green!60!black}{\faPagelines} each model is in terms of carbon efficiency---we evaluate a range of FUs by adjusting serving constraints. QPS is from 1 to 20 req/s. The Qscore ranges are set to [-5, 15] for Qwen and [-5, 10] for Llama, based on the Qscore distribution of each model family (\Cref{fig:size_newsqa_score_dist} in \Cref{sec:appendix_model_size_newsqa}). These ranges ensure broad coverage while providing sufficient outputs across model sizes that meet quality requirements. TTFT is at 1s and TPOT is at 200ms to align with human reading speed. 


\subsection{Evaluation Results}

\fuelq{Are smaller models always greener? }
We first investigate whether smaller models are always greener. \Cref{fig:size_carbon_per_token_quality} shows carbon emissions per FU across model sizes under different Qscore settings at QPS = 1 req/s. We choose a relatively low QPS to ensure all models generate enough tokens without violating performance constraints. The results indicate that the answer is \textbf{no}.  

For Qwen, at a low Qscore of -5, smaller models emit less carbon. However, as Qscore increases, carbon emissions increase for all model sizes, with smaller models increasing at a faster rate. When Qscore exceeds 5, the smallest 7B model becomes the highest emitter. At Qscore 15, the 32B model has the lowest emissions, while the 7B model emits over 1.8× more. A similar trend is seen in Llama, where larger models become greener as quality requirement rise. We confirm that larger models produce higher-quality outputs with higher Qscores in \Cref{fig:size_newsqa_score_dist} in \Cref{sec:appendix_model_size_newsqa}. This underscores the need to balance model size and output quality for lower environmental impact.
% \Cref{fig:size_newsqa_score_dist} in \Cref{sec:appendix_model_size_newsqa} shows the Qscore distributions of different model sizes within each model family, confirming that larger models produce higher-quality outputs with higher Qscores.
\begin{figure}[!t]
    \centering
    \includegraphics[width=0.42\textwidth]{llmfu-acl2025/figs/size/carbonsaving_newsqa_qwen_multiscore.pdf}
    \vspace{-0.15in}
    \caption{Carbon savings of Qwen 14B and 32B compared to 7B with Qscore low (-5) and high (15). Data for Qwen 32B are missing at QPS > 4 req/s, as larger models cannot serve intensive workloads.}
    \label{fig:size_newsqa_cs_qwen}
\end{figure}

\begin{figure}[!t]
    \centering
    \includegraphics[width=0.42\textwidth]{llmfu-acl2025/figs/size/carbonsaving_newsqa_llama2_multiscore.pdf}
    \vspace{-0.15in}
    \caption{Carbon savings of Llama 13B compared to 7B with Qscore low (-5) and high (10).}
    \label{fig:size_newsqa_cs_llama}
\end{figure}

\fuelq{When are larger models greener?}
To examine when larger models become greener, we set FUs with a broader QPS range and two quality requirements: low (Qscore = -5) and high (Qscore = 15 for Qwen, 10 for Llama). \Cref{fig:size_newsqa_cs_qwen} shows that for Qwen, larger models (14B and 32B) save more carbon compared to the 7B model under high Qscore, with the 32B saving over 40\%. However, under a low-quality requirement (Qscore = -5), larger models offer no advantage. A similar trend is seen for Llama, where the 13B model saves over 20\% carbon compared to the 7B model at high quality. Thus, \textbf{larger models become greener when output quality requirements are high}.

To explain the carbon savings shift with varying QPS, we analyze its impact on \emph{service level objective (SLO) attainment}, which refers to meeting TTFT and TPOT constraints. In \Cref{fig:size_slo_attn}, we observe that once QPS exceeds a certain threshold, SLO attainment drops, as the system becomes saturated. This explains why larger models can be greener at lower QPS: they meet performance constraints while producing higher-quality output.



\begin{figure}[!t]
    \centering
    \includegraphics[width=0.42\textwidth]{llmfu-acl2025/figs/size/slo_attn.pdf}
    \vspace{-0.15in}
    \caption{SLO attainment of Qwen and Llama families across QPS range.}
    \label{fig:size_slo_attn}
\end{figure}

\begin{figure}[!t]
    \centering
    \includegraphics[width=0.42\textwidth]{llmfu-acl2025/figs/size/heatmap_qwen_new.pdf}
    \vspace{-0.15in}
    \caption{Comparison of Qwen 7B, 14B, and 32B in FUEL. Tile colors indicate the model size with the lowest carbon per FU. Tile values are carbon savings (\%) of greenest model size compared to the second greenest.}
    \label{fig:size_heatmap}
\end{figure}


\fuelq{Does a universal greenest model size exist?}

\noindent The answer is \textbf{no}. \Cref{fig:size_heatmap} shows the relative carbon savings of Qwen 7B, 14B, and 32B across various QPS and Qscore values. No model size consistently has the lowest carbon emissions. At low QPS (1-4 req/s) with high Qscore, Qwen 32B can save up to 49\% in carbon emissions compared to the second greenest. However, as QPS increases, the 32B fails to meet the performance constraints, making the 14B the greenest. When the quality requirement is low (Qscore = 0), the 7B model is always the greenest, especially at high QPS.

\takeawaybox{
Larger models are greener under high-quality, low-QPS conditions. Smaller models become greener as QPS increases. No single model size is the greenest across all scenarios.
}


\section{Case Study: Quantization}\label{sec:case2} 

In this section, we explore how quantization affects the environmental impact of LLM serving. By reducing model weight and activation precision, quantization significantly decreases model size. For example, 4-bit quantization cuts model size by 4× compared to FP16. This reduction lowers memory usage and computational costs while maintaining accuracy. Using \SYSTEM{}, we investigate whether quantization, especially weight-only \cite{lin2024awq} and activation \cite{frantar2022gptq} quantization techniques, can improve carbon efficiency while maintaining output quality.

\subsection{Evaluation Methodology}

\paragraph{Setup.} We evaluate two widely used quantization methods: 4-bit AWQ \cite{lin2024awq} (weight-only) and W8A8 \cite{frantar2022gptq} (INT8 quantization for both weights and activations). We evaluate Qwen2.5 (7B, 14B, 32B) and Llama2 (7B, 13B) on an NVIDIA H100 GPU with an Intel Xeon 8480+ CPU. Qwen provides an official AWQ version, while Llama’s AWQ is from Hugging Face~\cite{huggingface_llama7_awq_2023, huggingface_llama13_awq_2023}. For W8A8, we quantize the models using LLM Compressor~\cite{vllm_llm_compressor_2023}, an open-source library designed for vLLM.


\paragraph{Benchmarking configurations.} Same as in \Cref{sec:case1}.

\subsection{Evaluation Results}

\fuelq{Is weight-only quantization always greener?}

\noindent The answer is \textbf{no}. \Cref{fig:quant_cs_qwen} shows the relative carbon emission savings per FU for AWQ compared to the FP16 version of Qwen under high (10) and low (-5) Qscores. Overall, AWQ’s carbon savings decline as QPS increases. For the 7B model, AWQ consistently reduces emissions, even under high Qscore. At QPS = 1 req/s and Qscore = 10, AWQ cuts emissions by over 20\% compared to FP16. This is because AWQ slightly increases the output quality of 7B (\Cref{tab:quant_qscore} in \Cref{sec:appendix_quant}), resulting in an increased number of FUs. On the other hand, the 14B model shows positive carbon savings at low Qscore (-5) but negative savings at high Qscore (10). The 32B model never achieves positive carbon savings, regardless of Qscore. We observe a similar trend for Llama in~\Cref{fig:quant_cs_llama}. As QPS increases, the carbon savings of AWQ over FP16 decline and can even become negative at high QPS. 

\begin{figure}[!t]
    \centering
    \includegraphics[width=0.45\textwidth]{llmfu-acl2025/figs/quant/cs_qwen.pdf}
    \vspace{-0.15in}
    \caption{Carbon savings of AWQ Qwen compared to the FP16 version with Qscore low (-5) and high (10). Data are missing at higher QPS for 14B and 32B, as larger models cannot serve intensive workloads.}
    \label{fig:quant_cs_qwen}
\end{figure}

\begin{figure}[!t]
    \centering
    \includegraphics[width=0.45\textwidth]{llmfu-acl2025/figs/quant/cs_llama.pdf}
    \vspace{-0.15in}
    \caption{Carbon savings of AWQ Llama compared to the FP16 version with Qscore low (-5) and high (10). Data are missing at higher QPS for 13B, as larger models cannot serve intensive workloads.}
    \label{fig:quant_cs_llama}
\end{figure}

To understand why AWQ does not always outperform FP16 in carbon savings, we analyze its impact on TTFT and TPOT speedup. Figures \ref{fig:quant_speedup_qwen} and \ref{fig:quant_speedup_llama} show that TPOT sees some speedup at low QPS but slows down at high QPS, while TTFT is always slower than FP16. This is because quantization reduces weight size, but weights are dequantized back to 16-bit during inference, adding overhead. AWQ improves TPOT in memory-bound cases at low QPS by reducing memory transfer, but this advantage diminishes as QPS increases and computation grows. Since TTFT is compute-intensive, AWQ provides no speedup.


\begin{figure}[!t]
    \centering
    \includegraphics[width=0.45\textwidth]{llmfu-acl2025/figs/quant/qwen_awq_speedup.pdf}
    \vspace{-0.15in}
    \caption{Latency speedup of AWQ Qwen compared to the FP16 version.}
    \label{fig:quant_speedup_qwen}
\end{figure}

\begin{figure}[!t]
    \centering
    \includegraphics[width=0.45\textwidth]{llmfu-acl2025/figs/quant/llama_awq_speedup.pdf}
    \vspace{-0.15in}
    \caption{Latency speedup of AWQ Llama compared to the FP16 version.}
    \label{fig:quant_speedup_llama}
\end{figure}


\takeawaybox{
Weight-only quantization reduces carbon emissions at low QPS but loses its advantage as QPS increases.
}



% \fuelq{When does AWQ become greener?}

% \begin{figure}[htbp]
%     \centering
%     \includegraphics[width=0.48\textwidth]{llmfu-acl2025/figs/quant/heatmap_qwen7_awq.pdf}
%     \caption{Comparison of Qwen-7B and 7B-AWQ in FUEL. Tile colors indicates the model with the greatest Carbon per Functional Unit. Tile values are Carbon Savings (\%) of Greenest Model in FUEL vs. Second Greenest}
%     \label{fig:quant_heat_awq_qwen7}
% \end{figure}
% \begin{figure}[htbp]
%     \centering
%     \includegraphics[width=0.48\textwidth]{llmfu-acl2025/figs/quant/heatmap_qwen14_awq.pdf}
%     \caption{Comparison of Qwen-14B and 14B-AWQ in FUEL. Tile colors indicates the model with the greatest Carbon per Functional Unit. Tile values are Carbon Savings (\%) of Greenest Model in FUEL vs. Second Greenest}
%     \label{fig:quant_heat_awq_qwen14}
% \end{figure}


% \subsubsection{Weight and Activation Quantization}
\fuelq{Is activation quantization always greener?}

\noindent Unlike weight-only quantization, activation quantization applies to both weights and activations. We compared the relative carbon savings of W8A8 compared to the FP16 version under different Qscores and QPS, and the results show that the answer is \textbf{yes}. As shown in~\Cref{fig:quant_cs_qwen_w8a8}, W8A8 consistently reduces carbon emissions for Qwen models, regardless of quality requirements. Despite some accuracy loss in the 7B model (\Cref{tab:quant_qscore} in \Cref{sec:appendix_quant}), it still achieves a 5\% carbon reduction at Qscore = 10. Unlike AWQ, W8A8 maintains stable savings even as QPS increases.


\begin{figure}[!t]
    \centering
    \includegraphics[width=0.45\textwidth]{llmfu-acl2025/figs/quant/cs_qwen_w8a8.pdf}
    \vspace{-0.15in}
    \caption{Carbon savings of W8A8 Qwen compared to the FP16 version with Qscore low (-5) and high (10). Data are missing at higher QPS for 14B and 32B, as larger models cannot serve intensive workloads.}
    \label{fig:quant_cs_qwen_w8a8}
\end{figure}

We observe a similar trend for Llama in~\Cref{fig:quant_cs_llama_w8a8}. Notably, Llama 7B improved in output quality after quantization (\Cref{tab:quant_qscore} in \Cref{sec:appendix_quant}), saving over 15\% of carbon at Qscore = 10. This shows activation quantization can break the tradeoff between FP16 and AWQ, ensuring consistent carbon savings across different FUs.

\begin{figure}[!t]
    \centering
    \includegraphics[width=0.45\textwidth]{llmfu-acl2025/figs/quant/cs_llama_w8a8.pdf}
    \caption{Carbon savings of W8A8 Llama compared to the FP16 version with Qscore low (-5) and high (10).}
    \vspace{-0.15in}
    \label{fig:quant_cs_llama_w8a8}
\end{figure}

To understand why W8A8 always outperforms FP16 in carbon savings, we analyze its impact on TTFT and TPOT speedup. Figures \ref{fig:quant_speedup_qwen_w8a8} and \ref{fig:quant_speedup_llama_w8a8} show that W8A8 consistently speeds up TPOT and TTFT across all QPS ranges. This improvement comes from reducing both weight and activation precision, which decreases the amount of data movement and computation during inference. This makes W8A8 a more sustainable choice for LLM serving, as it strikes a balance between quality and performance.


\begin{figure}[!t]
    \centering
    \includegraphics[width=0.45\textwidth]{llmfu-acl2025/figs/quant/qwen_w8a8_speedup.pdf}
    \vspace{-0.15in}
    \caption{Latency speedup of W8A8 Qwen compared to the FP16 version.}
    \label{fig:quant_speedup_qwen_w8a8}
\end{figure}

\begin{figure}[!t]
    \centering
    \includegraphics[width=0.45\textwidth]{llmfu-acl2025/figs/quant/llama_w8a8_speedup.pdf}
    \vspace{-0.15in}
    \caption{Latency speedup of W8A8 Llama compared to the FP16 version.}
    \label{fig:quant_speedup_llama_w8a8}
\end{figure}


\fuelq{Does a universal greenest quantization method exist?}

\begin{figure}[!t]
    \centering
    \includegraphics[width=0.48\textwidth]{llmfu-acl2025/figs/quant/heatmap_qwen_quant.png}
    \vspace{-0.2in}
    \caption{Comparison of FP16, AWQ and W8A8 versions of Qwen 7B/14B in \SYSTEM{}. Tile colors indicate the model with the lowest carbon per FU. Tile values are carbon savings (\%) of greenest quantization versioncompared to the second greenest.}
    \label{fig:quant_heat_w8a8_qwen}
\end{figure}

\noindent The answer is \textbf{no}. \Cref{fig:quant_heat_w8a8_qwen} shows the relative carbon savings of FP16, AWQ, and W8A8 models across various QPS and Qscores for Qwen 7B and 14B. For Qwen 14B, W8A8 outperforms in all scenarios, with carbon savings increasing as QPS rises. However, for Qwen 7B, AWQ maintains slightly better quality at low QPS, while W8A8 lags behind at high QPS and high-quality requirements due to its slight accuracy loss (\Cref{tab:quant_qscore} in \Cref{sec:appendix_quant}).

\takeawaybox{Weight and activation quantization methods, like W8A8, hold significant potential for reducing carbon emissions in LLM serving, particularly for larger models. }




% \begin{figure}[htbp]
%     \centering
%     \includegraphics[width=0.48\textwidth]{llmfu-acl2025/figs/quant/heatmap_qwen7_w8a8.pdf}
%     \caption{Comparison of Qwen-7B, 7B-W8A8 and 7B-AWQ in FUEL. Tile colors indicate the model with the greatest Carbon per Functional Unit. Tile values are Carbon Savings (\%) of Greenest Model in FUEL vs. Second Greenest}
%     \label{fig:quant_heat_w8a8_qwen}
% \end{figure}

% \begin{figure}[htbp]
%     \centering
%     \includegraphics[width=0.48\textwidth]{llmfu-acl2025/figs/quant/heatmap_qwen14_w8a8.pdf}
%     \caption{Comparison of Qwen-14B, 14B-W8A8 and 14B-AWQ in FUEL. Tile colors indicate the model with the greatest Carbon per Functional Unit. Tile values are Carbon Savings (\%) of Greenest Model in FUEL vs. Second Greenest}
%     \label{fig:quant_heat_w8a8_qwen14}
% \end{figure}




\section{Case Study: Hardware}\label{sec:case3} 

In this section, we examine how hardware platform affects the environmental impact of LLM serving. Using \SYSTEM{}, we investigate whether more advanced hardware can enhance carbon efficiency while maintaining output quality.

\subsection{Evaluation Methodology}

\paragraph{Setup.} We conduct experiments on two GPU servers with different hardware configurations, one older and one newer, as detailed in~\Cref{tab:hardware_specifications}. For fair comparisons, we use a single GPU per server for all experiments. We evaluate the Qwen2.5 (7B, 14B) and Llama2 (7B, 13B).

\paragraph{Benchmarking configurations.} Same as in \Cref{sec:case1}.

\begin{table}[!t]
    \centering
    \caption{Hardware platform specifications in this paper.}
    \footnotesize
    \begin{tabular}{l|p{2.5cm}|p{2.5cm}}
        \toprule
        \textbf{Specification} & \textbf{L40 server} & \textbf{H100 server} \\
        \midrule        
        \textbf{GPU} & 4 $\times$ L40 & 8$\times$ H100 \\
        \footnotesize{TDP} & \footnotesize{300W} & \footnotesize{350W} \\
        \footnotesize{Process size} & \footnotesize{5nm} & \footnotesize{5nm} \\
        \footnotesize{Die size} & \footnotesize{609 mm$^2$} & \footnotesize{814 mm$^2$} \\
        \footnotesize{GPU memory} & \footnotesize{40GB} & \footnotesize{80GB} \\
        \footnotesize{Release Year} & \footnotesize{2022} & \footnotesize{2023} \\
        \hline
        \textbf{CPU} & AMD EPYC 7443 & Intel Xeon 8480+ \\
        \footnotesize{TDP} & \footnotesize{200W} & \footnotesize{350W} \\
        \footnotesize{Process size} & \footnotesize{7nm} & \footnotesize{10nm} \\
        \footnotesize{Die size} & \footnotesize{4$\times$81 mm$^2$} & \footnotesize{4$\times$477 mm$^2$} \\
        \footnotesize{CPU memory} & 504GB & 1031GB \\
         \footnotesize{Release Year} & \footnotesize{2021} & \footnotesize{2023} \\
        \bottomrule
    \end{tabular}
    \label{tab:hardware_specifications}
\end{table}

\subsection{Evaluation Results}

\fuelq{How does different hardware contribute to total carbon emissions?}

\noindent \Cref{fig:hw_carbon_breakdown} shows the breakdown of carbon emissions per FU for Qwen and Llama 7B models on different hardware platforms, separating operational and embodied carbon. It is worth noting that different hardware contributes to different embodied carbon per FU, due to differences in the total embodied carbon for each hardware. The L40 platform has lower total embodied carbon than the H100, with values of 26.6 and 29.92 kgCO\textsubscript{2}eq repectively. These differences are based on calculations using the ACT modeling tool~\cite{gupta2022act} and are due to hardware factors such as process and die size. The difference is even more pronounced between the AMD EPYC 7443 and Intel Xeon 8480+ CPUs, with the AMD CPU having 9.98 kgCO\textsubscript{2}eq, compared to the Intel's 42.81 kgCO\textsubscript{2}eq, over 4x higher.

\begin{figure}[!t]
    \centering
    \includegraphics[width=0.45\textwidth]{llmfu-acl2025/figs/hw/carbon_breakdown.pdf}
    \vspace{-0.15in}
    \caption{Breakdown of Carbon emission per FU for Qwen and Llama 7B models on different hardware platforms, evaluated in FUEL with Qscore=0.}
    \label{fig:hw_carbon_breakdown}
\end{figure}

Advanced hardware like the H100 offers better performance but higher embodied carbon. Extending hardware lifetime can yield more carbon savings, especially considering the large difference in embodied carbon between older and newer devices.


\fuelq{Is LLM serving on advanced hardware greener?}

\noindent \Cref{fig:hw_carbon_models} shows the carbon emissions per FU for the Qwen and Llama model families on two hardware platforms. At low QPS, the L40 server consistently has lower carbon emissions than the H100. This means that the answer is \textbf{no}: advanced hardware is not necessarily greener.

\begin{figure}[!t]
    \centering
    \includegraphics[width=0.42\textwidth]{llmfu-acl2025/figs/hw/carbon_models.pdf}
    \vspace{-0.15in}
    \caption{Carbon emission per FU of Qwen and Llama model families on different hardware platforms, evaluated in FUEL with Qscore=0.}
    \label{fig:hw_carbon_models}
\end{figure}

The main advantage of advanced hardware like the H100 is its ability to produce higher-quality outputs and meet performance constraints, as shown in~\Cref{fig:hw_slo_attn}. Although advanced hardware may not be greener, it provides better performance and supports higher-quality LLM serving at scale.

\begin{figure}[!t]
    \centering
    \includegraphics[width=0.42\textwidth]{llmfu-acl2025/figs/hw/slo_attn.pdf}
    \vspace{-0.15in}
    \caption{SLO attainment of Qwen and Llama model families on different hardware platforms.}
    \label{fig:hw_slo_attn}
\end{figure}

\begin{figure}[!t]
    \centering
    \includegraphics[width=0.48\textwidth]{llmfu-acl2025/figs/hw/hw_qwen_heatmap.png}
    \vspace{-0.2in}
    \caption{Comparison of Qwen 7B and 14B on different hardware platforms in FUEL. Tile colors indicate the hardware with the lowest carbon emission per FU. Tile values are carbon savings (\%) of the greenest hardware compared to the second greenest.}
    \label{fig:hw_heatmap_qwen_sep}
\end{figure}

\fuelq{How to choose greener hardware?}

\noindent To answer this question, we run experiments across different FUs with varying QPS and Qscores. \Cref{fig:hw_heatmap_qwen_sep} shows the relative carbon savings of L40 and H100 servers for Qwen 7B and 14B. Hardware carbon efficiency depends mainly on model size and QPS, with a minor influence from Qscore. Newer hardware is more carbon efficient at high QPS, while older hardware is better at low QPS. These findings underscore the sustainability benefits of reusing older hardware to cut carbon emissions while maintaining performance and quality.


\takeawaybox{
Advanced hardware offers higher performance but is not always greener due to higher embodied carbon. Older hardware can achieve lower carbon emissions while still meeting quality and performance constraints.
}


\section{Conclusion}
In this work, we propose a simple yet effective approach, called SMILE, for graph few-shot learning with fewer tasks. Specifically, we introduce a novel dual-level mixup strategy, including within-task and across-task mixup, for enriching the diversity of nodes within each task and the diversity of tasks. Also, we incorporate the degree-based prior information to learn expressive node embeddings. Theoretically, we prove that SMILE effectively enhances the model's generalization performance. Empirically, we conduct extensive experiments on multiple benchmarks and the results suggest that SMILE significantly outperforms other baselines, including both in-domain and cross-domain few-shot settings.


%\section*{Acknowledgments}

% Bibliography entries for the entire Anthology, followed by custom entries
%\bibliography{anthology,custom}
% Custom bibliography entries only
\bibliography{refs}

\appendix

\subsection{Lloyd-Max Algorithm}
\label{subsec:Lloyd-Max}
For a given quantization bitwidth $B$ and an operand $\bm{X}$, the Lloyd-Max algorithm finds $2^B$ quantization levels $\{\hat{x}_i\}_{i=1}^{2^B}$ such that quantizing $\bm{X}$ by rounding each scalar in $\bm{X}$ to the nearest quantization level minimizes the quantization MSE. 

The algorithm starts with an initial guess of quantization levels and then iteratively computes quantization thresholds $\{\tau_i\}_{i=1}^{2^B-1}$ and updates quantization levels $\{\hat{x}_i\}_{i=1}^{2^B}$. Specifically, at iteration $n$, thresholds are set to the midpoints of the previous iteration's levels:
\begin{align*}
    \tau_i^{(n)}=\frac{\hat{x}_i^{(n-1)}+\hat{x}_{i+1}^{(n-1)}}2 \text{ for } i=1\ldots 2^B-1
\end{align*}
Subsequently, the quantization levels are re-computed as conditional means of the data regions defined by the new thresholds:
\begin{align*}
    \hat{x}_i^{(n)}=\mathbb{E}\left[ \bm{X} \big| \bm{X}\in [\tau_{i-1}^{(n)},\tau_i^{(n)}] \right] \text{ for } i=1\ldots 2^B
\end{align*}
where to satisfy boundary conditions we have $\tau_0=-\infty$ and $\tau_{2^B}=\infty$. The algorithm iterates the above steps until convergence.

Figure \ref{fig:lm_quant} compares the quantization levels of a $7$-bit floating point (E3M3) quantizer (left) to a $7$-bit Lloyd-Max quantizer (right) when quantizing a layer of weights from the GPT3-126M model at a per-tensor granularity. As shown, the Lloyd-Max quantizer achieves substantially lower quantization MSE. Further, Table \ref{tab:FP7_vs_LM7} shows the superior perplexity achieved by Lloyd-Max quantizers for bitwidths of $7$, $6$ and $5$. The difference between the quantizers is clear at 5 bits, where per-tensor FP quantization incurs a drastic and unacceptable increase in perplexity, while Lloyd-Max quantization incurs a much smaller increase. Nevertheless, we note that even the optimal Lloyd-Max quantizer incurs a notable ($\sim 1.5$) increase in perplexity due to the coarse granularity of quantization. 

\begin{figure}[h]
  \centering
  \includegraphics[width=0.7\linewidth]{sections/figures/LM7_FP7.pdf}
  \caption{\small Quantization levels and the corresponding quantization MSE of Floating Point (left) vs Lloyd-Max (right) Quantizers for a layer of weights in the GPT3-126M model.}
  \label{fig:lm_quant}
\end{figure}

\begin{table}[h]\scriptsize
\begin{center}
\caption{\label{tab:FP7_vs_LM7} \small Comparing perplexity (lower is better) achieved by floating point quantizers and Lloyd-Max quantizers on a GPT3-126M model for the Wikitext-103 dataset.}
\begin{tabular}{c|cc|c}
\hline
 \multirow{2}{*}{\textbf{Bitwidth}} & \multicolumn{2}{|c|}{\textbf{Floating-Point Quantizer}} & \textbf{Lloyd-Max Quantizer} \\
 & Best Format & Wikitext-103 Perplexity & Wikitext-103 Perplexity \\
\hline
7 & E3M3 & 18.32 & 18.27 \\
6 & E3M2 & 19.07 & 18.51 \\
5 & E4M0 & 43.89 & 19.71 \\
\hline
\end{tabular}
\end{center}
\end{table}

\subsection{Proof of Local Optimality of LO-BCQ}
\label{subsec:lobcq_opt_proof}
For a given block $\bm{b}_j$, the quantization MSE during LO-BCQ can be empirically evaluated as $\frac{1}{L_b}\lVert \bm{b}_j- \bm{\hat{b}}_j\rVert^2_2$ where $\bm{\hat{b}}_j$ is computed from equation (\ref{eq:clustered_quantization_definition}) as $C_{f(\bm{b}_j)}(\bm{b}_j)$. Further, for a given block cluster $\mathcal{B}_i$, we compute the quantization MSE as $\frac{1}{|\mathcal{B}_{i}|}\sum_{\bm{b} \in \mathcal{B}_{i}} \frac{1}{L_b}\lVert \bm{b}- C_i^{(n)}(\bm{b})\rVert^2_2$. Therefore, at the end of iteration $n$, we evaluate the overall quantization MSE $J^{(n)}$ for a given operand $\bm{X}$ composed of $N_c$ block clusters as:
\begin{align*}
    \label{eq:mse_iter_n}
    J^{(n)} = \frac{1}{N_c} \sum_{i=1}^{N_c} \frac{1}{|\mathcal{B}_{i}^{(n)}|}\sum_{\bm{v} \in \mathcal{B}_{i}^{(n)}} \frac{1}{L_b}\lVert \bm{b}- B_i^{(n)}(\bm{b})\rVert^2_2
\end{align*}

At the end of iteration $n$, the codebooks are updated from $\mathcal{C}^{(n-1)}$ to $\mathcal{C}^{(n)}$. However, the mapping of a given vector $\bm{b}_j$ to quantizers $\mathcal{C}^{(n)}$ remains as  $f^{(n)}(\bm{b}_j)$. At the next iteration, during the vector clustering step, $f^{(n+1)}(\bm{b}_j)$ finds new mapping of $\bm{b}_j$ to updated codebooks $\mathcal{C}^{(n)}$ such that the quantization MSE over the candidate codebooks is minimized. Therefore, we obtain the following result for $\bm{b}_j$:
\begin{align*}
\frac{1}{L_b}\lVert \bm{b}_j - C_{f^{(n+1)}(\bm{b}_j)}^{(n)}(\bm{b}_j)\rVert^2_2 \le \frac{1}{L_b}\lVert \bm{b}_j - C_{f^{(n)}(\bm{b}_j)}^{(n)}(\bm{b}_j)\rVert^2_2
\end{align*}

That is, quantizing $\bm{b}_j$ at the end of the block clustering step of iteration $n+1$ results in lower quantization MSE compared to quantizing at the end of iteration $n$. Since this is true for all $\bm{b} \in \bm{X}$, we assert the following:
\begin{equation}
\begin{split}
\label{eq:mse_ineq_1}
    \tilde{J}^{(n+1)} &= \frac{1}{N_c} \sum_{i=1}^{N_c} \frac{1}{|\mathcal{B}_{i}^{(n+1)}|}\sum_{\bm{b} \in \mathcal{B}_{i}^{(n+1)}} \frac{1}{L_b}\lVert \bm{b} - C_i^{(n)}(b)\rVert^2_2 \le J^{(n)}
\end{split}
\end{equation}
where $\tilde{J}^{(n+1)}$ is the the quantization MSE after the vector clustering step at iteration $n+1$.

Next, during the codebook update step (\ref{eq:quantizers_update}) at iteration $n+1$, the per-cluster codebooks $\mathcal{C}^{(n)}$ are updated to $\mathcal{C}^{(n+1)}$ by invoking the Lloyd-Max algorithm \citep{Lloyd}. We know that for any given value distribution, the Lloyd-Max algorithm minimizes the quantization MSE. Therefore, for a given vector cluster $\mathcal{B}_i$ we obtain the following result:

\begin{equation}
    \frac{1}{|\mathcal{B}_{i}^{(n+1)}|}\sum_{\bm{b} \in \mathcal{B}_{i}^{(n+1)}} \frac{1}{L_b}\lVert \bm{b}- C_i^{(n+1)}(\bm{b})\rVert^2_2 \le \frac{1}{|\mathcal{B}_{i}^{(n+1)}|}\sum_{\bm{b} \in \mathcal{B}_{i}^{(n+1)}} \frac{1}{L_b}\lVert \bm{b}- C_i^{(n)}(\bm{b})\rVert^2_2
\end{equation}

The above equation states that quantizing the given block cluster $\mathcal{B}_i$ after updating the associated codebook from $C_i^{(n)}$ to $C_i^{(n+1)}$ results in lower quantization MSE. Since this is true for all the block clusters, we derive the following result: 
\begin{equation}
\begin{split}
\label{eq:mse_ineq_2}
     J^{(n+1)} &= \frac{1}{N_c} \sum_{i=1}^{N_c} \frac{1}{|\mathcal{B}_{i}^{(n+1)}|}\sum_{\bm{b} \in \mathcal{B}_{i}^{(n+1)}} \frac{1}{L_b}\lVert \bm{b}- C_i^{(n+1)}(\bm{b})\rVert^2_2  \le \tilde{J}^{(n+1)}   
\end{split}
\end{equation}

Following (\ref{eq:mse_ineq_1}) and (\ref{eq:mse_ineq_2}), we find that the quantization MSE is non-increasing for each iteration, that is, $J^{(1)} \ge J^{(2)} \ge J^{(3)} \ge \ldots \ge J^{(M)}$ where $M$ is the maximum number of iterations. 
%Therefore, we can say that if the algorithm converges, then it must be that it has converged to a local minimum. 
\hfill $\blacksquare$


\begin{figure}
    \begin{center}
    \includegraphics[width=0.5\textwidth]{sections//figures/mse_vs_iter.pdf}
    \end{center}
    \caption{\small NMSE vs iterations during LO-BCQ compared to other block quantization proposals}
    \label{fig:nmse_vs_iter}
\end{figure}

Figure \ref{fig:nmse_vs_iter} shows the empirical convergence of LO-BCQ across several block lengths and number of codebooks. Also, the MSE achieved by LO-BCQ is compared to baselines such as MXFP and VSQ. As shown, LO-BCQ converges to a lower MSE than the baselines. Further, we achieve better convergence for larger number of codebooks ($N_c$) and for a smaller block length ($L_b$), both of which increase the bitwidth of BCQ (see Eq \ref{eq:bitwidth_bcq}).


\subsection{Additional Accuracy Results}
%Table \ref{tab:lobcq_config} lists the various LOBCQ configurations and their corresponding bitwidths.
\begin{table}
\setlength{\tabcolsep}{4.75pt}
\begin{center}
\caption{\label{tab:lobcq_config} Various LO-BCQ configurations and their bitwidths.}
\begin{tabular}{|c||c|c|c|c||c|c||c|} 
\hline
 & \multicolumn{4}{|c||}{$L_b=8$} & \multicolumn{2}{|c||}{$L_b=4$} & $L_b=2$ \\
 \hline
 \backslashbox{$L_A$\kern-1em}{\kern-1em$N_c$} & 2 & 4 & 8 & 16 & 2 & 4 & 2 \\
 \hline
 64 & 4.25 & 4.375 & 4.5 & 4.625 & 4.375 & 4.625 & 4.625\\
 \hline
 32 & 4.375 & 4.5 & 4.625& 4.75 & 4.5 & 4.75 & 4.75 \\
 \hline
 16 & 4.625 & 4.75& 4.875 & 5 & 4.75 & 5 & 5 \\
 \hline
\end{tabular}
\end{center}
\end{table}

%\subsection{Perplexity achieved by various LO-BCQ configurations on Wikitext-103 dataset}

\begin{table} \centering
\begin{tabular}{|c||c|c|c|c||c|c||c|} 
\hline
 $L_b \rightarrow$& \multicolumn{4}{c||}{8} & \multicolumn{2}{c||}{4} & 2\\
 \hline
 \backslashbox{$L_A$\kern-1em}{\kern-1em$N_c$} & 2 & 4 & 8 & 16 & 2 & 4 & 2  \\
 %$N_c \rightarrow$ & 2 & 4 & 8 & 16 & 2 & 4 & 2 \\
 \hline
 \hline
 \multicolumn{8}{c}{GPT3-1.3B (FP32 PPL = 9.98)} \\ 
 \hline
 \hline
 64 & 10.40 & 10.23 & 10.17 & 10.15 &  10.28 & 10.18 & 10.19 \\
 \hline
 32 & 10.25 & 10.20 & 10.15 & 10.12 &  10.23 & 10.17 & 10.17 \\
 \hline
 16 & 10.22 & 10.16 & 10.10 & 10.09 &  10.21 & 10.14 & 10.16 \\
 \hline
  \hline
 \multicolumn{8}{c}{GPT3-8B (FP32 PPL = 7.38)} \\ 
 \hline
 \hline
 64 & 7.61 & 7.52 & 7.48 &  7.47 &  7.55 &  7.49 & 7.50 \\
 \hline
 32 & 7.52 & 7.50 & 7.46 &  7.45 &  7.52 &  7.48 & 7.48  \\
 \hline
 16 & 7.51 & 7.48 & 7.44 &  7.44 &  7.51 &  7.49 & 7.47  \\
 \hline
\end{tabular}
\caption{\label{tab:ppl_gpt3_abalation} Wikitext-103 perplexity across GPT3-1.3B and 8B models.}
\end{table}

\begin{table} \centering
\begin{tabular}{|c||c|c|c|c||} 
\hline
 $L_b \rightarrow$& \multicolumn{4}{c||}{8}\\
 \hline
 \backslashbox{$L_A$\kern-1em}{\kern-1em$N_c$} & 2 & 4 & 8 & 16 \\
 %$N_c \rightarrow$ & 2 & 4 & 8 & 16 & 2 & 4 & 2 \\
 \hline
 \hline
 \multicolumn{5}{|c|}{Llama2-7B (FP32 PPL = 5.06)} \\ 
 \hline
 \hline
 64 & 5.31 & 5.26 & 5.19 & 5.18  \\
 \hline
 32 & 5.23 & 5.25 & 5.18 & 5.15  \\
 \hline
 16 & 5.23 & 5.19 & 5.16 & 5.14  \\
 \hline
 \multicolumn{5}{|c|}{Nemotron4-15B (FP32 PPL = 5.87)} \\ 
 \hline
 \hline
 64  & 6.3 & 6.20 & 6.13 & 6.08  \\
 \hline
 32  & 6.24 & 6.12 & 6.07 & 6.03  \\
 \hline
 16  & 6.12 & 6.14 & 6.04 & 6.02  \\
 \hline
 \multicolumn{5}{|c|}{Nemotron4-340B (FP32 PPL = 3.48)} \\ 
 \hline
 \hline
 64 & 3.67 & 3.62 & 3.60 & 3.59 \\
 \hline
 32 & 3.63 & 3.61 & 3.59 & 3.56 \\
 \hline
 16 & 3.61 & 3.58 & 3.57 & 3.55 \\
 \hline
\end{tabular}
\caption{\label{tab:ppl_llama7B_nemo15B} Wikitext-103 perplexity compared to FP32 baseline in Llama2-7B and Nemotron4-15B, 340B models}
\end{table}

%\subsection{Perplexity achieved by various LO-BCQ configurations on MMLU dataset}


\begin{table} \centering
\begin{tabular}{|c||c|c|c|c||c|c|c|c|} 
\hline
 $L_b \rightarrow$& \multicolumn{4}{c||}{8} & \multicolumn{4}{c||}{8}\\
 \hline
 \backslashbox{$L_A$\kern-1em}{\kern-1em$N_c$} & 2 & 4 & 8 & 16 & 2 & 4 & 8 & 16  \\
 %$N_c \rightarrow$ & 2 & 4 & 8 & 16 & 2 & 4 & 2 \\
 \hline
 \hline
 \multicolumn{5}{|c|}{Llama2-7B (FP32 Accuracy = 45.8\%)} & \multicolumn{4}{|c|}{Llama2-70B (FP32 Accuracy = 69.12\%)} \\ 
 \hline
 \hline
 64 & 43.9 & 43.4 & 43.9 & 44.9 & 68.07 & 68.27 & 68.17 & 68.75 \\
 \hline
 32 & 44.5 & 43.8 & 44.9 & 44.5 & 68.37 & 68.51 & 68.35 & 68.27  \\
 \hline
 16 & 43.9 & 42.7 & 44.9 & 45 & 68.12 & 68.77 & 68.31 & 68.59  \\
 \hline
 \hline
 \multicolumn{5}{|c|}{GPT3-22B (FP32 Accuracy = 38.75\%)} & \multicolumn{4}{|c|}{Nemotron4-15B (FP32 Accuracy = 64.3\%)} \\ 
 \hline
 \hline
 64 & 36.71 & 38.85 & 38.13 & 38.92 & 63.17 & 62.36 & 63.72 & 64.09 \\
 \hline
 32 & 37.95 & 38.69 & 39.45 & 38.34 & 64.05 & 62.30 & 63.8 & 64.33  \\
 \hline
 16 & 38.88 & 38.80 & 38.31 & 38.92 & 63.22 & 63.51 & 63.93 & 64.43  \\
 \hline
\end{tabular}
\caption{\label{tab:mmlu_abalation} Accuracy on MMLU dataset across GPT3-22B, Llama2-7B, 70B and Nemotron4-15B models.}
\end{table}


%\subsection{Perplexity achieved by various LO-BCQ configurations on LM evaluation harness}

\begin{table} \centering
\begin{tabular}{|c||c|c|c|c||c|c|c|c|} 
\hline
 $L_b \rightarrow$& \multicolumn{4}{c||}{8} & \multicolumn{4}{c||}{8}\\
 \hline
 \backslashbox{$L_A$\kern-1em}{\kern-1em$N_c$} & 2 & 4 & 8 & 16 & 2 & 4 & 8 & 16  \\
 %$N_c \rightarrow$ & 2 & 4 & 8 & 16 & 2 & 4 & 2 \\
 \hline
 \hline
 \multicolumn{5}{|c|}{Race (FP32 Accuracy = 37.51\%)} & \multicolumn{4}{|c|}{Boolq (FP32 Accuracy = 64.62\%)} \\ 
 \hline
 \hline
 64 & 36.94 & 37.13 & 36.27 & 37.13 & 63.73 & 62.26 & 63.49 & 63.36 \\
 \hline
 32 & 37.03 & 36.36 & 36.08 & 37.03 & 62.54 & 63.51 & 63.49 & 63.55  \\
 \hline
 16 & 37.03 & 37.03 & 36.46 & 37.03 & 61.1 & 63.79 & 63.58 & 63.33  \\
 \hline
 \hline
 \multicolumn{5}{|c|}{Winogrande (FP32 Accuracy = 58.01\%)} & \multicolumn{4}{|c|}{Piqa (FP32 Accuracy = 74.21\%)} \\ 
 \hline
 \hline
 64 & 58.17 & 57.22 & 57.85 & 58.33 & 73.01 & 73.07 & 73.07 & 72.80 \\
 \hline
 32 & 59.12 & 58.09 & 57.85 & 58.41 & 73.01 & 73.94 & 72.74 & 73.18  \\
 \hline
 16 & 57.93 & 58.88 & 57.93 & 58.56 & 73.94 & 72.80 & 73.01 & 73.94  \\
 \hline
\end{tabular}
\caption{\label{tab:mmlu_abalation} Accuracy on LM evaluation harness tasks on GPT3-1.3B model.}
\end{table}

\begin{table} \centering
\begin{tabular}{|c||c|c|c|c||c|c|c|c|} 
\hline
 $L_b \rightarrow$& \multicolumn{4}{c||}{8} & \multicolumn{4}{c||}{8}\\
 \hline
 \backslashbox{$L_A$\kern-1em}{\kern-1em$N_c$} & 2 & 4 & 8 & 16 & 2 & 4 & 8 & 16  \\
 %$N_c \rightarrow$ & 2 & 4 & 8 & 16 & 2 & 4 & 2 \\
 \hline
 \hline
 \multicolumn{5}{|c|}{Race (FP32 Accuracy = 41.34\%)} & \multicolumn{4}{|c|}{Boolq (FP32 Accuracy = 68.32\%)} \\ 
 \hline
 \hline
 64 & 40.48 & 40.10 & 39.43 & 39.90 & 69.20 & 68.41 & 69.45 & 68.56 \\
 \hline
 32 & 39.52 & 39.52 & 40.77 & 39.62 & 68.32 & 67.43 & 68.17 & 69.30  \\
 \hline
 16 & 39.81 & 39.71 & 39.90 & 40.38 & 68.10 & 66.33 & 69.51 & 69.42  \\
 \hline
 \hline
 \multicolumn{5}{|c|}{Winogrande (FP32 Accuracy = 67.88\%)} & \multicolumn{4}{|c|}{Piqa (FP32 Accuracy = 78.78\%)} \\ 
 \hline
 \hline
 64 & 66.85 & 66.61 & 67.72 & 67.88 & 77.31 & 77.42 & 77.75 & 77.64 \\
 \hline
 32 & 67.25 & 67.72 & 67.72 & 67.00 & 77.31 & 77.04 & 77.80 & 77.37  \\
 \hline
 16 & 68.11 & 68.90 & 67.88 & 67.48 & 77.37 & 78.13 & 78.13 & 77.69  \\
 \hline
\end{tabular}
\caption{\label{tab:mmlu_abalation} Accuracy on LM evaluation harness tasks on GPT3-8B model.}
\end{table}

\begin{table} \centering
\begin{tabular}{|c||c|c|c|c||c|c|c|c|} 
\hline
 $L_b \rightarrow$& \multicolumn{4}{c||}{8} & \multicolumn{4}{c||}{8}\\
 \hline
 \backslashbox{$L_A$\kern-1em}{\kern-1em$N_c$} & 2 & 4 & 8 & 16 & 2 & 4 & 8 & 16  \\
 %$N_c \rightarrow$ & 2 & 4 & 8 & 16 & 2 & 4 & 2 \\
 \hline
 \hline
 \multicolumn{5}{|c|}{Race (FP32 Accuracy = 40.67\%)} & \multicolumn{4}{|c|}{Boolq (FP32 Accuracy = 76.54\%)} \\ 
 \hline
 \hline
 64 & 40.48 & 40.10 & 39.43 & 39.90 & 75.41 & 75.11 & 77.09 & 75.66 \\
 \hline
 32 & 39.52 & 39.52 & 40.77 & 39.62 & 76.02 & 76.02 & 75.96 & 75.35  \\
 \hline
 16 & 39.81 & 39.71 & 39.90 & 40.38 & 75.05 & 73.82 & 75.72 & 76.09  \\
 \hline
 \hline
 \multicolumn{5}{|c|}{Winogrande (FP32 Accuracy = 70.64\%)} & \multicolumn{4}{|c|}{Piqa (FP32 Accuracy = 79.16\%)} \\ 
 \hline
 \hline
 64 & 69.14 & 70.17 & 70.17 & 70.56 & 78.24 & 79.00 & 78.62 & 78.73 \\
 \hline
 32 & 70.96 & 69.69 & 71.27 & 69.30 & 78.56 & 79.49 & 79.16 & 78.89  \\
 \hline
 16 & 71.03 & 69.53 & 69.69 & 70.40 & 78.13 & 79.16 & 79.00 & 79.00  \\
 \hline
\end{tabular}
\caption{\label{tab:mmlu_abalation} Accuracy on LM evaluation harness tasks on GPT3-22B model.}
\end{table}

\begin{table} \centering
\begin{tabular}{|c||c|c|c|c||c|c|c|c|} 
\hline
 $L_b \rightarrow$& \multicolumn{4}{c||}{8} & \multicolumn{4}{c||}{8}\\
 \hline
 \backslashbox{$L_A$\kern-1em}{\kern-1em$N_c$} & 2 & 4 & 8 & 16 & 2 & 4 & 8 & 16  \\
 %$N_c \rightarrow$ & 2 & 4 & 8 & 16 & 2 & 4 & 2 \\
 \hline
 \hline
 \multicolumn{5}{|c|}{Race (FP32 Accuracy = 44.4\%)} & \multicolumn{4}{|c|}{Boolq (FP32 Accuracy = 79.29\%)} \\ 
 \hline
 \hline
 64 & 42.49 & 42.51 & 42.58 & 43.45 & 77.58 & 77.37 & 77.43 & 78.1 \\
 \hline
 32 & 43.35 & 42.49 & 43.64 & 43.73 & 77.86 & 75.32 & 77.28 & 77.86  \\
 \hline
 16 & 44.21 & 44.21 & 43.64 & 42.97 & 78.65 & 77 & 76.94 & 77.98  \\
 \hline
 \hline
 \multicolumn{5}{|c|}{Winogrande (FP32 Accuracy = 69.38\%)} & \multicolumn{4}{|c|}{Piqa (FP32 Accuracy = 78.07\%)} \\ 
 \hline
 \hline
 64 & 68.9 & 68.43 & 69.77 & 68.19 & 77.09 & 76.82 & 77.09 & 77.86 \\
 \hline
 32 & 69.38 & 68.51 & 68.82 & 68.90 & 78.07 & 76.71 & 78.07 & 77.86  \\
 \hline
 16 & 69.53 & 67.09 & 69.38 & 68.90 & 77.37 & 77.8 & 77.91 & 77.69  \\
 \hline
\end{tabular}
\caption{\label{tab:mmlu_abalation} Accuracy on LM evaluation harness tasks on Llama2-7B model.}
\end{table}

\begin{table} \centering
\begin{tabular}{|c||c|c|c|c||c|c|c|c|} 
\hline
 $L_b \rightarrow$& \multicolumn{4}{c||}{8} & \multicolumn{4}{c||}{8}\\
 \hline
 \backslashbox{$L_A$\kern-1em}{\kern-1em$N_c$} & 2 & 4 & 8 & 16 & 2 & 4 & 8 & 16  \\
 %$N_c \rightarrow$ & 2 & 4 & 8 & 16 & 2 & 4 & 2 \\
 \hline
 \hline
 \multicolumn{5}{|c|}{Race (FP32 Accuracy = 48.8\%)} & \multicolumn{4}{|c|}{Boolq (FP32 Accuracy = 85.23\%)} \\ 
 \hline
 \hline
 64 & 49.00 & 49.00 & 49.28 & 48.71 & 82.82 & 84.28 & 84.03 & 84.25 \\
 \hline
 32 & 49.57 & 48.52 & 48.33 & 49.28 & 83.85 & 84.46 & 84.31 & 84.93  \\
 \hline
 16 & 49.85 & 49.09 & 49.28 & 48.99 & 85.11 & 84.46 & 84.61 & 83.94  \\
 \hline
 \hline
 \multicolumn{5}{|c|}{Winogrande (FP32 Accuracy = 79.95\%)} & \multicolumn{4}{|c|}{Piqa (FP32 Accuracy = 81.56\%)} \\ 
 \hline
 \hline
 64 & 78.77 & 78.45 & 78.37 & 79.16 & 81.45 & 80.69 & 81.45 & 81.5 \\
 \hline
 32 & 78.45 & 79.01 & 78.69 & 80.66 & 81.56 & 80.58 & 81.18 & 81.34  \\
 \hline
 16 & 79.95 & 79.56 & 79.79 & 79.72 & 81.28 & 81.66 & 81.28 & 80.96  \\
 \hline
\end{tabular}
\caption{\label{tab:mmlu_abalation} Accuracy on LM evaluation harness tasks on Llama2-70B model.}
\end{table}

%\section{MSE Studies}
%\textcolor{red}{TODO}


\subsection{Number Formats and Quantization Method}
\label{subsec:numFormats_quantMethod}
\subsubsection{Integer Format}
An $n$-bit signed integer (INT) is typically represented with a 2s-complement format \citep{yao2022zeroquant,xiao2023smoothquant,dai2021vsq}, where the most significant bit denotes the sign.

\subsubsection{Floating Point Format}
An $n$-bit signed floating point (FP) number $x$ comprises of a 1-bit sign ($x_{\mathrm{sign}}$), $B_m$-bit mantissa ($x_{\mathrm{mant}}$) and $B_e$-bit exponent ($x_{\mathrm{exp}}$) such that $B_m+B_e=n-1$. The associated constant exponent bias ($E_{\mathrm{bias}}$) is computed as $(2^{{B_e}-1}-1)$. We denote this format as $E_{B_e}M_{B_m}$.  

\subsubsection{Quantization Scheme}
\label{subsec:quant_method}
A quantization scheme dictates how a given unquantized tensor is converted to its quantized representation. We consider FP formats for the purpose of illustration. Given an unquantized tensor $\bm{X}$ and an FP format $E_{B_e}M_{B_m}$, we first, we compute the quantization scale factor $s_X$ that maps the maximum absolute value of $\bm{X}$ to the maximum quantization level of the $E_{B_e}M_{B_m}$ format as follows:
\begin{align}
\label{eq:sf}
    s_X = \frac{\mathrm{max}(|\bm{X}|)}{\mathrm{max}(E_{B_e}M_{B_m})}
\end{align}
In the above equation, $|\cdot|$ denotes the absolute value function.

Next, we scale $\bm{X}$ by $s_X$ and quantize it to $\hat{\bm{X}}$ by rounding it to the nearest quantization level of $E_{B_e}M_{B_m}$ as:

\begin{align}
\label{eq:tensor_quant}
    \hat{\bm{X}} = \text{round-to-nearest}\left(\frac{\bm{X}}{s_X}, E_{B_e}M_{B_m}\right)
\end{align}

We perform dynamic max-scaled quantization \citep{wu2020integer}, where the scale factor $s$ for activations is dynamically computed during runtime.

\subsection{Vector Scaled Quantization}
\begin{wrapfigure}{r}{0.35\linewidth}
  \centering
  \includegraphics[width=\linewidth]{sections/figures/vsquant.jpg}
  \caption{\small Vectorwise decomposition for per-vector scaled quantization (VSQ \citep{dai2021vsq}).}
  \label{fig:vsquant}
\end{wrapfigure}
During VSQ \citep{dai2021vsq}, the operand tensors are decomposed into 1D vectors in a hardware friendly manner as shown in Figure \ref{fig:vsquant}. Since the decomposed tensors are used as operands in matrix multiplications during inference, it is beneficial to perform this decomposition along the reduction dimension of the multiplication. The vectorwise quantization is performed similar to tensorwise quantization described in Equations \ref{eq:sf} and \ref{eq:tensor_quant}, where a scale factor $s_v$ is required for each vector $\bm{v}$ that maps the maximum absolute value of that vector to the maximum quantization level. While smaller vector lengths can lead to larger accuracy gains, the associated memory and computational overheads due to the per-vector scale factors increases. To alleviate these overheads, VSQ \citep{dai2021vsq} proposed a second level quantization of the per-vector scale factors to unsigned integers, while MX \citep{rouhani2023shared} quantizes them to integer powers of 2 (denoted as $2^{INT}$).

\subsubsection{MX Format}
The MX format proposed in \citep{rouhani2023microscaling} introduces the concept of sub-block shifting. For every two scalar elements of $b$-bits each, there is a shared exponent bit. The value of this exponent bit is determined through an empirical analysis that targets minimizing quantization MSE. We note that the FP format $E_{1}M_{b}$ is strictly better than MX from an accuracy perspective since it allocates a dedicated exponent bit to each scalar as opposed to sharing it across two scalars. Therefore, we conservatively bound the accuracy of a $b+2$-bit signed MX format with that of a $E_{1}M_{b}$ format in our comparisons. For instance, we use E1M2 format as a proxy for MX4.

\begin{figure}
    \centering
    \includegraphics[width=1\linewidth]{sections//figures/BlockFormats.pdf}
    \caption{\small Comparing LO-BCQ to MX format.}
    \label{fig:block_formats}
\end{figure}

Figure \ref{fig:block_formats} compares our $4$-bit LO-BCQ block format to MX \citep{rouhani2023microscaling}. As shown, both LO-BCQ and MX decompose a given operand tensor into block arrays and each block array into blocks. Similar to MX, we find that per-block quantization ($L_b < L_A$) leads to better accuracy due to increased flexibility. While MX achieves this through per-block $1$-bit micro-scales, we associate a dedicated codebook to each block through a per-block codebook selector. Further, MX quantizes the per-block array scale-factor to E8M0 format without per-tensor scaling. In contrast during LO-BCQ, we find that per-tensor scaling combined with quantization of per-block array scale-factor to E4M3 format results in superior inference accuracy across models. 


\end{document}
