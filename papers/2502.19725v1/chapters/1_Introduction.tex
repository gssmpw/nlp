\section{Introduction}
Industrial Control Systems (ICS) rely heavily on \textit{Programmable Logic Controllers (PLCs)} to manage critical infrastructure such as manufacturing, power generation, and transportation~\cite{erickson2016programmable, alphonsus2016review}.
Despite the advent of newer systems, many industrial sites continue to operate legacy PLCs that lack up-to-date documentation and source code~\cite{wang2023survey}. This creates significant challenges for security analysis and maintenance, particularly in facilities that must remain operational around the clock~\cite{rupprecht2021concepts,keliris2019icsref, chivilikhin2020automatic}.
High-profile incidents like Stuxnet and Triton demonstrate how attackers can target the PLC layer to disrupt physical processes with severe real-world consequences~\cite{giles2019triton,makrakis2021vulnerabilities}.
In these cases, threat actors exploited vulnerabilities in the toolchain or the deployed PLC program.
Such attacks underscore the urgent need for methods to inspect and analyze PLC executables even when source code is unavailable~\cite{giles2019triton,makrakis2021vulnerabilities,keliris2019icsref, wang2023survey}.

Binary code analysis offers a promising solution by examining executables directly without requiring source code access~\cite{song2008bitblaze, shoshitaishvili2016sok, 7985685, yu2023cfg2vec}. 
This technique examines the executable itself to uncover structure, behavior, and potential vulnerabilities.
However, while binary analysis is well-established for standard platforms like Linux or Windows, its application to PLCs remains challenging.
PLCs use vendor-specific compilers~\cite{codesys, openplc, geb, step7} that produce proprietary binary formats, making traditional analysis tools ineffective~\cite{keliris2019icsref, chang2018constructing, 7006408}. 
The challenge is compounded by limited access to source code and documentation in legacy systems, along with diverse programming languages and standards like IEC 61131-3~\cite{international3international}. Most critically, the field lacks standardized datasets for developing machine learning-based analysis techniques.

To bridge this gap, 
% as part of a joint industry-academia initiative between Siemens and UCI,
we introduce \textbf{PLC-BEAD (PLC Binary Evaluation and Analysis Dataset)}, a comprehensive dataset containing \textit{2431 compiled binaries} from \textit{over 700 PLC programs} across four major industrial compilers. 
This novel dataset uniquely pairs each binary with its original Structured Text source code and standardized functionality labels, \textit{enabling} both \textit{binary-level} and \textit{source-level} analysis for machine learning research. 
The dataset covers binaries compiled using CoDeSys~\cite{codesys}, GEB~\cite{geb}, OpenPLC-V2~\cite{openplc}, and OpenPLC-V3~\cite{openplc}, with programs derived from the industry-standard OSCAT library~\cite{oscat}. 
Each binary is labeled across 22 functionality categories, providing rich metadata for supervised learning tasks.

To demonstrate the dataset's utility, we present \textbf{PLCEmbed}, a transformer-based embedding framework that ingests raw bytes of PLC binaries to accomplish tasks such as \textbf{toolchain provenance identification} (i.e., discovering which compiler produced a given binary) and \textbf{functionality classification} (e.g., distinguishing a timer routine from a network communication block). These tasks are vital in \textbf{ICS digital forensics}, where investigators often must quickly determine whether a suspicious binary matches known legitimate code, or whether it stems from a vulnerable or maliciously modified compiler.

\textbf{Contributions.} Our work provides: \begin{itemize} \item \textbf{PLC-BEAD Dataset.} The first well-documented, open-source PLC binary dataset, comprising over 700 programs compiled with four different toolchains.

    \item \textbf{PLCEmbed.} A framework that leverages machine learning for multi-class classification of binary identity. 

    \item \textbf{Benchmarks \& Findings.} Experimental results show PLCEmbed can extract forensic information from various vendor-specific PLC binaries. PLCEmbed achieves 93\% accuracy for toolchain provenance and about 42\% for functionality classification.
    
    \item \textbf{Open-Source Release \& Future Impact.} Both the dataset and the PLCEmbed code are publicly released. Our repository is available at \url{\repourl}.

\end{itemize}

Overall, our work bridges an important gap in ICS security research by providing an all-in-one PLC binary dataset and a flexible embedding approach that can adapt to new compiler versions or vendor libraries.
Our findings highlight the inherent complexity of PLC binary analysis, while showing how data-driven methods, guided by open benchmarks, can strengthen both academic research and industrial cybersecurity practices.

% The rest of the paper is organized as follows. Section 2 provides essential background on PLC binary analysis challenges. Section 3 details the PLC-BEAD dataset construction and organization. Section 4 describes the PLCEmbed framework and its implementation. Section 5 presents experimental results and analysis. Finally, Section 6 concludes with discussion of impact and future directions.
