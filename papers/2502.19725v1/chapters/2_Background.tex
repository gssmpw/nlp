\section{Related Works and Background}
\label{sec:related}
Binary analysis has long been essential in software security, enabling vulnerability assessment and malware detection without source code access. For general-purpose software, established tools~\cite{hex2017ida, radare2book, nsfghidra} and techniques leverage standardized formats like ELF on Linux or PE on Windows. However, PLC binary analysis presents unique challenges due to proprietary formats and diverse industrial requirements.

\subsection{PLC Binary Analysis Challenges}
PLCs serve as a cornerstone in various industrial sectors due to their extensive integration and controlling capability in complex automation systems. Engineers develop PLC programs using standardized languages defined by the International Electrotechnical Commission (IEC 61131-3)~\cite{international3international}, with Structured Text (ST) being a high-level programming language syntactically resembling Pascal. These programs are then compiled using vendor-specific toolchains into proprietary binary formats.

In practice, \textbf{PLC toolchains} vary widely. Each vendor typically provides its own proprietary compiler and Integrated Development Environment (IDE) for programming. 
Examples include CoDeSys, Siemens STEP7, GEB, and OpenPLC~\cite{codesys,step7,geb,openplc}. Although these vendor-specific environments simplify the engineering workflow, they also produce binary code in nonstandard formats. 
This heterogeneity complicates efforts to develop universal analysis or reverse-engineering tools.

\subsection{Machine Learning for Binary Analysis}
Recent advances in machine learning (ML) have shown promise in binary analysis tasks such as functionality classification and compiler identification~\cite{pizzolotto2020identifying, zuo2018neural,marcelli2022machine, ding2019asm2vec, duan2020deepbindiff,yu2023cfg2vec,  gao2018vulseeker, marcelli2022machine}. Neural architectures like Convolutional Neural Networks (CNNs) and Transformers have demonstrated effectiveness in learning patterns from raw binary data. However, these approaches typically rely on large, labeled datasets, a resource notably absent in the PLC domain.

Prior, non-ML, PLC binary analysis efforts have focused on specific toolchains, particularly CoDeSys~\cite{keliris2019icsref, ICSFuzz}. 
For example, ICSREF provides a framework for automated reverse engineering of CoDeSys-generated binaries, while ICSFuzz enables fuzzing of control applications~\cite{keliris2019icsref}. 
However, these solutions are limited to single compilers or narrow subsets of PLC platforms. The lack of comprehensive, cross-compiler datasets has hindered the development of more general machine-learning approaches.

\subsection{Need for Standardized PLC Datasets}
The scarcity of publicly available, well-documented PLC binary datasets presents a significant barrier to research. Several factors contribute to this gap: First, PLC systems often control critical infrastructure, making organizations reluctant to share operational code. Second, intellectual property concerns restrict 
%the distribution of proprietary compiler outputs. 
the free exchange and availability of these resources and stifling the potential for broader, more inclusive research and understanding of PLC systems.
Third, the diversity of vendor-specific formats requires expertise across multiple industrial platforms to create representative datasets. The absence of data-sharing incentives further exacerbates the challenge.
These challenges underscore the need for an open, comprehensive dataset that enables:

(1) Development of machine learning models for cross-compiler binary analysis

(2) Standardized evaluation of PLC binary analysis techniques

(3) Research into industrial control system security and forensics

The following sections introduce PLC-BEAD, addressing these needs through a carefully curated collection of PLC binaries spanning multiple compilers and industrial applications.

