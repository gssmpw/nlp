\section{Previous works}
Previous works have studied the generation of networks with specific combinations of features considered in our study. 

Arguing that real-world networks are often highly clustered while showing small average distances between nodes, Watts and Strogatz~\cite{1998Watts} proposed a model to reproduce such characteristics. Starting with a set of $N$ nodes in a circular order where each node is connected with an undirected edge to $k$ neighbors, the authors rewired each edge with some fixed and small probability $p$. The average clustering coefficient remained quite high while the average distance dropped to a small value approximately proportional to $\log N$. 

Latent-space network models have also been employed to investigate small-world networks with non-vanishing clustering~\cite{Bogu2021}. 
In its simplest version, also called random geometric graph model, nodes are distributed uniformly at random in some metric space, and two nodes $i$ and $j$ are connected if and only if the distance $x_{i j}$ between them is less than some parameter $\mu$, leading to high clustering and large average shortest paths. 
% that scales linearly with network size (these networks are large worlds). 
However, the model can become small-world by introducing a probability $p_{i j}$ of the existence of a link between the nodes. 
For example, in $\mathbb{R}^d$ space, choosing $p_{i j} \propto x_{i j}^{-\beta}$ with $\beta \in (d, 2d)$ results in non-vanishing clustering coefficients and small-world networks, with average distances scaling proportionally to $\log N$~\cite{Bogu2020}. 

On the other hand, real-world networks also have high-degree nodes called hubs, which are absent in the above models. 
%Watts and Strogatz' model and in latent-space network models. 
Barabási and Albert~\cite{1999Barabasi} proposed a preferential attachment process that generates such long-tailed degree distributions. 
Starting from some small graph, one new node $v$ is added at each iteration with $k$ new edges linking $v$ to $k$ different nodes chosen with probabilities proportional to node degrees. That is, the likelihood of node $v$ choosing node $w$ is proportional to the degree of $w$ at that iteration, generating scale-free networks.

To increase the clustering coefficient of the networks generated by the above BA model, Holme and Kim~\cite{2002Holme} introduced a Triad Formation step performed with probability $P_t$ after a node and its edges are added to the network. When a new edge is added linking the new node $v$ to a node $w$, an edge is added linking $v$ to a randomly selected neighbor of $w$, thus creating a triad between the three nodes and increasing the clustering of the network. 

Although previous models have presented solid results in scale-free network construction, a significant concern we raise is that they require global information at each step (e.g., the degrees of all nodes to calculate the preferential attachment probabilities), which may be unrealistic since in real-world networks links emerge naturally, without knowing global information about network topology. 

Saramäki and Kaski~\cite{Saram_ki_2004} proposed using random walkers to generate undirected scale-free networks, showing that it is not necessary to have global information about node degrees at each step to achieve such results. 
Herrera and Zufiria~\cite{2011Herrera} improved this process by using the number of steps in the random walks to guide triangle generation, introducing a way to control the network's clustering coefficient using again only local information. 



%Herrera and Zufiria~cite{aaa} fizeram não sei o que. 
%Herrera et al.~cite{aaa} fizeram não sei o que. 
%Não sei o que foi proposto pelo Herrera et al.~cite{aaa}. 
%Estrategia X foi usada para conseguir resultado A~cite{aaa}. 

%Random walks are a prominent way to generate power-law networks. 

The random walk process proposed by Saramäki and Kaski~\cite{Saram_ki_2004} begins from a typically small initial graph with $n_0$ nodes. At each iteration, a new node $v$ is added to the graph, linking $v$ to existing nodes that will be chosen using random walks. 
%\st{where $m$ is the first algorithm parameter.} \rubenF{Não precisa, fica claro pelo contexto.} 
These chosen nodes (called ``marked nodes'' from now on) are identified as follows: beginning from a randomly selected node $w$, $l$ random steps are taken from $w$, allowing to revisit previous nodes. After each walk, the endpoint is marked, and the process continues until $m$ nodes are marked. The new node $v$ is then connected to marked nodes. The process finalizes after adding $N$ new nodes to the graph. 

Herrera and Zufiria~\cite{2011Herrera} noted that by changing the value of the parameter $l$, the number of steps in the random walk, it is possible to control the network's clustering coefficient. If $l = 1$, the neighbor of a marked node will also be marked, generating a triangle between these two neighbors and the added node, thus affecting the clustering coefficient. Each node $v$ has an associated value $p_v$, the probability of $l = 1$ if the random walk starts from that node. %, according to some probability distribution. 

Random walks controlling the walk length $l$ proved an efficient way to create power-law networks while regulating the clustering coefficient. However, the lack of attention to the average distances on the network remained an open question for real-world network generation using random walks.