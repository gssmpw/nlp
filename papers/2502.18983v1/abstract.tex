\begin{abstract}

The Von Neumann bottleneck, which relates to the energy cost of moving data from memory to on-chip core and vice versa, is a serious challenge in state-of-the-art AI architectures, like Convolutional Neural Networks' (CNNs) accelerators. 
Systolic arrays exploit distributed processing elements that exchange data with each other, thus mitigating the memory cost. However, when involved in convolutions, data redundancy must be carefully managed to avoid significant memory access overhead. To overcome this problem, TrIM has been recently proposed. It features a systolic array based on an innovative dataflow, where input feature map (ifmap) activations are locally reused through a triangular movement. However, ifmaps still suffer from memory accesses overhead.
This work proposes 3D-TrIM, an upgraded version of TrIM that addresses the memory access overhead through few extra shadow registers. In addition, due to a change in the architectural orientation, the local shift register buffers are now shared between different slices, thus improving area and energy efficiency. An architecture of 576 processing elements is implemented on commercial 22 nm technology and achieves an area efficiency of 4.47 TOPS/mm$^2$ and an energy efficiency of 4.54 TOPS/W. Finally, 3D-TrIM outperforms TrIM by up to $3.37\times$ in terms of operations per memory access considering CNN topologies like VGG-16 and AlexNet.

\end{abstract}