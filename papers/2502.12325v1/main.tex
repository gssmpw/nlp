% This must be in the first 5 lines to tell arXiv to use pdfLaTeX, which is strongly recommended.
\pdfoutput=1
% In particular, the hyperref package requires pdfLaTeX in order to break URLs across lines.

\documentclass[11pt]{article}

% Change "review" to "final" to generate the final (sometimes called camera-ready) version.
% Change to "preprint" to generate a non-anonymous version with page numbers.
\usepackage[preprint]{acl}

% Standard package includes
\usepackage{times}
\usepackage{latexsym}

% For proper rendering and hyphenation of words containing Latin characters (including in bib files)
\usepackage[T1]{fontenc}
% For Vietnamese characters
% \usepackage[T5]{fontenc}
% See https://www.latex-project.org/help/documentation/encguide.pdf for other character sets

% This assumes your files are encoded as UTF8
\usepackage[utf8]{inputenc}

% This is not strictly necessary, and may be commented out,
% but it will improve the layout of the manuscript,
% and will typically save some space.
\usepackage{microtype}

% This is also not strictly necessary, and may be commented out.
% However, it will improve the aesthetics of text in
% the typewriter font.
\usepackage{inconsolata}

%Including images in your LaTeX document requires adding
%additional package(s)

% ================ Custom imports for our paper =========
\usepackage{graphicx}
\usepackage{subcaption}
\usepackage{multirow}
\usepackage{multicol}
\usepackage{booktabs}
% ========================

% If the title and author information does not fit in the area allocated, uncomment the following
%
%\setlength\titlebox{<dim>}
%
% and set <dim> to something 5cm or larger.

% \title{CAM: Content-Adaptive MoEfication to transform a Dense LLM to an MoE variant for faster inference
% }
\title{From Dense to Dynamic: Token-Difficulty Driven MoEfication of Pre-Trained LLMs}
% \title{DynaMoE: Token-Difficulty Driven MoEfication of Pre-Trained LLMs}


\usepackage[textwidth=2.5cm]{todonotes}
\newcommand{\mf}[1]{\todo[color=orange!30,size=\tiny]{MF: #1}}

\author{
\textbf{Kumari Nishu}, \textbf{Sachin Mehta}\thanks{Contributed when employed by Apple.}, \textbf{Samira Abnar}, \textbf{Mehrdad Farajtabar} \\ \textbf{Maxwell Horton}, \textbf{Mahyar Najibi}, \textbf{Moin Nabi}, \textbf{Minsik Cho}, \textbf{Devang Naik}
\\
Apple
\\
 %  \small{
 %   \textbf{Correspondence:} \href{mailto:knishu@apple.com}{knishu@apple.com}, 
 %    \textsuperscript{1}contributed when employed by Apple
 % }
}
 
% Author information can be set in various styles:
% For several authors from the same institution:
% \author{Author 1 \and ... \and Author n \\
%         Address line \\ ... \\ Address line}
% if the names do not fit well on one line use
%         Author 1 \\ {\bf Author 2} \\ ... \\ {\bf Author n} \\
% For authors from different institutions:
% \author{Author 1 \\ Address line \\  ... \\ Address line
%         \And  ... \And
%         Author n \\ Address line \\ ... \\ Address line}
% To start a separate ``row'' of authors use \AND, as in
% \author{Author 1 \\ Address line \\  ... \\ Address line
%         \AND
%         Author 2 \\ Address line \\ ... \\ Address line \And
%         Author 3 \\ Address line \\ ... \\ Address line}

% \author{First Author \\
%   Affiliation / Address line 1 \\
%   Affiliation / Address line 2 \\
%   Affiliation / Address line 3 \\
%   \texttt{email@domain} \\\And
%   Second Author \\
%   Affiliation / Address line 1 \\
%   Affiliation / Address line 2 \\
%   Affiliation / Address line 3 \\
%   \texttt{email@domain} \\}

%\author{
%  \textbf{First Author\textsuperscript{1}},
%  \textbf{Second Author\textsuperscript{1,2}},
%  \textbf{Third T. Author\textsuperscript{1}},
%  \textbf{Fourth Author\textsuperscript{1}},
%\\
%  \textbf{Fifth Author\textsuperscript{1,2}},
%  \textbf{Sixth Author\textsuperscript{1}},
%  \textbf{Seventh Author\textsuperscript{1}},
%  \textbf{Eighth Author \textsuperscript{1,2,3,4}},
%\\
%  \textbf{Ninth Author\textsuperscript{1}},
%  \textbf{Tenth Author\textsuperscript{1}},
%  \textbf{Eleventh E. Author\textsuperscript{1,2,3,4,5}},
%  \textbf{Twelfth Author\textsuperscript{1}},
%\\
%  \textbf{Thirteenth Author\textsuperscript{3}},
%  \textbf{Fourteenth F. Author\textsuperscript{2,4}},
%  \textbf{Fifteenth Author\textsuperscript{1}},
%  \textbf{Sixteenth Author\textsuperscript{1}},
%\\
%  \textbf{Seventeenth S. Author\textsuperscript{4,5}},
%  \textbf{Eighteenth Author\textsuperscript{3,4}},
%  \textbf{Nineteenth N. Author\textsuperscript{2,5}},
%  \textbf{Twentieth Author\textsuperscript{1}}
%\\
%\\
%  \textsuperscript{1}Affiliation 1,
%  \textsuperscript{2}Affiliation 2,
%  \textsuperscript{3}Affiliation 3,
%  \textsuperscript{4}Affiliation 4,
%  \textsuperscript{5}Affiliation 5
%\\
%  \small{
%    \textbf{Correspondence:} \href{mailto:email@domain}{email@domain}
%  }
%}



% ============ math commands
\usepackage{amssymb, amsmath, mathtools, bm}
\def\R{\mathbb{R}}
\def\bx{\bm{X}}
\def\by{\bm{Y}}
\def\L{\mathcal{L}}
\def\calR{\mathcal{R}}
\newcommand{\win}{\bm{W}^{(IN)}}
\newcommand{\wout}{\bm{W}^{(OUT)}}
\newcommand{\bp}[1]{\left( #1 \right)}
\newcommand{\bsb}[1]{\left[ #1 \right]}
\newcommand{\ip}[2]{\left< #1, #2 \right>}
\DeclarePairedDelimiter\ceil{\lceil}{\rceil}
\DeclarePairedDelimiter\floor{\lfloor}{\rfloor}
\DeclareMathOperator{\mname}{DynaMoE}
%===========================

\begin{document}
\maketitle
\begin{abstract}
\begin{abstract}  
Test time scaling is currently one of the most active research areas that shows promise after training time scaling has reached its limits.
Deep-thinking (DT) models are a class of recurrent models that can perform easy-to-hard generalization by assigning more compute to harder test samples.
However, due to their inability to determine the complexity of a test sample, DT models have to use a large amount of computation for both easy and hard test samples.
Excessive test time computation is wasteful and can cause the ``overthinking'' problem where more test time computation leads to worse results.
In this paper, we introduce a test time training method for determining the optimal amount of computation needed for each sample during test time.
We also propose Conv-LiGRU, a novel recurrent architecture for efficient and robust visual reasoning. 
Extensive experiments demonstrate that Conv-LiGRU is more stable than DT, effectively mitigates the ``overthinking'' phenomenon, and achieves superior accuracy.
\end{abstract}  
\end{abstract}

\section{Introduction}

%LLM useful
% costly to run, required memory compute budget to run.. Resource specific LLM, or many LLMs are trained and used based on resource bandwidth. This is costly.
% Many models train multiple models of different size in one go to tackle this.
% however, once the model is deployed, it becomes static in general. We advocate for dynamic processing of inputs by a given LLM based on the difficulty of token. Recent work [sharc] proposed to adjust the network width based on the input hardness. Flextron uses elastic sub-networks for dynamic inference. However router training using heuristics or using only language loss is sub-optimal. Additionaly, such methods do provide any tunability factor to adjust the router's sensitivity to route tokens to different experts.


 Large language models (LLMs) have significantly advanced the field of natural language processing, showcasing strong capabilities in addressing complex tasks \cite{LLM1, LLM2_LLama, LLM3_Wei2022ChainOT}. However, their large size presents challenges, particularly in terms of high memory and computational demands, which can limit their deployment in resource-constrained settings. To address this, LLMs must be optimized for specific memory and computational constraints \cite{Touvron2023Llama2O}. However, designing multi-billion-parameter models for every use case is not cost-effective, as it demands substantial training time, data, and resources.

% leveraging existing models for many use cases
Some prior works have focused on adapting large LLMs for resource-constrained use cases by distilling knowledge from larger models into smaller ones~\citep{hsieh-etal-2023-distilling} or pruning model parameters to reduce computational demands~\citep{sun2024a}. While these methods effectively enable the use of large LLMs in low-resource scenarios, they often lead to performance degradation and require careful balancing between efficiency and accuracy. Alternatively, other approaches have investigated many-in-one LLM designs, MatFormer~\citep{Devvrit2023MatFormerNT} and SortedNet~\citep{Sortednet}, to employ multiple sub-networks within a single model to accommodate different computational budgets. These architectures use nested structures integrated into the standard LLM framework. However, they require non-standard methodologies and significantly longer, more resource-intensive training processes, which can offset the intended efficiency benefits.

% MoEs --> reducing inference... also not adaptive
Mixture-of-Experts (MoE) models~\citep{shazeer2017,Du2021GLaMES,fedus_switch,zoph2022st,he2024mixture} have emerged as a promising alternative to dense models, offering improved efficiency by sparsely activating select sub-modules or experts. This selective activation enables MoEs to achieve high performance while using fewer computational resources during inference. However, training MoEs from scratch remains resource-intensive and each expert becomes static, often requiring fixed compute budget irrespective of the input complexity.
% In this work, we explore post-training optimization to adapt a base LLM with minimal fine-tuning cost and enable dynamic compute based on the input data.

\begin{figure*}[!t]
    \centering
    \includegraphics[width=0.95\textwidth]{sections/image/overview.png} 
    \caption{Overview of our proposed post-training optimization framework, $\mname$. The left part represents the base pre-trained LLM, while the right part shows the adapted $\mname$ model.
    % The token-difficulty-driven router predicts the difficulty label for processing each token. During training, the token difficulty label generator generates the difficulty label for each token which is used to train the router.
    }
    \label{fig:overview}
\end{figure*}

Flextron~\citep{Cai2024FlextronMF} explored a post-training methodology by integrating the MoE concept into a nested elastic structure within the MLP layers, creating heterogeneous experts of different sizes, selected by a router conditioned on the input data. However, the lack of supervision in the router training leads to sub-optimal input complexity adaptation. Furthermore, the router lacks a parameter to customize its sensitivity to token complexity, limiting its flexibility and performance in handling diverse use-cases. \citet{Salehi2023SHARCSET} proposed an input-adaptive approach that predicts the difficulty of input data and dynamically adjusts the network's width accordingly. In the absence of ground-truth difficulty labels, they relied on heuristic methods for label generation, which may limit precision and consistency in difficulty estimation.



%  our method
To address their shortcomings, we introduce $\mname$, a post-training optimization framework designed to transform a dense LLM into a token-difficulty-driven MoE model. $\mname$ leverages the insight that not all tokens require the full capacity of a model's weights. For example, in the sentence ``Geoffrey did his PhD at the university of Edinburgh'', simpler tokens like ``at the university of'' are predictable using prior context, while more complex tokens like "Edinburgh" demand broader contextual understanding. To maximize efficiency, $\mname$ selectively activates nested sub-components of the MLP, referred as experts, based on the predicted difficulty of each token. To this end, we make the following contributions:

\begin{itemize}
    \item The framework includes a novel token-difficulty-aware router, trained to predict token hardness and assign it to the appropriate expert dynamically.
    
    \item Due to the lack of ground truth notion of hardness, we introduce a method to derive token difficulty labels which serve as supervision signals for training the router. This approach allows a token to have varying difficulty labels across different layers.

    % \item A simplified post-training optimization framework, $\mname$, to easily adapt a pre-trained dense LLM to a token-difficulty-driven MoE model, featuring a sensitivity parameter to customize the efficiency vs accuracy trade-off.
    \item A simplified post-training optimization framework that efficiently adapts a pre-trained dense LLM into a token-difficulty-driven MoE model, featuring a sensitivity parameter to customize the efficiency vs accuracy trade-off.

    
\end{itemize}






\section{Method}
In this section, we describe our proposed post-training optimization framework, $\mname$, which transforms a dense LLM into an MoE model for adaptive inference based on token difficulty. The process involves three key steps: $(1)$ defining heterogeneous experts by splitting the MLP layers of the dense LLM; $(2)$ generating token labels during training to represent token difficulty; and $(3)$ training a router to predict token difficulty while fine-tuning the model. We detail these steps in the below sub-sections.

\subsection{Defining Heterogeneous Experts}
In this work, we focus on defining experts into the MLP layers of the LLM \cite{Devvrit2023MatFormerNT}, as these layers account for the majority of the compute and operate on a token-by-token basis. The overview of $\mname$ is depicted in Fig. \ref{fig:overview}. The left part of the figure denotes the base pre-trained model which consists of the normalization layers, attention layers and the MLP layers in each transformer block. The right part shows the adapted $\mname$ model, where the original single MLP layer is transformed into multiple nested FFN blocks or experts. Such expert formation introduces no additional parameters to the base model, aside from the router. This design draws inspiration from adaptive width reduction in transformer \cite{Salehi2023SHARCSET} and recent works like Matformer \cite{Devvrit2023MatFormerNT} and Flextron \cite{Cai2024FlextronMF}. The attention layers remain frozen, and the MLP layers adapted to nested experts are fine-tuned in $\mname$.

Let $D$ and $H$ denote the embedding and the hidden dimensions of the MLP layer respectively. The input to the MLP layer is $\bx \in \R^{B \times D}$ and the output is $\by \in \R^{B \times D}$, where $B$ is the batch dimension. The MLP layer with two fully connected layers is represented by weight matrices $\win \in \R^{H \times D}$ and $\wout \in \R^{D \times H}$. In order to get best results, we first rearrange these fully-connected layers,  $\win$ and $\wout$, to have the most important rows/columns in the beginning of the matrix so that they can be included in all of the experts \cite{Samragh2023WeightSD}. There are a total of $E$ experts indexed using $e \in  \{0, 1, \ldots, E-1\}$. Each expert gets a portion $H_e$ of the weight matrices $\win$ and $\wout$, sliced over the hidden dimension $H$. The value $H_e$ is obtained as a fraction of $H$ as, 
\begin{equation}
    H_e = \floor*{\bp{\frac{e+1}{E}}\cdot H},
\end{equation}
consequently, $H_0 < H_1 < \cdots < H_{E-1}$ and $H_{E-1}=H$. Note that the expert with index $E-1$ utilizes the full MLP layer. The restriction of the matrices $\win$ and $\wout$ to the expert width $H_e$ is obtained using the slicing operator that selects the first $H_e$ rows and columns respectively as 
\begin{align}
    \win_e &= \win[0:H_e,\, :], \\
    \wout_e &= \wout[:,\, 0:H_e].
\end{align}
    
With $\sigma$ as the activation function, the output $\by_e$ of the MLP layer corresponding to the expert $e$ can thus be obtained as, 
\begin{equation}\label{eq:expert}
    \by_e = \sigma \bp{ \bx \cdot \bp{ \win_e}^T  } \cdot \bp{ \wout_e}^T.
\end{equation}


% # Table
\begingroup
    \renewcommand{\arraystretch}{1.5}
    \begin{table*}[!t]
        \centering
        \resizebox{\textwidth}{!}{%
        % \small
        % \footnotesize
        % \scriptsize 
        \begin{tabular}{c|c|c|c|c|c|c|c|c|c|c|c}
            \toprule
            & \textbf{Cost (\#Tokens)} & \textbf{Params} & \textbf{ARC-e} & \textbf{LAMBADA} & \textbf{PIQA} & \textbf{WinoGrande} & \textbf{Avg4} & \textbf{SciQ} & \textbf{HellaSwag} & \textbf{ARC-c}  & \textbf{Avg7} \\
            \toprule
            Base Mistral 7B & - & 7B & 80.2 & 75.1 & 80.8 & 75.5 & \textbf{77.8} & 96.4 & 61.4 & 50.5  & 74.2 \\
             
            \hline
            $\mname$ $\theta=0.9$ & \textbf{10B}  & 6B & 75.0 & 71.0 & 78.3 & 71.8 & \textbf{74.0} & 95.2 & 57.9 & 41.5 & 70.1 \\
            
            $\mname$ $\theta=0.8$ & \textbf{10B}  & 5.1B & 69.9 & 68.0 & 78.0 & 66.0 & \textbf{70.5} & 94.2 & 54.5 & 35.2 & 66.5 \\
            
            % $\mname$ $\theta=0.6$ & 10B  & -B & 66.5 & 61.4 & 74.5 & 62.4 & 66.2 & 93.4 & 49.4 & 32 & 62.8\\
            
           $\mname$ $\theta=0.7$ & \textbf{10B}  & 4.6B & 66.0 & 65.9 & 75.4 & 63.4 & 67.7 & 93.6 & 52.1 & 31.7 & 64.0 \\
            
           \hline
        Base Llama2-7B $^\dagger$ & -  & 6.5B & 75.1	& 71.5	& 77.5 &	69.1 & 73.3 &  &  &  &  \\
        
        Flextron $^\dagger$ & 93.57B & 4.1B & 68.6	& 65.1	& 76.1 &	63.7 & 68.3 &  &  &  &  \\
        % 6.7\% drop in acc
            % \textbf{Rel. Imprv}$^{\S}$  & \textbf{10.1} & \textbf{15.9} & \textbf{2.05}  & \textbf{38.3} & \textbf{50.6} & \textbf{76.9} \\
            % % \bottomrule
            \bottomrule
        \end{tabular}
        }
        \vspace{0.1 in}
        \caption{Evaluation of $\mname$ models with different sensitivity factor $\theta$ on downstream tasks, using 0-shot non-normalized accuracy metric. Our base model is Mistral 7B \citep{Jiang2023Mistral7}. $(^\dagger)$: results from Flextron \citep{Cai2024FlextronMF} used as our baseline. \textit{Params} denotes the average number of total activated parameters, aggregated over the downstream tasks. \textit{Avg4} averages over \textit{ARC-e, LAMBDA, PIQA, WinoGrande}, while \textit{Avg7} averages over all tasks.
        }
        \label{tab:result}
    \end{table*}
\endgroup




\subsection{Generating Token Difficulty Label}\label{subsec:generate_gt_label}

% \begin{figure*}[ht!]
%     \centering
%     \includegraphics[width=\textwidth]{sections/image/router_training.png} % Full page width
%     \caption{Router training in the proposed $\mname$ model.}
%     \label{fig:router_training}
% \end{figure*}

We aim to train a token-difficulty-aware router to dynamically assign tokens to an appropriate expert. But there is no ground-truth label denoting token difficulty to train such a router. To this end, we propose a method to estimate the token difficulty and generate a derived-ground-truth difficulty label during training. This is shown  as ``Token Difficulty Label Generator'' in Fig. \ref{fig:overview}. 

First, we pass the input to all experts and generate the output $\by_e$ for each $e \in [E]$. Then, for each token $b \in [B]$ and each expert $e \in [E]$, we compute a similarity score $S_{b,e}$ that measures how similar is the output of the expert $e$ compared to the output of the full MLP layer $e={E-1}$ for that token. We calculate this similarity as,
\begin{equation}
    S_{b,e} = \frac{\ip{\by_e[b,\,:]}{\by_{E-1}[b,\,:]}}{\ip{\by_{E-1}[b,\,:]}{\by_{E-1}[b,\,:]}}.
\end{equation}
Here, $\ip{\cdot}{\cdot}$ denotes the dot-product between two vectors. We use dot-product in this calculation as it accounts for both the magnitude and the direction of the tensors being compared.

Finally, we generate a derived ground-truth hardness label $l_b$, representing the target expert index for token $b$. Given a threshold $\theta$, we assign $l_b$ as the smallest expert index $e$ satisfying $S_{b,e} > \theta$, that is, $l_b=\min\{e \in [E] \mid S_{b,e}>\theta\}$. We say that a token is easier if it has a smaller label $l_b$, that is the similarity score for a smaller expert is higher than threshold $\theta$. In such cases, processing the token with the smaller expert incurs less compute without much compromise in the accuracy. During the forward pass of fine-tuning of $\mname$, we generate the token difficulty labels. These labels are then used in the backward pass to compute the router loss, which trains the router.

  

\subsection{Training a Token-Difficulty-Aware Router}
The output of a router is in $\R^{B \times E}$, denoting logits over the $E$ experts. Each router is parameterized by two linear layers, projecting the token embedding from dimension $D$ to $U$ and subsequently to $E$. In our experiments, we use $U=256$, resulting in total parameters added by the routers across all layers to $33.6M$, which is only $0.51\%$ of the base model size. 

We train the router using the derived labels from Section~\ref{subsec:generate_gt_label}. The objective is to learn the expert prediction using the derived-ground-truth labels to mimic token assignment based on their complexity and need. Hence, we impose the cross-entropy loss on the router to guide to this behavior and call it as the router loss. The overall objective function of $\mname$ is given as,
\begin{equation}\label{eq:loss}
    \L = \lambda_{LLM} \cdot \L_{LLM} + \lambda_{Router} \cdot \L_{Router}.
\end{equation}
Here, $\L_{LLM}$ is the main LLM Cross-entropy loss and $\L_{Router}$ is the router loss.  $\lambda_{LLM}$ and $\lambda_{Router}$ are hyper-parameters, denoting the weights of the respective losses.


\section{Experimental Set up}
\section{RESULTS}

\begin{figure*}[t!]
    %\vspace{-0.5cm}
    \centering
    \includegraphics[width=1\linewidth]{images/SystemArchitecture_2.png}
    \caption{From a single user demonstration, the system extracts the desired task goal state with the help of user interaction to solve ambiguities. Using the created environment variation, the system computes a task execution plan to bring new environments into the goal state. It sends the plan to agents in the environment to execute.} \label{fig:system_architecture}
\end{figure*}

Figure \ref{fig:system_architecture} shows our proposed framework to define a task goal, i.e. an environment goals state, and to turn a given environment into this goal state. The system visually observes a task execution by a user and segments this \underline{single} demonstration into \skills. \actions\ and \skills\ are defined in \ref{ssec:actions_skills}. The demonstration changed one or several properties of entities in the environment; environment which is now in the goal state. This information and the differences in entity properties from the start and end environment states are used to represent the task goal state. More on that in \ref{ssec:exp_model_def}. To turn a new environment into the defined goal state, a planning problem must be solved. This entails computing the differences between the environment's current state and the goal state, finding \actions\ that solve these differences, instantiating \skills\ that implement the \actions\ in the environment, selecting the \skills\ to execute by minimizing a given metric, and finally, sending the \skills\ to the agents in the environment to execute. This process is detailed in \ref{ssec:exp_model_use}.

% To prove the usability of our model, we present experiments to create a new goal state and turn the current environment into an (already-defined) goal state.

\subsection{Actions and Skills}\label{ssec:actions_skills}
A change in the environment is modeled using \actions, i.e. \textbf{what} has happened, and \skills, i.e. \textbf{how} did the change happen \cite{conceptHierarchyGeriatronicsSummit24}. Like in STRIPS \cite{strips} and PDDL \cite{pddl}, we represent \actions\ by their effects on entity properties and \skills\ by their preconditions and effects. \actions\ do not need preconditions because they only describe the \textbf{what} part of a change, not which conditions must be satisfied to perform the change. Besides preconditions and effects, \skills\ have a list of checks that tell our system if the \skill\ is executed in the environment. These checks allow the creation of a \skill\ recognition program, like the one presented in \cite{conceptHierarchyGeriatronicsSummit24}.
%\todo{citation of Geriatronics summit paper or the journal/unsubmitted paper?}
Using the \skill\ recognition output, we capture the changes from a task demonstration.

A \skill\ is thus the physical enactment of an abstract \action\ in an environment. Hence, \skills\ are correlated with \actions\ via their effects. A \skill\ can have more effects than a corresponding \action. For example, the \skill\ of scooping jam from a jar with a spoon implements the \action\ of \textit{TransferringContents}, but it also \textit{Dirties} the spoon.

\subsection{How To Parameterize The Model}\label{ssec:exp_model_def}
Creating a new goal state should be easier than manually specifying all variations wanted from the goal state. Doing so requires programming knowledge, which should not be needed to define goal states. One can let the system, which knows how to represent goal states, question the user about the desired state of the environment. However, this tedious process requires many questions from the system, also leading to decreased system usability.

Therefore, our approach is to let the user turn a given environment into a desired goal state and analyze the differences between the initial and final environment state to create the goal state representation. This single demonstration highlights the entity property values that were not in the desired goal state before being changed by the user.

We capture the demonstration via an Intel Realsense 3D camera \cite{realsense}, analyze the human skeleton via the OpenPose human pose estimation method \cite{openpose}, and determine the 3d pose of objects with AprilTag markers \cite{aprilTag}.

One demonstration contains the initial environment, not in the task goal state, and the final environment, in the goal state. The final environment state alone is not enough to create the environment variation. Thus, additional questions, guided by the differences between the two environment values, are posed by the system to the user to determine the desired variation in the environment state.

In a demonstration in which the user pours milk into a bowl, as shown in the top of Figure \ref{fig:system_architecture}, the initial question posed to the user is which entities that have changed properties are relevant for the goal state. If the goal state is to have more milk in the bowl, the milk carton is irrelevant; it is a means to achieve the goal state but not relevant to the goal itself. The bowl is thus selected as a relevant entity. 

Next, the list of relevant modified properties must also be determined for each relevant entity. It could have happened that during pouring of the milk into the bowl, the bowl's location also changed, e.g. touched accidentally by the user. Thus, not all modified properties could be relevant to the task. After selecting the relevant properties, the system knows from the knowledge base \cite{conceptHierarchyGeriatronicsSummit24} their \textit{ValueDomain} and the list of implemented \textbf{variations} for that \textit{ValueDomain}. Thus, the user parametrizes a selected \textbf{variation} from the list: choosing either a fixed value, a \textit{ValueDomain}-specific \textbf{RangeVariation} that must be parametrized, a conjunction or disjunction of \textbf{RangeVariations}, or the whole \textit{ValueDomain}.

In the example above, the user chooses the \textit{contentLevel} property as relevant. The system knows this property's defined set of values: a non-negative real number, and the possible range variation types: an open interval, a closed interval, an open-closed or closed-open interval, an intersection or union of intervals, etc. The user chooses a closed interval of $[0.28, 0.32]$ around the final \textit{contentLevel} value of $0.3L$. The user also specifies a variation for the entity's concept. It is generalized from that specific bowl instance to a \textit{LiquidContainer}.

After each modified property of each entity has a represented \textbf{variation}, the system automatically collects the entities into a variation of type $A$, see \ref{ssec:variations}, which is the assigned \textbf{variation} for the collection of entities in the environment.

Thus, the environment variation is determined in $\mathcal{O}\left(n\times m \times p\right)$ questions to the user, where $n$ is the number of entities in the environment, $m$ is the maximal number of properties that an entity can have, and $p$ is the maximal number of parameters that a \textbf{RangeVariation} needs to be represented. In the example above, $10$ questions were necessary to determine the task goal state shown in Figure \ref{fig:system_architecture} of a \textit{LiquidContainer} with \textit{contentLevel} between $0.28$ and $0.32L$. Figure \ref{fig:task_goal_state} shows the internal JSON-like representation of the goal state as the environment variation.
% 1 question which entities are relevant -> just bowl
% 1 question which properties are relevant -> contentLevel and concept
% 1 question about concept values being the same; should create variation?
% 1 question which ConceptValue-variation to select -> ConceptValue in Environment
% 1 question: which generalized concept?
% 1 question -> add other range-variation
% 1 question which Number-variation to select -> Interval
% 1 question: min-bound?
% 1 question: max-bound?
% 1 question -> add other range-variation

\begin{figure}[t!]
    %\vspace{-0.5cm}
    \centering
    \includegraphics[width=1\linewidth]{images/TaskDefinition_7.png}
    \caption{The goal state is a \textbf{RangeVariation} of the environment, of type EnvironmentDataRangeEntityVariation, which contains a \textbf{variation} of entities. This sub-variation is a \textbf{RangeVariation} of type MapRangeInstanceSubset (\textbf{variation} of type $A$, see \ref{ssec:variations}) and contains one instance \textbf{RangeVariation} of type InstanceRangePropertiesVariation. It defines the instance's concept \textbf{RangeVariation}, a \textit{LiquidContainer} to be found in the environment, and the \textit{contentLevel} property \textbf{RangeVariation}, the closed interval $\left[0.28, 0.32\right]$.} \label{fig:task_goal_state}
\end{figure}

\subsection{How To Use The Model}\label{ssec:exp_model_use}
Assuming the representation of a task's goal state is given, i.e. an environment variation, we detail our procedure (see Figure \ref{fig:experiment_description}) to turn the current environment into the goal state.

First, a Comparison between the environment and the goal variation is computed. This leads, as described in \ref{ssec:comparisons}, to a list of reasons why the environment is not in the variation. These reasons, i.e. differences $\delta$ of concept properties $p$, must be fixed to turn the environment into the goal state.

% Computing the differences between an EnvironmentData and an EnvironmentData-Variation, that has a Collection-Variation of type $A$, see \ref{ssec:variations}, is done via a maximal matching algorithm, where an edge between an entity $e$ an an entity variation $v_e$ means $e \in v_e$. 
For an EnvironmentData-Variation $v_{env}$ that defines a Collection-RangeVariation of type $A$, see \ref{ssec:variations}, computing the Comparison between an EnvironmentData $env$ and this target $v_{env}$ leads to a list of reasons for each entity $e_{env}$ in the entity collection of $env$, why $e_{env} \not\in v, \forall v \in A$. This can be seen in Figure \ref{fig:experiment_description}, where for each entity of \textit{LiquidContainer} concept in the environment, there is a list of differences, i.e. Comparisons, created for why the respective entity does not match the defined variation on the top-right.

\begin{figure}[t!]
    %\vspace{-0.5cm}
    \centering
    \includegraphics[width=1\linewidth]{images/Experiment_DescriptionUsingVariations_2.png}
    \caption{The procedure to turn an environment into its goal state is divided into 5 steps: computing differences, finding abstract solutions (i.e. \actions), computing practical solutions for the abstract ones (i.e. \actions\ $\rightarrow$ \skills), selecting the best practical solution, and executing the solution.} \label{fig:experiment_description}
\end{figure}
The second step of the procedure is to turn the list of differences into a list of \actions\ that can fix them. In notation, \action\ $A_x$ solves a difference in the concept property $p_x$. The system knows which properties \actions\ modify by analyzing the definition of their effects. Thus, \actions\ are created (parametrized) to fix the differences in entity properties.

% Because multiple instances can fit the instance variation, the third step is to match instances with the variations. Our matching optimization criterion is to minimize the amount of \textit{Actions} needed to fix the instances' property differences. \todo{continue!}

In the third step, each \action\ $A_x$ is converted into an execution plan $P_x$ that implements solving the difference $\delta_{p_x}$ in the environment. It is also possible that there is no possibility to implement the \action\ $A_x$ in the environment; this is represented as an execution plan $P_x = \emptyset$. An execution plan $P_x$ is otherwise, in its simplest form, a set of \skill\ alternatives $\left\{S_y\right\}$, where the \skill\ $S_y$ implements the \action\ $A_x$. There is the case to consider that the \skill\ $S_y$ has preconditions that are not met. And so, before executing the skill $S_y$, a different execution plan $P_{S_y}$ has to be computed and executed to allow the \skill\ $S_y$ to solve the property difference $\delta_{p_x}$. It is also possible that one single \skill\  $S_y$ is not enough to implement the \action\ $A_x$. Consider the case where the environment contains three cups with $0.1L$ of water, and the goal is to have one cup with $0.3L$ of content. One single \textit{Pouring} \skill\ is not enough to fulfill the goal; two \textit{Pouring} \skills\ must be executed. Thus, in the most general form, an execution plan $P_x = \left[\left\{ S_{iy}, P_{S_{iy}} \right\}_i\right]$ is a list of skill alternatives $\left\{ S_{iy}, P_{S_{iy}} \right\}_i$, that possibly contain other execution plans $P_{S_{iy}}$ to solve the skill's preconditions.

Our procedure to parameterize the \skills\ $S_y$ that implement the \action\ $A_x$ is a custom solution for each property $p_x$. One could backtrack through all possible parameter values of all possible skills to create a general solution that works for all properties. Another idea is to invert \skill\ effects and thus guide the \skill\ parameter search from the target variation to the value. However, both approaches would be computationally intense and would not create execution plans in a reasonable time. 
% reinforcement learning with policy for each property

The procedure to solve an entity $e$'s \underline{contentLevel} property difference searches for other \textit{Container} object instances in the environment, sorts them according to their content volume, and iterates through them in ascending order if $e.contentLevel \le target.contentLevel$; otherwise, in descending order. If a \skill\ $S$ can be executed with the two objects, that reduces the difference between $e.contentLevel$ and $target.contentLevel$, the \skill\ is added to the execution plan. If, after checking all objects, $e.contentLevel \not\in target.contentLevel$, there is no solution to solve this property difference.

Thus, the result of the third step is an execution plan $P_x$ for each entity property difference.

Fourth, after having the execution plans $P_x$ per entity-variation and entity, a \underline{solution selector} scores all solutions according to defined metrics and then, via a maximal matching algorithm, selects the solutions to execute to satisfy all variations of the Collection-RangeVariation of type $A$. The edges in the maximal matching have the cost of the solution score. For this paper, the scoring metric by the \underline{solution selector} is the number of steps of the execution plan.

The fifth and final step is to pass the execution plan to the agent(s) to execute in the environment. Figure \ref{fig:data_flow} presents the flow of data through the five steps.
We have used the Franka Emika Panda robot in CoppeliaSim \cite{coppeliaSim} to perform the computed execution plan.
% Note that the approach is independent of the used robot; only when instantiating \skills\ must the robot's abilities, manipulability region, and workspace be considered. How the \skills\ are executed in the environment is separated from the modeling of what must be done.

\begin{figure}[t!]
    % \vspace{-0.2cm}
    \centering
    \includegraphics[width=1\linewidth]{images/Experiment_DescriptionUsingVariations_DataFlow_2.png}
    \caption{Data flow when transforming an environment into a given goal state. $\Delta$ are differences of entity properties $p$, $A$ are \actions, $P$ is an execution plan and $S$ are \skills.} \label{fig:data_flow}
\end{figure}

The experiments aim to compute solution plans for solving the difference of the \textbf{contentLevel} property of \textit{Container} objects. For this, we consider the following criteria. $C1$: \textbf{variation} type = $\left\{\text{fixed},\text{interval},\text{interval union}\right\}$. $C2$: target relative to content = \{$\left\{t < cL \le cV \right\}$, $\left\{cL < t < cV \right\}$, $\left\{cL < t \ni cV \right\}$, $\left\{cL \le cV < t \right\}$\}, where $t$ is the \textbf{variation} value and $cL$ and $cV$ are the \textit{contentLevel} and \textit{contentVolume} properties respectively. $C3$: achievable in environment $ = \left\{\text{yes}, \text{no}\right\}$. Figure \ref{fig:experiment_table} presents planning results for different environments and the criteria described above. The lower table shows cases where the computed solution does not match the actual solution. This only happens when multiple instance variations are defined. The reason is that the implemented procedure to turn the list of differences into an execution plan treats each difference independently. Thus, dependencies between two variations are not accurately solved.

In the upper table of Figure \ref{fig:experiment_table}, there are two solutions for $C1.3$, $C2.3$, $C3.1$: one with the bowl $B$ as the instance in the \textbf{variation} $V1$, the other with $M$. The solution when $B$ is the matched instance has three steps: 1) pouring $0.1L$ from $M$ into $B$, 2) pouring $0.1L$ from $C1$ into $B$, and, finally, 3) pouring  $0.02L$ from $C2$ into $B$. This plan is sent to the robot in simulation and is executed as shown in Figure \ref{fig:robot_plan_execution}.

% \begin{figure}[t!]
%     % \vspace{-0.2cm}
%     \centering
%     \includegraphics[width=1\linewidth]{images/Experiment_Table_1Variation_compressed.png}
%     \caption{$B$ is a bowl with $0.5L$ \textit{contentVolume}, $M$ is a milk carton with $1.0L$ \textit{contentVolume}, $C1$ and $C2$ are cups with $0.3L$ \textit{contentVolume} each. Times, in seconds, averaged across 10 runs. Criteria $C2.4$ and $C3.1$ are mutually exclusive (a solution does not exist to let a container have more \textit{contentLevel} than its \textit{contentVolume}); thus, they are not included in the table.} \label{fig:experiment_table}
% \end{figure}
\begin{figure}[t!]
    % \vspace{-0.2cm}
    \centering
    \includegraphics[width=1\linewidth]{images/Experiment_Table_Results.png}
    \caption{$B$ is a bowl with $0.5L$ \textit{contentVolume}, $M$ is a milk carton with $1.0L$ \textit{contentVolume}, $C1$ and $C2$ are cups with $0.3L$ \textit{contentVolume} each. Times, in seconds, averaged across 10 runs. Criteria $C2.4$ and $C3.1$ are mutually exclusive (a solution does not exist to let a container have more \textit{contentLevel} than its \textit{contentVolume}); thus, they are not included in the upper table. The lower table presents results for open intervals and multiple variations in the environment.} \label{fig:experiment_table}
\end{figure}

\begin{figure}[t!]
    %\vspace{-0.1cm}
    \centering
    \includegraphics[width=1\linewidth]{images/Robot_PouringInBowl_M_PC1_PC2.png}
    \caption{Robot executing plan to bring $B$, the bowl, into the goal state. Because no liquids were simulated, the pouring amount was associated with the pouring time via: $t_{pour} = 10 * amount_{pour}$.} \label{fig:robot_plan_execution}
\end{figure}

\section{Results}
\subsection{Evaluation}
\label{subsec:Evaluation}
We evaluate the $\mname$ models on $7$ downstream tasks using LM Evaluation Harness \cite{eval-harness} and report the 0-shot non-normalized accuracy metric in Table~\ref{tab:result}. The selected evaluation tasks include ARC (Easy and Challenge) \cite{arc-c-e}, HellaSwag \cite{Zellers2019HellaSwagCA}, PIQA \cite{Bisk2019PIQARA}, SciQ \cite{SciQ}, WinoGrande \cite{Winograd}, and LAMBADA \cite{lambda}.

$\mname$ is compared to two baselines, the base Mistral 7B model and the Flextron model ~\citep{Cai2024FlextronMF} using Avg4 and Avg7 in Table \ref{tab:result}. Compared to Mistral 7B, $\mname$ with $\theta=0.8$ improves efficiency by activating only $5.1B$ of $7B$ parameters on average, with an $7.3$ point accuracy drop after fine-tuning on only $10B$ tokens on the downstream tasks. The number of activated parameters adapts dynamically to token difficulty. For reference, Flextron fine-tunes on $93.57B$ tokens, activating $4.1B$ of $6.5B$ parameters, with a $5$ point accuracy drop from its base model, Llama2-7B \cite{LLM2_LLama}. We emphasize that with only $\frac{1}{9}\text{th}$ of the Flextron's fine-tuning cost, our results  for $\mname$ with $\theta=0.7$  are comparable to Flextron. Accuracy improves with increase in fine-tuning cost, but to keep the adaption lightweight, we opt for a smaller cost.

% We evaluate the $\mname$ models on $7$ downstream tasks (Appendix~\ref{app:downstream_tasks}) using LM Evaluation Harness \cite{eval-harness} and report the 0-shot accuracy metric in Table~\ref{tab:result}. $\mname$ is compared to two baselines, the base Mistral 7B model and the Flextron model ~\citep{Cai2024FlextronMF}. Compared to the Mistral 7B base LLM, $\mname$ offers variants with different accuracy-efficiency trade-offs. At $\theta = 0.8$, $\mname$ achieves significant efficiency gains with minimal accuracy loss. For reference, we also include the Flextron results, which uses $93.57B$ tokens for fine-tuning and activates $63\%$ of the parameters with $6.7\%$ drop in accuracy compared to its base model, Llama2-7B \cite{LLM2_LLama}. We emphasize that with only $\frac{1}{18}\text{th}$ of the Flextron's fine-tuning cost, our results are comparable to Flextron.

\subsection{Analysis of Token-Difficulty-Aware Router}
We assess the performance of the Token-Difficulty-Aware router by gathering its predictions from all layers across the 7 downstream tasks outlined in Section \ref{subsec:Evaluation}. These predictions are then compared to the ground truth labels derived in Section \ref{subsec:generate_gt_label}. Using both sets of labels, we compute the router's overall classification accuracy. 

We present the confusion matrices for the router's classification tasks across all $\mname$ models in Fig. \ref{fig:router_confusion_matrix}. Notably, the matrices exhibit a strong diagonal pattern, indicating high classification accuracy. Furthermore, when the router misclassifies tokens, the errors predominantly occurred in neighboring expert classes, underscoring the router's effectiveness in distinguishing token difficulty levels.


\subsection{Experts usage analysis}
We visualize the expert usage patterns across all layers in  Fig. \ref{fig:cam_expert_load}. For each model, the Y-axis represents the percentage of tokens routed to a specific expert, while the X-axis indicates the layer index. Notably, a token's perceived difficulty may vary across layers, hence it can be routed to different experts as the token progresses through the model. The visualization shows that all experts are utilized in varying proportions across layers, reflecting an aggregated behavior over the $7$ tasks. However, during inference, the model adapts to the data, with simpler queries predominantly engaging lower-compute experts to maximize efficiency.


The parameter $\theta$ affects expert usage in $\mname$ models by controlling how quickly tokens are routed to larger experts based on difficulty. At lower $\theta$ values (e.g., $\theta=0.7$), smaller experts ($e=2$) dominate across layers, optimizing for efficiency. In contrast, at higher $\theta$ values (e.g., $\theta=0.9$), larger experts ($e=3$) are utilized more frequently, prioritizing accuracy over efficiency. This shift demonstrates $\theta$'s role in balancing computational resource allocation and prediction accuracy.


\textbf{Ablating the Router Loss}: 
To examine the router loss's role in expert allocation, we train a $\mname$ model without it, relying solely on the LLM loss to train the router. Tokens are routed using the router’s predicted expert indices without explicit difficulty supervision. The resulting expert usage pattern, shown in Fig. \ref{fig:nocam_gate}, reveals that the model converges to using specific experts per layer instead of dynamically allocating experts based on token difficulty. In contrast, when router loss is applied, expert usage adapts dynamically to token difficulty across layers.

% In the absence of router loss, Fig. \ref{fig:nocam_gate}, the model converges to using specific experts per layer instead of dynamically allocating experts based on token difficulty.

% \section{Related Work}
% \paragraph{Errors in machine learning benchmarks} Previous works have studied the identification of errors in machine learning benchmarks, as well as the resulting impact of these label errors on the quality of model evaluations. \citet{tsipras2020from} investigate the original ImageNet labeling process and %
release a refined, multi-label re-labeling of the ImageNet validation set. \citet{northcutt2021pervasive} identify errors in commonly used machine learning benchmarks, and then find that evaluating on benchmarks with significant rates of errors can lead practitioners to incorrectly select less performant models. \citet{bowman2021will} raise concerns similar to ours over issues in benchmarking for NLP tasks, and lay out a set of criteria that good benchmarks should satisfy.
However, they focus on overall design and social impact of benchmarks, whereas we focus specifically on better assessing model reliability. Additionally, while their work identifies similar flaws in NLP benchmarks, we propose an approach that can address some of these flaws (in particular, erroneous examples and ambiguity). %

Recently, \citet{gema2024we} released MMLU-Redux, a re-annotated subset of the MMLU benchmark~\cite{hendrycks2020measuring} created through manual assessment by 14 human experts. Our re-labeling of the MMLU high school mathematics subset actually intersects with MMLU-Redux on 100 examples. Our revised annotations align with MMLU-Redux on all but one example, on which we find that one of their human experts accidentally re-annotated a correct solution to make it incorrect\footnote{See the question marked ``wrong\_groundtruth'' here: \url{https://huggingface.co/datasets/edinburgh-dawg/mmlu-redux/viewer/high_school_mathematics?row=52}}. This slight remaining inconsistency highlights the difficulty of avoiding errors when creating and revising benchmarks.

\paragraph{LLM failures on simple tasks} It is generally known and often discussed that LLMs fail in surprising and unintuitive ways even on simple tasks. For instance, the common example of LLMs failing on the query ``how many r's are there in the word strawberry'' has circulated both social media and news outlets \cite{silberling2024strawberry}. Previous works have investigated specific instantiations of such failures. \citet{yang2024can} find that models frequently fail on simple problems even when they can solve harder versions of these same problems, suggesting inconsistency in their reasoning abilities. 
\citet{nezhurina2024alice} raise similar concerns over breakdowns in LLM reasoning behavior by identifying a specific category of logic tasks on which current frontier LLMs fail consistently.


\paragraph{Adversarial examples}
Adversarial attacks are small, sometimes imperceptible perturbations to model inputs that can drastically change their behavior---these lightly perturbed inputs are referred to as \textit{adversarial examples}. There has been significant work studying adversarial attacks and defenses against them in computer vision (and other) domains \cite{szegedy2014intriguing,carlini2017towards,madry2018towards,papernot16jsma}, and recent work has demonstrated successful adversarial attacks on LLMs, especially in the context of breaking safety alignment~\cite{zou2023universal,xu2023llm}. As a result, one possible approach to identifying LLM failures on simple queries might be to adversarially optimize for queries that result in model failures. However, we aim for our benchmark to assess whether models can be deployed reliably on real-world tasks, and adversarial examples generally do not align with the ``corner cases'' that models might face in the real world (unless the users themselves adversarially optimize their own inputs).




%Work related to adaptive inference with MoEs...

\section{Conclusion}
\section{Conclusion}
We introduce a novel approach, \algo, to reduce human feedback requirements in preference-based reinforcement learning by leveraging vision-language models. While VLMs encode rich world knowledge, their direct application as reward models is hindered by alignment issues and noisy predictions. To address this, we develop a synergistic framework where limited human feedback is used to adapt VLMs, improving their reliability in preference labeling. Further, we incorporate a selective sampling strategy to mitigate noise and prioritize informative human annotations.

Our experiments demonstrate that this method significantly improves feedback efficiency, achieving comparable or superior task performance with up to 50\% fewer human annotations. Moreover, we show that an adapted VLM can generalize across similar tasks, further reducing the need for new human feedback by 75\%. These results highlight the potential of integrating VLMs into preference-based RL, offering a scalable solution to reducing human supervision while maintaining high task success rates. 

\section*{Impact Statement}
This work advances embodied AI by significantly reducing the human feedback required for training agents. This reduction is particularly valuable in robotic applications where obtaining human demonstrations and feedback is challenging or impractical, such as assistive robotic arms for individuals with mobility impairments. By minimizing the feedback requirements, our approach enables users to more efficiently customize and teach new skills to robotic agents based on their specific needs and preferences. The broader impact of this work extends to healthcare, assistive technology, and human-robot interaction. One possible risk is that the bias from human feedback can propagate to the VLM and subsequently to the policy. This can be mitigated by personalization of agents in case of household application or standardization of feedback for industrial applications. 

\section*{Limitations}
\section{Limitations}

\paragraph{Reliance on a Stronger LLM. }
Our framework relies on a stronger LLM to synthesize data. While this enables the synthesis of high quality data, removing this dependency could help lead to a more robust and independent framework, possibly at the cost of performance degradation. Additionally, LLM-generated data may amplify existing biases or include inappropriate content.

\paragraph{Seed Data Quality. }
The quality of our synthesized data is tied to that of our seed data. We select concise, high-quality datasets from prior works to use as the seed data. A more comprehensive exploration of seed data selection and its impact on synthetic data remains an important direction for future work.

Furthermore, our work does not fully address the scalability our framework. There likely exists a limit to how much data we can synthesize from our seed data, until the synthesized data becomes repetitive and lacks diversity.

\paragraph{LLM-Based Evaluation. }
Our evaluation relies on benchmarks that use LLMs as a judge. Although they correlate highly with human judgments, it is important to acknowledge that they may still have limitations, such as biases towards longer responses or their own outputs.


\section{Acknowledgments}
This work has benefited from the Microsoft Accelerate Foundation Models Research (AFMR) grant program, through which leading foundation models hosted by Microsoft Azure and access to Azure credits were provided to conduct the research.



% \section*{Acknowledgments}

% This document

% Bibliography entries for the entire Anthology, followed by custom entries
%\bibliography{anthology,custom}
% Custom bibliography entries only
\bibliography{reference}

% \appendix
% \section{Example Appendix}
% \label{sec:appendix}
% This is an appendix.

\end{document}
