% This must be in the first 5 lines to tell arXiv to use pdfLaTeX, which is strongly recommended.
\pdfoutput=1
% In particular, the hyperref package requires pdfLaTeX in order to break URLs across lines.

\documentclass[11pt]{article}

% Change "review" to "final" to generate the final (sometimes called camera-ready) version.
% Change to "preprint" to generate a non-anonymous version with page numbers.
\usepackage[preprint]{acl}

% Standard package includes
\usepackage{times}
\usepackage{latexsym}

% For proper rendering and hyphenation of words containing Latin characters (including in bib files)
\usepackage[T1]{fontenc}
% For Vietnamese characters
% \usepackage[T5]{fontenc}
% See https://www.latex-project.org/help/documentation/encguide.pdf for other character sets

% This assumes your files are encoded as UTF8
\usepackage[utf8]{inputenc}

% This is not strictly necessary, and may be commented out,
% but it will improve the layout of the manuscript,
% and will typically save some space.
\usepackage{microtype}

% This is also not strictly necessary, and may be commented out.
% However, it will improve the aesthetics of text in
% the typewriter font.
\usepackage{inconsolata}

%Including images in your LaTeX document requires adding
%additional package(s)

% ================ Custom imports for our paper =========
\usepackage{graphicx}
\usepackage{subcaption}
\usepackage{multirow}
\usepackage{multicol}
\usepackage{booktabs}
% ========================

% If the title and author information does not fit in the area allocated, uncomment the following
%
%\setlength\titlebox{<dim>}
%
% and set <dim> to something 5cm or larger.

% \title{CAM: Content-Adaptive MoEfication to transform a Dense LLM to an MoE variant for faster inference
% }
\title{From Dense to Dynamic: Token-Difficulty Driven MoEfication of Pre-Trained LLMs}
% \title{DynaMoE: Token-Difficulty Driven MoEfication of Pre-Trained LLMs}


\usepackage[textwidth=2.5cm]{todonotes}
\newcommand{\mf}[1]{\todo[color=orange!30,size=\tiny]{MF: #1}}

\author{
\textbf{Kumari Nishu}, \textbf{Sachin Mehta}\thanks{Contributed when employed by Apple.}, \textbf{Samira Abnar}, \textbf{Mehrdad Farajtabar} \\ \textbf{Maxwell Horton}, \textbf{Mahyar Najibi}, \textbf{Moin Nabi}, \textbf{Minsik Cho}, \textbf{Devang Naik}
\\
Apple
\\
 %  \small{
 %   \textbf{Correspondence:} \href{mailto:knishu@apple.com}{knishu@apple.com}, 
 %    \textsuperscript{1}contributed when employed by Apple
 % }
}
 
% Author information can be set in various styles:
% For several authors from the same institution:
% \author{Author 1 \and ... \and Author n \\
%         Address line \\ ... \\ Address line}
% if the names do not fit well on one line use
%         Author 1 \\ {\bf Author 2} \\ ... \\ {\bf Author n} \\
% For authors from different institutions:
% \author{Author 1 \\ Address line \\  ... \\ Address line
%         \And  ... \And
%         Author n \\ Address line \\ ... \\ Address line}
% To start a separate ``row'' of authors use \AND, as in
% \author{Author 1 \\ Address line \\  ... \\ Address line
%         \AND
%         Author 2 \\ Address line \\ ... \\ Address line \And
%         Author 3 \\ Address line \\ ... \\ Address line}

% \author{First Author \\
%   Affiliation / Address line 1 \\
%   Affiliation / Address line 2 \\
%   Affiliation / Address line 3 \\
%   \texttt{email@domain} \\\And
%   Second Author \\
%   Affiliation / Address line 1 \\
%   Affiliation / Address line 2 \\
%   Affiliation / Address line 3 \\
%   \texttt{email@domain} \\}

%\author{
%  \textbf{First Author\textsuperscript{1}},
%  \textbf{Second Author\textsuperscript{1,2}},
%  \textbf{Third T. Author\textsuperscript{1}},
%  \textbf{Fourth Author\textsuperscript{1}},
%\\
%  \textbf{Fifth Author\textsuperscript{1,2}},
%  \textbf{Sixth Author\textsuperscript{1}},
%  \textbf{Seventh Author\textsuperscript{1}},
%  \textbf{Eighth Author \textsuperscript{1,2,3,4}},
%\\
%  \textbf{Ninth Author\textsuperscript{1}},
%  \textbf{Tenth Author\textsuperscript{1}},
%  \textbf{Eleventh E. Author\textsuperscript{1,2,3,4,5}},
%  \textbf{Twelfth Author\textsuperscript{1}},
%\\
%  \textbf{Thirteenth Author\textsuperscript{3}},
%  \textbf{Fourteenth F. Author\textsuperscript{2,4}},
%  \textbf{Fifteenth Author\textsuperscript{1}},
%  \textbf{Sixteenth Author\textsuperscript{1}},
%\\
%  \textbf{Seventeenth S. Author\textsuperscript{4,5}},
%  \textbf{Eighteenth Author\textsuperscript{3,4}},
%  \textbf{Nineteenth N. Author\textsuperscript{2,5}},
%  \textbf{Twentieth Author\textsuperscript{1}}
%\\
%\\
%  \textsuperscript{1}Affiliation 1,
%  \textsuperscript{2}Affiliation 2,
%  \textsuperscript{3}Affiliation 3,
%  \textsuperscript{4}Affiliation 4,
%  \textsuperscript{5}Affiliation 5
%\\
%  \small{
%    \textbf{Correspondence:} \href{mailto:email@domain}{email@domain}
%  }
%}



% ============ math commands
\usepackage{amssymb, amsmath, mathtools, bm}
\def\R{\mathbb{R}}
\def\bx{\bm{X}}
\def\by{\bm{Y}}
\def\L{\mathcal{L}}
\def\calR{\mathcal{R}}
\newcommand{\win}{\bm{W}^{(IN)}}
\newcommand{\wout}{\bm{W}^{(OUT)}}
\newcommand{\bp}[1]{\left( #1 \right)}
\newcommand{\bsb}[1]{\left[ #1 \right]}
\newcommand{\ip}[2]{\left< #1, #2 \right>}
\DeclarePairedDelimiter\ceil{\lceil}{\rceil}
\DeclarePairedDelimiter\floor{\lfloor}{\rfloor}
\DeclareMathOperator{\mname}{DynaMoE}
%===========================

\begin{document}
\maketitle
\begin{abstract}
\begin{abstract}
Retrieval-Augmented Generation (RAG) is often used with Large Language Models (LLMs) to infuse domain knowledge or user-specific information. In RAG, given a user query, a retriever extracts chunks of relevant text from a knowledge base. These chunks are sent to an LLM as part of the input prompt. Typically, any given chunk is repeatedly retrieved across user questions. However, currently, for every question, attention-layers in LLMs fully compute the key values (KVs) repeatedly for the input chunks, as state-of-the-art methods cannot reuse KV-caches when chunks appear at arbitrary locations with arbitrary contexts. Naive reuse leads to output quality degradation.  This leads to potentially redundant computations on expensive GPUs and increases latency. In this work, we propose \sys, a system for managing and reusing precomputed KVs corresponding to the text chunks (we call \textit{chunk-caches}) in RAG-based systems. We present how to identify \hl{\textit{chunk-caches} that are reusable}, how to efficiently perform a small fraction of recomputation to \textit{fix} the cache to maintain output quality, and how to efficiently store and evict \textit{chunk-caches} in the hardware for maximizing reuse while masking any overheads. With real production workloads as well as synthetic datasets, we show that \sys reduces redundant computation by \textbf{51\%} over SOTA prefix-caching and \textbf{75\%} over full recomputation.
\hl{Additionally, with continuous batching on a real production workload, we get a \textbf{1.6$\times$} speedup in throughput and a \textbf{2$\times$} reduction in end-to-end response latency over prefix-caching while maintaining quality, for both the \llama-3-8B and \llama-3-70B models. 
}
\end{abstract}





\end{abstract}

\section{Introduction}

%LLM useful
% costly to run, required memory compute budget to run.. Resource specific LLM, or many LLMs are trained and used based on resource bandwidth. This is costly.
% Many models train multiple models of different size in one go to tackle this.
% however, once the model is deployed, it becomes static in general. We advocate for dynamic processing of inputs by a given LLM based on the difficulty of token. Recent work [sharc] proposed to adjust the network width based on the input hardness. Flextron uses elastic sub-networks for dynamic inference. However router training using heuristics or using only language loss is sub-optimal. Additionaly, such methods do provide any tunability factor to adjust the router's sensitivity to route tokens to different experts.


 Large language models (LLMs) have significantly advanced the field of natural language processing, showcasing strong capabilities in addressing complex tasks \cite{LLM1, LLM2_LLama, LLM3_Wei2022ChainOT}. However, their large size presents challenges, particularly in terms of high memory and computational demands, which can limit their deployment in resource-constrained settings. To address this, LLMs must be optimized for specific memory and computational constraints \cite{Touvron2023Llama2O}. However, designing multi-billion-parameter models for every use case is not cost-effective, as it demands substantial training time, data, and resources.

% leveraging existing models for many use cases
Some prior works have focused on adapting large LLMs for resource-constrained use cases by distilling knowledge from larger models into smaller ones~\citep{hsieh-etal-2023-distilling} or pruning model parameters to reduce computational demands~\citep{sun2024a}. While these methods effectively enable the use of large LLMs in low-resource scenarios, they often lead to performance degradation and require careful balancing between efficiency and accuracy. Alternatively, other approaches have investigated many-in-one LLM designs, MatFormer~\citep{Devvrit2023MatFormerNT} and SortedNet~\citep{Sortednet}, to employ multiple sub-networks within a single model to accommodate different computational budgets. These architectures use nested structures integrated into the standard LLM framework. However, they require non-standard methodologies and significantly longer, more resource-intensive training processes, which can offset the intended efficiency benefits.

% MoEs --> reducing inference... also not adaptive
Mixture-of-Experts (MoE) models~\citep{shazeer2017,Du2021GLaMES,fedus_switch,zoph2022st,he2024mixture} have emerged as a promising alternative to dense models, offering improved efficiency by sparsely activating select sub-modules or experts. This selective activation enables MoEs to achieve high performance while using fewer computational resources during inference. However, training MoEs from scratch remains resource-intensive and each expert becomes static, often requiring fixed compute budget irrespective of the input complexity.
% In this work, we explore post-training optimization to adapt a base LLM with minimal fine-tuning cost and enable dynamic compute based on the input data.

\begin{figure*}[!t]
    \centering
    \includegraphics[width=0.95\textwidth]{sections/image/overview.png} 
    \caption{Overview of our proposed post-training optimization framework, $\mname$. The left part represents the base pre-trained LLM, while the right part shows the adapted $\mname$ model.
    % The token-difficulty-driven router predicts the difficulty label for processing each token. During training, the token difficulty label generator generates the difficulty label for each token which is used to train the router.
    }
    \label{fig:overview}
\end{figure*}

Flextron~\citep{Cai2024FlextronMF} explored a post-training methodology by integrating the MoE concept into a nested elastic structure within the MLP layers, creating heterogeneous experts of different sizes, selected by a router conditioned on the input data. However, the lack of supervision in the router training leads to sub-optimal input complexity adaptation. Furthermore, the router lacks a parameter to customize its sensitivity to token complexity, limiting its flexibility and performance in handling diverse use-cases. \citet{Salehi2023SHARCSET} proposed an input-adaptive approach that predicts the difficulty of input data and dynamically adjusts the network's width accordingly. In the absence of ground-truth difficulty labels, they relied on heuristic methods for label generation, which may limit precision and consistency in difficulty estimation.



%  our method
To address their shortcomings, we introduce $\mname$, a post-training optimization framework designed to transform a dense LLM into a token-difficulty-driven MoE model. $\mname$ leverages the insight that not all tokens require the full capacity of a model's weights. For example, in the sentence ``Geoffrey did his PhD at the university of Edinburgh'', simpler tokens like ``at the university of'' are predictable using prior context, while more complex tokens like "Edinburgh" demand broader contextual understanding. To maximize efficiency, $\mname$ selectively activates nested sub-components of the MLP, referred as experts, based on the predicted difficulty of each token. To this end, we make the following contributions:

\begin{itemize}
    \item The framework includes a novel token-difficulty-aware router, trained to predict token hardness and assign it to the appropriate expert dynamically.
    
    \item Due to the lack of ground truth notion of hardness, we introduce a method to derive token difficulty labels which serve as supervision signals for training the router. This approach allows a token to have varying difficulty labels across different layers.

    % \item A simplified post-training optimization framework, $\mname$, to easily adapt a pre-trained dense LLM to a token-difficulty-driven MoE model, featuring a sensitivity parameter to customize the efficiency vs accuracy trade-off.
    \item A simplified post-training optimization framework that efficiently adapts a pre-trained dense LLM into a token-difficulty-driven MoE model, featuring a sensitivity parameter to customize the efficiency vs accuracy trade-off.

    
\end{itemize}






\section{Method}

\section{Study Methodology}
\label{sec:methodology}

This study aims to investigate how the issue resolution process is implemented in practice at Mozilla Firefox to solve various software problems and tasks described in issue reports.  We investigate the major stages of the issue resolution process, described in \Cref{sub:background_issue_res}, and how developers\footnote{We hereon use \textit{developers} to refer to all stakeholders involved in issue resolution: programmers, reporters, QA members, \etc}  follow them to solve a variety of problem categories (\eg crashes, UI issues, or refactoring changes) reported in various issue report types (defects, enhancements, or tasks). 
The study addresses the following research questions (RQs):
\looseness=-1

\begin{enumerate}[label=\textbf{RQ$_\arabic*$:}, ref=\textbf{RQ$_\arabic*$}, itemindent=0cm,leftmargin=1cm]
	\item \label{rq:stages}{\textit{What issue resolution stages are found in issue reports?}} 
	\item \label{rq:interactions}{\textit{How do the resolution stages interact with each other?}}
	\item \label{rq:process}{\textit{What is the overall process of issue resolution?}}
	\item \label{rq:patterns}{\textit{What resolution patterns are found in issue reports?}}
	\item \label{rq:pattern_usefulness}{\rev{\textit{What are the potential use cases of the patterns?}} }
\end{enumerate}

\ref{rq:stages} investigates the major stages that Mozilla developers go through to address reported issues and how frequently these stages are discussed in issue reports. \ref{rq:interactions} investigates how these stages interact with one another, including how frequently these stages co-occur in issue reports. \ref{rq:process} investigates the overall issue resolution process at Mozilla Firefox.  \ref{rq:patterns} investigates recurrent instances of the resolution process, expressed as sequences of stages. \rev{\ref{rq:pattern_usefulness} examines the potential applications of the derived patterns for Mozilla developers.}
\looseness=-1



\subsection{Issue Collection}
\label{sub:issue_collection}



Mozilla's BMO is the centralized system for managing the issues of Firefox desktop and mobile~\cite{mozilla-products}. In this study, we focused on the desktop version of Mozilla Firefox, studying the issues of its two main components: \textit{Firefox} and \textit{Core}.  The \textit{Firefox} component (\aka \textit{product} in BMO) implements the graphical user interface (GUI) of the web browser, while the \textit{Core} component includes essential functionality such as web page rendering, web browsing, and networking services. 


Our study focused on \textit{FIXED} and \textit{RESOLVED} issue reports
for the selected components. To obtain recent issues within a significant period of system evolution, we downloaded all the issues created from January 1st, 2010 to April 30th, 2023 using Bugzilla's API~\cite{bugzilla-api}, including their title/summary, comments (which contain the issue description), and relevant metadata: creation time, resolution time, and others.
\rev{From 199,271 downloaded issues ($\approx$164.7k/34.5k for Core/Firefox), we randomly sampled 384 issues for analysis. This is a statistically significant sample, at a 95\% confidence level and 5\% error margin, that captures the diversity and characteristics of the entire population of \textit{Core} and \textit{Firefox} issues. This is evidenced by comparing our sample and the entire issue population in terms of the proportion of issue types (defects: 71.1\% vs 70.1\%, enhancements: 16.9\% vs 16.1\%, and tasks: 12\% vs 13.5\%), the proportion of issues per product (Core: 81.5\% vs 82.7\% and Firefox: 18.5\% vs 17.3\%), average \# of comments per issue (13.4 vs 14.6), and average resolution time (81 vs 88 days). 
	} 
The 384 issues contain 13.4 (9) comments, 30.27 (16) paragraphs, and 56.73 (25) sentences on average (median). 
\looseness=-1













\subsection{Issue Annotation}
\label{sub:issue_annotation}





\subsubsection{\textbf{Goals and Overview}}
\rev{
We qualitatively analyzed all the information provided in the issues, annotating textual content related to issue resolution by employing an iterative \textit{open coding} methodology~\cite{spencer2009card}. The annotation process was conducted by six Ph.D. students and one professor (\aka \textit{annotators}), including the authors of this paper.  The annotators have 1-9 years of research experience (particularly in qualitative text analysis), and five of them have 1-4 years of industry experience.}
\looseness=-1






\rev{The annotation targeted all the textual content written by different stakeholders in issue comments and aimed to identify: (1) \textit{themes} or \textit{codes} about different activities performed to resolve the issue 
(\eg reproduction attempts or a code review), and (2) the types of problems described in the issues (\eg crashes, UI issues, \etc), which we call \textbf{problem categories}.}


\rev{\subsubsection{\textbf{Annotation Tool and Unit}} We used the Hypothesis annotation tool~\cite{hypothesis} to directly annotate the web pages of the issue reports. The tool allowed us to collaboratively assign \textit{codes} to text snippets in the issue threads, modify the assigned codes, and discuss the annotations.

We coded \textit{text snippets} in the issue comments. The minimal annotation unit was a complete sentence. Since one or more sentences may convey the same type of
information (\ie a given resolution activity), the annotation included individual sentences, multiple sentences, paragraphs, or even entire comments. A single textual snippet was allowed to be coded with one or more codes.

\subsubsection{\textbf{Code Catalog and Coding Guidelines}}
We maintained a  \textit{code catalog} via a  Google spreadsheet shared among the annotators. The catalog included a list of codes,  code descriptions, rules to apply the codes, and text snippets from annotated issues used as examples. The code catalog also included a list of problem categories, with detailed definitions and examples of annotated issues. We also maintained a shared Google document with detailed guidelines of the annotation procedure, coding rules, and necessary resources for annotating the issues (\eg official Mozilla documentation to get familiar with Firefox's resolution process and a glossary of annotation terminology). Both the catalog and  guidelines were built from scratch and developed by all the annotators incrementally and collaboratively.}
\looseness=-1



\rev{\subsubsection{\textbf{Annotation Procedure}} 
We adopted an iterative multi-coder open-coding methodology wherein each issue report was annotated and validated by at least two annotators.  The 384 issues were distributed evenly among the seven annotators, who iteratively examined, annotated, and validated the issue comments in batches of 30-50 issues.  
The first annotator assigned codes to text snippets in the comments, and a second annotator reviewed these annotations for accuracy and completeness. Discrepancies were resolved in reconciliation sessions. Annotator roles alternated across batches, with each person either annotating from scratch or reviewing the annotations by the first annotator. To avoid fatigue and reduce potential mistakes, the annotators annotated small sets of issues with breaks in between.
\looseness=-1

The overall process for a single issue involved the first annotator thoroughly reviewing the issue, including attached patches, linked commits, and metadata (\eg\ issue commentators, tags, and status), to identify/annotate relevant content and the problem category. Codes were assigned based on the content's meaning and the code catalog. The second annotator then reviewed these annotations, verifying their accuracy, suggesting additional codes, or flagging mistakes. After processing a batch, both annotators discussed disagreements to reach a consensus.

To establish the initial coding framework, two researchers annotated the first batch of 30 issues, creating an initial set of codes and problem categories. These were refined through discussion sessions, resulting in complete definitions, examples, and rules for applying the codes. This initial annotation informed the creation of the coding guidelines, which included resources for understanding issues and general annotation rules.

Before annotating the remaining issues, training sessions were conducted with the other annotators to review the coding guidelines, discuss examples from the initial batch, and solve misunderstandings. Throughout the entire annotation process, the code catalog was continuously updated, with changes such as new codes, code merges, or renames collectively agreed upon and promptly communicated. When the catalog was updated, previously coded issues were revisited to ensure consistency. Regular communication via Zoom meetings and Slack discussions was essential to maintain the accuracy and uniformity of the catalog and annotated content.

}


















\rev{\subsubsection{\textbf{Annotation Results and Inter-coder Agreement}} During the annotation process, 28 issues, that were pull requests (PR) automatically created by the issue tracker, were discarded.
In summary, we annotated \textbf{3,707 textual snippets} in 2,574 issue comments across 356 issue reports. The annotation process resulted in 
\textbf{22 issue resolution codes}, and \textbf{17 problem categories} which we further grouped into \textbf{3 problem classes}. \Cref{tab:stages,tab:problem_categories} show examples of these elements;
our replication package contains the full catalog of codes and problems~\cite{repl_pack}. 

The annotators agreed on 3,438 annotations with an agreement rate of $\approx$93\% and a Cohen's kappa of 0.92, which indicates a high overall agreement~\cite{Cohen}. Common
sources of disagreement (269/3,707 text snippets) included misunderstandings due to ambiguous comments or unclear code definitions. If both annotators were unable to reach an agreement, a third annotator reviewed the issue to resolve the conflict. 
\looseness=-1
}













  















\subsection{Inferring and Analyzing Issue Resolution Stages}
\label{sub:resolution_stages}




The 22 codes obtained from the issue report annotation represent the information about activities performed by developers during issue resolution.
We implemented two steps for inferring the issue resolution stages from the annotation codes. In the first step, we qualitatively analyzed the code catalog and annotated issues and identified 13 codes (\ie actionable codes) that signified specific actions performed to directly address the problems (\eg reproducing the problem or implementing a solution as a code change). In the second step, we engaged in an analysis of issues/codes and a discussion to categorize the 13 codes for inferring \textit{issue resolution stages}. 
\looseness=-1

\begin{table}[t]
\caption{Issue Resolution Stages}

\label{tab:stages}
\resizebox{\columnwidth}{!}{%
\begin{tabular}{c|c|c|c}
\hline
\textbf{Stage}                & \textbf{Description}                                                                                                                                             & \textbf{Annotation Codes}                                                                                        & \textbf{\# of Issues} \\ \hline
\textbf{\ir (\texttt{\textbf{R}})}      & \begin{tabular}[c]{@{}c@{}}Developers attempt to\\ reproduce the issue.\end{tabular}                                                                             & \sc{REP\_ATT}                                                                                                        & 47 (13.2\%)        \\ \hline
\textbf{\ia (\texttt{\textbf{A}})}          & \begin{tabular}[c]{@{}c@{}}Developers analyze the issue\\ by reviewing the problem,\\ identifying the problem cause,\\ or locating the relevant code.
	\\
 \end{tabular}   & \begin{tabular}[c]{@{}c@{}}\sc{PROB\_LOC},\\ \sc{PROB\_REV},\\ \sc{CAUS\_IDENT}\end{tabular}                               & 134 (37.6\%)       \\ \hline
\textbf{\sd (\texttt{\textbf{SD}})} & \begin{tabular}[c]{@{}c@{}}Developers discuss how to\\solve the issue, \ie propose\\a potential solution or\\review a proposed solution.\end{tabular} & \begin{tabular}[c]{@{}c@{}}\sc{POT\_SOL\_DES},\\ \sc{SOL\_REV}\end{tabular}                                                & 150 (42.1\%)       \\ \hline
\textbf{\impl (\texttt{\textbf{I}})}    & \begin{tabular}[c]{@{}c@{}}Developers make the\\ necessary code changes\\ to resolve the issue.\end{tabular}                                           & \sc{CODE\_IMPL}                                                                                                       & 328 (92.1\%)       \\ \hline
\textbf{\crv (\texttt{\textbf{CR}})}     & \begin{tabular}[c]{@{}c@{}}Developers review the\\ implemented code changes.\end{tabular}                                                                                & \sc{CODE\_REV}                                                                                                        & 264 (74.2\%)       \\ \hline
\textbf{\ver (\texttt{\textbf{V}})}      & \begin{tabular}[c]{@{}c@{}}Developers verify the\\solution by testing the\\ implemented code changes.\end{tabular}                                           & \begin{tabular}[c]{@{}c@{}}\sc{SOL\_VER},\\ \sc{UPLIFT\_APRV},\\ \sc{IMPL\_REV},\\ \sc{COL\_PROB\_ANA},\\ \sc{COL\_POT\_SOL}\end{tabular} & 146 (41\%)         \\ \hline
\end{tabular}%
}
\end{table}


\begin{table}[]
\caption{Problem Categories and Classes}
\label{tab:problem_categories}
\resizebox{\columnwidth}{!}{%
\begin{tabular}{c|c|c|c}
\hline
\textbf{Problem Class}  & \textbf{Problem Categories (Examples)}        & \textbf{\# of Categ.} & \textbf{\# of Issues} \\ \hline
{Implementation} & UI Issue, Feature Development, Crash       & 12                 & 261                \\ \hline
{Refactoring}    & Code Improvement, Unnecessary Code Removal & 2                  & 51                 \\ \hline
{Testing}        & Test Failure, Test Update, Flaky Tests     & 3                  & 44                 \\ \hline
\end{tabular}%
}
\end{table}


The first step was necessary because 9 of the 22 codes were either: (1) requests to perform an action, not an action in itself; or (2) cross-cutting actions, which can be performed at any stage of the resolution process. {\sc{solution\_review\_request}} \rev{is an example of a request}, which represents a petition, made by a developer to another one, to review a proposed solution to the problem. 
{\sc{solved\_by\_other\_issue}} is an example of a cross-cutting code that represents an issue resolved in another issue. 
\looseness=-1

Based on the qualitative analysis, we identified six different issue resolution stages, namely:
\ir (\texttt{\textbf{R}}), \ia (\texttt{\textbf{A}}), \sd (\texttt{\textbf{SD}}), \impl(\texttt{\textbf{I}}), \crv (\texttt{\textbf{CR}}), and \ver (\texttt{\textbf{V}}). Each stage is represented by one to five actionable codes, each code belonging to a single stage. Examples of codes for the \ia stage (\texttt{\textbf{A}}) are {\sc{problem\_localization}} and 
{\sc{cause\_identification}}.
\Cref{tab:stages} shows all the stages with their description and codes.

To answer \ref{rq:stages}, we analyze the frequency in which the six stages appear in the issue reports, across different report types and problem classes and categories.

\subsection{Analysis of Stages Sequences and Process Inference}
\label{sub:process}

When the identified stages are aggregated in the order in which codes appear in the issue report (\ie chronologically), they create a sequence of codes, which we can then examine to understand the process adopted to resolve the issue. For example, issue \#1363344's~\cite{firefox-bug} annotation code sequence is: 
{\sc{code\_implementation}},  {\sc{code\_review}},  {\sc{code\_review}}, {\sc{code\_review}}. We created a stage sequence by utilizing the code sequence and the code-stage mapping for each issue. 
For example, for the above code sequence of issue \#1363344~\cite{firefox-bug}, the derived stage sequence is: \texttt{\textbf{I,CR,CR,CR}} which we simplified as \texttt{\textbf{I,CR}} by merging consecutive repeating stages. This process was applied to all the issues.

To answer \ref{rq:interactions}, we counted the bi-grams and tri-grams appearing in the stage sequences, as well as the number of issues where these n-grams appear. Bi-grams are pairs of consecutive stages, while tri-grams are triplets of consecutive stages in the sequences. We also analyzed the frequency with which the stages appear at the beginning or end of the sequences.
\looseness=-1

To answer \ref{rq:process}, we constructed a graph representing the overall issue resolution process, where the nodes correspond to the stages and the edges represent the transitions between stages. This graph was constructed based on the most frequent bi-grams found in the sequences and serves to validate the patterns of issue resolution we derive as part of \ref{rq:patterns} (see \Cref{sub:resolution_patterns}). 
\looseness=-1


\subsection{Inferring Issue Resolution Patterns}
\label{sub:resolution_patterns}
To answer \ref{rq:patterns}, we engaged in a qualitative analysis of the stage sequences and derived issue resolution patterns by grouping similar stage sequences into coarse-grained sequences. The derived patterns correspond to instances of the derived issue resolution process in \ref{rq:process}.

\subsubsection{\textbf{Pattern Notations}} To communicate the issue resolution patterns clearly and analyze them in different dimensions, we represent the patterns as a string based on  three  notations: 
\begin{itemize}
    \item \textbf{A?}  indicates that stage A is optional;
    \item \textbf{(A$\mid$B)} indicates that either A or B or both stages appear; 
    \item \textbf{(A,B,...,Z)+} indicates that stages A, B, ..., and Z appear more than once, and at least one subsequence of two or more stages (A,B or B,Z or A,B,Z, \etc) appears more than once.
    \looseness=-1
\end{itemize}




\subsubsection{\textbf{Deriving Issue Resolution Patterns}} At first, we identified the stage sequences where the 3rd notation, (A,B,...,Z)+, is applicable and created issue resolution patterns for those stage sequences applying the notation. For example, issue \#991812~\cite{firefox-bug-991812} with the stage sequence \texttt{\textbf{I,CR,I,CR,I,CR,V,I,V}} has \texttt{\textbf{I}}, \texttt{\textbf{CR}}, and \texttt{\textbf{V}} appearing more than once and the sub-sequence \texttt{\textbf{I,CR}} appears more than once. Hence, the sequence can be collapsed to create the issue resolution pattern \texttt{\textbf{(I,CR,V)+}}. With this notation, the order of the stages does not matter.

Second, we created groups of stage sequences that differ only by one or two stages in order and qualitatively analyzed each sequence to understand the differences among the sequences. We aimed to represent the sequences using the first two notations (\ie A? and (A$\mid$B)) to form a coarse-grained sequence. For example, issues \#698552~\cite{firefox-bug-698552}, \#676248~\cite{firefox-bug-676248}, and \#730907~\cite{firefox-bug-730907} have the stage sequences ``\texttt{\textbf{SD,I,CR}}", ``\texttt{\textbf{SD,I,CR,I}}'', and ``\texttt{\textbf{SD,I,CR,V}}", respectively. Here, all three stages, \sds, \impls, and \crvs, are included in the three issues. However, the sequences only differ by the last stage: \impls or \vers is present for the last two issues while it is not present in the first one. Hence, we can create a common pattern for these three issues, \ie\ \texttt{\textbf{SD,I,CR,(I$\mid$V)?}} which will represent all three sequences.

\rev{We meticulously created this grouping} by considering several factors (\eg the \# of issues per sequence, the presence of unique stages per sequence, and the issue resolution process of each issue in the group) so that we would not lose information or create any misleading sequence that does not represent the actual resolution process. For example, we could create a group for the sequences \texttt{\textbf{I}} and \texttt{\textbf{I,CR}} by making \texttt{\textbf{CR}} as an optional stage (\ie  \texttt{\textbf{I,CR?}}). However, the first sequence is found for 21 issues and the second sequence is found for 50 issues which implies these two sequences are already widely used and can represent two distinct ways of issue resolution. In the first sequence, no \texttt{\textbf{CR}} is performed, whereas in the second, it is performed to resolve the issue. Hence, we did not create a group from these two sequences.
\looseness=-1

In all qualitative steps, one researcher qualitatively analyzed the issue and made necessary changes by documenting the rationale behind each change which was reviewed and validated by the second researcher. Both researchers continuously discussed the patterns and solved any disagreements.













\rev{\subsubsection{\textbf{Pattern Derivation Results and Pattern Categorization}} Our analysis resulted in 47 distinct issue resolution patterns -- the  10 most frequent patterns are shown in \Cref{tab:patterns}. The patterns contain 1-6 stages and appear in 1-64 issues (7.6 issues on average). 
The more unique stages and the more interacting stages a pattern has, the more complicated a pattern is. We argue that the complexity of a pattern reflects the effort developers invest in resolving an issue, which can be quantified by the number of stages in the sequences associated with the pattern. Therefore,  we categorized the patterns as \textit{simple} or \textit{complex} based on the average number of stages in their sequences. Since the distribution of these averages is not skewed (see our replication package for the distribution~\cite{repl_pack}), the mean serves as a threshold for classification. Specifically, the process involves calculating the average number of stages ($P_a$) for each pattern, determining the overall mean across the patterns ($T=6.2$ stages), and classifying a pattern as complex if $P_a > T$ or simple if $P_a \leq T$. In \Cref{sub:results_patterns}, we discuss the pattern catalog and compare it with the derived process from \ref{rq:process} to answer \ref{rq:patterns}.}
\looseness=-1

\rev{
\subsection{Investigating Potential Use Cases of the Derived Patterns}

To answer \ref{rq:pattern_usefulness}, we conducted semi-structured interviews with two Mozilla developers, aimed to gather detailed feedback from them on the usefulness of the resolution patterns. The interviews were conducted over Zoom for 60 minutes and were structured into four sections: 

\begin{enumerate}
	\item \underline{Participant's Background}: Participants were asked to share their background and experience in software development and issue resolution at Mozilla and other companies.
	
	\item \underline{Mozilla's Issue Resolution Process}: Participants were asked to describe Mozilla's issue resolution process (both prescribed by Mozilla and implemented by developers) as well as the specific approaches they follow.
	
	\item \underline{Research Presentation}: The research team presented the study's goals, methodology, and findings, including the identified patterns. Participants were encouraged to ask questions about the patterns and findings.
	
	\item \underline{Question-Answering}: Participants were asked 11 questions that prompted for feedback on the identified patterns, with a focus on understanding their potential benefits for Mozilla.
\end{enumerate}

Follow-up questions were asked when additional information was needed. The interview questionnaire, protocol, and anonymized responses are found in our replication package~\cite{repl_pack}.

\subsubsection{\textbf{Finding Participants}} Our target population consisted of Mozilla stakeholders with experience in issue resolution.
To identify potential participants, we explored the developers' profiles from Mozilla Research's website~\cite{mozilla_research}, LinkedIn, Mozilla's issue tracker, Mozilla’s Forums~\cite{mozilla_forums}, and Matrix~\cite{mozilla_matrix}.
We created a shortlist of 42 potential participants, all of whom were invited to participate via email.


\subsubsection{\textbf{Participants' Background}} 
Two developers responded to our call and participated in the interview (\ie referred to as D1 and D2). They are current Mozilla developers with 7 to 11 years of experience at the company.
They have extensive issue resolution experience, having resolved around 1.4K issues and contributed to approximately 19K issues in total. %


\subsubsection{\textbf{Response Analysis}} We recorded and transcribed the interviews using Zoom to facilitate response analysis. We corrected inaccuracies in the transcripts, \eg\ misspellings, incorrect phrases, and punctuation. 
Using the revised transcripts, one author analyzed and grouped the participants’ answers to each question into themes representing use cases of the patterns. A second author reviewed the answers and themes for accuracy. Misinterpretations were resolved through discussion.
\looseness=-1
}




\section{Experimental Set up}
\section{Experimental Results}
We demonstrate the effectiveness of STAIR through extensive experiments on multiple benchmarks that reflect both the safety guardrails and general capabilities of LLMs. 

\subsection{Experimental Settings}

We hereby introduce the key experimental settings, with more details explained in~\cref{sec:appendix_data} and~\ref{sec:appendix_exp}.


\textbf{Models and Datasets.} We take two base LLMs for safety alignment, LLaMA-3.1-8B-Instruct~\cite{dubey2024llama} and Qwen-2-7B-Instruct~\cite{qwen2}. For test-time scaling and ablation studies, only LLaMA is utilized. All experiments use a seed dataset $\mathcal{D}$ comprising 50k samples from three sources. For safety-focused data, we use a modified version of 22k preference samples from PKU-SafeRLHF~\cite{ji2024pku} along with 3k jailbreak data from JailbreakV-28k~\cite{luo2024jailbreakv}. Additionally, 25k pairwise data are drawn from UltraFeedback~\cite{cui2024ultrafeedback} to maintain helpfulness, as done in prior works~\cite{qi2024safety,wu2024thinking}. Note that responses in $\mathcal{D}$ are in normal conversational style rather than reasoning-oriented. While we use the whole dataset with labels for training baselines, we only take 10k samples each from PKU-SafeRLHF and UltraFeedback to construct structured CoT data $\mathcal{D}_{\text{CoT}}$. During each self-improvement iteration, 5k safety and 5k helpfulness samples are utilized. Jailbreak prompts are used only in the final two iterations, with 1k and 2k samples, respectively.

\textbf{Baselines.} We first evaluate the performance of CoT prompting~\cite{wei2022chain} to assess the contribution of available reasoning capability to safety consolidation. We then include SFT and DPO~\cite{rafailov2024direct} on standard datasets as representative alignment techniques, both of which are employed in our framework. Besides, SACPO~\cite{wachi2024stepwise}, designed to mitigate the safety-performance trade-off with two-step DPO, and Self-Rewarding~\cite{yuanself}, which leverages self-generated and self-rewarded data in iterative DPO, are also used as baselines for comparison.


\textbf{Evaluation.} We use 10 popular benchmarks to evaluate harmlessness and general performance of the trained models. For harmlessness, models are required to provide clear refusals to harmful queries, following~\cite{guan2024deliberative}. We test the models on StrongReject~\cite{souly2024strongreject}, XsTest~\cite{rottger2023xstest}, highly toxic prompts from WildChat~\cite{zhaowildchat}, and the stereotype-related split from Do-Not-Answer~\cite{wang2023not}. We report the average goodness score on the top-2 jailbreak methods of PAIR~\cite{chaojailbreaking} and PAP~\cite{zeng2024johnny} for StrongReject, and refusal rates for the rest. For general performance, we use benchmarks reflecting diverse aspects of trustworthiness in addition to the popular ones for helpfulness like GSM8k~\cite{hendrycks2measuring}, AlpacaEval2.0~\cite{dubois2024length} and BIG-bench HHH~\cite{zhou2024beyond}. We take SimpleQA~\cite{wei2024measuring} for truthfulness, InfoFlow~\cite{mireshghallahcan} for privacy awareness, and AdvGLUE~\cite{wang2adversarial} for adversarial robustness. Official metrics are reported for all.

% We leave other details including hyperparameters and evaluation strategies in~\cref{sec:appendix_exp}.


\begin{table*}[ht]
    \centering
    \caption{Performance on diverse benchmarks reflecting both harmlessness and general performance. CoT Style represents whether the method adopt Chain-of-Thought reasoning, while Self Gen. denotes whether the method use self-generated data for training. For all reported metrics, the best results are marked in \textbf{bold} and the second best results are marked by \underline{underline}.}
    \renewcommand{\arraystretch}{1.1} % Increase row height
    
\resizebox{\textwidth}{!}{%
    \begin{tabular}{l@{\;\,}|@{\;\,}c@{\;\,}|@{\;\,}c@{\;\,}|c@{\;\,}c@{\;\,}c@{\;\,}c|c@{\;\,}c@{\;\,}c@{\;\,}c@{\;\,}c@{\;\,}c}
        \toprule[1.5pt]
       & \multirow{2}{*}{\makecell{CoT\\Style}} & \multirow{2}{*}{\makecell{Self\\Gen.}}  &  \multicolumn{4}{c|}{\textbf{Harmlessness}} & \multicolumn{6}{c}{\textbf{General}}  \\ \cmidrule(lr){4-7}\cmidrule(lr){8-13}
       & & & StrongReject  & XsTest  & WildChat  & Stereotype  &  SimpleQA 	&  InfoFlow  &  AdvGLUE  & GSM8k  & AlpacaEval  & HHH  \\\midrule
        \multicolumn{13}{c}{\sc Llama-3.1-8B-Instruct} \\ \midrule
        Base &  - & - & 0.4054 & 88.00\% & 47.94\% & 87.37\% & 2.52\% & 0.4229 & 58.33\% &85.60\% &  25.55\% & 82.50\%\\ 
        CoT & \cmark & - & 0.3790 & 87.00\% & 50.23\% & 65.26\% & 4.09\% &  0.7041 & 58.40\% & 87.11\% &22.04\% & 81.63\% \\
        SFT & \xmark & \xmark & 0.4698 & 94.50\% & 50.68\% & 94.74\% & 4.72\% &  0.7134 & 57.53\% &72.02\% & 9.21\% & 82.63\% \\
        DPO & \xmark & \xmark & 0.5054 & 86.00\% & 54.79\% & \bf 97.89\% & 4.46\% & 0.7081 & 66.27\% &84.15\% &  15.26\% & 83.84\% \\ 
        SACPO & \xmark & \xmark  & 0.7264 & 88.50\% & 58.45\% & 96.84\% & 0.74\% &  0.0503 & 65.60\% &86.50\% & 20.44\% & 85.21\%\\ 
        Self-Rewarding & \xmark & \cmark & 0.4633 & \bf 99.00\% & 49.77\% & 94.74\% & 2.70\%  & 0.6618 & 59.10\% & \bf 88.10\%& 26.41\% & 82.09\%\\\midrule
        STAIR-SFT & \cmark & \xmark & 0.6536 & 85.50\% & 50.68\% & 94.74\% & \underline{6.31\%} & \underline{0.7876} & \bf 70.57\% & 86.05\%  &  31.21\% & 83.13\%\\
        +DPO-1 & \cmark & \cmark & 0.6955 & 94.00\% & 57.99\% & \bf 97.89\% & 6.08\% & \bf 0.7998 & 65.93\% & 86.81\% & 34.48\% & 84.53\% \\
        +DPO-2 & \cmark & \cmark & \underline{0.7973} & 96.50\% & \underline{68.95\%} & 96.84\% & 6.00\% &  0.7700 & \underline{69.43\%} & 87.26\% &\underline{36.24\%} & \bf 87.09\% \\
        +DPO-3 & \cmark & \cmark & \bf  0.8798 &  \bf 99.00\% & \bf 69.86\% & 96.84\% & \bf 6.38\% &  0.7395 & 69.20\% &\underline{87.64\%} &\bf  38.66\% & \underline{85.66\%} \\ \midrule
        \multicolumn{13}{c}{\sc Qwen-2-7B-Instruct} \\ \midrule
        Base &  - & - & 0.3808 & 72.50\% & 47.49\% & 90.53\% & 3.79\% & 0.7221 & 66.50\%& \underline{87.49\%}  & 20.06\% & 87.87\%\\ 
        CoT & \cmark & -  & 0.3792 & 70.00\% & 42.92\% & 88.42\% & 3.03\%& 0.7628 & 65.60\% & \bf 88.10\%  & \underline{25.97\%} & 88.30\%\\
        SFT & \xmark & \xmark & 0.4952 & 84.00\% & 58.45\% & 91.58\% & 3.47\% & 0.6267 & 66.90\% &82.34\% &  8.94\% & 89.74\% \\
        DPO & \xmark & \xmark & 0.5026 & 69.00\% & 66.21\% & 88.42\% & 2.59\% &  0.6793 & 70.97\% & 81.43\% & 11.48\% & 88.08\% \\
        SACPO & \xmark & \xmark & 0.5577 & 75.00\% & 60.27\% & 95.79\% & 0.62\%  & 0.6213 & 64.10\% & 85.22\% & 17.04\% & 89.60\% \\ 
        Self-Rewarding & \xmark & \cmark & 0.5062 & 96.00\% & 52.51\% &  94.74\% & 3.37\% & 0.7140 & 66.13\% & 87.34\% & 14.69\% & 88.31\% \\\midrule
        STAIR-SFT & \cmark & \xmark & 0.7356 & 83.50\% & 62.56\% & 95.79\% & 3.81\% &  0.8215 & 70.57\% &84.61\% & 20.31\% & \underline{90.38\%} \\
        +DPO-1 & \cmark & \cmark & 0.7606 & 96.50\% & 65.19\% & 95.79\% & \underline{3.88\%} & \underline{0.8235} & \underline{73.10\%} & 84.76\% & 23.29\% & 90.21\% \\
        +DPO-2 & \cmark & \cmark & \underline{0.8137} & \underline{98.50\%} & \underline{67.90\%} & \underline{97.89\%} & 3.79\% & \bf 0.8646 & 72.83\% & 86.05\% & 24.86\% & 90.11\% \\
        +DPO-3 & \cmark & \cmark & \bf 0.8486 & \bf 99.00\% & \bf 80.56\% & \bf 98.95\% & \bf 4.07\% & 0.7644 & \bf 74.13\% & 85.75\% & \bf 26.31\% & \bf 90.71\% \\ \bottomrule[1.5pt]
    \end{tabular}}
    \label{tab:benchmarks}
    \vspace{-2ex}
\end{table*}



\subsection{Main Results}

We present the results on diverse benchmarks evaluating both the harmlessness and the general performance in~\cref{tab:benchmarks}, which shows the superiority of STAIR, attributed to the incorporation of introspective reasoning to safety alignment and the self-improvement on stepwise data generated with SI-MCTS. 
We use STAIR-SFT to represent the model trained on $\mathcal{D}_\text{CoT}$ with SFT and DPO-k to denote the model after the k-th iteration of self-improvement. Some qualitative examples are displayed in~\cref{sec:appendix_examples}.

First of all, though initially aligned with instruction tuning, the base LLMs remain vulnerable to harmful queries, especially jailbreak attacks. This is evidenced by the goodness scores below 0.40 on StrongReject. We then explore CoT prompting to stimulate the existing reasoning capability in LLMs. While it leads to improvements in reasoning-dependent tasks like GSM8k and InfoFlow, it shows no enhancement in safety. When applying SFT or DPO to the whole dataset $\mathcal{D}$, we observe significant safety-performance trade-offs due to the conflicting objectives. For instance, for both LLaMA-3.1 and Qwen-2 trained with SFT and DPO, their winning rates against GPT-4 on AlpacaEval decline sharply compared to base models. By employing safety-constrained optimization, the trade-off issue is mitigated to a large extent by SACPO, with better safety enhancements compared to previous methods. However, the performance on SimpleQA and InfoFlow degrades, reflecting losses in factual knowledge and over-refusals to benign privacy-related queries. For Self-Rewarding, their improvements on XsTest, which contains queries apparently harmful, are considerable due to the original behaviors of direct refusals in base LLMs. Nevertheless, the behaviors of refusals fail to generalize to jailbreak attacks, as they lack sufficient capabilities to analyze the underlying risks. 

In comparison, STAIR demonstrates more balanced and continuous improvements on diverse benchmarks. With CoT format alignment, the models acquire the basic ability of safety-aware reasoning, enhancing their resilience against harmful inputs. Further training with stepwise preference data generated by SI-MCTS leads to consistent safety enhancements while maintaining or even improving general performance. For example, LLaMA-3.1 achieves an increase of over 20\% in refusal rate on WildChat after three iterations of self-improvement, while its winning rate against GPT-4 on AlpacaEval reaches 38.66\%, a significant improvement compared to 25.55\% for the base model. Similar trends are observed on other benchmarks like SimpleQA and GSM8k. Besides, the accuracy on AdvGLUE is substantially higher than other baselines, highlighting the benefit to robustness from step-by-step reasoning. On StrongReject, both LLMs eventually reach goodness scores of 0.8798 and 0.8486 respectively, which firmly confirm the positive impact of integrating reasoning with safety alignment.

\subsection{Test-time Scaling}

Using the trained process reward model, we investigate the impact of test-time scaling. Since both stepwise and full-trajectory data are used for training, we employ Best-of-N (BoN) and Beam Search, with results presented in~\cref{fig:tts-safe} and~\ref{fig:tts-helpful} for StrongReject and AlpacaEval respectively. Extra computational costs are estimated based on the number of generated steps relative to one-time greedy decoding, expressed in logarithmic form. For example, Bo8 and beam search generating 4 successors with a beam width of 2 correspond to $\log_2(N)=3$. The results indicate that test-time scaling consistently improves both safety and helpfulness. Both searching methods bring improvements of 0.06 for goodness on StrongReject and more than 3.0\% for winning rates on Alpaca.
This supports that the effect of test-time scaling can generalize from math and coding~\cite{snell2024scaling,xie2024self} to more general scenarios like safety.


\begin{figure*}[t]
     \centering
     \begin{minipage}{0.3\textwidth}
         \centering
         \includegraphics[width=\textwidth, trim={1cm 1cm 1cm 1cm}]{images/draft/strongreject.png}
         \vspace{-4ex}
         \caption{Changes in goodness scores on StrongReject with test-time scaling.}
         \label{fig:tts-safe}
     \end{minipage}
     \hfill
     \begin{minipage}{0.3\textwidth}
         \centering
         \includegraphics[width=\textwidth, trim={1cm 1cm 1cm 1cm}]{images/draft/alpaca.png}
         \vspace{-4ex}
         \caption{Changes in winning rates on AlpacaEval when with test-time scaling.}
         \label{fig:tts-helpful}
     \end{minipage}
     \hfill
     \begin{minipage}{0.3\textwidth}
         \centering
         \includegraphics[width=\textwidth, trim={1cm 1cm 1cm 1cm}]{images/draft/balance.png}
         \vspace{-4ex}
         \caption{Results on StrongReject and AlpacaEval as the ratio of safety data varies.}
         \label{fig:data}
     \end{minipage}
        % \caption{Three simple graphs}
        % \label{fig:three graphs}
    \vspace{-1ex}
\end{figure*}


\subsection{Detailed Analysis}

We then conduct some ablation studies to confirm the effectiveness of our framework.

\textbf{Balance between Safety and Helpfulness Data.} To evaluate the impact of the ratio between safety and helpfulness data in the training dataset, we conduct a study during the CoT format alignment stage as a representative. We plot the performance in terms of safety and helpfulness to the varying ratios in~\cref{fig:data}. While a trade-off between safety and helpfulness is observed, consistent with prior findings~\cite{bai2022training}, the performance in both dimensions consistently exceeds that of the base model. This highlights the effectiveness of training with structured CoT data.

\textbf{Step-level Optimization.} To verify the effectiveness of stepwise preference data in the stage of self-improvement, we compare the performance of DPO-1, which is trained on stepwise data based on STAIR-SFT using DPO, with models trained on full trajectory data using either SFT or DPO. The full trajectory data is selected from the same search trees of SI-MCTS, with the total number of training samples kept equal to that of DPO-1. Results in~\cref{tab:iterative} support our strategy of step-level optimization, which brings more fine-grained supervision to safety-aware reasoning.

\textbf{Iterative Training.} We adopt iterative optimization for continuous improvement, motivated by the belief that data generated in later iterations is of higher quality. To validate this, we compare the results of DPO-3 with the model trained using data crafted from all prompts in a single iteration and the model trained on data from the first iteration for three times as many epochs. Results in~\cref{tab:iterative} demonstrate superior improvements on different benchmarks, confirming the improving data quality throughout iterations.




\begin{table}[ht]
\vspace{-1ex}
    \centering
    \caption{Ablation studies on iterative training on stepwise data}
    % \renewcommand{\arraystretch}{1.2} % Increase row height
\resizebox{\linewidth}{!}{%
    \begin{tabular}{l@{\;\,}|@{\;\,}c@{\;\,}c@{\;\,}c@{\;\,}c}
    \toprule[1.5pt]
         & StrongReject & XsTest & GSM8k & AlpacaEval  \\ \midrule
      \multicolumn{5}{c}{Stepwise Data}\\\midrule
      STAIR-SFT + Full (SFT) &  0.6222 & 87.00\% & 85.29\% & 28.10\% \\
      STAIR-SFT + Full (DPO) &  0.6663 & 92.50\% & 86.50\% & 32.87\%\\\midrule
      STAIR-SFT + Step (DPO) & \bf 0.6955 & \bf 94.00\% & \bf 86.81\% & \bf 34.48\% \\\midrule
      \multicolumn{5}{c}{Iterative Training}\\\midrule
      1st Split, 3$\times$ Epochs & 0.6745 & 97.50\%  & 85.75\% & 37.28\% \\
      Full Dataset, 1 Iteration   & 0.7342 & 90.00\%  & 86.58\% & 36.96\%\\\midrule
      STAIR-DPO-3 & \bf 0.8798 & \bf 99.00\% &  \bf 87.64\% & \bf 38.66\% \\\bottomrule[1.5pt]
    \end{tabular}}
    \label{tab:iterative}
    \vspace{-2ex}
\end{table}


\section{Results}
\subsection{Evaluation}
\label{subsec:Evaluation}
We evaluate the $\mname$ models on $7$ downstream tasks using LM Evaluation Harness \cite{eval-harness} and report the 0-shot non-normalized accuracy metric in Table~\ref{tab:result}. The selected evaluation tasks include ARC (Easy and Challenge) \cite{arc-c-e}, HellaSwag \cite{Zellers2019HellaSwagCA}, PIQA \cite{Bisk2019PIQARA}, SciQ \cite{SciQ}, WinoGrande \cite{Winograd}, and LAMBADA \cite{lambda}.

$\mname$ is compared to two baselines, the base Mistral 7B model and the Flextron model ~\citep{Cai2024FlextronMF} using Avg4 and Avg7 in Table \ref{tab:result}. Compared to Mistral 7B, $\mname$ with $\theta=0.8$ improves efficiency by activating only $5.1B$ of $7B$ parameters on average, with an $7.3$ point accuracy drop after fine-tuning on only $10B$ tokens on the downstream tasks. The number of activated parameters adapts dynamically to token difficulty. For reference, Flextron fine-tunes on $93.57B$ tokens, activating $4.1B$ of $6.5B$ parameters, with a $5$ point accuracy drop from its base model, Llama2-7B \cite{LLM2_LLama}. We emphasize that with only $\frac{1}{9}\text{th}$ of the Flextron's fine-tuning cost, our results  for $\mname$ with $\theta=0.7$  are comparable to Flextron. Accuracy improves with increase in fine-tuning cost, but to keep the adaption lightweight, we opt for a smaller cost.

% We evaluate the $\mname$ models on $7$ downstream tasks (Appendix~\ref{app:downstream_tasks}) using LM Evaluation Harness \cite{eval-harness} and report the 0-shot accuracy metric in Table~\ref{tab:result}. $\mname$ is compared to two baselines, the base Mistral 7B model and the Flextron model ~\citep{Cai2024FlextronMF}. Compared to the Mistral 7B base LLM, $\mname$ offers variants with different accuracy-efficiency trade-offs. At $\theta = 0.8$, $\mname$ achieves significant efficiency gains with minimal accuracy loss. For reference, we also include the Flextron results, which uses $93.57B$ tokens for fine-tuning and activates $63\%$ of the parameters with $6.7\%$ drop in accuracy compared to its base model, Llama2-7B \cite{LLM2_LLama}. We emphasize that with only $\frac{1}{18}\text{th}$ of the Flextron's fine-tuning cost, our results are comparable to Flextron.

\subsection{Analysis of Token-Difficulty-Aware Router}
We assess the performance of the Token-Difficulty-Aware router by gathering its predictions from all layers across the 7 downstream tasks outlined in Section \ref{subsec:Evaluation}. These predictions are then compared to the ground truth labels derived in Section \ref{subsec:generate_gt_label}. Using both sets of labels, we compute the router's overall classification accuracy. 

We present the confusion matrices for the router's classification tasks across all $\mname$ models in Fig. \ref{fig:router_confusion_matrix}. Notably, the matrices exhibit a strong diagonal pattern, indicating high classification accuracy. Furthermore, when the router misclassifies tokens, the errors predominantly occurred in neighboring expert classes, underscoring the router's effectiveness in distinguishing token difficulty levels.


\subsection{Experts usage analysis}
We visualize the expert usage patterns across all layers in  Fig. \ref{fig:cam_expert_load}. For each model, the Y-axis represents the percentage of tokens routed to a specific expert, while the X-axis indicates the layer index. Notably, a token's perceived difficulty may vary across layers, hence it can be routed to different experts as the token progresses through the model. The visualization shows that all experts are utilized in varying proportions across layers, reflecting an aggregated behavior over the $7$ tasks. However, during inference, the model adapts to the data, with simpler queries predominantly engaging lower-compute experts to maximize efficiency.


The parameter $\theta$ affects expert usage in $\mname$ models by controlling how quickly tokens are routed to larger experts based on difficulty. At lower $\theta$ values (e.g., $\theta=0.7$), smaller experts ($e=2$) dominate across layers, optimizing for efficiency. In contrast, at higher $\theta$ values (e.g., $\theta=0.9$), larger experts ($e=3$) are utilized more frequently, prioritizing accuracy over efficiency. This shift demonstrates $\theta$'s role in balancing computational resource allocation and prediction accuracy.


\textbf{Ablating the Router Loss}: 
To examine the router loss's role in expert allocation, we train a $\mname$ model without it, relying solely on the LLM loss to train the router. Tokens are routed using the router’s predicted expert indices without explicit difficulty supervision. The resulting expert usage pattern, shown in Fig. \ref{fig:nocam_gate}, reveals that the model converges to using specific experts per layer instead of dynamically allocating experts based on token difficulty. In contrast, when router loss is applied, expert usage adapts dynamically to token difficulty across layers.

% In the absence of router loss, Fig. \ref{fig:nocam_gate}, the model converges to using specific experts per layer instead of dynamically allocating experts based on token difficulty.

% \section{Related Work}
% \section{Related Work}
Behavior control for Humanoid robots is a long-standing problem, initially explored with simplified humanoid agent \citep{tunyasuvunakool2020dm_control} and recently with full-size humanoid robot \citep{zhuang2024humanoid,fu2024humanplus} such as Unitree H1.
Humanoid robots are of particular interest to the reinforcement learning community because of the high-dimensional action space \citep{merel2017learning, hansen2022temporal, hansen2023td, hansen2024hierarchical}.
To overcome the challenges of exploration in high-dimensional action spaces, some algorithms learn policies by imitating human behavior \citep{fu2024humanplus} or enhance exploration through massive parallelization \citep{zhuang2024humanoid}.
In contrast, our proposed algorithm attempts to learn from scratch without the aid of massive parallelization \citep{makoviychuk2021isaac}. 
We have extensively evaluated our algorithm on the HumanoidBench \citep{sferrazza2024humanoidbench}, a benchmark built on humanoid robot with dexterous hands \citep{menagerie2022github} that contains not only  14 locomotion tasks but also  17 whole-body manipulation tasks.
In the LocoMujoco \citep{al2023locomujoco}, the H1 robot is not equipped with dexterous hands and only focus on locomotion tasks.

Confronted with tasks involving high-dimensional action spaces, model-based RL algorithms \citep{ha2018recurrent, hansen2022temporal, hafner2023mastering, hafner2019dream} often prove to be more sample-efficient compared to model-free alternatives \citep{haarnoja2018soft, fujimoto2018addressing}. 
However, when it comes to humanoid robots with dexterous hands, even the SOTA model-based algorithms struggle to solve it \citep{sferrazza2024humanoidbench}. 
Our algorithm integrates the concept of imitation learning \citep{liu2023ceil, zhang2024context} with the reinforcement learning framework, introducing a loss term of behavioral cloning \citep{pomerleau1988alvinn}.  It may bear a  resemblance to the offline RL \citep{zhuang2024reinformer, fujimoto2019off} algorithm TD3+BC \citep{fujimoto2021minimalist} but our problem setting is   completely different to theirs.
Additionally, it should be noted that the SIRL framework is fundamentally an online RL paradigm that does not rely on expert data, different from IBRL \citep{hu2023imitation} or MoDem \citep{hansen2022modem}.




%Work related to adaptive inference with MoEs...

\section{Conclusion}
\section{Conclusion}

%In this paper, w
We propose a new PEFT method called DiffoRA, which enables efficient and adaptive LLM fine-tuning based on LoRA. 
Instead of adjusting every interior rank, 
%of the decomposition matrices 
%of all modules, 
we argue that adopting LoRA module-wisely is sufficient. 
To achieve this, we construct a DAM to select the modules that are most suitable and essential to fine-tune. We theoretically analyze how the DAM impacts the convergence rate and generalization capability.
%of the pre-trained model. 
Furthermore, we adopt continuous relaxation and discretization to establish DAM.
%for each task. 
To alleviate the issue of discretization discrepancy, we utilize the weight-sharing strategy for optimization. 
%We fully implement our method and t
The experimental results demonstrate that our DiffoRA works consistently better than the baselines across all benchmarks. 

\section*{Limitations}
\section{Limitations}

Our method imposes certain constraints on its applicability to existing decoder-only large language models (LLMs) due to its reliance on parallel encoding/decoding capabilities during the pre-filling stage. This requirement limits its direct adoption in conventional autoregressive LLMs. However, it is worth noting that many high-performance language models with parallel encoding/decoding capabilities have already become standard choices in various Retrieval-Augmented Generation (RAG) systems, such as FiD~\cite{DBLP:conf/eacl/IzacardG21}, CEPE~\cite{DBLP:conf/acl/YenG024}, and Parallel Windows~\cite{DBLP:conf/acl/RatnerLBRMAKSLS23}. Furthermore, our approach requires such models only during the reranker training phase; once trained, the reranker itself is independent of any specific LLM and can be flexibly adapted to other decoder-only models. Therefore, our method primarily serves as a general training framework rather than imposing architectural constraints on the final inference model. Additionally, our approach introduces extra hyperparameters in the Gumbel-Softmax process, including the temperature parameter $\tau$ and the scaling factor $\kappa$, which require tuning to achieve optimal performance. However, through empirical studies, we find that $\tau=0.5$ and $\kappa=1.0$ provide robust and stable performance across different model architectures and datasets. We provide a further discussion on the effect of $\tau$ and $\kappa$ in \autoref{sec: Effect of hyper-parameters on the Training Process}.

\section{Ethical Considerations}
While our method aims to improve the accuracy of the RAG system, it does not eliminate the inherent risks of biased data or model outputs, as the performance of RAG systems still heavily depends on the quality of training data and underlying models. The potential for bias in the training data, particularly for domain-specific queries, can lead to the amplification of these biases in the retrieved results, which can impact downstream applications.


% \section*{Acknowledgments}

% This document

% Bibliography entries for the entire Anthology, followed by custom entries
%\bibliography{anthology,custom}
% Custom bibliography entries only
\bibliography{reference}

% \appendix
% \section{Example Appendix}
% \label{sec:appendix}
% This is an appendix.

\end{document}
