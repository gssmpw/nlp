\section{Related Works}
\label{sec:review}

Modeling vehicle dynamics has been studied for decades. Here, we classify and review related work into two folders as in Sec. \ref{sec:intro}, namely, DSE methods and DRC methods.

\subsection{Vehicle Dynamics State Estimation (DSE) methods}

The DSE methods can be further spitted into three classes, \textit{i.e.}, pure physics-based ones, pure DNN-based ones, and the hybrid ones. Each of them has achieved remarkable success in the last few decades. 

Physics-based DSE methods, or analytic methods, model vehicle dynamics as mathematical rules, and they can vary greatly in complexity and precision. 
As the most simplistic model, point-mass model ____ neglects the size of vehicles and load transfer caused by lateral and longitudinal acceleration, and only considers the vehicle as an infinitesimally small point with a mass, which is commonly used for online trajectory generation ____. 
Single-track models, or 3 DoF bicycle models, ____ improve realism by considering the geometry of the vehicle, providing a more accurate representation of the vehicle’s steering. They project the front and rear wheels on two virtual wheels located along the longitudinal axis of the vehicle and estimate the forces applied to different components of the vehicle. Single-track approaches can capture three fundamental DoF of a vehicle, so they are widely used in planning and control ____. Later, to capture additional effects like yaw moments from different drive and brake forces across an axle, scholars develop double-track models ____. 
% In ____, the authors found a little high-level variation in optimal trajectories generated with single-track and double-track models, while ____ highlighted the importance of longitudinal load transfer by comparing point-mass and single-track models. 
In ____, the authors quantified the impacts of model fidelity on the effectiveness of trajectory optimization and found that the dynamics model used does not have a major effect on the optimal path or speed for the vehicle, but simple models fail to capture transient dynamics.
The trade-off problem between accuracy and efficiency remains an unsolved problem for physics modeling.
Furthermore, System identification is necessary for physics-based DSE models.
Current solutions widely adopt Gaussian process ____ or NN methods ____ to identify parameters of the dynamics systems, but it is still difficult to capture the overall dynamics of the complex vehicle systems due to the limited dimensions of the state-space systems ____, and the identification process requires significant time and commercial costs of real vehicle calibrations. 

Inspired by recent achievements in many research fields ____, AD researchers and engineers have developed DNN-based DSE approach to directly regress vehicle dynamical systems, instead of deriving traditional analytic formula. 
As early as 2015, Punjani \textit{et al.} represented the helicopter dynamics with a Rectified Linear Unit (ReLU) Network Model ____.
In ____, Spielberg \textit{et al.} proposed to employ a DNN to predict yaw and lateral acceleration given a sequence of past states and control inputs, and also experimentally demonstrated that the DNN achieved better performance than the physical model when implemented in the same feedforward-feedback control architecture. 
As an early exploration by Baidu Apollo team, ____ evaluated several novel DNN architectures in vehicle dynamics regression and demonstrated the DNN-based models can achieve higher accuracy with significantly reduced re-development effort compared to rule-based ones. 
Hermansdorfer \textit{et al.} investigated the performance of Gated Recurrent Unit (GRU) networks in DSE task ____.
Cao \textit{et al.} further improved by designing two DNN models to separately learn longitudinal and lateral dynamical properties of the vehicle ____. 
Since the neural network generally contains millions of parameters and high non-linearity, it has demonstrated significant potential to learn to faithfully describe the complex and non-linear dynamics of vehicles from large volumes of data, thus achieving improved state estimation accuracy then physics-based DSE methods, but its accuracy and generalization ability are challenged limited due to domain shift between different conditions.

In recent five years, aiming at high precision and deployable vehicle dynamics modeling, researchers have explored hybrid DSE methods combining physics-based dynamics models with advanced DNN models, sometimes called Physics-Informed Neural Network (PINN) in this area. 
For instance, in ____, Kim \textit{et al.} proposed a full differentiable physics model with prior knowledge by the Pacejka tire model ____ and 3DoF bicycle model. later, Chrosniak \textit{et al.} improved by ensuring that the internal coefficient estimates remain within their nominal physical ranges ____. Similar to system identification approaches in physics-based methods, these two methods also estimated physics coefficient and then leverage well-established dynamical equations to accurately predict vehicle states, but the overall processes are achieved by neural networks in end-to-end manner. These hybrid DSE methods achieve higher precision than traditional pure physics-based or DNN-based alternates, but its upper-bound is still limited by the involved physical dynamics model.

\subsection{Vehicle Dynamics Residual Correction (DRC) methods}

In theory, the DRC methods are also hybrid methods, but they combine physics-based dynamics models with advanced DNN models in another strategy that the DNN model commonly takes physics model's estimation as input and regress the errors (or residuals) of physics model, instead of directly regress dynamics states.
The Baidu Apollo team firstly proposed the DRC framework ____ in 2021. The framework is composed of two models: 1) an open-loop physics-based vehicle dynamics model, and 2) a DNN model that uses Stochastic Variational Gaussian Process (SVGP) to model estimation residuals of physics model.
Ning \textit{et al.} also used Gaussian Process (GP) to model dynamics residual, but they further combined deep kernel learning with GP for DRC task ____.
Chen \textit{et al.} focused on lateral velocity estimation tasks and utilized transfer learning scheme to broaden the applicability of DRC methods, allowing fast model transfer across different vehicle classes ____. 
Baier \textit{et al.} employ DRC framework for interpretable vehicle state estimation, restricting the output range of the DNN model as part of the hybrid system, which limits the uncertainty of the DNN model ____.
These methods all explore the ability of DNN model to regress residuals to refine the estimation of base methods, thus pushing the limitation of physics-based dynamics model.
 
Our work also employ the same DRC framework, but we put our attention on designing a proper DNN model to achieve significant improvements, filling the gap in the DRC approach to specific network designing, and we work towards more complex distributed electric-driven vehicles.