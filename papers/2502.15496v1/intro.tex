\section{Introduction}
Scientific machine learning (SciML) integrates machine learning (ML) into scientific workflows to enhance system simulation and analysis, with an emphasis on computational modeling of physical systems.
This field emerged from Department of Energy workshops and initiatives starting in 2018, which also identified the need to increase ``the scale, rigor, robustness, and reliability of SciML necessary for routine use in science and engineering applications''~\cite{baker2019workshop}.
The field's subsequent growth through funding initiatives, conference themes, and high-profile publications stems from its ability to unite ML's predictive power with the domain knowledge and mathematical rigor of computational science and engineering (\CSE{}).
However, this surge in SciML development has outpaced good practices and reporting standards for building trust~\cite{Mcgreivy_H_arxiv_2024,Kapoor_et_al_SA_2024,Wang_MKetal_COM_2020,Zhu_YR_EST_2023}.

SciML models must demonstrate trustworthiness to be safe and useful~\cite{Jacovi_MMG_ACM_2021}.
Organizational and computational trust definitions~\cite{schneider1999trust,vashney2022trustworthy} inform our criteria for trustworthy SciML: competence in basic performance, reliability across conditions, transparency about processes and limitations, and alignment with scientific objectives.
These criteria span technical attributes (correctness, reliability, safety) and human-centric qualities (comprehensibility, transparency).\footnote{The notion of trustworthiness in SciML differs from ``trustworthy AI'' in socio-technical systems. While trustworthy AI addresses societal impacts, ethical considerations, and human behavioral aspects, trustworthiness in SciML focuses on mathematical rigor, physical consistency, and computational reliability in scientific and engineering applications.}
Trustworthy SciML performs consistently across operating conditions while providing insight into decision-making and adhering to scientific principles.
The model's decision process must align with prior information (intrinsic trustworthiness) and generalize to unseen cases (extrinsic trustworthiness)~\cite{Jacovi_MMG_ACM_2021}.

Rigorous modeling practices require sustained effort from the scientific community.
\CSE{} has established standards over decades for model reporting and assessment, including verification and validation, across biology~\cite{Patterson_W_CVM_1017}, earth sciences~\cite{Oreskes_SB_Science_1994, Jakeman_LN_EMS_2006}, and engineering~\cite{Schwer_EWC_2007, Roy_O_CMAME_2011, Sandkararaman_M_RESS_2015, AIAA_validation_report_1998}.
These standards help users and decision-makers document limitations, uncertainties, and modeling choices to prevent resource misallocation and maintain model credibility.
Similar SciML efforts have only recently begun~\cite{Kapoor_et_al_SA_2024,Wang_MKetal_COM_2020,Zhu_YR_EST_2023}.
Developers and users of SciML must establish best practices for the development, reporting, and evaluation of new SciML paradigms.

This paper initiates a dialogue toward consensus-based practices for predictive SciML—the use of ML models to learn, improve, or accelerate physical system predictions. Rather than creating a rigid checklist, we guide transparent modeling processes by adapting existing \CSE{} verification and validation (V\&V) standards~\cite{AIAA_validation_report_1998,VV10-2006,Oberkampf_T_PAS_2002} while addressing SciML's unique challenges. Our guidance applies to both the development and deployment in support of a scientific claim and new algorithm development.

The remainder of the paper is structured as follows. Section~\ref{sec:background} defines our scope, compares \CSE{} and SciML development processes, and presents our four-component framework for SciML model development.
Sections~\ref{sec:problem-def}--\ref{sec:ongoing} detail these components and develop the recommendations.
The resulting 16 recommendations are listed below for quick reference.
Section~\ref{sec:conclusions} summarizes the key messages.


\begin{tcolorbox}[
    %enhanced, % causes latex error
    title=Recommendations for Trustworthy Scientific Machine Learning,
    colback=white,
    colframe=qblue,
    colbacktitle=qblue,
    coltitle=white,
    fonttitle=\bfseries
]

\noindent\textbf{Problem Definition}
\begin{enumerate}
\item Specify prior knowledge and model purpose
\item Specify verification, calibration, validation, and application domains
\item Carefully select and specify quantities of interest
\item Select and document model structure
\end{enumerate}

\noindent\textbf{Verification}
\begin{enumerate}[resume]
\item Verify code implementation with idealized test problems
\item Verify solution accuracy with realistic benchmarks
\end{enumerate}

\noindent\textbf{Validation}
\begin{enumerate}[resume]
\item Perform probabilistic calibration
\item Validate model against purpose-specific requirements
\item Quantify prediction uncertainties
\end{enumerate}

\noindent\textbf{Continuous Credibility Building}
\begin{enumerate}[resume]
\item Document data characteristics and impact
\item Document data processing procedures
\item Quantify SciML model sensitivities
\item Document the hyperparameter selection process
\item Use software testing and ensure reproducibility
\item Compare developed SciML model against alternatives
\item Explain the SciML prediction mechanism
\end{enumerate}
\end{tcolorbox}
