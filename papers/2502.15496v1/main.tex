\pdfoutput=1 

% Arxiv categories
% Computer science - Machine Learning
% Physics - Computational Physics; 

\documentclass[a4paper,10pt]{article}
\usepackage[utf8x]{inputenc}
\usepackage{amsmath,amssymb}
\usepackage{url}
\usepackage{tcolorbox}
\usepackage{xcolor, graphicx}
\usepackage[left=1.0in, right=1.0in, top=1.0in,bottom=1.0in]{geometry}
\usepackage{enumitem}
\setlist{nolistsep}
\usepackage{authblk}

\usepackage{fontawesome}
\usepackage{pifont}

\definecolor{qblue}{RGB}{26, 31, 113}
\definecolor{qred}{RGB}{139, 0, 0}
\usepackage[hidelinks]{hyperref}

\newcommand{\CSE}{CSE}

\newcounter{recno}
\newtcolorbox{essrec}[1][]{colframe=qblue, title={\refstepcounter{recno} \therecno~$\bullet$~#1}}

\title{Verification and Validation for Trustworthy \\ Scientific Machine Learning}
\author[1]{John~D.~Jakeman} %% correspondence 
\author[2]{Lorena~A.~Barba}
\author[3]{Joaquim~R.~R.~A.~Martins}
\author[4]{Thomas~O'Leary-Roseberry}

\affil[1]{Optimization and Uncertainty Quantification, Sandia National Laboratories, Albuquerque, NM, 87123, USA}
\affil[2]{Department of Mechanical and Aerospace Engineering, The George Washington University, Washington, DC, 20052, USA}
\affil[3]{Department of Aerospace Engineering, University of Michigan, Ann Arbor, MI, 48109, USA}
\affil[4]{Oden Institute for Computational Engineering and Sciences, The University of Texas at Austin, TX, 78712, USA}

\date{}
\begin{document}
\maketitle


\begin{abstract}
Scientific machine learning (SciML) models are transforming many scientific disciplines. However, the development of good modeling practices to increase the trustworthiness of SciML has lagged behind its application, limiting its potential impact. The goal of this paper is to start a discussion on establishing consensus-based good practices for predictive SciML. We identify key challenges in applying existing computational science and engineering guidelines, such as verification and validation protocols, and provide recommendations to address these challenges. Our discussion focuses on predictive SciML, which uses machine learning models to learn, improve, and accelerate numerical simulations of physical systems. While centered on predictive applications, our 16 recommendations aim to help researchers conduct and document their modeling processes rigorously across all SciML domains.
\end{abstract}

\section{Introduction}
\label{sec:intro}

\begin{figure*}[tb]
    \centering
    \includegraphics[width=0.848\linewidth]{figs/circuitnn.pdf} 
    \caption{Illustration of differentiable CircuitNN. CircuitNN is designed based on differentiable NAND gates. After DAS is guided by PI and PO pairs of the truth table, CircuitNN can get the precise circuit architecture logic equivalent to the truth table.}
    \label{fig:circuitnn}
\end{figure*}

% 1. Describe the importance of logic synthesis
% 2. Existing Problems
% (a) Neural Architecture Search: Unstable, Predefined Setting, etc.
% (b) Circuit Generation: Probabilistic Model, Logic Equivalence

With the rapid advancement of technology, the scale of integrated circuits (ICs) has expanded exponentially. 
This expansion has introduced significant challenges in chip manufacturing, particularly concerning power and area metrics.
A primary objective in IC design is achieving the same circuit function with fewer transistors, thereby reducing power usage and area occupancy.

Logic synthesis~\cite{hachtel2005logicsynth}, a critical step in electronic design automation (EDA), transforms behavioral-level circuit designs into optimized gate-level circuits, ultimately yielding the final IC layout. 
The primary goal of logic synthesis is to identify the physical implementation with the fewest gates for a given circuit function. 
This task constitutes a challenging NP-hard combinatorial optimization problem. 
Current logic synthesis tools~\cite{brayton2010abc, wolf2013yosys} rely on human-designed heuristics, often leading to sub-optimal outcomes.

Differentiable architecture search (DAS) techniques~\cite{liu2018darts, chu2020darts} offer novel perspectives on addressing challenges in this problem.
Circuit functions can be represented through truth tables, which map binary inputs to their corresponding outputs. 
Truth tables provide a precise representation of input-output relationships, ensuring the design of functionally equivalent circuits.
Inspired by this, researchers~\cite{deepmind2024ai4sys, wang2024tnet} have begun exploring the application of DAS to synthesize circuits directly from truth tables.
Specifically, \citet{deepmind2024ai4sys} proposed CircuitNN, a framework that learns differentiable connection structures with logic gates, enabling the automatic generation of logic circuits from truth tables.
This approach significantly reduces the complexity of traditional circuit generation. 
Building on this, \citet{wang2024tnet} introduced T-Net, a triangle-shaped variant of CircuitNN, incorporating regularization techniques to enhance the efficiency of DAS.

Despite these advancements, several challenges remain. 
The computational complexity of DAS grows quadratically with the number of gates, posing scalability issues.
Although triangle-shaped architecture~\cite{wang2024tnet} partially mitigates this problem, redundancy persists. 
%Additionally, DAS is susceptible to converging to local optima, limiting the ability to search architectures that satisfy the given truth tables~\cite{liu2018darts}. 
%Furthermore, hyperparameters (network depth and layer width) require extensive searches, introducing complexity and prolonging the synthesis process. 
Additionally, DAS is susceptible to converging to local optima~\cite{liu2018darts} and hyperparameters (network depth and layer width) require extensive searches. 
The challenges arise from the vast search space in DAS. 
% Even with predefined settings for CircuitNN, finding a configuration that meets the truth table requires extensive trial and error during the DAS process. 
Intuitively, limiting the search space through predefined parameters (network depth, gates per layer, and connection probabilities) can significantly reduce the complexity.

Recent advances~\cite{openai2023gpt4, abramson2024alphafold3, esser2024sd3, li2024mar} in conditional generative models have demonstrated remarkable performance across language, vision, and graph generation tasks. 
Motivated by these developments, we propose a novel approach to circuit generation that generates preliminary circuit structures to guide DAS in generating refined circuits matching specified truth tables. 
Firstly, we introduce CircuitVQ, a tokenizer with a discrete codebook for circuit tokenization. 
Built upon our Circuit AutoEncoder framework~\cite{hou2022graphmae,li2023maskgae,wu2025mgvga}, CircuitVQ is trained through a circuit reconstruction task. 
Specifically, the CircuitVQ encoder encodes input circuits into discrete tokens using a learnable codebook, while the decoder reconstructs the circuit adjacency matrix based on these tokens.
Subsequently, the CircuitVQ encoder serves as a circuit tokenizer for CircuitAR pretraining, which employs a masked autoregressive modeling paradigm~\cite{chang2022maskgit, li2023mage}. 
In this process, the discrete codes function as supervision signals. 
After training, CircuitAR can generate discrete tokens progressively, which can be decoded into initial circuit structures by the decoder of the CircuitVQ. 
These prior insights can guide DAS in producing refined circuits that match the target truth tables precisely.

Our key contributions can be summarized as follows:
\begin{itemize}
\item We introduce CircuitVQ, a circuit tokenizer that facilitates graph autoregressive modeling for circuit generation, based on our Circuit AutoEncoder framework;
\item Develop CircuitAR, a model trained using masked autoregressive modeling, which generates initial circuit structures conditioned on given truth tables;
\item Propose a refinement framework that integrates differentiable architecture search to produce functionally equivalent circuits guided by target truth tables;
\item Comprehensive experiments demonstrating the scalability and capability emergence of our CircuitAR and the superior performance of the proposed circuit generation approach.
\end{itemize}

% Motivation
% (a) Diffusion (Vision, Graph), Autoregressive (Language, Vision)
% (b) Circuit Generation for Predefined Setting
% (c) Neural Architecture Search for Strict Logic Equivalence

% Contribution
% (a) Circuit Tokenizer (new transformer arch, training strategy)
% (b) CircuitAR (train and gen strategies, post-ar strategy)
% (c) Extensive Evaluation including BitD (Bit Distance) for Scalability

\section{Basic Background: Supervised Learning and the PAC Model}
\label{sec:background}

At this point almost everyone has heard of machine learning (ML). Anyone likely to stumble upon this article will have also heard of its most influential special case, supervised learning, and those theoretically inclined will also be familiar with the PAC model. Nonetheless, I will set the stage by  recapping the basics.

\subsection{Basics of Supervised Learning}%Let's set the stage in any case

\emph{Supervised Learning} is the task of ``coming up'' with a function $f: \X \to \Y$ to ``explain'' or ``fit'' a sequence of input/output examples   $(x_1,y_1), \ldots, (x_n,y_n)$, with $x_i \in \X$ and $y_i \in \Y$.  Here $\X$ is a \emph{data domain} consisting of \emph{datapoints} $x \in \X$, $\Y$ is a \emph{label set} consisting of \emph{labels} $y \in \Y$, and the sequence $(x_1,y_1),\ldots,(x_n,y_n)$ is the \emph{training data} consisting of \emph{labeled examples (a.k.a. samples)}~$(x_i,y_i)$.  I~will refer to the chosen function $f$ as a \emph{predictor}, and to $n$ as the \emph{sample size}. A \emph{learning algorithm} takes as input training data, and outputs (some representation of) a predictor $f \in \Y^\X$.\footnote{Note that this describes the usual \emph{batch}, a.k.a.~\emph{offline}, setting of supervised learning. I do not discuss other paradigms such as online or active learning in this article.} 



Success in supervised learning is defined as \emph{generalization} to  future examples: For a typical \emph{test example}  $(x_{\tst},y_{\tst})$, the predicted label $y'_{\tst}=f(x_{\tst})$ should ``equal'' $y_{\tst}$, perhaps approximately. We usually assume the test example is drawn from the same  ``source'' as the training data  --- commonly, i.i.d.~from the same distribution. The quality of the prediction is quantified by $\ell(y'_{\tst},y_{\tst})$, where $\ell:~\Y~\times~\Y \to \RR_{\geq 0}$ is a \emph{loss function} chosen as part of the problem definition. Common loss functions include the 0-1 loss $\ell_{0-1}(y',y) = [y' \neq y]$ for \emph{classification} problems,\footnote{The notation $[P]$ denotes $1$ when predicate $P$ is true, and denotes $0$ when $P$ is false.} as well as the absolute loss $|y'-y|$ or squared loss $(y'-y)^2$ for \emph{regression problems} featuring $\Y  \sse \RR$.

Nontrivial generalization properties are typically only possible if one assumes something about the data.\footnote{The need for such an assumption is formalized by the  \emph{no free lunch theorems} of supervised learning \cite{wolpert_connection_1992,wolpert_lack_1996,schaffer_conservation_1994}.} The Bayesian approach to  machine learning, common in many applications, assumes some parametric form for the distribution generating the data, and postulates a prior on the parameters. This is not the approach I will take in this article. Instead, I will focus on the frequentist --- and some would say ``worst-case'' or ``adversarial'' ---  approach that is common in the computational learning theory community, embodied by the PAC model. Here we assume that the (training and test) data can be explained, perhaps approximately, by a function in some ``simple enough to learn'' class of functions $\H \sse \Y^\X$, often called the \emph{hypotheses}. Equivalently, we  seek a predictor which explains the unseen data roughly  as well as the best hypothesis $h^* \in \H$, whether or not we assume that $h^*$ itself provides a perfect explanation.



 \paragraph{Common Algorithmic Templates.} Perhaps the best known general-purpose supervised learning algorithm is \emph{empirical risk minimization (ERM)}, which chooses as its predictor a hypothesis $f \in \H$ minimizing $\frac{1}{n} \sum_{i=1}^n \ell(f(x_i),y_i)$ --- a quantity called the \emph{training error}, \emph{empirical error}, or \emph{empirical risk} of $f$. %\footnote{When multiple hypotheses minimize the empirical risk, we assume ERM breaks ties arbitrarily.}
A common template for generalizing ERM involves adding a \emph{regularization term} $\psi(f)$ to the  objective function, typically chosen to measure some notion of ``hypothesis complexity.'' An algorithm instantiating this template is known as a \emph{structural risk minimizer (SRM)}, and chooses as its predictor the hypothesis $f \in \H$ minimizing the \emph{structural risk} $\frac{1}{n} \sum_{i=1}^n \ell(f(x_i),y_i) + \psi(f)$. Other well-known algorithms, such as gradient descent and its variations,  can frequently be interpreted as approximate implementations of ERM or SRM.


\paragraph{Proper vs Improper Learning.} A learning algorithm is said to be \emph{proper} if its predictor $f$ is always chosen from the hypothesis class, i.e., $f \in \H$, otherwise it is said to be \emph{improper}. ERM  is an example of a proper learning algorithm, as are SRM algorithms of the form described above.  In the \emph{proper regime} of learning, algorithms are required to be proper. This article will be concerned with the more flexible \emph{improper regime} (a.k.a \emph{representation-independent learning}), where no such constraint is placed on the learner. In other words, all we care about is predictive power at test time, rather than any insights derived from the functional form or representation of the predictor~itself.


\subsection{The PAC Model}
A standard mathematical setup for evaluation of supervised learning algorithms, at least in the theoretical computer science community, is Valiant's \emph{Probably Approximately Correct (PAC) model} of learning (see e.g.~\cite{kearns_introduction_1994,mohri_foundations_2018}). Here, we assume there is an unknown distribution $\D$ on $\X \times \Y$ from which training and test data are  drawn.  Specifically, the labeled datapoints of the training set  $(x_1,y_1), \ldots, (x_n,y_n)$, as well as the test data  $(x_\tst,y_\tst)$, are i.i.d.~from $\D$. Often it is assumed that $\D$ lies in some class of distributions of interest. The \emph{true expected loss}, or simply \emph{loss}, of a predictor $f: \X \to \Y$ is the expected loss it incurs on draws from $\D$, written $L_\D(f) = \Ex_{(x,y) \sim \D} \ell(f(x),y)$.


There are two main ``settings'' in PAC learning. The  \emph{realizable setting} only requires that the data be perfectly explained by some hypothesis in $\H$. More generally, the \emph{agnostic setting} makes no assumption relating the data to the hypotheses, but shifts the goalposts as necessary to allow nontrivial guarantees: the expected loss at test time is evaluated only ``relative'' to that of the best hypothesis $h^* \in \H$. There are other settings which make more nuanced assumptions, such as $\D$ being of a particular parametric form or its support living in some (unknown) lower-dimensional space, etc. I will mostly discuss the realizable and agnostic settings in this article, those being the simplest and most studied from a theoretical perspective. %TODO:We will briefly discuss other settings in Section ??

The PAC model demands high probability guarantees of learners, in the worst case over distributions of interest. Consider first the realizable setting, where $\D$ is such that $\min_{h \in \H} L_{\D}(h) = 0$. A PAC learner has \emph{error} $\epsilon=\epsilon(n)$ and \emph{confidence} $\delta=\delta(n)$ if, when training data consists of $n$ i.i.d~samples from a realizable distribution $\D$, it produces a predictor $f$  satisfying $L_\D(f) \leq \epsilon$ with probability at least $1-\delta$. In the agnostic setting, where $\D$ can be arbitrary, we require $L_\D(f) - \min_{h \in \H} L_\D(h) \leq \epsilon$ with probability $1-\delta$.

In both the realizable and agnostic settings, we look for PAC learners with small $\epsilon$ and $\delta$ as a function of the sample size $n$. An equivalent perspective looks at the sample complexity $m(\epsilon,\delta)$, which is the minimum sample size which guarantees error  at most $\epsilon$ with probability at least $1-\delta$. We say a problem is \emph{PAC learnable} if its PAC sample complexity is finite whenever $\epsilon,\delta > 0$.

For most PAC learning problems, learnability and sample complexity are characterized in terms of a  ``dimension'' of the hypothesis class. Most prominently this is the \emph{VC dimension} for binary classification, the \emph{fat shattering dimension} for agnostic regression, and the \emph{DS dimension} for multiclass classification (see \cite{anthony_neural_1999,daniely_optimal_2014,brukhim_characterization_2022}). Treatment of these is beyond the scope of this article. The unfamiliar reader need not worry, however,  as dimensions will feature only tangentially in our~discussion.




%\paragraph{Learning settings: Realizable, Agnostic, etc.} In learning theory, evaluating a supervised learning algorithm requires specifying a data model and an objective. We will leave the details of the data model flexible for now, to allow for both the PAC model and the adversarial transductive model. Nonetheless we will describe two variations, which we call ``settings'', which cut across different models. The  \emph{realizable setting}  requires only that the data be perfectly explained by some hypothesis $h \in \H$ --- i.e., there exists a hypothesis which is guaranteed to suffer a loss of $0$ on training and test data. The performance of the learning algorithm is its expected loss at test time for some ``worst case'' realizable instance. More generally, the \emph{agnostic setting} makes no assumption relating the data to the hypotheses, but shifts the goalposts as necessary to allow nontrivial guarantees: the expected loss at test time is evaluated only ``relative'' to that of the best hypothesis $h^* \in \H$, again for some ``worst case'' instance. There are other settings which make more nuanced assumptions about the data, such as it is drawn from a distribution of a particular parametric form, or that it lives in some (unknown) lower-dimensional space, etc. We will mostly discuss the realizable and agnostic settings, those being the simplest and most studied from a theoretical perspective.




%%% Local Variables:
%%% mode: latex
%%% TeX-master: "learning_matching"
%%% End:
 
\begin{tikzpicture}[
    node distance=2em and 3em,
    every node/.style={rectangle, draw, rounded corners, text centered, minimum height=1em},
    arrow/.style={-Stealth, thick},
    decision/.style={diamond, draw, text centered, inner sep=0pt, aspect=2},
]

% Nodes
\node[decision] (decision) {\faUser};
\node[above right=1em and 6em of decision] (generator) {\ggen ($\S\text{\ref{sec:generator}}$)};
\node[below right=1em and 6em of decision] (manual) {Write Manually};
% \node[right=of decision, text width=12em] (fuzzing) {Automated obtaining from programs\\ or Manually Written};
\node[right=22em of decision] (sippy) {\tool ($\S\text{\ref{sec:positive} \&}~\S\text{\ref{sec:SLsynthesis}}$)} ;
\node[above right=1em and 12em of sippy] (verifiers) {Verification ($\S\text{\ref{sec:verification}}$)};
\node[right=12em of sippy] (synthesizers) {Synthesis ($\S\text{\ref{sec:synthesis}}$)};
\node[below right=1em and 12em of sippy] (others) {Other applications};


% Edges with labels
\draw[arrow] (decision) -- node[above, draw=none, align=center, sloped] {Program} (generator);
\draw[arrow] (decision) -- node[below, draw=none, align=center, sloped] {or} (manual);
\draw[arrow] (manual) -- node[above, draw=none, align=left] {Memory Graphs~~} (sippy);
\draw[arrow] (generator) --  (sippy);
\draw[arrow] (sippy) -- node[above, draw=none, align=center] {Heap Predicates} node[below,draw=none,align=center] {for further applications} ++(14em, 0) coordinate (branch);
% \draw[arrow] (mid) -- node[above, draw=none, rectangle, align=center, sloped] {2} ++(1em, 0) coordinate (branch);
\draw[arrow] (branch) |- (verifiers);
\draw[arrow] (branch) |- (synthesizers);
\draw[arrow] (branch) |- (others);

\end{tikzpicture}

This work identifies signal collapse as a critical bottleneck in one-shot neural network pruning. Performance loss in pruned networks is due to \textbf{signal collapse} in addition to the removal of critical parameters. We propose \textbf{REFLOW} (\textbf{Re}storing \textbf{F}low of \textbf{Low}-variance signals), a simple yet effective method that mitigates signal collapse without computationally expensive weight updates. By focusing on signal preservation, REFLOW highlights the importance of mitigating signal collapse in sparse networks and enables magnitude pruning to match or surpass state-of-the-art one-shot pruning methods such as CHITA, CBS, and WF.

REFLOW consistently achieves state-of-the-art accuracy across diverse architectures, restoring ResNeXt-101 from under 4.1\% to 78.9\% top-1 accuracy at 80\% sparsity on ImageNet. Its lightweight design makes it a practical solution for both research and deployment, delivering high-quality sparse models without the overhead of traditional approaches. These findings challenge the traditional emphasis on weight selection strategies and underscore the critical role of signal propagation for achieving high-quality sparse networks in the context of one-shot pruning.




\section{Concluding remarks}
\label{sec:conclusions}
Scientific machine learning (SciML) is an emerging field that integrates machine learning into scientific workflows, creating a powerful synergy between traditional and new methodologies.
More specifically, we define SciML as the application of machine learning methods to enhance, accelerate, or improve the computational modeling and simulation of physical systems.
This definition is applicable across a wide array of scientific disciplines, including physics, earth sciences, and engineering.
However, we acknowledge that this definition may not be universally embraced in certain fields, such as psychology and social sciences.

While computational science and engineering (CSE) has developed solid theoretical foundations over decades, SciML's rapid empirical advances have often outpaced formal theoretical analysis.
Successfully advancing SciML requires thoughtfully integrating the complementary strengths of both fields: CSE's mathematical rigor and ML's data-driven innovations.
Moreover, the CSE community has established rigorous guidelines for model development, verification, and validation (V\&V) to support scientific claims.
In contrast, such standards are still emerging in SciML.
Consequently, 
this paper addresses the critical need for developing V\&V  practices for SciML that builds upon CSE frameworks while addressing SciML's unique challenges.

In this paper, we identified key challenges to developing trusted SciML practices, including the field's heavy reliance on data, the optimization of non-convex functions, and gaps in theoretical understanding.
To address these challenges, we presented a four-component framework for developing and deploying trustworthy SciML models.
As part of this framework, we established 16 recommendations to provide concrete guidance on critical aspects like reporting computational requirements and preparing detailed documentation. We also used real-world examples to illustrate concepts and offer actionable guidance for both development and deployment phases.

The recommendations in this paper are particularly crucial for high-consequence applications where model credibility is paramount. 
These include safety-critical systems in aerospace and nuclear engineering, digital twins for infrastructure monitoring, and medical applications that use model predictions guide treatment decisions. In such domains, the cost of model failure---whether from inadequate verification, poor validation, or incomplete uncertainty quantification---can be severe. Our framework provides a foundation for building the necessary trust in SciML models for these critical applications.


Beyond these domain-specific challenges, the rapid evolution of the field and the drive to publish quickly have led to poor reproducibility, often compromising rigor and transparency.
Here, we aim to catalyze broader community dialogue around establishing consensus-based practices for predictive SciML.
As such, this paper makes a meaningful contribution by being one of the first comprehensive attempts to address systematic V\&V practices specifically for SciML.
It fills an important gap in the literature, as while both CSE and ML have their own established practices, the intersection of these fields requires new approaches to ensure trustworthiness.

We encourage journal editors to adopt these, or similar, recommendations in their submission guidelines, funding agencies to require V\&V plans in proposals, conference organizers to promote sessions on SciML credibility, and researchers to implement these practices in their work. 
Only through collective effort can we establish rigorous standards needed to fully realize SciML's potential for accelerating scientific discovery and enabling trustworthy predictions in high-consequence applications.

\section*{Acknowledgments}
Funding: John Jakeman's work was funded by Sandia National Laboratories’ Laboratory Directed Research and Development (LDRD) program.
Thomas O'Leary-Roseberry's work was funded by the National Science Foundation under DMS Award 2324643 and OAC Award 2313033.
Sandia National Laboratories is a multi-mission laboratory managed and operated by National Technology \& Engineering Solutions of Sandia, LLC (NTESS), a wholly owned subsidiary of Honeywell International Inc., for the U.S. Department of Energy’s National Nuclear Security Administration (DOE/NNSA) under contract DE-NA0003525.
This written work is authored by an employee of NTESS. 
The employee, not NTESS, owns the right, title and interest in and to the written work and is responsible for its contents.
Any subjective views or opinions that might be expressed in the written work do not necessarily represent the views of the U.S. Government.
The publisher acknowledges that the U.S. Government retains a non-exclusive, paid-up, irrevocable, world-wide license to publish or reproduce the published form of this written work or allow others to do so, for U.S. Government purposes.
The DOE will provide public access to results of federally sponsored research in accordance with the DOE Public Access Plan.

AI Disclosure: During the preparation of this work the authors used Sandia National Laboratories' SandiaAI Chat and Claude.ai in order to correct spelling and grammar and reduce the length of the manuscript.
After using these tools, the authors reviewed and edited the content as needed and take full responsibility for the content of the publication. 

\bibliographystyle{plain}
\bibliography{references}
\end{document}
