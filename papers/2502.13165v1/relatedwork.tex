\section{Related Work}
\subsection{Quantitative Finance}
%\textbf{Quantitative Finance.} 
Quantitative finance is an interdisciplinary field that merges finance with mathematical and statistical methods to address complex financial challenges \cite{kou2019machine,kanamura2021pricing}. It plays a crucial role in the valuation of sophisticated financial derivatives, portfolio optimization, and the analysis of market dynamics \cite{tavella2003quantitative,horvath2021deep}. With the advent of machine learning, the field has seen a surge in predictive modeling capabilities, enhancing both the accuracy of market forecasts and the efficiency of algorithmic trading systems \cite{mieg2022volatility}, such as MV \cite{YU2011367}, DeepTrader \cite{wang2021deeptrader}. As financial markets continue to evolve, the integration of quantitative techniques is essential for developing effective strategies that can withstand market uncertainties \cite{finreportacm}.

\subsection{LLM-based Agent}
%\textbf{LLM-based Agent.} 
LLM-based Agents, leveraging the cognitive and generative prowess of Large Language Models \cite{touvron2023llama, chowdhery2023palm}, promising advancements towards Artificial General Intelligence (AGI)\cite{yang2023autogpt}. LLMs play a vital role in enhancing the autonomy\cite{Wang_2024}, reactivity, and social interaction capabilities of agents. This enables agents to perform a range of complex tasks, including natural language interaction, knowledge integration, information memory, logical reasoning, and strategic planning\cite{sumers2023cognitive, park2023generative} . LLM-based agent systems based on LLMs hold promise for impactful contributions across sectors like finance (e.g. FinGPT \cite{yang2023fingpt}, FinAgent \cite{zhang2024multimodal}), offering novel approaches and sophisticated solutions to intricate challenges \cite{hong2023metagpt,wu2023bloomberggpt,zhang2023appagent}.