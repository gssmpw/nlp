%\documentclass[sigconf,review]{acmart}
%\documentclass[sigconf,natbib=true,anonymous=true]{acmart}
%\documentclass[sigconf,review]{acmart}
\documentclass[sigconf]{acmart}
% \documentclass[sigconf,natbib,anonymous,review]{acmart}
% \documentclass[sigconf]{acmart}
\usepackage{tabularx}
\usepackage{mdwlist}
\usepackage{subcaption}
\hypersetup{colorlinks=false,linkcolor=blue,urlcolor=blue,citecolor=red}
\usepackage[labelfont=bf,textfont=bf]{caption}
\usepackage{multirow}
\usepackage{array}
\usepackage[linesnumbered,ruled]{algorithm2e}
\SetKwComment{Comment}{$\triangleright$\ }{}
\usepackage{xcolor}
%\usepackage{amssymb}
\usepackage{epstopdf}
\usepackage{enumitem}
\usepackage[utf8]{inputenc} % allow utf-8 input
\usepackage[T1]{fontenc}    % use 8-bit T1 fonts
\usepackage{hyperref}       % hyperlinks
\usepackage{url}            % simple URL typesetting
\usepackage{booktabs}       % professional-quality tables
\usepackage{amsfonts}       % blackboard math symbols
\usepackage{nicefrac}       % compact symbols for 1/2, etc.
\usepackage{microtype}      % microtypography
\usepackage{xcolor}         % colors
\usepackage{enumitem}
\usepackage{listings}
\usepackage{graphicx}  %插入图片的宏包
\usepackage{float}  %设置图片浮动位置的宏包
%\usepackage{subfigure}  %插入多图时用子图显示的宏包
\usepackage{subcaption} 
%\usepackage{algorithm}
%\usepackage{algorithmic}
\usepackage{placeins}
\usepackage{amsmath}
\usepackage{adjustbox}
\usepackage[utf8]{inputenc}

\copyrightyear{2025}
\acmYear{2025}
\setcopyright{acmlicensed}\acmConference[WWW Companion '25]{Companion Proceedings of the ACM Web Conference 2025}{April 28-May 2, 2025}{Sydney, NSW, Australia}
\acmBooktitle{Companion Proceedings of the ACM Web Conference 2025 (WWW Companion '25), April 28-May 2, 2025, Sydney, NSW, Australia} \acmDOI{10.1145/3701716.3715232} \acmISBN{979-8-4007-1331-6/2025/04}
%Conference



% The default list of authors is too long for headers}
%\renewcommand{\shortauthors}{Paper XXX}

\begin{document}

%\fancyhead{}

\title{HedgeAgents: A Balanced-aware Multi-agent Financial Trading System}

%\author{Paper ID: xxx}
\author{Xiangyu Li}
\authornote{Both authors contributed equally to this paper}
\email{65603605lxy@gmail.com}
\orcid{0009-0002-8261475r}
\affiliation{
  \institution{South China University of Technology}
  \city{Guangzhou}
  \country{China}
}

\author{Yawen Zeng}
\orcid{0000-0003-1908-1157}
\email{yawenzeng11@gmail.com}
\authornotemark[1]
\affiliation{%
  \institution{ByteDance}
  \city{Beijing}
  \country{China}
}

\author{Xiaofen Xing}
%\authornote{Corresponding authors}
\email{xfxing@scut.edu.cn}
\orcid{0000-0002-0016-9055}
\affiliation{
  \institution{South China University of Technology}
  \city{Guangzhou}
  \country{China}
}

\author{Jin Xu}
\orcid{0009-0001-8735-3532}
\email{jinxu@scut.edu.cn}
%\authornotemark[2]
\affiliation{%
  \institution{South China University of Technology}
  \institution{Pazhou Lab}
  \city{Guangzhou}
  \country{China}
}

\author{Xiangmin Xu}
\authornote{Corresponding author}
\orcid{0009-0001-8735-3532}
\email{xmxu@scut.edu.cn}
\affiliation{%
  \institution{South China University of Technology}
  \city{Guangzhou}
  \country{China}
}



\hyphenpenalty=2000
\tolerance=6000

\begin{CCSXML}
<ccs2012>
   <concept>
       <concept_id>10002951.10003317.10003371.10003386</concept_id>
       <concept_desc>Information systems~Financial Trading Systems</concept_desc>
       <concept_significance>500</concept_significance>
       </concept>
   <concept>
       <concept_id>10003752.10010070.10010071.10010261.10010276</concept_id>
       <concept_desc>Theory of computation~Multi-agent Systems</concept_desc>
       <concept_significance>500</concept_significance>
       </concept>
   <concept>
       <concept_id>10003752.10010070.10010071.10010261</concept_id>
       <concept_desc>Theory of computation~Reinforcement learning</concept_desc>
       <concept_significance>500</concept_significance>
       </concept>
 </ccs2012>
\end{CCSXML}

\ccsdesc[500]{Information systems~Financial Trading Systems}


\begin{abstract}
As automated trading gains traction in the financial market, algorithmic investment strategies are increasingly prominent. While Large Language Models (LLMs) and Agent-based models exhibit promising potential in real-time market analysis and trading decisions, they still experience a significant -20\% loss when confronted with rapid declines or frequent fluctuations, impeding their practical application. Hence, there is an imperative to explore a more robust and resilient framework. This paper introduces an innovative multi-agent system, \textbf{HedgeAgents}, aimed at bolstering system robustness via ``hedging'' strategies. In this well-balanced system, an array of hedging agents has been tailored, where HedgeAgents consist of a central fund manager and multiple hedging experts specializing in various financial asset classes. These agents leverage LLMs' cognitive capabilities to make decisions and coordinate through three types of conferences. Benefiting from the powerful understanding of LLMs, our HedgeAgents attained a 70\% annualized return and a 400\% total return  over a period of 3 years. Moreover, we have observed with delight that HedgeAgents can even formulate investment experience comparable to those of human experts (\url{https://hedgeagents.github.io/}).
\end{abstract}

\keywords{Quantization Finance, Large Language Model, Multi-agent Systems}

\maketitle
%real-world

\begin{figure}[h]
    \center
    \vspace{-0.8cm}
    \includegraphics[width=0.38\textwidth]{fig1.pdf}
    \vspace{-0.5cm}
    \caption{Our method outperformed all baselines on the PRUDEX benchmark \cite{sun2023prudexcompass} across six dimensions. Marks on the inner circle represent the market average, while the outer layer details the measures evaluated.
    %Our method has comprehensively surpassed all baselines on the PRUDEX\cite{sun2023prudexcompass} evaluation benchmark. The internal star chart within the figure provides a visual representation of the relative strength of different methods across six dimensions. A mark on the inner circle of the star chart indicates the market average, while the outer layer specifies the particular measures assessed in practice to showcase the state of the evaluation.
    }
    \label{fig:compass}
    \vspace{-0.5cm}
\end{figure}


%%%%%%%%%%%%%%%%%%%%%%%%%%%%%%%%%%%%%%%%%%%%%%%%%%s

\section{Introduction}

\begin{figure*}[htbp]
  \centering
   \includegraphics[width=0.99\textwidth,keepaspectratio]{fig2.pdf}
  \vspace{-0.6cm}
  \caption{Performance of state-of-the-art models on Total Return (TR), Sharpe Ratio (SR) and Cumulative Returns (CR) in real-world scenarios, with the majority exhibiting negative scores.}
  \label{fig:intro1}
  \vspace{-10pt}
\end{figure*}
In contemporary financial markets, automated trading has emerged as a pivotal investment strategy \cite{bankivestment}. Leveraging advanced algorithms, traders can actively track market trends in real-time, enabling swift and precise decision-making, thereby enhancing investment yields while mitigating risk \cite{kanamura2021pricing}. The integration of artificial intelligence into financial transactions has unlocked significant opportunities, notably with the advent of large language models (LLMs) \cite{touvron2023llama,chowdhery2023palm,zeng2021multi}. These cutting-edge models possess the capability to analyze extensive financial and news information, forecast market variations, and even produce financial analyses and reports \cite{finreportacm}. Moreover, LLM-based agents can refine trading strategies through the simulation of market dynamics. % with greater precision

Despite the impressive performance of these cutting-edge models \cite{yang2023autogpt,pan2023llms}, they still lack the robustness required to withstand real-world fluctuations. As depicted in Figure \ref{fig:intro1}, the performance of state-of-the-art models is notably poor on Total Return (TR) and Sharpe Ratio (SR) metrics. For instance, in ``rapid decline'' scenarios, 8 out of 9 baselines exhibit negative scores below zero. Meanwhile, both RL-based DeepTrader \cite{wang2021deeptrader} and LLM-based FinGPT \cite{yang2023fingpt} experience financial losses ranging from {\color{red}-15\% to -20\%} in ``frequent fluctuations'' scenarios.
%utilizing LLMs or Agents
%As depicted in Figure \ref{fig:intro1}, the performance of state-of-the-art models on Total Return (TR) and Sharpe Ratio (SR) metrics in various real-world scenarios (such as rapid rises, rapid declines, and frequent fluctuations) reveals a notable limitation. 
This limitation stems from the absence of risk management mechanisms within these models or systems, impeding their practical application in real markets. Compounded by the inherent complexity and uncertainty already prevalent in financial markets, the challenge of designing and optimizing trading strategies is further exacerbated.


In this paper, our aim is to incorporate the concept of ``hedging'' in the development of a robust and reliable financial trading system. Empirically, hedging serves as a strategic tool that enables traders to sustain returns amidst market volatility by constructing diversified portfolios. Along this line, we are committed to building a multi-agent system that mirrors real-world market, with a LLM acting as the brain \cite{hong2023metagpt,wu2023bloomberggpt}. A key challenge lies in the configuration and coordination of these hedge agents, which constitutes our primary focus. Illustrated in Figure \ref{fig:intro1}, unconstrained agent system (e.g. FinAgent \cite{zhang2024multimodal}) can only achieve suboptimal performance.

We are devoted to developing a multi-agent hedging system named \textbf{HedgeAgents}. Within this well-balanced framework, we have established a comprehensive array of hedge agents, comprising a fund manager and a team of specialized experts in various fields such as Stocks, Forex, and Bitcoin. Each expert is tasked with overseeing their designated area, while the fund manager plays a pivotal role in orchestrating their efforts through facilitating discussions, conducting reviews, and consolidating insights. Leveraging the profound understanding capabilities of LLM, these agents (from their memories) have distilled invaluable investment experience akin to those crafted by human counterparts! Drawing from these rich experiences and hedging expertise, our HedgeAgents not only demonstrates unwavering stability even in the face of extreme conditions (Figure \ref{fig:intro1}), but also wins across all metrics, achieving a total return of {\color{teal}400\% over 3 years} (Figure \ref{fig:compass}). We hold a strong belief that this research will pave the way for a dependable automated system that enhances the security of financial investments. 
% a pioneering endeavor in this domain

The contributions are summarized as follows: 
%(i)-To the best of our knowledge, this work represents a pioneering effort in integrating ``hedging'' within a multi-agent environment to develop a robust and reliable financial trading system. (ii)-We have established a hedging portfolio consisting of three experts and one manager, collaborating through three types of conference, with a LLM serving as the central cognitive hub. (iii)-Extensive experiments has demonstrated that our framework delivers optimal performance on all metrics, achieving a 70\% annualized return and a total return of 400\% over 3 years. Furthermore, the hedge experts within our framework have produced an invaluable investment experience akin to human expertise. 
%The implementation codes and datasets are available at here\footnote{https://hedgeagents.github.io/}.
\begin{itemize}[leftmargin=*]
\item {}To the best of our knowledge, this work represents a pioneering effort in integrating ``hedging'' within a multi-agent environment to develop a robust and reliable financial trading system.
\item{} We have established a hedging portfolio consisting of three experts and one manager, collaborating through three types of conference, with a LLM serving as the central cognitive hub.
\item{} Extensive experiments has demonstrated that our framework delivers optimal performance on all metrics, achieving a 70\% annualized return and a total return of 400\% over 3 years. Furthermore, the hedge experts within our framework have produced an invaluable investment experience in their memories akin to human expertise. The codes are available at here\footnote{https://hedgeagents.github.io/}.
\end{itemize}


\begin{figure*}[htbp]
  \centering
   \includegraphics[width=0.95\textwidth,keepaspectratio]{fig3.pdf}
  \vspace{-0.4cm}
  \caption{Our HedgeAgents comprise 3 hedging agents and 1 manager. Each agent is equipped with 23 tools and possesses 3 types of memory to execute 8 actions. Furthermore, collaboration among multiple agents can be categorized into three types of conferences: {\color{teal}budget allocation}, {\color{olive}experience sharing}, and {\color{red}extreme market conference}.}
  \label{fig:framework}
  \vspace{-0.4cm}
\end{figure*}


\section{Related Work}
\subsection{Quantitative Finance}
%\textbf{Quantitative Finance.} 
Quantitative finance is an interdisciplinary field that merges finance with mathematical and statistical methods to address complex financial challenges \cite{kou2019machine,kanamura2021pricing}. It plays a crucial role in the valuation of sophisticated financial derivatives, portfolio optimization, and the analysis of market dynamics \cite{tavella2003quantitative,horvath2021deep}. With the advent of machine learning, the field has seen a surge in predictive modeling capabilities, enhancing both the accuracy of market forecasts and the efficiency of algorithmic trading systems \cite{mieg2022volatility}, such as MV \cite{YU2011367}, DeepTrader \cite{wang2021deeptrader}. As financial markets continue to evolve, the integration of quantitative techniques is essential for developing effective strategies that can withstand market uncertainties \cite{finreportacm}.

\subsection{LLM-based Agent}
%\textbf{LLM-based Agent.} 
LLM-based Agents, leveraging the cognitive and generative prowess of Large Language Models \cite{touvron2023llama, chowdhery2023palm}, promising advancements towards Artificial General Intelligence (AGI)\cite{yang2023autogpt}. LLMs play a vital role in enhancing the autonomy\cite{Wang_2024}, reactivity, and social interaction capabilities of agents. This enables agents to perform a range of complex tasks, including natural language interaction, knowledge integration, information memory, logical reasoning, and strategic planning\cite{sumers2023cognitive, park2023generative} . LLM-based agent systems based on LLMs hold promise for impactful contributions across sectors like finance (e.g. FinGPT \cite{yang2023fingpt}, FinAgent \cite{zhang2024multimodal}), offering novel approaches and sophisticated solutions to intricate challenges \cite{hong2023metagpt,wu2023bloomberggpt,zhang2023appagent}.



\section{Proposed Method}

\subsection{Preliminaries}
The objective is to optimize returns while minimizing risk using hedging strategies. Specifically, this task involves utilizing financial data, encompassing prices and news, as input. Our framework then generates and implements trading actions, such as buying and selling, across three domains: Bitcoin, Stocks, and Forex.

\subsection{Overall Framework}
The HedgeAgents framework simulates the architecture of a real hedge fund company, aiming to optimize risk hedging for multi-asset investment portfolios. As depicted in Figure \ref{fig:framework}, the framework comprises four agents: Bitcoin Analyst Dave, Dow Jones Analyst Bob, Forex Analyst Emily, and Hedge Fund Manager Otto. Each of the three analysts is in charge of managing a specific asset. Otto, as their supervisor, is responsible for the overall risk management of the investment portfolio and the distribution of the asset investment budget.
%\footnote{For detailed explanation of each member's profile,please refer to Appendix~\ref{app:Profile}.}.
To achieve effective collaboration, we have established three types of multi-agent coordination meetings: Budget Allocation Conference (BAC), Experience Sharing Conference (ESC) and Extreme Market Conference (EMC). These conferences serve to facilitate budget allocation, experience summary, and swift emergency actions.
% \begin{itemize}[leftmargin=*]
%   \item \textbf{Extreme Market Conference}: Held when there are severe market fluctuations to discuss countermeasures.
%   \item \textbf{Experience Sharing Conference}: Conducted regularly, where analysts share investment cases and summarize general investment experiences.
%   \item \textbf{Budget Allocation Conference}: Scheduled periodically to allocate the next phase's investment budget based on market conditions and the performance of the analysts.
% \end{itemize}



\subsection{Definitions of Single Agent}
In this section, we will provide a comprehensive overview of the composition and execution process of a single agent, designed to simulate the human decision-making process in investments. Each agent comprises a range of financial analysis tools $\mathcal{T}$, along with definitions for profile, action $\mathcal{A}$, memory and reflection $\mathcal{M}$. Among them, there are 23 tools, encompassing Indicator Analysis, Cryptocurrency Market Analysis, and Risk Management. Actions comprise 8 types, such as Buy/Sell/Hold for current assets. Memory is categorized into 3 types: basic Market Information Memory $\mathcal{M}_{MI}$, Investment Reflection $\mathcal{M}_{IR}$, and General Experience $\mathcal{M}_{GE}$.
%\footnote{For a detailed introduction to each component, please refer to Appendix~\ref{app:agent}.}.

%\vspace{-0.5cm}
\begin{center}
\fcolorbox{black}{gray!10}{\parbox{1\linewidth}{
<Simplified Prompt Template> \\
You are \{Dave Profile\}. The market environment today includes \{Prices\}, \{News\}. Through financial analysis tools, \{Tool Results\} can be obtained. The output format should be JSON, such as \{Examples\}.
}}
\end{center}

%In our approach, we have divided the Memory module $\mathcal{M}$ into three distinct parts: Market Information Memory $\mathcal{M}_{MI}$, Intelligent Investment Reflection Memory $\mathcal{M}_{IIR}$, and General Investment Experience Pool $\mathcal{M}_{GIE}$. For a detailed introduction to each component, please refer to Appendix~\ref{app:agent}.
% In this section, we will introduce the Investment Decision Agent (IDA) module in detail. This module simulates the human investment decision-making process, based on large language models, and integrates financial analysis tools, memory retrieval, and reflective decision-making, providing an automated solution for complex market investment decision problems. As shown in Figure XXX, the Investment Decision Agent module includes Tool, Action, Memory, Reflection, and Profile.

The following will introduce the workflow for a single agent. We have integrated the LLM-based intelligent investment agent into a reinforcement learning framework through a reflection-driven decision-making process, enabling the flexible definition of the reasoning processes. The reflection-driven process includes three steps: memory retrieval, decision making, and reflection update.

%The following will introduce the workflow for a single agent. We have integrated the LLM-based intelligent investment agent into a reinforcement learning framework through a reflection-driven decision-making process, enabling the flexible definition of the reasoning processes. The reflection-driven process includes three steps: memory retrieval, decision making, and reflection update.
\textbf{Memory Retrieval.} We employ retrieval strategy to search for reliable experiences to enhance decision-making. Notably, due to the substantial input received by each agent at time step $t$ (comprising market information, financial news, and tool results), we will first compile a summarized query $Q_t$ and then utilize it to retrieve K=5 similar cases $\mathcal{M}_{ret}$ from all memories $\mathcal{M}=\{\mathcal{M}_{MI}, \mathcal{M}_{GE}, \mathcal{M}_{IR}\}$.
%In the specific workflow, the agent generates the current investment reflection $IR_t^{\lambda}$ based on the input market information, financial news and analysis tool output , and then generates the query text $Q_t^{\lambda}$ . Next, the agent uses $Q_t^{\lambda}$ to retrieve the $K$ most relevant past investment cases from the memory $\mathcal{M}$. The retrieved results are then summarized to generate the retrieval text sequence $R\mathcal{M}_t^\lambda$.
% \begin{equation}
%     \begin{gathered}
%   R\mathcal{M}_t^\lambda=\{\mathcal{M}_{MI}, \mathcal{M}_{GIE}, \mathcal{M}_{IIR}\}^k \leftarrow \text{TopK Retrieve}(Q_t^{\lambda},\mathcal{M})\\
%     \end{gathered}
% \end{equation}

\textbf{Decision Making.} Based on the experience $\mathcal{M}_{ret}$, we extend reinforcement learning as follows:
\begin{equation}
    \begin{aligned}
	\pi_{{\theta}^*}^{}=\arg\max_{{\pi}_{\theta}}{\mathbb{E}           {_{\pi}}_{\theta}}\left[\sum_{t=0}^T\gamma r_{t}\left|s_t=s,\mu_t=\mu \right.\right],
    \end{aligned}
\end{equation}
where $r_t$ is the reward at time step $t$, which depends on the environment state $s_t$ (e.g. market indicators) and action $a_t$ (e.g. bug/sell). An action $a_t$ is determined through the following process:
%$\mu(\cdot)$ is a specialized module that encapsulates the beneficial internal reasoning process
\begin{equation}\label{eqn:2}
    \begin{aligned}
        {\pi}_{Agent}\left(a_t|s_t,\mu_t\right) & \equiv {\mathcal{D}}\left(LLM\left(\phi_D\left(s_t,\mathcal{M}_{ret}\right)\right)\right) , \\
    \end{aligned}
\end{equation} 
where the prompt template $\phi_D(\cdot)$ is designed to receive responses via a LLM, which are then processed by parsing function $D(\cdot)$ to determine compatible actions in this environment. Therefore, the goal of a agent is to find the policy $\mu$ to optimize the total return.
% as described in Eqn.\ref{eqn:3}
% \begin{equation}\label{eqn:3}
% \begin{aligned}
%     \pi_{Agent}^* &= \arg\max_{\pi(\cdot),\mu(\cdot)} \mathbb{E} \pi \left[ \sum_{i=0}^T \gamma^i r_{t+i} \left| s_t=s, a_t=\mu \right. \right] , \\
%     \text{s.t.} \quad &\pi(a_t|s_t,\mu_t) = \mathcal{D}^{\lambda}\left(LLM\left(\phi_D^{\lambda}\left(s_t,\mu_t\right)\right)\right) \ \quad \forall t .
% \end{aligned}
% \end{equation}

\textbf{Reflection Update.} The basic market information and summarized queries $Q_t$ will be stored in the basic memory $\mathcal{M}_{MI}$, while the reflection and actions during decision-making will be updated in the investment reflection $\mathcal{M}_{IR}$.

\subsection{Coordination of Hedging Agents}
In this section, we design three collaborative meetings to implement the hedging strategies of multiple agents. Among them, the periodic Budget Allocation Conference realizes the trading of multiple assets, with Dave (Bitcoin), Bob (Stocks), Emily (Forex), and manager Otto participating. The Extreme Market Conference serves as an emergency mechanism, enhancing the overall robustness of our system, while the Experience Sharing Conference fosters the exchange of valuable investment insights.

\subsubsection{Budget Allocation Conference}
%In the Budget Allocation Conference, the three investment agents will proceed with the Individual Report in sequence. Each report includes information on the following two parts: Current Profit Situation $P_{it}$ and Budget Expectations \& Reasons $B_{it}$. Then, the Hedge Fund Manager Otto combines the results of each agent's Current Profit Situation $P_{it}$ and Budget Expectations and Reasons $B_{it}$, Otto integrates the comprehensive evaluation opinions $S_{Aux_i}$ from Auxiliary Agents regarding the performance of the three agents. Based on this information, an assessment is made of the expected future returns ${\rho}_{it}$ for the assets managed by the three agents.
This conference spans a 30-day cycle, wherein Dave, Bob, and Emily present reports to Manager Otto aimed at budget allocation. Each report encompasses the current profit situation $R_{p}$ alongside budget expectations \& reasons $R_{b}$. Subsequently, manager Otto consolidates all reports and auxiliary tools $R_{a}$ via a prompt template ${\phi}_{\rho}$, to evaluate expected future returns ${\rho}_{i}$.
\begin{equation}
    \begin{aligned}
        {\rho}_{i}=LLM\left({\phi}_{\rho}\left(R_{p},R_{b},R_{a}\right)\right), i \in \{Dave, Bob, Emily\} .
    \end{aligned}
\end{equation}
%After obtaining the expected future returns ${\rho}_{mt}$, we introduce three indicators for the investment portfolio: the expected total return $\zeta_t$, the overall portfolio risk $\sigma_t$, and the conditional expected draw down risk $\mathrm{CVaR}_t$, in order to prepare for the allocation of weights for the assets managed by the three agents within the total assets of the portfolio.
Furthermore, to refine the asset portfolio weighting, three additional indicators are introduced: expected total return $I_{etr}$, overall portfolio risk $I_{pr}$, and conditional expected draw down risk $I_{cvar}$.
\begin{equation}
    I_{etr} = \sum_{i=1}^{3} \omega_{i} \rho_{i}, \quad
    {I_{pr}^2} = \sum_{i=1}^{3} \sum_{j=1}^{3} \omega_i \omega_j I_{ij}, \quad
    I_{cvar} = \frac{1}{1-\alpha} \int_{1}^{\alpha} \mathrm{VaR}(p) \, dp ,
\end{equation}
where $w_i$ represents the weight of asset $i$, $\sigma_{ij}$ is the covariance between the returns of asset $i$ and asset $j$, and $\mathrm{VaR}(\alpha)$ is the Value at Risk at confidence level $\alpha$, estimated using the historical simulation method.
Finally, budget allocations are determined via an optimization objective function:
\begin{equation}
\begin{aligned}
    \omega^* & =\arg\max_{\omega}I_{etr}-\lambda_{1}{I_{pr}}-\lambda_{2}I_{cvar},\quad
    \text{s.t.} \underset{}{\overset{}{\underset{i}{\overset{}{\sum}}{\omega}_i=1,{\omega}_i\ge0}} , %i \in [1,3]
\end{aligned}
\end{equation}
where $\lambda_1$ and $\lambda_2$ are risk aversion coefficients, which ensure the sum of the allocated budget weights is 1 and non-negative.

\subsubsection{Experience Sharing Conference}
%At the end of each investment cycle, three investment agents convene for an experience sharing meeting. The meeting proceeds in multiple rounds, in each round, all agents summarize and share a Typical Action Case $C_{it}$, where $\quad i \in \{Dave, Bob, Emily\}$, based on their Intelligent Investment Reflection Memory $\mathcal{M}_{IIR}$. The remaining two agents study and discuss the case. After the final round, each agent $i$ updates its General Investment Experience Pool $\mathcal{M}_{GIE}$ using the function $\mathcal{U}$, considering the entire meeting's accumulated discussion content $D_{dt}$, its own case $C_{it}$, and the previous $\mathcal{M}_{GIE_{i(t-1)}}$:
This is a regular meeting held at the end of each investment cycle, where 3 hedging agents engage in multi-round knowledge accumulation. In each round, each agent will recapitulate a typical case $C_{t}$ from the memory $\mathcal{M}_{IR}$ for discussion among peers, with all insights $D_{t}$ being archived in memory $\mathcal{M}_{GE}$. In this way, the invaluable experience is consolidated to bolster future decision-making.
\begin{equation}
    \begin{aligned}
    \mathcal{M}_{GE}=LLM({\phi}_{c}(D_{t},C_{t})),
    \end{aligned}
\end{equation}
where the LLM aims to update the general experience memory $\mathcal{M}_{GE}$ via a prompt template $\phi_{c}$.

\subsubsection{Extreme Market Conference}
% Hedge Fund Manager Otto will continuously monitor the performance of the overall investment portfolio. Once any asset is detected to be in an extreme market condition, Otto will initiate an Extreme Market Investment Conference request based on preset thresholds. In the conference, {\color{red}the agent $A$} in the extreme market condition will need to address the following three questions: The current holdings of the investment portfolio $\mathcal{H}_{At}^\varepsilon$;
% The main reasons causing the asset to be in an extreme market condition $\mathcal{E}_{At}^ \varepsilon $;
% The disposal plan for the remaining assets $\mathcal{P}_{At}^ \varepsilon $.
% {\color{red}The other two agents} will serve as advisory agents in this Extreme Market Investment Conference. Their suggestions for addressing the current loss situation are denoted as $\mathcal{S}_{Bt}^\varepsilon$ and $\mathcal{S}_{Ct}^\varepsilon$.
% Next, the hedge fund manager Otto will engage in self-reflection based on the information provided, and generate a course of action $EMR_t^ \varepsilon$ for dealing with the current extreme market conditions $EMR_t^ \varepsilon$, and transmit it to agent $A$.Agent $A$ will integrate the received instructions with the methods described in section 3.3.6, make decisions regarding the current extreme market situation.
This is an interim session convened to address volatile market conditions, as identified by manager Otto, characterized by a daily amplitude exceeding 5\% or a cumulative three-day amplitude surpassing 10\%. During this session, the crisis agent facing challenges are required to present their current portfolio holdings, articulate the reasons behind the crisis, and outline the proposed plans. Following this, Otto and other agents will offer suggestions $\mathcal{S}_{B}$, $\mathcal{S}_{C}$, $S_{E}$ aimed at fostering reflection and optimizing the actions, with the ultimate goal of navigating through the turbulent period effectively.
\begin{equation}
    \begin{aligned}
        \mathcal{S}=(1-{\lambda_3}){\phi_B}(\mathcal{S}_{B},\mathcal{S}_{C})+{\lambda_3}{\phi_B}(\mathcal{S}_{E}),
    \end{aligned}
\end{equation}
where ${\lambda_3}$ is used to balance the importance of suggestions $\mathcal{S}$, and $\phi_B(\cdot)$ is a prompt template. The optimization process is similar to Eqn.\ref{eqn:2}.
%signifies the parameter representing the degree of emphasis on extreme market conditions. 
%After fully considering the extreme market conditions, agent A makes decisions $ \pi_{EMI}^*$ and takes actions$\pi_{EMI}(a_t|s_t,\mu_t)$ regarding the disposal of assets in extreme market situations. 
\begin{equation}
\begin{aligned}
    \pi^* &= \arg\max_{\pi(\cdot),\mu(\cdot)} \mathbb{E} \pi \left[ \sum_{t=0}^T \gamma r_{t} \left| s_t=s, \mu_t=\mu \right. \right] \\
    \text{s.t.} \quad &\pi(a_t|s_t,\mu_t) =\mathcal{D}\left(LLM\left(\phi_B\left(s_t,\mathcal{S}\right)\right)\right)  \quad \forall t
\end{aligned}
\end{equation}



\begin{table*}[]
\caption{Performance comparison of all baselines on 9 evaluation metrics. \textbf{Bold} represents optimal performance, while \underline{underline} represents suboptimal.}
\label{tab:main}
\small
\centering
\normalsize
\setlength{\tabcolsep}{6pt} % 调整列间距
\renewcommand{\arraystretch}{0.99} % 调整行间距
\resizebox{1\textwidth}{!}{ 
\begin{tabular}{cccccccccccl}
\hline
\hline
Categories       & Models           & ARR(\% )& TR(\%)  & SR      & CR      & SoR     & MDD(\%) & Vol(\%) & ENT     & ENB     \\
\midrule
                 & Bitcoin          & 12.92 & 43.97 & 0.54 & 1.19   & 12.49 & 76.63 & 3.40  &    --     &    --     \\
Market Trends           & FX               & 4.08& 12.74& 0.61& 0.93& 11.99& \underline{15.56}&\textbf{0.38}&    --     &    --     \\
                 & DJIA             & 7.64& 24.70& 0.59& 1.06    & 11.72& 21.94& \underline{0.78}&    --     &     --    \\\midrule
                 & MV               & 13.03& 44.39& 0.71& 1.25& 16.14& 32.04& \textcolor{black}{1.13}& 1.09& 1.02\\
Rule-based        & ZMR              & -7.25& -20.21& -0.52& -3.13& -5.15& 61.52& 1.98& 1.55& 1.11\\
                 & TSM              & 19.13& 69.09& 0.78& 1.53& 18.21& 39.14& 1.55& 1.10& 1.09\\\midrule
                 & SAC              & 24.71& 93.94& 1.16& 3.12& 23.15& 21.56& 1.16& 1.62& 1.14\\
RL-based         & DeepTrader       & 32.78& 134.11& 1.41& 4.06& 30.43& 20.95& 1.21& 2.02& 1.30\\
                 & AlphaMix+        & 37.59& 160.47& 1.62& 3.69    & 35.52& 25.56& 1.17& \underline{2.93}& 1.22\\\midrule
                 & FinGPT           & 34.22& 141.82& \underline{1.93}& \underline{7.64}& \underline{39.57}   &\textcolor{black}{ 17.08}& 8.76& 1.76& \textcolor{black}{1.33}\\
LLM-based        & FinMem           & \textcolor{black}{47.67}& \textcolor{black}{221.99}& 1.20& 4.02& 25.42& 32.39& 2.16& 1.99& 1.25\\
                 & FinAgent         & \underline{53.54}& \underline{261.98}  & \textcolor{black}{1.80}&\textcolor{black}{ 4.52}&\textcolor{black}{ 39.12}& 28.24& 1.42& \textcolor{black}{2.85}& \underline{1.41}\\\midrule
Ours             &\textbf{HedgeAgents}     & \textbf{71.60}& \textbf{405.34}& \textbf{2.41}& \textbf{11.02}& \textbf{58.00}& \textbf{14.21}& 1.30& \textbf{3.13}& \textbf{1.53}\\
\midrule
\multicolumn{2}{c}{Improvement(\%)} & 33.75& 54.72& 24.49& 44.28& 46.58& 16.76&  \multicolumn{1}{c}{--}& 6.85& 8.51\\
\hline
\hline
\end{tabular}
}
\end{table*}

\begin{figure}[htbp]
  \centering
   \includegraphics[width=0.49\textwidth,keepaspectratio]{fig4.pdf}
  \caption{Cumulative Returns Comparison of all baselines and our HedgeAgents.}
  \label{fig:CRAll}
\end{figure}

\section{Experiments}
\subsection{Datasets}\label{sec:datasets}
We have compiled a comprehensive financial dataset comprising Bitcoin, foreign exchange, and the Dow Jones component stocks. These data were sourced from reputable financial databases, namely Yahoo Finance and the Alpaca News API. The dataset spans from January 1, 2015, to December 31, 2023, encompassing daily data points such as open, high, low, and close prices, as well as volume and adjusted close prices. Additionally, daily news updates\footnote{Following \cite{finreportacm}, we only use processed news headlines that cover the key information of the news.} and 60 standard technical analysis indicators are included for each asset.
%, with further details provided in the Appendix~\ref{app:dataset}.
% To comprehensively assess the performance of the proposed HedgeAgents architecture across different asset categories , we constructs a large financial dataset . The dataset is composed of three subsets: the Bitcoin dataset, the foreign exchange dataset, and the Dow Jones 30 dataset. The statistical information is summarized in Table~\ref{tab:dataset}.

% The data for the three subsets are sourced from authoritative financial databases, Yahoo Finance and the Alpaca News API, ensuring the reliability and authenticity of the data. During the construction process, we adhered to the following criteria:
% i) The time span is from January 1, 2015, to December 31, 2023, covering nearly nine years of financial market data.
% ii) All asset price data are on a daily frequency, including open, high, low, close prices, volume, and adjusted close prices.
% iii) Daily updated news data are provided for each asset, ensuring a diversified and comprehensive perspective on the financial markets.
% iv) 60 common technical analysis indicators are assigned to each asset, with the specific construction method detailed in Appendix~\ref{app:dataset}.
% v) For each dataset, financial assets with missing values were filtered out.


\subsection{Evaluation Metrics}\label{subsec:eva}
We compare HedgeAgents and baselines in terms of 9 financial metrics following \cite{sun2023trademaster,qin2023earnhft}, which include 2 profit metrics: Total Return (TR), Annual Return Rate (ARR); 3 risk-adjusted profit metrics: Sharpe Ratio (SR), Calmar Ratio (CR), Sortino Ratio (SOR); 2 risk metrics: Maximum Drawdown (MDD), Volatility (VOl); and 2 diversity metrics: Entropy (ENT) and Effect Number of Bets (ENB).
%\footnote{Detailed definitions and formulas are available in Appendix ~\ref{app:metrics}.}.

% \subsection{Baselines}
% To comprehensively evaluate the performance of HedgeAgents in investment decision-making and asset allocation, we selected a variety of classical and cutting-edge baseline models for comparison. These include three classical rule-based quantitative investment strategies (Classical methods): MV, ZMR, and TSM; three reinforcement learning-based financial agents (RL-based methods): SAC, DeepTrader, and AlphaMix+; and three investment methods based on LLM models (LLM-based methods): FinGPT, FinMem, and FinAgent. A brief introduction to each method is available in Appendix ~\ref{app:baseline}.

\subsection{Implementation Details}
%Although the HedgeAgents architecture can operate in environments without a GPU, to accelerate benchmarking, we utilized an NVIDIA RTX 4090 GPU. 
The dataset is divided using January 1, 2021, as the cutoff point, with data from January 1, 2015, to December 31, 2020, designated as the training set, and data from January 1, 2021, to December 31, 2023, as the testing set. Notably, during the testing phase, only historical prices can be access to avoid data leakage issues. To ensure the fairness of the evaluation results, all baseline models are trained and tested in the same reinforcement learning environment.
%\footnote{The parameter settings can be obtained in Appendix ~\ref{app:Implementation}.}.

For baseline configurations, we employed Optuna \cite{akiba2019optuna} for hyperparameter optimization.
%fine-tuning classic quantitative strategies and reinforcement learning methods to adapt to the trading environment, with evaluations conducted using the optimal hyperparameters. 
For LLM-based approaches and HedgeAgents, we consistently use the GPT-4-1106-preview version with a temperature setting of 0.7 to balance consistency and creativity. The memory module in our framework is designed as a text similarity-based storage and retrieval mechanism, utilizing the text-embedding-3-large model \cite{openai2023textembedding3}, with a top-k value of 5. We ensured LLM-based methods are adjusted to their optimal configurations as suggested in their respective studies. 


\begin{table*}[]
\caption{Ablation analysis on three conference. \checkmark denote the inclusion of components.}
%within the variant.}
\label{tab:rq2}
\small
\centering
\normalsize
\setlength{\tabcolsep}{6pt} % 调整列间距
\renewcommand{\arraystretch}{0.99} % 调整行间距
\resizebox{1\textwidth}{!}{ 
\begin{tabular}{ccc|ccccccccc}
\hline \hline
ESC & BAC & EMC & ARR(\%) & TR(\%) & SR   & CR    & SoR   & MDD(\%) & Vol(\%) & ENT  & ENB  \\ \hline
 \checkmark    &    &     & 43.88   & 197.88 & 1.90  & 4.89  &  42.20 & 21.70    & 1.12    &  2.56 &  1.23 \\
    &  \checkmark  &     &  45.58   &  208.52 & 1.67 & 4.54  & 36.36 & 24.68   & 1.34    & 2.39 & 1.17 \\
    &    &  \checkmark   & 40.97   & 180.13 &  1.96 & 5.07 & 39.58 & 19.59  & 1.02   & 1.97 & 1.08 \\ \hline
    &  \checkmark   &  \checkmark    & 50.68   & 242.11 & 2.01 & 6.92  & 45.60  & 17.26   & 1.19    & \textcolor{black}{2.61} & \textcolor{black}{1.28} \\
 \checkmark   &    &  \checkmark    & \textcolor{black}{44.57}   & \textcolor{black}{202.17} & 2.24 & 8.78  & 55.01 & 12.02   & 0.94    & 2.72 & 1.39 \\
 \checkmark    &  \checkmark   &     & 59.81   & 308.12 & \textcolor{black}{1.93} &\textcolor{black}{ 5.69 } & \textcolor{black}{40.16} & \textcolor{black}{24.44}   & \textcolor{black}{1.44 }   & 2.91 & 1.42 \\ \hline
 \checkmark    &  \checkmark   &  \checkmark    & \textbf{71.60}    & \textbf{405.34} & \textbf{2.41} & \textbf{11.02} & \textbf{58.00}    & \textbf{8.68}   & \textbf{1.30}     & \textbf{3.13} & \textbf{1.53} \\ \hline \hline
\end{tabular}
}
\end{table*}

\begin{figure*}[htbp]
  \centering
   \includegraphics[width=1\textwidth,keepaspectratio]{fig5.pdf}
  \caption{Ablation analysis on several LLM backbones, from open-source to closed-source models.}
  \label{fig:rq3}
\end{figure*}

\subsection{Overall Performance Comparison} 
We selected a diverse range of classical and state-of-the-art baseline models for comparison purposes. These encompass three classical rule-based quantitative investment strategies: MV\cite{YU2011367}, ZMR\cite{eeckhoudt2018dual}, and TSM\cite{moskowitz2012time}; three reinforcement learning-based financial agents: SAC \cite{haarnoja2018soft}, DeepTrader\cite{Wang_Huang_Tu_Zhang_Xu_2021}, and AlphaMix+\cite{sun2023prudexcompass}; and three LLM-based methods: FinGPT\cite{yang2023fingpt}, FinMem\cite{yu2023finmem}, and FinAgent\cite{zhang2024multimodal}. To ensure a fair comparison, we meticulously adhered to the specific requirements of each baseline, providing them with the necessary conditions for optimal performance. 
%We carefully adapted our experimental datasets and environment to accommodate the unique needs of each model, ensuring that all baselines could fully utilize the available data and achieve their best possible performance within our trading framework.
%A brief introduction to each method is available in Appendix ~\ref{app:baseline}.
%We conducted a comprehensive assessment of HedgeAgents' overall investment performance using the evaluation metrics introduced in Section \ref{subsec:eva}. Table \ref{tab:results} and Figure \ref{fig:performance} demonstrate that our method has surpassed existing benchmark methods in this field, setting a new benchmark. A more intuitive comparison of capabilities can be found in Appendix \ref{app:comparison}.


The experimental results are presented in Table \ref{tab:main}. The following observations can be made: 1) Compared to rule-based strategies, RL-based methods demonstrate a superior capacity to comprehend the intricacies and uncertainties in financial markets . For instance, DeepTrader and AlphaMix+ achieve ARR of 32.78\% and 37.59\%, respectively, which are significantly higher than rule-based models MV (13.03\%), ZMR (-7.25\%), and TSM (19.13\%), indicating the superior learning and decision-making processes of RL-based agents.
2) Leveraging LLMs, as exemplified by FinGPT, FinMem, and FinAgent, enables the formulation of well-informed decisions, surpassing RL-based methods in risk-adjusted metrics such as SR and CR. FinMem achieves an ARR of 47.67\% and a TR of 221.99\%, figures that are substantially higher than RL-based agents, suggesting that LLMs' ability to process and contextualize information leads to more effective investment strategies.


3) Our HedgeAgents exhibits exceptional performance across all metrics, attributable to the following factors:
i) \textbf{Strategic Budget Allocation}.
The dynamic budget allocation plays a pivotal role in optimizing investments across diverse asset categories. This is evidenced by the model achieving the highest ARR of 71.60\% and TR of 405.34\% among all baselines, showcasing the efficacy of strategic budget distribution in enhancing financial outcomes;
ii) \textbf{Sophisticated Risk Management}.
HedgeAgents demonstrate an impressive MDD of 14.21\%, surpassing all other baseline models in risk management.
%HedgeAgents exhibit an impressive MDD of 14.21\%, which is the lowest among all baseline models. This indicates a sophisticated risk management framework capable of protecting capital during market downturns, highlighting the model's resilience and robustness in adverse market conditions;
iii) \textbf{Diversification and Robustness}.
With the highest ENT and ENB, HedgeAgents have demonstrated a diversified investment portfolio that is resilient to the unique risks associated with individual assets. The consistent outperformance of HedgeAgents, as further substantiated by the cumulative returns depicted in Figure \ref{fig:CRAll}. Notably, during May 2022, while most portfolios managed by benchmark models experienced significant losses, HedgeAgents adeptly navigated through this challenging period, demonstrating its superior capabilities.

In summary, these analyses confirm that HedgeAgents has not only surpassed benchmarks in individual metrics but also achieved a harmonious balance of returns, risk management, and diversification. This positions HedgeAgents as a cutting-edge solution, adeptly equipped to operate within complex and dynamic investment environments.

\begin{figure*}[t]
    \centering
    \begin{subfigure}{0.31\textwidth}
        \centering
        \includegraphics[width=\textwidth]{fig6_1.pdf}
        \caption{Performance comparison on AAPL stock}
    \end{subfigure}
    \begin{subfigure}{0.31\textwidth}
        \centering
        \includegraphics[width=\textwidth]{fig6_2.pdf}
        \caption{Performance comparison on Bitcoin}
    \end{subfigure}
    \begin{subfigure}{0.31\textwidth}
        \centering
        \includegraphics[width=\textwidth]{fig6_3.pdf}
        \caption{Performance comparison on YUAN}
    \end{subfigure}
    \vspace{-0.3cm}
    \caption{Comparative analysis of cumulative returns for all models in single-asset trading scenarios. These curves show performance differences for AAPL, Bitcoin, and YUAN over a three-year period.}
    \label{fig:single_asset_performance}
    \vspace{-0.5cm}
\end{figure*}

\subsection{Ablation Study}
% In this subsection, we perform two types of ablation studies, namely 1) ablation of three conferences, which coordinate all agents; and 2) ablation of the LLM backbone, which is the brain of our framework.

\subsubsection{Effectiveness of Each Conference}
To comprehensively evaluate the utility of Experience Sharing Conference (ESC), Budget Allocation Conference (BAC), and Extreme Market Conference (EMC) in our proposed multi-agent architecture, we conducted a systematic ablation study. 

The experimental results are shown in Table \ref{tab:rq2}, We have the following observations:
1) The ESC module is crucial for portfolio diversification and hedging tail risks associated with individual assets, which is beneficial for enhancing returns and risk resistance capabilities. When this module is removed, the ENT metric decreases by 16.63\% to 2.61, showing the largest drop; although the annualized return and Sharpe ratio decrease, they remain at a moderate level. 2) The BAC module has the greatest impact on returns. After removing this module, the annualized return drops by 37.75\% to 44.57\%, the lowest among the three single-module experiments; however, when only this module is retained, the annualized return reaches the highest value of 45.58\% for a single module. 3) The EMC module is vital for risk aversion and return robustness. After removing this module, the maximum drawdown rate increases by 71.93\% to 24.44\%, and the Sharpe ratio decreases by 19.86\% to 1.93; however, when only this module is retained, both the maximum drawdown rate and Sharpe ratio achieve the best levels among single modules. 4) The synergistic effect of the three modules significantly impacts the overall performance. Taking the Sharpe ratio as an example, compared to the removal of the other modules alone, the Sharpe ratio of the complete architecture improves by 41.29\%, 60.65\%, and 19.72\%, respectively, highlighting the synergistic effect of each module and ensuring a balance between risk and return. 
%This demonstrates the superior performance endowed to HedgeAgents by our effective multi-agent architecture.
This demonstrates the superior performance of our effective multi-agent architecture.


\subsubsection{Effectiveness of LLM Backbone}
To evaluate the performance of different LLMs as the backbone in our HedgeAgents, we selected 6 representative models, including ChatGLM-6B\cite{zeng2022glm}, Baichuan-13B\cite{baichuan_inc_2023}, Qwen-72B\cite{qwen_72b_2023}, Gemini 1.5 Pro\cite{geminiteam2024gemini}, gpt-3.5-turbo-16k\cite{openai_gpt_3_5_turbo}, and gpt-4-1106-preview\cite{openai2024gpt4}. We incorporated each of these models into the HedgeAgents framework and tested their investment performance. %The results are summarized in Figure~\ref{fig:rq3}. % For detailed data, please refer to Appendix \ref{app:details}.

\textbf{1) Robustness Analysis:} As shown in Figure~\ref{fig:rq3}, despite variations in individual metrics across backbones, the curves shapes are strikingly similar, indicating that our framework is not entirely contingent upon the capabilities of a certain language model.

\textbf{2) Impact of Parameter Size:} Experimental results demonstrate that an increase in parameters contributes to enhancing the investment returns. The gpt-4-1106-preview significantly outperforms others in terms of ARR and TR metrics. Notably, models with more parameters tend to adopt more aggressive investment strategies, which can lead to higher risk. Conversely, ChatGLM-6B achieves the best scores in MDD and SR, showcasing strong risk control capabilities. 
%despite having only 6B parameters,
%The diversity of the models does not significantly vary with changes in parameter size.

 \textbf{3) Comparison between Open-source and Closed-source:} Closed-source models exhibit excellent performance across all metrics, likely due to their proprietary optimization algorithms and datasets, which provide them with a competitive edge in decision-making capabilities. In the realm of open-source models, Qwen-72B demonstrates performance comparable to the closed-source model.
 %which suggests that open-source models can compete with their closed-source counterparts.

Therefore, we have selected GPT-4 as the core of our framework. Notably, our system has accumulated a total cost of \$15 over the three years, averaging only 2 cents per day!

%After a comprehensive consideration of all metrics, the closed-source commercial model gpt-4-1106-preview was chosen as the backbone algorithm for the HedgeAgents framework due to its balanced approach to investment returns and risk control.


%{\color{red}In terms of returns, the gpt-4-1106-preview model outperformed others with an Annualized Return Rate (ARR) of 71.60\% and a Total Return (TR) of 405.34\%. In contrast, the Chat-GLM-6B model's ARR and TR were only 49.14\% and 231.73\%, respectively, indicating that the parameter size significantly influences the decision-making and investment returns of language models;
%For risk control, the Chat-GLM-6B and gpt-4-1106-preview models demonstrated clear advantages. The Chat-GLM-6B model had the best Maximum Drawdown (MDD) of 9.19\% and Sharpe Ratio (SR) of 2.88, likely due to more conservative decisions from smaller parameter models, which, despite lower returns, offered stronger risk control;
%In terms of risk-adjusted return indicators, the Chat-GLM-6B and Gemini 1.5 Pro models both excelled, showing a concave relationship between performance and parameter size;
%Regarding portfolio diversification, there was no evident linear relationship between the ENT and ENB metrics and parameter size. The medium-parameter Baichuan-13B model surpassed the larger-parameter Qwen-72B model in both ENT and ENB metrics, suggesting that increasing parameter size does not necessarily enhance diversification levels. Considering all metrics, we selected the gpt-4-1106-preview as the optimal backbone model.

%It is noteworthy that, as shown in Figure~\ref{fig:XXXXX}, despite variations in individual metrics across models, the overall shapes of them are strikingly similar. This indicates the effectiveness of our HedgeAgents framework in managing the returns and risks of the investment portfolio, rather than being entirely reliant on the interpretive capabilities of the language models.}

\begin{figure*}[t]
  \centering
\includegraphics[width=0.96\textwidth,keepaspectratio]{fig7.pdf}
\vspace{-0.4cm}
  \caption{Workflow of a single agent, taking Bitcoin Analyst Dave as an example. Blue represents the analysis content, green indicates the query process, and red shows the specific decisions and outcomes.}
  \label{fig:sigleagentworkflow}
  %\vspace{-0.4cm}
\end{figure*}

\begin{figure*}[htbp]
  \centering
   \includegraphics[width=0.95\textwidth,keepaspectratio]{fig8.pdf}
  %\vspace{-0.4cm}
    \caption{Otto's hedging decisions in extreme markets. Green represents market decline, blue indicates the current portfolio in extreme conditions, and red shows the final execution outcome.}
  \label{fig:ottohedge}
  %\vspace{-0.4cm}
\end{figure*}

\subsection{Single-Asset Performance without Hedging}
In the previous setup, we examined performance in a multi-asset hedging scenario. In this subsection, we will divide our dataset into three categories: Bitcoin-only, AAPL-only (stocks), and YUAN-only (forex). This will allow us to assess the capabilities of all models under single-asset conditions without any hedging strategy.

% To evaluate the robustness of HedgeAgents in non-hedged, single-asset scenarios, we conducted experiments on Bitcoin-only (cryptocurrencies), AAPL-only (stocks), and YUAN-only (forex). This assessment isolates the model's performance on individual assets without diversification benefits. 
%For fair comparison, we modified the HedgeAgents framework: each agent started with equal capital and was restricted to trading a single asset. Otto's role was adjusted to a cross-asset expert providing portfolio suggestions. The Experience Sharing Conference remained unchanged, while Otto offered cross-asset insights in the Budget Allocation and Extreme Market Conferences. This adaptation allowed direct comparison with baseline models while preserving HedgeAgents' core collaborative aspects.


%To evaluate the robustness of HedgeAgents in non-hedged, single-asset scenarios, we conducted experiments on Bitcoin (cryptocurrencies),  AAPL (stocks)and YUAN (forex). This assessment aims to isolate the model's performance on individual assets without the benefits of diversification. To ensure a fair comparison, we modified the HedgeAgents framework for these experiments. Each agent was initialized with equal capital and restricted to trading a single asset, Otto's role was adjusted to a cross-asset expert providing portfolio suggestions. The Experience Sharing Conference mechanism remained unchanged. In the Budget Allocation Conference and Extreme Market Conference, Otto's role shifted from budget allocation to offering cross-asset insights. This adaptation allowed for a direct comparison with baseline models while preserving the core collaborative aspects of HedgeAgents.

%Figure \ref{fig:single_asset_performance} illustrates the cumulative returns of HedgeAgents and baseline models over a three-year period for each asset.
Figure \ref{fig:single_asset_performance} illustrates the cumulative returns of all models over a three-year period for each single asset. 1) The Bitcoin-only setting showcased HedgeAgents' exceptional management of high-volatility assets, achieving a cumulative return of about 210\%, far exceeding all baseline models. The Bitcoin expert, Dave, excelled in interpreting blockchain data, anticipating regulatory changes, and analyzing global adoption trends, which facilitated stable growth and resilience during market downturns. 2) In the stock market, represented by AAPL, HedgeAgents attained a cumulative return of approximately 160\%, significantly outperforming competitors like FinAgent and FinMem, which recorded returns of about 135\% and 120\%, respectively. The stock expert, Bob, adeptly navigated both bullish and bearish periods, maintaining consistent growth through his skills in analyzing financial statements and predicting market movements. 3) For the forex market, represented by YUAN, HedgeAgents achieved a modest but positive return of around 9\%, surpassing several baselines that showed negative returns. The forex expert, Emily, demonstrated a deep understanding of macroeconomic indicators, geopolitical events, and central bank policies, enabling the model to capitalize on small price movements and achieve profitability.
%These results underscore HedgeAgents' versatility and effectiveness across diverse asset classes and market conditions.

%1) For cryptocurrencies, the Bitcoin experiment highlighted HedgeAgents' exceptional capability in managing high-volatility assets. HedgeAgents achieved a cumulative return of about 210\%, substantially surpassing all baselines. The cryptocurrency expert Dave in HedgeAgents exhibited remarkable skill in interpreting blockchain data, anticipating regulatory changes, and analyzing global adoption trends. This expertise allowed for more stable growth, especially during market downturns, where HedgeAgents showed resilience compared to the more volatile performance of baselines.

%2) In the stock market, represented by AAPL, HedgeAgents demonstrated superior performance, achieving a cumulative return of approximately 160\%. This significantly outperformed the closest competitors, FinAgent and FinMem, which reached about 135\% and 120\% returns respectively. The stock market expert Bob within HedgeAgents showed particular prowess in navigating both bullish and bearish periods, maintaining consistent growth even during market corrections. Bob's ability to analyze complex financial statements, interpret market sentiment, and predict earnings surprises contributed significantly to the overall performance.

%3) In the forex market, as demonstrated by the YUAN experiment, HedgeAgents navigated the complex and often subtle movements of currency exchange rates effectively. While overall returns were more modest compared to other asset classes, HedgeAgents maintained a positive return of about 9\% over the period, outperforming several baselines that experienced negative returns. The forex expert Emily within HedgeAgents demonstrated a nuanced understanding of macroeconomic indicators, geopolitical events, and central bank policies, enabling the model to capitalize on small price movements and maintain profitability in this challenging market.
These results highlight the versatility and robustness of HedgeAgents across various asset classes and market conditions. 

%Finally, these results underscore the versatility and robustness of HedgeAgents across diverse asset classes and market conditions. The consistent outperformance can be attributed not only to the sophisticated analytical capabilities of the LLM-based agents and effective coordination mechanisms but also to the specialized expertise of individual asset experts within the HedgeAgents framework. This performance in single-asset scenarios provides strong evidence of the model's potential effectiveness in more complex, multi-asset environments, where the synergy between different expert agents can be fully leveraged.

\subsection{Visualization}
To examine our framework's cognitive processes during execution, we have employed visualizations to elucidate the workflow of a single agent and how Otto achieves hedging strategies.

%both single-agent and multi-agent collaborative processes. Surprisingly, our model demonstrates the capacity to valuable experiences akin to human experts.
\subsubsection{Visualization of single agent} \label{sec:vsa}
Figure \ref{fig:sigleagentworkflow} illustrates the decision-making process of a single agent, exemplified by Bitcoin Analyst Dave. Dave's workflow begins with a market analysis noting significant price increases and a robust growth trajectory for Bitcoin. This prompts him to query about historical performance under similar optimistic conditions influenced by macroeconomic policies. Through memory retrieval, Dave recalls past instances where optimistic expectations led to full investment in Bitcoin, but also highlights the challenges of swift position liquidation during subsequent market changes. This historical context informs Dave's decision-making phase, where he considers the positive influence of current macroeconomic policies. Balancing optimism with caution, Dave decides to purchase 50\% of the current position in Bitcoin, complemented by continuous market monitoring. This measured approach reflects Dave's synthesis of historical lessons and present market dynamics. The reflection phase reveals a 7.43\% increase in asset value over four trading days, validating the effectiveness of Dave's strategy. Consequently, he advocates for maintaining a prudent investment posture and implementing dynamic asset management strategies. Dave's workflow exemplifies an iterative learning process that combines historical insights with real-time analysis, enhancing decision quality in the volatile market.

%Figure \ref{fig:sigleagentworkflow} illustrates a single-agent decision-making workflow applied to Bitcoin investment. This case study examines Dave's approach to navigating Bitcoin market fluctuations, highlighting an innovative integration of real-time analysis with historical data.A key feature of this process is its ability to contextualize current decisions within the framework of past performance and prevailing market conditions. This comprehensive approach potentially enhances Dave's adaptive learning capabilities, enabling continuous refinement of investment strategies based on accumulated experiences. The iterative nature of the workflow suggests a mechanism for ongoing improvement in decision quality, which is particularly valuable in the volatile cryptocurrency market. By systematically incorporating lessons from past actions and their outcomes, this approach offers a robust framework for decision-making in dynamic financial environments.
%This section delves into a specific case illustrating how a single agent navigates investment decisions within a dynamic and volatile financial landscape. As depicted in Figure \ref {fig:sigleagentworkflow}, this case study centers on Bitcoin analyst Dave and illustrates how Dave integrates market information to formulate queries, retrieve relevant memories and experiences, make decisions, and summarize reflections.
%As shown in Figure~\ref{fig:sigleagentworkflow}, the case study centers on the decision-making trajectory of Bitcoin analyst Dave, spanning from September 2021 and encompassing the entirety of his investment decision-making process.
%Figure~\ref{fig:sigleagentworkflow} clearly illustrates how a single agent within the HedgeAgents system makes investment decisions in a complex and volatile financial environment. The case study focuses on a typical decision-making trajectory by Bitcoin Analyst Dave from September 2021, encompassing the entire investment decision-making process. 
%A comprehensive analysis of the case and detailed decision-making procedures can be found in Appendix~\ref{app:hedgeagentworkflow}.
%\subsubsection{Visualization of multi-agent}
%Figure~\ref{fig:multiagentworkflow} offers a visualization of the multi-agent collaboration within our HedgeAgents system. 1) During the budget allocation conference on April 30, 2023, Dave presented on Bitcoin report, Bob on Stocks, and Emily on Forex, after which Manager Otto made the budget decision of 0.5:0.3:0.2. 2) During the experience sharing conference on the same day, only three hedging experts participated, with each person attentively listening and summarizing the experiences of others. 3) On April 25, 2021, Dave experienced significant losses with Bitcoin, prompting Otto to convene an extreme market conference, which ultimately helped mitigate the losses.

%Throughout these processes, each agent continuously accumulates its own experiences, fostering a synergistic relationship in ``hedging'' among all agents, which is crucial to the success of our framework\footnote{More cases are provided in Appendix~\ref{app:hedgeagentworkflow}.}.

%Figure~\ref{fig:multiagentworkflow} offers a visualization of the multi-agent collaboration within our HedgeAgents system, capturing the essence of the cooperative processes across three critical meetings. The depicted case studies, including the Budget Allocation and Experience Sharing Conferences on April 30, 2023, and the Extreme Market Conference on April 25, 2021, following significant losses in Dave's Bitcoin management, exemplify the decision-making trajectories within our framework.

%Through this visualization process, the high interpretability of our multi-agent framework is revealed, as the strategic interactions and decision-making pathways among agents are made transparent. The efficacy of the collaborative efforts is palpable, as evidenced by the system's adept response to market volatility and the intelligent aggregation of individual insights into collective strategies.
%An extensive analysis of these cases, detailing the intricate collaborative processes and decision-making protocols among agents, is provided in Appendix~\ref{app:hedgeagentworkflow}.
\subsubsection{Visualization of hedging expert}
Figure \ref{fig:ottohedge} illustrates the decision-making process of Otto, our hedging expert agent, during a significant Bitcoin market decline of 10.22\%. In this scenario, Otto first assesses the market condition, noting the current portfolio composition of 13.2\% cash and 56.8\% Bitcoin. Otto's risk assessment identifies three key factors: the recent Bitcoin price surge and subsequent correction, the impact of Federal Reserve policies on market liquidity, and Bitcoin's inherent volatility. Based on this analysis, Otto concludes that a cautious and balanced approach is necessary to protect the portfolio. The situation analysis reveals Otto's concern about the high volatility of the Bitcoin market and the current losses, leading to the decision to avoid excessive trading and risky bottom-fishing strategies. In response to these conditions, Otto proposes a comprehensive hedging strategy. This strategy combines portfolio diversification, options trading, and risk management techniques to create a multi-layered hedge against market volatility. By employing this balanced approach, Otto aims to mitigate downside risk while preserving upside potential in adverse market conditions. The effectiveness of Otto's approach is evident in the final operation result, where total assets increased by 8.6\% over two days, despite the initial market turbulence. 
%This outcome demonstrates how expert hedging expert can transform market volatility into an opportunity for effective risk management and potential profit.


\section{Conclusions}
This paper presents HedgeAgents, an innovative multi-agent system designed to bolster the robustness of financial actions via hedging strategies. Within this crafted framework, we have developed a suite of specialized hedging agents and organized regular conferences to foster collaboration among multiple agents. Through extensive experiments, our framework has consistently demonstrated superior and robust performance. Furthermore, our observations via visualization indicate that our HedgeAgents is capable of generating investment experience in their memories on par with those achieved by human experts. 

% \section{Limitations}
% We still have the following limitations: 1) In terms of generalization, our HedgeAgents is modular and flexible,  designed to adapt to various market scenarios. Although our experiments focus on stocks, forex, and Bitcoin, this framework can be easily extended to other industries (e.g., technology, energy) and regions (e.g., US, EU, Japan) by modifying the asset expert modules without major changes to the core system. We will explore generalization further in future work. 2) Furthermore, in fact, our framework has not yet been used in real-world market. Therefore, we plan to deploy this HedgeAgents in actual transactions, and we are open to disclosing our returns at upcoming time intervals.

\section{Acknowledgments}
This work is supported in part by the National Natural Science Foundation of China (62372187), in part by the National Key Research and Development Program of China (2022YFC3601005) and in part by the Guangdong Provincial Key Laboratory of Human Digital Twin (2022B1212010004).
The authors are highly grateful to the anonymous reviewers for their careful reading and insightful comments.


\clearpage
\newpage
\bibliographystyle{ACM-Reference-Format}
\balance
\bibliography{hedge_www}

\appendix

\begin{table*}[]
\caption{Performance comparison of different LLM as the backbone for HedgeAgents on 9 evaluation metrics. }
\label{tab:llm}
\small
\centering
\normalsize
\setlength{\tabcolsep}{6pt} % 调整列间距
\renewcommand{\arraystretch}{0.99} % 调整行间距
\resizebox{1\textwidth}{!}{ 
\begin{tabular}{cccccccccc}
\hline
\hline
LLM & ARR(\%) & TR(\%) & SR   & CR    & SoR   & MDD(\%) & Vol(\%) & ENT  & ENB           \\ \hline

Chat-GLM-6B        & 49.14         & 231.73          & \textbf{2.88} & \textbf{12.26} & \textbf{66.64} & \textbf{9.19} & \textbf{0.78} & 2.46          & 1.25          \\
Baichuan-13B       & 53.24         & 259.85          & 2.39          & 8.85           & 55.59          & 13.81         & 1.02          & 2.99          & 1.46          \\
Qwen-72B           & 61.34         & 319.99          & 2.16          & 8.01           & 50.21          & 17.44         & 1.29          & 2.35          & 1.18          \\
Gemini1.5 Pro      & 68.61         & 379.37          & 2.49          & 11.48          & 59.94          & 13.12         & 1.21          & \textbf{3.31} & \textbf{1.57} \\
gpt-3.5-turbo-16k  & 65.44         & 352.81          & 2.27          & 8.44           & 53.49          & 17.35         & 1.29          & 2.93          & 1.49          \\
gpt-4-1106-preview & \textbf{71.6} & \textbf{405.34} & 2.41          & 11.02          & 58             & 14.21         & 1.3           & 3.13          & 1.53          \\
\hline
\hline
\end{tabular}
}
\end{table*}

\section{Experiment of Ablation Study} \label{app:ablation}
\subsection{Effectiveness of Each Conference}
Cumulative returns of ablation analysis on three conference, as shown in Figure \ref{fig:CRaba} .
\begin{figure}[htbp]
  \centering
   \includegraphics[width=0.49\textwidth,keepaspectratio]{fig9.pdf}
  \caption{Cumulative Returns of Ablation Analysis on Three Conference}
  \label{fig:CRaba}
\end{figure}
\subsection{Effectiveness of LLM Backbone}
For using different LLM as the backbone for HedgeAgents, their experimental results are presented in Table \ref{tab:llm}, and the cumulative returns chart is in Figure \ref{fig:CRabb}.


\begin{figure}[htbp]
  \centering
   \includegraphics[width=0.48\textwidth,keepaspectratio]{fig10.pdf}
  \caption{Cumulative Returns of HedgeAgents Based on Different LLM}
  \label{fig:CRabb}
\end{figure}
\FloatBarrier


\section{Experiment of Memory Retrieval}
\label{app:memroy}
To quantify the impact of the Memory Retrieval (MR) module on the overall performance of HedgeAgents, we have specifically designed comparative experiments with and without this module.
\begin{figure}[htbp]
  \centering
   \includegraphics[width=0.43\textwidth,keepaspectratio]{fig11.pdf}
  \caption{(a) Performance of HedgeAgents with/without MR. (b) Visualization of MR embedding by t-SNE on Dow Jones analyst Bob.}
  \label{fig:Xa}
\end{figure}

As depicted in Figure~\ref{fig:Xa} (a), the Annualized Return Rate (ARR) and Sharpe Ratio (SR) have both significantly increased upon the activation of the MR. Compared to the ARR and SR without the module the enhancements reached 57.71\% and 39.3\%, directly confirming the positive role of the MR module in promoting returns and hedging risks.
To delve deeper into the operational mechanics of the MR module, let's consider the case of DJ30 analyst Bob. We'll illustrate the distribution of various types of memory embeddings when employing this module, as depicted in Figure~\ref{fig:Xa} (b). By employing t-SNE for dimensionality reduction and visualization, we have observed that embeddings derived by LLM from hierarchical memory contents, including Market Information Memory, General Investment Experience Pool, and Intelligent Investment Reflection Memory, showcase a discernible clustering distribution in semantic features. This indicates a certain level of distinctiveness between different categories. Such findings substantiate the efficacy of our Hierarchical Memory Retrieval approach in effectively capturing the semantic attributes of memory content, thereby furnishing the agent with valuable contextual memory support.
\section{Experiment of Temporal Isolation}
To minimize the risk of inadvertent information leakage, we designed context-rich prompts incorporating real-time conditions, market variables, and asset-specific data.
Figure \ref{fig:hedgeagents_performance} shows the performance of HedgeAgents during Q1-Q3 of 2024, demonstrating its ability to adapt to rapidly changing market with high predictive accuracy and robustness. The model achieved a Total Return of 68.44\%, a Sharpe Ratio of 2.1. These results validate the model’s robustness and confirm the effectiveness of our information leakage mitigation strategies.
\begin{figure}[H]
    \centering
    \includegraphics[width=0.45\textwidth]{fig12.pdf}
    \caption{Cumulative Returns of HedgeAgents. Red markers indicate dates of EMC meetings, which occurred 36 times.}
    \label{fig:hedgeagents_performance}
\end{figure}

\end{document}
