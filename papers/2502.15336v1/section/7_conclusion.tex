\section{Conclusions}
\label{sec:CON}
In conclusion, EMLMs represent a cutting-edge frontier in AI research, combining language, vision, perception, and action to tackle complex, real-world problems. This review has explored the development of large models in language, vision, and multimodal domains, with a specific focus on how embodied tasks, including perception, navigation, interaction, and simulation, are advancing the field.

The progress made thus far demonstrates the transformative potential of embodied multimodal models across diverse applications, from autonomous systems to robotics. However, significant challenges remain in terms of cross-modal alignment, computational efficiency, generalization, and data acquisition. Furthermore, the ethical implications of deploying such technologies must be carefully considered.

Looking ahead, the field holds immense promise. Advancements in cross-modal pre-training, self-supervised learning, and reinforcement learning will likely drive the next generation of more capable, adaptable, and efficient models. In particular, the integration of these models in real-world, dynamic environments promises to revolutionize fields such as autonomous robotics, virtual agents, and interactive systems.

As the field matures, addressing the technical and ethical challenges will be essential for the responsible and impactful deployment of EMLMs. Continued research and collaboration across disciplines will play a pivotal role in shaping the future of this exciting area of AI.
