%File: formatting-instructions-latex-2025.tex
%release 2025.0
\documentclass[letterpaper]{article} % DO NOT CHANGE THIS
\usepackage{aaai25}  % DO NOT CHANGE THIS
\usepackage{times}  % DO NOT CHANGE THIS
\usepackage{helvet}  % DO NOT CHANGE THIS
\usepackage{courier}  % DO NOT CHANGE THIS
\usepackage[hyphens]{url}  % DO NOT CHANGE THIS
% \usepackage{tabu}
\usepackage{subcaption}
\usepackage{float}
\usepackage{mathrsfs}
\usepackage{tabularx}
\usepackage{xcolor}
% \usepackage{titlesec}

\usepackage{graphicx} % DO NOT CHANGE THIS
\urlstyle{rm} % DO NOT CHANGE THIS
\def\UrlFont{\rm}  % DO NOT CHANGE THIS
\usepackage{natbib}  % DO NOT CHANGE THIS AND DO NOT ADD ANY OPTIONS TO IT
\usepackage{caption} % DO NOT CHANGE THIS AND DO NOT ADD ANY OPTIONS TO IT
\frenchspacing  % DO NOT CHANGE THIS
\setlength{\pdfpagewidth}{8.5in}  % DO NOT CHANGE THIS
\setlength{\pdfpageheight}{11in}  % DO NOT CHANGE THIS
\usepackage{booktabs}
\usepackage{algorithm}
\usepackage{algorithmic}
\usepackage{newfloat}
\usepackage{listings}
\usepackage{pifont}
\usepackage{amsmath,amssymb}
\usepackage{xcolor}
\usepackage{pifont}

\newcommand{\tick}{\textcolor{red}{\ding{51}}} % Red tick
\newcommand{\cross}{\ding{55}} % Regular cross

\DeclareCaptionStyle{ruled}{labelfont=normalfont,labelsep=colon,strut=off} % DO NOT CHANGE THIS
\lstset{%
	basicstyle={\footnotesize\ttfamily},% footnotesize acceptable for monospace
	numbers=left,numberstyle=\footnotesize,xleftmargin=2em,% show line numbers, remove this entire line if you don't want the numbers.
	aboveskip=0pt,belowskip=0pt,%
	showstringspaces=false,tabsize=2,breaklines=true}
\floatstyle{ruled}
\newfloat{listing}{tb}{lst}{}
\floatname{listing}{Listing}
\pdfinfo{
/TemplateVersion (2025.1)
}


\setcounter{secnumdepth}{0} 

\title{Spatial Distribution-Shift Aware Knowledge-Guided Machine Learning\vspace{-1.25em}}
\author {
    % Authors
    Arun Sharma\textsuperscript{\rm 1},
    Majid Farhadloo\textsuperscript{\rm 1},
    Mingzhou Yang\textsuperscript{\rm 1},
    Ruolei Zeng\textsuperscript{\rm 1},
    Subhankar Ghosh\textsuperscript{\rm 1},
    Shashi Shekhar\textsuperscript{\rm 1}
}
\affiliations {
    % Affiliations
    \textsuperscript{\rm 1}Department of Computer Science, University of Minnesota, Twin Cities, USA\\

    \{sharm485, farha043, yang7492, zeng0208, ghosh117, shekhar\}@umn.edu
}
\usepackage{bibentry}
\begin{document}

\maketitle
\vspace{-5em}
\begin{abstract}
Given inputs of diverse soil characteristics, and climate data gathered from various regions, we aimed to build a model to predict accurate land emissions. The problem is important since accurate quantification of the carbon cycle in agroecosystems is crucial for mitigating climate change and ensuring sustainable food production. Predicting accurate land emissions is challenging due to since calibrating heterogeneous nature of soil properties, moisture, and environmental conditions is hard at decision-relevant scales. Traditional approaches do not adequately estimate land emissions due to location-independent parameters failing to leverage the spatial heterogeneity and also require large datasets. To overcome these limitations, we proposed Spatial Distribution-Shift Aware Knowledge-Guided Machine Learning (SDSA-KGML) which leverage location-dependent parameters which accounts significant spatial heterogeneity in soil moisture from multiple sites within the same region. Experimental results demonstrate that SDSA-KGML models achieve higher local accuracy for the specified states in the Midwest Region.
\end{abstract}
\vspace{-1.5em}
\section{Introduction}
Given inputs of diverse types of information, including climate data, soil characteristics, and Gross Primary Productivity (GPP) gathered from various regions, we aimed to build a model to predict land emissions. These emissions include the amount of carbon dioxide released by vegetation through Autotrophic Respiration (Ra) and the carbon dioxide released by soil microorganisms \cite{janssens2007spatial} during the decomposition of organic matter, termed Heterotrophic Respiration (Rh), for various regions \cite{liu2024knowledge}. Figure \ref{fig1} shows estimated land emissions generated by a knowledge-guided machine learning model using climate data and soil characteristics \cite{liu2024knowledge}.

\begin{figure}[h]
\centering
\begin{subfigure}[b]{0.40\linewidth} % Adjust width to fit both figures side by side
    \centering
    \includegraphics[width=\linewidth]{Figures/Input.png}
    \caption{Input}
    \label{fig:subfig1}
\end{subfigure}
\hfill % Adds horizontal space between the subfigures
\begin{subfigure}[b]{0.59\linewidth} % Adjust width to fit within the row
    \centering
    \includegraphics[width=\linewidth]{Figures/Output.png}
    \caption{Output}
    % \caption{Jerk values with 6 m/s\(^2\) bound}
    \label{fig:subfig2}
\end{subfigure}
\caption{Problem Statement \cite{liu2024knowledge}}
\label{fig1}
\vspace{-2em} % Adjust vertical space below the figure
\end{figure}

Land emissions estimations, such as carbon dioxide, methane, and nitrogen oxides, are important for sustainable agriculture. They also support climate change mitigation, optimized crop management, and informed decision-making to sustain food production. Given these initiatives, it is essential to establish reliable and scalable methods for accurately quantifying carbon sequestration at the field level. This will help evaluate its impact on climate mitigation and ensure that farmers' mitigation efforts are compensated fairly and precisely. The problem is challenging since calibrating the heterogeneous nature of soil properties, moisture, and environmental conditions results in lower accuracy. In addition, such models requires large dataset \cite{2022a,2022b}.

\textbf{Limitations of Related Work:} Knowledge Guided Machine Learning (KGML) \cite{karpatne2017theory,karpatne2022knowledge} has shown promise in modeling earth systems governed by established equations \cite{trivedi2020knowledge}. For example, KGML-ag \cite{liu2022kgml} combines process-based models to improve agroecosystem dynamics predictions, including emission estimates, while \citet{liu2024knowledge} refines carbon cycle estimations by integrating process-based models. However grounded in principles like mass and energy conservation are commonly used for predicting land emissions but struggle in regions with high spatial heterogeneity \cite{gupta2021spatial} and require large datasets \cite{2022c}. In this work, we propose a region-based knowledge-guided machine learning framework to extract more precise and meaningful land emission estimates for precision agriculture and agroecosystems.

\textbf{Contributions:} This paper makes three contributions:
\begin{itemize}
    \item We introduced a taxonomy based for spatial variability to study the impact of location-based model parameters.
    \item We briefly introduce the Spatial Distribution-Shift Aware Knowledge-Guided Machine Learning (SDSA-KGML) which consider location-dependent parameters.
    \item We validated SDSA-KGML achieving higher local accuracy across Illinois, Iowa, and Indianna.
\end{itemize}

% First, we introduced a taxonomy based for spatial variability to study the impact of location-based model parameters. 

% Second, we briefly introduce the Spatial Distribution-Shift Aware Knowledge-Guided Machine Learning (SDSA-KGML) which consider location-dependent parameters. 

% Third, we validated SDSA-KGML achieving higher local accuracy across Illinois, Iowa, and Indianna.

\begin{table*}[t]
\small
\centering
\caption{Spatial Variability Awareness Levels based on Location Dependent (\tick) and Location Independent (\cross)}
\label{tab:table1}
\begin{tabular}{ccccccc}
\toprule
Level & Taxonomy & Example & Inputs ($x$) & Outputs ($y$) & Parameters ($\theta$)\\
\midrule
1 & One Size Fit All & Data-Driven Models & \cross & \cross & \cross\\
2 & Spatial Explicit & KGML-Ag \cite{liu2024knowledge}) & \tick & \tick & \cross \\
3 & Spatial Variability-Aware & \textcolor{red}{\textbf{Proposed Approach}} & \tick & \tick & \tick\\
\bottomrule
\end{tabular}
\end{table*}

\vspace{-1em}
\section{Proposed Approach}\label{sec2}
\textbf{Taxonomy:} Land emission forecasting has been widely studied using methods such as decision trees \cite{adegun2023state}, random forests \cite{fang2018modeling,ardeshir2014gis}, and neural networks \cite{feng2023survey}. While these data-driven models handle geographic heterogeneity, they often suffer from overfitting, limiting their ability to generalize to unseen data. Table \ref{tab:table1} we present a taxonomy of ML models based on their level of spatial variability awareness. Data-driven models are one-size-fits-all as they do not consider location whereas methods like KGML-ag\cite{liu2024knowledge}) are spatial explicit but are location-independent. By contrast, the proposed approach (SDSA-KGML) incorporates location dependence into the model itself.

% presents a taxonomy of Spatial Variability Awareness models in the literature. Data-driven models disregard geographic location or model parameters, unlike KGML-based models (e.g., process-based methods \cite{liu2024knowledge}). However, KGML-based model parameters are invariant to geographic location, whereas the proposed SDSA-KGML accounts for input, output, and model parameters as location-dependent.
% \textbf{Taxonomy:} Land emission forecasting has been widely studied, and methods such as decision trees \cite{adegun2023state}, random forests \cite{fang2018modeling,ardeshir2014gis}, and neural networks\cite{feng2023survey} can accommodate considerable geographic heterogeneity. However, these data-driven models are susceptible to overfitting, which limits their ability to generalize to new, unseen data \cite{feng2023survey}. Table \ref{tab:table1} introduced a taxonomy of Spatial Variability Awareness models used by current literature where data-driven models does not consider location or model parameters into account in contrast with KGML-based models (e.g., process-based methods \cite{liu2024knowledge}). However, KGML-based model parameters are in-variant with geographic location as compared to proposed SDSA-KGML which consider input, output, and model parameters as location dependent.

\begin{figure}[ht!]
    \centering
    \includegraphics[width=0.85\linewidth]{Figures/Fig1.png} % Ensure this path is correct
    \caption{Illustration of the SDSA-KGML framework.}
    \label{fig:Fig1b}
\end{figure}

\textbf{SDSA-KGML Framework:} The proposed approach leverages KGML-Ag architecture \cite{liu2024knowledge} which is pre-trained using synthetic data generated by the process-based model, enabling it to capture fundamental patterns and relationships. Figure \ref{fig:Fig1b} illustrate SDSA-KGML framework which first processes climate, remote sensing \cite{ma2019remote}, soil, and carbon flux data through an \textbf{Auto Region Detector} to extract geographic sub-area of of a Midwest Region i.e., \textit{Illinois}, \textit{Iowa}, and \textit{Indiana} which is later feeds to \textbf{Predictor Framework} using GRU layers and attention mechanisms to predict land emissions. The model is then fine-tuned with observed low-resolution crop yield data and carbon flux measurements from sparsely distributed eddy-covariance sites aided by knowledge-guided loss functions to  ensure that the target variables respond appropriately to input variables.  we employed a streamlined 5-step training protocol, selected the GRU architecture for the neural network, and incorporated regularization to mitigate overfitting.

\vspace{-.8em}
\section{Preliminary Results}\label{sec2}

\textbf{Dataset:} We integrated diverse datasets into a knowledge-guided learning framework and compared its performance against a one-size-fits all approach. The task was to model agroecosystem dynamics and carbon cycles for the Midwest region of the US. We utilized daily climate data from NLDAS-2\footnote{\url{https://ldas.gsfc.nasa.gov/nldas/nldas-2-forcing-data}}, soil characteristics from gSSURGO\footnote{\url{https://www.nrcs.usda.gov/resources/data-and-reports/gridded-soil-survey-geographic-gssurgo-database}}, annual crop yield data from National Agricultural Statistics Service (NASS), high-resolution GPP data from the Soil-Adjusted Near-Infrared Reflectance SANIRv model, and carbon flux measurements from EC flux towers in the U.S. Midwest. These datasets were preprocessed using Z-normalization, interpolation, and aggregation. The SDSA-KGML framework combines these datasets to predict carbon fluxes, crop yields, and soil organic carbon changes with precision, training based on 100 samples from Illinois, Indiana, and Iowa.

\begin{figure}[ht!]
    \centering
    \includegraphics[width=0.7\linewidth]{Figures/Fig2.png} % Ensure this path is correct
    \caption{MSE Loss across  Illinois, Iowa, and Indiana}
    \label{fig:fig1a}
\end{figure}

Our findings show that SDSA-KGML models trained on state-specific data outperform the global model (KGML-Ag), achieving higher R1 and R2 values and lower MSE losses. Figure \ref{fig:fig1a} highlights that models trained on Iowa, Indiana, and Illinois data have lower MSE losses when tested within these states compared to training on combined data. This demonstrates that regional data segmentation enhances geographic information integration, capturing finer nuances and improving prediction accuracy.

\vspace{-1em}
\section{Conclusion and Future Work}\label{sec5}
This study demonstrates SDSA-KGML model employs location-based parameter values to effectively capture spatial heterogeneity. Results demonstrated that models trained on data specific to individual states better accuracy than those using location-independent parameters. 

\textbf{Future Work:} We plan to add novel partitioning techniques and new datasets \cite{2015d} for detailed experiments to better capture spatial variability \cite{ghosh2024towardssigspatial, ghosh2024towardsarxiv, ghosh2024reducingarxivcoloc, ghosh2024towardscosit, ghosh2023reducing, ghosh2022towards, ghosh2017video, yang2025climate}. In addition, we also plan to incorporate Task-Adaptive Meta Learning \cite{liu2023task} for model generalization corresponding to any new tasks.

\section*{Acknowledgments}{This material is based on work supported by the USDA under Grant No. 2023-67021-39829, the National Science Foundation under Grant No. 1901099, and the USDOE Office of Energy Efficiency and Renewable Energy under FOA No. DE-FOA0002044. We also thank Kim Kofolt and the Spatial Computing Research Group for their valuable comments and contributions..}
\bibliography{aaai25}

\end{document}
