At the end of the 50s, Hanbury-Brown and Twiss showed evidence that, when a detector of single photons is placed in the electric field generated by a source of thermal light, the detection times tend to lump together; this is the so-called \emph{bunching effect}.
In what physicists call the \emph{semi-classical} picture, photon detection times are modeled by an inhomogeneous Poisson point process of intensity function given by the electric field, and the assumption of thermal light corresponds to taking this intensity function to be a realization of a Gaussian process with smooth samples. 
In modern parlance, this hierarchical model makes the detection times a Cox point process. 
The smoothness of the samples from the Gaussian process explains why the detection times appear to be lumped together.

While her thesis initially was concerned with finding a stochastic description of this bunching effect, without the \emph{semiclassical} description of the electric field as a function-valued stochastic process, Odile Macchi then turned to model a similar situation where photons are replaced by fermions, like electrons. 
There, she found out that detection events are negatively correlated, evidencing an \emph{anti-bunching} effect that was later experimentally observed.
Macchi formalized DPPs precisely to model that detection process \cite{Mac72}; see also \cite{BFBDHRRSW22Sub} for a modern review. 
Mathematically, DPPs arise from the combined action of Born's rule --the quantum physics axiom that lab measurements are draws from a random variable parametrized by two operators on a Hilbert space-- and the canonical anti-commutation relations satisfied by the building blocks of operators describing fermions.

In this section, we walk in Macchi's footprints and illustrate how DPPs are related to quantum fermionic models in a simple setting, showing in passing that we can design an algorithm to sample a given DPP on a quantum computer.
This section paraphrases \cite{BaFaFe24}, to which we refer the reader for more details and important references, such as \cite{JSKSB18, KePr22}.

\subsection{The formalism of quantum physics}
A minimalistic physical model is $(i)$ a correspondence between a set of physical notions and a set of mathematical objects, as well as $(ii)$ a law to connect these mathematical objects to a measurement.
In quantum physics, the mathematical objects to describe an experiment are a Hilbert space $(\mathbb{H},\braket{\cdot}{\cdot})$ and a collection of self-adjoint operators from $\mathbb{H}$ to $\mathbb{H}$ called \emph{observables}.
A state of the physical system, labeled as some string of characters $\psi$, is described by the projection onto the line $\mathbb{C}\ket{\psi}$ in $\mathbb{H}$. 
Here $\ket{\psi}$ is a vector of $\mathbb{H}$ of unit norm that we associate to the label $\psi$. 
Writing vectors in the form $\ket{\psi}$ (a \emph{ket}) and linear forms as $\bra{\psi}:\ket{\phi} \mapsto \braket{\psi}{\phi}$ (a \emph{bra}) is a useful notational trick known as the \emph{bra-ket} notation.
It is compatible with our denoting the inner product as $\braket{\cdot}{\cdot}$.
In particular, the projection onto $\mathbb{C}\ket{\psi}\subset\mathbb{H}$ simply writes $\ketbra{\psi}$, because
$$
\ketbra{\psi} \ket{\phi} = \braket{\psi}{\phi} \ket{\psi}.
$$

For simplicity, we henceforth restrict to the $2^N$-dimensional example $\mathbb{H} = (\mathbb{C}^2)^{\otimes N}$, for some $N\geq 0$. 
This is the Hilbert space that is used in quantum computing to describe a system of $N$ qubits, i.e. $N$ particles that can each assume two states.
In this finite-dimensional setting, an observable $A$ is simply a self-adjoint matrix, so that it diagonalizes in an orthonormal basis $A = \sum_{i=1}^{M} \lambda_i u_i u_i^*$, where we let $M=2^N$.
Observables are meant to correspond to physical properties that one can measure.
The final ingredient of the model is \emph{Born's rule}, which associates a state-observable pair to a random measurement.
More precisely, Born's rule stipulates that measuring the physical property described by $A$ in the state described by $\ketbra{\psi}$ is equivalent to drawing from the random variable $X$ supported on the spectrum $\Lambda = \{\lambda_1, \dots, \lambda_{M}\} \subset \mathbb{R}$ of $A$, and described by 
\begin{equation}
    \label{e:born}
    \mathbb{E} h(X) = \mathrm{Tr}\left(\ketbra{\psi} \sum_{i=1}^M h(\lambda_i)u_i u_i^*\right ) = \sum_{i=1}^M h(\lambda_i) \vert\braket{\psi}{u_i}\vert^2,
\end{equation}
where $h$ is any real-valued function on $\Lambda$.
Furthermore, independently preparing $\ket{\psi}$ and measuring $A$ several times in a row yields statistically independent draws of the random variable $X$.

\subsection{The canonical anticommutation relations lead to DPPs}

In the rest of this section, we describe a state $\ket{\psi}$ and an observable $A$ on $\mathbb{H} = (\mathbb{C}^2)^{\otimes N}$ that can be efficiently prepared and measured on a quantum computer, and such that $X$ in \eqref{e:born} is a user-defined projection DPP on $\{1, \dots, N\}$.
Generic DPPs on finite ground sets with Hermitian kernels, and more generally Pfaffian point processes, can also be obtained \cite{BaFaFe24}, but we focus here on projection DPPs for simplicity.

Let $c_1, \dots, c_N$ be any operators on $\mathbb{H}$ that satisfy the canonical anticommutation relations
\begin{equation}
        \label{e:CAR}
        \lbrace c_i,c_j \rbrace = \lbrace c^*_i,c^*_j \rbrace  = 0
        \qquad \text{and} \qquad
        \lbrace c_i,c^*_j \rbrace = \delta_{ij} \mathbb{I},
\end{equation}
where $\{u,v\} \triangleq uv+vu$ is the anti-commutator of operators $u$ and $v$.   
Assuming the existence of a set of operators satisfying \eqref{e:CAR} for the moment, one can show \cite{Nie05} the existence of a simultaneous eigenvector to all $c_i^*c_i$, which we denote by $\ket{\emptyset}$, always with eigenvalue zero. Moreover, the vectors 
$$
    \ket{\mathbf{n}} = \prod_{i=1}^N (c_i^*)^{n_i}\ket{\emptyset}, \quad \mathbf{n}\in\{0,1\}^N, 
$$
form a basis of $\mathbb{H}$, such that 
$$
    c_i^* c_i \ket{\mathbf{n}} = n_i\ket{\mathbf{n}}.
$$
This basis is called the \emph{Fock basis} corresponding to $c_1, \dots, c_N$. 

Coming back to the existence of operators satisfying \eqref{e:CAR}, an example is given by the Jordan-Wigner operators $a_1, \dots, a_N$; see e.g. Definition 3.3 in \cite{BaFaFe24}. 
The Jordan-Wigner operators come with additional tractability, in the sense that any product of $a_i$s applied to $\ket{\emptyset}$ is easy to prepare on a quantum computer, and any observable of the form $a_i^*a_i$ is easy to measure in such a state. 

We are ready to define the state-observable pair that will correspond to a given projection DPP. Let $ V = ((v_{ij}))$ be an $N\times N$ unitary matrix, and consider 
$$
    b_k = \sum_{j=1}^N v_{kj}^* a_j, \quad 1\leq k \leq N.
$$
One can show that $b_1, \dots, b_k$ also satisfy \eqref{e:CAR}. 
Now, consider the state $\ketbra{\psi}$, where
$$
    \ket{\psi} = b_1^* \dots b_r^* \ket{\emptyset},
$$
and consider the observable $A = \prod_{i=1}^N a_i^* a_i$.
Since all $a_i^*a_i$ are simultaneously diagonalizable in the Fock basis, they commute. 
In particular, $A$ is self-adjoint, making it a \emph{bona fide} observable. 
Decomposing the trace appearing in Born's rule \eqref{e:born} in the Fock basis, and carefully applying the canonical anti-commutation relations, one obtains a random variable $X$ that is a projection DPP with kernel $K = V_{[r],:}^*V_{[r],:}$.
The determinantal character of the distribution precisely comes from this manipulation of anticommuting operators in Born's rule, which physicists have come to call \emph{Wick's theorem}.
We refer to Appendix A.7 of \cite{BaFaFe24} for a detailed computation, with the same notations as we used here.


\begin{figure}[h]
    \centering
    \includegraphics[width=1\linewidth]{Figures/quantum.jpg}
    \caption{A circuit sampling a DPP with projection kernel of rank $r = 3$, with $N = 5$ items. On the left-hand side, Pauli $X$ gates are used to create fermionic modes on the three first qubits. Note also the parallel QR Givens rotations on
neighbouring qubits indicated by parametrized $XX + YY$ gates. On the right-hand side, measurements of occupation numbers are denoted by black squares (c.f. \cite{BaFaFe24}).}
    \label{fig:quantum}
\end{figure}


In addition to a more extensive introduction to the quantum computing tools used here, the paper \cite{BaFaFe24} also contains a derivation of state-observable pairs corresponding to generic DPPs with Hermitian kernels and Pfaffian point processes, as well as experimental demonstrations on IBM quantum computers.
