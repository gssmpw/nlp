In time-frequency analysis, Short Time Fourier Transforms (abbrv. STFTs) are foundational objects \cite{TFA_Grochenig, TFA_Flandrin, TFA_Cohen}. 
Given a signal $f\in L^2(\R)$ and a window function $\phi \in L^2(\R)$, the STFT $\cF_\phi f(\cdot,\cdot)$ of $f$ with respect to $\phi$ is defined as the inner product between f and shifted $\phi$ (in the sense of time and frequency). More precisely,
\begin{equation}
    \cF_\phi f(u,v) := \int_\R f(x) \overline{\phi (x-u)} e^{-2\pi i x v} \d x = \langle f , M_v T_u \phi \rangle,
\end{equation}
where $T_u g (x) := g(x- u)$ and $M_v g(x) := e^{2\pi i v x} g(x) $ and $\langle f,g \rangle := \int_\R f(x)\overline{g(x)}\d x$.
\begin{remark}
    It is convenient to work with complex notation sometimes. By setting $z= u + iv$, we will regard $\cF_\phi f$ as function of $z\in \C$.
\end{remark}

Popular choices for window functions $\phi$ include Gaussians
\[\phi(x) \propto e^{-x^2/2\sigma^2}, \quad \sigma >0.\]
We will particularly choose  $\phi (x) = 2^{1/4}e^{-\pi x^2}$ as in \cite{BARDENET_Spec,GhoshGAF}, 
to make use of the \emph{Bargmann transform}, which will be introduced later. In this setting, the STFT is also known as the \emph{Gabor transform} \cite{TFA_Grochenig}.

Given the STFT of a signal, the function $u,v\mapsto|\cF_\phi f(u,v)|^2$ is  fundamentally important, called the \emph{spectrogram} of the signal $f$ with respect to the window function $\phi$.
We will be particularly interested in the \emph{level sets} of the spectrogram of $f$
\[\mathcal L (a) := \{(u,v) \in \R^2: |\cF_\phi f(u,v)| \ge a \}, \quad a\ge 0.\]

Spectrogram analysis is highly effective in many applications \cite{TFA_Grochenig, Spectrogram_Flandrin}. Classically, the maxima of the spectrogram have been extensively studied. This is related to the understanding that these capture greater energy from the spectrogram and, therefore, provide more information about the signal  \cite{TFA_Grochenig, TFA_Flandrin, TFA_Cohen, Flandrin_2018}. However, in recent years, a different approach to spectrogram analysis has been investigated, focusing on the zeros of the spectrogram instead of the maxima. This new line of investigation originated from a seminal paper by Flandrin \cite{Spectrogram_Flandrin}. 

In relation to this direction, Bardenet et al. \cite{BARDENET_Spec} observed that the zero set of the spectrogram of the Gaussian white noise has the same statistical distribution as the zero set of the planar Gaussian Analytic Function (GAF). In \cite{BARDENET_GAF}, Bardenet and Hardy further explored the connection between several time-frequency transforms of Gaussian white noises and different models of GAFs (planar, elliptic, hyperbolic). In particular, the STFT with respect to a Gaussian window corresponds to the planar GAF, and the analytic wavelet transform \cite{Daubechies_1988} is related to the hyperbolic GAF. Thus, spectrogram analysis is linked to the theory of Gaussian Analytic Functions, a rich and elegant topic of independent mathematical interest \cite{hough2009zeros, PeresVirag, sodin}. 

\subsection{Zeros of Gaussian Analytic Functions}
To be self-contained and for the sake of the readers, we will briefly review the Gaussian analytic functions and their zeros.
\begin{definition}[Gaussian analytic functions, \cite{hough2009zeros}]
    Let $D \subset \mathbb C$ be a domain. A Gaussian analytic function (GAF) on $D$ is a random variable $f$ taking values in the space of analytic functions on $D$ (endowed with the topology of uniform convergence on compact subsets), such that for every $z_1,\ldots,z_n \in D$ and  $n \ge 1$, the random vector $(f(z_1),\ldots,f(z_n))$ has a mean zero complex Gaussian distribution.
\end{definition}

Given a GAF $f$ on a domain $D$, it is natural to consider the covariance kernel of $f$, defined by
\[ K(z,w) := \mathbb E[f(z) \overline{f(w)}], \quad z,w \in \mathbb C.\]
Due to the Gaussian nature and the continuity of $f$, the covariance kernel $K$ completely determines the distribution of $f$.

A standard way to construct a GAF is as follows.

\begin{proposition}
    Let $D$ be a domain in $\mathbb C$ and $\Psi_n, n \ge 1$ be holomorphic functions on $D$ such that $\sum_{n} |\Psi_n(z)|^2$ converges uniformly on compact subsets of $D$. Let $\xi_n, n \ge 1$ be i.i.d. complex Gaussian random variables. Then, almost surely, the random series
    \[ f(z) := \sum_{n=1}^\infty \xi_n \Psi_n(z)\]
    converges uniformly on compact subsets of $D$, and hence defines a Gaussian analytic function $f$ with covariance kernel
    \[ K(z,w) = \sum_{n\ge 1} \Psi_n(z) \overline{\Psi_n(w)}.\]
\end{proposition}

Our main interest is in the zeros of GAFs. There are three particular models of GAFs, for which the corresponding zero set interacts nicely with the geometry of the ambient domain. In what follows, $\xi_n, n\ge 0$ are i.i.d. $\mathcal N_{\mathbb C}(0,1)$-random variables.

\begin{enumerate}
    \item(Planar GAF) Let $D= \mathbb C$, we define
    \[ f_L(z) := \sum_{n=0}^\infty \xi_n \sqrt{\frac{L^n}{n!}}z^n, \quad L>0.\]
    For each $L>0$, $f_L(z)$ is a GAF on $\mathbb C$ with covariance kernel $\exp(Lz\bar w)$.
    \item(Spherical GAF)
    Let $D= \mathbb C \mathbb P^1 = \mathbb C \cup \{\infty\}$, we define 
    \[f_L(z) := \sum_{n=0}^L \xi_n \sqrt{\frac{L(L-1)\ldots(L-n+1)}{n!}}z^n, \quad L \in \mathbb N=\{1,2,3,\ldots\}.\]
    For each $L\in \mathbb N$, $f_L(z)$ is a GAF on $\mathbb C$ with covariance kernel $(1+z\bar w)^L$, and it has a pole at $\infty$.
    \item(Hyperbolic GAF) Let $D= \mathbb D = \{|z|<1\}$, we define 
    \[f_L(z) := \sum_{n=0}^L \xi_n \sqrt{\frac{L(L+1)\ldots(L+n-1)}{n!}}z^n, \quad L >0.\]
    For each $L>0$, $f_L(z)$ is a GAF on $\mathbb D$ with covariance kernel $(1-z\bar w)^{-L}$. 
\end{enumerate}

\begin{figure}[h]
    \centering
    \includegraphics[width=0.80\linewidth]{Figures/Zeros.jpg}
    \caption{Zeros of 3 canonical models of GAFs (cf. \cite{BARDENET_GAF})}
    \label{fig:zeros}
\end{figure}

These 3 models are characterized by the property that the zero sets are invariant (in distribution) and ergodic under the action of the group of isomorphisms of the ambient domains (see \cite{hough2009zeros}, Chapter 2). 

The behavior of the zeros of GAFs is quite similar to that of determinantal point processes (DPPs), as both exhibit the so-called \emph{hyperuniformity} in the spatial distribution of points, which is of independent interest in statistical physics \cite{Ghosh2016NumberRI, GhoshLebowitz, PhysRevE}. 
In fact, for the hyperbolic GAF with $L=1$, it has been shown by Peres and Virag \cite{PeresVirag} that the zeros of $f_1$ is a DPP on $\mathbb D$ with the Bergman kernel $K(z,w) = \pi^{-1} (1-z\bar w)^{-2}$ and the background measure is the Lebesgue measure on $\mathbb D$. The zeros of GAFs also exhibit repulsiveness, but at short ranges. In particular, the probability of having 2 zeros in a same small disk of radius $\varepsilon$ is of order $\mathcal{O}(\varepsilon^6)$ (see \cite{hough2009zeros}, Chapter 3). In contrast, from \eqref{eq:Poisson_repel} with $d=2$, that probability for Poisson point processes is of order $\varepsilon^4$.
For more details about GAFs, we refer the readers to \cite{hough2009zeros, sodin}.

\subsection{Bargmann transform and Gaussian white noise}
\subsubsection{Bargmann transform and Hermite functions}
\begin{definition}[Bargmann transform]
Given $f\in L^2(\R)$, the Bargmann transform of $f$ is a function on $\C$ defined by
\[ \mathcal Bf(z) = 2^{1/4} \int_\R f(x) \exp \Big ( 2\pi x z - \pi x^2 - \frac{\pi}{2} z^2 \Big ) \d x, \quad z\in \C.\]
\end{definition}

The connection of the STFT (with respect to Gaussian window functions) to the Bargmann transform is given by the following proposition:  

\begin{proposition}
For $f\in L^2(\R)$ and $z= u+iv$, we have 
\[ \cF_\phi f (z) = \exp \Big ( -\pi i u v - \frac{\pi}{2} |z|^2 \Big ) \mathcal B f(\bar z), \]
where $\phi (x) := 2^{1/4}e^{-\pi x^2}$.
\end{proposition}

The Hermite functions $h_k(x), k=0,1,2,\ldots$ are defined as 
\begin{equation} \label{eq:hermite}
 h_k(x) := \frac{(-1)^k}{2^{2k-1/2} \pi^k k!} e^{\pi x^2} \frac{ \d^k}{\d x^k} \Big (e^{-2\pi x^2} \Big ), \quad k \in \N.
\end{equation}
It is known that $\{h_k\}_k$ is an orthonormal basis of $L^2(\R)$ and they behave nicely under Bargmann transform:

\begin{proposition}[\cite{BARDENET_Spec,GhoshGAF}] \label{p:Bargmann}
We have
\begin{enumerate}
\item[(i)] $ \mathcal Bh_k(z) = \frac{\pi^{k/2}}{\sqrt{k!}} z^k , \quad k\in \N$.
\item[(ii)] $\cF_\phi h_k (z) = \exp  (-\pi i uv - \frac{\pi}{2} |z|^2  ) \sqrt{\frac{\pi^k}{k!}} \bar{z}^k$, where $z= u+iv$ and $k\in \N$.
\end{enumerate}
\end{proposition}

\subsubsection{Gaussian white noise}

To define the white noise, we recall the notions of Schwartz space and tempered distributions. Let $C^\infty(\R,\C)$ be the space of smooth functions from $\R$ to $\C$, the Schwartz space $\cS(\R)$ is defined as
\[ \cS(\R) = \Big \{ f \in C^\infty(\R, \C) : \sup_{x\in \R} |x^n f^{(m)}(x)| < \infty, \: \forall m,n \in \N \Big \}.\]
The Schwartz space particularly contains Gaussian density functions.

The space of tempered distributions on $\R$ is the continuous dual of $\cS(\R)$, for which we will denote by $\cS'(\R)$. For $\psi \in \cS'(\R)$ and $\phi\in \cS(\R)$, we denote the action of $\psi$ on $\phi$ by $\langle \psi , \phi \rangle $.

\begin{definition}[White noise]
Let $\mathcal B(\cS'(\R))$ be the Borel $\sigma$-field on $\cS'(\R)$. The white noise $\xi$ is the random element of $(\cS'(\R), \mathcal B(\cS'(\R)))$ satisfying
\[ \E[e^{i\langle \xi, g \rangle } ] = e^{-\frac{1}{2} \|g\|^2_{L^2(\R)}}, \quad \forall g\in \cS(\R). \]
In other words, $\langle \xi ,g \rangle$ is a $\mathcal N(0,\|g\|_{L^2(\R)}^2)$-random variable for all $g\in \cS(\R)$. We will denote by $\mu_1$ the law of $\xi$.
\end{definition}

\subsection{The short-time Fourier transforms of white noise}

We note that the notion of STFT extends naturally for $f\in \cS'(\R)$ and $\phi \in \cS(\R)$ as
\[ \cF_\phi f(u,v ) := \langle f , M_v T_u \phi \rangle , \quad u,v\in \R.\]
Using Bargmann transform, the STFT of the white noise $\xi$ is then given by the following theorem (see \cite{BARDENET_Spec, GhoshGAF}).
\begin{theorem} \label{p:whitenoise}
For $z= u+iv \in \C$ and $\phi (x) = 2^{1/4}e^{-\pi x^2}$, we have
\begin{equation} \label{eq:stft_whitenoise}
\cF_\phi \xi (z) = \sqrt{\pi} \exp\Big (i\pi uv - \frac{\pi}{2}|z|^2 \Big ) \sum_{k=0}^\infty \langle \xi, h_k \rangle \sqrt{\frac{\pi^k}{k!}} z^k,
\end{equation}
where the convergence in the RHS is in $L^2(\mu_1)$. In particular, the random series
\[ \sum_{k=0}^\infty \langle \xi, h_k \rangle  \sqrt{\frac{\pi^k}{k!}} z^k\]
defines a random entire function $\mu_1$-a.s.
\end{theorem}

Thus, the STFT of white noise is a Gaussian analytic function on $\mathbb C$. The covariance kernel of $\{\cF_\phi \xi (z)\}_{z\in \C}$ is given by
\begin{equation}
    K(z,w) := \Cov[\cF_\phi \xi(z) , \cF_\phi \xi(w)] = \pi \exp\Big ( i\pi(u_1v_1 - u_2v_2) - \frac{\pi}{2}(|z|^2 + |w|^2) + \pi z\bar w \Big ),
\end{equation}
where $z= u_1 + i v_1, w = u_2 + i v_2$. In particular,
    $K(z,z) = \Var[\cF_\phi \xi (z)] = \pi$.

Theorem \ref{p:whitenoise} particularly implies that the zeros of the STFT of the white noise are the same as those of 
\[ \sum_{k=0}^\infty \langle \xi, h_k \rangle  \sqrt{\frac{\pi^k}{k!}} z^k.\]
Since $\langle \xi, h_k \rangle$ are i.i.d. real Gaussians, this is also called the \emph{symmetric planar Gaussian analytic function}. However, the zero set in this case is not stationary, which poses an obstacle in using techniques from spatial statistics. Motivated by early results by Prosen, the authors of \cite{BARDENET_Spec} argue that the zero set of the spectrogram of real Gaussian white noise could be effectively approximated by the zero set of the spectrogram of \emph{complex} Gaussian white noise. It turns out that the latter is precisely the zero set of the planar GAF, which has been shown to be invariant under isomorphisms of the complex plane.

\subsection{Applications in signal analysis}
The connection to the zeros of GAFs provides a useful approach for studying the stochastic geometry of spectrogram level sets of a signal. In \cite{GhoshGAF}, Ghosh et al. provided a theoretical guarantee for using GAFs in the problem of signal detection. To elaborate,
let $f$ be a noisy signal generated as
\begin{equation}\label{eq:signal}
     f = \lambda h_k + \xi
\end{equation}
where $\lambda \in \R$ is a real parameter, $h_k$ is an Hermite function defined as \eqref{eq:hermite} and $\xi$ is the white noise. We are interested in the spectrogram level sets of $f$ and their restrictions
\begin{eqnarray} \label{eq:spectrogram}
    \mathcal L(a) &:=& \{z\in \C: |\cF_\phi f(z)| \ge a \}, \quad a\ge 0 \\
    \mathcal L_L(a) &:=& \mathcal L(a) \cap \B_L, \quad \text{where } \B_L:=\{z\in \C: |\re(z)|\le L, |\im(z)|\le L\}.
\end{eqnarray}

Utilising Equation \eqref{eq:stft_whitenoise} and techniques from maxima of Gaussian random fields, Ghosh et al. in \cite{GhoshGAF} showed the following result for the spectrogram level set of the white noise
\begin{theorem} \label{t:sup_whitenoise}
    For $L\ge \pi$, we have that for any $\tau > 0$
    \[ \P \Big ( \sup_{z\in \B_L}|\cF_\phi \xi (z)| \le \sqrt 2 (14K +\tau) \sqrt{\log L}\Big ) \ge 1 - 4 \exp \Big (- \frac{\tau^2}{2 \pi} \log L \Big )\]
    where $K>0$ is a constant, and
    \[ \B_L:= \{z\in \C: |\re(z)|\le L, |\im(z)|\le L\}.\]
\end{theorem}

We note that  $\cF_\phi$ is linear and $f$ is a mixture of an Hermite function and the white noise. Thus, to study spectrogram level sets of $f$, one can utilise Theorem \ref{t:sup_whitenoise} and the fact that
\begin{equation}
    \sup_{z\in \C} |\cF_\phi h_k(z)| = \prod_{n=1}^k\sqrt{\frac{k}{en}},
\end{equation}
and the supremum is attained at $|z|=\sqrt{k/\pi}$ (see \cite{GhoshGAF}). 
Combining the ingredients above, we have the following theorem.
\begin{theorem} \label{t:noisy-signal}
    Let $f$ be a noisy signal generated as \eqref{eq:signal}. Let $L\ge \max \{ \sqrt{k/\pi}, \pi\}$ and 
    \[ |\lambda | \ge \frac{5\sqrt{2}(14K + \tau) \sqrt{\log L}}{\prod_{n=1}^k \sqrt{k/(en)}}\]
    where $K>0$ is the constant in  Theorem \ref{t:sup_whitenoise} and $\tau>0$ is a parameter. Then with probability at least $1-4\exp(-\frac{\tau^2}{2\pi} \log L )$, we have
    \[ \varnothing  \neq \mathcal L_L\Big (3\sqrt{2}(14K+\tau)\log L \Big ) \subset \Big \{z\in \C: \frac{|\cF_\phi h_k(z)|}{\prod_{n=1}^k \sqrt{k/(en)}} > \alpha\Big \}\]
    where
    \[\alpha := \frac{\sqrt{2}(14K + \tau) \sqrt{\log L}}{|\lambda|\prod_{n=1}^k \sqrt{k/(en)}} \in (0,1/5]. \]
\end{theorem}


Theorem \ref{t:noisy-signal} provides useful technical tools in signal detection. In particular, the authors in \cite{GhoshGAF} present two applications: one in hypothesis testing and one in signal estimation. For the first application, they design an effective hypothesis test to distinguish between pure white noise and the presence of a fundamental signal. They also provide theoretical guarantees for its effectiveness using non-asymptotic bounds on the power of the test. For the second application, they demonstrate an estimation procedure for identifying a fundamental signal, if present, and provide error bounds on the accuracy of this estimation based on the sample size. For more details, we refer readers to Section 4 of \cite{GhoshGAF}.

