\subsection{Background}
The use of point processes to model stochastic network configurations has a fairly long history. Indeed, a key aspect of wireless communication networks is their spatial nature. These networks consist of numerous nodes distributed across space, typically in two- or three-dimensional Euclidean space. Each node of the network, modelling a wireless base station, broadcasts signals that interfere with one another, making it crucial to understand the distribution of signal strength across different locations, accounting for these interferences. One main objective is to design network layouts that maximize signal strength for the majority of locations.


This area is also intertwined closely with the stochastic geometry of point processes. The stochastic geometry of negatively dependent point processes has attracted considerable attention in recent years in the applied probability community, including but not limited to the problems of continuum percolation or the Gilbert disk model \cite{yogesh, Baccelli-Blasz, Blaszczyszyn_Haenggi_Keeler_Mukherjee_2018, shirai-miyoshi, ghosh_aop}.

\subsection{Stochastic Networks and hyperuniformity}
The classical and most popularly studied model for stochastic networks is based on the Poisson point process \cite{Baccelli-Blasz,Blaszczyszyn_Haenggi_Keeler_Mukherjee_2018,Haenggi}. However, Poissonian networks exhibit similar issues to independent random points, thereby limiting their performance. 

In view of this, dependent point fields have been investigated in this area; most notably the so-called \textit{Ginibre network} which is based on the Ginibre ensemble -- a DPP that is intimately related to random matrices and 2D Coulomb gases. In the celebrated work \cite{shirai-miyoshi}, the authors proposed a stochastic network based on the eigenvalues from the Ginibre ensemble, and demonstrated rigorously that such a network provides provably improved performance over standard Poissonian networks. To this end, they observed that for quantities that depend solely on the absolute values of the Ginibre points, such as the coverage probability or link success probability, we can take advantage of the fact that the process of absolute values can be equivalently described by independent gamma random variables.

However, Ginibre networks have difficulties vis-a-vis tractability and robustness issues. For instance, from an application point of view, the network should be easy to simulate, so that large scale statistical behaviour can be easily understood from empirical studies. It is reasonable to believe that many of the salutary properties of a Ginibre network are consequences of not the specific, delicate structure of the Ginibre model itself, but of the more general phenomenon of \textit{hyperuniformity}. 

Hyperuniformity refers to the suppression of fluctuations in particle numbers in a spatial point process. In Poissonian systems, the variance in the number of points within a large spatial domain increases proportionally to the volume (a phenomenon described in physics as extensive fluctuations). However, in hyperuniform systems, these fluctuations are of a smaller order relative to the volume (sub-extensive), often scaling only with the surface area of the domain. Such behaviour manifests itself in a wide array of natural systems  \cite{hough2009zeros, GhoshLebowitz, TORQUATO, torquato2, ghosh-lebowitz-cmp}, and allows for elegant and effective interpretations in terms of their \textit{power spectra}  \cite{PhysRevE,baake, ghoshlebowitz-cmp2}. We refer the readers to \cite{GhoshLebowitz, coste} for a detailed discussion on hyperuniformity, and to \cite{lacieze_hyperuniform} for statistical tests of hyperuniformity for point processes.

\subsection{Stochastic networks based on disordered lattices}
In \cite{ghosh-shirai}, the authors investigate disordered lattice networks, aiming to find a best-of-both-worlds compromise between Poissonian and Ginibre networks. As the name indicates, these networks are based on point processes that are i.i.d. random perturbations of Euclidean lattices. It is known that under very general conditions -- as soon as the tail of the perturbing random variable decays faster that $\|x\|^{-d}$ for a $d$-dimensional lattice -- the resulting disordered lattice constitutes a hyperuniform process \cite{GhoshLebowitz, Ghosh2016NumberRI}. 

By modulating the statistical law of the perturbations to vary within a parametric family of distributions (e.g. an exponential family), disordered lattices allow for the developments of a parametric statistical theory in the context of stochastic networks. In the particular setting of Gaussian perturbations, modulating the scale parameter of the Gaussian provides a smooth interpolation between stochastic networks of varying degrees of disorder, with a rigid lattice at one end and a Poisson network at the other. 

A typical instance of the network statistics we want to investigate is the so-called \textit{Signal to Interference plus Noise Ratio} (abbrv. SINR) at a typical point (let's say the origin). Due to the randomness of the network, this is a random variable, and therefore it is of interest to investigate its tails, also known as coverage probabilities:
\begin{equation} \label{eq:infiniteprod_d}
p_c(\theta, \beta) =  \P(\mathrm{SINR} > \theta)
= \E\left[\prod_{j \not= B} \left(1 +\theta
 \frac{|X_B|^{d\beta}}{|X_j|^{d\beta}}  \right)^{-1}
\right],
\end{equation} 
where $X_B$ is the location of the node of the network nearest to the origin, $d$ is the ambient dimension, and $\beta>0$ is a fixed modelling parameter related to the decay of the so-called \textit{path loss function} \cite{Baccelli-Blasz, shirai-miyoshi}.

Focusing on Gaussian disorder for the perturbed lattice networks, the authors in \cite{ghosh-shirai} demonstrate that for an appropriate choice of the disorder strength, the coverage probabilities roughly match those of a Ginibre random network, thereby achieving the \textit{best of both worlds} phenomenon aspired for. 

Interestingly, at the same disorder strength, the perturbed lattice point process appears to be the \textit{closest} to the Ginibre random point field. In order to achieve this comparison, \cite{ghosh-shirai} introduces an approach towards robust measurement of the ``distance'' between point processes by a quantitative comparison of their \textit{persistence homology}, a technique that we expect to see increasing usefulness in coming years. 

In addition to elaborate theoretical analysis demonstrating power law asymptotics for the coverage probability of disordered lattice networks and of interpolation properties discussed above, \cite{ghosh-shirai} obtains stochastic approximations to the SINR variable in various limiting regimes of interest. An interesting finding in this vein is the emergence of the Epstein Zeta Function of the underlying lattice as a key determinant of network statistics. This leads to the optimal choices of lattices as the minimizers of this lattice energy function, which is given by the triangular lattice in 2D and face centered cubic (FCC) lattice in 3D. 

