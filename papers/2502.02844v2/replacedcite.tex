\section{Related Works}
\textbf{Robust MARL Strategies:}
Recent research has focused on robust MARL to address unexpected changes in multi-agent environments. Max-min optimization ____ has been applied to traditional MARL algorithms for robust learning ____. Robust Nash equilibrium has been redefined to better suit multi-agent systems ____. Regularization-based approaches have also been explored to improve MARL robustness ____, alongside distributional reinforcement learning methods to manage uncertainties ____.

\textbf{Adversarial Attacks for Resilient RL:}
To strengthen RL, numerous studies have explored adversarial learning to train policies under worst-case scenarios ____. These attacks introduce perturbations to various MDP components, including state ____, action ____, and reward ____. Adversarial attacks have recently been extended to multi-agent setups, introducing uncertainties to state or observation ____, actions ____, and rewards ____. Further research has applied adversarial attacks to value decomposition frameworks ____, selected critical agents for targeted attacks ____, and analyzed their effects on inter-agent communication ____.

\textbf{Model-based Frameworks for Robust RL:}
Model-based methods have been extensively studied to enhance RL robustness ____, including adversarial extensions ____. Transition models have been leveraged to improve robustness ____, and offline setups have been explored for robust training ____. In multi-agent systems, model-based approaches address challenges like constructing worst-case sets ____ and managing transition kernel uncertainty ____.


\begin{figure*}[ht!]
    \centering
    \vspace{-0.1in}
    \includegraphics[width=0.8\textwidth]{figures/concept.pdf}
    \vspace{-0.2in}
    \caption{Visualization of Wolfpack attack strategy during combat in the Starcraft II environment: (a) The initial agent is attacked, disrupting its original action (b) Responding (follow-up) agents to help the initially attacked agent and (c) Wolfpack adversarial attack that disrupts help actions of follow-up agents.}
    \vspace{-0.2in}
    \label{fig:concept}
\end{figure*}