\section{Related Work}
In this section, we summarize prior relevant work in HCI and information visualization with a focus on (1) critical and feminist thinking for visualization research and HCI, (2) visualizations of demographic data, and (3) practices of visualization designers. 

\subsection{Critical and Feminist Approaches in Information Visualization}
Early work by ____ proposed a critical approach for visualization research that centered values such as disclosure, plurality, contingency, and empowerment. Emphasizing the potential of data visualizations to influence high-stakes decision-making, Correll____ argued that visualizations are not neutral and that researchers in this space should carefully consider the ethical implications of their work. Correll proposed a list of obligations for the visualization community, including visualizing hidden labor, collecting data with empathy, and challenging structures of power. Extending this call from a broader data perspective, D'Ignazio and Klein____ proposed the data feminism framework, drawing on feminist epistemologies to question the role of power in data processes. Prior critical work in HCI has also examined the challenges of working with racially marginalized populations and their data, including work that adapted critical race theory for HCI____ and a historical breakdown of the dimensions of race and their construction in current sociotechnical systems and research practices____.

A growing body of research has also aimed to broaden current visualization practices to include concepts such as care, equity, and feminist entanglement. Schwabish and Feng's Do No Harm guide____ offers guiding principles for data practitioners that center equity particularly when visualizing demographic data. Using the lens of matters of care, Akbaba et al.____ highlighted the tensions related to power and responsibility in visualization research collaborations between researchers and domain experts. Recent work by Akbaba et al.____ traced the genealogical lineage of feminist entanglement theory and illustrated a case study of applying feminist epistemologies to visualization research. In this work, Akbaba et al.____ introduced situated knowledges, a feminist epistemology proposed by____ that emphasizes that all knowledge is situated in the perspective of the person producing said knowledge, which in turn determines what, how, and how much they can know and can thus communicate to others. We examine how visualization artifacts are shaped by the beliefs, values, politics, and positionalities of designers and frame our findings using Haraway's concept of situated knowledges____. 

\subsection{Visualizing Demographic Data}
Research on visualizing demographic data has broadly focused on best practices (such as using color and language mindfully____) and open questions around equitably representing these categories____. Specifically, this work has included visualizations of gender, with ____ summarizing visualization practices in five academic communities, ____ investigating gender representation in 30 years of IEEE VIS publications, and ____ presenting an artistic visualization of sexual harassment. Although these efforts reflect growing interest in equitably visualizing demographic data, the visualization research community has yet to critically engage with how the positionalities of researchers and practitioners, who remain predominantly male and White____, might influence their approaches in working with demographic data of groups they do not belong to. Prior work has also explored anthropographics (human-shaped visualizations) for their potential to evoke prosocial feelings from audiences in specific humanitarian contexts ____. In response to homogeneous anthropographics____ with generic human shapes that obscure the demographic differences of the people being visualized, Dhawka et al.____ proposed diverse anthropographics---visualizations that communicate demographic diversity using physical characteristics---and provided a broad overview of the research challenges around representing racial data. Extending this work____, we investigate the specific practices and values of designers producing visualizations of demographic data of race and gender, including their use of diverse anthropographics. 

\subsection{Practices of Visualization Designers}
Existing research on visualization practitioners has largely examined their design processes in attempts to bridge knowledge gaps between the academic and practitioner communities. Parsons investigated the design processes of professional data visualization designers, including their decision-making steps, strategies, and familiarity with popular design methods____. Parsons' findings revealed that visualization designers do not follow pre-determined steps but rather engage in a reflexive and intuitive process, as a result of their ``situated knowing''____ that are often absent from idealized models of visualization workflows. Subsequent work by Parsons et al.____ also investigated how data visualization creators engage in design fixation in their process, focusing largely on creative design practices. Work by Bako et al.____ has examined how visualization designers use examples in their process, particularly in data storytelling. Within the context of the COVID-19 pandemic, Zhang et al.____ investigated the experiences of visualization dashboard designers and their strategies for handling potential misinterpretation of their work. While prior research has focused on the strategies and design processes of visualization practitioners more broadly, we specifically investigate how practitioners experience the production of visualizations of race and gender, focusing on how the nature of this demographic data interacts with issues of power, politics, and neutrality.
%