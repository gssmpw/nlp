\pdfoutput=1
% \documentclass{article} % For LaTeX2e
\documentclass[11pt, letterpaper, shortlabels]{berkeley}
% Optional math commands from https://github.com/goodfeli/dlbook_notation.
% %%%%% NEW MATH DEFINITIONS %%%%%

\usepackage{amsmath,amsfonts,bm}
\usepackage{derivative}
% Mark sections of captions for referring to divisions of figures
\newcommand{\figleft}{{\em (Left)}}
\newcommand{\figcenter}{{\em (Center)}}
\newcommand{\figright}{{\em (Right)}}
\newcommand{\figtop}{{\em (Top)}}
\newcommand{\figbottom}{{\em (Bottom)}}
\newcommand{\captiona}{{\em (a)}}
\newcommand{\captionb}{{\em (b)}}
\newcommand{\captionc}{{\em (c)}}
\newcommand{\captiond}{{\em (d)}}

% Highlight a newly defined term
\newcommand{\newterm}[1]{{\bf #1}}

% Derivative d 
\newcommand{\deriv}{{\mathrm{d}}}

% Figure reference, lower-case.
\def\figref#1{figure~\ref{#1}}
% Figure reference, capital. For start of sentence
\def\Figref#1{Figure~\ref{#1}}
\def\twofigref#1#2{figures \ref{#1} and \ref{#2}}
\def\quadfigref#1#2#3#4{figures \ref{#1}, \ref{#2}, \ref{#3} and \ref{#4}}
% Section reference, lower-case.
\def\secref#1{section~\ref{#1}}
% Section reference, capital.
\def\Secref#1{Section~\ref{#1}}
% Reference to two sections.
\def\twosecrefs#1#2{sections \ref{#1} and \ref{#2}}
% Reference to three sections.
\def\secrefs#1#2#3{sections \ref{#1}, \ref{#2} and \ref{#3}}
% Reference to an equation, lower-case.
\def\eqref#1{equation~\ref{#1}}
% Reference to an equation, upper case
\def\Eqref#1{Equation~\ref{#1}}
% A raw reference to an equation---avoid using if possible
\def\plaineqref#1{\ref{#1}}
% Reference to a chapter, lower-case.
\def\chapref#1{chapter~\ref{#1}}
% Reference to an equation, upper case.
\def\Chapref#1{Chapter~\ref{#1}}
% Reference to a range of chapters
\def\rangechapref#1#2{chapters\ref{#1}--\ref{#2}}
% Reference to an algorithm, lower-case.
\def\algref#1{algorithm~\ref{#1}}
% Reference to an algorithm, upper case.
\def\Algref#1{Algorithm~\ref{#1}}
\def\twoalgref#1#2{algorithms \ref{#1} and \ref{#2}}
\def\Twoalgref#1#2{Algorithms \ref{#1} and \ref{#2}}
% Reference to a part, lower case
\def\partref#1{part~\ref{#1}}
% Reference to a part, upper case
\def\Partref#1{Part~\ref{#1}}
\def\twopartref#1#2{parts \ref{#1} and \ref{#2}}

\def\ceil#1{\lceil #1 \rceil}
\def\floor#1{\lfloor #1 \rfloor}
\def\1{\bm{1}}
\newcommand{\train}{\mathcal{D}}
\newcommand{\valid}{\mathcal{D_{\mathrm{valid}}}}
\newcommand{\test}{\mathcal{D_{\mathrm{test}}}}

\def\eps{{\epsilon}}


% Random variables
\def\reta{{\textnormal{$\eta$}}}
\def\ra{{\textnormal{a}}}
\def\rb{{\textnormal{b}}}
\def\rc{{\textnormal{c}}}
\def\rd{{\textnormal{d}}}
\def\re{{\textnormal{e}}}
\def\rf{{\textnormal{f}}}
\def\rg{{\textnormal{g}}}
\def\rh{{\textnormal{h}}}
\def\ri{{\textnormal{i}}}
\def\rj{{\textnormal{j}}}
\def\rk{{\textnormal{k}}}
\def\rl{{\textnormal{l}}}
% rm is already a command, just don't name any random variables m
\def\rn{{\textnormal{n}}}
\def\ro{{\textnormal{o}}}
\def\rp{{\textnormal{p}}}
\def\rq{{\textnormal{q}}}
\def\rr{{\textnormal{r}}}
\def\rs{{\textnormal{s}}}
\def\rt{{\textnormal{t}}}
\def\ru{{\textnormal{u}}}
\def\rv{{\textnormal{v}}}
\def\rw{{\textnormal{w}}}
\def\rx{{\textnormal{x}}}
\def\ry{{\textnormal{y}}}
\def\rz{{\textnormal{z}}}

% Random vectors
\def\rvepsilon{{\mathbf{\epsilon}}}
\def\rvphi{{\mathbf{\phi}}}
\def\rvtheta{{\mathbf{\theta}}}
\def\rva{{\mathbf{a}}}
\def\rvb{{\mathbf{b}}}
\def\rvc{{\mathbf{c}}}
\def\rvd{{\mathbf{d}}}
\def\rve{{\mathbf{e}}}
\def\rvf{{\mathbf{f}}}
\def\rvg{{\mathbf{g}}}
\def\rvh{{\mathbf{h}}}
\def\rvu{{\mathbf{i}}}
\def\rvj{{\mathbf{j}}}
\def\rvk{{\mathbf{k}}}
\def\rvl{{\mathbf{l}}}
\def\rvm{{\mathbf{m}}}
\def\rvn{{\mathbf{n}}}
\def\rvo{{\mathbf{o}}}
\def\rvp{{\mathbf{p}}}
\def\rvq{{\mathbf{q}}}
\def\rvr{{\mathbf{r}}}
\def\rvs{{\mathbf{s}}}
\def\rvt{{\mathbf{t}}}
\def\rvu{{\mathbf{u}}}
\def\rvv{{\mathbf{v}}}
\def\rvw{{\mathbf{w}}}
\def\rvx{{\mathbf{x}}}
\def\rvy{{\mathbf{y}}}
\def\rvz{{\mathbf{z}}}

% Elements of random vectors
\def\erva{{\textnormal{a}}}
\def\ervb{{\textnormal{b}}}
\def\ervc{{\textnormal{c}}}
\def\ervd{{\textnormal{d}}}
\def\erve{{\textnormal{e}}}
\def\ervf{{\textnormal{f}}}
\def\ervg{{\textnormal{g}}}
\def\ervh{{\textnormal{h}}}
\def\ervi{{\textnormal{i}}}
\def\ervj{{\textnormal{j}}}
\def\ervk{{\textnormal{k}}}
\def\ervl{{\textnormal{l}}}
\def\ervm{{\textnormal{m}}}
\def\ervn{{\textnormal{n}}}
\def\ervo{{\textnormal{o}}}
\def\ervp{{\textnormal{p}}}
\def\ervq{{\textnormal{q}}}
\def\ervr{{\textnormal{r}}}
\def\ervs{{\textnormal{s}}}
\def\ervt{{\textnormal{t}}}
\def\ervu{{\textnormal{u}}}
\def\ervv{{\textnormal{v}}}
\def\ervw{{\textnormal{w}}}
\def\ervx{{\textnormal{x}}}
\def\ervy{{\textnormal{y}}}
\def\ervz{{\textnormal{z}}}

% Random matrices
\def\rmA{{\mathbf{A}}}
\def\rmB{{\mathbf{B}}}
\def\rmC{{\mathbf{C}}}
\def\rmD{{\mathbf{D}}}
\def\rmE{{\mathbf{E}}}
\def\rmF{{\mathbf{F}}}
\def\rmG{{\mathbf{G}}}
\def\rmH{{\mathbf{H}}}
\def\rmI{{\mathbf{I}}}
\def\rmJ{{\mathbf{J}}}
\def\rmK{{\mathbf{K}}}
\def\rmL{{\mathbf{L}}}
\def\rmM{{\mathbf{M}}}
\def\rmN{{\mathbf{N}}}
\def\rmO{{\mathbf{O}}}
\def\rmP{{\mathbf{P}}}
\def\rmQ{{\mathbf{Q}}}
\def\rmR{{\mathbf{R}}}
\def\rmS{{\mathbf{S}}}
\def\rmT{{\mathbf{T}}}
\def\rmU{{\mathbf{U}}}
\def\rmV{{\mathbf{V}}}
\def\rmW{{\mathbf{W}}}
\def\rmX{{\mathbf{X}}}
\def\rmY{{\mathbf{Y}}}
\def\rmZ{{\mathbf{Z}}}

% Elements of random matrices
\def\ermA{{\textnormal{A}}}
\def\ermB{{\textnormal{B}}}
\def\ermC{{\textnormal{C}}}
\def\ermD{{\textnormal{D}}}
\def\ermE{{\textnormal{E}}}
\def\ermF{{\textnormal{F}}}
\def\ermG{{\textnormal{G}}}
\def\ermH{{\textnormal{H}}}
\def\ermI{{\textnormal{I}}}
\def\ermJ{{\textnormal{J}}}
\def\ermK{{\textnormal{K}}}
\def\ermL{{\textnormal{L}}}
\def\ermM{{\textnormal{M}}}
\def\ermN{{\textnormal{N}}}
\def\ermO{{\textnormal{O}}}
\def\ermP{{\textnormal{P}}}
\def\ermQ{{\textnormal{Q}}}
\def\ermR{{\textnormal{R}}}
\def\ermS{{\textnormal{S}}}
\def\ermT{{\textnormal{T}}}
\def\ermU{{\textnormal{U}}}
\def\ermV{{\textnormal{V}}}
\def\ermW{{\textnormal{W}}}
\def\ermX{{\textnormal{X}}}
\def\ermY{{\textnormal{Y}}}
\def\ermZ{{\textnormal{Z}}}

% Vectors
\def\vzero{{\bm{0}}}
\def\vone{{\bm{1}}}
\def\vmu{{\bm{\mu}}}
\def\vtheta{{\bm{\theta}}}
\def\vphi{{\bm{\phi}}}
\def\va{{\bm{a}}}
\def\vb{{\bm{b}}}
\def\vc{{\bm{c}}}
\def\vd{{\bm{d}}}
\def\ve{{\bm{e}}}
\def\vf{{\bm{f}}}
\def\vg{{\bm{g}}}
\def\vh{{\bm{h}}}
\def\vi{{\bm{i}}}
\def\vj{{\bm{j}}}
\def\vk{{\bm{k}}}
\def\vl{{\bm{l}}}
\def\vm{{\bm{m}}}
\def\vn{{\bm{n}}}
\def\vo{{\bm{o}}}
\def\vp{{\bm{p}}}
\def\vq{{\bm{q}}}
\def\vr{{\bm{r}}}
\def\vs{{\bm{s}}}
\def\vt{{\bm{t}}}
\def\vu{{\bm{u}}}
\def\vv{{\bm{v}}}
\def\vw{{\bm{w}}}
\def\vx{{\bm{x}}}
\def\vy{{\bm{y}}}
\def\vz{{\bm{z}}}

% Elements of vectors
\def\evalpha{{\alpha}}
\def\evbeta{{\beta}}
\def\evepsilon{{\epsilon}}
\def\evlambda{{\lambda}}
\def\evomega{{\omega}}
\def\evmu{{\mu}}
\def\evpsi{{\psi}}
\def\evsigma{{\sigma}}
\def\evtheta{{\theta}}
\def\eva{{a}}
\def\evb{{b}}
\def\evc{{c}}
\def\evd{{d}}
\def\eve{{e}}
\def\evf{{f}}
\def\evg{{g}}
\def\evh{{h}}
\def\evi{{i}}
\def\evj{{j}}
\def\evk{{k}}
\def\evl{{l}}
\def\evm{{m}}
\def\evn{{n}}
\def\evo{{o}}
\def\evp{{p}}
\def\evq{{q}}
\def\evr{{r}}
\def\evs{{s}}
\def\evt{{t}}
\def\evu{{u}}
\def\evv{{v}}
\def\evw{{w}}
\def\evx{{x}}
\def\evy{{y}}
\def\evz{{z}}

% Matrix
\def\mA{{\bm{A}}}
\def\mB{{\bm{B}}}
\def\mC{{\bm{C}}}
\def\mD{{\bm{D}}}
\def\mE{{\bm{E}}}
\def\mF{{\bm{F}}}
\def\mG{{\bm{G}}}
\def\mH{{\bm{H}}}
\def\mI{{\bm{I}}}
\def\mJ{{\bm{J}}}
\def\mK{{\bm{K}}}
\def\mL{{\bm{L}}}
\def\mM{{\bm{M}}}
\def\mN{{\bm{N}}}
\def\mO{{\bm{O}}}
\def\mP{{\bm{P}}}
\def\mQ{{\bm{Q}}}
\def\mR{{\bm{R}}}
\def\mS{{\bm{S}}}
\def\mT{{\bm{T}}}
\def\mU{{\bm{U}}}
\def\mV{{\bm{V}}}
\def\mW{{\bm{W}}}
\def\mX{{\bm{X}}}
\def\mY{{\bm{Y}}}
\def\mZ{{\bm{Z}}}
\def\mBeta{{\bm{\beta}}}
\def\mPhi{{\bm{\Phi}}}
\def\mLambda{{\bm{\Lambda}}}
\def\mSigma{{\bm{\Sigma}}}

% Tensor
\DeclareMathAlphabet{\mathsfit}{\encodingdefault}{\sfdefault}{m}{sl}
\SetMathAlphabet{\mathsfit}{bold}{\encodingdefault}{\sfdefault}{bx}{n}
\newcommand{\tens}[1]{\bm{\mathsfit{#1}}}
\def\tA{{\tens{A}}}
\def\tB{{\tens{B}}}
\def\tC{{\tens{C}}}
\def\tD{{\tens{D}}}
\def\tE{{\tens{E}}}
\def\tF{{\tens{F}}}
\def\tG{{\tens{G}}}
\def\tH{{\tens{H}}}
\def\tI{{\tens{I}}}
\def\tJ{{\tens{J}}}
\def\tK{{\tens{K}}}
\def\tL{{\tens{L}}}
\def\tM{{\tens{M}}}
\def\tN{{\tens{N}}}
\def\tO{{\tens{O}}}
\def\tP{{\tens{P}}}
\def\tQ{{\tens{Q}}}
\def\tR{{\tens{R}}}
\def\tS{{\tens{S}}}
\def\tT{{\tens{T}}}
\def\tU{{\tens{U}}}
\def\tV{{\tens{V}}}
\def\tW{{\tens{W}}}
\def\tX{{\tens{X}}}
\def\tY{{\tens{Y}}}
\def\tZ{{\tens{Z}}}


% Graph
\def\gA{{\mathcal{A}}}
\def\gB{{\mathcal{B}}}
\def\gC{{\mathcal{C}}}
\def\gD{{\mathcal{D}}}
\def\gE{{\mathcal{E}}}
\def\gF{{\mathcal{F}}}
\def\gG{{\mathcal{G}}}
\def\gH{{\mathcal{H}}}
\def\gI{{\mathcal{I}}}
\def\gJ{{\mathcal{J}}}
\def\gK{{\mathcal{K}}}
\def\gL{{\mathcal{L}}}
\def\gM{{\mathcal{M}}}
\def\gN{{\mathcal{N}}}
\def\gO{{\mathcal{O}}}
\def\gP{{\mathcal{P}}}
\def\gQ{{\mathcal{Q}}}
\def\gR{{\mathcal{R}}}
\def\gS{{\mathcal{S}}}
\def\gT{{\mathcal{T}}}
\def\gU{{\mathcal{U}}}
\def\gV{{\mathcal{V}}}
\def\gW{{\mathcal{W}}}
\def\gX{{\mathcal{X}}}
\def\gY{{\mathcal{Y}}}
\def\gZ{{\mathcal{Z}}}

% Sets
\def\sA{{\mathbb{A}}}
\def\sB{{\mathbb{B}}}
\def\sC{{\mathbb{C}}}
\def\sD{{\mathbb{D}}}
% Don't use a set called E, because this would be the same as our symbol
% for expectation.
\def\sF{{\mathbb{F}}}
\def\sG{{\mathbb{G}}}
\def\sH{{\mathbb{H}}}
\def\sI{{\mathbb{I}}}
\def\sJ{{\mathbb{J}}}
\def\sK{{\mathbb{K}}}
\def\sL{{\mathbb{L}}}
\def\sM{{\mathbb{M}}}
\def\sN{{\mathbb{N}}}
\def\sO{{\mathbb{O}}}
\def\sP{{\mathbb{P}}}
\def\sQ{{\mathbb{Q}}}
\def\sR{{\mathbb{R}}}
\def\sS{{\mathbb{S}}}
\def\sT{{\mathbb{T}}}
\def\sU{{\mathbb{U}}}
\def\sV{{\mathbb{V}}}
\def\sW{{\mathbb{W}}}
\def\sX{{\mathbb{X}}}
\def\sY{{\mathbb{Y}}}
\def\sZ{{\mathbb{Z}}}

% Entries of a matrix
\def\emLambda{{\Lambda}}
\def\emA{{A}}
\def\emB{{B}}
\def\emC{{C}}
\def\emD{{D}}
\def\emE{{E}}
\def\emF{{F}}
\def\emG{{G}}
\def\emH{{H}}
\def\emI{{I}}
\def\emJ{{J}}
\def\emK{{K}}
\def\emL{{L}}
\def\emM{{M}}
\def\emN{{N}}
\def\emO{{O}}
\def\emP{{P}}
\def\emQ{{Q}}
\def\emR{{R}}
\def\emS{{S}}
\def\emT{{T}}
\def\emU{{U}}
\def\emV{{V}}
\def\emW{{W}}
\def\emX{{X}}
\def\emY{{Y}}
\def\emZ{{Z}}
\def\emSigma{{\Sigma}}

% entries of a tensor
% Same font as tensor, without \bm wrapper
\newcommand{\etens}[1]{\mathsfit{#1}}
\def\etLambda{{\etens{\Lambda}}}
\def\etA{{\etens{A}}}
\def\etB{{\etens{B}}}
\def\etC{{\etens{C}}}
\def\etD{{\etens{D}}}
\def\etE{{\etens{E}}}
\def\etF{{\etens{F}}}
\def\etG{{\etens{G}}}
\def\etH{{\etens{H}}}
\def\etI{{\etens{I}}}
\def\etJ{{\etens{J}}}
\def\etK{{\etens{K}}}
\def\etL{{\etens{L}}}
\def\etM{{\etens{M}}}
\def\etN{{\etens{N}}}
\def\etO{{\etens{O}}}
\def\etP{{\etens{P}}}
\def\etQ{{\etens{Q}}}
\def\etR{{\etens{R}}}
\def\etS{{\etens{S}}}
\def\etT{{\etens{T}}}
\def\etU{{\etens{U}}}
\def\etV{{\etens{V}}}
\def\etW{{\etens{W}}}
\def\etX{{\etens{X}}}
\def\etY{{\etens{Y}}}
\def\etZ{{\etens{Z}}}

% The true underlying data generating distribution
\newcommand{\pdata}{p_{\rm{data}}}
\newcommand{\ptarget}{p_{\rm{target}}}
\newcommand{\pprior}{p_{\rm{prior}}}
\newcommand{\pbase}{p_{\rm{base}}}
\newcommand{\pref}{p_{\rm{ref}}}

% The empirical distribution defined by the training set
\newcommand{\ptrain}{\hat{p}_{\rm{data}}}
\newcommand{\Ptrain}{\hat{P}_{\rm{data}}}
% The model distribution
\newcommand{\pmodel}{p_{\rm{model}}}
\newcommand{\Pmodel}{P_{\rm{model}}}
\newcommand{\ptildemodel}{\tilde{p}_{\rm{model}}}
% Stochastic autoencoder distributions
\newcommand{\pencode}{p_{\rm{encoder}}}
\newcommand{\pdecode}{p_{\rm{decoder}}}
\newcommand{\precons}{p_{\rm{reconstruct}}}

\newcommand{\laplace}{\mathrm{Laplace}} % Laplace distribution

\newcommand{\E}{\mathbb{E}}
\newcommand{\Ls}{\mathcal{L}}
\newcommand{\R}{\mathbb{R}}
\newcommand{\emp}{\tilde{p}}
\newcommand{\lr}{\alpha}
\newcommand{\reg}{\lambda}
\newcommand{\rect}{\mathrm{rectifier}}
\newcommand{\softmax}{\mathrm{softmax}}
\newcommand{\sigmoid}{\sigma}
\newcommand{\softplus}{\zeta}
\newcommand{\KL}{D_{\mathrm{KL}}}
\newcommand{\Var}{\mathrm{Var}}
\newcommand{\standarderror}{\mathrm{SE}}
\newcommand{\Cov}{\mathrm{Cov}}
% Wolfram Mathworld says $L^2$ is for function spaces and $\ell^2$ is for vectors
% But then they seem to use $L^2$ for vectors throughout the site, and so does
% wikipedia.
\newcommand{\normlzero}{L^0}
\newcommand{\normlone}{L^1}
\newcommand{\normltwo}{L^2}
\newcommand{\normlp}{L^p}
\newcommand{\normmax}{L^\infty}

\newcommand{\parents}{Pa} % See usage in notation.tex. Chosen to match Daphne's book.

\DeclareMathOperator*{\argmax}{arg\,max}
\DeclareMathOperator*{\argmin}{arg\,min}

\DeclareMathOperator{\sign}{sign}
\DeclareMathOperator{\Tr}{Tr}
\let\ab\allowbreak


% Recommended, but optional, packages for figures and better typesetting:

%\usepackage{algorithmic}
% hyperref makes hyperlinks in the resulting PDF.
% If your build breaks (sometimes temporarily if a hyperlink spans a page)
% please comment out the following usepackage line and replace
% \usepackage{icml2023} with \usepackage[nohyperref]{icml2023} above.
\usepackage{hyperref}
\usepackage{color-edits}

\newcommand{\ysy}[1]{\textcolor{blue}{{[ysy: #1]}}}
\newcommand{\czh}[1]{{\color{red} #1}}
\newcommand{\todo}{$\blacksquare$ TODO\xspace}
\newcommand{\eg}{{e.g.}}
\newcommand{\ie}{{i.e.}} 
\newcommand{\vs}{{v.s.}}
% \linespread{.98}
\usepackage{algorithm2e}
\usepackage{algorithm}
\usepackage{algorithmicx}
\usepackage{algpseudocode}
\newcommand\numberthis{\addtocounter{equation}{1}\tag{\theequation}}
\ifx\assumption\undefined
\newtheorem{assumption}{Assumption}
\fi

% if you use cleveref..
\usepackage[capitalize,noabbrev]{cleveref}
\makeatletter
\def\adl@drawiv#1#2#3{%
        \hskip.5\tabcolsep
        \xleaders#3{#2.5\@tempdimb #1{1}#2.5\@tempdimb}%
                #2\z@ plus1fil minus1fil\relax
        \hskip.5\tabcolsep}
\newcommand{\cdashlinelr}[1]{%
  \noalign{\vskip\aboverulesep
           \global\let\@dashdrawstore\adl@draw
           \global\let\adl@draw\adl@drawiv}
  \cdashline{#1}
  \noalign{\global\let\adl@draw\@dashdrawstore
           \vskip\belowrulesep}}
\makeatother
\newcommand{\method}{Agent-R\xspace}
%%%%%%%%%%%%%%%%%%%%%%%%%%%%%%%%
% THEOREMS
%%%%%%%%%%%%%%%%%%%%%%%%%%%%%%%%
% \theoremstyle{plain}
% \theoremstyle{definition}
% \newtheorem{theorem}{Theorem}[section]
% \newtheorem{proposition}[theorem]{Proposition}
% \newtheorem{lemma}[theorem]{Lemma}
% \newtheorem{corollary}[theorem]{Corollary}
% \newtheorem{definition}[theorem]{Definition}
% \newtheorem{assumption}[theorem]{Assumption}
% \theoremstyle{remark}
% \newtheorem{remark}[theorem]{Remark}

% Todonotes is useful during development; simply uncomment the next line
%    and comment out the line below the next line to turn off comments
%\usepackage[disable,textsize=tiny]{todonotes}
%\usepackage[textsize=tiny]{todonotes}
\usepackage{wrapfig}
\captionsetup[figure]{font=small,skip=0pt}
\setlength{\belowcaptionskip}{0pt}


% ##########################################

% ##########################################

% ##########################################

% ##########################################

% ##########################################
% \documentclass{article} % For LaTeX2e
% \documentclass[11pt]{berkeley}

% \usepackage[preprint]{neurips_2023}
\title{Breaking Focus: Contextual Distraction Curse in Large Language Models}

\usepackage[all]{hypcap}

\usepackage[authoryear, round]{natbib}
% \bibliographystyle{plainnat}

\usepackage{hyperref}[citecolor=magenta,linkcolor=magenta]


\usepackage{multirow, makecell, caption}\usepackage{microtype}
\usepackage{graphicx}
\usepackage{booktabs} % for professional tables
%\usepackage{floatrow}
%\usepackage{float}


\usepackage{amsmath}
\usepackage{amssymb}
\usepackage{mathtools}
\usepackage{amsthm}
\usepackage{mathrsfs}
\usepackage{nicefrac}
\usepackage{dsfont}
\usepackage{enumitem}
% \usepackage{minted}
%\usepackage{float}
\usepackage{arydshln}
\setlength\parindent{0pt}
% \setminted[python]{frame=lines, breaklines, framesep=2mm, fontsize=\footnotesize, numbersep=5pt}
%\usepackage{paralist}

% \usepackage[authoryear, sort&compress, round]{natbib}
\usepackage{xspace}
\usepackage[capitalize,noabbrev]{cleveref}
\bibliographystyle{plainnat}
\usepackage{subcaption}
\usepackage{wrapfig}
\usepackage{lipsum}
\usepackage{listings}
\usepackage{xcolor}
\usepackage{tcolorbox}
\usepackage{fontawesome5} % For \faLightbulb
\usepackage{amsmath}
\usepackage{amssymb}
\usepackage{mathtools}
\usepackage{amsthm}
\usepackage{bbm}
\usepackage{soul}
\tcbuselibrary{skins, breakable, listings, theorems}
\usepackage{algpseudocode}
\usepackage{setspace}

\usepackage{color}
\definecolor{deepblue}{rgb}{0,0,0.5}
\definecolor{deepred}{rgb}{0.6,0,0}
\definecolor{deepgreen}{rgb}{0,0.5,0}


% Python style for highlighting
\newcommand\pythonstyle{\lstset{
basicstyle=\ttfamily\footnotesize,
language=Python,
morekeywords={self, clip, exp, mse_loss, uniform_sample, concatenate, logsumexp},              % Add keywords here
keywordstyle=\color{deepblue}, % Custom highlighting style
stringstyle=\color{deepgreen},
frame=single,                         % Any extra options here
showstringspaces=false
}}

% Python environment
\lstnewenvironment{python}[1][]
{
\pythonstyle
\lstset{#1}
}
{}

% Python for external files
\newcommand\pythonexternal[2][]{{
\pythonstyle
\lstinputlisting[#1]{#2}}}

% Python for inline
\newcommand\pythoninline[1]{{\pythonstyle\lstinline!#1!}}

\definecolor{promptgray}{RGB}{200,200,200}
\definecolor{promptblue}{RGB}{25,118,210}
\definecolor{darkblue}{HTML}{0C2340}
\definecolor{gold}{HTML}{AE9142}


\newtcolorbox{promptbox}[2][]{%
    enhanced,
    unbreakable,
    before skip=2mm,
    after skip=2mm,
    colback=darkblue!5!white, 
    colframe=darkblue, 
    coltitle=white, 
    boxrule=0.5mm,
    sharp corners,
    arc=5pt,
    attach boxed title to top center={yshift=-3mm},
    boxed title style={
        enhanced,
        colback=gold, 
        colframe=darkblue,
        arc=5pt,
        outer arc=5pt,
        boxrule=0pt,
    },
    title={\faLightbulb[solid]\space #2},
    fonttitle=\bfseries\color{white}, 
    #1
}

% \newtcolorbox{promptbox}[2][]{%
%     enhanced,
%     unbreakable,
%     before skip=2mm,
%     after skip=2mm,
%     colback=promptgray!20!white,
%     colframe=promptblue!30!black,
%     coltitle=white,
%     boxrule=0.5mm,
%     sharp corners,
%     arc=5pt,
%     attach boxed title to top center={yshift=-3mm},
%     boxed title style={
%         enhanced,
%         colback=promptblue!50!white,
%         colframe=promptblue,
%         arc=5pt,
%         outer arc=5pt,
%         boxrule=0pt,
%     },
%     title={\faLightbulb[solid]\space #2},
%     fonttitle=\bfseries\color{white},
%     #1
% }


% % \usepackage{bbm}
\usepackage{graphicx}
\usepackage{amsmath,amssymb,amsthm,amsfonts}

\usepackage{paralist}
\usepackage{bm}
\usepackage{xspace}
\usepackage{url}
\usepackage{prettyref}
\usepackage{boxedminipage}
\usepackage{wrapfig}
\usepackage{ifthen}
\usepackage{color}
\usepackage{xspace}

\newcommand{\ii}{{\sc Indicator-Instance}\xspace}
\newcommand{\midd}{{\sf mid}}


\usepackage{amsmath,amsthm,amsfonts,amssymb}
\usepackage{mathtools}
\usepackage{graphicx}


% \usepackage{fullpage}

\usepackage{nicefrac}

\newtheorem{inftheorem}{Informal Theorem}
\newtheorem{claim}{Claim}
\newtheorem*{definition*}{Definition}
\newtheorem{example}{Example}

\DeclareMathOperator*{\argmax}{arg\,max}
\DeclareMathOperator*{\argmin}{arg\,min}
\usepackage{subcaption}

\newtheorem{problem}{Problem}
\usepackage[utf8]{inputenc}
\newcommand{\rank}{\mathsf{rank}}
\newcommand{\tr}{\mathsf{Tr}}
\newcommand{\tv}{\mathsf{TV}}
\newcommand{\opt}{\mathsf{OPT}}
\newcommand{\rr}{\textsc{R}\space}
\newcommand{\alg}{\textsf{Alg}\space}
\newcommand{\sd}{\textsf{sd}_\lambda}
\newcommand{\lblq}{\mathfrak{lq} (X_1)}
\newcommand{\diag}{\textsf{diag}}
\newcommand{\sign}{\textsf{sgn}}
\newcommand{\BC}{\texttt{BC} }
\newcommand{\MM}{\texttt{MM} }
\newcommand{\Nexp}{N_{\mathrm{exp}}}
\newcommand{\Nrep}{N_{\mathrm{replay}}}
\newcommand{\Drep}{D_{\mathrm{replay}}}
\newcommand{\Nsim}{N_{\mathrm{sim}}}
\newcommand{\piBC}{\pi^{\texttt{BC}}}
\newcommand{\piRE}{\pi^{\texttt{RE}}}
\newcommand{\piEMM}{\pi^{\texttt{MM}}}
\newcommand{\mmd}{\texttt{Mimic-MD} }
\newcommand{\RE}{\texttt{RE} }
\newcommand{\dem}{\pi^E}
\newcommand{\Rlint}{\mathcal{R}_{\mathrm{lin,t}}}
\newcommand{\Rlipt}{\mathcal{R}_{\mathrm{lip,t}}}
\newcommand{\Rlin}{\mathcal{R}_{\mathrm{lin}}}
\newcommand{\Rlip}{\mathcal{R}_{\mathrm{lip}}}
\newcommand{\Rmax}{R_{\mathrm{max}}}
\newcommand{\Rall}{\mathcal{R}_{\mathrm{all}}}
\newcommand{\Rdet}{\mathcal{R}_{\mathrm{det}}}
\newcommand{\Fmax}{F_{\mathrm{max}}}
\newcommand{\Nmax}{\mathcal{N}_{\mathrm{max}}}
\newcommand{\piref}{\pi^{\mathrm{ref}}}
\newcommand{\green}{\text{\color{green!75!black} green}\;}
\newcommand{\thetaBC}{\widehat{\theta}^{\textsf{BC}}}
\newcommand{\ent}{\mathcal{E}_{\Theta,n,\delta}}
\newcommand{\eNt}{\mathcal{E}_{\Theta_t,\Nexp,\delta}}
\newcommand{\eNtH}{\mathcal{E}_{\Theta_t,\Nexp,\delta/H}}

\newcommand{\eref}[1]{(\ref{#1})}
\newcommand{\sref}[1]{Sec. \ref{#1}}
\newcommand{\dr}{\widehat{d}_{\mathrm{replay}}}
\newcommand{\figref}[1]{Fig. \ref{#1}}

\usepackage{xcolor}
\definecolor{expert}{HTML}{008000}
\definecolor{error}{HTML}{f96565}
\newcommand{\GKS}[1]{{\textcolor{violet}{\textbf{GKS: #1}}}}
\newcommand{\Q}[1]{{\textcolor{red}{\textbf{Question #1}}}}
\newcommand{\ZSW}[1]{{\textcolor{orange}{\textbf{ZSW: #1}}}}
\newcommand{\JAB}[1]{{\textcolor{teal}{\textbf{JAB: #1}}}}
\newcommand{\jab}[1]{{\textcolor{teal}{\textbf{JAB: #1}}}}
\newcommand{\SAN}[1]{{\textcolor{blue}{\textbf{SC: #1}}}}
\newcommand{\scnote}[1]{\SAN{#1}}
\newcommand{\norm}[1]{\left\lVert #1 \right\rVert}

\usepackage{color-edits}
\addauthor{sw}{blue}

\usepackage{thmtools}
\usepackage{thm-restate}

\usepackage{tikz}
\usetikzlibrary{arrows,calc} 
\newcommand{\tikzAngleOfLine}{\tikz@AngleOfLine}
\def\tikz@AngleOfLine(#1)(#2)#3{%
\pgfmathanglebetweenpoints{%
\pgfpointanchor{#1}{center}}{%
\pgfpointanchor{#2}{center}}
\pgfmathsetmacro{#3}{\pgfmathresult}%
}

\declaretheoremstyle[
    headfont=\normalfont\bfseries, 
    bodyfont = \normalfont\itshape]{mystyle} 
\declaretheorem[name=Theorem,style=mystyle,numberwithin=section]{thm}

% \usepackage{algorithm}
% \usepackage{algorithmic}
\usepackage[linesnumbered,algoruled,boxed,lined,noend]{algorithm2e}

\usepackage{listings}
\usepackage{amsmath}
\usepackage{amsthm}
\usepackage{tikz}
\usepackage{caption}
\usepackage{mdwmath}
\usepackage{multirow}
\usepackage{mdwtab}
\usepackage{eqparbox}
\usepackage{multicol}
\usepackage{amsfonts}
\usepackage{tikz}
\usepackage{multirow,bigstrut,threeparttable}
\usepackage{amsthm}
\usepackage{bbm}
\usepackage{epstopdf}
\usepackage{mdwmath}
\usepackage{mdwtab}
\usepackage{eqparbox}
\usetikzlibrary{topaths,calc}
\usepackage{latexsym}
\usepackage{cite}
\usepackage{amssymb}
\usepackage{bm}
\usepackage{amssymb}
\usepackage{graphicx}
\usepackage{mathrsfs}
\usepackage{epsfig}
\usepackage{psfrag}
\usepackage{setspace}
\usepackage[%dvips,
            CJKbookmarks=true,
            bookmarksnumbered=true,
            bookmarksopen=true,
%						bookmarks=false,
            colorlinks=true,
            citecolor=red,
            linkcolor=blue,
            anchorcolor=red,
            urlcolor=blue
            ]{hyperref}
%\usepackage{algorithm}
\usepackage[linesnumbered,algoruled,boxed,lined]{algorithm2e}
\usepackage{algpseudocode}
\usepackage{stfloats}
\RequirePackage[numbers]{natbib}

\usepackage{comment}
\usepackage{mathtools}
\usepackage{blkarray}
\usepackage{multirow,bigdelim,dcolumn,booktabs}

\usepackage{xparse}
\usepackage{tikz}
\usetikzlibrary{calc}
\usetikzlibrary{decorations.pathreplacing,matrix,positioning}

\usepackage[T1]{fontenc}
\usepackage[utf8]{inputenc}
\usepackage{mathtools}
\usepackage{blkarray, bigstrut}
\usepackage{gauss}

\newenvironment{mygmatrix}{\def\mathstrut{\vphantom{\big(}}\gmatrix}{\endgmatrix}

\newcommand{\tikzmark}[1]{\tikz[overlay,remember picture] \node (#1) {};}

%% Adapted form https://tex.stackexchange.com/questions/206898/braces-for-cases-in-tabular-environment/207704#207704
\newcommand*{\BraceAmplitude}{0.4em}%
\newcommand*{\VerticalOffset}{0.5ex}%  
\newcommand*{\HorizontalOffset}{0.0em}% 
\newcommand*{\blocktextwid}{3.0cm}%
\NewDocumentCommand{\InsertLeftBrace}{%
	O{} % #1 = draw options
	O{\HorizontalOffset,\VerticalOffset} % #2 = optional brace shift options
	O{\blocktextwid} % #3 = optional text width
	m   % #4 = top tikzmark
	m   % #5 = bottom tikzmark
	m   % #6 = node text
}{%
	\begin{tikzpicture}[overlay,remember picture]
	\coordinate (Brace Top)    at ($(#4.north) + (#2)$);
	\coordinate (Brace Bottom) at ($(#5.south) + (#2)$);
	\draw [decoration={brace, amplitude=\BraceAmplitude}, decorate, thick, draw=black, #1]
	(Brace Bottom) -- (Brace Top) 
	node [pos=0.5, anchor=east, align=left, text width=#3, color=black, xshift=\BraceAmplitude] {#6};
	\end{tikzpicture}%
}%
\NewDocumentCommand{\InsertRightBrace}{%
	O{} % #1 = draw options
	O{\HorizontalOffset,\VerticalOffset} % #2 = optional brace shift options
	O{\blocktextwid} % #3 = optional text width
	m   % #4 = top tikzmark
	m   % #5 = bottom tikzmark
	m   % #6 = node text
}{%
	\begin{tikzpicture}[overlay,remember picture]
	\coordinate (Brace Top)    at ($(#4.north) + (#2)$);
	\coordinate (Brace Bottom) at ($(#5.south) + (#2)$);
	\draw [decoration={brace, amplitude=\BraceAmplitude}, decorate, thick, draw=black, #1]
	(Brace Top) -- (Brace Bottom) 
	node [pos=0.5, anchor=west, align=left, text width=#3, color=black, xshift=\BraceAmplitude] {#6};
	\end{tikzpicture}%
}%
\NewDocumentCommand{\InsertTopBrace}{%
	O{} % #1 = draw options
	O{\HorizontalOffset,\VerticalOffset} % #2 = optional brace shift options
	O{\blocktextwid} % #3 = optional text width
	m   % #4 = top tikzmark
	m   % #5 = bottom tikzmark
	m   % #6 = node text
}{%
	\begin{tikzpicture}[overlay,remember picture]
	\coordinate (Brace Top)    at ($(#4.west) + (#2)$);
	\coordinate (Brace Bottom) at ($(#5.east) + (#2)$);
	\draw [decoration={brace, amplitude=\BraceAmplitude}, decorate, thick, draw=black, #1]
	(Brace Top) -- (Brace Bottom) 
	node [pos=0.5, anchor=south, align=left, text width=#3, color=black, xshift=\BraceAmplitude] {#6};
	\end{tikzpicture}%
}%

\usetikzlibrary{patterns}

\definecolor{cof}{RGB}{219,144,71}
\definecolor{pur}{RGB}{186,146,162}
\definecolor{greeo}{RGB}{91,173,69}
\definecolor{greet}{RGB}{52,111,72}

% provide arXiv number if available:
% \arxiv{cs.IT/1502.00326}

% put your definitions there:

%\newtheorem{remark}{Remark} \def\remref#1{Remark~\ref{#1}}
%\newtheorem{conjecture}{Conjecture} \def\remref#1{Remark~\ref{#1}}
%\newtheorem{example}{Example}

%\theorembodyfont{\itshape}
%\newtheorem{theorem}{Theorem}
%\newtheorem{proposition}{Proposition}
%\newtheorem{lemma}{Lemma} \def\lemref#1{Lemma~\ref{#1}}
%\newtheorem{corollary}{Corollary}


%\theorembodyfont{\rmfamily}
%\newtheorem{definition}{Definition}
%\numberwithin{equation}{section}
% \theoremstyle{plain}
% \newtheorem{theorem}{Theorem}
% \newtheorem{Example}{Example}
% \newtheorem{lemma}{Lemma}
% \newtheorem{remark}{Remark}
% \newtheorem{corollary}{Corollary}
% \newtheorem{definition}{Definition}
% \newtheorem{conjecture}{Conjecture}
% \newtheorem{question}{Question}
% \newtheorem*{induction}{Induction Hypothesis}
% \newtheorem*{folklore}{Folklore}
% \newtheorem{assumption}{Assumption}

\def \by {\bar{y}}
\def \bx {\bar{x}}
\def \bh {\bar{h}}
\def \bz {\bar{z}}
\def \cF {\mathcal{F}}
\def \bP {\mathbb{P}}
\def \bE {\mathbb{E}}
\def \bR {\mathbb{R}}
\def \bF {\mathbb{F}}
\def \cG {\mathcal{G}}
\def \cM {\mathcal{M}}
\def \cB {\mathcal{B}}
\def \cN {\mathcal{N}}
\def \var {\mathsf{Var}}
\def\1{\mathbbm{1}}
\def \FF {\mathbb{F}}


\newenvironment{keywords}
{\bgroup\leftskip 20pt\rightskip 20pt \small\noindent{\bfseries
Keywords:} \ignorespaces}%
{\par\egroup\vskip 0.25ex}
\newlength\aftertitskip     \newlength\beforetitskip
\newlength\interauthorskip  \newlength\aftermaketitskip















%%%%%%%%%%%%%%%%%%%%%%%%%%%% by Wu %%%%%%%%%%%%%%%%%%%%%%%%%%%%
\usepackage{xspace}

\newcommand{\Lip}{\mathrm{Lip}}
\newcommand{\stepa}[1]{\overset{\rm (a)}{#1}}
\newcommand{\stepb}[1]{\overset{\rm (b)}{#1}}
\newcommand{\stepc}[1]{\overset{\rm (c)}{#1}}
\newcommand{\stepd}[1]{\overset{\rm (d)}{#1}}
\newcommand{\stepe}[1]{\overset{\rm (e)}{#1}}
\newcommand{\stepf}[1]{\overset{\rm (f)}{#1}}


\newcommand{\floor}[1]{{\left\lfloor {#1} \right \rfloor}}
\newcommand{\ceil}[1]{{\left\lceil {#1} \right \rceil}}

\newcommand{\blambda}{\bar{\lambda}}
\newcommand{\reals}{\mathbb{R}}
\newcommand{\naturals}{\mathbb{N}}
\newcommand{\integers}{\mathbb{Z}}
\newcommand{\Expect}{\mathbb{E}}
\newcommand{\expect}[1]{\mathbb{E}\left[#1\right]}
\newcommand{\Prob}{\mathbb{P}}
\newcommand{\prob}[1]{\mathbb{P}\left[#1\right]}
\newcommand{\pprob}[1]{\mathbb{P}[#1]}
\newcommand{\intd}{{\rm d}}
\newcommand{\TV}{{\sf TV}}
\newcommand{\LC}{{\sf LC}}
\newcommand{\PW}{{\sf PW}}
\newcommand{\htheta}{\hat{\theta}}
\newcommand{\eexp}{{\rm e}}
\newcommand{\expects}[2]{\mathbb{E}_{#2}\left[ #1 \right]}
\newcommand{\diff}{{\rm d}}
\newcommand{\eg}{e.g.\xspace}
\newcommand{\ie}{i.e.\xspace}
\newcommand{\iid}{i.i.d.\xspace}
\newcommand{\fracp}[2]{\frac{\partial #1}{\partial #2}}
\newcommand{\fracpk}[3]{\frac{\partial^{#3} #1}{\partial #2^{#3}}}
\newcommand{\fracd}[2]{\frac{\diff #1}{\diff #2}}
\newcommand{\fracdk}[3]{\frac{\diff^{#3} #1}{\diff #2^{#3}}}
\newcommand{\renyi}{R\'enyi\xspace}
\newcommand{\lpnorm}[1]{\left\|{#1} \right\|_{p}}
\newcommand{\linf}[1]{\left\|{#1} \right\|_{\infty}}
\newcommand{\lnorm}[2]{\left\|{#1} \right\|_{{#2}}}
\newcommand{\Lploc}[1]{L^{#1}_{\rm loc}}
\newcommand{\hellinger}{d_{\rm H}}
\newcommand{\Fnorm}[1]{\lnorm{#1}{\rm F}}
%% parenthesis
\newcommand{\pth}[1]{\left( #1 \right)}
\newcommand{\qth}[1]{\left[ #1 \right]}
\newcommand{\sth}[1]{\left\{ #1 \right\}}
\newcommand{\bpth}[1]{\Bigg( #1 \Bigg)}
\newcommand{\bqth}[1]{\Bigg[ #1 \Bigg]}
\newcommand{\bsth}[1]{\Bigg\{ #1 \Bigg\}}
\newcommand{\xxx}{\textbf{xxx}\xspace}
\newcommand{\toprob}{{\xrightarrow{\Prob}}}
\newcommand{\tolp}[1]{{\xrightarrow{L^{#1}}}}
\newcommand{\toas}{{\xrightarrow{{\rm a.s.}}}}
\newcommand{\toae}{{\xrightarrow{{\rm a.e.}}}}
\newcommand{\todistr}{{\xrightarrow{{\rm D}}}}
\newcommand{\eqdistr}{{\stackrel{\rm D}{=}}}
\newcommand{\iiddistr}{{\stackrel{\text{\iid}}{\sim}}}
%\newcommand{\var}{\mathsf{var}}
\newcommand\indep{\protect\mathpalette{\protect\independenT}{\perp}}
\def\independenT#1#2{\mathrel{\rlap{$#1#2$}\mkern2mu{#1#2}}}
\newcommand{\Bern}{\text{Bern}}
\newcommand{\Poi}{\mathsf{Poi}}
\newcommand{\iprod}[2]{\left \langle #1, #2 \right\rangle}
\newcommand{\Iprod}[2]{\langle #1, #2 \rangle}
\newcommand{\indc}[1]{{\mathbf{1}_{\left\{{#1}\right\}}}}
\newcommand{\Indc}{\mathbf{1}}
\newcommand{\regoff}[1]{\textsf{Reg}_{\mathcal{F}}^{\text{off}} (#1)}
\newcommand{\regon}[1]{\textsf{Reg}_{\mathcal{F}}^{\text{on}} (#1)}

\definecolor{myblue}{rgb}{.8, .8, 1}
\definecolor{mathblue}{rgb}{0.2472, 0.24, 0.6} % mathematica's Color[1, 1--3]
\definecolor{mathred}{rgb}{0.6, 0.24, 0.442893}
\definecolor{mathyellow}{rgb}{0.6, 0.547014, 0.24}


\newcommand{\red}{\color{red}}
\newcommand{\blue}{\color{blue}}
\newcommand{\nb}[1]{{\sf\blue[#1]}}
\newcommand{\nbr}[1]{{\sf\red[#1]}}

\newcommand{\tmu}{{\tilde{\mu}}}
\newcommand{\tf}{{\tilde{f}}}
\newcommand{\tp}{\tilde{p}}
\newcommand{\tilh}{{\tilde{h}}}
\newcommand{\tu}{{\tilde{u}}}
\newcommand{\tx}{{\tilde{x}}}
\newcommand{\ty}{{\tilde{y}}}
\newcommand{\tz}{{\tilde{z}}}
\newcommand{\tA}{{\tilde{A}}}
\newcommand{\tB}{{\tilde{B}}}
\newcommand{\tC}{{\tilde{C}}}
\newcommand{\tD}{{\tilde{D}}}
\newcommand{\tE}{{\tilde{E}}}
\newcommand{\tF}{{\tilde{F}}}
\newcommand{\tG}{{\tilde{G}}}
\newcommand{\tH}{{\tilde{H}}}
\newcommand{\tI}{{\tilde{I}}}
\newcommand{\tJ}{{\tilde{J}}}
\newcommand{\tK}{{\tilde{K}}}
\newcommand{\tL}{{\tilde{L}}}
\newcommand{\tM}{{\tilde{M}}}
\newcommand{\tN}{{\tilde{N}}}
\newcommand{\tO}{{\tilde{O}}}
\newcommand{\tP}{{\tilde{P}}}
\newcommand{\tQ}{{\tilde{Q}}}
\newcommand{\tR}{{\tilde{R}}}
\newcommand{\tS}{{\tilde{S}}}
\newcommand{\tT}{{\tilde{T}}}
\newcommand{\tU}{{\tilde{U}}}
\newcommand{\tV}{{\tilde{V}}}
\newcommand{\tW}{{\tilde{W}}}
\newcommand{\tX}{{\tilde{X}}}
\newcommand{\tY}{{\tilde{Y}}}
\newcommand{\tZ}{{\tilde{Z}}}

\newcommand{\sfa}{{\mathsf{a}}}
\newcommand{\sfb}{{\mathsf{b}}}
\newcommand{\sfc}{{\mathsf{c}}}
\newcommand{\sfd}{{\mathsf{d}}}
\newcommand{\sfe}{{\mathsf{e}}}
\newcommand{\sff}{{\mathsf{f}}}
\newcommand{\sfg}{{\mathsf{g}}}
\newcommand{\sfh}{{\mathsf{h}}}
\newcommand{\sfi}{{\mathsf{i}}}
\newcommand{\sfj}{{\mathsf{j}}}
\newcommand{\sfk}{{\mathsf{k}}}
\newcommand{\sfl}{{\mathsf{l}}}
\newcommand{\sfm}{{\mathsf{m}}}
\newcommand{\sfn}{{\mathsf{n}}}
\newcommand{\sfo}{{\mathsf{o}}}
\newcommand{\sfp}{{\mathsf{p}}}
\newcommand{\sfq}{{\mathsf{q}}}
\newcommand{\sfr}{{\mathsf{r}}}
\newcommand{\sfs}{{\mathsf{s}}}
\newcommand{\sft}{{\mathsf{t}}}
\newcommand{\sfu}{{\mathsf{u}}}
\newcommand{\sfv}{{\mathsf{v}}}
\newcommand{\sfw}{{\mathsf{w}}}
\newcommand{\sfx}{{\mathsf{x}}}
\newcommand{\sfy}{{\mathsf{y}}}
\newcommand{\sfz}{{\mathsf{z}}}
\newcommand{\sfA}{{\mathsf{A}}}
\newcommand{\sfB}{{\mathsf{B}}}
\newcommand{\sfC}{{\mathsf{C}}}
\newcommand{\sfD}{{\mathsf{D}}}
\newcommand{\sfE}{{\mathsf{E}}}
\newcommand{\sfF}{{\mathsf{F}}}
\newcommand{\sfG}{{\mathsf{G}}}
\newcommand{\sfH}{{\mathsf{H}}}
\newcommand{\sfI}{{\mathsf{I}}}
\newcommand{\sfJ}{{\mathsf{J}}}
\newcommand{\sfK}{{\mathsf{K}}}
\newcommand{\sfL}{{\mathsf{L}}}
\newcommand{\sfM}{{\mathsf{M}}}
\newcommand{\sfN}{{\mathsf{N}}}
\newcommand{\sfO}{{\mathsf{O}}}
\newcommand{\sfP}{{\mathsf{P}}}
\newcommand{\sfQ}{{\mathsf{Q}}}
\newcommand{\sfR}{{\mathsf{R}}}
\newcommand{\sfS}{{\mathsf{S}}}
\newcommand{\sfT}{{\mathsf{T}}}
\newcommand{\sfU}{{\mathsf{U}}}
\newcommand{\sfV}{{\mathsf{V}}}
\newcommand{\sfW}{{\mathsf{W}}}
\newcommand{\sfX}{{\mathsf{X}}}
\newcommand{\sfY}{{\mathsf{Y}}}
\newcommand{\sfZ}{{\mathsf{Z}}}


\newcommand{\calA}{{\mathcal{A}}}
\newcommand{\calB}{{\mathcal{B}}}
\newcommand{\calC}{{\mathcal{C}}}
\newcommand{\calD}{{\mathcal{D}}}
\newcommand{\calE}{{\mathcal{E}}}
\newcommand{\calF}{{\mathcal{F}}}
\newcommand{\calG}{{\mathcal{G}}}
\newcommand{\calH}{{\mathcal{H}}}
\newcommand{\calI}{{\mathcal{I}}}
\newcommand{\calJ}{{\mathcal{J}}}
\newcommand{\calK}{{\mathcal{K}}}
\newcommand{\calL}{{\mathcal{L}}}
\newcommand{\calM}{{\mathcal{M}}}
\newcommand{\calN}{{\mathcal{N}}}
\newcommand{\calO}{{\mathcal{O}}}
\newcommand{\calP}{{\mathcal{P}}}
\newcommand{\calQ}{{\mathcal{Q}}}
\newcommand{\calR}{{\mathcal{R}}}
\newcommand{\calS}{{\mathcal{S}}}
\newcommand{\calT}{{\mathcal{T}}}
\newcommand{\calU}{{\mathcal{U}}}
\newcommand{\calV}{{\mathcal{V}}}
\newcommand{\calW}{{\mathcal{W}}}
\newcommand{\calX}{{\mathcal{X}}}
\newcommand{\calY}{{\mathcal{Y}}}
\newcommand{\calZ}{{\mathcal{Z}}}

\newcommand{\bara}{{\bar{a}}}
\newcommand{\barb}{{\bar{b}}}
\newcommand{\barc}{{\bar{c}}}
\newcommand{\bard}{{\bar{d}}}
\newcommand{\bare}{{\bar{e}}}
\newcommand{\barf}{{\bar{f}}}
\newcommand{\barg}{{\bar{g}}}
\newcommand{\barh}{{\bar{h}}}
\newcommand{\bari}{{\bar{i}}}
\newcommand{\barj}{{\bar{j}}}
\newcommand{\bark}{{\bar{k}}}
\newcommand{\barl}{{\bar{l}}}
\newcommand{\barm}{{\bar{m}}}
\newcommand{\barn}{{\bar{n}}}
\newcommand{\baro}{{\bar{o}}}
\newcommand{\barp}{{\bar{p}}}
\newcommand{\barq}{{\bar{q}}}
\newcommand{\barr}{{\bar{r}}}
\newcommand{\bars}{{\bar{s}}}
\newcommand{\bart}{{\bar{t}}}
\newcommand{\baru}{{\bar{u}}}
\newcommand{\barv}{{\bar{v}}}
\newcommand{\barw}{{\bar{w}}}
\newcommand{\barx}{{\bar{x}}}
\newcommand{\bary}{{\bar{y}}}
\newcommand{\barz}{{\bar{z}}}
\newcommand{\barA}{{\bar{A}}}
\newcommand{\barB}{{\bar{B}}}
\newcommand{\barC}{{\bar{C}}}
\newcommand{\barD}{{\bar{D}}}
\newcommand{\barE}{{\bar{E}}}
\newcommand{\barF}{{\bar{F}}}
\newcommand{\barG}{{\bar{G}}}
\newcommand{\barH}{{\bar{H}}}
\newcommand{\barI}{{\bar{I}}}
\newcommand{\barJ}{{\bar{J}}}
\newcommand{\barK}{{\bar{K}}}
\newcommand{\barL}{{\bar{L}}}
\newcommand{\barM}{{\bar{M}}}
\newcommand{\barN}{{\bar{N}}}
\newcommand{\barO}{{\bar{O}}}
\newcommand{\barP}{{\bar{P}}}
\newcommand{\barQ}{{\bar{Q}}}
\newcommand{\barR}{{\bar{R}}}
\newcommand{\barS}{{\bar{S}}}
\newcommand{\barT}{{\bar{T}}}
\newcommand{\barU}{{\bar{U}}}
\newcommand{\barV}{{\bar{V}}}
\newcommand{\barW}{{\bar{W}}}
\newcommand{\barX}{{\bar{X}}}
\newcommand{\barY}{{\bar{Y}}}
\newcommand{\barZ}{{\bar{Z}}}

\newcommand{\hX}{\hat{X}}
\newcommand{\Ent}{\mathsf{Ent}}
\newcommand{\awarm}{{A_{\text{warm}}}}
\newcommand{\thetaLS}{{\widehat{\theta}^{\text{\rm LS}}}}

\newcommand{\jiao}[1]{\langle{#1}\rangle}
\newcommand{\gaht}{\textsc{GoodActionHypTest}\;}
\newcommand{\iaht}{\textsc{InitialActionHypTest}\;}
\newcommand{\true}{\textsf{True}\;}
\newcommand{\false}{\textsf{False}\;}

% \usepackage[capitalize,noabbrev]{cleveref}
% \crefname{lemma}{Lemma}{Lemmas}
% \Crefname{lemma}{Lemma}{Lemmas}
% \crefname{thm}{Theorem}{Theorems}
% \Crefname{thm}{Theorem}{Theorems}
% \Crefname{assumption}{Assumption}{Assumptions}
% \Crefname{inftheorem}{Informal Theorem}{Informal Theorems}
% \crefformat{equation}{(#2#1#3)}

% % if you use cleveref..
% \usepackage[capitalize,noabbrev]{cleveref}
% \crefname{lemma}{Lemma}{Lemmas}
% \crefname{proposition}{Proposition}{Propositions}
% \crefname{remark}{Remark}{Remarks}
% \crefname{corollary}{Corollary}{Corollaries}
% \crefname{definition}{Definition}{Definitions}
% \crefname{conjecture}{Conjecture}{Conjectures}
% \crefname{figure}{Fig.}{Figures}

% % \input{}

\makeatletter
\def\mathcolor#1#{\@mathcolor{#1}}
\def\@mathcolor#1#2#3{%
  \protect\leavevmode
  \begingroup
    \color#1{#2}#3%
  \endgroup
}
\makeatother

\definecolor{NDblue}{RGB}{12, 35, 64} % ND Blue
\definecolor{NDgold}{RGB}{174, 145, 66} % ND Metallic Gold

\hypersetup{
    colorlinks = true,    
    linkcolor = NDblue,    
    citecolor = NDgold,   
    urlcolor = NDblue,     
    filecolor = NDblue     
}


% \usepackage[textsize=tiny]{todonotes}
% \setlength{\parskip}{3pt}

% \usepackage[skins,theorems]{tcolorbox}
% %%%%%%%%%%%%%%%%%%%%%%%%%%%%%%%%
% % from Tengyang

% tweaking cleveref
\Crefformat{equation}{#2Eq.\;(#1)#3}

\Crefformat{figure}{#2Figure #1#3}
\Crefformat{assumption}{#2Assumption #1#3}
\Crefname{assumption}{Assumption}{Assumptions}

% Fix hyperref in section titles.
\usepackage{crossreftools}
\pdfstringdefDisableCommands{%
    \let\Cref\crtCref
    \let\cref\crtcref
}
\newcommand{\creftitle}[1]{\crtcref{#1}}

\usepackage{dsfont}
\usepackage{nicefrac}
% \usepackage[shortlabels]{enumitem}


\author[1,*]{Yue Huang}
\author[2,*]{Yanbo Wang}
\author[2,*]{Zixiang Xu}
\author[2]{Chujie Gao}
\author[4]{Siyuan Wu}
\author[4]{Jiayi Ye}
\author[2]{Xiuying Chen}
\author[3]{Pin-Yu Chen}
\author[1]{Xiangliang Zhang}

\affil[1]{University of Notre Dame}
\affil[2]{MBZUAI}
\affil[3]{IBM Research}
\affil[4]{Independent Researcher}

\affil[*]{Equal Contribution}

\correspondingauthor{xzhang33@nd.edu (Xiangliang Zhang)}



% \correspondingauthor{syyuan21@m.fudan.edu.cn, lovesnow@mail.ustc.edu.cn}


\newcommand{\fix}{\marginpar{FIX}}
\newcommand{\new}{\marginpar{NEW}}


\begin{abstract}
\textbf{Abstract:} Recent advances in Large Language Models (LLMs) have revolutionized generative systems, achieving excellent performance across diverse domains. Although these models perform well in controlled environments, their real-world applications frequently encounter inputs containing both essential and irrelevant details. Our investigation has revealed a critical vulnerability in LLMs, which we term Contextual Distraction Vulnerability (CDV). This phenomenon arises when models fail to maintain consistent performance on questions modified with semantically coherent but irrelevant context. To systematically investigate this vulnerability, we propose an efficient tree-based search methodology to automatically generate CDV examples. Our approach successfully generates CDV  examples across four datasets, causing an average performance degradation of approximately 45\% in state-of-the-art LLMs. To address this critical issue, we explore various mitigation strategies and find that post-targeted training approaches can effectively enhance model robustness against contextual distractions. Our findings highlight the fundamental nature of CDV as an ability-level challenge rather than a knowledge-level issue since models demonstrate the necessary knowledge by answering correctly in the absence of distractions. This calls the community's attention to address CDV during model development to ensure reliability.
Code is available at \href{https://github.com/wyf23187/LLM_CDV}{https://github.com/wyf23187/LLM\_CDV}.
\end{abstract}

\begin{document}
\maketitle


%!TEX root = gcn.tex
\section{Introduction}
Graphs, representing structural data and topology, are widely used across various domains, such as social networks and merchandising transactions.
Graph convolutional networks (GCN)~\cite{iclr/KipfW17} have significantly enhanced model training on these interconnected nodes.
However, these graphs often contain sensitive information that should not be leaked to untrusted parties.
For example, companies may analyze sensitive demographic and behavioral data about users for applications ranging from targeted advertising to personalized medicine.
Given the data-centric nature and analytical power of GCN training, addressing these privacy concerns is imperative.

Secure multi-party computation (MPC)~\cite{crypto/ChaumDG87,crypto/ChenC06,eurocrypt/CiampiRSW22} is a critical tool for privacy-preserving machine learning, enabling mutually distrustful parties to collaboratively train models with privacy protection over inputs and (intermediate) computations.
While research advances (\eg,~\cite{ccs/RatheeRKCGRS20,uss/NgC21,sp21/TanKTW,uss/WatsonWP22,icml/Keller022,ccs/ABY318,folkerts2023redsec}) support secure training on convolutional neural networks (CNNs) efficiently, private GCN training with MPC over graphs remains challenging.

Graph convolutional layers in GCNs involve multiplications with a (normalized) adjacency matrix containing $\numedge$ non-zero values in a $\numnode \times \numnode$ matrix for a graph with $\numnode$ nodes and $\numedge$ edges.
The graphs are typically sparse but large.
One could use the standard Beaver-triple-based protocol to securely perform these sparse matrix multiplications by treating graph convolution as ordinary dense matrix multiplication.
However, this approach incurs $O(\numnode^2)$ communication and memory costs due to computations on irrelevant nodes.
%
Integrating existing cryptographic advances, the initial effort of SecGNN~\cite{tsc/WangZJ23,nips/RanXLWQW23} requires heavy communication or computational overhead.
Recently, CoGNN~\cite{ccs/ZouLSLXX24} optimizes the overhead in terms of  horizontal data partitioning, proposing a semi-honest secure framework.
Research for secure GCN over vertical data  remains nascent.

Current MPC studies, for GCN or not, have primarily targeted settings where participants own different data samples, \ie, horizontally partitioned data~\cite{ccs/ZouLSLXX24}.
MPC specialized for scenarios where parties hold different types of features~\cite{tkde/LiuKZPHYOZY24,icml/CastigliaZ0KBP23,nips/Wang0ZLWL23} is rare.
This paper studies $2$-party secure GCN training for these vertical partition cases, where one party holds private graph topology (\eg, edges) while the other owns private node features.
For instance, LinkedIn holds private social relationships between users, while banks own users' private bank statements.
Such real-world graph structures underpin the relevance of our focus.
To our knowledge, no prior work tackles secure GCN training in this context, which is crucial for cross-silo collaboration.


To realize secure GCN over vertically split data, we tailor MPC protocols for sparse graph convolution, which fundamentally involves sparse (adjacency) matrix multiplication.
Recent studies have begun exploring MPC protocols for sparse matrix multiplication (SMM).
ROOM~\cite{ccs/SchoppmannG0P19}, a seminal work on SMM, requires foreknowledge of sparsity types: whether the input matrices are row-sparse or column-sparse.
Unfortunately, GCN typically trains on graphs with arbitrary sparsity, where nodes have varying degrees and no specific sparsity constraints.
Moreover, the adjacency matrix in GCN often contains a self-loop operation represented by adding the identity matrix, which is neither row- nor column-sparse.
Araki~\etal~\cite{ccs/Araki0OPRT21} avoid this limitation in their scalable, secure graph analysis work, yet it does not cover vertical partition.

% and related primitives
To bridge this gap, we propose a secure sparse matrix multiplication protocol, \osmm, achieving \emph{accurate, efficient, and secure GCN training over vertical data} for the first time.

\subsection{New Techniques for Sparse Matrices}
The cost of evaluating a GCN layer is dominated by SMM in the form of $\adjmat\feamat$, where $\adjmat$ is a sparse adjacency matrix of a (directed) graph $\graph$ and $\feamat$ is a dense matrix of node features.
For unrelated nodes, which often constitute a substantial portion, the element-wise products $0\cdot x$ are always zero.
Our efficient MPC design 
avoids unnecessary secure computation over unrelated nodes by focusing on computing non-zero results while concealing the sparse topology.
We achieve this~by:
1) decomposing the sparse matrix $\adjmat$ into a product of matrices (\S\ref{sec::sgc}), including permutation and binary diagonal matrices, that can \emph{faithfully} represent the original graph topology;
2) devising specialized protocols (\S\ref{sec::smm_protocol}) for efficiently multiplying the structured matrices while hiding sparsity topology.


 
\subsubsection{Sparse Matrix Decomposition}
We decompose adjacency matrix $\adjmat$ of $\graph$ into two bipartite graphs: one represented by sparse matrix $\adjout$, linking the out-degree nodes to edges, the other 
by sparse matrix $\adjin$,
linking edges to in-degree nodes.

%\ie, we decompose $\adjmat$ into $\adjout \adjin$, where $\adjout$ and $\adjin$ are sparse matrices representing these connections.
%linking out-degree nodes to edges and edges to in-degree nodes of $\graph$, respectively.

We then permute the columns of $\adjout$ and the rows of $\adjin$ so that the permuted matrices $\adjout'$ and $\adjin'$ have non-zero positions with \emph{monotonically non-decreasing} row and column indices.
A permutation $\sigma$ is used to preserve the edge topology, leading to an initial decomposition of $\adjmat = \adjout'\sigma \adjin'$.
This is further refined into a sequence of \emph{linear transformations}, 
which can be efficiently computed by our MPC protocols for 
\emph{oblivious permutation}
%($\Pi_{\ssp}$) 
and \emph{oblivious selection-multiplication}.
% ($\Pi_\SM$)
\iffalse
Our approach leverages bipartite graph representation and the monotonicity of non-zero positions to decompose a general sparse matrix into linear transformations, enhancing the efficiency of our MPC protocols.
\fi
Our decomposition approach is not limited to GCNs but also general~SMM 
by 
%simply 
treating them 
as adjacency matrices.
%of a graph.
%Since any sparse matrix can be viewed 

%allowing the same technique to be applied.

 
\subsubsection{New Protocols for Linear Transformations}
\emph{Oblivious permutation} (OP) is a two-party protocol taking a private permutation $\sigma$ and a private vector $\xvec$ from the two parties, respectively, and generating a secret share $\l\sigma \xvec\r$ between them.
Our OP protocol employs correlated randomnesses generated in an input-independent offline phase to mask $\sigma$ and $\xvec$ for secure computations on intermediate results, requiring only $1$ round in the online phase (\cf, $\ge 2$ in previous works~\cite{ccs/AsharovHIKNPTT22, ccs/Araki0OPRT21}).

Another crucial two-party protocol in our work is \emph{oblivious selection-multiplication} (OSM).
It takes a private bit~$s$ from a party and secret share $\l x\r$ of an arithmetic number~$x$ owned by the two parties as input and generates secret share $\l sx\r$.
%between them.
%Like our OP protocol, o
Our $1$-round OSM protocol also uses pre-computed randomnesses to mask $s$ and $x$.
%for secure computations.
Compared to the Beaver-triple-based~\cite{crypto/Beaver91a} and oblivious-transfer (OT)-based approaches~\cite{pkc/Tzeng02}, our protocol saves ${\sim}50\%$ of online communication while having the same offline communication and round complexities.

By decomposing the sparse matrix into linear transformations and applying our specialized protocols, our \osmm protocol
%($\prosmm$) 
reduces the complexity of evaluating $\numnode \times \numnode$ sparse matrices with $\numedge$ non-zero values from $O(\numnode^2)$ to $O(\numedge)$.

%(\S\ref{sec::secgcn})
\subsection{\cgnn: Secure GCN made Efficient}
Supported by our new sparsity techniques, we build \cgnn, 
a two-party computation (2PC) framework for GCN inference and training over vertical
%ly split
data.
Our contributions include:

1) We are the first to explore sparsity over vertically split, secret-shared data in MPC, enabling decompositions of sparse matrices with arbitrary sparsity and isolating computations that can be performed in plaintext without sacrificing privacy.

2) We propose two efficient $2$PC primitives for OP and OSM, both optimally single-round.
Combined with our sparse matrix decomposition approach, our \osmm protocol ($\prosmm$) achieves constant-round communication costs of $O(\numedge)$, reducing memory requirements and avoiding out-of-memory errors for large matrices.
In practice, it saves $99\%+$ communication
%(Table~\ref{table:comm_smm}) 
and reduces ${\sim}72\%$ memory usage over large $(5000\times5000)$ matrices compared with using Beaver triples.
%(Table~\ref{table:mem_smm_sparse}) ${\sim}16\%$-

3) We build an end-to-end secure GCN framework for inference and training over vertically split data, maintaining accuracy on par with plaintext computations.
We will open-source our evaluation code for research and deployment.

To evaluate the performance of $\cgnn$, we conducted extensive experiments over three standard graph datasets (Cora~\cite{aim/SenNBGGE08}, Citeseer~\cite{dl/GilesBL98}, and Pubmed~\cite{ijcnlp/DernoncourtL17}),
reporting communication, memory usage, accuracy, and running time under varying network conditions, along with an ablation study with or without \osmm.
Below, we highlight our key achievements.

\textit{Communication (\S\ref{sec::comm_compare_gcn}).}
$\cgnn$ saves communication by $50$-$80\%$.
(\cf,~CoGNN~\cite{ccs/KotiKPG24}, OblivGNN~\cite{uss/XuL0AYY24}).

\textit{Memory usage (\S\ref{sec::smmmemory}).}
\cgnn alleviates out-of-memory problems of using %the standard 
Beaver-triples~\cite{crypto/Beaver91a} for large datasets.

\textit{Accuracy (\S\ref{sec::acc_compare_gcn}).}
$\cgnn$ achieves inference and training accuracy comparable to plaintext counterparts.
%training accuracy $\{76\%$, $65.1\%$, $75.2\%\}$ comparable to $\{75.7\%$, $65.4\%$, $74.5\%\}$ in plaintext.

{\textit{Computational efficiency (\S\ref{sec::time_net}).}} 
%If the network is worse in bandwidth and better in latency, $\cgnn$ shows more benefits.
$\cgnn$ is faster by $6$-$45\%$ in inference and $28$-$95\%$ in training across various networks and excels in narrow-bandwidth and low-latency~ones.

{\textit{Impact of \osmm (\S\ref{sec:ablation}).}}
Our \osmm protocol shows a $10$-$42\times$ speed-up for $5000\times 5000$ matrices and saves $10$-2$1\%$ memory for ``small'' datasets and up to $90\%$+ for larger ones.

\section{Related Works}

\textbf{Enhancing LLMs' Theory of Mind.} There has been systematic evaluation that revealed LLMs' limitations in achieving robust Theory of Mind inference \citep{ullman2023large, shapira2023clever}. To enhance LLMs' Theory of Mind capacity, recent works have proposed various prompting techniques. For instance, SimToM \citep{wilf2023think} encourages LLMs to adopt perspective-taking, PercepToM \citep{jung2024perceptions} improves perception-to-belief inference by extracting relevant contextual details, and \citet{huang2024notion} utilize an LLM as a world model to track environmental changes and refine prompts. Explicit symbolic modules also seem to improve LLM's accuracy through dynamic updates based on inputs. Specifically, TimeToM \citep{hou2024timetom} constructs a temporal reasoning framework to support inference, while SymbolicToM \citep{sclar2023minding} uses graphical representations to track characters' beliefs. Additionally, \citet{wagner2024mind} investigates ToM's necessity and the level of recursion required for specific tasks. However, these approaches continue to exhibit systematic errors in long contexts, complex behaviors, and recursive reasoning due to inherent limitations in inference and modeling \citep{jin2024mmtom,shi2024muma}. Most of them rely on domain-specific designs, lacking open-endedness.


\textbf{Model-based Theory of Mind inference.} Model-based Theory of Mind inference, in particular, Bayesian inverse planning (BIP) \citep{baker2009action,ullman2009help,baker2017rational,zhi2020online}, explicitly constructs representations of agents' mental states and how mental states guide agents' behavior via Bayesian Theory of Mind (BToM) models. These methods can reverse engineer human ToM inference in simple domains \citep[e.g.,][]{baker2017rational,netanyahu2021phase,shu2021agent}. Recent works have proposed to combine BIP with LLMs to achieve robust ToM inference in more realistic settings \citep{ying2023neuro, jin2024mmtom, shi2024muma}. However, these methods require manual specification of the BToM models as well as rigid, domain-specific implementations of Bayesian inference, limiting their adaptability to open-ended scenarios. To overcome this limitation, we propose \ours, a method capable of automatically modeling mental variables across diverse conditions and conducting automated BIP without domain-specific knowledge or implementations.


\begin{figure*}[ht]
  \centering
  \includegraphics[width=\linewidth]{figures/benchmarks_and_models.pdf}
    \vspace{-15pt}
  \caption{Examples questions (top panels) and the necessary Bayesian Theory of Mind (BToM) model for Bayesian inverse planning (bottom panels) in diverse Theory of Mind benchmarks. \ours aims to answer any Theory of Mind question in a variety of benchmarks, encompassing different mental variables, observable contexts, numbers of agents, the presence or absence of utterances, wording styles, and modalities. It proposes and iteratively adjusts an appropriate BToM and conducts automated Bayesian inverse planning based on the model.
  There can be more types of questions/models in each benchmark beyond the examples shown in this figure.}
  \label{fig:benchmarks_and_models}
  %\vspace{-0.75em}
  \vspace{-10pt}
\end{figure*}



\textbf{Automated Modeling with LLMs.} There has been an increasing interest in integrating LLMs with inductive reasoning and probabilistic inference for automated modeling. \citet{piriyakulkij2024doing} combine LLMs with Sequential Monte Carlo to perform probabilistic inference about underlying rules. Iterative hypothesis refinement techniques \citep{qiu2023phenomenal} further enhance LLM-based inductive reasoning by iteratively proposing, selecting, and refining textual hypotheses of rules. Beyond rule-based hypotheses, \citet{wang2023hypothesis} prompt LLMs to generate natural language hypotheses that are then implemented as verifiable programs, while \citet{li2024automated} propose a method in which LLMs construct, critique, and refine statistical models represented as probabilistic programs for data modeling. \citet{cross2024hypothetical} leverage LLMs to propose and evaluate agent strategies for multi-agent planning but do not specifically infer individual mental variables. Our method also aims to achieve automated modeling with LLMs. Unlike prior works, we propose a novel automated model discovery approach for Bayesian inverse planning, where the objective is to confidently infer any mental variable given any context via constructing a suitable Bayesian Theory of Mind model.
\vspace{-5pt}
\section{Method}
\label{sec:method}
\section{Overview}

\revision{In this section, we first explain the foundational concept of Hausdorff distance-based penetration depth algorithms, which are essential for understanding our method (Sec.~\ref{sec:preliminary}).
We then provide a brief overview of our proposed RT-based penetration depth algorithm (Sec.~\ref{subsec:algo_overview}).}



\section{Preliminaries }
\label{sec:Preliminaries}

% Before we introduce our method, we first overview the important basics of 3D dynamic human modeling with Gaussian splatting. Then, we discuss the diffusion-based 3d generation techniques, and how they can be applied to human modeling.
% \ZY{I stopp here. TBC.}
% \subsection{Dynamic human modeling with Gaussian splatting}
\subsection{3D Gaussian Splatting}
3D Gaussian splatting~\cite{kerbl3Dgaussians} is an explicit scene representation that allows high-quality real-time rendering. The given scene is represented by a set of static 3D Gaussians, which are parameterized as follows: Gaussian center $x\in {\mathbb{R}^3}$, color $c\in {\mathbb{R}^3}$, opacity $\alpha\in {\mathbb{R}}$, spatial rotation in the form of quaternion $q\in {\mathbb{R}^4}$, and scaling factor $s\in {\mathbb{R}^3}$. Given these properties, the rendering process is represented as:
\begin{equation}
  I = Splatting(x, c, s, \alpha, q, r),
  \label{eq:splattingGA}
\end{equation}
where $I$ is the rendered image, $r$ is a set of query rays crossing the scene, and $Splatting(\cdot)$ is a differentiable rendering process. We refer readers to Kerbl et al.'s paper~\cite{kerbl3Dgaussians} for the details of Gaussian splatting. 



% \ZY{I would suggest move this part to the method part.}
% GaissianAvatar is a dynamic human generation model based on Gaussian splitting. Given a sequence of RGB images, this method utilizes fitted SMPLs and sampled points on its surface to obtain a pose-dependent feature map by a pose encoder. The pose-dependent features and a geometry feature are fed in a Gaussian decoder, which is employed to establish a functional mapping from the underlying geometry of the human form to diverse attributes of 3D Gaussians on the canonical surfaces. The parameter prediction process is articulated as follows:
% \begin{equation}
%   (\Delta x,c,s)=G_{\theta}(S+P),
%   \label{eq:gaussiandecoder}
% \end{equation}
%  where $G_{\theta}$ represents the Gaussian decoder, and $(S+P)$ is the multiplication of geometry feature S and pose feature P. Instead of optimizing all attributes of Gaussian, this decoder predicts 3D positional offset $\Delta{x} \in {\mathbb{R}^3}$, color $c\in\mathbb{R}^3$, and 3D scaling factor $ s\in\mathbb{R}^3$. To enhance geometry reconstruction accuracy, the opacity $\alpha$ and 3D rotation $q$ are set to fixed values of $1$ and $(1,0,0,0)$ respectively.
 
%  To render the canonical avatar in observation space, we seamlessly combine the Linear Blend Skinning function with the Gaussian Splatting~\cite{kerbl3Dgaussians} rendering process: 
% \begin{equation}
%   I_{\theta}=Splatting(x_o,Q,d),
%   \label{eq:splatting}
% \end{equation}
% \begin{equation}
%   x_o = T_{lbs}(x_c,p,w),
%   \label{eq:LBS}
% \end{equation}
% where $I_{\theta}$ represents the final rendered image, and the canonical Gaussian position $x_c$ is the sum of the initial position $x$ and the predicted offset $\Delta x$. The LBS function $T_{lbs}$ applies the SMPL skeleton pose $p$ and blending weights $w$ to deform $x_c$ into observation space as $x_o$. $Q$ denotes the remaining attributes of the Gaussians. With the rendering process, they can now reposition these canonical 3D Gaussians into the observation space.



\subsection{Score Distillation Sampling}
Score Distillation Sampling (SDS)~\cite{poole2022dreamfusion} builds a bridge between diffusion models and 3D representations. In SDS, the noised input is denoised in one time-step, and the difference between added noise and predicted noise is considered SDS loss, expressed as:

% \begin{equation}
%   \mathcal{L}_{SDS}(I_{\Phi}) \triangleq E_{t,\epsilon}[w(t)(\epsilon_{\phi}(z_t,y,t)-\epsilon)\frac{\partial I_{\Phi}}{\partial\Phi}],
%   \label{eq:SDSObserv}
% \end{equation}
\begin{equation}
    \mathcal{L}_{\text{SDS}}(I_{\Phi}) \triangleq \mathbb{E}_{t,\epsilon} \left[ w(t) \left( \epsilon_{\phi}(z_t, y, t) - \epsilon \right) \frac{\partial I_{\Phi}}{\partial \Phi} \right],
  \label{eq:SDSObservGA}
\end{equation}
where the input $I_{\Phi}$ represents a rendered image from a 3D representation, such as 3D Gaussians, with optimizable parameters $\Phi$. $\epsilon_{\phi}$ corresponds to the predicted noise of diffusion networks, which is produced by incorporating the noise image $z_t$ as input and conditioning it with a text or image $y$ at timestep $t$. The noise image $z_t$ is derived by introducing noise $\epsilon$ into $I_{\Phi}$ at timestep $t$. The loss is weighted by the diffusion scheduler $w(t)$. 
% \vspace{-3mm}

\subsection{Overview of the RTPD Algorithm}\label{subsec:algo_overview}
Fig.~\ref{fig:Overview} presents an overview of our RTPD algorithm.
It is grounded in the Hausdorff distance-based penetration depth calculation method (Sec.~\ref{sec:preliminary}).
%, similar to that of Tang et al.~\shortcite{SIG09HIST}.
The process consists of two primary phases: penetration surface extraction and Hausdorff distance calculation.
We leverage the RTX platform's capabilities to accelerate both of these steps.

\begin{figure*}[t]
    \centering
    \includegraphics[width=0.8\textwidth]{Image/overview.pdf}
    \caption{The overview of RT-based penetration depth calculation algorithm overview}
    \label{fig:Overview}
\end{figure*}

The penetration surface extraction phase focuses on identifying the overlapped region between two objects.
\revision{The penetration surface is defined as a set of polygons from one object, where at least one of its vertices lies within the other object. 
Note that in our work, we focus on triangles rather than general polygons, as they are processed most efficiently on the RTX platform.}
To facilitate this extraction, we introduce a ray-tracing-based \revision{Point-in-Polyhedron} test (RT-PIP), significantly accelerated through the use of RT cores (Sec.~\ref{sec:RT-PIP}).
This test capitalizes on the ray-surface intersection capabilities of the RTX platform.
%
Initially, a Geometry Acceleration Structure (GAS) is generated for each object, as required by the RTX platform.
The RT-PIP module takes the GAS of one object (e.g., $GAS_{A}$) and the point set of the other object (e.g., $P_{B}$).
It outputs a set of points (e.g., $P_{\partial B}$) representing the penetration region, indicating their location inside the opposing object.
Subsequently, a penetration surface (e.g., $\partial B$) is constructed using this point set (e.g., $P_{\partial B}$) (Sec.~\ref{subsec:surfaceGen}).
%
The generated penetration surfaces (e.g., $\partial A$ and $\partial B$) are then forwarded to the next step. 

The Hausdorff distance calculation phase utilizes the ray-surface intersection test of the RTX platform (Sec.~\ref{sec:RT-Hausdorff}) to compute the Hausdorff distance between two objects.
We introduce a novel Ray-Tracing-based Hausdorff DISTance algorithm, RT-HDIST.
It begins by generating GAS for the two penetration surfaces, $P_{\partial A}$ and $P_{\partial B}$, derived from the preceding step.
RT-HDIST processes the GAS of a penetration surface (e.g., $GAS_{\partial A}$) alongside the point set of the other penetration surface (e.g., $P_{\partial B}$) to compute the penetration depth between them.
The algorithm operates bidirectionally, considering both directions ($\partial A \to \partial B$ and $\partial B \to \partial A$).
The final penetration depth between the two objects, A and B, is determined by selecting the larger value from these two directional computations.

%In the Hausdorff distance calculation step, we compute the Hausdorff distance between given two objects using a ray-surface-intersection test. (Sec.~\ref{sec:RT-Hausdorff}) Initially, we construct the GAS for both $\partial A$ and $\partial B$ to utilize the RT-core effectively. The RT-based Hausdorff distance algorithms then determine the Hausdorff distance by processing the GAS of one object (e.g. $GAS_{\partial A}$) and set of the vertices of the other (e.g. $P_{\partial B}$). Following the Hausdorff distance definition (Eq.~\ref{equation:hausdorff_definition}), we compute the Hausdorff distance to both directions ($\partial A \to \partial B$) and ($\partial B \to \partial A$). As a result, the bigger one is the final Hausdorff distance, and also it is the penetration depth between input object $A$ and $B$.


%the proposed RT-based penetration depth calculation pipeline.
%Our proposed methods adopt Tang's Hausdorff-based penetration depth methods~\cite{SIG09HIST}. The pipeline is divided into the penetration surface extraction step and the Hausdorff distance calculation between the penetration surface steps. However, since Tang's approach is not suitable for the RT platform in detail, we modified and applied it with appropriate methods.

%The penetration surface extraction step is extracting overlapped surfaces on other objects. To utilize the RT core, we use the ray-intersection-based PIP(Point-In-Polygon) algorithms instead of collision detection between two objects which Tang et al.~\cite{SIG09HIST} used. (Sec.~\ref{sec:RT-PIP})
%RT core-based PIP test uses a ray-surface intersection test. For purpose this, we generate the GAS(Geometry Acceleration Structure) for each object. RT core-based PIP test takes the GAS of one object (e.g. $GAS_{A}$) and a set of vertex of another one (e.g. $P_{B}$). Then this computes the penetrated vertex set of another one (e.g. $P_{\partial B}$). To calculate the Hausdorff distance, these vertex sets change to objects constructed by penetrated surface (e.g. $\partial B$). Finally, the two generated overlapped surface objects $\partial A$ and $\partial B$ are used in the Hausdorff distance calculation step.

Our goal is to increase the robustness of T2I models, particularly with rare or unseen concepts, which they struggle to generate. To do so, we investigate a retrieval-augmented generation approach, through which we dynamically select images that can provide the model with missing visual cues. Importantly, we focus on models that were not trained for RAG, and show that existing image conditioning tools can be leveraged to support RAG post-hoc.
As depicted in \cref{fig:overview}, given a text prompt and a T2I generative model, we start by generating an image with the given prompt. Then, we query a VLM with the image, and ask it to decide if the image matches the prompt. If it does not, we aim to retrieve images representing the concepts that are missing from the image, and provide them as additional context to the model to guide it toward better alignment with the prompt.
In the following sections, we describe our method by answering key questions:
(1) How do we know which images to retrieve? 
(2) How can we retrieve the required images? 
and (3) How can we use the retrieved images for unknown concept generation?
By answering these questions, we achieve our goal of generating new concepts that the model struggles to generate on its own.

\vspace{-3pt}
\subsection{Which images to retrieve?}
The amount of images we can pass to a model is limited, hence we need to decide which images to pass as references to guide the generation of a base model. As T2I models are already capable of generating many concepts successfully, an efficient strategy would be passing only concepts they struggle to generate as references, and not all the concepts in a prompt.
To find the challenging concepts,
we utilize a VLM and apply a step-by-step method, as depicted in the bottom part of \cref{fig:overview}. First, we generate an initial image with a T2I model. Then, we provide the VLM with the initial prompt and image, and ask it if they match. If not, we ask the VLM to identify missing concepts and
focus on content and style, since these are easy to convey through visual cues.
As demonstrated in \cref{tab:ablations}, empirical experiments show that image retrieval from detailed image captions yields better results than retrieval from brief, generic concept descriptions.
Therefore, after identifying the missing concepts, we ask the VLM to suggest detailed image captions for images that describe each of the concepts. 

\vspace{-4pt}
\subsubsection{Error Handling}
\label{subsec:err_hand}

The VLM may sometimes fail to identify the missing concepts in an image, and will respond that it is ``unable to respond''. In these rare cases, we allow up to 3 query repetitions, while increasing the query temperature in each repetition. Increasing the temperature allows for more diverse responses by encouraging the model to sample less probable words.
In most cases, using our suggested step-by-step method yields better results than retrieving images directly from the given prompt (see 
\cref{subsec:ablations}).
However, if the VLM still fails to identify the missing concepts after multiple attempts, we fall back to retrieving images directly from the prompt, as it usually means the VLM does not know what is the meaning of the prompt.

The used prompts can be found in \cref{app:prompts}.
Next, we turn to retrieve images based on the acquired image captions.

\vspace{-3pt}
\subsection{How to retrieve the required images?}

Given $n$ image captions, our goal is to retrieve the images that are most similar to these captions from a dataset. 
To retrieve images matching a given image caption, we compare the caption to all the images in the dataset using a text-image similarity metric and retrieve the top $k$ most similar images.
Text-to-image retrieval is an active research field~\cite{radford2021learning, zhai2023sigmoid, ray2024cola, vendrowinquire}, where no single method is perfect.
Retrieval is especially hard when the dataset does not contain an exact match to the query \cite{biswas2024efficient} or when the task is fine-grained retrieval, that depends on subtle details~\cite{wei2022fine}.
Hence, a common retrieval workflow is to first retrieve image candidates using pre-computed embeddings, and then re-rank the retrieved candidates using a different, often more expensive but accurate, method \cite{vendrowinquire}.
Following this workflow, we experimented with cosine similarity over different embeddings, and with multiple re-ranking methods of reference candidates.
Although re-ranking sometimes yields better results compared to simply using cosine similarity between CLIP~\cite{radford2021learning} embeddings, the difference was not significant in most of our experiments. Therefore, for simplicity, we use cosine similarity between CLIP embeddings as our similarity metric (see \cref{tab:sim_metrics}, \cref{subsec:ablations} for more details about our experiments with different similarity metrics).

\vspace{-3pt}
\subsection{How to use the retrieved images?}
Putting it all together, after retrieving relevant images, all that is left to do is to use them as context so they are beneficial for the model.
We experimented with two types of models; models that are trained to receive images as input in addition to text and have ICL capabilities (e.g., OmniGen~\cite{xiao2024omnigen}), and T2I models augmented with an image encoder in post-training (e.g., SDXL~\cite{podellsdxl} with IP-adapter~\cite{ye2023ip}).
As the first model type has ICL capabilities, we can supply the retrieved images as examples that it can learn from, by adjusting the original prompt.
Although the second model type lacks true ICL capabilities, it offers image-based control functionalities, which we can leverage for applying RAG over it with our method.
Hence, for both model types, we augment the input prompt to contain a reference of the retrieved images as examples.
Formally, given a prompt $p$, $n$ concepts, and $k$ compatible images for each concept, we use the following template to create a new prompt:
``According to these examples of 
$\mathord{<}c_1\mathord{>:<}img_{1,1}\mathord{>}, ... , \mathord{<}img_{1,k}\mathord{>}, ... , \mathord{<}c_n\mathord{>:<}img_{n,1}\mathord{>}, ... , $
$\mathord{<}img_{n,k}\mathord{>}$,
generate $\mathord{<}p\mathord{>}$'', 
where $c_i$ for $i\in{[1,n]}$ is a compatible image caption of the image $\mathord{<}img_{i,j}\mathord{>},  j\in{[1,k]}$. 

This prompt allows models to learn missing concepts from the images, guiding them to generate the required result. 

\textbf{Personalized Generation}: 
For models that support multiple input images, we can apply our method for personalized generation as well, to generate rare concept combinations with personal concepts. In this case, we use one image for personal content, and 1+ other reference images for missing concepts. For example, given an image of a specific cat, we can generate diverse images of it, ranging from a mug featuring the cat to a lego of it or atypical situations like the cat writing code or teaching a classroom of dogs (\cref{fig:personalization}).
\vspace{-2pt}
\begin{figure}[htp]
  \centering
   \includegraphics[width=\linewidth]{Assets/personalization.pdf}
   \caption{\textbf{Personalized generation example.}
   \emph{ImageRAG} can work in parallel with personalization methods and enhance their capabilities. For example, although OmniGen can generate images of a subject based on an image, it struggles to generate some concepts. Using references retrieved by our method, it can generate the required result.
}
   \label{fig:personalization}\vspace{-10pt}
\end{figure}



\section{Experiments}
\seclabel{experiments}
Our experiments are designed to test a) the extent to which open loop execution is an issue for precise mobile manipulation tasks, b) how effective are blind proprioceptive correction techniques, c) do object detectors and point trackers perform reliably enough in wrist camera images for reliable control, d) is occlusion by the end-effector an issue and how effectively can it be mitigated through the use of video in-painting models, and e) how does our proposed \name methodology compare to large-scale imitation learning? 


\subsection{Tasks and Experimental Setup}
We work with the Stretch RE2 robot. Stretch RE2 is a commodity mobile manipulator with a 5DOF arm mounted on top of a non-holomonic base. We upgrade the robot to use the Dex Wrist 3, which has an eye-in-hand RGB-D camera (Intel D405). 
We consider 3 task families for a total
of 6 different tasks: a) holding a knob to pull open a cabinet or drawer, b) holding a
handle to pull open a cabinet, and c) pushing on objects (light buttons, books
in a book shelf, and light switches). Our focus is on generalization. {\it
Therefore, we exclusively test on previously unseen instances, not used during
development in any way.} 
\figref{tasks} shows the instances that we test on. 

All tasks involve some precise manipulation, followed by execution of a motion
primitive. {\bf For the pushing tasks}, the precise motion is to get the
end-effector exactly at the indicated point and the motion primitive is to push
in the direction perpendicular to the surface and retract the end-effector 
upon contact. The robot is positioned such
that the target position is within the field of view of the wrist camera. A user
selects the point of pushing via a mouse click on the wrist camera image. The
goal is to push at the indicated location. Success is determined by whether the
push results in the desired outcome (light turns on / off or book gets pushed in). 
The original rubber gripper bends upon contact, we use a rigid known tool
that sticks out a bit. We take the geometry of the tool into account while servoing.

{\bf For the opening articulated object tasks}, the precise manipulation is grasping the
knob / handle, while the motion primitive is the whole-body motion that opens
the cupboard. Computing and executing this full body motion is difficult. We
adopt the modular approach to opening articulated objects (MOSART) from Gupta \etal~\cite{gupta2024opening} and invoke it
after the gripper has been placed around the knob / handle. The whole tasks 
starts out with the robot about 1.5m way from the target object, with the 
target object in view
from robot's head mounted camera. We use MOSART to compute articulation
parameters and convey the robot to a pre-grasp
location with the target handle in view of the wrist camera. At this point,
\name (or baseline) is used to center the gripper around the knob / handle, 
before resuming MOSART: extending the gripper till contact, close the gripper, and play rest of the predicted motion plan. Success is 
determined by whether the cabinet opens by more than $60^\circ$
or the drawer is pulled out by more than $24cm$, similar to the criteria used in \cite{gupta2024opening}.


For the precise manipulation part, all baselines consume the current and
previous RGB-D images from the wrist camera and output full body motor
commands.

% % Please add the following required packages to your document preamble:
% % \usepackage{graphicx}
% \begin{table*}[!ht]
% \centering
% \caption{}
% \label{tab:my-table}
% \resizebox{\textwidth}{!}{%
% \begin{tabular}{lcccccc}
% \toprule
%  & \multicolumn{2}{c}{ours} & \multicolumn{2}{c}{Gurobi} & \multicolumn{2}{c}{MOSEK} \\
%  & \multicolumn{1}{l}{time (s)} & \multicolumn{1}{l}{optimality gap (\%)} & \multicolumn{1}{l}{time (s)} & \multicolumn{1}{l}{optimality gap (\%)} & \multicolumn{1}{l}{time (s)} & \multicolumn{1}{l}{optimality gap (\%)} \\ \hline
% \begin{tabular}[c]{@{}l@{}}Linear Regression\\ Synthetic \\ (n=16000, p=16000)\end{tabular} & 57 & 0.0 & 3351 & - & 2148 & - \\ \hline
% \begin{tabular}[c]{@{}l@{}}Linear Regression\\ Cancer Drug Response\\ (n=822, p=2300)\end{tabular} & 47 & 0.0 & 1800 & 0.31 & 212 & 0.0 \\ \hline
% \begin{tabular}[c]{@{}l@{}}Logistic Regression\\ Synthetic\\ (n=16000, p=16000)\end{tabular} & 271 & 0.0 & N/A & N/A & 1800 & - \\ \hline
% \begin{tabular}[c]{@{}l@{}}Logistic Regression\\ Dorothea\\ (n=1150, p=91598)\end{tabular} & 62 & 0.0 & N/A & N/A & 600 & 0.0 \\
% \bottomrule
% \end{tabular}%
% }
% \end{table*}

% Please add the following required packages to your document preamble:
% \usepackage{multirow}
% \usepackage{graphicx}
\begin{table*}[]
\centering
\caption{Certifying optimality on large-scale and real-world datasets.}
\vspace{2mm}
\label{tab:my-table}
\resizebox{\textwidth}{!}{%
\begin{tabular}{llcccccc}
\toprule
 &  & \multicolumn{2}{c}{ours} & \multicolumn{2}{c}{Gurobi} & \multicolumn{2}{c}{MOSEK} \\
 &  & time (s) & opt. gap (\%) & time (s) & opt. gap (\%) & time (s) & opt. gap (\%) \\ \hline
\multirow{2}{*}{Linear Regression} & \begin{tabular}[c]{@{}l@{}}synthetic ($k=10, M=2$)\\ (n=16k, p=16k, seed=0)\end{tabular} & 79 & 0.0 & 1800 & - & 1915 & - \\ \cline{2-8}
 & \begin{tabular}[c]{@{}l@{}}Cancer Drug Response ($k=5, M=5$)\\ (n=822, p=2300)\end{tabular} & 41 & 0.0 & 1800 & 0.89 & 188 & 0.0 \\ \hline
\multirow{2}{*}{Logistic Regression} & \begin{tabular}[c]{@{}l@{}}Synthetic ($k=10, M=2$)\\ (n=16k, p=16k, seed=0)\end{tabular} & 626 & 0.0 & N/A & N/A & 2446 & - \\ \cline{2-8}
 & \begin{tabular}[c]{@{}l@{}}DOROTHEA ($k=15, M=2$)\\ (n=1150, p=91598)\end{tabular} & 91 & 0.0 & N/A & N/A & 634 & 0.0 \\
 \bottomrule
\end{tabular}%
}
% \vspace{-3mm}
\end{table*}

\begin{figure*}
\insertW{1.0}{figures/figure_6_cropped_brighten.pdf}
\caption{{\bf Comparison of \name with the open loop (eye-in-hand) baseline} for opening a cabinet with a knob. Slight errors in getting to the target cause the end-effector to slip off, leading to failure for the baseline, where as our method is able to successfully complete the task.}
\figlabel{rollout}
\end{figure*}

\begin{table}
\setlength{\tabcolsep}{8pt}
  \centering
  \resizebox{\linewidth}{!}{
  \begin{tabular}{lcccg}
  \toprule
                              & \multicolumn{2}{c}{\bf Knobs} & \bf Handle & \bf \multirow{2}{*}{\bf Total} \\
                              \cmidrule(lr){2-3} \cmidrule(lr){4-4}
                              & \bf Cabinets & \bf Drawer & \bf Cabinets & \\
  \midrule
  RUM~\cite{etukuru2024robot}  & 0/3    & 1/4         & 1/3         & 2/10 \\
  \name (Ours) & 2/3    & 2/4         & 3/3     &  7/10 \\
  \bottomrule
  \end{tabular}}
  \caption{Comparison of \name \vs RUM~\cite{etukuru2024robot}, a recent large-scale end-to-end imitation learning method trained on 1200 demos for opening cabinets and 525 demos for opening drawers across 40 different environments. Our evaluation spans objects from three environments across two buildings.}
  \tablelabel{rum}
\end{table}

\subsection{Baselines}
We compare against three other methods for the precise manipulation part of
these tasks. 
\subsubsection{Open Loop (Eye-in-Hand)} To assess the precision requirements of
the tasks and to set it in context with the manipulation capabilities of the
robot platform, this baseline uses open loop execution starting from estimates
for the 3D target position from the first wrist camera image.
\subsubsection{MOSART~\cite{gupta2024opening}}
The recent modular system for opening cabinets and drawers~\cite{gupta2024opening}
reports impressive performance with open-loop control (using the head camera from 1.5m away), combined with proprioception-based feedback to 
compensate for errors in perception and control when interacting with handles. 
We test if such correction is also sufficient for interacting with knobs. Note 
that such correction is not possible for the smaller buttons and pliable books.

\subsubsection{\name (no inpainting)} To understand how much of an issue
occlusion due to the end-effector is during manipulation, we ablate the use of
inpainting. %

\subsubsection{Robot Utility Models (RUM)~\cite{etukuru2024robot}}
For the opening articulated object tasks, we also compare to Robot Utility Models (RUM), 
a closed-loop imitation learning method recently proposed by Etukuru et al. \cite{etukuru2024robot}.
RUM is trained on a substantial dataset comprising expert demonstrations, including 
1,200 instances of cabinet opening and 525 of drawer opening, gathered from roughly 
40 different environments.
This dataset stands as the most extensive imitation 
learning dataset for articulated object manipulation to date, establishing RUM as a 
strong baseline for our evaluation.

Similar to our method, we use MOSART to compute articulation
parameters and convey the robot to a pre-grasp location
with the target handle in view of the wrist camera.
One of the assumptions of RUM is a good view of the handle.
To benefit RUM, we try out three different heights of the wrist camera,
and \textit{report the best result for RUM.}

\begin{figure*}
\insertW{1.0}{figures/figure_9_cropped_brighten.pdf}
\caption{{\bf \name \vs open loop (eye-in-hand) baseline for pushing on user-clicked points}. Slight errors in getting to the target cause failure, where as \name successfully turns the lights off. Note the quality of CoTracker's track ({\color{blue} blue dot}).}
\figlabel{rollout_v2}
\end{figure*}

\begin{figure*}
\insertW{1.0}{figures/figure_5_v2_cropped_brighten.pdf}
\caption{{\bf Comparison of \name with and without inpainting}. Erroneous detection without inpainting causes execution to fail, where as with inpainting the target is correctly detected leading to a successful grasp and a successful execution.}
\figlabel{rollouts2}
\end{figure*}


\subsection{Results}
\tableref{results} presents results from our experiments. 
Our training-free approach \name successfully 
solves over 85\% of task instances that we test on.
As noted, all these
tests were conducted on unseen object instances in unseen
environments that were not used for development in any way. We discuss our key
experimental findings below.

\subsubsection{Closing the loop is necessary for these precise tasks} 
While the proprioception-based strategies proposed in MOSART~\cite{gupta2024opening}
work out for handles, they are inadequate for targets like knobs and just
don't work for tasks like pushing buttons. Using estimates from the wrist
camera is better, but open loop execution still fails for knobs and pushing
buttons. 

\subsubsection{Vision models work reasonably well even on wrist camera images}
Inpainting works well on wrist camera images (see \figref{occlusion} and \figref{inpainting}).
Closing the loop using feedback from vision detectors and point trackers on
wrist camera images also work well, particularly when we use in-painted images.
See some examples detections and point tracks in \figref{rollout} and \figref{rollout_v2}. 
Detic~\cite{zhou2022detecting} was able to reliably detect the knobs and
handles and CoTracker~\cite{karaev2023cotracker} was able to successfully track
the point of interaction letting us solve 24/28 task instances.

\subsubsection{Erroneous detections without inpainting hamper performance on 
handles and our end-effector out-painting strategy effectively mitigates it} 
As shown in \figref{rollouts2}, presence of the end-effector caused the object
detector to miss fire leading to failed execution. Our out painting approach
mitigates this issue leading to a higher success rate than the 
approach without out-painting. Interestingly, CoTracker~\cite{karaev2023cotracker} is quite robust
to occlusion (possibly because it tracks multiple points) and doesn't benefit
from in-painting. 


\subsubsection{Closed-loop imitation learning struggles on novel objects}
As presented in \tableref{rum}, \name significantly outperforms RUM in a paired evaluation on unseen objects across three novel environments. A common failure mode of RUM is its inability to grasp the object's handle, even when it approaches it closely.
Another failure mode we observe is RUM misidentifying keyholes or cabinet edges as handles, also resulting in failed grasp attempts.
These result demonstrate that a modular approach that leverages the broad generalization capabilities of vision foundation models is able to generalize much better than an end-to-end imitation learning approach trained on 1000+ demonstrations, which must learn all aspects of the task from scratch.



\section{Conclusion}

In this paper, we propose a sample weight averaging strategy to address variance inflation of previous independence-based sample reweighting algorithms. 
We prove its validity and benefits with theoretical analyses. 
Extensive experiments across synthetic and multiple real-world datasets demonstrate its superiority in mitigating variance inflation and improving covariate-shift generalization.  


\bibliography{reference}

%%%%%%%%%%%%%%%%%%%%%%%%%%%%%%%%%%%%%%%%%%%%%%%%%%%%%%%%%%%%%%%%%%%%%%%%%%%%%%%
%%%%%%%%%%%%%%%%%%%%%%%%%%%%%%%%%%%%%%%%%%%%%%%%%%%%%%%%%%%%%%%%%%%%%%%%%%%%%%%
% APPENDIX
%%%%%%%%%%%%%%%%%%%%%%%%%%%%%%%%%%%%%%%%%%%%%%%%%%%%%%%%%%%%%%%%%%%%%%%%%%%%%%%
%%%%%%%%%%%%%%%%%%%%%%%%%%%%%%%%%%%%%%%%%%%%%%%%%%%%%%%%%%%%%%%%%%%%%%%%%%%%%%%
\newpage
\appendix
\onecolumn
% \input{sections/app_frameworksummary}
% \section{Framework Details}
% Our framework is described in Algorithm~\ref{algorithm}, and compared with former baselines in Table~\ref{table:comparison}. Distinct with several methods generating Python code for visualization directly, we use VQL as an intermediate representation to bridge natural language queries and visualization code. Additionally, our framework can be easily optimized by adding some useful tools such as Retrieval Augmented Generation. Moreover, our method supports handling multi-table data and the visualization can be customized according to humans' preferences. Our framework utilizes the agent-based collaborative workflow, which consists of data preprocessing, generation, and error correction, organized with the modular design.

% \begin{algorithm}
% \small
% \caption{\system Framework}
% \label{algorithm}
% \begin{algorithmic}[1]
% \Function{\nlvis}{$Q$, $S$}
%     \State Initialize $Mem \gets \{Q,S\}$
%     \State $(S', A) \gets \textsc{Processor}(Mem)$
%     \State $Mem.update(S', A)$
%     \State $V \gets \textsc{Composer}(Mem)$
%     \State $Mem.update(V)$
%     \State $Chart, isValid \gets \textsc{Validator}(Mem)$
%     \While{not $isValid$}
%         \State $V \gets \textsc{Refine}(Mem)$
%         \State $Mem.update(V)$
%         \State $Chart, isValid \gets \textsc{Validator}(Mem)$
%     \EndWhile
%     \State \Return $Chart$
% \EndFunction
% \end{algorithmic}

% \end{algorithm}




% \begin{table*}[!t]
%     \centering
    
%     \vspace{-1em}
%     \scalebox{0.68}{
%     \begin{tabular}{lccccccc}
%         \toprule[1.5pt]
%         \multirow{3}{*}{\textbf{Framework}} & \multicolumn{2}{c}{\textbf{System Features}} & \multicolumn{2}{c}{\textbf{Visualization Capabilities}} & \multicolumn{3}{c}{\textbf{Agentic Workflow}} \\
%         \cmidrule(lr){2-3} \cmidrule(lr){4-5} \cmidrule(lr){6-8}
%         & \textbf{VQL as} & \textbf{Extensible} & \textbf{Multi-Table} & \textbf{Customizable} & \textbf{Data} & \textbf{Modular} & \textbf{Error-} \\
%         & \textbf{Thoughts} & \textbf{Optimization} & \textbf{Support} & \textbf{Styling} & \textbf{Preprocess} & \textbf{Design} & \textbf{Correction} \\
%         \midrule
%         Chat2VIS~\cite{chat2vis} & \textcolor{red}{\ding{56}} & \textcolor{red}{\ding{56}} & \textcolor{red}{\ding{56}} & \textcolor{red}{\ding{56}} & \textcolor{green!60!black}{\ding{52}} & \textcolor{red}{\ding{56}} & \textcolor{red}{\ding{56}} \\
%         Mirror~\cite{mirror} & \textcolor{red}{\ding{56}} & \textcolor{red}{\ding{56}} & \textcolor{red}{\ding{56}} & \textcolor{red}{\ding{56}} & \textcolor{red}{\ding{56}} & \textcolor{green!60!black}{\ding{52}} & \textcolor{red}{\ding{56}} \\
        
%         LIDA~\cite{lida} & \textcolor{red}{\ding{56}} & \textcolor{green!60!black}{\ding{52}} & \textcolor{red}{\ding{56}} & \textcolor{green!60!black}{\ding{52}} & \textcolor{green!60!black}{\ding{52}} & \textcolor{green!60!black}{\ding{52}} & \textcolor{red}{\ding{56}} \\
%         CoML4VIS~\cite{coml} & \textcolor{red}{\ding{56}} & \textcolor{red}{\ding{56}} & \textcolor{green!60!black}{\ding{52}} & \textcolor{red}{\ding{56}} & \textcolor{green!60!black}{\ding{52}} & \textcolor{red}{\ding{56}} & \textcolor{red}{\ding{56}} \\
        
%         Prompt4VIS~\cite{prompt4vis} & \textcolor{green!60!black}{\ding{52}} & \textcolor{red}{\ding{56}} & \textcolor{green!60!black}{\ding{52}} & \textcolor{red}{\ding{56}} & \textcolor{green!60!black}{\ding{52}} & \textcolor{green!60!black}{\ding{52}} & \textcolor{red}{\ding{56}} \\
        
%         CoT-Vis~\cite{cotvis} & \textcolor{green!60!black}{\ding{52}} & \textcolor{red}{\ding{56}} & \textcolor{red}{\ding{56}} & \textcolor{red}{\ding{56}} & \textcolor{green!60!black}{\ding{52}} & \textcolor{red}{\ding{56}} & \textcolor{red}{\ding{56}} \\

%         \midrule
%         \SystemName (Ours) & \textcolor{green!60!black}{\ding{52}} & \textcolor{green!60!black}{\ding{52}} & \textcolor{green!60!black}{\ding{52}} & \textcolor{green!60!black}{\ding{52}} & \textcolor{green!60!black}{\ding{52}} & \textcolor{green!60!black}{\ding{52}} & \textcolor{green!60!black}{\ding{52}} \\
%         \bottomrule[1.5pt]
%     \end{tabular}}
% \caption{Comparison of various \nlvis frameworks. }  \label{table:comparison}
% \vspace{-1em}
% \end{table*}

\section{Detailed Experiment Setups}
\label{detailed_experiment_setups}
\paragraph{Baselines.}
\label{detailed_baselines}
% We implemented our experiment compared with three recent baselines. Note that, we also tried to use Code Interpreter as a baseline, but due to the rate limit of API constraint, the evaluation failed to generate visualizations via direct .csv files.
This study compares our approach with three state-of-the-art baselines. We also attempted to include Code Interpreter as a baseline; however, API rate limitations prevent the direct generation of visualizations from CSV files.

\begin{itemize}[leftmargin=*, itemsep=0pt] 
    \item \textbf{Chat2Vis} \cite{chat2vis}: It generates data visualizations by leveraging prompt engineering to translate natural language descriptions into visualizations. It uses a language-based table description, which includes column types and sample values, to inform the visualization generation process.\item \textbf{LIDA} \cite{lida}: It structures visualization generation as a four-step process, where each step builds on the previous one to incrementally translate natural language inputs into visualizations. It uses a JSON format to describe column statistics and samples, making it adaptable across various visualization tasks.
    \item \textbf{CoML4Vis} \cite{coml}: 
    % Building on a data science code generation framework, CoML4Vis 
    It utilizes a few-shot prompt that integrates multiple tables into a single visualization task. It summarizes data table information, including column names and samples, and then applies a few-shot prompt to guide visualization generation.
\end{itemize}

\paragraph{Metrics.}
\label{detailed_metrics}
Our evaluation framework involves five main metrics:
\begin{itemize}[leftmargin=*, itemsep=0pt] 
    \item \textbf{Invalid Rate} represents the percentage of visualizations that fail to render due to issues like incorrect API usage or other code errors.
    \item \textbf{Illegal Rate} indicates the percentage of visualizations that do not meet query requirements, which can include incorrect data transformations, mismatched chart types, or improper visualizations.
    \item \textbf{Readability Score} is the average score (range 1-5) assigned by a vision language model, like GPT-4V, for valid and legal visualizations, assessing their visual clarity and ease of interpretation.
    \item \textbf{Pass Rate} measures the proportion of visualizations in the evaluation set that are both valid (able to render) and legal (meet the query requirements).
    \item \textbf{Quality Score} is set to 0 for invalid or illegal visualizations; otherwise, it is equal to the readability score, providing an overall assessment of visualization quality factoring in both functionality and clarity.
\end{itemize}
To thoroughly evaluate each main metric, we further break them down into the following detailed assessment criteria:
\begin{itemize}[leftmargin=4mm, itemsep=0.05mm] 
    \item \textbf{Code Execution Check} verifies that the Python code generated by the model can be successfully executed.
    \item \textbf{Surface-form Check} ensures that the generated code includes necessary elements to produce a visualization like function calls to display the chart.
    \item \textbf{Chart Type Check} verifies whether the extracted chart type from the visualization matches the ground truth.
    \item \textbf{Data Check} assesses if the data used in the visualization matches the ground truth, taking into consideration potential channel swaps based on specified channels.
    \item \textbf{Order Check} evaluates whether the sorting of visual elements follows the specified query requirements.
    \item \textbf{Layout Check} examines issues like text overflow or element overlap within visualizations.
    \item \textbf{Scale \& Ticks Check} ensures that scales and ticks are appropriately chosen, avoiding unconventional representations.
    \item \textbf{Overall Readability Rating} integrates various readability checks to provide a comprehensive score considering layout, scale, text clarity, and arrangement.
\end{itemize}

% For all evaluation results, these metrics are averaged across the dataset to provide an overarching view of model performance. These metrics collectively ensure that visualizations are not only correct in terms of execution but also effective in communicating the intended data narratives.
The evaluation metrics are averaged across the dataset to provide a comprehensive overview of the model's performance. Together, these metrics ensure that the visualizations are both accurate in execution and effective in conveying the intended data narratives.



\begin{table}[!t]
\centering
\setlength{\belowcaptionskip}{0em} 
% \vspace{-1em}
\begin{tabular}{lcc}
\toprule[1.5pt]
\textbf{Model} & \textbf{P-corr} & \textbf{P-value} \\
\midrule
GPT-4o-mini & \textbf{0.6503} & 0.000 \\
GPT-4o & 0.5648 & 0.000 \\
\bottomrule[1.5pt]
\end{tabular}
\caption{ The Pearson correlations of GPT-4o-mini and GPT-4o with human judgments on readability scores. }
\label{tab:pearson_corr}
\vspace{-1em}
\end{table}

\begin{table*}[!ht]
\centering

\vspace{-1em}
\begin{tabular}{l|ccc|ccc}
\toprule
\multirow{2}{*}{Method} & \multicolumn{3}{c|}{Single Table} & \multicolumn{3}{c}{Multiple Tables} \\
\cmidrule(l){2-4} \cmidrule(l){5-7}
 & prompt & response & total & prompt & response & total \\
\midrule
LIDA & 1386.23 & 237.90 & 1624.13 & \multicolumn{3}{c}{N/A} \\
Chat2Vis & 414.35 & 451.30 & 865.65 & \multicolumn{3}{c}{N/A} \\
CoML4Vis & 2614.76 & 279.86 & 2894.62 & 3069.62 & 307.67 & 3377.29 \\
\system & 5122.99 & 777.63 & 5900.62 & 5613.96 & 1014.10 & 6628.06 \\
\bottomrule
\end{tabular}
\caption{Token usage comparison for different methods. N/A indicates that LIDA and Chat2Vis cannot handle multiple table scenarios.}
\label{tab:token_usage}
\end{table*}

\begin{table}[ht]
\centering
\scalebox{1}{
\begin{tabular}{l|ccc}
\toprule
Agent & \#Input & \#Output & \#Total \\
\midrule
Processor & 1486.07 & 569.58 & 1755.65\\
Composer & 3268.32 & 221.74 & 3490.07 \\
Validator & 1051.82 & 127.85 & 1179.67  \\
\bottomrule
\end{tabular}}
\caption{Token usage of three agents in \system.} \label{tab:token_agent} 
\vspace{-1em}
\end{table}

\paragraph{Implement Details.}
Our system is implemented in Python 3.9, utilizing GPT-4o \citep{openai_gpt4o_2024}, GPT-4o-mini~\cite{openai2024gpt4omini}, and GPT-3.5-turbo~\cite{chatgpt3.5} as the backbone model for all approaches, with the temperature set to 0 for consistent outputs. GPT-4o-mini serves as the vision language model for readability evaluation. We interact with these models through the Azure OpenAI API. The specific prompt templates for each agent, crucial for guiding their respective roles in the visualization generation process, are detailed in Appendix~\ref{prompt_details}. Token usages of \system and baselines are demonstrated in Table~\ref{tab:token_usage}, and usage for each agent in our \system is shown in Table~\ref{tab:token_agent}. Additionally, our evaluations are conducted in VisEval Benchmark (with MIT license).

\paragraph{Human Annotation.}
\label{human}
The annotation is conducted by 5 authors of this paper independently. As acknowledged, the diversity of annotators plays a crucial role in reducing bias and enhancing the reliability of the benchmark. These annotators have knowledge in the data visualization domain, with different genders, ages, and educational backgrounds. The educational backgrounds of annotators are above undergraduate. To ensure the annotators can proficiently mark the data, we provide them with detailed tutorials, teaching them how to judge the quality of data visualization. We also provide them with detailed criteria and task requirements in each annotation process shown in Figure~\ref{fig:annotation}. Two experiments requiring human annotation are detailed as follows:

\begin{figure}[!ht]
    \centering
    \includegraphics[width=\linewidth]{figure/score_distribution.pdf}
    \caption{Comparison of score density distribution between GPT-4o, GPT-4o-mini and human average score.}
    \label{fig:score_distribution}
\end{figure}

\begin{table*}[!ht]
\centering
\begin{tabular}{l|ccc}
\toprule
& Invalid Rate & Illegal Rate & Pass Rate \\
\midrule
\system & 4.66\% & 23.97\% & 71.35\% \\
w. CoT for Validator & 5.82\% & 23.39\% & 70.78\% \\
w. original schema for Validator & 4.80\% & 24.22\% & 70.97\% \\
\bottomrule
\end{tabular}
\caption{Additional exploration for Validator (using GPT-3.5-turbo).} 
\vspace{-1em} 
\label{tab:ablation_validator}
\end{table*}

\begin{itemize}[leftmargin=*, itemsep=0pt]
    \item \textbf{Pearson Correlation of Visual Language Model.} We conduct human annotation frameworks to compare the ability of the visual language model for MLLM-as-a-Judge~\cite{chen2024mllm}, providing the readability score. Our annotation framework is shown in Figure~\ref{fig:annotation}. The final Pearson scores are demonstrated in Table~\ref{tab:pearson_corr}, with its density distribution in Figure~\ref{fig:score_distribution}. The detailed instructions can be found in Figure~\ref{fig:scoring_instructions}.
    \item \textbf{Qualitative comparison to calculate ELO Scores.} We conduct human-judgments evaluations to compare which visualization generated by different models meets the query requirement more precisely. The leaderboard is shown in Table~\ref{tab:elo_rankings}, and Figure~\ref{fig:elo} shows the judgment framework. Each model starts with a base ELO score of 1500. After each pairwise comparison, the scores are updated based on the outcome and the current scores of the models involved. The hyperparameters are set as follows: the $K$-factor is set to 32, which determines the maximum change in rating after a single comparison. We conduct two sets of evaluations: one for single-table queries and another for multiple-table queries, with 1000 bootstrap iterations for each set to ensure statistical robustness. For each model's ELO rating, we report the 95\% confidence intervals computed through bootstrap resampling, providing a measure of rating stability. The evaluation process involves presenting human judges with a query and two visualizations, asking them to select the one that better meets the query requirements. This process is repeated across all model pairs and queries in our test set. The detailed guidance provides to the human evaluators can be found in Figure~\ref{fig:evaluation_instructions}, which outlines the criteria for judging visualization quality and relevance to the given query.


\end{itemize}

\begin{figure}[!ht]
	\centering
    \setlength{\belowcaptionskip}{-1em}
	\includegraphics[width=0.98\linewidth,scale=1.0]
    {./figure/library.pdf}
    \vspace{-1em}
	\caption{Performance of different models using \texttt{Matplotlib} and \texttt{Seaborn} libraries, using GPT-3.5-turbo.
    % \yao{larger fontsize?}
    }
\label{fig: library}
\end{figure}

\begin{figure*}[!h]
    \centering
    \includegraphics[width=0.98\linewidth]{figure/annotation.pdf}
    \caption{Screenshot of human annotation process in readability score.}
    \label{fig:annotation}
\end{figure*}

\begin{figure*}[ht]
\centering
\vspace{1em}
\begin{tcolorbox}[enhanced,attach boxed title to top center={yshift=-3mm,yshifttext=-1mm},boxrule=0.9pt, 
  colback=gray!00,colframe=black!50,colbacktitle=gray,
  title=Readability Scoring Instruction,
  boxed title style={size=small,colframe=gray} ]
\small
\textbf{Scoring Instructions:} Please evaluate the charts based on the following criteria, with a score range from 1 to 5, where 1 indicates very poor quality and 5 indicates excellent quality. You should focus on the following aspects:

\vspace{0.5em}
\textbf{1. Chart Colors:}
\begin{itemize}
    \item Are the colors clear and natural, effectively conveying the information?
    \item Color blindness accessibility: Are the color combinations easy to distinguish, especially for users with color blindness?
\end{itemize}

\vspace{0.5em}
\textbf{2. Title and Axis Labels:}
\begin{itemize}
    \item Ensure the chart has a clear title.
    \item Do the X-axis and Y-axis labels exist, and are they complete?
    \item Check if the labels are difficult to read, e.g., are they written vertically instead of horizontally?
    \item The title should not be a direct question; instead, it should describe the data or trends being presented.
\end{itemize}

\vspace{0.5em}
\textbf{3. Legend Completeness:}
\begin{itemize}
    \item Is the legend complete, and does it clearly indicate the color labels for different data series?
    \item Ensure each color has a corresponding legend, making it easy for users to understand what the data represents.
\end{itemize}

\vspace{0.5em}
\textbf{Scoring Scale:}
\begin{itemize}
    \item \textbf{1 Point:} Very poor, unable to understand or severely lacking information.
    \item \textbf{2 Points:} Poor quality, multiple issues present, difficult to extract information.
    \item \textbf{3 Points:} Fair, conveys some information but still has room for improvement.
    \item \textbf{4 Points:} Good, generally clear charts with minor areas for improvement.
    \item \textbf{5 Points:} Excellent, outstanding chart design with clear and effective information presentation.
\end{itemize}

Please consider the above factors when assessing the charts and provide the appropriate score. Thank you for your cooperation and effort!
\end{tcolorbox}
\vspace{-7pt}
\caption{Instructions for human annorators in annotating readability scoring.}
\label{fig:scoring_instructions}
\vspace{1em}
\end{figure*}

\begin{figure*}[!ht]
    \centering
    \includegraphics[width=0.98\linewidth]{figure/elo.pdf}
    \caption{Screenshot of ELO score evaluation framework for Human-as-a-Judge.}
    \label{fig:elo}
\end{figure*}

\begin{figure*}[ht]
\centering
\vspace{1em}
\begin{tcolorbox}[enhanced,attach boxed title to top center={yshift=-3mm,yshifttext=-1mm},boxrule=0.9pt, 
  colback=gray!00,colframe=black!50,colbacktitle=gray,
  title=Visualization Comparison Guidance,
  boxed title style={size=small,colframe=gray} ]
\small
Welcome to the visualization comparison evaluation. Your task is to judge which model-generated visualization better meets the requirements of the natural language query.

\vspace{0.5em}
\textbf{Evaluation criteria:}
\begin{enumerate}
    \item \textbf{Appropriateness of chart type:} Check if the selected chart type is suitable for expressing the data and relationships required by the query.
    \item \textbf{Data completeness:} Ensure the chart includes all necessary data required by the query.
    \item \textbf{Readability:} Assess the clarity of the chart, accuracy of labels, and overall layout.
    \item \textbf{Aesthetics:} Consider if the chart's color scheme, proportions, and overall design are visually pleasing.
    \item \textbf{Information conveyance:} Judge if the chart effectively conveys the main information or insights required by the query.
\end{enumerate}

\vspace{0.5em}
\textbf{Evaluation process:}
\begin{enumerate}
    \item Carefully read the natural language query.
    \item Observe the visualization results generated by two models.
    \item Based on the above criteria, choose the better visualization or select a tie if they are equally good.
    \item If neither visualization satisfies the query requirements well, please choose the relatively better one.
\end{enumerate}

Remember, your evaluation will help us improve and compare different visualization models. Thank you for your participation!
\end{tcolorbox}
\vspace{-7pt}
\caption{Instructions for human annorators in visualization comparison.}
\label{fig:evaluation_instructions}
\vspace{1em}
\end{figure*}


\section{Additional Experiment Results}
\label{additional_experiment_result}

We also conducted a comparison experiment of different methods using matplotlib or seaborn library. Figure~\ref{fig: library} demonstrates the results, indicating that our method outperforms obviously other baselines not only with matplotlib but also seaborn.

In addition, we test techniques in the Validator Agent, such as Chain-of-Thought. As is shown in Table~\ref{tab:ablation_validator}, integrating Chain-of-Thought reasoning, may affect its performance badly, likely due to the simple refining task with complex reasoning. Moreover, using the original schema to check for false schema filtering seems to be useless in this case.

\section{Evaluation Results with Detailed Metrics}
We demonstrated the main results in Table~\ref{tab:performance_comparison}, and here we reported more detailed results of other metrics in Table~\ref{tab:detailed_results}, which underscored the error rates for each stage, including \textit{Invalid}, \textit{Illegal}, and \textit{Low Readability}. 

\begin{table*}[!ht]
\centering
\footnotesize
\scalebox{0.98}{
\begin{tabular}{ll|cc|cccc|cc}
\toprule[1.5pt]
\multirow{2}{*}{Method} & \multirow{2}{*}{Dataset} & \multicolumn{2}{c|}{Invalid} & \multicolumn{4}{c|}{Illegal} & \multicolumn{2}{c}{Low Readability} \\
&  & Execution & Surface. & Decon. & Chart Type & Data & Order & Layout & Scale\&Ticks \\
\midrule
\multicolumn{10}{c}{ \textbf{\textit{GPT-4o}}}\\
\midrule
\multirow{3}{*}{CoML4Vis} & All & 1.15 & 0.00 & 0.26 & 1.75 & 14.28 & 10.36 & 32.02 & 32.55 \\
& Single & 0.67 & 0.00 & 0.43 & 1.93 & 13.54 & 10.16 & 31.08 & 32.76 \\
& Multiple & 1.87 & 0.00 & 0.00 & 1.48 & 15.39 & 10.66 & 33.43 & 32.23 \\
\cmidrule{2-10}
\multirow{3}{*}{LIDA} & All & 6.61 & 0.00 & 1.60 & 3.24 & 40.53 & 4.07 & 32.68 & 15.77 \\
& Single & 1.13 & 0.00 & 2.11 & 0.89 & 12.26 & 6.79 & 53.93 & 26.22 \\
& Multiple & 14.80 & 0.00 & 0.79 & 8.51 & 80.53 & 0.00 & 1.24 & 0.21 \\
\cmidrule{2-10}
\multirow{3}{*}{Chat2Vis} & All & 16.05 & 0.00 & 0.62 & 3.99 & 30.14 & 5.96 & 2.37 & 20.88 \\
& Single & 0.86 & 0.00 & 0.75 & 2.30 & 10.78 & 9.73 & 3.97 & 34.63 \\
& Multiple & 38.74 & 0.00 & 0.43 & 6.51 & 59.08 & 0.32 & 0.00 & 0.34 \\
\cmidrule{2-10}
\multirow{3}{*}{nvAgent} & All & 0.97 & 0.00 & 0.08 & 1.28 & 11.07 & 4.05 & 5.07 & 40.03 \\
& Single & 0.72 & 0.00 & 0.14 & 1.27 & 9.88 & 3.60 & 3.92 & 39.36 \\
& Multiple & 1.34 & 0.00 & 0.00 & 1.30 & 12.84 & 4.73 & 6.79 & 41.03 \\
\midrule
\multicolumn{10}{c}{ \textbf{\textit{GPT-4o-mini}}}\\
\midrule
\multirow{3}{*}{CoML4Vis} & All & 4.23 & 0.00 & 0.20 & 2.31 & 16.64 & 11.83 & 35.23 & 29.35 \\
& Single & 0.36 & 0.00 & 0.26 & 2.32 & 13.80 & 11.67 & 35.92 & 32.22 \\
& Multiple & 10.01 & 0.00 & 0.10 & 2.31 & 20.87 & 12.07 & 34.19 & 25.05 \\
\cmidrule{2-10}
\multirow{3}{*}{LIDA} & All & 12.50 & 0.00 & 0.40 & 4.92 & 40.02 & 5.80 & 27.87 & 17.05 \\
& Single & 9.09 & 0.00 & 0.44 & 2.53 & 12.91 & 9.68 & 45.69 & 28.32 \\
& Multiple & 17.61 & 0.00 & 0.33 & 8.51 & 80.53 & 0.00 & 1.24 & 0.21 \\
\cmidrule{2-10}
\multirow{3}{*}{Chat2Vis} & All & 15.45 & 0.17 & 0.17 & 4.21 & 31.90 & 8.20 & 2.14 & 18.97 \\
& Single & 2.14 & 0.29 & 0.41 & 2.53 & 11.99 & 9.68 & 45.69 & 28.32 \\
& Multiple & 35.78 & 0.00 & 0.00 & 6.70 & 61.66 & 0.00 & 0.92 & 0.32 \\
\cmidrule{2-10}
\multirow{3}{*}{nvAgent} & All & 5.14 & 0.00 & 0.00 & 2.40 & 16.33 & 10.61 & 41.06 & 27.00 \\
& Single & 1.97 & 0.00 & 0.14 & 2.97 & 15.21 & 7.49 & 39.30 & 32.39 \\
& Multiple & 8.15 & 0.00 & 0.00 & 2.31 & 20.87 & 12.07 & 34.19 & 25.05 \\
\midrule
\multicolumn{10}{c}{ \textbf{\textit{GPT-3.5-turbo}}}\\
\midrule
\multirow{3}{*}{CoML4Vis} & All & 9.28 & 0.00 & 0.62 & 1.91 & 15.83 & 12.86 & 25.09 & 27.73 \\ 
& Single & 6.17 & 0.00 & 0.89 & 2.50 & 14.71 & 13.20 & 26.10 & 29.93 \\ 
& Multiple & 13.92 & 0.00 & 0.21 & 1.04 & 17.51 & 12.36 & 23.57 & 24.43 \\ 
\cmidrule{2-10} 
\multirow{3}{*}{LIDA} & All & 53.43 & 0.00 & 1.27 & 3.56 & 22.33 & 0.53 & 14.90 & 6.62 \\ 
& Single & 47.32 & 0.00 & 1.91 & 2.81 & 13.03 & 0.89 & 24.43 & 11.05 \\ 
& Multiple & 62.57 & 0.00 & 0.32 & 4.68 & 36.23 & 0.00 & 0.65 & 0.00 \\ 
\cmidrule{2-10} 
\multirow{3}{*}{Chat2Vis} & All & 18.68 & 0.00 & 0.28 & 3.66 & 32.47 & 7.20 & 25.45 & 20.15 \\ 
& Single & 3.90 & 0.00 & 0.47 & 2.78 & 15.62 & 12.01 & 41.74 & 33.38 \\ 
& Multiple & 40.77 & 0.00 & 0.00 & 4.97 & 57.66 & 0.00 & 1.12 & 0.37 \\ 
\cmidrule{2-10} 
\multirow{3}{*}{nvAgent} & All & 4.66 & 0.00 & 0.08 & 3.06 & 18.24 & 5.64 & 5.25 & 35.34 \\ 
& Single & 2.98 & 0.00 & 0.14 & 2.84 & 15.08 & 5.69 & 3.62 & 37.57 \\ 
& Multiple & 7.18 & 0.00 & 0.00 & 3.38 & 22.95 & 5.56 & 7.69 & 32.02 \\
\bottomrule[1.5pt]
\end{tabular}
}
\caption{Detailed error rates (\%) for different methods.} 
\label{tab:detailed_results}
\end{table*}

\section{Case Study}
\label{example}
% To demonstrate our approach's effectiveness, we present several illustrative examples. Figure~\ref{fig:nl_vql} shows how our system translates natural language into a structured VQL representation. Figure~\ref{python code} and Figure~\ref{fig:example_chart} demonstrate the complete pipeline from query to visualization.
Figure~\ref{fig:nl_vql} shows an example of a natural language query with its corresponding VQL representation. The output Python code for visualization and the final bar chart are demonstrated in Figure~\ref{python code} and Figure~\ref{fig:example_chart}, respectively.
Furthermore, we provide a case study of \system performance on four hardness-level NL2Vis problems in VisEval in Figure \ref{hardness case}.

The easy case demonstrates accurate grouping in scatter plot relationships. The medium case shows correct handling of multi-table joins for continent-wise statistics. The hard case exhibits temporal data visualization with proper filtering. The extra hard case showcases complex operations including weekday binning and stacked visualization. These cases highlight our system's consistent performance across varying task complexities, particularly excelling in multiple table scenarios and complex aggregations.

\begin{figure*}[htbp]
\centering
\begin{tcolorbox}[enhanced,attach boxed title to top center={yshift=-3mm,yshifttext=-1mm},boxrule=0.9pt, 
  colback=gray!00,colframe=black!50,colbacktitle=gray,
  title=An Example of Natural Language Query and  Corresponding VQL,,
  boxed title style={size=small,colframe=gray} ]

\textbf{Natural Language Query:}\\
How many documents are stored? Bin the store date by weekday in a bar chart.\\
\tcbline
\textbf{Corresponding VQL:}\\
Visualize BAR \\
SELECT Date\_Stored, COUNT(Document\_ID)\\
FROM All\_Documents \\
GROUP BY Date\_Stored \\
BIN Date\_Stored BY WEEKDAY\\
\end{tcolorbox}
\caption{The natural language query case and its corresponding output VQL representation.}
\label{fig:nl_vql}
\end{figure*}

\lstset{
    basicstyle=\ttfamily\small,
    breaklines=true,
    numbers=left,
    numberstyle=\tiny,
    frame=single,
    showstringspaces=false,
    tabsize=4,
    keywordstyle=\color{blue},
    commentstyle=\color{green!60!black},
    stringstyle=\color{purple},
    breakatwhitespace=false,
    keepspaces=true
}
\begin{figure*}[htbp]
\begin{lstlisting}[
    language=Python,
    frame=single,
    numbers=left,
    numberstyle=\tiny,
    basicstyle=\tiny\ttfamily,    
    breaklines=true,
    breakatwhitespace=false,
    showstringspaces=false,
    framesep=1mm,                 % 减小框架边距
    framexleftmargin=1mm,         % 减小框架左边距
    columns=flexible,             % 灵活列宽
    keepspaces=true
]
import matplotlib.pyplot as plt 
import pandas as pd 
import os 
import duckdb

# Set data folder path
data_folder = 'E:/visEval_dataset/databases/cre_Doc_Tracking_DB'  

# Connect to database 
con = duckdb.connect(database=':memory:')  

# Read all CSV files and create view
csv_files = [f for f in os.listdir(data_folder) if f.endswith('.csv')] 
for file in csv_files:
    table_name = os.path.splitext(file)[0]
    con.execute(f"CREATE VIEW {table_name} AS SELECT * FROM read_csv_auto('{os.path.join(data_folder, file)}')")  
    
# Execute SQL query 
sql = f''' 
    SELECT Date_Stored, COUNT(Document_ID) AS count_Document_ID 
    FROM All_Documents 
    GROUP BY Date_Stored 
''' 
df = con.execute(sql).fetchdf() 
con.close()  

# Rename columns 
df.columns = ['Date_Stored','count_Document_ID'] 

# Apply binning operation
flag = True 
df['Date_Stored'] = pd.to_datetime(df['Date_Stored']) 
df['Date_Stored'] = df['Date_Stored'].dt.day_name()  

# Group by and calculate count 
if flag:
    df = df.groupby('Date_Stored').sum().reset_index() 

# Ensure all seven days of the week are included 
weekday_order = ['Monday', 'Tuesday', 'Wednesday', 'Thursday', 
                 'Friday', 'Saturday', 'Sunday'] 
df = df.set_index('Date_Stored').reindex(weekday_order, fill_value=0).reset_index()
df['Date_Stored'] = pd.Categorical(df['Date_Stored'], 
                                  categories=weekday_order, ordered=True) 
df = df.sort_values('Date_Stored')

# Create visualization 
fig, ax = plt.subplots(1, 1, figsize=(10, 4)) 
ax.spines['top'].set_visible(False) 
ax.spines['right'].set_visible(False) 
ax.bar(df['Date_Stored'], df['count_Document_ID']) 
ax.set_xlabel('Date_Stored') 
ax.set_ylabel('count_Document_ID') 
ax.set_title(f'BAR Chart of count_Document_ID by Date_Stored') 
plt.xticks(rotation=45) 
plt.tight_layout()  
plt.show()
\end{lstlisting}
\caption{An example of python code generating module within \system.}
\label{python code}
\end{figure*}


\begin{figure*}[!ht]
    \centering
    \includegraphics[width=0.98\linewidth,scale=1.0]{figure/bar_chart.pdf}
    \caption{An example of generated bar chart using \system.}
    \label{fig:example_chart}
\end{figure*}

\begin{figure*}[htbp]
\centering
\begin{tcolorbox}[enhanced,attach boxed title to top center={yshift=-3mm,yshifttext=-1mm},boxrule=0.9pt, 
  colback=gray!00,colframe=black!50,colbacktitle=gray,
  title=Examples of \textsc{nvAgent} performance on different hardness levels,
  boxed title style={size=small,colframe=gray} ]
  
\textbf{Hardness Level:} Easy \\
\begin{minipage}{0.45\linewidth}
    \textbf{Dataset}: \textit{Single}\\
    \textbf{Input Tables}: basketball\_match\\
    \textbf{Input Query}: Show the relation between acc percent and all\_games\_percent for each ACC\_Home using a grouped scatter chart.\\
\end{minipage}\hfill
\begin{minipage}{0.45\linewidth}
    \centering
    \textbf{Response}:
    \includegraphics[width=\linewidth]{figure/easy_3085.pdf} 
\end{minipage}
\tcbline

\textbf{Hardness Level:} Medium \\
\begin{minipage}{0.45\linewidth}
    \textbf{Dataset}: \textit{Multiple}\\
    \textbf{Input Tables}: car\_makers, car\_names, cars\_data, continents, countries, model\_list\\
    \textbf{Input Query}: Display a pie chart for what is the name of each continent and how many car makers are there in each one?\\
\end{minipage}\hfill
\begin{minipage}{0.55\linewidth}
    \centering
    \textbf{Response}:
    \includegraphics[width=\linewidth]{figure/medium_433.pdf} 
\end{minipage}
\tcbline

\textbf{Hardness Level:} Hard \\[1em]
\begin{minipage}{0.45\linewidth}
    \textbf{Dataset}: \textit{Multiple}\\
    \textbf{Input Tables}: advisor, classroom, course, department, instructor, prereq, section, student, takes, teaches, time\_slot\\
    \textbf{Input Query}: Find the number of courses offered by Psychology department in each year with a line chart.\\
\end{minipage}\hfill
\begin{minipage}{0.45\linewidth}
    \centering
    \textbf{Response}:
    \includegraphics[width=\linewidth]{figure/hard_611.pdf} 
\end{minipage}
\tcbline

\textbf{Hardness Level:} Extra Hard \\[1em]
\begin{minipage}{0.45\linewidth}
    \textbf{Dataset}: \textit{Multiple}\\
    \textbf{Input Tables}: Accounts, Documents, Documents\_with\_Expenses, Projects, Ref- \_Budget\_Codes, Ref\_Document\_Types, Statements\\
    \textbf{Input Query}: How many documents are created in each day? Bin the document date by weekday and group by document type description with a stacked bar chart, I want to sort Y in desc order.\\
\end{minipage}\hfill
\begin{minipage}{0.45\linewidth}
    \centering
    \textbf{Response}:
    \includegraphics[width=\linewidth]{figure/extra_851.pdf} 
\end{minipage}

\end{tcolorbox}
    \caption{Examples of \textsc{nvAgent}'s performance on different hardness levels in VisEval (easy, medium, hard, and extra hard.}
    \label{hardness case}
\end{figure*}


\clearpage
\onecolumn
\section{Prompts Details}
\label{prompt_details}
We provide detailed prompt design of our \system as follows.



\begin{promptbox}[Prompt template for Processor Agent]
You are an experienced and professional database administrator. Given a database schema and a user query, your task is to analyze the query, filter the relevant schema, generate an optimized representation, and classify the query difficulty. \\
\\
Now you can think step by step, following these instructions below. \\
\textbf{[Instructions]} \\
1. Schema Filtering: \\
\text{\ \ \ \ }- Identify the tables and columns that are relevant to the user query.\\
\text{\ \ \ \ }- Only exclude columns that are completely irrelevant.\\
\text{\ \ \ \ }- The output should be \{\{tables: [columns]\}\}.\\
\text{\ \ \ \ }- Keep the columns needed to be primary keys and foreign keys in the filtered schema.\\
\text{\ \ \ \ }- Keep the columns that seem to be similar with other columns of another table.\\
\\
2. New Schema Generation:\\
\text{\ \ \ \ }- Generate a new schema of the filtered schema, based on the given database schema and your filtered schema.\\
\\
3. Augmented Explanation:\\
\text{\ \ \ \ }- Provide a concise summary of the filtered schema to give additional knowledge.\\
\text{\ \ \ \ }- Include the number of tables, total columns, and any notable relationships or patterns.\\
\\
4. Classification:\\
For the database new schema, classify it as SINGLE or MULTIPLE based on the tables number.\\
\text{\ \ \ \ }- if tables number >= 2: predict MULTIPLE\\
\text{\ \ \ \ }- elif only one table: predict SINGLE\\
\\
==============================\\
Here is a typical example:\\
\textbf{[Database Schema]}\\
\textbf{[DB\_ID]} dorm\_1\\
\textbf{[Schema]}\\
\# Table: Student\\
\text{[}\\
  \text{\ \ \ \ }(stuid, And This is a id type column),\\
  \text{\ \ \ \ }(lname, Value examples: [`Smith', `Pang', `Lee', `Adams', `Nelson', `Wilson'].),\\
  \text{\ \ \ \ }(fname, Value examples: [`Eric', `Lisa', `David', `Sarah', `Paul', `Michael'].),\\
  \text{\ \ \ \ }(age, Value examples: [18, 20, 17, 19, 21, 22].),\\
  \text{\ \ \ \ }(sex, Value examples: [`M', `F'].),\\
  \text{\ \ \ \ }(major, Value examples: [600, 520, 550, 50, 540, 100].),\\
  \text{\ \ \ \ }(advisor, And this is a number type column),\\
  \text{\ \ \ \ }(city code, Value examples: [`PIT', `BAL', `NYC', `WAS', `HKG', `PHL'].)\\
\text{]}\\
% \end{promptbox}
% \end{figure*}
% \begin{figure*}[!h]
% \begin{promptbox}[Prompt template for Processor Agent]
\# Table: Dorm\\
\text{[}\\
  \text{\ \ \ \ }(dormid, And This is a id type column),\\
  \text{\ \ \ \ }(dorm name, Value examples: [`Anonymous Donor Hall', `Bud Jones Hall', `Dorm-plex 2000', `Fawlty Towers', `Grad Student Asylum', `Smith Hall'].),\\
  \text{\ \ \ \ }(student capacity, Value examples: [40, 85, 116, 128, 256, 355].),
  (gender, Value examples: [`X', `F', `M'].)\\
\text{]}\\
\# Table: Dorm\_amenity\\
\text{[}\\
  \text{\ \ \ \ }(amenid, And This is a id type column),\\
  \text{\ \ \ \ }(amenity name, Value examples: [`4 Walls', `Air Conditioning', `Allows Pets', `Carpeted Rooms', `Ethernet Ports', `Heat'].)\\
\text{]}\\
\# Table: Has\_amenity\\
\text{[}\\
  \text{\ \ \ \ }(dormid, And This is a id type column),\\
  \text{\ \ \ \ }(amenid, And This is a id type column)\\
\text{]}\\
\# Table: Lives\_in\\
\text{[}\\
  \text{\ \ \ \ }(stuid, And This is a id type column),\\
  \text{\ \ \ \ }(dormid, And This is a id type column),\\
  \text{\ \ \ \ }(room number, And this is a number type column)\\
\text{]}\\
\\
\textbf{[Query]}\\
Find the first name of students who are living in the Smith Hall, and count them by a pie chart\\
\\
Now we can think step by step\\
\textbf{[Filtered Schema]}\\
\{\\
  "Student": ["stuid", "fname"],\\
  "Dorm": ["dormid", "dorm name"],\\
  "Lives\_in": ["stuid", "dormid"]\\
\}\\
\\
\textbf{[New Schema]}\\
\# Table: Student\\
\text{[}\\
  (stuid, And This is a id type column),\\
  (fname, Value examples: [`Eric', `Lisa', `David', `Sarah', `Paul', `Michael'].),\\
\text{]}\\
% \end{promptbox}
% \end{figure*}
% \begin{figure*}[!h]
% \begin{promptbox}[Prompt template for Processor Agent]
\# Table: Dorm
\text{[}\\
  (dormid, And This is a id type column),\\
  (dorm name, Value examples: [`Anonymous Donor Hall', `Bud Jones Hall', `Dorm-plex 2000', `Fawlty Towers', `Grad Student Asylum', `Smith Hall'].),\\
\text{]}\\
\# Table: Lives\_in\\
\text{[}\\
  (stuid, And This is a id type column),\\
  (dormid, And This is a id type column),\\
\text{]}\\
\textbf{[Augmented Explanation]}\\
The filtered schema consists of 3 tables (Student, Dorm, and Lives\_in) with a total of 6 relevant columns. There is a many-to-one relationship between Student and Dorm through the Lives\_in junction table. The query involves joining these three tables to find students living in a specific dorm (Smith Hall).\\
\\
Key points:\\
1. The Lives\_in table acts as a bridge between Student and Dorm, allowing for the association of students with their dorms.\\
2. The `dorm name' column in the Dorm table is crucial for filtering the specific dorm (Smith Hall).\\
3. The `fname' column from the Student table is required for the final output.\\
\\
\textbf{[Classification]}\\
MULTIPLE\\
\\
==============================\\
Here is a new question:\\
\\
\textbf{[DB\_ID]} \{db\_id\}\\
\textbf{[Database Schema]}\\
\{db\_schema\}\\
\\
\textbf{[Query]}\\
\{query\}\\
\\
Now give your answer following this format strictly without other explanation:\\
\\
\textbf{[Filtered Schema]}\\
\\
\textbf{[New Schema]}\\
\\
\textbf{[Augmented Explanation]}\\
\\
\textbf{[Classification]}\\
\\
\end{promptbox}
% \end{figure*}

% \subsection{Composer Agent Prompt}
% \label{composer_prompt}
% \begin{figure*}[!h]
\begin{promptbox}[Prompt template for multiple classification]
Given a [Database schema] with [Augmented Explanation] and a [Question], generate a valid VQL (Visualization Query Language) sentence. VQL is similar to SQL but includes visualization components. \\
\\
Now you can think step by step, following these instructions below. \\
\textbf{[Background]} \\
VQL Structure:\\
Visualize [TYPE] SELECT [COLUMNS] FROM [TABLES] [JOIN] [WHERE] [GROUP BY] [ORDER BY] [BIN BY]\\
\\
You can consider a VQL sentence as "VIS TYPE + SQL + BINNING"\\
You must consider which part in the sketch is necessary, which is unnecessary, and construct a specific sketch for the natural language query.\\
\\
Key Components:\\
1. Visualization Type: bar, pie, line, scatter, stacked bar, grouped line, grouped scatter\\
2. SQL Components: SELECT, FROM, JOIN, WHERE, GROUP BY, ORDER BY\\
3. Binning: BIN [COLUMN] BY [INTERVAL], [INTERVAL]: [YEAR, MONTH, DAY, WEEKDAY]\\
\\
When generating VQL, we should always consider special rules and constraints:\\
\textbf{[Special Rules]} \\
a. For simple visualizations:\\
    \text{\ \ \ \ }- SELECT exactly TWO columns, X-axis and Y-axis(usually aggregate function)\\
b. For complex visualizations (STACKED BAR, GROUPED LINE, GROUPED SCATTER):\\
    \text{\ \ \ \ }- SELECT exactly THREE columns in this order!!!:\\
        \text{\ \ \ \ }\text{\ \ \ \ }1. X-axis\\
        \text{\ \ \ \ }\text{\ \ \ \ }2. Y-axis (aggregate function)\\
        \text{\ \ \ \ }\text{\ \ \ \ }3. Grouping column\\
c. When "COLORED BY" is mentioned in the question:\\
    \text{\ \ \ \ }- Use complex visualization type(STACKED BAR for bar charts, GROUPED LINE for line charts, GROUPED SCATTER for scatter charts)\\
    \text{\ \ \ \ }- Make the "COLORED BY" column the third SELECT column\\
    \text{\ \ \ \ }- Do NOT include "COLORED BY" in the final VQL\\     
d. Aggregate Functions:\\
    \text{\ \ \ \ }- Use COUNT for counting occurrences\\
    \text{\ \ \ \ }- Use SUM only for numeric columns\\
    \text{\ \ \ \ }- When in doubt, prefer COUNT over SUM\\
e. Time based questions:\\
    \text{\ \ \ \ }- Always use BIN BY clause at the end of VQL sentence\\
    \text{\ \ \ \ }- When you meet the questions including "year", "month", "day", "weekday"\\
    \text{\ \ \ \ }- Avoid using window function, just use BIN BY to deal with time base queries\\
% \end{promptbox}
% \end{figure*}
% \begin{figure*}[!h]
% \begin{promptbox}[Prompt template for multiple classification]
\textbf{[Constraints]} \\
- In SELECT <column>, make sure there are at least two selected!!!\\
- In FROM <table> or JOIN <table>, do not include unnecessary table\\
- Use only table names and column names from the given database schema\\
- Enclose string literals in single quotes\\
- If [Value examples] of <column> has `None' or None, use JOIN <table> or WHERE <column> is NOT NULL is better\\
- Ensure GROUP BY precedes ORDER BY for distinct values\\
- NEVER use window functions in SQL\\
\\
Now we could think step by step:\\
1. First choose visualize type and binning, then construct a specific sketch for the natural language query\\
2. Second generate SQL components following the sketch.\\
3. Third add Visualize type and BINNING into the SQL components to generate final VQL\\
\\
==============================\\
Here is a typical example:\\
\textbf{[Database Schema]}\\
\# Table: Orders, (orders)\\
\text{[}\\
  \text{\ \ \ \ }(order\_id, order id, And this is a id type column),\\
  \text{\ \ \ \ }(customer\_id, customer id, And this is a id type column),\\
  \text{\ \ \ \ }(order\_date, order date, Value examples: [`2023-01-15', `2023-02-20', `2023-03-10'].),\\
  \text{\ \ \ \ }(total\_amount, total amount, Value examples: [100.00, 200.00, 300.00, 400.00, 500.00].)\\
\text{]}\\
\# Table: Customers, (customers)\\
\text{[}\\
  \text{\ \ \ \ }(customer\_id, customer id, And this is a id type column),\\
  \text{\ \ \ \ }(customer\_name, customer name, Value examples: [`John', `Emma', `Michael', `Sophia', `William'].),\\
  \text{\ \ \ \ }(customer\_type, customer type, Value examples: [`Regular', `VIP', `New'].)\\
\text{]}\\
\textbf{[Augmented Explanation]}\\
The filtered schema consists of 2 tables (Orders and Customers) with a total of 7 relevant columns. There is a one-to-many relationship between Customers and Orders through the customer\_id foreign key.\\
\\
Key points:\\
1. The Orders table contains information about individual orders, including the order date and total amount.\\
2. The Customers table contains customer information, including their name and type (Regular, VIP, or New).\\
3. The customer\_id column links the two tables, allowing us to associate orders with specific customers.\\
% \end{promptbox}
% \end{figure*}
% \begin{figure*}[!h]
% \begin{promptbox}[Prompt template for multiple classification]
4. The order\_date column in the Orders table will be used for monthly grouping and binning.\\
5. The total\_amount column in the Orders table needs to be summed for each group.\\
6. The customer\_type column in the Customers table will be used for further grouping and as the third dimension in the stacked bar chart.\\
\\

The query involves joining these two tables to analyze order amounts by customer type and month, which requires aggregation and time-based binning.\\
\\
\textbf{[Question]}\\
Show the total order amount for each customer type by month in a stacked bar chart.\\
\\
Decompose the task into sub tasks, considering [Background] [Special Rules] [Constraints], and generate the VQL after thinking step by step:\\
\\
\textbf{Sub task 1:} First choose visualize type and binning, then construct a specific sketch for the natural language query\\
Visualize type: STACKED BAR, BINNING: True\\
VQL Sketch:\\
Visualize STACKED BAR SELECT \_ , \_ , \_ FROM \_ JOIN \_ ON \_ GROUP BY \_ BIN \_ BY MONTH\\
\\
\textbf{Sub task 2:} Second generate SQL components following the sketch.\\
Let's think step by step:\\
1. We need to select 3 columns for STACKED BAR chart, order\_date as X-axis, SUM(total\_amout) as Y-axis, customer\_type as group column.\\
2. We need to join the Orders and Customers tables.\\
3. We need to group by customer type.\\
4. We do not need to use any window function for MONTH.\\
\\
\text{sql}\\
```sql\\
SELECT O.order\_date, SUM(O.total\_amount), C.customer\_type\\
FROM Orders AS O\\
JOIN Customers AS C ON O.customer\_id = C.customer\_id\\
GROUP BY C.customer\_type\\
```\\
\\
\textbf{Sub task 3:} Third add Visualize type and BINNING into the SQL components to generate final VQL\\
\textbf{Final VQL:}\\
Visualize STACKED BAR SELECT O.order\_date, SUM(O.total\_amount), C.customer\_type FROM Orders O JOIN Customers C ON O.customer\_id = C.customer\_id GROUP BY C.customer\_type BIN O.order\_date BY MONTH\\
\\
% \end{promptbox}
% \end{figure*}
% \begin{figure*}[!h]
% \begin{promptbox}[Prompt template for multiple classification]
==============================\\
Here is a new question:\\
\\
\textbf{[Database Schema]}\\
\{desc\_str\}\\
\\
\textbf{[Augmented Explanation]}\\
\{augmented\_explanation\}\\
\\
\textbf{[Query]}\\
\{query\}\\
\\
Now, please generate a VQL sentence for the database schema and question after thinking step by step.\\

\end{promptbox}
% \end{figure*}


% \begin{figure*}[!h]
\begin{promptbox}[Prompt template for single classification]
Given a [Database schema] with [Augmented Explanation] and a [Question], generate a valid VQL (Visualization Query Language) sentence. VQL is similar to SQL but includes visualization components. \\
\\
Now you can think step by step, following these instructions below. \\
\textbf{[Background]} \\
VQL Structure:\\
Visualize [TYPE] SELECT [COLUMNS] FROM [TABLES] [JOIN] [WHERE] [GROUP BY] [ORDER BY] [BIN BY]\\
\\
You can consider a VQL sentence as "VIS TYPE + SQL + BINNING"\\
You must consider which part in the sketch is necessary, which is unnecessary, and construct a specific sketch for the natural language query.\\
\\
Key Components:\\
1. Visualization Type: bar, pie, line, scatter, stacked bar, grouped line, grouped scatter\\
2. SQL Components: SELECT, FROM, JOIN, WHERE, GROUP BY, ORDER BY\\
3. Binning: BIN [COLUMN] BY [INTERVAL], [INTERVAL]: [YEAR, MONTH, DAY, WEEKDAY]\\
\\
When generating VQL, we should always consider special rules and constraints:\\
\textbf{[Special Rules]} \\
a. For simple visualizations:\\
    \text{\ \ \ \ }- SELECT exactly TWO columns, X-axis and Y-axis(usually aggregate function)\\
b. For complex visualizations (STACKED BAR, GROUPED LINE, GROUPED SCATTER):\\
    \text{\ \ \ \ }- SELECT exactly THREE columns in this order!!!:\\
        \text{\ \ \ \ }\text{\ \ \ \ }1. X-axis\\
        \text{\ \ \ \ }\text{\ \ \ \ }2. Y-axis (aggregate function)\\
        \text{\ \ \ \ }\text{\ \ \ \ }3. Grouping column\\
c. When "COLORED BY" is mentioned in the question:\\
    \text{\ \ \ \ }- Use complex visualization type(STACKED BAR for bar charts, GROUPED LINE for line charts, GROUPED SCATTER for scatter charts)\\
    \text{\ \ \ \ }- Make the "COLORED BY" column the third SELECT column\\
    \text{\ \ \ \ }- Do NOT include "COLORED BY" in the final VQL\\     
d. Aggregate Functions:\\
    \text{\ \ \ \ }- Use COUNT for counting occurrences\\
    \text{\ \ \ \ }- Use SUM only for numeric columns\\
    \text{\ \ \ \ }- When in doubt, prefer COUNT over SUM\\
e. Time based questions:\\
    \text{\ \ \ \ }- Always use BIN BY clause at the end of VQL sentence\\
    \text{\ \ \ \ }- When you meet the questions including "year", "month", "day", "weekday"\\
    \text{\ \ \ \ }- Avoid using window function, just use BIN BY to deal with time base queries\\
% \end{promptbox}
% \end{figure*}
% \begin{figure*}[!h]
% \begin{promptbox}[Prompt template for single classification]
\textbf{[Constraints]} \\
- In SELECT <column>, make sure there are at least two selected!!!\\
- In FROM <table> or JOIN <table>, do not include unnecessary table\\
- Use only table names and column names from the given database schema\\
- Enclose string literals in single quotes\\
- If [Value examples] of <column> has `None' or None, use JOIN <table> or WHERE <column> is NOT NULL is better\\
- Ensure GROUP BY precedes ORDER BY for distinct values\\
- NEVER use window functions in SQL\\
\\
Now we could think step by step:\\
1. First choose visualize type and binning, then construct a specific sketch for the natural language query\\
2. Second generate SQL components following the sketch.\\
3. Third add Visualize type and BINNING into the SQL components to generate final VQL\\
\\
==============================\\
Here is a typical example:\\
\textbf{[Database Schema]}\\
\# Table: course, (course)\\
\text{[}\\
  \text{\ \ \ \ }(course\_id, course id, Value examples: [101, 696, 656, 659]. And this is an id type column),\\
  \text{\ \ \ \ }(title, title, Value examples: [`Geology', `Differential Geometry', `Compiler Design', `International Trade', `Composition and Literature', `Environmental Law'].),\\
  \text{\ \ \ \ }(dept\_name, dept name, Value examples: [`Cybernetics', `Finance', `Psychology', `Accounting', `Mech. Eng.', `Physics'].),\\
  \text{\ \ \ \ }(credits, credits, Value examples: [3, 4].)\\
\text{]}\\
\# Table: section, (section)\\
\text{[}\\
  \text{\ \ \ \ }(course\_id, course id, Value examples: [362, 105, 960, 468]. And this is an id type column),\\
  \text{\ \ \ \ }(sec\_id, sec id, Value examples: [1, 2, 3]. And this is an id type column),\\
  \text{\ \ \ \ }(semester, semester, Value examples: [`Fall', `Spring'].),\\
  \text{\ \ \ \ }(year, year, Value examples: [2002, 2006, 2003, 2007, 2010, 2008].),\\
  \text{\ \ \ \ }(building, building, Value examples: [`Saucon', `Taylor', `Lamberton', `Power', `Fairchild', `Main'].),\\
  \text{\ \ \ \ }(room\_number, room number, Value examples: [180, 183, 134, 143].),\\
  \text{\ \ \ \ }(time\_slot\_id, time slot id, Value examples: [`D', `J', `M', `C', `E', `F']. And this is an id type column)\\
\text{]}\\
\textbf{[Augmented Explanation]}\\
The filtered schema consists of 2 tables (course and section) with a total of 11 relevant columns. There is a one-to-many relationship between course and section through the course\_id foreign key.\\
\\
% \end{promptbox}
% \end{figure*}
% \begin{figure*}[!h]
% \begin{promptbox}[Prompt template for single classification]
Key points:\\
1. The course table contains information about individual courses, including the course title, department, and credits.\\
2. The section table contains information about specific sections of courses, including the semester, year, building, room number, and time slot.\\
3. The course\_id column links the two tables, allowing us to associate sections with specific courses.\\
4. The dept\_name column in the course table will be used to filter for Psychology department courses.\\
5. The year column in the section table will be used for yearly grouping and binning.\\
6. We need to count the number of courses offered each year, which requires aggregation and time-based binning.\\
\\
The query involves joining these two tables to analyze the number of courses offered by the Psychology department each year, which requires aggregation and time-based binning.\\
\\
\textbf{[Question]}\\
Find the number of courses offered by Psychology department in each year with a line chart.\\
\\
Decompose the task into sub tasks, considering [Background] [Special Rules] [Constraints], and generate the VQL after thinking step by step:\\
\\
\textbf{Sub task 1:} First choose visualize type and binning, then construct a specific sketch for the natural language query\\
Visualize type: LINE, BINNING: True\\
VQL Sketch:\\
Visualize LINE SELECT \_ , \_ FROM \_ JOIN \_ ON \_ WHERE \_ BIN \_ BY YEAR\\
\\
\textbf{Sub task 2:} Second generate SQL components following the sketch.\\
Let's think step by step:\\
1. We need to select 2 columns for LINE chart, year as X-axis, COUNT(year) as Y-axis.\\
2. We need to join the course and section tables to get the number of courses offered by the Psychology department in each year.\\
3. We need to filter the courses by the Psychology department.\\
4. We do not need to use any window function for YEAR.\\
\\
\text{sql}\\
```sql\\
SELECT S.year, COUNT(S.year)\\
FROM course AS C\\
JOIN section AS S ON C.course\_id = S.course\_id\\
WHERE C.dept\_name = `Psychology'\\
```\\
\\
% \end{promptbox}
% \end{figure*}
% \begin{figure*}[!h]
% \begin{promptbox}[Prompt template for single classification]
\textbf{Sub task 3:} Third add Visualize type and BINNING into the SQL components to generate final VQL\\
\textbf{Final VQL:}\\
Visualize LINE SELECT S.year, COUNT(S.year) FROM course C JOIN section S ON C.course\_id = S.course\_id WHERE C.dept\_name = `Psychology' BIN S.year BY YEAR\\
\\
==============================\\
Here is a new question:\\
\\
\textbf{[Database Schema]}\\
\{desc\_str\}\\
\\
\textbf{[Augmented Explanation]}\\
\{augmented\_explanation\}\\
\\
\textbf{[Query]}\\
\{query\}\\
\\
Now, please generate a VQL sentence for the database schema and question after thinking step by step.\\

\end{promptbox}
% \end{figure*}

% \subsection{Validator Agent Prompt}
% \label{validator_prompt}
% \begin{figure*}
\begin{promptbox}[Prompt template for Validator Agent]
As an AI assistant specializing in data visualization and VQL (Visualization Query Language), your task is to refine a VQL query that has resulted in an error. Please approach this task systematically, thinking step by step.\\
\textbf{[Background]}\\
VQL Structure:\\
Visualize [TYPE] SELECT [COLUMNS] FROM [TABLES] [JOIN] [WHERE] [GROUP BY] [ORDER BY] [BIN BY]\\
\\
You can consider a VQL sentence as "VIS TYPE + SQL + BINNING"\\
\\
Key Components:\\
1. Visualization Type: bar, pie, line, scatter, stacked bar, grouped line, grouped scatter\\
2. SQL Components: SELECT, FROM, JOIN, WHERE, GROUP BY, ORDER BY\\
3. Binning: BIN [COLUMN] BY [INTERVAL], [INTERVAL]: [YEAR, MONTH, DAY, WEEKDAY]\\
\\
When refining VQL, we should always consider special rules and constraints:\\
\textbf{[Special Rules]} \\
a. For simple visualizations:\\
    \text{\ \ \ \ }- SELECT exactly TWO columns, X-axis and Y-axis(usually aggregate function)\\
b. For complex visualizations (STACKED BAR, GROUPED LINE, GROUPED SCATTER):\\
    \text{\ \ \ \ }- SELECT exactly THREE columns in this order!!!:\\
        \text{\ \ \ \ }\text{\ \ \ \ }1. X-axis\\
        \text{\ \ \ \ }\text{\ \ \ \ }2. Y-axis (aggregate function)\\
        \text{\ \ \ \ }\text{\ \ \ \ }3. Grouping column\\
c. When "COLORED BY" is mentioned in the question:\\
    \text{\ \ \ \ }- Use complex visualization type(STACKED BAR for bar charts, GROUPED LINE for line charts, GROUPED SCATTER for scatter charts)\\
    \text{\ \ \ \ }- Make the "COLORED BY" column the third SELECT column\\
    \text{\ \ \ \ }- Do NOT include "COLORED BY" in the final VQL\\     
d. Aggregate Functions:\\
    \text{\ \ \ \ }- Use COUNT for counting occurrences\\
    \text{\ \ \ \ }- Use SUM only for numeric columns\\
    \text{\ \ \ \ }- When in doubt, prefer COUNT over SUM
% \end{promptbox}
% \end{figure*}

% \begin{figure*}
% \begin{promptbox}[Prompt template for Validator Agent]
e. Time based questions:\\
    \text{\ \ \ \ }- Always use BIN BY clause at the end of VQL sentence\\
    \text{\ \ \ \ }- When you meet the questions including "year", "month", "day", "weekday"\\
    \text{\ \ \ \ }- Avoid using time function, just use BIN BY to deal with time base queries\\
\\
\textbf{[Constraints]} \\
- In FROM <table> or JOIN <table>, do not include unnecessary table\\
- Use only table names and column names from the given database schema\\
- Enclose string literals in single quotes\\
- If [Value examples] of <column> has `None' or None, use JOIN <table> or WHERE <column> is NOT NULL is better\\
- ENSURE GROUP BY clause cannot contain aggregates\\
- NEVER use date functions in SQL\\
\\
\textbf{[Query]} \\
\{query\}\\
\\
\textbf{[Database info]} \\
\{db\_info\}\\
\\
\textbf{[Current VQL]} \\
\{vql\}\\
\\
\textbf{[Error]} \\
\{error\}\\
\\
Now, please analyze and refine the VQL, please provide:\\
\\
\textbf{[Explanation]}\\
\text{[}Provide a detailed explanation of your analysis process, the issues identified, and the changes made. Reference specific steps where relevant.\text{]}\\
\\
\textbf{[Corrected VQL]}\\
\text{[}Present your corrected VQL here. Ensure it's on a single line without any line breaks.\text{]}\\
\\
Remember:\\
- The SQL components must be parseable by DuckDB.\\
- Do not change rows when you generate the VQL.\\
- Always verify your answer carefully before submitting.\\
\end{promptbox}
% \end{figure*}
%%%%%%%%%%%%%%%%%%%%%%%%%%%%%%%%%%%%%%%%%%%%%%%%%%%%%%%%%%%%%%%%%%%%%%%%%%%%%%%
%%%%%%%%%%%%%%%%%%%%%%%%%%%%%%%%%%%%%%%%%%%%%%%%%%%%%%%%%%%%%%%%%%%%%%%%%%%%%%%

\end{document}
