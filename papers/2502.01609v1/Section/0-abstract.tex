\begin{abstract}
Recent advances in Large Language Models (LLMs) have revolutionized generative systems, achieving excellent performance across diverse domains. Although these models perform well in controlled environments, their real-world applications frequently encounter inputs containing both essential and irrelevant details. Our investigation has revealed a critical vulnerability in LLMs, which we term Contextual Distraction Vulnerability (CDV). This phenomenon arises when models fail to maintain consistent performance on questions modified with semantically coherent but irrelevant context. To systematically investigate this vulnerability, we propose an efficient tree-based search methodology to automatically generate CDV examples. Our approach successfully generates CDV  examples across four datasets, causing an average performance degradation of approximately 45\% in state-of-the-art LLMs. To address this critical issue, we explore various mitigation strategies and find that post-targeted training approaches can effectively enhance model robustness against contextual distractions. Our findings highlight the fundamental nature of CDV as an ability-level challenge rather than a knowledge-level issue since models demonstrate the necessary knowledge by answering correctly in the absence of distractions. This calls the community's attention to address CDV during model development to ensure reliability.
\end{abstract}
