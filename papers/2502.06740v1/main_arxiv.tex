% !TeX program = lualatex
% !TeX spellcheck = en_GB
\documentclass[11pt, DIV=15]{scrarticle}
\usepackage[utf8]{luainputenc}
\usepackage[T1]{fontenc}
\usepackage[UKenglish]{babel}
\usepackage{amsmath, amsfonts, amsthm, amssymb, mathtools, stmaryrd}

\usepackage[hyphens]{url}
\usepackage{breakurl}
\usepackage{tikz}
\usepackage{tikz-cd}
\usepackage{tkz-graph}
\usetikzlibrary{arrows,backgrounds,positioning,fit,cd,babel}

\usepackage[pdfusetitle,hidelinks]{hyperref}
\usepackage[capitalise,noabbrev]{cleveref}
\newcommand{\creflastconjunction}{, and\nobreakspace} % https://tex.stackexchange.com/questions/161338/can-cleveref-be-made-to-use-the-oxford-comma-for-multiple-citations


\usepackage{csquotes}
%\usepackage{lmodern}
\usepackage[osf,sc]{mathpazo}
\addtokomafont{disposition}{\normalfont\bfseries} % \scshape \rmfamily
\usepackage{relsize}
\usepackage{microtype}
% \usepackage{pdfpages}
%\usepackage[a4paper,margin=1in]{geometry}

\usepackage{thm-restate}
\usepackage{subcaption}
\usepackage{enumitem}
\usepackage{comment}


\usepackage{todonotes}


\linespread{1.06}
\recalctypearea

\usepackage[style=numeric-comp,maxbibnames=99,backend=biber,bibencoding=utf8,sorting=nty]{biblatex} % ,sorting=none,backref=true

\usepackage{orcidlink}
\usepackage{booktabs}


\addbibresource{literature.bib}

\theoremstyle{definition}
\newtheorem{definition}{Definition}[section]
\newtheorem{notation}[definition]{Notation}
\newtheorem{example}[definition]{Example}
\newtheorem{remark}[definition]{Remark}

\theoremstyle{plain}
\newtheorem{lemma}[definition]{Lemma}
\newtheorem{corollary}[definition]{Corollary}
\newtheorem{theorem}[definition]{Theorem}
\newtheorem{conjecture}[definition]{Conjecture}
\newtheorem{question}[definition]{Question}
\newtheorem{proposition}[definition]{Proposition}
\newtheorem{observation}[definition]{Observation}
\newtheorem{fact}[definition]{Fact}
\Crefname{fact}{Fact}{Facts}

\theoremstyle{remark}
\newtheorem{claim}{Claim}[definition]
\renewcommand*{\theclaim}{\thetheorem\alph{claim}}
\Crefname{claim}{Claim}{Claims}
\newenvironment{claimproof}[1][Proof of Claim]{\begin{proof}[#1] \renewcommand{\qedsymbol}{$\lrcorner$}}{ \end{proof}}
\setlist[enumerate, 1]{font=\upshape, noitemsep, nolistsep}
\setlist[enumerate, 2]{font=\upshape, noitemsep, nolistsep}
\setlist[itemize, 1]{noitemsep, nolistsep,font=\upshape}
\setlist[itemize, 2]{noitemsep, nolistsep,font=\upshape}

\DeclarePairedDelimiter{\norm}{\lVert}{\rVert}
\DeclarePairedDelimiter{\abs}{\lvert}{\rvert}

\newcommand{\cart}{\mathbin\square}

\newcommand{\Aa}{{\cal A}}
\newcommand{\Bb}{{\cal B}}
\newcommand{\Cc}{{\cal C}}
\newcommand{\Dd}{{\cal D}}
\newcommand{\Ee}{{\cal E}}
\newcommand{\Ff}{{\cal F}}
\newcommand{\Gg}{{\cal G}}
\newcommand{\Hh}{{\cal H}}
\newcommand{\Ii}{{\cal I}}
\newcommand{\Jj}{{\cal J}}
\newcommand{\Kk}{{\cal K}}
\newcommand{\Ll}{{\cal L}}
\newcommand{\Mm}{{\cal M}}
\newcommand{\Nn}{{\cal N}}
\newcommand{\Oo}{{\cal O}}
\newcommand{\Pp}{{\cal P}}
\newcommand{\Qq}{{\cal Q}}
\newcommand{\Rr}{{\cal R}}
\newcommand{\Ss}{{\cal S}}
\newcommand{\Tt}{{\cal T}}
\newcommand{\Uu}{{\cal U}}
\newcommand{\Vv}{{\cal V}}
\newcommand{\Ww}{{\cal W}}
\newcommand{\Xx}{{\cal X}}
\newcommand{\Zz}{{\cal Z}}


\DeclareMathOperator{\tw}{tw}
\DeclareMathOperator{\hdtw}{hdtw}
\DeclareMathOperator{\mn}{mn}
\DeclareMathOperator{\sub}{sub}
\DeclareMathOperator{\emb}{emb}
\DeclareMathOperator{\perm}{perm}
\DeclareMathOperator{\ml}{ml}
\DeclareMathOperator{\hdim}{hdim}
\DeclareMathOperator{\vc}{vc}
\DeclareMathOperator{\mln}{mln}
\DeclareMathOperator{\tr}{tr}
\DeclareMathOperator{\aut}{aut}
\DeclareMathOperator{\Sub}{Sub}
\DeclareMathOperator{\cl}{cl}
\DeclareMathOperator{\id}{id}
\DeclareMathOperator{\cc}{bcc}



\tikzset{
	vertex/.style={draw,circle,fill=gray},
	every node/.style={anchor=center},
	lbl/.style={color=lightgray}
}

\title{Symmetric Algebraic Circuits and Homomorphism Polynomials}
\author{Anuj Dawar \and Benedikt Pago \and Tim Seppelt}


\renewcommand{\phi}{\varphi}
\renewcommand{\epsilon}{\varepsilon}

\newcommand{\Aut}{\mathbf{Aut}}
\newcommand{\Sym}{\mathbf{Sym}}
\newcommand{\Alt}{\mathbf{Alt}}
\newcommand{\Stab}{\mathbf{Stab}}
\newcommand{\StabP}{\Stab^{\bullet}}
\newcommand{\Orb}{\mathbf{Orb}}

\DeclareMathOperator{\child}{children}




\newcommand{\bbN}{\mathbb{N}}
\newcommand{\bbQ}{\mathbb{Q}}
\newcommand{\bbF}{\mathbb{F}}




\definecolor{lightgray}{rgb}{0.60, 0.60, 0.61} % lipicsLightGray
\definecolor{gray}{rgb}{0.31, 0.31, 0.33} % lipicsBulletGray
\definecolor{yellow}{rgb}{0.99, 0.78, 0.07}

\usepackage{xargs}   
\newcommandx{\tim}[2][1=]{\todo[author=Tim,color=yellow,#1]{#2}}
\newcommandx{\benedikt}[2][1=]{\todo[author=Benedikt,color=orange,#1]{#2}}


\let\sup\relax
\DeclareMathOperator{\sup}{sup}

\DeclareMathOperator{\imm}{imm}
\DeclareMathOperator{\sgn}{sgn}
\DeclareMathOperator{\Match}{\# Match}
\DeclareMathOperator{\maxOrb}{maxOrb}
\DeclareMathOperator{\maxSup}{maxSup}



\newcommand{\VP}{\mathsf{VP}}
\newcommand{\VNP}{\mathsf{VNP}}
\newcommand{\VFPT}{\mathsf{VFPT}}
\newcommand{\VW}{\mathsf{VW}[1]}

\newcommand{\FPT}{\mathsf{FPT}}
\newcommand{\sharpW}{\#\mathsf{W}[1]}

\begin{document}
	\maketitle


	\begin{abstract}
	The central open question of algebraic complexity is whether $\VP \neq \VNP$, which is saying that the permanent cannot be represented by families of polynomial-size algebraic circuits. 
	For symmetric algebraic circuits, this has been confirmed by Dawar and Wilsenach (2020) who showed exponential lower bounds on the size of symmetric circuits for the permanent. In this work, we set out to develop a more general symmetric algebraic complexity theory. Our main result is that a family of symmetric polynomials admits small symmetric circuits if and only if they can be written as a linear combination of homomorphism counting polynomials of graphs of bounded treewidth. We also establish a relationship between the symmetric complexity of subgraph counting polynomials and the vertex cover number of the pattern graph. As a concrete example, we examine the symmetric complexity of immanant families (a generalisation of the determinant and permanent) and show that a known conditional dichotomy due to Curticapean (2021) holds unconditionally in the symmetric setting. 
\end{abstract}	


\section{Introduction}	
The study of \emph{algebraic circuit complexity} (also called \emph{arithmetic circuit complexity}) aims to undestand the power of circuits to succinctly express (or compute) polynomials.  In short, we are interested in establishing how many operations of addition and multiplication are needed in a circuit that computes a polynomial $p \in \bbF[\Xx]$, for some field $\bbF$ and set of variables $\Xx$.  We are usually interested in how this complexity grows with $n$ for a family of polynomials $(p_n)_{n \in \bbN}$.  The central conjecture in the field (known as $\VP \neq \VNP$) is that there are no polynomial-size circuits for the \emph{permanent} in the way that  there are for the \emph{determinant}.

Both the determinant and permanent are examples of polynomials over matrices.  That is to say that we can treat the set of variables $\Xx$ as the entries $x_{ij}$ of a square matrix.  Moreover, the permanent is \emph{symmetric} in the sense that permuting the rows and columns of the matrix does not change the polynomial.  This motivates the study of \emph{symmetric algebraic circuits} introduced in~\cite{dawar_symmetric_2020}, which are circuits where the symmetries of the polynomial computed are reflected in the automorphisms of the circuits.  This was used to show an exponential gap between the complexity of the determinant and the permanent for so-called square symmetric circuits, that is those that are symmetric under the action of the same permutation applied to the rows and columns.  For these, it was shown that there are polynomial-size circuits that compute the determinant but that any family of circuits computing the permanent is of exponential size.  It is the latter, exponential lower bound, that is the main technical achievement of the paper. The possibility of proving such strong lower bounds which seem presently out of reach for general circuits motivates the focus on \emph{symmetric} circuits.

In the present paper we consider polynomials over (not necessarily square) rational matrices which are invariant under arbitrary permutations of the rows and columns.  That is to say, we consider families $(p_{n,m})_{n,m \in \bbN}$ of polynomials where $p_{n,m} \in \bbQ[\Xx_{n,m}]$ and $\Xx_{n,m}$ is the set of variables $\{x_{ij} \mid i \in [n], j \in [m]\}$.  The invariance condition we require is that for any pair of permutations $\pi \in \Sym_n, \sigma \in \Sym_m$, if $p_{n,m}^{(\pi,\sigma)}$ denotes the polynomial obtained from $p_{n,m}$ by replacing every occurrence of $x_{ij}$ with $x_{\pi(i)\sigma(j)}$ then $p_{n,m} = p_{n,m}^{(\pi,\sigma)}$. An assignment of values in $\bbQ$ to the variables $\Xx_{n,m}$ can be seen as a bipartite graph with rational edge weights.  The polynomial $p_{n,m}$ is then a formula for calculating a numerical value of the graph and the invariance condition is stating that this numerical value is indeed a property of the graph and not dependent on a specific ordering of its vertices. An example (with $n=m$) is the permanent $\sum_{\pi \in \Sym_n} \prod_{i \in [n]} x_{i\pi(i)}$.  When the variables are instantiated with the weights of a bipartite graph, it evaluates to the weighted sum of perfect matchings in the graph, where the weight of a perfect matching is the product of the weights of edges in it.

The question we address is when such a family $(p_{n,m})_{n,m \in
	\bbN}$ of polynomials is computable by a family of small circuits
$(C_{n,m})_{n,m \in \bbN}$ which are themselves invariant under
permutations in $\Sym_n \times \Sym_m$.  To be precise, say that a
circuit $C$ whose inputs come from $\Xx_{n,m} \cup \bbQ$ is
\emph{$\Sym_n \times \Sym_m$-symmetric} if every permutation in
$\Sym_n \times \Sym_m$ acting on $\Xx_{n,m}$ can be extended to an
automorphism of $C$.  The measure of size we are interested in is
the \emph{orbit size} of $C$, that is, the number of gates of $C$ in
the largest orbit of the action of $\Sym_n \times \Sym_m$ on $C$.
Indeed, the known lower bounds on the size of symmetric circuits,
such as for the permanent in~\cite{dawar_symmetric_2020}, establish lower bounds on the orbit size.  And, for this measure of size we are able to give a complete characterisation (Theorem~\ref{thm:main1} below) of those families of polynomials which admit circuits of polynomial orbit size.

Our characterisation is in terms of \emph{homomorphism polynomials}.  Let  $F$ be a bipartite graph with bipartition $A \uplus B$ and $n,m$ be positive integers.  We define the homomorphism polynomial $\hom_{F,n,m} \in \bbQ[\Xx_{n,m}]$ to be the polynomial
$$ \sum_{h \colon A \uplus B \to [n] \uplus [m]} \prod_{ab \in E(F)} x_{h(a)h(b)}.$$
The name comes from the fact that if we evaluate the matrix of variables $\Xx_{n,m}$ as the biadjacency matrix of a bipartite graph $G$, then $\hom_{F,n,m}$ evaluates to the number of homomorphisms from $F$ to $G$.  For our characterisation, we are particularly interested in the case when the \emph{treewidth} of $F$ is bounded.
Let $\mathfrak{T}_{n,m}^k$ denote the collection of all polynomials that can be obtained as linear combinations of homomorphism polynomials $\hom_{F,n,m}$ for graphs $F$ of treewidth at most $k$.    We can now state our main result.
\begin{theorem}[restate=thmMain,label=] \label{thm:main1}
	For every family of polynomials $p_{n,m} \in \mathbb{Q}[\mathcal{X}_{n,m}]$, the following are equivalent:
	\begin{enumerate}
		\item there exists a constant $k \in \mathbb{N}$ such that $p_{n,m} \in \mathfrak{T}_{n,m}^k$ for all $n,m \in \mathbb{N}$,\label{it:main1}
		%			\item $p$ has counting width at most $k-1$ on $(n,m)$-vertex edge-coloured graphs,\label{it:main2}
		\item the $p_{n,m}$ admit $\Sym_n \times \Sym_m$-symmetric circuits of orbit size polynomial in $n+m$.\label{it:main2}
	\end{enumerate}
\end{theorem}

It is instructive to consider again the permanent $\perm_n = \sum_{\pi \in \Sym_n} \prod_{i \in [n]} x_{i\pi(i)}$.  This is not a homomorphism polynomial but it can be seen as counting, given a bipartite graph $G$ instantiated as a biadjacency matrix $\Xx_{n,m}$, the number of \emph{subgraphs} isomorphic to an $n$-matching. The $n$-matching is of small treewidth but one consequence of Theorem~\ref{thm:main1} together with the exponential lower bound on symmetric circuits for the permament is that $\perm_n$ cannot be expressed as a linear combination of homomorphism polynomials of bounded treewidth.  Alternatively, we can understand one direction of Theorem~\ref{thm:main1} as a vast generalization of the lower bound on the permanent as it shows lower bounds on the orbit size of symmetric circuits for many other polynomials.

The lower bound on orbit size of symmetric circuits for $\perm_n$ was
established in~\cite{dawar_symmetric_2020} by showing a lower bound on
the \emph{counting width} of the number of perfect matchings in a
bipartite graph.  This is a measure of the complexity of graph
parameters introduced in~\cite{dawar_definability_2017}.  Formally,
for a parameter $\mu$ that maps graphs to natural numbers, the
counting width $\omega_{\mu}$ of $\mu$ is a function that maps $n$ to
the smallest $k$ such that among graphs of order $n$, $\mu$ is
constant on the $k$-Weisfeiler-Leman equivalence classes. The
$k$-Weisfeiler-Leman equivalence on graphs is a standard relaxation of
isomorphism (see~\cite{CFI,kiefer2020WL}).   For Boolean
circuits (and Boolean valued graph parameters), a tight relationship
between the counting width of $\mu$ and the orbit size  of symmetric
circuits is known~\cite{anderson_symmetric_2017}.  For algebraic
circuits, similar arguments show that if a family of polynomials is
computed by symmetric circuits of polynomial orbit size then the graph
parameter it evaluates has counting width bounded by a constant.  The
converse is not known in general, but we are able to establish it in
some special cases. 
\begin{theorem}
	\label{thm:main2}
	Let $p_{n,m} \in \mathbb{Q}[\mathcal{X}_{n,m}]$ be a family of polynomials satisfying any of the assertions below.
	Then $p_{n,m}$ has constant counting width if, and only if,
        $p_{n,m} \in \mathfrak{T}_{n,m}^k$ for some constant $k$.
	\begin{enumerate}
		\item $p_{n,m}$ is a linear combination of homomorphism polynomials of connected graphs whose size is at most logarithmic in $n, m$,
		\item $p_{n,m}$ is a homomorphism polynomial of a single graph.
	\end{enumerate}
\end{theorem}
In the above, we can understand the counting width of a polynomial family
$p_{n,m}$ to mean the counting width of the parameter on bipartite graphs defined by
evaluating $p_{n,m}$ on $\{0,1\}$-matrices.  We extend the notion of
counting width to matrices over $\bbQ$ in
\cref{def:counting-width} below.

The relationship with counting width also casts an interesting light on Theorem~\ref{thm:main1}.  
It is known that two graphs $G$ and $H$ are equivalent with respect to $k$-dimensional Weisfeiler-Leman equivalence if, and only if, 
for every graph $F$ of treewidth at most $k$, the number of homomorphisms from $F$ to $G$ and $H$ is the same~\cite{dvorak_recognizing_2010,dell-grohe-rattan}.
It follows from this that any class of graphs that is recognized by a family of symmetric (Boolean) circuits with polynomial-size orbits, and therefore has bounded counting width, is determined by counting homomorphisms from a class of graphs of bounded treewidth. 
Theorem~\ref{thm:main1} can then be seen as an analogue of this in the context of algebraic circuits. 

Our results should also be compared with
\cite{curticapean_homomorphisms_2017}, which studies the
\emph{computational} complexity of subgraph counting and related
parameters. Its key message is that many graph parameters can be
expressed as linear combinations of homomorphism counts, and that the
complexity is entirely controlled (subject to complexity-theoretic
assumptions like the exponential time hypothesis ETH) by the
treewidth of the graphs that appear among these homomorphism counts.

An essential limitation of the results in \cite{curticapean_homomorphisms_2017} is that they apply only to graph parameters $\mu$ which can be \emph{uniformly} expressed
as linear combinations of homomorphism counts, i.e.\ they are of the form $\mu(G) = \sum \alpha_i \hom(F_i, G)$ for all graphs $G$.
Our results extend this framework to graph parameters which can be \emph{non-uniformly} expressed as linear combinations of homomorphism counts, i.e.\ they are such that 
for all $n$ there exist coefficients $\alpha_{n,i}$ and pattern graphs $F_{n,i}$ such that $\mu(G) = \sum \alpha_{n,i} \hom(F_{n,i}, G)$ for all graphs $G$ on $n$ vertices.
Our \cref{thm:main1} shows that, as for the classical complexity of uniform graph motif parameters, 
the symmetric circuit complexity of non-uniform graph motif parameters is determined by the treewidth of the pattern graphs.
Moreover, this holds without any complexity-theoretic assumptions.

In order to make these observations more concrete,
we consider subgraph polynomials, i.e.\ polynomials which express
subgraph counts  (see \cref{sec:circuitsAndHomPolynomials}).
It is known that the number of occurrences of a subgraph $F$ in a
given graph $G$ can be computed in time $|V(G)|^{O(\vc(F))}$, where
$\vc(F)$ denotes the size of the smallest vertex cover number of
$F$~\cite{Williams13}.  This result is improved
in~\cite{curticapean_homomorphisms_2017} to show that the number of
occurrences of $F$ in $G$ can be computed in time $O(|V(F)|^{O(V(F)}|V(G)|^{t+1})$
where $t$ is the maximum treewidth of a homomorphic image of $F$.  The
corresponding lower bound states that, assuming ETH, this cannot be
much improved.  It is not too difficult to see that the number of
occurrences of $F$ as a subgraph of $G$ can be (uniformly) expressed as a linear
combination of the homomorphism counts of the homomorphic images of
$H$.  Moreover, there is a tight relationship between the vertex cover
number of $F$ and the maximum treewidth of its homomorphic images,
which also then relates the counting width of the subgraph count
of $F$ to its vertex cover number (see~\cite[Theorem
1.3]{neuen_homomorphism-distinguishing_2024}).
We prove the analogous results for syntactic complexity of subgraph
polynomials in the more general non-uniform setting.  In the case
where the size of the pattern graph $F$ grows sublinearly in $n$ and
$m$ we have the following complete picture:
\begin{theorem}[Informal]
	For graphs $F_{n,m}$ of size sublinear in $n,m$, the following are equivalent:
	\begin{enumerate}
		\item the subgraph polynomials of the $F_{n,m}$ are in $\mathfrak{T}_{n,m}^k$ for some constant $k$,
		\item the subgraph polynomials of the $F_{n,m}$ admit polynomial-size $\Sym_n \times \Sym_m$-symmetric circuits,
		\item the subgraph polynomials of the $F_{n,m}$ have constant counting width,
		\item the graphs $F_{n,m}$ have constant vertex cover number.
	\end{enumerate}	
      \end{theorem}
 To extend beyond the sublinear case, we introduce another parameter,
 which we call the \emph{biclique cover number} of a graph, which can
 be understood as closing the vertex cover measure under bipartite
 graph complementation (see \cref{def:bcc}).

Finally, to give another concrete example for the merits of the
symmetric circuit framework, in~\cref{sec:immanants} we show that a
complexity dichotomy for the \emph{immanant} families due to~\cite{curticapean2021full}, whose hard cases are conditional on certain complexity-theoretic assumptions, holds unconditionally for symmetric circuits. The immanants generalise the permanent and determinant polynomials, which are in fact the ``extreme cases''. 
Thus, the symmetric dichotomy for the immanants completes the picture begun in~\cite{dawar_symmetric_2020}.

\paragraph*{Acknowledgments}We acknowledge fruitful discussions with Radu Curticapean, Filip Kučerák, Deepanshu Kush, and Benjamin Rossman.
	
	

	

	\section{Preliminaries}
	
	We write $\mathbb{N} = \{0,1,\dots\}$ and $[n] = \{1, \dots, n\}$ for $n \geq 1$.
	For a tuple $\boldsymbol{x} \in X^k$ and $i \in [k]$, $x \in X$, write $\boldsymbol{x}[i/x] = x_1 \dots x_{i-1} x x_{i+1} \dots x_k \in X^k$ and $\boldsymbol{x}[i/] = x_1 \dots x_{i-1} x_{i+1} \dots x_k \in X^{k-1}$.
	For a set $A$,
	write $\Pi(A)$ for the set of partitions of $A$.
	For a partition $\pi \in \Pi(A)$,
	write $A/\pi$ for its set of parts.
	Abusing notation, we also write $\pi$ for the canonical map $A \to A/\pi$.
	




	\subsection{Permutation groups and supports}
	\label{sec:supports}
	Let $\Gamma$ be a group acting on a set $X$. 
	For $S \subseteq X$, let $\Stab_\Gamma(S) \coloneqq \{ \pi \in \Gamma \mid \pi(S) = S\}$ be the \emph{stabiliser} of $S$ and let $\StabP_\Gamma(S) \coloneqq \{ \pi \in \Gamma \mid \pi(s) = s \text{ for every } s\in S\}$ be the \emph{pointwise stabiliser} of $S$.
	The \emph{orbit} of $S$ is denoted $\Orb_{\Gamma}(S) = \{ \pi(S) \mid \pi \in \Gamma  \}$. It holds $|\Orb_\Gamma(S)| = |\Gamma| / |\Stab_\Gamma(S)| = [\Gamma : \Stab_\Gamma(S)]$ by the Orbit-Stabiliser Theorem.
	We drop the subscript $\Gamma$ if the group is clear from the context.
	
	Let $\Delta \leq \Gamma$ be a subgroup. A set $S \subseteq X$ is a \emph{support} of $\Delta$ if $\StabP_\Gamma(S) \leq \Delta$. 
	For certain special cases of $\Gamma$, it is known that there always exists a unique minimal support.
	\begin{lemma}
		\label{lem:minSupportExists}
		Every subgroup $\Delta \leq \Sym_n$ that has a support of size $< n/2$ has a unique minimal support  $\sup(\Delta)$.
		Every subgroup $\Delta \leq \Sym_I \times \Sym_J$ that has a support $S$ with $|S \cap I| < |I|/2$ and $|S \cap J| < |J|/2$ has a unique minimal support $\sup(\Delta)$.
		In both cases $\StabP(\sup(\Delta)) \leq \Delta \leq \Stab(\sup(\Delta))$.
	\end{lemma}	
	\begin{proof}
		The existence of a unique minimal support is shown by proving that the intersection of any two supports with the respective size bound is again a support. For $\Delta \leq \Sym_n$, this is proved in Lemma 26 in \cite{blass1999choiceless}. 
		For $\Delta \leq \Sym_I \times \Sym_J$, it is easy to check that the proof of that lemma also goes through.
		The assertion $\StabP(\sup(\Delta)) \leq \Delta \leq \Stab(\sup(\Delta))$ is shown in \cite{anderson_symmetric_2017}.
	\end{proof}	
	
	
	
	We also use the following fact without explicitly referring to it later. 
	\begin{lemma}[\cite{anderson_symmetric_2017}]
		\label{lem:groupActionOnSupports}
		Let $\Delta \leq \Gamma$ act on $X$ such that the minimal support $\sup(\Delta)$ is defined. Let $\Delta' \coloneqq \pi \Delta \pi^{-1}$ for $\pi \in \Gamma$. Then $\sup(\Delta') = \pi(\sup(\Delta))$.
	\end{lemma}	
	We mostly deal with the case $\Gamma = \Sym_I \times \Sym_J$ where $I = [n], J = [m]$ denote the bipartition of a bipartite graph. 
	In that case we write $\sup_L(\Delta) \coloneqq \sup(\Delta) \cap I$ and $\sup_R(\Delta) \coloneqq \sup(\Delta) \cap J$ to denote the restrictions of the support to the left and right part of the bipartition, respectively. Generally, for any set $S \subseteq I \uplus J$, we write $S_L$ to denote $S \cap I$ and $S_R$ for $S \cap J$.
	
	We also often need to think of supports as ordered tuples rather than unordered sets, in which case we write $\vec{\sup}(\Delta)$, or $\vec{\sup}_L(\Delta), \vec{\sup}_R(\Delta)$. The ordering of the tuples is be specified in the respective context.
	
	
	
	
	\subsection{Symmetric algebraic circuits}
	
	An \emph{algebraic circuit} over a variable set $\Xx$ and a field $\bbF$ is a connected \textsmaller{DAG} with multiedges
	such that each vertex is labelled with an element of $\Xx \cup \bbF \cup \{+,\times\}$. 	The vertices of a circuit are called \emph{gates}, the edges \emph{wires}.
	Gates labelled with elements from $\Xx \cup \bbF$ are called
        \emph{input gates} and they are not allowed to have incoming
        edges. Every element of $\Xx \cup \bbF$ is allowed to appear at most once as the label of a gate -- this is no restriction because one can simply identify the respective gates.  We assume that a circuit contains exactly one gate without outgoing wires, and this is called the \emph{output gate}.
	
	 An algebraic circuit represents (or computes) a polynomial in
         $\bbF[\Xx]$ in the obvious way: the gates $+$ and $\times$
         are simply interpreted as addition and multiplication on
         polynomials, and we care about the polynomial at the output
         gate. Arrows are directed from a computation result towards
         the next gate where that result is used as an input. The set
         of \emph{children} of a gate $g$, denoted $\child(g)$, is the
         set of gates $h$ such that $(h,g)$ is a wire of the circuit.  
	 Multiedges are allowed so that for example the polynomial
         $x^2$ can be represented with just one input and one
         multiplication gate. The \emph{size} of a circuit $C$, is
         defined as the number of gates plus the number of wires, counted with multiplicities. It is denoted $\norm{C}$.
	
	Let $\Gamma$ be a group that acts on $\Xx$. Then a circuit $C$ is \emph{$\Gamma$-symmetric} if the action of every $\pi \in \Gamma$ on the input gates $\mathcal{X}$ extends to an automorphism of the circuit:
	For every input gate $g$, let $\ell(g)$ denote its label. This is either a variable or a field element. 
	Field elements are fixed by every $\pi \in \Gamma$.
	We say that a permutation $\pi \in \Gamma$ \emph{extends to a circuit automorphism} of the  circuit $C$ 
	if there exists a $\sigma \in \Sym(V(C))$ such that $\ell(\sigma(g)) = \pi(\ell(g))$ for every input gate, 
	and such that $\sigma$ is an automorphism of the multigraph structure of $C$ (i.e.\ it maps edges to edges, non-edges to non-edges, and preserves the operation types of the gates). 
	For more details, see \cite{dawar_symmetric_2024}. 
	
	A $\Gamma$-symmetric circuit is called \emph{rigid} if it has no non-trivial circuit automorphism that fixes every input gate. This is equivalent to saying that for every $\pi \in \Gamma$, there is a \emph{unique} circuit automorphism $\sigma$ that extends $\pi$. In that case, we also write $\pi(g)$ for internal gates $g$, to denote the application of the unique circuit automorphism that extends $\pi$. Symmetric circuits can always be assumed to be rigid:
	%We will actually need a stronger property than rigidity, let's call it \emph{strong rigidity}. A circuit is strongly rigid if there do not exist two distinct gates in it that have the same set of children. A strongly rigid circuit is always rigid because for there to be a non-trivial automorphism that fixes all input gates, there must exist two internal gates with the same set of children. It is also easy to construct examples of circuits that are rigid but not strongly rigid. 

\begin{lemma}
	\label{lem:rigidifyCircuits}
	Let $C$ be a $\Gamma$-symmetric circuit.
	Then there exists a $\Gamma$-symmetric \emph{rigid} circuit
        $C'$ that represents the same polynomial with $\norm{C'} \leq
        \norm{C}$.
\end{lemma}	
\begin{proof}
	For Boolean circuits, the transformation of a non-rigid into a rigid symmetric circuit is described in the proof of \cite[Lemma 7]{anderson_symmetric_2017}. 
	It is not hard to see that a similar transformation can be carried out for algebraic circuits, and that it satisfies $\norm{C'} \leq \norm{C}$.
\end{proof}	



%Let $C$ be a $\Gamma$-symmetric circuit that is not necessarily rigid. 
%	For a gate $g$, let $\Orb_{\id}(g)$ denote the orbit of $g$ under the action of the subgroup of $\Aut(C)$ that fixes every input gate.
%	We show by induction on $d$ that there exists a $\Gamma$-symmetric circuit $C_d$ and a map $f \colon V(C) \to V(C_d)$ such that the following is true:
%	\begin{enumerate}
%		\item $\norm{C_d} \leq \norm{C}$.
%		\item For every two gates $g,g'$ in $C_d$ from which all paths towards an input gate have length $\leq d$, $\child(g) \neq \child(g')$.
%		\item For every gate $g$ in $C$, the gate $f(g)$ in $C_d$ represents the same polynomial as $g$.
%	\end{enumerate}	
%	For $d=1$, we can let $C_1 = C$, and all conditions hold for $f$ being the identity map. Now suppose that for some $d \geq 1$, the circuit $C_d$ has been constructed. To construct $C_{d+1}$, consider a gate $g$ in $C_d$ from which all paths towards input gates have length $\leq d+1$.
%	Let $G \subseteq V(C_d)$ be the set of gates $g'$ such that $\child(g') = \child(g)$, and suppose $|G| \geq 2$.
%	Then to construct $C_{d+1}$, we remove all gates in $G \setminus \{g\}$ from the circuit, which is possible because their outputs are identical.
%	It remains to connect the outgoing wires of $g$ in $C_{d+1}$ correctly: For every gate $h$ in $C_d$, let $w(h) \in \bbN$ denote the total number of incoming wires of $h$ that come from gates in $G$. In $C_{d+1}$, we simply put $w(h)$ many wires from $g$ to $h$, for every $h$ with $w(h) > 0$ (here we exploit that multiedges are allowed). The above procedure is done for every such gate $g$. 
%	We then define the function $f_{d+1}\colon C \to C_{d+1}$ like $f_d\colon C \to C_d$, with the difference that whenever $f_d(g') \in G \setminus \{g\}$, we set $f_{d+1}(g') \coloneqq g$.
	
%	It is obvious by construction that the third item is true with respect to this function $f_{d+1}$.
%	It is also clear that the first and second item are true for $C_{d+1}$. The only property that remains to check is that $C_{d+1}$ is still $\Gamma$-symmetric. 
%	So let $\pi \in \Gamma$ and $\sigma$ be an automorphism of $C_d$ extending $\pi$. We need to show that also $C_{d+1}$ has a circuit automorphism $\sigma'$ that extends $\pi$. On gates that were not affected by the construction step from $C_d$ to $C_{d+1}$, $\sigma'$ will do the same as $\sigma$.
%	Consider a set $G \subseteq V(C_d)$ from which we removed all but one gate $g$ in the construction of $C_{d+1}$. Either, $\sigma$ is a permutation of the gates in $G$ or it maps $G$ to another such set $\sigma(G)$ disjoint from $G$. In the first case, $\sigma'$ will just fix $g$. It is easy to see that this preserves all outgoing and all incoming wires of $g$, with multiplicities. If $\sigma(G)$ is disjoint from $G$, then $\sigma'(g)$ will be the unique gate that remains in $C_{d+1}$ from the set $\sigma(G)$ (by symmetry, $\sigma(G)$ is also a set of gates in $C_d$ that have the same children, so all but one of them are removed in $C_{d+1}$). It can be checked that this $\sigma'$ is indeed a circuit automorphism of $C_{d+1}$.
%\end{proof}	
Working with rigid circuits has several advantages. One of them is
that we obtain a well-defined notion of \emph{support} of gates, and
that there is a strong link between the size of these supports and the
orbit sizes of the gates.  Another is that we can identify $\Gamma$
with the automorphism group of $C$ and write $g^{\pi}$ for $g$ a gate
of $C$ and $\pi \in \Gamma$ to denote the image of $g$ under the
action of the automorphism of $C$ extending $\pi$.
In the following, we always assume that $\Gamma$ is $\Sym_n$ or $\Sym_n \times \Sym_m$.
In the former case, our variable set is $\Xx_n = \{ x_{ij} \mid i,j \in [n] \}$, and $\pi \in \Sym_n$ acts simultaneously on both indices, so $\pi(x_{ij}) = x_{\pi(i)\pi(j)}$.
In the latter case, the variable set is $\Xx_{n,m} = \{x_{ij} \mid i \in [n], j \in [m]\}$. Then a pair of permutations $(\pi, \pi') \in \Sym_n \times \Sym_m$ applied to $x_{ij}$ yields $x_{\pi(i)\pi'(j)}$.

\paragraph*{Complexity measures for symmetric circuits}
We study the complexity of symmetric polynomials in terms of properties of their symmetric circuits. 
An important measure for symmetric circuits is $\maxOrb(C)$, the \emph{maximum orbit size} of any gate in $C$, formally:
\[
\maxOrb(C) \coloneqq \max_{g \in V(C)} | \{ g^{\sigma} \mid \sigma \in \Gamma \}  |.
\]
	
If $C$ is rigid, and $\Gamma$ and $C$ are such that every stabiliser group of a gate admits a unique minimal support, then we can also speak about the \emph{support size} of $C$. In particular, this is possible if $\Gamma \in \{\Sym_n, \Sym_n \times \Sym_m\}$ and all gates have a support of size $\leq n/2$, because then, Lemma \ref{lem:minSupportExists} applies.
We write $\sup(g) \subseteq [n] \uplus [m]$ to denote the \emph{minimal support} of the group $\Stab_{\Sym_n \times \Sym_m}(g)$ in $\Sym_n \times \Sym_m$, and as before, $\sup(g)_L$ and $\sup_R(g)$ are its restrictions to $[n]$ and $[m]$, respectively. Similarly, in a $\Sym_n$-symmetric circuit, $\sup(g) \subseteq [n]$ denotes the minimal support of the group $\Stab_{\Sym_n}(g)$.
The \emph{support size} of a circuit $C$ is then:	
\[
\maxSup(C) \coloneqq \max_{g \in V(C)} | \sup(g) |.
\]	
We have the following relations between $\maxOrb$ and $\maxSup$. 
\begin{lemma}
	\label{lem:constantSupportOfGates}
	Let $(\Gamma_n)_{n \in \bbN}$ be either the sequence $(\Sym_n)_{n \in \bbN}$ or $(\Sym_{n} \times \Sym_{m})_{n \in \bbN}$ with $m \leq n$.
	Let $(C_n)_{n \in \bbN}$ be a sequence of $\Gamma_n$-symmetric rigid circuits.
	\begin{enumerate}
		\item If $\maxOrb(C_n)$ is polynomially bounded in $n$, then there exists a constant $k \in \bbN$ such that for all large enough $n \in \bbN$, it holds $\maxSup(C_n) \leq k$.
		\item If $\maxOrb(C_n)$ is at most $2^{o(n)}$, then $\maxSup(C_n) \leq o(n)$.
	\end{enumerate}	
\end{lemma}	
\begin{proof}
	In the case $\Gamma_n = \Sym_n$, this is stated in
	Theorems 14 and 15 in \cite{dawar_symmetric_2024}.
	One can check that for $\Gamma_n = \Sym_n \times \Sym_m$, the proofs of these theorems also go through. We refrain from spelling out all details but the core argument is this: 
	If the orbit size of a gate is polynomially bounded, then by the orbit-stabiliser theorem, the index of its stabiliser group in $\Gamma_n$ is also polynomial, i.e.\ can be bounded by $\binom{n}{k}$, for some constant $k$.
	Theorem 14 in \cite{dawar_symmetric_2024} relies on Theorems 4.6 and 4.10 in \cite{dawar_rank_logic}. The former of them states the following: If $\Delta \leq \Sym_n$ is a subgroup of index at most $\binom{n}{k}$, then $\Delta$ contains an alternating group $A$ on $(n-k)$ points. The proof of \cite[Theorem 4.10]{dawar_rank_logic} then shows that if $\Delta$ is the stabiliser group of a gate in a circuit, then the gate is stabilised not only by $A$, but by the full symmetric group on the $n-k$ many points. 
	This means that the remaining $k$ points constitute a support for the gate. Now if $\Delta \leq \Sym_n \times \Sym_m$ is the stabiliser group of a gate, then we can consider $\Delta_1, \Delta_2 \leq \Delta$ defined as the subgroups of $\Delta$ consisting of those members of the product that have the neutral element in the first or second component, respectively. 
	Because the orbit size of the gate is polynomially bounded,
        the index of $\Delta_1, \Delta_2$ in $\Sym_n, \Sym_m$,
        respectively, is at most $\binom{n}{k}$ (or $\binom{m}{k}$),
        for some constant $k$. Then the same reasoning as before
        applie (this is analogous to the argument made, in the context of
        the alternating group, in~\cite{dawar2021lower}).
	This finishes the first part of the theorem. The second part follows from the first part as shown in the proof of  \cite[Theorem 15]{dawar_symmetric_2024}, and this is independent of whether $\Gamma_n = \Sym_n$ or $\Gamma_n = \Sym_n \times \Sym_m$.
\end{proof}	

The converse of \cref{lem:constantSupportOfGates}, part 1, is also true:

\begin{lemma}
	\label{lem:constantSupportImpliesPolyOrbit}
	Let $(C_n)_{n \in \bbN}$ be a sequence of rigid $\Gamma$-symmetric circuits, for $\Gamma \in \{\Sym_n, \Sym_n \times \Sym_m\}$. Assume that there is a constant $k \in \bbN$ such that $\maxSup(C_n) \leq k$, for all $n \in \bbN$.
	Then $\maxOrb(C_n) \leq (n+m)^k$ (or $\maxOrb(C_n) \leq n^k$ if $\Gamma = \Sym_n$).
\end{lemma}
\begin{proof}		
	We consider the case $\Gamma = \Sym_n \times \Sym_m$, the other case is similar.
	Since $C_n$ is rigid, every $(\pi, \pi') \in \Sym_n \times \Sym_m$ has a unique extension to a circuit automorphism $\sigma$. 
	Let $g \in V(C_n)$ be a gate. Its stabiliser subgroup in $\Sym_n \times \Sym_m$ contains $\Stab
_{\Sym_n \times \Sym_m}(\vec{\sup}(g))$. 
	This means that for any two pairs $(\pi_1, \pi_1'), (\pi_2, \pi_2') \in \Sym_n \times \Sym_m$, it holds: If $(\pi_1, \pi_1')(\vec{\sup}(g)) = (\pi_2, \pi_2')(\vec{\sup}(g))$, and $\sigma_1, \sigma_2 \in \Aut(C_n)$ are the unique circuit automorphisms that $(\pi_1, \pi_1'), (\pi_2, \pi_2')$ extend to, then $\sigma_1(g) = \sigma_2(g)$. 
	Thus:
	\begin{align*}
		&|\{ \sigma(g) \mid \sigma \text{ is the unique circuit automorphism extending a } (\pi, \pi') \in \Sym_n \times \Sym_m   \}|\\
		= 
		&|\{ (\pi, \pi')(\vec{\sup}(g)) \mid (\pi, \pi') \in \Sym_n \times \Sym_m   \}| \leq (n+m)^k.
	\end{align*}
	Since $C_n$ does not have other circuit automorphisms due to its rigidity, this is indeed an upper bound on the orbit size of the gate.
\end{proof}
	
	

%\begin{lemma}
%		\label{lem:combinationOfRigidCircuits}
%		Let $C, C'$ be rigid $\Gamma$-symmetric algebraic circuits over the same variable set $\Xx$ and field $\bbF$.
%		Then for every $a,b \in \bbF$, there is a $\Gamma$-symmetric rigid circuit $C_+$ that computes $a \cdot C + b \cdot C'$.
%		In particular, $\maxOrb(C_+) \leq \max \{ \maxOrb(C), \maxOrb(C') \}$.
%		Likewise there is a $\Gamma$-symmetric rigid circuit $C_\times$ that computes $C \cdot C'$ and satisfies $\maxOrb(C_\cdot) \leq \max \{ \maxOrb(C), \maxOrb(C') \}$.
%\end{lemma}	
%\begin{proof}
%	Define $C_+$ in the obvious way as $C_+ = a \cdot C + b \cdot C'$. This circuit is $\Gamma$-symmetric because $C$ and $C'$ are. It may be that $C_+$ is not rigid. If this is the case, then there is at least one gate $g \in V(C_+)$ whose orbit under $\StabP_{\Aut(C_+)}(\Xx)$ is not a singleton (where $\Xx$ denotes the set of input gates in this context). Because $C$ and $C'$ are rigid, this orbit can only consist of exactly $2$ elements $g \in V(C), g' \in V(C')$. 
%	Now we rigidify $C_+$ using Lemma \ref{lem:rigidifyCircuits}. Let $C_+^r$ be the resulting circuit. In $C_+^r$, these gates $g$ and $g'$ are identified. If this happens, then by symmetry, every gate in $\Orb_\Gamma(g)$ in $C$ has such a counterpart in $\Orb_\Gamma(g') \subseteq V(C')$. So the two orbits $\Orb_\Gamma(g)$ and $\Orb_\Gamma(g')$ are identified in $C_+^r$ (gate-wise). 
%	Because $C_+^r$ is rigid, every permutation $\pi \in \Gamma$ acting on $\Xx$ has a unique extension $\sigma \in \Aut(C_+^r)$, and $\Aut(C_+^r)$ does not contain any other permutations. 
%	 The orbits of this action of $\Gamma$ on $C_+^r$ are exactly the orbits of $\Gamma$ in $C$ and $C'$, respectively, with the only difference that some of these orbits may now be shared between $C$ and $C'$, due to their identification. In any case, the sizes of these orbits are exactly like in $C$ and $C'$, so $C_+^r$ is the circuit with the desired properties. 
%	 The argument for $C_\times$ is the same.
%	\end{proof}

	
	
	

	
If we are considering a sequence $(C_n)_{n \in \bbN}$ of $\Gamma_n$-symmetric circuits, then we say they have \emph{polynomially bounded orbit size} if there is a polynomial $p(n)$ such that asymptotically, $\maxOrb(C_n) \leq p(n)$. Note that the overall size $\norm{C_n}$ may grow super-polynomially even if the orbit size is polynomially bounded. In that case, symmetry itself is not the reason why the circuits are large: They simply contain a super-polynomial number of distinct orbits.
	
	
	
\paragraph*{Relationship between supports of children and parent gates}
	%\begin{lemma}
	%	\label{lem:supportOfParentContainedInUnionOfChildSupports}
%		Let $C$ be a strongly rigid $\Gamma$-symmetric circuit, and $g \in V(C)$ an internal gate. Then $\sup(g) \subseteq \bigcup_{h \in \child(g)} \sup(h)$.
%	\end{lemma}	
%	\begin{proof}
	%	We just have to prove that $\bigcup_{h \in \child(g)} \sup(h)$ is a support of $g$. So let $\pi \in \Gamma$ such that $\pi$ pointwise fixes every element of $\bigcup_{h \in \child(g)} \sup(h)$. Then it fixes every child of $g$. Because $C$ is strongly rigid, $\pi$ must then also fix $g$ -- otherwise, $\pi(g) \neq g$ would be another gate with exactly the same set of children, and this would contradict the definition of strong rigidity. So $\bigcup_{h \in \child(g)} \sup(h)$ is a support of $g$, and its minimal support must be a subset of this.
%	\end{proof}	
	The children of a gate $g$ can be partitioned into orbits with respect to the action of $\Stab(g)$ because any permutation that induces a circuit automorphism fixing $g$ must also fix the children of $g$ setwise. We can also consider the orbit partition of children of $g$ with respect to $\StabP(\sup(g)) \leq \Stab(g)$, which may be finer. 
	\begin{lemma}
		\label{lem:intersectionOfChildSupports}
		Let $C$ be a $\Sym_n \times \Sym_m$-symmetric circuit such that $n,m \geq 3 \cdot \maxSup(C)$. 
		Let $g \in V(C)$ an internal gate, $h$ a child of $g$.
		Let $O_h \coloneqq \Orb_{\StabP(\sup(g))}(h)$ and $S(h) \coloneqq \bigcap_{h' \in O_h} \sup(h')$.
		Then $S(h) \subseteq \sup(g)$, and for every $h' \in O_h$, $(\sup(h') \setminus S(h)) \cap \sup(g) = \emptyset$. 
	\end{lemma}	
	\begin{proof}
		For the first part, let $s \in S(h)$, so in particular, $s \in \sup(h)$. Assume for a contradiction that $s \notin \sup(g)$. Then let $(\pi, \pi') \in \Sym_n \times \Sym_m$ be such that $(\pi, \pi')(\sup(h) \setminus \sup(g))$ does not contain $s$, and $(\pi, \pi')$ fixes $\sup(g)$ pointwise. Such a permutation exists because we are assuming $n$ and $m$ to be large enough compared to the support sizes. Then $h' \coloneqq (\pi, \pi')(h) \in O_h$ but $s \notin \sup(h')$, so $s \notin S(h)$. Contradiction.
		For the second part, assume for a contradiction that there exists $s \in (\sup(h') \setminus S(h)) \cap \sup(g)$, for some $h' \in O_h$. Since $s \in \sup(g)$, every permutation in $\StabP(\sup(g))$ fixes $s$. So $s$ must then be in $\sup(h')$ for every $h' \in O_h$. So it is in $S(h)$. Contradiction.	
	\end{proof}	
	
	
	

	

	%\begin{lemma}
	%\label{lem:supportOfParentExactlySupportOfChildren}
	%Let $C$ be a $\Sym_I \times \Sym_J$-symmetric circuit as in Lemma \ref{lem:intersectionOfChildSupports} that is additionally strongly rigid. Let $g \in V(C)$ an internal gate.
	%For a child $h$, let $S(h)$ be defined as in the previous lemma.
	%Then 
	%\[
	%\sup(g) = \bigcup_{h \in \child(g)} S(h). 
	%\]
	%\end{lemma}
	%\begin{proof}
	%By Lemma \ref{lem:intersectionOfChildSupports}, $\bigcup_{h \in \child(g)} S(h) \subseteq \sup(g)$.
	%For the reverse direction, assume for a contradiction that there exists some $s \in \sup(g)$ that is not in $S(h)$ for any $h \in \child(g)$. 
	%By Lemma \ref{lem:supportOfParentContainedInUnionOfChildSupports}, $s \in \sup(h)$ for some $h \in \child(g)$. But since $s \in \sup(g)$, it follows that $s \in \sup(h')$ for every $h' \in \Orb_{\StabP(\sup(g))}(h)$. Hence $s \in S(h)$. Contradiction.
	%\end{proof}
	


	
	\subsection{Graphs and Homomorphisms}
	
	A \emph{simple graph} is a tuple $G = (V(G), E(G))$ of a finite set $V(G)$ and a set $E(G) \subseteq \binom{V(G)}{2}$.
	A \emph{multigraph} is a tuple $G = (V(G), E(G))$ of a finite set $V(G)$ and a multiset $E(G)$ of elements in $\binom{V(G)}{2}$. 
	When a distinction is not necessary, we call both simple graphs and multigraphs \emph{graphs}.
	For a graph $G$, write $\norm{G} \coloneqq \abs{V(G)} + \abs{E(G)}$.
	
	Let $F$ and $G$ be graphs.
	A \emph{homomorphism} $h \colon F\to G$ is a map $h \colon V(F) \to V(G)$ such that $h(uv) \in E(G)$ for all $uv \in E(F)$.
	We write $\hom(F, G)$ for the number of homomorphisms from $F$ to $G$.
	
	Homomorphism counts behave nicely with respect to graph operations.
	For two graphs $F_1$, $F_2$, write $F_1 + F_2$ for their \emph{disjoint union}, i.e.\ $V(F_1 + F_2) \coloneqq V(F_1) \uplus V(F_2)$ and $E(F_1 + F_2) \coloneqq E(F_1) \uplus E(F_2)$.
	For multiple graphs $F_1, \dots, F_n$, we write $\coprod_{i=1}^n F_i$ for $F_1 + \dots + F_n$.
	For two graphs $G_1$, $G_2$, write $G_1 \times G_2$ for their \emph{categorical product}, i.e.\ $V(G_1 \times G_2) \coloneqq V(G_1) \times V(G_2)$ and $(v_1, v_2)$ and $(w_1, w_2)$ are adjacent in $G_1 \times G_2$ if, and only if, $v_1w_1 \in E(G_1)$ and $v_2w_2 \in E(G_2)$.
	The following identities hold for all graphs $F_1$, $F_2$, $G_1$, $G_2$, and all connected graphs $K$, cf.\ e.g.\ \cite[(5.28)--(5.30)]{lovasz_large_2012}:
		\begin{align}
		\hom(F_1 + F_2, G) &= \hom(F_1, G) \hom(F_2, G), \label{eq:coproduct} \\
		\hom(F, G_1 \times G_2) &= \hom(F, G_1) \hom(F, G_2), \text{ and }\label{eq:product} \\
		\hom(K, G_1 + G_2) &= \hom(K, G_1) + \hom(K, G_2). \label{eq:disjoint}
	\end{align}

	For graphs $G$ and $H$, write $G \boxplus H$ for the graph with vertex set $V(G \boxplus H) \coloneqq V(G) \uplus V(H)$ and $E(G \boxplus H) \coloneqq E(G) \uplus E(H) \uplus \{vw \mid v \in V(G), w\in V(H)\}$.
	For a graph $F$ and $U \subseteq V(F)$,
	write $F[U]$ for the \emph{subgraph induced by $U$} and $F -U$ for the subgraph obtained by deleting $U$, i.e.\ $V(F[U]) \coloneqq U$, $E(F[U]) \coloneqq E(F) \cap \binom{U}{2}$, and $F-U \coloneqq F[V(F) \setminus U]$.
	For every graph $F$, it holds that
	\begin{equation}\label{eq:boxplus}
		\hom(F, G \boxplus H) = \sum_{U \subseteq V(F)} \hom(F[U], G) \hom(F -U, H).
	\end{equation}

	For graphs $G_1$ and $G_2$, write $G_1 \cdot G_2$ for their \emph{lexicographic product}, i.e.\ $V(G_1 \cdot G_2) \coloneqq V(G_1) \times V(G_2)$ and $(v_1, v_2)$ and $(w_1, w_2)$ are adjacent in $G_1 \cdot G_2$ if, and only if, $v_1 = w_1$ and $v_2w_2 \in E(G_2)$ or $v_1w_1 \in E(G_1)$.
	For a graph $F$ and a partition $\mathcal{R}$ of $V(F)$,
	write $F/\mathcal{R}$ for the graph with vertex set $\mathcal{R}$ and $RS \in E(F/\mathcal{R})$ if, and only if, $R \neq S$ and there exist $r \in R$ and $s \in S$ such that $rs \in E(F)$.
	By \cite[Theorem~16]{seppelt_logical_2024},
	for every graph $F$,
	\begin{equation}\label{eq:lexprod}
		\hom(F, G_1 \cdot G_2) = \sum_{\mathcal{R} \in \Gamma(F)} \hom(F/\mathcal{R}) \hom(\coprod_{R \in \mathcal{R}} F[R]).
	\end{equation}	
	where $\Gamma(F)$ denotes the set of all partitions $\mathcal{R}$ of $V(F)$ such that for all $R \in \mathcal{R}$ the graph $F[R]$ is connected.

\subsection{Counting Logic and Bijective Games}        


Counting logic is first-order logic
augmented by counting quantifiers of the form $\exists^{\geq i} x$,
for every $i \in \bbN$. A graph (or another finite structure) $G$
satisfies a formula $\exists^{\geq i} x \phi(x)$, written as $G
\models \exists^{\geq i} x \phi(x)$, if there exist at least $i$ many
distinct $v \in V(G)$ such that $G \models \phi(v)$. The $k$-variable
fragment of counting logic is denoted $\Cc^k$.
More details on counting logic and references may be found
in~\cite{dawar_siglog15}).  For our purposes, the main interest is the
equivalence relation that this induces on graphs.  Two structures $G,H$
are called \emph{$\Cc^k$-equivalent}, denoted $G
\equiv_{\mathcal{C}^k} H$, if $G$ and $H$ satisfy exactly the same
sentences of $\Cc^k$.  This family of equivalence relations has many
characterizations, in combinatorics, logic, algebra and linear
optimization.  In particular, it is known that $G
\equiv_{\mathcal{C}^{k+1}} H$ if, and only if, $G$ and $H$ are not
  distinguished by the $k$-dimensional Weisfeiler-Leman test for graph
  isomorphism (see~\cite{CFI}).

 For our purposes, a particularly useful characterization is in terms
 of \emph{$k$-pebble bijective games}, introduced by
 Hella~\cite{hella96}.  This is a game played by two players called
 \emph{Spoiler} and \emph{Duplicator} on the graphs $G$ and $H$ using~$k$
pairs of pebbles~$(a_1,b_1),\dots,(a_k,b_k)$.  In a game position,
some (or all) of the pebbles~$a_1,\ldots,a_k$ are placed on vertices 
of~$G$ while the matching pebbles among~$b_1,\ldots,b_k$ are
placed on vertices of~$H$.  Where it causes no confusion, we
do not distinguish notationally between the pebble~$a_i$ (or~$b_i$)
and the vertex on which it is placed.  In each move of the game, Spoiler
chooses a pair of pebbles~$(a_i,b_i)$ and Duplicator has to respond by
giving a bijection~$f: V(G) \rightarrow V(H)$ which agrees with
the map~$a_j \mapsto b_j$ for all~$j \neq i$.    Spoiler chooses $a
\in V(G)$ and the pebbles $a_i$ and $b_i$ are placed on $a$ and $f(a)$
respectively.  If the resulting partial map from~$G$ to~$H$ given
by~$a_i \mapsto b_i$ is not a partial isomorphism, then Spoiler has
won the game.   We say that Duplicator has a \emph{winning
  strategy} if, no matter how Spoiler plays, she can play forever
without losing.  It is known that $G  \equiv_{\mathcal{C}^k} H$ if, and only if, Duplicator has a winning
strategy in the~$k$-pebble bijective game played on $G$ and $H$.

Cai, F\"urer and Immerman~\cite{CFI} showed that for each $k$ there
are graphs $G$ and $H$ with $O(k)$ vertices which are not
isomorphic but such that $G  \equiv_{\mathcal{C}^k} H$.  The graphs
$G$ and $H$ are constructed from a base graph $F$ which is
sufficiently richly connected (in particular it has treewidth larger
than $k$) by replacing each vertex $v$ of $F$ by a small gadget whose
size depends on the degree of $v$.  The construction is known as the
\textsmaller{CFI} construction, and the gadgets as the
\textsmaller{CFI} gadgets.  There are many variations of the
construction in the literature, and we rely on the construction
from~\cite{roberson_oddomorphisms_2022}, known as the
\textsmaller{CFI} construction without without internal
vertices (see als~\cite{rossman_csl25}).


	
	
	\section{Circuits with Bounded Support and Homomorphism Polynomials}
	\label{sec:circuitsAndHomPolynomials}
	In this section, we prove \cref{thm:main1} which characterises the polynomials which admit $\Sym_n \times \Sym_m$-symmetric circuits of polynomial orbit size.
	To that end, we first observe that  the polynomials which admit $\Sym_n \times \Sym_m$-symmetric circuits of arbitrary orbit size are precisely the linear combinations of subgraph or homomorphism polynomials.
	
	Let $F$ be a bipartite multigraph with bipartition $A \uplus B$.
	For integers $n,m \in \mathbb{N}$,
	the \emph{homomorphism polynomial $\hom_{F, n,m} \in \mathbb{Q}[\mathcal{X}_{n,m}]$} and the \emph{subgraph polynomial $\sub_{F, n,m} \in \mathbb{Q}[\mathcal{X}_{n,m}]$} of $F$ are defined as
	\begin{align*}
		\hom_{F, n, m} &\coloneqq \sum_{h \colon A \uplus B \to [n] \uplus [m]} \prod_{ab \in E(F)} x_{h(a)h(b)}, \\
		\sub_{F, n, m} &\coloneqq \frac{1}{|\Aut(F)|}\sum_{h \colon A \uplus B \hookrightarrow [n] \uplus [m]} \prod_{ab \in E(F)} x_{h(a)h(b)}.
	\end{align*}
	Here, we think of $h$ as mapping $A$ to $[n]$ and $B$ to $[m]$.
	For example, $\hom_{K_2, n, m} = \sum_{v \in [n]} \sum_{w \in [m]} x_{vw}$ and $\sub_{K_{n,m}, n,m} = \prod_{v\in [n]} \prod_{w \in [m]} x_{vw}$. 
	If $A_G \in \{0,1\}^{[n] \times [m]}$ is the bi-adjacency matrix of an undirected bipartite graph $G$ with bipartition $V(G) = [n] \uplus [m]$, then $\hom_{F,n,m}(A_G)$ evaluates to the number of bipartition-preserving homomorphisms from $F$ to $G$, and $\sub_{F, n,m}(A_G)$ is the number of occurrences of $F$ as a subgraph in $G$. We also write $\hom_{F,n,m}(G)$ and $\sub_{F, n,m}(G)$ for the polynomials evaluated in the bi-adjacency matrix of $G$.
	
	For every multigraph $F$, the coefficients of all monomials in $\sub_{F,n,m}$ are $1$. If $F$ is a simple graph, then $\sub_{F,n,m}$ is multilinear. 
	This is generally not the case for $\hom_{F, n,m}$, not even when $F$ is simple.
%	Note that every monomial in $\sub_{F, n, m}$ has coefficient $1$ and not $\frac{1}{|\Aut(F)|}$.
	
	As a first step, we show that the set
	\[
		\mathfrak{G}_{n,m} \coloneqq \left\{\sum \alpha_i \hom_{F_i, n,m} \mid \text{bipartite multigraphs } F_i, \alpha_i \in \mathbb{Q} \right\} \subseteq \mathbb{Q}[\mathcal{X}_{n,m}].
	\]
	contains precisely the $\Sym_n \times \Sym_m$-symmetric polynomials:
	
	\begin{lemma}\label{lem:gnm-sym}
		For $n,m \in \mathbb{N}$ and $p \in \mathbb{Q}[\mathcal{X}_{n,m}]$,
		the following are equivalent:
		\begin{enumerate}
			\item the polynomial $p$ is $\Sym_n \times \Sym_m$-symmetric,\label{it:gnm1}
			\item $p \in \mathfrak{G}_{n,m}$,\label{it:gnm2}
			\item $p = \sum \alpha_i \sub_{F_i, n,m}$ for some bipartite multigraphs $F_i$ and $\alpha_i \in \mathbb{Q}$.\label{it:gnm3}
		\end{enumerate}
	\end{lemma}
	\begin{proof}
		For \ref{it:gnm1} $\Rightarrow$ \ref{it:gnm3},
		partition the monomial set of $p$ into its $\Sym_n \times \Sym_m$-orbits. 
		We aim to show that each orbit is the polynomial $\sub_{F,n,m}$, for some $F$, multiplied with some non-zero constant. 
		Let $\Omega$ be such an orbit, i.e.\ a set of monomials. 
		Let $q \in \Omega$ denote an arbitrary representative. 
		It describes a bipartite graph $F$ with bipartition $V(F) = \{ i \in [n] \mid \text{ there exists } j \text{ s.t. } x_{ij} \text{ divides } q \}  \allowbreak  \uplus \allowbreak \{ j \in [m] \mid \text{ there exists } i \text{ s.t. } x_{ij} \text{ divides } q \}$.
		An edge $ij$ appears in $F$ with the multiplicity given by the degree of $x_{ij}$ in $q$.
		
		Now $q$ encodes the identity embedding $\iota_{\id} \colon V \hookrightarrow [n] \uplus [m]$. 
		For every $(\pi,\sigma) \in \Sym_n \times \Sym_m$, 
		we have $(\pi, \sigma)(q) = \prod_{vw \in E(F)} x_{\pi(\iota_{\id}(v))\sigma(\iota_{\id}(w))}$. 
		Each embedding $\iota \colon V \hookrightarrow [n] \uplus [m]$ can be written as $(\pi,\sigma) \circ \iota_{\id}$ for some $(\pi,\sigma) \in \Sym_n \times \Sym_m$. 
		Therefore, the orbit $\Omega$ is indeed $\sub_{F,n,m}$, multiplied with some constant.
		
		The implication \ref{it:gnm3} $\Rightarrow$ \ref{it:gnm2} follows from \cref{thm:sub-hom}.
		For \ref{it:gnm2} $\Rightarrow$ \ref{it:gnm1}, let $F$ be a bipartite multigraph and $(\pi, \sigma) \in \Sym_n \times \Sym_m$.
		Then
		\[
			(\pi,\sigma)(\hom_{F, n, m}) = \sum_{h \colon A \uplus B \to [n] \uplus [m]} \prod_{ab \in E(F)} x_{\pi(h(a)) \sigma(h(b))}
			= \sum_{h \colon A \uplus B \to [n] \uplus [m]} \prod_{ab \in E(F)} x_{h(a)) h(b)} = \hom_{F, n,m}
		\]
		observing that the set of maps $h \colon A \uplus B \to [n] \uplus [m]$ is in bijection with the set of maps $(\pi, \sigma) \circ h$ for $h \colon A \uplus B \to [n] \uplus [m]$.
	\end{proof}
	\begin{remark} 
		\label{rem:squareSymmetricPolynomials}
	One can similarly characterise the $\Sym_n$-symmetric polynomials in terms of homomorphism/subgraph polynomials for \emph{directed} and not necessarily bipartite graphs. A $\Sym_n$-symmetric polynomial is one that is fixed by every $\pi \in \Sym_n$ acting on $\Xx_{n,n}$ by mapping $x_{ij}$ to $x_{\pi(i)\pi(j)}$. In fact, we believe that all or most of the theory we develop in this paper for $\Sym_n \times \Sym_m$-symmetric polynomials applies to $\Sym_n$-symmetric ones as well, but we prove our results only for the $\Sym_n \times \Sym_m$-symmetric setting.
	\end{remark}	
	Even though the definition of $\mathfrak{G}_{n,m}$ involves homomorphism polynomials of infinitely many graphs,
	it follows from \cref{lem:gnm-sym} that, for every $n,m\in \mathbb{N}$, finitely many such polynomials already generate $\mathfrak{G}_{n,m}$ as a $\mathbb{Q}$-algebra:
	
	\begin{corollary}
		For every $n,m \in \mathbb{N}$,
		there exist finitely many bipartite multigraphs $F_1, \dots, F_r$ such that, for every $p \in \mathfrak{G}_{n,m}$,
		there exists a polynomial $q \in \mathbb{Q}[y_1, \dots, y_r]$ such that 
		\(
		p = q(\hom_{F_1, n,m}, \dots, \hom_{F_r, n,m}).
		\)
	\end{corollary}
	\begin{proof}
		By the invariant theory of finite groups, cf.\ e.g.\ \cite[Proposition~3.0.1]{derksen_computational_2015}, 
		the set of $\Sym_n \times \Sym_m$-symmetric polynomials in $\mathbb{Q}[\mathcal{X}_{n,m}]$ is finitely generated as a $\mathbb{Q}$-algebra.
		By \cref{lem:gnm-sym}, each of its generators is a finite linear combination of homomorphism polynomials from some bipartite multigraphs.
		Hence, the set of $\Sym_n \times \Sym_m$-symmetric polynomials in $\mathbb{Q}[\mathcal{X}_{n,m}]$ is equal to the $\mathbb{Q}$-algebra generated by some finite set of homomorphism polynomials of bipartite multigraphs.
	\end{proof}
	
	We now turn to the characterisation of symmetric polynomials admitting symmetric circuits of polynomial orbit size.
	To that end, for $k,n,m \in \mathbb{N}$, write
	\[
		\mathfrak{T}_{n,m}^k \coloneqq \left\{\sum \alpha_i \hom_{F_i, n,m} \ \middle|\  \text{bipartite multigraphs } F_i \text{ such that } \tw(F_i) < k, \alpha_i \in \mathbb{Q} \right\} \subseteq \mathfrak{G}_{n,m}.
	\]
	for the set of finite $\mathbb{Q}$-linear combinations of homomorphism polynomials of bipartite multigraphs of treewidth less than~$k$.
	Our main theorem is the following:

	\thmMain*
	
	As a first step towards \cref{thm:main1}, we remark that homomorphism polynomials of graphs of bounded treewidth admit polynomial-size symmetric circuits. 
	
	\begin{theorem}[restate=thmHomCircuit, label=] \label{thm:hom-circuit-main}
		Let $k, n, m \in \mathbb{N}$.
		Let $F$ be a bipartite multigraph of treewidth less than~$k$.
		Then there exists a $\Sym_n \times \Sym_{m}$-symmetric rigid circuit $C_F$ for $\hom_{F,n,m}$ of size at most $5k!^2k \cdot (n+m)^{k+1} \cdot \norm{F}^2$ satisfying $\maxSup(C_F) \leq k$.
	\end{theorem}
	
	\Cref{thm:hom-circuit-main} yields the forward implication in \cref{thm:main1}: If $p \in \mathfrak{T}_{n,m}^k$, then it is a linear combination of polynomials $\hom_{F,n,m}$, where $\tw(F) \leq k$ for each $F$ in the linear combination. With \cref{thm:hom-circuit-main}, we get a rigid symmetric circuit $C_F$ with $\maxSup(C_F) \leq k$ for each of these $\hom_{F,n,m}$. To obtain $p$, we just take a linear combination of these circuits and rigidify it again using \cref{lem:rigidifyCircuits}. This yields a rigid circuit $C$ representing $p$ that still satisfies $\maxSup(C) \leq k$. By \cref{lem:constantSupportImpliesPolyOrbit}, it has orbit size at most $(n+m)^k$.  
	
	The remainder of this section is concerned with the backward implication in \cref{thm:main1} and the proof of \cref{thm:hom-circuit-main}.
	
	
	


	
	\subsection{Homomorphism Polynomials of Labelled Graphs}

	The main tool for proving \cref{thm:main1} are homomorphism polynomials of labelled graphs.
		
	\begin{definition}\label{def:labelled-hom-poly}
		Let $\ell, r \in \mathbb{N}$.
		An \emph{$(\ell, r)$-labelled bipartite graph} $\boldsymbol{F} = (F, \boldsymbol{a}, \boldsymbol{b})$ is a tuple of a bipartite multigraph $F$ with bipartition $A \uplus B = V(F)$,
		an $\ell$-tuple $\boldsymbol{a} \in A^\ell$, and an $r$-tuple $\boldsymbol{b} \in B^r$.
		Write $\mathcal{G}(\ell, r)$ for the class of all $(\ell, r)$-labelled bipartite graphs.
		
		For $n, m \in \mathbb{N}$ and tuples $\boldsymbol{v} \in [n]^\ell$ and $\boldsymbol{w} \in [m]^r$, write
		\[
		\boldsymbol{F}_{n,m}(\boldsymbol{v}, \boldsymbol{w}) \coloneqq \sum_{\substack{h \colon A \uplus B \to [n] \uplus [m]\\ h(\boldsymbol{a}) = \boldsymbol{v} \\ h(\boldsymbol{b}) = \boldsymbol{w}}} \prod_{ab \in E(F)} x_{h(a)h(b)} \in \mathbb{Q}[\mathcal{X}_{n,m}]
		\]
		for the \emph{homomorphism polynomial of $\boldsymbol{F}$ at $\boldsymbol{v},  \boldsymbol{w}$}.
		The map $\boldsymbol{F}_{n,m} \colon [n]^\ell \times [m]^r \to \mathbb{Q}[\mathcal{X}_{n,m}]$ given by $(\boldsymbol{v}, \boldsymbol{w}) \mapsto \boldsymbol{F}_{n,m}(\boldsymbol{v}, \boldsymbol{w})$ is the \emph{homomorphism polynomial map} of $\boldsymbol{F}$. Write
		\[
		\mathfrak{G}_{n,m}(\ell, r) \coloneqq \left\{\sum \alpha_i \boldsymbol{F}^i_{n,m} \ \middle|\  \alpha_i \in \mathbb{Q}, \boldsymbol{F}^i \in \mathcal{G}(\ell, r)\right\} \subseteq \mathbb{Q}[\mathcal{X}_{n,m}]^{[n]^\ell \times [m]^r}
		\]
		for the set of finite $\mathbb{Q}$-linear combinations of homomorphism polynomial maps of $(\ell,r)$-labelled bipartite graphs.
	\end{definition}

	Note that for $\ell = 0 = r$, the homomorphism polynomial map $\boldsymbol{F}_{n,m}$ simplifies to the homomorphism polynomial $\hom_{F, n, m}$, i.e.\ essentially $\mathfrak{G}_{n,m}(0,0) = \mathfrak{G}_{n,m}$. 
	
	\begin{example}\label{ex:edge}
		Let $\boldsymbol{F} = (F, a, b)$ denote the $(1,1)$-labelled bipartite graph with $V(F) = \{a,b\}$ and $E(F) = \{ab\}$.
		For $n,m \in \mathbb{N}$, $v \in [n]$, and $w \in [m]$,
		we have $\boldsymbol{F}_{n,m}(v, w) = x_{vw} \in \mathbb{Q}[\mathcal{X}_{n,m}]$.
	\end{example}
	
	Next, we define what it means for a labelled bipartite graph to have bounded treewidth.
	Furthermore, we generalise $\mathfrak{T}^k_{n,m}$.


	\begin{definition}\label{def:treewidth-labelled}
		Let $\ell, r, k \in \mathbb{N}$.
		An \emph{$(\ell, r)$-labelled bipartite graph} $\boldsymbol{F} = (F, \boldsymbol{a}, \boldsymbol{b})$ has \emph{treewidth} less than $k$ if there exists a tree decomposition $(T, \beta)$ of $F$ of width less than $k$ with a vertex $s \in V(T)$ such that $a_1, \dots, a_\ell, b_1, \dots, b_r \in \beta(s)$.
		Write $\mathcal{T}^k(\ell, r) \subseteq \mathcal{G}(\ell, r)$ for the class of all $(\ell, r)$-labelled bipartite graphs of treewidth less than~$k$.
		For $n,m \in \mathbb{N}$, write
		\[
			\mathfrak{T}_{n,m}^k(\ell, r) \coloneqq \left\{\sum \alpha_i \boldsymbol{F}^i_{n,m} \ \middle|\  \alpha_i \in \mathbb{Q}, \boldsymbol{F}^i \in \mathcal{T}^k(\ell, r)\right\}
			\subseteq \mathfrak{G}_{n,m}(\ell, r)
		\]
		for the set of $\mathbb{Q}$-linear combinations of homomorphism polynomial maps of $(\ell,r)$-labelled bipartite graphs of treewidth less than~$k$.
	\end{definition}

	Note that essentially $\mathfrak{T}^k_{n,m} = \mathfrak{T}^k_{n,m}(0,0)$. For $\boldsymbol{v} \in [n]^\ell, \boldsymbol{w} \in [m]^r$, we also define the set of \emph{instantiated homomorphism polynomial maps} as  
	\(
	\mathfrak{T}_{n,m}^k(\boldsymbol{v},\boldsymbol{w}) \coloneqq \left\{ \phi(\boldsymbol{v}, \boldsymbol{w}) \ \middle|\  \phi \in \mathfrak{T}^k_{n,m}(\ell,r) \right\} \subseteq \mathbb{Q}[\mathcal{X}_{n,m}].
	\)
	
	

	\begin{example}\label{ex:one}
		For $\ell, r \in \mathbb{N}$,
		let $\boldsymbol{J} = (J, \boldsymbol{a}, \boldsymbol{b})$ be the $(\ell, r)$-labelled edge-less $(\ell, r)$-vertex bipartite graphs with labels residing on distinct vertices. 
		Clearly, $\boldsymbol{J} \in \mathcal{T}^{\ell+r}(\ell, r)$.
		For all $n, m \in \mathbb{N}$, $\boldsymbol{v} \in [n]^\ell$, and $\boldsymbol{w} \in [m]^r$,
		\[
			\boldsymbol{J}_{n,m}(\boldsymbol{v}, \boldsymbol{w}) \coloneqq \sum_{\substack{h \colon A \uplus B \to [n] \uplus [m]\\ h(\boldsymbol{a}) = \boldsymbol{v} \\ h(\boldsymbol{b}) = \boldsymbol{w}}} 1 = 1
		\]
		That is, $\boldsymbol{J}_{n,m} \in \mathfrak{T}^{\ell+r}_{n,m}(\ell, r)$ is the homomorphism polynomial map which maps all tuples to $1$.
	\end{example}

	
	

	\subsection{Operations on Labelled Graphs and Homomorphism Polynomial Maps}
	\label{sec:ops}
	
	Fix  $n,m, \ell, r, k \in \mathbb{N}$ throughout this section.
	In this section, we prove various closure properties of $\mathfrak{T}_{n,m}^k(\ell, r)$.
	\subsubsection{Swapping Left and Right Labels}
	
	We first note that the left and right labels can be interchanged.
	In order to simplify notation, we subsequently focus on operations involving only the left labels.
	The following purely syntactic lemma justifies this approach.
	To that end,
	define for $\boldsymbol{F} = (F, \boldsymbol{a}, \boldsymbol{b}) \in \mathcal{G}(\ell, r)$
	the $(r,\ell)$-labelled bipartite graph
	 $\boldsymbol{F}^* \coloneqq (F^*, \boldsymbol{b}, \boldsymbol{a})  \in \mathcal{G}(r, \ell)$ where $F^*$ is the bipartite multigraph obtained from $F$ by interchanging the parts of the bipartition.
	Analogously, for $\phi \in \mathfrak{G}_{n,m}(\ell, r)$,
	define $\phi^* \in \mathfrak{G}_{m,n}(r, \ell)$ via
        $\phi^*(\boldsymbol{w}, \boldsymbol{v}) \coloneqq
        \phi(\boldsymbol{v}, \boldsymbol{w})$ for all $\boldsymbol{v}
        \in [n]^\ell$ and $\boldsymbol{w} \in [m]^r$.   The following
        is then immediate.
	%TODO: Think about what this is really needed for?
	\begin{lemma}\label{lem:reverse}
		Let $\boldsymbol{F} \in \mathcal{G}(\ell, r)$
		and $\phi \in \mathfrak{G}_{n,m}(\ell, r)$.
		\begin{enumerate}
			\item If $\boldsymbol{F} \in \mathcal{T}^k(\ell, r)$,
			then $\boldsymbol{F}^* \in \mathcal{T}^k(r, \ell)$.
			\item For $\boldsymbol{v} \in [n]^\ell$ and
                          $\boldsymbol{w} \in [m]^r$, we have $\boldsymbol{F}_{n,m}(\boldsymbol{v}, \boldsymbol{w}) = \boldsymbol{F}^*_{m,n}(\boldsymbol{w}, \boldsymbol{v})$.
			\item If $\phi \in \mathfrak{T}^k_{n,m}(\ell, r)$, then $\phi^* \in \mathfrak{T}^k_{m,n}(r, \ell)$.
		\end{enumerate}
		
	\end{lemma}
	
	\subsubsection{Unlabelling and Sums}
	Next, we consider the unlabelling operation.
	For $\ell \geq 1$, $\boldsymbol{F} = (F, \boldsymbol{a}, \boldsymbol{b}) \in \mathcal{G}(\ell, r)$, and $i \in [\ell]$,
	define $\Sigma_i \boldsymbol{F} \coloneqq (F, \boldsymbol{a}[i/], \boldsymbol{b}) \in \mathcal{G}(\ell-1, r)$
	as the graph obtained by dropping the $i$-th left label.
	Analogously, 
	for $\phi \in \mathfrak{G}_{n,m}(\ell, r)$,
	define $\Sigma_i\phi \in \mathfrak{G}_{n,m}(\ell-1, r)$ via $(\Sigma_i \phi)(\boldsymbol{v}[i/], \boldsymbol{w}) \coloneqq \sum_{v \in [n]}\phi(\boldsymbol{v}[i/v], \boldsymbol{w})$ for all $\boldsymbol{v} \in [n]^\ell$ and $\boldsymbol{w} \in [m]^r$. 

	\begin{lemma}\label{lem:unlabelling}
		For $\ell \geq 1$,
		let $\boldsymbol{F} \in \mathcal{G}(\ell, r)$
		and $\phi \in \mathfrak{G}_{n,m}(\ell, r)$.
		\begin{enumerate}
			\item If $\boldsymbol{F} \in \mathcal{T}^k(\ell, r)$,
			then $\Sigma_i\boldsymbol{F} \in \mathcal{T}^k(\ell-1, r)$.
			\item For $\boldsymbol{v} \in [n]^\ell$ and $\boldsymbol{w} \in [m]^r$,  it is
					\[
			(\Sigma_i \boldsymbol{F})_{n,m}(\boldsymbol{v}[i/],\boldsymbol{w}) = \sum_{v \in [n]} \boldsymbol{F}_{n,m}(\boldsymbol{v}[i/v], \boldsymbol{w}).
			\]
			\item If $\phi \in \mathfrak{T}^k_{n,m}(\ell, r)$, then $\Sigma_i \phi \in \mathfrak{T}^k_{n,m}(\ell-1, r)$. 
		\end{enumerate}
	\end{lemma}

	\subsubsection{Disjoint Union and Tensor Products}

	The \emph{disjoint union} of labelled graphs is defined as follows:
	For elements $\boldsymbol{F} = (F, \boldsymbol{a}, \boldsymbol{b}) \in \mathcal{G}(\ell, r)$ and $\boldsymbol{F}' = (F', \boldsymbol{a}', \boldsymbol{b}') \in \mathcal{G}(\ell', r')$,
	define $\boldsymbol{F} \otimes \boldsymbol{F}' \in \mathcal{G}(\ell+\ell', r+r')$
	as $(F \uplus F', \boldsymbol{a}\boldsymbol{a}', \boldsymbol{b}\boldsymbol{b}')$.
	Analogously, for $\phi \in \mathfrak{G}(\ell ,r )$ and $\psi \in \mathfrak{G}(\ell', r')$,
	define $\phi \otimes \psi \in \mathfrak{G}(\ell+\ell', r+r')$ by $(\phi \otimes \psi)(\boldsymbol{v}, \boldsymbol{w}) \coloneqq \phi(v_1 \dots v_\ell, w_1 \dots w_\ell) \psi(v_{\ell+1} \dots v_{\ell+\ell'}, w_{r+1} \dots w_{r+r'})$ for all $\boldsymbol{v} \in [n]^{\ell+\ell'}$ and $\boldsymbol{w} \in [m]^{r+r'}$.
	\begin{lemma}\label{lem:disjoint-union}
		Let $\boldsymbol{F} \in \mathcal{G}(\ell, r)$ and $\boldsymbol{F}' \in \mathcal{G}(\ell', r')$.

		Let $\phi \in \mathfrak{G}(\ell, r)$ and $\psi \in \mathfrak{G}(\ell', r')$.
		Then
		\begin{enumerate}
			\item If $\boldsymbol{F} \in \mathcal{T}^k(\ell, r)$ and $\boldsymbol{F}' \in \mathcal{T}^{k'}(\ell', r')$, then $\boldsymbol{F} \otimes \boldsymbol{F}' \in \mathcal{T}^{k''}(\ell+\ell', r+r')$ for $k'' \coloneqq \max\{k,k', \ell+\ell'+r+r'\}$.
			\item For $\boldsymbol{v} \in [n]^{\ell+\ell'}$ and $\boldsymbol{w} \in [m]^{r+r'}$, 
			\[
				(\boldsymbol{F} \otimes \boldsymbol{F}')_{n,m}(\boldsymbol{v}, \boldsymbol{w}) = \boldsymbol{F}_{n,m}(v_1 \dots v_\ell, w_1 \dots w_r) \boldsymbol{F}'_{n,m}(v_{\ell+1} \dots v_{\ell+ \ell'}, w_{r+1} \dots w_{r+r'}).
			\]
			\item If $\phi \in \mathfrak{T}^k(\ell, r)$ and $\phi' \in \mathfrak{T}^{k'}(\ell', r')$,
			then $\phi \otimes \phi' \in \mathfrak{T}^{k''}(\ell+\ell', r+r')$ for $k'' \coloneqq \max\{k,k', \ell+\ell'+r+r'\}$.
		\end{enumerate}
	\end{lemma}
	\begin{proof}
		For the first assertion, note that a tree decomposition for $\boldsymbol{F} \otimes \boldsymbol{F}'$ can be constructed by making the root nodes of the tree decompositions of $\boldsymbol{F}$ and $\boldsymbol{F}'$ adjacent to a fresh node containing the vertices labelled in $\boldsymbol{F} \otimes \boldsymbol{F}'$.
		This yields a tree decomposition of width $k'' \coloneqq \max\{k,k', \ell+\ell'+r+r'\}$.
	\end{proof}

	\subsubsection{Gluing and Point-wise Products}
	The final elementary operation is gluing.
	For elements $\boldsymbol{F} = (F, \boldsymbol{a}, \boldsymbol{b})$ and $\boldsymbol{F}' = (F', \boldsymbol{a}', \boldsymbol{b}')$ of $\mathcal{G}(\ell, r)$,
	define $\boldsymbol{F} \odot \boldsymbol{F}' \in \mathcal{G}(\ell, r)$
	by taking the disjoint union of $F$ and $F'$ and identifying for $i \in [\ell]$ and $j \in [r]$ the vertices $a_i$ with $a'_i$ and $b_i$ with $b'_i$.
	Since left/right labels come from the left/right side of the bipartitions the resulting graph is indeed bipartite.
	If $\ell = 0 = r$, then gluing amounts to taking a disjoint union.
	
	For $\phi, \phi' \in \mathfrak{G}_{n,m}(\ell, r)$, define  the point-wise product $\phi \odot \phi' \in \mathfrak{G}_{n,m}(\ell, r)$ via $(\phi \odot \phi')(\boldsymbol{v}, \boldsymbol{w}) \coloneqq \phi(\boldsymbol{v}, \boldsymbol{w}) \phi'(\boldsymbol{v}, \boldsymbol{w})$.
	When convenient, we write $\phi \phi'$ for the point-wise product $\phi \odot \phi'$.

	\begin{lemma}\label{lem:gluing}
		Let $\boldsymbol{F}, \boldsymbol{F}' \in \mathcal{G}(\ell, r)$.
		Let $\phi, \phi' \in \mathfrak{T}^k_{n,m}(\ell, r)$.
		\begin{enumerate}
			\item If $\boldsymbol{F}, \boldsymbol{F}' \in \mathcal{T}^k(\ell, r)$,
			then $\boldsymbol{F} \odot \boldsymbol{F}' \in \mathcal{T}^k(\ell, r)$.
			\item For $\boldsymbol{v} \in [n]^\ell$ and $\boldsymbol{w} \in [m]^r$,  			\[
				(\boldsymbol{F} \odot \boldsymbol{F}')_{n,m}(\boldsymbol{v}, \boldsymbol{w})  = \boldsymbol{F}_{n,m}(\boldsymbol{v}, \boldsymbol{w}) \boldsymbol{F}'_{n,m}(\boldsymbol{v}, \boldsymbol{w}).
			\]
			\item The point-wise product $\phi \odot \phi'$ of $\phi$ and $\phi'$ is in $\mathfrak{T}^k_{n,m}(\ell, r)$.
		\end{enumerate}
	\end{lemma}

	\Cref{lem:gluing} shows that $\mathfrak{T}^k_{n,m}(\ell,r)$ forms a $\mathbb{Q}$-algebra for all $k,n,m, \ell, r \in \mathbb{N}$.
	If $\ell +r \leq k$, then this algebra is unital by \cref{ex:one}.
		
	 
 	\subsubsection{Restricted Sums}
	As a modification of \cref{lem:unlabelling},
	we consider the following restricted sum operation.
	It is needed when we analyse the semantics of summation gates in dependence of their support.
	Let $i \in [\ell]$ and $J \subseteq [\ell] \setminus \{i\}$.
	For $\phi \in \mathfrak{G}_{n,m}(\ell, r)$,
	define $\Sigma_{i, J} \phi \in \mathfrak{G}_{n,m}(\ell-1, r)$
	via \[ (\Sigma_{i, J} \phi)(\boldsymbol{v}[i/], \boldsymbol{w}) \coloneqq \sum_{v \in [n] \setminus \{v_j \mid j \in J\}} \phi(\boldsymbol{v}[i/v], \boldsymbol{w}). \]
	for all $\boldsymbol{v} \in [n]^\ell$ and $\boldsymbol{w} \in [m]^r$.
	Note that $\Sigma_i \phi = \Sigma_{i, \emptyset} \phi$.
	
	\begin{lemma}\label{lem:sum-exclude-lincomb}
		Suppose that $\ell+r \leq k$.
		For $\phi \in \mathfrak{T}^k_{n,m}(\ell,r )$,  $i \in [\ell]$, and $J \subseteq [\ell] \setminus \{i\}$,
		we have that $\Sigma_{i, J} (\phi) \in \mathfrak{T}^k_{n,m}(\ell-1, r)$.
	\end{lemma}
	\begin{proof}
		For distinct $i, j \in [\ell]$, 
		let $\boldsymbol{D}^{i,j} \in \mathcal{T}^{k}(\ell, r)$ be the $(\ell, r)$-labelled edge-less $(\ell-1,r)$-vertex bipartite graph whose labels are placed such that the $i$-th and the $j$-th left label
		reside on the same vertex while all other labels are carried by distinct vertices.
		Note that, for all $\boldsymbol{v} \in [n]^{\ell}$ and $\boldsymbol{w} \in [m]^r$,
		\begin{equation}\label{eq:diell}
			\boldsymbol{D}^{i,j}_{n,m}(\boldsymbol{v}, \boldsymbol{w}) = \begin{cases}
				1,& v_i = v_j, \\
				0,& \text{otherwise}.
			\end{cases}
		\end{equation}
		Let $p \in \mathbb{Q}[x]$ be a polynomial such that $p(0) = 1$ and $p(1) = \dots = p(\ell) = 0$.
		By applying \cref{lem:gluing,ex:one},
		define $\delta \coloneqq p(\sum_{j \in J} \boldsymbol{D}^{i,j}_{n,m}) \in \mathfrak{T}^k_{n,m}(\ell, r)$.
		Note that, for all $\boldsymbol{v} \in [n]^{\ell}$ and $\boldsymbol{w} \in [m]^r$,
		\[
			\delta(\boldsymbol{v}, \boldsymbol{w}) = \begin{cases}
				1, & v_i  \not\in \{v_j \mid j \in J\},\\
				0, & \text{otherwise}.
			\end{cases}
		\]
		Hence, $\sum_{i, J} \phi = \sum_{i} (\delta \odot \phi)$.
		The claim follows from \cref{lem:gluing,lem:unlabelling}.
	\end{proof}

	\subsubsection{Products}
	
	Suppose that $\ell \geq 1$.
	In this section, we consider the following product operator:
	For $\phi \in \mathfrak{G}_{n,m}(\ell, r)$ and $i \in [\ell]$,
	define $\Pi_i \phi \in \mathfrak{G}_{n,m}(\ell-1, r)$ via 
	\[ (\Pi_i \phi)(\boldsymbol{v}[i/], \boldsymbol{w}) \coloneqq \prod_{v \in [n]} \phi(\boldsymbol{v}[i/v], \boldsymbol{w}) \] 
	for all $\boldsymbol{v} \in [n]^\ell$ and $\boldsymbol{w} \in [m]^k$.
	Note that, in contrast to $\Sigma_i$, and $\odot$,
	the operator $\Pi_i$ is not linear.
	Despite that, the following \cref{thm:product-lincomb} shows that applying $\Pi_i$ to a linear combination of homomorphism polynomial maps yields a linear combination of homomorphism polynomial maps.
	This theorem is the most important technical novelty in the proof of \cref{thm:main1}.
	
	\begin{theorem}\label{thm:product-lincomb}
		For $\phi \in \mathfrak{T}^k_{n,m}(\ell,r )$ and $i \in [\ell]$,
		we have that $\Pi_i\phi \in \mathfrak{T}^k_{n,m}(\ell-1, r)$.
	\end{theorem}
	
	
	\Cref{thm:product-lincomb} is proved in several steps.
	We first show in \cref{lem:product-one-graph} that $\Pi_i \boldsymbol{F}_{n,m} \in \mathfrak{T}^k_{n,m}(\ell-1, r)$ for all $\boldsymbol{F} \in \mathcal{T}^k(\ell, r)$.
	To that end, consider the following lemma.
	
	\begin{lemma}\label{lem:pi-i-pi}
		Let $\boldsymbol{F}^1, \dots, \boldsymbol{F}^n \in \mathcal{T}^k(\ell, r)$ and $i \in [\ell]$.
		For a partition $\pi$ of $[n]$,
		define the $(\ell-1, r)$-labelled bipartite graph
		\(
			\Pi_i^\pi(\boldsymbol{F}^1, \dots, \boldsymbol{F}^n) \coloneqq \bigodot_{P \in [n]/\pi} \Sigma_i(\bigodot_{v \in P} \boldsymbol{F}^v) \in \mathcal{G}(\ell-1, r).
		\)
		Then
		\[
			(\Pi_i^\pi(\boldsymbol{F}^1, \dots, \boldsymbol{F}^n))_{n,m}(\boldsymbol{v}[i/], \boldsymbol{w}) = \sum_{h \colon [n]/\pi \to [n]}  \prod_{v \in [n]} \boldsymbol{F}^v_{n, m}(\boldsymbol{v}[i/(h\circ \pi)(v)], \boldsymbol{w})
		\]
		for all $\boldsymbol{v} \in [n]^\ell$ and $\boldsymbol{w} \in [m]^r$.
		Furthermore, $\Pi_i^\pi(\boldsymbol{F}^1, \dots, \boldsymbol{F}^n) \in \mathcal{T}^k(\ell-1, r) $. 
	\end{lemma}
	\begin{proof}
		The final assertion follows from \cref{lem:gluing,lem:unlabelling}.
		For the first assertion, observe that, by \cref{lem:gluing,lem:unlabelling},
		\begin{align*}
			(\Pi_i^\pi(\boldsymbol{F}^1, \dots, \boldsymbol{F}^n))_{n,m}(\boldsymbol{v}[i/], \boldsymbol{w})
		%	&= \prod_{P \in [n]/\pi} \Sigma_i\left(\prod_{v \in P} \boldsymbol{F}^v_{n,m}(\boldsymbol{v}, \boldsymbol{w}) \right) \\
			&= \prod_{P \in [n]/\pi} \sum_{v' \in [n]} \prod_{v \in P} \boldsymbol{F}^v_{n,m}(\boldsymbol{v}[i/v'], \boldsymbol{w}) \\
			&= \sum_{h \colon [n]/\pi \to [n]}  \prod_{P \in [n]/\pi} \prod_{v \in P} \boldsymbol{F}^v_{n,m}(\boldsymbol{v}[i/h(P)], \boldsymbol{w}) \\
			&= \sum_{h \colon [n]/\pi \to [n]}  \prod_{v \in [n]} \boldsymbol{F}_{n, m}^v(\boldsymbol{v}[i/(h\circ \pi)(v)], \boldsymbol{w}). \qedhere
		\end{align*}
	\end{proof}
	
	\Cref{lem:pi-i-pi} is used to prove the following \cref{lem:product-one-graph}.
	Here,  the $\boldsymbol{F}^1, \dots, \boldsymbol{F}^n$ are taken to be all the same $(\ell, r)$-labelled bipartite graph $\boldsymbol{F}$.
	We abbreviate $\Pi_i^\pi \boldsymbol{F} \coloneqq \Pi_i^\pi(\boldsymbol{F}, \dots, \boldsymbol{F})$.
	\Cref{lem:pi-i-pi} then yields
	\begin{equation}\label{eq:pi-i-pi-simplified}
		(\Pi_i^\pi \boldsymbol{F})_{n,m}(\boldsymbol{v}[i/], \boldsymbol{w}) = \sum_{h \colon [n]/\pi \to [n]}  \prod_{v \in [n]} \boldsymbol{F}_{n, m}(\boldsymbol{v}[i/(h\circ \pi)(v)], \boldsymbol{w})
	\end{equation}
	
	\begin{lemma}\label{lem:product-one-graph}
		For $\boldsymbol{F} \in \mathcal{T}^k(\ell, r)$ and $i \in [\ell]$,
		 $\Pi_i (\boldsymbol{F}_{n,m}) \in \mathfrak{T}^k_{n, m}(\ell-1, r)$.
	\end{lemma}
	\begin{proof}
		Write $\mu$ for the Möbius function of the partition lattice of $[n]$, cf.\ \cref{eq:frs}.
		More concretely, 
		we show that $\Pi_i \boldsymbol{F}_{n,m} = \frac{1}{n!} \sum_{\pi \vdash n} \mu_\pi (\Pi_i^\pi \boldsymbol{F})_{n,m}$.
		By \cref{lem:pi-i-pi},  $\Pi^\pi_i\boldsymbol{F} \in \mathcal{T}^k(\ell-1, r)$.
		Hence, it remains to show that
		\[
		\prod_{v \in [n]} \boldsymbol{F}_{n,m}(\boldsymbol{v}[i/v], \boldsymbol{w})
		=  \frac{1}{n!} \sum_{\pi \vdash n} \mu_{\pi} (\Pi^\pi_i\boldsymbol{F})_{n, m}(\boldsymbol{v}[i/], \boldsymbol{w})
		\]
		for all $\boldsymbol{v} \in [n]^\ell$ and $\boldsymbol{w} \in [m]^r$.
		Note that, crucially, the coefficients in the above expression do not depend on $\boldsymbol{v} \in [n]^\ell$ and $\boldsymbol{w} \in [m]^r$.
		
		 
		
		The left hand-side expression can be rewritten as follows noting that, for every bijection $h \colon [n] \hookrightarrow [n]$, it is 
		$\prod_{v \in [n]} \boldsymbol{F}_{n,m}(\boldsymbol{v}[i/v], \boldsymbol{w}) = \prod_{v \in [n]} \boldsymbol{F}_{n,m}(\boldsymbol{v}[i/h(v)], \boldsymbol{w})$
		by commutativity of the product.
		The second equality follows from \cref{lem:moebius-polynomial},
		the third from \cref{eq:pi-i-pi-simplified}.
		\begin{align*}
			\prod_{v \in [n]} \boldsymbol{F}_{n,m}(\boldsymbol{v}[i/v], \boldsymbol{w})
			&= \frac{1}{n!} \sum_{h \colon [n] \hookrightarrow [n]} \prod_{v \in [n]} \boldsymbol{F}_{n,m}(\boldsymbol{v}[i/h(v)], \boldsymbol{w}) \\
			&= \frac{1}{n!} \sum_{\pi \vdash n} \mu_\pi \sum_{h \colon [n]/\pi \to [n]} \prod_{v \in [n]} \boldsymbol{F}_{n,m}(\boldsymbol{v}[i/(h \circ \pi)(v)], \boldsymbol{w}) \\
			&= \frac{1}{n!} \sum_{\pi \vdash n} \mu_\pi (\Pi^\pi_i \boldsymbol{F})_{n,m}(\boldsymbol{v}[i/], \boldsymbol{w}). \qedhere
		\end{align*}
	\end{proof}
%
%	\begin{example}
%		Let $n = 2$.
%		Let $\boldsymbol{F}$ denotes the $1$-labelled $K_2$ whose label is on the left part of the bipartition.
%		Then $\boldsymbol{F}_{n,m}(v) = \sum_{w \in [m]} x_{vw}$ for $v \in [n]$.
%		Over $\{0,1\}$-adjacency matrices, $\boldsymbol{F}_{n,m}(v) = \deg(v)$.
%		The graph $\Sigma_1 \boldsymbol{F}$ is unlabelled, hence $(\Sigma_1 \boldsymbol{F})^{\odot 2}$ is just $2K_2$.
%		It follows that
%		\[
%			\prod_{v \in [n]} \boldsymbol{F}_{n,m}(v) = \frac12 \left( \hom(2K_2, -)  - \hom(K_{1,2}, -) \right).
%		\]
%		This translates over $\{0,1\}$-adjacency matrices with $n = 2$ to the identity
%		\[
%			2\deg(1)\deg(2) = \left(\deg(1) + \deg(2)\right)^2 - \left( \deg(1)^2 + \deg(2)^2 \right).
%		\]
%	\end{example}
%
%	\begin{example}
%		Consider the $2$-labelled edge $\boldsymbol{F}$.
%		Then $\boldsymbol{F}_{n, m}(v,w) = x_{v,w}$.
%		We expand $\prod_{w \in [m]} \boldsymbol{F}_{n, m}(v, w)$ using \cref{lem:product-one-graph}.
%		The $1$-labelled graphs appearing in the expansion for $n = m = 2$ are the $1$-labelled star $\boldsymbol{F}^1$ with two tips and label at the centre and $\boldsymbol{F}^2$ the $1$-labelled graph on two vertices connected by two parallel edges.
%		We have $\boldsymbol{F}^1_{n, m}(v) = \sum_{w_1, w_2 \in [m]} x_{vw_1} x_{vw_2}$ and $\boldsymbol{F}^2_{n, m}(v) = \sum_{w \in [m]} x_{vw}^2$. 
%		For $n = m = 2$,
%		\begin{align*}
%			\prod_{w \in [2]} \boldsymbol{F}_{2,2}(v, w)
%			= x_{v1} x_{v2}
%			&= \frac{1}{2} \left( x_{v1}x_{v1} + x_{v1} x_{v2} + x_{v2} x_{v1} + x_{v2} x_{v2} - x_{v1}^2 - x_{v2}^2 \right) \\
%			&= \frac{1}{2} \left(\boldsymbol{F}^1_{2,2}(v) - \boldsymbol{F}^2_{2,2}(v) \right).
%		\end{align*}
%	\end{example}

	Now we consider the general case: applying the product operator to a linear combination of labelled homomorphism polynomials.
	In preparation, we note the following \cref{fact:orbits}.

	\begin{fact}\label{fact:orbits}
		Let $\Gamma$ denote a finite group acting on a finite set $X$.
		For an element $x \in X$, write $\beta_x$ for the number of pairs $(z, \gamma) \in X \times \Gamma$ such that $\gamma(z) = x$.
		If $x, y \in X$ are in the same orbit under $\Gamma$,
		then $\beta_x = \beta_y$.
	\end{fact}
	\begin{proof}
		Suppose that $\iota(x) = y$ for some $\iota \in \Gamma$.
		Then the map $(z, \gamma) \mapsto (z, \iota \gamma )$ sends a pair $(z, \gamma)$ such that $\gamma(z) = x$ to the pair $(z, \iota \gamma)$ satisfying $\iota \gamma (z) = \iota(x) = y$.
		This map is bijective.
	\end{proof}



	
	\begin{proof}[Proof of \cref{thm:product-lincomb}]
		Write $\phi = \sum_{t \in T} \alpha_t \boldsymbol{F}^t_{n,m}$ for some finite set $T$,
		coefficients $\alpha_t \in \mathbb{Q}$ and $\boldsymbol{F}^t \in \mathcal{T}^k(\ell, r)$.
		The symmetric group $\Sym_n$ acts on the set of functions $f \colon [n] \to T$ by composition.
		The orbits of this action are in bijection with maps $\lambda \colon T \to \{0, \dots, n\}$ such that $\sum_{t\in T} \lambda(t) = n$.
		Write $\Lambda$ for the set of all such functions.
		A function $f \colon [n] \to T$ belongs to the orbit $O_\lambda$ for $\lambda \in \Lambda$ if $|f^{-1}(t)| = \lambda(t)$ for all $t \in T$.
		
		Let $\boldsymbol{v} \in [n]^\ell$ and $\boldsymbol{w} \in [m]^r$.
		First, product and sum are interchanged. The resulting sum over all functions $f \colon [n] \to T$ is then grouped by orbits:
		\begin{align*}
			(\Pi_i \phi)(\boldsymbol{v}[i/], \boldsymbol{w})
			&= \prod_{v \in [n]} \sum_{t \in T} \alpha_t \boldsymbol{F}_{n,m}^t(\boldsymbol{v}[i/v], \boldsymbol{w}) \\
			&= \sum_{f \colon [n] \to T} \prod_{v \in [n]} \alpha_{f(v)} \boldsymbol{F}_{n,m}^{f(v)}(\boldsymbol{v}[i/v], \boldsymbol{w}) \\
			&= \sum_{\lambda \in \Lambda} \sum_{f \in O_\lambda} \prod_{v \in [n]} \alpha_{f(v)} \boldsymbol{F}_{n,m}^{f(v)}(\boldsymbol{v}[i/v], \boldsymbol{w}) \\
			&= \sum_{\lambda \in \Lambda} \alpha_\lambda  \sum_{f \in O_\lambda} \prod_{v \in [n]} \boldsymbol{F}_{n,m}^{f(v)}(\boldsymbol{v}[i/v], \boldsymbol{w}).
		\end{align*}
		where $\alpha_\lambda \coloneqq \prod_{t \in T} \alpha_t^{\lambda(t)} \in \mathbb{Q}$.
		
		From now on, we consider each orbit $\lambda \in \Lambda$ separately.
		By \cref{fact:orbits}, there exists a positive number $\beta_\lambda \in \mathbb{N}$ such that, for every $f \in O_\lambda$, the number of pairs $g \colon [n] \to T$ and $h \in \Sym_n$ such that $g \circ h = f$ is $\beta_\lambda$.	
		In any such pair, we necessarily have $g \in O_\lambda$.
		Towards applying Möbius inversion, we denote permutations $h \in \Sym_n$ as injective maps $[n] \hookrightarrow [n]$.
		\begin{align*}
			\beta_\lambda \sum_{f \in O_\lambda} \prod_{v \in [n]} \boldsymbol{F}_{n,m}^{f(v)}(\boldsymbol{v}[i/v], \boldsymbol{w})
			&= \sum_{g \in O_\lambda} \sum_{h \colon [n] \hookrightarrow [n]} \prod_{v \in [n]} \boldsymbol{F}_{n,m}^{(g \circ h)(v)}(\boldsymbol{v}[i/v], \boldsymbol{w}) \\
			&= \sum_{g \in O_\lambda} \sum_{h \colon [n] \hookrightarrow [n]} \prod_{v \in [n]} \boldsymbol{F}_{n,m}^{g(v)}(\boldsymbol{v}[i/h(v)], \boldsymbol{w}) \\
			&= \sum_{g \in O_\lambda} \sum_{\pi \vdash n} \mu_\pi \sum_{h \colon [n]/\pi \to [n]} \prod_{v \in [n]} \boldsymbol{F}_{n,m}^{g(v)}(\boldsymbol{v}[i/(h \circ \pi)(v)], \boldsymbol{w}) \\
			&= \sum_{g \in O_\lambda} \sum_{\pi \vdash n} \mu_\pi (\Pi^\pi_i(\boldsymbol{F}^{g(1)}, \dots, \boldsymbol{F}^{g(n)}))_{n,m}(\boldsymbol{v}[i/], \boldsymbol{w})
		\end{align*}
		Here, the first equality follows from \cref{fact:orbits}.
		The second equality is obtained by first replacing the product over $v \in [n]$ by a product over $h(v)$ for $v \in [n]$ and then replacing the sum over $h \colon [n] \hookrightarrow [n]$ by a sum over $h^{-1}$ for $h \colon [n] \hookrightarrow [n]$.
		The third equality follows from \cref{lem:moebius-polynomial}.
		The final equality is implied by \cref{lem:pi-i-pi}.
		
		Combining the identities above, we obtain
		\begin{equation}\label{eq:product-lincomb}
		(\Pi_i \phi)(\boldsymbol{v}[i/], \boldsymbol{w})
		= \sum_{\lambda \in \Lambda} \frac{\alpha_\lambda}{\beta_\lambda} \sum_{g \in O_\lambda} \sum_{\pi \vdash n} \mu_\pi \Pi_i^\pi(\boldsymbol{F}^{g(1)}, \dots, \boldsymbol{F}^{g(n)})_{n,m}(\boldsymbol{v}[i/], \boldsymbol{w}).
		\end{equation}
		Note that neither the $\alpha_\lambda$ nor $\beta_\lambda$ depend on $\boldsymbol{v}$.
		Hence, $\Pi_i \phi \in \mathfrak{T}^k_{n,m}(\ell-1, r)$.
	\end{proof}

	\begin{remark}
		\cref{thm:product-lincomb} indeed generalises \cref{lem:product-one-graph}.
		If $T = \{\boldsymbol{F}\}$ is a singleton and there are no coefficients,
		then $\Lambda$ consists of a single map $\lambda$.
		Then, $\alpha_\lambda = 1$ and  $\beta_\lambda = n!$.
		The orbit $O_\lambda$ contains only one map $g$ and $\Pi_i^\pi(\boldsymbol{F}^{g(1)}, \dots, \boldsymbol{F}^{g(n)}) = \Pi_i^\pi \boldsymbol{F}$.
	\end{remark}

	\subsubsection{Restricted Products}
	
	Finally, we derive a corollary of \cref{thm:product-lincomb} concerned with the restricted product operator $\Pi_{i, J}$ which is defined as follows:
	Let $i \in [\ell]$ and $J \subseteq [\ell] \setminus \{i\}$.
	For $\phi \in \mathfrak{G}_{n,m}(\ell, r)$,
	define $\Pi_{i, J} \phi \in \mathfrak{G}_{n,m}(\ell-1, r)$
	via \[ (\Pi_{i, J} \phi)(\boldsymbol{v}[i/], \boldsymbol{w}) \coloneqq \prod_{v \in [n] \setminus \{v_j \mid j \in J\}} \phi(\boldsymbol{v}[i/v], \boldsymbol{w}). \]
	for all $\boldsymbol{v} \in [n]^\ell$ and $\boldsymbol{w} \in [m]^r$.
	Note that $\Pi_i \phi = \Pi_{i, \emptyset} \phi$.

	
	\begin{corollary}\label{cor:product-exclude-lincomb}
		Suppose that $\ell+r \leq k$.
		For $\phi \in \mathfrak{T}^k_{n,m}(\ell,r )$,  $i \in [\ell]$ and $J \subseteq [\ell] \setminus \{i\}$,
		 $\Pi_{i, J} (\phi) \in \mathfrak{T}^k_{n,m}(\ell-1, r)$.
	\end{corollary}

	\begin{proof}
		For $J \subseteq [\ell]$, write $V_J \coloneqq [n] \setminus \{v_j \mid j \in J\}$.
		The proof is by induction on $|J|$.
		The base case $|J| = 0$ is established in \cref{thm:product-lincomb}.
		
		For the inductive step, let $j \in J$ and write $J' \coloneqq J \setminus \{j\}$.
		We first distinguish cases based on the form of the tuple $\boldsymbol{v}$.
		This is a priori problematic since the desired expression is required not to depend on $\boldsymbol{v}$.
		We overcome this problem by implementing the case distinction itself using homomorphism polynomial maps.
		
		Write $\phi = \sum_{t \in T} \alpha_t \boldsymbol{F}^t$ and
		suppose first that $v_j \not\in \{v_{j'} \mid j' \in J'\}$, i.e.\ $V_{J'} = V_{J} \uplus \{v_j\}$.
		Recall the definition of $\boldsymbol{D}^{i,j} \in \mathcal{T}^k(\ell, r)$ from the proof of \cref{lem:sum-exclude-lincomb}.
		We rewrite $\Pi_{i, J} (\phi)$ in terms of expressions to which the inductive hypothesis applies by considering the following polynomial:
		\begin{align*}
			&\prod_{v \in V_{J'}}
			\left( \phi(\boldsymbol{v}[i/v], \boldsymbol{w}) + \boldsymbol{D}^{i, j}_{n, m}(\boldsymbol{v}[i/v], \boldsymbol{w}) \right)  \\
			&= \left( \phi(\boldsymbol{v}[i/v_j], \boldsymbol{w}) + \boldsymbol{D}^{i, j}_{n, m}(\boldsymbol{v}[i/v_j], \boldsymbol{w}) \right)  \prod_{v \in V_{J}}
			\left( \phi(\boldsymbol{v}[i/v], \boldsymbol{w}) + \boldsymbol{D}^{i, j}_{n, m}(\boldsymbol{v}[i/v], \boldsymbol{w}) \right) \\
			&\overset{\eqref{eq:diell}}{=} \left( \phi(\boldsymbol{v}[i/v_j], \boldsymbol{w}) + 1 \right) \prod_{v \in V_{J}} \phi(\boldsymbol{v}[i/v], \boldsymbol{w}) \\
			&= \prod_{v \in V_{J'}}  \phi(\boldsymbol{v}[i/v], \boldsymbol{w}) + \prod_{v \in V_{J}} \phi(\boldsymbol{v}[i/v], \boldsymbol{w}).
		\end{align*}
		If otherwise $v_j \in \{v_{j'} \mid j' \in J'\}$,
		then the expression for $J$ equals the expression for $J'$.
		Formally,
		\[
			\prod_{v \in V_{J'}}\phi(\boldsymbol{v}[i/v], \boldsymbol{w}) 
			= \prod_{v \in V_{J}}\phi(\boldsymbol{v}[i/v], \boldsymbol{w}). 
		\]
		In order to combine the two cases considered above,
		observe that, for every  $\boldsymbol{v} \in [n]^{\ell}$ and $\boldsymbol{w} \in [m]^r$, by \cref{lem:unlabelling},
		\[
			\frac1n \sum_{j' \in J'} (\Sigma_i\boldsymbol{D}^{j', j})_{n,m}(\boldsymbol{v}[i/], \boldsymbol{w}) 
			= \frac1n \sum_{v \in [n]} \sum_{j' \in J'}\boldsymbol{D}^{j', j}_{n,m}(\boldsymbol{v}[i/v], \boldsymbol{w})
			= \sum_{j' \in J'} \boldsymbol{D}^{j', j}_{n,m}(\boldsymbol{v}, \boldsymbol{w})
		\]
		is equal to the number of indices $j' \in J'$ such that $v_j = v_{j'}$.
		This is a value from $\{0, \dots, \ell\}$.
		Here, it is crucial that $i \not\in J$, as assumed.
		Let $p \in \mathbb{Q}[x]$ denote a polynomial such that $p(0) = 0$ and $p(1) = p(2) = \dots = p(\ell) = 1$.
		By \cref{lem:unlabelling,lem:gluing},
		$\psi \coloneqq p(\frac1n \sum_{j' \in J'} (\Sigma_i\boldsymbol{D}^{j', j})_{n,m}) \in \mathfrak{T}_{n,m}^k(\ell-1, r)$.
		Then
		\begin{equation*}
			\psi(\boldsymbol{v}[i/], \boldsymbol{w}) = \begin{cases}
				1, & v_j \in \{v_{j'} \mid j' \in J'\}, \\
				0, & \text{otherwise}.
			\end{cases}
		\end{equation*}
		Combining the above,
		it follows that the expression $\prod_{v \in V_{J}}
		\phi(\boldsymbol{v}[i/v], \boldsymbol{w})$ of interest is equal to
		\begin{align*}
			&\psi(\boldsymbol{v}[i/], \boldsymbol{w})
			\prod_{v \in V_{J'}} \phi(\boldsymbol{v}[i/v], \boldsymbol{w}) \\
			&+
			(1-\psi(\boldsymbol{v}[i/], \boldsymbol{w}))
			\left(\prod_{v \in V_{J'}}
			\left( \phi(\boldsymbol{v}[i/v], \boldsymbol{w}) + \boldsymbol{D}^{i, j}_{n, m}(\boldsymbol{v}[i/v], \boldsymbol{w}) \right) 
			- \prod_{v \in V_{J'}}  \phi(\boldsymbol{v}[i/v], \boldsymbol{w}) \right).
		\end{align*}
		This expression depends uniformly on $\boldsymbol{v}$ and $\boldsymbol{w}$.
		In other words,
		\begin{align*}
			\Pi_{i, J} (\phi) &= \psi \ \Pi_{i, J'}( \phi) + (1- \psi) \left( \Pi_{i,J'}(\phi + \boldsymbol{D}^{i,j}_{n,m}) - \Pi_{i, J'} (\phi) \right).
%			&= \Pi_{i, J'}(\phi) + (1- \psi) \left( \Pi_{i,J'}(\phi + \boldsymbol{D}^{i,j}_{n,m}) \right).\label{eq:lincomb-product-restricted}
		\end{align*}
		Here, we regard $1$ as the element of $\mathfrak{T}^k_{n,m}(\ell-1, r)$ which is uniformly $1$, cf.\ \cref{ex:one}.
		The induction hypothesis applies to $\Pi_{i, J'}$.
		Hence, $\Pi_{i, J'}( \phi),\Pi_{i,J'}(\phi + \boldsymbol{D}^{i,j}_{n,m}) \in \mathfrak{T}^{k}_{n,m}(\ell-1,r)$.
		By \cref{lem:gluing}, $\mathfrak{T}^k_{n,m}(\ell-1, r)$ is closed under pointwise products.
		Hence, $\Pi_{i, J} (\phi) \in \mathfrak{T}^k_{n,m}(\ell-1, r)$.
	\end{proof}



 	\subsection{Homomorphism Polynomials from Symmetric Circuits}
 
 	In this section we prove the direction \ref{it:main2} $\Rightarrow$ \ref{it:main1} of \cref{thm:main1} using the results we have established so far. For the proof, we start with a rigid symmetric circuit $C$ with $\maxSup(C) \leq k$, and show via induction from the inputs to the output that at each gate $g$, the computed polynomial is in $\mathfrak{T}^{2k}_{n,m}(\vec{\sup}_L(g), \vec{\sup}_R(g))$.  
 	Recall that $\vec{\sup}_L(g)$ and $\vec{\sup}_R(g)$ denote ordered tuples whose entries are the elements of $\sup_L(g)$ and $\sup_R(g)$ (see \cref{sec:supports}). The ordering of the tuples represents the assignment of the labels to the support elements:
 	The label $i \in[\ell]$ is mapped to the $i$-th entry of $\vec{\sup}_L(g)$, and analogously for $\vec{\sup}_R(g)$. 
 	
 	For the inductive step of the proof, we assume that $g$ is a multiplication or summation gate such that every child $h$ of $g$ computes a polynomial in $\mathfrak{T}_{n,m}^{2k}(\vec{\sup}_L(h), \vec{\sup}_R(h))$. Then we show that the polynomial computed at $g$ is in $\mathfrak{T}_{n,m}^{2k}(\vec{\sup}_L(g), \vec{\sup}_R(g))$. In the following, if $g$ is a gate, then $p_g$ denotes the polynomial that this gate outputs.
  	
  	\begin{observation}\label{lem:labelsMovedByPermutations}
   		Let $n,m, \ell, r \in \mathbb{N}$.
  		Let $\boldsymbol{F} \in \mathcal{G}(\ell, r)$ and let $(\pi,\sigma) \in \Sym_n \times \Sym_m$. 
  		Then $(\pi, \sigma)(\boldsymbol{F}_{n,m}(\boldsymbol{v}, \boldsymbol{w})) = \boldsymbol{F}_{n,m}(\pi(\boldsymbol{v}), \sigma(\boldsymbol{w}))$ for all $\boldsymbol{v} \in [n]^\ell$ and $\boldsymbol{w} \in [m]^r$.
  	\end{observation}	
  	\begin{proof}
 		For $\boldsymbol{F} = (F, \boldsymbol{a}, \boldsymbol{b})$,
  		we have $(\pi,\sigma)( \{ h\colon A \uplus B \to [n] \uplus [m] \mid h(\boldsymbol{a}) = \boldsymbol{v}, h(\boldsymbol{b}) = \boldsymbol{w} \} ) =  \{ h\colon A \uplus B \to [n] \uplus [m] \mid h(\boldsymbol{a}) = \pi(\boldsymbol{v}) , h(\boldsymbol{b}) = \sigma(\boldsymbol{v}) \}$.
  	\end{proof}	
  	
 	\begin{lemma}
 		\label{lem:supportOfGateDeterminesLabels}
 		Let $C$ be a rigid $\Sym_n \times \Sym_m$-symmetric circuit and $g \in V(C)$ a gate such that the polynomial computed at $g$ is of the form $p_g = \phi(\vec{\sup}_L(g),\vec{\sup}_R(g))$ for some $\phi \in \mathfrak{T}^{2k}_{n,m}(\ell, r)$.
 		Let $g' \in \Orb_{\Sym_{n} \times \Sym_{m}}(g)$. Then the polynomial computed by $g'$ is $p_{g'} = \phi(\vec{\sup}_L(g'),\vec{\sup}_R(g'))$.
 	\end{lemma}	
 	\begin{proof}
 		Let $(\pi, \sigma) \in \Sym_{n} \times \Sym_{m}$ such that $(\pi, \sigma)(g)  = g'$. So the polynomial computed by $g'$ is $(\pi, \sigma)(p)$. 
 		Then the lemma follows from \cref{lem:labelsMovedByPermutations}.
 	\end{proof}	
 	
 	
 	 	
 	The next lemma shows the inductive step in case that the children of the gate $g$ form a single orbit.
 	\begin{lemma}
 		\label{lem:productOfOneChildOrbit}
 		Let $k \in \bbN$.
 		Let $C$ be a rigid $\Sym_{n} \times \Sym_{m}$-symmetric circuit, and $g \in V(C)$ a multiplication or a summation gate with $|\sup(g)| \leq k$.
 		Assume that for every $h \in \child(g)$, it also holds
                that $|\sup(h)| \leq k$ and the polynomial computed at $h$ is $p_h = \phi(\vec{\sup}_L(h), \vec{\sup}_R(h))$ for some $\phi \in \mathfrak{T}^{2k}_{n,m}(|\vec{\sup}_L(h)|, |\vec{\sup}_R(h)|)$.
 		Let $h \in \child(g)$ and let $O_h \coloneqq \Orb_{\StabP(\sup(g))}(h)$.\\
 		Then $\prod_{h' \in O_h} p_{h'}$ and $\sum_{h' \in O_h} p_{h'}$ are in $\mathfrak{T}_{n,m}^{2k}(\vec{\sup}_L(g), \vec{\sup}_R(g))$.
 	\end{lemma}	
 	\begin{proof}
 		As in \cref{lem:intersectionOfChildSupports}, let $S(h) = \bigcap_{h' \in O_h} \sup(h')$, and $S(h)_L = S(h) \cap [n]$, $S(h)_R = S(h) \cap [m]$.
  		Note that for every $h' \in O_h$, the tuples $\vec{\sup}_L(h')$ and $\vec{\sup}_R(h')$ have the elements of $S(h)$ in the same positions because $S(h) \subseteq \sup(g)$ by Lemma \ref{lem:intersectionOfChildSupports}. Since all these gates are in the same $\StabP(\sup(g))$-orbit, the positions of $S(h)$ in $\vec{\sup}_L(h')$ and $\vec{\sup}_R(h')$ are the same. In the following, when we write $\vec{S}(h)_L$ and $\vec{S}(h)_R$, we refer to the ordering of $S(h)_L$ and $S(h)_R$ as in $\vec{\sup}_L(h')$ and $\vec{\sup}_R(h')$, respectively. 
 		
		For simplicity, assume in the following that the first $|S(h)_L|$ positions in each tuple $\vec{\sup}_L(h')$ are occupied by the elements of $S(h)_L$ and the first $|S(h)_R|$ positions in $\vec{\sup}_R(h')$ are occupied by $S(h)_L$. 
 		Since $\Orb_{\StabP(\sup(g))}(h)$ is a subset of $\Orb_{\Sym_n \times \Sym_m}(h)$, by \cref{lem:supportOfGateDeterminesLabels}, all these gates compute the same instantiated homomorphism polynomial map, just with different instantiation of the labels.
 		Let $\phi \in \mathfrak{T}_{n,m}^{2k}(\ell , r)$ denote this homomorphism polynomial map, where $\ell = |\sup_L(h)|, r = |\sup_R(h)|$. 
 		Recall the definition of $\boldsymbol{J}$ from \cref{ex:one}.
 		First consider $\prod_{h' \in O_h} p_{h'}$.
 		\begin{align*}
 		 \prod_{h' \in O_h} p_{h'} 
 		 &= \prod_{h' \in O_h}  \phi(\vec{\sup}_L(h'), \vec{\sup}_R(h')) \\
 		 &= \prod_{\boldsymbol{v} \in ([n] \setminus \sup_L(g))^{\ell - |S(h)_L|}} \prod_{\boldsymbol{w} \in ([m] \setminus \sup_R(g))^{r - |S(h)_R|}} \phi(\vec{S}(h)_L \boldsymbol{v}, \vec{S}(h)_R \boldsymbol{w})  \\
 		 &= \prod_{\boldsymbol{v} \in ([n] \setminus \sup_L(g))^{\ell - |S(h)_L|}} \prod_{\boldsymbol{w} \in ([m] \setminus \sup_R(g))^{r - |S(h)_R|}} \phi(\vec{S}(h)_L \boldsymbol{v}, \vec{S}(h)_R \boldsymbol{w})
 		  \cdot \boldsymbol{J}_{n,m}(\vec{\sup}_L(g) \setminus \vec{S}(h)_L, \vec{\sup}_R(g) \setminus \vec{S}(h)_R)\\
 		  &= \prod_{\boldsymbol{v} \in ([n] \setminus \sup_L(g))^{\ell - |S(h)_L|}} \prod_{\boldsymbol{w} \in ([m] \setminus \sup_R(g))^{r - |S(h)_R|}} (\phi \otimes \boldsymbol{J}_{n,m})(\vec{\sup}_L(g) \boldsymbol{v}, \vec{\sup}_R(g) \boldsymbol{w}). 
 		\end{align*}
 		The second equality is true because by Lemma \ref{lem:intersectionOfChildSupports}, for each $h' \in O_h$, 
 		the set $\sup(h') \setminus S(h)$ does not intersect $\sup(g)$, and $O_h$ is the orbit of $h$ under the pointwise stabiliser of $\sup(g)$ in $\Sym_n \times \Sym_m$. Hence, there is a bijection between the gates in $O_h$ and the set $\{  \boldsymbol{v}\boldsymbol{w} \mid  \boldsymbol{v} \in ([n] \setminus \sup_L(g))^{\ell - |S(h)_L|},  \boldsymbol{w} \in ([m] \setminus \sup_R(g))^{r - |S(h)_R|}  \}$.
 		The third equality holds because by \cref{ex:one}, $\boldsymbol{J}_{n,m}(\vec{\sup}_L(g) \setminus \vec{S}(h)_L, \vec{\sup}_R(g) \setminus \vec{S}(h)_R)= 1$. The last equality is due to the definition of $\otimes$ on homomorphism polynomial maps, cf.\ \cref{lem:disjoint-union}.
 		Analogously,
 		it holds that
 		\[
 			\sum_{h' \in O_h} p_{h'}
 			=
 			\sum_{\boldsymbol{v} \in ([n] \setminus \sup_L(g))^{\ell - |S(h)_L|}} 
 			\sum_{\boldsymbol{w} \in ([m] \setminus \sup_R(g))^{r - |S(h)_R|}} (\phi \otimes \boldsymbol{J}_{n,m})(\vec{\sup}_L(g) \boldsymbol{v}, \vec{\sup}_R(g) \boldsymbol{w}).
 		\]
 		
 		By \cref{lem:disjoint-union}, $(\phi \otimes \boldsymbol{J}) \in \mathfrak{T}_{n,m}^{2k}(|\sup_L(g)|+|\sup_L(h) \setminus \sup_L(g)|, |\sup_R(g)|+|\sup_R(h) \setminus \sup_R(g)|)$. 
 		Here it is important that $| \sup(g)| + |\sup(h)| \leq 2k$.
 		By repeatedly applying \cref{cor:product-exclude-lincomb} to the last expression above,
 		it follows that $\prod_{h' \in O_h} p_{h'} \in \mathfrak{T}^{2k}_{n,m}(\vec{\sup}_L(g), \vec{\sup}_R(g))$.
 		If $g$ is a summation gate, then similarly, with \cref{lem:sum-exclude-lincomb}, $\sum_{h' \in O_h} p_{h'} \in \mathfrak{T}^{2k}_{n,m}(\vec{\sup}_L(g), \vec{\sup}_R(g))$.
 	\end{proof}
 	
 	We now generalise the above lemma so that it covers all orbits of children of~$g$.
 	
 	 \begin{lemma}
 		\label{lem:productOfAllChildOrbits}
 		Let $k \in \bbN$. 
 		Let $C$ be a rigid $\Sym_{n} \times \Sym_{m}$-symmetric circuit. 
 		Let $g \in V(C)$ be a multiplication or summation gate with $|\sup(g)| \leq k$.
 		Assume that for every $h \in \child(g)$, the polynomial $p_h$ is in $\mathfrak{T}_{n,m}^{2k}(\vec{\sup}_L(h), \vec{\sup}_R(h))$.
 		Then the polynomial computed at $g$ is in $\mathfrak{T}_{n,m}^{2k}(\vec{\sup}_L(g), \vec{\sup}_R(g))$.
 	\end{lemma}	
 	\begin{proof}
 		We treat the case when $g$ is a multiplication gate in detail.
 		The other case is analogous.
 		Partition the set $\child(g)$ into orbits under $\StabP(\sup(g))$. 
		To each such orbit $O_h$, we apply \cref{lem:productOfOneChildOrbit}, 
		which shows that $p_{O_h} \coloneqq \prod_{h' \in O_h} p_{h'} \in \mathfrak{T}^{2k}_{n,m}(\vec{\sup}_L(g), \vec{\sup}_R(g))$.
		Let $\Omega \coloneqq \{O_h \mid h \in \child(g)\}$ and let $\phi_{O_h}(\vec{\sup}_L(g), \vec{\sup}_R(g))$ be the instantiated homomorphism polynomial map in $\mathfrak{T}^{2k}_{n,m}(\vec{\sup}_L(g), \vec{\sup}_R(g))$ that is equal to $p_{O_h}$. 
 		The polynomial computed at $g$ is 
 		%$\prod_{O_h \in \Omega} p_{O_h}$.
 		%We show that $\prod_{O_h \in \Omega} p_{O_h} \in \mathfrak{T}^{2k}_{n,m}(\vec{S}_L, \vec{S}_R)$, for $S \coloneqq \bigcup_{h \in \child(g)} S(h)$, where $S(h)$ is like in Lemma \ref{lem:productOfOneChildOrbit}. For each $O_h \in \Omega$, let $\phi_{O_h}(\vec{\sup}_L(g), \vec{\sup}_R(g)) \in \mathfrak{T}_{n,m}^{2k}(\vec{\sup}_L(g), \vec{\sup}_R(g))$ be the polynomial that we get from \cref{lem:productOfOneChildOrbit}.
 		%Let $T_{O_h}$ be an index set such that $\phi_{O_h}= \sum_{i \in T_{O_h}} \alpha_i\boldsymbol{F}^i_{n,m}$. Let $T = \biguplus_{O_h \in \Omega} T_{O_h}$.
 		\begin{align*}
 			p_g = \prod_{O_h \in \Omega} p_{O_h} &= \prod_{O_h \in \Omega} \phi_{O_h}(\vec{\sup}_L(g), \vec{\sup}_R(g)).
 			 %\sum_{\stackrel{f\colon \Omega \to T}{f(O_h) \in T_{O_h} \text{ for each } O_h \in \Omega}} \prod_{O_h \in \Omega} \boldsymbol{F}^{f(O_h)}_{n,m}(\vec{S}(h)_L, \vec{S}(h)_R)\\
 			%&= \sum_{\stackrel{f\colon \Omega \to T}{f(O_h) \in T_{O_h} \text{ for each } O_h \in \Omega}} \boldsymbol{G}^{f}_{n,m}(\vec{S}_L, \vec{S}_R).
 		\end{align*}	
 		%In the last line, we applied \cref{lem:semiGluing} to the product: $G^f$ denotes the graph that is obtained by gluing the graphs in $\{ F^{f(O_h)} \mid O_h \in \Omega \}$ together, where vertices of distinct graphs $F^{f(O_{h_1})}, F^{f(O_{h_2})}$ are identified whenever their labels are assigned to the same element of $[n] \uplus [m]$ in the tuples $\vec{S}(h_1)$ and $\vec{S}(h_2)$ (see \cref{lem:semiGluing}). 
 		This polynomial is in $\mathfrak{T}_{n,m}^{2k}(\vec{\sup}_L(g), \vec{\sup}_R(g))$ by \cref{lem:gluing}.
 	\end{proof}	
 	
 	
 	\subsection{Proof of the backward direction of \cref{thm:main1}}
 	\begin{proof}
 		Let $(C_{n,m})_{n,m \in \bbN}$ be a family of circuits such that each $C_{n,m}$ is $\Sym_n \times \Sym_m$-symmetric, and $\maxOrb(C_{n,m})$ is polynomially bounded in $(n+m)$. By \cref{lem:rigidifyCircuits}, we can assume that the circuits are rigid. Then by \cref{lem:constantSupportOfGates}, there exists a constant $k \in \bbN$ such that $\maxSup(C_{n,m}) \leq k$ for all $n,m \in \bbN$. Consider now fixed $n, m \in \bbN$.
 		By induction on the structure of $C_{n,m}$, we show
                that for every gate $g \in V(C_{n,m})$, we have $p_g \in \mathfrak{T}^{2k}_{n,m}(\vec{\sup}(g)_L, \vec{\sup}(g)_R)$. If $g$ is an input gate labelled with a constant, then $g$ is fixed by every permutation, so $\sup(g) = \emptyset$. Indeed, any constant is a polynomial in $ \mathfrak{T}^{2k}_{n,m}(0, 0)$ by \cref{ex:one}. If $g$ is an input gate labelled with a variable $x_{ij}$, then $\sup_L(g) = \{i\}, \sup_R(g) = \{j\}$. It is clear that the polynomial $x_{ij}$ is in $\mathfrak{T}^{2k}_{n,m}(i, j)$ because $x_{ij} = \boldsymbol{F}_{n,m}(i,j)$ where $\boldsymbol{F}$ is the graph that consists of a single edge. This finishes the base cases. 
 		In the inductive step, $g$ is either a summation or multiplication gate. Both these cases are handled by \cref{lem:productOfAllChildOrbits}. Finally, the output gate of $C_{n,m}$ is stabilised by $\Sym_n \times \Sym_m$, so its minimal support is empty, which means that the polynomial computed by $C_{n,m}$ is in $\mathfrak{T}^{2k}_{n,m}$. 
 	\end{proof}	
 	
 	
 	
 	
 	
 	
 	
 	\subsection{Symmetric Circuits for Homomorphism Polynomials}
 	In this section, we prove \cref{thm:hom-circuit-main}. 	
 	Let $F$ be a graph with a tree decomposition $(T, \beta)$ and $r \in V(T)$.
 	For $s \in V(T)$, write $T^s$ for the subtree of $T$ induced by $s$ and all its descendents in the rooted tree $(T, r)$.
 	Write $F^s$ for the subgraph of $F$ induced by $\bigcup_{t \in V(T^s)} \beta(t)$.
 	
 	\begin{lemma} \label{lem:hom-circuit-core-bipartite}
 		Let $k, n, m \in \mathbb{N}$.
 		Let $F$ be a bipartite multigraph with bipartition $A \uplus B$, tree decomposition $(T, \beta)$, and $r \in V(T)$ such that 
 		\begin{enumerate}
 			\item $|\beta(t)| = k$ for all $t \in V(T)$,
 			\item $|\beta(s) \cap \beta(t)| = k-1$ for all $st \in E(T)$, 
% 			\item if $s, t \in V(T)$ are siblings in $(T,r)$, then $\beta(s) \neq \beta(t)$, and
% Removed as this follows from the previous two conditions
			\item every vertex in the rooted tree $(T,r)$ has out-degree at most $k$.
 		\end{enumerate}
 		There exists a rigid $\Sym_n \times \Sym_m$-symmetric circuit of size at most $4  k!^2 k \cdot (n+m)^{k+1} \cdot \abs{V(T)} \cdot \norm{F} $
 		and support size at most $k$
 		which contains, for every 
 		$s \in V(T)$,
 		$\boldsymbol{a} \in A^\ell$,
 		$\boldsymbol{b} \in B^r$,
 		$\ell +r = k$ such that
 		$\beta(s) = \{a_1, \dots, a_\ell, b_1, \dots, b_r\}$, and $\boldsymbol{v} \in [n]^\ell$,
 		$\boldsymbol{w} \in [m]^r$,
 		a gate which computes $\boldsymbol{F}^s_{n,m}(\boldsymbol{v}, \boldsymbol{w})$
 		where $\boldsymbol{F}^s \coloneqq (F^s, \boldsymbol{a}, \boldsymbol{b}) \in \mathcal{T}^{k}(\ell, r)$.
 	\end{lemma}
 	\begin{proof}
 		We build the circuit by induction on the depth of $s$ in $(T, r)$.
 		For the base case, consider all leaves of $T$.
 		For every leaf $s \in V(T)$,
 		all vertices in $ \boldsymbol{F}^s$ are labelled.
 		Hence,
 		$ \boldsymbol{F}^s_{n,m}(\boldsymbol{v},\boldsymbol{w})$ can be represented by a single multiplication gate whose fan-in equals the number of edges in $F^s$.
 		For every leaf, this yields a subcircuit of size at most $\norm{F} n^\ell m^r k! \leq \norm{F} (n+m)^k k!$.
 		
 		For the inductive step, suppose that a circuit has been constructed which contains the required gates for all $s \in V(T)$ of depth $\leq d$.
 		Let $s \in V(T)$ be a vertex at depth $d+1$.
 		Fix $\boldsymbol{v} \in [n]^\ell$,
 		$\boldsymbol{w} \in [m]^r$, $\boldsymbol{a} \in A^\ell$,
 		$\boldsymbol{b} \in B^r$,
 		$\ell +r = k$ such that
 		$\beta(s) = \{a_1, \dots, a_\ell, b_1, \dots, b_r\}$.
 		
 		Let $t \in V(T)$ be a child of $s$.
 		Write $u$ for the unique vertex in $\beta(s) \setminus \beta(t)$.
 		Suppose wlog that $u = a_\ell \in A$.
 		Write $K^t$ for the subgraph of $F$ induced by $\beta(s) \cup \bigcup_{t' \in V(T^t)} \beta(t')$.
 		Let $\boldsymbol{K}^t \coloneqq (K^t, \boldsymbol{a}, \boldsymbol{b})$.
 		We first construct a circuit for $\boldsymbol{K}^t_{n,m}(\boldsymbol{v}, \boldsymbol{w})$.
 		Let $u'$ denote the unique vertex in $\beta(t) \setminus \beta(s)$.
 		Distinguish cases:
 		\begin{enumerate}
 			\item If $u' \in A$, let $\boldsymbol{a}' \coloneqq \boldsymbol{a}[\ell/u']$ and define $\boldsymbol{F}^t \coloneqq (F^t, \boldsymbol{a}', \boldsymbol{b})$.
 				  Then
 				  \begin{equation}\label{eq:circuit1}
 				  \boldsymbol{K}^t_{n,m}(\boldsymbol{v}, \boldsymbol{w}) = \sum_{v \in [n]} \boldsymbol{F}^t_{n,m}(\boldsymbol{v}[\ell/v], \boldsymbol{w}) \prod_{j \in [r]} \prod_{\substack{a b \in E(F) \\ a = a_\ell \\ b = b_j}} x_{v_\ell w_j}.
 				  \end{equation}
 			\item If $u' \in B$, let $\boldsymbol{a}' \coloneqq \boldsymbol{a}[\ell/]$ and define $\boldsymbol{F}^t \coloneqq (F^t, \boldsymbol{a}', \boldsymbol{b}u')$.
 				  Then
 				  \begin{equation}\label{eq:circuit2}
 				  \boldsymbol{K}^t_{n,m}(\boldsymbol{v}, \boldsymbol{w}) = \sum_{w \in [m]} \boldsymbol{F}^t_{n,m}(\boldsymbol{v}[\ell/], \boldsymbol{w}w) \prod_{j \in [r]} \prod_{\substack{a b \in E(F) \\ a = a_\ell \\ b = b_j}} x_{v_\ell w_j}.
 				  \end{equation}
 		\end{enumerate}
 		In both cases, the inductive hypothesis applies to $\boldsymbol{F}^t$.
 		In order to construct $\boldsymbol{K}^t_{n,m}(\boldsymbol{v}, \boldsymbol{w})$, we require one summation gate of fan-in $\max\{n,m\}$
 		and a product gate of fan-in at most $\norm{F} + 1$.
 		
 		Let now $t_1, \dots, t_\ell$ for $\ell \leq k$ denote the children of $s$.
 		By \cref{lem:gluing},
 		\begin{equation}\label{eq:circuit3}
 			\boldsymbol{F}^s_{n,m}(\boldsymbol{v},\boldsymbol{w}) = \prod_{i=1}^\ell \boldsymbol{K}^{t_i}(\boldsymbol{v}, \boldsymbol{w}).
 		\end{equation}
 		This polynomial can be realised by one additional product gate of fan-in at most $k$.
 		
 		Thus, in order to construct the additional gates for $s \in V(T)$, 
 		we require at most \[ n^\ell m^r \ell! r! (4 + k  + \max\{n,m\} + \norm{F})  \leq 4(n+m)^{k+1} k!^2 k \cdot \norm{F} \] gates and edges.
 		It follows that the overall circuit has the desired size. In the end, we invoke \cref{lem:rigidifyCircuits} to make the circuit rigid, which allows us to speak about the supports of gates.
 		
 		Let $\boldsymbol{v} \in [n]^\ell$ and $\boldsymbol{w} \in [m]^r$.
 		At any stage, the subcircuit computing $\boldsymbol{F}^s_{n,m}(\boldsymbol{v}, \boldsymbol{w})$ is by construction $\StabP_{\Sym_n \times \Sym_m}(\boldsymbol{v}, \boldsymbol{w})$-symmetric, so $\boldsymbol{v}\boldsymbol{w}$ is a support of the gate computing $\boldsymbol{F}^s_{n,m}(\boldsymbol{v}, \boldsymbol{w})$. Note that $|\boldsymbol{v}\boldsymbol{w}| \leq k$.
 		
 		The overall circuit is $\Sym_n \times \Sym_m$-symmetric,
 		since, by \cref{lem:labelsMovedByPermutations}, $(\pi, \sigma)(\boldsymbol{F}^s_{n,m}(\boldsymbol{v}, \boldsymbol{w})) = \boldsymbol{F}^s_{n,m}(\pi(\boldsymbol{v}), \sigma(\boldsymbol{w}))$ is computed as well for $(\pi, \sigma) \in \Sym_n \times \Sym_m$.
 	\end{proof}
 	
 	The lemma implies the theorem:
 	
 	\thmHomCircuit*
 	\begin{proof}
 		By \cite[Lemma~6]{seppelt_algorithmic_2024} and its proof, 
 		there exists a tree decomposition $(T, \beta)$ of $F$ with a vertex $r \in V(T)$ possessing the properties required by \cref{lem:hom-circuit-core-bipartite}.
 		Note that $|V(T)| \leq |V(F)|$.
 		
 		Write $A \uplus B$ for the bipartition of $F$.
 		Let $\ell, r \in \mathbb{N}$ be such that there exists $\boldsymbol{a} \in A^\ell$ and $\boldsymbol{b} \in B^r$ are such that $\beta(r) = \{a_1, \dots, a_\ell, b_1, \dots, b_r\}$.
 		Let $\boldsymbol{F} \coloneqq (F, \boldsymbol{a}, \boldsymbol{b}) \in \mathcal{T}^k(\ell, r)$.
 		By \cref{lem:hom-circuit-core-bipartite}, 
 		for $n,m \in \mathbb{N}$,
 		there exists a $\Sym_n \times \Sym_m$-symmetric circuit computing $\boldsymbol{F}_{n,m}(\boldsymbol{v}, \boldsymbol{w})$
 		for every $\boldsymbol{v} \in [n]^\ell$ and $\boldsymbol{w} \in [m]^r$.
 		The circuit is of size at most $4 k!^2 k \cdot (n+m)^{k+1} \cdot \norm{F}^2$ and support size at most $k$.
 		By \cref{lem:unlabelling}, the desired circuit can be constructed from this circuit by adding one additional summation gate of fan-in $(n+m)^k$.
 		That is,
 		\[
 			\hom_{F, n,m} = \sum_{\boldsymbol{v} \in [n]^\ell} \sum_{\boldsymbol{w} \in [m]^r} \boldsymbol{F}_{n,m}(\boldsymbol{v}, \boldsymbol{w}). \qedhere
 		\]
 	\end{proof}
 	
	
	\subsection{Polynomial Orbit Size Versus Polynomial Size}
	
	\cref{thm:main1} gives a characterisation for when symmetric circuits with polynomial \emph{orbit size}, rather than polynomial total size, exist. A super-polynomial lower bound on the orbit size of course implies a super-polynomial lower bound on the circuit size itself, but for upper bounds, this is not true: There can exist symmetric circuits with polynomial orbit size but whose total size is super-polynomial. The following theorem shows that under the common assumption $\VP \neq \VNP$, this situation indeed arises: There exist polynomials in $\mathfrak{T}_{n,m}^k$ that do not admit polynomial size circuits (neither symmetric nor asymmetric), unless $\VP = \VNP$. Thus, \cref{thm:main1} is best-possible, and obtaining a characterisation of \emph{total} circuit size rather than orbit size can be expected to be as difficult for symmetric circuits as for general ones.
	
	\begin{theorem}
		There is a $\VNP$-hard family of polynomials $(p_{n})_{n \in \bbN}$ such that $p_{n} \in \mathfrak{T}_{n,n}^2$ for all $n \in \mathbb{N}$.
	\end{theorem}	
	
	
	
	\begin{proof}
		We consider a family of polynomials that was shown to be $\VNP$-hard in \cite[Theorem~1]{curticapean_et_al2022}, and modify it. 
		For a partition $\lambda \vdash d$, write $R$ for the sets of parts of $\lambda$,
		and, for $i \in [d]$, write $s_i$ for the number of size-$i$ parts in $\lambda$.
		Let $n \geq d$ be an integer. 
		Consider the \emph{monomial symmetric polynomial}:
		\[
		m_\lambda(y_1, \dots, y_n) \coloneqq \alpha_\lambda \sum_{f \colon R \hookrightarrow [n]} \prod_{r \in R} y_{f(r)}^{|r|}.
		\]	
		Here, $\alpha_\lambda \coloneqq \prod_{i=1}^d \frac{1}{s_i!}$ is such that all coefficients in $m_\lambda$ are $1$.
		
		In \cite[Theorem~1]{curticapean_et_al2022}, the existence of polynomial functions $r,s \colon \mathbb{N} \to \mathbb{N}$ and of partitions $\lambda_n \vdash r(n)$, $n \in \mathbb{N}$,
		is established such that the family of polynomials $m_{\lambda_n}(y_1, \dots, y_{s(n)})$ is $\VNP$-complete.
		
		Given $m_\lambda(y_1, \dots, y_n)$, we define a polynomial $p_\lambda \in \mathbb{Q}[\mathcal{X}_{n,n}]$ by substituting, for every $v \in [n]$, the expression $\sum_{w \in [n]} x_{vw}$ for $y_v$.
		Clearly, $m_\lambda$ can be recovered from $p_\lambda$ by substituting $\frac1n y_v$ for $x_{vw}$ for all $v, w \in [n]$.
		Hence, the $\VNP$-complete family in \cite[Theorem~1]{curticapean_et_al2022} gives rise to a $\VNP$-complete family of polynomials of the form $p_\lambda \in \mathbb{Q}[\mathcal{X}_{n,n}]$.
		It remains to prove membership in $\mathfrak{T}^2_{n,n}$.
		
		To that end, let $\lambda \vdash d$ and $n \geq d$ be arbitrary.
		Recall that $R$ denotes the set of parts of $\lambda$
		and apply Möbius inversion to the outer sum.
		By \cref{lem:moebius-polynomial},
		\begin{align*}
			m_\lambda(y_1, \dots, y_n) &= \alpha_\lambda \sum_{\pi \in \Pi(R)} \mu_\pi \sum_{f \colon R/\pi \to [n]} \prod_{r \in R} y_{f\pi(r)}^{|r|} \\
			&= \alpha_\lambda \sum_{\pi \in \Pi(R)} \mu_\pi \sum_{f \colon R/\pi \to [n]} \prod_{s \in R/\pi} y_{f(s)}^{\sum_{r \in s}|r|} \\
			&= \alpha_\lambda  \sum_{\pi \in \Pi(R)} \mu_\pi \prod_{s \in R/\pi} \sum_{v \in [n]} y_v^{\sum_{r \in s}|r|}.
		\end{align*}
		For $\phi(v) \coloneqq \sum_{w \in [n]} x_{vw}$ where $v \in [n]$,
		we have $\phi \in \mathfrak{T}^2_{n,n}(1,0)$ by \cref{lem:unlabelling,ex:edge}.
		Similarly, $\sum_{v \in [n]} \phi(v)^{\ell} \in \mathfrak{T}^2_{n,n}$ for every  integer $\ell \geq 1$ by  \cref{lem:gluing,lem:unlabelling}.
		Hence, the desired polynomial $p_{\lambda} = m_{\lambda}(\phi(1), \dots, \phi(n))$ is a linear combination of polynomials in $\mathfrak{T}^2_{n,m}$ and thus in $\mathfrak{T}^2_{n,m}$ itself.
	\end{proof}	
	
	\section{Counting Width and Orbit Size Lower Bounds}
	
	\Cref{thm:main1}, the main result of the preceding section, characterises the polynomials which are computed by symmetric circuits of polynomial orbit size. Ultimately, the goal of the symmetric circuit programme is to show super-polynomial lower bounds on the orbit and hence circuit size, for polynomials for which this is not currently possible in the non-symmetric circuit setting. However, \cref{thm:main1} itself does not give us an immediate tool to prove such lower bounds in general. 
	
	In this section and in the following ones, we consider more restricted families of polynomials with the goal of determining their orbit size.
	To the best of our knowledge, the only available tool for proving orbit size lower bounds is the \emph{counting width} as introduced by \textcite{dawar_definability_2017}.
	It describes the descriptive complexity of a graph parameter.
	
	In order to define counting width, we view $(n,m)$-vertex bipartite graphs as elements of $\{0,1\}^{n \times m}$.
	More generally, an \emph{edge-weighted} $(n,m)$-vertex bipartite graph is an element of $\mathbb{Q}^{n \times m}$. 
	
	Let $\tau \coloneqq \{A, B\} \cup \{R_q \mid q \in \mathbb{Q}\}$ denote the infinite relational signature with two unary symbols $A$ and $B$ and one binary symbol $R_q$ for every $q \in \mathbb{Q}$.
	We view elements $G \in \mathbb{Q}^{n \times m}$ as $\tau$-structures with universe $n \times m$ by interpreting $A^G \coloneqq [n]$, $B^G \coloneqq [m]$, and $R_q^G = \{ij \in n \times m \mid G(ij) = q\}$ for $q \in \mathbb{Q}$.
	% By default, whenever we speak of bipartite graphs, we mean that they are equipped with two unary relations marking the bipartition.
	An \emph{isomorphism} between $G, H \in \mathbb{Q}^{n \times m}$ is a bijection $h \colon [n] \uplus [m] \to [n] \uplus [m]$ such that $G \circ h = H$.
	
	
	\begin{definition}\label{def:counting-width}
		Let $n,m\in \mathbb{N}$.
		Let $p$ be an isomorphism-invariant function with domain $\mathbb{Q}^{n \times m}$. 
		\begin{enumerate}
			\item The \emph{counting width of $p$ on $(n,m)$-vertex simple graphs} is the least integer $k$ such that, for all $G, H \in \{0,1\}^{n \times m}$, if $G$ and $H$ are $\mathcal{C}^k$-equivalent, then $p(G) = p(H)$.
			\item The \emph{counting width of $p$ on $(n,m)$-vertex edge-weighted graphs} is the least integer $k$ such that, for all $G, H \in \mathbb{Q}^{n \times m}$, if $G$ and $H$ are $\mathcal{C}^k$-equivalent, then $p(G) = p(H)$.
		\end{enumerate}
	\end{definition} 
	
	Since $p$ is assumed to be isomorphism-invariant and every edge-weighted graph can be defined in $\mathcal{C}^{n+m}$ up to isomorphism, the counting width on $(n,m)$-vertex graphs is at most $n+m$ and well-defined.
	Clearly, the counting width on edge-weighted graphs is an upper bound on the counting width on simple graphs.
	
	The following theorem establishes the desired connection between counting width and orbit size.	
	It is essentially due to \textcite{anderson_symmetric_2017}.
	
	\begin{theorem}\label{thm:counting-width}
		Let $(p_{n,m})_{n,m\in \mathbb{N}}$ be a family of polynomials.
		\begin{enumerate}
			\item If the $p_{n,m}$ admit $\Sym_n \times \Sym_m$-symmetric circuits of orbit size polynomial in $n+m$,
			then there is a constant $k \in \bbN$ that bounds the counting width of $p_{n,m}$ on $(n,m)$-vertex edge-weighted graphs, for all $n,m \in \bbN$.
			\item If the $p_{n,m}$ admit $\Sym_n \times \Sym_m$-symmetric circuits of orbit size $2^{o(n)}$, then the counting width of $p_{n,m}$ on $(n,m)$-vertex edge-weighted graphs is at most $o(n)$.
		\end{enumerate}
	\end{theorem}
	\begin{proof} 
		We prove the contrapositive of the first statement. Suppose the counting width of $p_{n,m}$ on $(n,m)$-vertex edge-weighted graphs is not bounded by a constant. Then for every $k \in \bbN$, there exist $n,m \in \bbN$ such that we find $\mathcal{C}^{k}$-equivalent $(n,m)$-vertex edge-weighted graphs $G,H$ with $p_{n,m}(G) \neq p_{n,m}(H)$.
		The graphs being $\mathcal{C}^{k}$-equivalent means that Duplicator has a winning strategy in the bijective $k$-pebble played on $(G,H)$. Then by \cite[Theorem~4.8]{dawar2021lower}, any $\Sym_n \times \Sym_m$-symmetric circuit whose support sizes are bounded by $k/2$ yields the same output on the adjacency matrices of $G$ and $H$ (the game in that theorem is not the standard bijective $k$-pebble game, but an equivalent one). Since $p_{n,m}(G) \neq p_{n,m}(H)$, no such circuit can represent $p_{n,m}$. Since this argument can be made for every $k \in \bbN$, this shows that there do not exist $\Sym_n \times \Sym_m$-symmetric circuits with constant support size for all $p_{n,m}$.
		Then by the first part of \cref{lem:constantSupportOfGates} (and the fact that circuits can always be assumed to be rigid), there do not exist symmetric circuits with polynomial orbit size.
		The second statement can be shown similarly: Suppose the counting width of $p_{n,m}$ on $(n,m)$-vertex edge-weighted graphs is $\Omega(n)$. Then for every function $f(n) \in o(n)$, there exist $n,m \in \bbN$ such that we find $\mathcal{C}^{f(n)}$-equivalent $(n,m)$-vertex edge-weighted graphs $G,H$ with $p_{n,m}(G) \neq p_{n,m}(H)$. With the same reasoning as before, using the second part of \cref{lem:constantSupportOfGates} now, we conclude that there do not exist $\Sym_n \times \Sym_m$-symmetric circuits for $p_{n,m}$ with orbit size $2^{o(n)}$.
	\end{proof}	
	
	
	
	By \cref{thm:counting-width}, if a family of polynomials $p_{n,m} \in \mathbb{Q}[\mathcal{X}_{n,m}]$ admits $\Sym_n\times \Sym_m$-symmetric circuits of orbit size polynomial in $n+m$,
	then the counting width of $p_{n,m}$ is bounded on edge-weighted bipartite graphs.
	In fact, the counting width corresponds precisely to the treewidth of the graphs in the homomorphism representation:
	%TODO: read this
	\begin{theorem}\label{thm:counting-width-explicit}
		Let $k,n,m \in \mathbb{N}$. If $p \in \mathfrak{T}^k_{n,m}$,
		then the counting width of $p$ on edge-weighted $(n,m)$-vertex graphs is at most $k$.
	\end{theorem}
	\begin{proof}
		Since $p$ is a linear combination of homomorphism polynomials,
		it suffices to show that if $G, H \in \mathbb{Q}^{n\times m}$ are $\mathcal{C}^k$-equivalent, then $\hom_{F, n,m}(G) = \hom_{F, n,m}(H)$ for every bipartite multigraph $F$ with $\tw(F) < k$.
		To that end, consider the following claim.

		For relational structures $G$ and $H$ over the same universe $U$,
		we say that $(G,\boldsymbol{v})$  and $(H,\boldsymbol{w})$ are \emph{$\mathcal{C}^k$-equivalent} for $\boldsymbol{v},\boldsymbol{w} \in U^\ell$ and $\ell \leq k$
		if, for every $\mathcal{C}^k$-formula $\phi$ with $\ell$ free variables, $G\models \phi(\boldsymbol{v}) \iff H \models \phi(\boldsymbol{w})$.
		We prove the following:
		
		\begin{claim}
			Let $\ell, r \in \mathbb{N}$ such that $\ell +r \leq k$ and $\boldsymbol{F} \in \mathcal{T}^k(\ell, r)$.
			Let $\boldsymbol{v} \in [n]^\ell$ and $\boldsymbol{w} \in [m]^r$. 
			If $G, H \in \mathbb{Q}^{n\times m}$ are such that $(G, \boldsymbol{v}, \boldsymbol{w})$ and $(H, \boldsymbol{v}, \boldsymbol{w})$ are $\mathcal{C}^k$-equivalent,
			then $\boldsymbol{F}_{n,m}(\boldsymbol{v}, \boldsymbol{w})(G) = \boldsymbol{F}_{n,m}(\boldsymbol{v}, \boldsymbol{w})(H)$.
		\end{claim}
		\begin{claimproof}
			Write $\boldsymbol{F} = (F, \boldsymbol{a}, \boldsymbol{b})$ and $V(F) = A \uplus B$ for its bipartition.
			By \cref{def:treewidth-labelled} and \cite[Lemma~7]{bodlaender_partial_1998},
			there exists a tree decomposition $(T, \beta)$ of $F$ such that
			\begin{enumerate}
				\item there exists a vertex $q \in V(T)$ such that $\beta(q)= \{a_1, \dots, a_\ell, b_1, \dots, b_r \}$,
				\item for every $s \in V(T)$ which has
                                  two children $t_1, t_2 \in V(T)$ in
                                  the rooted tree $(T, q)$, we have $\beta(t_1) = \beta(s) = \beta(t_2)$,
				\item for every $s \in V(T)$ which has
                                  a single child $t \in V(T)$ in the
                                  rooted tree $(T, q)$, we have $\beta(t) \subseteq \beta(s)$ and $|\beta(s)| = |\beta(t)|+1$ or  $\beta(s) \subseteq \beta(t)$ and $|\beta(t)| = |\beta(s)|+1$,
				\item for every $s \in V(T)$, $|\beta(s)| \leq k$.
			\end{enumerate}
			We prove the claim by induction on the height of the rooted tree $(T, q)$.
			
			For the base case, suppose that $q$ is the only vertex in $T$.
			Since all vertices of $F$ are labelled, 
			$\boldsymbol{F}_{n,m}(\boldsymbol{v}, \boldsymbol{w})(G)$ depends solely on the isomorphism type of the edge-weighted bipartite subgraph of $G$ induced by $\boldsymbol{v}$ and $\boldsymbol{w}$.
			Since $\ell + r \leq k$, this isomorphism type can be defined in $\mathcal{C}^k$.
			This implies the desired statement.
			
			For the inductive step, consider three cases:
			\begin{enumerate}
				\item The vertex $q$ has a single child $t \in V(T)$, $\beta(t) \subseteq \beta(q)$, and $|\beta(q)| = |\beta(t)|+1$.
				
				Define $F'$ as the subgraph of $F$ induced by $\bigcup_{s \in V(T) \setminus \{q\}} \beta(s)$.
				Suppose without loss of generality that $\beta(q) \cap B = \beta(t) \cap B$.
				Let $i \in [\ell]$ be such that $a_i \not\in \beta(t)$.
				Then $\boldsymbol{F}' \coloneqq (F', \boldsymbol{a}[i/], \boldsymbol{b})$ is a graph to which the inductive hypothesis applies.
				
				If $(G, \boldsymbol{v}, \boldsymbol{w})$ and $(H, \boldsymbol{v}, \boldsymbol{w})$ are $\mathcal{C}^k$-equivalent,
				then so are $(G, \boldsymbol{v}[i/], \boldsymbol{w})$ and $(H, \boldsymbol{v}[i/], \boldsymbol{w})$.
				By the inductive hypothesis,
				$\boldsymbol{F}'_{n,m}(\boldsymbol{v}[i/], \boldsymbol{w})(G) = \boldsymbol{F}'_{n,m}(\boldsymbol{v}[i/], \boldsymbol{w})(H)$.

				Now $\boldsymbol{F}$ differs from $\boldsymbol{F}'$ only in the vertex $a_i$ and the potential edges connecting $a_i$ to the vertices in $\boldsymbol{b}$.
				More precisely,
				\[
					\boldsymbol{F}_{n,m}(\boldsymbol{v}, \boldsymbol{w})(G)
					= \boldsymbol{F}'_{n,m}(\boldsymbol{v}[i/], \boldsymbol{w})(G) \prod_{a_ib_j \in E(F)} G(v_i w_j).
				\]
				Since $(G, \boldsymbol{v}, \boldsymbol{w})$ and $(H, \boldsymbol{v}, \boldsymbol{w})$ are $\mathcal{C}^k$-equivalent, $G(v_i w_j) = H(v_i w_j)$ for all $i \in [\ell]$ and $j \in [r]$.
				This implies the claim.
				
				\item The vertex $q$ has a single child $t \in V(T)$, $\beta(t) \supseteq \beta(q)$, and $|\beta(q)|+1 = |\beta(t)|$.
				
				Without loss of generality $\beta(t) \cap B = \beta(q) \cap B$. Write $a$ for the unique element of $\beta(t) \setminus \beta(q)$. 
				The inductive hypothesis applies to $\boldsymbol{F}' \coloneqq (F, \boldsymbol{a}a, \boldsymbol{b})$.
				Here, we insert the vertex $a$ at the end of the tuple only to ease notation.
				If  $(G, \boldsymbol{v}, \boldsymbol{w})$ and $(H, \boldsymbol{v}, \boldsymbol{w})$ are $\mathcal{C}^k$-equivalent and $\boldsymbol{v}$ and $\boldsymbol{w}$ are tuples of total length at most $k-1$,
				then there exists a bijection $\pi \colon [n] \to [n]$ such that 
				$(G, \boldsymbol{v}v, \boldsymbol{w})$ and $(H, \boldsymbol{v}\pi(v), \boldsymbol{w})$ are $\mathcal{C}^k$-equivalent for all $v \in [n]$.
				By the inductive hypothesis,
				$\boldsymbol{F}'_{n,m}(\boldsymbol{v}v, \boldsymbol{w})(G) = \boldsymbol{F}'_{n,m}(\boldsymbol{v}\pi(v), \boldsymbol{w})(H)$ for all $v \in [n]$.
				Hence, by \cref{lem:unlabelling},
				\[
					\boldsymbol{F}_{n,m}(\boldsymbol{v}, \boldsymbol{w})(G)	
					= \sum_{v \in [n]} \boldsymbol{F}'_{n,m}(\boldsymbol{v}v, \boldsymbol{w})(G)	
					= \sum_{v \in [n]} \boldsymbol{F}'_{n,m}(\boldsymbol{v}v, \boldsymbol{w})(H)
					= 	\boldsymbol{F}_{n,m}(\boldsymbol{v}, \boldsymbol{w})(H).	
				\]
				
				\item The vertex $q$ has several children $t_1, \dots, t_\ell$.
				
				For $i \in [\ell]$,
				define $F^i$ as the subgraph of $F$ induced by $\beta(t_i)$, and $\beta(s)$ for all descendents $s$ of $t$.
				Then the inductive hypothesis applies to $\boldsymbol{F}^i \coloneqq (F^i, \boldsymbol{a}, \boldsymbol{b})$.
				By \cref{lem:gluing}, $\boldsymbol{F}_{n,m}(\boldsymbol{v}, \boldsymbol{w}) = \prod_{i=1}^\ell \boldsymbol{F}^i_{n,m}(\boldsymbol{v}, \boldsymbol{w})$.
				The claim follows.\qedhere
			\end{enumerate}
		\end{claimproof}
		The claim implies the theorem with $\ell = 0 = r$.	
	\end{proof}
	
We are going to establish the converse direction in various restricted cases, as summarised in \cref{thm:main2}. To start with, we note that if we only care about the semantics of the polynomials as functions on simple graphs, then counting width and treewidth indeed coincide:
 	
	
	\subsection{Characterisation of polynomials up to multilinearisation}
	The \emph{multilinearisation} of a polynomial $p$ is obtained from $p$ by setting all non-zero exponents to $1$. 
	If we consider polynomials only up to multilinearisation, i.e.\ up to their evaluation under $\{0,1\}$-assignments, then
	the following converse of \cref{thm:counting-width-explicit} holds.	
	\begin{theorem}\label{thm:multilinear}
		Let $k,n,m \in \mathbb{N}$.
		For $p \in \mathbb{Q}[\mathcal{X}_{n,m}]$, the following are equivalent:
		\begin{enumerate}
			\item there exists $q \in \mathfrak{T}^k_{n,m}$ such that $p$ and $q$ are equal up to multilinearisation,
			\item the counting width of $p$ on $(n,m)$-vertex simple bipartite graphs is at most $k$.
		\end{enumerate}
	\end{theorem}
	\begin{proof}
		For the forward direction, observe that, by \cref{thm:counting-width-explicit},
		the desired statement holds for $q$.
		The polynomials $p$ and $q$ assume the same values on the Boolean cube $\{0,1\}^{n \times m}$.
		Hence, the counting width of $p$ on $(n,m)$-vertex simple bipartite graphs is at most $k$.
		
		For the converse direction,
		we write the function $p \colon \{0,1\}^{n\times m} \to \mathbb{Q}$ as a linear combination of $\mathcal{C}^k$-type indicator functions.
		This argument requires interpolation and thus relies on the fact that there are only finitely many simple $(n,m)$-vertex bipartite graphs.
		
		Let $C_1 \uplus \dots \uplus C_r = \{0,1\}^{n\times m}$ be an enumeration of the $\mathcal{C}^k$-types of simple $(n,m)$-vertex bipartite graphs.
		Pick representatives $G_1, \dots, G_r$.
		By \cite{dvorak_recognizing_2010},
		for $i \neq j$, there  exists a bipartite graph $F_{ij}$ of treewidth at most $k-1$ such that $\hom(F_{ij}, G_i) \neq \hom(F_{ij}, G_j)$.
		For $i \in [r]$, define the polynomial
		\[
			q_i\coloneqq \prod_{j \neq i} \frac{\hom_{F_{ij}, n,m} - \hom(F_{ij}, G_j)}{\hom(F_{ij}, G_i) - \hom(F_{ij}, G_j)} \in \mathfrak{T}^k_{n,m}.
		\]
		Here, $\hom_{F_{ij}, n,m}$ is a polynomial in $\mathfrak{T}^k_{n,m}$ while $\hom(F_{ij}, G_j)$ is merely an integer.
		For all $G \in \{0,1\}^{n\times m}$, $q_i(G) = 1$ if, and only if, $G \in C_i$, and $q_i(G) =0$ otherwise.
		Since $p$ is constant on $\mathcal{C}^k$-types,
		it is
		\[
			p = \sum_{i=1}^r \alpha_i q_i
		\]
		as functions on $\{0,1\}^{n\times m}$ for some coefficients $\alpha_i \in \mathbb{Q}$.
		Hence, $p$ is equal to the polynomial $\sum_{i=1}^r \alpha_i q_i \in \mathfrak{T}^k_{n,m}$ up to multilinearisation.
	\end{proof}
	We see in \cref{ex:matching} that the statements in \cref{thm:multilinear} do not imply that the counting width of $p$ on $(n,m)$-vertex \emph{edge-weighted} graphs is bounded.
	The argument yielding \cref{thm:multilinear} is bound to fail when multilinearisation is not applied:
	\begin{example}\label{ex:simple-interpolation}
		Let $F$ be a graph and $n,m\in \mathbb{N}$.
		Write $C = \{\hom(F, G) \mid G \in \{0,1\}^{n \times m}\}$ for the finite set of homomorphism counts from $F$ to any simple $(n,m)$-vertex graph.
		Consider the polynomial
		\begin{align*}
		p \coloneqq \hom_{F, n,m} \prod_{c \in C} (\hom_{F, n,m} - c)
		&= \hom_{F, n,m} \sum_{C' \subseteq C} \hom_{|C \setminus C'| F, n,m} \prod_{c \in C'} (-1)^{|C'|} \cdot c\\
		&=   \sum_{C' \subseteq C} \hom_{(|C \setminus C'|+1) F, n,m} \prod_{c \in C'} (-1)^{|C'|} \cdot c
		\end{align*}
		Then $p$ is a linear combination of homomorphism polynomials and $p(G) = 0$ for all $G \in \{0,1\}^{n \times m}$.
		Hence, the counting width of $p$ on simple $(n,m)$-vertex graphs is zero.
		However, $p \not\in \mathfrak{T}^{k}_{n,m}$ for $k < \tw(F) - 1$ when $n,m \geq |E(F)|$.
	\end{example}
	
	

	
	\section{Towards a Dichotomy for Homomorphism Polynomials}
	
	\subsection{Linear Combinations of Homomorphism Polynomials of Logarithmic-Size Patterns}
	
	Towards a converse of \cref{thm:counting-width-explicit},
	we first consider elements of $\mathfrak{T}^k_{n,m}$ that contain only homomorphism polynomials of connected graphs of logarithmic size.
	For a simple bipartite graph $F$ with bipartition $A \uplus B$, write $\gamma_A(F) \coloneqq \sum_{v \in A} 2^{\deg_F(v)-1}$ and $\gamma_B(F) \coloneqq \sum_{v \in B} 2^{\deg_F(v)-1}$.
	If $F$ is a bipartite multigraph, then $\gamma_A(F) \coloneqq \gamma_A(F')$ and $\gamma_B(F) \coloneqq \gamma_B(F')$ where $F'$ is the simple graph obtained from $F$ by discarding multiedges.
	
	\begin{theorem}\label{thm:logarithmic-lincomb}
		Let $k,n,m \in \mathbb{N}$.
		Let $p = \sum_{i\in [r]} \alpha_i \hom_{F_i, n,m}$ be a linear combination of homomorphism polynomials of connected bipartite multigraphs $F_i$ with bipartition $A_i \uplus B_i$ such that $\gamma_{A_i}(F_i) \leq n$ and $\gamma_{B_i}(F_i) \leq m$ for every $i \in [r]$.
		The following are equivalent:
		\begin{enumerate}
			\item $p \in \mathfrak{T}^k_{n,m}$,
			\item the counting width of $p$ on simple $(n,m)$-vertex graphs is at most $k$,
			\item the counting width of $p$ on edge-weighted $(n,m)$-vertex graphs is at most $k$.
		\end{enumerate}
	\end{theorem}	
	\begin{proof}
		Given \cref{def:counting-width,thm:counting-width-explicit}, it suffices to prove that if the counting width of $p$ on simple $(n,m)$-vertex bipartite graphs is at most $k$, then $p \in \mathfrak{T}^k_{n,m}$.
		For contraposition, suppose that $p \not\in \mathfrak{T}^k_{n,m}$.
		Hence, one of the graphs $F_i$ is of treewidth at least $k$, wlog $F_1$.
		The objective is to construct simple $(n,m)$-vertex bipartite graphs $G$ and $H$ such that $G \equiv_{\mathcal{C}^{k}} H$ and $p(G) \neq p(H)$.
		To that end, we may replace the multigraphs $F_i$ with the simple graphs obtained by discarding multiple edges.
		
		By grouping summands, we may suppose that the $F_i$ are pairwise non-isomorphic.
		Let $\mathcal{F}$ denote the finite directed graph whose vertices are the $F_i$ and which contains an arc $F_i \to F_j$ if, and only if, there is a weak oddomorphism from $F_i$ to $F_j$, cf.~\cref{def:oddomorphism}.
		By \cref{obs:oddo-surjective}, there do not exist distinct $F_i$ and $F_j$ such that both the arc $F_i \to F_j$ and the arc $F_j \to F_i$ are in $\mathcal{F}$, as their composition would yield an isomorphism $F_i \cong F_j$.
		Since weak oddomorphisms compose \cite[Lemma~5.3]{roberson_oddomorphisms_2022}, $\mathcal{F}$ is acyclic.
		
		Pick a graph $G$ of in-degree zero in $\mathcal{F}$ among the $F_i$ such that there exists an arc $G \to F_1$ to the graph $F_1$ with $\tw(F_1) \geq k$.
		Such a $G$ exists since $\mathcal{F}$ is a finite directed acyclic graph.
		By \cref{thm:neuen},
		it holds that $\tw(G) \geq \tw(F_1) \geq k$.
		
		By \cref{thm:neuen-bipartite}, the \textsmaller{CFI} graphs $G_0$ and $G_1$ of $G$ are $\mathcal{C}^k$-equivalent  as bipartite graphs with the fixed bipartition.
		By \cref{thm:rob3.13-bipartite},
		since $G$ is of in-degree zero in $\mathcal{F}$,
		it holds that $\hom(F, G_0) = \hom(F, G_1)$ for all $F$ in $\mathcal{F}$ other than $G$.
		Furthermore, $\hom(G, G_0) \neq \hom(G, G_1)$, again by \cref{thm:rob3.13-bipartite}.
		Here, only homomorphisms respecting the fixed bipartitions are counted.
		It follows that $p(G_0) \neq p(G_1)$, as desired.
	\end{proof}
	
	
	
\subsection{Single homomorphism polynomials}
We now turn to the case of single homomorphism polynomials, as opposed to linear combinations of them. 
Like in \cref{thm:logarithmic-lincomb}, it turns out that treewidth and counting width coincide, but when we are only dealing with one hom-polynomial, we do not need to assume any size bounds on the pattern graphs to prove this.

	\begin{theorem}\label{thm:dichotomy-chromatic}
		For every family  $(F_{n,m})_{n,m \in \mathbb{N}}$ of bipartite multigraphs,
		the following are equivalent:
		\begin{enumerate}
			\item the treewidth $\tw(F_{n,m})$ is bounded,\label{it:hom1}
			\item the counting width of $\hom_{F_{n,m},n,m}$ on $(n,m)$-vertex simple graphs is bounded,\label{it:hom2}
			\item the counting width of $\hom_{F_{n,m},n,m}$ on $(n,m)$-vertex edge-weighted graphs is bounded,\label{it:hom3}
			\item the $\hom_{F_{n,m}, n, m}$ admit $\Sym_n \times \Sym_m$-symmetric circuits of orbit size polynomial in $n+m$,\label{it:hom5}
			\item the $\hom_{F_{n,m}, n, m}$ admit $\Sym_n \times \Sym_m$-symmetric circuits of size polynomial in $\norm{F_{n,m}} + n +m$ and orbit size polynomial in $n+m$.\label{it:hom4}
			\end{enumerate}
	\end{theorem}

	Remarkably, \cref{thm:dichotomy-chromatic} shows that for single homomorphism polynomials there is no difference between counting width on simple graphs and counting width on edge-weighted graphs.


	Given the results of the preceding section,  it remains to show \enquote{\ref{it:hom2} $\Rightarrow$ \ref{it:hom1}} via the following \cref{lem:minor-chromatic}.
	An \emph{induced minor} of a simple graph $F$ is a graph which can be obtained from an induced subgraph of $F$ by contracting edges.
	
	
	\begin{lemma}\label{lem:minor-chromatic}
		Let $F$ be a $c$-colourable simple graph for $c \geq 1$.
		Let $P$ be an induced minor of $F$ such that $P$ is connected.
		Write $k \coloneqq \tw(P)$, $\gamma(P) \coloneqq \sum_{v \in V(P)}2^{\deg_P(v)-1}$, and $n \coloneqq (1+\gamma(P)) c$.
		Then there exist $n$-vertex simple graphs $G_0$ and $G_1$ such that
		\begin{enumerate}
			\item $G_0$ and $G_1$ are $\mathcal{C}^k$-equivalent,
			\item $\hom(F, G_0) \neq \hom(F, G_1)$,
			\item $\hom(F', G_0) \geq \hom(F', G_1)$ for every graph $F'$.
		\end{enumerate}
	\end{lemma}
	\begin{proof}
		Let $G'_0$ and $G'_1$ denote the \textsmaller{CFI} graphs of $P$.
		Construct $G_0$ and $G_1$ by making the lexicographic products $G'_0 \cdot K_c$ and $G'_1 \cdot K_c$, respectively,
		adjacent with a copy of $K_c$, i.e.\ 
		$G_i \coloneqq (G'_i \cdot K_c) \boxplus K_c$ for $i \in \{0,1\}$.
		The resulting graphs have $\gamma(P) c + c = n$ vertices.
		Regardless of whether $G'_0$ and $G'_1$ are connected,
		it holds that $G_0$ and $G_1$ are connected, as $c \geq 1$.
		
		By \cref{thm:neuen}, $G'_0$ and $G'_1$ are $\mathcal{C}^k$-equivalent.
		By \cite{dvorak_recognizing_2010} and \cite[Theorem~14]{seppelt_logical_2024}, $G'_0 \cdot K_c$ and $G'_1 \cdot K_c$ are $\mathcal{C}^k$-equivalent.
		By applying the argument yielding \cite[Theorem~14]{seppelt_logical_2024} to \cref{eq:boxplus}, it follows that $G_0$ and $G_1$ are also $\mathcal{C}^k$-equivalent.
		
		We proceed by showing that $\hom(F', G_0) \geq  \hom(F', G_1)$ for every graph $F'$.
		For $i \in \{0,1\}$,
		\begin{align}
			\hom(F', G_i)
			&\overset{\eqref{eq:boxplus}}{=} \sum_{U \subseteq V(F')} \hom(F'[U], G'_i \cdot K_c) \hom(F - U, K_c) \notag \\
			&\overset{\eqref{eq:lexprod}}{=} \sum_{U \subseteq V(F')} \sum_{\mathcal{R} \in \Gamma(F'[U])} \hom(F'[U]/\mathcal{R}, G'_i) \hom(\coprod_{R \in \mathcal{R}} F'[R], K_c) \hom(F' - U, K_c) \label{eq:lem-induced-minor}
		\end{align}
		Recall that $\Gamma(K)$ for a graph $K$ denotes the set of all partitions $\mathcal{R}$ of $V(K)$ such that $K[R]$ is connected for all $R \in \mathcal{R}$.
		
		By \cref{thm:rob3.13},
		$\hom(F', G'_0) \geq \hom(F', G'_1)$ for every graph $F'$.
		Since all coefficients are non-negative,
		it follows that  $\hom(F', G_0) \geq \hom(F', G_1)$ for every graph $F'$.
		
		It remains to show that $\hom(F, G_0) \neq \hom(F, G_1)$.
		To that end note that $\hom(\coprod_{R \in \mathcal{R}} F[R], K_c) > 0$ for all $\mathcal{R}$ and $\hom(F - U, K_c)  > 0$ for all $U$ by construction.
		Furthermore, $\hom(F[U]/\mathcal{R}, G_0) > \hom(F[U]/\mathcal{R}, G_1)$ for $U$ and $\mathcal{R}$ such that $F[U]/\mathcal{R} \cong P$ by \cref{thm:rob3.13}.
		This choice of $U$ and $\mathcal{R}$ exists as $P$ is an induced minor of $F$.
		It follows that $\hom(F, G_0) \neq \hom(F, G_1)$.
	\end{proof}

		
			The following \cref{lem:bipartite3} is well-known.
		\begin{fact}\label{lem:bipartite3}
			For every connected bipartite graph $F$,
			 $\hom(F, K_2) = 2$.
		\end{fact}
		In other words, every connected bipartite graph admits a bipartition which is unique up to flipping the sides. 
		

	In the following, we deviate from our previous convention to consider only bipartite graphs with distinguished bipartition.
	For a bipartite graph $F$ without distinguished bipartition and a bipartition $V(F) = A \uplus B$,
	we write $F_{A, B}$ for the bipartite graph $F$ with distinguished bipartition $A \uplus B$.
	Thus, in the following \cref{lem:bipartite1},
	$\hom(F, G)$ denotes the number of all graph homomorphisms
        from $F$ to $G$ which do not necessarily respect any fixed bipartition while $\hom(F_{A, B}, G_{X, Y})$ denotes the number of homomorphisms $h \colon F \to G$ such that $h(A) \subseteq X$ and $h(B) \subseteq Y$.

	\begin{lemma}\label{lem:bipartite1}
		Let $F$ and $G$ be bipartite graphs
		with bipartitions $V(F) = A \uplus B$ and $V(G) = X \uplus Y$. Suppose that $F$ is connected.
		Then
		$
		\hom(F, G) = \hom(F_{A, B}, G_{X, Y}) + \hom(F_{A, B}, G_{Y, X}).
		$ 
	\end{lemma}
	\begin{proof}
		First observe that $
		\hom(F, G) \geq \hom(F_{A, B}, G_{X, Y}) + \hom(F_{A, B}, G_{Y, X}).
		$ 
		This is because bipartition-preserving homomorphisms are homomorphisms and the sets of homomorphisms counted by $\hom(F_{A, B}, G_{X, Y}) $ and $\hom(F_{A, B}, G_{Y, X})$ are disjoint.
		
		For the converse inequality, observe that for every homomorphism $h \colon F \to G$, it holds $h(A) \cap h(B) = \emptyset$.
		Indeed, if there are $a \in A$ and $b \in B$ such that $h(a) = h(b)$, then the odd-length path connecting $a$ and $b$ in $F$ is mapped to an odd-length cycle in $G$,
		contradicting that $G$ is bipartite.
		
		Let $G'$ denote the subgraph of $G$ containing the edges and vertices in the image of $F$ under $h$.
		As the image of a connected graph,  $G'$ is connected.
		As subgraph of $G$, $G'$ is bipartite.
		Hence, the bipartition of $G'$ is unique up to flipping sides by \cref{lem:bipartite3}.
		
		This bipartition is, on the one hand, given by $(X \cap V(G')) \uplus (Y \cap V(G'))$ and, on the other hand, 
		given by $h(A) \uplus h(B)$.
		It follows that $h(A) \subseteq X$ and $h(B) \subseteq Y$, or $h(B) \subseteq X$ and $h(A) \subseteq Y$, as desired.
	\end{proof}
	

	
	The \emph{bipartite double cover} of a graph $G$ is the graph $G \times K_2$ where  $\times$ denotes the categorical graph product.
	Its vertex set is $V(G) \times \{0,1\}$ and $(v, i)$ and $(w,j)$ are adjacent if, and only if, $vw \in E(G)$ and $i \neq j$.
	
	
	\begin{lemma}\label{lem:bipartite2}
		Let $F$ and $G$ be bipartite graphs.
		Let $H$ denote the bipartite double cover of $G$.
		Let $V(F) = A \uplus B$ be the bipartition of $F$ and $V(H) = X \uplus Y \coloneqq (V(G) \times \{0\}) \uplus (V(G) \times \{1\})$ be the canonical bipartition of the bipartite double cover of $G$.
		Then
		$
		\hom(F_{A, B}, H_{X, Y})
		= \hom(F_{A, B}, H_{Y, X}).
		$
	\end{lemma}
	\begin{proof}
		Write $V(H) = V(G) \times \{0,1\}$. 
		The automorphism $\phi$ of $H$ given by $(v, i) \mapsto (v, 1-i)$ for all $v \in V(G)$ and $i \in \{0,1\}$ induces a bijection between the set of homomorphisms $F_{A, B}  \to H_{X, Y}$ and the set of homomorphisms $F_{A, B} \to H_{Y, X}$.
	\end{proof}

	\begin{lemma}\label{lem:ck-double-cover}
		Let $G$ and $H$ be $\mathcal{C}^k$-equivalent simple graphs.
		Then their bipartite double covers are $\mathcal{C}^k$-equivalent as bipartite graphs with fixed bipartitions.
	\end{lemma}
	\begin{proof}


Let $G'$ and $H'$ denote the bipartite double covers of $G, H$, respectively, where the bipartitions are marked with unary relations. 		We show that Duplicator has a winning strategy in the bijective $k$-pebble game played on $G'$ and $H'$, using her winning strategy in the $k$-pebble game on $G$ and $H$.  Consider the two projection maps $\pi_1$ and $\pi_2$ on the vertices of $V(G')$ and $V(H')$.  In particular $\pi_1$ maps $V(G')$ and $V(H')$ to $V(G)$ and $V(H)$ respectively and $\pi_2$ maps $V(G')$ and $V(H')$ to $\{0,1\}$.

We argue that Duplicator can maintain the following invariant in the game on $G'$ and $H'$: for any position $(\vec{a}, \vec{b})$ in the game, where $\vec{a}$ and $\vec{b}$ are the tuples of pebbled positions in $V(G')$, the position $(\pi_1(\vec{a}),\pi_1(\vec{b}))$ is winning for Duplicator in the $k$-pebble bijection game on $G$ and $H$, and $\pi_2(\vec{a}) = \pi_2(\vec{b})$.

To see that this invariant can be maintained, suppose $(\vec{a}, \vec{b})$ is a position satisfying the condition and Spoiler chooses to move pebble pair $p$.  Consider the position $(\pi_1(\vec{a}),\pi_1(\vec{b}))$ in the game on $G$ and $H$.  Since this is a winning position for Duplicator by assumption, there is a bijection $f: V(G) \rightarrow V(H)$ which is winning for Duplicator in response to Spoiler moving pebble pair $p$.  If Duplicator plays the bijection $f': V(G') \rightarrow V(H')$ taking $(v,i)$ to $(f(v),i)$, this maintains the invariant.  Indeed, if Spoiler now places the pebbles on $a$ and $f'(a)$, then the position $(\pi_1(\vec{a}[p/a]), \pi_1(\vec{b}[p/b]))$ is winning for Duplicator in the game on $G$ and $H$ since $f$ is a winning move.  And $\pi_2(\vec{a}[p/a]) = \pi_2(\vec{b}[p/b])$ since $f'$ is the identity on the second component.

Finally, it remains to argue that the invariant implies that Duplicator wins the game on $G'$ and $H'$.  For this we only need to argue that the map taking $\vec{a}$ to $\vec{b}$ is necessarily a partial isomorphism if the invariant is maintained.  Suppose $v$ and $v'$ are a pair of vertices in $\vec{a}$ and $w$ and $w'$ the corresponding vertices in $\vec{b}$, then the invariant guarantees that there is an edge between $\pi_1(v)$ and $\pi_1(v')$ in $G$ if, and only if, there is an edge between $\pi_1(w)$ and $\pi_1(w')$ in $H$ (since $(\pi_1(\vec{a}),\pi_1(\vec{b}))$ is winning for Duplicator) and $\pi_2(v) \neq \pi_2(v')$ if, and only if, $\pi_2(w) \neq \pi_2(w')$ since $\pi_2(\vec{a}) = \pi_2(\vec{b})$.  Together these imply that $(v,v') \in E(G)$ if, and only if, $(w,w') \in E(H)$.  Moreover, the condition $\pi_2(\vec{a}) = \pi_2(\vec{b})$ also guarantees that the bipartition is always respected.

			
	\end{proof}	
	
	
	
	
	
	
	
	\begin{proof}[Proof of \cref{thm:dichotomy-chromatic}]
		\Cref{thm:hom-circuit-main} yields that \ref{it:hom1} implies \ref{it:hom4}.
		That \ref{it:hom4} implies \ref{it:hom5} is immediate.
		That \ref{it:hom5} implies \ref{it:hom3} follows from \cref{thm:counting-width}.
		By \cref{def:counting-width},
		it holds that \ref{it:hom3} implies \ref{it:hom2}.
		Thus, it remains to show that \ref{it:hom2} implies \ref{it:hom1}.
		
		To that end, we argue by contraposition. Assuming $\tw((F_{n,m})_{n,m \in \bbN})$ is unbounded, we show that for every  sufficiently large $k \in \mathbb{N}$, the counting width of $\hom(F_{n,m},-)$ on  $(n,m)$-vertex simple graphs exceeds $k$ for some $n,m\in \mathbb{N}$.
		That is, there exist $(n,m)$-vertex simple bipartite graphs $G$ and $H$ which are $\mathcal{C}^k$-equivalent and such that $\hom(F_{n,m}, G) \neq \hom(F_{n,m}, H)$.
		Since $G$ and $H$ are simple graphs,
		we may suppose without loss of generality that $F_{n,m}$ is simple.
		This is because for every multigraph $K$ and every
                simple graph $G$, we have $\hom(K, G) = \hom(K', G)$ where $K'$ is the simple graph obtained from $K$ by forgetting edge multiplicities.
		 
				
		By assumption, there exist sufficiently large $n, m \in \mathbb{N}$ such that $F_{n,m}$ satisfies $\tw(F_{n,m}) \geq k$.
		Let $F$ denote a connected component of $F_{n,m}$ such that $\tw(F) \geq k$.

		
		By the Grid Minor Theorem \cite[(2.1)]{robertson_graph_1986},
		there exists a grid minor $Q$ of $F$ whose size is $f(k)$ for some unbounded function $f \colon \mathbb{N} \to \mathbb{N}$.
		By omitting edge deletions,
		the minor $Q$ yields an induced minor $R$ of $F$ such that $R$ is a supergraph of $Q$ with the same vertex set.
		The graph $R$ is connected and has treewidth at least $f(k)$.
		By taking the subgraph of $R$ induced by a principal subgrid,
		one may obtain a graph $P$ which is connected and satisfies $\gamma(P) \leq \frac{\min\{n,m\}}{2} -1$.
		The treewidth of $P$ is $f'(k)$ for some unbounded function $f' \colon \mathbb{N} \to \mathbb{N}$.
		
		By applying \cref{lem:minor-chromatic} to $F$ and $P$,
		there exist simple graphs $G_0$ and $G_1$ which are $\mathcal{C}^{f'(k)}$-equivalent and such that $\hom(F, G_0) \neq \hom(F, G_1)$.
		The number of vertices in $G_0$ and $G_1$ is at most $\frac{\min\{n,m\}}{2}$.
		
		
		Finally, we construct bipartite graphs from $G_0$ and $G_1$ which satisfy the same properties.
		To that end, let $G'_0$ and $G'_1$ denote the bipartite double covers of $G_0$ and $G_1$, respectively.
		The number of vertices of $G'_0$ and $G'_1$ is at most $\min\{n,m\}$.
		By \cref{eq:product} and the arguments in \cite{seppelt_logical_2024}, 
		$G'_0$ and $G'_1$ are $\mathcal{C}^{f'(k)}$-equivalent.
		
		Write $F_{n,m} = F^1 + \dots + F^r$ as the disjoint union of its connected components.
		Recall \cref{lem:bipartite3}.
		It holds that
		\[
			\hom(F_{n,m}, G'_0)
			\overset{\eqref{eq:coproduct}}{=} \prod_{j=1}^r \hom(F^j, G'_0)
			\overset{\eqref{eq:product}}{=} 2^r\prod_{j=1}^r \hom(F^j, G_0)
			\geq 2^r \prod_{j=1}^r \hom(F^j, G_1)
			= \hom(F_{n,m}, G'_1).
		\]
		The inequality holds by the final assertion of \cref{lem:minor-chromatic}.
		Note that one of the $F^j$ is the graph $F$ to which \cref{lem:minor-chromatic} was applied.
		Since the $F^j$ are bipartite and $G_0$ and $G_1$ both contain at least one edge, $\hom(F^j,G_i) > 0$ for all $j \in [r]$ and $i \in \{0,1\}$ and hence
		\begin{equation}\label{eq:in-eq-doubly-cover}
			\hom(F_{n,m}, G'_0) > \hom(F_{n,m}, G'_1).
		\end{equation}
		
		It remains to show that \cref{eq:in-eq-doubly-cover} also holds when counting homomorphisms respecting a fixed bipartition.
		To that end, write $V(G'_i) = X_i \uplus Y_i \coloneqq (V(G_i) \times \{0\}) \uplus  (V(G_i) \times \{1\})$ for the canonical bipartition of $G'_i$ for $i \in \{0,1\}$.
		By \cref{lem:bipartite1,lem:bipartite2}, it is for every $j \in [r]$ and $i \in \{0,1\}$,
		\[
			\hom(F^j, G'_i) = 2 \hom(F^j_{A, B}, (G'_i)_{X_i, Y_i}). 
		\]
		Hence,
		\begin{align*}
			\hom(F_{n,m}, G'_i)
			\overset{\eqref{eq:coproduct}}{=} \prod_{j =1}^r \hom(F^j, G'_i)
			= 2^r \prod_{j=1}^r  \hom(F^j_{A, B}, (G'_i)_{X_i, Y_i})
			= 2^{r} \hom((F_{n,m})_{A, B}, (G'_i)_{X_i, Y_i}).
		\end{align*}
		Finally, \cref{eq:in-eq-doubly-cover} implies that $\hom((F_{n,m})_{A, B}, (G'_0)_{X_0, Y_0}) \neq \hom((F_{n,m})_{A, B}, (G'_1)_{X_1, Y_1})$.\\
		By \cref{lem:ck-double-cover},
		$(G'_0)_{X_0, Y_0}$ and $(G'_1)_{X_1, Y_1}$ are $\mathcal{C}^{f'(k)}$-equivalent.
	\end{proof}
	
	The strategy employed for proving \cref{thm:dichotomy-chromatic} falls short of yielding a dichotomy for arbitrary linear combinations of homomorphism polynomials since it cannot deal with cancellations between homomorphism counts.
	It is essential that all homomorphism counts appear with non-negative coefficients in \cref{eq:lem-induced-minor}.
	This is what allows to amplify the contributions from homomorphism counts of high-treewidth induced minors.
	Such an argument does not carry through in the general case. 
	Also, as we argued in \cref{ex:simple-interpolation},
	it does not suffice to consider homomorphism counts into simple bipartite graphs.
	
	
	\section{Towards a Dichotomy for Subgraph Polynomials}
	
	
	\Cref{thm:main1} characterises the polynomials admitting symmetric circuits of polynomial orbit size in terms of their representation as linear combinations of homomorphism polynomials.
	By \cref{lem:gnm-sym}, all symmetric polynomials also admit an alternative presentation as linear combinations of \emph{subgraph polynomials}.
	The results from this section suggest that in this subgraph representation, $\mathfrak{T}^k_{n,m}$ can be characterised via another natural graph parameter, namely a variant of the \emph{vertex cover number} that is closed under graph complementation. 
	We establish this characterisation at least in the case of subgraph polynomials for single pattern graphs of sublinear size.
		
	As a first step, we give a sufficient criterion for a subgraph polynomial to be in $\mathfrak{T}^k_{n,m}$.
	This criterion involves the \emph{hereditary treewidth} $\hdtw(F)$ of a bipartite multigraph $F$ defined as the maximum of the treewidth $\tw(F')$ of all bipartite multigraphs $F'$ which admit a vertex- and edge-surjective bipartition-preserving homomorphism $F \to F'$.
	As already observed in \cite{curticapean_homomorphisms_2017}, these graphs $F'$ are precisely those that appear when writing the function $\sub(F, -)$ as a linear combination of functions $\hom(F', -)$ via Möbius inversion, cf.\ \cref{thm:sub-hom}.
	
	\begin{theorem}\label{thm:sub-hom-poly}
		Let $F$ be a bipartite multigraph and $n,m \in \mathbb{N}$.
		Then $\sub_{F, n,m} \in \mathfrak{T}^{k+1}_{n,m}$ for  $k \coloneqq \hdtw(F)$.
	\end{theorem}
	\begin{proof}
		By definition of the hereditary treewidth and \cref{thm:sub-hom}.
	\end{proof}

	The expression for $\sub_{F, n,m}$ derived in \cref{thm:sub-hom-poly} does not depend on $n,m$.
	For this reason, \cref{thm:sub-hom-poly} fails to characterise membership of subgraph polynomials in $\mathfrak{T}^k_{n,m}$,
	as illustrated by the following example which is provided by the machinery developed in \cref{sec:ops}.
	\begin{example} \label{ex:knm}
		For all $n,m \in \mathbb{N}$,
		$\sub_{K_{n,m}, n,m} \in \mathfrak{T}^2_{n,m}$.
	\end{example}
	\begin{proof}
		Since $\sub_{K_{n,m}, n,m} = \prod_{v\in [n]}
                \prod_{w \in [m]} x_{vw}$, this polynomial is in $\mathfrak{T}^2_{n,m}$ by \cref{ex:edge,thm:product-lincomb}.
	\end{proof}
	
	\Cref{ex:knm} shows that $\mathfrak{T}^k_{n,m}$ contains subgraph polynomials of dense graphs.
	Hence, a characterisation of $\mathfrak{T}^k_{n,m}$ in terms of subgraph polynomials cannot involve a monotone graph parameter such as (hereditary) treewidth.
	Towards such a characterisation, we make the following observation:
	
	\begin{theorem}\label{thm:knn}
		For a function $f \colon \mathbb{N} \to \mathbb{N}$ such that $f(n) \leq n$ for all $n \in \mathbb{N}$,
		the following are equivalent:
		\begin{enumerate}
			\item there exists a constant $k \in \mathbb{N}$ such that $\min\{f(n), n- f(n)\} \leq k$ for all $n \in \mathbb{N}$,\label{it:knn1}
			\item the counting width of $\sub_{K_{f(n), f(n)}, n,n}$ on $(n,n)$-vertex simple graphs is bounded,\label{it:knn2}
			\item the counting width of $\sub_{K_{f(n), f(n)}, n,n}$ on $(n,n)$-vertex edge-weighted graphs is bounded,\label{it:knn3}
			\item the $\sub_{K_{f(n), f(n)}, n,n}$ admit $\Sym_n \times \Sym_n$-symmetric circuits of orbit size polynomial in $n$,\label{it:knn4}
			\item there exists a constant $k \in \mathbb{N}$ such that $\sub_{K_{f(n), f(n)}, n,n} \in \mathfrak{T}^k_{n,n}$ for all $n \in \mathbb{N}$,\label{it:knn5}
			\item the polynomials $\sub_{K_{f(n), f(n)}, n,n}$ admit $\Sym_n \times \Sym_n$-symmetric circuits of size polynomial in $n$.\label{it:knn6}
		\end{enumerate}
	\end{theorem}
	\begin{proof}
		First consider the implication \ref{it:knn1} $\Rightarrow$ \ref{it:knn6}:
		Observe that
		\begin{align*}
			\sub_{K_{k,k}, n,n} &= \sum_{\substack{A \subseteq [n] \\ |A| = k}}\sum_{\substack{B \subseteq [n] \\ |B| = k}} \prod_{a \in A} \prod_{b \in B} x_{ab}, \\
			\sub_{K_{n-k,n-k}, n,n} &= \sum_{\substack{A \subseteq [n] \\ |A| = k}}\sum_{\substack{B \subseteq [n] \\ |B| = k}} \prod_{a \in [n] \setminus A} \prod_{b \in [n] \setminus B} x_{ab}.
		\end{align*}
		These formulas represent $\Sym_n \times \Sym_n$-symmetric circuits of size $O(n^{2k})$.
		
		The implications \ref{it:knn6} $\Rightarrow$ \ref{it:knn4} $\Leftrightarrow$ \ref{it:knn5} follow from \cref{thm:main1}.
		The implications \ref{it:knn4} $\Rightarrow$ \ref{it:knn3} $\Rightarrow$ \ref{it:knn2} follow from \cref{thm:counting-width,def:counting-width}.
		The remaining implication \ref{it:knn2} $\Rightarrow$ \ref{it:knn1} is proved by contraposition.
		
		Let $k \geq 2$ be arbitrary.
		By assumption, there exists $n \in \mathbb{N}$ large enough such that
		\[
			\min\{ f(n), n - f(n) \} \geq  k2^{k-1} - k.
		\]
		In particular,  $f(n) \geq k$.
		Let $F \coloneqq K_{f(n), f(n)}$ and write $P \subseteq V(F)$ for a set of vertices such that $F[P] \cong K_{k,k}$.
		Blow up the vertices in $P$ to \textsmaller{CFI} gadgets and call the resulting bipartite graphs $G_0^P$ and $G_1^P$.
		Note that both graphs have $k 2^{k-1} + f(n) - k \leq n$ vertices on each side.
		It follows from \cref{thm:neuen-bipartite} that
		$G_0^P$ and $G_1^P$ are $\mathcal{C}^k$-equivalent as bipartite graphs with fixed bipartitions.
		
		Write $\rho \colon G_i^P \to F$ for the projection map, cf.\ \cref{sec:cfi}.
		We count embeddings $\emb(F, G_i^P)$.
		For a homomorphism $h \colon F \to F$, 
		write $\emb_h(F, G_i^P)$ for the number of embeddings $e \colon F \to G_i^P$ such that $\rho \circ e = h$.
		Consider the following claims:
		
		\begin{claim}\label{cl:knn1}
			If $h \colon F\to F$ is not surjective onto $P$,
			then $\emb_h(F, G_0^P) = \emb_h(F, G_1^P)$.
		\end{claim}
		\begin{claimproof}
			For $u \in V(P)$, write $G_u^P$ for the \textsmaller{CFI} graph where the vertex $u$ carries the odd weight.
			By e.g.\ \cite[Lemma~11]{neuen_homomorphism-distinguishing_2024}, for every $u, v \in P$, there exists an isomorphism $\phi \colon G_u^P \to G_v^P$ such that $\rho \circ \phi = \rho$.
			Hence, $\emb_h(F, G_1^P) = \emb_h(F, G_u^P)$ for every vertex $u \in P$.
			The graphs obtained from $G_0^P$ and $G_u^P$ by removing the gadgets corresponding to $u$ are isomorphic.
			Hence,
			$\emb_h(F, G_0^P) = \emb_h(F, G_u^P) = \emb_h(F, G_1^P)$ for every $u \in P$ such that $u \not\in h(V(F))$.
		\end{claimproof}
		
		\begin{claim}\label{cl:knn2}
			If $h \colon F \to F$ is surjective onto $P$
			and non-injective,
			then $\emb_h(F, G^P_0) = 0 = \emb_h(F, G^P_1)$.
		\end{claim}
		\begin{claimproof}
			By assumption, there exist distinct $a, b \in V(F)$ such that $h(a) = h(b) \eqqcolon v$.
			If $v \not\in P$,
			then $\emb_h(F, G_0^P) = 0 = \emb_h(F, G_1^P)$ since any map $e \colon F \to G_i^P$ such that $\rho \circ e = h$ is non-injective and thus not an embedding.
			Hence, $v \in P$.

			Let $e \colon F \to G_i^P$ be an embedding such that $\rho \circ e = h$.
			Since $h$ is surjective onto $P$,
			both $e(a)$ and $e(b)$ have a shared neighbour in every \textsmaller{CFI} gadget on the opposing side of the bipartition.
			This implies that $e(a) = e(b)$ by \cref{def:cfi},
			contradicting that $e$ is injective. 
		\end{claimproof}
		
		By \cref{cl:knn1,cl:knn2},
		\begin{align*}
			\emb(F, G_0^P) -
			\emb(F, G_1^P)
			& = \sum_{h \colon F \to F} \emb_h(F, G_0^P) - \sum_{h \colon F \to F} \emb_h(F, G_1^P) \\
			&= \sum_{\substack{h \colon F \to F \\ \text{surjective onto } P \\ \text{injective}}} \emb_h(F, G_0^P)
			- \sum_{\substack{h \colon F \to F \\ \text{surjective onto } P \\ \text{injective}}} \emb_h(F, G_1^P)
		\end{align*}
		Here, all $h$ are bipartition-respecting. For $i \in \{0,1\}$, 
		\[
			\sum_{\substack{h \colon F \to F \\ \text{surjective onto } P \\ \text{injective}}} \emb_h(F, G_i^P)
			= f(n)^{\underline{k}} \cdot f(n)^{\underline{k}} \cdot \emb(K_{k,k}, (K_{k,k})_i) \cdot
							 \emb(K_{f(n)-k,f(n)-k}, K_{f(n)-k, f(n)-k})
		\]
		where $n^{\underline{k}} \coloneqq n(n-1) \cdots (n-k+1)$ denotes the falling factorial and $(K_{k,k})_i$ the \textsmaller{CFI} graphs of $K_{k,k}$.
		By \cref{cor:sub-cfi},
		$\emb(F, G_0^P) \neq \emb(F, G_1^P)$.
		Thus, the counting width of $\sub_{K_{f(n), f(n)}, n,n}$ on $(n,n)$-vertex simple graphs is at least $k$.
		As $k$ was chosen arbitrarily, the desired implication follows.
	\end{proof}
	
	\Cref{thm:knn} shows that $\sub_{K_{k,k}, n,n}$ admits small symmetric circuits if, and only if, $k$ is small or $n-k$ is small.
	We generalise the backward direction of this argument for all pattern graphs by introducing a new graph parameter which captures non-uniformity and relaxes hereditary treewidth.
	To that end, we first recall the \emph{vertex cover number} $\vc(F)$ of graph $F$ which is defined as the minimum size of a set $C \subseteq V(F)$ such that every edge in $F$ is incident to a vertex in $C$.
	Vertex cover number, hereditary treewidth, and matching number are functionally equivalent, cf.\ \cref{sec:hdtw} and \cite{curticapean_homomorphisms_2017}.
	\begin{lemma}\label{lem:vc-mn-hdtw}
		For every graph $F$,
		\( \frac12 \hdtw(F) \leq \mn(F) \leq \vc(F) \leq 2\mn(F) \leq (\hdtw(F) +2)^2. \)
	\end{lemma}

	We relax the graph parameters above as follows:
	
	\begin{definition}
		\label{def:bcc}
		Let $n,m \in \mathbb{N}$.
		Let $F$ be a simple $(n,m)$-vertex bipartite graph with bipartition $A \uplus B$.
		Define the simple $(n,m)$-vertex bipartite graph $\overline{F}$ via $V(\overline{F}) \coloneqq V(F)$ and $E(\overline{F}) \coloneqq (A \times B) \setminus E(F)$.
		The \emph{$(n,m)$-biclique cover number of $F$} is
		\[
			\cc_{n,m}(F) \coloneqq \min\{\vc(F), \vc(\overline{F})\}.
		\]
		For $n' \geq n$ and $m' \geq m$, define $\cc_{n',m'}(F) \coloneqq \cc_{n',m'}(F')$ where $F'$ is obtained from $F$ by adding $n'-n$ isolated vertices on the left side and $m'-m$ isolated vertices on the right side.
		If $n' < n$ or $m' < m$, let $\cc_{n',m'}(F) \coloneqq 0$.
	\end{definition}
	
	For an $(n,m)$-vertex graph $F$, we have $\cc_{n,m}(F) = k$ if, and only if, there exists a set $C \subseteq V(F)$ of size at most $k$ such that $F - C$ is an independent set or a biclique.
	Equipped with this definition,
	we first show that a bound on the biclique cover number of a pattern yields polynomial size symmetric circuits for their subgraph polynomials.
	
	\begin{theorem}\label{thm:sub-small-circuit}
		For every family  $(F_{n,m})_{n,m \in \mathbb{N}}$ of simple bipartite graphs,
		if $\cc_{n,m}(F_{n,m})$ is bounded,
		then the $\sub_{F_{n,m}, n,m}$ admit  $\Sym_n \times \Sym_m$-symmetric circuits of size polynomial in $n+m$.
	\end{theorem}
	\begin{proof}
		Fix $F \coloneqq F_{n,m}$ for some $n,m \in \mathbb{N}$ and write $A \uplus B$ for the fixed bipartition of $F$.
		We assume first that $F$ has no isolated vertices and that removing the biclique cover results in an independent set rather than a biclique.
		Let $K = K_A \uplus K_B$ be a biclique cover of $F$, such that $|K| \leq k$ and $K_A \subseteq A$, $K_B \subseteq B$. We perform the following circuit construction for every $\iota\colon K \hookrightarrow [n] \uplus [m]$. 
		Given~$\iota$, let $p_{K,\iota} \coloneqq \prod_{uv \in E(F[K])} x_{\iota(u)\iota(v)}$. 
		This polynomial is represented by a constant-size circuit which is symmetric under $\StabP(\iota(K))$, the pointwise stabiliser of the set $\iota(K)$ in $\Sym_n \times \Sym_m$.
		Now for every type $S \subseteq K_A$ or $S \subseteq K_B$, introduce a fresh variable $t_S$. If $S \subseteq K_A$, we call $S$ an $A$-type, otherwise a $B$-type. Let
		\[
		q_{\iota} \coloneqq p_{K,\iota} \cdot \prod_{j \in [n] \uplus [m] \setminus \iota(K)} \left( \sum_{S \subseteq K \text{ an A-type} } t_S \cdot \prod_{i \in \iota(S)} x_{ij} + \sum_{S \subseteq K \text{ a B-type} }  t_S \cdot \prod_{i \in \iota(S)} x_{ji}\right)
		\]
		The obvious circuit representation of this has size $O((n+m) \cdot 2^k \cdot k)$ (the number of different types is at most $2^k$), and the circuit is also $\StabP(\iota(K))$-symmetric.
		Now we view $q_{\iota}$ as a polynomial in the variables $\{t_S \mid S \text{ a type} \}$ and with coefficients in $\bbQ[\Xx_{n,m}]$.
		For a type $S$, let $\#(S)$ denote the number of vertices in $V(F) \setminus K$ whose $E(F)$-neighbourhood in $K$ is precisely $S$.
		
		The coefficient of the monomial $\prod_{S \text{ a type}} t^{\#(S)}_S$ in $q_{\iota}$ is the part of $\sub_{F,n,m}$ that sums over all injections $V(F) \hookrightarrow [n] \uplus [m]$ which extend~$\iota$. More formally, this coefficient is
		\[
		\sub_{F,n,m;\iota} = \sum_{\substack{\iota' \colon V(F) \hookrightarrow [n] \uplus [m] \\ \iota'|_K = \iota}} \prod_{vw \in E(F)} x_{\iota'(v)\iota'(w)} .
		\]
		Given a symmetric circuit for $q_{\iota}$, we can compute $\sub_{F,n,m;\iota}$ by interpolation. 
		The polynomial $q_\iota$ has at most $2^k$ variables.
		In each of these variables, its maximal degree is $n+m-k$.
		Thus, by \cref{cor:interpolationTrickCircuits}, $\sub_{F,n,m;\iota}$ can be computed with a $\StabP(\iota(K))$-symmetric circuit of size $O((n+m)^{2^k+1} \cdot 2^k \cdot k)$.
		
		Let $C_{\iota}$ denote the circuit that realises this. The polynomial $\sub_{F,n,m}$ is then computed by the circuit $C$ that simply sums up all $C_{\iota}$, for all injections $\iota: K \hookrightarrow [n] \uplus [m]$. The size of $C$ is $O((n+m)^{2^k+k+1} \cdot 2^k \cdot k)$. It remains to argue that $C$ is symmetric. 
		For this, we firstly observe that for any two $\iota_1, \iota_2 \colon K \hookrightarrow [n] \uplus [m]$, the polynomials $q_{\iota_1}$ and $q_{\iota_2}$ are symmetric to each other via any $(\pi, \sigma) \in \Sym_n \times \Sym_m$ that maps $\iota_1$ to $\iota_2$. Since we use the same points $a_1,...,a_{n+m-k} \in \bbQ$ for the interpolation in each subcircuit $C_{\iota}$, and the coefficients in the linear combination given by \cref{lem:multivariate-polynomial-interpolation} are the same, $C_{\iota_1}$ and $C_{\iota_2}$ are also symmetric to each other. This finishes the case where removing the biclique cover results in an independent set. 
		
		In the other case, the graph after removing the biclique cover is a biclique.
		Then we perform the same construction as above, with the difference that now, $p_{K,\iota} = \prod_{uv \in E(F[K])} x_{\iota(u)\iota(v)} \cdot \prod_{i \in [n] \setminus \iota(K_A), j\in [m] \setminus \iota(K_B)} x_{ij}$.
		Finally, if isolated vertices are present in $F$, then we just multiply our circuit with the appropriate constant factor to obtain the subgraph count polynomial.
	\end{proof}	
	
	We conjecture that the converse of \cref{thm:sub-small-circuit} holds.
	That is, the parameter $\cc_{n,m}$ is the right parameter for measuring the symmetric circuit complexity of subgraph polynomials.
	\begin{conjecture}\label{conj:sub}
		For every family $(F_{n,n})_{n,m \in \mathbb{N}}$ of simple bipartite graphs,
		the following are equivalent:
		\begin{enumerate}
			\item $\cc_{n,m}(F_{n,m})$ is bounded,\label{it:subconj1}
			\item the counting width of $\sub_{F_{n,m},n,m}$ on $(n,m)$-vertex edge-weighted graphs is bounded,\label{it:subconj2}
			\item the $\sub_{F_{n,m}, n, m}$ admit $\Sym_n \times \Sym_m$-symmetric circuits of orbit size polynomial in $n+m$,\label{it:subconj3}
			\item the $\sub_{F_{n,m}, n, m}$ admit $\Sym_n \times \Sym_m$-symmetric circuits of size polynomial in $n+m$.\label{it:subconj4}
		\end{enumerate}
	\end{conjecture}
	
	The implications \ref{it:subconj1} $\Rightarrow$ \ref{it:subconj4} $\Rightarrow$ \ref{it:subconj3} $\Rightarrow$ \ref{it:subconj2} follow from \cref{def:counting-width,thm:counting-width,thm:sub-small-circuit}.
	In the remainder of this section,
	we give evidence for the remaining implication.
	
	\subsection{Invariance of Counting Width of Subgraph Polynomials under Complements}
	
	\Cref{conj:sub} predicts that, for every simple $(n,m)$-vertex bipartite graph $F$, $\sub_{F,n,m}$ has bounded counting width if, and only if, $\sub_{\overline{F},n,m}$ has bounded counting width.
	We prove this consequence.
	
	
	\begin{theorem}\label{thm:complements}
		Let $n,m \in \mathbb{N}$.
		For every $(n,m)$-vertex simple bipartite graph $F$,
		the polynomials $\sub_{F, n,m}$ and $\sub_{\overline{F}, n,m}$ have the same counting width on $(n,m)$-vertex edge-weighted graphs.
	\end{theorem}
	\begin{proof}
		The proof is by constructing, given $(n,m)$-vertex bipartite edge-weighted graphs $G$ and $H$ such that $G \equiv_{\mathcal{C}^k} H$ and $\sub_{F, n,m}(G) \neq \sub_{F, n,m}(H)$,
		two $(n,m)$-vertex bipartite edge-weighted graphs $G'$ and $H'$ such that $G' \equiv_{\mathcal{C}^k} H'$ and $\sub_{\overline{F}, n,m}(G') \neq \sub_{\overline{F}, n,m}(H')$.
		
		To that end, we turn the edge-weights from $\mathbb{Q}$ in $G$ into formal variables.
		Let $y_1, \dots, y_r$ denote formal variables, one for every value appearing as an edge weight in $G$ and $H$.
		Let $G''$ and $H''$ denote the $(n,m)$-vertex bipartite graphs whose edges are annotated by these formal variables according to the corresponding values in $G$ and $H$.
		Write $C_1 \uplus \dots \uplus C_r = E(G)$ for the partition of the edges according to this correspondence.
		Write $V(F) = A \uplus B$ and $V(G) = [n] \uplus [m]$ for the bipartitions.
		Consider the following polynomials in $\mathbb{Q}[y_1, \dots, y_r]$:
		\begin{align*}
		\sub_{F, n,m}(G'')
		&= \frac{1}{|\Aut(F)|} \sum_{\substack{h \colon A \uplus B \to [n] \uplus [m] \\ \text{bijective}}} \prod_{i = 1}^r y_i^{|h(E(F)) \cap C_i|}, \\
		\sub_{\overline{F}, n,m}(G'') 
		&= \frac{1}{|\Aut(F)|} \sum_{\substack{h \colon A \uplus B \to [n] \uplus [m] \\ \text{bijective}}} \prod_{i = 1}^r y_i^{|h(E(\overline{F})) \cap C_i|}
		= \frac{1}{|\Aut(F)|}  \sum_{\substack{h \colon A \uplus B \to [n] \uplus [m] \\ \text{bijective}}} \prod_{i = 1}^r y_i^{|C_i| - |h(E(F)) \cap C_i|}.
		\end{align*}
		The final equality holds since $h(E(\overline{F})) \cap C_i = ([n] \times [m] \setminus h(E(F))) \cap C_i = C_i \setminus (h(E(F)) \cap C_i)$  as $F$ and $G$ have the same number of vertices.
		
		It follows that $\sub_{F, n,m}(G''), \sub_{\overline{F}, n,m}(G'') \in \mathbb{Q}[y_1, \dots, y_r]$ are the same polynomials up to a permutation of coefficients.
		More precisely,
		the coefficient of $y_1^{n_1} \cdots y_r^{n_r}$ in $\sub_{F, n,m}(G'')$ for $n_i \in \{0, \dots, |C_i|\}$, $i \in [r]$, 
		equals the number of bijective maps $h \colon A \uplus B \to [n] \uplus [m]$ such that $|h(E(F)) \cap C_i| = n_i$ for $i \in [r]$.
		This number equals the coefficient of $y_1^{|C_1| - n_1} \cdots y_r^{|C_r| - n_r}$ in $ \sub_{\overline{F}, n,m}(G'')$.
		Hence, as formal polynomials in $\mathbb{Q}[y_1, \dots, y_r]$, 
		\[
			 \sub_{F, n,m}(G'') =  \sub_{F, n,m}(H'') \iff  \sub_{\overline{F}, n,m}(G'') =  \sub_{\overline{F}, n,m}(H'').
		\]
		Finally, the edge-weighted graphs $G$ and $H$ such that $\sub_{F, n,m}(G) \neq \sub_{F, n,m}(H)$
		give assignments to the variables $y_1, \dots, y_r$ such that the polynomials $ \sub_{F, n,m}(G'')$ and $\sub_{F, n,m}(H'')$ evaluate to different values.
		Hence,  $ \sub_{\overline{F}, n,m}(G'') \neq  \sub_{\overline{F}, n,m}(H'')$
		and there exist assignments of rationals to $y_1, \dots, y_r$ such that these polynomials evaluate to different values.
		These assignments correspond to $(n,m)$-vertex edge-weighted graphs $G'$ and $H'$ such that $\sub_{\overline{F}, n,m}(G') \neq \sub_{\overline{F}, n,m}(H')$.
		The graphs $G'$ and $H'$ can be obtained from $G$ and $H$ by replacing the edge weights according to some function.
		Hence, $G'$ and $H'$ are $\mathcal{C}^k$-equivalent if $G$ and $H$ are.
	\end{proof}
	\Cref{thm:complements} allows us to separate counting width on simple graphs and counting width on edge-weighted graphs.
	This is in stark contrast to \cref{thm:dichotomy-chromatic}, which shows that, for single homomorphism polynomials, the simple and edge-weighted counting width coincide.
	\begin{example}\label{ex:matching}
		Let $n \in \mathbb{N}$
		and write $F \coloneqq \overline{nK_{1,1}}$ for the bipartite complement of the matching on $(n,n)$-vertices.
		The counting width of the multilinear polynomial $\sub_{F, n,n}$ on $(n,n)$-vertex simple graphs is $2$ while its counting width on $(n,n)$-vertex edge-weighted graphs is $\Theta(n)$.
	\end{example}
	\begin{proof}
		For the first claim, let $G$ be an arbitrary simple $(n,n)$-vertex bipartite graph with bipartition $V(G) = X \uplus Y$.
		If $G$ contains a subgraph isomorphic to $F$,
		then all its vertices have degree $n$ or $n-1$.
		This property can be defined in $\mathcal{C}^2$.
		Write $X' \subseteq X$ and $Y' \subseteq Y$ for the sets of vertices of degree $n$ on each side of the bipartition.
		By the handshaking lemma,  $|X'| = |Y'| \eqqcolon \ell$.
		It holds that $\sub(F,G) = \ell!$.
		Clearly, the number $\ell$ is definable in $\mathcal{C}^2$.
		Hence, the counting width of $\sub_{F,n,n}$ is $2$.

		The second claim follows from \cref{thm:complements} and \cite[Theorem~15]{dawar_symmetric_2020} which states that the counting width of the permanent, which equals $\sub_{\overline{F}, n,n}$, on simple graphs is $\Theta(n)$.
	\end{proof}

	\subsection{Patterns of Sublinear Size}

	We prove \cref{conj:sub} for families of patterns of sublinear size.

	\begin{theorem}\label{thm:sublinear}
		Let $(F_{n,m})_{n,m \in \mathbb{N}}$ be a family of simple bipartite $(a_{n,m}, b_{n,m})$-vertex graphs such that 
		$a_{n,m} \in o(n)$ for all $m \in \mathbb{N}$ and $b_{n,m} \in o(m)$ for all $n \in \mathbb{N}$.
		The following are equivalent:
		\begin{enumerate}
			\item $\cc_{n,m}(F_{n,m})$ is bounded,\label{it:sublinear1}
			\item $\vc(F_{n,m})$ is bounded,\label{it:sublinear1a}
			\item the counting width of $\sub_{F_{n,m},n,m}$ on $(n,m)$-vertex simple graphs is bounded,\label{it:sublinear2}
			\item the counting width of $\sub_{F_{n,m},n,m}$ on $(n,m)$-vertex edge-weighted graphs is bounded,\label{it:sublinear3}
			\item the $\sub_{F_{n,m}, n, m}$ admit $\Sym_n \times \Sym_m$-symmetric circuits of orbit size polynomial in $n+m$,\label{it:sublinear5}
			\item the $\sub_{F_{n,m}, n, m}$ admit $\Sym_n \times \Sym_m$-symmetric circuits of size polynomial in $n+m$.\label{it:sublinear4}
		\end{enumerate}
	\end{theorem}
	\begin{proof}
		The implications \ref{it:sublinear1} $\Rightarrow$ \ref{it:sublinear4} $\Rightarrow$ \ref{it:sublinear5} $\Rightarrow$ \ref{it:sublinear3} $\Rightarrow$ \ref{it:sublinear2} follow from \cref{thm:sub-small-circuit,thm:counting-width,def:counting-width}.
		By \cref{def:bcc}, the vertex cover number of $F$ is generally an upper bound for $\cc_{n,m}(F)$, so \ref{it:sublinear1a} implies \ref{it:sublinear1}. 
		%The assumptions on $a_{n,m}$ and $b_{n,m}$ imply that the complement of $F_{n,m}$ when padded to be an $(n,m)$-vertex graph contains bicliques of unbounded size.
		The remaining implication \ref{it:sublinear2} $\Rightarrow$ \ref{it:sublinear1a} is proved by contraposition.
		
		By \cref{lem:vc-mn-hdtw}, suppose that the matching number of the $F_{n,m}$ is unbounded.
		For arbitrary $k \in \mathbb{N}$ and sufficiently large $n$ and $m$,
		simple bipartite  $\mathcal{C}^k$-equivalent $(n,m)$-vertex graphs $G$ and $H$ are constructed such that $\sub(F_{n,m}, G) \neq \sub(F_{n,m}, H)$.
	%	Here and throughout this proof, $\sub$ and $\hom$ denote subgraph or homomorphism counts respecting fixed bipartitions.
		
		Let $k \in \mathbb{N}$ be arbitrary.
		Let $n,m \in \mathbb{N}$ large enough such that $k \leq \mn(F_{n,m})$, $k2^{k} \leq \frac{n}{a_{n,m} +1}$, and $k2^{k} \leq \frac{m}{b_{n,m} +1}$.
		Let $F \coloneqq F_{n,m}$ and write $A \uplus B = V(F)$ for its fixed bipartition.
		Let $a \coloneqq |A|$ and $b \coloneqq |B|$.
		
		Let $\ell \in \mathbb{N}$ be maximal such that $(\ell+1)^2 \leq k$.
		Since the matching number of $F$ is at least $k$, 
		there exist partitions $\pi \in \Pi(A)$ and $\sigma \in \Pi(B)$ such that $F/(\pi, \sigma)$ is, up to multiedges, isomorphic to $K_{\ell+1,\ell} \eqqcolon P$.
		The treewidth of $P$ is $\ell \geq \sqrt{k}$.
		By \cref{thm:neuen-bipartite},
		the \textsmaller{CFI} graphs $P_0$ and $P_1$ of $P$ are $\mathcal{C}^{\sqrt{k}}$-equivalent as graphs with fixed bipartitions.
		
		Consider the following claim, which is proved using an interpolation argument due to \cite{curticapean_count_2024}.
		For a bipartite graph $G$ with bipartition $X \uplus Y$ and integers $q, p \geq 1$,
		write $G \cdot (\overline{K_q}, \overline{K_p})$
		for the graph obtained from $G$ by replacing every vertex in $X$ by $q$ disconnected vertices and every vertex in $Y$ by $p$ disconnected vertices.
		These gadgets are connected according to the adjacencies in $G$.
		The bipartition  $X \uplus Y$ of $G$ induces the bipartition $(X \times [q]) \uplus (Y \times [p])$ of $G \cdot (\overline{K_q}, \overline{K_p})$.
		
		\begin{claim}\label{cl:sublinear}
			If $\sub(F, P_0 \cdot  (\overline{K_q}, \overline{K_p})) = \sub(F, P_1 \cdot  (\overline{K_q}, \overline{K_p}))$ for all $1 \leq q \leq a+1$ and $1 \leq p \leq b+1$,
			then, for all $1 \leq r \leq a$ and $1 \leq s \leq b$,
			\[
			\sum_{\substack{\pi \in \Pi(A) \\ |A/\pi| = r}}\sum_{\substack{\sigma \in \Pi(B) \\ |B/\sigma| = s}} |\mu_{\pi} \mu_\sigma| \hom(F/(\pi,\sigma), P_0)
			=
			\sum_{\substack{\pi \in \Pi(A) \\ |A/\pi| = r}}\sum_{\substack{\sigma \in \Pi(B) \\ |B/\sigma| = s}} |\mu_{\pi} \mu_\sigma| \hom(F/(\pi,\sigma), P_1).
			\]
		\end{claim}
		\begin{claimproof}
		By \cref{thm:sub-hom},
		\begin{align*}
			\sub(F, P_i \cdot \overline{K_q}) 
			&= \sum_{\pi \in \Pi(A)} \sum_{\sigma \in \Pi(B)} \mu_{\pi} \mu_{\sigma} \hom(F/(\pi, \sigma), P_i \cdot(\overline{K_q}, \overline{K_p})) \\
			& \overset{\eqref{eq:lexprod}}{=} \sum_{\pi \in \Pi(A)} \sum_{\sigma \in \Pi(B)} \mu_{\pi} \mu_{\sigma} \hom(F/(\pi, \sigma), P_i) q^{|A/\pi|}p^{|B/\sigma|} \\
			&= \sum_{r = 1}^{a}\sum_{s=1}^b q^r p^s \sum_{\substack{\pi \in \Pi(A) \\ |A/\pi| = r}}\sum_{\substack{\sigma \in \Pi(B) \\ |B/\sigma| = s}} \mu_{\pi}\mu_{\sigma} \hom(F/(\pi, \sigma), P_i).
		\end{align*}
		By invertibility of the Vandermonde matrices $(q^r)_{1 \leq q \leq a+1, 0 \leq r \leq a}$ and $(p^s)_{1 \leq p \leq b+1, 0 \leq s \leq b}$, 
		it follows that 
		 \[
		\sum_{\substack{\pi \in \Pi(A) \\ |A/\pi| = r}}\sum_{\substack{\sigma \in \Pi(B) \\ |B/\sigma| = s}} \mu_{\pi} \mu_\sigma \hom(F/(\pi,\sigma), P_0)
		=
		\sum_{\substack{\pi \in \Pi(A) \\ |A/\pi| = r}}\sum_{\substack{\sigma \in \Pi(B) \\ |B/\sigma| = s}} \mu_{\pi} \mu_\sigma \hom(F/(\pi,\sigma), P_1).
		\]
		for all $1 \leq r \leq a$ and $1 \leq s \leq b$.
		By \cref{eq:frs},
		the sign of $\mu_\pi$ is $(-1)^{|A| - |A/\pi|}$.
		Hence, all coefficients in the identity above have the same sign $(-1)^{a - r} (-1)^{b - s} = (-1)^{a + b - r - s}$.
		The desired equality follows by multiplying with $(-1)^{a + b - r - s}$.
	\end{claimproof}
		
		By \cref{thm:rob3.13-bipartite},
		$\hom(F', P_0) \geq \hom(F', P_1)$ for all graphs $F'$ and $\hom(P, P_0) > \hom(P, P_1)$.
		This contradicts the identity in \cref{cl:sublinear} for $r = \ell+1$ and $s = \ell$.
		
		Thus, by contraposition, 
		there exist integers $1 \leq q \leq a+1$ and $1 \leq p \leq b+1$ 
		such that $\sub(F, P_0 \cdot (\overline{K_q}, \overline{K_q})) \neq \sub(F, P_1 \cdot (\overline{K_q}, \overline{K_q}))$.
		The graphs $P_0 \cdot (\overline{K_q}, \overline{K_q})$ and $P_1 \cdot (\overline{K_q}, \overline{K_q})$ are $\mathcal{C}^{\sqrt{k}-1}$\nobreakdash-equivalent.
		On their left sides, 
		they have at most $(\ell+1)2^{\ell-1} q \leq k2^{k}(a+1) \leq n$ vertices.
		Similarly,
		on their right sides,
		they have at most $\ell2^{\ell} p \leq k2^{k}(b+1) \leq m$ vertices.
		
		Hence, the counting width of $\sub(F_{n,m}, -)$ on $(n,m)$-vertex simple bipartite graphs is at least $\sqrt{k}-1$.
		Since $k$ was chosen arbitrarily, it follows that the counting width of $\sub_{F_{n,m},n,m}$ on $(n,m)$-vertex simple graphs is unbounded.
	\end{proof}
	
	\section{Symmetric complexity of the immanants}
	\label{sec:immanants}
	The permanent and determinant are the extreme intractable and tractable case of \emph{immanant} polynomials. Immanants are families of $\Sym_n$-symmetric polynomials $(p_n)_{n \in \bbN}$ that are typically defined via so-called \emph{irreducible characters} of the symmetric group. An irreducible character $f\colon \Sym_n \to \mathbb{C}$ is a class function, i.e.\ it only depends on the multiset of cycle lengths of the input permutation. Given such an $f$, the corresponding immanant is defined as
	\[
	\imm_{f} = \sum_{\pi \in \Sym_n} f(\pi) \prod_{i \in [n]} x_{i,\pi(i)}
	\]
	We view these as polynomials over the field $\mathbb{C}$. 
	If $f$ is constantly $1$, then $\imm_f$ is the permanent. The determinant is obtained by letting $f = \sgn$.
	
	Note that while the permanent is $\Sym_n \times
        \Sym_n$-symmetric, this is not true for every immanant
        (indeed, not even of the determinant).
	But for every $f$, $\imm_{f}$ is $\Sym_n$-symmetric because for any $\sigma \in \Sym_n$, the monomial $\prod_{i \in [n]} x_{\sigma(i),\sigma(\pi(i))}$ encodes a permutation from the same conjugacy class as $\pi$, so it has the same value under $f$. Therefore, we study the complexity of $\Sym_n$-symmetric algebraic circuits for $\imm_{f}$. 
	The $\Sym_n \times \Sym_m$-symmetric polynomials that we have mostly been working with naturally expressed properties of weighted undirected bipartite graphs; now, $\Sym_n$-symmetric polynomials express properties of weighted \emph{directed} (not necessarily bipartite) graphs (see also \cref{rem:squareSymmetricPolynomials}).
	
	In \cite{curticapean2021full}, Curticapean shows a complexity dichotomy for the immanants: In the tractable case, $\imm_{f}$ is in $\VP$ (i.e.\ admits polynomial size algebraic circuits) and is computable in polynomial time. In the intractable case, the family of polynomials is not in $\VP$, unless $\VFPT = \VW$, and also, it is not computable in polynomial time, unless $\FPT = \sharpW$. The complexity is controlled by a certain parameter of $f$. 
	
	We show that the same parameter of $f$ produces the analogous dichotomy with regards to the complexity of $\Sym_n$-symmetric algebraic circuits for the immanants. The benefit of considering symmetric circuits is that the lower bound is not conditional on any complexity-theoretic assumptions. 
	We now introduce the relevant parameter of $f$. As explained in \cite{curticapean2021full}, the irreducible characters $f$ correspond naturally to partitions of $[n]$. If $\lambda$ is a partition of $[n]$, we denote by $\chi^{\lambda}\colon \Sym_n \to \mathbb{C}$ its corresponding irreducible character, and we also write $\imm_{\lambda}$ for $\imm_{\chi^\lambda}$. Partitions of $[n]$ are denoted as tuples $(k_1,\dots ,k_s)$, where each entry $k_i$ denotes a part of size $k_i$.
	For example, the permanent is $\imm_{(n)}$, and the determinant is $\imm_{(1,\dots ,1)}$. For other partitions, the rule for computing their corresponding irreducible characters is more involved and not needed here (for details, see \cite{curticapean2021full}).
	 If $\lambda$ is a partition of $[n]$, let $b(\lambda) = n-s$, where $s$ denotes the number of parts. 
	
	Formally, \cite{curticapean2021full} considers a family $\Lambda$ of partitions. It is said that $\Lambda$ \emph{supports growth} $g \colon \bbN \to \bbN$ if for every $n \in \bbN$ there is a partition $\lambda^{(n)}$ in $\Lambda$ with $b(\lambda^{(n)}) \geq g(n)$ and size $\Theta(n)$. According to the dichotomy from \cite{curticapean2021full}, $(\imm_{\lambda})_{\lambda \in \Lambda}$ is tractable if there is a constant that bounds $b(\lambda)$ for all $\lambda \in \Lambda$. Otherwise, if $\Lambda$ supports growth $\omega(1)$, then the immanant family is (conditionally) intractable. If $\Lambda$ even supports growth $\omega(n^k)$, then $(\imm_{\lambda})_{\lambda \in \Lambda}$ is $\VNP$-complete.
	
Our analogous result for symmetric circuits is the following.
\begin{theorem}
\label{thm:immanantDichotomy}
Let $\Lambda$ be a family of partitions.
\begin{enumerate}
	\item If there exists a $k \in \bbN$ such that $b(\lambda) \leq k$ for all $\lambda \in \Lambda$, then the $(\imm_{\lambda})_{\lambda \in \Lambda}$ admit $\Sym_n$-symmetric circuits of polynomial size in $n$.
	\item If $\Lambda$ supports growth $g \in \omega(1)$, then the counting width of $(\imm_{\lambda})_{\lambda \in \Lambda}$ on directed $\bbQ$-edge-weighted $n$-vertex graphs is unbounded, where $n$ is the integer that is partitioned by the respective $\lambda$.
	 Thus, all $\Sym_n$-symmetric circuits for the immanant family have super-polynomial size in $n$. 
	 \item If $\Lambda$ supports growth $g \in \omega(n^k)$ for a constant $k$, then the counting width of $(\imm_{\lambda})_{\lambda \in \Lambda}$ on directed $\bbQ$-edge-weighted $n$-vertex graphs is $\Omega(n)$, so all $\Sym_n$-symmetric circuits have size $2^{\Omega(n)}$.
\end{enumerate}	
\end{theorem}	
This theorem subsumes and generalises the results from \cite{dawar_symmetric_2020}. 
	 
	


\subsection{The tractable case}

We show the first part of \cref{thm:immanantDichotomy}.
In \cite{hartmann1985complexity}, Hartmann gives an algorithm for $\imm_{\lambda}$ that runs in polynomial time if $b(\lambda)$ is bounded. This algorithm is essentially a formula that expresses the immanant in terms of the determinant. For the determinant (over fields of characteristic 0), we know that efficient symmetric circuits exist: 
\begin{lemma}[\cite{dawar_symmetric_2020}]
	\label{lem:determinantEfficient}
	The family $(\det_n)_{n \in \bbN}$ admits $\Sym_n$-symmetric algebraic circuits of polynomial size. 
\end{lemma}	

We now present the relevant details of Hartmann's formula and show that using the above result, we can express it with a symmetric circuit. 
Let $m \leq n$ and let $I = (i_1,\dots ,i_m)$ be a tuple of natural numbers. Define a class function $f_I\colon \Sym_n \to \mathbb{C}$ as follows. 
\[
f_I(\pi) \coloneqq \sgn(\pi)\cdot \prod_{\ell \in [m]} \alpha_\ell(\pi)^{i_\ell}, 
\]
where $\alpha_\ell(\pi)$ denotes the multiplicity of the cycle of length $\ell$ in the cycle decomposition of $\pi$.\\ 

Our first subgoal is a circuit for $\imm_{f_I}$. Following \cite{hartmann1985complexity}, we define certain matrices and use their determinants to compute $\imm_{f_I}$.
A cycle in the graph corresponding to the matrix $(x_{ij})_{1 \leq i,j \leq n}$ is a sequence of variables such that each $x_{ij}$ in this sequence is followed by a variable $x_{jk}$, and the right index of the last variable is equal to the left index of the first variable of the sequence. 
Given a fixed tuple $I = (i_1,\dots ,i_m)$ of cycle multiplicities, we consider tuples of cycles in the graph $(x_{ij})_{1 \leq i,j \leq n}$:
Let $(\sigma_1^{(1)}, \sigma_1^{(2)}, \dots , \sigma_1^{(i_1)}, \sigma_2^{(1)}, \dots , \sigma_2^{(i_2)}, \dots , \sigma_m^{(i_m)})$ denote a tuple of (directed) cycles in the graph $(x_{ij})_{1\leq i,j \leq n}$, where the subscript $\ell$ of each $\sigma$ denotes the length of the cycle, and the superscript indexes the cycles of length $\ell$ from $1$ up to $i_\ell$. 

For a given tuple of cycles, we define a matrix $A(\sigma_1^{(1)},\dots ,\sigma_{m}^{(i_m)})$ in variables $(x_{ij})_{1 \leq i,j \leq n}$ and $(t_\ell^{(k)})$, where $\ell$ and $k$ range over the corresponding sub- and superscripts of $\sigma$. For $i,j \in [n]$, let $X_{ij}$ denote the set of index pairs $(k,\ell)$ such that the cycle $\sigma_{\ell}^{(k)}$ contains the variable $x_{ij}$. The matrix $A(\sigma_1^{(1)},\dots ,\sigma_{m}^{(i_m)})$ is defined by letting
\[
a_{ij} \coloneqq x_{ij}\prod_{(k,\ell) \in X_{ij}} t^{(k)}_{\ell}. 
\]
We extend the action of $\Sym_n$ from the variables $x_{ij}$ to the variables $t_\ell^{(k)}$ by just letting $\pi(t_\ell^{(k)}) = t_\ell^{(k)}$.
It can be checked that every $\pi \in \Sym_n$ maps $A \coloneqq A(\sigma_1^{(1)},\dots ,\sigma_{m}^{(i_m)})$ to $A' \coloneqq A(\pi(\sigma_1^{(1)},\dots ,\sigma_{m}^{(i_m)}))$ such that $\pi(a_{ij}) = a'_{\pi(i),\pi(j)}$. 
The following is a $\Sym_n$-invariant polynomial:
\[
S \coloneqq \sum_{(\sigma_1^{(1)},\dots ,\sigma_{m}^{(i_m)}) \in \Cc} \det A(\sigma_1^{(1)},\dots ,\sigma_{m}^{(i_m)}),
\]
where $\Cc$ is the set of all tuples of cycles $(\sigma_1^{(1)},\dots ,\sigma_{m}^{(i_m)})$ (with the respective lengths as given by the subscripts) such that the union of their edges is a partial cycle cover of $(x_{ij})_{1 \leq i,j \leq n}$ (i.e.\ a disjoint union of cycles). 
Note that $\Cc$ is $\Sym_n$-invariant. 
By \cref{lem:determinantEfficient}, the polynomial $S$ is computable by a $\Sym_n$-symmetric circuit $C_S$ of size $|\Cc| \cdot \text{poly}(n)$: Each $\det A(\sigma_1^{(1)},\dots ,\sigma_{m}^{(i_m)})$ in the sum can be represented by a poly-size $\Sym_n$-symmetric circuit $C(\sigma_1^{(1)},\dots ,\sigma_{m}^{(i_m)})$ that is just like the circuit for $\det_n$ where the inputs are modified to be the products $a_{ij}$ instead of single variables $x_{ij}$. If $\pi \in \Sym_n$ does not fix the tuple of cycles $(\sigma_1^{(1)},\dots ,\sigma_{m}^{(i_m)})$, then the circuit automorphism of $C(\sigma_1^{(1)},\dots ,\sigma_{m}^{(i_m)})$ that $\pi$ extends to turns $C(\sigma_1^{(1)},\dots ,\sigma_{m}^{(i_m)})$ into $C(\pi(\sigma_1^{(1)},\dots ,\sigma_{m}^{(i_m)}))$. 
Thus, because $\Cc$ is $\Sym_n$-invariant, $C_S$ is $\Sym_n$-symmetric.\\
\begin{claim}\label{imm:claim1} If we view $S$ as a polynomial in the $t_\ell^{(k)}$-variables, then the coefficient of the monomial $\prod (t_\ell^{(k)})^\ell$ in $S$ (where the product ranges over the entire tuple of variables $(t_1^{(1)},\dots ,t_1^{(i_1)},\dots ,t_m^{(i_m)})$) is exactly $\imm_{f_I}$.
\end{claim}
\begin{claimproof} Let $M$ be a monomial in $\det A(\sigma_1^{(1)},\dots ,\sigma_{m}^{(i_m)})$. Then its submonomial consisting of $x_{ij}$-variables is a monomial in $\det_n$, so it describes a permutation $\pi \in \Sym_n$. Its submonomial in $t_\ell^{(k)}$-variables is equal to $\prod (t_\ell^{(k)})^\ell$ if and only if every cycle in $(\sigma_1^{(1)},\dots ,\sigma_{m}^{(i_m)})$ is a cycle in the cycle decomposition of $\pi$ (note that $(\sigma_1^{(1)},\dots ,\sigma_{m}^{(i_m)})$ may contain the same cycle multiple times, and then this is also true). 
Thus, the coefficient of $\prod (t_\ell^{(k)})^\ell$ in $S$ is of the form
\[
\sum_{\pi \in P}  \sgn(\pi) \cdot \beta_{\pi} \cdot \prod_{i \in [n]} x_{i\pi(i)},  
\]
where $P \subseteq \Sym_n$ is the set of all permutations whose cycle decomposition contains at least one cycle of length $\ell$, for each $\ell \in [m]$. The coefficient $\beta_\pi$ is the number of tuples 
$(\sigma_1^{(1)},\dots ,\sigma_{m}^{(i_m)}) \in \Cc$ such that each cycle $\sigma_\ell^{(k)}$ appears in the cycle decomposition of $\pi$. Since duplicate cycles can appear in tuples in $\Cc$, we have that $\sgn(\pi) \cdot \beta_\pi = \sgn(\pi) \cdot \prod_{\ell \in [m]} \alpha_\ell(\pi)^{i_\ell} = f_I(\pi)$ because for every cycle length $\ell$, we have $i_\ell$ many slots in each tuple in $\Cc$, each of which can be filled with one of the $\alpha_\ell(\pi)$ many cycles of $\pi$. This proves the claim.
\end{claimproof}

This means that it remains to compute the coefficient of the monomial $\prod (t_\ell^{(k)})^\ell$ in $S$, which can be done by the interpolation method.\\

%One could do this via interpolation but this would be too inefficient. Instead, in \cite{hartmann1985complexity}, this is achieved by taking a series of partial derivatives:
%\textbf{Claim 2:} The coefficient of $\prod (t_\ell^{(k)})^\ell$ in $S$ is equal to
%\[
%\prod \frac{1}\frac{(\ell!)^{i_\ell}}\cdot \frac{\delta}{\delta t_1^{(1)}}\dots \Big(\frac{\delta}{\delta t_m^{(i_m)}}\Big^m S
%\]
%\textit{Proof of claim.} The monomial $\prod (t_\ell^{(k)})^\ell$ in $S$ is maximal in the following sense: There is no monomial in $S$ containing a variable $(t_\ell^{(k)})^{\ell'}$, for an $\ell' > \ell$. This would only be possible if there were a monomial in $\det_n$ that contains $\ell'$ many variables which are in the cycle $\sigma_\ell^{(k)}$ that $t_\ell^{(k)}$ stands for. But the length of $\sigma_\ell^{(k)}$ is $\ell$, so at most $\ell$ variables in any monomial in $\det_n$ can be on this cycle. 
%For this reason, every monomial in $S$ that does not contain $\prod (t_\ell^{(k)})^\ell$ as a submonomial will vanish after taking all these derivatives. The only monomials that remain in the end are the ones in $S$ that do contain $\prod (t_\ell^{(k)})^\ell$, with $\prod (t_\ell^{(k)})^\ell$ being removed. The factor $\prod \frac{1}\frac{(\ell!)^{i_\ell}}$ just corrects for the fact that taking the derivatives introduces the factorials of the exponents of the $t$-variables as coefficients.\\
%\end{comment}
\begin{claim}\label{imm:claim2} Let $d$ be the maximum degree of a $t_\ell^{(k)}$-variable in $S$, and let $t$ denote the number of $t_\ell^{(k)}$-variables. Then $d^t \leq m^{\sum_{\ell \in [m]} i_\ell}$.
\end{claim}
%When $S$ is viewed as a polynomial in the $t_\ell^{(k)}$-variables, it has at most 
%$\prod_{\ell \in [m]} (\ell+1)^{i_\ell}$ many distinct monomials.\\
\begin{claimproof}
We know $t = \sum_{\ell \in [m]} i_\ell$. It remains to show that $d \leq m$.
Let $M$ be a monomial in $S$. Then it is a monomial that appears in some $\det A(\sigma_1^{(1)},\dots ,\sigma_{m}^{(i_m)})$.
This means that the exponent of a variable $t_\ell^{(k)}$ in $M$ can be at most $\ell \leq m$ because $t_\ell^{(k)}$ corresponds to the cycle $\sigma_\ell^{(k)}$ of length $\ell$ and can only appear in a monomial in $\det A(\sigma_1^{(1)},\dots ,\sigma_{m}^{(i_m)})$ together with a variable on the cycle $\sigma_\ell^{(k)}$.
\end{claimproof}
\begin{claim}\label{imm:claim3}
\(
|\Cc| \leq \prod_{\ell \in [m]} \left(\frac{n!}{(n-\ell)!} \right)^{i_\ell}.
\)
\end{claim}
\begin{claimproof} The quantity $\frac{n!}{(n-\ell)!}$ is just the number of different ordered length-$\ell$ cycles in the $n$-vertex graph $(x_{ij})_{1 \leq i,j \leq n}$.
\end{claimproof}


Using \cref{imm:claim1,cor:interpolationTrickCircuits}, we can obtain a symmetric circuit $C_I$ for $\imm_{f_I}$ from the circuit $C_S$. With \cref{imm:claim2,imm:claim3} it follows that 
\[
\norm{C_I} \leq m^{\sum_{\ell \in [m]} i_\ell} \cdot \norm{C_S} \leq  m^{\sum_{\ell \in [m]} i_\ell} \cdot \prod_{\ell \in [m]} \left(\frac{n!}{(n-\ell)!} \right)^{i_\ell} \cdot \operatorname{poly}(n).
\]

This finishes the first subgoal of the proof. Now we use $C_I$ to obtain a circuit for $\imm_{\lambda}$.
The proof of Theorem 1 in \cite{hartmann1985complexity} continues as follows. Let $r$ denote the size of the largest part of $\lambda$, and let $m \coloneqq n-s$, where $s$ is the number of parts of $\lambda$.
Let $\Ii$ be the set of all tuples $(k_1,\dots ,k_m)$ such that $m-r+1 \leq \sum_{\ell \in [m]} \ell k_\ell \leq m$.

 
Let $\eta$ be the function that takes as input a tuple $(k_1,\dots ,k_m) \in \Ii$ and is defined as
\begin{align*}
\eta(k_1,\dots ,k_m) \coloneqq \Big\{ (c,i_1,\dots ,i_m) \mid &c \in \mathbb{C}, i_1,\dots ,i_m \in \bbN,\\
 &\text{ the monomial } c\cdot \prod_{\ell \in [m]} \alpha_\ell^{i_\ell}  \text{ appears in }  \binom{\alpha_1}{k_1} \dots   \binom{\alpha_m}{k_m} \Big\}.
\end{align*}
In the above definition,  $\binom{\alpha_1}{k_1} \dots  \binom{\alpha_m}{k_m}$ is viewed as a polynomial in formal variables $\alpha_1,\dots ,\alpha_m$.
The following identity is from \cite{hartmann1985complexity}:
\begin{align}
\imm_{\lambda} = \sum_{(k_1,\dots ,k_m) \in \Ii} \chi(k_1,\dots ,k_m) \cdot \Big( \sum_{(c,i_1,\dots ,i_m) \in \eta(k_1,\dots ,k_m)}  c \cdot \imm_{f_{(i_1,\dots ,i_m)}} \Big),\label{eq:star1}
\end{align}
where $\chi(k_1,\dots ,k_m)$ is some $\mathbb{C}$-valued function that we do not need to elaborate on further. In a circuit, we can simply hardwire its respective value.
We have 
\[
| \Ii | \leq \sum_{m-r+1\leq i \leq m} p(i),
\]
where $p(i)$ denotes the number of partitions of $[i]$.
As argued in \cite{hartmann1985complexity}, the polynomial $\binom{\alpha_1}{k_1}\cdot \dots  \cdot \binom{\alpha_m}{k_m}$ has at most $\prod_{\ell \in [m]} (k_\ell+1) \leq \binom{2m}{m}$ many monomials and the degree of each $\alpha_\ell$ is at most $k_\ell$.
Therefore, $|\eta(k_1,\dots ,k_m)| \leq \binom{2m}{m}$, and for every tuple $(c,i_1,\dots ,i_m) \in \eta(k_1,\dots ,k_m)$, it holds $i_\ell \leq k_\ell$ for every $\ell \in [m]$. Using this, we can upper-bound one of the products in the size bound of $C_I$ as follows:
\[
m^{\sum_{\ell \in [m]} i_\ell} \leq m^{\sum_{\ell \in [m]} k_\ell} \leq m^m.
\]	

In total, we can use \cref{eq:star1} and plug in the symmetric circuits $C_I$ for the respective $\imm_{f_{(i_1,\dots ,i_m)}}$. Thus we obtain a symmetric circuit $C_\lambda$ for $\imm_\lambda$ with
\[
\norm{C_\lambda} \in O\Big( \Big(\sum_{m-r+1\leq i \leq m} p(i)\Big) \cdot \binom{2m}{m} \cdot  m^m \cdot \prod_{\ell \in [m]} \Big(\frac{n!}{(n-\ell)!} \Big)^{i_\ell} \cdot \text{poly}(n) \Big).
\]
From \cite{hartmann1985complexity}, we get the following estimation:
\begin{align*}
 \Big(\sum_{m-r+1\leq i \leq m} p(i)\Big) \cdot \binom{2m}{m} \cdot m^m \cdot \prod_{\ell \in [m]} \Big(\frac{n!}{(n-\ell)!} \Big)^{i_\ell} &\leq (m+1)e^{3m^{1/2}}e^{2m} \cdot m^m \cdot n^m\\
  &\leq n^{6(n-s)+4} \cdot (n-s)^{(n-s)}.
\end{align*}
The last step holds for $n > 3$ and uses the fact that $m+1 \leq n$ and $m=n-s$. Since we are in the first case of \cref{thm:immanantDichotomy}, we assume that $m = n-s$ is bounded by a constant $k$. In particular, the factor $(n-s)^{(n-s)}$ and the exponent of $n$ become constant.
So $\norm{C_\lambda}$ is in $O( n^{6k+4})$, multiplied with the symmetric complexity of the determinant, which is also polynomial in $n$.

	
	

\subsection{The hard case}
Let us first focus on the second part of \cref{thm:immanantDichotomy}.
Fix a family $\Lambda$ of partitions such that $b(\Lambda)$ is unbounded, i.e.\ it supports growth $g \in \omega(1)$.
We exploit the detailed analysis in \cite{curticapean2021full, curticapean2021complexitydichotomyimmanantfamilies} to show that the counting width of $(\imm_{\lambda})_{\lambda \in \Lambda}$ on edge-weighted directed graphs is unbounded. The hardness proof in \cite{curticapean2021full} reduces the problem of counting $k$-matchings, for growing $k$, to the immanant. 
Consider the following family of graph parameters, for a given function $h\colon \bbN \to \bbN$. Let $n \in \bbN$ and let $G$ be an undirected graph with $n$ vertices. Then $\Match_{n,h}(G)$ denotes the number of matchings in $G$ consisting of exactly $h(n)$ many edges.
\begin{lemma}
	\label{lem:kMatchingUnboundedCountingWidth}
	For any super-constant function $h(n)$ with $h(n) \leq n/2$, the graph parameter $\Match_{n,h}$ has unbounded counting width on undirected $(n,n)$-vertex bipartite graphs with fixed bipartition.
	Moreover, the number of \emph{perfect} matchings is a graph parameter with counting width $\Omega(n)$ in undirected $(n,n)$-vertex bipartite graphs with fixed bipartition.
\end{lemma}	
\begin{proof}
	Let $k \in \bbN$ be arbitrary.
	For a large enough $n \in \bbN$, we can do the following:
	Let $H'_0, H'_1$ be large enough \textsmaller{CFI} graphs from \cite[Theorem 19]{dawar_symmetric_2020} in which the size of a perfect matching is precisely $h(n)$, and such that $H'_0$ and $H'_1$ are $\mathcal{C}^k$-equivalent. Such graphs can be found because $h$ is super-constant. The graphs are bipartite (and we can assume the bipartition to be marked with unary relation symbols without breaking $\mathcal{C}^k$-equivalence, cf.\ \cref{thm:neuen-bipartite}).
	Let $H_0, H_1$ be defined by padding $H_0', H_1'$, respectively, with isolated vertices so that they both have exactly $n$ vertices. 
	Then still, $H_0 \equiv_{\Cc^k} H_1$. Moreover, the number of size-$h(n)$ matchings in $H_0$ and $H_1$, respectively, is equal to the number of perfect matchings in $H_0'$ and $H_1'$, respectively, and these numbers are different by \cite{dawar_symmetric_2020}.
	Therefore, $\Match_{n,h}(H_0) \neq \Match_{n,h}(H_1)$. Since $k$ was arbitrary, this shows that the counting width of $\Match_{n,h}$ is unbounded.
	For the second part of the lemma, we do not even have to modify $H'_0, H'_1$ that we obtain from \cite[Theorem 19]{dawar_symmetric_2020}. They are $n$-vertex bipartite graphs with different numbers of perfect matchings. It also follows from \cite{dawar_symmetric_2020} that $H'_0 \equiv_{\mathcal{C}^{o(n)}} H'_1$.
\end{proof}	
Now one has to go through the proof of Theorem 2 in \cite{curticapean2021complexitydichotomyimmanantfamilies} in some detail: Let $h(n) = \sqrt{g(n)}/24$.
Fix an arbitrary $k \in \bbN$.
By \cref{lem:kMatchingUnboundedCountingWidth}, there exists an $n \in \bbN$ and $\mathcal{C}^k$-equivalent undirected bipartite $(n,n)$-vertex graphs $H_0, H_1$ such that 
$\Match_{n,h}(H_0) \neq \Match_{n,h}(H_1)$. Due to the growth condition on $\Lambda$, there is a $\lambda \in \Lambda$ with $b(\lambda) \geq 24h(n) = \sqrt{g(n)}$ and such that $\lambda$ satisfies certain further properties that we do not need to introduce here. The properties of $\lambda$ fall into two cases, and in each of them, it is shown in \cite{curticapean2021complexitydichotomyimmanantfamilies} that $\Match_{n,h}$ is reducible to $\imm_{\lambda}$. 
In the following, the term \emph{digon} refers to a directed $2$-cycle.

\paragraph*{First case}
In the first case, the second part of Lemma 22 from \cite{curticapean2021full} is used. It states that for any given undirected bipartite $(n,n)$-vertex graph $H$, there exists a directed edge-weighted graph $G(H)$ with edge weights $\{0,1,x\}$ ($x$ being a formal variable), such that for the $\lambda$ at hand we have:
\begin{equation}
\Match_{n,h}(H) = c \cdot [x^{2h(n)}] \imm_{\lambda}(G(H)) \label{eq:star2}
\end{equation}
Here, $[x^{2h(n)}] \imm_{\lambda}(G(H))$ denotes the coefficient of the monomial $x^{2h(n)}$ in $\imm_{\lambda}(G(H))$, and $c \in \bbQ$ is a constant. 
The precise construction of $G(H)$ can be found in Lemma 20 in \cite{curticapean2021complexitydichotomyimmanantfamilies}:
Let $L \uplus R = V(H)$ be the bipartition of $H$. To obtain $G(H)$, all edges in $H$ are directed from $L$ to $R$, and all edges $ R \times L$ are added to $G$. A set $T$ of fresh vertices is added (whose size depends only on $\lambda$), together with all edges $R \times T$ and $T \times L$. For every $v \in V(H)$, a so-called switch vertex $s_v$ is added, with a self-loop of weight $x$, and a digon between $v$ and $s_v$. Finally, a number (that depends only on $\lambda$) of padding digons and padding vertices with self-loops is added. 
For $q \in \bbQ$, denote by $G_q$ the graph constructed as just described, but with edge weight $q$ instead of $x$.
\begin{lemma}
	\label{lem:raduReductionPreservesCkEquivalence}
	For every $q \in \bbQ$ and bipartite graphs $H_0$, $H_1$ with fixed bipartitions,
	if $H_0 \equiv_{\mathcal{C}^k} H_1$, then $G_q(H_0) \equiv_{\mathcal{C}^{k}} G_q(H_1)$.
\end{lemma}	
\begin{proof}
	It can be checked that after every construction step of $G_q$, the Duplicator still has a winning strategy in the bijective $k$-pebble game played on the two graphs.  
	The first step is directing the edges from $L$ to $R$, and adding all edges $ R \times L$. This new edge relation is definable by a quantifier-free FO-formula because $H_0$ and $H_1$ have their bipartitions marked with unary relations.
	It is clear that $\mathcal{C}^k$-equivalence is preserved through such quantifier-free FO-definitions because any formula speaking about the new graphs can be translated into a formula speaking about the original graphs by replacing the symbol for the edge relation with its defining formula. This does not increase the number of variables, so no $k$-variable formula can distinguish the new pair of graphs.
	 
	The second step is to add a set $T$ of fresh isolated vertices. We can assume them to be coloured with a new unary relation symbol $T$. It is clear that if the two graphs up to this point are $\mathcal{C}^k$-equivalent, then adding the same number of $T$-coloured isolated vertices to each graph does not change this fact because Duplicator still has a winning strategy. In the next step, the edges $R \times T$ and $T \times L$ are inserted. These edges are again definable with a quantifier-free formula, using the colours, so again, $\mathcal{C}^k$-equivalence is preserved.
	The next step is to add a switch vertex $s_v$ for each $v \in L \cup R$, to connect it with $v$ via a digon, and to add a self-loop with weight $q$ (represented as a new binary relation) to every $s_v$.
	Here, one can easily define a winning strategy for Duplicator in the bijective $k$-pebble game on the new pair of graphs, using the Duplicator strategy on the pair before inserting the switch vertices: Any position in the game on the modified structures corresponds to a position in the game on the original structures: If a switch vertex $s_v$ is pebbled, this corresponds to $v$ being pebbled in the original graph. Then if $f$ is Duplicator's chosen bijection in the game on the original graphs, this can be lifted to a bijection on the modified graphs by just mapping $s_{v}$ to $s_{f(v)}$. It is easy to see that Duplicator wins with this strategy. 
	Finally, the same number of isolated digons and vertices with self-loops is added to both graphs, which clearly does not change the fact that Duplicator has a winning strategy.
\end{proof}	

The constant $c$ in \cref{eq:star2} only depends on $\lambda$ and $h(n)$, but not on the graph $H$ (see Lemma 26 in \cite{curticapean2021complexitydichotomyimmanantfamilies}). Thus, putting \cref{eq:star2}  together with the equation $\Match_{n,h}(H_0) \neq \Match_{n,h}(H_1)$, we have:
\[
 [x^{2h(n)}] \imm_{\lambda}(G(H_0)) \neq [x^{2h(n)}] \imm_{\lambda}(G(H_1)).
\]
It follows that there must exist a $q \in \bbQ$ such that $\imm_{\lambda}(G_q(H_0)) \neq \imm_{\lambda}(G_q(H_1))$, for if we had $\imm_{\lambda}(G_q(H_0)) = \imm_{\lambda}(G_q(H_1))$ for every $q \in \bbQ$, then $\imm_{\lambda}(G(H_0))$ and $\imm_{\lambda}(G(H_1))$ would be the same polynomial in $\bbQ[x]$, contradicting the statement above. 
By \cref{lem:raduReductionPreservesCkEquivalence}, we have $G_q(H_0) \equiv_{\mathcal{C}^k} G_q(H_1)$, which proves that $\imm_{\lambda}$ is not $\mathcal{C}^k$-invariant on $\bbQ$-edge-weighted graphs. 

\paragraph*{Second case}
In the second case, the properties of $\lambda$ are different, and the second part of Lemma 28 in \cite{curticapean2021full} has to be applied. 
This describes another graph construction, let us call it $G'(H)$, such that
\begin{equation}
\Match_{n,h}(H) = c' \cdot \imm_{\lambda}(G'(H)) \label{eq:2stars}
\end{equation}
The graph $G'(H)$ is directed and has edge weights $\{0,1,-1\}$ if $H$ is unweighted.
The construction of $G'(H)$ is the following (Lemma 28 in
\cite{curticapean2021complexitydichotomyimmanantfamilies}): Replace
every edge $uv$ in $H$ with an edge gadget consisting of $u, v$, two
fresh vertices, and five digons, as depicted below (see also page 6 of \cite{curticapean2021complexitydichotomyimmanantfamilies}).
\begin{figure}[h!]
\centering
\begin{tikzpicture}[->, thick, shorten <= 2pt, shorten >= 2pt, every node/.style={circle, draw, fill=black, minimum size=0.3cm}]
	% Nodes
	\node[label=$u$] (A) {};
	\node[above right = 1cm and 3cm of A] (B) {};
	\node[below right = 1cm and 3cm of A] (C) {};
	\node[label=$v$, right = 6cm of A] (D) {};
	% Edges
	\draw (A) edge[bend left = 20] (B); 
	\draw (B) edge[bend left = 20] (A); 
	
	\draw (A) edge[bend left = 20] (C); 
	\draw (C) edge[bend left = 20] (A);
	
	\draw (B) edge[bend left = 20] node[draw=none, fill=none, midway, right]{$-1$} (C); 
	\draw (C) edge[bend left = 20] (B); 
	
	\draw (B) edge[bend left = 20] (D); 
	\draw (D) edge[bend left = 20] (B); 
	
	\draw (C) edge[bend left = 20] (D); 
	\draw (D) edge[bend left = 20] (C); 
\end{tikzpicture}
\caption{The gadget that replaces an edge $uv$}
\end{figure}

This edge gadget is symmetric between $u$ and $v$. Next, for each vertex $v \in V(H)$, add a fresh \enquote{switch vertex} $s_v$ connected to $v$ with a digon. Add another $2h(n)$ many fresh vertices, called \enquote{receptor vertices}, and add an edge between each pair of receptor and switch vertex. Replace each of these new edges with the aforementioned edge gadget. 
Finally, a number of isolated digons and a number of isolated vertices is added. These numbers depend on $\lambda, |V(H)|, |E(H)|$, and $h(n)$.
\begin{lemma}
	\label{lem:raduReductionPreservesCkEquivalence2}
	Let $k \geq 4$.
	If $H_0, H_1$ are undirected simple bipartite graphs with fixed bipartition, and $H_0 \equiv_{\mathcal{C}^k} H_1$, then $G'(H_0) \equiv_{\mathcal{C}^{k/2}} G'(H_1)$.
\end{lemma}	
\begin{proof}
Again, we verify that every construction step of $G'$ preserves $\mathcal{C}^k$-equivalence. 
Most of it works just like in \cref{lem:raduReductionPreservesCkEquivalence}. The only additional construction step here is the replacement of every edge with the gadget depicted above. 
Since these edge gadgets contain ``internal'' vertices, they give Spoiler the possibility to fix an edge using just one pebble. This is why we need to halve the number of pebbles in this construction step to ensure that Duplicator can still win. 
The argument is the following. We now start with $H_0$ and $H_1$ and
consider only the construction step where each edge is replaced with
the gadget. Consider a position in the bijective $k$-pebble game on
$H_0$ and $H_1$. The Duplicator winning strategy gives us a bijection
$f\colon V(H_0) \to V(H_1)$ that respects the $\mathcal{C}^k$-types of
the vertices, with the $\leq k-1$ pebbles on the board as
parameters. This is the bijection that Duplicator plays. If at most
$k-2$ pebbles are currently on the board, then in addition to $f$,
there also exists a bijection $f_2\colon V(H_0)^2 \to V(H_1)^2$ on the
pairs of vertices which respects the parametrised
$\mathcal{C}^k$-types of the pairs. If such a bijection did not exist,
then the tuples of pebbled vertices would have different
$\mathcal{C}^k$-types and Duplicator would not be able to win from
this position. (Note that $f$ and $f_2$ in general do not agree on the vertices, unless $H_0$ and $H_1$ are isomorphic.)   
With this in mind, we can define a Duplicator winning strategy in the $k/2$-pebble game on the modified graphs: A position on the modified graphs translates back into a pebble position on the original graphs as follows: Pebbles on original vertices are kept, and each pebble on an internal vertex of the gadget of an edge $uv$ is replaced by two pebbles, one on $u$ and one on $v$, in the original graph. Before Duplicator gets to specify a bijection in the game on the modified structures, there are at most $k/2-1$ many pebbles on each graph, so the corresponding pebble position in the original graphs has at most $k-2$ many pebbles on each graph. Because Duplicator has a winning strategy in the $k$-pebble game on the original graphs, there exist a $\mathcal{C}^k$-type preserving bijection $f$ on vertices, and a $\mathcal{C}^k$-type preserving bijection $f_2$ on the vertex pairs, as argued before. Because the existence of an edge is part of the $\mathcal{C}^k$-type of a pair, $f_2$ preserves edges. We use $f$ and $f_2$ to define Duplicator's bijection in the game on the modified graphs: The internal vertices of the edge gadget of an edge $e$ are mapped to the corresponding internal vertices of the gadget of the edge $f_2(e)$. The non-gadget vertices are mapped according to $f$. Because $f$ and $f_2$ are type-preserving, after Spoiler's move in the game on the modified graphs, when the position is translated back to the other game, the pebbled tuples again have the same $\mathcal{C}^k$-type, so Duplicator can indeed maintain this invariant and win the game.

The other construction steps can be dealt with as in \cref{lem:raduReductionPreservesCkEquivalence}, noting that the introduction of edges between all pairs of receptor and switch vertices is FO-definable if we assume the set of receptor and switch vertices to be coloured with a distinguished colour, respectively. 
In the final construction step, we just have to note that the number of isolated vertices and digons that are added depend on $\lambda, |V(H)|, |E(H)|$, and $h(n)$, and we have $|V(H_0)| = |V(H_1)|$ and $|E(H_0)| = |E(H_1)|$ because $H_0$ and $H_1$ are $\mathcal{C}^k$-equivalent, and $k \geq 2$.
\end{proof}	
The constant $c' \in \bbQ$ in \cref{eq:2stars} depends on $n=|V(H)|, |E(H)|, h(n)$ and $\lambda$ (see Lemma 28 in \cite{curticapean2021complexitydichotomyimmanantfamilies}), so it is in particular the same constant for $\mathcal{C}^k$-equivalent graphs $H_0, H_1$. 
Thus, as before, \cref{eq:2stars} implies that $\imm_{\lambda}(G'(H_0)) \neq \imm_{\lambda}(G'(H_1))$. The two graphs are $\mathcal{C}^{k/2}$-equivalent by \cref{lem:raduReductionPreservesCkEquivalence2}, so $\imm_{\lambda}$ is not $\mathcal{C}^{k/2}$-invariant on edge-weighted directed graphs. Since $k$ was arbitrary, we find such an example of graphs for every $k$, so the counting width of $\imm_{\lambda}$ is unbounded.

The two cases together show the second part of \cref{thm:immanantDichotomy} because they prove that the family $\imm_{\Lambda}$ is not $\mathcal{C}^k$-invariant for any fixed $k \in \bbN$. 
	
The first part of \cref{thm:counting-width} then implies that $\imm_{\Lambda}$ does not admit polynomial size symmetric circuits.

The third part of \cref{thm:immanantDichotomy} is shown analogously. The proof in \cite{curticapean2021complexitydichotomyimmanantfamilies} is then again divided into two cases, in each of which we get \cref{eq:star2,eq:2stars}, respectively, but where $\Match_{n,h}(H)$ is now replaced with the number of \emph{perfect} matchings. \cref{lem:kMatchingUnboundedCountingWidth} then gives us graphs that differ with respect to the number of perfect matchings and are $\mathcal{C}^{o(n)}$-equivalent. It is easy to check that \cref{lem:raduReductionPreservesCkEquivalence} and \cref{lem:raduReductionPreservesCkEquivalence2} are also true for the graph constructions from \cite{curticapean2021complexitydichotomyimmanantfamilies} that are used here (they use exactly the same gadgets as before).
So in total, if $g \in \Omega(n^k)$, then it follows that for any sublinear function $f(n)$, and every large enough $n$, we can find graphs $G(H_0) \equiv_{\mathcal{C}^{f(n)}} G(H_1)$ with $\imm_\lambda(G(H_0)) \neq \imm_\lambda(G(H_1))$. Hence, the counting width of the immanant is $\Omega(n)$.

The $2^{\Omega(n)}$ size lower bound on symmetric circuits follows from the second part of \cref{thm:counting-width}.

\section{Conclusion}
	

In recent years, a rich theory of symmetric computation has been emerging which has established a tight and surprising relationship between logical definability, circuit complexity and also computation in other models such as linear programming (see~\cite{dawar_csl2020,dawar_icalp2024} for pointers).  A central plank of this is the use of variations of the Cai-F\"urer-Immerman construction to estabish unconditional lower bounds in the context of models of computation with natural symmetries.  The work in~\cite{dawar_symmetric_2020} is a signficant example of this, showing an unconditional exponential separation between the complexity of the determinant and the permanent polynomials in the context of symmetric algebraic circuits.

The present work gives a sweeping generalization of these results in two different directions.  First, considering matrix polynomials symmetric under aribtrary permutations of the rows and columns (i.e.\ the $\Sym_n \times \Sym_m$ action), we give a complete characterisation of the tractable cases in terms of homomorphism polynomials of bounded treewidth, linking this study to other complexity classifications such as~\cite{curticapean_homomorphisms_2017}.  For polynomials on \emph{square} matrices symmetric under the simultaneous application of a permutation to the rows and columns (i.e.\ the $\Sym_n$ action), we generalise the separation of the determinant and permanent by giving a complete classification of the symmetric complexity of all immanants.

The work raises a number of directions for future research.  It suggests that there is a close connection between the counting width of polynomials, which we define, and their symmetric algebraic complexity.  We have not established this in full generality and thus the first important direction is to prove or disprove \cref{conj:sub}.  A step in this direction would be to understand the complexity of linear combinations of homomorphism polynomials where neither of \cref{thm:logarithmic-lincomb} or \cref{thm:dichotomy-chromatic} applies.  That is to say, there is more than one homomorphism polynomial involved and the size of the pattern graphs is not logarithmically bounded.

Our most general results are for matrix polynomials symmetric under $\Sym_n \times \Sym_m$, where we have a complete characterization of tractable families.  The polynomials on square matrices symmetric under the $\Sym_n$ action include many that are not $\Sym_n \times \Sym_n$-symmetric (for instance, the determinant and many other immanants).  While we give a complete classification of the immanant polynomials, there are many other such polynomial families. It seems plausible that a complete classification in terms of homomorphism polynomials from directed graphs might be achievable in this case, and we leave it for future work.

It would be interesting to relate this to other methods for
establishing lower bounds in algebraic complexity and to natural
algorithmic methods in that field.  Indeed, one of the striking
features of the study of symmetric computation models is that a
surprising range of algorithms that are used (for instance in
combinatorial optimization) are indeed symmetric and thus the
unconditional lower bounds show the limitations of standard
techniques.

Finally, this work may provide the means to prove lower bounds for
symmetric circuits for interesting polynomial families beyond the
determinant and the permanent.  One potential application would be in
proof complexity.  There has been considerable recent interest in the
\emph{Ideal Proof System}~\cite{grochow_pitassi}, in which proofs of tautologies take the
form of algebraic circuits.  One could conceivably use the methods
developed here to show that certain families of tautologies with natural
symmetries do not admit succinct symmetric proofs.

	
	
	
	
	
	
	
	\newpage
	\printbibliography
	\newpage
	
	\appendix 
	
	\section{Möbius Inversion over Polynomial Rings}
	
	Let $L$ be a finite poset and $R$ be a ring.
	Define the \emph{Möbius function} $\mu \colon L \times L \to R$ of $L$ recursively via
	\begin{align*}
		\mu(s,s) &= 1 && \text{for all } s \in L, \\
		\mu(s,u) &= - \sum_{s \leq t < u} \mu(s, t) && \text{for all } s < u \text{ in } L.
	\end{align*}
	
	\begin{lemma}[{\cite[Proposition~3.7.2]{stanley_enumerative_1997}}]\label{lem:moebius}
		Let $L$ be a finite poset, $R$ a ring, and $M$ an $R$-module.
		Let $f, g \colon L \to M$. Then
		\[
		g(s) = \sum_{t \geq s} f(t) \text{ for all } s\in L
		\iff
		f(s) = \sum_{t \geq s} g(t) \mu(s,t) \text{ for all } s \in L.
		\]
	\end{lemma}
	\begin{proof}
		Assume the first identity holds. Then
		\begin{align*}
			\sum_{t \geq s} g(t) \mu(s,t) &= \sum_{t \geq s} \mu(s,t) \sum_{u \geq t} f(u) \\
			&= \sum_{u \geq s} f(u) \sum_{u \geq t \geq s} \mu(s, t) \\ 
			&= \sum_{u \geq s} f(u) \delta_{u = s} \\
			&= f(s).
		\end{align*}
		The converse follows analogously.
	\end{proof}

	\Cref{lem:moebius} is applied to the lattice $\Pi(A)$ of partitions of some finite set $A$.
	The Möbius function of the partition lattice is given by 
	the Frucht--Rota--Schützenberger formula \cite[(A.2)]{lovasz_large_2012}.
	For a partition $\pi \in \Pi(A)$,
	\begin{equation}\label{eq:frs}
		\mu_{\pi} \coloneqq (-1)^{|A| - |A/\pi|} \prod_{ P \in A/\pi} (|P|-1)!.
	\end{equation}

	
	\begin{lemma} \label{lem:moebius-polynomial}
		Let $A$ and $I$ be finite sets. 
		For every map $h \colon A \to I$, let $p_h \in M$.
		Then, the following hold:
		\begin{align*}
			\sum_{h \colon A \to I} p_h &= \sum_{\pi \in \Pi(A)} \sum_{h \colon A/\pi \hookrightarrow I} p_{h \circ \pi}, \\
			\sum_{h \colon A \hookrightarrow I} p_h &= \sum_{\pi \in \Pi(A)} \mu_\pi \sum_{h \colon A/\pi \to I} p_{h \circ \pi}.
		\end{align*}
	\end{lemma}
	\begin{proof}
		We first verify the first identity.
		To that end, note that the set of maps $h \colon A \to I$ can be partitioned according to the partition $\pi \in \Pi(A)$ the map $h$ induces.
		In other words,
		\begin{align*}
			\{(\pi, h) \mid \pi \in \Pi(A), h \colon A/\pi \hookrightarrow I \} &\to I^A \\
			(\pi, h) & \mapsto h \circ \pi
		\end{align*}
		is a bijection.
		
		
		To obtain the second identity, we verify the assumptions of \cref{lem:moebius}.
		For a partition $\pi \in \Pi(A)$,
		let
		\[
			f(\pi) \coloneqq \sum_{h \colon A/\pi \hookrightarrow I} p_{h \circ \pi}, \quad \text{and} \quad
			g(\pi) \coloneqq \sum_{h \colon A/\pi \to I} p_{h \circ \pi}.
		\]
		Now, the first identity reads as $g(\bot) = \sum_{\pi \in \Pi(A)} f(\pi)$
		while the second identity reads as $f(\bot) = \sum_{\pi \in \Pi(A)} \mu_\pi f(\pi)$.
		Towards the assumptions of \cref{lem:moebius}, let $\pi \in \Pi(A)$ be arbitrary.
		Then
		\begin{align*}
			g(\pi) &= \sum_{h \colon A/\pi \to I} p_{h \circ \pi} \\
				   &= \sum_{\sigma \in \Pi(A/\pi)} \sum_{h \colon (A/\pi)/\sigma \hookrightarrow I} p_{(h \circ \sigma) \circ \pi}  \\
				   &= \sum_{\substack{\sigma \in \Pi(A) \\ \sigma \geq \pi}} \sum_{h \colon A/\sigma \hookrightarrow I} p_{h \circ \sigma} \\
				   &= \sum_{\substack{\sigma \in \Pi(A) \\ \sigma \geq \pi}} f(\sigma),
		\end{align*}
		as desired.
		\Cref{lem:moebius} yields the second identity.
	\end{proof}
	
	Consider the following corollary.
	Let $F$ be a bipartite graph with bipartition $A \uplus B = V(F)$.
	For $n, m\in \mathbb{N}$, consider the \emph{embedding polynomial}
	\[
	\emb_{F, n,m} \coloneqq \sum_{h \colon A \uplus B \hookrightarrow [n] \uplus [m]} \prod_{vw \in E(F)} x_{h(v)h(w)}.
	\]
	Clearly, $\emb_{F, n,m} = |\Aut(F)| \cdot \sub_{F, n,m}$.
	
	
	\begin{corollary} \label{thm:sub-hom}
		Let $F$ be a bipartite multigraph with bipartition $A \uplus B = V(F)$.
		Let $n,m \in \mathbb{N}$.
		Then
		\[
		\sub_{F, n,m} = \frac{1}{|\Aut(F)|} \sum_{\pi \in \Pi(A)} \mu_{\pi} \sum_{\sigma \in \Pi(B)} \mu_{\sigma} \hom_{F/(\pi, \sigma), n, m}.
		\]
	\end{corollary}
	
	\begin{proof}
		Applying \cref{lem:moebius-polynomial},
		we treat the two parts of the bipartition separately.
	
		For maps $h \colon A \to [n]$ and $h' \colon B \to [m]$, define the polynomial
		\[
			p_{h,h'} \coloneqq \prod_{vw \in E(F)} x_{h(v)h'(w)}.
		\]
		Applying \cref{lem:moebius-polynomial} twice,
		\begin{align*}
			\emb_{F, n, m} 
			&= \sum_{h \colon A \hookrightarrow [n]} \sum_{h' \colon B \hookrightarrow [m]} p_{h,h'} \\
			&= \sum_{\pi \in \Pi(A)} \mu_\pi \sum_{h \colon A/\pi \to [n]} \sum_{h' \colon B \hookrightarrow [m]} p_{h \circ \pi,h'} \\
			&= \sum_{\pi \in \Pi(A)} \mu_\pi \sum_{h \colon A/\pi \to [n]} \sum_{\sigma \in \Pi(B)} \mu_\sigma \sum_{h' \colon B/\sigma \to [m]} p_{h \circ \pi, h' \circ \sigma} \\
			&= \sum_{\pi \in \Pi(A)} \sum_{\sigma \in \Pi(B)} \mu_\pi \mu_\sigma \hom_{F/(\pi, \sigma), n,m}. \qedhere
		\end{align*}
	\end{proof}
	
	\begin{example}
		Consider the $F = P_3$, i.e.\ the path on three vertices where $A$ is a singleton and $B$ is of size two. Then
		\[
		2\sub_{F, n, m} = \sum_{i \in [n]} \sum_{\substack{k, j \in [m] \\ k \neq j}} x_{ij}x_{ik}.
		\]
		There is only one partition in $\Pi(A)$, namely $\bot = \top$ with $\mu_\bot = 1$.
		In $\Pi(B)$, there are two partitions, namely $\bot$ and $\top$ with $\mu_\bot = 1$ and $\mu_\top = -1$.
		It follows that
		\[
		2\sub_{F, n, m} =  \hom_{F/(\bot, \bot), n,m} - \hom_{F/(\bot, \top), n,m}
		= \hom_{P_3, n, m} - \hom_{K_2^2, n, m}
		=  \sum_{i \in [n]} \sum_{k, j \in [m]} x_{ij}x_{ik} - \sum_{i \in [n]} \sum_{k \in [m]} x_{ik}^2
		\]
		where $K_2^2$ denotes the loopless multigraph with two vertices and two edges.
		In particular, for $m = 1$, the polynomial $\sub_{F, n, m}$ is the zero polynomial.
	\end{example}
	
	
	
	\section{Interpolation}
	
	\begin{lemma}[Multivariate Polynomial Interpolation] \label{lem:multivariate-polynomial-interpolation}
		Let $\mathbb{K}$ be a field and $A$ be a $\mathbb{K}$-algebra.
		Let $p \in A[x_1, \dots, x_t]$ be a multivariate polynomial such that the degree in each of the variables is less than~$n$.
		Let $a_1, \dots, a_n \in \mathbb{K}$ be distinct.
		Then the coefficients of $p$ are $\mathbb{K}$-linear combinations of the $p(a_{j_1}, \dots, a_{j_t})$ for $j_1, \dots, j_t \in [n]$. The coefficients of the respective linear combination only depend on $a_1, \dots, a_n$ and on the coefficient of $p$ to be expressed by the linear combination.
	\end{lemma}
	
	Note that the polynomial is evaluated at $n^t$ points in $\mathbb{K}^t$.
	We think of applying the lemma to $\mathbb{K} = \mathbb{Q}$ and $A = \mathbb{Q}[y_1,\dots, y_r]$.
	
	\begin{proof}
		Write $p = \sum_{0 \leq i_1,\dots, i_t < n} \alpha_{i_1\dots i_t} x^{i_1}_1 \cdots x^{i_t}_t$ 
		for $\alpha_{i_1,\dots, i_t} \in A$.
		Then for all $j_1, \dots, j_t \in [n]$
		\[
		p(a_{j_1}, \dots, a_{j_t}) = \sum_{0 \leq i_1,\dots, i_t < n}  a^{i_1}_{j_1} \cdots a^{i_t}_{j_t} \alpha_{i_1\dots i_t}
		\]
		Define the $\{1, \dots, n\} \times \{0, \dots, n-1\}$-matrix $V$ such that $V_{ji} \coloneqq a_{j}^i$.
		By assumption, $V$ is invertible with entries in $\mathbb{K}$.
		Hence, the Kronecker product $W \coloneqq V \otimes \dots \otimes V$ of $t$ copies of $V$  is invertible. 
		The above equation now reads as
		\[
		p(a_{j_1}, \dots, a_{j_t}) = \sum_{0 \leq i_1,\dots, i_t < n}  W_{j_1\dots j_t, i_1\dots i_t} \alpha_{i_1\dots i_t}
		\]
		Hence, for all $0 \leq i_1,\dots, i_t < n$,
		\[
		\alpha_{i_1\dots i_t} = \sum_{j_1, \dots, j_t \in [n]} W^{-1}_{i_1\dots i_t, j_1\dots j_t} p(a_{j_1}, \dots, a_{j_t}).
		\]
		Thus, the coefficients $\alpha_{i_1\dots i_t}$ are linear combinations of the $p(a_{j_1}, \dots, a_{j_t})$ with coefficients in $\mathbb{K}$.
	\end{proof}
	
	
	\begin{corollary}
		\label{cor:interpolationTrickCircuits}
		Let $\bbF$ be a field. 	
		Let $p \in \bbF[x_1, \dots, x_t, y_1,\dots y_r]$. 
		Let $n$ be an upper bound on the degree of any of the variables $x_i$ in $p$.
		Let $q \in \bbF[y_1,\dots y_r]$ be the coefficient of a monomial in the $x_i$-variables in $p$, when $p$ is viewed as a polynomial in $(\bbF[y_1,\dots y_r])[x_1,\dots, x_t]$.
		If $p$ admits a circuit representation of size $s$, then $q$ admits a circuit representation of size at most $n^t \cdot s$.
		If the circuit for $p$ is $\Gamma$-symmetric for a group $\Gamma$ that acts on the $y_i$-variables and fixes the $x_i$-variables pointwise, then so is the the circuit for $q$.
	\end{corollary}
	\begin{proof}
		\cref{lem:multivariate-polynomial-interpolation} tells us that we can express the desired $q$ as a linear combination of $n^t$ many different evaluations of $p(x_1, \dots, x_t)$. Therefore, we just need $n^t$ many copies of the circuit for $p$, where different constants are hardwired to the inputs in each copy.
		If the circuit for $p$ is symmetric under a group $\Gamma$ that acts on the $y$-variables and fixes the $x$-variables pointwise, then substituting the $x$-variables with constants does not change these symmetries, so the resulting circuit are a sum of $\Gamma$-symmetric circuits.
	\end{proof}	
	
	
	
	
	
	
	
	
	
	
	
	\section{\textsmaller{CFI} Graphs}
	\label{sec:cfi}
	In this section, we recall the \textsmaller{CFI} graphs studied in \cite{roberson_oddomorphisms_2022} and some of their properties.
	In the literature, this construction is known as \textsmaller{CFI} graphs without internal vertices.
	Throughout, we highlight properties of bipartite \textsmaller{CFI} graphs.
	
	For a connected simple graph $G$ and $v \in V(G)$, write $E(v) \subseteq E(G)$ for the set of edges incident to~$v$.
	
	\begin{definition} \label{def:cfi}
		For a connected simple graph $G$ and a vector $U \in \mathbb{Z}_2^{V(G)}$, define the \textsmaller{CFI} graph $G_U$ via
		\begin{align*}
			V(G_U) &\coloneqq \left\{(v, T) \ \middle| \ v \in V(G), T \in \mathbb{Z}_2^{E(v)}, \sum_{e \in E(v)} T(e) = U(v) \right\}, \\
			E(G_U) &\coloneqq \left\{(v, T)(u, S) \ \middle|\ vu \in E(G), T(uv) + S(uv) = 0 \right\}.
		\end{align*}
	\end{definition}
	
	We write $\gamma(G) \coloneqq \sum_{v\in V(G)}2^{\deg_G(v) -1}$ for the size of $G_U$.
	Even though we have defined one graph $G_U$ for every vector $U \in \mathbb{Z}_2^{V(G)}$, there are in fact only two such graphs up to isomorphism.
	\begin{lemma}[{\cite[Corollary~3.7]{roberson_oddomorphisms_2022}}] \label{lem:cfi-hom-base-graph}
		For a connected simple graph $G$ and a vector $U \in \mathbb{Z}_2^{V(G)}$, the following are equivalent:
		\begin{enumerate}
			\item $\sum_{v \in V(G)} U(v) = 0$,
			\item $G_0 \cong G_U$,
			\item $\hom(G, G_0) = \hom(G, G_U)$.
		\end{enumerate}
	\end{lemma}
	
	By virtue of \cref{lem:cfi-hom-base-graph}, we define $G_1$ as any of the graph $G_U$ for $\sum_{v\in V(G)} U(v) = 1$.
	The graphs $G_0$ and $G_1$ are the \emph{even} and the \emph{odd \textsmaller{CFI} graphs} of $G$.
	
	In \cite{roberson_oddomorphisms_2022}, a characterisation of the graphs $F$ was given that are such that $\hom(F, G_0) = \hom(F, G_1)$ for some connected graph $G$.
	We recall this characterisation for the purpose of proving \cref{thm:logarithmic-lincomb}.
	
	\begin{definition}[{\cite[Definition~3.9]{roberson_oddomorphisms_2022}}]\label{def:oddomorphism}
		Let $F$ and $G$ be simple graphs and $\phi \colon F \to G$ be a homomorphism.
		A vertex $a \in V(F)$ is \emph{$\phi$-even} / \emph{$\phi$-odd} if $|N_F(a) \cap \phi^{-1}(v)|$ is even / odd for all $v \in N_G(\phi(a))$.
		The map $\phi$ is an \emph{oddomorphism} if 
		\begin{enumerate}
			\item every vertex $a \in V(F)$ is $\phi$-odd or $\phi$-even and
			\item every fibre $\phi^{-1}(v) \subseteq V(F)$ for $v \in V(G)$ contains an odd number of $\phi$-odd vertices.
		\end{enumerate}
		The map $\phi$ is a \emph{weak oddomorphism} if there exists a subgraph $F' \subseteq F$ such that $\phi|_{F'} \colon F' \to G$ is an oddomorphism.
	\end{definition}

	For example, the identity map $G \to G$ is an oddomorphism.
	
	\begin{theorem}[{\cite[Theorem~3.13]{roberson_oddomorphisms_2022}}]\label{thm:rob3.13}
		Let $G$ be a connected simple graph.
		For every simple graph $F$,
		\[
			\hom(F, G_0) \geq \hom(F, G_1)
		\]
		with strict inequality if, and only if, there exists a weak oddomorphism $F \to G$.
	\end{theorem}

	The details of \cref{def:oddomorphism} are irrelevant for this article.
	In the following observation, we state some straightforward properties of weak oddomorphisms.
	\begin{observation} \label{obs:oddo-surjective}
		Let $F$ and $G$ be simple graphs. Every weak oddomorphism $\phi \colon F\to G$ is surjective on vertices and edges.
	\end{observation}
	\begin{proof}
		For every $v \in V(G)$, $\phi^{-1}(v)$ contains an odd number of $\phi$-odd vertices. Hence, $\phi^{-1}(v)$ is non-empty and $\phi$ is surjective on vertices.
		Let $uv \in E(G)$ be an edge and $a \in \phi^{-1}(v)$ be some $\phi$-odd vertex. 
		Then $a$ has an odd number of neighbours in $\phi^{-1}(u)$.
		In particular, there is a neighbour $b \in
                \phi^{-1}(u)$ of $a$ and $\phi(ab) = uv$.
		Hence, $\phi$ is surjective on edges.
	\end{proof}
	
	\begin{theorem}[{\cite[Lemma~12, Corollary~13]{neuen_homomorphism-distinguishing_2024}}]\label{thm:neuen}
		\begin{enumerate}
			\item Let $G$ be a connected simple graph and $k \in \mathbb{N}$.
			If $\tw(G) \geq k$, then $G_0$ and $G_1$ are $\mathcal{C}^k$-equivalent.
			\item For all simple graphs $F$ and $G$ for
                          which there exists a weak oddomorphism $F
                          \to G$, we have $\tw(F) \geq \tw(G)$.
		\end{enumerate}
	\end{theorem}

	\subsection{Bipartite \textsmaller{CFI} Graphs}
	
	In this section, the results stated above are adjusted for bipartite graphs with fixed bipartition.
	Recall \cref{def:cfi}.
	The map $\rho \colon G_U \to G$ given by $(v, T) \mapsto v$ is a homomorphism.
	Hence, if $G$ is bipartite,
	then so is $G_U$.
	If the fixed bipartition of $G$ is $A \uplus B$,
	then we fix the bipartition $\rho^{-1}(A) \uplus \rho^{-1}(B)$ of~$G_U$.
	We write $\gamma_A(G) \coloneqq \sum_{v\in A}2^{\deg_G(v) -1}$ and $\gamma_B(G) \coloneqq \sum_{v\in B}2^{\deg_G(v) -1}$ for the size of left and the right side of the bipartition of $G_U$.
	
	\begin{corollary}\label{thm:rob3.13-bipartite}
		Let $G$ be a connected simple bipartite graph.
		For every simple bipartite graph $F$,
		\[
		\hom(F, G_0) \geq \hom(F, G_1)
		\]
		with strict inequality if, and only if, there exists a bipartition-respecting weak oddomorphism $F \to G$.
	\end{corollary}
	Note that the identity map $G \to G$ is a bipartition-respecting weak oddomorphism. Hence, $\hom(G, G_0) > \hom(G, G_1)$.
	\begin{proof}
		For a homomorphism $\psi \colon F\to G$, write $\hom_\psi(F, G_i)$ for the number of homomorphisms $h \colon F \to G_i$ such that $\rho h = \psi$.
		Note that
		\[
			\hom(F, G_i) = \sum_{\substack{\psi \colon F \to G \\ \psi \text{ respects bipartitions}}} \hom_\psi(F, G_i). 
		\]
		By \cite[Theorem~3.6]{roberson_oddomorphisms_2022},
		$\hom_\psi(F, G_0) \geq \hom_\psi(F, G_1)$ for every homomorphism $\psi \colon F \to G$.
		This implies that $\hom(F, G_0) \geq \hom(F, G_1)$ where $F$, $G_0$, and $G_i$ are regarded as graphs with fixed bipartition.
		
		It holds that $\hom(F, G_0) > \hom(F, G_1)$ if, and only if, there exists a bipartition-respecting homomorphism $\psi \colon F\to G$ such that $\hom_\psi(F, G_0) > \hom_\psi(F, G_1)$.
		By the proof of \cite[Theorem~3.13]{roberson_oddomorphisms_2022}, $\hom_\psi(F, G_0) > \hom_\psi(F, G_1)$ if, and only if, $\psi$ is a weak oddomorphism.
	\end{proof}

	\begin{corollary}\label{cor:sub-cfi}
		For every connected simple bipartite graph $G$,
		$\emb(G, G_0) > \emb(G, G_1)$.
	\end{corollary}
	\begin{proof}
		Write $A \uplus B$ for the fixed bipartition of $G$.
		By \cref{thm:sub-hom}, for $i \in \{0,1\}$,
		\[
		\emb(G, G_i) = \sum_{\pi \in \Pi(A)} \sum_{\sigma \in \Pi(B)} \mu_\pi \mu_\sigma \hom(G/(\pi, \sigma), G_i).
		\]
		Here, $G/(\pi, \sigma)$ can be assumed to be simple by discarding possible multiedges.
		The number of vertices in $G/(\pi, \sigma)$ is $|A/\pi|$ on the left side and $|B/\sigma|$ on the right side.
		By \cref{thm:rob3.13-bipartite,obs:oddo-surjective},
		\begin{align*}
			\emb(G, G_0) - \emb(G, G_1) 
			&= \mu_\bot \sigma_\bot \left( \hom(G/(\bot, \bot), G_0) - \hom(G/(\bot, \bot), G_1) \right) \\
			&= \hom(G, G_0) - \hom(G, G_1)
		\end{align*}
		where $\bot$ denotes the discrete partitions of $A$ and $B$.
	\end{proof}

	\begin{corollary}\label{thm:neuen-bipartite}
		Let $G$ be a connected simple bipartite graph and $k \in \mathbb{N}$.
		If $\tw(G) \geq k$, then $G_0$ and $G_1$ are $\mathcal{C}^k$-equivalent as bipartite graphs with fixed bipartition.
	\end{corollary}
	\begin{proof}
		Note that the Duplicator strategy constructed in \cite[Lemmas~11 and~12]{neuen_homomorphism-distinguishing_2024} respects the gadgets of the \textsmaller{CFI} graphs.
		Thus, this strategy allows Duplicator to win even when the bipartitions are fixed.
	\end{proof}
	
	
	
	
	\section{Hereditary Treewidth, Vertex Cover Number, and Matching Number}
	\label{sec:hdtw}
	In this section, we show that the three graph parameters hereditary treewidth, vertex cover number, and matching number are functionally equivalent.
	That is, we prove \cref{lem:vc-mn-hdtw}.
	
	
	A \emph{matching} of $F$ is a set $M \subseteq E(F)$ such that for all distinct $e_1, e_2 \in M$ it holds that $e_1 \cap e_2 = \emptyset$.
	Define the \emph{matching number} $\mn(F)$ of $F$ as the size of the largest matching in $F$.
	\begin{theorem} \label{thm:mn-hdtw}
		For every graph $F$,
		$\frac12 \hdtw(F) \leq \mn(F) \leq \binom{\hdtw(F) + 2}{2}$.
	\end{theorem}
	
	We first consider the upper bound:
	
	\begin{lemma}
		For every graph $F$,
		$\mn(F) \leq \binom{\hdtw(F)+2}{2}$.
	\end{lemma}
	\begin{proof}
		Let $F$ be a graph.
		By \cite[Fact~3.4]{curticapean_homomorphisms_2017},
		if $F'$ is a graph with at most $\mn( F)$ edges and no isolated vertices then $\tw(F') \leq \hdtw(F)$.
		Let $k \in \mathbb{N}$ be such that $\binom{k+2}{2} \geq \mn(F) \geq \binom{k+1}{2}$.
		Then $F'$ can be taken to be the $(k+1)$-clique, which has treewidth $k$.
		Hence, $\hdtw(F) \geq k$.
		Thus, $\binom{\hdtw(F) +2}{2} \geq \binom{k+2}{2} \geq \mn(F)$.
	\end{proof}
	
	By \cite[220]{curticapean_homomorphisms_2017}, one gets an asymptotic linear upper bound on $\mn(F)$ in terms of $\hdtw(F)$.
	
	\begin{lemma}
		For every graph $F$,
		$\hdtw(F) \leq 2\mn(F)$.
	\end{lemma}
	\begin{proof}
		We first show that $\tw(F) \leq 2\mn(F)$ by constructing a tree decomposition of width $\mn(F)$ from a maximum matching.
		
		Let $M$ be a maximum matching in $F$, i.e.\ a matching of cardinality $\mn(F)$.
		Write $V \coloneqq \bigcup_{e \in M} e$ for the set of vertices incident to matched edges.
		Every edge $e \in V(F)$ is incident to at least one vertex in $V$.
		Let $W \coloneqq V(F) \setminus V$ denote the set of vertices not incident to any matched edge.
		
		Define a tree decomposition $\beta$ over the star graph with centre $x$ and tips indexed by $w \in W$ as follows.
		Let $\beta(x) \coloneqq V$ and $\beta(w) \coloneqq \{w\} \cup V$.
		Since no two vertices in $W$ are adjacent, this is a valid tree decomposition of width $2 \mn(F)$.
		
		Now consider a surjective homomorphism $h \colon F \to F'$.
		Let $V' \coloneqq h(V) \subseteq V(F')$ and write $W' \coloneqq h(W) \setminus V \subseteq V(F')$.
		Define a tree decomposition $\beta'$ for $F'$ over the star graph with centre $x'$ and tips indexed by $w' \in W'$.
		Let $\beta'(x') \coloneqq V'$ and $\beta(w') \coloneqq \{w'\} \cup V'$.
		Every bag is of size at most $2\mn(F) + 1$ and the bags cover the entire graph $F'$ as $h$ is an edge- and vertex-surjective homomorphism.
		It is a valid decomposition because all $w'_1 \neq w'_2$ in $W'$ are non-adjacent since their preimages under $h$ are non-adjacent.
	\end{proof}
	
	%	\Cref{lem:tw-mn} can be strengthened as follows:
	%	
	%	\begin{lemma}
		%		For every graph $F$,
		%		$\hdtw(F) \leq 2\mn(F)$.
		%	\end{lemma}
	%	\begin{proof}
		%		Let 
		%		We show that the width-$2\mn(F)$ tree decomposition $(T, \beta)$ of $F$ constructed in \cref{lem:tw-mn} induces a tree decomposition of the same width of $F'$.
		%		To that end, define $\gamma \colon V(T) \to 2^{V(F')}$ via $\gamma(t) \coloneqq \{h(v) \mid v \in \beta(t)\}$.
		%		Since $h$ is surjective on vertices on edges, the decomposition $(T, \gamma)$ covers all vertices on edges.
		%		However, it may fall short of being a valid tree decomposition since vertices $w \neq w'$  in $W$, as defined in the proof of \cref{lem:tw-mn},
		%		may be mapped to the same vertex under $h$.
		%		In this case, the two corresponding bags have the same image and can be identified. \todo{Why do they have the same image? Couldn't the neighbours of $w$ and $w'$ in $F$ still be mapped to different vertices by $h$?}
		%		Hence, we may obtain a tree decomposition of same width.
		%	\end{proof}
	
	
	A \emph{vertex cover} is a set $C \subseteq V(F)$ such that every edge $e \in E(F)$ is incident to $C$, i.e.\ $e \cap C \neq \emptyset$.
	Write $\vc(F)$ for the size of the smallest vertex cover in $F$.
	
	\begin{fact} \label{fact:mn-vc}
		For every graph $F$, $\mn(F) \leq \vc(F) \leq 2\mn(F)$.
	\end{fact}
	
	\Cref{thm:mn-hdtw,fact:mn-vc} yield \cref{lem:vc-mn-hdtw}.
\end{document}
