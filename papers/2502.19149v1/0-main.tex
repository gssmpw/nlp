\pdfoutput=1

\documentclass[11pt]{article}

\usepackage[final]{acl}

\usepackage{times}
\usepackage{latexsym}

\usepackage[T1]{fontenc}

\usepackage[utf8]{inputenc}

\usepackage{microtype}

\usepackage{inconsolata}
\usepackage{pifont}
\usepackage{graphicx}

%\usepackage{minted}
\usepackage{listings}
\usepackage{markdown}
\definecolor{mygreen}{rgb}{0,0.6,0}
\definecolor{mymauve}{rgb}{0.58,0,0.82}
\lstset{
backgroundcolor=\color{white},
basicstyle=\footnotesize\ttfamily,
columns=fullflexible,
breaklines=true,
postbreak=\mbox{$\hookrightarrow$\space},
columns=fullflexible,
captionpos=b,
tabsize=4,
commentstyle=\color{mygreen},
escapeinside={\%*}{*)},
keywordstyle=\color{blue},
stringstyle=\color{mymauve}\ttfamily,
frame=single,
rulesepcolor=\color{red!20!green!20!blue!20},
keywordstyle=\color{blue!70},
commentstyle=\color{red!50!green!50!blue!50},
numbers=none, 
rulesepcolor=\color{red!20!green!20!blue!20},
xleftmargin=2em,
xrightmargin=2em
}

\usepackage{xspace}
\usepackage{xcolor}
\usepackage{booktabs}
\usepackage{nicematrix}
\usepackage{cleveref}
\usepackage{adjustbox}
%\usepackage[most]{tcolorbox}
\usepackage{listings-rust}
\usepackage{caption}
\usepackage{subcaption}


\newcommand{\tocheck}[1]{\textcolor{red}{#1}}

\newcommand{\benchname}{\textsc{PseudoEval}\xspace}
\newcommand{\livecb}{LiveCodeBench\xspace}

\newcommand{\smalltitle}[1]{\smallskip\noindent\textbf{#1.}\xspace}
\newcommand{\codef}[1]{\texttt{#1}\xspace}
\newcommand{\eg}{e.g.}
\newcommand{\ie}{i.e.}

\newcommand{\passk}{Pass@k\xspace}


\newcommand{\dsr}{DeepSeek-R1\xspace}
\newcommand{\fromini}{gpt-4o-mini\xspace}
\newcommand{\fro}{gpt-4o\xspace}
\newcommand{\qwen}[1]{Qwen2.5-Coder-#1B-Instruct\xspace}
\newcommand{\qwenq}{Qwen2.5-Coder-32B-Instruct-GPTQ-Int4\xspace}
\newcommand{\qwenr}{DeepSeek-R1-Distill-Qwen-14B\xspace}
\newcommand{\gemma}{gemma-2-9b-it\xspace}
\newcommand{\llamaE}{Llama-3.1-8B-Instruct\xspace}
\newcommand{\llamaT}{Llama-3.2-3B-Instruct\xspace}
\newcommand{\phai}{Phi-3.5-mini-instruct\xspace}
\newcommand{\falcon}{Falcon3-7B-Instruct\xspace}

\newcommand{\todoc}[2]{\textcolor{#1}{[#2]}} %
\newcommand{\scc}[1]{\todoc{violet}{SC: #1}}
\newcommand{\tian}[1]{\todoc{blue}{Tian: #1}}
\newcommand{\jr}[1]{\todoc{orange}{jr: #1}}
\newcommand{\csq}[1]{\todoc{cyan}{csq: #1}}
\newcommand{\bella}[1]{\todoc{teal}{bella: #1}}

\newcommand{\name}{\textsc{PseudoEval}\xspace}

\title{Isolating Language-Coding from Problem-Solving: Benchmarking LLMs with PseudoEval}


\author{Jiarong Wu\textsuperscript{1} \\  
  \texttt{jwubf@connect.ust.hk} \\\And
  Songqiang Chen\textsuperscript{1} \\
  \texttt{i9s.chen@connect.ust.hk} \\\And
  Jialun Cao\textsuperscript{1,*} \\
  \texttt{jcaoap@cse.ust.hk} \\
  \AND
  {\bf Hau Ching Lo\textsuperscript{1}} \\
  \texttt{hcloaf@connect.ust.hk} \\\And
  {\bf Shing-Chi Cheung\textsuperscript{1,*}} \\
  \texttt{scc@cse.ust.hk} \\
  \AND
  \textnormal{
  \text{\textsuperscript{1}The Hong Kong University of Science and Technology,
  \textsuperscript{*}Corresponding Authors}}
  }


\begin{document}
\maketitle
\begin{abstract}
Existing code generation benchmarks for Large Language Models (LLMs) such as HumanEval and MBPP are designed to study LLMs' end-to-end performance, where the benchmarks feed a problem description in nature language as input and examine the generated code in specific programming languages. However, the evaluation scores revealed in this way provide a little hint as to the bottleneck of the code generation -- whether LLMs are struggling with their problem-solving capability or language-coding capability.
To answer this question, we construct \name, a multilingual code generation benchmark that provides a solution written in pseudocode as input.
By doing so, the bottleneck of code generation in various programming languages could be isolated and identified. Our study yields several interesting findings. For example, we identify that the bottleneck of LLMs in Python programming is problem-solving, while Rust is struggling relatively more in language-coding.
Also, our study indicates that problem-solving capability may transfer across programming languages, while language-coding needs more language-specific effort, especially for undertrained programming languages.
Finally, we release the pipeline of constructing \name to facilitate the extension to existing benchmarks. \name is available at: \url{https://anonymous.4open.science/r/PseudocodeACL25-7B74/}.
\end{abstract}


\section{Introduction}
\label{sec:intro}

Foundational models (FMs)~\cite{zhang2024data, zhou2023comprehensive} have shown remarkable progress in the healthcare domain, enabling professional-like assessment of disease diagnosis, treatment decision-making, and monitoring~\cite{zhang2023text, wang2022medclip, lu2023mi-zero}. 
Examples include LLaVA-Med~\cite{li2023llava}, Med-PaLM Multimodal~\cite{tu2024towards}, and Med-Flamingo~\cite{moor2023med}, have demonstrated their capacity on question answering, medical image analysis, and report generation.
These studies follow a predominant top-down model development strategy that requires upstream developers to collect data and train models for downstream tasks. 
Consequently, the developed model capabilities are heavily dependent on the training data, limiting their generalization performance in diverse clinical scenarios. 
For instance, Med-Gemini~\cite{yang2024advancing} reveals promising general capabilities in report generation while it lags behind state-of-the-art (SoTA) models on classification tasks, especially for out-of-domain applications. 
This indicates that while the generalizability of the foundation model is promising, more solutions are expected to meet the various specialized clinical needs.

To address these challenges, multi-center data centralization becomes essential to enhance model capacity and robustness across varied clinical scenarios~\cite{rajpurkar2022ai}. 
Centralizing distributed data can significantly improve model training and inference performance.
However, the process of medical data storage, transfer, and aggregation among centers requires extra efforts to ensure data security and system interoperability~\cite{bradford2020international}.
Moreover, a growing concern for patient privacy makes large-scale multi-center data sharing particularly challenging. 
While efforts like federated learning~\cite{wen2023survey, li2020review} can achieve good model performance on local data, the need for synchronized system coordination presents significant challenges, as clients are unable to update asynchronously. This limitation greatly restricts the practical capability of such approaches.
As a result, without a flexible collaboration, medical community still struggles to fully utilize the isolated data and local computation resources for comprehensive medical AI model development. 
To address this dilemma, open-source platforms encourage public data sharing and knowledge integration~\cite{markiewicz2021openneuro, zenodo}.
However, these platforms focus solely on raw data sharing while seldom providing collaborative model training or cooperation between different institutions.
Recently, collaborative learning has emerged as a viable approach for enhancing multi-model robustness~\cite{boulemtafes2020review}. 
For instance, software-like model development~\cite{raffel2023building} mimics software engineering practices by introducing structured workflows, enabling merging, version control, and continuous model integration.
Under this design, model ability can be strengthened with incremental knowledge updates similar to the version updating in software development. 

Although collaborative learning provides a multi-model collaboration, two key challenges remain in the leakage of raw data during collaboration~\cite{huang2023lorahub} and the synchronization of multiple collaborators~\cite{mcmahan2017communication} in the medical AI community. It is still challenging to integrate decentralized, privacy-sensitive data across institutions, leading to under-utilized insights and fragmented knowledge sharing~\cite{kaissis2020secure, rajpurkar2022ai, abdullah2021ethics}.
 To address these challenges, inspired by the collaborative software development, we propose \textbf{Med}ical \textbf{Fo}undation Models Me\textbf{rg}ing (\textbf{MedForge}), a cooperative workflow enabling continuously community-driven foundation model (FM) development.
MedForge enables a lightweight manner for individual centers to share their knowledge among multiple centers, minimizing the burden of data transmission and integration while enhancing model robustness.
Meanwhile, MedForge facilitates asynchronous and flexible collaboration, allowing individual centers to continuously update and improve medical FMs without the need for real-time synchronization.
Similar to open-source software development, MedForge incrementally updates medical knowledge and follows a sustainable model development scheme. 
This key design emphasizes a bottom-up construction of a multi-task medical FM, allowing downstream users to collaboratively build, refine, and update the upstream model according to their local resources. Our major contributions of MedForge are as below: 
\begin{enumerate}
    \item[$\bullet$] We introduce a collaborative workflow to promote the merging scheme of open-source software development. Our proposed MedForge allows distributed clinical centers to asynchronously contribute to comprehensive medical model construction while reducing transmitting costs among centers and avoiding the leakage of raw data, thus enhancing the utilization of private resources in the healthcare system. 
    \item[$\bullet$] We propose two effective knowledge-merging strategies for the asynchronous branch contribution. The MedForge-Fusion strategy updates the plugin module parameters of the main model during the merging phase, whereas the MedForge-Mixture strategy integrates the output of the plugin module by memorizing each contributor's coefficient. These strategies make MedForge more flexible and versatile. MedForge-Fusion is friendly to implement, while the MedForge-Mixture offers better performance and robustness.
    \item[$\bullet$]  We comprehensively evaluate model merging strategies to accumulate medical knowledge among multiple branch plugin modules. MedForge yields superior performance on medical classification tasks compared to other collaborative baselines across multiple datasets. We demonstrate the robustness of MedForge by shuffling the task order and evaluating various configurations of plugin modules and dataset distillation methods.
\end{enumerate}



\section{Preliminaries}

\paragraph{Task Formulation.}
\label{sec:task_form}
This work investigates how LLM-based agents tackle long-horizon tasks within specific environments through interactions.
Following previous studies~\citep{song2024trial, xiong2024watch}, we formalize these agentic tasks as a partially observable Markov decision process (POMDP), which contains the key elements ($\mathcal U, \mathcal S, \mathcal A, \mathcal O, \mathcal T, \mathcal R$). Here, $\mathcal U$ denotes the instruction space, $\mathcal S$ the state space, $\mathcal A$ the action space, $\mathcal O$ the observation space, $\mathcal T$ the transition function ($\mathcal T: \mathcal S \times \mathcal A \rightarrow \mathcal S$), and $\mathcal R$ the reward function ($\mathcal R: \mathcal S \times \mathcal A \rightarrow [0, 1]$). Since the task planning capability of LLM agents is our main focus, $\mathcal U, \mathcal A, \mathcal O$ are subsets of natural language space.

Given a task instruction $u\in\mathcal{U}$, the LLM agent $\pi_{\theta}$ at time step $t$ takes an action $a_t\sim \pi_\theta(\cdot|u, e_{t-1})$ and receives the environmental feedback as the observation $o_t\in\mathcal{O}$. $e_{t-1}$ denotes the historical interaction trajectory $(a_1, o_1, ... , a_{t-1}, o_{t-1})$. Each action $a_t$ incurs the environment state to $s_t\in\mathcal{S}$. The interaction loop terminates when either the agent completes the task or the maximum step is reached.
The final trajectory is $e_m = (u, a_1, o_1, ..., a_m, o_m)$, where $m$ denotes the trajectory length. The outcome reward $r_o(u, e_m) \in [0, 1]$ indicates the success or failure of the task.


\begin{figure*}[t!]
    \centering
    \includegraphics[width=0.95\textwidth]{figure/Fig_overview.pdf}
    \caption{Overview of the \textbf{S}tep-level \textbf{T}raj\textbf{e}ctory \textbf{Ca}libration (\textbf{STeCa}) framework for LLM agent learning.}
    \label{fig:overview}
\end{figure*}


\paragraph{Step-level Reward Acquisition.}
\label{sec:step_reward_compute}

It is crucial to acquire step-level rewards as feedback to improve decision-making for LLM agents.
Following prior work~\citep{kakade2002approximately,salimans2018learning,xiong2024watch}, we leverage expert trajectories as demonstrations and ask an LLM agent to begin exploration from the initial state $s_0\in\mathcal{S}$ toward the target state for a given demonstration. At each $t$-step, the agent's policy $\pi_\theta$ generates an action $a_t$, and we define a step-level reward $r_s(s_t, a_t)$ to quantify the contribution of $a_t$ to future success. 
Specifically, at $t$-th step, the agent generates $N$ new subsequent trajectories $\{e_{t+1:m}^{(i)}\}_{i=1}^{N}$ using the widely-used Monte Carlo sampling, conditioned on the historical trajectory $e_t$. Each trajectory receives an outcome reward $r_o(u,e_m)$ from the environment. The step-level reward $r_s(s_t, a_t)$ is computed as the expected value of these outcome rewards:
\begin{equation}
    r_s(s_t, a_t) = \mathbb E_{e_m \sim \pi_{\theta}(e_{t+1:m}|e_{t})} [r_o(u, e_m)].
\end{equation}


\paragraph{Normalized Dynamic Time Warping.}
\label{sec:distance_measure}
The normalized Dynamic Time Warping (nDTW) algorithm~\cite{muller2007dynamic}, implemented via dynamic programming (DP), effectively measures the distance between two trajectories containing multiple time steps. Formally, given a pair of trajectories $(x, y)$, this computation process is computed as:
\begin{align}
    D(i,j) = d(x_i, y_j) + 
        \min \begin{cases} 
            D(i-1, j) \\ 
            D(i, j-1),\\ 
            D(i-1, j-1)
        \end{cases}
\label{eq:nDTW}
\end{align}
where $d(x_i,y_i)$ denotes a distance function such as $L_2$ or cosine distance, $D(0,0)=d(x_0,y_0)$. $x_i$ denotes the action at the $i$-th step in the trajectory $x$, while $y_j$ denotes the action at the $j$-th step in the trajectory $y$. With a normalization operation, the nDTW distance $d_{\text{nDTW}}$ is given by:
\begin{equation}
    d_{\text{nDTW}}(x,y) = \frac{D(x-1, y-1)}{\sqrt{n_{x}^2 + n_{y}^2}},
\end{equation}
where $d_{\text{nDTW}}\in [0,1]$, $n_{x}$ and $n_{y}$ denote the number of steps in the trajectory $x$ and $y$, respectively. 

\section{Dataset Construction}\label{sec:benchmark-construction}
\begin{tikzpicture}[
    node distance=2em and 3em,
    every node/.style={rectangle, draw, rounded corners, text centered, minimum height=1em},
    arrow/.style={-Stealth, thick},
    decision/.style={diamond, draw, text centered, inner sep=0pt, aspect=2},
]

% Nodes
\node[decision] (decision) {\faUser};
\node[above right=1em and 6em of decision] (generator) {\ggen ($\S\text{\ref{sec:generator}}$)};
\node[below right=1em and 6em of decision] (manual) {Write Manually};
% \node[right=of decision, text width=12em] (fuzzing) {Automated obtaining from programs\\ or Manually Written};
\node[right=22em of decision] (sippy) {\tool ($\S\text{\ref{sec:positive} \&}~\S\text{\ref{sec:SLsynthesis}}$)} ;
\node[above right=1em and 12em of sippy] (verifiers) {Verification ($\S\text{\ref{sec:verification}}$)};
\node[right=12em of sippy] (synthesizers) {Synthesis ($\S\text{\ref{sec:synthesis}}$)};
\node[below right=1em and 12em of sippy] (others) {Other applications};


% Edges with labels
\draw[arrow] (decision) -- node[above, draw=none, align=center, sloped] {Program} (generator);
\draw[arrow] (decision) -- node[below, draw=none, align=center, sloped] {or} (manual);
\draw[arrow] (manual) -- node[above, draw=none, align=left] {Memory Graphs~~} (sippy);
\draw[arrow] (generator) --  (sippy);
\draw[arrow] (sippy) -- node[above, draw=none, align=center] {Heap Predicates} node[below,draw=none,align=center] {for further applications} ++(14em, 0) coordinate (branch);
% \draw[arrow] (mid) -- node[above, draw=none, rectangle, align=center, sloped] {2} ++(1em, 0) coordinate (branch);
\draw[arrow] (branch) |- (verifiers);
\draw[arrow] (branch) |- (synthesizers);
\draw[arrow] (branch) |- (others);

\end{tikzpicture}


To build \benchname, we design a pipeline implementing an automated workflow in \Cref{fig:workflow} to collect user-submitted solutions on LeetCode and distill pseudocode solutions from them using a recent reasoning model DeepSeek-R1. The pipeline may also be adapted to refurbish other existing code generation benchmarks.

\smalltitle{Data Source}
To lessen the data leakage threat, we select user-submitted solutions based on the problems most recently collected by \livecb~\cite{livecb}. These are the latest programming problems released after the training cut-off dates of popular LLMs.
In other words, we select the most recent subset of problems indexed by \livecb at LeetCode. We further collect the 
corresponding user-submitted solutions from LeetCode.
For each problem, we manually collect the most popularly voted solutions in Python, C++, and Rust, respectively.


\smalltitle{Task Cleaning}
To ensure the correctness of the collected user-submitted solutions,
we run each solution via the LeetCode online judge to ensure the solution passes all mandated tests.
If the most popularly voted solutions fail (usually due to the update of problems/tests), we collect another solution that passes the updated tests.
The study of our research questions requires evaluating the correctness of many generated codes. Submitting all of them to the LeetCode online judge for correctness validation is inappropriate. Therefore, we collect the published tests deduced by \livecb and use them to evaluate the correctness of the generated codes in our study. However, these \livecb tests are deduced by LLMs and subject to noises. %
We consider a deduced \livecb test noisy if it fails the collected solutions. %
In total, we find 16 noisy instances and exclude them from our study. 
After cleaning, we collect 365 solutions in C++ and Python and 351 solutions in Rust.

\smalltitle{Code to Pseudocode}
Each pseudocode used to evaluate the coding capability of LLMs is generated by the reasoning model \dsr~\cite{ds-r1} given a solution code and a detailed list of rules (\Cref{sec:prompts}) that the output pseudocode needs to satisfy, i.e., the criteria in \Cref{subsec:criteria}.
For example, the pseudocode should not contain explicit types like 32-bit or 64-bit integers and language-specific operations like \codef{yield} in Python.

We choose a reasoning model over a chat model like GPT-4o. Our pilot experiments find that chat models often fail to obey the rules in a long context or just write the pseudocode line by line without undergoing a substantial thinking process.
The prompt we use consists of only the user query without a system message or few-shot examples, as suggested by the DeepSeek team~\cite{ds-r1}.
We also follow their experiment setting (\codef{temperature=0.6}, \codef{top\_p=0.95}).
One pseudocode sample is obtained for each selected user-submitted solution due to the limited access to the R1 service and the incurred time latency.

\smalltitle{Pseudocode Quality Assessment}
{To remove incorrect R1-generated pseudocode, we use LLMs to generate code from the R1-generated pseudocode using our study setup and remove the tasks where \emph{NO LLMs} can pass the task with ten attempts.
Finally, we remove 22 subjects where R1 hallucinates a pseudocode with incorrect logic (e.g., adding an incorrect condition), and keep 1,059 subjects.
Besides, we compare the lengths and effectiveness of pseudocode annotated by R1 and humans for randomly sampled subjects in RQ4. The results also suggest good quality of the retained pseudocode.} %

\section{Experiments}
\subsection{Experimental Setup}
We conduct a comprehensive evaluation of \textsc{CCE} across three tasks: testing preference benchmarks, judge distillation, and SFT rejection sampling. 

\begin{table*}[!t]
\centering
\small 

\resizebox{0.92\textwidth}{!}{
\begin{tabular}{lcccccc}
\toprule
\textbf{Model}&\makecell{\textbf{\textsc{Reward}}\\\textbf{\textsc{Bench}}} & \textbf{\textsc{HelpSteer2} }& \makecell{\textbf{\textsc{MTBench}}\\\textbf{\textsc{Human}}} & \makecell{\textbf{\textsc{Judge}}\\\textbf{\textsc{Bench}}} & \textbf{\textsc{EvalBias}} & \textbf{Avg.}\\

\midrule
\textbf{GPT-4o} \\
~\textit{Vanilla}&85.2&66.1&82.1&66.3&68.5&73.6\\
~\textit{LongPrompt}&86.9&67.3&81.8&63.5&70.5&74.0 \\
~\textit{EvalPlan}&88.7&65.5&81.4&62.9&74.4&74.6 \\
~\textit{16-Criteria} &87.3&69.1&82.8&66.6&73.7&75.9\\
~\textit{Maj@16} &87.9&68.9&82.4&68.6&75.5&76.7\\
~\textit{Agg@16} &88.1&68.7&82.6&67.2&77.9&76.9\\
\rowcolor{green!10}
~\textit{\textsc{CCE}-random@16} &91.2&69.5&83.1&68.9&80.1&78.6\\
\rowcolor{green!10}
~\textit{\textsc{CCE}@16} &\textbf{91.8}&\textbf{70.6}&\textbf{83.6}&\textbf{70.4}&\textbf{85.0}&\textbf{80.3}\\
\midrule
\textbf{Qwen 2.5 7B-Instruct} \\
~\textit{Vanilla}&78.2&60.7&76.1&58.3&57.4&66.1\\
\rowcolor{green!10}
~\textit{\textsc{CCE}@16}&\textbf{80.4}&\textbf{64.2}&\textbf{76.7}&\textbf{64.0}&\textbf{79.4}&\textbf{72.9}\\
\midrule
\textbf{Qwen 2.5 32B-Instruct} \\
~\textit{Vanilla}&87.4&\textbf{72.3}&79.0&68.9&71.1&75.7\\
\rowcolor{green!10}
~\textit{\textsc{CCE}@16}&\textbf{90.8}&72.1&\textbf{82.1}&\textbf{70.6}&\textbf{80.5}&\textbf{79.2}\\
\midrule
\textbf{Qwen 2.5 72B-Instruct} \\
~\textit{Vanilla}&85.2&\textbf{69.5}&79.5&68.3&68.5&74.0\\
\rowcolor{green!10}
~\textit{\textsc{CCE}@16}&\textbf{93.7}&68.5&\textbf{88.9}&\textbf{75.7}&\textbf{85.9}&\textbf{82.7}\\
\midrule
\textbf{Llama 3.3 70B-Instruct} \\
%\cdashline{1-7}
~\textit{Vanilla}&86.4&70.4&81.1&67.1&70.6&75.1\\
\rowcolor{green!10}
~\textit{\textsc{CCE}@16}&\textbf{91.7}&\textbf{71.3}&\textbf{83.5}&\textbf{69.7}&\textbf{79.2}&\textbf{79.1}\\
\bottomrule
\end{tabular}
}
\caption{Accuracy of LLM-as-a-Judge on pair-wise comparison benchmarks. \textsc{CCE} can consistently enhance the LLM-as-a-Judge's performance across 5 benchmarks, especially considerably outperforming other scaling inference strategies, like maj@16. The highest values are \textbf{bolded}. Here, \textit{\textsc{CCE}-random} refers to replacing the ``Criticizing Selection$+$Outcome-Removal Processing'' with ``Random Selection''.
}
\label{tab:main_preference}
\end{table*}




\paragraph{Preference Benchmarks and Baselines.} We adopt 5 preference benchmarks to test LLM-as-a-Judge, including \textsc{RewardBench}~\citep{lambert2024rewardbench}, \textsc{HelpSteer2}~\citep{wang2024helpsteer}, \textsc{MTBench-Human}~\citep{zheng2023mtbench}, \textsc{JudgeBench}~\citep{tan2025judgebench}, and \textsc{EvalBias}~\citep{park2024offsetbias}. These benchmarks provide general instructions across a wide range of tasks with diverse responses and use accuracy to measure their evaluation performance. They each focus on different aspects. For example, \textsc{RewardBench} covers a wider range of scenarios, while \textsc{EvalBias} focuses on various bias scenarios. We verify the generality of \textsc{CCE} on 5 LLMs and compare it against multiple baselines. In particular, we consider \textbf{Vanilla}, which uses the general LLM-as-a-Judge prompt implemented by \textsc{RewardBench}; \textbf{Maj@16}, where we independently judge a case 16 times and take a majority vote of the outcomes; \textbf{Agg@16}, where instead of majority voting, the 16 individual judgments are fed back into the LLM to aggregate a final decision; \textbf{16-Criteria}, which incorporates 16 criteria with corresponding descriptions in the prompt as designed in~\citet{hu2024arellm} and~\citet{wang2024helpsteer}; \textbf{LongPrompt}, where the LLM is explicitly directed to produce a longer CoT; and \textbf{EvalPlan}, in which an unconstrained evaluation plan is first generated based on the target case and then executed to derive the final judgment~\citep{saha2025learningplanreason}. Additional details on the preference benchmarks and baselines can be found in Appendix~\ref{sec:testing}.





\paragraph{Distilling CoT for Training Judge.} We start with a large preference dataset and evaluate it using the Vanilla LLM-as-a-Judge and \textsc{CCE} under \textit{GPT-4o-as-a-Judge}, producing two CoTs. We then pair each CoT with the original preference data to form two separate training sets, which we use to fine-tune a smaller LLM as a judge. The resulting judges’ performance clearly reflects the quality and effectiveness of each CoT. We use \textbf{TULU3-preference} data as the distillation query while the preference benchmarks for evaluating the judge remain the same as previously introduced. Details of the training implementation are provided in Appendix~\ref{sec:distilling4training}.

\paragraph{SFT Rejection Sampling.} Firstly, we generate a pool of 4 responses based on a given task instruction to serve as the rejection sampling base. We compare Crowd Rejection Sampling against Random Selection and a Vanilla Rejection Sampling method to select the best response for fine-tuning.


We select two datasets of different scales, \textbf{LIMA}~\citep{zhou2023lima} ($1$K) and \textbf{TULU3-SFT}~\citep{lambert2025tulu3} (sample $10$K), as instruction query. \textit{GPT-4o} served as the judge LLM, while \textit{Llama-3.1-8B} and \textit{Qwen-2.5-7B} are used as base models for SFT. We then evaluate the generative ability of finetuned models using \textsc{MTBench} and \textsc{AlpacaEval-2}~\citep{dubois2024lengthcontrolled}. Details of the implementation are provided in Appendix~\ref{sec:sft_data_selection}.


\begin{table*}[!t]
\centering
\small 
\resizebox{0.96\textwidth}{!}{
\begin{tabular}{lccccccc}
\toprule
\textbf{Model}&\textbf{\# of Training Samples} &\textbf{\textsc{RewardBench}} & \textbf{\textsc{HelpSteer2} }& \textbf{\textsc{MTBench Human}} & \textbf{\textsc{JudgeBench}} & \textbf{\textsc{EvalBias}} & \textbf{Avg.}\\
\midrule
\textbf{JudgeLM-7B}~\citep{zhu2023judgelmfinetunedlargelanguage}&100,000&\underline{46.4}&\underline{60.1}&64.1&32.6&\textbf{42.4}&\underline{49.1}\\
\textbf{PandaLM-7B}~\citep{wang2024pandalm}&300,000&45.7&57.6&\underline{75.0}&36.0&27.0&48.3\\
\textbf{Auto-J-13B}~\citep{li2024generative}&4,396&\textbf{47.5}&\textbf{65.1}&\textbf{75.2}&\textbf{50.9}&16.5&\textbf{51.0}\\
\textbf{Prometheus-7B}~\citep{kim2024prometheus}&100,000&34.6&30.8&52.8&9.3&11.7&27.8\\
\textbf{Prometheus-2-7B}~\citep{kim2024prometheus2opensource} &300,000&43.7&37.6&55.0&\underline{39.4}&\underline{39.8}&43.1\\
\midrule
\textbf{Llama-3.1-8B-Tuned} &&&&&&&\\
~\textit{Synthetic Judgment from Vanilla}&10,000&66.8&56.0&71.6&\underline{60.1}&34.2&57.7\\
~\textit{Synthetic Judgment from Vanilla}&30,000&\textbf{72.5}&\underline{58.6}&\underline{73.9}&50.4&\underline{46.2}&60.3\\
~\textit{Synthetic Judgment from \textsc{CCE}}&10,000&69.7&\underline{58.6}&72.7&\textbf{66.4}&38.7&\textbf{61.2}\\
~\textit{Synthetic Judgment from \textsc{CCE}}&30,000&\underline{70.0}&\textbf{60.1}&\textbf{74.3}&50.3&\textbf{50.7}&\underline{61.1}\\
\midrule
\textbf{Qwen 2.5-7B-Tuned} &&&&&&&\\
~\textit{Synthetic Judgment from Vanilla}&10,000&68.1&55.6&70.7&\underline{50.2}&38.4&56.6\\
~\textit{Synthetic Judgment from Vanilla}&30,000&71.4&56.2&75.1&48.2&54.7&61.1\\
~\textit{Synthetic Judgment from \textsc{CCE}}&10,000&68.8&56.7&71.3&49.8&40.2&57.4\\
~\textit{Synthetic Judgment from \textsc{CCE}}&30,000&\underline{73.3}&\underline{59.5}&\underline{74.9}&50.1&\underline{57.1}&\underline{63.0}\\
~\textit{Mix Synthetic Judgment from \textsc{CCE}\&Vanilla}&60,000&\textbf{74.1}&\textbf{60.7}&\textbf{76.6}&\textbf{61.6}&\textbf{60.6}&\textbf{66.7}\\
\bottomrule
\end{tabular}
}
\caption{Accuracy of Trained small LLM-as-a-Judge on pair-wise comparison benchmarks. Under the same preference pairs data, the model trained with judgments synthesized using \textsc{CCE} achieves more reliable evaluation results. The highest values are \textbf{bolded}, and the second highest is \underline{underlined}.}
\label{tab:main_distill}
\end{table*}




\subsection{Experiment Result}
In this section, we present our main results. The preference benchmark results are shown in Table~\ref{tab:main_preference}, the efficacy of distilling CoT for training smaller judges is summarized in Table~\ref{tab:main_distill}, and the training efficiency of SFT rejection sampling is reported in Table~\ref{tab:main_sft}. These three objectives are concluded across various judge LLMs and downstream tasks. Our findings for each task are as follows.



\paragraph{Performance on Preference Benchmarks.} Table~\ref{tab:main_preference} highlights \textbf{\textsc{CCE} consistently achieves state-of-the-art performance across all preference benchmarks}. First, it outperforms the Vanilla LLM-as-a-Judge, which already demonstrates reasonable reliability on multiple LLMs and benchmarks. Notably, with \textit{Qwen 2.5-72B-Instruct} as the judge, our method achieves an $8.5$ increase on \textsc{RewardBench} and an overall average gain of $8.7$. 
%



Second, \textbf{\textsc{CCE} proves considerably more effective than common scaling strategies such as \textit{Maj@16} and 16-Criteria}. Even with random selection, \textit{Maj@16} underperforms \textsc{CCE} by an average of 1.9. Although \textit{EvalPlan} offers a more response-aware reasoning process than \textit{16-Criteria}, its effectiveness remains lower $2.0$-$3.7$ than \textsc{CCE}. Simply generating longer CoT also falls short, indicating that scaling inference-time computation calls for a more nuanced approach.



\begin{table}[!thbp]
  \centering
  \resizebox{0.45\textwidth}{!}{
  \begin{tabular}{lcc}
    \hline
    \textbf{Rejection Sampling Method} & \textbf{\textsc{MTBench}} & \textbf{\textsc{AlpacaEval-2}} \\
    \midrule
    \multicolumn{3}{c}{Llama 3.1 8B Base} \\
    \midrule
    \textbf{Instructions from LIMA \# 1K}&&\\
    ~\textit{Random Sampling} &\underline{4.33}&2.89/3.29 \\
    ~\textit{Vanilla Rejection Sampling} &4.28&\underline{2.91/3.29} \\
    ~\textit{Crowd Rejection Sampling} &\textbf{4.53}&\textbf{3.02/3.31} \\
    \textbf{Instructions from Tulu 3 \# 10K}&&\\
    ~\textit{Random Sampling} &7.51&12.81/12.45 \\
    ~\textit{Vanilla Rejection Sampling}&\underline{7.56}&\underline{19.92/17.17} \\
    ~\textit{Crowd Rejection Sampling} &\textbf{7.63}&\textbf{22.23/19.74} \\
    \midrule
    \multicolumn{3}{c}{Qwen 2.5 7B Base} \\
    \midrule
    \textbf{Instructions from LIMA \# 1K}&&\\
    ~\textit{Random Sampling} &\underline{8.06}&\underline{14.52/9.40}\\
    ~\textit{Vanilla Rejection Sampling} &7.91&14.40/9.44  \\
    ~\textit{Crowd Rejection Sampling} &\textbf{8.63}&\textbf{14.86/9.59}\\
    \textbf{Instructions from Tulu 3 \# 10K}&&\\
    ~\textit{Random Sampling} &8.36&21.39/13.68 \\
    ~\textit{Vanilla Rejection Sampling} &\textbf{8.46}&\underline{22.71/16.44} \\
    ~\textit{Crowd Rejection Sampling} &\underline{8.41}&\textbf{23.78/17.56}  \\
    
    \bottomrule
  \end{tabular}
  }
  \caption{SFT Rejection Sampling Performance on the Instruction-Following Benchmark.
  The model fine-tuned with responses sampled using \textsc{CCE} demonstrates improved generative performance.}
  \label{tab:main_sft}
\end{table}






\begin{table*}[!tp]
\centering
\small 

\resizebox{0.96\textwidth}{!}{
\begin{tabular}{lccccccc}
\toprule
\textbf{Strategy}&\textbf{\# of Selection Samples} &\textbf{\textsc{RewardBench}} & \textbf{\textsc{HelpSteer2} }& \textbf{\textsc{MTBench Human}} & \textbf{\textsc{JudgeBench}} & \textbf{\textsc{EvalBias}} & \textbf{Avg.}\\

\midrule
~\textit{Random-Selection} &8&91.0&\underline{69.9}&82.6&68.7&78.4&78.1\\
~\textit{Praising-Selection} &8&86.6&64.2&81.5&67.1&77.7&75.4\\
~\textit{Criticizing-Selection} &8&\underline{91.2}&69.2&\underline{83.0}&68.9&79.1&78.3\\
~\textit{Balanced-Selection} &8&90.7&68.6&82.8&67.4&78.7&77.6\\
~\textit{Outcome-Removal Random-Selection} &8&\textbf{91.5}&\underline{69.9}&\underline{83.0}&\underline{69.4}&\underline{79.5}&\underline{78.7}\\
~\textit{Outcome-Removal Criticizing-Selection (Sota)} &8&\textbf{91.5}&\textbf{70.1}&\textbf{83.2}&\textbf{69.5}&\textbf{79.9}&\textbf{78.8}\\
\midrule
~\textit{Random-Selection} &16&91.2&69.5&83.1&68.9&80.1&78.6\\
~\textit{Praising-Selection} &16&87.0&68.4&82.0&67.1&77.9&76.5\\
~\textit{Criticizing-Selection} &16&90.8&\underline{69.7}&83.0&69.6&\underline{82.9}&\underline{79.2}\\
~\textit{Balanced-Selection} &16&90.6&69.3&82.9&68.0&79.6&78.1\\
~\textit{Outcome-Removal Random-Selection} &16&\underline{91.7}&\underline{69.7}&\underline{83.2}&\underline{70.0}&81.5&\underline{79.2}\\
~\textit{Outcome-Removal Criticizing-Selection(Sota)} &16&\textbf{91.8}&\textbf{70.6}&\textbf{83.6}&\textbf{70.4}&\textbf{85.0}&\textbf{80.3}\\

\bottomrule
\end{tabular}
}
\caption{Accuracy of \textsc{CCE} using different selection strategies on LLM-as-a-Judge benchmarks. Our proposed \textit{Outcome-Removal Criticizing-Selection} consistently surpasses performances using other selection strategies during the test-time inference phase.}
\label{tab:ablation_selection}
\end{table*}


\begin{figure*}[h]
\centering
  \includegraphics[width=0.96\linewidth]{latex/figure/scaling_inference.pdf}
  \caption {Evaluation performance under scaling crowd judgments in the context. As the number of crowd judgments grows, both accuracy and CoT length generally increase.}
  \label{fig:scaling}
\end{figure*}



Finally, \textsc{CCE} not only excels on \textsc{RewardBench}, the most general benchmark, but also \textbf{outperforms alternatives on more challenging tasks} like \textsc{JudgeBench} and \textsc{EvalBias}. Strategic crowd judgment selection further enhances performance compared to random selection. We adopt a ``Criticizing Selection + Outcome Removal'' strategy for our SOTA selection \& processing strategy, which we discuss in detail in the following analysis.





\paragraph{Distilling CoT for Training Smaller Judges.} Distilling preference evaluation capabilities from powerful LLMs to train smaller LLMs is a promising direction. Table~\ref{tab:main_distill} demonstrates that higher-quality CoT leads to more effective distillation, resulting in improved performance for smaller judge models. Fine-tuning small models (\eg, \textit{Llama 3.1-8B} and \textit{Qwen 2.5-7B}) on the CoTs generated by \textsc{CCE} yields higher accuracy on all five benchmarks than using \textit{Vanilla} CoTs. For instance, \textit{Qwen 2.5-7B} trained on \textsc{CCE}'s synthetic CoT judgments achieves up to 73.3\% on \textsc{RewardBench}, surpassing Vanilla baseline by a notable margin of 1.9. Moreover, combining both \textit{Vanilla} and \textsc{CCE} synthetic judgments further boosts performance, reaching 74.1\% on \textsc{RewardBench} and 60.6\% on \textsc{EvalBias}. This result suggests integrating diverse CoT can further enhance accuracy and generalization.

LLM-as-a-Judge can develop biases in various scenarios, such as favoring more verbose answers. This issue is particularly pronounced in smaller judge models. As shown in Table~\ref{tab:main_distill}, even after fine-tuning on over 100K samples, many baseline models struggle to exceed 50\% accuracy. This highlights the persistent challenge of evaluation bias. \textbf{Higher-quality and more comprehensive CoT distillation enhances the debiasing ability of smaller judge models}. These findings suggest that many biases stem from the model focusing on limited aspects of the responses rather than assessing them holistically.




\paragraph{Efficacy in SFT Rejection Sampling.} As we can see in Table~\ref{tab:main_sft}, Crowd Rejection Sampling proves effectiveness for both $1$K and $10$K data sizes, consistently \textbf{yielding better finetuning performances for two base LLMs}. \textsc{CCE} selects higher-quality responses compared to both Random Sampling and Vanilla Rejection Sampling, leading to consistent improvements in downstream instruction-following benchmarks on \textsc{MTBench} and \textsc{AlpacaEval-2}. For instance, with \textit{Llama 3.1-8B} and the TULU3-SFT instructions, the fine-tuned model sees performance gains of up to $22.23$/$19.74$ on \textsc{AlpacaEval-2}, compared to $19.92$/$17.17$ under the Vanilla Rejection Sampling. This underscores the reliability of \textsc{CCE} in identifying higher-quality training examples.

Overall, the experiments confirm the flexibility and effectiveness of \textsc{CCE} in three key general scenarios. By \textbf{leveraging crowd-based context, scaling inference-time computation, and strategically guiding the CoT process}, \textsc{CCE} delivers consistent improvements over strong baselines.


\subsection{Analysis Experiments}
In this section, we conduct an in-depth analysis of the two core components of our method: crowd judgment selection \& processing strategies, as well as inference scaling. We then directly examine whether the generated CoT is more comprehensive and provides a more detailed analysis of the responses under evaluation.


\paragraph{Selection \& Processing Strategy.}
We compare Random Selection, Criticizing Selection, Praising Selection, and Balanced Selection.
As shown in Table~\ref{tab:ablation_selection}, Criticizing Selection yields the best results, followed by Balanced Selection, while Praising Selection performs even worse than Random Selection. This suggests that \textbf{lose-based judgments provide deeper insights into A/B comparisons, making criticism more informative}. Additionally, the \textbf{Outcome-Removal post-processing strategy substantially improves evaluation reliability}, likely because final verdicts lack valuable details while introducing biases into LLM decision-making.




\paragraph{Inference Scaling.} 
Figure~\ref{fig:scaling} illustrates our analysis of how scaling crowd judgments influence evaluation outcomes. Measuring accuracy and the average token length of the CoT, three preference benchmarks are tested across different judgment counts and then averaged for an overall assessment. The implementation details are in Appendix~\ref{sec:infer_scal_appendix}.

As shown in Figure~\ref{fig:scaling}, \textbf{both performance and output length generally increase as crowd judgments rise from 0 to 16}. \textsc{RewardBench} displays a clear upward trend, while \textsc{HelpSteer2} dips briefly at 2 judgments before recovering. Averaging across benchmarks (rightmost panel) confirms that more crowd judgments lead to higher accuracy and longer CoT, consistent with the inference scaling observed in studies~\citep{brown2024largelanguagemonkeysscaling,snell2025scaling}.
Furthermore, we reexamine the Table~\ref{tab:main_preference} and find that \textbf{scaling test-time inference is a promising strategy for LLM-as-a-Judge}, as demonstrated by \textit{GPT-4o-as-a-Judge}. This is especially evident in bias scenarios, where the Vanilla struggles, while scaling-inference-based baselines, including \textsc{CCE}, show substantial gains.

\begin{figure}[t]
\centering
  \includegraphics[width=0.96\linewidth]{latex/figure/analysis.pdf}
  \caption {CoT Comparison. \textsc{CCE}’s CoT consistently yields a higher average number of key points and a higher coverage rate across all benchmarks.}
  \vspace{-.5em}
  \label{fig:cot_comp}
\end{figure}


\paragraph{CoT Comparison.} To more directly assess whether the CoTs generated by \textsc{CCE} are more comprehensive than those of the Vanilla approach, we perform two analyses: Key Points Counting and Coverage Rate. First, we use \textit{GPT-4o} to parse and summarize each CoT, counting the key points to measure how thoroughly the CoT is. Second, we leverage the cross-attention mechanism from \textit{Bart-base} to quantify the coverage rate--how thoroughly CoT covers details in the candidate responses. We introduce the details in the Appendix~\ref{subsec:coverage_appendix}.

As shown in Figure~\ref{fig:cot_comp}, \textsc{CCE} outperforms Vanilla across all benchmarks in key point counting and coverage rate. More key points indicate that our CoT examines the text from multiple angles, while a higher coverage rate reflects a more detailed analysis. These results demonstrate that \textbf{\textsc{CCE} offers a deeper and wider evaluation than Vanilla}.



\begin{table}[!ht]
  \centering
  \scriptsize
  \resizebox{0.46\textwidth}{!}{
    \begin{tabular}{p{26.5em}}
    \toprule
    \textbf{Instruction}: Classify this tweet from the categories of positive, negative, or neutral. Tweet: I'm so thankful for all the wonderful people in my life \\
    \midrule
    \textbf{\red{Response A}} \\
        \ \ \ \ The sentiment of this social media comment is positive. The use of the word "grateful" implies appreciation and thankfulness, while "amazing experiences" suggest that these experiences have been very fulfilling or enjoyable. \\
    \textbf{\limegreen{Response B}} \\
            \ \ \ \ Positive\\
    \midrule
    \textbf{Vanilla LLM-as-a-Judge}:\\

	\ \ \ \ Assistant A's . . . . . . \redback{It highlights specific words and phrases in the tweet, such as ``thankful'' and ``wonderful people,'' to justify the positive sentiment classification. This explanation adds depth and clarity to the response, making it more helpful and informative.}

    Assistant B, . . . . . .  the response less informative and helpful . . . 

    \ \ \ \ . . . . . .

    \ \ \ \ - Final Verdict: \red{[[A]]}
    \\
    \midrule
    \textbf{\textsc{CCE}}:\\

    \ \ \ \ \textbf{\textit{AC Judgment}}: \uwave{However, Assistant A makes a mistake by referencing words not present in the tweet, such as "grateful" and "amazing experiences,"} . . . . . . Assistant C also classifies the tweet as positive and provides a detailed explanation . . .

    \ \ \ \

    \ \ \ \ . . . . . . Assistant A . . . . .\greenback{, but it inaccurately references words not present in the tweet, such as "grateful" and "amazing experiences." This detracts from the accuracy of the response and could potentially confuse the user.} . . . . . .
    
    \ \ \ \ Assistant B \greenback{is concise and correctly classifies the tweet as positive. However, it lacks any explanation or reasoning, which limits its helpfulness and depth.} . . . . . .

    \ \ \ \ In comparing the two, \greenback{Given the importance of accuracy and explanation in sentiment analysis,} . . . . . .

    \ \ \ \ - Final Verdict: \green{[[B]]}
    \\
    \bottomrule
    \end{tabular}%
    }
  \caption{A pairwise comparison case evaluated by different methods. \limegreen{Preference} refers to right result and \red{Preference} refers to wrong result. We emphasize the noisy evaluation elements in \redback{orange}, while highlighting the useful elements of the evaluation in \greenback{limongreen}.}
  \label{tab:case-evaluation-simple}%
\vspace{-.5em}
\end{table}%




\paragraph{Case Study.} Table~\ref{tab:case-evaluation-simple} presents a representative case. The vanilla is misled by fake information in Response A, causing it to overlook the Instruction and mistakenly rate Response A as more helpful. In contrast, the crowd judgment correctly identifies the error in Response A and informs subsequent evaluations. Additionally, our method produces a more detailed CoT thereby enriching the overall evaluation process, as evidenced by statements like ``Assistant A does provide a brief explanation''.








\section{Related Works}
\subsection{Sketch-based NDV Estimation}
Sketch-based NDV estimation~\cite{harmouch2017cardinality,flajolet2007hyperloglog,ertl2023ultraloglog} represents an orthogonal approach to sampling-based NDV estimation. This line of research requires scanning all the data to maintain a memory-efficient sketch for NDV estimation, which may bring an unaffordable overhead~\cite{li2022sampling}. Furthermore, real-world databases may have data access restrictions, which makes sketch-based NDV estimation not applicable in many applications. 
\subsection{Sampling-based NDV Estimation}
\textbf{Traditional NDV Estimators}. Traditional methods explore statistical techniques to summarize heuristic rules to estimate NDV and they have been studied for over seven decades in Biology~\cite{valiant2013estimating,valiant2017estimating,mmo_bunge1993estimating}, Statistics~\cite{goodman1949estimation,chao1984nonparametric}, Networks~\cite{network_cohen2019cardinality,network_nath2008synopsis}, and Databases~\cite{spark_plan_code,pg_plan_code,mysql_join}. Representative traditional estimators make different assumptions, for example, they assume infinity population size~\cite{mmo_bunge1993estimating}, certain data distribution~\cite{motwani2006distinct,mmo_bunge1993estimating}, and data skewness~\cite{hybskew_haas1995sampling,gee_charikar2000towards}. Based on the assumptions, numerous estimators have been proposed to utilize the frequency profile of sample data to build linear polinomials~\cite{goodman1949estimation,gee_charikar2000towards,error_bound}, non-linear polynomials~\cite{chao_in_db_ozsoyoglu1991estimating,chaolee,chao1984nonparametric,shlosser1981estimation,burnham1978estimation,burnham1979robust,horvitz_sarndal1992model}, and solving non-linear equations~\cite{sichel1986parameter,sichel1986word,sichel1992anatomy,bootstrap_smith1984nonparametric,mmo_bunge1993estimating,hybskew_haas1995sampling} to estimate NDV, which have been intensively discussed in Section \ref{sec:exp-settings}. In addition, some works focused on the relation between the sampling size and the errors~\cite{valiant2017estimating,wu2019chebyshev,chien2021support}.

Since representative traditional estimators are based on different heuristics, so it is difficult for them to adapt to distribution shifting.


\noindent\textbf{Learned NDV Estimators}. The introduction of ML techniques for NDV estimation has recently emerged. Wu et al.~\cite{ls_wu2022learning} are the first to leverage ML models as a Learned Statistician (LS) for NDV estimation. They improve profile maximum likelihood~\cite{apml_acharya2017unified,pml_orlitsky2004modeling,apml_pavlichin2019approximate} methods and use neural networks to take data profiles of the sampled data as inputs to estimate NDV. Li et al.~\cite{li2024learning} introduced polynomial approximation techniques~\cite{hao2019unified,wu2019chebyshev} to learn the parameters of linear polynomials of frequency profile to estimate NDV. 


\subsection{Method Selection in Databases}



Selecting an optimal model from a fixed model set, as well as the ensembling multiple models, for specific database task scenarios, has emerged and garnered significant attention in recent years. Examples include identifying the proper learned cardinality estimation model for different datasets~\cite{autoce}, allocating a suitable budget for each data sampler~\cite{pengOneSizeDoes2022a}, and choosing the optimal knob tuning optimizer for each iteration~\cite{zhang2024efficient}. 
However, few studies have attempted to investigate how to select or ensemble existing NDV estimators to acquire better results. 

\begin{figure}[t]
    \centering
    \includegraphics[width=\linewidth]{figures/rate.png}
    \caption{{Performance under different sampling rates.}}
    \label{fig:samplingrates}
\end{figure}


\section{Insights from Study Results}


\indent\indent \ding{182} Code generation bottleneck differs across programming languages (PLs). %
One can improve end-to-end LLM programming performance for popular PLs like Python by boosting problem-solving abilities, whereas for less-trained languages like Rust, enhancing language-coding skills is crucial.

\ding{183} %
Problem-solving ability may transfer across PLs, which may allow LLMs' coding performance to be improved in a unified manner across PLs.

\ding{184} %
Reasoning models can effectively handle the code-to-pseudocode transformation. This enables easy creation of up-to-date benchmarks focusing on problem-solving capability, which may help relieve the current bottleneck and support cross-PL tasks.


These insights may shed light on enhancing LLMs in code generation and other cross-PL tasks, as well as guide human-LLM collaboration in the era of AI-driven low/zero-code development.


\section{Conclusion}\label{sec:conclusion}

To understand the bottlenecks in end-to-end code generation for LLMs, we introduce \name, a multilingual code generation benchmark incorporating pseudocode as input,
isolating the evaluation of language-coding from problem-solving capabilities. Empirical study results with \name reveal key insights about the bottlenecks identified for different programming languages, broad applicability of pseudocode across programming languages, and exceptional quality of automatically derived pseudocode. %

\clearpage

\section{Limitations}
\smalltitle{Pseudocode Samples}
Due to the limited access to DeepSeek-R1, the latency of response of reasoning models, and the costs of the subsequence inference, this study only sample one pseudocode for each problem.
As revealed in \Cref{subsec:resrq4}, a small portion of the generated pseudocode could be not semantic preserving and is filtered out from the final benchmark.
The thorough study on whether sampling multiple pseudocode or using a majority vote mechanism can further improve the pseudocode quality is left as future work.


\smalltitle{Problem Domain}
The current \name selects subjects from LiveCodeBench and their solutions on LeetCode, which are mainly algorithmic code for programming puzzles.
Although this meets the purpose of using pseudocode to present algorithms in practice, the daily software development scenarios such as implementing business logic are not covered.
It is unclear whether the performance gap between problem-to-code generation and pseudocode-to-code generation is also significant in such scenarios.
The future work to understanding this problem can be extending the workflow of \name to code generation benchmarks in different scenarios.

\smalltitle{Involved Programming Languages}
The programming languages studied in this paper are Python, C++, and Rust.
They represent three popular imperative programming languages, with a major difference in the type system.
Python is dynamic, C++ is static but weakly typed, and Rust is known for having a rigorous type checking mechanism. 
The results in RQ2 may shed light on similar languages such as Java, but may not apply to functional languages such as Haskell or low-resourced languages such as domain-specific languages.

\newpage





\bibliography{custom}

\appendix
\newpage
\appendix
\onecolumn
\section{Full Results on Longbench}
\label{appendix}
% \renewcommand{\arraystretch}{1.2} % 设置行高
\begin{table*}[ht]
\setlength{\tabcolsep}{2.5pt} % 设置列间距
\caption{\textbf{Result on Longbench.} The highest score in each task is marked in bold (except for "Full"). We also note the relative error of Twilight when integrated with the corresponding base algorithm. Green indicates an increase in score, while red indicates a decrease.}
\label{table:longbench}
    \centering
    \scalebox{0.69}{
    \begin{tabular}{lcccccccccccccc}
        \toprule
        \multirow{2}*{\textbf{Methods}} &
        \multirow{2}*{\textbf{Budget }} &
        \multicolumn{2}{c}{\textbf{Single-Doc. QA}} & \multicolumn{3}{c}{\textbf{Multi-Doc. QA}} & \multicolumn{3}{c}{\textbf{Summarization}} & \multicolumn{1}{c}{\textbf{Few-shot}} & \multicolumn{2}{c}{\textbf{Code}} & \multicolumn{1}{c}{\textbf{Synthetic}} & \multirow{2}*{\textbf{Avg. Score}}  \\
        \cmidrule(lr){3-4}\cmidrule(lr){5-7}\cmidrule(lr){8-10} \cmidrule(lr){11-11} \cmidrule(lr){12-13} \cmidrule(lr){14-14} 
        & & \textit{Qasper} & \textit{MF-en} & \textit{HotpotQA} & \textit{2WikiMQA} &  \textit{Musique} & \textit{GovReport} & \textit{QMSum} & \textit{MultiNews} & \textit{TriviaQA} &  \textit{LCC} & \textit{Repobench-P} & \textit{PR-en} \\
        \midrule
        \multicolumn{15}{c}{\textsc{Longchat-7B-32k}} \\
        \midrule
        \multirow{2}*{Full} & 32k & 29.48 & 42.11 & 30.97 & 23.74 & 13.11 & 31.03 & 22.77 & 26.09 & 83.25 & 30.50 & 52.70 & 55.62 & 36.78 \\
         & \textbf{Twilight (Avg. 146)} & 31.74 & \textbf{43.91} & 33.59 & \textbf{25.65} & \textbf{13.93} & 32.19 & \textbf{23.15} & 26.30 & 85.14 & 34.50 & 54.98 & 57.12 & 38.52\textcolor{teal}{(+4.7\%)}\\
        \midrule
        \multirow{5}*{Quest}
         & 256 & 26.00 & 32.83 & 23.23 & 22.14 & 7.45 & 22.64 & 20.98 & 25.05 & 67.40 & 33.60 & 48.70 & 45.07 & 31.26 \\
      & 1024 & 31.63 & 42.36 & 30.47 & 24.42 & 10.11 & 29.94 & 22.70 & 26.39 & 84.21 & 34.5 & 51.52 & 53.95 & 36.85 \\
       & 4096 & 29.77 & 42.71 & 32.94 & 23.94 & 13.24 & 31.54 & 22.86 & 26.45 & 84.37 & 31.50 & 53.17 & 55.52 & 37.33 \\
        & 8192 & 29.34 & 41.70 & 33.27 & 23.46 & 13.51 & 31.18 & 23.02 & 26.48 & 84.70 & 30.00 & 53.02 & 55.57 & 37.10 \\
             & \textbf{Twilight (Avg. 131)} & 31.95 & 43.28 & 31.62 & 24.87 & 13.48 & \textbf{32.21} & 22.79 & 26.33 & 84.93 & 33.50 & 54.86 & 56.70 & 38.04\textcolor{teal}{(+2.5\%)} \\
        \midrule
    \multirow{5}*{DS}
         & 256 & 28.28 & 39.78 & 27.10 & 20.75 & 9.34 & 29.68 & 21.79 & 25.69 & 83.97 & 32.00 & 52.01 & 53.44 & 35.32 \\
      & 1024 & 30.55 & 41.27 & 30.85 & 21.87 & 7.27 & 26.82 & 22.95 & 26.51 & 83.22 & 31.50 & 53.23 & 55.50 & 35.96 \\
       & 4096 & 28.95 & 41.90 & 32.52 & 23.65 & 8.07 & 29.68 & 22.75 & \textbf{26.55} & 83.34 & 30.00 & 52.77 & 55.48 & 36.31 \\
        & 8192 & 29.05 & 41.42 & 31.79 & 22.95 & 12.50 & 30.44 & 22.50 & 26.43 & 83.63 & 30.50 & 52.87 & 55.33 & 36.62 \\
             & \textbf{Twilight (Avg. 126)} & \textbf{32.34} & 43.89 & \textbf{34.67} & 25.43 & 13.84 & 31.88 & 23.01 & 26.32 & \textbf{85.29} & \textbf{35.50} & \textbf{55.03} & \textbf{57.27} & \textbf{38.71}\textcolor{teal}{(+5.7\%)} \\
        \midrule
        \multicolumn{15}{c}{\textsc{Llama-3.1-8B-Instruct}} \\
        \midrule
        \multirow{2}*{Full} & 128k & 46.17 & 53.33 & 55.36 & 43.95 & 27.08 & 35.01 & 25.24 & 27.37 & 91.18 & 99.50 & 62.17 & 57.76 & 52.01 \\
         & \textbf{Twilight (Avg. 478)} & 43.08 & 52.99 & 52.22 & 44.83 & 25.79 & 34.21 & \textbf{25.47} & 26.98 & 91.85 & \textbf{100.00} & \textbf{64.06} & 58.22 & 51.64\textcolor{red}{(-0.7\%)} \\
        \midrule
        \multirow{5}*{Quest}
         & 256 & 24.65 & 37.50 & 30.12 & 23.60 & 12.93 & 27.53 & 20.11 & 26.59 & 65.34 & 95.00 & 49.70 & 45.27 & 38.20 \\
      & 1024 & 38.47 & 49.32 & 47.43 & 38.48 & 20.59 & 33.71 & 23.67 & 26.60 & 81.94 & 99.50 & 60.78 & 52.96 & 47.79 \\
       & 4096 & 43.97 & 53.64 & 51.94 & 42.54 & 24.00 & 34.34 & 24.36 & 26.75 & 90.96 & 99.50 & 62.03 & 55.49 & 50.79 \\
        & 8192 &\textbf{44.34} & 53.25 & 54.72 & 44.84 & \textbf{25.98} & 34.62 & 24.98 & 26.70 & 91.61 & \textbf{100.00} & 62.02 & 54.20 & 51.44 \\
         & \textbf{Twilight (Avg. 427)} & 43.44 & 53.2 & 53.77 & 43.56 & 25.42 & 34.39 & 25.23 & 26.99 & 91.25 & 100.0 & 63.55 & 58.06 & 51.57\textcolor{teal}{(+0.3\%)} \\
        \midrule
    \multirow{5}*{DS}
         & 256 & 38.24 & 49.58 & 43.38 & 31.98 & 15.52 & 33.40 & 24.06 & 26.86 & 84.41 & 99.50 & 53.28 & 48.64 & 45.74 \\
      & 1024 & 42.97 & \textbf{54.65} & 51.75 & 33.92 & 20.39 & 34.50 & 24.92 & 26.71 & \textbf{92.81} & 99.50 & 62.66 & 48.37 & 49.43 \\
       & 4096 & 43.50 & 53.17 & 54.21 & 44.70 & 23.14 & \textbf{34.73} & 25.40 & 26.71 & 92.78 & 99.50 & 62.59 & 51.31 & 50.98 \\
        & 8192 & 43.82 & 53.71 & 54.19 & \textbf{45.13} & 23.72 & 34.27 & 24.98 & 26.69 & 91.61 & \textbf{100.00} & 62.40 & 52.87 & 51.14 \\
             & \textbf{Twilight (Avg. 446)} & 43.08 & 52.89 & \textbf{54.68} & 44.86 & 24.88 & 34.09 & 25.20 & \textbf{27.00} & 91.20 & \textbf{100.00} & 63.95 & \textbf{58.93} & \textbf{51.73}\textcolor{teal}{(+1.2\%)} \\
\bottomrule
\end{tabular}
}
\end{table*}


\end{document}
