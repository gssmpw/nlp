\pdfoutput=1

\documentclass[11pt]{article}

\usepackage[final]{acl}

\usepackage{times}
\usepackage{latexsym}

\usepackage[T1]{fontenc}

\usepackage[utf8]{inputenc}

\usepackage{microtype}

\usepackage{inconsolata}
\usepackage{pifont}
\usepackage{graphicx}

%\usepackage{minted}
\usepackage{listings}
\usepackage{markdown}
\definecolor{mygreen}{rgb}{0,0.6,0}
\definecolor{mymauve}{rgb}{0.58,0,0.82}
\lstset{
backgroundcolor=\color{white},
basicstyle=\footnotesize\ttfamily,
columns=fullflexible,
breaklines=true,
postbreak=\mbox{$\hookrightarrow$\space},
columns=fullflexible,
captionpos=b,
tabsize=4,
commentstyle=\color{mygreen},
escapeinside={\%*}{*)},
keywordstyle=\color{blue},
stringstyle=\color{mymauve}\ttfamily,
frame=single,
rulesepcolor=\color{red!20!green!20!blue!20},
keywordstyle=\color{blue!70},
commentstyle=\color{red!50!green!50!blue!50},
numbers=none, 
rulesepcolor=\color{red!20!green!20!blue!20},
xleftmargin=2em,
xrightmargin=2em
}

\usepackage{xspace}
\usepackage{xcolor}
\usepackage{booktabs}
\usepackage{nicematrix}
\usepackage{cleveref}
\usepackage{adjustbox}
%\usepackage[most]{tcolorbox}
\usepackage{listings-rust}
\usepackage{caption}
\usepackage{subcaption}


\newcommand{\tocheck}[1]{\textcolor{red}{#1}}

\newcommand{\benchname}{\textsc{PseudoEval}\xspace}
\newcommand{\livecb}{LiveCodeBench\xspace}

\newcommand{\smalltitle}[1]{\smallskip\noindent\textbf{#1.}\xspace}
\newcommand{\codef}[1]{\texttt{#1}\xspace}
\newcommand{\eg}{e.g.}
\newcommand{\ie}{i.e.}

\newcommand{\passk}{Pass@k\xspace}


\newcommand{\dsr}{DeepSeek-R1\xspace}
\newcommand{\fromini}{gpt-4o-mini\xspace}
\newcommand{\fro}{gpt-4o\xspace}
\newcommand{\qwen}[1]{Qwen2.5-Coder-#1B-Instruct\xspace}
\newcommand{\qwenq}{Qwen2.5-Coder-32B-Instruct-GPTQ-Int4\xspace}
\newcommand{\qwenr}{DeepSeek-R1-Distill-Qwen-14B\xspace}
\newcommand{\gemma}{gemma-2-9b-it\xspace}
\newcommand{\llamaE}{Llama-3.1-8B-Instruct\xspace}
\newcommand{\llamaT}{Llama-3.2-3B-Instruct\xspace}
\newcommand{\phai}{Phi-3.5-mini-instruct\xspace}
\newcommand{\falcon}{Falcon3-7B-Instruct\xspace}

\newcommand{\todoc}[2]{\textcolor{#1}{[#2]}} %
\newcommand{\scc}[1]{\todoc{violet}{SC: #1}}
\newcommand{\tian}[1]{\todoc{blue}{Tian: #1}}
\newcommand{\jr}[1]{\todoc{orange}{jr: #1}}
\newcommand{\csq}[1]{\todoc{cyan}{csq: #1}}
\newcommand{\bella}[1]{\todoc{teal}{bella: #1}}

\newcommand{\name}{\textsc{PseudoEval}\xspace}

\title{Isolating Language-Coding from Problem-Solving: Benchmarking LLMs with PseudoEval}


\author{Jiarong Wu\textsuperscript{1} \\  
  \texttt{jwubf@connect.ust.hk} \\\And
  Songqiang Chen\textsuperscript{1} \\
  \texttt{i9s.chen@connect.ust.hk} \\\And
  Jialun Cao\textsuperscript{1,*} \\
  \texttt{jcaoap@cse.ust.hk} \\
  \AND
  {\bf Hau Ching Lo\textsuperscript{1}} \\
  \texttt{hcloaf@connect.ust.hk} \\\And
  {\bf Shing-Chi Cheung\textsuperscript{1,*}} \\
  \texttt{scc@cse.ust.hk} \\
  \AND
  \textnormal{
  \text{\textsuperscript{1}The Hong Kong University of Science and Technology,
  \textsuperscript{*}Corresponding Authors}}
  }


\begin{document}
\maketitle
\begin{abstract}
Existing code generation benchmarks for Large Language Models (LLMs) such as HumanEval and MBPP are designed to study LLMs' end-to-end performance, where the benchmarks feed a problem description in nature language as input and examine the generated code in specific programming languages. However, the evaluation scores revealed in this way provide a little hint as to the bottleneck of the code generation -- whether LLMs are struggling with their problem-solving capability or language-coding capability.
To answer this question, we construct \name, a multilingual code generation benchmark that provides a solution written in pseudocode as input.
By doing so, the bottleneck of code generation in various programming languages could be isolated and identified. Our study yields several interesting findings. For example, we identify that the bottleneck of LLMs in Python programming is problem-solving, while Rust is struggling relatively more in language-coding.
Also, our study indicates that problem-solving capability may transfer across programming languages, while language-coding needs more language-specific effort, especially for undertrained programming languages.
Finally, we release the pipeline of constructing \name to facilitate the extension to existing benchmarks. \name is available at: \url{https://anonymous.4open.science/r/PseudocodeACL25-7B74/}.
\end{abstract}


%!TEX root = gcn.tex
\section{Introduction}
Graphs, representing structural data and topology, are widely used across various domains, such as social networks and merchandising transactions.
Graph convolutional networks (GCN)~\cite{iclr/KipfW17} have significantly enhanced model training on these interconnected nodes.
However, these graphs often contain sensitive information that should not be leaked to untrusted parties.
For example, companies may analyze sensitive demographic and behavioral data about users for applications ranging from targeted advertising to personalized medicine.
Given the data-centric nature and analytical power of GCN training, addressing these privacy concerns is imperative.

Secure multi-party computation (MPC)~\cite{crypto/ChaumDG87,crypto/ChenC06,eurocrypt/CiampiRSW22} is a critical tool for privacy-preserving machine learning, enabling mutually distrustful parties to collaboratively train models with privacy protection over inputs and (intermediate) computations.
While research advances (\eg,~\cite{ccs/RatheeRKCGRS20,uss/NgC21,sp21/TanKTW,uss/WatsonWP22,icml/Keller022,ccs/ABY318,folkerts2023redsec}) support secure training on convolutional neural networks (CNNs) efficiently, private GCN training with MPC over graphs remains challenging.

Graph convolutional layers in GCNs involve multiplications with a (normalized) adjacency matrix containing $\numedge$ non-zero values in a $\numnode \times \numnode$ matrix for a graph with $\numnode$ nodes and $\numedge$ edges.
The graphs are typically sparse but large.
One could use the standard Beaver-triple-based protocol to securely perform these sparse matrix multiplications by treating graph convolution as ordinary dense matrix multiplication.
However, this approach incurs $O(\numnode^2)$ communication and memory costs due to computations on irrelevant nodes.
%
Integrating existing cryptographic advances, the initial effort of SecGNN~\cite{tsc/WangZJ23,nips/RanXLWQW23} requires heavy communication or computational overhead.
Recently, CoGNN~\cite{ccs/ZouLSLXX24} optimizes the overhead in terms of  horizontal data partitioning, proposing a semi-honest secure framework.
Research for secure GCN over vertical data  remains nascent.

Current MPC studies, for GCN or not, have primarily targeted settings where participants own different data samples, \ie, horizontally partitioned data~\cite{ccs/ZouLSLXX24}.
MPC specialized for scenarios where parties hold different types of features~\cite{tkde/LiuKZPHYOZY24,icml/CastigliaZ0KBP23,nips/Wang0ZLWL23} is rare.
This paper studies $2$-party secure GCN training for these vertical partition cases, where one party holds private graph topology (\eg, edges) while the other owns private node features.
For instance, LinkedIn holds private social relationships between users, while banks own users' private bank statements.
Such real-world graph structures underpin the relevance of our focus.
To our knowledge, no prior work tackles secure GCN training in this context, which is crucial for cross-silo collaboration.


To realize secure GCN over vertically split data, we tailor MPC protocols for sparse graph convolution, which fundamentally involves sparse (adjacency) matrix multiplication.
Recent studies have begun exploring MPC protocols for sparse matrix multiplication (SMM).
ROOM~\cite{ccs/SchoppmannG0P19}, a seminal work on SMM, requires foreknowledge of sparsity types: whether the input matrices are row-sparse or column-sparse.
Unfortunately, GCN typically trains on graphs with arbitrary sparsity, where nodes have varying degrees and no specific sparsity constraints.
Moreover, the adjacency matrix in GCN often contains a self-loop operation represented by adding the identity matrix, which is neither row- nor column-sparse.
Araki~\etal~\cite{ccs/Araki0OPRT21} avoid this limitation in their scalable, secure graph analysis work, yet it does not cover vertical partition.

% and related primitives
To bridge this gap, we propose a secure sparse matrix multiplication protocol, \osmm, achieving \emph{accurate, efficient, and secure GCN training over vertical data} for the first time.

\subsection{New Techniques for Sparse Matrices}
The cost of evaluating a GCN layer is dominated by SMM in the form of $\adjmat\feamat$, where $\adjmat$ is a sparse adjacency matrix of a (directed) graph $\graph$ and $\feamat$ is a dense matrix of node features.
For unrelated nodes, which often constitute a substantial portion, the element-wise products $0\cdot x$ are always zero.
Our efficient MPC design 
avoids unnecessary secure computation over unrelated nodes by focusing on computing non-zero results while concealing the sparse topology.
We achieve this~by:
1) decomposing the sparse matrix $\adjmat$ into a product of matrices (\S\ref{sec::sgc}), including permutation and binary diagonal matrices, that can \emph{faithfully} represent the original graph topology;
2) devising specialized protocols (\S\ref{sec::smm_protocol}) for efficiently multiplying the structured matrices while hiding sparsity topology.


 
\subsubsection{Sparse Matrix Decomposition}
We decompose adjacency matrix $\adjmat$ of $\graph$ into two bipartite graphs: one represented by sparse matrix $\adjout$, linking the out-degree nodes to edges, the other 
by sparse matrix $\adjin$,
linking edges to in-degree nodes.

%\ie, we decompose $\adjmat$ into $\adjout \adjin$, where $\adjout$ and $\adjin$ are sparse matrices representing these connections.
%linking out-degree nodes to edges and edges to in-degree nodes of $\graph$, respectively.

We then permute the columns of $\adjout$ and the rows of $\adjin$ so that the permuted matrices $\adjout'$ and $\adjin'$ have non-zero positions with \emph{monotonically non-decreasing} row and column indices.
A permutation $\sigma$ is used to preserve the edge topology, leading to an initial decomposition of $\adjmat = \adjout'\sigma \adjin'$.
This is further refined into a sequence of \emph{linear transformations}, 
which can be efficiently computed by our MPC protocols for 
\emph{oblivious permutation}
%($\Pi_{\ssp}$) 
and \emph{oblivious selection-multiplication}.
% ($\Pi_\SM$)
\iffalse
Our approach leverages bipartite graph representation and the monotonicity of non-zero positions to decompose a general sparse matrix into linear transformations, enhancing the efficiency of our MPC protocols.
\fi
Our decomposition approach is not limited to GCNs but also general~SMM 
by 
%simply 
treating them 
as adjacency matrices.
%of a graph.
%Since any sparse matrix can be viewed 

%allowing the same technique to be applied.

 
\subsubsection{New Protocols for Linear Transformations}
\emph{Oblivious permutation} (OP) is a two-party protocol taking a private permutation $\sigma$ and a private vector $\xvec$ from the two parties, respectively, and generating a secret share $\l\sigma \xvec\r$ between them.
Our OP protocol employs correlated randomnesses generated in an input-independent offline phase to mask $\sigma$ and $\xvec$ for secure computations on intermediate results, requiring only $1$ round in the online phase (\cf, $\ge 2$ in previous works~\cite{ccs/AsharovHIKNPTT22, ccs/Araki0OPRT21}).

Another crucial two-party protocol in our work is \emph{oblivious selection-multiplication} (OSM).
It takes a private bit~$s$ from a party and secret share $\l x\r$ of an arithmetic number~$x$ owned by the two parties as input and generates secret share $\l sx\r$.
%between them.
%Like our OP protocol, o
Our $1$-round OSM protocol also uses pre-computed randomnesses to mask $s$ and $x$.
%for secure computations.
Compared to the Beaver-triple-based~\cite{crypto/Beaver91a} and oblivious-transfer (OT)-based approaches~\cite{pkc/Tzeng02}, our protocol saves ${\sim}50\%$ of online communication while having the same offline communication and round complexities.

By decomposing the sparse matrix into linear transformations and applying our specialized protocols, our \osmm protocol
%($\prosmm$) 
reduces the complexity of evaluating $\numnode \times \numnode$ sparse matrices with $\numedge$ non-zero values from $O(\numnode^2)$ to $O(\numedge)$.

%(\S\ref{sec::secgcn})
\subsection{\cgnn: Secure GCN made Efficient}
Supported by our new sparsity techniques, we build \cgnn, 
a two-party computation (2PC) framework for GCN inference and training over vertical
%ly split
data.
Our contributions include:

1) We are the first to explore sparsity over vertically split, secret-shared data in MPC, enabling decompositions of sparse matrices with arbitrary sparsity and isolating computations that can be performed in plaintext without sacrificing privacy.

2) We propose two efficient $2$PC primitives for OP and OSM, both optimally single-round.
Combined with our sparse matrix decomposition approach, our \osmm protocol ($\prosmm$) achieves constant-round communication costs of $O(\numedge)$, reducing memory requirements and avoiding out-of-memory errors for large matrices.
In practice, it saves $99\%+$ communication
%(Table~\ref{table:comm_smm}) 
and reduces ${\sim}72\%$ memory usage over large $(5000\times5000)$ matrices compared with using Beaver triples.
%(Table~\ref{table:mem_smm_sparse}) ${\sim}16\%$-

3) We build an end-to-end secure GCN framework for inference and training over vertically split data, maintaining accuracy on par with plaintext computations.
We will open-source our evaluation code for research and deployment.

To evaluate the performance of $\cgnn$, we conducted extensive experiments over three standard graph datasets (Cora~\cite{aim/SenNBGGE08}, Citeseer~\cite{dl/GilesBL98}, and Pubmed~\cite{ijcnlp/DernoncourtL17}),
reporting communication, memory usage, accuracy, and running time under varying network conditions, along with an ablation study with or without \osmm.
Below, we highlight our key achievements.

\textit{Communication (\S\ref{sec::comm_compare_gcn}).}
$\cgnn$ saves communication by $50$-$80\%$.
(\cf,~CoGNN~\cite{ccs/KotiKPG24}, OblivGNN~\cite{uss/XuL0AYY24}).

\textit{Memory usage (\S\ref{sec::smmmemory}).}
\cgnn alleviates out-of-memory problems of using %the standard 
Beaver-triples~\cite{crypto/Beaver91a} for large datasets.

\textit{Accuracy (\S\ref{sec::acc_compare_gcn}).}
$\cgnn$ achieves inference and training accuracy comparable to plaintext counterparts.
%training accuracy $\{76\%$, $65.1\%$, $75.2\%\}$ comparable to $\{75.7\%$, $65.4\%$, $74.5\%\}$ in plaintext.

{\textit{Computational efficiency (\S\ref{sec::time_net}).}} 
%If the network is worse in bandwidth and better in latency, $\cgnn$ shows more benefits.
$\cgnn$ is faster by $6$-$45\%$ in inference and $28$-$95\%$ in training across various networks and excels in narrow-bandwidth and low-latency~ones.

{\textit{Impact of \osmm (\S\ref{sec:ablation}).}}
Our \osmm protocol shows a $10$-$42\times$ speed-up for $5000\times 5000$ matrices and saves $10$-2$1\%$ memory for ``small'' datasets and up to $90\%$+ for larger ones.

%!TEX root = gcn.tex
\section{Preliminary}

\paragraph{Notations.}
Table~\ref{tab:notation} summarizes the main notations.
$\Abb$ denotes an Abelian group.
$\Sbb_n$ denotes a permutation group of $n$ elements.
$\Mbb_{m,n}(\Rcal)$ denotes a matrix ring, which defines a set of $m\times n$ matrices with entries in a ring $\Rcal$, forming a ring under matrix addition and
%matrix 
multiplication.
$\Msf_{m \times n}=(\Msf[i,j])_{i,j=1}^{m,n}$ denotes an $m\times n$ matrix\footnote{
For simplicity, we omit the subscript of 
$\Msf_{m \times n}$ when the values of $m$ and $n$ are clear from the context.
Also, we write 
$\Msf = (\Msf[i, j])_{i, j = 1}^{n}$ if $m = n$.
} 
where row indices are $\{1, 2, \ldots, m\}$ and column indices are $\{1, 2, \ldots, n\}$, and $\Msf[i,j]$ is the value at the $i$-th row and $j$-th column.
$\Pi(\ ;\ )$ denotes a protocol execution between two parties, $\pp_0$ and $\pp_1$, where $\pp_0$'s inputs are the left part of `;' and $\pp_1$'s inputs are the right part of `;'.

\paragraph{Secret Sharing.} 
We use $2$-out-of-$2$ additive secret sharing over a ring, where
the floating-point
%input 
values are encoded to be fixed-point numbers $x\in \Zbb/2^f\Zbb$, with $L = 64$ bits representing decimals and $f = 18$ bits representing the fractional part~\cite{sp/MohasselZ17}.
Specifically, one party $\pp_0$ holds the share $\l x\r _0\in\Zbb $, while the other party $\pp_1$ holds the share $\l x\r_1\in\Zbb$ such that $x\cdot 2^f = \l x\r_0 + \l x\r_1$.
The shares can be arithmetic or binary.


%$e$ &The degree of the permutation group
%$\Abb$& Abelian group

\paragraph{Graph Convolutional Networks.}
GCN~\cite{iclr/KipfW17} has been proposed for training over graph data, using graph structure and node features as input.
Like most neural networks, GCN consists of multiple linear and non-linear layers.
Compared to CNN, GCN replaces convolutional layers with graph convolution layers
%to learn the graph topology 
(more details on GCN architecture and its training/inference are in Section~\ref{sec::secgcn}).
Graph convolution can be computed by SMM, often yielding many $0$-value results.

Let $\adjmat \in \Mbb_{\numnode,\numnode}(\Rcal)$ be a (normalized) adjacency matrix of a graph with $\numnode$ nodes and $\Xsf \in \Mbb_{\numnode, d}(\Rcal)$ be the feature matrix (with dimensionality $d$) of the nodes.
The graph convolution layer (with output dimensionality $k$) is defined as $\mathsf{Y} = \adjmat \feamat \weimat$,
where $\mathsf{Y} \in \Mbb_{\numnode, k}(\Rcal)$ is the output and $\weimat \in \Mbb_{d, k}(\Rcal)$ is a trainable parameter.
As matrix multiplication costs increase linearly with input size and $k \ll \numnode$ in practice, the challenge of secure GCN
%training 
lies in the
%secure 
SMM of $\adjmat \feamat$.
Multiplying (dense) $\weimat$ can be done using Beaver's approach.

\begin{table}[!t]
\centering
\caption{Notation and Definition}
\label{tab:notation}
\setlength\tabcolsep{2pt}
\begin{tabular}{l|l}
\hline
%\textbf{Notation} & \textbf{Definition}
%\\\hline
$\pp_i, \l \ \r_i$& Party $i$ and its share ($i \in \{0, 1\})$
\\\hline
$\bitlen$ & The bit-length of data
\\\hline
$\pi,b,\uvec,\l u \r$ & Pre-computed randomnesses
%$\pi,b,\uvec,u$ & \multirow{2}{*}{Offline-generated randomness (\eg., vectors, values)}
\\\hline
$\delta_x$ & Masked version of value/vector $x$
\\\hline
$\Mbb_{m, n}{(\Rcal)}$ & A set of ${m\times n}$ matrices with entries in a ring $\Rcal$
\\\hline
$\Msf_{m,n}$ & A matrix $\Msf$ of size $m\times n$
\\\hline
$\Msf[i, j]$ & The value of $\Msf$ at the $i$-th row and $j$-th column
\\\hline
$\sigma\xvec$& Permutation operation $\sigma$ over a matrix/vector $\xvec$
\\\hline
$\Sbb_n$ & Permutation group of $n$ elements
\\\hline
\end{tabular}
\end{table}

\section{Dataset Construction}\label{sec:benchmark-construction}
\section{Problem definition}
\label{sec:problem-def}

The first step in building and using a \CSE{} or SciML model is defining the problem scope: the model's intended purpose, application domain and operating environment, required quantities of interest (QoI) and their scales, and how prior knowledge informs model conceptualization.

\subsection{Model purpose}

\begin{essrec}[Specify prior knowledge and model purpose]
Define the model's intended use and document the essential model properties that must be satisfied. Document any known limitations and constraints of the chosen approach. This ensures appropriate data selection and physics-informed objectives while preventing model misuse outside its intended scope.
\end{essrec}

A SciML model's purpose, as discussed in Section~\ref{sec:scope}, dictates all subsequent modeling choices.
This purpose determines required outer-loop processes and essential properties.

To highlight the importance of specifying the target outer-loop process, consider a model used for explanatory modeling. An explanatory model must simulate all system processes, like ice-sheet thickness and velocity evolution. In contrast, a risk assessment model needs only decision-relevant quantities, like ice-sheet mass loss under varying emissions scenarios. Design and control models, meanwhile, have different requirements than those for risk assessment.
The model purpose dictates the data types and formulations needed to train a SciML model. The impact of this purpose on data requirements and physics-informed objectives varies by application domain. Thus, the exact model formulation should be chosen in light of these problem-specific considerations.


\subsection{Verification, calibration, validation and application domain}

\begin{essrec}[Specify verification, calibration, validation, and application domains]
Define the specific conditions under which the model will operate across the verification, calibration, validation, and prediction phases. These domains are specified by relevant boundary conditions, forcing functions, geometry, and timescales. Account for potential differences between these domains and address any data distribution shifts that could affect model performance. This ensures the selected model architecture and training data align with the intended use while preventing unreliable predictions when operating outside validated conditions.
\end{essrec}

The trustworthy development and deployment of a model (see Figure~\ref{fig:model-development}) requires using the model in regards to verification, calibration, validation, and application domains. These domains are defined by the conditions under which the model operates during these respective phases and must be defined before model construction because they determine viable model classes. Key features include boundary conditions, forcing functions, geometry, and timescales. For ice sheets, examples include surface mass balance, land mass topography, and ocean temperatures.

Each domain will often require the prediction of different quantities of interest under different conditions. Moreover the complexity of the processes being modeled typically increase when transitioning from verification to calibration, to validation to prediction. Additionally the amount of data to complement or inform the model decreases as we move through these domains. For example, the verification domain for our ice-sheet examplar predicts the entire state of the ice-sheet for simple manufactured or analytical solutions. The calibration domain predicts Humboldt Glacier surface velocity under steady-state preindustrial conditions. The validation domain predicts grounding-line change rates from the first decade of this century. The application domain predicts glacier mass change in 2100. Figure~\ref{fig:computational-domains} illustrates these distinct domains. When transitioning between domains, data shifts across domains must be considered. For example, a model trained only on calibration data from recent ice-sheet forcings may fail to predict ice-sheet properties under different future conditions.

\begin{figure}[htb]
    \centering
    \includegraphics[width=0.65\linewidth]{application-domain.pdf}
    \caption{Verification, validation, calibration and application domains.}
    \label{fig:computational-domains}
\end{figure}


\subsection{Quantities of interest}

\begin{essrec}[Carefully select and specify the quantities of interest]
Select and specify the model outputs (quantities of interest, QoI) required for the intended use, considering their form and scale. For risk assessment and design applications, identify the minimal set of QoIs needed for decision-making or optimization. For explanatory modeling, specify the broader range of QoIs needed to capture system behavior. This choice fundamentally determines the required model complexity, training data requirements, and computational approach needed to achieve reliable predictions.
\end{essrec}

Quantities of interest (QoI) are the model outputs required by users. Their form and scale depend on modeling purpose and application domain. We now discuss key considerations for QoI selection.

Risk assessment requires reproducing only decision-critical QoI. For ice-sheets, these include sea-level rise from mass loss and infrastructure damage costs. Design applications similarly need few QoI to evaluate objectives and constraints, like thermal and structural stresses in aerospace vehicles. Design models need accurate QoI predictions only along optimizer trajectories\footnote{For each design iteration the model may still need to be accurate across all uncertain model inputs}, while risk assessment models must predict across many conditions. Explanatory modeling demands more extensive QoI sets, such as complete ice-sheet depth and velocity fields for studying calving. Therefore, simple surrogates often suffice for risk assessment and design, but explanatory modeling may require operators or reduced order models.


\subsection{Model conceptualization}


\begin{essrec}[Select and document model structure]
Select a model structure that fits the model's purpose, domain, and quantities of interest based on relevant prior knowledge such as conservation laws or system properties. Document the alternative model structures considered and the reasoning behind the final selection, including how available resources and computational constraints influenced the choice. This systematic approach ensures the model balances usability, reliability, and feasibility while maintaining transparency about structural assumptions and limitations.
\end{essrec}

Model conceptualization, which follows problem definition, involves selecting model structure based on prior knowledge. While essential to \CSE{} model development~\cite{Jakeman_LN_EMS_2006}, this step requires clear identification of the application domain and relevant QoI.

Model structure selection draws on key prior knowledge: conservation laws, system invariances like rotational and translational symmetries. These guide method selection---for example, symplectic time integrators~\cite{ruth1983canonical} preserve system dynamics properties. Moreover, this knowledge informs and justifies the selection of candidate model structures.
A \CSE{} modeler chooses between model types like lumped versus distributed PDE models, and linear versus nonlinear PDEs. The optimal choice depends on application domain, QoI, and available resources. For example, linear PDEs may introduce more error but their lower computational cost enables better error and uncertainty characterization for tasks like optimal design.
Similar considerations guide SciML model selection. For example, Gaussian processes excel at predicting scalar QoI with few inputs and limited data, but become intractable for larger datasets without variational inference approximations~\cite{Liu_CO_KBS_2018}. In contrast, deep neural networks handle high-dimensional data but require large datasets. The intended use also shapes model structure and training, e.g., optimization applications require controlling derivative errors~\cite{bouhlel2020scalable,o2024derivative} to ensure convergence~\cite{cao2024lazy,luo2023efficient}. These approximation errors must be understood and quantified where possible.

\CSE{} has a strong history of using prior knowledge to formulate governing equations for complex phenomena and deriving numerical methods that respect important physical properties. However, all models are approximate and the best model must balance usability, reliability, and feasibility~\cite{Hamilton_PSFJEMS_2022}. While SciML methods can be usable and feasible, more attention is needed to establish their trustworthiness. In the following two sections, we discuss how \CSE{} V\&V can improve the trustworthiness of SciML research.

\section{Verification}
\label{sec:verification}

Verification increases the trustworthiness of numerical models by demonstrating that the numerical method can adequately solve the equations of the desired mathematical model and the code correctly implements the algorithm. Verification consists of code verification and solution verification, which enhance credibility and trust in the model's predictions. Code and solution verification are well-established in \CSE{} to reduce algorithmic errors. However, verification for SciML models has received less attention due to the field's young age and unique challenges. Moreoever, because SciML models heavily rely on data, unlike \CSE{} models, existing \CSE{} verification notions need to be adapted for SciML.

\subsection{Code verification}
\label{sec:code-verification}

\begin{essrec}[Verify code implementation with test problems]
Evaluate the SciML model's accuracy on simple manufactured test problems using verification data that is independent from training data. Assess how the model error responds to variations in training data samples and optimization parameters while increasing both model complexity and training data size. This systematic testing approach reveals implementation issues, quantifies the impact of sampling and optimization choices, and establishes confidence in the model's numerical implementation.
\end{essrec}

Code verification ensures that a computer code correctly implements the intended mathematical model. For \CSE{} models, this involves confirming that numerical methods and algorithms are free from programming errors (``bugs"). PDE-based \CSE{} models commonly use the method of manufactured solutions (MMS) to verify code on arbitrarily complex solutions. MMS substitutes a user-provided solution into the governing equations, then differentiates it to obtain the exact forcing function and boundary conditions. These solutions check if the code produces the known theoretical convergence rate as the numerical discretization is refined. If the observed order of convergence is less than theoretical, causes such as software bugs, insufficient mesh refinement, or singularities and discontinuities affecting convergence must be identified.

Code verification for SciML models is important but challenging due to the large role of data and nonconvex numerical optimization. Three main challenges limit code verification for many SciML models.
First, while theoretical analysis of SciML models is increasing~\cite{schwab2019deep,schwab2023deep,opschoor2022exponential,leshno1993multilayer,lanthaler2023curse,kovachki2021universal,kovachki2023neural}, many models like neural networks do not generally admit known convergence rates outside specific map classes~\cite{schwab2019deep,schwab2023deep,opschoor2022exponential,herrmann2024neural}, despite their universal approximation properties~\cite{hornik1989multilayer,cybenko1989approximation,leshno1993multilayer}.
Second, generalizable procedures to refine models, such as consistently refining neural-network width and depth as data increases, do not exist.
Finally, regardless of data amount and model unknowns, modeling error often plateaus at a much higher level than machine precision due to nonconvex optimization issues like local minima and saddle points~\cite{Dauphin_PGCGB_NIPS_2014,Bottouleon_CN_SIAMR_2018}.

Developing theoretical and algorithmic advances to address the three main challenges limiting code verification can substantially improve the trustworthiness of SciML models. Convergence-based code verification is currently possible only for certain SciML models with theory that bounds approximation errors in terms of model complexity and training data amount, such as operator methods~\cite{Turnage_et_al_arxiv_2024}, polynomial chaos expansions~\cite{Cohen_M_SMAIJCM_2017,xiu2002wiener}, and Gaussian processes~\cite{Burt_RV_PMLR_2019}.

For SciML models without supporting theory, convergence tests should still be conducted and reported. Studies providing evidence of model convergence engender greater trustworthiness than those that do not, even when the empirically estimated convergence rate cannot be compared to theoretical rates. For example, observing Monte Carlo-type sampling rates in a regime of interest for a fixed overparametrized model can provide intuition into whether the model should be enhanced.

To account for the heavy reliance of SciML models on training data optimization, code verification should be adapted in two ways.
First, report errors in the ML model for a given complexity and data amount for different realizations of the training data to quantify the impact of sampling error, which is not present in \CSE{} models.
Second, because most SciML algorithms introduce optimization error, conduct verification studies that artificially generate data from a random realization of an ML model, then compare the recovered parameter values with the true parameter values or compare the predictions of the learned and true approximations, or at the very least compare the predictions of the two models. Additionally, quantify the sensitivity of the SciML model error to randomness in the optimization by varying the random seed and initial guess passed to the optimizer (see Section~\ref{sec:loss-and-opt}).
All verification tests must employ test data or \emph{verification data}, independent of the training data, to measure the accuracy of the SciML model.


\subsection{Solution verification}

\begin{essrec}[Verify solution accuracy with realistic benchmarks]
Test the model's performance on well-designed, realistic benchmark problems that reflect the intended application domain. Quantify how the model error varies with different training data samples and optimization parameters. When feasible, examine error patterns across different model complexities and data amounts; otherwise, focus on verifying the specific configuration intended for deployment. This ensures the model meets accuracy requirements under realistic conditions while accounting for uncertainties in training and optimization.
\end{essrec}

Code verification establishes a code's ability to reproduce known idealized solutions, while solution verification, performed after code verification, assesses the code's accuracy on more complex yet tractable problems defined by more realistic boundary conditions, forcing, and data. For example, code verification of ice sheets may use manufactured solutions, whereas solution verification may use more realistic MISMIP benchmarks~\cite{Cornford_et_al_TC_2020}. In solution verification, the numerical solution cannot be compared to a known exact solution, and the convergence rate to a known solution cannot be established. Instead, solution verification must use other procedures to estimate the error introduced by the numerical discretization.

Solution verification establishes whether the exact conditions of a model result in the expected theoretical convergence rate or if unexpected features like shocks or singularities prevent it. The most common approach for \CSE{} models compares the difference between consecutive solutions as the numerical discretization is refined and uses Richardson extrapolation to estimate errors. A posteriori error estimation techniques that require solving an adjoint equation can also be used.

While thorough solution verification of CSE models is challenging, these difficulties are further amplified for SciML models. Currently, solution verification of SciML models simply consists of evaluating a trained model's performance using test data separate from the training data. However, this is insufficient as solution verification requires quantifying the impact of increasing data and model complexity on model error. Yet, unfortunately, performing a posteriori error estimation for many SciML models using techniques like Richardson extrapolation is difficult due to the confounding of model expressivity, statistical sampling errors, and variability introduced by converging to local solutions or saddle points of nonconvex optimization, making it challenging to monotonically decrease the error of SciML models such as neural networks. 

Until convergence theory for SciML models improves and automated procedures are developed to change SciML model hyperparameters as data increases, solution verification of SciML models should repeat the sensitivity tests proposed for code verification (Section~\ref{sec:code-verification}) with two key differences:
First, verification experiments used to generate verification data must be specifically designed for solution verification, as not all verification data equally informs solution verification efforts, similar to observations made when creating validation datasets for \CSE{} models~\cite{Oberkampf_T_PAS_2002}. See Section~\ref{sec:data-sources} for more information on important properties of verification benchmarks.
Second, while ideally the convergence of SciML errors on realistic benchmarks should be investigated, it may be computationally impractical. Thus, solution verification should prioritize quantifying errors using the model complexity and data amount that will be used when deploying the SciML model to its application domain.

\section{Validation}
\label{sec:validation}

Verification establishes if a model can accurately produce the behavior of a system described by governing equations. In contrast, validation assesses whether a \CSE{} model's governing equations---or data for SciML models---and the model's implementation can reproduce the physical system's important properties, as determined by the model's purpose.

Validation requires three main steps: (1) solve an inverse problem to calibrate the model to observational data; (2) compare the model's output with observational data collected explicitly for validation; and (3) quantify the uncertainty in model predictions when interpolating or extrapolating from the validation domain to the application domain. We will expand on these steps below.
But first note that the issues affecting the verification of SciML models also affect calibration and validation. Consequently, we will not revisit them here but rather will highlight the unique challenges in validating SciML models.

\subsection{Calibration}

\begin{essrec}[Perform probabilistic calibration]
Calibrate the trained SciML model using observational data to optimize its predictive accuracy for the application domain. Implement Bayesian inference when possible to generate probabilistic parameter estimates and quantify model uncertainty. Choose calibration metrics that account for both model and experimental uncertainties, and select calibration data strategically to maximize information content within experimental constraints. This approach enables reliable uncertainty estimation and optimal use of available observational data.
\end{essrec}

Once a \CSE{} model has been verified, it must be calibrated to match experimental data that contains observational noise. This calibration requires solving an inverse problem~\cite{Stuart_AN_2010}, which can be either deterministic or statistical (e.g., Bayesian). The deterministic approach formulates the inverse problem as a (nonlinear) optimization problem that minimizes the mismatch between model and experimental data. This formulation requires regularization to ensure well-posedness, typically chosen using the L-curve~\cite{hansen1999curve} or the Morozov discrepancy principle~\cite{anzengruber2009morozov}. The Bayesian approach replaces the misfit with a likelihood function based on the noise model, while using a prior distribution for regularization. This prior distribution ensures well-posedness while encoding typical parameter ranges and correlation lengths. We recommend Bayesian methods for calibration because they provide insight into the uncertainty of the reconstructed model parameters. 

The calibration of SciML operator, reduced-order, and hybrid CSE-SciML models is distinct from SciML training and follows similar principles to \CSE{} model calibration. These models are first trained using simulation data for solution verification. Next, observational data (called \emph{calibration data}) determines the optimal model input values that match experimental outputs. For instance, calibrating a SciML ice-sheet model such as that of Ref.~\cite{He_PKS_JCP_2023} requires finding optimal friction field parameters of a trained SciML model, which best predict observed glacier surface velocities, given the noise in the observational data.

Calibration typically improves a model's predictive accuracy on its application domain, but the informative value of calibration data varies significantly. Therefore, researchers should select calibration data strategically to maximize information content within their experimental budget. See Section~\ref{sec:data-sources} for further discussion on collecting informative data.

\subsection{Model validation}

\begin{essrec}[Validate model against purpose-specific requirements]
Define validation metrics that align with the model's intended purpose. Then validate the model using independent data that was not used for training or calibration, ensuring it captures essential physics and boundary conditions of interest. If validation reveals inadequate performance, iterate by collecting additional training data, refining the model structure, or gathering more calibration data until the model achieves satisfactory accuracy for its intended application. This systematic approach will help ensure the model meets stakeholder requirements while maintaining scientific rigor.
\end{essrec}

Model validation is the ``substantiation that a model within its domain of applicability possesses a satisfactory range of accuracy consistent with the intended application of the model''~\cite{Refsgaard_H_AWR_2004}. Validation involves comparing computational results with observational data, then determining if the agreement meets the model's intended purpose~\cite{Lee_et_all_AIAA_2016}. For \CSE{} models with unacceptable validation agreement, modelers must either collect additional calibration data or refine the model structure until reaching acceptable accuracy. SciML models follow a similar iterative process but offer an additional option: to collect more training data.

Model validation must occur after calibration and requires independent data not used for calibration or training. For our conceptual ice-sheet model, calibration matches surface velocities assumed to represent pre-industrial conditions, while validation assesses the calibrated model's ability to predict grounding line change rates at the start of this decade. Performance metrics must target the specific modeling purpose. For optimization tasks, metrics should measure the distance from true optima obtained via the SciML model or bound the associated error~\cite{cao2024lazy}. For uncertainty estimation, metrics should quantify errors in uncertainty statistics through moment discrepancies or density-based measures like (shifted) reverse and forward Kullback--Leibler divergences.
For explanatory SciML modeling, validation metrics must also assess physical fidelity: adherence to physical laws, conservation properties (such as mass and energy), and other constraints. As with verification, the validation concept should encompass \emph{data validation}, particularly whether training data adequately represents the application space.

Validation determines whether a model is acceptable for its specific purpose rather than universally correct. The definition of acceptable is subjective, depending on validation metrics and accuracy requirements established by model stakeholders in alignment with the problem definition and model purpose (see Section~\ref{sec:problem-def}). Moreoever, validation itself does not constitute final model acceptance, which must be based on model accuracy in the application domain, as discussed in Section~\ref{sec:prediction}.

Two additional considerations complete our discussion of model validation. First, this validation differs from the concept of \emph{cross validation}, which estimates ML model accuracy on data representative of the training domain during development. The validation described here assesses accuracy in a separate validation domain. Second, validation data varies in informative value. Validation experiments should ``capture the essential physics of interest, including all relevant physical modeling data and initial and boundary conditions required by the code''~\cite{Oberkampf_T_NED_2008}. Most critically, validation data must remain independent from training and calibration data. 

\subsection{Prediction}
\label{sec:prediction}

\begin{essrec}[Quantify prediction uncertainties]
Assess and quantify all sources of uncertainty affecting model predictions in the application domain, including numerical approximation errors, input and parameter uncertainties, sampling errors from finite training data, and optimization errors. Propagate these uncertainties through the model using appropriate techniques to estimate relevant statistics that match validation criteria. Define acceptance thresholds for prediction uncertainty to ensure the model's reliability for its intended use while acknowledging inherent limitations in uncertainty quantification.
\end{essrec}

Although extensive data may be available for model calibration, validation data is typically scarcer and may not represent the model's intended application domain. According to Schwer~\cite{Schwer_EWC_2007}, ``The original reason for developing a model was to make predictions for applications of the model where no experimental data could, or would, be obtained.'' Therefore, minimizing validation metrics at nominal conditions cannot sufficiently validate a model. Modelers must also quantify accuracy and uncertainty when predictions are extrapolated to the application domain.

SciML models, like \CSE{} models, are subject to numerous sources of uncertainty. The impact of these uncertainties on model predictions must be quantified. Several sources of uncertainty affect \CSE{} models. These include: numerical errors, from approximating the solution to governing equations; input uncertainty arises, which is caused by inexact knowledge of model inputs; parameter uncertainty, which stems from inexact knowledge of model coefficients; and model structure error representing the difference between the model and reality. SciML models contain all these uncertainties. They also incorporate additional uncertainties from sampling and optimization errors, as discussed previously.

Sampling error arises from training a model with a finite amount of possibly noisy data. For a fixed ML model structure and zero optimization error, this error decreases as the amount of data increases. Optimization error represents the difference between the optimized solution, which is often a local optimum, and the global solution for fixed training data. Optimization error can enter \CSE{} models during calibration. Optimization error affects SciML models more significantly because it occurs both during calibration and training. Linear approximations, for example, based on polynomials, achieve zero optimization error during training to machine precision. However, nonlinear approximations such as neural networks often produce non-trivial optimization errors. Stochastic gradient descent demonstrates this by producing different parameter estimates due to stochastic optimization randomness and initial guesses.

The identified sources of modeling uncertainty require parameterization for sampling. Expert knowledge typically guides the construction of prior distributions that represent parametric uncertainty. This parameterization should occur during problem definition. Bayesian calibration updates these priors into posterior distributions using calibration data. The model must then propagate all uncertainties onto predictions in the application domain. Monte Carlo quadrature accomplishes this propagation by drawing random samples from the uncertainty distributions. The method collects model predictions at these samples and computes empirical estimates of important statistics defined by validation criteria, such as mean and variance.

We emphasize the impact of all sources of error and uncertainty must be quantified. Simply estimating the impact of error caused by using finite sample sets, for example estimated by generative models such as variational autoencoders of Gaussian processes is insufficient. Moreover, complete elimination of uncertainty is impossible. Consequently, model acceptance, like validation, must rely on subjective accuracy criteria established through stakeholder communication. For instance, acceptance criteria for predicted sea-level change from melting ice-sheets at year 2100 may specify that prediction precision reaches 1\% of the mean value. Yet, engineering applications, such as those focused on aerospace design, may have much higher accuracy requirements.

The aforementioned Monte-Carlo based UQ procedure effectively quantifies the impact of parameterized uncertainties on model predictions. However, model structure error remains difficult to parameterize in both SciML and \CSE{} modeling. Validation can partially assess model structure error. However, experiments rarely cover all conditions of use. Specifically, validation tests only the model's interpolation ability within the convex hull of available data and assumptions. This limitation creates challenges when applying the model outside its validation domain. Some progress exists in quantifying extrapolation error for ``models based upon highly-reliable theory that is augmented with less-reliable embedded models''~\cite{Oliver_TSM_CMAME_2015}. However, such hybrid CSE-SciML models rely on well-established physics-based governing equations to support extrapolation confidence. Pure SciML models still require substantial research to develop reliable methods for estimating model structure uncertainty.



To build \benchname, we design a pipeline implementing an automated workflow in \Cref{fig:workflow} to collect user-submitted solutions on LeetCode and distill pseudocode solutions from them using a recent reasoning model DeepSeek-R1. The pipeline may also be adapted to refurbish other existing code generation benchmarks.

\smalltitle{Data Source}
To lessen the data leakage threat, we select user-submitted solutions based on the problems most recently collected by \livecb~\cite{livecb}. These are the latest programming problems released after the training cut-off dates of popular LLMs.
In other words, we select the most recent subset of problems indexed by \livecb at LeetCode. We further collect the 
corresponding user-submitted solutions from LeetCode.
For each problem, we manually collect the most popularly voted solutions in Python, C++, and Rust, respectively.


\smalltitle{Task Cleaning}
To ensure the correctness of the collected user-submitted solutions,
we run each solution via the LeetCode online judge to ensure the solution passes all mandated tests.
If the most popularly voted solutions fail (usually due to the update of problems/tests), we collect another solution that passes the updated tests.
The study of our research questions requires evaluating the correctness of many generated codes. Submitting all of them to the LeetCode online judge for correctness validation is inappropriate. Therefore, we collect the published tests deduced by \livecb and use them to evaluate the correctness of the generated codes in our study. However, these \livecb tests are deduced by LLMs and subject to noises. %
We consider a deduced \livecb test noisy if it fails the collected solutions. %
In total, we find 16 noisy instances and exclude them from our study. 
After cleaning, we collect 365 solutions in C++ and Python and 351 solutions in Rust.

\smalltitle{Code to Pseudocode}
Each pseudocode used to evaluate the coding capability of LLMs is generated by the reasoning model \dsr~\cite{ds-r1} given a solution code and a detailed list of rules (\Cref{sec:prompts}) that the output pseudocode needs to satisfy, i.e., the criteria in \Cref{subsec:criteria}.
For example, the pseudocode should not contain explicit types like 32-bit or 64-bit integers and language-specific operations like \codef{yield} in Python.

We choose a reasoning model over a chat model like GPT-4o. Our pilot experiments find that chat models often fail to obey the rules in a long context or just write the pseudocode line by line without undergoing a substantial thinking process.
The prompt we use consists of only the user query without a system message or few-shot examples, as suggested by the DeepSeek team~\cite{ds-r1}.
We also follow their experiment setting (\codef{temperature=0.6}, \codef{top\_p=0.95}).
One pseudocode sample is obtained for each selected user-submitted solution due to the limited access to the R1 service and the incurred time latency.

\smalltitle{Pseudocode Quality Assessment}
{To remove incorrect R1-generated pseudocode, we use LLMs to generate code from the R1-generated pseudocode using our study setup and remove the tasks where \emph{NO LLMs} can pass the task with ten attempts.
Finally, we remove 22 subjects where R1 hallucinates a pseudocode with incorrect logic (e.g., adding an incorrect condition), and keep 1,059 subjects.
Besides, we compare the lengths and effectiveness of pseudocode annotated by R1 and humans for randomly sampled subjects in RQ4. The results also suggest good quality of the retained pseudocode.} %

\section{Simulations and Experiment}
In this section, we conduct comprehensive experiments in both simulation and the real-world robot to address the following questions:
\begin{itemize}[leftmargin=*]
    \item \textbf{Q1(Sim)}: How does the \our policy perform in tracking across different commands?
    \item  \textbf{Q2(Sim)}: How to reasonably combine various commands in the general command space? % Command Analysis
    \item \textbf{Q3(Sim)}: How does large-scale noise intervention training help in policy robustness? % Ablation Study
    \item \textbf{Q4(Real)}: How does \our behave in the real world? % Real World Demo
\end{itemize}

\noindent\textbf{Robot and Simulator.} 
Our main experiments in this paper are conducted on the Unitree H1 robot, which has 19 Degrees of Freedom (DOF) in total, including 
two 3-DOF shoulder joints, two elbow joints, one waist joint, two 3-DOF hip joints, two knee joints, and two ankle joints.
The simulation training is based on the NVIDIA IsaacGym simulator~\citep{makoviychuk2021isaac}. It takes 16 hours on a single RTX 4090 GPU to train one policy.

\noindent\textbf{Command analysis principle and metric.}
One of the main contributions of this paper is an extended and general command space for humanoid robots. Therefore, we pay much attention to command analysis (regarding Q1 and Q2). This includes analysis of single command tracking errors, along with the combination of different commands under different gaits.
% we categorize the commands into three groups: \emph{movement}, \emph{foot}, and \emph{posture}. The \emph{movement} commands include the linear velocity and angular velocity, forming the foundational locomotion commands and are considered the most critical aspect of the tasks. The \emph{foot} commands include the gait frequency and foot swing height, representing the mode of leg movement. The \emph{posture} commands include body height, body pitch and waist yaw, which determine the desired body posture.
For analysis, we evaluate the averaged episodic command tracking error (denoted as $E_\text{cmd}$), which measures the discrepancy between the actual robot states and the command space using $L_1$ norm.
% The tracking error is measured in units of $m/s$, $rad/s$, $Hz$, $m$, and $rad$, corresponding to linear velocity, angular velocity, frequency, position, and rotation, respectively.
All commands are uniformly sampled within a pre-defined command range, as shown in \tb{tab:commands}\footnote{Note that the hopping gait keeps a different command range, due to its asymmetric type of motion. More details can be referred to \ap{ap:Hopping}.}.

%%%%%%%%%%%---SETME-----%%%%%%%%%%%%%
%replace @@ with the submission number submission site.
\newcommand{\thiswork}{INF$^2$\xspace}
%%%%%%%%%%%%%%%%%%%%%%%%%%%%%%%%%%%%


%\newcommand{\rev}[1]{{\color{olivegreen}#1}}
\newcommand{\rev}[1]{{#1}}


\newcommand{\JL}[1]{{\color{cyan}[\textbf{\sc JLee}: \textit{#1}]}}
\newcommand{\JW}[1]{{\color{orange}[\textbf{\sc JJung}: \textit{#1}]}}
\newcommand{\JY}[1]{{\color{blue(ncs)}[\textbf{\sc JSong}: \textit{#1}]}}
\newcommand{\HS}[1]{{\color{magenta}[\textbf{\sc HJang}: \textit{#1}]}}
\newcommand{\CS}[1]{{\color{navy}[\textbf{\sc CShin}: \textit{#1}]}}
\newcommand{\SN}[1]{{\color{olive}[\textbf{\sc SNoh}: \textit{#1}]}}

%\def\final{}   % uncomment this for the submission version
\ifdefined\final
\renewcommand{\JL}[1]{}
\renewcommand{\JW}[1]{}
\renewcommand{\JY}[1]{}
\renewcommand{\HS}[1]{}
\renewcommand{\CS}[1]{}
\renewcommand{\SN}[1]{}
\fi

%%% Notion for baseline approaches %%% 
\newcommand{\baseline}{offloading-based batched inference\xspace}
\newcommand{\Baseline}{Offloading-based batched inference\xspace}


\newcommand{\ans}{attention-near storage\xspace}
\newcommand{\Ans}{Attention-near storage\xspace}
\newcommand{\ANS}{Attention-Near Storage\xspace}

\newcommand{\wb}{delayed KV cache writeback\xspace}
\newcommand{\Wb}{Delayed KV cache writeback\xspace}
\newcommand{\WB}{Delayed KV Cache Writeback\xspace}

\newcommand{\xcache}{X-cache\xspace}
\newcommand{\XCACHE}{X-Cache\xspace}


%%% Notions for our methods %%%
\newcommand{\schemea}{\textbf{Expanding supported maximum sequence length with optimized performance}\xspace}
\newcommand{\Schemea}{\textbf{Expanding supported maximum sequence length with optimized performance}\xspace}

\newcommand{\schemeb}{\textbf{Optimizing the storage device performance}\xspace}
\newcommand{\Schemeb}{\textbf{Optimizing the storage device performance}\xspace}

\newcommand{\schemec}{\textbf{Orthogonally supporting Compression Techniques}\xspace}
\newcommand{\Schemec}{\textbf{Orthogonally supporting Compression Techniques}\xspace}



% Circular numbers
\usepackage{tikz}
\newcommand*\circled[1]{\tikz[baseline=(char.base)]{
            \node[shape=circle,draw,inner sep=0.4pt] (char) {#1};}}

\newcommand*\bcircled[1]{\tikz[baseline=(char.base)]{
            \node[shape=circle,draw,inner sep=0.4pt, fill=black, text=white] (char) {#1};}}

\subsection{Single Command Tracking}
We first analyze each command separately while keeping all other commands held at their default values. The results are shown in \tb{tab:Single commands}.
It is easily observed that the tracking errors in the walking and standing gaits are significantly lower than those in the jumping and hopping, with hopping exhibiting the largest tracking errors.
For hopping gaits, the robot may fall during the tracking of specific commands, like high-speed tracking, body pitch, and waist-yaw control.
This can be attributed to the fact that hopping requires rather high stability. Moreover, the complex postures and motions further exacerbate the risk of instability. Consequently, the policy prioritizes learning to maintain the balance, which, to some extent, compromises the accuracy of command tracking.

We conclude that the tracking accuracy of each gait aligns with the training difficulty of that gait in simulation. For example, the walking and standing patterns can be learned first during training, while the jumping and hopping gaits appear later and require an extended training period for the robot to acquire proficiency.
Similarly, the tracking accuracy of robots under low velocity is significantly better than those under high velocity, since 1) the locomotion skills under low velocity are much easier to master, and 2) the dynamic stability of the robot decreases at high speeds, leading to a trade-off with tracking accuracy.

We also found that the tracking accuracy for longitudinal velocity commands $v_x$ surpasses that of horizontal velocity commands $v_y$, which is due to the limitation of the hardware configuration of the selected Unitree H1 robots. In addition, the {foot swing height} $l$ is the least accurately tracked.
Furthermore, the tracking reward related to foot placement outperforms the tracking performance associated with posture control, since adjusting posture introduces greater challenges to stability. In response, the policy adopts more conservative actions to mitigate balance-threatening postural changes.
% In contrast, the influence of foot placement on stability is comparatively less pronounced, allowing for more precise tracking.

\begin{table}[t]
\setlength{\abovecaptionskip}{0.cm}
\setlength{\belowcaptionskip}{-0.cm}
\centering
\caption{\small \textbf{Single command tracking error.} The tracking errors for foot commands are calculated over a complete gait cycle, and the remaining ones are over one environmental step. For standing gait, we only tested the body height, body pitch, and waist yaw tracking error. $E^\text{high}$ and $E^\text{low}$ represents high-speed ($v_x > 1m/s$) and low-speed ($v_x \le 1m/s$) modes categorized by the linear velocity $v$. 
The tracking error is computed by sampling each command in a predefined range (\tb{tab:commands}) while keeping all other commands held at their default values.}
\label{tab:Single commands}
\resizebox{\columnwidth}{!}{
\begin{tabular}{@{}c|cccc|cc|ccc@{}}
\toprule
\multirow{3}{*}{Gait} & \multicolumn{4}{c|}{Movement} & \multicolumn{2}{c|}{Foot} & \multicolumn{3}{c}{Posture} \\
\cmidrule(l){2-5} \cmidrule{6-7} \cmidrule{8-10} 
& \multirow{2}{*}{\makecell{$E_{v_x}^\text{low}$\\($m/s$)}} & \multirow{2}{*}{\makecell{$E_{v_x}^\text{high}$\\($m/s$)}} & \multirow{2}{*}{\makecell{$E_{v_y}$\\($m/s$)}} & \multirow{2}{*}{\makecell{$E_{\omega}$\\$rad/s$}} & \multirow{2}{*}{\makecell{$E_{f}$\\($HZ$)}} & \multirow{2}{*}{\makecell{$E_{l}$\\($m$)}} & \multirow{2}{*}{\makecell{$E_{h}$\\($m$)}}  & \multirow{2}{*}{\makecell{$E_{p}$\\($rad$)}} & \multirow{2}{*}{\makecell{$E_{w}$\\($rad$)}}   \\ 
&  &  &  &  &  &  &  &  &    \\ 
\midrule
Standing  & - & - & - & - & - & - & 0.035 & 0.047 & 0.022  \\
Walking   & 0.030 & 0.216 & 0.085 & 0.054 & 0.028 & 0.011 & 0.064 & 0.038 & 0.075  \\
Jumping  & 0.090 & 0.532 & 0.069 & 0.077 & 0.027 & 0.012 & 0.058 & 0.048 & 0.022 \\
Hopping   & 0.033 & - & 0.046 & 0.078 & - & - & 0.103 & - & - \\
\bottomrule
\end{tabular}}
\end{table}



\begin{table*}[t]
\setlength{\abovecaptionskip}{0.cm}
\setlength{\belowcaptionskip}{-0.cm}
\centering
\caption{\small \textbf{Tracking errors with different intervention strategies under the walking gait}. We evaluate three upper-body intervention training strategies: Noise (\our), the AMASS dataset, and no intervention at all. The tracking errors across various task and behavior commands reflect the intervention tolerance, \textit{i.e.}, the ability of precise locomotion control under external intervention.}
\label{tab:Intervetion Tracking Error}
\begin{tabular}{c|c|ccc|cc|ccc}
\toprule
\multirow{3}{*}{Training Strategy} & \multirow{3}{*}{Intervention Task} & \multicolumn{3}{c|}{Task Commands}                        & \multicolumn{5}{c}{Behavior Commands}\\ \cmidrule{3-10}
 & & \multicolumn{3}{c|}{Movement}                        & \multicolumn{2}{c|}{Foot}          & \multicolumn{3}{c}{Posture}                         \\ \cmidrule{3-10}
                                      &                                      &$E_{v_x}$ ($m/s$)     & $E_{v_y}$ ($m/s$)   & $E_{\omega}$ ($rad/s$)    & $E_{f}$ ($Hz$)         & $E_{l}$ ($m$)         & $E_{h}$ ($m$)        & $E_{p}$ ($rad$)     & $E_{w}$ ($rad$)         \\ \midrule
\multirow{3}{*}{\makecell{Noise Curriculum\\(\our)}}        & Noise                        & \textbf{0.0483} & \textbf{0.0962} & \textbf{0.1879} & \textbf{0.0471} & \textbf{0.0542} & \textbf{0.0402} & \textbf{0.0432} & \textbf{0.0552} \\
                                      & AMASS                                & \textbf{0.0391} & \textbf{0.0920} & \textbf{0.1039} & \textbf{0.0464} & \textbf{0.0543} & \textbf{0.0387} & \textbf{0.0364} & \textbf{0.0540} \\
                                      & None                                 & \textbf{0.0264} & \textbf{0.0863} & \textbf{0.0543} & \textbf{0.0447} & \textbf{0.0522} & 0.0372          & 0.0375          & 0.0475          \\ \cmidrule{1-10}
\multirow{3}{*}{AMASS}                & Noise                        & 0.1697          & 0.1055          & 0.2156          & 0.0621          & 0.0542          & 0.0620          & 0.0812          & 0.0694          \\
                                      & AMASS                                & 0.0567          & 0.0965          & 0.1593          & 0.0466          & 0.0555          & 0.0579          & 0.0458          & 0.0554          \\
                                      & None                                 & 0.0645          & 0.0916          & 0.0802          & 0.0460          & 0.0531          & 0.0577          & 0.0455          & 0.0568          \\ \cmidrule{1-10}
\multirow{3}{*}{No Intervention}                 & Noise                        & 0.8658          & 0.7511          & 0.9116          & 0.1930          & 0.1913          & 0.1658          & 0.3622          & 0.2241          \\
                                      & AMASS                                & 0.6299          & 0.4026          & 0.5758          & 0.2245          & 0.2527          & 0.1305          & 0.2367          & 0.1112          \\
                                      & None                                 & 0.0755          & 0.1076          & 0.1151          & 0.0450          & 0.0678          & \textbf{0.0255} & \textbf{0.0211} & \textbf{0.0380} \\ \bottomrule
\end{tabular}
\end{table*}



\begin{table}[t]
\setlength{\abovecaptionskip}{0.cm}
\setlength{\belowcaptionskip}{-0.cm}
\centering
\caption{ \small
\textbf{Averaged foot displacement under intervention}. We compare foot displacement $D_\text{cmd}$ of different training strategies under various intervention tasks, which computes the total movement of both feet in one episode with sampled posture behavior commands.
}
\label{tab:Intervention Mean Foot Movement}
\resizebox{\linewidth}{!}{
\begin{tabular}{ccccc}
\toprule
Training Strategy                 & Intervention Task     & $D_{h}$ ($m/s$)                  & $D_{p}$ ($m/s$)      & $D_{w}$ ($m/s$)       \\ \midrule
\multirow{3}{*}{\makecell{Noise Curriculum\\(\our)}}  & Noise & \textbf{0.0339}             & \textbf{0.0892} & \textbf{0.0199} \\
                       & AMASS         & \textbf{0.0454}             & \textbf{0.0728} & \textbf{0.0196} \\
                       & None          & \textbf{0.0003}             & \textbf{0.0016} & \textbf{0.0007} \\ \midrule
\multirow{3}{*}{AMASS only} & Noise         & 2.0815                      & 2.8978          & 3.2630          \\
                       & AMASS         & 0.0536                      & 0.1743          & 0.0396          \\
                       & None          & 0.0139                      & 0.0160          & 0.0013          \\ \midrule
\multirow{3}{*}{No Intervention}  & Noise         & 17.5358                     & 17.9732         & 25.7132         \\
                       & AMASS         & 25.3802 & 26.3496         & 21.3078         \\
                       & None          & 0.0159  & 1.7065          & 1.7152          \\ \bottomrule
\end{tabular}}
\end{table}

\subsection{Command Combination Analysis}
To provide an in-depth analysis of the command space and to 
reveal the underlying interaction of various commands under different gaits.
Here, we aim to analyze the \emph{orthogonality} of commands based on the interference or conflict between the tracking errors of these commands across their reasonable ranges. For instance, when we say that a set of commands are \emph{orthogonal}, each command does not significantly affect the tracking performance of each other in its range. To this end, we plot the tracking error $E_\text{cmd}$ as heat maps, generated by systematically scanning the command values for each pair of parameters, revealing the correlation of each command.
We leave the full heat maps at \ap{ap:heatmaps}, and conclude our main observation for all gaits.

\noindent\textbf{Walking.} Walking is the most basic gait, which preserves the best performance of the robot hardware.
\begin{itemize}[leftmargin=*]
    \item The {linear velocity} $v_x$, the {angular velocity yaw} $\omega$, the {body height} $h$, and the {waist yaw} $w$ are orthogonal during walking.
    \item When the {linear velocity} $v_x$ exceeds $1.5m/s$, the orthogonality between $v_x$ and other commands decreases due to reduced dynamic stability and the robot's need to maintain body stability over tracking accuracy.
    \item The {gait frequency} $f$ shows discrete orthogonality, with optimal tracking performance at frequencies of 1.5 or 2. High-frequency gait conditions reduce tracking accuracy.
    \item The {linear velocity} $v_y$, the {foot swing height} $l$, and the {body pitch} $p$ are orthogonal to other commands only within a narrow range.
\end{itemize}

\noindent\textbf{Jumping.} The command orthogonality in jumping is similar to walking, but the overall orthogonal range is smaller, due to the increased challenge of the jumping gait, especially in high-speed movement modes.
During each gait cycle, the robot must leap forward significantly to maintain its speed. To execute this complex jumping action continuously, the robot must adopt an optimal posture at the beginning of each cycle. Both legs exert substantial torque to propel the body forward. Upon landing, the robot must quickly readjust its posture to maintain stability and repeat the actions. Consequently, during movement, the robot can only execute other commands within a relatively narrow range.

\noindent\textbf{Hopping.}
The hopping gait introduces more instability, and the robot's control system must focus more on maintaining balance, making it difficult to simultaneously handle complex, multi-dimensional commands.
\begin{itemize}[leftmargin=*]
    \item Hopping gait commands lack clear orthogonal relationships.
    \item Effective tracking is limited to the x-axis {linear velocity} $v_x$, the y-axis {linear velocity} $v_y$, the {angular velocity yaw} $\omega$, and the {body height} $h$.
    \item Adjustments to $h$ can be understood that a lower body height improves dynamic stability, therefore, it plays a positive role in maintaining the target body posture.
    % enhancing the robot's hopping performance.
\end{itemize}

\noindent\textbf{Standing.} As for the standing gait, we tested the tracking errors of commands related to posture. The results showed that the tracking errors were similar to those observed during walking with zero velocity.

\begin{itemize}[leftmargin=*]
    \item The {waist yaw} $w$ command is almost orthogonal to the other two commands.
    \item As the range of commands increases, orthogonality between the {body height} $h$ and the {body pitch} $p$ decreases. This is because the H1 robot has only one degree of freedom at the waist, limiting posture adjustments to the hip pitch joint.
    \item A 0.3 m decrease of the body height relative to the default height reduces the range of motion of the hip pitch joint to almost zero, hindering precise tracking of body pitch.
\end{itemize}

Furthermore, we conclude that {gait frequency} $f$ highly affects the tracking accuracy of \emph{movement} commands when it is excessively high and low; the \emph{posture} commands can significantly impact the tracking errors of other commands, especially when they are near the range limits.
% We categorize the commands into three groups: \emph{movement}, \emph{foot}, and \emph{posture}. 1) The \emph{movement} commands include the linear velocity $v_x, v_y$ and angular velocity $\omega$, forming the foundational locomotion commands, and are considered the most critical aspect of the tasks. 2) The \emph{foot} commands include the {foot swing height} $l$, which is the least accurately tracked; and the {gait frequency} $f$, which can affect the tracking accuracy of \emph{movement} commands when it is excessively high and low. 3) The \emph{posture} commands, which include body height $h$, the body pitch $p$, and waist yaw $w$, determine the desired body posture, and can significantly impact the tracking errors of other commands, especially when the command is challenging. 
For different gaits, the orthogonality range between commands is greatest in the walking gait and smallest in the hopping gait.

\subsection{Ablation on Intervention Training Strategy}
\label{sec:InterventionExp}
% The three policies use the same random seeds and training time.
To validate the effectiveness of the intervention training strategy on the policy robustness when external upper-body intervention is involved, we compare the policies trained with different strategies, including noise curriculum (\our), filtered AMASS data~\citep{he2024omnih2o}, and no intervention. We test the tracking errors under two different intervention tasks, \textit{i.e.}, uniform noise, AAMAS dataset, along with a no-intervention setup. The results under the walking gait are shown in \tb{tab:Intervetion Tracking Error}, and we leave other gaits in \ap{ap:SingleCommandsTracking-REMAIN}. 
It is obvious that the noise curriculum strategy of \our achieved the best performance under almost all test cases, except the posture-related tracking with no intervention. 
In particular, \our showed less of a decrease in tracking accuracy with various interventions, indicating our noise curriculum intervention strategy enables the control policy to handle a large range of arm movements, making it very useful and supportive for loco-manipulation tasks.
In comparison, the policy trained with AMASS data shows a significant decrease in the tracking accuracy when intervening with uniform noise, due to the limited motion in the training data. The policy trained without any intervention only performs well without external upper-body control.

It is worth noting that when intervention training is involved, the tracking error related to the movement and foot is also better than those of the policy trained without intervention, and \our provides the most accurate tracking. This shows that intervention training also contributes to the robustness of the policy. During our real robot experiments, we further observed that the robot behaves with a harder force when in contact with the floor, indicating a possible trade-off between motion regularization and tracking accuracy when involving intervention.

\noindent\textbf{Stability under standing gait.}
Adjusting posture in the standing state introduces additional requirements for stability, since the robot pacing to maintain balance may increase the difficulty of achieving manipulation tasks that require stand still. To investigate the necessity of noise curriculum for manipulation, we further measured the averaged foot displacement (in meters) under the standing gait, which computes the total movement of both feet in one episode (20 seconds) while tracking the posture behavior commands. Results in \tb{tab:Intervention Mean Foot Movement} show that \our exhibits minimal foot displacement. On the contrary, the strategy trained on AMASS data requires frequent small steps to adjust the posture and maintain stability for noise interventions. 
Without intervention training, the policy tends to tip over when involving intervention, leading to failure of the entire task.

%  鲁棒性测试的结果分析
\begin{figure}[t]
    \centering
    \includegraphics[width=\linewidth]{imgs/radar_chart_V2.pdf}
    \vspace{-13pt}
    \caption{\small \textbf{External disturbance tolerance}. Left: A constant and continuous force is applied to the robot. Right: A one-second force is exerted on the robot. The experiment is conducted under a standing gait with default commands. If the robot's survival ratio exceeds $98\%$, it is deemed capable of tolerating such external disturbance. 
    The survival ratio computes the trajectory ratio of non-termination (ends of timeout) during 4096 rollouts.}
    \label{fig:Robust}
    \vspace{-12pt}
\end{figure}
\noindent\textbf{Robustness for external disturbance.}
Finally, we test the contribution of intervention training and noise curriculum to the robustness of external disturbance. In particular, we evaluated the robot's maximum tolerance to external disturbance forces in eight directions and compared the policy trained without intervention. Results illustrated in \fig{fig:Robust} demonstrate that \our preserves greater tolerance for external disturbances in both pushing and loading scenarios across most of the directions. The reason behind this is that the intervention brings the robot exposed to various disturbances originating from its upper body, and thereby enhances the overall stability by dynamically adjusting leg strength.

% \our has a significantly higher tolerance for external disturbance forces in almost all directions compared to the strategy without intervention training.
% This is attributed to the fact that, during large-scale noise intervention training, the robot effectively explored a wide range of extreme scenarios and learned to enhance body stability by adjusting leg movements.

\subsection{Real-World Experiments}
We deploy \our on a real-world robot to verify its effectiveness. In \fig{fig:teaser}, we illustrate the humanoid capabilities supported by \our, showing the versatile behavior of the Unitree H1 robot. In particular, we demonstrate the intriguing potential of the comprehensive task range that \our is able to achieve, with a flexible combination of commands in high dynamics. To qualitatively analyze the performance of \our, we estimate the tracking error of two pose parameters (body pitch $p$ and waist rotation $w$ from the motor readings) on real robots, since other commands are hard to measure without a highly accurate motion capture system. The results are shown in \tb{tb:track-real}, where $E^{\text{real}}_{\text{cmd}}$ illustrates the tracking error of the posture command.
We observe that the tracking error in real-world experiments is slightly higher than in simulation environments, primarily due to sensor noise and the wear of the robot's hardware. Among different gaits, the tracking error for the waist rotation $w$ is smaller compared to that for the body pitch $p$, as waist control has less impact on the robot’s overall stability. In both error tests, the jumping gait exhibited the smallest $E_{cmd}$, while the walking gait showed slightly higher errors, consistent with the findings observed in the simulation environment.

\begin{table}[t]
\centering
\caption{\small \textbf{Tracking error in real world.} We conducted five tests to measure the tracking error for each command under three gaits. The tracking error for each command was calculated during each control step. The tested commands gradually increased from the minimum to the maximum values within a predefined range, while the remaining commands were kept at their default values.} % To account for the impact of communication delays on the actual tracking error, we introduced a 0.1-second delay in the command execution.
\label{tb:track-real}
\begin{tabular}{c|cc} \toprule
Gait     & $E_p^{\text{real}}$ & $E_w^{\text{real}}$ \\ \midrule
Standing & 0.0712 $\pm$ 0.0425 & 0.0718 $\pm$ 0.0614 \\
Walking  & 0.1006 $\pm$ 0.0581  & 0.0571 $\pm$ 0.0489 \\
Jumping  & 0.0674 $\pm$ 0.0569  & 0.0552 $\pm$ 0.0469 \\ \bottomrule
\end{tabular}
\end{table}

\section{Related Work}\label{sec:related-work}


\smalltitle{Benchmarking End-to-End Code Generation}
Various benchmarks have been developed to assess LLMs in end-to-end code generation -- %
some benchmarks {\textit{broader the programming languages}} to evaluate. Classical benchmarks focus on Python programming \cite{humaneval, mbpp}; later, benchmarks considering other programming languages, e.g., Java \cite{JavaBench} and even multilingual \cite{humanevalx}, emerge. %
Some studies evaluate LLM programming {\textit{across different contexts}}, %
such as class-level \cite{classeval}, project-level \cite{deveval}, and repository-level \cite{repocoder, repoeval}, pushing the boundaries of LLM capabilities in real-world scenarios.
The performance of generating \textit{{code in different domains}} also attracts studies \cite{domaineval}. %
Several recent studies explored LLMs' code-generation capabilities {\textit{incorporating external techniques}}, for example, using RAG to retrieve codes \cite{ase24retrieval_repo, codesearchallyouneed_hu} and documents \cite{docragarxiv, docragccwan}, allowing LLMs to code with external resources.

Though these studies assess LLM's performance in various scenarios, they reveal relatively limited information about LLM's ability at steps within the end-to-end pipeline, e.g., coding a solution logic.

\smalltitle{Benchmarking Code Generation Using Pseudocode}
Only a few works have studied translating pseudocode into code. \citet{Dir17} propose a conceptual framework that breaks down pseudocode into XML elements. \citet{Kul19} explore potential mappings of pseudocode and C++ code using test cases. The SPoC dataset with 18K line-to-line mappings is built in the work. However, the fairly trivial line-by-line pseudocode may not accurately reflect the human-written pseudocode typically appearing in real-world software development. SPoC was later utilized by \citet{Ach22} to train two basic deep-learning models for pseudocode-to-code translation.
These studies worked on relatively small and trivial pseudocode snippets. They also barely compared the performance of code generators (in particular the advanced LLMs) or discussed the detailed abilities. %





\section{Insights from Study Results}


\indent\indent \ding{182} Code generation bottleneck differs across programming languages (PLs). %
One can improve end-to-end LLM programming performance for popular PLs like Python by boosting problem-solving abilities, whereas for less-trained languages like Rust, enhancing language-coding skills is crucial.

\ding{183} %
Problem-solving ability may transfer across PLs, which may allow LLMs' coding performance to be improved in a unified manner across PLs.

\ding{184} %
Reasoning models can effectively handle the code-to-pseudocode transformation. This enables easy creation of up-to-date benchmarks focusing on problem-solving capability, which may help relieve the current bottleneck and support cross-PL tasks.


These insights may shed light on enhancing LLMs in code generation and other cross-PL tasks, as well as guide human-LLM collaboration in the era of AI-driven low/zero-code development.


\section{Conclusion}\label{sec:conclusion}

To understand the bottlenecks in end-to-end code generation for LLMs, we introduce \name, a multilingual code generation benchmark incorporating pseudocode as input,
isolating the evaluation of language-coding from problem-solving capabilities. Empirical study results with \name reveal key insights about the bottlenecks identified for different programming languages, broad applicability of pseudocode across programming languages, and exceptional quality of automatically derived pseudocode. %

\clearpage

\section{Limitations}
\smalltitle{Pseudocode Samples}
Due to the limited access to DeepSeek-R1, the latency of response of reasoning models, and the costs of the subsequence inference, this study only sample one pseudocode for each problem.
As revealed in \Cref{subsec:resrq4}, a small portion of the generated pseudocode could be not semantic preserving and is filtered out from the final benchmark.
The thorough study on whether sampling multiple pseudocode or using a majority vote mechanism can further improve the pseudocode quality is left as future work.


\smalltitle{Problem Domain}
The current \name selects subjects from LiveCodeBench and their solutions on LeetCode, which are mainly algorithmic code for programming puzzles.
Although this meets the purpose of using pseudocode to present algorithms in practice, the daily software development scenarios such as implementing business logic are not covered.
It is unclear whether the performance gap between problem-to-code generation and pseudocode-to-code generation is also significant in such scenarios.
The future work to understanding this problem can be extending the workflow of \name to code generation benchmarks in different scenarios.

\smalltitle{Involved Programming Languages}
The programming languages studied in this paper are Python, C++, and Rust.
They represent three popular imperative programming languages, with a major difference in the type system.
Python is dynamic, C++ is static but weakly typed, and Rust is known for having a rigorous type checking mechanism. 
The results in RQ2 may shed light on similar languages such as Java, but may not apply to functional languages such as Haskell or low-resourced languages such as domain-specific languages.

\newpage





\bibliography{custom}

\appendix
\section{Hard Threshold of EAC}\label{threshhold}
In constructing a weighted-gradient saliency map, the value of \(\gamma\) determines the number of the dimensions we select where important feature anchors are located. As the value of \(\gamma\) increases, the number of selected dimensions decreases, requiring the editing information to be compressed into a smaller space during the compression process. 
During compression, it is desired for the compression space to be as small as possible to preserve the general abilities of the model. However, reducing the compression space inevitably increases the loss of editing information, which reduces the editing performance of the model.
Therefore, to ensure editing performance in a single editing scenario, different values of \(\gamma\) are determined for various models, methods, and datasets. Fifty pieces of knowledge were randomly selected from the dataset, and reliability, generalization, and locality were measured after editing. The averages of these metrics were then taken as a measure of the editing performance of the model.
Table~\ref{value} presents the details of \(\gamma\), while Table~\ref{s} illustrates the corresponding editing performance before and after the introduction of EAC. $P_{x}$ denotes the value below which x\% of the values in the dataset.


\begin{table}[!htb]
\caption{The value of $\gamma$.}
\centering
\resizebox{0.45\textwidth}{!}{
\begin{tabular}{lcccc}
\toprule
\textbf{Datasets} & \textbf{Model} & \textbf{ROME} & \textbf{MEMIT} \\
\midrule
\multirow{2}{*}{\textbf{ZSRE}} & GPT-2 XL & $P_{80}$ & $P_{80}$ \\
 & LLaMA-3 (8B) & $P_{90}$ & $P_{95}$ \\
\midrule
\multirow{2}{*}{\textbf{COUNTERFACT}} & GPT-2 XL & $P_{85}$ & $P_{85}$ \\
 & LLaMA-3 (8B) & $P_{95}$ & $P_{95}$ \\
\bottomrule
\end{tabular}}
\label{value}
\end{table}


\begin{table}[!htb]
\caption{The value of $\gamma$.}
\centering
\resizebox{\textwidth}{!}{%
\begin{tabular}{lccccccccccccc}
\toprule
\multirow{1}{*}{Dataset} & \multirow{1}{*}{Method} & \multicolumn{3}{c}{\textbf{GPT-2 XL}} & \multicolumn{3}{c}{\textbf{LLaMA-3 (8B)}} \\
\cmidrule(lr){3-5} \cmidrule(lr){6-8}
& & \multicolumn{1}{c}{Reliability} & \multicolumn{1}{c}{Generalization} & \multicolumn{1}{c}{Locality} & \multicolumn{1}{c}{Reliability} & \multicolumn{1}{c}{Generalization} & \multicolumn{1}{c}{Locality} \\
\midrule
\multirow{1}{*}{ZsRE} & ROME & 1.0000 & 0.9112 & 0.9661 & 1.0000 & 0.9883 & 0.9600  \\
& ROME-EAC & 1.0000 & 0.8923 & 0.9560 & 0.9933 & 0.9733 & 0.9742  \\
\cmidrule(lr){2-8}
& MEMIT & 0.6928 & 0.5208 & 1.0000 & 0.9507 & 0.9333 & 0.9688  \\
& MEMIT-EAC & 0.6614 & 0.4968 & 0.9971 & 0.9503 & 0.9390 & 0.9767  \\
\midrule
\multirow{1}{*}{CounterFact} & ROME & 1.0000 & 0.4200 & 0.9600 & 1.0000 & 0.3600 & 0.7800  \\
& ROME-EAC & 0.9800 & 0.3800 & 0.9600 & 1.0000 & 0.3200 & 0.8800  \\
\cmidrule(lr){2-8}
& MEMIT & 0.9000 & 0.2200 & 1.0000 & 1.0000 & 0.3800 & 0.9500  \\
& MEMIT-EAC & 0.8000 & 0.1800 & 1.0000 & 1.0000 & 0.3200 & 0.9800  \\
\bottomrule
\end{tabular}%
}
\label{s}
\end{table}

\section{Optimization Details}\label{b}
ROME derives a closed-form solution to achieve the optimization:
\begin{equation}
\text{minimize} \ \| \widehat{W}K - V \| \ \text{such that} \ \widehat{W}\mathbf{k}_* = \mathbf{v}_* \ \text{by setting} \ \widehat{W} = W + \Lambda (C^{-1}\mathbf{k}_*)^T.
\end{equation}

Here \( W \) is the original matrix, \( C = KK^T \) is a constant that is pre-cached by estimating the uncentered covariance of \( \mathbf{k} \) from a sample of Wikipedia text, and \( \Lambda = (\mathbf{v}_* - W\mathbf{k}_*) / ( (C^{-1}\mathbf{k}_*)^T \mathbf{k}_*) \) is a vector proportional to the residual error of the new key-value pair on the original memory matrix.

In ROME, \(\mathbf{k}_*\) is derived from the following equation:
\begin{equation}
\mathbf{k}_* = \frac{1}{N} \sum_{j=1}^{N} \mathbf{k}(x_j + s), \quad \text{where} \quad \mathbf{k}(x) = \sigma \left( W_{f_c}^{(l^*)} \gamma \left( a_{[x],i}^{(l^*)} + h_{[x],i}^{(l^*-1)} \right) \right).
\end{equation}

ROME set $\mathbf{v}_* = \arg\min_z \mathcal{L}(z)$, where the objective $\mathcal{L}(z)$ is:
\begin{equation}
\frac{1}{N} \sum_{j=1}^{N} -\log \mathbb{P}_{G(m_{i}^{l^*}:=z))} \left[ o^* \mid x_j + p \right] + D_{KL} \left( \mathbb{P}_{G(m_{i}^{l^*}:=z)} \left[ x \mid p' \right] \parallel \mathbb{P}_{G} \left[ x \mid p' \right] \right).
\end{equation}

\section{Experimental Setup} \label{detail}

\subsection{Editing Methods}\label{EM}

In our experiments, Two popular editing methods including ROME and MEMIT were selected as baselines.

\textbf{ROME} \cite{DBLP:conf/nips/MengBAB22}: it first localized the factual knowledge at a specific layer in the transformer MLP modules, and then updated the knowledge by directly writing new key-value pairs in the MLP module.

\textbf{MEMIT} \cite{DBLP:conf/iclr/MengSABB23}: it extended ROME to edit a large set of facts and updated a set of MLP layers to update knowledge.

The ability of these methods was assessed based on EasyEdit~\cite{DBLP:journals/corr/abs-2308-07269}, an easy-to-use knowledge editing framework which integrates the released codes and hyperparameters from previous methods.

\subsection{Editing Datasets}\label{dat}
In our experiment, two popular model editing datasets \textsc{ZsRE}~\cite{DBLP:conf/conll/LevySCZ17} and \textsc{CounterFact}~\cite{DBLP:conf/nips/MengBAB22} were adopted.

\textbf{\textsc{ZsRE}} is a QA dataset using question rephrasings generated by back-translation as the equivalence neighborhood.
Each input is a question about an entity, and plausible alternative edit labels are sampled from the top-ranked predictions of a BART-base model trained on \textsc{ZsRE}.

\textbf{\textsc{CounterFact}} accounts for counterfacts that start with low scores in comparison to correct facts. It constructs out-of-scope data by substituting the subject entity for a proximate subject entity sharing a predicate. This alteration enables us to differentiate between superficial wording changes and more significant modifications that correspond to a meaningful shift in a fact. 

\subsection{Metrics for Evaluating Editing Performance}\label{Mediting performance}
\paragraph{Reliability} means that given an editing factual knowledge, the edited model should produce the expected predictions. The reliability is measured as the average accuracy on the edit case:
\begin{equation}
\mathbb{E}_{(x'_{ei}, y'_{ei}) \sim \{(x_{ei}, y_{ei})\}} \mathbf{1} \left\{ \arg\max_y f_{\theta_{i}} \left( y \mid x'_{ei} \right) = y'_{ei} \right\}.
\label{rel}
\end{equation}

\paragraph{Generalization} means that edited models should be able to recall the updated knowledge when prompted within the editing scope. The generalization is assessed by the average accuracy of the model on examples uniformly sampled from the equivalence neighborhood:
\begin{equation}
\mathbb{E}_{(x'_{ei}, y'_{ei}) \sim N(x_{ei}, y_{ei})} \mathbf{1} \left\{ \arg\max_y f_{\theta_{i}} \left( y \mid x'_{ei} \right) = y'_{ei} \right\}.
\label{gen}
\end{equation}

\paragraph{Locality} means that the edited model should remain unchanged in response to prompts that are irrelevant or the out-of-scope. The locality is evaluated by the rate at which the edited model's predictions remain unchanged compared to the pre-edit model.
\begin{equation}
\mathbb{E}_{(x'_{ei}, y'_{ei}) \sim O(x_{ei}, y_{ei})} \mathbf{1} \left\{ f_{\theta_{i}} \left( y \mid x'_{ei} \right) = f_{\theta_{i-1}} \left( y \mid x'_{ei} \right) \right\}.
\label{loc}
\end{equation}

\subsection{Downstream Tasks}\label{pro}

Four downstream tasks were selected to measure the general abilities of models before and after editing:
\textbf{Natural language inference (NLI)} on the RTE~\cite{DBLP:conf/mlcw/DaganGM05}, and the results were measured by accuracy of two-way classification.
\textbf{Open-domain QA} on the Natural Question~\cite{DBLP:journals/tacl/KwiatkowskiPRCP19}, and the results were measured by exact match (EM) with the reference answer after minor normalization as in \citet{DBLP:conf/acl/ChenFWB17} and \citet{DBLP:conf/acl/LeeCT19}.
\textbf{Summarization} on the SAMSum~\cite{gliwa-etal-2019-samsum}, and the results were measured by the average of ROUGE-1, ROUGE-2 and ROUGE-L as in \citet{lin-2004-rouge}.
\textbf{Sentiment analysis} on the SST2~\cite{DBLP:conf/emnlp/SocherPWCMNP13}, and the results were measured by accuracy of two-way classification.

The prompts for each task were illustrated in Table~\ref{tab-prompt}.

\begin{table*}[!htb]
% \small
\centering
\begin{tabular}{p{0.95\linewidth}}
\toprule

NLI:\\
\{\texttt{SENTENCE1}\} entails the \{\texttt{SENTENCE2}\}. True or False? answer:\\

\midrule

Open-domain QA:\\
Refer to the passage below and answer the following question. Passage: \{\texttt{DOCUMENT}\} Question: \{\texttt{QUESTION}\}\\

\midrule

Summarization:\\
\{\texttt{DIALOGUE}\} TL;DR:\\

\midrule


Sentiment analysis:\\
For each snippet of text, label the sentiment of the text as positive or negative. The answer should be exact 'positive' or 'negative'. text: \{\texttt{TEXT}\} answer:\\

\bottomrule
\end{tabular}
\caption{The prompts to LLMs for evaluating their zero-shot performance on these general tasks.}
\label{tab-prompt}
\end{table*}

\subsection{Hyper-parameters for Elastic Net}\label{hy}

In our experiment, we set \(\lambda = 5 \times 10^{-7} \), \(\mu = 5 \times 10^{-1} \) for GPT2-XL\cite{radford2019language} and \(\lambda = 5 \times 10^{-7} \), \(\mu = 1 \times 10^{-3} \) for LLaMA-3 (8B)\cite{llama3}.

\begin{figure*}[!hbt]
  \centering
  \includegraphics[width=0.5\textwidth]{figures/legend_edit.pdf}
  \vspace{-4mm}
\end{figure*}

\begin{figure*}[!hbt]
  \centering
  \subfigure{
  \includegraphics[width=0.23\textwidth]{figures/ROME-GPT2XL-CF-edit.pdf}}
  \subfigure{
  \includegraphics[width=0.23\textwidth]{figures/ROME-LLaMA3-8B-CF-edit.pdf}}
  \subfigure{
  \includegraphics[width=0.23\textwidth]{figures/MEMIT-GPT2XL-CF-edit.pdf}}
  \subfigure{
  \includegraphics[width=0.23\textwidth]{figures/MEMIT-LLaMA3-8B-CF-edit.pdf}}
  \caption{Edited on CounterFact, editing performance of edited models using the ROME~\cite{DBLP:conf/nips/MengBAB22} and MEMIT~\cite{DBLP:conf/iclr/MengSABB23} on GPT2-XL~\cite{radford2019language} and LLaMA-3 (8B)~\cite{llama3}, as the number of edits increases before and after the introduction of EAC.}
  \vspace{-4mm}
  \label{edit-performance-cf}
\end{figure*}

\begin{figure*}[!hbt]
  \centering
  \includegraphics[width=0.75\textwidth]{figures/legend.pdf}
  \vspace{-4mm}
\end{figure*}

\begin{figure*}[!htb]
  \centering
  \subfigure{
  \includegraphics[width=0.23\textwidth]{figures/ROME-GPT2XL-CounterFact.pdf}}
  \subfigure{
  \includegraphics[width=0.23\textwidth]{figures/ROME-LLaMA3-8B-CounterFact.pdf}}
  \subfigure{
  \includegraphics[width=0.23\textwidth]{figures/MEMIT-GPT2XL-CounterFact.pdf}}
  \subfigure{
  \includegraphics[width=0.23\textwidth]{figures/MEMIT-LLaMA3-8B-CounterFact.pdf}}
  \caption{Edited on CounterFact, performance on general tasks using the ROME~\cite{DBLP:conf/nips/MengBAB22} and MEMIT~\cite{DBLP:conf/iclr/MengSABB23} on GPT2-XL~\cite{radford2019language} and LLaMA-3 (8B)~\cite{llama3}, as the number of edits increases before and after the introduction of EAC.}
  \vspace{-4mm}
  \label{task-performance-cf}
\end{figure*}

\section{Experimental Results}\label{app}

\subsection{Results of Editing Performance}\label{cf-performance}
Applying CounterFact as the editing dataset, Figure~\ref{edit-performance-cf} presents the editing performance of the ROME~\cite{DBLP:conf/nips/MengBAB22} and MEMIT~\cite{DBLP:conf/iclr/MengSABB23} methods on GPT2-XL~\cite{radford2019language} and LLaMA-3 (8B)~\cite{llama3}, respectively, as the number of edits increases before and after the introduction of EAC.
The dashed line represents the ROME or MEMIT, while the solid line represents the ROME or MEMIT applying the EAC.


\subsection{Results of General Abilities}\label{cf-ability}
Applying CounterFact as the editing dataset, Figure~\ref{task-performance-cf} presents the performance on general tasks of edited models using the ROME~\cite{DBLP:conf/nips/MengBAB22} and MEMIT~\cite{DBLP:conf/iclr/MengSABB23} methods on GPT2-XL~\cite{radford2019language} and LLaMA-3 (8B)~\cite{llama3}, respectively, as the number of edits increases before and after the introduction of EAC. 
The dashed line represents the ROME or MEMIT, while the solid line represents the ROME or MEMIT applying the EAC.

\subsection{Results of Larger Model}\label{13 B}
To better demonstrate the scalability and efficiency of our approach, we conducted experiments using the LLaMA-2 (13B)~\cite{DBLP:journals/corr/abs-2307-09288}.
Figure~\ref{13B-edit} presents the editing performance of the ROME~\cite{DBLP:conf/nips/MengBAB22} and MEMIT~\cite{DBLP:conf/iclr/MengSABB23} methods on LLaMA-2 (13B)
~\cite{DBLP:journals/corr/abs-2307-09288}, as the number of edits increases before and after the introduction of EAC.
Figure~\ref{13B} presents the performance on general tasks of edited models using the ROME and MEMIT methods on LLaMA-2 (13B), as the number of edits increases before and after the introduction of EAC.
The dashed line represents the ROME or MEMIT, while the solid line represents the ROME or MEMIT applying the EAC.

\begin{figure*}[!hbt]
  \centering
  \includegraphics[width=0.5\textwidth]{figures/legend_edit.pdf}
  \vspace{-4mm}
\end{figure*}

\begin{figure*}[!hbt]
  \centering
  \subfigure{
  \includegraphics[width=0.23\textwidth]{figures/ROME-LLaMA2-13B-ZsRE-edit.pdf}}
  \subfigure{
  \includegraphics[width=0.23\textwidth]{figures/MEMIT-LLaMA2-13B-ZsRE-edit.pdf}}
  \subfigure{
  \includegraphics[width=0.23\textwidth]{figures/ROME-LLaMA2-13B-CF-edit.pdf}}
  \subfigure{
  \includegraphics[width=0.23\textwidth]{figures/MEMIT-LLaMA2-13B-CF-edit.pdf}}
  \caption{Editing performance of edited models using the ROME~\cite{DBLP:conf/nips/MengBAB22} and MEMIT~\cite{DBLP:conf/iclr/MengSABB23} on LLaMA-2 (13B)~\cite{DBLP:journals/corr/abs-2307-09288}, as the number of edits increases before and after the introduction of EAC.}
  \vspace{-4mm}
  \label{13B-edit}
\end{figure*}

\begin{figure*}[!hbt]
  \centering
  \includegraphics[width=0.75\textwidth]{figures/legend.pdf}
  \vspace{-4mm}
\end{figure*}

\begin{figure*}[!htb]
  \centering
  \subfigure{
  \includegraphics[width=0.23\textwidth]{figures/ROME-LLaMA2-13B-ZsRE.pdf}}
  \subfigure{
  \includegraphics[width=0.23\textwidth]{figures/MEMIT-LLaMA2-13B-ZsRE.pdf}}
  \subfigure{
  \includegraphics[width=0.23\textwidth]{figures/ROME-LLaMA2-13B-CounterFact.pdf}}
  \subfigure{
  \includegraphics[width=0.23\textwidth]{figures/MEMIT-LLaMA2-13B-CounterFact.pdf}}
  \caption{Performance on general tasks using the ROME~\cite{DBLP:conf/nips/MengBAB22} and MEMIT~\cite{DBLP:conf/iclr/MengSABB23} on LLaMA-2 (13B)~\cite{DBLP:journals/corr/abs-2307-09288}, as the number of edits increases before and after the introduction of EAC.}
  \vspace{-4mm}
  \label{13B}
\end{figure*}

\section{Analysis of Elastic Net}
\label{FT}
It is worth noting that the elastic net introduced in EAC can be applied to methods beyond ROME and MEMIT, such as FT~\cite{DBLP:conf/emnlp/CaoAT21}, to preserve the general abilities of the model.
Unlike the previously mentioned fine-tuning, FT is a model editing approach. It utilized the gradient to gather information about the knowledge to be updated and applied this information directly to the model parameters for updates.
Similar to the approaches of ROME and MEMIT, which involve locating parameters and modifying them, the FT method utilizes gradient information to directly update the model parameters for editing. Therefore, we incorporate an elastic net during the training process to constrain the deviation of the edited matrix.
Figure~\ref{ft} shows the sequential editing performance of FT on GPT2-XL and LLaMA-3 (8B) before and after the introduction of elastic net.
The dashed line represents the FT, while the solid line represents the FT applying the elastic net.
The experimental results indicate that when using the FT method to edit the model, the direct use of gradient information to modify the parameters destroys the general ability of the model. By constraining the deviation of the edited matrix, the incorporation of the elastic net effectively preserves the general abilities of the model.

\begin{figure*}[t]
  \centering
  \subfigure{
  \includegraphics[width=0.43\textwidth]{figures/legend_FT.pdf}}
\end{figure*}

\begin{figure*}[t]%[!ht]
  \centering
  \subfigure{
  \includegraphics[width=0.22\textwidth]{figures/FT-GPT2XL-ZsRE.pdf}}
  \subfigure{
  \includegraphics[width=0.22\textwidth]{figures/FT-GPT2XL-CounterFact.pdf}}
  \vspace{-2mm}
  \caption{Edited on the ZsRE or CounterFact datasets, the sequential editing performance of FT~\cite{DBLP:conf/emnlp/CaoAT21} and FT with elastic net on GPT2-XL before and after the introduction of elastic net.}
  \vspace{-2mm}
  \label{ft}
\end{figure*}



\end{document}
