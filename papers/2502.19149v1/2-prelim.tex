\section{Problem Definition}
\subsection{Task Definition}

As shown in \Cref{fig:motivating-example}, the problem of LLM code generation comprises two tasks.

\emph{(1) Problem Solving,} which analyzes the problem and reasons a \emph{solution} as output.
The granularity of a solution can vary from a one-sentence description of the core algorithm to a pseudocode with a clear control flow and data manipulation.%
 
\emph{(2) Language Coding,} which transforms the solution into a piece of compilable and executable code that implements the key logic and data manipulation in a target programming language.

The two tasks exercise distinct abilities of LLMs, and previous code generation benchmarks such as LiveCodeBench~\cite{livecb} evaluate them inseparably.
This paper studies the coding ability of LLMs by isolating it from their problem-solving ability using pseudocode. %

\subsection{Pseudocode Criteria}\label{subsec:criteria}
Although there is no universal concrete standard for pseudocode, conventions such as guidebooks have been commonly adopted. %
To study the coding ability of LLMs, we adopt a set of criteria to prepare the pseudocode for \benchname. The criteria are designed based on the features of pseudocode in textbooks, guidebooks, and research papers. %


\smalltitle{Completeness}
The pseudocode should be mapped to a piece of implementation code \emph{without ambiguity}, \eg, a competent programmer should be able to implement the pseudocode solving the given problem in a specific programming language.
If given a piece of implementation code, one can obtain a trivial but complete pseudocode by line-by-line code translation~\cite{spoc}.



\smalltitle{Language-agnostic}
The pseudocode should describe a language-agnostic solution. It should not be tied to specific language features, such as the \codef{yield} expression in Python and the pointer manipulations in C++.
In particular, explicit type information (\eg, \codef{vector} in C++) and type conversion should be absent.
The language-agnostic criterion facilitates a fair evaluation of LLMs' coding abilities in different target programming languages with the same pseudocode.

\smalltitle{Conciseness}
A pseudocode should be concise, which can be measured by the lines of code and the number of tokens.
In practice, software developers tend to sketch solutions concisely.
Also, verbose pseudocode with implementation details may not help differentiate the abilities of stronger models and weaker models.
An interesting case (\Cref{subsec:simplication}) in our study is to simplify a well-known algorithm, Sieve of Eratosthenes, and customize its use in pseudocode.
LLMs with higher coding capability can successfully implement the pseudocode, while weaker LLMs have lower success rates and even drop to zero when the target language is Rust.

Following the above criteria, we define more specific rules (\Cref{sec:prompts}) to prompt DeepSeek-R1 to convert an implementation code into a pseudocode to construct \benchname (\Cref{sec:benchmark-construction}).
