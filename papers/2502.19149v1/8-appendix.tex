\newpage

\section{Dataset}

\subsection{Legal Compliance and License}
The problems we use are from the LiveCodeBench, and the solutions we use to generate pseudocode are from LeetCode,
which are the publicly visible portions.
We did not include the user-submitted solutions in our final benchmark but their extracted pseudocode.
Following \citet{henry21} and LiveCodeBench~\cite{livecb}, we abide by Fair Use 107: ``the fair use of a copyrighted work, including such use by \ldots\xspace scholarship, or research, is not an infringement of copyright'', where fair use is determined by the purpose and character of the use, including whether such use is of a commercial nature or is for nonprofit educational purposes'', ``the amount and substantiality of the portion used in relation to the copyrighted work as a whole'', and ``the effect of the use upon the potential market for or value of the copyrighted work.''
The collected data in \name is used only for academic purposes.
Moreover, \name is used for benchmarking, and we do not use it for training models.

\subsection{Basic Stats}
\Cref{tab:loc-tokens-full} shows the number of files and statistics of source code and pseudocode from different languages in \name.
The statistics is basically consistent with the sampled subset in \Cref{subsec:resrq4}.
Each pseudocode corresponds to a problem in LiveCodeBench and can use its testcases to test the correctness of the code generated from the pseudocode.

\begin{table}[h]
    \centering
    \begin{adjustbox}{max width=\linewidth}
    \setlength{\tabcolsep}{4pt}
    \begin{NiceTabular}{c c | cc | cc  }
    \CodeBefore
    \Body
        \toprule
        \Block{2-1}{Language} & \Block{2-1}{\#Files} & \Block{1-2}{Source Code} && \Block{1-2}{Pseudocode} &  \\
        & & LoC & Tokens & LoC & Tokens \\
        \midrule
C++     & 355  & 18.20 & 192.23 & 13.05 & 122.58 \\
Rust    & 348  & 18.59 & 215.71 & 13.66 & 129.76 \\
Python	& 357  & 13.75 & 140.30 & 11.33 & 108.29 \\
    \bottomrule
    \end{NiceTabular}
    \end{adjustbox}
    \caption{Statistics of source code and pseudocode from different languages in \name }
    \label{tab:loc-tokens-full}
\end{table}

\section{Human Annotations}
Six programmers with more than five years of C++ experience participate in the annotation of pseudocode on 55 sampled C++ solution code.
Each annotated piece of pseudocode is validated by two other participants from the same group.

The approval from the ethics review board is exempted because the annotation procedure is not physically or mentally harmful and does not impose an intense workload in a short time.
The participants have been compensated according to the local legislation.
The consent to use the annotated data has been obtained from the participants.

\section{Case Study}

\subsection{Motivating Example}\label{subsec:motv-ex}
\Cref{lst:problem-motiv} lists the full problem and \Cref{lst:cpp-motiv} lists the user-submitted C++ solution where the pseudocode is converted from.
Note that the pseudocode simplifies the solution by replacing the map structure with a set structure.

\subsection{Simplifying Common Procedures}\label{subsec:simplication}
\Cref{lst:cpp-simp} shows a user-submitted C++ solution with detailed steps, and \Cref{lst:pseudo-simp} shows a concise but semantic-preserving pseudocode converted from the long solution.
Powerful LLMs such as GPT-4o-mini and Qwen32B can implement code correctly in all three languages, while smaller LLMs such as Phi-3.5 have lower success rates and even drop to zero when writing Rust codes. 

\subsection{Underflow in Rust}\label{subsec:rust-underflow}
\Cref{lst:rust-underflow} shows an example of a user-submitted Rust solution with the subtraction underflow problem.
Specifically, the variable \codef{pos} is from \codef{.len()} (line~5) and should be a \codef{usize} (unsigned) variable.
The user uses a nonstandard way to control the loop termination: when \codef{pos} is 0 and subtracts 1 from it in the release mode, it becomes the biggest unsigned integer, so the loop terminates because \codef{pos > arr.len()}.
However, such a coding style is not encouraged in Rust.
In the debug mode, the Rust program will panic (\ie, running into an invalid state because \codef{pos} is unsigned and should not underflow) and terminate the execution.

The pseudocode generated by DeepSeeek-R1 (\Cref{lst:pseudo-rust-underflow}) focuses on the solution logic, which does not contain such detailed type information and uses a more standard coding style (loop until \codef{pos} is negative).
Based on the pseudocode, only Qwen32B notices the possible sign problem and can generate code that correctly converts the type as \codef{isize} (line~6, \Cref{lst:rust-fix-underflow}), while all other less powerful models failed to do so.



\subsection{Worsening Pseudocode}\label{subsec:worsening}
\Cref{lst:python-acc}, \ref{lst:pseudo-python-acc}, and \ref{lst:python-acc-wrong-gen} show a case where the pseudocode generated from a Python solution misleads LLMs and causes a lower pass@1 compared with generating Python code from the problem.

\subsection{Failure of Pseudocode Generation}\label{subsec:fail-pseudogen}
\Cref{lst:python-being-wrongly-pseudogen} and \ref{lst:pseudo-wrong-condition} show a case of a Python solution and its generated pseudocode that is not semantic preserving.
The problem is at the last line, where the Python code will return \codef{max\_sum} if it is negative but not \codef{-inf}, while the pseudocode incorrectly assumes \codef{max\_sum} to be non-negative, possibly due to the hallucination problem in LLMs.


\section{Prompts}\label{sec:prompts}
\smalltitle{Generating Pseudocode}
\Cref{lst:pseudogen-prompt} is the prompt (a single user query as suggested by the DeepSeek team~\cite{ds-r1}) we use to query \dsr to generate pseudocode from Python code.
The prompts to generate pseudocode from C++ and Rust are similar with minor difference in the example code snippets.

\smalltitle{Generating Code from Pseudocode}
\Cref{lst:query-prompt-zero} is the zero-shot prompt, and \Cref{lst:query-prompt-one} is the one-shot prompt for generating Python code.
The prompts to generate C++ and Rust code are similar with language difference in the one-shot example.

\section{Additional Experiment Results}\label{sec:add-exp-results}
\Cref{fig:compare-config-cpp}, \ref{fig:compare-config-python}, and \ref{fig:compare-config-rust} show the pass@k of code generation from pseudocode from C++, Python, and Rust, respectively, compared with the Pass@k of code generation from the problem.



\putsec{motiv}{Motivation}

\putssec{}{3D Gaussian Splatting on Graphics Hardware}

To understand the performance implications of exploiting the hardware pipeline
for Gaussian splatting, we implement its rendering process using {OpenGL} and
evaluate it across synthetic and real-world scenes on mobile and desktop GPUs.

%%%%%%%%%%%%%%%%%%%%%%%%%%%%%%%%%%%%%%%%%%%%%%%%%%%%%%%%%%%%%%%%%%%%%%%%%%%%%%%%%%%%%%%%%%%%%%%%%%%%%%%
\begin{figure}[t]
  \centering
  \includegraphics[trim=0in 0.15in 0in 0in, clip=True, width=0.95\columnwidth]{figures/opengl-3dgs.pdf}
  \caption{Gaussian splatting rendering via graphics APIs.}
  \vspace{-0.20in}
  \label{fig:opengl-3dgs}
\end{figure}

\myparagraph{Rendering 3D Gaussians via Graphics APIs.}
%
\figref{opengl-3dgs} illustrates how 3D Gaussians are rendered using graphics
APIs. To begin with, we implement custom CUDA kernels and use the NVIDIA CUB
library for the preprocessing and sorting steps, similar to the software-based
rendering~\cite{ker:kop23}. For the subsequent steps, we use {OpenGL} to
implement vertex and fragment shaders that execute on graphics hardware.

In the preprocessing and sorting steps, we first perform frustum culling to
exclude invisible Gaussians and calculate a depth value, which is the $z$-value
of the center of each Gaussian in camera space. We then project (i.e.,
``splat'') the Gaussians onto screen space and compute an RGB color of each
splat using SH coefficients and the viewing direction.
%
The splats are subsequently sorted by the depth values to perform alpha
blending front-to-back in a draw call.

Once we obtain a sorting order of the splats, we exploit fixed-function units
in graphics hardware for rendering. This requires each splat to be represented
as conventional graphics primitives (e.g., triangles). To achieve this, we
specify an Oriented Bounding Box (OBB)~\cite{got:lin96} that surrounds each
splat using two triangles with four vertices.
%
By using the center of the splat, two semi-axis vectors of the ellipse, and the
vertex indices, the vertex shader computes a 2D coordinate for each vertex in
screen space. Each vertex is also assigned the color and opacity values of the
corresponding Gaussian, which were previously computed during the preprocessing
step. 

Subsequently, the hardware rasterizer produces fragments that overlap with the
triangles. Afterward, the fragment shader computes the alpha value of each
fragment by evaluating a Gaussian function at the pixel position
(\eqnref{volumerender}). The fragments whose alpha values are small enough
(i.e., $\alpha<\frac{1}{255}$) are excluded from blending, which we call
\emph{alpha pruning} in this paper, and the remaining fragments are blended
into the corresponding pixels in the ROP units.

\begin{figure}[t]
  \centering
  \includegraphics[trim=0in 0.00in 0in 0in, clip=True, width=\columnwidth]{figures/cuda-vs-opengl.pdf}
  \caption{Performance comparison between software-based (CUDA) and hardware-based (OpenGL) graphics rendering on Jetson AGX Orin and RTX 3090. See~\tabref{workloads} for the details of each scene.}
  \vspace{-0.20in}
  \label{fig:cuda-vs-opengl}
\end{figure}

\myparagraph{Performance Analysis.}
%
\figref{cuda-vs-opengl} compares the performance of the state-of-the-art
software-based rendering using custom CUDA kernels~\cite{ker:kop23} and our
OpenGL-based rendering that exploits fixed-function graphics units.
%
To reduce the amount of work in rasterization, we use a tight OBB\footnote{The
Gaussian's boundary is defined where the alpha value is equal to
$\frac{1}{255}$.} that accounts for the opacity of each Gaussian in both CUDA
and OpenGL implementations. Specifically, for CUDA, this approach significantly
reduces the number of ineffective Gaussian-tile assignments, resulting in a
notable speedup in sorting and rasterization compared to the axis-aligned
bounding box (AABB)-based method used in the original CUDA implementation, as
discussed in previous studies~\cite{lee:lee24,rad:ste24}.
%
Note that this optimization does not alter the rendered image, as it only
reduces ineffective computations.

Results show that using fixed-function units generally offers better
performance than software-based rendering.
%
In software-based rendering, preprocessing is inefficient, requiring per-tile
buffers and duplicating depth and index data for Gaussians spanning multiple
tiles, which is time-consuming.
%
In contrast, hardware-based rendering eliminates the inefficiencies, as the
graphics hardware \emph{automatically manages} tiling and duplication.
%
This also leads to a reduction in sorting time, as we only need to sort the
entire Gaussians by depth values without the need of duplication and per-tile
sorting.
%
Furthermore, as noted in prior work~\cite{lee:lee24}, the lockstep execution of
threads in CUDA leads to \emph{ineffective} alpha computation during
rasterization.
%
Hardware-based rendering, operating at a finer granularity (e.g., a
2$\times$2-fragment quad) than a warp (i.e., 32 threads), allows for more
effective utilization of shader cores and ROPs, thereby improving performance
during the rasterization step.

While rendering using graphics hardware in GPUs is generally faster than
software-based rendering, it still falls short of real-time rendering,
particularly on mobile/edge devices with limited power and resources. 
%
As shown in~\figref{cuda-vs-opengl}(a), hardware-based rendering achieves less
than 25 FPS for real-world scenes.
%
In the following section, we further discuss our observations and the
inefficiencies of Gaussian-based rendering on the hardware pipeline.

\putssec{opportunities}{Observations and Opportunities}

\begin{figure}[t]
  \centering
  \includegraphics[trim=0in 0.00in 0in 0in, clip=True, width=\columnwidth]{figures/unit-utilization.pdf}
  \caption{
%
    Throughput utilization of each hardware unit for OpenGL-based rendering, which is computed by $(\frac{\textrm{Measured
    Throughput}}{\textrm{Max Throughput}}){\times}100$. 
%
  }
  \vspace{-0.20in}
  \label{fig:unit-utilization}
\end{figure}

\textbf{\textit{Observation I: ROP Pressure and Inefficient Use of Shader
Cores.}}
%
\figref{unit-utilization} shows the average throughput utilization of several
key graphics hardware units during a draw call.
%
We observe that the rendering performance of Gaussian splatting is dictated by
the ROP units (i.e., PROP and CROP) rather than by the shader units (SM).
%
This is because there are a huge number of fragments to blend per pixel in
Gaussian splatting.
%
The PROP (Pre-ROP) orchestrates the flow of depth and color fragments for a
final pixel, while the CROP (Color ROP) performs blending in the order of
fragments received from the PROP.

In addition, the shader units are relatively underutilized because of two reasons.
First, the ROP units are the pipeline bottleneck, so the shader units are often
not fully utilized due to back pressure. Second, the vertex and fragment
shaders used for Gaussian splatting are relatively simple compared to
CUDA-based renderers. In detail, the vertex shader only computes the 2D screen
coordinate using the center of the splat and axis vectors, and the vertex
colors are shared for all vertices of the splat, which are already computed in
the preprocessing step. 
%
The fragment shader computes the alpha value of each fragment by applying a dot
product to a normalized pixel coordinate and performing an exponential
operation of the Gaussian function.
%
Compared to other shading programs that require lighting calculations and
texture fetching for each fragment~\cite{kru:wes03}, the shaders for Gaussian
splatting are computationally cheaper.

%%%%%%%%%%%%%%%%%%%%%%%%%%%%%%%%%%%%%%%%%%%%%%%%%%%%%%%%%%%%%%%%%%%%%%%
\begin{figure}[t]
  \centering
  \includegraphics[trim=0in 0.00in 0in 0in, clip=True, width=\columnwidth]{figures/frag-heatmap.pdf}
  \caption{Number of fragments per pixel with and without early termination (scene: Bonsai).}
  \vspace{-0.20in}
  \label{fig:frag-heatmap}
\end{figure}

\myparagraph{\textit{Observation II: Challenges of Supporting Early
Termination.}}
% 
Gaussian splatting is a volume rendering technique that accumulates RGBA colors
of fragments at the same pixel location \emph{front-to-back} to produce the
final pixel color.
%
Consequently, similar to other volume rendering methods, it can benefit from
\emph{early termination}, a widely-used optimization technique in the volume
rendering process.
%
\figref{frag-heatmap} illustrates the impact of early termination on Gaussian
splatting by comparing the number of fragments per pixel blended with
\emph{and} without early termination.
%
With early termination, if the alpha value of a pixel surpasses a predefined
threshold after blending, the subsequent fragments are discarded as they do not
make a noticeable contribution to the final pixel color.
%
As a result, this technique effectively reduces the amount of computation by
avoiding unnecessary shading and blending operations.

\begin{figure}[t]
  \centering
  \includegraphics[trim=0in 0.00in 0in 0in, clip=True, width=\columnwidth]{figures/ert-in-cuda.pdf}
  \caption{Speedup of CUDA-based rendering and the reduction in the number of fragments with early termination.}
  \vspace{-0.10in}
  \label{fig:ert-in-cuda}
\end{figure}

\begin{figure}[t]
  \centering
  \includegraphics[trim=0in 0.00in 0in 0in, clip=True, width=\columnwidth]{figures/warp-divergence.pdf}
  \caption{Average percentage of threads in a warp that participate in blending in software-based (CUDA) rendering.}
  \vspace{-0.15in}
  \label{fig:warp-divergence}
\end{figure}


\figref{ert-in-cuda} shows the speedup of CUDA-based rendering and the
reduction in per-pixel fragment count when using early termination.
%
In software-based rendering, while early termination helps improve rendering
performance, its full benefits are difficult to realize due to the lockstep
execution of threads working on different pixels.
%
In the worst case, even if only one thread (pixel) in a warp is not terminated,
all other threads in the warp still ineffectively consume shader cores
until the active thread finishes.
%
\figref{warp-divergence} shows the percentage of threads in a warp performing
blending operations for the evaluated scenes.
%
With the combined effects of alpha pruning and early termination, less than
40\% of threads perform effective work (i.e., blending) in a warp across all
scenes.

On the other hand, hardware-based rendering currently does \emph{not} natively
support early termination. However, adding this capability to graphics hardware
would significantly improve rendering performance for two reasons while better
exploiting its benefits compared to software-based rendering.
%
First, the hardware graphics pipeline performs fragment blending at a finer
granularity in ROPs (i.e., a quad; 2$\times$2 fragments) than a warp, thereby
enabling more effective utilization of compute units when early termination is
applied than software-based rendering.
% 
Second, since volume rendering places substantial pressure on ROPs, reducing
the number of fragments sent to them would greatly streamline the hardware
pipeline by alleviating their processing burden.
%
To harness this benefit in the hardware pipeline, \ssecref{early-term-hw}
discusses our proposed architecture, which minimally extends the graphics
hardware by \emph{repurposing} fixed-function units, such as stencil test
hardware.

In the following section, we discuss the performance of in-shader blending and
software-based early termination via graphics APIs to demonstrate the need for
architectural support for volume rendering.

\input{listings/case-simp-common-alg-3233}

\begin{figure*}
\centering
\begin{lstlisting}[breaklines=true, language=Rust, frame=shadowbox, numbers=left,]
impl Solution {
  pub fn maximum_points(mut arr: Vec<i32>, mut curr: i32) -> i64 {
    arr.sort_unstable();
    let mut res = 0;
    let mut pos = arr.len() - 1;

    while pos <= arr.len() && arr[0] <= curr {
      res += (curr / arr[0]) as i64;
      curr %
      curr += arr[pos];
      pos -= 1;
    }

    return res;
  }
}
\end{lstlisting}
\captionof{lstlisting}{Example user-submitted Rust code with subtraction underflow problem (line 11)}
\label{lst:rust-underflow}
\end{figure*}

\begin{figure*}
\centering
\begin{lstlisting}[breaklines=true, frame=shadowbox, numbers=left,]
function maximum_points(arr, curr)
    sort arr in ascending order
    res = 0
    pos = arr.len() - 1

    while pos >= 0 and arr[0] <= curr:
        res += curr // arr[0]
        curr = curr %
        curr += arr[pos]
        pos -= 1

    return res
\end{lstlisting}
\captionof{lstlisting}{Pseudocode converted from \Cref{lst:rust-underflow}}
\label{lst:pseudo-rust-underflow}
\end{figure*}

\begin{figure*}
\centering
\begin{lstlisting}[breaklines=true, language=Rust, frame=shadowbox, numbers=left,]
impl Solution {
    pub fn maximum_points(enemy_energies: Vec<i32>, current_energy: i32) -> i64 {
        let mut arr = enemy_energies.clone();
        arr.sort();
        let mut res: i64 = 0;
        let mut pos = arr.len() as isize - 1;
        let mut curr = current_energy;

        while pos >= 0 && arr[0] <= curr {
            res += (curr / arr[0]) as i64;
            curr = curr %
            if pos >= 0 {
                curr += arr[pos as usize];
            }
            pos -= 1;
        }

        res
    }
}
\end{lstlisting}
\captionof{lstlisting}{Correct Rust code implemented from \Cref{lst:pseudo-rust-underflow}}
\label{lst:rust-fix-underflow}
\end{figure*}



\begin{figure*}
\centering
\begin{lstlisting}[breaklines=true, language=Python, frame=shadowbox, numbers=left,]
class Solution:
    def maximumSubarraySum(self, nums: List[int], k: int) -> int:
        max_sum, prefix_sum, val_to_min_prefix_sum = -inf, 0, defaultdict(lambda: inf)
        for i, num in enumerate(nums):
            if val_to_min_prefix_sum[num] > prefix_sum:
                val_to_min_prefix_sum[num] = prefix_sum
            prefix_sum += num
            max_sum = max(max_sum, prefix_sum - val_to_min_prefix_sum[num + k], prefix_sum - val_to_min_prefix_sum[num - k])
        return max_sum if max_sum > -inf else 0 
\end{lstlisting}
\captionof{lstlisting}{Python code whose converted pseudocode has negative impact}
\label{lst:python-acc}
\end{figure*}

\begin{figure*}
\centering
\begin{lstlisting}[breaklines=true, frame=shadowbox, numbers=left,]
function circularGameLosers(n, k):
    s = {1, 2, ..., n}
    for each mul in cumulative sums of integers from 0 to n-1:
        position = (k * mul) mod n + 1
        if position not in s:
            break
        remove position from s
    return s as a list
\end{lstlisting}
\captionof{lstlisting}{Pseudocode converted from \Cref{lst:python-acc}}
\label{lst:pseudo-python-acc}
\end{figure*}



\begin{figure*}
\centering
\begin{lstlisting}[breaklines=true, language=Python, frame=shadowbox, numbers=left,]
class Solution:
    def circularGameLosers(self, n: int, k: int) -> List[int]:
        players = set(range(1, n + 1))
        mul = 0
        while True:
            position = (k * mul) %
            if position not in players:
                break
            players.remove(position)
            mul += 1
        return sorted(players)
\end{lstlisting}
\captionof{lstlisting}{Wrong Python code generated from \Cref{lst:pseudo-python-acc} }
\label{lst:python-acc-wrong-gen}
\end{figure*}


\input{listings/python-being-wrongly-pseudogen}
\begin{figure*}
\centering
\begin{lstlisting}[breaklines=true, frame=shadowbox, numbers=left,]
function maximumSubarraySum(nums, k):
    max_sum = -infinity
    prefix_sum = 0
    val_to_min_prefix = empty map  // default value is infinity
    for each num in nums:
        if val_to_min_prefix[num] > prefix_sum:
            val_to_min_prefix[num] = prefix_sum
        prefix_sum += num
        candidate1 = prefix_sum - val_to_min_prefix[num + k]
        candidate2 = prefix_sum - val_to_min_prefix[num - k]
        max_sum = max(max_sum, candidate1, candidate2)
    return max(max_sum, 0)
\end{lstlisting}
\captionof{lstlisting}{Pseudocode converted from \Cref{lst:python-being-wrongly-pseudogen} with errors (line 12) }
\label{lst:pseudo-wrong-condition}
\end{figure*}


\begin{figure*}
\centering
\footnotesize
\begin{lstlisting}
I am a Python programmer.
Please help me convert Python code into a semantic-preserving and concise pseudocode.
Instead of translating line by line, you should simplify the pseudocode as much as possible and also readable.
Below are specific rules:

1. Use indents to represent control structures.
```
if a == b:
    c += 1
```

2. The pseudocode should not be tied to a specific programming language and should not contain any language-specific stuffs such as `yield` in Python.

3. The pseudocode does not need to preserve concrete type info: (a) The concrete names such as `vector` and `i64` should not appear. Usually, general names such as array/list and int are enough for describing algorithms. (b) Do not involve type casting.

4. You should omit the implementation of common algorithms/data structures/operations.

For example, the customized binary search subroutine
```
def search_square_geq(nums, val):
    left = 0
    right = len(nums) - 1
    while left < right:
        mid = left + (right - left) // 2
        if nums[mid]**2 < val:
            left = mid + 1
        else:
            right = mid
    return left
    
target = search_square_geq(xs, 9)
```
can be simplified as
```
target = binary search for the index i such that xs[i] * xs[i] >= 9
```

5. You can use natural language to simplify code, in particular loops. For example,
```
for x in xs:
    if x == 233:
        flag = true
```
can be simplified as `flag = whether 233 exists in xs`

6. Do not use natural language if that is verbose. For example, `let n be the size of list_a` is less compact and readable than `n = list_a.size()`

7. A function definition should be formatted like `function max(a, b)`. Functions can be nested and can use variables in the outer scope.

Finally, recall that the principles are **semantic-preserving** and **concise and readable**.
Do not change the name of the given function.
You can iterate the writing of pseudocode to ensure it follows the above rules.
Wrap only the final version with code blocks (```) in the response.

Below is the Python code to convert into pseudocode.
{code}
\end{lstlisting}
\captionof{lstlisting}{Prompt (a single user query) to generate pseudocode from DeepSeek-R1 }\label{lst:pseudogen-prompt}
\end{figure*}

\begin{figure*}
\centering
\footnotesize
\begin{lstlisting}
===System===
You are a proficient {lang} programmer and familiar with various algorithms.
Your task is to implement a {lang} code given a pseudocode illustrating an algorithm and a {lang} signature.
If some subroutines are provided in the standard library, you can use them.
For subroutines that are not provided in the standard library, you need to implement them.

===User===
```
{pseudocode}
```

Please implement the above pseudocode with the following template.
```
{signature}
```
\end{lstlisting}
\captionof{lstlisting}{Zero-shot prompt to generate code from pseudocode  }\label{lst:query-prompt-zero}
\end{figure*}

\begin{figure*}
\centering
\footnotesize
\begin{lstlisting}
===System===
You are a proficient Python programmer and familiar with various algorithms.
Your task is to implement a Python code given a pseudocode illustrating an algorithm and a Python signature.
If some subroutines are provided in the standard library, you can use them.
For subroutines that are not provided in the standard library, you need to implement them.
Below is a simple example.

===User===
```pseudocode
Input:
    nums is a list of positive integers;
    k is a positive integer
Output:
    return x such that x is the smallest in nums and x * x > k
function findTarget(nums, k):
    sorted = sort nums in ascending order
    res = binary search on sorted for the leftmost value res such that res * res > k
    return res
```

Please implement the function with the following template.
```python
class Solution:
    def findTarget(self, nums: List[int], k: int) -> int:
        
```

===Assistant===
The pseudocode involves two subroutines, sort and binary search.
Sort can be implemented using the standard libray, and binary search needs to be implemented by ourselves.
Finally, the Python implementation is as follows.

```python
class Solution:
    def findTarget(self, nums: List[int], k: int) -> int:
        nums = sorted(nums)
        left = 0
        right = len(nums) - 1
        while left < right:
            mid = left + (right - left) // 2
            if nums[mid] * nums[mid] <= k:
                left = mid + 1
            else:
                right = mid
        return nums[left]
```

===User===
```
{pseudocode}
```

Please implement the function with the following template.
```
{signature}
```
\end{lstlisting}
\captionof{lstlisting}{One-shot prompt to generate code from pseudocode  }\label{lst:query-prompt-one}
\end{figure*}


\begin{figure*}[t]
\centering
    \begin{subfigure}[b]{\textwidth}
         \centering
         \includegraphics[width=0.6\linewidth]{figs/compare_config_from_C++_to_C++.pdf} %
         \caption{To C++}
    \end{subfigure}
    \begin{subfigure}[b]{\textwidth}
         \centering
         \includegraphics[width=0.6\linewidth]{figs/compare_config_from_C++_to_Python.pdf}
         \caption{To Python}
    \end{subfigure}
    \begin{subfigure}[b]{\textwidth}
         \centering
         \includegraphics[width=0.6\linewidth]{figs/compare_config_from_C++_to_Rust.pdf}
         \caption{To Rust}
    \end{subfigure}
  \caption {Pass@k of code generation from pseudocode from C++ to all languages, compared with direct generation from problems}
  \label{fig:compare-config-cpp}
\end{figure*}

\begin{figure*}[t]
\centering
    \begin{subfigure}[b]{\textwidth}
         \centering
         \includegraphics[width=0.6\linewidth]{figs/compare_config_from_Python_to_C++.pdf} %
         \caption{To C++}
    \end{subfigure}
    \begin{subfigure}[b]{\textwidth}
         \centering
         \includegraphics[width=0.6\linewidth]{figs/compare_config_from_Python_to_Python.pdf}
         \caption{To Python}
    \end{subfigure}
    \begin{subfigure}[b]{\textwidth}
         \centering
         \includegraphics[width=0.6\linewidth]{figs/compare_config_from_Python_to_Rust.pdf}
         \caption{To Rust}
    \end{subfigure}
  \caption {Pass@k of code generation from pseudocode from Python to all languages, compared with direct generation from problems}
  \label{fig:compare-config-python}
\end{figure*}

\begin{figure*}[t]
\centering
    \begin{subfigure}[b]{\textwidth}
         \centering
         \includegraphics[width=0.6\linewidth]{figs/compare_config_from_Rust_to_C++.pdf} %
         \caption{To C++}
    \end{subfigure}
    \begin{subfigure}[b]{\textwidth}
         \centering
         \includegraphics[width=0.6\linewidth]{figs/compare_config_from_Rust_to_Python.pdf}
         \caption{To Python}
    \end{subfigure}
    \begin{subfigure}[b]{\textwidth}
         \centering
         \includegraphics[width=0.6\linewidth]{figs/compare_config_from_Rust_to_Rust.pdf}
         \caption{To Rust}
    \end{subfigure}
  \caption {Pass@k of code generation from pseudocode from Rust to all languages, compared with direct generation from problems}
  \label{fig:compare-config-rust}
\end{figure*}

