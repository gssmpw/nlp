\begin{figure*}
\centering
\footnotesize
\begin{lstlisting}
You are given an integer array nums and an integer k.
An integer h is called valid if all values in the array that are strictly greater than h are identical.
For example, if nums = [10, 8, 10, 8], a valid integer is h = 9 because all nums[i] > 9 are equal to 10, but 5 is not a valid integer.
You are allowed to perform the following operation on nums:

Select an integer h that is valid for the current values in nums.
For each index i where nums[i] > h, set nums[i] to h.

Return the minimum number of operations required to make every element in nums equal to k. If it is impossible to make all elements equal to k, return -1.
 
Example 1:

Input: nums = [5,2,5,4,5], k = 2
Output: 2
Explanation:
The operations can be performed in order using valid integers 4 and then 2.

Example 2:

Input: nums = [2,1,2], k = 2
Output: -1
Explanation:
It is impossible to make all the values equal to 2.

Example 3:

Input: nums = [9,7,5,3], k = 1
Output: 4
Explanation:
The operations can be performed using valid integers in the order 7, 5, 3, and 1.

 
Constraints:

1 <= nums.length <= 100 
1 <= nums[i] <= 100
1 <= k <= 100
\end{lstlisting}
\captionof{lstlisting}{Full problem in the motivating example}\label{lst:problem-motiv}
\end{figure*}

\begin{figure*}
\centering
\begin{lstlisting}[breaklines=true, language=C++, frame=shadowbox, numbers=left,]
class Solution {
public:
    int minOperations(vector<int>& nums, int k) {
        int mn = *min_element(nums.begin(), nums.end()); 
        if (mn < k) {
            return -1; 
        }
        unordered_map<int,int> mp; 
        for (auto &it: nums) {
            mp[it] = 1; 
        }
        int ans = mp.size(); 
        if (mp[k]) {
            ans--; 
        }
        return ans; 
    }
};
\end{lstlisting}
\captionof{lstlisting}{User-submitted C++ solution to \Cref{lst:problem-motiv}}
\label{lst:cpp-motiv}
\end{figure*}
