\begin{figure*}
\centering
\footnotesize
\begin{lstlisting}
I am a Python programmer.
Please help me convert Python code into a semantic-preserving and concise pseudocode.
Instead of translating line by line, you should simplify the pseudocode as much as possible and also readable.
Below are specific rules:

1. Use indents to represent control structures.
```
if a == b:
    c += 1
```

2. The pseudocode should not be tied to a specific programming language and should not contain any language-specific stuffs such as `yield` in Python.

3. The pseudocode does not need to preserve concrete type info: (a) The concrete names such as `vector` and `i64` should not appear. Usually, general names such as array/list and int are enough for describing algorithms. (b) Do not involve type casting.

4. You should omit the implementation of common algorithms/data structures/operations.

For example, the customized binary search subroutine
```
def search_square_geq(nums, val):
    left = 0
    right = len(nums) - 1
    while left < right:
        mid = left + (right - left) // 2
        if nums[mid]**2 < val:
            left = mid + 1
        else:
            right = mid
    return left
    
target = search_square_geq(xs, 9)
```
can be simplified as
```
target = binary search for the index i such that xs[i] * xs[i] >= 9
```

5. You can use natural language to simplify code, in particular loops. For example,
```
for x in xs:
    if x == 233:
        flag = true
```
can be simplified as `flag = whether 233 exists in xs`

6. Do not use natural language if that is verbose. For example, `let n be the size of list_a` is less compact and readable than `n = list_a.size()`

7. A function definition should be formatted like `function max(a, b)`. Functions can be nested and can use variables in the outer scope.

Finally, recall that the principles are **semantic-preserving** and **concise and readable**.
Do not change the name of the given function.
You can iterate the writing of pseudocode to ensure it follows the above rules.
Wrap only the final version with code blocks (```) in the response.

Below is the Python code to convert into pseudocode.
{code}
\end{lstlisting}
\captionof{lstlisting}{Prompt (a single user query) to generate pseudocode from DeepSeek-R1 }\label{lst:pseudogen-prompt}
\end{figure*}
