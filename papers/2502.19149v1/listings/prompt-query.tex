\begin{figure*}
\centering
\footnotesize
\begin{lstlisting}
===System===
You are a proficient {lang} programmer and familiar with various algorithms.
Your task is to implement a {lang} code given a pseudocode illustrating an algorithm and a {lang} signature.
If some subroutines are provided in the standard library, you can use them.
For subroutines that are not provided in the standard library, you need to implement them.

===User===
```
{pseudocode}
```

Please implement the above pseudocode with the following template.
```
{signature}
```
\end{lstlisting}
\captionof{lstlisting}{Zero-shot prompt to generate code from pseudocode  }\label{lst:query-prompt-zero}
\end{figure*}

\begin{figure*}
\centering
\footnotesize
\begin{lstlisting}
===System===
You are a proficient Python programmer and familiar with various algorithms.
Your task is to implement a Python code given a pseudocode illustrating an algorithm and a Python signature.
If some subroutines are provided in the standard library, you can use them.
For subroutines that are not provided in the standard library, you need to implement them.
Below is a simple example.

===User===
```pseudocode
Input:
    nums is a list of positive integers;
    k is a positive integer
Output:
    return x such that x is the smallest in nums and x * x > k
function findTarget(nums, k):
    sorted = sort nums in ascending order
    res = binary search on sorted for the leftmost value res such that res * res > k
    return res
```

Please implement the function with the following template.
```python
class Solution:
    def findTarget(self, nums: List[int], k: int) -> int:
        
```

===Assistant===
The pseudocode involves two subroutines, sort and binary search.
Sort can be implemented using the standard libray, and binary search needs to be implemented by ourselves.
Finally, the Python implementation is as follows.

```python
class Solution:
    def findTarget(self, nums: List[int], k: int) -> int:
        nums = sorted(nums)
        left = 0
        right = len(nums) - 1
        while left < right:
            mid = left + (right - left) // 2
            if nums[mid] * nums[mid] <= k:
                left = mid + 1
            else:
                right = mid
        return nums[left]
```

===User===
```
{pseudocode}
```

Please implement the function with the following template.
```
{signature}
```
\end{lstlisting}
\captionof{lstlisting}{One-shot prompt to generate code from pseudocode  }\label{lst:query-prompt-one}
\end{figure*}
