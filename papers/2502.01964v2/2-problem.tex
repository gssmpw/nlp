\section{Model, Problem, and Related Work}
\label{sec:problem}

In this section, we define the quantum network model and the problem, as well as discuss the related work.

\subsection{Quantum Network Model} 
We define a quantum network as a graph $G=(V, E)$, with $V=\{v_1, ..., v_n\}$ and $E=\{e(v_i, v_j)\}$ denoting the set of nodes and links, respectively.
Pairs of nodes connected by a link are defined as \emph{neighbor} nodes.
An EP between neighbors (i.e., on a link $e(u, v)$) is called an ``elementary link EP'' or a ``link EP'' for short.
% A node with two neighbors is called a quantum repeater, while a node with three or more neighbors is called quantum router \CZ{differentiating repeater and router is irrelevant}.
Each node has a finite number of emissive quantum memories.
These memories generate atom-photon entanglement, where the atomic part of the EP is stored in the memory and the photonic part is transmitted to a distant device for photonic Bell-state measurement (BSM).
The atomic part of the EP has a time-dependent decoherence model.
Moreover, each memory has a maximum photon emission rate and multiple memories can emit photons simultaneously.
Entanglement swapping and purification are performed on the atomic memory qubits.

Each link $e(u, v)$ in the quantum network is comprised of two quantum channels and four classical channels.
The two quantum channels connect nodes $u$ and $v$ to the middle photonic BSM device, where photons emitted by quantum memories are measured.
Two of the four classical channels connect the photonic BSM device to both $u$ and $v$ to send the measurement results,
while the other two classical channels connect $u$ to $v$ and vice versa to send messages related to protocol coordination.
We assume lossless classical communication, while quantum bits have three sources of loss (see Table~\ref{tab:sim-parameter} in \S\ref{sec:result}): (1) failure during the memory emission, (2) loss during the transmission in the quantum channel, and (3) failure of the detector in the photonic BSM device. 
%\red{See .}




\subsubsection{User Request and Time-to-serve (TTS)} 
Fig.~\ref{fig:request} shows an example of a \emph{user request} arriving
at an \texttt{initiator} node, with the following attributes: \texttt{responder}, \texttt{start time}, \texttt{end time}, \texttt{number of EPs}, and \texttt{fidelity}.
The request asks the quantum network to generate a certain number of end-to-end EPs between the initiator node and the responder node with fidelity above a threshold within a time range [\texttt{start time}, \texttt{end time}].
If the generation of the requested end-to-end EPs succeeds at time $t$ and $t$ is within the time range, then the request's time-to-serve (TTS) is
\begin{equation}
    \mathrm{TTS} = t - \mathrm{start~time}
    \label{eqn:tts}
\end{equation}
i.e., the time elapsed from the request start time to the time $t$ when all the end-to-end EPs are distributed.
% If $t$ is larger than the end time, then the request fails to serve.
The TTS includes the time for elementary link entanglement generation and entanglement swapping.


\subsubsection{Entanglement Path Computation} 
%The TTS will definitely be affected by 
The path computed by the entanglement routing algorithm directly affects the TTS. 
For example, consider end nodes nodes 0 and 4 in Fig.~\ref{fig:request}; 
the path (0,2,4) will have a smaller expected TTS than path (0,1,2,5,4).
In this work, %we assume each node has a \emph{distributed static routing protocol} where the forwarding table on each node is initialized 
\red{we assume all nodes use static routing (i.e., the forwarding table of each node is pre-configured following the shortest distance path and remains fixed throughout the simulation).}
% \CZ{confusing}the forwarding tables of each node are populated at initialization by running Dijkstra's algorithm using the link's physical length as the cost. %) at the configuration time in the beginning.
We assume there is a small time interval between the request arrival at the initiator node and the request's start time.
The path computation starts after the request's arrival and is quickly completed before the request's start time.
Thus, the time for path computation is not included in the TTS. %, and the TTS starts counting from the elementary link entanglements generation given the computed path.

\begin{figure}
    \centering
    \includegraphics[width=0.75\linewidth]{figures/request.png}
    \vspace{-0.1in}
    \caption{Toy example of a quantum network, request, and time to serve (TTS).
    \red{A request arrives at node 0 (initiator), asking node 0 to generate 1 EP with node 4 (responder) within the [1, 2] seconds time range, and requires the fidelity greater than 0.7.}
    % A request to generate one EP with fidelity greater than 0.7 between nodes 0 (initiator) and 4 (responder) arrives~\red{[issue]} within the [1, 2] seconds time range to node 0.
    % The quantum network attempts to generate the EP between nodes 0 and 4 starting at time $t = 1$ second.
    If the EP is generated at time $t = 1.2$ seconds, then the TTS of this request is 0.2 seconds.}
    \vspace{-0.1in}
    \label{fig:request}
\end{figure}

\subsubsection{Motivation} 
We observe that a major component of the TTS is the time to generate link EPs. 
We aim to reduce this time by pre-generating EPs before the request's start time,
such that when a request arrives, pre-generated EPs can be immediately used.
However, nodes do not have the request's information beforehand; thus, nodes ought to continuously generate EPs with their neighbors regardless of incoming requests.
Inspired by~\cite{kolar-ac-infocom22}, an adaptive control scheme is introduced to guide the nodes to select more frequently used links to improve TTS.
% In summary, the goal is to design a continuous entanglement generation protocol with adaptiveness that works in tandem with the on-demand entanglement generation process.
% \footnote{In this paper, the on-demand entanglement generation process assumes the path is computed by a static entanglement routing protocol. However, the path computation could also use more advanced entanglement routing algorithms.}.


\subsection{Problem Statement} 
%\softpara{Setting.} 
We assume a quantum network with a \emph{distributed}, \emph{online}, and \emph{asynchronous} setting:
\begin{itemize}
    \item \textit{Distributed:} Each node only has a local view of the network (i.e., its neighbors).
    \item \textit{Online:} Each node can be adaptive to request arrival patterns and available elementary link EPs, which are unknown in advance.
    \item \textit{Asynchronous:} %There is no concept of time slots. 
    The design and simulation of protocols operate in discrete events (i.e., no time slots are assumed).
\end{itemize}
Under the above setting, the problem is to design and implement a continuous entanglement generation protocol, working in conjunction with an entanglement routing protocol, to reduce the request's TTS.
The proposed protocol will allow a node to continuously generate EPs with its neighbor regardless of request arrivals.
As a result, when a request arrives and a path has been computed, pre-generated EPs could be immediately used instead of generating EPs on demand after the request's start time.
% each node only knows neighbor nodes and communicates with neighbor nodes only.
% Also, each node does not know the request pattern in advance.
A naive protocol could randomly select neighbors to generate EPs with.
%To improve efficiency
However, an adaptive control that leverages the request information from the past can guide a node to select neighbors more wisely.
This approach will generate link EPs that have a higher chance of being used to serve future requests, thus improving the efficiency of continuous EP generation (see Fig.~\ref{fig:acp-highlevel}).
% Given both the adaptiveness and continuousness, we call the protocol \emph{Adaptive, Continuous entanglement generation Protocol} (\acp).


\begin{figure}
    \centering
    \includegraphics[width=0.9\linewidth]{figures/acp-highlevel.png}
    \vspace{-0.1in}
    \caption{Toy example of the problem. The \acp\ runs on Node-A, which has five neighbors in total. \acp\ is allowed to use a maximum of four quantum memories for continuous generation of link EPs. Node-A in (a) has to decide what neighbors to generate EP with.
    In (b), the user requests result in paths (computed from the entanglement routing algorithm) that frequently include the segments [2, A, 3] and [3, A, 5]. Thus, Node-A should select neighbor Node-3 the most, followed by Node-2 and Node-5.}
    \vspace{-0.1in}
    \label{fig:acp-highlevel}
\end{figure}



\subsection{Related Work}

The \acp\ is complementary to entanglement routing, which can be performed in an on-demand or continuous manner. 

In on-demand entanglement routing, entanglement generation only starts after a request arrives~\cite{chakra-routing-19}. The typical objective is to find an optimal path, together with a swapping policy (i.e., the order of entanglement swapping), to satisfy throughput, latency, and fidelity requirements for a user request~\cite{abane-routingsurvey}.
Caleffi~\cite{caleffi-routing-access17} proposed an exponential runtime algorithm that examines all possible paths with a metric assuming a balanced tree swapping policy. Shi and Qian~\cite{shi-routing-sigcomm20} created a Dijkstra-like algorithm with a metric considering multiple quantum channels between neighbor nodes and they considered a sequential swapping policy. Ghaderibaneh~\cite{ghader-swappingtree-tqe22,fan-purification} used dynamic programming to find the optimal ``combination of path and swapping policy'', i.e., ``swapping tree''. 

On-demand entanglement routing has also been explored beyond the bipartite regime. In general, the distribution of multipartite entanglement across a network~\cite{pirker-multipartite-19,fan-graphstate} starts from the generation of small entangled states shared by few parties (e.g., normal 2-qubit Bell state or 3-qubit GHZ states~\cite{lee-graphstate}). Then multipartite entanglement routing mainly deals with the optimal policy for assembling the small entangled states into large multipartite entangled states.

In continuous entanglement routing, elementary link EPs are generated continuously over all links in the background~\cite{abane-routingsurvey}. Then, the routing algorithm computes the path on the instant logical topology formed by the created link entanglements~\cite{cicco-routing-21}. Kolar et al~\cite{kolar-ac-infocom22} first considered continuous generation of elementary link EPs guided by an adaptive scheme, but the evaluation was ad hoc and quantum memory decoherence was not considered. 

Continuous entanglement generation can also be extended from between neighbor nodes (physical link) to between non-neighbor nodes (virtual link), with the help of entanglement swapping. This allows the network to go beyond the physical connection and create arbitrary virtual topologies, which offers the potential to increase the connectivity of the network, reduce the diameter of the network, and ultimately reduce the latency of end-to-end entanglement generation~\cite{ghader-predistribution-qce22}. Inesta and Wehner~\cite{inesta-continuous-pra23} defined two metrics to measure the quality of continuous entanglement generation and distribution protocols: (1) virtual neighborhood size and (2) virtual node degree. The higher the values of the two metrics are, the better the continuous protocol is.

\begin{comment}
The \acp\ is complementary to entanglement routing, so this section starts with entanglement routing, including on-demand and continuous entanglement routing.
Then, virtual links and continuous entanglement distribution are discussed.

\subsubsection{On-demand Entanglement Routing}
In on-demand entanglement routing, entanglement generation does not start until a request arrives~\cite{chakra-routing-19}. 
The typical objective is to find an optimal path, 
% with allocated capacity for each edge in the path, 
together with a swapping policy (i.e., order of entanglement swapping operation) to satisfy throughput, latency, and fidelity requirements for a user request~\cite{abane-routingsurvey}.
Caleffi~\cite{caleffi-routing-access17} uses an exponential runtime algorithm (examine all possible paths) in which the path metric is based on a balanced tree swapping policy, where the order of swapping corresponds to a balanced tree of height $\lceil \log_2 n \rceil$.
Shi and Qian~\cite{shi-routing-sigcomm20} use a Dijkstra-like algorithm with a metric considering multiple quantum channels between neighbor nodes and they consider a sequential swapping policy.
Ghaderibaneh et al.~\cite{ghader-swappingtree-tqe22} use dynamic programming to find the optimal ``combination of path and swapping policy'' that they call ``swapping tree''.
Note that for a path with heterogeneous links, the balanced tree swapping policy used in~\cite{caleffi-routing-access17} is not necessarily optimal~\cite{ghader-swappingtree-tqe22}.
\end{comment}
% For the swapping policy,
% The synchronous setting is assumed in~\cite{shi-routing-sigcomm20}, while asynchronous setting is assumed in~\cite{calaffi-dqc-cn24,ghader-swappingtree-tqe22}.
% Synchronous setting assumes that all the link EP generation and swapping operations must succeed in a ``small time slot''.
% Asynchronous setting does does not consider time slot and each EP generation and swapping operates independently.
% A qubit of an EP may wait for its counterpart to become available so that a swapping operation can be performed~\cite{ghader-swappingtree-tqe22}.



\begin{figure*}[ht]
    \centering
    \includegraphics[width=0.96\linewidth]{figures/acp-fsm.png}
    \vspace{-0.1in}
    \caption{The two FSMs of \acp. (a) depicts the FSM at Node-A, while (b) shows the FSM at Node-A's neighbors.
    Note that \acp\ is a symmetric \red{peer-to-peer} protocol, so the FSM in (a) is also running on Node-A's neighbor, and the FSM in (b) is also running on Node-A.
    % In other words, nodes could assume the role of ``Node-A'' or ``Node-$x$'' depending on the situation.
    \red{This figures depicts ``one connection''. We achieved multiple connections by having multiple $S_2$ states in the Node-A FSM.}
    }
    \vspace{-0.1in}
    \label{fig:acp-fsm}
\end{figure*}

% Multipartite entanglement routing has been explored beyond the aforementioned bipartite entanglement routing work.
% To generate multipartite entanglement across a network~\cite{fan-graphstate}, first we must generate bipartite EPs (the base resource), then we must find an optimal fusion policy via linear programming to fuse smaller graph states into larger graph states.
% In~\cite{lee-graphstate}, 3-qubit GHZ states are used as the base resource to generate graph states.

% Fidelity describes the quality of the entanglement and is an important metric in entanglement routing.
% Since the fidelity of an entanglement drops with each swapping, a minimum fidelity requirement can be converted to a path length constraint~\cite{chakra-routing-20}.
% Another source of fidelity drop is the time-dependent decoherence. 
% The decoherence risk can be handled by discarding qubits that aged beyond some cutoff time.
% Quantum network simulators such as~\cite{netsquid,sequence} provide physical models to compute the decoherence process.


\begin{comment}

\subsubsection{Continuous Entanglement Routing}
In continuous entanglement routing, elementary link EPs are generated continuously over all links in the background~\cite{abane-routingsurvey}.
Then the routing algorithm computes the path on the instant logical topology formed by the created link entanglements~\cite{cicco-routing-21}.
Here, the logical topology is dynamic because the generation of link entanglements is probabilistic.
Continuous generation of elementary link EPs is guided by an adaptive scheme in~\cite{kolar-ac-infocom22}, followed by a distributed local best-effort routing algorithm in~\cite{chakra-routing-19}.


\subsubsection{Virtual Link}
A virtual link is an EP between non-neighbor nodes.
Continuous entanglement generation can be extended from neighbor link entanglement (physical link) to non-neighbor (virtual link) entanglement with the help of entanglement swapping.
This allows the network to go beyond the physical connection and create arbitrary virtual topologies.
Wisely selecting and pre-generating virtual links can increase the connectivity of the network, reduce the diameter of the network, and ultimately reduce the latency of end-to-end entanglement generation~\cite{ghader-predistribution-qce22}.
% Chakraborty~\cite{chakra-routing-19} considers a hybrid routing in a graph with both physical links and virtual links.
% The routing algorithm prioritizes virtual links and only picks physical link generation on-demand when virtual links are unavailable.

\subsubsection{Continuous Distribution of Entanglement}
The goal of protocols for continuous distribution (CD) of entanglement is to provide a continuous supply of entanglement for nodes to run applications without the need for explicitly demanding entanglement~\cite{inesta-continuous-pra23}.
Inesta and Wehner~\cite{inesta-continuous-pra23} define two metrics to measure the quality of CD protocols: (1) virtual neighborhood size and (2) virtual node degree.
The higher the value of the two metrics, the better the CD protocol.
CD and on-demand entanglement generation can co-exist: CD will supply the EPs to applications running at a constant rate, while on-demand support this process during peak demands from sporadic applications.

\end{comment}
