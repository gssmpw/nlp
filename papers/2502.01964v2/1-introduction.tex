\section{Introduction}
\label{sec:intro}

Quantum communications hold enormous potential for many ground-breaking scientific and technological advances. 
A quantum network~\cite{wehner2018quantum} serves as the backbone for applications such as % the transmission of an unknown quantum state~\cite{hu-teleportation-nrp23}, 
distributed quantum computing~\cite{calaffi-dqc-cn24} and distributed quantum sensing~\cite{zang-dqs,zhan-opt-24,hillery-qsn-pra23,zhan-localization-qce23}. 
A core functionality of a quantum network is to distribute entangled pairs (EPs) between two distant network nodes, providing a key resource for quantum network applications.
However, distributing an EP across remote nodes can introduce significant latency due to the probabilistic nature of the underlying physical processes. 
% When a user requests the quantum network for EPs, the quantum network will serve the request by generating the EPs, while the number of the EPs to generate depends on the specific application.
Thus, a key performance metric of a quantum network is the time required to serve a user request.
We call this metric time-to-serve (TTS), which is the time required to successfully distribute end-to-end EPs at or above the requested fidelity.
%the  for user requests (i.e., the time cost to serve the user request).
% and this metric is determined by the time to generate EPs.
Minimizing TTS is essential for quantum applications due to the limited qubit lifetime.
This will reduce wait time for quantum network users, while improving overall service quality.
%For instance, superconducting qubits have a coherence time in the order of ten milliseconds~\cite{milul-superconducting}, while trapped-ion qubits have a coherence time in the order of seconds~\cite{drmota-trappedion}.

Motivated by the above considerations, the primary goal of this work is to reduce TTS for user requests. 
A common approach to reduce TTS involves optimizing entanglement routing~\cite{caleffi-routing-access17,shi-routing-sigcomm20,ghader-swappingtree-tqe22,abane-routingsurvey}, however, we take a different approach.
% To this end, we combine continuous generation of EPs in the back end, entanglement purification to mitigate quantum memory decoherence, and online adaptivity towards request pattern. For the problem formulation see Section~\ref{sec:problem}. In the following we discuss our main contributions.
% In this paper, we use a Dijkstra algorithm based on the shortest physical path for routing.
% \para{Reduce TTS via entanglement pre-generation.} 
%Instead, we aim to reduce the TTS by 
We focus on \emph{continuously generating EPs between neighbor nodes} in the background, independent of user requests. 
%arriving and using 
These pre-generated EPs~\cite{kolar-ac-infocom22,inesta-continuous-pra23,ghader-predistribution-qce22} are then used to expedite the time required to fulfill the requests. 

To further improve the efficiency, we design an \emph{adaptive scheme} that leverages past information %from previous requests 
to dynamically guide nodes 
in selecting more optimal neighbors for EP generation ~\cite{kolar-ac-infocom22}.
We call this approach Adaptive, Continuous entanglement generation Protocol (\acp).
% With the above continuousness and adaptiveness, we introduce the Adaptive, Continuous entanglement generation Protocol (\acp).
% \para{Improve fidelity via entanglement purification.}
While 
%the usage of pre-generated EPs by the 
\acp\ improves TTS through pre-generated EPs, the fidelity of these 
%pre-generated 
EPs may degrade as they remain idle in quantum memories before being used to serve future requests.
%Therefore, 
To mitigate this, we use entanglement purification~\cite{bbpssw} to improve the fidelity of the pre-generated EPs. %, ultimately leading to fidelity improvement of the EPs for user requests.


% \para{Implementation \& quantum network simulator extension.}
% We implement the \acp\ in a discrete-event quantum network simulator, SeQUeNCe~\cite{sequence}.
We make substantial extensions to the discrete-event quantum network simulator, SeQUeNCe~\cite{sequence}, to support the implementation of the \acp.
The three key extensions are: 1) a Single Heralded Entanglement Generation Protocol that is aware of pre-generated EPs, 2) a Resource Reservation Protocol to make reservations for the \acp, and a Resource Manager that configures the entanglement purification policy.
% an extended Quantum Manager that supports the Bell-diagonal state (BDS)~\cite{zang-dqs}; 
% an extended Quantum Memory Manager that supports memory re-assignment and time-dependent decoherence; 
% a Heralded Entanglement Generation protocol that supports BDS and is aware of pre-generated EPs; 
% an entanglement purification protocol~\cite{bbpssw,dejmps} that supports BDS and purifies the pre-generated EPs; 
% an entanglement swapping protocol that supports BDS.
% The implementation of the \acp\ and the extensions of the quantum network simulator will be opened source upon acceptance.
% \footnote{The URL link is hidden in accordance with anonymous policy.}.
We evaluate \acp\ in SeQUeNCe across various network scales by simulating a multi-user quantum network, with user requests sampled from a traffic matrix.
Ou simulation results show that the \acp\ can indeed reduce the request TTS by \red{$57\%\sim94\%$} while simultaneously improving the fidelity by $0.01\sim0.05$ through entanglement purification.


\para{Contributions and Paper Organization.} We define a quantum network model and formulate the problem of adaptive, continuous entanglement generation in~\S\ref{sec:problem}.
%In this context, we make the following contributions.
Our key contributions are as follows:

\begin{enumerate}
    \item We introduce the Adaptive, Continuous entanglement generation Protocol (\acp) in \S\ref{sec:acp}, which reduces the TTS of user requests. 
    Additionally, entanglement purification (\S\ref{subsec:resource}) enhances the fidelity of EPs.
    % instead of being traded off for TTS, the fidelity is improved.
    \item We extend the SeQUeNCe quantum network simulator (\S\ref{sec:simulator}) with significant enhancements. 
    In particular, a single-heralded entanglement generation protocol is included that accounts for pre-generated EPs. 
    \item We conduct extensive evaluation of \acp\ using SeQUeNCe and demonstrate its effectiveness in reducing TTS and improving the fidelity of distributed EPs in \S\ref{sec:result}.
\end{enumerate}
