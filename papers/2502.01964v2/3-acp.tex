\section{Adaptive, Continuous Entanglement Generation Protocol}
\label{sec:acp}

In this section, we formulate the \acp, which allows a node to continuously generate EPs with their neighbors, with neighbor selection guided by an adaptive control.
The \acp\ runs in the background in complement to entanglement routing protocols for on-demand requests.

\subsection{Continuousness and Finite State Machine}
\label{subsec:continuousness}

We use finite state machines (FSMs) to describe the continuousness of the \acp.
A FSM is defined by \emph{state}, \emph{transition}, \emph{event}, and \emph{action}.
A state describes the system's present situation, including what it is doing or waiting for. 
It is depicted as a circle, and the circle's text describes the behaviors or processes that occur while the system is in the state (see Fig.~\ref{fig:acp-fsm}).
A transition is the change from one state to another and is depicted by an arrow.
An event causing the transition is shown above the horizontal line labeling the transition, and the actions taken when the event occurs are shown below the horizontal line~\cite{topdown-approch}.
We use two FSMs to define the \acp: one for Node-A and one for Node-A's neighbor Node-x.
\red{It resembles the classic ``sender-receiver'' FSMs~\cite{topdown-approch} where Node-A is the sender and Node-x is the receiver.}

\para{States.}
% The two states in Node-A's\CZ{Edge property} FSM are:
\red{The two states of the FSM at Node-A are}:
\begin{itemize}
    \item $S_1$: Sleep for a small random period $\tau$.
    \item $S_2$: Wait for a reply from Node-$x$.
\end{itemize}
% The single state in Node-$x$'s (neighbor of Node-A) FSM is:
\red{The state of FSM at Node-$x$ (neighbor of Node-A) is:}
\begin{itemize}
    \item $S_3$: Wait for a query from Node-A.
\end{itemize}

\para{Transition.} Three factors determine the transition:
\begin{enumerate}
    \item \emph{Counter}: each node has a counter to keep track of the number of quantum memories occupied by the \acp.
    \item \emph{MAX\_MEMORY\_ACP}: determines the maximum number of quantum memories  that the \acp\ is allowed to occupy.
    This number must be smaller than the total number of quantum memories at the node.
    \item Available quantum memory $q_A^i$ ($i$th memory at node-A) not occupied by any protocols, including \acp\ as well as entanglement generation protocol, entanglement swapping protocol, and purification protocol (see Section~\ref{sec:simulator}). 
\end{enumerate}


\softpara{$S_1 \rightarrow S_2$ and $S_1 \rightarrow S_1$}: At Node-A, the FSM starts at state $S_1$, where the FSM sleeps for a small random period $\tau$.
When the period $\tau$ is up, the FSM will take different actions depending on the condition.
If the condition ``\emph{Counter} is less than \emph{MAX\_MEMORY\_ACP} and Node-A has available q-memories'' is satisfied, then the FSM will take several actions and it will transition to state $S_2$, where it waits for a reply from Node-$x$.
The actions include: incrementing \emph{Counter} by one, picking an available q-memory (e.g., $q_A^i$), selecting a neighbor Node-$x$ according to a probability table (see Fig.~\ref{fig:acp-probtable}(a)), and querying the availability of a q-memory to Node-$x$ while notifying it that $q_A^i$ has been picked on Node-A.
If the condition is not satisfied, the FSM will do nothing and transition back to $S_1$.

\softpara{$S_3 \rightarrow S_3$}: At Node-$x$, the FSM starts at state $S_3$, where it waits for a query from Node-A. 
When a query arrives, the FSM will transition back to $S_3$, and it performs different actions depending on the condition. 
If the condition ``\emph{Counter} is smaller than \emph{MAX\_MEMORY\_ACP} and Node-$x$ has available q-memory'' is satisfied, the FSM will increment \emph{Counter} by one, pick an available q-memory (e.g., $q_x^j$), reply to Node-A stating \emph{YES} and that $q_x^j$ is available to pair with $q_A^i$, and create an entanglement generation protocol for $(q_A^i, q_x^j)$.
After the entanglement generation protocol finishes in the future, \emph{Counter} is decremented by one (not shown in the FSM in Fig.~\ref{fig:acp-fsm}).
If the condition is not satisfied, the FSM replies with a message stating \emph{NO}.

\softpara{$S_2 \rightarrow S_1$}: When Node-$x$'s reply arrives at Node-A, the FSM will transition to $S_1$ and perform different actions depending on the reply's message. 
If the reply is \emph{YES} with $q_x^j$ being available to pair with $q_A^i$, \red{Node-A will create a rule that generates entanglement generation protocol for $(q_A^i, q_x^j)$.
After the rule expires, the memory is released and \emph{Counter} is decremented by one.}
If the reply is \emph{NO}, then \emph{Counter} is decremented by one.



\subsection{Adaptive Control}
\label{subsec:adaptive-control}
The adaptive control of \acp\ at each node dynamically updates a local \emph{probability table} in real-time to adjust to request patterns. 
When a node needs to select a neighbor for pre-generating entanglement, it runs a roulette wheel selection algorithm informed by its probability table.
See Fig.~\ref{fig:acp-probtable}(a) for an example of a probability table.
The first column of the probability table is the neighbor, and the second column is the probability of selecting that neighbor.
%Note that there is a non-existent neighbor denoted \emph{None} and it is also associated with a probability.
Our implementation requires a ``phantom'' neighbor denoted \emph{None} that is also associated with a probability.
In an online setting, the probability table can adapt to user request patterns, which are unknown beforehand.

\para{Adaptiveness Mechanism.} We want Node-A to have a higher chance of selecting neighbor Node-$x$ if the link (Node-A, Node-$x$) frequently appears in the entanglement path of the user requests.
The adaptive control will update the probability tables after each request is served.
When Node-$i$ (request initiator) receives a request to generate an EP with Node-$r$ (request responder), the entanglement routing protocol will compute a path and start the on-demand entanglement generation and swapping along the path.
The request is served after the end-to-end EP is distributed.
Once the request is served, Node-$i$ and Node-$r$ will use Algorithm~\ref{algo:adaptive} to update their local probability tables directly;
as end nodes, Node-$i$ and Node-$r$ have the path information $p=[i, \cdots, r]$.
After Node-$i$ updates its probability table, Node-$i$ will send a message to all intermediate nodes containing $p$ (see Fig.~\ref{fig:acp-probtable}(b)).
Upon receiving the message containing $p$, each intermediate node will run Algorithm~\ref{algo:adaptive} to update their probability table.

\para{Algorithm~\ref{algo:adaptive}} is designed to give a reward to $P_x$ of Node-A's probability table by adding a small positive number $\delta$ to $P_x$ if the link (Node-A, Node-$x$) is indeed part of the entanglement path of the request.
Thus, the \acp\ will have a higher chance of generating EP for the link (Node-A, Node-$x$), assuming future requests follow a similar path.
\red{A larger value of $\alpha$ lets the algorithm adapt to the new observations (new EPs) more quickly, while also making the algorithm less stable. We observe that a value around 0.05 is a good balance.}


\begin{figure}[t]
    \centering
    \includegraphics[width=\linewidth]{figures/acp-probtable.png}
    \vspace{-0.2in}
    \caption{(a) Node-$A$'s probability table. (b) Node-$i$ receives a request to generate EPs with Node-$r$. 
    After the request is served, Node-$i$ and Node-$r$ will update their probability tables. 
    Meanwhile, Node-$i$ sends a message containing the path to the intermediate nodes.
    Intermediate nodes update their probability tables after receiving the message. }
    \vspace{-0.1in}
    \label{fig:acp-probtable}
\end{figure}


\begin{algorithm}[t] 
        \KwIn{This node $Node\text{-}A$, entanglement path $path$}
        \KwIn{Probability table $P$, delta $\delta$}
        \KwOut{$P$, updated probability table of this node}
        $neighbors \leftarrow$ All neighbors of $Node\text{-}A$ \\
	\For{$Node$-$x$ in neighbors}{
           \If{Link ($Node\text{-}x$, $Node\text{-}A$) in $path$}{
                $P_x \leftarrow P_x + \delta$ 
           }
        }
        $P \leftarrow$ Normalized $P$ \ \ \ \ \tcp{$\sum P_* = 1$} 
        \Return $P$
	\caption{Update($Node\text{-}A$, $path$, $P$, $\delta$)}
\label{algo:adaptive}
\end{algorithm}



% puesdo code for the adaptive control

% after a request is served



