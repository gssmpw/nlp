\section{Related Work}
The \acp\ is complementary to entanglement routing, which can be performed in an on-demand or continuous manner. 

In on-demand entanglement routing, entanglement generation only starts after a request arrives____. The typical objective is to find an optimal path, together with a swapping policy (i.e., the order of entanglement swapping), to satisfy throughput, latency, and fidelity requirements for a user request____.
Caleffi____ proposed an exponential runtime algorithm that examines all possible paths with a metric assuming a balanced tree swapping policy. Shi and Qian____ created a Dijkstra-like algorithm with a metric considering multiple quantum channels between neighbor nodes and they considered a sequential swapping policy. Ghaderibaneh____ used dynamic programming to find the optimal ``combination of path and swapping policy'', i.e., ``swapping tree''. 

On-demand entanglement routing has also been explored beyond the bipartite regime. In general, the distribution of multipartite entanglement across a network____ starts from the generation of small entangled states shared by few parties (e.g., normal 2-qubit Bell state or 3-qubit GHZ states____). Then multipartite entanglement routing mainly deals with the optimal policy for assembling the small entangled states into large multipartite entangled states.

In continuous entanglement routing, elementary link EPs are generated continuously over all links in the background____. Then, the routing algorithm computes the path on the instant logical topology formed by the created link entanglements____. Kolar et al____ first considered continuous generation of elementary link EPs guided by an adaptive scheme, but the evaluation was ad hoc and quantum memory decoherence was not considered. 

Continuous entanglement generation can also be extended from between neighbor nodes (physical link) to between non-neighbor nodes (virtual link), with the help of entanglement swapping. This allows the network to go beyond the physical connection and create arbitrary virtual topologies, which offers the potential to increase the connectivity of the network, reduce the diameter of the network, and ultimately reduce the latency of end-to-end entanglement generation____. Inesta and Wehner____ defined two metrics to measure the quality of continuous entanglement generation and distribution protocols: (1) virtual neighborhood size and (2) virtual node degree. The higher the values of the two metrics are, the better the continuous protocol is.

\begin{comment}
The \acp\ is complementary to entanglement routing, so this section starts with entanglement routing, including on-demand and continuous entanglement routing.
Then, virtual links and continuous entanglement distribution are discussed.

\subsubsection{On-demand Entanglement Routing}
In on-demand entanglement routing, entanglement generation does not start until a request arrives____. 
The typical objective is to find an optimal path, 
% with allocated capacity for each edge in the path, 
together with a swapping policy (i.e., order of entanglement swapping operation) to satisfy throughput, latency, and fidelity requirements for a user request____.
Caleffi____ uses an exponential runtime algorithm (examine all possible paths) in which the path metric is based on a balanced tree swapping policy, where the order of swapping corresponds to a balanced tree of height $\lceil \log_2 n \rceil$.
Shi and Qian____ use a Dijkstra-like algorithm with a metric considering multiple quantum channels between neighbor nodes and they consider a sequential swapping policy.
Ghaderibaneh et al.____ use dynamic programming to find the optimal ``combination of path and swapping policy'' that they call ``swapping tree''.
Note that for a path with heterogeneous links, the balanced tree swapping policy used in____ is not necessarily optimal____.
\end{comment}
% For the swapping policy,
% The synchronous setting is assumed in____, while asynchronous setting is assumed in____.
% Synchronous setting assumes that all the link EP generation and swapping operations must succeed in a ``small time slot''.
% Asynchronous setting does does not consider time slot and each EP generation and swapping operates independently.
% A qubit of an EP may wait for its counterpart to become available so that a swapping operation can be performed____.



\begin{figure*}[ht]
    \centering
    \includegraphics[width=0.96\linewidth]{figures/acp-fsm.png}
    \vspace{-0.1in}
    \caption{The two FSMs of \acp. (a) depicts the FSM at Node-A, while (b) shows the FSM at Node-A's neighbors.
    Note that \acp\ is a symmetric \red{peer-to-peer} protocol, so the FSM in (a) is also running on Node-A's neighbor, and the FSM in (b) is also running on Node-A.
    % In other words, nodes could assume the role of ``Node-A'' or ``Node-$x$'' depending on the situation.
    \red{This figures depicts ``one connection''. We achieved multiple connections by having multiple $S_2$ states in the Node-A FSM.}
    }
    \vspace{-0.1in}
    \label{fig:acp-fsm}
\end{figure*}

% Multipartite entanglement routing has been explored beyond the aforementioned bipartite entanglement routing work.
% To generate multipartite entanglement across a network____, first we must generate bipartite EPs (the base resource), then we must find an optimal fusion policy via linear programming to fuse smaller graph states into larger graph states.
% In____, 3-qubit GHZ states are used as the base resource to generate graph states.

% Fidelity describes the quality of the entanglement and is an important metric in entanglement routing.
% Since the fidelity of an entanglement drops with each swapping, a minimum fidelity requirement can be converted to a path length constraint____.
% Another source of fidelity drop is the time-dependent decoherence. 
% The decoherence risk can be handled by discarding qubits that aged beyond some cutoff time.
% Quantum network simulators such as____ provide physical models to compute the decoherence process.


\begin{comment}

\subsubsection{Continuous Entanglement Routing}
In continuous entanglement routing, elementary link EPs are generated continuously over all links in the background____.
Then the routing algorithm computes the path on the instant logical topology formed by the created link entanglements____.
Here, the logical topology is dynamic because the generation of link entanglements is probabilistic.
Continuous generation of elementary link EPs is guided by an adaptive scheme in____, followed by a distributed local best-effort routing algorithm in____.


\subsubsection{Virtual Link}
A virtual link is an EP between non-neighbor nodes.
Continuous entanglement generation can be extended from neighbor link entanglement (physical link) to non-neighbor (virtual link) entanglement with the help of entanglement swapping.
This allows the network to go beyond the physical connection and create arbitrary virtual topologies.
Wisely selecting and pre-generating virtual links can increase the connectivity of the network, reduce the diameter of the network, and ultimately reduce the latency of end-to-end entanglement generation____.
% Chakraborty____ considers a hybrid routing in a graph with both physical links and virtual links.
% The routing algorithm prioritizes virtual links and only picks physical link generation on-demand when virtual links are unavailable.

\subsubsection{Continuous Distribution of Entanglement}
The goal of protocols for continuous distribution (CD) of entanglement is to provide a continuous supply of entanglement for nodes to run applications without the need for explicitly demanding entanglement____.
Inesta and Wehner____ define two metrics to measure the quality of CD protocols: (1) virtual neighborhood size and (2) virtual node degree.
The higher the value of the two metrics, the better the CD protocol.
CD and on-demand entanglement generation can co-exist: CD will supply the EPs to applications running at a constant rate, while on-demand support this process during peak demands from sporadic applications.

\end{comment}