\section{Extensions of the SeQUeNCe Simulator}
\label{sec:simulator}

%There exist quite a few quantum network simulators 
Recently, several quantum network simulators have been developed such as SeQUeNCe~\cite{sequence}, QuISP~\cite{quisp}, and NetSquid~\cite{netsquid}.
We chose SeQUeNCe for our simulation because it is open-source, it is easy to understand (Python source code), it has good documentation, and most importantly it is customizable and easy to extend.
In this section, we present the extensions of SeQUeNCe we make to support the implementation of the \acp.
The extensions are open-source at {\url{https://github.com/caitaozhan/adaptive-continuous}}.


\subsection{SeQUeNCe Overview}
SeQUeNCe adopts a modular design~\cite{sequence} that separates functionality into modules to support the development of future quantum network architectures, new hardware~\cite{zang2022simulation,davossa2024simulation}, new protocols, and new applications while allowing parallelization~\cite{wu2024parallel}.
Fig.~\ref{fig:sequence} shows the six main modules of SeQUeNCe: Application, Network Management, Resource Management, Entanglement Management, Hardware, and Simulation Kernel.
Not all components of each module are depicted in Fig.~\ref{fig:sequence}.
%The connection and the dependency of the modules and components are detailed in Section 3\&4 of~\cite{sequence}.
For a detailed description of the connections and dependency of the modules and components, we refer the reader to Sections 3 and 4 of~\cite{sequence}.
We highlight three main aspects of SeQUeNCe modules:
\begin{enumerate}
    \item \textit{Application module:}
    It creates requests and sends them to the Network Management module (see \ding{172} in Fig.~\ref{fig:sequence}).
    
    \item \textit{Network Management module:}
    Upon receiving a request, the Resource Reservation protocol creates rules and installs them in the Rule Manager of the Resource Manager (see \ding{173} Fig.~\ref{fig:sequence}).
    
    \item \textit{Resource Management module:}
    When the condition of a rule is satisfied, the rule will create the entanglement protocols (see \ding{174} in Fig.~\ref{fig:sequence}).

\end{enumerate}


\begin{figure}[b]
    \vspace{-0.2in}
    \centering
    \includegraphics[width=0.75\linewidth]{figures/sequence.png}
    \vspace{-0.1in}
    \caption{SeQUeNCe architecture has six modules.
    Each module has several components (not all components are shown).
    Subsection~\ref{subsec:quantum_manager} to~\ref{subsec:reservation} discuss the extensions in the eight highlighted components in yellow.}
    \vspace{-0.1in}
    \label{fig:sequence}
\end{figure}

\begin{figure*}
    \centering
    \includegraphics[width=0.82\linewidth]{figures/eg.png}
    \vspace{-0.1in}
    \caption{
    The entanglement generation protocol handles two scenarios (a) and (b).
    The yellow-highlighted ``Check EP'' in the figure determines which scenario will happen.
    (a) shows the scenario without pre-generated EPs (i.e., \acp\ has not generated an EP yet). \red{The photonic BSM device indicates a meet-in-the-middle protocol.}
    (b) shows the case of having pre-generated EPs (red and blue) before a request's arrival.
    Note that the request and response messages in green are the coordination messages of the Resource Manager to connect two entanglement generation protocols at two nodes (e.g., EG[0]), not the user request.}
    \vspace{-0.1in}
    \label{fig:eg}
\end{figure*}



The \texttt{Rule} --which is composed of \emph{priority}, \emph{condition}, and \emph{action}-- plays an important role in SeQUeNCe. 
A Rule will take \emph{action} once the \emph{condition} is satisfied.
If multiple rules are satisfied simultaneously, then the \emph{priority} will break the tie.
Similar to the RuleSet in~\cite{matsuo-ruleset-pra19}, when an action is taken, the entanglement protocols will be created, including generation, swapping, and purification.
Each entanglement protocol is associated with a set of quantum memories on a node.




\subsection{Location of the \acp\ in the SeQUeNCe Architecture}
In Fig.~\ref{fig:sequence}, the \acp\  is placed ``between'' the Application module and Network Management module.
Technically, the \acp\ is implemented outside of all modules and is a direct attribute of a quantum node. 
Logically, the \acp\ belongs to the Application module because the \acp\ calls the Resource Reservation Protocol just like any typical application (e.g., quantum teleportation) by making a request to the Network Management module.
% Here, typical applications include the transfer of unknown qubits via teleportation, distributed quantum computing, sensing, etc.
The difference is that typical applications usually make a request on demand due to user behavior, while the \acp\ continuously runs in the background regardless of the user's request.



\subsection{Code Extensions}
This section describes the necessary code extensions for implementing the \acp\ in SeQUeNCe.

\subsubsection{Quantum Manager}
\label{subsec:quantum_manager}
We add a new class \texttt{BellDiagonalState} to the Simulation Kernel to represent a 2-qubit entanglement pair as Bell Diagonal State (BDS).
% This class has four diagonal elements of the density matrix in the Bell basis, and the value of the first element is the fidelity~\cite{zang-dqs} of the state.
We also add a new class \texttt{QuantumManagerBellDiagonal} to the Simulation Kernel to track and manage the quantum states with the BDS formalism.
\red{The assumption of BDS is valid as any 2-qubit state can be transformed into one via Pauli twirling with fidelity unchanged.
The value of the first element of the BDS vector equals the fidelity.}


\subsubsection{Quantum Memory}
\label{subsec:memory}
We extend the \texttt{Memory} class with a new method \texttt{bds\_decohere},
which enables time-dependent decoherence for quantum memories according to single-qubit error pattern, i.e., the probability distribution of $X, Y, Z$ Pauli errors $\{p_{X,Y,Z}\}$.
% When this method is called, it will first fetch the current time $t_c$, and then deduct the decoherence time $\Delta t$ from the time when \texttt{bds\_decohere} was previously called, $t_p$, through $\Delta t=t_c-t_p$.
When this method is called, it will first fetch the current time $t_c$ and the time $t_p$ when \texttt{bds\_decohere} was previously called. 
Then the decoherence time $\Delta t$ is computed as $\Delta t=t_c-t_p$.
Given the previous BDS state, $\Delta t$, $\{p_{X,Y,Z}\}$, and the memory coherence time (See Table~\ref{tab:sim-parameter}), the new BDS state after decoherence is analytically computed (see~\cite{zang-dqs} for details). 
% \red{While the error pattern is tunable, in this work we focus on Werner states under depolarizing channel.}


\subsubsection{Entanglement Generation Protocol}
\label{subsec:generation}


We add a new class \texttt{EntanglementGenerationACP}.
Compared with SeQUeNCe's original class \texttt{EntanglementGeneration}, the new class is single heralded, it uses BDS to represent the quantum state, and most importantly it is aware of existing EPs pre-generated by the \acp. 
Fig.~\ref{fig:eg}(b) shows two pre-generated EPs in the colors red and blue.
The primary node of the pair (the node that has a larger name alphabetically) will be in charge of selecting one EP among potentially multiple EPs between two neighbor nodes.
% Note that in SeQUeNCe, the primary node among two nodes is the node that has a ``larger'' name alphabetically.
Compared with Fig.~\ref{fig:eg}(a) that needs to generate a new EP, Fig.~\ref{fig:eg}(b) shows how reusing existing EPs significantly decreases the latency since link EP generation is probabilistic.
% Note that in Fig.~\ref{fig:eg}(a), an EP is successfully generated in one trial.
Table~\ref{tab:sim-parameter} shows the parameters used during the simulation.


\subsubsection{Entanglement Swapping Protocol}
% A new class named \texttt{EntanglementSwappingBDS} is added.
% Compared with the SeQUeNCe's original class \texttt{EntanglementSwapping}, the new class for entanglement swapping uses BDS~\cite{zang-dqs} to represent the quantum state.
We add a new class \texttt{EntanglementSwappingBDS} that uses BDS~\cite{zang-dqs} to represent the quantum states for entanglement swapping and perform analytical derivation of imperfect swapping results.
During entanglement swapping, we assume the Bell state measurement on memory qubits is always successful but will introduce additional noise in gates and measurements. 
See swapping success rate, gate and measure fidelity in Table~\ref{tab:sim-parameter}.

\subsubsection{Entanglement Purification Protocol}
\label{subsec:purification}
% A new class named \texttt{BBPSSWBDS} is added.
% Compared with the SeQUeNCe's original class \texttt{BBPSSW}, the new class for entanglement purification uses BDS~\cite{zang-dqs} to represent the quantum state.
We add a new class \texttt{BBPSSWBDS} that uses BDS~\cite{zang-dqs} to represent the quantum states for entanglement purification. 
\red{In \texttt{BBPSSWBDS}, the circuit is fundamentally the same as the BBPSSW protocol~\cite{bbpssw}, while only assuming BDS states instead of Werner states.
}
% \red{The naming of the class follows the convention of SeQUeNCe, while in literature it is also known as the DEJMPS protocol~\cite{deutsch1996quantum}. The circuit is fundamentally the same as the BBPSSW protocol~\cite{bbpssw}, while only assuming BDS states instead of Werner states.}
Entanglement purification has a less-than-one success rate and is analytically calculated given two input EPs.

\subsubsection{Resource Manager}
\label{subsec:resource}
The \texttt{ResourceManager} at each node allows \acp\ to keep track of the EPs it generates with neighbor nodes.
We adopt the common as-soon-as-possible strategy for entanglement purification~\cite{ent_dist_epp}.
% due to the non-universality~\cite{zang2024no} of the protocol we use under our assumed error model.
Once a new EP between Node-A and Node-$x$ is generated and there exists an older EP between Node-A and Node-$x$ generated by \acp, the \acp\ of the primary node between Node-A and Node-$x$ will be responsible for selecting one of the older EP for purification.
\red{This process is also known as entanglement pumping~\cite{dur2003entanglement,zang2024no}.}
The newer EP will be the kept EP, and the older EP will be the measurement EP.
Given the two EPs, the \texttt{ResourceManager} at the primary node will initiate the creation of the purification protocol. 
First, it creates an entanglement purification protocol locally (the primary node), and then it sends a message containing the information of the two EPs to the non-primary node, so the counterpart protocol can be created therein.


\subsubsection{Memory Manager}
We add the capability of reallocating quantum memories to the \texttt{MemoryManager}.
When a pre-generated EP is chosen to be reused for entanglement generation, the qubits of the pre-generated EP at two nodes must be reallocated.
As shown in Fig.~\ref{fig:eg}(b), when the red pre-generated EP is chosen, it will be reallocated to the gray memory slot, which is allocated for the on-demand EP generation for Application requests.


\subsubsection{Resource Reservation Protocol}
\label{subsec:reservation}
\texttt{create\_rules\_acp} is added to the \texttt{ResourceReservationProtocol} class to create rules only for entanglement generation protocols in response to the \acp.
The Resource Reservation Protocol originally has a method \texttt{create\_rules} that creates the rules for entanglement generation, swapping, and purification in response to the Application request. 
The \acp, however, does not need swapping because it only creates link entanglement pairs with neighbor nodes.
Entanglement purification protocols are created without the rules (see Section~\ref{subsec:resource}).