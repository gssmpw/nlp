\section{Simulation}
\label{sec:result}
In this section, we present our simulation results,
which show that \acp\ is very effective in reducing the user request TTS.
Moreover, the fidelity of end-to-end EPs is improved with the help of entanglement purification.

\begin{table}[ht]
    \vspace{-0.1in}
    \caption{Simulation parameter}
    \vspace{-0.1in}
    \centering
    \begin{tabular}{|c|c||c|c|}
        \hline
        End node processing delay & 100 \unit{\us} & Forward delay & 20 \unit{\us}  \\
        \hline
         \# of memory per node & 10  & Fiber attenuation & 0.2 dB/km  \\
        \hline
         Max \# of memory for \acp & 5 & Link distance & 10 km \\
        \hline 
        Quantum memory efficiency & 0.6 &  Speed of light & $2\text{e}8$ m/s \\
        \hline
        Photon detector efficiency & 0.95  & Coherence time & 2 s \\
        \hline
        Initial EP fidelity & 0.95 &Pauli errors & [$\frac{1}{3},\frac{1}{3},\frac{1}{3}$] \\
        \hline
        Swapping success rate & 1 & Gate fidelity & 0.99\\
        \hline
        Photonic BSM success rate & 0.5 & Measure fidelity & 0.99 \\
        \hline
        Request arrival rate & 10 Hz &  $\delta$ in Algo.~\ref{algo:adaptive} & 0.05  \\
        \hline
    \end{tabular}
    \vspace{-0.1in}
    \label{tab:sim-parameter}
\end{table}


\subsection{Simulation Setting}

In this subsection, we discuss the network topology, request pattern, classical communication latency, and the entanglement generation strategies under study.
Table~\ref{tab:sim-parameter} summarizes the key parameters for our simulations. 

\begin{wrapfigure}{r}{1.45in}
    \vspace{-0.2in}
    \hspace{-0.2in}
    \centering
    \includegraphics[width=1.05\linewidth]{figures/topology.png}
    \vspace{-0.1in}
    \caption{Network topologies.}
    \vspace{-0.1in}
    \label{fig:topology}
\end{wrapfigure}

\subsubsection{Network Topology} 
We consider three network topologies: a small-scale two-node network, a medium-scale 20-node bottleneck network, and a large-scale 200-node autonomous system (AS) network as shown in Fig.~\ref{fig:topology}(a)$\sim$(c).
We assume that all links have a uniform distance of 10~km, each node has 10 quantum memories, and the \acp\ is allowed to use a maximum of five quantum memories at each node.





% \begin{figure}
%     \centering
%     \includegraphics[width=0.3\linewidth]{figures/topology.png}
%     \caption{Will use the wrapped figure}
%     \label{fig:topology}
% \end{figure}

\subsubsection{Request Pattern}
Requests are sampled from a \emph{traffic matrix} whose elements sum to one.
The element $[i,j]$ represents the probability of generating a request in which the initiator is Node-$i$ and the responder is Node-$j$.
The request arrival rate is 10~Hz throughout all experiments (i.e., a new request arrives every 0.1 s).
We assume each request asks for one end-to-end EP with fidelity $> 0.5$, and the difference between the \texttt{start time} and \texttt{end time} is slightly $< 0.1$ s (i.e., ideally a request is served before a new request arrives).
For the two-node topology in Fig.~\ref{fig:topology}(a), there is only one entanglement path with zero hops.
For the bottleneck topology in Fig.~\ref{fig:topology}(b), all the request initiators are on the left side of the bottleneck link, while the responders are on the right side.
As a result, the entanglement paths of the end-to-end EP have two hops (intermediate nodes).
For the AS topology in Fig.~\ref{fig:topology}(c), we create a traffic matrix such that all the entanglement paths have exactly four hops.



\subsubsection{Classical Communication Latency} 
Classical communication is needed for sending measurement results during entanglement generation, swapping, and purification, as well as for protocol coordination tasks.
% However, classical communication is often naively simplified in the literature, such as assuming all pair of nodes have direct fiber connection.
It is considered a major bottleneck for quantum networks~\cite{rozped-thesis-19}.
In this paper, the classical communication latency between node $u$ and $v$ is
\begin{equation}
    l_{(u,v)} = \frac{d_{(u,v)}}{c} + \mathrm{hop}_{(u,v)} \times D_\mathrm{forward} + D_\mathrm{end~process},
    \label{eqn:latency}
\end{equation}
where $d_{(u,v)}$ is the length of the path from $u$ to $v$,
$c$ is the speed of light in optical fiber,
$\mathrm{hop}_{(u,v)}$ is the number of hops along the path,
$D_\mathrm{forward}$ is the delay for forwarding the packet at intermediate nodes, and
$D_\mathrm{end~process}$ is the delay for processing the packets at the two end nodes.
The values of $D_\mathrm{end~process}$ and $D_\mathrm{forward}$ are in the first row of Table~\ref{tab:sim-parameter}.
We assume the packet size of classical communication in quantum networks is \red{under a hundred bytes, considering the payload being a few bytes to account for a few measurement results}. 
For simplicity, we neglect transmission and queueing delays.

\subsubsection{Methods Compared}
We compare the \acp\ with an ``On Demand Only'' (\odo) strategy, where the quantum network starts generating link EPs only \emph{after} a request arrives
\red{and does not include entanglement purification.}
We also compare the \acp\ with a ``Uniform Continuous Entanglement Generation Protocol'' (\ucp), which has the same continuousness (Sec.~\ref{subsec:continuousness}) of the \acp\ but it lacks the adaptive control (Sec.~\ref{subsec:adaptive-control}), i.e. a node continuously generates EPs with neighbor nodes where the neighbors are selected by a uniform probability distribution.
Moreover, the \acp\ has two variations without entanglement purification, where multiple EPs could have been generated on the same neighbor link.
These two variations apply two different strategies: (1) select the freshest EP and (2) select a random EP.
% \acp\ has purification enabled.
All these strategies will be compared by two metrics: request TTS (see Eqn.~\ref{eqn:tts}) and end-to-end EP fidelity.
% \red{\odo\ does not include purification. Purification could increase the TTS at least two times for \odo, but has minimal impact for \ucp\ and \acp.}

\begin{figure}
    \centering
    \vspace{-0.1in}
    \includegraphics[width=0.96\linewidth]{figures/2node.png}
    \vspace{-0.1in}
    \caption{Simulation results for the two-node topology network.}
    \vspace{-0.1in}
    \label{fig:2node}
\end{figure}


\subsection{Simulation Results}

\subsubsection{Two-Node Network Setting}
The \acp\ reduces the TTS by $94\%$ at most compared to the \odo.
As shown in Fig.~\ref{fig:2node}(a), the average TTS for the \odo\ is 5.95~ms, while the average TTS for the \acp\ with and without entanglement purification is 0.39~ms and 0.3~ms, respectively.
0.3~ms is the delay of a classical communication round trip time between two neighbors, see Fig.~\ref{fig:eg} and Eqn.~\ref{eqn:latency}.
The overhead of 0.09 ms is mainly because of failures in entanglement purification.
Occasionally, EPs generated by the \acp\ do not serve any request, because they are all destroyed during a failed entanglement purification attempt.
Despite a small overhead in TTS, the fidelity in Fig.~\ref{fig:2node}(b) shows an improvement of 0.01$\sim$0.02. The \acp's fidelity is 0.955, while the fidelities of the two variations of the \acp\ without entanglement purification are 0.943 and 0.937.
The \odo\ has a constant fidelity 0.949.
%If compare \acp\ with \odo, we are delighted to observe that 
This simulation demonstrates that the \acp\ can not only drastically reduce the TTS, but also improve the fidelity of end-to-end EPs.

\subsubsection{20-Node Network Setting}
The \acp\ reduces the TTS by $70\%$ compared to the \odo\ and $60\%$ compared to the \ucp.
In Fig.~\ref{fig:20node}(a), the TTS for the \acp\ starts at 9~ms, then it gradually decreases to 4~ms before a change in the traffic matrix occurs. 
After the traffic matrix changes, the TTS suddenly increases to 10~ms --which is worse than the beginning-- then it gradually goes down to 4~ms again.
This trend shows the effectiveness of the adaptive control (i.e., it allows a node to dynamically adjust to changing traffic patterns).
In comparison, the TTS for the \odo\ remains oscillating around 13.3~ms, while for the \ucp\ it fluctuates at 10.3~ms.
For fidelity in Fig.~\ref{fig:20node}(b), the \acp's fidelity starts at 0.84 and it gradually increases before the traffic matrix changes, which leads to a sudden decrease in fidelity.
After this sudden decrease, the fidelity gradually increases again and reaches 0.86, which is nearly 0.02 higher than the \odo\ and the \ucp's fidelity.


\subsubsection{200-Node Network Setting}
The trend depicted in Fig.\ref{fig:200as} is very similar to the one observed in Fig.~\ref{fig:20node}.
\red{The \acp\ (6.8~ms) reduces the TTS to $57\%$ and $50\%$ compared to the \odo\ (15.7~ms) and the \ucp\ (13.6~ms), respectively.}
Furthermore, we observe that the fidelity is increased by nearly 0.05 (see Fig.~\ref{fig:200as}(b)).
This improvement is significantly larger than the two-node and the 20-node scenarios.
This is because the fidelity improvement by the specific entanglement purification we use is the highest when the input fidelity is between 0.7 and 0.8 and in Fig.~\ref{fig:200as}(b), the fidelity of EPs happens to fall in this range.

% Another observation is that although the 200-node scenario has a larger network size and more hops in entanglement paths compared to the 20-node scenario, the TTS is lower. This is because the bottleneck in the 20-node scenario is the critical resource and the amount of memory \acp\ at each node is limited to five.
% This highlights the important role of simulations for capacity planning of future quantum networks.

\begin{figure}
    \centering
    \includegraphics[width=0.96\linewidth]{figures/20bottleneck.png}
    \vspace{-0.1in}
    \caption{Simulation results for the 20-node bottleneck topology network.}
    \vspace{-0.1in}
    \label{fig:20node}
\end{figure}
% bottleneck is causing the delay to be higher compared with the autonomous systems, although the node number is a lot smaller
% also the variation is higher

\begin{figure}
    \centering
    \includegraphics[width=0.96\linewidth]{figures/200as.png}
    \vspace{-0.1in}
    \caption{\red{Simulation results for the 200-Node AS topology network.}}
    \vspace{-0.1in}
    \label{fig:200as}
\end{figure}

