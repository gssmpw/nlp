\section{Extended Results}



\begin{figure}[t]
    \centering
    \includegraphics[width=0.7\linewidth]{figures/round-strategy/plot.pdf}
    \caption{\textbf{Comparing strategies for maintaining context between \system rounds.} The x-axis represents the number of tokens processed by the \textit{remote model}, while the y-axis shows the accuracy achieved.}
    \label{fig:round-strategy}
\end{figure}




\subsection{Model Analysis}\label{app:model_history}
\begin{table*}[]
\centering
\scriptsize

\begin{tabular}{l l l c c c c}
\toprule
\multicolumn{1}{c}{\textbf{Local Model}} & \multicolumn{1}{c}{\textbf{Remote Model}} & \multicolumn{1}{c}{\textbf{Release Date}} & \multicolumn{1}{c}{\textbf{Accuracy (Longhealth)}} & \multicolumn{1}{c}{\textbf{Accuracy (QASPER)}} & \multicolumn{1}{c}{\textbf{Accuracy (Finance)}} \\ 
\midrule
llama-3B & gpt-4o & May 2024 & 0.7025 & 0.598 & 0.7826 \\ 
llama-3B & gpt-4-turbo & April 2024 & 0.6247 & 0.614 & 0.6304 \\ 
llama-3B & gpt-3.5-turbo-0125 & Jan  2024 & 0.2157 & 0.4314 & 0.1707 \\ 
llama-3B & gpt4o-mini & July 2024 & 0.6275 & 0.568 & 0.6522 \\ 
llama-3B & llama3-70B-Instruct-Turbo & April 2024 & 0.3525 & 0.144 & 0.1818 \\ 
llama-3B & llama3.1-70B-Instruct-Turbo & July 2024 & 0.6193 & 0.514 & 0.4348 \\ 
llama-3B & llama3.3-70B-Instruct-Turbo & December 2024 & 0.6658 & 0.534 & 0.6739 \\ 
\bottomrule
\end{tabular}
\caption{Accuracy Results for Longhealth, QASPER, and Finance across Various Models}
\label{tab:remote-model-variations}
\end{table*}
\begin{table*}[]
\centering
\scriptsize

\begin{tabular}{l l c c l}
\toprule
\multicolumn{1}{c}{\textbf{Local Model}} & \multicolumn{1}{c}{\textbf{Remote Model}} & \textbf{Accuracy (Longhealth)} & \textbf{Accuracy (QASPER)} & \multicolumn{1}{c}{\textbf{System Date}} \\ 
\midrule
Llama-2-7b-chat-hf & gpt-4-1106-preview & 0.340 & 0.178 & November 2023 \\ 
Llama-3.1-8B-Instruct & gpt-4-turbo & 0.645 & 0.528 & April 2024 \\ 
Llama-3.1-8B-Instruct & gpt-4o & 0.740 & 0.582 & July 2024 \\ 
--- & gpt-4-turbo & 0.768 & 0.391 & April 2024 \\ 
\bottomrule
\end{tabular}
\caption{Point in time results for \system configurations with best-in-class $\locallm$ and $\remotelm$}
\label{tab:system-snapshot}
\end{table*}
We include additional experiment results from Section~\ref{subsec:results-model}. In Table~\ref{tab:remote-model-variations} we show the effects of varying $\remotelm$ on \system.  In Table~\ref{tab:system-snapshot}, we show the performance of \system using the best in-class models at the time (from late 2023 to late 2024).


\subsection{\naive $\locallm$ Analysis}
% Accuracy vs Number of Chunks in Context
\begin{table}[h]
    \centering
    \begin{tabular}{cc}
        \toprule
        \textbf{Total Chunks In-Context} & \textbf{Accuracy} \\
        \midrule
        1   & 0.59375 \\
        16  & 0.53906 \\
        32  & 0.50000 \\
        64  & 0.48438 \\
        128 & 0.46094 \\
        \bottomrule
    \end{tabular}
    \caption{Accuracy vs. Number of Chunks in Context} Each chunk has 512 tokens.
    \label{tab:chunks_vs_accuracy}
\end{table}

% Accuracy vs Number of Sub Tasks
\begin{table}[h]
    \centering
    \begin{tabular}{cc}
        \toprule
        \textbf{Number of Sub Tasks} & \textbf{Accuracy} \\
        \midrule
        1 & 0.70313 \\
        2 & 0.39844 \\
        3 & 0.19531 \\
        4 & 0.14844 \\
        \bottomrule
    \end{tabular}
    \caption{Accuracy vs. Number of Sub Tasks}
    \label{tab:subtasks_vs_accuracy}
\end{table}


\label{app:naive-analysis}

We perform an empirical analysis evaluating the robustness of $\locallm$. We perform experiments to evaluate two axes of model capabilities: (1) ability to reason over long contexts and (2) ability to solve multi-part queries. To test (1) and (2) we curate a synthetic dataset built over the \finance dataset wherein we use \gpt to construct an extraction based question-answering dataset over chunks (length 512 tokens) of documents in the \finance dataset. We then construct two settings evaluating over \llamathreetwo-3B-Instruct.
\\

\textbf{Long Context Reasoning}: To evaluate long-context reasoning, we concatenate between \{1,16,32,64,128\} chunks to construct the context. At least one chunk in the concatenated context contains the ground truth result. As seen in Table~\ref{tab:chunks_vs_accuracy}, increasing the context length from 512 to ~65.5K tokens leads to a 13 point drop in accuracy.
\\

\textbf{Multi-step Queries} To evaluate the ability of $\locallm$ to fulfill multi-step queries, we construct queries that have between \{1,2,3,4\} sub-tasks. Our results indicate increasing from 1 to sub-tasks leads to a 56.3 point drop in accuracy (see Table~\ref{tab:subtasks_vs_accuracy}).

\begin{table*}[]
\centering
\tiny

\begin{tabular}{p{0.7cm} p{0.7cm} p{0.7cm} p{0.4cm} p{0.4cm} p{0.4cm} p{0.4cm} p{0.4cm} p{0.4cm} p{0.4cm} p{0.4cm} p{0.4cm} p{0.4cm} p{0.4cm} p{0.4cm}}
\toprule
\multicolumn{1}{c}{\tiny \textbf{Protocol}} & \multicolumn{1}{c}{\tiny  \textbf{Local Model}} & \multicolumn{1}{c}{\tiny  \textbf{Remote Model}} & \multicolumn{4}{c}{\finance} & \multicolumn{4}{c}{\longhealth} & \multicolumn{4}{c}{\qasper} \\
 &  &  & Acc. & Cost & In Tok. (1k) & Out Tok. (1k) & Acc. & Cost & In Tok. (1k) & Out Tok. (1k) & Acc. & Cost & In Tok. (1k) & Out Tok. (1k) \\
\hline
Remote Only & --- & \gpt & 0.826 & \$0.261 & 103.04 & 0.32 & 0.748 & \$0.301 & 120.10 & 0.07 & 0.598 & \$0.137 & 54.40 & 0.09\\
\midrule
Local Only & \llamaeight & --- & 0.326 & \$0.000 & 0.00 & 0.00 & 0.468 & \$0.000 & 122.58 & 0.07 & 0.538 & \$0.000 & 54.41 & 0.06\\
Local Only & \llamaone & --- & 0.000 & \$0.000 & 0.00 & 0.00 & 0.115 & \$0.000 & 122.58 & 0.07 & 0.000 & \$0.000 & 54.41 & 0.10\\
Local Only & \llamathree & --- & 0.130 & \$0.000 & 0.00 & 0.00 & 0.345 & \$0.000 & 122.58 & 0.08 & 0.164 & \$0.000 & 54.41 & 0.08\\
Local Only & \qwenthree & --- & 0.087 & \$0.000 & 0.00 & 0.00 & 0.177 & \$0.000 & 31.24 & 0.08 & 0.156 & \$0.000 & 32.58 & 0.08\\
\midrule

\naive & \llamaeight & \gpt & 0.804 & \$0.007 & 0.88 & 0.46 & 0.635 & \$0.010 & 1.85 & 0.50 & 0.450 & \$0.007 & 0.92 & 0.42\\

\naive & \llamathree & \gpt & 0.698 & \$0.010 & 1.74 & 0.52 & 0.482 & \$0.009 & 1.56 & 0.47 & 0.372 & \$0.011 & 2.26 & 0.53\\

\naive & \qwenthree & \gpt & 0.217 & \$0.029 & 8.28 & 0.82 & 0.281 & \$0.021 & 5.70 & 0.68 & 0.210 & \$0.035 & 10.51 & 0.87\\
\midrule
\system & \llamaeight & \gpt & 0.804 & \$0.053 & 15.99 & 1.29 & 0.740 & \$0.054 & 18.96 & 0.65 & 0.582 & \$0.019 & 5.10 & 0.61\\
\system & \llamathree & \gpt & 0.726 & \$0.079 & 24.67 & 1.77 & 0.703 & \$0.057 & 20.11 & 0.66 & 0.558 & \$0.020 & 5.62 & 0.60\\
\system & \qwenthree & \gpt & 0.783 & \$0.059 & 17.20 & 1.56 & 0.645 & \$0.043 & 14.43 & 0.65 & -- & -- & -- & --\\
\system & \qwenseven & \gpt & -- & -- & -- & -- & -- & -- & -- & -- & 0.600 & \$0.015 & 3.44 & 0.61\\
\hline
\end{tabular}


\caption{\textbf{Accuracy and cost of local-remote systems.} Evaluation of cost and accuracy on the \finance \cite{islam2023financebench}, \longhealth\cite{adams2024longhealth}, and \qasper ~\cite{dasigi2021dataset}. The table compares two edge-remote communication protocols --- Naïve (\cref{sec:naive}) and \system (\cref{sec:methods}) --- alongside edge-only and remote-only baselines. 
Three different edge models are considered (\llamaeight, \llamathree, \qwenthree, and \llamaone) and a remote model (\gpt). 
Accuracy (Acc.) is the fraction of correct predictions across the dataset. 
Cost (USD) is the average cost in USD per query in the dataset computed. Costs are incurred for any calls to the remote model at \gpt rates (January 2025: \$2.50 per million input tokens and \$10.00 per million output tokens). We assume that running the edge model is free; see \cref{sec:prelim-setup} for details on the cost model.
In tokens is the number of input (\textit{i.e.} prefill) tokens sent to the remote model. 
Out tokens is the number of output (\textit{i.e.} decode) tokens generated from the remote model. Both values are shown in thousands. 
}
\label{table:app-tradeoff}
\end{table*}


\subsection{Retrieval-Augmented Generation}
In this section, we describe a process of developing reliable RAG models with enhanced fluency, coherence, and factual accuracy from Kanana LLMs and introduce internal Korean RAG benchmark developed to more accurately measure the factual consistency.

% The evaluation of diverse models of korean RAG scenarios is presented in \autoref{fig:rag-main-performance}.
% After our training process, Kanana Essence 9.8B RAG achieved helpfulness while maintaining 91.4\% of GPT-4o's grounding performance. So we comfirmed that Kanana model can sufficiently achieve grounding performance against oracle model.

\begin{figure}[h]
    \small
    \centering
    \includegraphics[width=0.6\textwidth]{figures/rag/rag-main-performance-iter4.pdf}
    \caption{Performance Comparison of Various Models Based on averaged helpfulness and grounding in RAG-General-Bench.}
    % \caption{Performance Overview}
    \label{fig:rag-main-performance}
\end{figure}
Retrieval-Augmented Generation (RAG) methods \citep{lewis2021retrievalaugmentedgenerationknowledgeintensivenlp} enable large language models to access the latest external or proprietary information without altering model parameters \citep{liu2024chatqasurpassinggpt4conversational}.
% Retrieval-Augmented Generation (RAG) is a hybrid generation model that combines Large Language Models (LLMs) with access to non-parametric memory \citep{lewis2021retrievalaugmentedgenerationknowledgeintensivenlp}.
With access to reliable data sources, such as search results, data bases, and tool call, RAG greatly reduces the hallucination of LLMs \citep{Shuster2021RetrievalAR}.
% In other words, it enhances reliability and contextual accuracy by ensuring that generated responses are conditioned on the given information from credible sources, such as search, database, and tool call.
% To ensure advantages of RAG, additional training process is required to enhance factuality \citep{lin2024flamefactualityawarealignmentlarge}.
However, in order to endow RAG ability, models require further training using appropriate RAG data \citep{lin2024flamefactualityawarealignmentlarge}.
% Additionally, robust evaluation methods are crucial to accurately measure factual consistency for model ablation.

\begin{figure}[h]
\centering
\resizebox{0.9\textwidth}{!}{%
    \includegraphics[width=\textwidth]{figures/rag/data-generation-pipeline.pdf}
}
\caption{QA Generation Pipelines}
\label{fig:rag-data-generation-pipeline}
\end{figure}

% To encourage responses that leverage contextual information, 
To construct RAG dataset, we synthetically generate question-answer pairs  using high-quality bilingual documents as seed documents, following the pipeline in \autoref{fig:rag-data-generation-pipeline}.
% We build dataset around specialized knowledge to ensure that model relies on contextual information rather than its intrinsic knowledge.
% We construct a dataset that requires specialized knowledge to respond, to 
% After generation of large amount of QA pair, we filter low-grounded responses with LLM-Judge and then execute response refinement with reflection, ensuring that only high-grounded responses remained.
From the synthetically generated pairs, we filter out instances with low grounding scores and use LLM-judge to reflect and refine the low grounding instances.
At this point, we have obtained dataset for SFT, which we call SynQA-SFT. 
Along with SynQA-SFT, we augment StructLM \citep{zhuang2024structlmbuildinggeneralistmodels}, FollowRAG \citep{dong2024generalinstructionfollowingalignmentretrievalaugmented}, and SFT dataset from Section \ref{subsec:post-training-data} to train the instruction model to adapt to the RAG scenarios.
% while maintaining its instruction following capabilities.
Then, we augment low grounding responses to high grounding instances in SynQA-SFT to construct preference dataset for DPO.
Because the SFT and DPO process decays instruction following capability, we merge the DPO model with the instruction model to preserve both abilities.
% To execute the pipline cost-effectively, we utilized strong model for generation and LLM-Judge locally served.
Our comprehensive training process demonstrates improvements in grounding performance while maintaining the instruction following scores, as shown in \autoref{tab:rag-performance-comparison}.

% During training phase, we focus on improving grounding capability.
% In SFT phase, we utilize a mixture of StructLM \citep{zhuang2024structlmbuildinggeneralistmodels}, FollowRAG \citep{dong2024generalinstructionfollowingalignmentretrievalaugmented} and synthetic QA dataset for training to adapting diverse context format and instructions in RAG scenarios. Additionally, we replay SFT dataset from Section \ref{subsec:post-training-data} to maintain helpfulness and instruction-following capabilities.
% During DPO phase, we inject preference for better grounding responses by augmenting rejected responses.
% Our comprehensive training process demonstrates improvements in grounding performance as shown in \autoref{tab:rag-performance-comparison}.
    
% However, it also declines helpfulenss compared to base model.  
% To address this issue, we apply weight averaging at the end of process. To enhance overall performance, grounding-focused model was merged with helpfulness-focused model.
% When merging models with different concepts, task transfer could be possible in addition to the ensemble effect \citep{kim2024prometheus2opensource}.
% Through this approach, we can get also higher score of helpfulness compared to base model.

\begin{table}[h]
\centering
\resizebox{\columnwidth}{!}{
    \begin{tabular}{l|c|cc|c|c}
    \toprule
    \textbf{Models} & \multicolumn{1}{c|}{\textbf{FACTs}} & \multicolumn{2}{c|}{\textbf{RAG-General-Bench}} & \multicolumn{1}{c|}{\textbf{ContextualBench}} & \multicolumn{1}{c}{\textbf{IFEval}} \\
    & \small{grounding} & \small{grounding} & \small{helpfulness} & \small{em} &  \\
    \midrule
    Kanana Essence 9.8B & 40.66 & 32.63 & 55.86 & 20.22 & \textbf{79.93}\\
    \midrule
    % Phase1 (SFT)& 62.40& 59.29& 51.60& 48.08& 72.99\\
    % Phase2 (DPO)& \textbf{63.09}& \textbf{65.33}& 52.67& \textbf{48.76}& 75.00\\
    % Phase3 (Merge)& 53.09 & 57.38& \textbf{57.32}& 48.31& 78.44\\
    + SFT & 62.40& 59.29& 51.60& 48.08& 72.99\\
    + DPO & \textbf{63.09}& \textbf{65.33}& 52.67& \textbf{48.76}& 75.00\\
    + Merge (Kanana Essence 9.8B RAG) & 53.09 & 57.38& \textbf{57.32}& 48.31& 78.44\\
    \bottomrule
    \end{tabular}
}
    \caption{Performance change of each phase of recipe. Grounding score is average of RAGAS \citep{es2023ragasautomatedevaluationretrieval} Faithfulness and rubric based LLM-judge. Helpfulness score is average of RAGAS Answer Relevancy and rubric based LLM-judge.}
    \label{tab:rag-performance-comparison}
\end{table}
    

% \begin{wrapfigure}[14]{r}{0.45\textwidth}
%     \centering
%     \resizebox{0.44\textwidth}{!}{%
%         \includegraphics{figures/rag/eval-ko-data-portion.pdf}
%     }
%     \caption{Task distribution of RAG-General-Bench, which was developed by Kanana team.}
%     \label{rag-ko-eval-data}
% \end{wrapfigure}

For completeness of the evaluation, we collect and assess our model on diverse RAG scenario benchmarks of ContextualBench \citep{nguyen2024sfrragcontextuallyfaithfulllms}, FACTs \citep{jacovi2025factsgroundingleaderboardbenchmarking}, and IFEval \citep{zhou2023instructionfollowingevaluationlargelanguage}.
% ContextualBench \citep{nguyen2024sfrragcontextuallyfaithfulllms} is set of multi-hop QA testset that golden answers are highly concise rather than chat-style. 
% we utilized it for measure conciseness of response and intuitive evaluation method based on exact match.
% FACTs \citep{jacovi2025factsgroundingleaderboardbenchmarking} is consist of diverse task given context like Reasoning, QA, Summarization, Rewriting and Extraction. 
% We filtered within char-level length 20k since our base model was trained with token length limit of 8k. This dataset is not labeled golden answer, so we only measure grounding score with it.
% IFEval \citep{zhou2023instructionfollowingevaluationlargelanguage} measures the maintenance of instruction following performance. 
% While utilizing english public benchmarks, we developed internal RAG benchmark named RAG-General-Bench to address insufficient of korean evaluation set.
However, these benchmarks are all English-based, making them insufficient to judge the RAG abilities in Korean.
To this end, we have developed an internal FACTs-like Korean benchmark for evaluating RAG abilities, called RAG-General-Bench.
In RAG-General-Bench, human annotators manually constructed the dataset with context, instruction, and reference answer, allowing the evaluation of helpfulness as well. 
% Since reference answer exists, helpfulness can also be evaluated.
% It focued on RAG-specific tasks similar with FACTs.
% In \autoref{rag-ko-eval-data}, benchmark consists of a total of 115 samples, categorized into 27 subcategories, providing a diverse set of scenarios for evaluation. There are 4 samples of each category in \autoref{appendix:rag-bench-example}.
The benchmark consists of a total of 115 samples, categorized into 27 subcategories, providing a diverse set of scenarios for evaluation. There are 4 samples of each category in \autoref{appendix:rag-bench-example}.

% The evaluation of diverse models of korean RAG scenarios is presented in \autoref{fig:rag-main-performance}.
% After our training process, Kanana Essence 9.8B RAG achieved helpfulness while maintaining 91.4\% of GPT-4o's grounding performance. So we comfirmed that Kanana model can sufficiently achieve grounding performance against oracle model.

% The evaluation of diverse models of korean RAG scenarios is presented in \autoref{fig:rag-main-performance}.
% After our training process, Phase 2 model outperformed GPT-4o in grounding performance but its helpfulness fell short of both the base model and other open models.
% Meanwhile, Phase 3 improved helpfulness while maintaining 91.4\% of GPT-4o's grounding performance. So we comfirmed that Kanana model can sufficiently achieve grounding performance against oracle model.

    % We train Kanana Essence model following training process in \autoref{fig:model-recipe}.
    % We conduct ablations while fixing SFT-DPO flow and model merge is utilized after DPO to maintain helpfulness and instruction-following performance.

    % In phase1, we use 2.1M unique samples for training, including SynQA-SFT Collection.
    % To maintain the general capabilities of the base model, around 40\% of the total samples were replayed using Kanana-SFT dataset.
    % Sequence packing was employed to ensure contamination-free training, allowing the attention mechanism to function solely within individual samples.
    % We set maximum learning rate to $5\times10^{-6}$, the warmup steps to 200, minimum learning rate to 10\% of peak lr and trained with 3 epochs.
    % In phase2, We further train supervised fine-tuned model using 250K unique samples from SynQA-DPO Collection.
    % We set the maximum learning rate to $1\times10^{-7}$, the warmup steps to 200, and the minimum learning rate to 0.1 and trained with 3 epochs. In phase3, we merged Kanana-Inst and DPO model in a 1:1 ratio.
    % It enables an ensemble effect without increasing inference time \citep{wortsman2022modelsoupsaveragingweights}. 
    
  
    
% \begin{figure}[h]
%     \centering
%     \resizebox{0.8\textwidth}{!}{%
%         \includegraphics[width=\textwidth]{figures/rag/model-recipe.pdf}
%     }
%     \caption{Training Process}
%     \label{fig:model-recipe}
% \end{figure}