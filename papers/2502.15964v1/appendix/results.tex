\section{Extended Results}



\begin{figure}[t]
    \centering
    \includegraphics[width=0.7\linewidth]{figures/round-strategy/plot.pdf}
    \caption{\textbf{Comparing strategies for maintaining context between \system rounds.} The x-axis represents the number of tokens processed by the \textit{remote model}, while the y-axis shows the accuracy achieved.}
    \label{fig:round-strategy}
\end{figure}




\subsection{Model Analysis}\label{app:model_history}
\begin{table*}[]
\centering
\scriptsize

\begin{tabular}{l l l c c c c}
\toprule
\multicolumn{1}{c}{\textbf{Local Model}} & \multicolumn{1}{c}{\textbf{Remote Model}} & \multicolumn{1}{c}{\textbf{Release Date}} & \multicolumn{1}{c}{\textbf{Accuracy (Longhealth)}} & \multicolumn{1}{c}{\textbf{Accuracy (QASPER)}} & \multicolumn{1}{c}{\textbf{Accuracy (Finance)}} \\ 
\midrule
llama-3B & gpt-4o & May 2024 & 0.7025 & 0.598 & 0.7826 \\ 
llama-3B & gpt-4-turbo & April 2024 & 0.6247 & 0.614 & 0.6304 \\ 
llama-3B & gpt-3.5-turbo-0125 & Jan  2024 & 0.2157 & 0.4314 & 0.1707 \\ 
llama-3B & gpt4o-mini & July 2024 & 0.6275 & 0.568 & 0.6522 \\ 
llama-3B & llama3-70B-Instruct-Turbo & April 2024 & 0.3525 & 0.144 & 0.1818 \\ 
llama-3B & llama3.1-70B-Instruct-Turbo & July 2024 & 0.6193 & 0.514 & 0.4348 \\ 
llama-3B & llama3.3-70B-Instruct-Turbo & December 2024 & 0.6658 & 0.534 & 0.6739 \\ 
\bottomrule
\end{tabular}
\caption{Accuracy Results for Longhealth, QASPER, and Finance across Various Models}
\label{tab:remote-model-variations}
\end{table*}
\begin{table*}[]
\centering
\scriptsize

\begin{tabular}{l l c c l}
\toprule
\multicolumn{1}{c}{\textbf{Local Model}} & \multicolumn{1}{c}{\textbf{Remote Model}} & \textbf{Accuracy (Longhealth)} & \textbf{Accuracy (QASPER)} & \multicolumn{1}{c}{\textbf{System Date}} \\ 
\midrule
Llama-2-7b-chat-hf & gpt-4-1106-preview & 0.340 & 0.178 & November 2023 \\ 
Llama-3.1-8B-Instruct & gpt-4-turbo & 0.645 & 0.528 & April 2024 \\ 
Llama-3.1-8B-Instruct & gpt-4o & 0.740 & 0.582 & July 2024 \\ 
--- & gpt-4-turbo & 0.768 & 0.391 & April 2024 \\ 
\bottomrule
\end{tabular}
\caption{Point in time results for \system configurations with best-in-class $\locallm$ and $\remotelm$}
\label{tab:system-snapshot}
\end{table*}
We include additional experiment results from Section~\ref{subsec:results-model}. In Table~\ref{tab:remote-model-variations} we show the effects of varying $\remotelm$ on \system.  In Table~\ref{tab:system-snapshot}, we show the performance of \system using the best in-class models at the time (from late 2023 to late 2024).


\subsection{\naive $\locallm$ Analysis}
% Accuracy vs Number of Chunks in Context
\begin{table}[h]
    \centering
    \begin{tabular}{cc}
        \toprule
        \textbf{Total Chunks In-Context} & \textbf{Accuracy} \\
        \midrule
        1   & 0.59375 \\
        16  & 0.53906 \\
        32  & 0.50000 \\
        64  & 0.48438 \\
        128 & 0.46094 \\
        \bottomrule
    \end{tabular}
    \caption{Accuracy vs. Number of Chunks in Context} Each chunk has 512 tokens.
    \label{tab:chunks_vs_accuracy}
\end{table}

% Accuracy vs Number of Sub Tasks
\begin{table}[h]
    \centering
    \begin{tabular}{cc}
        \toprule
        \textbf{Number of Sub Tasks} & \textbf{Accuracy} \\
        \midrule
        1 & 0.70313 \\
        2 & 0.39844 \\
        3 & 0.19531 \\
        4 & 0.14844 \\
        \bottomrule
    \end{tabular}
    \caption{Accuracy vs. Number of Sub Tasks}
    \label{tab:subtasks_vs_accuracy}
\end{table}


\label{app:naive-analysis}

We perform an empirical analysis evaluating the robustness of $\locallm$. We perform experiments to evaluate two axes of model capabilities: (1) ability to reason over long contexts and (2) ability to solve multi-part queries. To test (1) and (2) we curate a synthetic dataset built over the \finance dataset wherein we use \gpt to construct an extraction based question-answering dataset over chunks (length 512 tokens) of documents in the \finance dataset. We then construct two settings evaluating over \llamathreetwo-3B-Instruct.
\\

\textbf{Long Context Reasoning}: To evaluate long-context reasoning, we concatenate between \{1,16,32,64,128\} chunks to construct the context. At least one chunk in the concatenated context contains the ground truth result. As seen in Table~\ref{tab:chunks_vs_accuracy}, increasing the context length from 512 to ~65.5K tokens leads to a 13 point drop in accuracy.
\\

\textbf{Multi-step Queries} To evaluate the ability of $\locallm$ to fulfill multi-step queries, we construct queries that have between \{1,2,3,4\} sub-tasks. Our results indicate increasing from 1 to sub-tasks leads to a 56.3 point drop in accuracy (see Table~\ref{tab:subtasks_vs_accuracy}).

\begin{table*}[]
\centering
\tiny

\begin{tabular}{p{0.7cm} p{0.7cm} p{0.7cm} p{0.4cm} p{0.4cm} p{0.4cm} p{0.4cm} p{0.4cm} p{0.4cm} p{0.4cm} p{0.4cm} p{0.4cm} p{0.4cm} p{0.4cm} p{0.4cm}}
\toprule
\multicolumn{1}{c}{\tiny \textbf{Protocol}} & \multicolumn{1}{c}{\tiny  \textbf{Local Model}} & \multicolumn{1}{c}{\tiny  \textbf{Remote Model}} & \multicolumn{4}{c}{\finance} & \multicolumn{4}{c}{\longhealth} & \multicolumn{4}{c}{\qasper} \\
 &  &  & Acc. & Cost & In Tok. (1k) & Out Tok. (1k) & Acc. & Cost & In Tok. (1k) & Out Tok. (1k) & Acc. & Cost & In Tok. (1k) & Out Tok. (1k) \\
\hline
Remote Only & --- & \gpt & 0.826 & \$0.261 & 103.04 & 0.32 & 0.748 & \$0.301 & 120.10 & 0.07 & 0.598 & \$0.137 & 54.40 & 0.09\\
\midrule
Local Only & \llamaeight & --- & 0.326 & \$0.000 & 0.00 & 0.00 & 0.468 & \$0.000 & 122.58 & 0.07 & 0.538 & \$0.000 & 54.41 & 0.06\\
Local Only & \llamaone & --- & 0.000 & \$0.000 & 0.00 & 0.00 & 0.115 & \$0.000 & 122.58 & 0.07 & 0.000 & \$0.000 & 54.41 & 0.10\\
Local Only & \llamathree & --- & 0.130 & \$0.000 & 0.00 & 0.00 & 0.345 & \$0.000 & 122.58 & 0.08 & 0.164 & \$0.000 & 54.41 & 0.08\\
Local Only & \qwenthree & --- & 0.087 & \$0.000 & 0.00 & 0.00 & 0.177 & \$0.000 & 31.24 & 0.08 & 0.156 & \$0.000 & 32.58 & 0.08\\
\midrule

\naive & \llamaeight & \gpt & 0.804 & \$0.007 & 0.88 & 0.46 & 0.635 & \$0.010 & 1.85 & 0.50 & 0.450 & \$0.007 & 0.92 & 0.42\\

\naive & \llamathree & \gpt & 0.698 & \$0.010 & 1.74 & 0.52 & 0.482 & \$0.009 & 1.56 & 0.47 & 0.372 & \$0.011 & 2.26 & 0.53\\

\naive & \qwenthree & \gpt & 0.217 & \$0.029 & 8.28 & 0.82 & 0.281 & \$0.021 & 5.70 & 0.68 & 0.210 & \$0.035 & 10.51 & 0.87\\
\midrule
\system & \llamaeight & \gpt & 0.804 & \$0.053 & 15.99 & 1.29 & 0.740 & \$0.054 & 18.96 & 0.65 & 0.582 & \$0.019 & 5.10 & 0.61\\
\system & \llamathree & \gpt & 0.726 & \$0.079 & 24.67 & 1.77 & 0.703 & \$0.057 & 20.11 & 0.66 & 0.558 & \$0.020 & 5.62 & 0.60\\
\system & \qwenthree & \gpt & 0.783 & \$0.059 & 17.20 & 1.56 & 0.645 & \$0.043 & 14.43 & 0.65 & -- & -- & -- & --\\
\system & \qwenseven & \gpt & -- & -- & -- & -- & -- & -- & -- & -- & 0.600 & \$0.015 & 3.44 & 0.61\\
\hline
\end{tabular}


\caption{\textbf{Accuracy and cost of local-remote systems.} Evaluation of cost and accuracy on the \finance \cite{islam2023financebench}, \longhealth\cite{adams2024longhealth}, and \qasper ~\cite{dasigi2021dataset}. The table compares two edge-remote communication protocols --- Naïve (\cref{sec:naive}) and \system (\cref{sec:methods}) --- alongside edge-only and remote-only baselines. 
Three different edge models are considered (\llamaeight, \llamathree, \qwenthree, and \llamaone) and a remote model (\gpt). 
Accuracy (Acc.) is the fraction of correct predictions across the dataset. 
Cost (USD) is the average cost in USD per query in the dataset computed. Costs are incurred for any calls to the remote model at \gpt rates (January 2025: \$2.50 per million input tokens and \$10.00 per million output tokens). We assume that running the edge model is free; see \cref{sec:prelim-setup} for details on the cost model.
In tokens is the number of input (\textit{i.e.} prefill) tokens sent to the remote model. 
Out tokens is the number of output (\textit{i.e.} decode) tokens generated from the remote model. Both values are shown in thousands. 
}
\label{table:app-tradeoff}
\end{table*}


\section{RAG-General-Bench Examples}\label{appendix:rag-bench-example}

\begin{CJK}{UTF8}{mj}
    % \begin{figure}[h]
    % \centering
    % \resizebox{0.7\textwidth}{!}{%
    %     \includegraphics[width=\textwidth]{figures/function_calling/fc-appendix-single.pdf}
    % }
    % \caption{FunctionChat-bench Example : Single-call(1\_exact)}
    % \label{fig:fc-appendix-single}
    % \end{figure}

\begin{figure}[h]
    \centering
    \subfloat[][Sample 1]{%
        \fbox{\includegraphics[width=0.45\textwidth]{figures/rag/rag-appendix-examples-1.pdf}}%
        \label{fig:rag_pre_stage1_data}
    }
    \hfill
    \subfloat[][Sample 2]{%
        \fbox{\includegraphics[width=0.45\textwidth]{figures/rag/rag-appendix-examples-2.pdf}}%
        \label{fig:rag_pre_stage2_data}
    }
    \caption{RAG-General-Bench example: QA}
    \label{fig:rag_sample1}
\end{figure}

% \begin{longtable}{lp{0.8\textwidth}}
% \toprule
% Context & \{doc1\} \newline

% \{doc2\} \newline
% \\\\

% Instruction & A씨는 최근에 요양병원에 계신 어머님을 두 번 면회하고 왔습니다. A씨는 두 번 다 마스크를 착용하지 않았지만 별다른 제지는 없었습니다. A씨와 요양병원 원장 B씨가 각각 내야 할 과태료는 얼마인가요?\\\\
% \midrule
% \endfirsthead

% \multicolumn{2}{c}{\tablename\ \thetable\ -- \textit{Continued from previous page}} \\
% \toprule

% \endhead

% \midrule
% \multicolumn{2}{r}{\textit{Continued on next page}} \\
% \endfoot

% \bottomrule
% \caption{RAG-General-Bench Example : QA}
% \label{tab:rag-bench-samples-qa} \\
% \endlastfoot

% Reference Answer & 요양병원은 마스크 착용 의무화된 곳으로 규정되어 있습니다. 따라서 A씨는 두 번 다 마스크를 착용하지 않았으므로, 위반 당사자인 A씨는 위반 횟수와 관계 없이 각각 10만원의 과태료를 부과받으므로, 총 20만원의 과태료를 내야 합니다.
% 요양병원 원장 B씨는 관리자로서 이용자에게 마스크 착용을 안내하고 강제하는 의무가 있습니다. 만약 B씨가 해당 의무를 다하지 못했다면, 과태료의 부과 기준에 따라 1차 위반 시 50만원, 2차 위반 시 100만원의 과태료를 부과받게 됩니다. 따라서 B씨는 총 150만원의 과태료를 내야합니다.\\\\

% \end{longtable}

% \begin{longtable}{lp{0.8\textwidth}}
% \toprule
% Context & \{doc1\}

% \{ "title": "암살", "director": "최동훈", "actors": ["전지현", "이정재", "하정우", "오달수", "조진웅", "이경영", "최덕문"], "release\_date": "2015년 7월 22일", "plot": "1933년, 조국이 사라진 시대. 대한민국 임시정부는 일본 측에 노출되지 않은 세 명을 암살 작전에 지목한다. 독립군 저격수 안옥윤, 신흥무관학교 출신 속사포, 폭탄 전문가 황덕삼! 김구의 두터운 신임을 받는 임시정부 경무국 대장 염석진은 이들을 찾아 나서기 시작한다. 암살단의 타겟은 조선주둔군 사령관 카와구치 마모루와 친일파 강인국. 한편, 누군가에게 거액의 의뢰를 받은 청부살인업자 하와이 피스톨이 암살단의 뒤를 쫓는데... 친일파 암살작전을 둘러싼 이들의 예측할 수 없는 운명이 펼쳐진다!" \}

% \dots
% \\\\

% Instruction & 영화 $<$암살$>$$, $$<$국제시장$>$$, $$<$밀정$>$ 중 시대적 배경이 가장 오래 전인 영화는 무엇인가요?\\\\
% \midrule
% \endfirsthead

% \multicolumn{2}{c}{\tablename\ \thetable\ -- \textit{Continued from previous page}} \\
% \toprule

% \endhead

% \midrule
% \multicolumn{2}{r}{\textit{Continued on next page}} \\
% \endfoot

% \bottomrule
% \caption{RAG-General-Bench Example : QA}
% \label{tab:rag-bench-samples-qa} \\
% \endlastfoot

% Reference Answer & 시대적 배경이 가장 오래 전인 영화는 $<$밀정$>$입니다.

% $<$암살$>$은 1933년을 배경으로 하고 있고, $<$국제시장$>$은 1950년 한국전쟁 이후, $<$밀정$>$은 1920년대 일제강점기의 이야기입니다.

% 따라서 $<$밀정$>$의 시대적 배경이 가장 오래되었습니다. \\\\

% \end{longtable}


% \begin{longtable}{lp{0.8\textwidth}}
% \toprule
% Context & \{doc1\}

% G-Dragon

% 본명 권지용 (權志龍, Kwon Jiyong, クォン・ジヨン)

% 출생 1988년 8월 18일

% 국적 대한민국

% 본관 안동 권씨 ((安東 權氏)

% 신체 172.7cm, 57.6kg, A형

% 소속사 갤럭시코퍼레이션

% 레이블 EMPIRE

% 소속그룹 BIGBANG $|$ GD X TAEYANG $|$ GD\&TOP

% 포지션 리더, 메인래퍼 

% 데뷔 2001년 1월 15일

%  2006년 8월 19일 BIGBANG

%  2009년 8월 18일 정규 1집 Heartbreaker

% MBTI ENTP

% \dots
% \\\\

% Instruction & GD의 영향력을 음악 부분과 패션 부분으로 나눠서 주제별로 요약해줘. 너무 길지 않게 각각 300자 이내로\\\\

% \midrule
% \endfirsthead

% \multicolumn{2}{c}{\tablename\ \thetable\ -- \textit{Continued from previous page}} \\
% \toprule

% \endhead

% \midrule
% \multicolumn{2}{r}{\textit{Continued on next page}} \\
% \endfoot

% \bottomrule
% \caption{RAG-General-Bench Example : Summarize}
% \label{tab:rag-bench-samples-summarize} \\
% \endlastfoot

% Reference Answer & 음악 부분:
% G-DRAGON은 BIGBANG의 리더로서 작곡가, 프로듀서 역할을 맡아 히트곡을 대거 선보였다. 그의 프로듀싱 능력을 증명하는 대표적인 곡은 '거짓말', '마지막 인사', '하루하루', '붉은 노을' 등이다. 이러한 능력을 바탕으로 그는 가요계 역사상 그룹, 유닛, 솔로 모두 15년간 음원 차트 1위를 한 번도 놓친 적이 없는 가수가 되었다. 특히 그는 아이돌 중 최초로 저작권협회 정회원 등록을 받고, 가장 많은 작사료를 받는 인물로 저작권 대상을 수상하였다.
% \newline\newline
% 패션 부분:
% G-DRAGON은 패션 아이콘으로서 뛰어난 센스를 보여주었다. 그는 명품 브랜드 글로벌 엠버서더로 선정되는 등 패션 업계에서는 상당한 위상을 가지고 있다. 나이키와의 콜라보레이션으로 통해 전 세계적으로 패션 위상을 제고했으며, 다양한 패션 아이템과 스타일을 유행시켰다. 남자가 트위드 자켓을 입고 샤넬 백을 매는 등 젠더리스 패션의 선두자이자 아이콘으로서 패션계 전반에 걸쳐 전례 없는 유행을 선도하였다. 스트릿 패션과 스니커즈 시장에서도 큰 영향력을 행사하였으며, K-POP 시장의 명품 진출의 길을 만들었다.
% \\\\

% \end{longtable}


% \begin{longtable}{lp{0.8\textwidth}}
% \toprule
% Context & \{doc1\}

% \#\#\# 라흐마니노프 1차 Casting Schedule

% $|$ 일자 $|$ 시간 $|$ 라흐마니노프 $|$ 니콜라이 $|$ 피아니스트 $|$

% $|$-------------$|$--------$|$--------------$|$-------------$|$------------$|$

% $|$ 10.25 (화) $|$ 15:00 $|$ 박신유 $|$ 서경수 $|$ 최지훈 $|$

% $|$ 10.26 (수) $|$ 19:00 $|$ 박신유 $|$ 서경수 $|$ 최지훈 $|$

% $|$ 10.27 (목) $|$ 15:00 $|$ 박신유 $|$ 임동화 $|$ 김여옥 $|$

% $|$ 10.28 (금) $|$ 15:00 $|$ 박신유 $|$ 서경수 $|$ 최지훈 $|$

% $|$ 10.29 (토) $|$ 15:00 $|$ 박신유 $|$ 서경수 $|$ 최지훈 $|$

% $|$ 10.29 (토) $|$ 19:00 $|$ 박신유 $|$ 임동화 $|$ 김여옥 $|$\\\\

% Instruction & 니콜라이 역 서경수 배우의 코로나 확진으로 인해 부득이하게 스케줄을 변경하게 되었습니다. 라흐마니노프 1차 캐스팅 스케줄에서 10월 25일부터 28일까지 니콜라이 역을 임동화 배우로 수정해 주세요.\\\\
% \midrule
% \endfirsthead

% \multicolumn{2}{c}{\tablename\ \thetable\ -- \textit{Continued from previous page}} \\
% \toprule

% \endhead

% \midrule
% \multicolumn{2}{r}{\textit{Continued on next page}} \\
% \endfoot

% \bottomrule
% \caption{RAG-General-Bench Example : Rewrite}
% \label{tab:rag-bench-samples-rewrite} \\
% \endlastfoot

% Reference Answer & \#\#\# 라흐마니노프 1차 Casting Schedule

% $|$ 일자 $|$ 시간 $|$ 라흐마니노프 $|$ 니콜라이 $|$ 피아니스트 $|$

% $|$-------------$|$--------$|$--------------$|$-------------$|$------------$|$

% $|$ 10.25 (화) $|$ 15:00 $|$ 박신유 $|$ 임동화 $|$ 최지훈 $|$

% $|$ 10.26 (수) $|$ 19:00 $|$ 박신유 $|$ 임동화 $|$ 최지훈 $|$

% $|$ 10.27 (목) $|$ 15:00 $|$ 박신유 $|$ 임동화 $|$ 김여옥 $|$

% $|$ 10.28 (금) $|$ 15:00 $|$ 박신유 $|$ 임동화 $|$ 최지훈 $|$

% $|$ 10.29 (토) $|$ 15:00 $|$ 박신유 $|$ 서경수 $|$ 최지훈 $|$

% $|$ 10.29 (토) $|$ 19:00 $|$ 박신유 $|$ 임동화 $|$ 김여옥 $|$\\\\

% \end{longtable}


% \begin{longtable}{lp{0.8\textwidth}}
% \toprule
% Context & \{doc1\}

% 흑백요리사 출연진

% - 심사위원 2인

%  1. 백종원

%  대한민국의 대표적인 요식업 경영인으로 더본코리아 대표이다.

%  2. 안성재

%  2024년 기준 국내 유일의 미쉐린 가이드 3스타 레스토랑이었던 파인 다이닝 '모수 서울'의 오너셰프이다.

% - 백수저 20인

%  1. 김도윤

%  국내 한식 다이닝 기반의 레스토랑 미쉐린 가이드 1스타 윤서울, 면서울의 오너 쉐프이다.

%  \dots
% \\\\

% Instruction & 흑백요리사에 출연한 백수저 중에서 미슐랭 셰프 이름만 뽑아주세요.\\\\
% \midrule
% \endfirsthead

% \multicolumn{2}{c}{\tablename\ \thetable\ -- \textit{Continued from previous page}} \\
% \toprule

% \endhead

% \midrule
% \multicolumn{2}{r}{\textit{Continued on next page}} \\
% \endfoot

% \bottomrule
% \caption{RAG-General-Bench Example : Extraction}
% \label{tab:rag-bench-samples-extraction} \\
% \endlastfoot

% Reference Answer &  흑백요리사에 출연한 백수저 중 미슐랭 셰프는 다음과 같습니다.

% 1. 김도윤

% 2. 방기수

% 3. 조셉 리저우드

% 4. 파브리치오 페라리

% 5. 황진선
% \\\\

% \end{longtable}

\end{CJK}
