\subsection{Relationship with Retrieval-Augmented Generation}\label{app:rag}
\begin{figure*}[t]
    \centering
    \includegraphics[width=1 \linewidth]{figures/rag-comparison/plotv2.pdf}
    \vspace{-0.8cm}
    \caption{
    \textbf{Relationship with retrieval-augmented generation.} 
    }
    \label{fig:rag-comparison}
\end{figure*}


In this section, we discuss the relationship between local-remote collaboration and retrieval-augmented generation (RAG), a technique that reduces the number of tokens processed by an LM by retrieving a subset of relevant documents or chunks LM~\cite{lewis2020retrieval,karpukhin2020dense,lee2019latent}.

Retrieval-augmented generation and local-remote collaboration (\textit{e.g.} \system) are complementary techniques. 
They both provide a means to reduce cost by providing an LLM with a partial view of a large context. 
But, as we discuss below, they also have different error profiles and can be used in conjunction to improve performance.

\subsubsection{Comparison of \system and RAG on \finance}\label{app:rag:finance}
In \Cref{fig:rag-comparison} (left), we plot the quality-cost trade-off on \finance for local-remote systems (\naive and \system) and RAG systems using BM25 and OpenAI's \texttt{text-embedding-3-small} embeddings~\cite{article,neelakantan2022text}.
For RAG, we use a chunk size of 1000 characters, which we found to be optimal for this dataset after sweeping over chunk sizes with the BM25 retriever (see \Cref{fig:rag-comparison} (center)). We show how a simple hyperparameter (number of retrieved chunks provided to the remote model) allows us to trade off quality of the RAG system for remote cost.
% 
Furthermore, we note that when the BM25 RAG system provides 50 or more chunks of the document to the remote model, it exceeds the performance of the remote model with the full context. This likely indicates that RAG helps in minimizing distractions from the long context. 
% 
For \finance, when compared to \system, the RAG system with OpenAI embeddings reaches similar points in the quality-cost trade-off space. Interestingly however, none of the RAG configurations are are able to match the quality of \naive at the same low cost. 

\subsubsection{Comparison of \system and RAG (Embeddings + BM25) on Summarization Tasks}\label{app:rag:summary}
RAG is a very suitable approach for \finance, since all of the questions heavily rely on information extraction from specific sections of financial statements. However, RAG will not be suitable for a summarization task, unlike small LMs. Therefore, we use the long-document summarization dataset, \textsc{BooookScore}~\citep{chang2023booookscore}. \textsc{BooookScore} which contains a set of 400 books published between 2023-2024. The average story length in \textsc{Booookscore} is 128179 tokens with a max of 401486 tokens and a minimum of 26926 tokens. We utilize both \system, RAG (w/Embeddings + BM25), and \gpt only to complete the task. We describe the set-up for all three approaches next.
\\

\textbf{\system for summarization} In applying \system to the task, the $\locallm$ (\llamathreetwo-3B-Instruct) provides summaries on chunks of the original text, passing a list of chunk summaries to the $\remotelm$ (\gpt). $\remotelm$ produces the final summary. 

\textbf{RAG (Embedding) for summarization} In our embedding-based RAG approach, we use the OpenAI \textsc{text-embedding-3-small} to embed chunks of the original text (of length 5000 characters) and we retrieve the top-15 most relevant chunks using the query ``Summarize the provided text''. We then prompt \gpt to generate a complete summary over the retrieved chunks.

\textbf{RAG (BM25) for summarization} In our BM25-based RAG approach, we use the BM25 to retrieve chunks of the original text (of length 5000 characters) based on the query: ``Summarize the provided text''. We retrieve the top-15 most relevant chunks and prompt \gpt to produce a final summary over the retrieved chunks. We choose top-15 to ensure the number of tokens passed up by the baseline is comparable with those passed up by \system.

\textbf{GPT-4o} In our final baseline, we use \gpt alone to create the story summaries. For texts that extend beyond the 128K context length window, we truncate the stories.

\paragraph{Evaluation}
\begin{itemize}
    \setlength{\itemindent}{-1em}
    \item \textbf{Qualitative} In Table~\ref{tab:story-summaries} we provide samples outputs from each of the 4 methods described above. We highlight major events in red, themes in green, locations in blue and names in indigo. The samples demonstrate that amongst all the methods, \system outputs contain the most entity mentions and story specific details. Moreover, when compared to \gpt-only $\remotelm$,
\system is $9.3\times$ more efficient --- 11,500 versus the full 108,185 prefill tokens.

The summaries from \system are generally $1.3\times$ longer and more verbose than the RAG systems' summaries, likely indicating that the former is more effective at ``passing forward'' salient information. Moreover, RAG systems' summaries are missing the main arc of the narrative in favor of what seems an assortment of facts.  
    \setlength{\itemindent}{-1em}
    \item \textbf{Quantitative} We additionally perform a quantitative analysis of the generated summaries using a LLM-as-a-judge framework. As an evaluator, we use the \textsc{claude-3.5-sonnet} model, to avoid any biases between the evaluator and the supervisor model. We prompt the model with the generated summary, ground truth summary (gpt4-4096-inc-cleaned) provided from the original \textsc{BooookScore} generations, and a grading rubric (see Figure~\ref{rubric:summary}). The rubric evaluates 7 criteria: coherence, relevance, conciseness, comprehensiveness, engagement \& readability, accuracy, and thematic depth. We prompt \textsc{claude-3.5-sonnet} to generate a score (1-5) for each of the criteria and average the scores. We find that summaries generated by \system score comparably with \textsc{GPT4o}-only generated summaries, while RAG based baselines perform worse. Our results can be found in Table~\ref{tab:summary_qualitative_scores}.
\end{itemize}


\begin{figure}[h]
    \centering
    \begin{tcolorbox}[width=\textwidth, colframe=black!50, colback=white, sharp corners]
        \textbf{Evaluation Rubric for Summaries}
        \begin{enumerate}
            \item \textbf{Coherence (1-5):} Summary is logically structured, with clear connections between events, avoiding abrupt jumps or inconsistencies.
            \item \textbf{Relevance (1-5):} Accurately reflects key themes, events, and characters, focusing on essential details without unnecessary plot points.
            \item \textbf{Conciseness (1-5):} Thorough yet avoids excessive detail, presenting necessary information without redundancy.
            \item \textbf{Comprehensiveness (1-5):} Covers all major characters, events, and themes, ensuring a complete overview without omissions.
            \item \textbf{Engagement \& Readability (1-5):} Engaging and easy to read, with well-constructed sentences and clear, precise language.
            \item \textbf{Accuracy (1-5):} Stays true to the book’s storyline, themes, and tone, with correct details, names, and events.
            \item \textbf{Thematic Depth (1-5):} Identifies underlying themes and messages, providing insights into conflicts, motivations, and resolutions.
        \end{enumerate}
    \end{tcolorbox}
    \caption{Evaluation Rubric for Summaries}
    \label{rubric:summary}
\end{figure}

% \\

% \textbf{Qualitative} In Table\ref{tab:story-summaries} we provide samples outputs from each of the 4 methods described above. We highlight major events in red, themes in green, locations in blue and names in indigo. The samples demonstrate that amongst all the methods, \system outputs contain the most entity mentions and story specific details. Moreover, when compared to \gpt-only \remotelm,
% \system is $9.3\times$ more efficient --- 11,500 versus the full 108,185 prefill tokens.

% The summaries from \system are generally $1.3\times$ longer and more verbose than the RAG systems' summaries, likely indicating that the former is more effective at ``passing forward'' salient information. Moreover, RAG systems' summaries are missing the main arc of the narrative in favor of what seems an assortment of facts.  

% \textbf{Quantitative} We additionally perform a quantitative analysis of the generated summaries using a LLM-as-a-judge framework. As an evaluator, we use the \textsc{claude-3.5-sonnet} model, to avoid any biases between the evaluator and the supervisor model. We prompt the model with the generated summary, ground truth summary (gpt4-4096-inc-cleaned) provided from the original \textsc{BooookScore} generations, and a grading rubric. The rubric evaluates 7 criteria: coherence, relevance, conciseness, comprehensiveness, engagement \& readability, accuracy, and thematic depth. We prompt \textsc{claude-3.5-sonnet} to generate a score (1-5) for each of the criteria and average the scores. We find that summaries generated by \system score comparably with \textsc{GPT4o}-only generated summaries, while RAG based baslines perform worse. Our results can be found in Table~\ref{tab:summary_qualitative_scores}.

\begin{table}[h]
    \centering
    \begin{tabular}{lc}
        \hline
        Method & Score \\
        \hline
        \system & 3.01 \\
        GPT4o & 3.06 \\
        RAG (BM25) & 2.48 \\
        RAG (Embedding) & 2.38 \\
        \hline
        \hline
    \end{tabular}
    \caption{Comparison of Methods and Rubric Scores}
    \label{tab:summary_qualitative_scores}
\end{table}






% it  utilizes 11.5K prefill tokens of $\remotelm$ where as \gpt-only uses 108,185 prefill tokens. 


% Define custom colors
\definecolor{character}{HTML}{4B0082}  % Indigo for character names
\definecolor{event}{HTML}{8B0000}      % Dark red for major events
\definecolor{theme}{HTML}{006400}      % Dark green for themes
\definecolor{location}{HTML}{000080}   % Navy for locations



\scriptsize
\begin{longtable}{|p{1cm}|p{3cm}|p{3cm}|p{3cm}|p{3cm}|}    \caption{Comparative Analysis of Story Summaries Across Different Methods}
    \label{tab:story-summaries} \\
    \hline
    \textbf{Story Title} & \textbf{Minions} & \textbf{GPT-4o} & \textbf{BM25} & \textbf{Embedding} \\
    \hline
    \endfirsthead

    \hline
    \multicolumn{5}{|c|}{\textit{Continued from previous page}} \\
    \hline
    \textbf{Story Title} & \textbf{Minions} & \textbf{GPT-4o} & \textbf{BM25} & \textbf{Embedding} \\
    \hline
    \endhead

    \hline
    \multicolumn{5}{|r|}{\textit{Continued on next page}} \\
    \hline
    \endfoot

    \hline
    \endlastfoot

    
    \textit{A Likely Story} & The novel follows the complex dynamics of the {\color{character}Manning family}, focusing on {\color{character}Isabelle Manning}, a struggling writer grappling with the legacy of her famous father, {\color{character}Ward Manning}, and the recent death of her mother, {\color{character}Claire}. As Isabelle navigates her career and personal life, she uncovers family secrets, including her father's {\color{event}plagiarism} and her mother's {\color{event}hidden manuscript}, which challenge her identity and ambitions. The narrative explores themes of {\color{theme}fame, family expectations, and personal growth}, culminating in Isabelle's journey towards self-discovery and independence. Set against the backdrop of {\color{location}New York City} and {\color{location}Sag Harbor}, the story delves into the intricacies of familial relationships and the pursuit of creative fulfillment. & {\color{character}Isabelle Manning}, daughter of a famous author, {\color{event}discovers an unfinished manuscript} by her late mother, {\color{character}Claire}. She completes and publishes it as her own, leading to success but also guilt and exposure. Her father, {\color{character}Ward}, reveals he sabotaged her first book to teach her resilience, deepening her sense of betrayal. Ultimately, Isabelle confesses her actions, seeks redemption, and begins to redefine her identity beyond her father's shadow. & {\color{character}Isabelle Manning}, overshadowed by her famous father {\color{character}Ward}, receives a {\color{event}threatening text}, "I KNOW WHAT YOU DID," hinting at a secret tied to her book, "{\color{location}Underpainting}." Meanwhile, her mother {\color{character}Claire's} past support and her father's {\color{event}hidden illness} add layers to Isabelle's challenges. & In a narrative spanning {\color{location}New York} and {\color{location}Sag Harbor}, {\color{character}Claire Cunningham} grapples with her identity beyond being a {\color{location}Vassar} graduate, while her daughter {\color{character}Isabelle} faces personal and professional challenges, including her father's illness and her own writing struggles. \\
    \hline
    \textit{All the Dangerous Things} & {\color{character}Isabelle Drake}, a woman grappling with the {\color{event}traumatic disappearance of her son Mason}, navigates a complex web of grief, guilt, and suspicion. As she becomes entangled with true crime enthusiasts and investigators, including {\color{character}podcast host Waylon} and {\color{character}Detective Dozier}, Isabelle's quest for truth reveals unsettling family secrets and personal betrayals. Her journey is marked by strained relationships, particularly with her ex-husband {\color{character}Ben} and his connections to other women, including {\color{character}Valerie} and {\color{character}Allison}. Throughout the narrative, themes of {\color{theme}motherhood, mental health, and societal judgment} are explored, culminating in a deeper understanding of the pressures and expectations faced by women. & {\color{character}Isabelle Drake}, plagued by {\color{theme}insomnia and guilt}, is desperate to find her missing son, {\color{character}Mason}. She suspects her husband, {\color{character}Ben}, and his new partner, {\color{character}Valerie}. With {\color{character}Waylon's} help, she discovers {\color{character}Abigail Fisher}, manipulated by {\color{character}Valerie}, took Mason believing she was rescuing him. & The narrative follows {\color{character}Isabelle}, dealing with {\color{event}Mason's disappearance}. She works with podcaster {\color{character}Waylon}, uncovering links to {\color{character}Ben's} deceased wife, {\color{character}Allison}. & {\color{character}Isabelle}, struggling with {\color{theme}grief and insomnia}, joins a {\color{location}grief counseling group}. She meets {\color{character}Valerie} and collaborates with {\color{character}Waylon}, but becomes wary after finding unsettling information on his laptop. \\
    \hline
    \textit{A Living Remedy: A Memoir} & {\color{character}Nicole Chung}, a {\color{theme}Korean American adoptee}, reflects on her complex relationships with her adoptive parents, her identity, and the challenges of navigating life as a minority in a predominantly white community in {\color{location}Oregon}. Her memoir explores themes of {\color{theme}family, loss, and resilience}, as she recounts her {\color{event}father's death from kidney failure}, and her {\color{event}mother's battle with cancer}. Amidst these personal challenges, Chung grapples with her own grief, financial struggles, and the impact of the {\color{event}COVID-19 pandemic}, while finding solace in her family, faith, and writing. Her journey is marked by a deep appreciation for her parents' sacrifices, the support of her husband and children, and the enduring legacy of love and forgiveness instilled by her mother. & {\color{character}Nicole Chung's} memoir explores her journey after the {\color{event}loss of her adoptive parents}. As a Korean adoptee, she reflects on family's financial struggles, parents' health battles, and their deaths' impact on her identity. She finds solace in writing and her own family. & The protagonist struggles with {\color{event}visiting her dying mother during the COVID-19 pandemic}. The story explores {\color{theme}grief, family responsibility, and cherishing life} amidst adversity. & A woman reflects on her {\color{event}parents' illnesses and deaths}, balancing her role as a daughter and mother. She finds solace in {\color{theme}childhood memories} and the legacy of her parents' love. \\
    \hline
        \textit{A House with Good Bones} & {\color{character}Samantha}, a 32-year-old archaeoentomologist, returns to her childhood home on {\color{location}Lammergeier Lane} in {\color{location}North Carolina}, where she confronts her family's dark past, including her grandmother {\color{character}Gran Mae's} mysterious and malevolent legacy. As Samantha navigates her mother's strange behavior and the {\color{event}eerie presence of vultures}, she uncovers secrets involving {\color{event}ritual magic}, a jar of human teeth, and the supernatural "underground children." With the help of her friend {\color{character}Gail} and handyman {\color{character}Phil}, Samantha faces the haunting manifestations of her family's history. The novel explores themes of {\color{theme}family, memory, and the supernatural}, blending elements of horror and fantasy. & {\color{character}Samantha Montgomery} returns home to find her mother acting strangely and the house devoid of insects. She uncovers a dark history involving her {\color{character}great-grandfather}, a sorcerer, and her grandmother, who used {\color{event}roses to wield power}. With help from {\color{character}Gail} and {\color{character}Phil}, she confronts the terrifying "{\color{event}underground children}," using rose power to banish threats. & The protagonist returns to their {\color{character}grandmother's} unchanged garden, filled with roses but {\color{event}mysteriously devoid of insects}. They uncover {\color{theme}unsettling truths about their grandmother's past} and their mother's current state of mind. The narrative explores themes of {\color{theme}family legacy and the passage of time}. & {\color{character}Samantha}, an archaeoentomologist, returns to her childhood home and finds herself investigating {\color{event}insect collections}. Dealing with {\color{theme}sleep paralysis} and memories of her grandmother, she discovers the {\color{event}peculiar absence of insects} in the garden. She navigates family dynamics and her mother's anxiety amid an eerie atmosphere. \\

    
\end{longtable}






