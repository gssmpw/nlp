

\vspace{-1em}
\section{Results}
\label{sec:results}
\begin{figure}[t]
    \centering
    \includegraphics[width=1\linewidth]{figures/scaling-edge-model/size_comm_efficiency.pdf}
    \vspace{-1em}
    \caption{\textbf{Trade-offs in edge model performance, communication efficiency, and cost of sequential communication.} \textbf{(Left)} Accuracy vs. local model size, with the purple dashed line showing the \gpt model baseline. \textbf{(Right)} Communication efficiency of \system with different local LMs, where larger models (7–8B) are more token-efficient.}
    \label{fig:scaling-edge-model}
  \end{figure}



%

\begin{figure}[h!]
    \centering
    \includegraphics[width=0.5\linewidth]{figures/lm-bottleneck/plot.pdf}
    \caption{\textbf{Communication efficiency of \system when utilizing different local LM's.} x-axis is the total number of token sent up to the remote model. We observe that larger local models (7-8B parameters) are more token efficient than smaller LMs.
    }
    \label{fig:lm-bottleneck}
  \end{figure}




% 


\begin{figure}[h!]
    \centering
    \includegraphics[width=\linewidth]{figures/model-longitudinal/plot.pdf}
    \caption{\textbf{Edge-remote quality over time.} The x-axis represents the date of release.
}
    \label{fig:model-longitudinal}
  \end{figure}



% \section{Experiments}
\label{sec:experiments}
The experiments are designed to address two key research questions.
First, \textbf{RQ1} evaluates whether the average $L_2$-norm of the counterfactual perturbation vectors ($\overline{||\perturb||}$) decreases as the model overfits the data, thereby providing further empirical validation for our hypothesis.
Second, \textbf{RQ2} evaluates the ability of the proposed counterfactual regularized loss, as defined in (\ref{eq:regularized_loss2}), to mitigate overfitting when compared to existing regularization techniques.

% The experiments are designed to address three key research questions. First, \textbf{RQ1} investigates whether the mean perturbation vector norm decreases as the model overfits the data, aiming to further validate our intuition. Second, \textbf{RQ2} explores whether the mean perturbation vector norm can be effectively leveraged as a regularization term during training, offering insights into its potential role in mitigating overfitting. Finally, \textbf{RQ3} examines whether our counterfactual regularizer enables the model to achieve superior performance compared to existing regularization methods, thus highlighting its practical advantage.

\subsection{Experimental Setup}
\textbf{\textit{Datasets, Models, and Tasks.}}
The experiments are conducted on three datasets: \textit{Water Potability}~\cite{kadiwal2020waterpotability}, \textit{Phomene}~\cite{phomene}, and \textit{CIFAR-10}~\cite{krizhevsky2009learning}. For \textit{Water Potability} and \textit{Phomene}, we randomly select $80\%$ of the samples for the training set, and the remaining $20\%$ for the test set, \textit{CIFAR-10} comes already split. Furthermore, we consider the following models: Logistic Regression, Multi-Layer Perceptron (MLP) with 100 and 30 neurons on each hidden layer, and PreactResNet-18~\cite{he2016cvecvv} as a Convolutional Neural Network (CNN) architecture.
We focus on binary classification tasks and leave the extension to multiclass scenarios for future work. However, for datasets that are inherently multiclass, we transform the problem into a binary classification task by selecting two classes, aligning with our assumption.

\smallskip
\noindent\textbf{\textit{Evaluation Measures.}} To characterize the degree of overfitting, we use the test loss, as it serves as a reliable indicator of the model's generalization capability to unseen data. Additionally, we evaluate the predictive performance of each model using the test accuracy.

\smallskip
\noindent\textbf{\textit{Baselines.}} We compare CF-Reg with the following regularization techniques: L1 (``Lasso''), L2 (``Ridge''), and Dropout.

\smallskip
\noindent\textbf{\textit{Configurations.}}
For each model, we adopt specific configurations as follows.
\begin{itemize}
\item \textit{Logistic Regression:} To induce overfitting in the model, we artificially increase the dimensionality of the data beyond the number of training samples by applying a polynomial feature expansion. This approach ensures that the model has enough capacity to overfit the training data, allowing us to analyze the impact of our counterfactual regularizer. The degree of the polynomial is chosen as the smallest degree that makes the number of features greater than the number of data.
\item \textit{Neural Networks (MLP and CNN):} To take advantage of the closed-form solution for computing the optimal perturbation vector as defined in (\ref{eq:opt-delta}), we use a local linear approximation of the neural network models. Hence, given an instance $\inst_i$, we consider the (optimal) counterfactual not with respect to $\model$ but with respect to:
\begin{equation}
\label{eq:taylor}
    \model^{lin}(\inst) = \model(\inst_i) + \nabla_{\inst}\model(\inst_i)(\inst - \inst_i),
\end{equation}
where $\model^{lin}$ represents the first-order Taylor approximation of $\model$ at $\inst_i$.
Note that this step is unnecessary for Logistic Regression, as it is inherently a linear model.
\end{itemize}

\smallskip
\noindent \textbf{\textit{Implementation Details.}} We run all experiments on a machine equipped with an AMD Ryzen 9 7900 12-Core Processor and an NVIDIA GeForce RTX 4090 GPU. Our implementation is based on the PyTorch Lightning framework. We use stochastic gradient descent as the optimizer with a learning rate of $\eta = 0.001$ and no weight decay. We use a batch size of $128$. The training and test steps are conducted for $6000$ epochs on the \textit{Water Potability} and \textit{Phoneme} datasets, while for the \textit{CIFAR-10} dataset, they are performed for $200$ epochs.
Finally, the contribution $w_i^{\varepsilon}$ of each training point $\inst_i$ is uniformly set as $w_i^{\varepsilon} = 1~\forall i\in \{1,\ldots,m\}$.

The source code implementation for our experiments is available at the following GitHub repository: \url{https://anonymous.4open.science/r/COCE-80B4/README.md} 

\subsection{RQ1: Counterfactual Perturbation vs. Overfitting}
To address \textbf{RQ1}, we analyze the relationship between the test loss and the average $L_2$-norm of the counterfactual perturbation vectors ($\overline{||\perturb||}$) over training epochs.

In particular, Figure~\ref{fig:delta_loss_epochs} depicts the evolution of $\overline{||\perturb||}$ alongside the test loss for an MLP trained \textit{without} regularization on the \textit{Water Potability} dataset. 
\begin{figure}[ht]
    \centering
    \includegraphics[width=0.85\linewidth]{img/delta_loss_epochs.png}
    \caption{The average counterfactual perturbation vector $\overline{||\perturb||}$ (left $y$-axis) and the cross-entropy test loss (right $y$-axis) over training epochs ($x$-axis) for an MLP trained on the \textit{Water Potability} dataset \textit{without} regularization.}
    \label{fig:delta_loss_epochs}
\end{figure}

The plot shows a clear trend as the model starts to overfit the data (evidenced by an increase in test loss). 
Notably, $\overline{||\perturb||}$ begins to decrease, which aligns with the hypothesis that the average distance to the optimal counterfactual example gets smaller as the model's decision boundary becomes increasingly adherent to the training data.

It is worth noting that this trend is heavily influenced by the choice of the counterfactual generator model. In particular, the relationship between $\overline{||\perturb||}$ and the degree of overfitting may become even more pronounced when leveraging more accurate counterfactual generators. However, these models often come at the cost of higher computational complexity, and their exploration is left to future work.

Nonetheless, we expect that $\overline{||\perturb||}$ will eventually stabilize at a plateau, as the average $L_2$-norm of the optimal counterfactual perturbations cannot vanish to zero.

% Additionally, the choice of employing the score-based counterfactual explanation framework to generate counterfactuals was driven to promote computational efficiency.

% Future enhancements to the framework may involve adopting models capable of generating more precise counterfactuals. While such approaches may yield to performance improvements, they are likely to come at the cost of increased computational complexity.


\subsection{RQ2: Counterfactual Regularization Performance}
To answer \textbf{RQ2}, we evaluate the effectiveness of the proposed counterfactual regularization (CF-Reg) by comparing its performance against existing baselines: unregularized training loss (No-Reg), L1 regularization (L1-Reg), L2 regularization (L2-Reg), and Dropout.
Specifically, for each model and dataset combination, Table~\ref{tab:regularization_comparison} presents the mean value and standard deviation of test accuracy achieved by each method across 5 random initialization. 

The table illustrates that our regularization technique consistently delivers better results than existing methods across all evaluated scenarios, except for one case -- i.e., Logistic Regression on the \textit{Phomene} dataset. 
However, this setting exhibits an unusual pattern, as the highest model accuracy is achieved without any regularization. Even in this case, CF-Reg still surpasses other regularization baselines.

From the results above, we derive the following key insights. First, CF-Reg proves to be effective across various model types, ranging from simple linear models (Logistic Regression) to deep architectures like MLPs and CNNs, and across diverse datasets, including both tabular and image data. 
Second, CF-Reg's strong performance on the \textit{Water} dataset with Logistic Regression suggests that its benefits may be more pronounced when applied to simpler models. However, the unexpected outcome on the \textit{Phoneme} dataset calls for further investigation into this phenomenon.


\begin{table*}[h!]
    \centering
    \caption{Mean value and standard deviation of test accuracy across 5 random initializations for different model, dataset, and regularization method. The best results are highlighted in \textbf{bold}.}
    \label{tab:regularization_comparison}
    \begin{tabular}{|c|c|c|c|c|c|c|}
        \hline
        \textbf{Model} & \textbf{Dataset} & \textbf{No-Reg} & \textbf{L1-Reg} & \textbf{L2-Reg} & \textbf{Dropout} & \textbf{CF-Reg (ours)} \\ \hline
        Logistic Regression   & \textit{Water}   & $0.6595 \pm 0.0038$   & $0.6729 \pm 0.0056$   & $0.6756 \pm 0.0046$  & N/A    & $\mathbf{0.6918 \pm 0.0036}$                     \\ \hline
        MLP   & \textit{Water}   & $0.6756 \pm 0.0042$   & $0.6790 \pm 0.0058$   & $0.6790 \pm 0.0023$  & $0.6750 \pm 0.0036$    & $\mathbf{0.6802 \pm 0.0046}$                    \\ \hline
%        MLP   & \textit{Adult}   & $0.8404 \pm 0.0010$   & $\mathbf{0.8495 \pm 0.0007}$   & $0.8489 \pm 0.0014$  & $\mathbf{0.8495 \pm 0.0016}$     & $0.8449 \pm 0.0019$                    \\ \hline
        Logistic Regression   & \textit{Phomene}   & $\mathbf{0.8148 \pm 0.0020}$   & $0.8041 \pm 0.0028$   & $0.7835 \pm 0.0176$  & N/A    & $0.8098 \pm 0.0055$                     \\ \hline
        MLP   & \textit{Phomene}   & $0.8677 \pm 0.0033$   & $0.8374 \pm 0.0080$   & $0.8673 \pm 0.0045$  & $0.8672 \pm 0.0042$     & $\mathbf{0.8718 \pm 0.0040}$                    \\ \hline
        CNN   & \textit{CIFAR-10} & $0.6670 \pm 0.0233$   & $0.6229 \pm 0.0850$   & $0.7348 \pm 0.0365$   & N/A    & $\mathbf{0.7427 \pm 0.0571}$                     \\ \hline
    \end{tabular}
\end{table*}

\begin{table*}[htb!]
    \centering
    \caption{Hyperparameter configurations utilized for the generation of Table \ref{tab:regularization_comparison}. For our regularization the hyperparameters are reported as $\mathbf{\alpha/\beta}$.}
    \label{tab:performance_parameters}
    \begin{tabular}{|c|c|c|c|c|c|c|}
        \hline
        \textbf{Model} & \textbf{Dataset} & \textbf{No-Reg} & \textbf{L1-Reg} & \textbf{L2-Reg} & \textbf{Dropout} & \textbf{CF-Reg (ours)} \\ \hline
        Logistic Regression   & \textit{Water}   & N/A   & $0.0093$   & $0.6927$  & N/A    & $0.3791/1.0355$                     \\ \hline
        MLP   & \textit{Water}   & N/A   & $0.0007$   & $0.0022$  & $0.0002$    & $0.2567/1.9775$                    \\ \hline
        Logistic Regression   &
        \textit{Phomene}   & N/A   & $0.0097$   & $0.7979$  & N/A    & $0.0571/1.8516$                     \\ \hline
        MLP   & \textit{Phomene}   & N/A   & $0.0007$   & $4.24\cdot10^{-5}$  & $0.0015$    & $0.0516/2.2700$                    \\ \hline
       % MLP   & \textit{Adult}   & N/A   & $0.0018$   & $0.0018$  & $0.0601$     & $0.0764/2.2068$                    \\ \hline
        CNN   & \textit{CIFAR-10} & N/A   & $0.0050$   & $0.0864$ & N/A    & $0.3018/
        2.1502$                     \\ \hline
    \end{tabular}
\end{table*}

\begin{table*}[htb!]
    \centering
    \caption{Mean value and standard deviation of training time across 5 different runs. The reported time (in seconds) corresponds to the generation of each entry in Table \ref{tab:regularization_comparison}. Times are }
    \label{tab:times}
    \begin{tabular}{|c|c|c|c|c|c|c|}
        \hline
        \textbf{Model} & \textbf{Dataset} & \textbf{No-Reg} & \textbf{L1-Reg} & \textbf{L2-Reg} & \textbf{Dropout} & \textbf{CF-Reg (ours)} \\ \hline
        Logistic Regression   & \textit{Water}   & $222.98 \pm 1.07$   & $239.94 \pm 2.59$   & $241.60 \pm 1.88$  & N/A    & $251.50 \pm 1.93$                     \\ \hline
        MLP   & \textit{Water}   & $225.71 \pm 3.85$   & $250.13 \pm 4.44$   & $255.78 \pm 2.38$  & $237.83 \pm 3.45$    & $266.48 \pm 3.46$                    \\ \hline
        Logistic Regression   & \textit{Phomene}   & $266.39 \pm 0.82$ & $367.52 \pm 6.85$   & $361.69 \pm 4.04$  & N/A   & $310.48 \pm 0.76$                    \\ \hline
        MLP   &
        \textit{Phomene} & $335.62 \pm 1.77$   & $390.86 \pm 2.11$   & $393.96 \pm 1.95$ & $363.51 \pm 5.07$    & $403.14 \pm 1.92$                     \\ \hline
       % MLP   & \textit{Adult}   & N/A   & $0.0018$   & $0.0018$  & $0.0601$     & $0.0764/2.2068$                    \\ \hline
        CNN   & \textit{CIFAR-10} & $370.09 \pm 0.18$   & $395.71 \pm 0.55$   & $401.38 \pm 0.16$ & N/A    & $1287.8 \pm 0.26$                     \\ \hline
    \end{tabular}
\end{table*}

\subsection{Feasibility of our Method}
A crucial requirement for any regularization technique is that it should impose minimal impact on the overall training process.
In this respect, CF-Reg introduces an overhead that depends on the time required to find the optimal counterfactual example for each training instance. 
As such, the more sophisticated the counterfactual generator model probed during training the higher would be the time required. However, a more advanced counterfactual generator might provide a more effective regularization. We discuss this trade-off in more details in Section~\ref{sec:discussion}.

Table~\ref{tab:times} presents the average training time ($\pm$ standard deviation) for each model and dataset combination listed in Table~\ref{tab:regularization_comparison}.
We can observe that the higher accuracy achieved by CF-Reg using the score-based counterfactual generator comes with only minimal overhead. However, when applied to deep neural networks with many hidden layers, such as \textit{PreactResNet-18}, the forward derivative computation required for the linearization of the network introduces a more noticeable computational cost, explaining the longer training times in the table.

\subsection{Hyperparameter Sensitivity Analysis}
The proposed counterfactual regularization technique relies on two key hyperparameters: $\alpha$ and $\beta$. The former is intrinsic to the loss formulation defined in (\ref{eq:cf-train}), while the latter is closely tied to the choice of the score-based counterfactual explanation method used.

Figure~\ref{fig:test_alpha_beta} illustrates how the test accuracy of an MLP trained on the \textit{Water Potability} dataset changes for different combinations of $\alpha$ and $\beta$.

\begin{figure}[ht]
    \centering
    \includegraphics[width=0.85\linewidth]{img/test_acc_alpha_beta.png}
    \caption{The test accuracy of an MLP trained on the \textit{Water Potability} dataset, evaluated while varying the weight of our counterfactual regularizer ($\alpha$) for different values of $\beta$.}
    \label{fig:test_alpha_beta}
\end{figure}

We observe that, for a fixed $\beta$, increasing the weight of our counterfactual regularizer ($\alpha$) can slightly improve test accuracy until a sudden drop is noticed for $\alpha > 0.1$.
This behavior was expected, as the impact of our penalty, like any regularization term, can be disruptive if not properly controlled.

Moreover, this finding further demonstrates that our regularization method, CF-Reg, is inherently data-driven. Therefore, it requires specific fine-tuning based on the combination of the model and dataset at hand.

Here, we analyze how the design of \system affects cost and quality. Our main takeaways are: 
\begin{itemize}
\item On average across three datasets, \system can recover $97.9\%$ of the performance of remote-only systems while spending $5.7\times$ less; 
\item We identify protocol hyper-parameters that let us flexibly trade-off cost and quality;
\item As local models grow stronger, \system becomes increasingly cost-effective.
  % \item \system would not have worked until mid-2024. With today's model, \system necessitates a local model that is $>3B$ parameters.
  % \item Communication efficiency is correlated with $\locallm$ capabilities: larger local models (\textit{i.e.} 8B parameters) are 1.53x more cost effective than their 1B parameter counter parts.
  % \item Scaling local workloads through aggressive parallelization improves performance, with chunking and sub-task generation being more cost effective than sampling.
  % \item Back-and-forth communication improves performance by up to \yell{XXX} points, but comes at a cost.
  % \item TODO
\end{itemize}

We structure our analysis around three core design choices:
%\vspace{-0.75em}
\begin{enumerate}
    \item \textbf{Model choice} \textit{How does the choice of local and remote model effect cost and quality?} We examine different model types and sizes for $\locallm$ and $\remotelm$ in \Cref{subsec:results-model}.\vspace{-0.5em}
    \item \textbf{Scaling parallel workloads on-device} \textit{How should we structure parallel workloads on the local device to maximize performance and minimize cost?} We highlight how scaling the local workloads can improve performance (\Cref{subsec:results-workloads}) and study the effects on cost.\vspace{-0.5em}
    \item \textbf{Sequential communication protocol} \textit{Can multiple rounds of communication improve quality? At what cost?} We explore this trade-off in \Cref{subsec:results-communication}.

\end{enumerate}

Our findings are detailed in Sections~\ref{subsec:results-model},~\ref{subsec:results-workloads}, and ~\ref{subsec:results-communication}. Finally, in \Cref{subsec:rag} we discuss the relationship between retrieval augemented generation and local-remote compute. 

\vspace{-0.5em}
\subsection{Experimental setup} 
\label{subsec:exp-setup}
\paragraph{Datasets and models} We evaluate \system on three benchmarks that are well suited for data-intensive reasoning: \finance, \longhealth, and \qasper. \finance tests financial document understanding with complex reasoning over reports. \longhealth focuses on tracking and interpreting longitudinal health records. \qasper assesses question answering over dense scientific papers. See Appendix~\ref{app:experiments-dataset} for details. We use two open-source model families (\llama, \qwen) as $\locallm$ and \gpt as $\remotelm$ (details in \Cref{app:experiments-models}).

\vspace{-0.5em}\subsection{Model choice}
\label{subsec:results-model}

This section explores the model requirements and generalization capabilities of \system, examining the local model sizes necessary for effective collaboration, the sensitivity of the communication protocol across different local-remote model pairings, and the longitudinal evolution of \system’ performance with advances in model capabilities over time.

\vspace{-0.5em}\paragraph{\textit{What size does $\locallm$ have to be in order to be effective in \system?}}
% \Cref{fig:scaling-edge-model} % x: parameters, y: accuracy, notes: vertical lines at different device types 

Our results demonstrate that \system starts being competitive with $\remotelm$-only baseline at the $3$B parameter model scale. When considering both the \qwen and \llama model families running locally, at $1$B scale, \system recovers 49.5\% of the \gpt-only baseline performance, 3B scale recovers 93.4\% and 8B recovers 97.9\% accuracy (see Table~\ref{table:main-tradeoff} for more details).

\vspace{-0.75em}\paragraph{\textit{How does the capacity of $\locallm$ affect the cost-accuracy tradeoff?}}
% x: parameters, y: accuracy, notes: vertical lines at different device types 
In our system, $\locallm$ implicitly acts as an information encoder, optimizing the Information Bottleneck objective~\citep{tishby2000information} by compressing input context while preserving predictive information (see Appendix~\ref{app:info_bottleneck}). To measure this, we analyze the tradeoff between remote ``prefill'' tokens (fewer tokens indicate greater compression) and accuracy (higher accuracy means better retention). Figure~\ref{fig:scaling-edge-model} shows that as $\locallm$ size increases, representations become more compressed and accurate, improving Information Bottleneck values. Larger $\locallm$ models trade local FLOPs for communication, with 7–8B models being 1.53× more token-efficient than 1B models. Additionally, the \qwen family follows a different tradeoff than \llama, yielding more compressed representations. This suggests that as small LMs improve, local-remote systems will become increasingly cost-efficient.


% We find that communication efficiency is correlated with size the of the $\locallm$ (see Figure~\ref{fig:lm-bottleneck}). Interestingly, averaged across datasets and model families, we find that the 7-8B parameter models are 1.53X more token efficient than their 1B parameter counter parts. This suggests an interesting trend: as the capabilities of local models increases, the cost of \system will further decrease.

\vspace{-0.75em}\paragraph{\textit{Is \system sensitive to different local/remote pairs?}} 
We ask whether the communication protocol in \system is invariant to changing the model types (\textit{i.e.} \llama vs \qwen locally and \llama vs \gpt remotely). Our results indicate that \system performs similarly with different local-remote LM combinations (see the Table~\ref{table:main-tradeoff}): varying the $\locallm$ from \qwen to \llamathreetwo, results in performances within $\pm$ .05 performance points (see Table~\ref{table:main-tradeoff}). Furthermore, we find that holding the $\locallm$ fixed as \llamathreetwo-3B and varying $\remotelm$ from \gpt to \llama-3.3-70B leads to similar overall performances within $\pm$ 0.07 points (see Table~\ref{tab:remote-model-variations} in Appendix).

\begin{figure*}[t]
    \centering
    \includegraphics[width=\linewidth]{figures/scaling-samples/plot.pdf}
    \caption{\textbf{Scaling parallel jobs on-device improves quality.} The x-axis represents tokens processed by the \textit{remote model}, and the y-axis shows macro-average accuracy across \longhealth and \qasper. The cloud model is \gpt. Each plot varies a different \system hyperparameter affecting parallelism, with annotated values. \textbf{(Left)} Varying the number of unique \textit{instructions}. \textbf{(Middle)} Varying the number of unique \textit{samples}. \textbf{(Right)} Varying the chunking granularity in code $\mathbf{f}$. See Section~\ref{sec:methods} for details.}
    \label{fig:scaling-samples}
\end{figure*}



\vspace{-0.75em}\paragraph{\textit{How have local / remote model capabilities changed over time, and what effects do they have on \system?}}
% would \system been possible in 2023?
% \cref{fig:model-longitudinal} x: date, y: accuracy, notes: staircase for baseline
In Table~\ref{tab:system-snapshot}, we provide a retrospective analysis demonstrating how the quality of \system would have changed with model releases over time. 
From 2023 to 2025, the average performance of \system with the best models available has improved from 0.26 to 0.66 (see Table~\ref{tab:system-snapshot} in Appendix). Interestingly, it was only in July 2024 --- with the release of \textsc{gpt4-turbo} and \llamathreeone-8B --- that \system could have come within 12\% of the best frontier model performance at the time (see Table~\ref{tab:system-snapshot} in Appendix). 



\vspace{-0.5em}\subsection{Scaling parallel workloads on-device}
\label{subsec:results-workloads}

% Scaling parallel local workloads
% three charts: (1) x-axis: tokens, y-axis: acuracy, labels=# of tasks per round (2) x-axis: tokens, y-axis: acuracy, labels=# of samples per task, (3)   x-axis: tokens, y-axis: acuracy, labels=chunks size

In \system, there are three levers for maximizing local compute resources through parallelized, batched processing: (1) number of tasks per round, (2) number of samples taken per task, and (3) number of chunks. We ablate each, showing their impact on performance. We find that (1) and (3) are more cost effective ways of increasing performance.

% \paragraph{\textit{At what point does edge compute become I/O bound?}}

\vspace{-0.75em}\paragraph{\textit{How does the number of tasks per round affect performance?}} Increasing tasks per round proxies task decomposition, with more sub-tasks enhancing decomposition. Raising tasks from 1 to 16 boosts performance by up to 14 points but doubles $\remotelm$ prefill costs. Optimal task count varies by query and model, but exceeding 16 reduces performance.


\vspace{-0.75em}\paragraph{\textit{How does scaling local samples affect performance?}} We explore whether increased sampling at an individual \{task, context\} level improves performance. Increased sampling enables us to better utilize the available compute resources while improving task-level accuracy~\citep{brown2024large}. Our results indicate that increasing the number samples from $1$ to $32$ can improve performance on average $7.4$ points, but comes at the cost of $5\times$ the $\remotelm$ prefill costs. This being said, increasing sampling beyond $16$ starts hurting task performance as the noise across samples is too large for the remote model to effectively distill the correct answer~\citep{kuratov2024babilong}.

\vspace{-0.75em}\paragraph{\textit{What effect does chunk size have on downstream performance?}} We test whether increasing local utilization by using more chunks per task improves performance. Our results indicate that increasing \# of chunks per task (by decreasing the number of ``pages'' per chunk from $100$ to $5$) leads to an $11.7$ point accuracy lift. However, this lift comes with a $2.41\times$ increase in $\remotelm$ prefill costs. 



\vspace{-0.5em}\subsection{Scaling sequential communication}
\label{subsec:results-communication}




\begin{wrapfigure}{r}{0.5\linewidth}
    \centering
    \includegraphics[width=\linewidth]{figures/scaling-rounds/plotv4.pdf}
    \caption{\textbf{Exploring the trade-off between cost and quality through multiple rounds.} The x-axis represents the remote model’s token cost, while the y-axis shows accuracy. Point labels indicate communication rounds. The purple dashed line marks \gpt’s performance as a benchmark.}
    \label{fig:scaling-rounds}
\end{wrapfigure}





% Figure~\ref{fig:scaling-rounds} x: tokens, y: accuracy, hue: different strategies 
Both the \naive and \system communication protocols feature \textit{sequential} communication: they allow for multiple rounds of exchange between the local and remote models. 



\vspace{-0.75em}\paragraph{\textit{Does performance improve as we increase the maximum number of rounds? At what cost?}} We vary the maximum communication rounds and find it is correlated with accuracy and cost (see \cref{fig:scaling-edge-model}). By simply increasing the maximum number of rounds in \naive from $1$ to $5$, we enable a $8.5$-point lift in average accuracy across the three tasks (with \llamathreetwo on-device).
However, this accuracy improvement comes at a cost: each additional round of communication increases the cost by \$$0.006$ per query while boosting accuracy by $4.2$ points.


% longer conversations also cost more: on \finance, each additional round of communication increases cost by \$$0.006$ per query and accuracy by $4.2$ accuracy points.



\vspace{-1em}\paragraph{\textit{How should we maintain context between \system rounds?}}
We experiment with two sequential protocol strategies: (1) simple retries and (2) scratchpad. 
% 
See Section~\ref{sec:methods} for details of these strategies.
% 
As shown in \Cref{fig:round-strategy}, both strategies show consistent increases in both accuracy and cost when increasing the maximum number of rounds, with the scratchpad strategy achieving a slightly better cost-accuracy tradeoff. 
% 
Notably, each additional round of communication with the scratchpad strategy leads to a larger improvement in accuracy ($6.1$ accuracy points) which are mostly offset by larger increases in cost ($8.6$ dollars). 


% Our experiments demonstrate for a given edge-remote pair, increasing the number of rounds of communication (\textit{i.e.} round trips between edge to remote model) improves accuracy by up to \yell{XX} points. However, this improved accuracy comes at the cost of greater utilization of remote resources (see Figure~\ref{fig:scaling-rounds}). In Section~\ref{sec:analysis}, we present a detailed slice analysis that identifies which problem types benefit most from sequential processing. 

% \paragraph{What kinds of tasks benefit from a sequential communication protocol?}
% \yell{Potentially make this part of the slice analysis.} 

\vspace{-0.5em}\subsection{Retrieval-Augmented Generation in the Context of Local-Remote Compute}
\label{subsec:rag}


We further investigate the interplay between local-remote compute paradigms (\textit{e.g.}, \system) and retrieval-augmented generation (RAG), analyzing their complementary strengths and trade-offs in data-intensive reasoning tasks. Through empirical evaluations, we examine these methods on financial document extraction (\finance) and long-document summarization (\textsc{Booookscore}). \textit{See Appendix ~\ref{app:rag} for a more detailed treatment.}

\subsubsection{Comparison of \system and RAG on \finance}

RAG performs well for financial document analysis, where relevant information is found in specific sections. In \Cref{fig:rag-comparison} (left), we compare \system, \naive, and RAG using BM25 and OpenAI's \texttt{text-embedding-3-small} embeddings~\citep{article,neelakantan2022text}. A chunk size of 1000 characters balances retrieval accuracy and efficiency (\Cref{fig:rag-comparison} (center)).

Adjusting the number of retrieved chunks allows RAG to optimize quality and cost. When BM25-based RAG retrieves 50+ chunks, it surpasses the performance of a fully context-fed remote model. However, RAG does not match the cost-effectiveness of \naive. When compared to \system, RAG using OpenAI embeddings achieves similar cost-quality trade-offs but may overlook nuanced financial signals across sections.

\subsubsection{Comparison of \system and RAG on Summarization Tasks}

Unlike \finance, RAG struggles with summarization due to its reliance on retrieval. Unlike the financial parsing tasks, summarization requires reasoning over information \textit{dispersed} across the document.  We evaluate \system, RAG (with BM25 andEmbedding retrievals), and a \gpt-only baseline on \textsc{BooookScore}~\citep{chang2023booookscore}, a long novel summarization dataset.

\paragraph{Evaluation}
\begin{itemize}
\item \textbf{Qualitative Analysis:} As shown in App. Table~\ref{tab:story-summaries}, \system generates summaries with richer entity mentions and more coherent narratives than RAG. It is also $9.3\times$ more token-efficient than \gpt-only (11,500 vs. 108,185 tokens).
\item \textbf{Quantitative Analysis:} Using \textsc{claude-3.5-sonnet} for evaluation, \system achieves near-parity with \gpt-only, while RAG-based methods underperform (Table~\ref{tab:summary_qualitative_scores} in Appendix).
\end{itemize}

These results highlight RAG's effectiveness in structured tasks like \finance but its shortcomings in long-form synthesis which requires reasoning over information dispersed across the document. 
%


\begin{wrapfigure}{r}{0.5\linewidth}
    \centering
    \includegraphics[width=\linewidth]{figures/scaling-rounds/plotv4.pdf}
    \caption{\textbf{Exploring the trade-off between cost and quality through multiple rounds.} The x-axis represents the remote model’s token cost, while the y-axis shows accuracy. Point labels indicate communication rounds. The purple dashed line marks \gpt’s performance as a benchmark.}
    \label{fig:scaling-rounds}
\end{wrapfigure}





% RAG \cite{lewis2020retrieval} reduces ``prefill'' tokens for the remote model by (1) chunking the document, (2) retrieving relevant chunks, and  (3) passing them in-context to \remotelm. In \cref{app:rag:finance}, we compare RAG to \system and \naive (all with \gpt as the their \remotelm) on the \finance dataset, whose problems rely on extracting financial information from 10K sections. We show that after a optimizing its hyperparameters, RAG can make similar cost-accuracy tradeoffs compared to \system, whereas \naive finds more cost-effective points compared to both methods under a similar budget of tokens sent remotely. However, some tasks cannot be reduced to retrieval. One such task is summarization which requires integrating information across chunks of the document. When comparing RAG to \system on the \textsc{BooookScore} dataset (\cref{app:rag:summary}, we find that RAG results in lower-quality summaries that comprise of collections of facts, whereas \system provides equivalent albeit more verbose summaries compared to \gpt-only.

% \section{Analysis}
% \label{sec:analysis}








