

\vspace{-1em}
\section{Results}
\label{sec:results}
\begin{figure}[t]
    \centering
    \includegraphics[width=1\linewidth]{figures/scaling-edge-model/size_comm_efficiency.pdf}
    \vspace{-1em}
    \caption{\textbf{Trade-offs in edge model performance, communication efficiency, and cost of sequential communication.} \textbf{(Left)} Accuracy vs. local model size, with the purple dashed line showing the \gpt model baseline. \textbf{(Right)} Communication efficiency of \system with different local LMs, where larger models (7–8B) are more token-efficient.}
    \label{fig:scaling-edge-model}
  \end{figure}



%

\begin{figure}[h!]
    \centering
    \includegraphics[width=0.5\linewidth]{figures/lm-bottleneck/plot.pdf}
    \caption{\textbf{Communication efficiency of \system when utilizing different local LM's.} x-axis is the total number of token sent up to the remote model. We observe that larger local models (7-8B parameters) are more token efficient than smaller LMs.
    }
    \label{fig:lm-bottleneck}
  \end{figure}




% 


\begin{figure}[h!]
    \centering
    \includegraphics[width=\linewidth]{figures/model-longitudinal/plot.pdf}
    \caption{\textbf{Edge-remote quality over time.} The x-axis represents the date of release.
}
    \label{fig:model-longitudinal}
  \end{figure}



% \section{Experiments: Planning outperforms Heuristics}
\label{sec:experiment}

We begin our empirical demonstrations by showcasing the effectiveness of our planning framework on both synthetic and real datasets. We focus on the simplest planning algorithm, 1-step lookaheads (Algorithm~\ref{alg:complete}), and show that even basic planning can hold great promise. 
We illustrate our framework using two uncertainty quantification modules---GPs and 
\ensembles/ \ensembleplus. 

Throughout this section, we focus on evaluating the mean squared error of 
a regression model $\model$,  and develop adaptive policies that minimize uncertainty on $g(f)$ defined in~\eqref{eqn:l2-g-f}.
When GPs provide a valid model of uncertainty, 
our experiments show that our planning framework significantly outperforms other baselines. 
We further demonstrate that our conceptual framework extends to deep learning-based uncertainty quantification methods such as  \ensembleplus while highlighting computational challenges that need to be resolved in order to scale our ideas. 
For simplicity, we assume a naive predictor, i.e., $\psi(\cdot) \equiv 0$. However, we emphasize that this problem is just as complex as if we were using a sophisticated model $\psi(.)$. The performance gap between the algorithms 
primarily depends
on the level  of uncertainty in our prior beliefs.

To evaluate the performance of our algorithm, we benchmark it against several baselines. 
%Active learning baselines use an acquisition function $\ac$ to select points that have the highest   function value: $X\opt_t \in \argmax_{X \in \xpoolj{t}} \ac({X})$ at every step $t$. These methods may also need an UQ module, which we simply use the same UQ module as in our algorithm, and it  outputs $V(X)$ that measures the the uncertainty of each point $X \in \xpoolj{t}$.
Our first set of baselines are from active learning~\citep{AggarwalKoGuHaPh14}:
\\ % \noindent\textbf{Active Learning Heuristics:} 
\textbf{(1)} 
\textsf{Uncertainty Sampling (Static):}  In this approach, we query the samples for which the model is least certain about. Specifically, we estimate the variance of the latent output $f(X)$ for each $X \in \xpool$ using the UQ module and select the top-$K$ points with the highest uncertainty. \\
\textbf{(2)} \textsf{Uncertainty Sampling (Sequential):} This is a greedy heuristic that sequentially selects the points with the highest uncertainty within a batch, while updating the posterior beliefs using pseudo labels from the current posterior state. Unlike \textsf{Uncertainty Sampling (Static)}, this method takes into account the information gained from each point within batch, and hence tries to diversify the selected points within a batch. 

 
We also compare our approach to the  \textbf{(3)} \textsf{Random Sampling}, which selects each batch uniformly at random from the pool. Additionally, we compare solving the planning problem using  \textsf{REINFORCE}-based policy gradients with   $\mathsf{Smoothed\text{-}Autodiff}$ policy gradients.\footnote{Our code repository is available at
  \url{https://github.com/namkoong-lab/adaptive-labeling}.}
%Detailed experimental setups are provided in Section \ref{sec:details-experiments}.

%We repeat all experiments with 10 random seeds.




\begin{figure}[t]
\centering
\begin{minipage}[b]{0.49\textwidth}
\centering
\includegraphics[width=\textwidth, height=5cm]{figures/original_scale/Var_of_l_2_loss.pdf}
\caption{(Synthetic data) Variance of mean squared loss evaluated through the posterior belief $\mu_t$ at each horizon $t$. This is the objective that policy gradient methods like \textsf{REINFORCE} and $\ouralgo$ optimizes. 1-step lookaheads are surprisingly effective even in long horizons.}
\label{fig:var-l2-sim}
\end{minipage}
\hfill
\begin{minipage}[b]{0.49\textwidth}
\centering \includegraphics[width=\textwidth, height=5cm]{figures/original_scale/Error_of_estimated_model_l_2_loss.pdf}
\caption{(Synthetic data) Error between MSE calculated based on collected data $\mc{D}^{0:T}$ vs. population oracle MSE over $\mc{D}_{\rm eval} \sim P_X$. Reducing uncertainty over posteriors directly leads to better OOD evaluations. 1-step lookaheads significantly outperform active learning heuristics in small horizons.}
\label{fig:mean-l2-sim}
\end{minipage}
%\caption{Simulated data for GPs}
%\label{fig:both_plots}
\end{figure}

\subsection{Planning with Gaussian processes}
\label{sec:experiment-plan-GP}
We now briefly describe the data generation process for the GP experiments,  deferring a more detailed discussion of the dataset generation to Section~\ref{sec:details-experiments}. 
We use both the synthetic data and the real data to test our methodology.
For the \emph{simulated data},  we construct a setting where the general population is distributed across \emph{51 non-overlapping clusters} while the initial labeled data $\dtrain$ just comes from one cluster. In contrast, both $\dpool \defeq (\xpool,\ypool),\deval \defeq (\xeval,\yeval)$ are generated   from all the clusters. 
We begin with a low-dimensional scenario, generating a one-dimensional regression setting using a GP. %Gaussian Process (GP).
Although the data-generating process is not known to the algorithms,  we assume that the GP hyperparameters are known to all the algorithms
to ensure fair comparisons. This can be viewed as a setting where our prior is well-specified, allowing us to isolate the effects
of different policy optimization approaches
 without any concerns about the misspecified priors. We select $10$ batches, each of size $K=5$ across $T = 10$ time horizons.

To examine the robustness of our method against the distributional assumptions made  in the simulated case, we then move to a real dataset where the correct prior is not known. We simulate selection bias from the eICU dataset~\citep{PollardJoRaCeMaBa18}, which contains real-world patient data with in-hospital mortality outcomes. 
We conduct a $k$-means clustering to generate 51 clusters and then select data from those clusters. We view this to be a credible replication of practice, as severe distribution shifts are common due to selection bias in clinical labels.  To convert the binary mortality labels into a regression setting, we train a  random forest classifier and fit a GP on predicted scores, which serves as the UQ module for all the algorithms. As before, the task is to select 10 batches, each consisting of 5 samples, across 10 time horizons.

 In Figures~\ref{fig:var-l2-sim} and~\ref{fig:mean-l2-sim}, we present results for the simulated data. 
Figure~\ref{fig:var-l2-sim} shows the variance of $\ell_2$ loss, and Figure~\ref{fig:mean-l2-sim} presents the error in the estimated $\ell_2$ loss using $\mu_t$ (relative to true $\ell_2$ loss, that is unknown to the algorithm). 
As we can see from these plots, our method one-step lookahead  gives substantial improvements  over active learning baselines and random sampling. In addition,
compared to the one-step lookahead planning approach using \textsf{REINFORCE}-based policy gradients, 
we observe that $\mathsf{Smoothed\text{-}Autodiff}$-based policy gradients provide significantly more robust performance over all horizons.

In Figures~\ref{fig:var-l2-real}~and~\ref{fig:mean-l2-real}, we observe similar findings on the eICU data. We see that planning policies (\textsf{REINFORCE} and $\mathsf{Smoothed\text{-}Autodiff}$) consistently outperform other heuristics by a large margin.  Active learning baselines perform poorly in these small-horizon batched problems and can sometimes be even worse than the random search baselines.  Overall, our results show the importance of careful planning in adaptive labeling for reliable model evaluation. 

We offer some intuition as to why one-step lookahead planning may outperform other heuristic algorithms. 
 First,  \textsf{Uncertainty sampling (Static)} while myopically selects the
 top-$K$ inputs with the highest uncertainty, it fails to consider 
the overlap in information content among the ``best” instances; see \citep{AggarwalKoGuHaPh14} for more details. 
In other words,  it might acquire points from the same region with high uncertainty while failing to induce diversity among the batch.
Although \textsf{Uncertainty Sampling (Sequential)} somewhat addresses the issue of information overlap, a significant drawback of 
this algorithm
is the disconnect between the objective we aim to optimize and the algorithm. For example, it might sample from a region with high uncertainty but very low density. 

\begin{figure}[t]
\centering
\begin{minipage}[b]{0.48\textwidth}
\centering
\includegraphics[width=\textwidth, height=5cm]{figures/original_scale/Var_of_l_2_loss_real.pdf}
\caption{(Real-world eICU data) Variance of mean squared loss evaluated through the posterior belief $\mu_t$ at each horizon $t$. Even 1-step lookaheads are extremely effective planners, and auto-differentiation-based pathwise policy gradients provide a reliable optimization algorithm based on low-variance gradient estimates.}
\label{fig:var-l2-real}
\end{minipage}
\hfill
\begin{minipage}[b]{0.48\textwidth}
\centering \includegraphics[width=\textwidth, height=5cm]{figures/original_scale/Error_of_estimated_model_l_2_loss_real.pdf}
\caption{(Real-world eICU data) Error between MSE calculated based on collected data $\mc{D}^{0:T}$ vs. population oracle MSE over $\mc{D}_{\rm eval} \sim P_X$. Reducing uncertainty over posteriors directly leads to better OOD evaluations. Our method significantly outperforms active learning-based heuristics, and random sampling.}
\label{fig:mean-l2-real}
\end{minipage}
%\caption{Real data for GPs}
\end{figure}
 
%\vspace{-1.5cm}
% \begin{wrapfigure}{r}{.32\columnwidth}
%   \vspace{-.5cm} 
%   \centering
% \includegraphics[scale=.29]{figures/Var of l2l_2 loss.pdf}
%   \vspace{-0.2cm}
%   \caption{Results of GP}
% \label{fig:var-l2-gp}
%   \vspace{-0.1cm}
% \end{wrapfigure}


% Attempts have been made  in the past to address these  drawbacks heuristically  (see \citep{AggarwalKoGuHaPh14}). We give a unified computational framework while approaching the problem in a more principled manner and solving it more optimally.




\subsection{Planning with  neural network-based uncertainty quantification methods ($\ensembleplus$)}


We now provide a proof-of-concept that shows the generalizability of our conceptual framework  to the deep learning-based UQ modules, specifically focusing on $\ensembleplus$ due to their previously observed superior performance~\citep{OsbandWenAsDwIbLuRo23}. Recall that implementing our framework with deep learning-based UQ modules  requires us to retrain the model across multiple possible random actions $\bm{a}(\theta)$ sampled from the current policy $\pi_\theta$.
This requires significant computational resources, in sharp contrast to the GPs where the posteriors are in closed form and can be readily updated and differentiated. 

Due to the computational constraints, we test $\ensembleplus$ on a toy setting to demonstrate the generalizability of our framework. We consider a setting where the general population consists of four clusters, while the initial labeled data only comes from one cluster. Again we generate data using GPs.  The task is to select a batch of 2 points in one horizon. We detail the $\ensembleplus$ architecture in Section \ref{sec:details-experiments}, and we assume prior uncertainty to be large (depends on the scaling of the prior generating functions). 
The results are summarized in the Table~\ref{tab:UQ_ensemble}.

% \begin{table}[H]
% \vspace{-10pt}
% \caption{Performance under \ensembleplus as UQ module}
%     \centering
%     \begin{tabular}{|m{3cm}|m{2.5cm}|m{2cm}|} 
%     \hline
%       Algorithm   & Variance of $\loss_2$ loss estimate & Error of $\loss_2$ loss estimate  \\ \hline Random Sampling 
%          & $1710.9 \pm 1352.1$ & $8.67\pm6.62$ 
%       \\ \hline \ouralgo & $1.30 \pm 0.68$ & $0.91\pm0.25$ \\ \hline
%     \end{tabular}
%     \label{tab:UQ_ensemble}
%     %\vspace{-10pt}
% \end{table}




\begin{table}[h]
\vspace{-10pt}
\caption{Performance under \ensembleplus as the UQ module}
\centering
\begin{tabular}{|l|l|l|}
\hline
Algorithm   & Variance of $\loss_2$ loss estimate & Error of $\loss_2$ loss estimate  \\
\hline
\textsf{Random sampling} & 7129.8 $\pm$ 1027.0 & 136.2 $\pm$ 8.28 \\ \hline
\textsf{Uncertainty sampling (Static)} & 10852 $\pm$ 0.0 & 162.156 $\pm$ 0.0 \\ \hline
\textsf{Uncertainty sampling (Sequential)} & 8585.5 $\pm$ 898.9 & 144 $\pm$ 6.93 \\ \hline
\textsf{REINFORCE} & 1697.1 $\pm$ 0.0 & 45.27 $\pm$ 0.0 \\ \hline
\ouralgo & 1697.1 $\pm$ 0.0 & 45.27 $\pm$ 0.0 \\ \hline
\end{tabular}
%\caption{Comparison of different algorithms based on variance   and   error in $\ell_2$ loss estimation with Ensemble $+$ as the UQ module. Our results demonstrate that {\ouralgo} and REINFORCE outperformthe other active learning based heuristics, confirming the benefits of our MDP formulation for the adaptive labeling problem, as also demonstrated in Section 4.\\
%\footnotesize{Experimental details: We use Gaussian Processes as our data generating process, GP parameters are the same as in Section D.3.  The task is to select a batch of 2 points along one horizon.The marginal distribution $p_X$ has 4 \textit{non-overlapping} clusters. Initial data comes from one cluster, while pool and evaluation points comes from all the clusters. We have $20$ initial labeled data points, $10$ pool points, and $252$ evaluation points.  Training procedures are similar to the one in Section D.3.} }
\label{tab:UQ_ensemble}
\end{table}



% We faced  issues in scaling up these experiments which will be our focus in the future. 





% \begin{itemize}
%     \item Posteriors should be consistent. Two dimensions: even with less training,  
%     \item the inference should be  fast enough
% \end{itemize}


% Potential research directions for uncertainty quantification

% In this section we consider a simple setting We consider a simpler setting and 


% For synthetic dataset generation, we use ...... For real datasets, we use ...... We compare our methodolgy to several baselines ()    This Section is structured as follows:
% \begin{itemize}
%     \item \textbf{GPs, square loss objective} (Section \ref{}): 
%     %the broad aim of the experiments  in this section is to isolate the performance of our methodology without any concerns for the inefficiencies induced due to a mis-specified prior or imperfect posterior inference. To accomplish this we generate synthetic datasets using GPs (detailed later). We use the well specified prior (GPs - with same hyperparameter setting) as our UQ module.   
%      As GPs provide differentaible posterior inference - any errors induced due to imperfect posterior updates are also isolated. We note that under this setting
%      \item In Section\ref{} we demonstrate why our methodology performs better than other baselines - by devising various synthetic experiments ()
%     \item  \textbf{UQ Benchmarking }(Section \ref{}): Before diving into the experiments using $\ensembleplus$ and ENNs,  we showcase our benchmarking experiments in Section \ref{}. We use real datasets We observe that ENNs perform better
%      \item \textbf{Ensemble $+$}, objective: recall, accuracy
%     \item \textbf{ENN}, objective: recall, accuracy
% \end{itemize}




% In Section {}, we test 
% \subsection{Experimental details}

% \begin{itemize}
%     \item UQ methodologies - GPs, ENNs
%     \item Objectives - Recall,  ATE
%     \item Datasets - ATE-synthetic datasets, Recall-synthetic, real datasets
%     \item Baselines - 
%     \begin{itemize}
%         \item Random sampling
%         \item Active learning - Uncertainty based sampling - In regression setting almost all of the 
%         \item Myopic greedy - Greedy Batch based sampling
%         \item Policy Gradient
%     \end{itemize}
    
% \end{itemize}

% \subsection{Experiments}
%     \begin{itemize}
%     \item GPs with square loss
%     \item Benchmarking ENN
%         \item ENNs with ATE
%         \item ENNs with Recall
%     \end{itemize}

% \subsection{Benefits over other algorithms - intuition and experiments}

%Active learning - Myopic greedy / Don't rely on the objective rather some entropy version.


%%% Local Variables:
%%% mode: latex
%%% TeX-master: "main"
%%% End:


Here, we analyze how the design of \system affects cost and quality. Our main takeaways are: 
\begin{itemize}
\item On average across three datasets, \system can recover $97.9\%$ of the performance of remote-only systems while spending $5.7\times$ less; 
\item We identify protocol hyper-parameters that let us flexibly trade-off cost and quality;
\item As local models grow stronger, \system becomes increasingly cost-effective.
  % \item \system would not have worked until mid-2024. With today's model, \system necessitates a local model that is $>3B$ parameters.
  % \item Communication efficiency is correlated with $\locallm$ capabilities: larger local models (\textit{i.e.} 8B parameters) are 1.53x more cost effective than their 1B parameter counter parts.
  % \item Scaling local workloads through aggressive parallelization improves performance, with chunking and sub-task generation being more cost effective than sampling.
  % \item Back-and-forth communication improves performance by up to \yell{XXX} points, but comes at a cost.
  % \item TODO
\end{itemize}

We structure our analysis around three core design choices:
%\vspace{-0.75em}
\begin{enumerate}
    \item \textbf{Model choice} \textit{How does the choice of local and remote model effect cost and quality?} We examine different model types and sizes for $\locallm$ and $\remotelm$ in \Cref{subsec:results-model}.\vspace{-0.5em}
    \item \textbf{Scaling parallel workloads on-device} \textit{How should we structure parallel workloads on the local device to maximize performance and minimize cost?} We highlight how scaling the local workloads can improve performance (\Cref{subsec:results-workloads}) and study the effects on cost.\vspace{-0.5em}
    \item \textbf{Sequential communication protocol} \textit{Can multiple rounds of communication improve quality? At what cost?} We explore this trade-off in \Cref{subsec:results-communication}.

\end{enumerate}

Our findings are detailed in Sections~\ref{subsec:results-model},~\ref{subsec:results-workloads}, and ~\ref{subsec:results-communication}. Finally, in \Cref{subsec:rag} we discuss the relationship between retrieval augemented generation and local-remote compute. 

\vspace{-0.5em}
\subsection{Experimental setup} 
\label{subsec:exp-setup}
\paragraph{Datasets and models} We evaluate \system on three benchmarks that are well suited for data-intensive reasoning: \finance, \longhealth, and \qasper. \finance tests financial document understanding with complex reasoning over reports. \longhealth focuses on tracking and interpreting longitudinal health records. \qasper assesses question answering over dense scientific papers. See Appendix~\ref{app:experiments-dataset} for details. We use two open-source model families (\llama, \qwen) as $\locallm$ and \gpt as $\remotelm$ (details in \Cref{app:experiments-models}).

\vspace{-0.5em}\subsection{Model choice}
\label{subsec:results-model}

This section explores the model requirements and generalization capabilities of \system, examining the local model sizes necessary for effective collaboration, the sensitivity of the communication protocol across different local-remote model pairings, and the longitudinal evolution of \system’ performance with advances in model capabilities over time.

\vspace{-0.5em}\paragraph{\textit{What size does $\locallm$ have to be in order to be effective in \system?}}
% \Cref{fig:scaling-edge-model} % x: parameters, y: accuracy, notes: vertical lines at different device types 

Our results demonstrate that \system starts being competitive with $\remotelm$-only baseline at the $3$B parameter model scale. When considering both the \qwen and \llama model families running locally, at $1$B scale, \system recovers 49.5\% of the \gpt-only baseline performance, 3B scale recovers 93.4\% and 8B recovers 97.9\% accuracy (see Table~\ref{table:main-tradeoff} for more details).

\vspace{-0.75em}\paragraph{\textit{How does the capacity of $\locallm$ affect the cost-accuracy tradeoff?}}
% x: parameters, y: accuracy, notes: vertical lines at different device types 
In our system, $\locallm$ implicitly acts as an information encoder, optimizing the Information Bottleneck objective~\citep{tishby2000information} by compressing input context while preserving predictive information (see Appendix~\ref{app:info_bottleneck}). To measure this, we analyze the tradeoff between remote ``prefill'' tokens (fewer tokens indicate greater compression) and accuracy (higher accuracy means better retention). Figure~\ref{fig:scaling-edge-model} shows that as $\locallm$ size increases, representations become more compressed and accurate, improving Information Bottleneck values. Larger $\locallm$ models trade local FLOPs for communication, with 7–8B models being 1.53× more token-efficient than 1B models. Additionally, the \qwen family follows a different tradeoff than \llama, yielding more compressed representations. This suggests that as small LMs improve, local-remote systems will become increasingly cost-efficient.


% We find that communication efficiency is correlated with size the of the $\locallm$ (see Figure~\ref{fig:lm-bottleneck}). Interestingly, averaged across datasets and model families, we find that the 7-8B parameter models are 1.53X more token efficient than their 1B parameter counter parts. This suggests an interesting trend: as the capabilities of local models increases, the cost of \system will further decrease.

\vspace{-0.75em}\paragraph{\textit{Is \system sensitive to different local/remote pairs?}} 
We ask whether the communication protocol in \system is invariant to changing the model types (\textit{i.e.} \llama vs \qwen locally and \llama vs \gpt remotely). Our results indicate that \system performs similarly with different local-remote LM combinations (see the Table~\ref{table:main-tradeoff}): varying the $\locallm$ from \qwen to \llamathreetwo, results in performances within $\pm$ .05 performance points (see Table~\ref{table:main-tradeoff}). Furthermore, we find that holding the $\locallm$ fixed as \llamathreetwo-3B and varying $\remotelm$ from \gpt to \llama-3.3-70B leads to similar overall performances within $\pm$ 0.07 points (see Table~\ref{tab:remote-model-variations} in Appendix).

\begin{figure*}[t]
    \centering
    \includegraphics[width=\linewidth]{figures/scaling-samples/plot.pdf}
    \caption{\textbf{Scaling parallel jobs on-device improves quality.} The x-axis represents tokens processed by the \textit{remote model}, and the y-axis shows macro-average accuracy across \longhealth and \qasper. The cloud model is \gpt. Each plot varies a different \system hyperparameter affecting parallelism, with annotated values. \textbf{(Left)} Varying the number of unique \textit{instructions}. \textbf{(Middle)} Varying the number of unique \textit{samples}. \textbf{(Right)} Varying the chunking granularity in code $\mathbf{f}$. See Section~\ref{sec:methods} for details.}
    \label{fig:scaling-samples}
\end{figure*}



\vspace{-0.75em}\paragraph{\textit{How have local / remote model capabilities changed over time, and what effects do they have on \system?}}
% would \system been possible in 2023?
% \cref{fig:model-longitudinal} x: date, y: accuracy, notes: staircase for baseline
In Table~\ref{tab:system-snapshot}, we provide a retrospective analysis demonstrating how the quality of \system would have changed with model releases over time. 
From 2023 to 2025, the average performance of \system with the best models available has improved from 0.26 to 0.66 (see Table~\ref{tab:system-snapshot} in Appendix). Interestingly, it was only in July 2024 --- with the release of \textsc{gpt4-turbo} and \llamathreeone-8B --- that \system could have come within 12\% of the best frontier model performance at the time (see Table~\ref{tab:system-snapshot} in Appendix). 



\vspace{-0.5em}\subsection{Scaling parallel workloads on-device}
\label{subsec:results-workloads}

% Scaling parallel local workloads
% three charts: (1) x-axis: tokens, y-axis: acuracy, labels=# of tasks per round (2) x-axis: tokens, y-axis: acuracy, labels=# of samples per task, (3)   x-axis: tokens, y-axis: acuracy, labels=chunks size

In \system, there are three levers for maximizing local compute resources through parallelized, batched processing: (1) number of tasks per round, (2) number of samples taken per task, and (3) number of chunks. We ablate each, showing their impact on performance. We find that (1) and (3) are more cost effective ways of increasing performance.

% \paragraph{\textit{At what point does edge compute become I/O bound?}}

\vspace{-0.75em}\paragraph{\textit{How does the number of tasks per round affect performance?}} Increasing tasks per round proxies task decomposition, with more sub-tasks enhancing decomposition. Raising tasks from 1 to 16 boosts performance by up to 14 points but doubles $\remotelm$ prefill costs. Optimal task count varies by query and model, but exceeding 16 reduces performance.


\vspace{-0.75em}\paragraph{\textit{How does scaling local samples affect performance?}} We explore whether increased sampling at an individual \{task, context\} level improves performance. Increased sampling enables us to better utilize the available compute resources while improving task-level accuracy~\citep{brown2024large}. Our results indicate that increasing the number samples from $1$ to $32$ can improve performance on average $7.4$ points, but comes at the cost of $5\times$ the $\remotelm$ prefill costs. This being said, increasing sampling beyond $16$ starts hurting task performance as the noise across samples is too large for the remote model to effectively distill the correct answer~\citep{kuratov2024babilong}.

\vspace{-0.75em}\paragraph{\textit{What effect does chunk size have on downstream performance?}} We test whether increasing local utilization by using more chunks per task improves performance. Our results indicate that increasing \# of chunks per task (by decreasing the number of ``pages'' per chunk from $100$ to $5$) leads to an $11.7$ point accuracy lift. However, this lift comes with a $2.41\times$ increase in $\remotelm$ prefill costs. 



\vspace{-0.5em}\subsection{Scaling sequential communication}
\label{subsec:results-communication}




\begin{wrapfigure}{r}{0.5\linewidth}
    \centering
    \includegraphics[width=\linewidth]{figures/scaling-rounds/plotv4.pdf}
    \caption{\textbf{Exploring the trade-off between cost and quality through multiple rounds.} The x-axis represents the remote model’s token cost, while the y-axis shows accuracy. Point labels indicate communication rounds. The purple dashed line marks \gpt’s performance as a benchmark.}
    \label{fig:scaling-rounds}
\end{wrapfigure}





% Figure~\ref{fig:scaling-rounds} x: tokens, y: accuracy, hue: different strategies 
Both the \naive and \system communication protocols feature \textit{sequential} communication: they allow for multiple rounds of exchange between the local and remote models. 



\vspace{-0.75em}\paragraph{\textit{Does performance improve as we increase the maximum number of rounds? At what cost?}} We vary the maximum communication rounds and find it is correlated with accuracy and cost (see \cref{fig:scaling-edge-model}). By simply increasing the maximum number of rounds in \naive from $1$ to $5$, we enable a $8.5$-point lift in average accuracy across the three tasks (with \llamathreetwo on-device).
However, this accuracy improvement comes at a cost: each additional round of communication increases the cost by \$$0.006$ per query while boosting accuracy by $4.2$ points.


% longer conversations also cost more: on \finance, each additional round of communication increases cost by \$$0.006$ per query and accuracy by $4.2$ accuracy points.



\vspace{-1em}\paragraph{\textit{How should we maintain context between \system rounds?}}
We experiment with two sequential protocol strategies: (1) simple retries and (2) scratchpad. 
% 
See Section~\ref{sec:methods} for details of these strategies.
% 
As shown in \Cref{fig:round-strategy}, both strategies show consistent increases in both accuracy and cost when increasing the maximum number of rounds, with the scratchpad strategy achieving a slightly better cost-accuracy tradeoff. 
% 
Notably, each additional round of communication with the scratchpad strategy leads to a larger improvement in accuracy ($6.1$ accuracy points) which are mostly offset by larger increases in cost ($8.6$ dollars). 


% Our experiments demonstrate for a given edge-remote pair, increasing the number of rounds of communication (\textit{i.e.} round trips between edge to remote model) improves accuracy by up to \yell{XX} points. However, this improved accuracy comes at the cost of greater utilization of remote resources (see Figure~\ref{fig:scaling-rounds}). In Section~\ref{sec:analysis}, we present a detailed slice analysis that identifies which problem types benefit most from sequential processing. 

% \paragraph{What kinds of tasks benefit from a sequential communication protocol?}
% \yell{Potentially make this part of the slice analysis.} 

\vspace{-0.5em}\subsection{Retrieval-Augmented Generation in the Context of Local-Remote Compute}
\label{subsec:rag}


We further investigate the interplay between local-remote compute paradigms (\textit{e.g.}, \system) and retrieval-augmented generation (RAG), analyzing their complementary strengths and trade-offs in data-intensive reasoning tasks. Through empirical evaluations, we examine these methods on financial document extraction (\finance) and long-document summarization (\textsc{Booookscore}). \textit{See Appendix ~\ref{app:rag} for a more detailed treatment.}

\subsubsection{Comparison of \system and RAG on \finance}

RAG performs well for financial document analysis, where relevant information is found in specific sections. In \Cref{fig:rag-comparison} (left), we compare \system, \naive, and RAG using BM25 and OpenAI's \texttt{text-embedding-3-small} embeddings~\citep{article,neelakantan2022text}. A chunk size of 1000 characters balances retrieval accuracy and efficiency (\Cref{fig:rag-comparison} (center)).

Adjusting the number of retrieved chunks allows RAG to optimize quality and cost. When BM25-based RAG retrieves 50+ chunks, it surpasses the performance of a fully context-fed remote model. However, RAG does not match the cost-effectiveness of \naive. When compared to \system, RAG using OpenAI embeddings achieves similar cost-quality trade-offs but may overlook nuanced financial signals across sections.

\subsubsection{Comparison of \system and RAG on Summarization Tasks}

Unlike \finance, RAG struggles with summarization due to its reliance on retrieval. Unlike the financial parsing tasks, summarization requires reasoning over information \textit{dispersed} across the document.  We evaluate \system, RAG (with BM25 andEmbedding retrievals), and a \gpt-only baseline on \textsc{BooookScore}~\citep{chang2023booookscore}, a long novel summarization dataset.

\paragraph{Evaluation}
\begin{itemize}
\item \textbf{Qualitative Analysis:} As shown in App. Table~\ref{tab:story-summaries}, \system generates summaries with richer entity mentions and more coherent narratives than RAG. It is also $9.3\times$ more token-efficient than \gpt-only (11,500 vs. 108,185 tokens).
\item \textbf{Quantitative Analysis:} Using \textsc{claude-3.5-sonnet} for evaluation, \system achieves near-parity with \gpt-only, while RAG-based methods underperform (Table~\ref{tab:summary_qualitative_scores} in Appendix).
\end{itemize}

These results highlight RAG's effectiveness in structured tasks like \finance but its shortcomings in long-form synthesis which requires reasoning over information dispersed across the document. 
%


\begin{wrapfigure}{r}{0.5\linewidth}
    \centering
    \includegraphics[width=\linewidth]{figures/scaling-rounds/plotv4.pdf}
    \caption{\textbf{Exploring the trade-off between cost and quality through multiple rounds.} The x-axis represents the remote model’s token cost, while the y-axis shows accuracy. Point labels indicate communication rounds. The purple dashed line marks \gpt’s performance as a benchmark.}
    \label{fig:scaling-rounds}
\end{wrapfigure}





% RAG \cite{lewis2020retrieval} reduces ``prefill'' tokens for the remote model by (1) chunking the document, (2) retrieving relevant chunks, and  (3) passing them in-context to \remotelm. In \cref{app:rag:finance}, we compare RAG to \system and \naive (all with \gpt as the their \remotelm) on the \finance dataset, whose problems rely on extracting financial information from 10K sections. We show that after a optimizing its hyperparameters, RAG can make similar cost-accuracy tradeoffs compared to \system, whereas \naive finds more cost-effective points compared to both methods under a similar budget of tokens sent remotely. However, some tasks cannot be reduced to retrieval. One such task is summarization which requires integrating information across chunks of the document. When comparing RAG to \system on the \textsc{BooookScore} dataset (\cref{app:rag:summary}, we find that RAG results in lower-quality summaries that comprise of collections of facts, whereas \system provides equivalent albeit more verbose summaries compared to \gpt-only.

% \section{Analysis}
% \label{sec:analysis}








