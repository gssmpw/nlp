\documentclass[10pt,journal,compsoc]{IEEEtran}
%
% If IEEEtran.cls has not been installed into the LaTeX system files,
% manually specify the path to it like:
% \documentclass[10pt,journal,compsoc]{../sty/IEEEtran}






% Some very useful LaTeX packages include:
% (uncomment the ones you want to load)


% *** MISC UTILITY PACKAGES ***
%
%\usepackage{ifpdf}
% Heiko Oberdiek's ifpdf.sty is very useful if you need conditional
% compilation based on whether the output is pdf or dvi.
% usage:
% \ifpdf
%   % pdf code
% \else
%   % dvi code
% \fi
% The latest version of ifpdf.sty can be obtained from:
% http://www.ctan.org/pkg/ifpdf
% Also, note that IEEEtran.cls V1.7 and later provides a builtin
% \ifCLASSINFOpdf conditional that works the same way.
% When switching from latex to pdflatex and vice-versa, the compiler may
% have to be run twice to clear warning/error messages.
\usepackage{colortbl}
\usepackage[utf8]{inputenc} % allow utf-8 input
\usepackage[T1]{fontenc}    % use 8-bit T1 fonts
% \usepackage{hyperref}       % hyperlinks
\usepackage{url}            % simple URL typesetting
\usepackage{booktabs}       % professional-quality tables
\usepackage{amsfonts}       % blackboard math symbols
\usepackage{nicefrac}       % compact symbols for 1/2, etc.
\usepackage{microtype}      % microtypography
%\usepackage{xcolor}         % colors
\usepackage[dvipsnames, svgnames, x11names,table]{xcolor}
\newcommand{\ricardo}[1]{\colorbox{ForestGreen}{\color{white}   \textsf{\textbf{Ricardo}}} \textcolor{ForestGreen}{#1}}
\definecolor{darkamber}{RGB}{195, 95, 0}
\usepackage[pagebackref=true,breaklinks=true,colorlinks,citecolor=ForestGreen]{hyperref}
\usepackage{graphicx}
\usepackage{tabularx}
\usepackage{amsmath}
% \usepackage{arydshln}
\usepackage{subfigure}
\usepackage{tcolorbox}
\usepackage{multirow}
\usepackage{pifont}
%\usepackage{minted}
\usepackage{enumitem}
\usepackage{amssymb}
\newcommand{\tabincell}[2]{\begin{tabular}{@{}#1@{}}#2\end{tabular}}
\usepackage{enumitem}
\usepackage[misc]{ifsym}
\usepackage{bm}
\usepackage{multirow}
\usepackage{bbding}
\usepackage{array}
\usepackage{hyperref}
\definecolor{LightGray}{gray}{0.95}
\usepackage{float}
\usepackage{wrapfig}
\usepackage{overpic}
\usepackage{stfloats}

% *** CITATION PACKAGES ***
%
\ifCLASSOPTIONcompsoc
  % IEEE Computer Society needs nocompress option
  % requires cite.sty v4.0 or later (November 2003)
  \usepackage[nocompress]{cite}
\else
  % normal IEEE
  \usepackage{cite}
\fi

% *** GRAPHICS RELATED PACKAGES ***
%
\ifCLASSINFOpdf
  % \usepackage[pdftex]{graphicx}
  % declare the path(s) where your graphic files are
  % \graphicspath{{../pdf/}{../jpeg/}}
  % and their extensions so you won't have to specify these with
  % every instance of \includegraphics
  % \DeclareGraphicsExtensions{.pdf,.jpeg,.png}
\else
  % or other class option (dvipsone, dvipdf, if not using dvips). graphicx
  % will default to the driver specified in the system graphics.cfg if no
  % driver is specified.
  % \usepackage[dvips]{graphicx}
  % declare the path(s) where your graphic files are
  % \graphicspath{{../eps/}}
  % and their extensions so you won't have to specify these with
  % every instance of \includegraphics
  % \DeclareGraphicsExtensions{.eps}
\fi



\def\eg{\emph{e.g.}}
\def\ie{\emph{i.e.}}
\def\etc{\emph{etc}}
\def\vs{\emph{v.s.}}
\def\wrt{\emph{w.r.t.}}
\def\etal{\emph{et al.}}
\def\j{\textcolor{red}{\{}}
\def\k{\textcolor{red}{\}}}
% \newtheorem{theorem}{Theorem}
% \newtheorem{corollary}{Corollary}
% \newtheorem{lemma}{Lemma}
% \newtheorem{definition}{Definition}


\def\eg{\emph{e.g.}}
%\def\etal{\emph{\etal}}
\def\vs{\emph{vs.}}
\def\ie{\emph{i.e.}}
\def\etc{\emph{etc}}
\def\wrt{\emph{w.r.t.}}
\def\etal{\emph{et al.}}
\DeclareMathOperator{\st}{s.t.}
\usepackage{booktabs}
\usepackage{multirow}
% \usepackage[table,xcdraw]{xcolor}
% \documentclass[xcolor=table]{beamer}
\usepackage{tabularx}
\usepackage{wasysym}
\usepackage{colortbl}
\usepackage{xcolor}
\usepackage{hhline}
\usepackage{graphicx}
\usepackage{xcolor}
\definecolor{deepgreen}{RGB}{0,100,0}
% \usepackage{subfig}

\usepackage{ragged2e}
\usepackage[edges]{forest}
\usetikzlibrary{shadows.blur}
\usetikzlibrary{shapes.geometric}
\definecolor{hiddendraw}{RGB}{10,128,122}
\definecolor{hidden-orange}{RGB}{224,224,224}
\usepackage{subcaption}
\usepackage{svg}

\usepackage{pifont}
\usepackage[perpage,symbol*]{footmisc}
\DefineFNsymbols{circled}{{\ding{192}}{\ding{193}}{\ding{194}}
{\ding{195}}{\ding{196}}{\ding{197}}{\ding{198}}{\ding{199}}{\ding{200}}{\ding{201}}}
\setfnsymbol{circled}

\newcommand\myfootnotestyle[1]{\ifcase#1 \or \ding{182}\or \ding{183}\or
\ding{184}\or \ding{185}\or \ding{186}\or \ding{187}%
\or \ding{188}\or \ding{189}\or \ding{190}\or \ding{191}\else *\fi\relax}

\hypersetup{hidelinks,colorlinks=true}

\newif\ifsubmit
\submitfalse

%\newcommand{\xinyun}[1]{\textcolor{red}{[todo: #1]}}
%\newcommand{\jiakai}[1]{\textcolor{blue}{ #1}}
\newcommand{\revision}[1]{\textcolor{blue}{#1}}
%\newcommand{\shiyu}[1]{\textcolor{blue}{ #1}}
%%%%%%%%%%%
%%%%% NEW MATH DEFINITIONS %%%%%

\usepackage{amsmath,amsfonts,bm}
\usepackage{derivative}
% Mark sections of captions for referring to divisions of figures
\newcommand{\figleft}{{\em (Left)}}
\newcommand{\figcenter}{{\em (Center)}}
\newcommand{\figright}{{\em (Right)}}
\newcommand{\figtop}{{\em (Top)}}
\newcommand{\figbottom}{{\em (Bottom)}}
\newcommand{\captiona}{{\em (a)}}
\newcommand{\captionb}{{\em (b)}}
\newcommand{\captionc}{{\em (c)}}
\newcommand{\captiond}{{\em (d)}}

% Highlight a newly defined term
\newcommand{\newterm}[1]{{\bf #1}}

% Derivative d 
\newcommand{\deriv}{{\mathrm{d}}}

% Figure reference, lower-case.
\def\figref#1{figure~\ref{#1}}
% Figure reference, capital. For start of sentence
\def\Figref#1{Figure~\ref{#1}}
\def\twofigref#1#2{figures \ref{#1} and \ref{#2}}
\def\quadfigref#1#2#3#4{figures \ref{#1}, \ref{#2}, \ref{#3} and \ref{#4}}
% Section reference, lower-case.
\def\secref#1{section~\ref{#1}}
% Section reference, capital.
\def\Secref#1{Section~\ref{#1}}
% Reference to two sections.
\def\twosecrefs#1#2{sections \ref{#1} and \ref{#2}}
% Reference to three sections.
\def\secrefs#1#2#3{sections \ref{#1}, \ref{#2} and \ref{#3}}
% Reference to an equation, lower-case.
\def\eqref#1{equation~\ref{#1}}
% Reference to an equation, upper case
\def\Eqref#1{Equation~\ref{#1}}
% A raw reference to an equation---avoid using if possible
\def\plaineqref#1{\ref{#1}}
% Reference to a chapter, lower-case.
\def\chapref#1{chapter~\ref{#1}}
% Reference to an equation, upper case.
\def\Chapref#1{Chapter~\ref{#1}}
% Reference to a range of chapters
\def\rangechapref#1#2{chapters\ref{#1}--\ref{#2}}
% Reference to an algorithm, lower-case.
\def\algref#1{algorithm~\ref{#1}}
% Reference to an algorithm, upper case.
\def\Algref#1{Algorithm~\ref{#1}}
\def\twoalgref#1#2{algorithms \ref{#1} and \ref{#2}}
\def\Twoalgref#1#2{Algorithms \ref{#1} and \ref{#2}}
% Reference to a part, lower case
\def\partref#1{part~\ref{#1}}
% Reference to a part, upper case
\def\Partref#1{Part~\ref{#1}}
\def\twopartref#1#2{parts \ref{#1} and \ref{#2}}

\def\ceil#1{\lceil #1 \rceil}
\def\floor#1{\lfloor #1 \rfloor}
\def\1{\bm{1}}
\newcommand{\train}{\mathcal{D}}
\newcommand{\valid}{\mathcal{D_{\mathrm{valid}}}}
\newcommand{\test}{\mathcal{D_{\mathrm{test}}}}

\def\eps{{\epsilon}}


% Random variables
\def\reta{{\textnormal{$\eta$}}}
\def\ra{{\textnormal{a}}}
\def\rb{{\textnormal{b}}}
\def\rc{{\textnormal{c}}}
\def\rd{{\textnormal{d}}}
\def\re{{\textnormal{e}}}
\def\rf{{\textnormal{f}}}
\def\rg{{\textnormal{g}}}
\def\rh{{\textnormal{h}}}
\def\ri{{\textnormal{i}}}
\def\rj{{\textnormal{j}}}
\def\rk{{\textnormal{k}}}
\def\rl{{\textnormal{l}}}
% rm is already a command, just don't name any random variables m
\def\rn{{\textnormal{n}}}
\def\ro{{\textnormal{o}}}
\def\rp{{\textnormal{p}}}
\def\rq{{\textnormal{q}}}
\def\rr{{\textnormal{r}}}
\def\rs{{\textnormal{s}}}
\def\rt{{\textnormal{t}}}
\def\ru{{\textnormal{u}}}
\def\rv{{\textnormal{v}}}
\def\rw{{\textnormal{w}}}
\def\rx{{\textnormal{x}}}
\def\ry{{\textnormal{y}}}
\def\rz{{\textnormal{z}}}

% Random vectors
\def\rvepsilon{{\mathbf{\epsilon}}}
\def\rvphi{{\mathbf{\phi}}}
\def\rvtheta{{\mathbf{\theta}}}
\def\rva{{\mathbf{a}}}
\def\rvb{{\mathbf{b}}}
\def\rvc{{\mathbf{c}}}
\def\rvd{{\mathbf{d}}}
\def\rve{{\mathbf{e}}}
\def\rvf{{\mathbf{f}}}
\def\rvg{{\mathbf{g}}}
\def\rvh{{\mathbf{h}}}
\def\rvu{{\mathbf{i}}}
\def\rvj{{\mathbf{j}}}
\def\rvk{{\mathbf{k}}}
\def\rvl{{\mathbf{l}}}
\def\rvm{{\mathbf{m}}}
\def\rvn{{\mathbf{n}}}
\def\rvo{{\mathbf{o}}}
\def\rvp{{\mathbf{p}}}
\def\rvq{{\mathbf{q}}}
\def\rvr{{\mathbf{r}}}
\def\rvs{{\mathbf{s}}}
\def\rvt{{\mathbf{t}}}
\def\rvu{{\mathbf{u}}}
\def\rvv{{\mathbf{v}}}
\def\rvw{{\mathbf{w}}}
\def\rvx{{\mathbf{x}}}
\def\rvy{{\mathbf{y}}}
\def\rvz{{\mathbf{z}}}

% Elements of random vectors
\def\erva{{\textnormal{a}}}
\def\ervb{{\textnormal{b}}}
\def\ervc{{\textnormal{c}}}
\def\ervd{{\textnormal{d}}}
\def\erve{{\textnormal{e}}}
\def\ervf{{\textnormal{f}}}
\def\ervg{{\textnormal{g}}}
\def\ervh{{\textnormal{h}}}
\def\ervi{{\textnormal{i}}}
\def\ervj{{\textnormal{j}}}
\def\ervk{{\textnormal{k}}}
\def\ervl{{\textnormal{l}}}
\def\ervm{{\textnormal{m}}}
\def\ervn{{\textnormal{n}}}
\def\ervo{{\textnormal{o}}}
\def\ervp{{\textnormal{p}}}
\def\ervq{{\textnormal{q}}}
\def\ervr{{\textnormal{r}}}
\def\ervs{{\textnormal{s}}}
\def\ervt{{\textnormal{t}}}
\def\ervu{{\textnormal{u}}}
\def\ervv{{\textnormal{v}}}
\def\ervw{{\textnormal{w}}}
\def\ervx{{\textnormal{x}}}
\def\ervy{{\textnormal{y}}}
\def\ervz{{\textnormal{z}}}

% Random matrices
\def\rmA{{\mathbf{A}}}
\def\rmB{{\mathbf{B}}}
\def\rmC{{\mathbf{C}}}
\def\rmD{{\mathbf{D}}}
\def\rmE{{\mathbf{E}}}
\def\rmF{{\mathbf{F}}}
\def\rmG{{\mathbf{G}}}
\def\rmH{{\mathbf{H}}}
\def\rmI{{\mathbf{I}}}
\def\rmJ{{\mathbf{J}}}
\def\rmK{{\mathbf{K}}}
\def\rmL{{\mathbf{L}}}
\def\rmM{{\mathbf{M}}}
\def\rmN{{\mathbf{N}}}
\def\rmO{{\mathbf{O}}}
\def\rmP{{\mathbf{P}}}
\def\rmQ{{\mathbf{Q}}}
\def\rmR{{\mathbf{R}}}
\def\rmS{{\mathbf{S}}}
\def\rmT{{\mathbf{T}}}
\def\rmU{{\mathbf{U}}}
\def\rmV{{\mathbf{V}}}
\def\rmW{{\mathbf{W}}}
\def\rmX{{\mathbf{X}}}
\def\rmY{{\mathbf{Y}}}
\def\rmZ{{\mathbf{Z}}}

% Elements of random matrices
\def\ermA{{\textnormal{A}}}
\def\ermB{{\textnormal{B}}}
\def\ermC{{\textnormal{C}}}
\def\ermD{{\textnormal{D}}}
\def\ermE{{\textnormal{E}}}
\def\ermF{{\textnormal{F}}}
\def\ermG{{\textnormal{G}}}
\def\ermH{{\textnormal{H}}}
\def\ermI{{\textnormal{I}}}
\def\ermJ{{\textnormal{J}}}
\def\ermK{{\textnormal{K}}}
\def\ermL{{\textnormal{L}}}
\def\ermM{{\textnormal{M}}}
\def\ermN{{\textnormal{N}}}
\def\ermO{{\textnormal{O}}}
\def\ermP{{\textnormal{P}}}
\def\ermQ{{\textnormal{Q}}}
\def\ermR{{\textnormal{R}}}
\def\ermS{{\textnormal{S}}}
\def\ermT{{\textnormal{T}}}
\def\ermU{{\textnormal{U}}}
\def\ermV{{\textnormal{V}}}
\def\ermW{{\textnormal{W}}}
\def\ermX{{\textnormal{X}}}
\def\ermY{{\textnormal{Y}}}
\def\ermZ{{\textnormal{Z}}}

% Vectors
\def\vzero{{\bm{0}}}
\def\vone{{\bm{1}}}
\def\vmu{{\bm{\mu}}}
\def\vtheta{{\bm{\theta}}}
\def\vphi{{\bm{\phi}}}
\def\va{{\bm{a}}}
\def\vb{{\bm{b}}}
\def\vc{{\bm{c}}}
\def\vd{{\bm{d}}}
\def\ve{{\bm{e}}}
\def\vf{{\bm{f}}}
\def\vg{{\bm{g}}}
\def\vh{{\bm{h}}}
\def\vi{{\bm{i}}}
\def\vj{{\bm{j}}}
\def\vk{{\bm{k}}}
\def\vl{{\bm{l}}}
\def\vm{{\bm{m}}}
\def\vn{{\bm{n}}}
\def\vo{{\bm{o}}}
\def\vp{{\bm{p}}}
\def\vq{{\bm{q}}}
\def\vr{{\bm{r}}}
\def\vs{{\bm{s}}}
\def\vt{{\bm{t}}}
\def\vu{{\bm{u}}}
\def\vv{{\bm{v}}}
\def\vw{{\bm{w}}}
\def\vx{{\bm{x}}}
\def\vy{{\bm{y}}}
\def\vz{{\bm{z}}}

% Elements of vectors
\def\evalpha{{\alpha}}
\def\evbeta{{\beta}}
\def\evepsilon{{\epsilon}}
\def\evlambda{{\lambda}}
\def\evomega{{\omega}}
\def\evmu{{\mu}}
\def\evpsi{{\psi}}
\def\evsigma{{\sigma}}
\def\evtheta{{\theta}}
\def\eva{{a}}
\def\evb{{b}}
\def\evc{{c}}
\def\evd{{d}}
\def\eve{{e}}
\def\evf{{f}}
\def\evg{{g}}
\def\evh{{h}}
\def\evi{{i}}
\def\evj{{j}}
\def\evk{{k}}
\def\evl{{l}}
\def\evm{{m}}
\def\evn{{n}}
\def\evo{{o}}
\def\evp{{p}}
\def\evq{{q}}
\def\evr{{r}}
\def\evs{{s}}
\def\evt{{t}}
\def\evu{{u}}
\def\evv{{v}}
\def\evw{{w}}
\def\evx{{x}}
\def\evy{{y}}
\def\evz{{z}}

% Matrix
\def\mA{{\bm{A}}}
\def\mB{{\bm{B}}}
\def\mC{{\bm{C}}}
\def\mD{{\bm{D}}}
\def\mE{{\bm{E}}}
\def\mF{{\bm{F}}}
\def\mG{{\bm{G}}}
\def\mH{{\bm{H}}}
\def\mI{{\bm{I}}}
\def\mJ{{\bm{J}}}
\def\mK{{\bm{K}}}
\def\mL{{\bm{L}}}
\def\mM{{\bm{M}}}
\def\mN{{\bm{N}}}
\def\mO{{\bm{O}}}
\def\mP{{\bm{P}}}
\def\mQ{{\bm{Q}}}
\def\mR{{\bm{R}}}
\def\mS{{\bm{S}}}
\def\mT{{\bm{T}}}
\def\mU{{\bm{U}}}
\def\mV{{\bm{V}}}
\def\mW{{\bm{W}}}
\def\mX{{\bm{X}}}
\def\mY{{\bm{Y}}}
\def\mZ{{\bm{Z}}}
\def\mBeta{{\bm{\beta}}}
\def\mPhi{{\bm{\Phi}}}
\def\mLambda{{\bm{\Lambda}}}
\def\mSigma{{\bm{\Sigma}}}

% Tensor
\DeclareMathAlphabet{\mathsfit}{\encodingdefault}{\sfdefault}{m}{sl}
\SetMathAlphabet{\mathsfit}{bold}{\encodingdefault}{\sfdefault}{bx}{n}
\newcommand{\tens}[1]{\bm{\mathsfit{#1}}}
\def\tA{{\tens{A}}}
\def\tB{{\tens{B}}}
\def\tC{{\tens{C}}}
\def\tD{{\tens{D}}}
\def\tE{{\tens{E}}}
\def\tF{{\tens{F}}}
\def\tG{{\tens{G}}}
\def\tH{{\tens{H}}}
\def\tI{{\tens{I}}}
\def\tJ{{\tens{J}}}
\def\tK{{\tens{K}}}
\def\tL{{\tens{L}}}
\def\tM{{\tens{M}}}
\def\tN{{\tens{N}}}
\def\tO{{\tens{O}}}
\def\tP{{\tens{P}}}
\def\tQ{{\tens{Q}}}
\def\tR{{\tens{R}}}
\def\tS{{\tens{S}}}
\def\tT{{\tens{T}}}
\def\tU{{\tens{U}}}
\def\tV{{\tens{V}}}
\def\tW{{\tens{W}}}
\def\tX{{\tens{X}}}
\def\tY{{\tens{Y}}}
\def\tZ{{\tens{Z}}}


% Graph
\def\gA{{\mathcal{A}}}
\def\gB{{\mathcal{B}}}
\def\gC{{\mathcal{C}}}
\def\gD{{\mathcal{D}}}
\def\gE{{\mathcal{E}}}
\def\gF{{\mathcal{F}}}
\def\gG{{\mathcal{G}}}
\def\gH{{\mathcal{H}}}
\def\gI{{\mathcal{I}}}
\def\gJ{{\mathcal{J}}}
\def\gK{{\mathcal{K}}}
\def\gL{{\mathcal{L}}}
\def\gM{{\mathcal{M}}}
\def\gN{{\mathcal{N}}}
\def\gO{{\mathcal{O}}}
\def\gP{{\mathcal{P}}}
\def\gQ{{\mathcal{Q}}}
\def\gR{{\mathcal{R}}}
\def\gS{{\mathcal{S}}}
\def\gT{{\mathcal{T}}}
\def\gU{{\mathcal{U}}}
\def\gV{{\mathcal{V}}}
\def\gW{{\mathcal{W}}}
\def\gX{{\mathcal{X}}}
\def\gY{{\mathcal{Y}}}
\def\gZ{{\mathcal{Z}}}

% Sets
\def\sA{{\mathbb{A}}}
\def\sB{{\mathbb{B}}}
\def\sC{{\mathbb{C}}}
\def\sD{{\mathbb{D}}}
% Don't use a set called E, because this would be the same as our symbol
% for expectation.
\def\sF{{\mathbb{F}}}
\def\sG{{\mathbb{G}}}
\def\sH{{\mathbb{H}}}
\def\sI{{\mathbb{I}}}
\def\sJ{{\mathbb{J}}}
\def\sK{{\mathbb{K}}}
\def\sL{{\mathbb{L}}}
\def\sM{{\mathbb{M}}}
\def\sN{{\mathbb{N}}}
\def\sO{{\mathbb{O}}}
\def\sP{{\mathbb{P}}}
\def\sQ{{\mathbb{Q}}}
\def\sR{{\mathbb{R}}}
\def\sS{{\mathbb{S}}}
\def\sT{{\mathbb{T}}}
\def\sU{{\mathbb{U}}}
\def\sV{{\mathbb{V}}}
\def\sW{{\mathbb{W}}}
\def\sX{{\mathbb{X}}}
\def\sY{{\mathbb{Y}}}
\def\sZ{{\mathbb{Z}}}

% Entries of a matrix
\def\emLambda{{\Lambda}}
\def\emA{{A}}
\def\emB{{B}}
\def\emC{{C}}
\def\emD{{D}}
\def\emE{{E}}
\def\emF{{F}}
\def\emG{{G}}
\def\emH{{H}}
\def\emI{{I}}
\def\emJ{{J}}
\def\emK{{K}}
\def\emL{{L}}
\def\emM{{M}}
\def\emN{{N}}
\def\emO{{O}}
\def\emP{{P}}
\def\emQ{{Q}}
\def\emR{{R}}
\def\emS{{S}}
\def\emT{{T}}
\def\emU{{U}}
\def\emV{{V}}
\def\emW{{W}}
\def\emX{{X}}
\def\emY{{Y}}
\def\emZ{{Z}}
\def\emSigma{{\Sigma}}

% entries of a tensor
% Same font as tensor, without \bm wrapper
\newcommand{\etens}[1]{\mathsfit{#1}}
\def\etLambda{{\etens{\Lambda}}}
\def\etA{{\etens{A}}}
\def\etB{{\etens{B}}}
\def\etC{{\etens{C}}}
\def\etD{{\etens{D}}}
\def\etE{{\etens{E}}}
\def\etF{{\etens{F}}}
\def\etG{{\etens{G}}}
\def\etH{{\etens{H}}}
\def\etI{{\etens{I}}}
\def\etJ{{\etens{J}}}
\def\etK{{\etens{K}}}
\def\etL{{\etens{L}}}
\def\etM{{\etens{M}}}
\def\etN{{\etens{N}}}
\def\etO{{\etens{O}}}
\def\etP{{\etens{P}}}
\def\etQ{{\etens{Q}}}
\def\etR{{\etens{R}}}
\def\etS{{\etens{S}}}
\def\etT{{\etens{T}}}
\def\etU{{\etens{U}}}
\def\etV{{\etens{V}}}
\def\etW{{\etens{W}}}
\def\etX{{\etens{X}}}
\def\etY{{\etens{Y}}}
\def\etZ{{\etens{Z}}}

% The true underlying data generating distribution
\newcommand{\pdata}{p_{\rm{data}}}
\newcommand{\ptarget}{p_{\rm{target}}}
\newcommand{\pprior}{p_{\rm{prior}}}
\newcommand{\pbase}{p_{\rm{base}}}
\newcommand{\pref}{p_{\rm{ref}}}

% The empirical distribution defined by the training set
\newcommand{\ptrain}{\hat{p}_{\rm{data}}}
\newcommand{\Ptrain}{\hat{P}_{\rm{data}}}
% The model distribution
\newcommand{\pmodel}{p_{\rm{model}}}
\newcommand{\Pmodel}{P_{\rm{model}}}
\newcommand{\ptildemodel}{\tilde{p}_{\rm{model}}}
% Stochastic autoencoder distributions
\newcommand{\pencode}{p_{\rm{encoder}}}
\newcommand{\pdecode}{p_{\rm{decoder}}}
\newcommand{\precons}{p_{\rm{reconstruct}}}

\newcommand{\laplace}{\mathrm{Laplace}} % Laplace distribution

\newcommand{\E}{\mathbb{E}}
\newcommand{\Ls}{\mathcal{L}}
\newcommand{\R}{\mathbb{R}}
\newcommand{\emp}{\tilde{p}}
\newcommand{\lr}{\alpha}
\newcommand{\reg}{\lambda}
\newcommand{\rect}{\mathrm{rectifier}}
\newcommand{\softmax}{\mathrm{softmax}}
\newcommand{\sigmoid}{\sigma}
\newcommand{\softplus}{\zeta}
\newcommand{\KL}{D_{\mathrm{KL}}}
\newcommand{\Var}{\mathrm{Var}}
\newcommand{\standarderror}{\mathrm{SE}}
\newcommand{\Cov}{\mathrm{Cov}}
% Wolfram Mathworld says $L^2$ is for function spaces and $\ell^2$ is for vectors
% But then they seem to use $L^2$ for vectors throughout the site, and so does
% wikipedia.
\newcommand{\normlzero}{L^0}
\newcommand{\normlone}{L^1}
\newcommand{\normltwo}{L^2}
\newcommand{\normlp}{L^p}
\newcommand{\normmax}{L^\infty}

\newcommand{\parents}{Pa} % See usage in notation.tex. Chosen to match Daphne's book.

\DeclareMathOperator*{\argmax}{arg\,max}
\DeclareMathOperator*{\argmin}{arg\,min}

\DeclareMathOperator{\sign}{sign}
\DeclareMathOperator{\Tr}{Tr}
\let\ab\allowbreak

% *** CITATION PACKAGES ***
%
% \ifCLASSOPTIONcompsoc
%   % IEEE Computer Society needs nocompress option
%   % requires cite.sty v4.0 or later (November 2003)
%   \usepackage[nocompress]{cite}
% \else
%   % normal IEEE
%   \usepackage{cite}
% \fi

\usepackage{booktabs}



% *** GRAPHICS RELATED PACKAGES ***
%
% \ifCLASSINFOpdf
%   % \usepackage[pdftex]{graphicx}
%   % declare the path(s) where your graphic files are
%   % \graphicspath{{../pdf/}{../jpeg/}}
%   % and their extensions so you won't have to specify these with
%   % every instance of \includegraphics
%   % \DeclareGraphicsExtensions{.pdf,.jpeg,.png}
% \else
%   % or other class option (dvipsone, dvipdf, if not using dvips). graphicx
%   % will default to the driver specified in the system graphics.cfg if no
%   % driver is specified.
%   % \usepackage[dvips]{graphicx}
%   % declare the path(s) where your graphic files are
%   % \graphicspath{{../eps/}}
%   % and their extensions so you won't have to specify these with
%   % every instance of \includegraphics
%   % \DeclareGraphicsExtensions{.eps}
% \fi
% graphicx was written by David Carlisle and Sebastian Rahtz. It is
% required if you want graphics, photos, etc. graphicx.sty is already
% installed on most LaTeX systems. The latest version and documentation
% can be obtained at: 
% http://www.ctan.org/pkg/graphicx
% Another good source of documentation is "Using Imported Graphics in
% LaTeX2e" by Keith Reckdahl which can be found at:
% http://www.ctan.org/pkg/epslatex
%
% latex, and pdflatex in dvi mode, support graphics in encapsulated
% postscript (.eps) format. pdflatex in pdf mode supports graphics
% in .pdf, .jpeg, .png and .mps (metapost) formats. Users should ensure
% that all non-photo figures use a vector format (.eps, .pdf, .mps) and
% not a bitmapped formats (.jpeg, .png). The IEEE frowns on bitmapped formats
% which can result in "jaggedy"/blurry rendering of lines and letters as
% well as large increases in file sizes.
%
% You can find documentation about the pdfTeX application at:
% http://www.tug.org/applications/pdftex





% *** MATH PACKAGES ***
%
\usepackage{amsmath}
% A popular package from the American Mathematical Society that provides
% many useful and powerful commands for dealing with mathematics.
%
% Note that the amsmath package sets \interdisplaylinepenalty to 10000
% thus preventing page breaks from occurring within multiline equations. Use:
%\interdisplaylinepenalty=2500
% after loading amsmath to restore such page breaks as IEEEtran.cls normally
% does. amsmath.sty is already installed on most LaTeX systems. The latest
% version and documentation can be obtained at:
% http://www.ctan.org/pkg/amsmath





% *** SPECIALIZED LIST PACKAGES ***
%
%\usepackage{algorithmic}
% algorithmic.sty was written by Peter Williams and Rogerio Brito.
% This package provides an algorithmic environment fo describing algorithms.
% You can use the algorithmic environment in-text or within a figure
% environment to provide for a floating algorithm. Do NOT use the algorithm
% floating environment provided by algorithm.sty (by the same authors) or
% algorithm2e.sty (by Christophe Fiorio) as the IEEE does not use dedicated
% algorithm float types and packages that provide these will not provide
% correct IEEE style captions. The latest version and documentation of
% algorithmic.sty can be obtained at:
% http://www.ctan.org/pkg/algorithms
% Also of interest may be the (relatively newer and more customizable)
% algorithmicx.sty package by Szasz Janos:
% http://www.ctan.org/pkg/algorithmicx




% *** ALIGNMENT PACKAGES ***
%
%\usepackage{array}
% Frank Mittelbach's and David Carlisle's array.sty patches and improves
% the standard LaTeX2e array and tabular environments to provide better
% appearance and additional user controls. As the default LaTeX2e table
% generation code is lacking to the point of almost being broken with
% respect to the quality of the end results, all users are strongly
% advised to use an enhanced (at the very least that provided by array.sty)
% set of table tools. array.sty is already installed on most systems. The
% latest version and documentation can be obtained at:
% http://www.ctan.org/pkg/array


% IEEEtran contains the IEEEeqnarray family of commands that can be used to
% generate multiline equations as well as matrices, tables, etc., of high
% quality.




% *** SUBFIGURE PACKAGES ***
%\ifCLASSOPTIONcompsoc
%  \usepackage[caption=false,font=footnotesize,labelfont=sf,textfont=sf]{subfig}
%\else
%  \usepackage[caption=false,font=footnotesize]{subfig}
%\fi
% subfig.sty, written by Steven Douglas Cochran, is the modern replacement
% for subfigure.sty, the latter of which is no longer maintained and is
% incompatible with some LaTeX packages including fixltx2e. However,
% subfig.sty requires and automatically loads Axel Sommerfeldt's caption.sty
% which will override IEEEtran.cls' handling of captions and this will result
% in non-IEEE style figure/table captions. To prevent this problem, be sure
% and invoke subfig.sty's "caption=false" package option (available since
% subfig.sty version 1.3, 2005/06/28) as this is will preserve IEEEtran.cls
% handling of captions.
% Note that the Computer Society format requires a sans serif font rather
% than the serif font used in traditional IEEE formatting and thus the need
% to invoke different subfig.sty package options depending on whether
% compsoc mode has been enabled.
%
% The latest version and documentation of subfig.sty can be obtained at:
% http://www.ctan.org/pkg/subfig





%\usepackage{stfloats}
% stfloats.sty was written by Sigitas Tolusis. This package gives LaTeX2e
% the ability to do double column floats at the bottom of the page as well
% as the top. (e.g., "\begin{figure*}[!b]" is not normally possible in
% LaTeX2e). It also provides a command:
%\fnbelowfloat
% to enable the placement of footnotes below bottom floats (the standard
% LaTeX2e kernel puts them above bottom floats). This is an invasive package
% which rewrites many portions of the LaTeX2e float routines. It may not work
% with other packages that modify the LaTeX2e float routines. The latest
% version and documentation can be obtained at:
% http://www.ctan.org/pkg/stfloats
% Do not use the stfloats baselinefloat ability as the IEEE does not allow
% \baselineskip to stretch. Authors submitting work to the IEEE should note
% that the IEEE rarely uses double column equations and that authors should try
% to avoid such use. Do not be tempted to use the cuted.sty or midfloat.sty
% packages (also by Sigitas Tolusis) as the IEEE does not format its papers in
% such ways.
% Do not attempt to use stfloats with fixltx2e as they are incompatible.
% Instead, use Morten Hogholm'a dblfloatfix which combines the features
% of both fixltx2e and stfloats:
%
% \usepackage{dblfloatfix}
% The latest version can be found at:
% http://www.ctan.org/pkg/dblfloatfix




% *** PDF, URL AND HYPERLINK PACKAGES ***
%
\usepackage{url}
% url.sty was written by Donald Arseneau. It provides better support for
% handling and breaking URLs. url.sty is already installed on most LaTeX
% systems. The latest version and documentation can be obtained at:
% http://www.ctan.org/pkg/url
% Basically, \url{my_url_here}.

\usepackage{xcolor}
\usepackage{cleveref}
\usepackage{graphicx}
\usepackage{subcaption}

%comments
\newcommand{\zijian}[1]{\textcolor{blue}{ZH: #1}}
\newcommand{\jiasi}[1]{\textcolor{red}{JC: #1}}
\newcommand{\yicheng}[1]{\textcolor{green}{YZ: #1}}
\newcommand{\sophie}[1]{\textcolor{purple}{SC: #1}}
% \newcommand{\eg}{\textit{e}.\textit{g}.,~}
%Latin shortcuts
\newcommand{\eg}{\emph{e.g.,} }
\newcommand{\ie}{\emph{i.e.,} }
\newcommand{\etal}{\emph{et al.} }

% correct bad hyphenation here
\hyphenation{op-tical net-works semi-conduc-tor}
% \usepackage{times}
% \usepackage[pagebackref=true,breaklinks=true,colorlinks,bookmarks=false]{hyperref}
% correct bad hyphenation here
\hyphenation{op-tical net-works semi-conduc-tor}
\begin{document}
%
% paper title
% Titles are generally capitalized except for words such as a, an, and, as,
% at, but, by, for, in, nor, of, on, or, the, to and up, which are usually
% not capitalized unless they are the first or last word of the title.
% Linebreaks \\ can be used within to get better formatting as desired.
% Do not put math or special symbols in the title.

\title{Class-Conditional Neural Polarizer: A Lightweight and Effective Backdoor Defense by Purifying Poisoned Features}

\author{%
  Mingli Zhu\footnote[1]{}, Shaokui Wei, Hongyuan Zha, Baoyuan Wu \textit{Senior Member, IEEE}
\thanks{M. Zhu, S. Wei, and B. Wu are with the School of Data Science, The Chinese University of Hong Kong, Shenzhen, Guangdong, 518172, P.R. China. (Email: minglizhu@link.cuhk.edu.cn, shaokuiwei@link.cuhk.edu.cn, wubaoyuan@cuhk.edu.cn)}
\thanks{H. Zha is with the School of Data Science, The Chinese University of Hong Kong, Shenzhen, and the Shenzhen Key Laboratory of Crowd Intelligence Empowered Low-Carbon Energy Network, China. (Email: zhahy@cuhk.edu.cn)}
\thanks{Corresponding author: Baoyuan Wu (wubaoyuan@cuhk.edu.cn).}
\thanks{M. Zhu and S. Wei contributed equally to this work.}
}


% The paper headers
\markboth{Submitted to IEEE TRANSACTIONS ON PATTERN ANALYSIS AND MACHINE INTELLIGENCE}%
{Shell \MakeLowercase{\textit{\etal}}: Bare Demo of IEEEtran.cls for Computer Society Journals}
% The only time the second header will appear is for the odd numbered pages
% after the title page when using the twoside option.
% 
% *** Note that you probably will NOT want to include the author's ***
% *** name in the headers of peer review papers.                   ***
% You can use \ifCLASSOPTIONpeerreview for conditional compilation here if
% you desire.



% The publisher's ID mark at the bottom of the page is less important with
% Computer Society journal papers as those publications place the marks
% outside of the main text columns and, therefore, unlike regular IEEE
% journals, the available text space is not reduced by their presence.
% If you want to put a publisher's ID mark on the page you can do it like
% this:
%\IEEEpubid{0000--0000/00\$00.00~\copyright~2015 IEEE}
% or like this to get the Computer Society new two part style.
%\IEEEpubid{\makebox[\columnwidth]{\hfill 0000--0000/00/\$00.00~\copyright~2015 IEEE}%
%\hspace{\columnsep}\makebox[\columnwidth]{Published by the IEEE Computer Society\hfill}}
% Remember, if you use this you must call \IEEEpubidadjcol in the second
% column for its text to clear the IEEEpubid mark (Computer Society jorunal
% papers don't need this extra clearance.)



\maketitle

\begin{abstract}


The choice of representation for geographic location significantly impacts the accuracy of models for a broad range of geospatial tasks, including fine-grained species classification, population density estimation, and biome classification. Recent works like SatCLIP and GeoCLIP learn such representations by contrastively aligning geolocation with co-located images. While these methods work exceptionally well, in this paper, we posit that the current training strategies fail to fully capture the important visual features. We provide an information theoretic perspective on why the resulting embeddings from these methods discard crucial visual information that is important for many downstream tasks. To solve this problem, we propose a novel retrieval-augmented strategy called RANGE. We build our method on the intuition that the visual features of a location can be estimated by combining the visual features from multiple similar-looking locations. We evaluate our method across a wide variety of tasks. Our results show that RANGE outperforms the existing state-of-the-art models with significant margins in most tasks. We show gains of up to 13.1\% on classification tasks and 0.145 $R^2$ on regression tasks. All our code and models will be made available at: \href{https://github.com/mvrl/RANGE}{https://github.com/mvrl/RANGE}.

\end{abstract}


\section{Introduction}

Video generation has garnered significant attention owing to its transformative potential across a wide range of applications, such media content creation~\citep{polyak2024movie}, advertising~\citep{zhang2024virbo,bacher2021advert}, video games~\citep{yang2024playable,valevski2024diffusion, oasis2024}, and world model simulators~\citep{ha2018world, videoworldsimulators2024, agarwal2025cosmos}. Benefiting from advanced generative algorithms~\citep{goodfellow2014generative, ho2020denoising, liu2023flow, lipman2023flow}, scalable model architectures~\citep{vaswani2017attention, peebles2023scalable}, vast amounts of internet-sourced data~\citep{chen2024panda, nan2024openvid, ju2024miradata}, and ongoing expansion of computing capabilities~\citep{nvidia2022h100, nvidia2023dgxgh200, nvidia2024h200nvl}, remarkable advancements have been achieved in the field of video generation~\citep{ho2022video, ho2022imagen, singer2023makeavideo, blattmann2023align, videoworldsimulators2024, kuaishou2024klingai, yang2024cogvideox, jin2024pyramidal, polyak2024movie, kong2024hunyuanvideo, ji2024prompt}.


In this work, we present \textbf{\ours}, a family of rectified flow~\citep{lipman2023flow, liu2023flow} transformer models designed for joint image and video generation, establishing a pathway toward industry-grade performance. This report centers on four key components: data curation, model architecture design, flow formulation, and training infrastructure optimization—each rigorously refined to meet the demands of high-quality, large-scale video generation.


\begin{figure}[ht]
    \centering
    \begin{subfigure}[b]{0.82\linewidth}
        \centering
        \includegraphics[width=\linewidth]{figures/t2i_1024.pdf}
        \caption{Text-to-Image Samples}\label{fig:main-demo-t2i}
    \end{subfigure}
    \vfill
    \begin{subfigure}[b]{0.82\linewidth}
        \centering
        \includegraphics[width=\linewidth]{figures/t2v_samples.pdf}
        \caption{Text-to-Video Samples}\label{fig:main-demo-t2v}
    \end{subfigure}
\caption{\textbf{Generated samples from \ours.} Key components are highlighted in \textcolor{red}{\textbf{RED}}.}\label{fig:main-demo}
\end{figure}


First, we present a comprehensive data processing pipeline designed to construct large-scale, high-quality image and video-text datasets. The pipeline integrates multiple advanced techniques, including video and image filtering based on aesthetic scores, OCR-driven content analysis, and subjective evaluations, to ensure exceptional visual and contextual quality. Furthermore, we employ multimodal large language models~(MLLMs)~\citep{yuan2025tarsier2} to generate dense and contextually aligned captions, which are subsequently refined using an additional large language model~(LLM)~\citep{yang2024qwen2} to enhance their accuracy, fluency, and descriptive richness. As a result, we have curated a robust training dataset comprising approximately 36M video-text pairs and 160M image-text pairs, which are proven sufficient for training industry-level generative models.

Secondly, we take a pioneering step by applying rectified flow formulation~\citep{lipman2023flow} for joint image and video generation, implemented through the \ours model family, which comprises Transformer architectures with 2B and 8B parameters. At its core, the \ours framework employs a 3D joint image-video variational autoencoder (VAE) to compress image and video inputs into a shared latent space, facilitating unified representation. This shared latent space is coupled with a full-attention~\citep{vaswani2017attention} mechanism, enabling seamless joint training of image and video. This architecture delivers high-quality, coherent outputs across both images and videos, establishing a unified framework for visual generation tasks.


Furthermore, to support the training of \ours at scale, we have developed a robust infrastructure tailored for large-scale model training. Our approach incorporates advanced parallelism strategies~\citep{jacobs2023deepspeed, pytorch_fsdp} to manage memory efficiently during long-context training. Additionally, we employ ByteCheckpoint~\citep{wan2024bytecheckpoint} for high-performance checkpointing and integrate fault-tolerant mechanisms from MegaScale~\citep{jiang2024megascale} to ensure stability and scalability across large GPU clusters. These optimizations enable \ours to handle the computational and data challenges of generative modeling with exceptional efficiency and reliability.


We evaluate \ours on both text-to-image and text-to-video benchmarks to highlight its competitive advantages. For text-to-image generation, \ours-T2I demonstrates strong performance across multiple benchmarks, including T2I-CompBench~\citep{huang2023t2i-compbench}, GenEval~\citep{ghosh2024geneval}, and DPG-Bench~\citep{hu2024ella_dbgbench}, excelling in both visual quality and text-image alignment. In text-to-video benchmarks, \ours-T2V achieves state-of-the-art performance on the UCF-101~\citep{ucf101} zero-shot generation task. Additionally, \ours-T2V attains an impressive score of \textbf{84.85} on VBench~\citep{huang2024vbench}, securing the top position on the leaderboard (as of 2025-01-25) and surpassing several leading commercial text-to-video models. Qualitative results, illustrated in \Cref{fig:main-demo}, further demonstrate the superior quality of the generated media samples. These findings underscore \ours's effectiveness in multi-modal generation and its potential as a high-performing solution for both research and commercial applications.
\section{Related Work}
\label{sec:relatedwork}

\subsection{Current AI Tools for Social Service}
\label{subsec:relatedtools}
% the title I feel is quite broad

Harnessing technology for social good has always been a grand challenge in social service \cite{berzin_practice_2015}. As early as the 90s, artificial neural networks and predictive models have been employed as tools for risk assessments, decision-making, and workload management in sectors like child protective services and mental health treatment \cite{fluke_artificial_1989, patterson_application_1999}. The recent rise of generative AI is poised to further advance social service practice, facilitating the automation of administrative tasks, streamlining of paperwork and documentation, optimisation of resource allocation, data analysis, and enhancing client support and interventions \cite{fernando_integration_2023, perron_generative_2023}.

Today, AI solutions are increasingly being deployed in both policy and practice \cite{goldkind_social_2021, hodgson_problematising_2022}. In clinical social work, AI has been used for risk assessments, crisis management, public health initiatives, and education and training for practitioners \cite{asakura_call_2020, gillingham2019can, jacobi_functions_2023, liedgren_use_2016, molala_social_2023, rice_piloting_2018, tambe_artificial_2018}. AI has also been employed for mental health support and therapeutic interventions, with conversational agents serving as on-demand virtual counsellors to provide clinical care and support \cite{lisetti_i_2013, reamer_artificial_2023}.
% commercial solutions include Woebot, which simulates therapeutic conversation, and Wysa, an “emotionally intelligent” AI coach, powered by evidenced-based clinical techniques \cite{reamer_artificial_2023}. 
% Non-clinical AI agents like Replika and companion robots can also provide social support and reduce loneliness amongst individuals \cite{ahmed_humanrobot_2024, chaturvedi_social_2023, pani_can_2024, ta_user_2020}.

Present research largely focuses on \textit{\textbf{AI-based decision support tools}} in social service \cite{james_algorithmic_2023, kawakami2022improving}, especially predictive risk models (PRMs) used to predict social service risks and outcomes \cite{gillingham2019can, van2017predicting}, like the Allegheny Family Screening Tool (AFST), which assesses child abuse risk using data from US public systems \cite{chouldechova_case_2018, vaithianathan2017developing}. Elsewhere, researchers have also piloted PRMs to predict social service needs for the homeless using Medicaid data\cite{erickson_automatic_2018, pourat_easy_2023}, and AI-powered algorithms to promote health interventions for at-risk populations, such as HIV testing among Californian homeless \cite{rice_piloting_2018, yadav_maximizing_2017}.

\subsection{Generative AI and Human-AI Collaboration}
\label{subsec:relatedworkhaicollaboration}
Beyond decision-making algorithms and PRMs, advancements in generative AI, such as large language models (LLMs), open new possibilities for human-AI (HAI) collaboration in social services. 
LLMs have been called "revolutionary" \cite{fui2023generative} and a "seismic shift" \cite{cooper2023examining}, offering "content support" \cite{memmert2023towards} by generating realistic and coherent responses to user inputs \cite{cascella2023evaluating}. Their vastly improved capabilities and ubiquity \cite{cooper2023examining} makes them poised to revolutionise work patterns \cite{fui2023generative}. Generative AI is already used in fields like design, writing, music, \cite{han2024teams, suh2021ai, verheijden2023collaborative, dhillon2024shaping, gero2023social} healthcare, and clinical settings \cite{zhang2023generative, yu2023leveraging, biswas2024intelligent}, with promising results. However, the social service sector has been slower in adopting AI \cite{diez2023artificial, kawakami2023training}.

% Yet, the social service sector is one that could perhaps stand to gain the most from AI technologies. As Goldkind \cite{goldkind_social_2021} writes, social service, as a "values-centred profession with a robust code of ethics" (p. 372), is uniquely placed to inform the development of thoughtful algorithmic policy and practice. 
Social service, however, stands to benefit immensely from generative AI. SSPs work in time-poor environments \cite{tiah_can_2024}, often overwhelmed with tedious administrative work \cite{meilvang_working_2023} and large amounts of paperwork and data processing \cite{singer_ai_2023, tiah_can_2024}. 
% As such, workers often work in time-poor environments and are burdened with information overload and administrative tasks \cite{tiah_can_2024, meilvang_working_2023}. 
Generative AI is well-placed to streamline and automate tasks like formatting case notes, formulating treatment plans and writing progress reports, which can free up valuable time for more meaningful work like client engagement and enhance service quality \cite{fernando_integration_2023, perron_generative_2023, tiah_can_2024, thesocialworkaimentor_ai_nodate}. 

Given the immense potential, there has been emerging research interest in HAI collaboration and teamwork in the Human-Computer Interaction and Computer Supported Cooperative Work space \cite{wang_human-human_2020}. HAI collaboration and interaction has been postulated by researchers to contribute to new forms of HAI symbiosis and augmented intelligence, where algorithmic and human agents work in tandem with one another to perform tasks better than they could accomplish alone by augmenting each other's strengths and capabilities  \cite{dave_augmented_2023, jarrahi_artificial_2018}.

However, compared to the focus on AI decision-making and PRM tools, there is scant research on generative AI and HAI collaboration in the social service sector \cite{wykman_artificial_2023}. This study therefore seeks to fill this critical gap by exploring how SSPs use and interact with a novel generative AI tool, helping to expand our understanding of the new opportunities that HAI collaboration can bring to the social service sector.

\subsection{Challenges in AI Use in Social Service}
\label{subsec:relatedworkaiuse}

% Despite the immense potential of AI systems to augment social work practice, there are multiple challenges with integrating such systems into real-life practice. 
Despite its evident benefits, multiple challenges plague the integration of AI and its vast potential into real-life social service practice.
% Numerous studies have investigated the use of PRMs to help practitioners decide on a course of action for their clients. 
When employing algorithmic decision-making systems, practitioners often experience tension in weighing AI suggestions against their own judgement \cite{kawakami2022improving, saxena2021framework}, being uncertain of how far they should rely on the machine. 
% Despite often being instructed to use the tool as part of evaluating a client, 
Workers are often reluctant to fully embrace AI assessments due to its inability to adequately account for the full context of a case \cite{kawakami2022improving, gambrill2001need}, and lack of clarity and transparency on AI systems and limitations \cite{kawakami2022improving}. Brown et al. \cite{brown2019toward} conducted workshops using hypothetical algorithmic tools 
% to understand service providers' comfort levels with using such tools in their work,
and found similar issues with mistrust and perceived unreliability. Furthermore, introducing AI tools can  create new problems of its own, causing confusion and distrust amongst workers \cite{kawakami2022improving}. Such factors are critical barriers to the acceptance and effective use of AI in the sector.

\citeauthor{meilvang_working_2023} (2023) cites the concept of \textit{boundary work}, which explores the delineation between "monotonous" administrative labour and "professional", "knowledge based" work drawing on core competencies of SSPs. While computers have long been used for bureaucratic tasks like client registration, the introduction of decision support systems like PRMs stirred debate over AI "threatening professional discretion and, as such, the profession itself" \cite{meilvang_working_2023}. Such latent concerns arguably drive the resistance to technology adoption described above. Generative AI is only set to further push this boundary, 
% these concerns are only set to grow in tandem with the vast capabilities of generative and other modern AI systems. Compared to the relatively primitive AI systems in past years, perceived as statistical algorithms \cite{brown2019toward} turning preset inputs like client age and behavioural symptoms \cite{vaithianathan2017developing} into simple numerical outputs indicating various risk scores, modern AI systems are vastly more capable: LLMs 
with its ability to formulate detailed reports and assessments that encroach upon the "core" work of SSPs.
% accept unrestricted and unstructured inputs and return a range of verbose and detailed evaluations according to the user's instructions. 
Introducing these systems exacerbate previously-raised issues such as understanding the limitations and possibilities of AI systems \cite{kawakami2022improving} and risk of overreliance on AI \cite{van2023chatgpt}, and requires a re-examination of where users fall on the algorithmic aversion-bias scale \cite{brown2019toward} and how they detect and react to algorithmic failings \cite{de2020case}. We address these critical issues through an empirical, on-the-ground study that to our knowledge is the first of its kind since the new wave of generative AI.

% W 

% Yet, to date, we have limited knowledge on the real-world impacts and implications of human-AI collaboration, and few studies have investigated practitioners’ experiences working with and using such AI systems in practice, especially within the social work context \cite{kawakami2022improving}. A small number of studies have explored practitioner perspectives on the use of AI in social work, including Kawakami et al. \cite{kawakami2022improving}, who interviewed social workers on their experiences using the AFST; Stapleton et al. \cite{stapleton_imagining_2022}, who conducted design workshops with caseworkers on the use of PRMs in child welfare; and Wassal et al. \cite{wassal_reimagining_2024}, who interviewed UK social work professionals on the use of AI. A common thread from all these studies was a general disregard for the context and users, with many practitioners criticising the failure of past AI tools arising from the lack of participation and involvement of social workers and actual users of such systems in the design and development of algorithmic systems \cite{wassal_reimagining_2024}. Similarly, in a scoping review done on decision-support algorithms in social work, Jacobi \& Christensen \cite{jacobi_functions_2023} reported that the majority of studies reveal limited bottom-up involvement and interaction between social workers, researchers and developers, and that algorithms were rarely developed with consideration of the perspective of social workers.
% so the \cite{yang_unremarkable_2019} and \cite{holten_moller_shifting_2020} are not real-world impacts? real-world means to hear practitioner's voice? I feel this is quite important but i didnt get this point in intro!

% why mentioning 'which have largely focused on existing ADS tools (e.g., AFST)'? i can see our strength is more localized, but without basic knowledge of social work i didnt get what's the 'departure' here orz
% the paragraph is great! do we need to also add one in line 20 21?

\subsection{Designing AI for Social Service through Participatory Design}
\label{subsec:relatedworkpd}
% i think it's important! but maybe not a whole subsection? but i feel the strong connection with practitioners is indeed one of our novelties and need to highlight it, also in intro maybe
% Participatory design (PD) has long been used extensively in HCI \cite{muller1993participatory}, to both design effective solutions for a specific community and gain a deep understanding of that community. Of particular interest here is the rich body of literature on PD in the field of healthcare \cite{donetto2015experience}, which in this regard shares many similarities and concerns with social work. PD has created effective health improvement apps \cite{ryu2017impact}, 

% PD offers researchers the chance to gather detailed user requirements \cite{ryu2017impact}...

Participatory design (PD) is a staple of HCI research \cite{muller1993participatory}, facilitating the design of effective solutions for a specific community while gaining a deep understanding of its stakeholders. The focus in PD of valuing the opinions and perspectives of users as experts \cite{schuler_participatory_1993} 
% In recent years, the tech and social work sectors have awakened to the importance of involving real users in designing and implementing digital technologies, developing human-centred design processes to iteratively design products or technologies through user feedback 
has gained importance in recent years \cite{storer2023reimagining}. Responding to criticisms and failures of past AI tools that have been implemented without adequate involvement and input from actual users, HCI scholars have adopted PD approaches to design predictive tools to better support human decision-making \cite{lehtiniemi_contextual_2023}.
% ; accordingly, in social service, a line of research has begun studying and designing for human-AI collaboration with real-world users (e.g. \cite{holten_moller_shifting_2020, kawakami2022improving, yang_unremarkable_2019}).
Section \ref{subsec:relatedworkaiuse} shows a clear need to better understand SSP perspectives when designing and implementing AI tools in the social sector. 
Yet, PD research in this area has been limited. \citeauthor{yang2019unremarkable} (2019), through field evaluation with clinicians, investigated reasons behind the failure of previous AI-powered decision support tools, allowing them to design a new-and-improved AI decision-support tool that was better aligned with healthcare workers’ workflows. Similarly, \citeauthor{holten_moller_shifting_2020} (2020) ran PD workshops with caseworkers, data scientists and developers in public service systems to identify the expectations and needs that different stakeholders had in using ADS tools.

% Indeed, it is as Wise \cite{wise_intelligent_1998} noted so many years ago on the rise of intelligent agents: “it is perhaps when technologies are new, when their (and our) movements, habits and attitudes seem most awkward and therefore still at the forefront of our thoughts that they are easiest to analyse” (p. 411). 
Building upon this existing body of work, we thus conduct a study to co-design an AI tool \textit{for} and \textit{with} SSPs through participatory workshops and focus group discussions. In the process, we revisit many of the issues mentioned in Section \ref{subsec:relatedworkaiuse}, but in the context of novel generative AI systems, which are fundamentally different from most historical examples of automation technologies \cite{noy2023experimental}. This valuable empirical inquiry occurs at an opportune time when varied expectations about this nascent technology abound \cite{lehtiniemi_contextual_2023}, allowing us to understand how SSPs incorporate AI into their practice, and what AI can (or cannot) do for them. In doing so, we aim to uncover new theoretical and practical insights on what AI can bring to the social service sector, and formulate design implications for developing AI technologies that SSPs find truly meaningful and useful.
% , and drive future technological innovations to transform the social service sector not just within [our country], but also on a global scale.

 % with an on-the-ground study using a real prototype system that reflects the state of AI in current society. With the presumption that AI will continue to be used in social work given the great benefits it brings, we address the pressing need to investigate these issues to ensure that any potential AI systems are designed and implemented in a responsible and effective manner.

% Building upon these works, this study therefore seeks to adopt a participatory design methodology to investigate social workers’ perspectives and attitudes on AI and human-AI collaboration in their social work practice, thus contributing to the nascent body of practitioner-centred HCI research on the use of AI in social work. Yet, in a departure from prior work, which have largely focused on existing ADS tools (e.g., AFST) and were situated in a Western context, our paper also aims to expand the scope by piloting a novel generative AI tool that was designed and developed by the researchers in partnership with a social service agency based in Singapore, with aims of generating more insights on wider use cases of AI beyond what has been previously studied.

% i may think 'While the current lacunae of research on applications of AI in social work may appear to be a limitation, it simultaneously presents an exciting opportunity for further research and exploration \cite{dey_unleashing_2023},' this point is already convincing enough, not sure if we need to quote here
% I like this end! it's a good transition to our study design, do we need to mention the localization in intro as well? like we target at singapore

% Given the increasing prominence and acceptance of AI in modern society, 

% These increased capabilities vastly exacerbate the issues already present with a simpler tool like the AFST: the boundaries and limitations of an LLM system are significantly more difficult to understand and its possible use cases are exponentially greater in scope. 

% Put this in discussion section instead?
% Kawakami et al's work "highlights the importance of studying how collaborative decision-making... impacts how people rely upon and make sense of AI models," They conclude by recommending designing tools that "support workers in understanding the boundaries of [an AI system's] capabilities", and implementing design procedures that "support open cultures for critical discussion around AI decision making". The authors outline critical challenges of implementing AI systems, elucidating factors that may hinder their effectiveness and even negatively affect operations within the organisation.


% Is this needed?:
% talk about the strengths of PD in eliciting user viewpoints and knowledge, in particular when it is a field that is novel or where a certain system has not been used or developed or tested before
\section{Study Design}
% robot: aliengo 
% We used the Unitree AlienGo quadruped robot. 
% See Appendix 1 in AlienGo Software Guide PDF
% Weight = 25kg, size (L,W,H) = (0.55, 0.35, 06) m when standing, (0.55, 0.35, 0.31) m when walking
% Handle is 0.4 m or 0.5 m. I'll need to check it to see which type it is.
We gathered input from primary stakeholders of the robot dog guide, divided into three subgroups: BVI individuals who have owned a dog guide, BVI individuals who were not dog guide owners, and sighted individuals with generally low degrees of familiarity with dog guides. While the main focus of this study was on the BVI participants, we elected to include survey responses from sighted participants given the importance of social acceptance of the robot by the general public, which could reflect upon the BVI users themselves and affect their interactions with the general population \cite{kayukawa2022perceive}. 

The need-finding processes consisted of two stages. During Stage 1, we conducted in-depth interviews with BVI participants, querying their experiences in using conventional assistive technologies and dog guides. During Stage 2, a large-scale survey was distributed to both BVI and sighted participants. 

This study was approved by the University’s Institutional Review Board (IRB), and all processes were conducted after obtaining the participants' consent.

\subsection{Stage 1: Interviews}
We recruited nine BVI participants (\textbf{Table}~\ref{tab:bvi-info}) for in-depth interviews, which lasted 45-90 minutes for current or former dog guide owners (DO) and 30-60 minutes for participants without dog guides (NDO). Group DO consisted of five participants, while Group NDO consisted of four participants.
% The interview participants were divided into two groups. Group DO (Dog guide Owner) consisted of five participants who were current or former dog guide owners and Group NDO (Non Dog guide Owner) consisted of three participants who were not dog guide owners. 
All participants were familiar with using white canes as a mobility aid. 

We recruited participants in both groups, DO and NDO, to gather data from those with substantial experience with dog guides, offering potentially more practical insights, and from those without prior experience, providing a perspective that may be less constrained and more open to novel approaches. 

We asked about the participants' overall impressions of a robot dog guide, expectations regarding its potential benefits and challenges compared to a conventional dog guide, their desired methods of giving commands and communicating with the robot dog guide, essential functionalities that the robot dog guide should offer, and their preferences for various aspects of the robot dog guide's form factors. 
For Group DO, we also included questions that asked about the participants' experiences with conventional dog guides. 

% We obtained permission to record the conversations for our records while simultaneously taking notes during the interviews. The interviews lasted 30-60 minutes for NDO participants and 45-90 minutes for DO participants. 

\subsection{Stage 2: Large-Scale Surveys} 
After gathering sufficient initial results from the interviews, we created an online survey for distributing to a larger pool of participants. The survey platform used was Qualtrics. 

\subsubsection{Survey Participants}
The survey had 100 participants divided into two primary groups. Group BVI consisted of 42 blind or visually impaired participants, and Group ST consisted of 58 sighted participants. \textbf{Table}~\ref{tab:survey-demographics} shows the demographic information of the survey participants. 

\subsubsection{Question Differentiation} 
Based on their responses to initial qualifying questions, survey participants were sorted into three subgroups: DO, NDO, and ST. Each participant was assigned one of three different versions of the survey. The surveys for BVI participants mirrored the interview categories (overall impressions, communication methods, functionalities, and form factors), but with a more quantitative approach rather than the open-ended questions used in interviews. The DO version included additional questions pertaining to their prior experience with dog guides. The ST version revolved around the participants' prior interactions with and feelings toward dog guides and dogs in general, their thoughts on a robot dog guide, and broad opinions on the aesthetic component of the robot's design. 


\section{Dataset}
\label{sec:dataset}

\subsection{Data Collection}

To analyze political discussions on Discord, we followed the methodology in \cite{singh2024Cross-Platform}, collecting messages from politically-oriented public servers in compliance with Discord's platform policies.

Using Discord's Discovery feature, we employed a web scraper to extract server invitation links, names, and descriptions, focusing on public servers accessible without participation. Invitation links were used to access data via the Discord API. To ensure relevance, we filtered servers using keywords related to the 2024 U.S. elections (e.g., Trump, Kamala, MAGA), as outlined in \cite{balasubramanian2024publicdatasettrackingsocial}. This resulted in 302 server links, further narrowed to 81 English-speaking, politics-focused servers based on their names and descriptions.

Public messages were retrieved from these servers using the Discord API, collecting metadata such as \textit{content}, \textit{user ID}, \textit{username}, \textit{timestamp}, \textit{bot flag}, \textit{mentions}, and \textit{interactions}. Through this process, we gathered \textbf{33,373,229 messages} from \textbf{82,109 users} across \textbf{81 servers}, including \textbf{1,912,750 messages} from \textbf{633 bots}. Data collection occurred between November 13th and 15th, covering messages sent from January 1st to November 12th, just after the 2024 U.S. election.

\subsection{Characterizing the Political Spectrum}
\label{sec:timeline}

A key aspect of our research is distinguishing between Republican- and Democratic-aligned Discord servers. To categorize their political alignment, we relied on server names and self-descriptions, which often include rules, community guidelines, and references to key ideologies or figures. Each server's name and description were manually reviewed based on predefined, objective criteria, focusing on explicit political themes or mentions of prominent figures. This process allowed us to classify servers into three categories, ensuring a systematic and unbiased alignment determination.

\begin{itemize}
    \item \textbf{Republican-aligned}: Servers referencing Republican and right-wing and ideologies, movements, or figures (e.g., MAGA, Conservative, Traditional, Trump).  
    \item \textbf{Democratic-aligned}: Servers mentioning Democratic and left-wing ideologies, movements, or figures (e.g., Progressive, Liberal, Socialist, Biden, Kamala).  
    \item \textbf{Unaligned}: Servers with no defined spectrum and ideologies or opened to general political debate from all orientations.
\end{itemize}

To ensure the reliability and consistency of our classification, three independent reviewers assessed the classification following the specified set of criteria. The inter-rater agreement of their classifications was evaluated using Fleiss' Kappa \cite{fleiss1971measuring}, with a resulting Kappa value of \( 0.8191 \), indicating an almost perfect agreement among the reviewers. Disagreements were resolved by adopting the majority classification, as there were no instances where a server received different classifications from all three reviewers. This process guaranteed the consistency and accuracy of the final categorization.

Through this process, we identified \textbf{7 Republican-aligned servers}, \textbf{9 Democratic-aligned servers}, and \textbf{65 unaligned servers}.

Table \ref{tab:statistics} shows the statistics of the collected data. Notably, while Democratic- and Republican-aligned servers had a comparable number of user messages, users in the latter servers were significantly more active, posting more than double the number of messages per user compared to their Democratic counterparts. 
This suggests that, in our sample, Democratic-aligned servers attract more users, but these users were less engaged in text-based discussions. Additionally, around 10\% of the messages across all server categories were posted by bots. 

\subsection{Temporal Data} 

Throughout this paper, we refer to the election candidates using the names adopted by their respective campaigns: \textit{Kamala}, \textit{Biden}, and \textit{Trump}. To examine how the content of text messages evolves based on the political alignment of servers, we divided the 2024 election year into three periods: \textbf{Biden vs Trump} (January 1 to July 21), \textbf{Kamala vs Trump} (July 21 to September 20), and the \textbf{Voting Period} (after September 20). These periods reflect key phases of the election: the early campaign dominated by Biden and Trump, the shift in dynamics with Kamala Harris replacing Joe Biden as the Democratic candidate, and the final voting stage focused on electoral outcomes and their implications. This segmentation enables an analysis of how discourse responds to pivotal electoral moments.

Figure \ref{fig:line-plot} illustrates the distribution of messages over time, highlighting trends in total messages volume and mentions of each candidate. Prior to Biden's withdrawal on July 21, mentions of Biden and Trump were relatively balanced. However, following Kamala's entry into the race, mentions of Trump surged significantly, a trend further amplified by an assassination attempt on him, solidifying his dominance in the discourse. The only instance where Trump’s mentions were exceeded occurred during the first debate, as concerns about Biden’s age and cognitive abilities temporarily shifted the focus. In the final stages of the election, mentions of all three candidates rose, with Trump’s mentions peaking as he emerged as the victor.
\section{Experimental Methodology}\label{sec:exp}
In this section, we introduce the datasets, evaluation metrics, baselines, and implementation details used in our experiments. More experimental details are shown in Appendix~\ref{app:experiment_detail}.

\textbf{Dataset.}
We utilize various datasets for training and evaluation. Data statistics are shown in Table~\ref{tab:dataset}.

\textit{Training.}
We use the publicly available E5 dataset~\cite{wang2024improving,springer2024repetition} to train both the LLM-QE and dense retrievers. We concentrate on English-based question answering tasks and collect a total of 808,740 queries. From this set, we randomly sample 100,000 queries to construct the DPO training data, while the remaining queries are used for contrastive training. During the DPO preference pair construction, we first prompt LLMs to generate expansion documents, filtering out queries where the expanded documents share low similarity with the query. This results in a final set of 30,000 queries.

\textit{Evaluation.}
We evaluate retrieval effectiveness using two retrieval benchmarks: MS MARCO \cite{bajaj2016ms} and BEIR \cite{thakur2021beir}, in both unsupervised and supervised settings.

\textbf{Evaluation Metrics.}
We use nDCG@10 as the evaluation metric. Statistical significance is tested using a permutation test with $p<0.05$.

\textbf{Baselines.} We compare our LLM-QE model with three unsupervised retrieval models and five query expansion baseline models.
% —

Three unsupervised retrieval models—BM25~\cite{robertson2009probabilistic}, CoCondenser~\cite{gao2022unsupervised}, and Contriever~\cite{izacard2021unsupervised}—are evaluated in the experiments. Among these, Contriever serves as our primary baseline retrieval model, as it is used as the backbone model to assess the query expansion performance of LLM-QE. Additionally, we compare LLM-QE with Contriever in a supervised setting using the same training dataset.

For query expansion, we benchmark against five methods: Pseudo-Relevance Feedback (PRF), Q2Q, Q2E, Q2C, and Q2D. PRF is specifically implemented following the approach in~\citet{yu2021improving}, which enhances query understanding by extracting keywords from query-related documents. The Q2Q, Q2E, Q2C, and Q2D methods~\cite{jagerman2023query,li2024can} expand the original query by prompting LLMs to generate query-related queries, keywords, chains-of-thought~\cite{wei2022chain}, and documents.


\textbf{Implementation Details.} 
For our query expansion model, we deploy the Meta-LLaMA-3-8B-Instruct~\cite{llama3modelcard} as the backbone for the query expansion generator. The batch size is set to 16, and the learning rate is set to $2e-5$. Optimization is performed using the AdamW optimizer. We employ LoRA~\cite{hu2022lora} to efficiently fine-tune the model for 2 epochs. The temperature for the construction of the DPO data varies across $\tau \in \{0.8, 0.9, 1.0, 1.1\}$, with each setting sampled eight times. For the dense retriever, we utilize Contriever~\cite{izacard2021unsupervised} as the backbone. During training, we set the batch size to 1,024 and the learning rate to $3e-5$, with the model trained for 3 epochs.

\section{Conclusion}
We introduce a novel approach, \algo, to reduce human feedback requirements in preference-based reinforcement learning by leveraging vision-language models. While VLMs encode rich world knowledge, their direct application as reward models is hindered by alignment issues and noisy predictions. To address this, we develop a synergistic framework where limited human feedback is used to adapt VLMs, improving their reliability in preference labeling. Further, we incorporate a selective sampling strategy to mitigate noise and prioritize informative human annotations.

Our experiments demonstrate that this method significantly improves feedback efficiency, achieving comparable or superior task performance with up to 50\% fewer human annotations. Moreover, we show that an adapted VLM can generalize across similar tasks, further reducing the need for new human feedback by 75\%. These results highlight the potential of integrating VLMs into preference-based RL, offering a scalable solution to reducing human supervision while maintaining high task success rates. 

\section*{Impact Statement}
This work advances embodied AI by significantly reducing the human feedback required for training agents. This reduction is particularly valuable in robotic applications where obtaining human demonstrations and feedback is challenging or impractical, such as assistive robotic arms for individuals with mobility impairments. By minimizing the feedback requirements, our approach enables users to more efficiently customize and teach new skills to robotic agents based on their specific needs and preferences. The broader impact of this work extends to healthcare, assistive technology, and human-robot interaction. One possible risk is that the bias from human feedback can propagate to the VLM and subsequently to the policy. This can be mitigated by personalization of agents in case of household application or standardization of feedback for industrial applications. 
\bibliographystyle{plain}
\bibliography{main}
% \input{sections/7_bio}


\clearpage
\renewcommand{\thefigure}{A\arabic{figure}}
\renewcommand{\thetable}{A\arabic{table}}
\renewcommand{\theequation}{A\arabic{equation}}
\setcounter{figure}{0}
\setcounter{table}{0}
\setcounter{equation}{0}

Our Appendix is organized as follows. First, we present the pseudocode for the key components of iGCT. We also include the proof for unit variance and boundary conditions in preconditioning iGCT's noiser. Next, we detail the training setups for our CIFAR-10 and ImageNet64 experiments. Additionally, we provide ablation studies on using guided synthesized images as data augmentation in image classification. Finally, we present more uncurated results comparing iGCT and CFG-EDM on inversion, editing and guidance, thoroughly of iGCT.

\vspace{-0.2cm}
\label{appendix:iGCT}
\section{Pseudocode for iGCT}
\vspace{-0.2cm}

iGCT is trained under a continuous-time scheduler similar to the one proposed by ECT \cite{ect}. Our noise sampling function follows a lognormal distribution, \(p(t) = \textit{LogNormal}(P_\textit{mean}, P_\textit{std})\), with \(P_\textit{mean}=-1.1\) and \( P_\textit{std}=2.0\). At training, the sampled noise is clamped at \(t_\text{min} = 0.002\) and \(t_\text{max} = 80.0\). Step function \(\Delta t (t)=\frac{t}{2^{\left\lfloor k/d \right\rfloor}}n(t)\), is used to compute the step size from the sampled noise \(t\), with \(k,d\) being the current training iteration and the number of iterations for halfing \(\Delta t\), and \(n(t) = 1 + 8 \sigma(-t)\) is a sigmoid adjusting function. 

In Guided Consistency Training, the guidance mask function determines whether the sampled noise \( t \) should be supervised for guidance training. With probability \( q(t) \in [0,1] \), the update is directed towards the target sample \( \boldsymbol{x}_0^{\text{tar}} \); otherwise, no guidance is applied. In practice, \( q(t) \) is higher in noisier regions and zero in low-noise regions, 
\begin{equation}
    q(t) = 0.9 \cdot \left( \text{clamp} \left( \frac{t - t_{\text{low}}}{t_{\text{high}} - t_{\text{low}}}, 0, 1 \right) \right)^2,
\end{equation}
where \( t_{\text{low}} = 11.0 \) and \( t_{\text{high}} = 14.3 \). For the range of guidance strength, we set \(w_\text{min} = 1\) and \(w_\text{max} = 15\). Guidance strengths are sampled uniformly at training, with \(w_\text{min} = 1\) means no guidance applied. 


\begin{algorithm}
\caption{Guided Consistency Training}
\label{alg:GCT}
\begin{algorithmic}[1]  % Adds line numbers
\setlength{\baselineskip}{0.9\baselineskip} % Adjust line spacing
\INPUT Dataset $\mathcal{D}$, weighting function $\lambda(t)$, noise sampling function $p(t)$, noise range $[t_\text{min}, t_\text{max}]$, step function $\Delta t(t)$, guidance mask function $q(t)$, guidance range $[w_\text{min}, w_\text{max}]$, denoiser $D_\theta$
\STATE \rule{0.96\textwidth}{0.45pt} 
\STATE Sample $(\boldsymbol{x}_0^{\text{src}}, c^{\text{src}}), (\boldsymbol{x}_0^{\text{tar}}, c^{\text{tar}}) \sim \mathcal{D}$ 
\STATE Sample noise $\boldsymbol{z} \sim \mathcal{N}(\boldsymbol{0},\mathbf{I})$, time step $t \sim p(t)$, and guidance weight $w \sim \mathcal{U}(w_\text{min}, w_\text{max})$
\STATE Clamp $t \leftarrow \text{clamp}(t,t_\text{min}, t_\text{max})$
\STATE Compute noisy sample: $\boldsymbol{x}_t = \boldsymbol{x}_0^{\text{src}} + t\boldsymbol{z}$
\STATE Sample $\rho \sim \mathcal{U}(0,1)$  
\vspace{0.3em}
\IF{$\rho > q(t)$}
    \STATE Compute step as normal CT: $\boldsymbol{x}_r = \boldsymbol{x}_t - \Delta t(t) \boldsymbol{z}$
    \STATE Set target class: $c \leftarrow c^{\text{src}}$
\ELSE
    \STATE Compute guided noise: $\boldsymbol{z}^* = (\boldsymbol{x}_t - \boldsymbol{x}_0^{\text{tar}}) / t$
    \STATE Compute guided step: $\boldsymbol{x}_r = \boldsymbol{x}_t - \Delta t(t) [w \boldsymbol{z}^* + (1-w)\boldsymbol{z}]$
    \STATE Set target class: $c \leftarrow c^{\text{tar}}$
\ENDIF
\vspace{0.3em} % Reduces extra vertical space before the loss line
\STATE Compute loss: 
\[
\mathcal{L}_\text{gct} = \lambda(t) \, d(D_{\theta}(\boldsymbol{x}_t, t, c, w), D_{{\theta}^-}(\boldsymbol{x}_r, r, c, w))
\]
\STATE Return $\mathcal{L}_\text{gct}$ 
\end{algorithmic}
\end{algorithm}



A \textit{noiser} trained under \textit{Inverse Consistency Training} maps an image to its latent noise in a single step. In contrast, DDIM Inversion requires multiple steps with a diffusion model to accurately produce an image's latent representation. Since the boundary signal is reversed, spreading from \( t_\text{max} \) down to \( t_\text{min} \), we design the importance weighting function \( \lambda'(t) \) to emphasize higher noise regions, defined as:
\begin{equation}
    \lambda'(t) = \frac{\Delta t (t)}{t_\text{max}},
\end{equation}
where the step size \( \Delta t (t) \) is proportional to the sampled noise level \(t\), and \( t_\text{max} \) is a constant that normalizes the scale of the inversion loss. The noise sampling function \( p(t) \) and the step function \( \Delta t (t) \) used in computing both \(\mathcal{L}_\text{gct}\) and \(\mathcal{L}_\text{ict}\) are the same.



\begin{algorithm}
\caption{Inverse Consistency Training}
\label{alg:iCT}
\begin{algorithmic}[1]  % Adds line numbers
\setlength{\baselineskip}{0.9\baselineskip} % Adjust line spacing
\INPUT Dataset $\mathcal{D}$, weighting function $\lambda'(t)$, noise sampling function $p(t)$, noise range $[t_\text{min}, t_\text{max}]$, step function $\Delta t(t)$, noiser $N_\varphi$
\STATE \rule{0.96\textwidth}{0.45pt} 
\STATE Sample $\boldsymbol{x}_0, c \sim \mathcal{D}$ 
\STATE Sample noise $\boldsymbol{z} \sim \mathcal{N}(\boldsymbol{0},\mathbf{I})$, time step $t \sim p(t)$
\STATE Clamp $t \leftarrow \text{clamp}(t,t_\text{min}, t_\text{max})$
\STATE Compute noisy sample: $\boldsymbol{x}_t = \boldsymbol{x}_0 + t\boldsymbol{z}$
\STATE Compute cleaner sample: $\boldsymbol{x}_r = \boldsymbol{x}_t - \Delta t(t) \boldsymbol{z}$
\vspace{0.3em} 
\STATE Compute loss: 
\[
\mathcal{L}_\text{ict} = \lambda'(t) \, d(N_{\varphi}(\boldsymbol{x}_r, r, c), D_{{\varphi}^-}(\boldsymbol{x}_t, t, c))
\]
\STATE Return $\mathcal{L}_\text{ict}$ 
\end{algorithmic}
\end{algorithm}

Together, iGCT jointly optimizes the two consistency objectives \(\mathcal{L}_\text{gct}, \mathcal{L}_\text{ict}\), and aligns the noiser and denoiser via a reconstruction loss, \(\mathcal{L}_\text{recon}\). To improve training efficiency, \(\mathcal{L}_\text{recon}\) is computed every \(i_\text{skip}\), reducing the computational cost of back-propagation through both the weights of the \textit{denoiser} \(\theta\) and the \textit{noiser} \(\varphi\). Alg. \ref{alg:iGCT} provides an overview of iGCT. 

\begin{algorithm}
\caption{iGCT}
\label{alg:iGCT}
\begin{algorithmic}[1]  % Adds line numbers
\setlength{\baselineskip}{0.9\baselineskip} % Adjust line spacing
\INPUT Dataset $\mathcal{D}$, learning rate $\eta$, weighting functions $\lambda'(t), \lambda(t), \lambda_{\text{recon}}$, noise sampling function $p(t)$, noise range $[t_\text{min}, t_\text{max}]$, step function $\Delta t(t)$, guidance mask function $q(t)$, guidance range $[w_\text{min}, w_\text{max}]$, reconstruction skip iters $i_\text{skip}$, models $N_\varphi, D_\theta$
\STATE \rule{0.9\textwidth}{0.45pt}  % Horizontal line to separate input from main algorithm
\STATE \textbf{Init:} $\theta, \varphi$, $\text{Iters} = 0$
\REPEAT
\STATE Do guided consistency training 
\[
\mathcal{L}_\text{gct}(\theta;\mathcal{D},\lambda(t),p(t),t_\text{min},t_\text{max},\Delta t(t),q(t),w_\text{min},w_\text{max})
\]
\STATE Do inverse consistency training
\[
\mathcal{L}_\text{ict}(\varphi;\mathcal{D},\lambda'(t),p(t),t_\text{min},t_\text{max},\Delta t(t))
\]
\IF{$(\text{Iters} \ \% \ i_\text{skip}) == 0$}
\STATE Compute reconstruction loss
\[
\mathcal{L}_\text{recon} = d(D_{\theta}(N_{\varphi}(\boldsymbol{x}_0,t_\text{min},c),t_\text{max},c,0), \boldsymbol{x}_0)
\]
\ELSE
\STATE \[
\mathcal{L}_\text{recon} = 0
\]
\ENDIF
\STATE Compute total loss: 
\[
\mathcal{L} = \mathcal{L}_\text{gct} + \mathcal{L}_\text{ict} + \lambda_{\text{recon}}\mathcal{L}_\text{recon}
\]
\STATE $\theta \leftarrow \theta - \eta \nabla_{\theta} \mathcal{L}, \ \varphi \leftarrow \varphi - \eta \nabla_{\varphi} \mathcal{L}$
\STATE $\text{Iters} = \text{Iters} + 1$
\UNTIL{$\Delta t \rightarrow dt$}
\end{algorithmic}
\end{algorithm}



\vspace{-0.3cm}
\section{Preconditioning for Noiser}
\label{appendix:unit-variance}
\vspace{-0.1cm}

We define 
\begin{equation}
    N_{\varphi}(\boldsymbol{x}_t, t, c) = c_\text{skip}(t) \, \boldsymbol{x}_t + c_\text{out}(t) \, F_{\varphi}(c_\text{in}(t) \, \boldsymbol{x}_t, t, c),
\end{equation}
where \( c_\text{in}(t) = \frac{1}{\sqrt{t^2 + \sigma_\text{data}^2}} \), \( c_\text{skip}(t) = 1 \), and \( c_\text{out}(t) = t_\text{max} - t \). This setup naturally serves as a boundary condition. Specifically:

\begin{itemize}
    \item When \( t = 0 \),
    \begin{equation}
        c_\text{out}(0) = t_\text{max} \gg c_\text{skip}(0) = 1,
    \end{equation}
    emphasizing that the model's noise prediction dominates the residual information given a relatively clean sample.

    \item When \( t = t_\text{max} \),
    \begin{equation}
        N_{\varphi}(\boldsymbol{x}_{t_\text{max}}, t_\text{max}, c) = \boldsymbol{x}_{t_\text{max}},
    \end{equation}
    satisfying the condition that \( N_{\varphi} \) outputs \( \boldsymbol{x}_{t_\text{max}} \) at the maximum time step.
\end{itemize}



We show that these preconditions ensure unit variance for the model’s input and target. First, \(\text{Var}_{\boldsymbol{x}_0, z}[\boldsymbol{x}_t] = \sigma_\text{data}^2 + t^2\), so setting \( c_\text{in}(t) = \frac{1}{\sqrt{\sigma_\text{data}^2 + t^2}} \) normalizes the input variance to 1. Second, we require the training target to have unit variance. Given the noise target for \( N_{\varphi} \) is \(\boldsymbol{x}_{t_\text{max}} = \boldsymbol{x}_0 + t_\text{max} z\), by moving of terms, the effective target for \( F_{\varphi} \) can be written as,
\begin{equation}
    \frac{\boldsymbol{x}_{t_\text{max}} - c_\text{skip}(t)\boldsymbol{x}_{t}}{c_\text{out}(t)}
\end{equation}
When \(c_\text{skip}(t) = 1\), \(c_\text{out}(t) = t_\text{max} - t \), we verify that target is unit variance,
\begin{align}
    &\text{Var}_{\boldsymbol{x}_0, \boldsymbol{z}} \left[ \frac{\boldsymbol{x}_{t_\text{max}} - c_\text{skip}(t) \, \boldsymbol{x}_{t}}{c_\text{out}(t)} \right] \\ \notag
    = \ &\text{Var}_{\boldsymbol{x}_0, \boldsymbol{z}} \left[ \frac{\boldsymbol{x}_0 + t_\text{max} \, \boldsymbol{z} - (\boldsymbol{x}_0 + t \, \boldsymbol{z})}{t_\text{max} - t} \right] \notag \\
    = \ &\text{Var}_{\boldsymbol{x}_0, \boldsymbol{z}} \left[ \frac{(t_\text{max} - t) \, \boldsymbol{z}}{t_\text{max} - t} \right] \notag \\
    = \ &\text{Var}_{\boldsymbol{x}_0, \boldsymbol{z}}[\boldsymbol{z}] \notag \\
    = \ &1. \notag
\end{align}

\vspace{-0.3cm}
\section{Baselines \& Training Details}
\label{appendix:bs-config}
\vspace{-0.1cm}

\begin{figure}[t!]  
    \centering
    \begin{subfigure}[b]{0.33\textwidth}
    \includegraphics[width=\textwidth]{fig/appendix/guidance_embed.pdf} 
        \caption{Guidance embedding.}
    \end{subfigure}
    \hfill
    \begin{subfigure}[b]{0.33\textwidth}
    \includegraphics[width=\textwidth]{fig/appendix/adm_arch.pdf} 
        \caption{NCSN++ architecture.}
    \end{subfigure}
    \hfill
    \begin{subfigure}[b]{0.33\textwidth}
    \includegraphics[width=\textwidth]{fig/appendix/ncsnpp_arch.pdf} 
        \caption{ADM architecture.}
    \end{subfigure}
    \hfill
    \caption{Design of guidance embedding, and conditioning under different network architectures.}
    \vspace{-1em}
    \label{fig:guidance_conditioning}
\end{figure}

For our diffusion model baseline, we follow \textit{EDM}'s official repository (\href{https://github.com/NVlabs/edm}{https://github.com/NVlabs/edm}) instructions for training and set \textit{label\_dropout} to 0.1 to optimize a CFG (classifier-free guided) DM. We will use this DM as the teacher model for our consistency model baseline via consistency distillation. 

The consistency model baseline \textit{Guided CD} is trained with a discrete-time schedule. We set the discretization steps \( N = 18 \) and use a Heun ODE solver to predict update directions based on the CFG EDM, as in \cite{song2023consistency}. Following \cite{luo2023latent}, we modify the model's architecture and iGCT's denoiser to accept guidance strength \(w\) by adding an extra linear layer. See the detailed architecture design for guidance conditioning of consistency model in Fig. \ref{fig:guidance_conditioning}. A range of guidance scales \(w \in [1,15]\) is uniformly sampled at training. Following \cite{song2023improved}, we replace LPIPS by Pseudo-Huber loss, with \(c=0.03 \) determining the breadth of the smoothing section between L1 and L2. See Table \ref{tab:training_configs} for a summary of the training configurations for our baseline models.


\begin{table}[t!]
\centering
\renewcommand{\arraystretch}{1.3} % Adjust vertical spacing
\small % Reduce text size
\caption{Summary of training configurations for baseline models.}
\begin{tabular}{lccc}
\toprule
\multirow{2}{*}{} & \multicolumn{2}{c}{\textbf{CIFAR-10}} & \textbf{ImageNet64}  \\
                  & EDM & Guided-CD & EDM \\
\midrule
\multicolumn{4}{l}{\textbf{\small Arch. config.}} \\
\hline
model arch.        & NCSN++ & NCSN++ & ADM     \\
channels mult.     & 2,2,2  & 2,2,2  & 1,2,3,4 \\
UNet size          & 56.4M  & 56.4M  & 295.9M  \\
\midrule
\multicolumn{4}{l}{\textbf{\small Training config.}} \\
\hline
lr             & 1e-3  & 4e-4  & 2e-4 \\
batch          & 512   & 512   & 4096 \\
dropout        & 0.13  & 0     & 0.1 \\
label dropout  & 0.1   & (n.a.) & 0.1 \\
loss           & L2    & Huber & L2    \\
training iterations & 390k  & 800k  & 800K \\
\bottomrule
\end{tabular}
\label{tab:training_configs}
\end{table}


\begin{table}[t!]
\centering
\renewcommand{\arraystretch}{1.3} % Adjust vertical spacing
\small % Reduce text size
\caption{Summary of training configurations for iGCT.}
\begin{tabular}{lcc}
\toprule
\multirow{2}{*}{} & \textbf{CIFAR-10} & \textbf{ImageNet64}  \\
                  & iGCT & iGCT \\
\midrule
\multicolumn{3}{l}{\textbf{\small Arch. config.}} \\
\hline
model arch.        & NCSN++ & ADM \\
channels mult.     & 2,2,2  & 1,2,2,3 \\
UNet size          & 56.4M  & 182.4M \\ 
Total size         & 112.9M & 364.8M \\ 
\midrule
\multicolumn{3}{l}{\textbf{\small Training config.}} \\
\hline
lr              & 1e-4 & 1e-4 \\
batch           & 1024 & 1024 \\
dropout            & 0.2 & 0.3 \\
loss               & Huber   & Huber \\
\(c\)                  & 0.03    &  0.06 \\
\(d\)                  & 40k     &  40k \\
\( P_\textit{mean} \) & -1.1 &  -1.1 \\
\( P_\textit{std} \) &  2.0  &  2.0  \\
\( \lambda_{\text{recon}} \) & 2e-5 & \parbox[t]{3.5cm}{\centering 2e-5, (\(\leq\) 180k)\\ 4e-5, (\(\leq\) 200k)\\ 6e-5, (\(\leq\) 260k) } \\  
\( i_{\text{skip}} \)        & 10 &  10 \\  
training iterations & 360k &  260k \\
\bottomrule
\end{tabular}
\label{tab:igct_training_configs}
\end{table}  

\begin{figure*}[t] 
    \centering
    \includegraphics[width=1.0\textwidth]{fig/appendix/inversion_collapse.pdf} 
    \caption{Inversion collapse observed during training on ImageNet64. The left image shows the input data. The middle image depicts the inversion collapse that occurred at iteration 220k, where leakage of signals in the noise latent can be visualized. The right image shows the inversion results at iteration 220k after appropriately increasing $\lambda_{\text{recon}}$ to 6e-5. The inversion images are generated by scaling the model's outputs by $1/80$, i.e., $ 1/t_\text{max}$, then clipping the values to the range [-3, 3] before denormalizing them to the range [0, 255]. }
    \vspace{-1.5em}
    \label{fig:inversion_collpase}
\end{figure*}

iGCT is trained with a continuous-time scheduler inspired by ECT \cite{ect}. To rigorously assess its independence from diffusion-based models, iGCT is trained from scratch rather than fine-tuned from a pre-trained diffusion model. Consequently, the training curriculum begins with an initial diffusion training stage, followed by consistency training with the step size halved every \(d\) iterations. In practice, we adopt the same noise sampling distribution \(p(t)\), same step function \(\Delta t (t) \), and same distance metric \( d(\cdot, \cdot) \) for both guided consistency training and inverse consistency training. 

For CIFAR-10, iGCT adopts the same UNet architecture as the baseline models. However, the overall model size is doubled, as iGCT comprises two UNets: one for the denoiser and one for the noiser. The Pseudo-Huber loss is employed as the distance metric, with a constant parameter \( c = 0.03 \). Consistency training is organized into nine stages, each comprising 400k iterations with the step size halved from the last stage. We found that training remains stable when the reconstruction weight \( \lambda_{\text{recon}} \) is fixed at \( 2 \times 10^{-5} \) throughout the entire training process.
 
For ImageNet64, iGCT employs a reduced ADM architecture \cite{dhariwal2021diffusionmodelsbeatgans} with smaller channel sizes to address computational constraints. A higher dropout rate and Pseudo-Huber loss with \( c = 0.06 \) is used, following prior works \cite{ect,song2023improved}. During our experiments, we observed that training on ImageNet64 is sensitive to the reconstruction weight. Keeping \(\lambda_{\text{recon}}\) fixed throughout training leads to inversion collapse, with significant signal leaked to the latent noise (see Fig. \ref{fig:inversion_collpase}). We found that increasing \(\lambda_{\text{recon}}\) to \( 4 \times 10^{-5} \) at iteration 1800 and to \( 6 \times 10^{-5} \) at iteration 2000 effectively stabilizes training and prevents collapse. This suggests that the reconstruction loss serves as a regularizer for iGCT. Additionally, we observed diminishing returns when training exceeded 240k iterations, leading us to stop at 260k iterations for our experiments. These findings indicate that alternative training strategies, such as framing iGCT as a multi-task learning problem \cite{kendall2018multi,liu2019loss}, and conducting a more sophisticated analysis of loss weighting, may be necessary to enhance stability and improve convergence. See Table \ref{tab:igct_training_configs} for a summary of the training configurations for iGCT.



\begin{table}[t]
\caption{Comparison of GPU hours across the methods used in our experiments on CIFAR-10.}
\centering
\begin{tabular}{|l|c|}
\hline
\textbf{Methods} & \textbf{A100 (40G) GPU hours} \\ \hline
CFG-EDM \cite{karras2022elucidating} & 312 \\ \hline
Guided-CD \cite{song2023consistency} & 3968 \\ \hline
iGCT (ours) & 2032 \\ \hline
\end{tabular}
\label{table:compute_resources}
\end{table}



\begin{figure*}[t!]  
    \centering
    \begin{subfigure}[b]{0.33\textwidth}
    \includegraphics[width=\textwidth]{fig/cls_exp_w1.png} 
        \caption{Accuracy on various ratios of augmented data, guidance scale w=1.}
    \end{subfigure}
    \begin{subfigure}[b]{0.33\textwidth}
    \includegraphics[width=\textwidth]{fig/cls_exp_w3.png} 
        \caption{Accuracy on various ratios of augmented data, guidance scale w=3.}
    \end{subfigure}
    \begin{subfigure}[b]{0.33\textwidth}
    \includegraphics[width=\textwidth]{fig/cls_exp_w5.png} 
        \caption{Accuracy on various ratios of augmented data, guidance scale w=5.}
    \end{subfigure}
    \begin{subfigure}[b]{0.33\textwidth}
    \includegraphics[width=\textwidth]{fig/cls_exp_w7.png} 
        \caption{Accuracy on various ratios of augmented data, guidance scale w=7.}
    \end{subfigure}
    \begin{subfigure}[b]{0.33\textwidth}
    \includegraphics[width=\textwidth]{fig/cls_exp_w9.png} 
        \caption{Accuracy on various ratios of augmented data, guidance scale w=9.}
    \end{subfigure}
    \caption{Comparison of synthesized methods, CFG-EDM vs iGCT, used for data augmentation in image classification. iGCT consistently improves accuracy. Conversely, augmentation data synthesized from CFG-EDM offers only limited gains.}
    \vspace{-1.5em}
    \label{fig:cls_results}
\end{figure*}


\vspace{-0.1cm}
\section{Application: Data Augmentation Under Different Guidance}
\vspace{-0.2cm}

In this section, we show the effectiveness of data augmentation with diffusion-based models, CFG-EDM and iGCT, across varying guidance scales for image classification on CIFAR-10 \cite{article}. High quality data augmentation has been shown to enhance classification performance \cite{yang2023imagedataaugmentationdeep}. Under high guidance, augmentation data generated from iGCT consistently improves accuracy. Conversely, augmentation data synthesized from CFG-EDM offers only limited gains. We describe the ratios of real to synthesized data, the classifier architecture, and the training setup in the following. 

\noindent{\bf Training Details.} We conduct classification experiments trained on six different mixtures of augmented data synthesized by iGCT and CFG-EDM: \(0\%\), \(20\%\), \(40\%\), \(80\%\), and \(100\%\). The ratio represents \(\textit{synthesized data} / \textit{real data}\). For example, \(0\%\) indicates that the training and validation sets contain only 50k of real samples from CIFAR-10, and \(20\%\) includes 50k real \textit{and} 10k synthesized samples. In terms of guidance scales, we choose \(w=1,3,5,7,9\) to synthesize the augmented data using iGCT and CFG-EDM. 
The augmented dataset is split 80/20 for training and validation. For testing, the model is evaluated on the CIFAR-10 test set with 10k samples and ground truth labels. 

The standard ResNet-18 \cite{he2015deepresiduallearningimage} is used to train on all different augmented datasets. All models are trained for 250 epochs, with batch size 64, using an Adam optimizer \cite{kingma2017adammethodstochasticoptimization}. For each augmentation dataset, we train the model six times under different seeds and report the average classification accuracy.

\noindent{\bf Results.} The classifier's accuracy, trained on augmented data synthesized by CFG-EDM and iGCT, is shown in Fig. \ref{fig:cls_results}. With \(w=1\) (no guidance), both iGCT and CFG-EDM provide comparable performance boosts. As guidance scale increases, iGCT shows more significant improvements than CFG-EDM. At high guidance and augmentation ratios, performance drops, but this effect occurs later for iGCT (e.g., at \(100\%\) augmentation and \(w=9\)), while CFG-EDM stops improving accuracy at \(w=7\). This experiment highlights the importance of high-quality data under high guidance, with iGCT outperforming CFG-EDM in data quality.

\section{Uncurated Results}
In this section, we present additional qualitative results to highlight the performance of our proposed iGCT method compared to the multi-step EDM baseline. These visualizations include both inversion and guidance tasks across the CIFAR-10 and ImageNet64 datasets. The results demonstrate iGCT's ability to maintain competitive quality with significantly fewer steps and minimal artifacts, showcasing the effectiveness of our approach.

\subsection{Inversion Results}
We provide additional visualization of the latent noise on both CIFAR-10 and ImageNet64 datasets. Fig. \ref{fig:CIFAR-10_inversion_reconstruction} and Fig. \ref{fig:im64_inversion_reconstruction} compare our 1-step iGCT with the multi-step EDM on inversion and reconstruction.  

\subsection{Editing Results}
In this section, we dump more uncurated editing results on ImageNet64's subgroups mentioned in Sec. \ref{sec:image-editing}. Fig. \ref{fig:im64_edit_1}--\ref{fig:im64_edit_4} illustrate a comparison between our 1-step iGCT and the multi-step EDM approach.

\subsection{Guidance Results}
In Section \ref{sec:guidance}, we demonstrated that iGCT provides a guidance solution without introducing the high-contrast artifacts commonly observed in CFG-based methods. Here, we present additional uncurated results on CIFAR-10 and ImageNet64. For CIFAR-10, iGCT achieves competitive performance compared to the baseline diffusion model, which requires multiple steps for generation. See Figs. \ref{fig:CIFAR-10_guided_1}--\ref{fig:CIFAR-10_guided_10}. For ImageNet64, although the visual quality of iGCT's generated images falls slightly short of expectations, this can be attributed to the smaller UNet architecture used—only 61\% of the baseline model size—and the need for a more robust training curriculum to prevent collapse, as discussed in Section \ref{appendix:bs-config}. Nonetheless, even at higher guidance levels, iGCT maintains style consistency, whereas CFG-based methods continue to suffer from pronounced high-contrast artifacts. See Figs. \ref{fig:im64_guided_1}--\ref{fig:im64_guided_4}.


\begin{figure*}[t]
    \centering
    \begin{subfigure}{0.48\textwidth}
        \centering
        \includegraphics[width=\linewidth]{fig/appendix/recon_c10_data.png}
        \caption{CIFAR-10: Original data}
    \end{subfigure}
    \begin{subfigure}{0.48\textwidth}
        \centering
        \includegraphics[width=\linewidth]{fig/appendix/recon_im64_data.png}
        \caption{ImageNet64: Original data}
    \end{subfigure}

    \begin{subfigure}{0.48\textwidth}
        \centering
        \includegraphics[width=\linewidth]{fig/appendix/inv_c10_edm.png}
    \end{subfigure}
    \begin{subfigure}{0.48\textwidth}
        \centering
        \includegraphics[width=\linewidth]{fig/appendix/inv_im64_edm.png}
    \end{subfigure}

    \begin{subfigure}{0.48\textwidth}
        \centering
        \includegraphics[width=\linewidth]{fig/appendix/recon_c10_edm.png}
        \caption{CIFAR-10: Inversion + reconstruction, EDM (18 NFE)}
    \end{subfigure}
    \begin{subfigure}{0.48\textwidth}
        \centering
        \includegraphics[width=\linewidth]{fig/appendix/recon_im64_edm.png}
        \caption{ImageNet64: Inversion + reconstruction, EDM (18 NFE)}
    \end{subfigure}

    \begin{subfigure}{0.48\textwidth}
        \centering
        \includegraphics[width=\linewidth]{fig/appendix/inv_c10_igct.png}
    \end{subfigure}
    \begin{subfigure}{0.48\textwidth}
        \centering
        \includegraphics[width=\linewidth]{fig/appendix/inv_im64_igct.png}
    \end{subfigure}

    \begin{subfigure}{0.48\textwidth}
        \centering
        \includegraphics[width=\linewidth]{fig/appendix/recon_c10_igct.png}
        \caption{CIFAR-10: Inversion + reconstruction, iGCT (1 NFE)}
    \end{subfigure}
    \begin{subfigure}{0.48\textwidth}
        \centering
        \includegraphics[width=\linewidth]{fig/appendix/recon_im64_igct.png}
        \caption{ImageNet64: Inversion + reconstruction, iGCT (1 NFE)}
    \end{subfigure}

    \caption{Comparison of inversion and reconstruction for CIFAR-10 (left) and ImageNet64 (right).}
    \label{fig:comparison_CIFAR-10_imagenet64}
\end{figure*}




\begin{figure*}[t]
    \centering

    % Left column: corgi -> golden retriever
    \begin{minipage}{0.48\textwidth}
        \centering
        \begin{subfigure}{0.48\textwidth}
            \includegraphics[width=\linewidth]{fig/appendix_edit_igct/src_corgi.png}
            \caption{Original: "corgi"}
        \end{subfigure}

        \begin{subfigure}{0.48\textwidth}
            \includegraphics[width=\linewidth]{fig/appendix_edit_edm/w=0_src_corgi_tar_golden_retriever.png}
            \caption{EDM (18 NFE), w=1}
        \end{subfigure}
        \begin{subfigure}{0.48\textwidth}
            \includegraphics[width=\linewidth]{fig/appendix_edit_edm/w=6_src_corgi_tar_golden_retriever.png}
            \caption{EDM (18 NFE), w=7}
        \end{subfigure}
        \begin{subfigure}{0.48\textwidth}
            \includegraphics[width=\linewidth]{fig/appendix_edit_igct/w=6_src_corgi_tar_golden_retriever.png}
            \caption{iGCT (1 NFE), w=7}
        \end{subfigure}
        \begin{subfigure}{0.48\textwidth}
            \includegraphics[width=\linewidth]{fig/appendix_edit_igct/w=0_src_corgi_tar_golden_retriever.png}
            \caption{iGCT (1 NFE), w=1}
        \end{subfigure}

        \caption{ImageNet64: "corgi" $\rightarrow$ "golden retriever"}
        \label{fig:im64_edit_1}
    \end{minipage}
    \hfill
    % Right column: zebra -> horse
    \begin{minipage}{0.48\textwidth}
        \centering
        \begin{subfigure}{0.48\textwidth}
            \includegraphics[width=\linewidth]{fig/appendix_edit_igct/src_zebra.png}
            \caption{Original: "zebra"}
        \end{subfigure}

        \begin{subfigure}{0.48\textwidth}
            \includegraphics[width=\linewidth]{fig/appendix_edit_edm/w=0_src_zebra_tar_horse.png}
            \caption{EDM (18 NFE), w=1}
        \end{subfigure}
        \begin{subfigure}{0.48\textwidth}
            \includegraphics[width=\linewidth]{fig/appendix_edit_edm/w=6_src_zebra_tar_horse.png}
            \caption{EDM (18 NFE), w=7}
        \end{subfigure}
        \begin{subfigure}{0.48\textwidth}
            \includegraphics[width=\linewidth]{fig/appendix_edit_igct/w=0_src_zebra_tar_horse.png}
            \caption{iGCT (1 NFE), w=1}
        \end{subfigure}
        \begin{subfigure}{0.48\textwidth}
            \includegraphics[width=\linewidth]{fig/appendix_edit_igct/w=6_src_zebra_tar_horse.png}
            \caption{iGCT (1 NFE), w=7}
        \end{subfigure}

        \caption{ImageNet64: "zebra" $\rightarrow$ "horse"}
        \label{fig:im64_edit_2}
    \end{minipage}

\end{figure*}

\begin{figure*}[t]
    \centering

    % Left column: broccoli -> cauliflower
    \begin{minipage}{0.48\textwidth}
        \centering
        \begin{subfigure}{0.48\textwidth}
            \includegraphics[width=\linewidth]{fig/appendix_edit_igct/src_broccoli.png}
            \caption{Original: "broccoli"}
        \end{subfigure}

        \begin{subfigure}{0.48\textwidth}
            \includegraphics[width=\linewidth]{fig/appendix_edit_edm/w=0_src_broccoli_tar_cauliflower.png}
            \caption{EDM (18 NFE), w=1}
        \end{subfigure}
        \begin{subfigure}{0.48\textwidth}
            \includegraphics[width=\linewidth]{fig/appendix_edit_edm/w=6_src_broccoli_tar_cauliflower.png}
            \caption{EDM (18 NFE), w=7}
        \end{subfigure}
        \begin{subfigure}{0.48\textwidth}
            \includegraphics[width=\linewidth]{fig/appendix_edit_igct/w=0_src_broccoli_tar_cauliflower.png}
            \caption{iGCT (1 NFE), w=1}
        \end{subfigure}
        \begin{subfigure}{0.48\textwidth}
            \includegraphics[width=\linewidth]{fig/appendix_edit_igct/w=6_src_broccoli_tar_cauliflower.png}
            \caption{iGCT (1 NFE), w=7}
        \end{subfigure}

        \caption{ImageNet64: "broccoli" $\rightarrow$ "cauliflower"}
        \label{fig:im64_edit_3}
    \end{minipage}
    \hfill
    % Right column: jaguar -> tiger
    \begin{minipage}{0.48\textwidth}
        \centering
        \begin{subfigure}{0.48\textwidth}
            \includegraphics[width=\linewidth]{fig/appendix_edit_igct/src_jaguar.png}
            \caption{Original: "jaguar"}
        \end{subfigure}

        \begin{subfigure}{0.48\textwidth}
            \includegraphics[width=\linewidth]{fig/appendix_edit_edm/w=0_src_jaguar_tar_tiger.png}
            \caption{EDM (18 NFE), w=1}
        \end{subfigure}
        \begin{subfigure}{0.48\textwidth}
            \includegraphics[width=\linewidth]{fig/appendix_edit_edm/w=6_src_jaguar_tar_tiger.png}
            \caption{EDM (18 NFE), w=7}
        \end{subfigure}
        \begin{subfigure}{0.48\textwidth}
            \includegraphics[width=\linewidth]{fig/appendix_edit_igct/w=0_src_jaguar_tar_tiger.png}
            \caption{iGCT (1 NFE), w=1}
        \end{subfigure}
        \begin{subfigure}{0.48\textwidth}
            \includegraphics[width=\linewidth]{fig/appendix_edit_igct/w=6_src_jaguar_tar_tiger.png}
            \caption{iGCT (1 NFE), w=7}
        \end{subfigure}

        \caption{ImageNet64: "jaguar" $\rightarrow$ "tiger"}
        \label{fig:im64_edit_4}
    \end{minipage}

\end{figure*}






\begin{figure*}[b]
    \centering
    % First image
    \begin{subfigure}{0.25\textwidth}
        \includegraphics[width=\linewidth]{fig/appendix_edm/0_0.0_middle_4x4_grid.png}
        \caption{CFG-EDM (18 NFE), w=1.0}
    \end{subfigure}
    \begin{subfigure}{0.25\textwidth}
        \includegraphics[width=\linewidth]{fig/appendix_edm/0_6.0_middle_4x4_grid.png}
        \caption{CFG-EDM (18 NFE), w=7.0}
    \end{subfigure}
    \begin{subfigure}{0.25\textwidth}
        \includegraphics[width=\linewidth]{fig/appendix_edm/0_12.0_middle_4x4_grid.png}
        \caption{CFG-EDM (18 NFE), w=13.0}
    \end{subfigure}
    \begin{subfigure}{0.25\textwidth}
        \includegraphics[width=\linewidth]{fig/appendix_igct/0_0.0_middle_4x4_grid.png}
        \caption{iGCT (1 NFE), w=1.0}
    \end{subfigure}
    \begin{subfigure}{0.25\textwidth}
        \includegraphics[width=\linewidth]{fig/appendix_igct/0_6.0_middle_4x4_grid.png}
        \caption{iGCT (1 NFE), w=7.0}
    \end{subfigure}
    % Third image
    \begin{subfigure}{0.25\textwidth}
        \includegraphics[width=\linewidth]{fig/appendix_igct/0_12.0_middle_4x4_grid.png}
        \caption{iGCT (1 NFE), w=13.0}
    \end{subfigure}
    \caption{CIFAR-10 "airplane"}
    \label{fig:CIFAR-10_guided_1}
\end{figure*}
\begin{figure*}[t]
    \centering
    % First image
    \begin{subfigure}{0.25\textwidth}
        \includegraphics[width=\linewidth]{fig/appendix_edm/1_0.0_middle_4x4_grid.png}
        \caption{CFG-EDM (18 NFE), w=1.0}
    \end{subfigure}
    \begin{subfigure}{0.25\textwidth}
        \includegraphics[width=\linewidth]{fig/appendix_edm/1_6.0_middle_4x4_grid.png}
        \caption{CFG-EDM (18 NFE), w=7.0}
    \end{subfigure}
    \begin{subfigure}{0.25\textwidth}
        \includegraphics[width=\linewidth]{fig/appendix_edm/1_12.0_middle_4x4_grid.png}
        \caption{CFG-EDM (18 NFE), w=13.0}
    \end{subfigure}
    \begin{subfigure}{0.25\textwidth}
        \includegraphics[width=\linewidth]{fig/appendix_igct/1_0.0_middle_4x4_grid.png}
        \caption{iGCT (1 NFE), w=1.0}
    \end{subfigure}
    % Second image
    \begin{subfigure}{0.25\textwidth}
        \includegraphics[width=\linewidth]{fig/appendix_igct/1_6.0_middle_4x4_grid.png}
        \caption{iGCT (1 NFE), w=7.0}
    \end{subfigure}
    % Third image
    \begin{subfigure}{0.25\textwidth}
        \includegraphics[width=\linewidth]{fig/appendix_igct/1_12.0_middle_4x4_grid.png}
        \caption{iGCT (1 NFE), w=13.0}
    \end{subfigure}
    \caption{CIFAR-10 "car"}
    \label{fig:CIFAR-10_guided_2}
\end{figure*}
\begin{figure*}[t]
    \centering
    % First image
    \begin{subfigure}{0.25\textwidth}
        \includegraphics[width=\linewidth]{fig/appendix_edm/2_0.0_middle_4x4_grid.png}
        \caption{CFG-EDM (18 NFE), w=1.0}
    \end{subfigure}
    \begin{subfigure}{0.25\textwidth}
        \includegraphics[width=\linewidth]{fig/appendix_edm/2_6.0_middle_4x4_grid.png}
        \caption{CFG-EDM (18 NFE), w=7.0}
    \end{subfigure}
    \begin{subfigure}{0.25\textwidth}
        \includegraphics[width=\linewidth]{fig/appendix_edm/2_12.0_middle_4x4_grid.png}
        \caption{CFG-EDM (18 NFE), w=13.0}
    \end{subfigure}
    \begin{subfigure}{0.25\textwidth}
        \includegraphics[width=\linewidth]{fig/appendix_igct/2_0.0_middle_4x4_grid.png}
        \caption{iGCT (1 NFE), w=1.0}
    \end{subfigure}
    % Second image
    \begin{subfigure}{0.25\textwidth}
        \includegraphics[width=\linewidth]{fig/appendix_igct/2_6.0_middle_4x4_grid.png}
        \caption{iGCT (1 NFE), w=7.0}
    \end{subfigure}
    % Third image
    \begin{subfigure}{0.25\textwidth}
        \includegraphics[width=\linewidth]{fig/appendix_igct/2_12.0_middle_4x4_grid.png}
        \caption{iGCT (1 NFE), w=13.0}
    \end{subfigure}
    \caption{CIFAR-10 "bird"}
    \label{fig:CIFAR-10_guided_3}
\end{figure*}
\begin{figure*}[t]
    \centering
    % First image
    \begin{subfigure}{0.25\textwidth}
        \includegraphics[width=\linewidth]{fig/appendix_edm/3_0.0_middle_4x4_grid.png}
        \caption{CFG-EDM (18 NFE), w=1.0}
    \end{subfigure}
    \begin{subfigure}{0.25\textwidth}
        \includegraphics[width=\linewidth]{fig/appendix_edm/3_6.0_middle_4x4_grid.png}
        \caption{CFG-EDM (18 NFE), w=7.0}
    \end{subfigure}
    \begin{subfigure}{0.25\textwidth}
        \includegraphics[width=\linewidth]{fig/appendix_edm/3_12.0_middle_4x4_grid.png}
        \caption{CFG-EDM (18 NFE), w=13.0}
    \end{subfigure}
    \begin{subfigure}{0.25\textwidth}
        \includegraphics[width=\linewidth]{fig/appendix_igct/3_0.0_middle_4x4_grid.png}
        \caption{iGCT (1 NFE), w=1.0}
    \end{subfigure}
    % Second image
    \begin{subfigure}{0.25\textwidth}
        \includegraphics[width=\linewidth]{fig/appendix_igct/3_6.0_middle_4x4_grid.png}
        \caption{iGCT (1 NFE), w=7.0}
    \end{subfigure}
    % Third image
    \begin{subfigure}{0.25\textwidth}
        \includegraphics[width=\linewidth]{fig/appendix_igct/3_12.0_middle_4x4_grid.png}
        \caption{iGCT (1 NFE), w=13.0}
    \end{subfigure}
    \caption{CIFAR-10 "cat"}
    \label{fig:CIFAR-10_guided_4}
\end{figure*}
\begin{figure*}[t]
    \centering
    % First image
    \begin{subfigure}{0.25\textwidth}
        \includegraphics[width=\linewidth]{fig/appendix_edm/4_0.0_middle_4x4_grid.png}
        \caption{CFG-EDM (18 NFE), w=1.0}
    \end{subfigure}
    \begin{subfigure}{0.25\textwidth}
        \includegraphics[width=\linewidth]{fig/appendix_edm/4_6.0_middle_4x4_grid.png}
        \caption{CFG-EDM (18 NFE), w=7.0}
    \end{subfigure}
    \begin{subfigure}{0.25\textwidth}
        \includegraphics[width=\linewidth]{fig/appendix_edm/4_12.0_middle_4x4_grid.png}
        \caption{CFG-EDM (18 NFE), w=13.0}
    \end{subfigure}
    \begin{subfigure}{0.25\textwidth}
        \includegraphics[width=\linewidth]{fig/appendix_igct/4_0.0_middle_4x4_grid.png}
        \caption{iGCT (1 NFE), w=1.0}
    \end{subfigure}
    % Second image
    \begin{subfigure}{0.25\textwidth}
        \includegraphics[width=\linewidth]{fig/appendix_igct/4_6.0_middle_4x4_grid.png}
        \caption{iGCT (1 NFE), w=7.0}
    \end{subfigure}
    % Third image
    \begin{subfigure}{0.25\textwidth}
        \includegraphics[width=\linewidth]{fig/appendix_igct/4_12.0_middle_4x4_grid.png}
        \caption{iGCT (1 NFE), w=13.0}
    \end{subfigure}
    \caption{CIFAR-10 "deer"}
    \label{fig:CIFAR-10_guided_5}
\end{figure*}
\begin{figure*}[t]
    \centering
    % First image
    \begin{subfigure}{0.25\textwidth}
        \includegraphics[width=\linewidth]{fig/appendix_edm/5_0.0_middle_4x4_grid.png}
        \caption{CFG-EDM (18 NFE), w=1.0}
    \end{subfigure}
    \begin{subfigure}{0.25\textwidth}
        \includegraphics[width=\linewidth]{fig/appendix_edm/5_6.0_middle_4x4_grid.png}
        \caption{CFG-EDM (18 NFE), w=7.0}
    \end{subfigure}
    \begin{subfigure}{0.25\textwidth}
        \includegraphics[width=\linewidth]{fig/appendix_edm/5_12.0_middle_4x4_grid.png}
        \caption{CFG-EDM (18 NFE), w=13.0}
    \end{subfigure}
    \begin{subfigure}{0.25\textwidth}
        \includegraphics[width=\linewidth]{fig/appendix_igct/5_0.0_middle_4x4_grid.png}
        \caption{iGCT (1 NFE), w=1.0}
    \end{subfigure}
    % Second image
    \begin{subfigure}{0.25\textwidth}
        \includegraphics[width=\linewidth]{fig/appendix_igct/5_6.0_middle_4x4_grid.png}
        \caption{iGCT (1 NFE), w=7.0}
    \end{subfigure}
    % Third image
    \begin{subfigure}{0.25\textwidth}
        \includegraphics[width=\linewidth]{fig/appendix_igct/5_12.0_middle_4x4_grid.png}
        \caption{iGCT (1 NFE), w=13.0}
    \end{subfigure}
    \caption{CIFAR-10 "dog"}
    \label{fig:CIFAR-10_guided_6}
\end{figure*}
\begin{figure*}[t]
    \centering
    % First image
    \begin{subfigure}{0.25\textwidth}
        \includegraphics[width=\linewidth]{fig/appendix_edm/6_0.0_middle_4x4_grid.png}
        \caption{CFG-EDM (18 NFE), w=1.0}
    \end{subfigure}
    \begin{subfigure}{0.25\textwidth}
        \includegraphics[width=\linewidth]{fig/appendix_edm/6_6.0_middle_4x4_grid.png}
        \caption{CFG-EDM (18 NFE), w=7.0}
    \end{subfigure}
    \begin{subfigure}{0.25\textwidth}
        \includegraphics[width=\linewidth]{fig/appendix_edm/6_12.0_middle_4x4_grid.png}
        \caption{CFG-EDM (18 NFE), w=13.0}
    \end{subfigure}
    \begin{subfigure}{0.25\textwidth}
        \includegraphics[width=\linewidth]{fig/appendix_igct/6_0.0_middle_4x4_grid.png}
        \caption{iGCT (1 NFE), w=1.0}
    \end{subfigure}
    % Second image
    \begin{subfigure}{0.25\textwidth}
        \includegraphics[width=\linewidth]{fig/appendix_igct/6_6.0_middle_4x4_grid.png}
        \caption{iGCT (1 NFE), w=7.0}
    \end{subfigure}
    % Third image
    \begin{subfigure}{0.25\textwidth}
        \includegraphics[width=\linewidth]{fig/appendix_igct/6_12.0_middle_4x4_grid.png}
        \caption{iGCT (1 NFE), w=13.0}
    \end{subfigure}
    \caption{CIFAR-10 "frog"}
    \label{fig:CIFAR-10_guided_7}
\end{figure*}
\begin{figure*}[t]
    \centering
    % First image
    \begin{subfigure}{0.25\textwidth}
        \includegraphics[width=\linewidth]{fig/appendix_edm/7_0.0_middle_4x4_grid.png}
        \caption{CFG-EDM (18 NFE), w=1.0}
    \end{subfigure}
    \begin{subfigure}{0.25\textwidth}
        \includegraphics[width=\linewidth]{fig/appendix_edm/7_6.0_middle_4x4_grid.png}
        \caption{CFG-EDM (18 NFE), w=7.0}
    \end{subfigure}
    \begin{subfigure}{0.25\textwidth}
        \includegraphics[width=\linewidth]{fig/appendix_edm/7_12.0_middle_4x4_grid.png}
        \caption{CFG-EDM (18 NFE), w=13.0}
    \end{subfigure}
    \begin{subfigure}{0.25\textwidth}
        \includegraphics[width=\linewidth]{fig/appendix_igct/7_0.0_middle_4x4_grid.png}
        \caption{iGCT (1 NFE), w=1.0}
    \end{subfigure}
    % Second image
    \begin{subfigure}{0.25\textwidth}
        \includegraphics[width=\linewidth]{fig/appendix_igct/7_6.0_middle_4x4_grid.png}
        \caption{iGCT (1 NFE), w=7.0}
    \end{subfigure}
    % Third image
    \begin{subfigure}{0.25\textwidth}
        \includegraphics[width=\linewidth]{fig/appendix_igct/7_12.0_middle_4x4_grid.png}
        \caption{iGCT (1 NFE), w=13.0}
    \end{subfigure}
    \caption{CIFAR-10 "horse"}
    \label{fig:CIFAR-10_guided_8}
\end{figure*}
\begin{figure*}[t]
    \centering
    % First image
    \begin{subfigure}{0.25\textwidth}
        \includegraphics[width=\linewidth]{fig/appendix_edm/8_0.0_middle_4x4_grid.png}
        \caption{CFG-EDM (18 NFE), w=1.0}
    \end{subfigure}
    \begin{subfigure}{0.25\textwidth}
        \includegraphics[width=\linewidth]{fig/appendix_edm/8_6.0_middle_4x4_grid.png}
        \caption{CFG-EDM (18 NFE), w=7.0}
    \end{subfigure}
    \begin{subfigure}{0.25\textwidth}
        \includegraphics[width=\linewidth]{fig/appendix_edm/8_12.0_middle_4x4_grid.png}
        \caption{CFG-EDM (18 NFE), w=13.0}
    \end{subfigure}
    \begin{subfigure}{0.25\textwidth}
        \includegraphics[width=\linewidth]{fig/appendix_igct/8_0.0_middle_4x4_grid.png}
        \caption{iGCT (1 NFE), w=1.0}
    \end{subfigure}
    % Second image
    \begin{subfigure}{0.25\textwidth}
        \includegraphics[width=\linewidth]{fig/appendix_igct/8_6.0_middle_4x4_grid.png}
        \caption{iGCT (1 NFE), w=7.0}
    \end{subfigure}
    % Third image
    \begin{subfigure}{0.25\textwidth}
        \includegraphics[width=\linewidth]{fig/appendix_igct/8_12.0_middle_4x4_grid.png}
        \caption{iGCT (1 NFE), w=13.0}
    \end{subfigure}
    \caption{CIFAR-10 "ship"}
    \label{fig:CIFAR-10_guided_9}
\end{figure*}
\begin{figure*}[t]
    \centering
    % First image
    \begin{subfigure}{0.25\textwidth}
        \includegraphics[width=\linewidth]{fig/appendix_edm/9_0.0_middle_4x4_grid.png}
        \caption{CFG-EDM (18 NFE), w=1.0}
    \end{subfigure}
    \begin{subfigure}{0.25\textwidth}
        \includegraphics[width=\linewidth]{fig/appendix_edm/9_6.0_middle_4x4_grid.png}
        \caption{CFG-EDM (18 NFE), w=7.0}
    \end{subfigure}
    \begin{subfigure}{0.25\textwidth}
        \includegraphics[width=\linewidth]{fig/appendix_edm/9_12.0_middle_4x4_grid.png}
        \caption{CFG-EDM (18 NFE), w=13.0}
    \end{subfigure}
    \begin{subfigure}{0.25\textwidth}
        \includegraphics[width=\linewidth]{fig/appendix_igct/9_0.0_middle_4x4_grid.png}
        \caption{iGCT (1 NFE), w=1.0}
    \end{subfigure}
    % Second image
    \begin{subfigure}{0.25\textwidth}
        \includegraphics[width=\linewidth]{fig/appendix_igct/9_6.0_middle_4x4_grid.png}
        \caption{iGCT (1 NFE), w=7.0}
    \end{subfigure}
    % Third image
    \begin{subfigure}{0.25\textwidth}
        \includegraphics[width=\linewidth]{fig/appendix_igct/9_12.0_middle_4x4_grid.png}
        \caption{iGCT (1 NFE), w=13.0}
    \end{subfigure}
    \caption{CIFAR-10 "truck"}
    \label{fig:CIFAR-10_guided_10}
\end{figure*}


\begin{figure*}[b]
    \centering
    % First image
    \begin{subfigure}{0.25\textwidth}
        \includegraphics[width=\linewidth]{fig/appendix_im64_edm/edm_class_291_w=0.0.png}
        \caption{CFG-EDM (18 NFE), w=1.0}
    \end{subfigure}
    \begin{subfigure}{0.25\textwidth}
        \includegraphics[width=\linewidth]{fig/appendix_im64_edm/edm_class_291_w=6.0.png}
        \caption{CFG-EDM (18 NFE), w=7.0}
    \end{subfigure}
    \begin{subfigure}{0.25\textwidth}
        \includegraphics[width=\linewidth]{fig/appendix_im64_edm/edm_class_291_w=12.0.png}
        \caption{CFG-EDM (18 NFE), w=13.0}
    \end{subfigure}
    \begin{subfigure}{0.25\textwidth}
        \includegraphics[width=\linewidth]{fig/appendix_im64_igct/class_291_w=0.0.png}
        \caption{iGCT (2 NFE), w=1.0}
    \end{subfigure}
    \begin{subfigure}{0.25\textwidth}
        \includegraphics[width=\linewidth]{fig/appendix_im64_igct/class_291_w=6.0.png}
        \caption{iGCT (2 NFE), w=7.0}
    \end{subfigure}
    % Third image
    \begin{subfigure}{0.25\textwidth}
        \includegraphics[width=\linewidth]{fig/appendix_im64_igct/class_291_w=12.0.png}
        \caption{iGCT (2 NFE), w=13.0}
    \end{subfigure}
    \caption{ImageNet64 "lion"}
    \label{fig:im64_guided_1}
\end{figure*}



\begin{figure*}[b]
    \centering
    % First image
    \begin{subfigure}{0.25\textwidth}
        \includegraphics[width=\linewidth]{fig/appendix_im64_edm/edm_class_292_w=0.0.png}
        \caption{CFG-EDM (18 NFE), w=1.0}
    \end{subfigure}
    \begin{subfigure}{0.25\textwidth}
        \includegraphics[width=\linewidth]{fig/appendix_im64_edm/edm_class_292_w=6.0.png}
        \caption{CFG-EDM (18 NFE), w=7.0}
    \end{subfigure}
    \begin{subfigure}{0.25\textwidth}
        \includegraphics[width=\linewidth]{fig/appendix_im64_edm/edm_class_292_w=12.0.png}
        \caption{CFG-EDM (18 NFE), w=13.0}
    \end{subfigure}
    \begin{subfigure}{0.25\textwidth}
        \includegraphics[width=\linewidth]{fig/appendix_im64_igct/class_292_w=0.0.png}
        \caption{iGCT (2 NFE), w=1.0}
    \end{subfigure}
    \begin{subfigure}{0.25\textwidth}
        \includegraphics[width=\linewidth]{fig/appendix_im64_igct/class_292_w=6.0.png}
        \caption{iGCT (2 NFE), w=7.0}
    \end{subfigure}
    % Third image
    \begin{subfigure}{0.25\textwidth}
        \includegraphics[width=\linewidth]{fig/appendix_im64_igct/class_292_w=12.0.png}
        \caption{iGCT (2 NFE), w=13.0}
    \end{subfigure}
    \caption{ImageNet64 "tiger"}
    \label{fig:im64_guided_2}
\end{figure*}


\begin{figure*}[b]
    \centering
    % First image
    \begin{subfigure}{0.25\textwidth}
        \includegraphics[width=\linewidth]{fig/appendix_im64_edm/edm_class_28_w=0.0.png}
        \caption{CFG-EDM (18 NFE), w=1.0}
    \end{subfigure}
    \begin{subfigure}{0.25\textwidth}
        \includegraphics[width=\linewidth]{fig/appendix_im64_edm/edm_class_28_w=6.0.png}
        \caption{CFG-EDM (18 NFE), w=7.0}
    \end{subfigure}
    \begin{subfigure}{0.25\textwidth}
        \includegraphics[width=\linewidth]{fig/appendix_im64_edm/edm_class_28_w=12.0.png}
        \caption{CFG-EDM (18 NFE), w=13.0}
    \end{subfigure}
    \begin{subfigure}{0.25\textwidth}
        \includegraphics[width=\linewidth]{fig/appendix_im64_igct/class_28_w=0.0.png}
        \caption{iGCT (2 NFE), w=1.0}
    \end{subfigure}
    \begin{subfigure}{0.25\textwidth}
        \includegraphics[width=\linewidth]{fig/appendix_im64_igct/class_28_w=6.0.png}
        \caption{iGCT (2 NFE), w=7.0}
    \end{subfigure}
    % Third image
    \begin{subfigure}{0.25\textwidth}
        \includegraphics[width=\linewidth]{fig/appendix_im64_igct/class_28_w=12.0.png}
        \caption{iGCT (2 NFE), w=13.0}
    \end{subfigure}
    \caption{ImageNet64 "salamander"}
    \label{fig:im64_guided_3}
\end{figure*}


\begin{figure*}[b]
    \centering
    % First image
    \begin{subfigure}{0.25\textwidth}
        \includegraphics[width=\linewidth]{fig/appendix_im64_edm/edm_class_407_w=0.0.png}
        \caption{CFG-EDM (18 NFE), w=1.0}
    \end{subfigure}
    \begin{subfigure}{0.25\textwidth}
        \includegraphics[width=\linewidth]{fig/appendix_im64_edm/edm_class_407_w=6.0.png}
        \caption{CFG-EDM (18 NFE), w=7.0}
    \end{subfigure}
    \begin{subfigure}{0.25\textwidth}
        \includegraphics[width=\linewidth]{fig/appendix_im64_edm/edm_class_407_w=12.0.png}
        \caption{CFG-EDM (18 NFE), w=13.0}
    \end{subfigure}
    \begin{subfigure}{0.25\textwidth}
        \includegraphics[width=\linewidth]{fig/appendix_im64_igct/class_407_w=0.0.png}
        \caption{iGCT (2 NFE), w=1.0}
    \end{subfigure}
    \begin{subfigure}{0.25\textwidth}
        \includegraphics[width=\linewidth]{fig/appendix_im64_igct/class_407_w=6.0.png}
        \caption{iGCT (2 NFE), w=7.0}
    \end{subfigure}
    % Third image
    \begin{subfigure}{0.25\textwidth}
        \includegraphics[width=\linewidth]{fig/appendix_im64_igct/class_407_w=12.0.png}
        \caption{iGCT (2 NFE), w=13.0}
    \end{subfigure}
    \caption{ImageNet64 "ambulance"}
    \label{fig:im64_guided_4}
\end{figure*}

\end{document}
