\documentclass[12pt]{article}%
\usepackage[letterpaper, top=1in, bottom=1in, left=1in, right=1in]{geometry}
\usepackage{amsfonts, amsmath, amsthm, amssymb}
\usepackage{enumitem, graphicx, stmaryrd, color, bbm}
\usepackage{url, algorithm}
\usepackage{dsfont}
\usepackage{float}
\floatstyle{ruled}

\usepackage[driverfallback=hypertex,pagebackref=true,colorlinks]{hyperref}
	\hypersetup{linkcolor=[rgb]{.7,0,.7}}
	\hypersetup{citecolor=[rgb]{.5,0,.5}}
	\hypersetup{urlcolor=[rgb]{.7,0,.7}}

\newtheorem{theorem}{Theorem}
\newtheorem{lemma}[theorem]{Lemma}
\newtheorem{claim}[theorem]{Claim}
\newtheorem{definition}[theorem]{Definition}
\newtheorem{observation}[theorem]{Observation}
\newtheorem{conjecture}[theorem]{Conjecture}
\newtheorem{corollary}[theorem]{Corollary}
\newtheorem{remark}[theorem]{Remark}
\newtheorem*{remarks}{Remarks}
\newtheorem{proposition}[theorem]{Proposition}
\newtheorem{example}[theorem]{Example}
\newtheorem{question}[theorem]{Question}
\newtheorem{fact}[theorem]{Fact}

\renewcommand{\implies}{\Longrightarrow}		% implies
\newcommand{\abs}[1]{\left|#1\right|}		% absolute value
\newcommand{\E}{\mathop{\mathbb{E}}}  		% expectation
\newcommand{\V}{\mathop{\mathbb{V}}}  		% variance
\newcommand{\R}{\mathop{\mathbb{R}}}  		% real numbers
\newcommand{\N}{\mathbb{N}} 			  		% natural numbers
\newcommand{\F}{\mathbb{F}}					% fields
\newcommand{\Z}{\mathbb{Z}}					% integers
\newcommand{\C}{\mathbb{C}}					% complex numbers

\newcommand{\mc}{\mathcal}
\newcommand{\mb}{\mathbb}

\newcommand{\eps}{\mathop{\epsilon}}
\newcommand{\inv}{{^{-1}}}                  % inverse
\newcommand{\set}[1]{\left\{ #1 \right\}}   % sets
\newcommand{\ip}[1]{\langle #1 \rangle}     % inner product
\newcommand{\brac}[1]{\left( #1 \right)}    % brackets
\newcommand{\sqbrac}[1]{\left[ #1 \right]}  % square brackets

\newcommand{\Inf}{\textnormal{Inf}} 				% influence
\newcommand{\I}{\mathbb{I}} 					% total influence (boolean analysis)
\newcommand{\Stab}{\textnormal{Stab}} 				% stability
\newcommand{\Var}{\textnormal{Var}}				% variance
\newcommand{\AND}{\textnormal{AND}}				% AND
\newcommand{\OR}{\textnormal{OR}} 	 				% OR

\newcommand{\todo}[1]{\textcolor{red}{TODO: #1}}

\newcommand{\norm}[1]{\left\vert #1 \right\vert}			% Single lined norm
\newcommand{\Norm}[1]{\left\Vert #1 \right\Vert}			% Norm
\newcommand{\Normlo}[1]{\left\Vert #1 \right\Vert_1}		% L1 Norm
\newcommand{\Normlt}[1]{\left\Vert #1 \right\Vert_2}		% L2 Norm
\newcommand{\up}[1]{^{\brac{#1}}}						

\DeclareMathOperator*{\argmin}{arg\,min}
\DeclareMathOperator*{\argmax}{arg\,max}

% Complexity shortcuts

\newcommand{\DTIME}{\mathrm{DTIME}}
\newcommand{\NTIME}{\mathrm{NTIME}}
\newcommand{\cP}{\mathrm{P}}
\newcommand{\NP}{\mathrm{NP}}
\newcommand{\RP}{\mathrm{RP}}
\newcommand{\BPP}{\mathrm{BPP}}
\newcommand{\Ppoly}{\mathrm{P/poly}}
\newcommand{\ind}{\mathds{1}}

\newcommand{\sgn}{\textnormal{sgn}} 	 	
\newcommand{\id}{\textnormal{I}} 
\newcommand{\dx}{\textnormal{dx}} 	
\newcommand{\supp}{\textnormal{supp}} 	
\newcommand{\modt}{\textnormal{ (mod 2)}} 
\newcommand{\modk}{\textnormal{ (mod $k-1$)}} 
\newcommand{\Lin}{\textnormal{Lin}} 
\newcommand{\Sym}{\textnormal{Sym}}  

\let\savedleq=\leq % save them in case you ever want them
\let\savedgeq=\geq %
\let\leq=\leqslant % redefine them
\let\geq=\geqslant %

\newcommand{\Kunal}[1]{{\color{red}[Kunal: #1]}}



%%%%%%%%%%%%%%%%%%%%%%%%%%%%%%%%%%%%%%%%%%%%%%%%%%%%%%%%%%%%%%%%%%%%%%%%%%%%%%%%
\begin{document}

%\sloppy

\title{Biased Linearity Testing in the 1\% Regime}
%\author{}
\author{Subhash Khot\thanks{Department of Computer Science, Courant Institute of Mathematical Sciences, New York University. E-mail: \href{khot@cs.nyu.edu}{\texttt{khot@cs.nyu.edu}.} Research supported by NSF Award CCF-1422159, NSF Award CCF-2130816, and the Simons Investigator Award.} \and Kunal Mittal\thanks{Department of Computer Science, Princeton University. E-mail: \href{kmittal@cs.princeton.edu}{\texttt{kmittal@cs.princeton.edu}.} Research supported by NSF Award CCF-2007462, and the Simons Investigator Award.}}
\date{\today}			 					% \today for date now
\maketitle
%\vspace{-1em}
\begin{abstract}
We study linearity testing over the $p$-biased hypercube $(\{0,1\}^n, \mu_p^{\otimes n})$ in the 1\% regime.
For a distribution $\nu$ supported over $\{x\in \{0,1\}^k:\sum_{i=1}^k x_i=0 \textnormal{ (mod 2)} \}$, with marginal distribution $\mu_p$ in each coordinate, the corresponding $k$-query linearity test $\textnormal{Lin}(\nu)$ proceeds as follows:
Given query access to a function $f:\{0,1\}^n\to \{-1,1\}$, sample $(x_1,\dots,x_k)\sim \nu^{\otimes n}$, query $f$ on $x_1,\dots,x_k$,  and accept if and only if $\prod_{i\in [k]}f(x_i)=1$.
	
Building on the work of Bhangale, Khot, and Minzer (STOC '23), we show, for $0<p\leq \frac{1}{2}$, that if $k \geq  1+\frac{1}{p}$, then there exists a distribution $\nu$ such that the test $\textnormal{Lin}(\nu)$ works in the 1\% regime; that is, any function $f:\{0,1\}^n\to \{-1,1\}$ passing the test $\textnormal{Lin}(\nu)$ with probability $\geq \frac{1}{2}+\epsilon$, for some constant $\epsilon>0$, satisfies $\Pr_{x\sim \mu_p^{\otimes n}}[f(x)=g(x)] \geq \frac{1}{2}+\delta$, for some linear function $g$, and a constant $\delta = \delta(\epsilon)>0$.
	
Conversely, we show that if $k < 1+\frac{1}{p}$, then no such test $\textnormal{Lin}(\nu)$ works in the 1\% regime. 
Our key observation is that the linearity test $\textnormal{Lin}(\nu)$ works if and only if the distribution $\nu$ satisfies a certain pairwise independence property.

\end{abstract}

\section{Introduction}


\begin{figure}[t]
\centering
\includegraphics[width=0.6\columnwidth]{figures/evaluation_desiderata_V5.pdf}
\vspace{-0.5cm}
\caption{\systemName is a platform for conducting realistic evaluations of code LLMs, collecting human preferences of coding models with real users, real tasks, and in realistic environments, aimed at addressing the limitations of existing evaluations.
}
\label{fig:motivation}
\end{figure}

\begin{figure*}[t]
\centering
\includegraphics[width=\textwidth]{figures/system_design_v2.png}
\caption{We introduce \systemName, a VSCode extension to collect human preferences of code directly in a developer's IDE. \systemName enables developers to use code completions from various models. The system comprises a) the interface in the user's IDE which presents paired completions to users (left), b) a sampling strategy that picks model pairs to reduce latency (right, top), and c) a prompting scheme that allows diverse LLMs to perform code completions with high fidelity.
Users can select between the top completion (green box) using \texttt{tab} or the bottom completion (blue box) using \texttt{shift+tab}.}
\label{fig:overview}
\end{figure*}

As model capabilities improve, large language models (LLMs) are increasingly integrated into user environments and workflows.
For example, software developers code with AI in integrated developer environments (IDEs)~\citep{peng2023impact}, doctors rely on notes generated through ambient listening~\citep{oberst2024science}, and lawyers consider case evidence identified by electronic discovery systems~\citep{yang2024beyond}.
Increasing deployment of models in productivity tools demands evaluation that more closely reflects real-world circumstances~\citep{hutchinson2022evaluation, saxon2024benchmarks, kapoor2024ai}.
While newer benchmarks and live platforms incorporate human feedback to capture real-world usage, they almost exclusively focus on evaluating LLMs in chat conversations~\citep{zheng2023judging,dubois2023alpacafarm,chiang2024chatbot, kirk2024the}.
Model evaluation must move beyond chat-based interactions and into specialized user environments.



 

In this work, we focus on evaluating LLM-based coding assistants. 
Despite the popularity of these tools---millions of developers use Github Copilot~\citep{Copilot}---existing
evaluations of the coding capabilities of new models exhibit multiple limitations (Figure~\ref{fig:motivation}, bottom).
Traditional ML benchmarks evaluate LLM capabilities by measuring how well a model can complete static, interview-style coding tasks~\citep{chen2021evaluating,austin2021program,jain2024livecodebench, white2024livebench} and lack \emph{real users}. 
User studies recruit real users to evaluate the effectiveness of LLMs as coding assistants, but are often limited to simple programming tasks as opposed to \emph{real tasks}~\citep{vaithilingam2022expectation,ross2023programmer, mozannar2024realhumaneval}.
Recent efforts to collect human feedback such as Chatbot Arena~\citep{chiang2024chatbot} are still removed from a \emph{realistic environment}, resulting in users and data that deviate from typical software development processes.
We introduce \systemName to address these limitations (Figure~\ref{fig:motivation}, top), and we describe our three main contributions below.


\textbf{We deploy \systemName in-the-wild to collect human preferences on code.} 
\systemName is a Visual Studio Code extension, collecting preferences directly in a developer's IDE within their actual workflow (Figure~\ref{fig:overview}).
\systemName provides developers with code completions, akin to the type of support provided by Github Copilot~\citep{Copilot}. 
Over the past 3 months, \systemName has served over~\completions suggestions from 10 state-of-the-art LLMs, 
gathering \sampleCount~votes from \userCount~users.
To collect user preferences,
\systemName presents a novel interface that shows users paired code completions from two different LLMs, which are determined based on a sampling strategy that aims to 
mitigate latency while preserving coverage across model comparisons.
Additionally, we devise a prompting scheme that allows a diverse set of models to perform code completions with high fidelity.
See Section~\ref{sec:system} and Section~\ref{sec:deployment} for details about system design and deployment respectively.



\textbf{We construct a leaderboard of user preferences and find notable differences from existing static benchmarks and human preference leaderboards.}
In general, we observe that smaller models seem to overperform in static benchmarks compared to our leaderboard, while performance among larger models is mixed (Section~\ref{sec:leaderboard_calculation}).
We attribute these differences to the fact that \systemName is exposed to users and tasks that differ drastically from code evaluations in the past. 
Our data spans 103 programming languages and 24 natural languages as well as a variety of real-world applications and code structures, while static benchmarks tend to focus on a specific programming and natural language and task (e.g. coding competition problems).
Additionally, while all of \systemName interactions contain code contexts and the majority involve infilling tasks, a much smaller fraction of Chatbot Arena's coding tasks contain code context, with infilling tasks appearing even more rarely. 
We analyze our data in depth in Section~\ref{subsec:comparison}.



\textbf{We derive new insights into user preferences of code by analyzing \systemName's diverse and distinct data distribution.}
We compare user preferences across different stratifications of input data (e.g., common versus rare languages) and observe which affect observed preferences most (Section~\ref{sec:analysis}).
For example, while user preferences stay relatively consistent across various programming languages, they differ drastically between different task categories (e.g. frontend/backend versus algorithm design).
We also observe variations in user preference due to different features related to code structure 
(e.g., context length and completion patterns).
We open-source \systemName and release a curated subset of code contexts.
Altogether, our results highlight the necessity of model evaluation in realistic and domain-specific settings.





\section{Preliminaries} \label{sec:prelims}
Before diving into the technical results, we state the basic graph notations used throughout the paper and recap the new non-standard definitions we have introduced throughout \Cref{sec:overview}.

\paragraph{Graphs.}
Throughout we consider directed simple graphs $G = (V, E)$, where $E \subseteq V^2$, with $n = |V|$ nodes and $m = |E|$ edges. The edges of the graph can be associated with some value: a length $\ell(e)$ or a capacity/cost $c(e)$, all of which we require to be positive. For any $U \subseteq V$, we write $\overline U = V \setminus U$. Let $G[U]$ be the subgraph induced by $U$. We denote with $\delta^{+}(U)$ the set of edges that have their starting point in $U$ and endpoint in~$\overline U$. We define $\delta^{-}(U)$ symmetrically. We also sometimes write $c(S) = \sum_{e \in S} c(e)$ (for a set of edges $S$) or $c(U, W) = \sum_{e \in E \cap (U \times W)} c(e)$ and $c(U) = c(U, U)$ (for sets of nodes $U, W$).

The distance between two nodes $v$ and $u$ is written $d_G(v,u)$ (throughout we consider only the \emph{length} functions to be relevant for distances). We may omit the subscript if it is clear from the context. The diameter of the graph is the maximum distance between any pair of nodes. For a subgraph $G'$ of $G$ we occasionally say that~$G'$ has \emph{weak diameter} $D$ if for all pairs of nodes $u, v$ in~$G'$, we have $d_G(u, v), d_G(v, u) \leq D$. A strongly connected component in a directed graph $G$ is a subgraph where for every pair of nodes $v,u$ there is a path from $v$ to $u$ and vise versa. Finally, for a radius $r \geq 0$ we write $B^+(v, r) = \set{x \in V : d_G(v, x) \leq r}$ and $B^-(v, r) = \set{y \in V : d_G(y, v) \leq r}$.


\paragraph{Polynomial Bounds.}
For graphs with edge lengths (or capacities), we assume that they are positive and the maximum edge length is bounded by $\poly(n)$. This is only for the sake of simplicity in \cref{sec:ldd-expander,sec:ldd-deterministic} (where in the more general case that all edge lengths are bounded by some threshold $W$ some logarithmic factors in $n$ become $\log (nW)$ instead), and is not necessary for our strongest LDD developed in \cref{sec:ldd-fast}.

\paragraph{Expander Graphs.}
Let $G = (V, E, \ell, c)$ be a directed graph with positive edge capacities $c$ and positive unit edge lengths $\ell$. We define the \emph{volume $\vol(U)$} by
\begin{equation*}
	\vol(U) = c(U, V) = \sum_{e \in E \cap (U \times V)} c(e),
\end{equation*}
and set $\minvol(U) = \min\set{\vol(U), \vol(\overline U)}$ where $\overline U = V \setminus U$. A node set $U$ naturally corresponds to a cut $(U, \overline U)$. The \emph{sparsity} (or \emph{conductance}) of $U$ is defined by
\begin{equation*}
	\phi(U) = \frac{c(U, \overline U)}{\minvol(U)}.
\end{equation*}
In the special cases that $U = \emptyset$ we set $\phi(U) = 1$ and in the special case that $U \neq \emptyset$ but $\vol(U) = 0$, we set $\phi(U) = 0$.
We say that $U$ is \emph{$\phi$-sparse} if $\phi(U) \leq \phi$. We say that a directed graph is a $\phi$-expander if it does not contain a $\phi$-sparse cut $U \subseteq V$. 
We define the \emph{lopsided sparsity} of $U$ as
\begin{equation*}
	\psi(U) = \frac{c(U, \overline U)}{\minvol(U) \cdot \log \frac{\vol(V)}{\minvol(U)}},
\end{equation*}
(with similar special cases), and we similarly say that $U$ is \emph{$\psi$-lopsided sparse} if $\psi(U) \leq \psi$. Finally, we call a graph a \emph{$\psi$-lopsided expander} if it does not contain a $\psi$-lopsided sparse cut $U \subseteq V$.

\section{A Gaussian Variant}\label{sec:gaussian_variant}

The first step towards proving Theorem~\ref{thm:intro_main} is to prove a Gaussian variant, stated below:

\begin{proposition}\label{prop:gaussian_counter_eg}	
	Let $k\in \N$, and let $\Sigma\in \R^{k\times k}$ be a symmetric positive semi-definite matrix such that:
	\begin{enumerate}
		\item For each $i\in [k]$, it holds that $\Sigma_{i,i} = 1$.
		\item The matrix $V = \Sigma - I$ has no row/column as all zeros.
	\end{enumerate}
	
	Then, there exists a Lipschitz continuous function $f:\R\to [-1,1]$ such that:
	\[\E_{X\sim \mc N(0,1)} \sqbrac{f(X)} = 0, \quad\text{and}\quad \abs{\E_{X\sim \mc N(0, \Sigma)}\sqbrac{\prod_{i\in [k]} f(X_i)}} > 0 .\]
\end{proposition}


%%%%%%%%%%%%%%%%%%%%%%%%%%%%%%%%%%%%%%%%%%%%%%%%%%%%%%%%%%%%%%%%%%%%%%%%%%%%%%%%
\subsection{Symmetric Powers of Polynomials}

Before we prove the above proposition, we first prove a lemma about (symmetrization of) powers of multivariate polynomials.
We show that if a polynomial $q(x_1,\dots,x_k)$ depends on all the variables $x_1,\dots,x_k$, then some power $\Sym(q^d)$ (see Definition~\ref{defn:sym_op}) contains a monomial divisible by $x_1x_2\cdots x_k$.


\begin{lemma}\label{lemma:polynomial_all_var}
	Let $k\in \N$, and let $q:\R^k\to \R$ be a polynomial such that for each $i\in [k]$, the polynomial $\ell_i=\partial_iq$ is not identically zero.
	Then, there exists some $d\in \N$, and positive integers $s_1,\dots,s_k \in \N$ such that the coefficient of $x_1^{s_1}\cdot x_2^{s_2}\cdots x_k^{s_k}$ in the polynomial $\Sym(q^d)$ is non-zero.
\end{lemma}

We start by proving the following lemma about derivates of powers of $q$.
\begin{lemma}\label{lemma:pow_derivatives}
	Let $k\in \N$, and let $q:\R^k\to \R$ be a polynomial. For each $i\in [k]$, let $\ell_i=\partial_iq$.
	
	Then, for every $s=(s_1,\dots,s_k)\in \Z_{\geq 0}^k$ with $\abs{s}=\sum_{i\in [k]}s_i$, there exist  polynomials $p_0, \dots, p_{\abs{s}}$, with $p_{\abs{s}} = \prod_{i\in [k]}\ell_i^{s_i}$, such that for each $d\geq \abs{s}$, it holds that 
	\[ \partial_1^{s_1}\cdot\partial_2^{s_2}\cdots \partial_k^{s_k}\brac{q^d} = q^{d-\abs{s}}\cdot \brac{\sum_{i=0}^{\abs{s}} d^i\cdot p_i}.\]
\end{lemma}
\begin{proof}
	The proof is by induction on $\abs{s}$.
	For the base case, if $\abs{s}=0$, we have $s = (0,0,\dots,0)$, and $p_0 = 1$ satisfies the statement of the lemma.
	
	For the inductive step, consider any $s=(s_1,\dots,s_k)\in \Z_{\geq 0}^k$ with $\abs{s}=\sum_{i\in [k]}s_i > 0$.
	Without loss of generality, by symmetry, we can assume that $s_1>0$.
	By the inductive hypothesis applied to $(s_1-1,s_2\dots,s_k)$, we have the existence of polynomials $p_0, \dots, p_{\abs{s}-1}$, with $p_{\abs{s}-1} = \ell_1^{s_1-1}\cdot \prod_{i=2}^k\ell_i^{s_i}$, and such that for each $d\geq \abs{s}-1$, we have 
	\[ \partial_1^{s_1-1}\cdot\partial_2^{s_2}\cdots \partial_k^{s_k}\brac{q^d} = q^{d-\abs{s}+1}\cdot \brac{\sum_{i=0}^{\abs{s}-1} d^i\cdot p_i}.\]
	Now, if $d\geq \abs{s}$, differentiating the above with respect to $x_1$, we get 
	\begin{align*}
		\partial_1^{s_1}\cdot\partial_2^{s_2}\cdots \partial_k^{s_k}\brac{q^d} 
		&= q^{d-\abs{s}}\cdot \brac{(d-\abs{s}+1)\cdot \ell_1\cdot \sum_{i=0}^{\abs{s}-1} d^i\cdot p_i} + q^{d-\abs{s}+1}\cdot \brac{\sum_{i=0}^{\abs{s}-1} d^i\cdot \partial_1(p_i)}
		\\&= q^{d-\abs{s}}\cdot \brac{\sum_{i=1}^{\abs{s}} d^{i}\cdot \ell_1\cdot p_{i-1} + \sum_{i=0}^{\abs{s}-1}d^i\cdot \brac{\brac{-\abs{s}+1}\cdot \ell_1\cdot p_i + q\cdot \partial_1 p_i} }
		\\&= q^{d-\abs{s}}\cdot \brac{\sum_{i=0}^{\abs{s}} d^{i}\cdot \tilde{p}_i },
	\end{align*}
	where the polynomials $\tilde{p}_1,\dots,\tilde{p}_{\abs{s}}$ do not depend on $d$, and are such that $\tilde{p}_{\abs{s}} = p_{\abs{s}-1}\cdot \ell_1  = \prod_{i\in [k]}\ell_i^{s_i}$, as desired.
\end{proof}

With the above lemma in hand, next we shall consider the symmetrization operation applied to derivatives of powers of $q$.

\begin{lemma}\label{lemma:pow_sec_der}
	Let $k\in \N$, and let $q:\R^k\to \R$ be a polynomial such that for each $i\in [k]$, the polynomial $\ell_i=\partial_iq$ is not identically zero.
	
	Then, for each large enough even integer $d\in \N$, the polynomial $\Sym\brac{\partial_1^2\cdot\partial_2^2\cdots \partial_k^2\brac{q^d}}$ is not identically zero.
\end{lemma}
\begin{proof}
	By applying Lemma~\ref{lemma:pow_derivatives} on $s = (2,2,\dots,2)$, we have the existence of polynomials $p_0, \dots, p_{2k}$, with $p_{2k} = \prod_{i\in [k]}\ell_i^2$, such that for each $d\geq 2k$, it holds that $\partial_1^{2}\cdot\partial_2^{2}\cdots \partial_k^{2}\brac{q^d} = q^{d-2k}\cdot \brac{\sum_{i=0}^{2k} d^i\cdot p_i}$.
	
	By Lemma~\ref{lemma:dim_var}, let $y\in \R^k$ be such that $y$ (and its permutations) don't lie in the zero set of any of the polynomials $\ell_1,\dots,\ell_k, q$.
	We define \[ A = \min_{\pi\in S_k}\sqbrac{\prod_{i\in [k]}\ell_i\brac{y_\pi}^2} > 0, \quad B = \max_{0\leq i\leq 2k-1,\ \pi\in S_k}\abs{p_i(y_\pi)}\geq 0. \]
	Then, for any even integer $d\geq \max\set{2k, \frac{4kB}{A}}$, it holds that
	\begin{align*}
		\Sym\brac{\partial_1^{2}\cdot\partial_2^{2}\cdots \partial_k^{2}\brac{q^d}}(y) 
		&=\sum_{\pi\in S_k} q\brac{y_\pi}^{d-2k}\cdot \brac{d^{2k}\cdot \prod_{i\in [k]}\ell_i\brac{y_{\pi}}^2 + \sum_{i=0}^{2k-1} d^i\cdot p_i\brac{y_\pi}}
		\\&\geq \sum_{\pi\in S_k} q\brac{y_\pi}^{d-2k}\cdot \brac{d^{2k}\cdot A - \sum_{i=0}^{2k-1}d^i\cdot B}
		\\&\geq  \brac{\sum_{\pi\in S_k} q\brac{y_\pi}^{d-2k}}\cdot \brac{d^{2k}\cdot A - 2k\cdot d^{2k-1}\cdot B}
		\\&\geq \brac{\sum_{\pi\in S_k} q\brac{y_\pi}^{d-2k}}\cdot \frac{d^{2k} A}{2} > 0.
	\end{align*}
	Hence, for even integers $d\geq \max\set{2k, \frac{4kB}{A}}$, the polynomial $\Sym\brac{\partial_1^{2}\cdot\partial_2^{2}\cdots \partial_k^{2}\brac{q^d}}$ is not identically zero.
\end{proof}

Finally, we prove the main lemma of this section.

\begin{proof}[Proof of Lemma~\ref{lemma:polynomial_all_var}]
	Let $k\in \N$, and let $q:\R^k\to \R$ be a polynomial such that for each $i\in [k]$, the polynomial $\ell_i=\partial_iq$ is not identically zero.
	It suffices to prove that for some $d\in \N$, the polynomial $\partial_1^{2}\cdot\partial_2^{2}\cdots \partial_k^{2}\brac{\Sym(q^d)}$ is not identically zero, since then the coefficient of some monomial divisible by $x_1^2\cdot x_2^2\cdots x_k^2$ is non-zero.
	
	For each polynomial $p:\R^k\to \R$, and each $\pi\in S_k$, we shall use $p_\pi$ to denote the polynomial given by $p_\pi(x) = p(x_\pi)$.
	Then, for all $s_1,\dots,s_k\in \Z_{\geq 0}$, we have that $\partial_1^{s_1}\cdot\partial_2^{s_2}\cdots \partial_k^{s_k} \brac{p_\pi} = \brac{\partial_{\pi^{-1}(1)}^{s_1}\cdot\partial_{\pi^{-1}(2)}^{s_2}\cdots \partial_{\pi^{-1}(k)}^{s_k} \brac{p}}_\pi$.
	
	By the above, we have that for each $d\in \N$,
	\begin{align*}
		\partial_1^{2}\cdot\partial_2^{2}\cdots \partial_k^{2}\brac{\Sym(q^d)} 
		&=  \partial_1^{2}\cdot\partial_2^{2}\cdots \partial_k^{2}\brac{\sum_{\pi\in S_k}q_\pi^d}
		\\&= \sum_{\pi\in S_k} \partial_1^{2}\cdot\partial_2^{2}\cdots \partial_k^{2}\brac{q_\pi^d}
		\\&= \sum_{\pi\in S_k} \brac{\partial_{\pi^{-1}(1)}^{2}\cdot\partial_{\pi^{-1}(2)}^{2}\cdots \partial_{\pi^{-1}(k)}^{2} \brac{q^d}}_\pi
		\\&= \sum_{\pi\in S_k} \brac{\partial_{1}^{2}\cdot\partial_{2}^{2}\cdots \partial_{k}^{2} \brac{q^d}}_\pi
		\\&= \Sym\brac{\partial_1^{2}\cdot\partial_2^{2}\cdots \partial_k^{2}\brac{q^d}}.
	\end{align*}	
	Now, the result follows from Lemma~\ref{lemma:pow_sec_der}.
\end{proof}

%%%%%%%%%%%%%%%%%%%%%%%%%%%%%%%%%%%%%%%%%%%%%%%%%%%%%%%%%%%%%%%%%%%%%%%%%%%%%%%%
\subsection{Proving the Gaussian Variant}

We start by proving a slight variant of Proposition~\ref{prop:gaussian_counter_eg}, where we allow $f$ to be an arbitrary (possibly unbounded) polynomial.

\begin{lemma}\label{lemma:hermite_counter_eg}
	Let $k\in \N$, and let $\Sigma\in \R^{k\times k}$ be a symmetric positive semi-definite matrix such that:
	\begin{enumerate}
		\item For each $i\in [k]$, it holds that $\Sigma_{i,i} = 1$.
		\item The matrix $V = \Sigma - I$ has no row/column as all zeros.
	\end{enumerate}
	Then, there exists a polynomial $f:\R\to \R$ such that $\E_{X\sim \mc N(0,1)} \sqbrac{f(X)} = 0$, and \[ \abs{\E_{X\sim \mc N(0, \Sigma)}\sqbrac{\prod_{i\in [k]} f(X_i)}} > 0 .\]
\end{lemma}

\begin{proof}
	For $s = (s_1,\dots,s_k)\in \N^k$ and $\alpha = (\alpha_1,\dots,\alpha_k)\in \R^{k}$, let $f_{s,\alpha}:\R\to \R$ be the polynomial defined by $f_{s,\alpha}(x) = \alpha_1 H_{s_1}(x)+\cdots + \alpha_k H_{s_k}(x)$, where the polynomials $H_{s_i}$ are Hermite polynomials (see Definition~\ref{defn:hermite_poly}).	
	Observe that since $s_1,\dots,s_k \geq 1$, this polynomial satisfies $\E_{X\sim \mc N(0,1)} \sqbrac{f(X)} = 0$.

	Suppose, for the sake of contradiction, that for every $s\in \N^k,\ \alpha\in \R^k$, it holds that 
	\[\E_{X\sim \mc N(0, \Sigma)}\sqbrac{\prod_{i\in [k]} f_{s,\alpha}(X_i)} = \E_{X\sim \mc N(0, \Sigma)}\sqbrac{\prod_{i\in [k]} \sum_{j\in [k]}\alpha_jH_{s_j}(X_i) } =  0.\]
	Observe that for every $s\in \N^k$, the above expression can be written as a multivariate polynomial in $\alpha_1,\dots,\alpha_k$.
	If the polynomial vanishes for all $\alpha\in \R^k$, the coefficient of $\alpha_1\cdot \alpha_2\cdots \alpha_k$ must be zero; that is,
	\[ \sum_{\pi\in S_k}\E_{X\sim \mc N(0, \Sigma)}\sqbrac{\prod_{i\in [k]} H_{s_{\pi(i)}}(X_i)} = 0.\]
	Now, applying Lemma~\ref{lemma:hermite_exp}, we get that for each $d\in \N$, and each $s_1,\dots,s_k \geq 1$ with $s_1+\dots+s_k=2d$,
	\[\sum_{\pi\in S_k}\sqbrac{\brac{t^{\top}V\ t}^d : t_1^{s_{\pi(1)}}\cdots t_k^{s_{\pi(k)}}}
		= \sum_{\pi\in S_k}\sqbrac{\brac{t_\pi^{\top}V\ t_\pi}^d : t_{1}^{s_k}\cdots t_{k}^{s_k}}
		= \sqbrac{ \Sym\brac{\brac{t^{\top}V\ t}^d} : t_{1}^{s_k}\cdots t_{k}^{s_k}} = 0. \]
	Note that the assumption that $V$ has no zero row/column implies that for every $i\in [k]$, the polynomial $\partial_i \brac{t^{\top}V\ t}$ is not identically zero.
	By Lemma~\ref{lemma:polynomial_all_var}, this is a contradiction.
\end{proof}

With the above, we now prove Proposition~\ref{prop:gaussian_counter_eg} via a standard truncation argument.

\begin{proof}[Proof of Proposition~\ref{prop:gaussian_counter_eg}]
By Lemma~\ref{lemma:hermite_counter_eg}, we know that there exists a polynomial $f:\R\to \R$ such that $\E_{X\sim \mc N(0,1)} \sqbrac{f(X)} = 0$, and $\abs{\E_{X\sim \mc N(0, \Sigma)}\sqbrac{\prod_{i\in [k]} f(X_i)}} > 0$.

For each integer $M\in \N$, we define the truncated function $f_M:\R\to [-M,M]$ by 
\[f_M(x) = f(x)\cdot \ind_{\abs{f(x)}\leq M} + M\cdot \ind_{f(x) > M} - M\cdot \ind_{f(x) < -M}.\]
Also, let $g_M:\R\to [-2M,2M]$, be given by $g_M(x) = f_M(x) - \E_{X\sim \mc N(0,1)}\sqbrac{f_M(X)}$.
Observe that
\begin{enumerate}
	\item For every $M$, it holds that $\E_{X\sim \mc N(0,1)}\sqbrac{g_M(X)} = 0$.
	\item For every $M$, the function $g_M$ is bounded and Lipschitz continuous.
	\item For every $x\in \R$, $f_M(x)\to f(x)$ as $M\to \infty$. Further, since $\abs{f_M(x)} \leq \abs{f(x)}$ for each $x\in \R, M\in \N$, by the dominated convergence theorem, we have  $\E_{X\sim \mc N(0,1)}\sqbrac{f_M(x)}\to \E_{X\sim \mc N(0,1)}\sqbrac{f(x)}=0$ as $M\to \infty$.
	This implies that for each $x\in \R$, $g_M(x)\to f(x)$ as $M\to \infty$.

	Also, for each $x\in \R, M\in \N$, we have $\abs{g_M(x)} \leq \abs{f(x)} + \E_{X\sim \mc N(0,1)}\sqbrac{\abs{f(X)}}$. Hence, by the dominated convergence theorem, we have that $\E_{X\sim \mc N(0, \Sigma)}\sqbrac{\prod_{i\in [k]} g_M(X_i)} \to \E_{X\sim \mc N(0, \Sigma)}\sqbrac{\prod_{i\in [k]} f(X_i)} \not= 0$ as $M\to \infty$.
\end{enumerate}

By the above, for some large enough $M$, the function $\frac{1}{2M}\cdot g_M:\R\to [-1,1]$ satisfies the desired properties.
\end{proof}

%%%%%%%%%%%%%%%%%%%%%%%%%%%%%%%%%%%%%%%%%%%%%%%%%%%%%%%%%%%%%%%%%%%%%%%%%%%%%%%%
\section{Linearity Testing Requires Pairwise Independence}\label{sec:lin_test_failure}

In this section, we prove Theorem~\ref{thm:intro_main}, which is restated below.

\begin{theorem}\label{thm:counter_eg}
	Let $k\in \N,\ p\in (0,1)$, and let $\nu \in \mc D(p,k)$ be a distribution having no pairwise independent coordinate (see Definition~\ref{defn:distr_class}).
	Then, there exists a constant $\alpha>0$, such that for every large enough $n\in \N$, there exists a function $f:\set{0,1}^n \to [-1,1]$ such that 
	\begin{enumerate}
		\item $\abs{\E_{X\sim \nu^{\otimes n}} \sqbrac{\prod_{i=1}^k f(X_i)} }\geq \alpha.$
		\item For every $S\subseteq [n]$, it holds that $ \abs{\E_{X\sim \mu_p^{\otimes n}}\sqbrac{f(X)\chi_S(X)}} \leq o_n(1)$.
	\end{enumerate}
	
	Moreover, if the distribution $\nu$ is such that $\eta:= \max_{i,j\in [k], i\not= j} \Pr_{X\sim \nu}\sqbrac{X_i=X_j} < 1$ (that is, no two coordinates are almost surely equal), the above holds for a function $f$ with range $\set{-1,1}$.
\end{theorem}

The remainder of this section is devoted to the proof of Theorem~\ref{thm:counter_eg}.
In Section~\ref{sec:counter_eg_1}, we prove the first part of the theorem, dealing with functions with range $[-1,1]$.
Then, in Section~\ref{sec:counter_eg_2}, we show how to round to functions with range $\set{-1,1}$.

%%%%%%%%%%%%%%%%%%%%%%%%%%%%%%%%%%%%%%%%%%%%%%%%%%%%%%%%%%%%%%%%%%%%%%

\subsection{Function with Range $[-1,1]$}\label{sec:counter_eg_1}

Let $k\in \N,\ p\in (0,1),$ and let $\nu \in \mc D(p,k)$ be a distribution having no pairwise independent coordinate.
Let $\Sigma \in \R^{k\times k}$ be the (normalized) covariance matrix corresponding to the distribution $\nu$, given by, $\Sigma_{i,j} = \E_{X\sim \nu}\sqbrac{\frac{ \brac{X_i-p}\cdot \brac{X_j-p}}{p-p^2}}$.
Observe that the matrix $\Sigma$ satisfies the conditions of Proposition~\ref{prop:gaussian_counter_eg}, and hence there exists a function $h:\R\to [-1,1]$ such that
\begin{enumerate}
	\item $\E_{Z\sim \mc N(0,1)}\sqbrac{h(Z)} = 0$
	\item The function $H:\R^k\to [-1,1]$ given by $H(x) = \prod_{i\in [k]}h\brac{x_i}$ is such that \[\alpha :=  \frac{1}{2}\cdot \abs{\E_{Z\sim \mc N(0, \Sigma)} \sqbrac{H(Z)}} > 0 .\]
	\item The function $h$ is $K$-Lipschitz for some $K>0$; in particular, both $h$ and $H$ are bounded continuous functions.
\end{enumerate}

Consider any large $n\in \N$.
We define $f:\set{0,1}^n \to [-1,1]$ by 
\[f(x) = h\brac{\frac{1}{\sqrt{n}}\cdot \sum_{j=1}^n \frac{x\up{j}-p}{\sqrt{p-p^2}}},\]
The function $f$ satisfies the two properties in the theorem statement, as follows:

\begin{itemize}
	\item Let $X \sim \nu^{\otimes n}$, and let $Y = (Y_1,\dots, Y_k)$ be a $\set{0,1}^k$-valued random vector, defined as $Y_i=\frac{1}{\sqrt{n}}\cdot \sum_{j=1}^n \frac{X_i\up{j}-p}{\sqrt{p-p^2}}$.
	 
	Let $F:\set{0,1}^{kn}\to [-1,1]$ be given by $F(x) = \prod_{i\in [k]}f\brac{x_i}$.
	Since $H$ is continuous and bounded, we have by the Multivariate CLT (Theorem~\ref{thm:multi_clt}) that
	\[ \abs{\ \E\sqbrac{F(X)} - \E_{Z \sim \mc N(0, \Sigma)} \sqbrac{H(Z)}\ } = \abs{\ \E\sqbrac{{H(Y)}} - \E_{Z \sim \mc N(0, \Sigma)} \sqbrac{H(Z)}\ } \leq o_n(1).\]
	Hence, for large $n$, we get $\abs{\E_{X\sim \nu^{\otimes n}} \sqbrac{\prod_{i=1}^k f(X_i)} }\geq 2\alpha-o_n(1)\geq \alpha$, as desired.
	
	
	\item Consider any subset $S\subseteq [n]$, and let $T\subseteq S$ be any subset of size $\abs{T}=\min\set{\lfloor n^{1/4}\rfloor, \abs{S}}$. Let $\tilde{f}:\set{0,1}^n\to [-1,1]$ be defined by $\tilde{f}(X) = h\brac{\frac{1}{\sqrt{n-\abs{T}}}\cdot \sum_{j\in [n]\setminus T} \frac{x\up{j}-p}{\sqrt{p-p^2}} }$; note that this function only depends on the coordinates of $x$ outside the set $T$. Further, for each $x\in \set{0,1}^n$, by the Lipschitz bound on $h$, we get
	\begin{align*}
		\abs{f(x)-\tilde{f}(x)} &\leq K\cdot  \abs{\frac{1}{\sqrt{n}}\cdot \sum_{j=1}^n \frac{x\up{j}-p}{\sqrt{p-p^2}} - \frac{1}{\sqrt{n-\abs{T}}}\cdot \sum_{j\in [n]\setminus T} \frac{x\up{j}-p}{\sqrt{p-p^2}}}
		\\&\leq \frac{K}{\sqrt{p-p^2}}\cdot\brac{\frac{\abs{T}}{\sqrt{n}}+ \brac{n-\abs{T}}\cdot \abs{\frac{1}{\sqrt{n-\abs{T}}} - \frac{1}{\sqrt{n}}}}
		\\&\leq \frac{K}{\sqrt{p-p^2}}\cdot\brac{\frac{\abs{T}}{\sqrt{n}}+ \frac{n-\abs{T}}{\sqrt{n}}\cdot \frac{\abs{T}}{n}}
		\\&\leq \frac{K}{\sqrt{p-p^2}}\cdot\frac{2\abs{T}}{\sqrt{n}} = o_n(1),
	\end{align*}
	where we used that $(1-t)^{-1/2} \leq 1+t$ for each $t\in [0,1/2]$.
 
	Now, for $X\sim \mu_p^{\otimes n}$, we have
	\begin{align*}
		\abs{\ \E_X\sqbrac{f(X)\cdot \chi_S(X)}\ } &\leq \abs{\ \E_X\sqbrac{\tilde{f}(X)\cdot \chi_S(X)}\ } + o_n(1)
		\\&= \abs{\ \E_X\sqbrac{\tilde{f}(X)\cdot \chi_{S\setminus T}(X)}\cdot \E_X\sqbrac{ \chi_{T}(X)}\ } + o_n(1)
		\\&= \abs{\ \E_X\sqbrac{\tilde{f}(X)\cdot \chi_{S\setminus T}(X)}\ }\cdot \abs{1-2p}^{\abs{T}} + o_n(1).
	\end{align*}
	If $\abs{S} \geq \lfloor n^{1/4} \rfloor$, then $\abs{1-2p}^{\abs{T}} = o_n(1)$.
	Otherwise, we have that $S=T$, and by the Central Limit Theorem (see Theorem~\ref{thm:multi_clt}) , the first term in the above product equals
	\[ \abs{\ \E_X\sqbrac{\tilde{f}(X)}\ } = \abs{\ \E_X\sqbrac{\tilde{f}(X)} - \E_{Z\sim \mc N(0,1)}\sqbrac{h(Z)}\ } = o_n(1). \pushQED{\qed}\qedhere\popQED \]	
\end{itemize}

%%%%%%%%%%%%%%%%%%%%%%%%%%%%%%%%%%%%%%%%%%%%%%%%%%%%%%%%%%%%%%%%%%%%%%%%%%%%%%%%

\subsection{Rounding to a Function with Range $\set{-1,1}$}\label{sec:counter_eg_2}

Now, we shall prove the second part of Theorem~\ref{thm:counter_eg}.

Let $k\in \N,\ p\in (0,1),$ and let $\nu \in \mc D(p,k)$ be a distribution having no pairwise independent coordinate.
Further suppose that the distribution $\nu$ is such that \[\eta := \max_{i,j\in [k], i\not= j} \Pr_{X\sim \nu}\sqbrac{X_i=X_j} < 1.\]
Let $\alpha>0$ be as obtained in Section~\ref{sec:counter_eg_1}.
Consider any large $n\in \N$, and let $f:\set{0,1}^n\to [-1,1]$ be the function obtained in Section~\ref{sec:counter_eg_1}.

Let $g:\set{0,1}^n \to \set{-1,1}$ be a random function, defined as $g(x) = \begin{cases} 1, & w.p.\  \frac{1+f(x)}{2} \\ -1, & w.p.\  \frac{1-f(x)}{2} \end{cases}$, independently for each $x\in \set{0,1}^n$.
Observe that this satisfies $\E_{g}\sqbrac{g(x)} = f(x)$ for each $x\in \set{0,1}^n$.
We will show that the function $g$ satisfies the two desired properties with probability $1-o_n(1)$, and hence by the probabilistic method, this guarantees the existence of a non-random $g$ as desired.
This is done as follows:
\begin{enumerate}
	\item Let $F,G:\set{0,1}^{kn}\to [-1,1]$ be defined as $F(x) = \prod_{i\in [k]}f\brac{x_i}$ and $G(x) = \prod_{i\in [k]}g\brac{x_i}$.
		Let $X,Y \sim \nu^{\otimes n}$ be independent (of each other and of $g$) and let $E$ be the event that $X_1, \dots, X_k, Y_1, \dots, Y_k$ are all distinct.
		Then, by a union bound, we have that $\Pr\sqbrac{\bar E} \leq 2\cdot \binom{k}{2} \cdot \eta^n + k^2\cdot \brac{p^2+\brac{1-p}^2}^n = o_n(1)$, and hence
		\begin{align*}
			\abs{\ \E_{g} \E_{X\sim \nu^{\otimes n}}\sqbrac{G(X)}-\E_{X\sim \nu^{\otimes n}}\sqbrac{F(X)}\ } 
			&\leq \Pr\sqbrac{\bar E} + \abs{\  \E_{g}\E_{X,Y}\sqbrac{G(X)\cdot \ind_{E}}-\E_{X}\sqbrac{F(X)}\ }
			\\&\leq \Pr\sqbrac{\bar E} + \abs{\ \E_{X,Y}\sqbrac{F(X)\cdot \ind_{E}}-\E_{X}\sqbrac{F(X)}\ } 
			\\&\leq 2\Pr\sqbrac{\bar E} = o_n(1).
		\end{align*}
		Similarly, we have
		\begin{align*}
			\abs{\ \E_{g} \sqbrac{\E_{X}\sqbrac{G(X)}}^2-\sqbrac{\E_{X}\sqbrac{F(X)}}^2\ } &= \abs{\ \E_{g} \E_{X,Y}\sqbrac{G(X)\cdot G(Y)}-\E_{X,Y}\sqbrac{F(X)\cdot F(Y)}\ } \\&\leq 2\Pr\sqbrac{\bar E} = o_n(1).
		\end{align*}
		
		Letting $\beta = \abs{\E_{X}\sqbrac{F(X)}} \geq \alpha$, we get 
		$ \Var_g\sqbrac{\E_{X}\sqbrac{G(X)}} \leq \beta^2+o_n(1) - \brac{\beta-o_n(1)}^2 = o_n(1)$.
		Hence, by Chebyshev's inequality (Fact~\ref{fact:chebyshev}), we have $\abs{\E_{X}\sqbrac{G(X)}} \geq \frac{\alpha}{2}$ with probability $1-o_n(1)$.
		
	\item Fix $S \subseteq [n]$. Let $X\sim \mu_p^{\otimes n}$, and let $W = \E_{X}\sqbrac{\chi_S(X)\cdot g(X)} = \sum_{x\in \set{0,1}^n}\Pr\sqbrac{X=x} \cdot \chi_S(x)\cdot g(x)$. Observe that $W$ is a sum of $2^n$ independent and bounded random variables, and such that $\E_g[W] = \E_{X}\sqbrac{\chi_S(X)\cdot f(X)}$. For $q = \max\set{p,1-p}<1$, it holds that $\sum_x (2\Pr[X=x])^2 \leq 4q^n\cdot \sum_x\Pr[X=x] = 4q^n$, and by Hoeffding's inequality (Fact~\ref{fact:hoeffding}), we have for each $t>0$ that
	\[ \Pr\sqbrac{ \abs{W-\E[W]}\geq t} \leq 2\cdot \exp\brac{-\frac{2t^2}{4q^n}}. \]
	Let $t=q^{n/4}$.
	Then, with probability at least $1-o_n(2^{-n})$, it holds that $\abs{W} = \abs{\E_{X}\sqbrac{\chi_S(X)\cdot g(X)}} \leq \abs{\E_{X}\sqbrac{\chi_S(X)\cdot f(X)}}+q^{n/4} = o_n(1).$
	
	Now, a union bound over $S\subseteq[n]$ shows that with probability $1-o_n(1)$, the above holds for every $S\subseteq [n]$.
	\qed
\end{enumerate}

\section{Queries vs. Bias Tradeoff}\label{sec:query_bias}

In this section, we analyze the relation between $p$ (the bias) and $k$ (the number of queries) for the existence of a distribution $\nu \in \mc D(p,k)$ with some pairwise independent coordinate, and with full even-weight support (see Definition~\ref{defn:distr_class}).

\subsection{Query Lower Bound}

We prove a lower bound on $k$ in terms of the $p$, as follows:
\begin{proposition}\label{prop:query_bias_lb}
	Let $k\in \N,\ p\in (0,1)$, and let $\nu \in \mc D(p,k)$ be a distribution that has some pairwise independent coordinate.
	Then, it holds that $k\geq 3$ and  $\frac{1}{k-1} \leq p \leq 1-\frac{1}{k-1}$.
\end{proposition}
\begin{proof}
	Let $X\sim \nu$, and let $i\in [k]$ be a pairwise independent coordinate under $\nu$.
	
	For $Z = \sum_{j\not=i} X_j$, we have by linearity of expectation, that $\E\sqbrac{X_i\cdot Z} = (k-1)p^2$.
	On the other hand, observe that if $X_i = 1$, then $Z = 1 \modt$ and so $Z\geq 1$.
	Hence,
	\[p = \E\sqbrac{X_i\cdot 1}\leq  \E\sqbrac{X_i\cdot Z} = (k-1)p^2, \]
	and we have $(k-1)p \geq 1$; in particular, this shows $k\geq 3$.
	
	For the upper bound on $p$, we consider the following cases:
	\begin{itemize}
		\item $k$ is odd: In this case, if $X_i=1$, then $Z = 1 \modt$ and so $Z\leq k-2$. Hence,
		\[ (k-1)p^2 = \E\sqbrac{X_i\cdot Z} \leq  \E\sqbrac{X_i\cdot (k-2)} = p(k-2),\]
		and we have $(k-1)p \leq (k-2)$, as desired.
		\item $k$ is even: In this case, observe that the distribution of the random variable $(1-X_1, \dots, 1-X_k)$ also satisfies the hypothesis of the proposition, with $p$ replaced by $1-p$. Hence, the above proof gives us $(k-1)\cdot (1-p) \geq 1$, as desired. \qedhere
	\end{itemize}
\end{proof}
\begin{remark}\label{remark:corner_case_full_support}
The proof of Proposition~\ref{prop:query_bias_lb} also shows that for $k>3$ and $p\in \set{\frac{1}{k-1}, 1-\frac{1}{k-1}}$, any distribution satisfying the assumptions of Proposition~\ref{prop:query_bias_lb} cannot have full even-weight support.
This is because if $p\in \set{\frac{1}{k-1}, 1-\frac{1}{k-1}}$, in all cases in the above proof, the random variable $Z$ must be constant under some value of $X_i$ (either $X_i=0$ or $X_i=1$); this cannot be the case for a distribution with full even-weight support when $k>3$.
\end{remark}
%%%%%%%%%%%%%%%%%%%%%%%%%%%%%%%%%%%%%%%%%%%%%%%%%%%%%%%%%%%%%%%%%%%%%%%%%%%%%%%%

\subsection{Query Upper Bound}

In this subsection, we shall prove the following proposition.
\begin{proposition}\label{prop:query_bias_ub}
	Let $k\geq 3$ be a positive integer, and let $p \in \sqbrac{\frac{1}{k-1},1-\frac{1}{k-1}}$ (note that this interval is non-empty for $k\geq 3$).
	
	Then, there exists a permutation-invariant\footnote{we say that a distribution $\nu$ over $\set{0,1}^k$ is \emph{permutation-invariant}, if for $X = (X_1,\dots,X_k)\sim \nu$, and any permutation $\pi:[k]\to [k]$, the distribution of $\brac{X_{\pi(1)}, \dots, X_{\pi(k)}}$ is the same as $\nu$.} and pairwise independent distribution $\nu (k,p) \in \mc D(p,k)$ (see Definition~\ref{defn:distr_class}).
	Furthermore, if $k=3$ or if $p\not\in \set{\frac{1}{k-1}, 1-\frac{1}{k-1}}$, then there exists such a distribution with full even-weight support.
\end{proposition}

The proof involves various cases, considered below in Lemma~\ref{lemma:ub_case_analysis} and Lemma~\ref{lemma:ub_add_ind_copies}.

\begin{lemma}\label{lemma:ub_case_analysis}
	Let $k\geq 4$ be a positive integer, and let $p \in \left[\frac{1}{k-1}, \frac{2}{k-1}\right)\cup \left( 1-\frac{2}{k-1}, 1-\frac{1}{k-1}\right]$ (note that this interval is contained in $\sqbrac{\frac{1}{k-1},1-\frac{1}{k-1}}$ for $k\geq 4$). 
	Then, there exists a pairwise independent distribution $\nu (k,p)\in \mc D(p,k)$.
	
	Moreover, if $p\not\in \set{\frac{1}{k-1}, 1-\frac{1}{k-1}}$, then there exists such a distribution with full even-weight support.
\end{lemma}
\begin{proof}
	Let $k\geq 4$ be a positive integer, and let $p \in \left[\frac{1}{k-1}, \frac{2}{k-1}\right)\cup \left( 1-\frac{2}{k-1}, 1-\frac{1}{k-1}\right]$.
	Let $s = \lfloor\frac{k}{2}\rfloor$; we shall exhibit a vector  $q = (q_0,q_1,\dots,q_s) \in [0,1]^{s+1}$ satisfying:
	\[
		\sum_{i=0}^s \binom{k}{2i}\cdot  q_i = 1,\ \quad\sum_{i=1}^s \binom{k-1}{2i-1}\cdot q_i = p,\quad \sum_{i=1}^s \binom{k-2}{2i-2}\cdot q_i = p^2.
	\]
	The distribution $\nu(p,k)$ is then defined 	by assigning probability $\begin{cases} q_{\abs{x}/2},& \abs{x}=0\modt\\0,& \abs{x}=1\modt \end{cases}$ to the point $x\in \set{0,1}^k$, where $\abs{x} = \sum_{i=1}^k x_i$.
	Note that the above properties correspond to $\nu(k,p)$ being a valid probability distribution supported on even-hamming-weight vectors, having marginals $\mu_p$, and pairwise independent coordinates.
	Furthermore, the distribution $\nu(p,k)$ has full even-weight support if and only if each $q_i \in (0,1)$.
	
	The vector $q$ is defined as follows in different cases (for brevity, we omit the verification of the above properties):
	\begin{enumerate}
		\item $k\geq 5$ is odd,  $p \in \left[\frac{1}{k-1}, \frac{2}{k-1}\right)$: Let $q_0 = 1 + \frac{kp^2}{2} - \frac{k^2p}{2(k-1)}$, $q_1 = \frac{(k-2)p-(k-1)p^2}{(k-1)(k-3)}$, $q_{(k-1)/2} = \frac{(k-1)p^2-p}{(k-1)(k-3)}$, and zero otherwise.
		\item $k\geq 5$ is odd,  $1-p\in \left[\frac{1}{k-1}, \frac{2}{k-1}\right)$: Let $q_0 = 1 + \frac{kp^2}{k-3} - \frac{k(2k-5)p}{(k-1)(k-3)}$, $q_{(k-3)/2} = \frac{3(k-2)p-3(k-1)p^2}{(k-1)(k-2)(k-3)}$, $q_{(k-1)/2} = \frac{(k-1)p^2-(k-4)p}{2(k-1)}$, and zero otherwise.
		\item $k\geq 4$ is even,  $p\in \left[\frac{1}{k-1}, \frac{2}{k-1}\right)$: Let $q_0 = \frac{(k-1)p^2-(k+1)p+2}{2}$,  $q_1 = \frac{p-p^2}{k-2}$, $q_{k/2} = \frac{(k-1)p^2-p}{k-2}$, and zero otherwise.
		\item $k\geq 4$ is even, $1-p\in \left[\frac{1}{k-1}, \frac{2}{k-1}\right)$: In this case, we define $\nu(k,p)$ to be the distribution obtained by flipping each coordinate of $\nu(k,1-p)$.
	\end{enumerate}
	
	Next, we show that if $p\not\in \set{\frac{1}{k-1}, 1-\frac{1}{k-1}}$, then such a distribution $\nu(p,k)$ with full even-weight support exists.
	We only need to do this for the first three cases, as the procedure described in the fourth case preserves the property of full even-weight support.
	 
	The same argument applies in all cases, and we present it for the first case: that is when $k\geq 5$ is odd, and $p \in \brac{\frac{1}{k-1}, \frac{2}{k-1}}$.
	We observe if $p \not= \frac{1}{k-1}$, each of the probabilities $q_0,q_1,q_{(k-1)/2}$ above lie in the interval $(0,1)$.
	Now, consider the equations
	\[
		\sum_{i=0}^s \binom{k}{2i}\cdot  \tilde{q}_i = 0,\ \quad\sum_{i=1}^s \binom{k-1}{2i-1}\cdot \tilde{q}_i = 0,\quad \sum_{i=1}^s \binom{k-2}{2i-2}\cdot \tilde{q}_i = 0.
	\]
	In these equations, the variables $\tilde{q}_0, \tilde{q}_1, \tilde{q}_{(k-1)/2}$ are linearly independent, and hence, there exists a vector $\tilde{q} \in \R^{s+1}$ satisfying these equations, which has all coordinates equal to 1, other than possibly $\tilde{q}_0, \tilde{q}_1, \tilde{q}_{(k-1)/2}$.
	Then, for some small $\delta > 0$, the vector $q+\delta\cdot \tilde{q}$ has all coordinates in $(0,1)$, and satisfies the required properties.	
\end{proof}

\begin{lemma}\label{lemma:ub_add_ind_copies}
	Let $k\geq 6$ be a positive integer, and let $p \in \sqbrac{\frac{2}{k-1},1-\frac{2}{k-1}} \setminus \set{\frac{1}{2}}$ (note that this interval is non-empty for $k\geq 6$).
	There, there exists a pairwise independent distribution $\nu (k,p)\in \mc D(p,k)$ with full even-weight support.
\end{lemma}
\begin{proof}
	Let $k\geq 6$ be a positive integer, and let $p \in \sqbrac{\frac{2}{k-1},1-\frac{2}{k-1}},\ p\not=\frac{1}{2}$.
	That is, for $q = \min\set{p,1-p} < \frac{1}{2}$, we have $k \geq 1 + \frac{2}{q}$.
	Let $\ell$ be the smallest odd integer satisfying $\ell > 1+\frac{1}{q} > 3$.
	Note that this satisfies $4\leq \ell \leq 3+\frac{1}{q} < 1 + \frac{2}{q} \leq k$, and we have $q \in \brac{\frac{1}{\ell-1}, \frac{2}{\ell-1}}$.	
	
	By Lemma~\ref{lemma:ub_case_analysis}, there exist pairwise independent distributions $\nu(\ell, p)$ and $\nu(\ell, 1-p)$, with full even-weight support.
	Let $\tilde{\nu}_0 = \nu(\ell, p)$, and let $\tilde{\nu}_1$ be the distribution obtained by flipping each coordinate of $\nu(\ell, 1-p)$.
	Since $\ell$ is odd, for each $b\in \set{0,1}$, it holds that $\tilde{\nu}_b$ has pairwise independent coordinates, each with marginal $\mu_p$, and such that $\supp(\tilde{\nu}_b) = \set{x\in \set{0,1}^k: \sum_{i=1}^k x_i = b \modt}$.
	Finally, we define $X \sim \nu(k,p)$ via the following random process: Let $(X_{\ell+1},\dots,X_k)\sim \mu_p^{\otimes\brac{k-\ell}}$, and with $Z = \sum_{i=\ell+1}^k X_i \modt$, we let $(X_1,\dots,X_\ell) \sim \tilde{\nu}_{Z}$.
	It is an easy check that this distribution satisfies the required properties.
\end{proof}

Finally, we prove Proposition~\ref{prop:query_bias_ub}.

\begin{proof}[Proof of Proposition~\ref{prop:query_bias_ub}]
	Note that it suffices to find such a distribution that is not necessarily permutation invariant, since averaging the distribution over all permutations preserves pairwise independence and full even-weight support.
	
	If $p=1/2$, for any $k\geq 3$, we let $\nu(k,p)$ be the uniform distribution on the set $\set{ x\in\set{0,1}^k : \sum_{i=1}^k x_i = 0 \modt}$.
	
	Now, for $k=3$, it must hold that $p=1/2$, in which case $\nu(k,p)$ is as above.
	For $k=4$ or $k=5$, and $p\not=1/2$, it must hold that $p \in \left[\frac{1}{k-1}, \frac{2}{k-1}\right)\cup \left( 1-\frac{2}{k-1}, 1-\frac{1}{k-1}\right]$, and the result follows from Lemma~\ref{lemma:ub_case_analysis}.
	For $k\geq 6$ and $p\not=\frac{1}{2}$, the result follows from Lemma~\ref{lemma:ub_case_analysis} and Lemma~\ref{lemma:ub_add_ind_copies}.
\end{proof}


%%%%%%%%%%%%%%%%%%%%%%%%%%%%%%%%%%%%%%%%%%%%%%%%%%%%%%%%%%%%%%%%%%%%%%%%%%%%%%%%

\section{Putting Everything Together}\label{sec:putting_together}

We are now ready to prove our main result.

\begin{proof}[Proof of Theorem~\ref{thm:intro_querybias_main_thm}]
Let $p\in (0,1)$.

\begin{enumerate}
	\item Consider any positive integer $k > 1 + \frac{1}{\min\set{p,1-p}} \geq 3$ (or $k=3$ with $p=\frac{1}{2}$).
		By Proposition~\ref{prop:query_bias_ub}, there exists a pairwise independent distribution $\nu\in \mc D(p,k)$ with full even-weight support.
		The result now follows by Theorem~\ref{thm:bkm23_in_section}.
	
	\item Suppose that $k\geq 3$ with $p=\frac{1}{k-1}$, or $k\geq 4$ is even with $\ p = 1-\frac{1}{k-1}$.
	In these cases, we observe that the distribution $\nu\in \mc D(p,k)$ constructed in Lemma~\ref{lemma:ub_case_analysis} is pairwise independent, and contains BLR (see Definition~\ref{defn:cont_BLR}):
	\begin{enumerate}
		\item If $k\geq 3, p=\frac{1}{k-1}$, the distribution $\nu$ contains all vectors in $\set{0,1}^k$ of hamming-weights 0 and 2 in its support. In this case, Definition~\ref{defn:cont_BLR} is satisfied with $\tilde{b}=0$ and $\tilde{z}$ as the all-zeros vector.
		\item If $k\geq 4$ is even, and $p=1-\frac{1}{k-1}$, the distribution $\nu$ contains all vectors in $\set{0,1}^k$ of hamming-weights $k-2$ and $k$ in its support. In this case, Definition~\ref{defn:cont_BLR} is satisfied with $\tilde{b}=1$ and $\tilde{z}$ as the all-ones vector.
	\end{enumerate}
	The result now follows by Theorem~\ref{thm:bkm23_in_section}.
	
	\item Suppose that $k < 1 + \frac{1}{\min\set{p,1-p}}$ is a positive integer, and let $\nu \in \mc D(p,k)$.
		We perform the following operation on the distribution $\nu$:
		if $i,j\in [k],\ i\not=j$ are such that $\Pr_{X\sim \nu}[X_i=X_j]=1$, we remove coordinates $i,j$ from $\nu$, and repeat until no such pairs remain.
		
		Finally, we are left with a distribution $\tilde{\nu}$ on $\tilde{k}\leq k$ coordinates.		
		We consider the following two cases:
		\begin{enumerate}
			\item Suppose that $\tilde{k}=0$. In this case, for every $n\in \N,$ and every $f:\set{0,1}^n\to \set{-1,1}$, it holds that $\E_{X\sim \nu^{\otimes n}} \sqbrac{\prod_{i=1}^kf(X_i)} = 1$, since the $k$ terms in the product cancel out in pairs.
			Hence, it suffices to show the existence of a function $f:\set{0,1}^n\to \set{-1,1}$ satisfying $\abs{\E_{X\sim \mu_p^{\otimes n}}\sqbrac{f(X)\cdot \chi_S(X)}} \leq o_n(1)$ for every $S\subseteq [n]$.
			Note that a (uniformly) random function $f:\set{0,1}^n\to \set{-1,1}$ satisfies this with high probability, by an argument similar to the one at the end of Section~\ref{sec:counter_eg_2} (a random function can be thought of as rounding the constant zero function as in Section~\ref{sec:counter_eg_2}).
			
			\item Now, suppose that $\tilde{k}\not=0$. 
			Then, it holds that $\tilde{\nu} \in \mc D(p,\tilde{k})$, and by Proposition~\ref{prop:query_bias_lb}, we have that $\tilde{\nu}$ has no pairwise independent coordinate.
			Now, by Theorem~\ref{thm:counter_eg} there exists a constant $\alpha>0$, such that for every large $n\in \N$, there exists a function $f:\set{0,1}^n\to \set{-1,1}$ such that \[ \abs{\E_{(X_1,\dots,X_k)\sim \nu^{\otimes n}} \sqbrac{\prod_{i\in [k]} f(X_i)} } = \abs{\E_{(X_1,\dots,X_{\tilde{k}})\sim \tilde{\nu}^{\otimes n}} \sqbrac{\prod_{i\in [\tilde{k}]} f(X_i)} }\geq \alpha,\]
			and such that $ \abs {\E_{X\sim \mu_p^{\otimes n}}\sqbrac{f(X)\cdot \chi_S(X)}} \leq o_n(1)$ for every $S\subseteq [n]$. \qedhere
			\end{enumerate}
\end{enumerate}
\end{proof}

%%%%%%%%%%%%%%%%%%%%%%%%%%%%%%%%%%%%%%%%%%%%%%%%%%%%%%%%%
\subsection{A Corner Case}\label{sec:corner_case}

In the above proof, we leave the case of odd $k\geq 5$ and $p=1-\frac{1}{k-1}$.
This turns out to be very interesting, and we discuss it next.
For the remainder of this section, we fix such a $k$ and $p$.

In this case, the pairwise independent distribution $\nu\in \mc D(p,k)$ constructed in Lemma~\ref{lemma:ub_case_analysis}, is supported on vectors of hamming weights 0 and $k-1$ (and does not contain BLR as in Definition~\ref{defn:cont_BLR}).
In particular, for every $x\in \supp(\nu)$, it holds that $\sum_{i=1}^k x_i = 0 \modk$.
For this reason, as we show next, the best we can expect from the test $\Lin(\nu)$, is to guarantee correlation with a character over $\Z/(k-1)\Z$, and this is indeed true.

\begin{definition}\label{defn:char}(Characters over $\Z/(k-1)\Z$)
Let $\omega$ be a primitive $(k-1)$\textsuperscript{th} root of unity.
For every $0\leq r\leq k-2$, we define the function $\phi_r: \set{0,1} \to \C$ as $\phi_r(x)=\omega^{rx}$.

For every $n\in \N$, and every integers $0\leq r\up{1},\dots,r\up{n}\leq k-2$, we define the product character $\phi_{r\up{1},\dots,r\up{n}}:\set{0,1}^n\to \C$ by $\phi_{r\up{1},\dots,r\up{n}}(x) = \prod_{j=1}^n\phi_{r\up{j}}(x\up{j}) = \omega^{\sum_{j=1}^n r\up{j}x\up{j}}$.
\end{definition} 

Now, consider the test $\Lin(\nu)$.
Observe that any character $f = \phi_{r\up{1},\dots,r\up{n}}$ passes this test with probability 1:
\[ \E_{X\sim\nu^{\otimes n}}\sqbrac{\prod_{i\in [k]}f(X_i)} = \prod_{j=1}^n \E_{Y\sim\nu}\sqbrac{\prod_{i\in [k]}\phi_{r\up{j}}(Y_i)} = \prod_{j=1}^n \E_{Y\sim\nu}\sqbrac{\omega^{r\up{j}\cdot \brac{\sum_{i\in [k]}{Y_i}}}} = 1.\]
Next, we claim that characters explain the success of $\Lin(\nu)$ for any function $f$:

\begin{theorem}
	For every constant $\epsilon > 0$, there exists a constant $\delta >0$ such that for every large enough $n\in \N$, the following is true:
	
	Let $f:\set{0,1}^n\to[-1,1]$ be a function such that $\abs{\E_{X\sim \nu^{\otimes n}} \sqbrac{\prod_{i=1}^k f(X_i)} }\geq \eps.$
	Then, there exist integers $0\leq r\up{1},\dots,r\up{n}\leq k-2$, such that
	\[ \abs {\E_{X\sim \mu_p^{\otimes n}}\sqbrac{f(X)\cdot \phi_{r\up{1},\dots,r\up{n}}(X)}} \geq \delta .\]
\end{theorem}
\begin{proof}
	The result follows from the work of Bhangale, Khot, Liu and Minzer~\cite{BKLM24a, BKLM24b}, and we omit the details.
	Very roughly speaking, the proof follows a similar strategy as in Section~\ref{sec:bkm_sketch}: first show that $f$ has good correlation with a character under random restrictions; then, use this to show that $f$ has good correlation with character times a low-degree function; finally, use that $\nu$ is pairwise independent to get rid of the low-degree function.
\end{proof}

Finally, we present an alternative solution to deal with this corner case of odd $k\geq 5$ and $p=1-\frac{1}{k-1}$.
Instead of the test $\Lin(\nu)$, we can perform the following test:

Let $f:\set{0,1}^n\to [-1,1]$, and let $\nu' \in \mc D(1-p,k)=\mc D(\frac{1}{k-1},k)$ be the pairwise independent distribution from Lemma~\ref{lemma:ub_case_analysis}.
\begin{enumerate}
	\item Sample $X=(X_1,\dots,X_k)\sim \nu'^{\otimes n}$.
	\item Let $X'$ be the vector obtained by negating each of the $kn$ coordinates of $X$.
	\item Query $f$ on $X'_1,\dots,X'_k$ and accept if and only if $\prod_{i\in [k]} f(X'_i) = 1$.  
\end{enumerate}
Each query $X_i'$ of the above test is distributed according to $\mu_p^{\otimes n}$, and the analysis of the test simply follows from the analysis for $\Lin(\nu')$ in Theorem~\ref{thm:intro_querybias_main_thm}.
The drawback here, though, is that the test does not accept all linear functions with probability 1, but only functions of the form $(-1)^{\abs{S}}\cdot \chi_S,$ for $S\subseteq[n]$.


\section{Analysis of the Linearity Test}\label{sec:bkm_sketch}

In this section, we shall state and prove a generalized version of Theorem~\ref{thm:bkm23}.
The proof follows the work of Bhangale, Khot and Minzer~\cite{BKM23b}, and hence we only give a rough outline (skipping many of the technical points), pointing out the places where the proof differs from the above work.
We start with the following definition:

\begin{definition}\label{defn:cont_BLR}
	Let $k\geq 3,p\in (0,1)$, and let $\nu\in \mc D(p,k)$ be a distribution.
	We say that $\nu$ \emph{contains BLR}, if there exists some $\tilde{b}\in \set{0,1},\ \tilde{z}\in \set{0,1}^{k-3}$, such that 
	\[ \set{(x_1,\ x_2,\ x_1\oplus x_2\oplus \tilde{b},\ \tilde{z}) : x_1,x_2\in \set{0,1}}\subseteq \supp(\nu)\subseteq \set{0,1}^k. \]	
	Furthermore, for technical reasons, we shall also require that \[\textnormal{span}_{\mathbb{F}_2}(\supp(\nu)) = \set{x\in \set{0,1}^k : \sum_{i=1}^k x_i = 0\modt} .\]
\end{definition}

Observe that any $\nu$ with full even-weight support contains BLR (with $\tilde{b}=0$, and $\tilde{z}$ the all-zeros vector).
With this, we state the following generalization of Theorem~\ref{thm:bkm23}:

\begin{theorem}\label{thm:bkm23_in_section}
	Let $k\geq 3$ be a positive integer, and let $p\in (0,1),\ \epsilon \in (0,1]$ be constants, and let $\nu \in \mc D(p,k)$ be a distribution containing BLR (see Definition~\ref{defn:cont_BLR}).
	Then, there exists constants $\delta>0,\ d\in \N$ (possibly depending on $k, p, \epsilon, \nu$), such that for every large enough $n\in \N$, the following is true:
	
	Let $f:\set{0,1}^n\to[-1,1]$ be a function such that \[ \abs{\E_{(X_1,\dots,X_k)\sim \nu^{\otimes n}} \sqbrac{\prod_{i=1}^k f(X_i)} }\geq \eps.\]
	Then, there exists a set $S\subseteq [n]$, and a polynomial $g:\set{0,1}^n\to \R$ of degree at most $d$ and with 2-norm $\E_{X\sim \mu_p^{\otimes n} }\sqbrac{g(X)^2}\leq 1$, such that
	\[ \abs {\E_{X\sim \mu_p^{\otimes n}}\sqbrac{f(X)\cdot \chi_S(X)\cdot g(X)}} \geq \delta .\]
	
	Moreover, if the distribution $\nu$ has some pairwise independent coordinate, then we may assume $g\equiv 1$; that is, $f$ correlates with a linear function $\chi_S$.
\end{theorem}

The remainder of this section is devoted to the proof of the above theorem.
Let $k\geq 3$ be an integer, and let $\ p\in (0,1),\ \epsilon \in (0,1]$ be constants, and let $\nu \in \mc D(p,k)$ be a distribution containing BLR (see Definition~\ref{defn:cont_BLR}).
Also, let $f:\set{0,1}^n\to[-1,1]$ be a function such that

\begin{equation}\label{eqn:test_pass}
	\abs{\E_{X=(X_1,\dots,X_k)\sim \nu^{\otimes n}} \sqbrac{\prod_{i=1}^k f(X_i)} }\geq \eps.
\end{equation}

\subsection*{Step 1: Large Fourier Coefficient under Random Restriction.}\label{sec:large_fcurr}
We note that the proof of this step is where we differ from~\cite{BKM23b}.

Since the distribution $\nu\in \mc D(p,k)$ contains BLR, we can write $\nu = (1-\beta)\cdot \nu' + \beta\cdot \mu$, for some small constant $0<\beta<\frac{1}{2}\min\set{p,1-p}$, some distribution $\nu'$ over $\set{0,1}^k$, and with $\mu$ the uniform distribution over $\set{(x_1,x_2,x_1\oplus x_2\oplus \tilde{b},\tilde{z}) : x_1,x_2\in \set{0,1}}$, where $\tilde{b},\tilde{z}$ are as in Definition~\ref{defn:cont_BLR}.
Using this, we can describe choosing $X \sim \nu^{\otimes n}$ as the following two step process. First choose a set $I\subseteq [n]$, denoted $I\sim_{1-\beta} [n]$, by choosing $i\in I$ with probability $1-\beta$, independently for each $i\in [n]$.
Then, choose $Z\sim \nu'^{\otimes I}$ and $Y\sim \mu^{\bar{I}}$, and set $X = (Y,Z)$.

With the above, we can prove that the function $f$ satisfies the property of having a large fourier coefficient under random restrictions; the reader is referred to~\cite{Don14} for an introduction to Fourier analysis over the hypercube.

\begin{lemma}\label{lemma:lfcurr}
	With $\delta = \epsilon/2$, it holds that
	\[ \Pr_{I\sim_{1-\beta}[n],\ Z\sim \nu'^{\otimes I}}\sqbrac{\exists S\subseteq [n]\setminus I:\ \abs{\widehat{f_{I\to Z_1}}(S)}\geq \delta\ } \geq \delta.\]
	Here, $f_{I\to Z_1}$ refers to the restriction of the function $f$, with the variables in $I$ \emph{set to} $Z_1$.
\end{lemma}	
\begin{proof}
	By Equation~\ref{eqn:test_pass}, we have
	\begin{align*}
		\epsilon &\leq \abs{\ \E_{X=(X_1,\dots,X_k)\sim \nu^{\otimes n}} \sqbrac{\prod_{i=1}^k f(X_i)}\ }
		\\&= \abs{\ \E_{I\sim_{1-\beta}[n],\ Z\sim \nu'^{\otimes I}}\E_{Y\sim \mu^{\otimes \bar{I}}} \sqbrac{\prod_{i=1}^k f_{I\to Z_i}(Y_i)}\ }
		\\&\leq \E_{I\sim_{1-\beta}[n],\ Z\sim \nu'^{\otimes I}}\abs{\ \E_{Y\sim \mu^{\otimes \bar{I}}} \sqbrac{\prod_{i=1}^k f_{I\to Z_i}(Y_i)}\ }		
	\end{align*}
	Observe that in the above expression, the random variables $Y_4,\dots,Y_k$ are constants (determined by $\tilde{z}$).
	Now, using a (classical) Fourier analytic argument to analyze the BLR linearity test over the uniform distribution (see Chapter 1 of~\cite{Don14}), we get
	\begin{align*}
		\epsilon &\leq \E_{I\sim_{1-\beta}[n],\ Z\sim \nu'^{\otimes I}}\abs{\ \E_{Y\sim \mu^{\otimes \bar{I}}} \sqbrac{\prod_{i=1}^3 f_{I\to Z_i}(Y_i)}\ }		
%		\\&= \E_{I\sim_{1-\beta}[n],\ Z\sim \nu'^{\otimes I}}\abs{\ \E_{Y_1,Y_2\sim \set{0,1}^{\otimes \bar{I}}} \sqbrac{ f_{I\to Z_1}(Y_1)\cdot f_{I\to Z_2}(Y_2)\cdot f_{I\to Z_3}(Y_1\oplus Y_2\oplus \tilde{b}^{\bar{I}}) }\ }
		\\&= \E_{I\sim_{1-\beta}[n],\ Z\sim \nu'^{\otimes I}}\abs{\ \sum_{S\subseteq \bar{I}} \widehat{f_{I\to Z_1}}(S)\cdot \widehat{f_{I\to Z_2}}(S)\cdot \widehat{f_{I\to Z_3}}(S)\cdot (-1)^{\tilde{b}\cdot \abs{S}} \ }
		\\&\leq \E_{I\sim_{1-\beta}[n],\ Z\sim \nu'^{\otimes I}} \sqbrac{\max_{S\subseteq \bar{I}}\abs{\widehat{f_{I\to Z_1}}(S)}}
		\\&\leq \Pr_{I\sim_{1-\beta}[n],\ Z\sim \nu'^{\otimes I}}\sqbrac{\exists S\subseteq \bar{I}:\ \abs{\widehat{f_{I\to Z_1}}(S)}\geq \epsilon/2}  + \epsilon/2.
		\qedhere
	\end{align*}
\end{proof}	

\subsection*{Step 2: Direct Product Test}
Using Theorem 1.1 in~\cite{BKM23b}, by Lemma~\ref{lemma:lfcurr} we get the existence of constants $d\in \N, \delta'>0$, a set $S\subseteq [n]$, and a polynomial $g:\set{0,1}^n\to \R$ of degree at most $d$, and with 2-norm $\E_{X\sim \mu_p^{\otimes n} }\sqbrac{g(X)^2}\leq 1$, such that \[ \abs {\E_{X\sim \mu_p^{\otimes n}}\sqbrac{f(X)\cdot \chi_S(X)\cdot g(X)}} \geq \delta' .\]
	
This proves the first part of Theorem~\ref{thm:bkm23}.
It remains to show that if $\nu$ has some pairwise independent coordinate, it is possible to remove the function $g$ in the above expression.
	
\subsection*{Step 3: List Decoding.} This step follows Section 4.2 and Section 4.3 in~\cite{BKM23b}.

Using an iterative list-decoding process, we can find a constant $r\in \N$, and functions $\chi_{S_1}, \dots, \chi_{S_r}$, and constant degree polynomials $g_1,\dots, g_r,$ such that it is possible to ``replace" $f$ by $\sum_{i\in [r]}\chi_{S_i}\cdot g_i$ in Equation~\ref{eqn:test_pass} (and lose at most some constant factor in $\epsilon$).
Now, this implies that for some constant $\epsilon'>0$, and some indices $j_1,\dots,j_k\in [r]$, we have  
\begin{equation}\label{eqn:after_list}
	\abs{\E_{(X_1,\dots,X_k)\sim \nu^{\otimes n}} \sqbrac{\prod_{i=1}^k \chi_{S_{j_i}}(X_i) g_{j_i}(X_i) } }\geq \epsilon'.
\end{equation}
We remark that for the next step, some extra structure on $S_{j_i}$'s is needed, and ensuring that it holds requires the condition on $\textnormal{span}_{\F_2}(\supp(\nu))$ in Definition~\ref{defn:cont_BLR}.
	
\subsection*{Step 4: Invariance Principle Argument.}
This step follows Section 4.4, Section 4.5, and Section 4.6 in~\cite{BKM23b}. 

Assume, for the sake of contradiction, that $f$ is not correlated well with any $\chi_S$; that is, $\E_{X\sim \mu_p^{\otimes n}} \sqbrac{f(X)\cdot \chi_S(X)} \leq o_n(1)$ for each $S\subseteq [n]$.
Using this, it can be shown, roughly, that for each $i\in [k]$, the expectation $\E_{X\sim \mu_p^{\otimes n}}\sqbrac{\chi_{S_{j_i}}(X) g_{j_i}(X)} \leq o_n(1)$; note that for this conclusion to hold, we might have to modify $S_{j_i}$'s and $g_{j_i}$'s, however it is possible to do so while maintaining Equation~\ref{eqn:after_list}.

Now, by an invariance principle argument~\cite{MOO10, Mos10, Mos20}, very roughly, it is possible to replace the expectation in Equation~\ref{eqn:after_list} over $(X_1,\dots,X_k)\sim \nu^{\otimes n}$, by an expectation over $(Z_1,\dots,Z_k) \sim \mc N(0,\Sigma)^{\otimes n}$, where $\Sigma\in \R^{k\times k}$ is the (normalized) covariance matrix of $\nu$.
Finally, we use that some coordinate $X_{i^*}$ is pairwise independent of each $X_i$, for $i\not=i^*$.
Since the Gaussian distribution is determined by its covariance matrix, this implies that $Z_{i^*}$ is mutually independent of $(Z_i)_{i\not=i^*}$.
We have
\begin{align*}
	\epsilon' &\leq \abs{\E_{X=(X_1,\dots,X_k)\sim \nu^{\otimes n}} \sqbrac{\prod_{i=1}^k \chi_{S_{j_i}}(X_i) g_{j_i}(X_i) } }
	\\ &\approx  \abs{\E_{Z=(Z_1,\dots,Z_k)\sim \mc N(0,\Sigma)^{\otimes n}} \sqbrac{\prod_{i=1}^k \chi_{S_{j_i}}(Z_i) g_{j_i}(Z_i) } }
	\\&\approx \abs{\ \E_{Z_{i^*}\sim \mc N(0,1)^{\otimes n}} \sqbrac{ \chi_{S_{j_{i^*}}}(Z_{i^*}) g_{j_{i^*}}(Z_{i^*})} }\cdot  \abs{\ \E_{Z} \sqbrac{\prod_{i\in [k], i\not=i^*} \chi_{S_{j_i}}(Z_i) g_{j_i}(Z_i) } }
	\\&\approx \abs{\ \E_{X_{i^*}\sim \mu_p^{\otimes n}} \sqbrac{ \chi_{S_{j_{i^*}}}(X_{i^*}) g_{j_{i^*}}(X_{i^*})} }\cdot  \abs{\ \E_{Z} \sqbrac{\prod_{i\in [k], i\not=i^*} \chi_{S_{j_i}}(Z_i) g_{j_i}(Z_i) } }
	\\ &\leq o_n(1),
\end{align*}
which is a contradiction.
\qed


\section*{Acknowledgements}
We thank Amey Bhangale, Yang P. Liu, and Dor Minzer for discussions that helped this project.
Amey and Dor politely declined to be co-authors.

\bibliographystyle{alpha}
\bibliography{main.bib}



\end{document}

