\section{Queries vs. Bias Tradeoff}\label{sec:query_bias}

In this section, we analyze the relation between $p$ (the bias) and $k$ (the number of queries) for the existence of a distribution $\nu \in \mc D(p,k)$ with some pairwise independent coordinate, and with full even-weight support (see Definition~\ref{defn:distr_class}).

\subsection{Query Lower Bound}

We prove a lower bound on $k$ in terms of the $p$, as follows:
\begin{proposition}\label{prop:query_bias_lb}
	Let $k\in \N,\ p\in (0,1)$, and let $\nu \in \mc D(p,k)$ be a distribution that has some pairwise independent coordinate.
	Then, it holds that $k\geq 3$ and  $\frac{1}{k-1} \leq p \leq 1-\frac{1}{k-1}$.
\end{proposition}
\begin{proof}
	Let $X\sim \nu$, and let $i\in [k]$ be a pairwise independent coordinate under $\nu$.
	
	For $Z = \sum_{j\not=i} X_j$, we have by linearity of expectation, that $\E\sqbrac{X_i\cdot Z} = (k-1)p^2$.
	On the other hand, observe that if $X_i = 1$, then $Z = 1 \modt$ and so $Z\geq 1$.
	Hence,
	\[p = \E\sqbrac{X_i\cdot 1}\leq  \E\sqbrac{X_i\cdot Z} = (k-1)p^2, \]
	and we have $(k-1)p \geq 1$; in particular, this shows $k\geq 3$.
	
	For the upper bound on $p$, we consider the following cases:
	\begin{itemize}
		\item $k$ is odd: In this case, if $X_i=1$, then $Z = 1 \modt$ and so $Z\leq k-2$. Hence,
		\[ (k-1)p^2 = \E\sqbrac{X_i\cdot Z} \leq  \E\sqbrac{X_i\cdot (k-2)} = p(k-2),\]
		and we have $(k-1)p \leq (k-2)$, as desired.
		\item $k$ is even: In this case, observe that the distribution of the random variable $(1-X_1, \dots, 1-X_k)$ also satisfies the hypothesis of the proposition, with $p$ replaced by $1-p$. Hence, the above proof gives us $(k-1)\cdot (1-p) \geq 1$, as desired. \qedhere
	\end{itemize}
\end{proof}
\begin{remark}\label{remark:corner_case_full_support}
The proof of Proposition~\ref{prop:query_bias_lb} also shows that for $k>3$ and $p\in \set{\frac{1}{k-1}, 1-\frac{1}{k-1}}$, any distribution satisfying the assumptions of Proposition~\ref{prop:query_bias_lb} cannot have full even-weight support.
This is because if $p\in \set{\frac{1}{k-1}, 1-\frac{1}{k-1}}$, in all cases in the above proof, the random variable $Z$ must be constant under some value of $X_i$ (either $X_i=0$ or $X_i=1$); this cannot be the case for a distribution with full even-weight support when $k>3$.
\end{remark}
%%%%%%%%%%%%%%%%%%%%%%%%%%%%%%%%%%%%%%%%%%%%%%%%%%%%%%%%%%%%%%%%%%%%%%%%%%%%%%%%

\subsection{Query Upper Bound}

In this subsection, we shall prove the following proposition.
\begin{proposition}\label{prop:query_bias_ub}
	Let $k\geq 3$ be a positive integer, and let $p \in \sqbrac{\frac{1}{k-1},1-\frac{1}{k-1}}$ (note that this interval is non-empty for $k\geq 3$).
	
	Then, there exists a permutation-invariant\footnote{we say that a distribution $\nu$ over $\set{0,1}^k$ is \emph{permutation-invariant}, if for $X = (X_1,\dots,X_k)\sim \nu$, and any permutation $\pi:[k]\to [k]$, the distribution of $\brac{X_{\pi(1)}, \dots, X_{\pi(k)}}$ is the same as $\nu$.} and pairwise independent distribution $\nu (k,p) \in \mc D(p,k)$ (see Definition~\ref{defn:distr_class}).
	Furthermore, if $k=3$ or if $p\not\in \set{\frac{1}{k-1}, 1-\frac{1}{k-1}}$, then there exists such a distribution with full even-weight support.
\end{proposition}

The proof involves various cases, considered below in Lemma~\ref{lemma:ub_case_analysis} and Lemma~\ref{lemma:ub_add_ind_copies}.

\begin{lemma}\label{lemma:ub_case_analysis}
	Let $k\geq 4$ be a positive integer, and let $p \in \left[\frac{1}{k-1}, \frac{2}{k-1}\right)\cup \left( 1-\frac{2}{k-1}, 1-\frac{1}{k-1}\right]$ (note that this interval is contained in $\sqbrac{\frac{1}{k-1},1-\frac{1}{k-1}}$ for $k\geq 4$). 
	Then, there exists a pairwise independent distribution $\nu (k,p)\in \mc D(p,k)$.
	
	Moreover, if $p\not\in \set{\frac{1}{k-1}, 1-\frac{1}{k-1}}$, then there exists such a distribution with full even-weight support.
\end{lemma}
\begin{proof}
	Let $k\geq 4$ be a positive integer, and let $p \in \left[\frac{1}{k-1}, \frac{2}{k-1}\right)\cup \left( 1-\frac{2}{k-1}, 1-\frac{1}{k-1}\right]$.
	Let $s = \lfloor\frac{k}{2}\rfloor$; we shall exhibit a vector  $q = (q_0,q_1,\dots,q_s) \in [0,1]^{s+1}$ satisfying:
	\[
		\sum_{i=0}^s \binom{k}{2i}\cdot  q_i = 1,\ \quad\sum_{i=1}^s \binom{k-1}{2i-1}\cdot q_i = p,\quad \sum_{i=1}^s \binom{k-2}{2i-2}\cdot q_i = p^2.
	\]
	The distribution $\nu(p,k)$ is then defined 	by assigning probability $\begin{cases} q_{\abs{x}/2},& \abs{x}=0\modt\\0,& \abs{x}=1\modt \end{cases}$ to the point $x\in \set{0,1}^k$, where $\abs{x} = \sum_{i=1}^k x_i$.
	Note that the above properties correspond to $\nu(k,p)$ being a valid probability distribution supported on even-hamming-weight vectors, having marginals $\mu_p$, and pairwise independent coordinates.
	Furthermore, the distribution $\nu(p,k)$ has full even-weight support if and only if each $q_i \in (0,1)$.
	
	The vector $q$ is defined as follows in different cases (for brevity, we omit the verification of the above properties):
	\begin{enumerate}
		\item $k\geq 5$ is odd,  $p \in \left[\frac{1}{k-1}, \frac{2}{k-1}\right)$: Let $q_0 = 1 + \frac{kp^2}{2} - \frac{k^2p}{2(k-1)}$, $q_1 = \frac{(k-2)p-(k-1)p^2}{(k-1)(k-3)}$, $q_{(k-1)/2} = \frac{(k-1)p^2-p}{(k-1)(k-3)}$, and zero otherwise.
		\item $k\geq 5$ is odd,  $1-p\in \left[\frac{1}{k-1}, \frac{2}{k-1}\right)$: Let $q_0 = 1 + \frac{kp^2}{k-3} - \frac{k(2k-5)p}{(k-1)(k-3)}$, $q_{(k-3)/2} = \frac{3(k-2)p-3(k-1)p^2}{(k-1)(k-2)(k-3)}$, $q_{(k-1)/2} = \frac{(k-1)p^2-(k-4)p}{2(k-1)}$, and zero otherwise.
		\item $k\geq 4$ is even,  $p\in \left[\frac{1}{k-1}, \frac{2}{k-1}\right)$: Let $q_0 = \frac{(k-1)p^2-(k+1)p+2}{2}$,  $q_1 = \frac{p-p^2}{k-2}$, $q_{k/2} = \frac{(k-1)p^2-p}{k-2}$, and zero otherwise.
		\item $k\geq 4$ is even, $1-p\in \left[\frac{1}{k-1}, \frac{2}{k-1}\right)$: In this case, we define $\nu(k,p)$ to be the distribution obtained by flipping each coordinate of $\nu(k,1-p)$.
	\end{enumerate}
	
	Next, we show that if $p\not\in \set{\frac{1}{k-1}, 1-\frac{1}{k-1}}$, then such a distribution $\nu(p,k)$ with full even-weight support exists.
	We only need to do this for the first three cases, as the procedure described in the fourth case preserves the property of full even-weight support.
	 
	The same argument applies in all cases, and we present it for the first case: that is when $k\geq 5$ is odd, and $p \in \brac{\frac{1}{k-1}, \frac{2}{k-1}}$.
	We observe if $p \not= \frac{1}{k-1}$, each of the probabilities $q_0,q_1,q_{(k-1)/2}$ above lie in the interval $(0,1)$.
	Now, consider the equations
	\[
		\sum_{i=0}^s \binom{k}{2i}\cdot  \tilde{q}_i = 0,\ \quad\sum_{i=1}^s \binom{k-1}{2i-1}\cdot \tilde{q}_i = 0,\quad \sum_{i=1}^s \binom{k-2}{2i-2}\cdot \tilde{q}_i = 0.
	\]
	In these equations, the variables $\tilde{q}_0, \tilde{q}_1, \tilde{q}_{(k-1)/2}$ are linearly independent, and hence, there exists a vector $\tilde{q} \in \R^{s+1}$ satisfying these equations, which has all coordinates equal to 1, other than possibly $\tilde{q}_0, \tilde{q}_1, \tilde{q}_{(k-1)/2}$.
	Then, for some small $\delta > 0$, the vector $q+\delta\cdot \tilde{q}$ has all coordinates in $(0,1)$, and satisfies the required properties.	
\end{proof}

\begin{lemma}\label{lemma:ub_add_ind_copies}
	Let $k\geq 6$ be a positive integer, and let $p \in \sqbrac{\frac{2}{k-1},1-\frac{2}{k-1}} \setminus \set{\frac{1}{2}}$ (note that this interval is non-empty for $k\geq 6$).
	There, there exists a pairwise independent distribution $\nu (k,p)\in \mc D(p,k)$ with full even-weight support.
\end{lemma}
\begin{proof}
	Let $k\geq 6$ be a positive integer, and let $p \in \sqbrac{\frac{2}{k-1},1-\frac{2}{k-1}},\ p\not=\frac{1}{2}$.
	That is, for $q = \min\set{p,1-p} < \frac{1}{2}$, we have $k \geq 1 + \frac{2}{q}$.
	Let $\ell$ be the smallest odd integer satisfying $\ell > 1+\frac{1}{q} > 3$.
	Note that this satisfies $4\leq \ell \leq 3+\frac{1}{q} < 1 + \frac{2}{q} \leq k$, and we have $q \in \brac{\frac{1}{\ell-1}, \frac{2}{\ell-1}}$.	
	
	By Lemma~\ref{lemma:ub_case_analysis}, there exist pairwise independent distributions $\nu(\ell, p)$ and $\nu(\ell, 1-p)$, with full even-weight support.
	Let $\tilde{\nu}_0 = \nu(\ell, p)$, and let $\tilde{\nu}_1$ be the distribution obtained by flipping each coordinate of $\nu(\ell, 1-p)$.
	Since $\ell$ is odd, for each $b\in \set{0,1}$, it holds that $\tilde{\nu}_b$ has pairwise independent coordinates, each with marginal $\mu_p$, and such that $\supp(\tilde{\nu}_b) = \set{x\in \set{0,1}^k: \sum_{i=1}^k x_i = b \modt}$.
	Finally, we define $X \sim \nu(k,p)$ via the following random process: Let $(X_{\ell+1},\dots,X_k)\sim \mu_p^{\otimes\brac{k-\ell}}$, and with $Z = \sum_{i=\ell+1}^k X_i \modt$, we let $(X_1,\dots,X_\ell) \sim \tilde{\nu}_{Z}$.
	It is an easy check that this distribution satisfies the required properties.
\end{proof}

Finally, we prove Proposition~\ref{prop:query_bias_ub}.

\begin{proof}[Proof of Proposition~\ref{prop:query_bias_ub}]
	Note that it suffices to find such a distribution that is not necessarily permutation invariant, since averaging the distribution over all permutations preserves pairwise independence and full even-weight support.
	
	If $p=1/2$, for any $k\geq 3$, we let $\nu(k,p)$ be the uniform distribution on the set $\set{ x\in\set{0,1}^k : \sum_{i=1}^k x_i = 0 \modt}$.
	
	Now, for $k=3$, it must hold that $p=1/2$, in which case $\nu(k,p)$ is as above.
	For $k=4$ or $k=5$, and $p\not=1/2$, it must hold that $p \in \left[\frac{1}{k-1}, \frac{2}{k-1}\right)\cup \left( 1-\frac{2}{k-1}, 1-\frac{1}{k-1}\right]$, and the result follows from Lemma~\ref{lemma:ub_case_analysis}.
	For $k\geq 6$ and $p\not=\frac{1}{2}$, the result follows from Lemma~\ref{lemma:ub_case_analysis} and Lemma~\ref{lemma:ub_add_ind_copies}.
\end{proof}


%%%%%%%%%%%%%%%%%%%%%%%%%%%%%%%%%%%%%%%%%%%%%%%%%%%%%%%%%%%%%%%%%%%%%%%%%%%%%%%%
