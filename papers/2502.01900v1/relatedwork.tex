\section{Related work}
The problem of linearity testing has been extensively studied, starting with the work of Blum, Luby and Rubinfeld~\cite{BLR93}, who gave a test for the uniform distribution, in the 99\% regime.
The analysis of their test was later extended to the 1\% regime~\cite{BCHKS96, KLX10}.
Tests for linearity have been also been studied in the low-randomness regime, and in the setting of non-abelian groups~\cite{BSVW03,BCLR08,SW06}.

For the $p$-biased case, in the 99\% regime, Halevy and Kushilevitz~\cite{HK07} gave a 3-query linearity test, that only uses random samples from the $p$-biased distribution!
However, the test is not tolerant, makes queries that are not distributed according to $\mu_p^{\otimes n}$, and hence may reject functions that are very close to linear (with respect to the $p$-biased measure).
Tolerant testers were analyzed later~\cite{KS09, DFH19}.
More strongly, the work of Dinur, Filmus and Harsha~\cite{DFH19} gives $2^d$-query tolerant tester for $p$-biased testing of degree $d$ functions over $\F_2$, a problem which has been well studied over the uniform distribution~\cite{AKKLR05, BKSSZ10}.

As a part of their work on approximability of satisfiable constraint satisfaction problems~\cite{BKM22, BKM23a, BKM23b, BKM24a, BKM24b}, Bhangale, Khot and Minzer study the $p$-biased version of linearity testing, in the 1\% regime.
As mentioned before, they give a 4-query test for $p\in \brac{\frac{1}{3},\frac{2}{3}}$. 

David, Dinur, Goldberg, Kindler and Shinkar~\cite{DDGKS17} study linearity testing on the $k$-slice (vectors of hamming-weight $k$), denoted by $L_{k,n}$, of the $n$-dimensional boolean hypercube, for even integers $k$. 
They show that if $f:\set{0,1}^n\to \set{-1,1}$ is such that $f(x\oplus y) = f(x)f(y)$ with probability $1-\epsilon$ over $x,y,x\oplus y$ (conditioned on all lying in $L_{k,n}$), then $f$ agrees with a linear function on $1-\delta$ fraction of $L_{k,n}$, where $\delta = \delta(\epsilon)\to 0$ as $\epsilon \to 0$.
In a recent work, Kalai, Lifshitz, Minzer and Ziegler~\cite{KLMZ24} prove a similar result for the $n/2$-slice, in the 1\% regime.