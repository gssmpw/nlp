\section{Putting Everything Together}\label{sec:putting_together}

We are now ready to prove our main result.

\begin{proof}[Proof of Theorem~\ref{thm:intro_querybias_main_thm}]
Let $p\in (0,1)$.

\begin{enumerate}
	\item Consider any positive integer $k > 1 + \frac{1}{\min\set{p,1-p}} \geq 3$ (or $k=3$ with $p=\frac{1}{2}$).
		By Proposition~\ref{prop:query_bias_ub}, there exists a pairwise independent distribution $\nu\in \mc D(p,k)$ with full even-weight support.
		The result now follows by Theorem~\ref{thm:bkm23_in_section}.
	
	\item Suppose that $k\geq 3$ with $p=\frac{1}{k-1}$, or $k\geq 4$ is even with $\ p = 1-\frac{1}{k-1}$.
	In these cases, we observe that the distribution $\nu\in \mc D(p,k)$ constructed in Lemma~\ref{lemma:ub_case_analysis} is pairwise independent, and contains BLR (see Definition~\ref{defn:cont_BLR}):
	\begin{enumerate}
		\item If $k\geq 3, p=\frac{1}{k-1}$, the distribution $\nu$ contains all vectors in $\set{0,1}^k$ of hamming-weights 0 and 2 in its support. In this case, Definition~\ref{defn:cont_BLR} is satisfied with $\tilde{b}=0$ and $\tilde{z}$ as the all-zeros vector.
		\item If $k\geq 4$ is even, and $p=1-\frac{1}{k-1}$, the distribution $\nu$ contains all vectors in $\set{0,1}^k$ of hamming-weights $k-2$ and $k$ in its support. In this case, Definition~\ref{defn:cont_BLR} is satisfied with $\tilde{b}=1$ and $\tilde{z}$ as the all-ones vector.
	\end{enumerate}
	The result now follows by Theorem~\ref{thm:bkm23_in_section}.
	
	\item Suppose that $k < 1 + \frac{1}{\min\set{p,1-p}}$ is a positive integer, and let $\nu \in \mc D(p,k)$.
		We perform the following operation on the distribution $\nu$:
		if $i,j\in [k],\ i\not=j$ are such that $\Pr_{X\sim \nu}[X_i=X_j]=1$, we remove coordinates $i,j$ from $\nu$, and repeat until no such pairs remain.
		
		Finally, we are left with a distribution $\tilde{\nu}$ on $\tilde{k}\leq k$ coordinates.		
		We consider the following two cases:
		\begin{enumerate}
			\item Suppose that $\tilde{k}=0$. In this case, for every $n\in \N,$ and every $f:\set{0,1}^n\to \set{-1,1}$, it holds that $\E_{X\sim \nu^{\otimes n}} \sqbrac{\prod_{i=1}^kf(X_i)} = 1$, since the $k$ terms in the product cancel out in pairs.
			Hence, it suffices to show the existence of a function $f:\set{0,1}^n\to \set{-1,1}$ satisfying $\abs{\E_{X\sim \mu_p^{\otimes n}}\sqbrac{f(X)\cdot \chi_S(X)}} \leq o_n(1)$ for every $S\subseteq [n]$.
			Note that a (uniformly) random function $f:\set{0,1}^n\to \set{-1,1}$ satisfies this with high probability, by an argument similar to the one at the end of Section~\ref{sec:counter_eg_2} (a random function can be thought of as rounding the constant zero function as in Section~\ref{sec:counter_eg_2}).
			
			\item Now, suppose that $\tilde{k}\not=0$. 
			Then, it holds that $\tilde{\nu} \in \mc D(p,\tilde{k})$, and by Proposition~\ref{prop:query_bias_lb}, we have that $\tilde{\nu}$ has no pairwise independent coordinate.
			Now, by Theorem~\ref{thm:counter_eg} there exists a constant $\alpha>0$, such that for every large $n\in \N$, there exists a function $f:\set{0,1}^n\to \set{-1,1}$ such that \[ \abs{\E_{(X_1,\dots,X_k)\sim \nu^{\otimes n}} \sqbrac{\prod_{i\in [k]} f(X_i)} } = \abs{\E_{(X_1,\dots,X_{\tilde{k}})\sim \tilde{\nu}^{\otimes n}} \sqbrac{\prod_{i\in [\tilde{k}]} f(X_i)} }\geq \alpha,\]
			and such that $ \abs {\E_{X\sim \mu_p^{\otimes n}}\sqbrac{f(X)\cdot \chi_S(X)}} \leq o_n(1)$ for every $S\subseteq [n]$. \qedhere
			\end{enumerate}
\end{enumerate}
\end{proof}

%%%%%%%%%%%%%%%%%%%%%%%%%%%%%%%%%%%%%%%%%%%%%%%%%%%%%%%%%
\subsection{A Corner Case}\label{sec:corner_case}

In the above proof, we leave the case of odd $k\geq 5$ and $p=1-\frac{1}{k-1}$.
This turns out to be very interesting, and we discuss it next.
For the remainder of this section, we fix such a $k$ and $p$.

In this case, the pairwise independent distribution $\nu\in \mc D(p,k)$ constructed in Lemma~\ref{lemma:ub_case_analysis}, is supported on vectors of hamming weights 0 and $k-1$ (and does not contain BLR as in Definition~\ref{defn:cont_BLR}).
In particular, for every $x\in \supp(\nu)$, it holds that $\sum_{i=1}^k x_i = 0 \modk$.
For this reason, as we show next, the best we can expect from the test $\Lin(\nu)$, is to guarantee correlation with a character over $\Z/(k-1)\Z$, and this is indeed true.

\begin{definition}\label{defn:char}(Characters over $\Z/(k-1)\Z$)
Let $\omega$ be a primitive $(k-1)$\textsuperscript{th} root of unity.
For every $0\leq r\leq k-2$, we define the function $\phi_r: \set{0,1} \to \C$ as $\phi_r(x)=\omega^{rx}$.

For every $n\in \N$, and every integers $0\leq r\up{1},\dots,r\up{n}\leq k-2$, we define the product character $\phi_{r\up{1},\dots,r\up{n}}:\set{0,1}^n\to \C$ by $\phi_{r\up{1},\dots,r\up{n}}(x) = \prod_{j=1}^n\phi_{r\up{j}}(x\up{j}) = \omega^{\sum_{j=1}^n r\up{j}x\up{j}}$.
\end{definition} 

Now, consider the test $\Lin(\nu)$.
Observe that any character $f = \phi_{r\up{1},\dots,r\up{n}}$ passes this test with probability 1:
\[ \E_{X\sim\nu^{\otimes n}}\sqbrac{\prod_{i\in [k]}f(X_i)} = \prod_{j=1}^n \E_{Y\sim\nu}\sqbrac{\prod_{i\in [k]}\phi_{r\up{j}}(Y_i)} = \prod_{j=1}^n \E_{Y\sim\nu}\sqbrac{\omega^{r\up{j}\cdot \brac{\sum_{i\in [k]}{Y_i}}}} = 1.\]
Next, we claim that characters explain the success of $\Lin(\nu)$ for any function $f$:

\begin{theorem}
	For every constant $\epsilon > 0$, there exists a constant $\delta >0$ such that for every large enough $n\in \N$, the following is true:
	
	Let $f:\set{0,1}^n\to[-1,1]$ be a function such that $\abs{\E_{X\sim \nu^{\otimes n}} \sqbrac{\prod_{i=1}^k f(X_i)} }\geq \eps.$
	Then, there exist integers $0\leq r\up{1},\dots,r\up{n}\leq k-2$, such that
	\[ \abs {\E_{X\sim \mu_p^{\otimes n}}\sqbrac{f(X)\cdot \phi_{r\up{1},\dots,r\up{n}}(X)}} \geq \delta .\]
\end{theorem}
\begin{proof}
	The result follows from the work of Bhangale, Khot, Liu and Minzer~\cite{BKLM24a, BKLM24b}, and we omit the details.
	Very roughly speaking, the proof follows a similar strategy as in Section~\ref{sec:bkm_sketch}: first show that $f$ has good correlation with a character under random restrictions; then, use this to show that $f$ has good correlation with character times a low-degree function; finally, use that $\nu$ is pairwise independent to get rid of the low-degree function.
\end{proof}

Finally, we present an alternative solution to deal with this corner case of odd $k\geq 5$ and $p=1-\frac{1}{k-1}$.
Instead of the test $\Lin(\nu)$, we can perform the following test:

Let $f:\set{0,1}^n\to [-1,1]$, and let $\nu' \in \mc D(1-p,k)=\mc D(\frac{1}{k-1},k)$ be the pairwise independent distribution from Lemma~\ref{lemma:ub_case_analysis}.
\begin{enumerate}
	\item Sample $X=(X_1,\dots,X_k)\sim \nu'^{\otimes n}$.
	\item Let $X'$ be the vector obtained by negating each of the $kn$ coordinates of $X$.
	\item Query $f$ on $X'_1,\dots,X'_k$ and accept if and only if $\prod_{i\in [k]} f(X'_i) = 1$.  
\end{enumerate}
Each query $X_i'$ of the above test is distributed according to $\mu_p^{\otimes n}$, and the analysis of the test simply follows from the analysis for $\Lin(\nu')$ in Theorem~\ref{thm:intro_querybias_main_thm}.
The drawback here, though, is that the test does not accept all linear functions with probability 1, but only functions of the form $(-1)^{\abs{S}}\cdot \chi_S,$ for $S\subseteq[n]$.

