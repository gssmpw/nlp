\section{Introduction}

Elliptic partial differential equations (PDEs) form the cornerstone of numerous scientific and engineering applications, from steady-state heat conduction to electrostatics and fluid dynamics. These equations also arise naturally when implicit time-stepping schemes are employed for parabolic or hyperbolic problems, where stability and accuracy requirements often necessitate the solution of elliptic systems at each time step. A fundamental challenge in solving these equations lies in handling complex geometries while maintaining computational efficiency and numerical accuracy.

Traditional numerical approaches for elliptic PDEs broadly fall into three categories:
\begin{itemize}
\item Body-fitted mesh methods (finite elements, boundary-fitted finite differences) \cite{barrett1987fitted} offer high accuracy but require complex mesh generation and management, particularly challenging for moving boundaries or topology changes.
\item Meshless methods (radial basis functions \cite{fornberg2015solving}, boundary integral methods \cite{mckenney1995fast,zhong2018implicit,zhou2024correction,ying2013kernel}, spectral method \cite{gu2021efficient}) provide geometric flexibility and often lead to dense linear systems and face challenges in handling variable coefficients.
\item Embedded boundary methods such as immersed finite difference/element \cite{peskin2002immersed,leveque1994immersed}, cut-cell methods \cite{chen2023arbitrarily,chen2021adaptive,hansbo2002unfitted,hansbo2014cut}, Difference Potentials Method (DPM) \cite{ryaben2012method}, combine the efficiency of structured grids with geometric flexibility but might struggle with accuracy near boundaries and stability issues.
\end{itemize}

Among these approaches, boundary integral methods (BIM) deserve special attention as they share important similarities with our proposed approach. BIM reduce volumetric PDEs to boundary integral equations, achieving dimension reduction and naturally handling unbounded domains. These methods excel in problems with constant coefficients where fundamental solutions are known analytically, and they provide high accuracy by incorporating the exact representation of solutions through Green's functions.
Our proposed approach shares the dimensional reduction advantage of BIM but operates in a discrete setting using lattice Green's functions instead of continuous ones. This discrete framework offers several distinct advantages: it naturally aligns with computational grids, avoids singular integral evaluations common in BIM, and maintains the ability to handle variable coefficients through the underlying difference potentials framework. 

Difference Potentials Method (DPM) \cite{ryaben2012method} offers a general framework of solving elliptic equations in arbitrary geometry using unfitted structured meshes. By the introduction of auxiliary domains and the discrete Green's function defined on the auxiliary domain, the discrete elliptic equations in the original domain can be reduced to boundary algebraic equations at grid points near the continuous boundary.  However, existing implementations face significant computational challenges, particularly in three dimensions. The primary bottleneck lies in constructing difference potentials operators, which traditionally requires multiple volume solves in the auxiliary domain. Typically, fast Poisson solvers such as those relied on FFTs are employed for acceleration of those multiple volume solves, which we aim to bypass in this work.

This work builds on previous work of using local basis function derived the Galerkin difference method \cite{banks2016galerkin} to construct difference potentials operators \cite{xia2023local}. The main novelty is the integration lattice Green's functions (LGFs) into the DPM framework. LGFs, which serve as fundamental solutions for discrete elliptic operators on infinite lattices, enable explicit representation of boundary potentials without relying on solving auxiliary problems defined in finite domains. This integration provides several key advantages, including reduction in computational complexity, matrix-free implementation capabilities for better memory efficiency and natural handling of unbounded domains.

Lattice Green's functions have been studied in many applications such as in \cite{liska2016fast,dorschner2020fast,liska2014parallel,cserti2000application,joyce2004exact,borwein2013lattice}.  In this work, we focus on Cartesian lattices, while triangular, honeycomb lattices are also possible \cite{zhao2011extension}. While lattice Green's functions have been previously studied for simple geometries and infinite domains, their systematic integration with difference potentials for complex geometries represents a significant advancement. 

This idea of using boundary algebraic equations to solve elliptic partial differential equations numerically can be traced back to a NASA technical report \cite{saltzer1958discrete}. Our approach builds on difference potentials and earlier work by Martinsson et al on boundary algebraic equations (BAE) \cite{martinsson2009boundary}. The key idea of BAE is to mimic its continuous counterpart, the boundary integral method, where the fundamental solution of the discrete Poisson equation is taken as the lattice Green's function. Discrete single and double layer potentials are defined similarly as in the boundary integral method, where a system of boundary algebraic equations are solved for unknown density at the boundary lattice points. One major difference is that no singularity occur in the BAE when the source point coincides with the target. Interior or exterior solutions at any point then can be obtained by summation of the source-target interactions, using discrete layer potentials. 

In the current work, we do not require grid nodes to align with boundary curves, hence the method is termed as ``unfitted'' BAE. We extend the BAE framework to handle arbitrary geometries through a rigorous theoretical foundation. In this work, BAE is also capable of handling nonhomogeneous source terms in finite domains via Fast Fourier Transform. For infinite domains and nonhomogeneous source functions, fast summation techniques such as fast multipole method \cite{beatson1997short} will be needed and studied in future work.

Our method establishes equivalence of boundary equations in DPM and boundary equations in BAE, so as to construct boundary equations directly without relying on auxiliary volume solves. Such direct formulation is also suitable for matrix-free implementation, offering memory efficiency for large-scale computations. Inheriting merits of both DPM and BAE, our method naturally admits property of dimension reduction, and offers a robust foundation for multiphysics simulations in complex domains.  The framework naturally extends to unbounded domains and provides pathways for applications to more challenging systems such as high-frequency Helmholtz problems, Stokes flows and etc.

The rest of the manuscript is organized as follows: Section \ref{sec:dpm} presents the fundamental theory of difference potentials and their extension to infinite domains using LGFs. Section \ref{sec:bae} develops direct and indirect boundary algebraic equation formulations, analyzing their spectral properties and establishing equivalence relationships. Section \ref{sec:numerical_results} validates the method through numerical experiments for Poisson and modified Helmholtz equations in 3D implicitly defined geometries. Section \ref{sec:conclusion} concludes with future directions and potential applications.
