\section{Difference Potentials Method}\label{sec:dpm}

In this work, we focus on efficient and accurate numerical solution of the elliptic partial differential equations in the form of:
\begin{subequations}\label{eqn:elliptic}
\begin{align}
-\Delta u +\sigma u &= f,\quad x\in\Omega\label{eqn:pde}\\
u &= g,\quad x\in\partial\Omega\label{eqn:pde_bc}
\end{align}
\end{subequations}
with $\sigma\geq0$. The geometry of $\Omega\subset\mathbb{R}^3$ is arbitrarily shaped and the PDE in \eqref{eqn:elliptic} is assumed well-posed. For time-dependent problems, when implicit time stepping is employed, $\sigma$ will be $1/\Delta t$ or $1/\Delta t^2$ where $\Delta t$ denotes the time stepping size. In this work, we will demonstrate the developed numerical algorithm with Dirichlet boundary condition \eqref{eqn:pde_bc}. Other types of boundary conditions can be handled similarly as discussed in \cite{xia2023local}.

\subsection{Difference Potentials in the finite domain}

Given a bounded domain $\Omega\subset \mathbb{R}^3$ of arbitrary shapes, we first embed it into a computationally simple domain $\Omega^0$ such as a cube. Then $\Omega^0$ is discretized using a Cartesian mesh. The grid points will be denoted: $M^0$ for points in $\Omega^0$, $M^+$ for points in $\Omega$, $M^-:=M^0\backslash M^+$. The stencil points $N^\pm$ are defined as all points needed in the formulation of a finite difference operator for points in $M^\pm$, respectively, and $N^0:=N^+\cup N^-$. In the second order case in 3D, $N^\pm$ consist of the seven-point stencil:
\begin{align}
N^\pm = \left\{(x_{i\pm 1},y_j,z_k),(x_i,y_{j\pm1},z_k),(x_i,y_j,z_{k\pm1}),(x_i,y_j,z_k) \ \big \vert\  (x_i,y_j,z_k)\in M^\pm\right\}.
\end{align}
The point sets $N^\pm$ have an intersection $\gamma:=N^+\cap N^-$ known as \emph{the discrete grid boundary} and $\gamma$ can be divided into two subsets, where $\gamma_+$ corresponds to the interior of $\Omega$ and $\gamma_-$ the exterior of $\Omega$. See Figure~\ref{fig:points} for an illustration of these two point sets $\gamma^\pm$ in 2D.

\begin{figure}[htbp]
\centering
\includegraphics[width=0.4\textwidth]{fig/point_sets}
\caption{Example of point sets in 2D (dot: $\gamma_{+}$, circle: $\gamma_-$)}\label{fig:points}
\end{figure}

With these point sets defined, we can discretize the PDE \eqref{eqn:pde} on point set $M^+\subset\Omega$
\begin{align}\label{eqn:discrete}
L_hu_{jkl}:= -\Delta_h u_{jkl}+\sigma u_{jkl} = f_{jkl},\quad x_{jkl}\in M^+
\end{align}
where $\Delta_h$ will be taken as the 7-point central finite difference stencil.
Note that values at $\gamma_-$ is unknown and boundary conditions need to be imposed for boundary closures, which will be discussed in details later. Given the discretization \eqref{eqn:discrete}, we introduce the auxiliary problem defined on the point set $N^0$
\begin{subequations}\label{eqn:aux}
\begin{align}
L_h u_{jkl} &= q_{jkl},\quad x_{jkl}\in M^0\\
u_{jkl} &= 0,\quad x_{jkl}\in N^0\backslash M^0
\end{align}
\end{subequations}
where $q_{jkl}$ is some grid function defined on $M^0$. We will denote the inverse operation of finding solutions to the difference equations \eqref{eqn:aux} with the given boundary condition as $u_{jkl}=G_hq_{jkl}$.

For a finite and computationally simple auxiliary domain $\Omega^0$, the operator $G_h$ is taken to be some fast inversion of the $L_h$. For example, FFT can be used to solve $L_h u_{jkl}=q_{jkl}$ for grid function $q_{jkl}$ and no explicit knowledge of the Green's function for the auxiliary domain $\Omega^0$ is needed. Now we will define the particular solution obtained with source function $q_{jkl}=\chi_{M^+}f_{jkl}$:
\begin{align}\label{eqn:ps}
G_hf_{jkl} = G_h[\chi_{M^+}f_{jkl}],\quad x_{jkl}\in M^0
\end{align}
where $\chi_{M^+}$ denotes a characteristic function for set $M^+$, and the difference potentials obtained with source function $q_{jkl}=\chi_{M^-}L_hw_{jkl}$
\begin{align}\label{eqn:dp}
P_{N^+\gamma}w_{jkl}=G_h[\chi_{M^-}L_hw_{jkl}]
\end{align}
where $w_{jkl}$ is any grid function defined on $N^+$.

It can be checked that the superposition of the particular solution \eqref{eqn:ps} and the difference potential \eqref{eqn:dp} satisfies the difference equation \eqref{eqn:discrete} in $M^+$.

The following key theorem establishes the dimension reduction relation from the volumetric difference equation \eqref{eqn:discrete} to algebraic equations at grid points only near the boundary, which can be found in many previous work, e.g. \cite{xia2023local,ryaben2006algorithm,medvinsky2012method,albright2015high,epshteyn2014algorithms,epshteyn2012upwind,epshteyn2019efficient}.
\begin{theorem}
The discrete equation \eqref{eqn:discrete} for points in $M^+$ is equivalent to the following boundary equation with projections:
\begin{align}\label{eqn:bep}
u_\gamma - P_\gamma u_\gamma = G_hf_\gamma.
\end{align}
where $P_\gamma u_\gamma:=Tr_{\gamma}G_h[\chi_{M^-}L_hu_\gamma]$ and $G_hf_\gamma:=Tr_\gamma G_h[\chi_{M^+}f_h]$ and $Tr_{\gamma}$ denotes the trace operator on the set $\gamma$, $\chi_{M^-}$ denotes the characteristic function for the set $M^-$.
\end{theorem}

Each column of the operator $P_\gamma$ can be constructed using a unit density
\begin{align}\label{eqn:unit_density}
u_{j^*k^*l^*}(j,k,l) = \left\{
\begin{array}{cc}
1,& j=j^*,k=k^*,l=l^*,\\
0,&\mbox{ elsewhere.}
\end{array}
\right.
\end{align}
and the corresponding column in $P_\gamma$ for point $x_{j^*k^*l^*}$ in the set $\gamma$ is thus
\begin{align}
P_\gamma Tr_{\gamma}u_{j^*k^*l^*}=Tr_{\gamma}G_h[\chi_{M^-}L_hu_{j^*k^*l^*}].
\end{align}
Essentially we are solving the auxiliary problem
\begin{subequations}\label{eqn:unit_finite_aux_prob}
\begin{align}
L_h u_{jkl} &= \chi_{M^-}L_hu_{j^*k^*l^*}\\
u_{jkl} &= 0 
\end{align}
\end{subequations}
and inject $Tr_{\gamma}u_{jkl}$ as the corresponding column for $P_\gamma$.

The boundary equations with projections \eqref{eqn:bep} can be further reduced to the interior set $\gamma_+$ only, as illustrated by the following theorem.
\begin{theorem}
The boundary equation with projections \eqref{eqn:bep} defined on $\gamma$ is equivalent to the following reduced boundary equation with projections \eqref{eqn:rbep} defined on $\gamma_+$.
\begin{align}\label{eqn:rbep}
u_{\gamma_+} - P_{\gamma_+} u_\gamma = G_hf_{\gamma_+}.
\end{align}
where $P_{\gamma_+} u_\gamma:=Tr_{\gamma_+}G_h\chi_{M^-}L_h[u_\gamma]$ and $G_hf_{\gamma_+}:=Tr_{\gamma_+} G_h[\chi_{M^+}f_h]$.
\end{theorem}
The proof can be found, for example, in \cite{epshteyn2019efficient}.

The reduced boundary equations with projections \eqref{eqn:rbep} is equivalent to the finite difference equations \eqref{eqn:discrete}. To close the linear system, one can use the local basis functions approach developed in \cite{xia2023local}. For example, the boundary condition can be discretized as
\begin{align}\label{eqn:bc}
\sum_{x_{jkl}\in\gamma} u_{jkl} \Phi_{jkl}(x^*,y^*,z^*) = g(x^*,y^*,z^*)
\end{align}
in the case of Dirichlet boundary condition, 
where $(x^*,y^*,z^*)$ is a boundary point in a cut cell and $\Phi_{jkl}$ is a nodal basis function defined at point $x_{jkl}$. In the second order case, we can take $\Phi_{jkl}$ to be the standard hat function in 3D, while higher order version will take the form of difference Galerkin basis function studied in \cite{jacangelo2020galerkin,banks2016galerkin}.

Now we have a closed square system with $|\gamma|$ number of equations and $|\gamma|$ number of unknowns, where $|\gamma|$ is the cardinality of point set $\gamma$.
\begin{subequations}\label{eqn:inhom}
\begin{align}
u_{\gamma_+} - P_{\gamma_+} u_\gamma &= G_hf_{\gamma_+}\\
\sum_{x_{jkl}\in\gamma} u_{jkl} \Phi_{jkl}(x^*,y^*,z^*) &= g(x^*,y^*,z^*)
\end{align}
\end{subequations}

The following lemma will serve to reduce the nonhomogeneous boundary equations \eqref{eqn:inhom} to homogeneous ones.

\begin{theorem}\label{lem:dp_gh}
The trace of the difference potential with density of the particular solution $G_hf_\gamma$ is 0, i.e.
\begin{align}
P_\gamma [G_hf_\gamma] = 0.
\end{align}
\end{theorem}

\begin{proof}
The difference potential for any density $u_\gamma$ admits the following definition
\begin{align}
P_{N^+\gamma}u_\gamma:=G_h \chi_{M^-}L_hu,\quad x_{jkl}\in N^+
\end{align}
where $u$ is some extension of the density $u_\gamma$ to the grid $N$, i.e. $Tr_{\gamma}u = u_\gamma$. It can be argued that the exact form of extension does not change the value of the difference potential. To see this, assume we have $u_1\neq u_2$ but $Tr_\gamma u_1=Tr_{\gamma} u_2 = u_\gamma$ then in $N^+$ and define $w=u_1-u_2$, then $Tr_{\gamma}w=0$, and
\begin{align}
L_hw = L_h(u_1-u_2) = 0,\quad x_{jkl}\in M^+
\end{align}
due to the restriction operator $\chi_{M^-}$. The above linear system admits only zero solution in $M^+$ as the boundary is zero. For $M^-$, similar arguments also show $w=0$ in $M^-$.
Hence the difference potential is the same no matter how the extension is formed. Other proofs of this result can be found in Ryabenkii's book as well.

Now for $P_\gamma [G_hf_\gamma]$, we will choose the particular solution as the extension $[G_h\chi_{M^+}f]$ then the difference potential for density $[G_hf_\gamma]$ will be computed as
\begin{align}
L_h[P_{N^+\gamma}G_hf_\gamma] = \chi_{M^-}L_h[G_h\chi_{M^+}f_h]=0
\end{align}
since $M^+$ and $M^-$ are dis-adjoint. When homogeneous BC are imposed for the auxiliary problem, the difference potential of the particular solution will be 0.
\end{proof}

Now we define the homogeneous density
\begin{align}\label{eqn:vgamma}
v_\gamma = u_\gamma - G_hf_\gamma
\end{align} 
and the following proposition holds.

\begin{proposition}\label{prop:vgamma}
The reduced boundary equation with projections \eqref{eqn:rbep} is equivalent to the boundary equations for $v_{\gamma}$ defined in \eqref{eqn:vgamma}:
\begin{align}\label{eqn:beq}
v_{\gamma_+} - P_{\gamma_+} v_\gamma = 0
\end{align}
\end{proposition}
\begin{proof}
``$\Rightarrow$'': 
By definition of $v_\gamma:=u_\gamma - G_hf_\gamma$, $u_\gamma=v_\gamma+G_hf_\gamma$, then the reduced boundary equation gives
\begin{subequations}
\begin{align}
u_{\gamma_+} - P_{\gamma_+} u_\gamma &= G_hf_{\gamma_+}\\
\Rightarrow v_{\gamma_+}+G_hf_{\gamma_+} - P_{\gamma_+} (v_\gamma+G_hf_\gamma) &= G_hf_{\gamma_+}\\
\Rightarrow v_{\gamma_+} - P_{\gamma_+} (v_\gamma+G_hf_\gamma) &= 0\\
\Rightarrow v_{\gamma_+} - P_{\gamma_+} v_\gamma &= P_{\gamma_+}G_hf_\gamma
\end{align}
\end{subequations}
The last equation is by Lemma~\ref{lem:dp_gh}.

``$\Leftarrow$'': Similarly, from the definition of $v_\gamma$,
\begin{subequations}
\begin{align}
v_{\gamma_+} - P_{\gamma_+} v_\gamma &= 0\\
\Rightarrow u_{\gamma_+} - G_hf_{\gamma_+} - P_{\gamma_+} [u_\gamma - G_hf_\gamma] &=0\\
\Rightarrow u_{\gamma_+} - G_hf_{\gamma_+} - P_{\gamma_+} u_\gamma &=0\\
\Rightarrow u_{\gamma_+} - P_{\gamma_+} u_\gamma &=G_hf_{\gamma_+}
\end{align}
due to Lemma~\ref{lem:dp_gh} $P_{\gamma_+}G_hf_\gamma=0$ as well.
\end{subequations}
\end{proof}

By Proposition~\ref{prop:vgamma}, the boundary systems \eqref{eqn:inhom} is equivalent to
\begin{subequations}\label{eqn:hom}
\begin{align}
v_{\gamma_+} - P_{\gamma_+} v_\gamma &= 0\\
\sum_{x_{jkl}\in\gamma} v_{jkl} \Phi_{jkl}(x^*,y^*,z^*) &= g(x^*,y^*,z^*)-\sum_{x_{jkl}\in\gamma}[G_hf_{\gamma}]_{jkl} \Phi_{jkl}(x^*,y^*,z^*)
\end{align}
\end{subequations}
where we can solve $v_\gamma$. Similar approach has been widely adopted in the boundary integral equation method for nonhomogeneous elliptic equations, see for example \cite{mayo1984fast}.

Once $v_\gamma$ is obtained, the density $u_\gamma$ is given by
\begin{align}
u_\gamma = v_\gamma + G_hf_\gamma,
\end{align}
and the approximation of the numerical solution to \eqref{eqn:elliptic} is given by the following discrete generalized Green's formula:
\begin{align}\label{eqn:discrete_green_u}
u_{jkl} = G_h[\chi_{M^-}L_hu_\gamma] + G_h[\chi_{M^+}f_h], \quad x_{jkl}\in N^+
\end{align}

\begin{remark}
Due to Lemma~\ref{lem:dp_gh}, we can simply replace the inhomogeneous density $u_\gamma$ with the homogeneous density $v_{\gamma}$ in the Green's formula.
\begin{align}\label{eqn:discrete_green_v}
u_{jkl} = G_h[\chi_{M^-}L_hv_\gamma] + G_h[\chi_{M^+}f_h], \quad x_{jkl}\in N^+,
\end{align}
which basically implies that we only need to solve two linear systems: one homogeneous system that accounts for the boundary data and one for the nonhomogeneous source functions.
\end{remark}

The most computationally expensive part in the above approximation of the discrete Generalized Green's formula lies in the construction of the density $u_\gamma$, which relies on the operator $P_{\gamma_+}$. Different approaches have been developed to obtain the operator, such as combining extension operators of Taylor's form with spectral method on the continuous boundary \cite{ryaben2012method,medvinsky2012method,epshteyn2012upwind}, etc., and method based on the Galerkin difference basis functions \cite{xia2023local}.

In \cite{xia2023local}, we demonstrated how to combine local Galerkin difference basis functions with difference potentials method and argued that the computational cost for constructing $P_{\gamma}$ or $P_{\gamma_+}$ would scale like a formidable $\mathcal{O}(N^5\log(N))$ in 3D, since we need to compute the difference potentials at $|\gamma|\sim\mathcal{O}(N^2)$ points in the set $\gamma$ using unit density defined at each point in $\gamma$. Here $N$ is the number of unknowns in either $x-$, $y-$ or $z-$ direction in the uniform mesh for the auxiliary domain. The computation of each difference potential corresponding to the unit density would be $\mathcal{N^3}\log(N)$ if FFT is employed to solve the auxiliary problem. In this work, we will continue using the Galerkin difference basis function and show how to avoid the computational cost of constructing $P_{\gamma}$ or $P_{\gamma_+}$ and how to obtain the operators directly from lattice Green's functions.


\subsection{Difference Potentials in the infinite domain}

In this subsection, we will focus on the homogeneous equation as we discussed in the previous subsection:
\begin{align}
Lv:=-\Delta v+\sigma v = 0, \quad x\in \Omega
\end{align}
in the same domain $\Omega$ of arbitrary shapes and with the same boundary condition as in $\eqref{eqn:discrete}$.

Instead of using the finite auxiliary domain $\Omega^0$, we now turn to the free space $\mathbb{R}^3$ as the auxiliary domain and introduce the infinite lattice ${\mathbb{Z}h}^3$ with spacing $h$. Here we assume that the infinite grid coincide with the finite grid on $M^+$. The point sets $N^\pm, M^\pm$ are defined similarly as in the finite auxiliary domain. The point set $\gamma$, $\gamma_\pm$ will be identically defined. To construct the difference potential operator $\widehat{P}_{\gamma}$ and $\widehat{P}_{\gamma_+}$, we use similar auxiliary problem as in \eqref{eqn:unit_finite_aux_prob} and use the unit density $v_{j^*,k^*,l^*}$. In other words, we solve 
\begin{subequations}\label{eqn:unit_infinite_aux_prob}
\begin{align}
\widehat{L}_h \hat{v}_{jkl} &= \chi_{M^-}\widehat{L}_h\hat{v}_{j^*k^*l^*},\\
\lim_{jkl\rightarrow\infty}v_{jkl} &= 0,
\end{align}
\end{subequations} 
and $\hat{v}_{jkl}=\widehat{G}_h\chi_{M^-}\widehat{L}_h\hat{v}_{j^*k^*l^*}$,
where the ``hat'' denotes that the operator is obtained in the free space. The fundamental solution to \eqref{eqn:unit_infinite_aux_prob} is commonly known as the lattice Green's function. In the case of infinite lattices, we will denote the lattice Green's function as $G_{\sigma,h}(jkl)$ and it satisfies
\begin{align}\label{eqn:lgf_delta}
\widehat{L}_h\widehat{G}_{\sigma,h}(jkl) = \delta_{jkl}
\end{align}
where the discrete delta function $\delta_{jkl}$ is 1 at point $x_{jkl}$ and 0 elsewhere. 

For constant-coefficient linear operators, the second order central finite difference scheme would give the discrete Green's functions in terms of a triple integral when one performs Fourier transform on the infinite lattice and the inverse Fourier transform will give:
\begin{align}\label{eqn:lgf}
\widehat{G}_{\sigma,h}(j,k,l) = \frac{h^2}{(2\pi)^3}\int_{-\pi}^{\pi}\int_{-\pi}^{\pi}\int_{-\pi}^{\pi} \frac{\cos(j\theta_1)\cos(k\theta_2)\cos(l\theta_3)}{6+\sigma h^2  - 2(\cos\theta_1+\cos\theta_2+\cos\theta_3)}d\theta_1d\theta_2d\theta_3
\end{align}

We will focus on the case of $h=1$, as the division by $h^2$ will cancel out in the left and right hand sides of \eqref{eqn:unit_infinite_aux_prob}. Thus we will drop the hat and $h$ in $\widehat{G}_{\sigma,h}(j,k,l)$, and use $G_{\sigma}(j,k,l)$ to denote the lattice Green's function, without risk of confusion. 

For $\sigma=0$, $G_{0}(j,k,l)$ reduces to the classical lattice Green's function and the well-known Watson's integral \cite{joyce2005evaluation,zucker201170}. The evaluation of different types of Watson's integral has been well-studied. For example, $G_{0}(0,0,0)$ is analytically computed as
\begin{align}
G_{0}(0,0,0) = \frac{\sqrt{6}}{96\pi^3}\Gamma\left(\frac{1}{24}\right)\Gamma\left(\frac{5}{24}\right)\Gamma\left(\frac{7}{24}\right)\Gamma\left(\frac{11}{24}\right)
\end{align}
or 
\begin{align}
G_{0}(0,0,0) = \frac{\sqrt{3}-1}{96\pi^3}\left[\Gamma\left(\frac{1}{24}\right)\Gamma\left(\frac{11}{24}\right)\right]^2
\end{align}
as in \cite{zucker201170}. Recursive formulas thus can be derived based on the definition \eqref{eqn:lgf_delta} and the exact value of $G_{0}(0,0,0)$.

Higher-order or non-standard finite difference stencils would give similar expressions of the lattice Green's function in \eqref{eqn:lgf}.  Evaluations of high order versions of the LGF \eqref{eqn:lgf} resulted from various forms of finite difference schemes have been studied in a recent work \cite{gabbard2024lattice}. The algorithm in this work can be extended naturally to high order accuracy once high order lattice Green's functions are obtained.

We will mainly focus on two cases: $\sigma=0$ and $\sigma>0$. 
\begin{itemize}
	\item 
For $\sigma=0$, the integrand in \eqref{eqn:lgf} is singular at the origin, but can be expressed in terms of an integral of Bessel's functions (Watson's transformation)
\begin{align}\label{eqn:lgf_bessel}
G_{0}(j,k,l)=\int_0^\infty\exp(-6t)I_j(2t)I_k(2t)I_l(2t)dt
\end{align}
where 
\begin{align}
I_n(t)  = \frac{1}{2\pi}\int_{-\pi}^{\pi} \exp(t\cos\theta)\cos(n\theta)d\theta
\end{align}
The improper integral \eqref{eqn:lgf_bessel} can be split into two integrals with a large enough $T^*$
\begin{align}
G_{0}(l,m,n) = \int_0^{T^*}\exp(-6t)I_l(2t)I_m(2t)I_n(2t)dt+\int_{T^*}^\infty\exp(-6t)I_l(2t)I_m(2t)I_n(2t)dt
\end{align}
where the first finite integral can be evaluated using quadrature rules and the second integral using asymptotic expansions of the Bessel functions. The asymptotic expansions for large argument $|t|$ and $\mu=4n^2$ (\cite[p375-377]{abramowitz1948handbook}):
\begin{align}
I_n(t) \sim \frac{\exp(t)}{\sqrt{2\pi t}}\left(1-\frac{\mu-1}{8t}+\frac{(\mu-1)(\mu-9)}{2!(8t)^2}-\frac{(\mu-1)(\mu-9)(\mu-25)}{3!(8t)^3}+\cdots\right)
\end{align}

The improper integral \eqref{eqn:lgf_bessel} can also be evaluated using adaptive Gauss Kronrod method. For large magnitude of $(l,m,n)$, asymptotic formula have been established in \cite{martinsson2002asymptotic}. The $q$-term asymptotic expansion of $G_{0}(\bm{n})$ with $\bm{n}=(j,k,l)$ is given by
\begin{align}
G_{0}(\bm{n}) = A_G^q(\bm{n})+\mathcal{O}(|n|^{-2q-1})
\end{align}
For large enough $\bm{n}$, we can take $q=2$:
\begin{align}
A^2_G(\bm{n})=-\frac{1}{4\pi|\bm{n}|}-\frac{j^4+k^4+l^4-3j^2k^2-3j^2l^2-3k^2l^2}{16\pi|\bm{n}|^7}
\end{align}
or $q=3$:
\begin{align}
A_G^3(\bm{n}) = A^2_G(\bm{n})&+\frac{1}{128\pi|\bm{n}|^{13}}\Big(-288(k^2l^2j^4+k^2l^4j^2+k^4l^2j^2)+621(j^4k^4+l^4k^4+j^4l^4)\nonumber\\
&-244(k^2j^6+l^2j^6+k^6j^2+l^6j^2+k^2l^6+k^6l^2)+23(j^8+k^8+l^8)\Big)
\end{align}
As $|\bm{n}|\rightarrow\infty$, $G_{0}(\bm{n})$ behaves like the fundamental solution of the continuous Poisson's equation $-1/4\pi |x|$.

\item For $\sigma>0$, the LGF \eqref{eqn:lgf} is well defined and the spectral accuracy of trapezoidal rule for periodic functions can be utilized:
\begin{subequations}
\begin{align}
G_{\sigma}(j,k,l)=\frac{h^2}{(2\pi)^3}\int_{-\pi}^{\pi}\int_{-\pi}^{\pi}\int_{-\pi}^{\pi} \frac{e^{ij\theta_1}e^{ik\theta_2}e^{il\theta_3}}{6+\sigma h^2  - 2(\cos\theta_1+\cos\theta_2+\cos\theta_3)}d\theta_1d\theta_2d\theta_3\\
\approx\frac{1}{N^3}\sum_{p=-N/2}^{\frac{N}{2}-1}\sum_{q=-\frac{N}{2}}^{\frac{N}{2}-1}\sum_{r=-\frac{N}{2}}^{\frac{N}{2}-1} \frac{e^{ij2\pi p/N}e^{ik2\pi q/N}e^{il2\pi r/N}}{6+\sigma h^2  - 2(\cos(2\pi p/N)+\cos(2\pi q/N)+\cos(2\pi r/N))}
\end{align}
\end{subequations}
Thus 3D FFTs can be employed to speed up evaluations of source-target interactions at multiple instances. For large enough $\sigma$, the values of $G_\sigma$ decays rapidly as $|\bm{n}|$ tends to infinity.

\end{itemize}

With lattice Green's function defined, we are able to solve \eqref{eqn:unit_infinite_aux_prob} with ease.
Notice that the right hand side in the difference equation is highly localized with at most six nonzero entries. Then the solution $\hat{v}_{jkl}$ can be explicitly written as
\begin{align}
\hat{v}_{jkl} =& (-6+\sigma h^2)\chi_{M^-}(p,q,r)G_0(p,q,r)\nonumber\\
&+\chi_{M^-}(p+1,q,r)G_0(p+1,q,r) +\chi_{M^-}(p-1,q,r)G_0(p-1,q,r) \nonumber\\
&+\chi_{M^-}(p,q+1,r)G_0(p,q+1,r) +\chi_{M^-}(p,q-1,r)G_0(p,q-1,r) \nonumber\\
&+\chi_{M^-}(p,q,r+1)G_0(p,q,r+1) +\chi_{M^-}(p,q,r-1)G_0(p,q,r-1) 
\end{align}
for any $x_{jkl}\in \gamma$ with unit density defined at $x_{pqr}\in\gamma$, thus for columns in $\widehat{P}_\gamma$. This significantly reduces the computational cost of the construction of $\widehat{P}_\gamma$ from $\mathcal{O}(N^5\log(N))$ to $\mathcal{O}(N^2)$ when unit density is used.

Likewise, we have the following theorem for homogeneous equations:
\begin{theorem}
The difference equation $\widehat{L}_h\hat{v}_{jkl}=0$ for $x_{jkl}\in M^+$ is equivalent to the boundary equation with projections
\begin{align}\label{eqn:inf_veq}
\hat{v}_{\gamma_+}-\widehat{P}_{\gamma_+}\hat{v}_{\gamma} = 0,
\end{align}
\end{theorem}
which we will not prove as this is a classical result in \cite{ryaben2012method}.

Replacing the boundary equations with projections in \eqref{eqn:hom} with \eqref{eqn:inf_veq} and $v_\gamma$ with $\hat{v}_\gamma$, we arrive at the following system of equations:
\begin{subequations}\label{eqn:inf_hom}
\begin{align}
\hat{v}_{\gamma_+} - \widehat{P}_{\gamma_+} \hat{v}_\gamma &= 0\\
\sum_{x_{jkl}\in\gamma} \hat{v}_{jkl} \Phi_{jkl}(x^*,y^*,z^*) &= g(x^*,y^*,z^*)-\sum_{x_{jkl}\in\gamma}[G_hf_{\gamma}]_{jkl} \Phi_{jkl}(x^*,y^*,z^*)
\end{align}
\end{subequations}
The difference between \eqref{eqn:hom} and \eqref{eqn:inf_hom} lies in the difference potentials operator $\widehat{P}_{\gamma_+}$ and $P_{\gamma_+}$, where the former with ``hat'' can be obtained with much less computational cost.

Now we establish the equivalence between \eqref{eqn:hom} and \eqref{eqn:inf_hom}.

\begin{theorem}
The densities $v_\gamma$ and $\hat{v}_\gamma$ from solving \eqref{eqn:hom} and \eqref{eqn:inf_hom} are identical, if the meshes align.
\end{theorem}

\begin{proof}
$v_{\gamma_+} - P_{\gamma_+} \hat{v}_\gamma = 0$ is equivalent to $L_hv_{jkl}=0$ for $x_{jkl}\in M^+$, thus \eqref{eqn:hom} is equivalent to
\begin{subequations}
\begin{align}
L_hv_{jkl}&=0,\quad x\in M^+\\
\sum_{x_{jkl}\in\gamma} v_{jkl} \Phi_{jkl}(x^*,y^*,z^*) &= g(x^*,y^*,z^*)-\sum_{x_{jkl}\in\gamma}[G_hf_{\gamma}]_{jkl} \Phi_{jkl}(x^*,y^*,z^*)
\end{align}
\end{subequations}

and $\hat{v}_{\gamma_+} - \widehat{P}_{\gamma_+} \hat{v}_\gamma = 0$ is equivalent to  $\widehat{L}_h\hat{v}_{jkl}=0$ for $x_{jkl}\in M^+$, thus  \eqref{eqn:inf_hom} is equivalent to 
\begin{subequations}
\begin{align}
\widehat{L}_h\hat{v}_{jkl} &= 0,\quad x\in M^+\\
\sum_{x_{jkl}\in\gamma} \hat{v}_{jkl} \Phi_{jkl}(x^*,y^*,z^*) &= g(x^*,y^*,z^*)-\sum_{x_{jkl}\in\gamma}[G_hf_{\gamma}]_{jkl} \Phi_{jkl}(x^*,y^*,z^*)
\end{align}
\end{subequations}
Notice that $L_h=\widehat{L}_h$ in $M^+$, the difference of $v$ and $\hat{v}$ satisfies 
\begin{align}
L_h (v-\hat{v})&=0,\\
\sum_{x_{jkl}\in\gamma} [v_{jkl}-\hat{v}_{jkl}] \Phi_{jkl}(x^*,y^*,z^*) &=0.
\end{align}
With well-posedness assumed, we conclude that $v_{jkl}=\hat{v}_{jkl}$ in $N^+$, thus also in $\gamma$.
\end{proof}

Once $\hat{v}_\gamma$ is obtained by solving \eqref{eqn:inf_hom}, the approximate solution in $M^+$ is given by the discrete generalized Green's formula:
\begin{align}\label{eqn:discrete_green}
u_{jkl} = G_h[\chi_{M^-}L_h\hat{v}_\gamma]+G_h[\chi_{M^+}f_h],\quad x_{jkl}\in M^+
\end{align}
where we only replaced $v_\gamma$ with $\hat{v}_\gamma$ in the discrete generalized Green's formula \eqref{eqn:discrete_green_v}.

To summarize, the operator $\widehat{P}_\gamma$ (essentially $\widehat{P}_{\gamma_+}$) is constructed using explicit knowledge of the lattice Green's function, thus the excessive computational cost mentioned in \cite{xia2023local} for computing the difference potentials operator in 3D is avoided. Only two instances that involve the operator $G_h$ in the finite auxiliary domain will be based on 3D FFTs in \eqref{eqn:discrete_green}.