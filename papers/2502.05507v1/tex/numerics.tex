\section{Numerical results}\label{sec:numerical_results}
We will use Algorithm~\ref{alg:bae} to perform the numerical tests.

The finite auxiliary domain is slightly larger than the torus and is chosen to be $[-1-\ell,1+\ell]^3$ and the grid size is taken as
\begin{align}
h = \frac{2+2\ell}{N},\quad N = 2^n-1
\end{align}
which gives a total of $N^3$ degrees of freedom in the auxiliary domain and the number of unknowns in the boundary equations is in the scale of $\mathcal{O}(N^2)$.

The Lattice Green's functions are precomputed on mesh $\mathbb{Z}^3$ and stored in a lookup matrix of size $N\times N\times N$ for only positive index values, and can be reused in different tests. Due to the symmetry of the Lattice Green's function, only 1/48 of the values are stored in 3D.

\subsection{Poisson's equation}
The geometry is an ellipsoid given implicitly by
\begin{align}
\phi(x,y,z) = \frac{x^2}{a^2}+\frac{y^2}{b^2}+\frac{z^2}{c^2}-1=0
\end{align}
with $a=1,b=0.8,c=0.4$. We choose $\ell=0.25$ for the auxiliary domain in this test. The spherical harmonics spectral approach developed in \cite{epshteyn2019efficient} will not work well for this ellipsoidal geometry.

The errors are computed against the exact solution
\begin{align}
u(x,y,z) = x^2+y^2+z^2
\end{align}
and boundary data $g(x,y,z)$ and the source term $f(x,y,z)$ are computed using the exact $u(x,y,z)$.

Due to memory limit, we do not store matrices $D_{\pm}$ or $S_\pm$, and we refer to \cite{martinsson2009boundary} for behaviors of singular values of matrices $S_-$ and $D_-$. In Figure~\ref{fig:poisson_res}, we present the relative residual of solving 
\begin{align}
(\Phi_+D_+ + \Phi_-D_-) q_d = b
\end{align}
using vanilla GMRES with no preconditioner and a zero initial guess. The tolerance is set to be $10^{-14}$ for test purpose, whereas a larger tolerance such as $10^{-8}$ would suffice for accuracy. It can be observed that for $N=31,63,127,255$, where the length of $q_d$ approximately quadruples over mesh refinement, the number of iterations grow mildly, even though not as good as the double layer formulation in the boundary integral method. This indicates that preconditioning techniques should be explored.


\begin{figure}[H]
    \centering
    % trim=left bottom right top
    \includegraphics[width=0.5\textwidth, trim = 0cm 5cm 0cm 5.5cm]{fig/Poisson3D_residuals}
    \caption{Relative residual of GMRES using double layer formulation for Poisson equations}
    \label{fig:poisson_res}
\end{figure}

In Figure~\ref{fig:poisson_results}, we plot the numerical solution and the error patterns on the finest mesh $255\times255\times255$ in our simulation. The error patterns hint that it might benefit from post-processing techniques studied such as in \cite{mirzaee2011smoothness}.

\begin{figure}[H]
    \centering
    \begin{subfigure}{0.45\textwidth}
        \centering
        % trim=left bottom right top
        \includegraphics[width=\textwidth]{fig/Poisson_3D_Solution_N256}
        % \vspace*{0.5mm}
        \caption{Numerical Solution $(N=255)$}
        \label{fig:poisson_solution}
    \end{subfigure}
    \quad
    \begin{subfigure}{0.45\textwidth}
        \centering
        \includegraphics[width=\textwidth]{fig/Poisson_3D_Error_N256}
        % \vspace*{0.5mm}
        \caption{Pointwise errors $(N=255)$}
        \label{fig:poisson_error}
    \end{subfigure}
    \caption{Numerical solution and pointwise errors for Poisson equation}\label{fig:poisson_results}
\end{figure}

In Table~\ref{table:poisson_convergence}, the max-norm error and convergence rates are presented, which shows the designed second order convergence.

\begin{table}[htbp]
    \centering
    \begin{tabular}{@{} r S[table-format=1.4e-1] S[table-format=1.2] @{}}
        \toprule
        {$N$} & {Max Error} & {Rate} \\
        \midrule
        31   & 1.3222e-03 & {--} \\
        63   & 3.5101e-04 & 1.91 \\
        127  & 9.1688e-05 & 1.94 \\
        255  & 2.2926e-05 & 2.00 \\
        \bottomrule
    \end{tabular}
    \caption{Max error and convergence rates for Poisson's equation in an ellipsoid}
    \label{table:poisson_convergence}
\end{table}

\subsection{Modified Helmholtz equation}

For modified Helmholtz equation, we test with $\sigma=10$. For the geometry, we use a multi-connected torus defined by the implicit function
\begin{align}
\phi(x,y,z) = (\sqrt{x^2+y^2}-R)^2+z^2-r^2 = 0
\end{align}
where $R=0.6$ is the distance between the center of the tube and the center of the torus and $r=0.3$ is the radius of the tube. The interior of the torus is categorized by $\phi(x,y,z)<0$. The auxiliary domain takes $\ell=0.1$.

\begin{remark}
A torus is simple enough, yet presenting sufficient numerical challenges for unfitted boundary methods. We chose this geometry to demonstrate the effectiveness of combing lattice Green's function in free space with local basis functions, where global basis functions will be difficult to construct for spectral approaches. We should also mention that using Non-Uniform Rational B-Splines (NURBS) on patches for 3D CAD geometry is also possible and has been studied in the difference potentials framework in \cite{PETROPAVLOVSKY2024112705}.
\end{remark}

The exact solution is
\begin{align}
u(x,y,z) = \sin(x)\cos(y)\sin(z),
\end{align}
and $f$ and $g$ are computed using the exact solution.

It can be seen in Figure~\ref{fig:poisson_res} and Figure~\ref{fig:mod_res} that the iteration numbers are similar regardless of the equation types $\sigma=0$ or $\sigma=10$ or different types of geometry or lattice Green's functions, corroborating the robustness of the boundary algebraic linear system.

The GMRES convergence behavior shown in Figure~\ref{fig:mod_res} demonstrates the effectiveness of the double layer formulation for the modified Helmholtz equation. The similar convergence patterns observed for different mesh sizes $(N = 31, 63, 127, 255)$ suggest that the conditioning of the system remains relatively stable under mesh refinement. This behavior is particularly noteworthy given the complex geometry of the torus and the challenging nature of the modified Helmholtz operator.

\begin{figure}[H]
    \centering
    % trim=left bottom right top
    \includegraphics[width=0.5\textwidth, trim = 0cm 5cm 0cm 5.5cm]{fig/Modified3D_residuals}
    \caption{Relative residual of GMRES using double layer formulation for modified Helmholtz equations}
    \label{fig:mod_res}
\end{figure}

In Figure~\ref{fig:mod_results}, the numerical solution and the error patterns on the finest mesh $255\times255\times255$ are presented. The large errors occur around where $u$ is large in magnitude.

\begin{figure}[H]
    \centering
    \begin{subfigure}{0.45\textwidth}
        \centering
        % trim=left bottom right top
        \includegraphics[width=\textwidth]{fig/Modified_3D_Solution_N256}
        % \vspace*{0.5mm}
        \caption{Numerical Solution $(N=255)$}
        \label{fig:mod_solution}
    \end{subfigure}
    \quad
    \begin{subfigure}{0.45\textwidth}
        \centering
        \includegraphics[width=\textwidth]{fig/Modified_3D_Error_N256}
        % \vspace*{0.5mm}
        \caption{Pointwise errors $(N=255)$}
        \label{fig:mod_error}
    \end{subfigure}
    \caption{Numerical solution and point errors for modified Helmholtz equation}\label{fig:mod_results}
\end{figure}

In Table~\ref{table:mod_convergence}, the max-norm error and convergence rates also show that the designed second order convergence is achieved.

\begin{table}[htbp]
    \centering
    \begin{tabular}{@{} r S[table-format=1.4e-1] S[table-format=1.2] @{}}
        \toprule
        {$N$} & {Max Error} & {Rate} \\
        \midrule
        31   & 7.5002e-05 & {--} \\
        63   & 2.0682e-05 & 1.86 \\
        127  & 5.6077e-06 & 1.88 \\
        255  & 1.3978e-06 & 2.00 \\
        \bottomrule
    \end{tabular}
    \caption{Max Error and Convergence Rates for modified Helmholtz equation in a torus}
    \label{table:mod_convergence}
\end{table}