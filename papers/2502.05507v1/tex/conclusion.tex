\section{Conclusion}\label{sec:conclusion}

We have developed an unfitted boundary algebraic equation method based on difference potentials for elliptic PDEs using lattice Green's functions that significantly advances the state of the art in handling complex 3D geometries. By replacing finite auxiliary domains with free-space LGFs, our approach dramatically reduces the computational complexity of constructing difference potentials operators, making three-dimensional simulations more tractable. The analytical derivation of LGF-based potentials, combined with rigorous proofs of equivalence between direct and indirect BAE formulations, provides a solid theoretical foundation for the method. Our spectral analysis demonstrates that the resulting systems are well-conditioned for iterative solvers, particularly when using double layer formulations. Numerical experiments confirm the method's efficiency and accuracy for both Poisson and modified Helmholtz equations in 3D implicitly defined geometry.

The developed approach successfully bridges the gap between structured grid efficiency and geometric flexibility, offering a robust foundation for multidisciplinary applications in computational physics and engineering. By combining the computational advantages of lattice Green's functions with the geometric flexibility of difference potentials, this method provides a promising path forward for solving complex PDEs in irregular domains with optimal efficiency. It should also be noted that the current approach only applies to constant-coefficient PDEs, while the difference potentials framework is suitable for variable-coefficient or nonlinear PDEs.

Future work will extend this framework to handle unbounded domains through fast multipole acceleration or FFT based libraries \cite{caprace2021flups} and explore applications to more challenging problems such as high-frequency Helmholtz equations using lattice Green's function for Helmholtz equation \cite{beylkin2009fast,beylkin2008fast,linton2010lattice} and Stokes flows. Additional developments will focus on implementing high-order stencils as studied in \cite{gabbard2024lattice}, developing fast summation techniques for matrix-free implementations similarly as in \cite{gillman2014fast}, and investigating various types of boundary conditions. 