\section{Boundary Algebraic Equations}\label{sec:bae}
In this section we focus on the reduced boundary equations with projections or the boundary algebraic equations \eqref{eqn:inf_veq}. We shall decompose the operator $\widehat{P}_{\gamma_+} $ in \eqref{eqn:beq} into two parts $P_+$ and $P_{-}$ that correspond to columns in $\gamma_+$ and $\gamma_-$:
\begin{align}
v_{\gamma_+} - \left[\begin{array}{cc}P_+ & P_-\end{array}\right]\left(\begin{array}{cc}v_{\gamma_+} & v_{\gamma_-}\end{array}\right)^T = 0
\end{align}
This gives a relation between $v_{\gamma_+}$ and $v_{\gamma_-}$:
\begin{align}
(I-P_+)v_{\gamma_+} = P_- v_{\gamma_-}\label{eqn:beq_bae}
\end{align}
where $P_+$ is a square matrix of size $|\gamma_+|\times|\gamma_+|$, $P_-$ is of size $|\gamma_+|\times|\gamma_-|$. Note that $P_+$ and $P_-$ are both dense. Even with the lattice Green's function, the computation of $P_+$ and $P_-$ and their storage can be expensive in 3D. Now we aim to establish some boundary algebraic equations like \eqref{eqn:beq_bae} directly from the lattice Green's function. The BAE \eqref{eqn:beq_bae} can also be regarded as the discrete version of Dirichlet to Neumann (DtN) map.

\subsection{Direct formulation}
Similarly to the boundary integral method, we will introduce the concept of direct and indirect formulations, where direct formulation is based on Green's third identity while indirect formulation is not. In this classification, the boundary algebraic equation \eqref{eqn:beq_bae} is a direct formulation. To see this, we introduce a lattice version of Green's second identity from \cite[Sec. 5, Lemma 1]{duffin1953discrete}:
\begin{lemma}
Let $u$ and $v$ be lattice/grid functions and let $E$ be a finite set of lattice points, then
\begin{align}
\sum_E (vLu - uLv) = \sum_S[v(p)u(q)-u(p)v(q)]
\end{align}
where $\sum_E$ denotes the summation over $E$, and $\sum_S$ denotes the summation over the set $S$ of pairs of points $(p,q)$ that are neighbors, $p$ being in $E$ and $q$ being in the complement of $E$.
\end{lemma}

Now we choose a lattice harmonic function $u$ such that $Lu = 0$ and $v(m)=G(m-p)$ the lattice Green's function of the difference operator $L$ at peaked at $p$. Let $p$ be any point in $\gamma_+$ and $q$ be any point in $\gamma_-$ and focus on set $p$, then the lattice version of Green's third identity would be 
\begin{align}\label{eqn:direct_bae}
% -u(p^*) = \sum_S[G(p-p^*)u(q)-u(p)G(q-p^*)]\\
v_{\gamma_+} - G_{\gamma_-}v_{\gamma_+} = -G_{\gamma_+}v_{\gamma_-}
\end{align}
which is exactly the form of BAE \eqref{eqn:beq_bae}, if one imposes $P_+:=G_{\gamma_-}$ and $P_-:=-G_{\gamma_+}$ that denote the pair interactions of $\gamma_+$ and $\gamma_-$. The matrix $G_{\gamma_-}$ needs proper reduction from $\gamma_-$ to $\gamma_+$. For example, if $q_1$ and $q_2$ both are neighbors of $p$, then the coefficient of $u(p)$ will be the addition of $G(q_1-p)$ and $G(q_2-p)$. Similarly, the matrix $G_{\gamma_+}$ will need to be extended from $\gamma_+$ to $\gamma_-$, as at both $q_1$ and $q_2$, $u(q)$ has the same coefficient.

\subsection{Indirect formulation}
As in the indirect boundary integral approach, we investigate indirect boundary algebraic method. We follow the work of boundary algebraic equations \cite{martinsson2009boundary}, but allow misalignment of grid points with the continuous boundary $\partial\Omega$.

We will start with the kernel for single layer potential:
\begin{align}\label{eqn:single}
S(m,n) = G_\sigma(|m-n|),\quad m,n\in \mathbb{Z}^3
\end{align} 
where $G_\sigma$ is the lattice Green's function in the free space.

Note that unlike the continuous case (the boundary integral method), no singularity appears when $m=n$ in \eqref{eqn:single}. The single layer potential for any density $q$ defined on $\gamma_-$ (source) is
\begin{align}\label{eqn:single_potential}
v_m=\sum_{x_n\in\gamma_-}S(m,n)q(n),\quad x_m\in N^+
\end{align}
where the target is any point in $N^+$.

\begin{proposition}\label{prop:single_potential}
The single layer potential \eqref{eqn:single_potential} satisfies the homogeneous difference equations $L_hv_m=0$ for $x_m\in M^+$.
\end{proposition}
\begin{proof}
For $x_m\in M^+$,
\begin{subequations}
\begin{align}
L_h v_m =& L_h\sum_{x_n\in\gamma_-}S(m,n)q(n)\\
=&\sum_{x_n\in\gamma_-}\Big[(-6+\sigma h^2)S(m,n)+S(m+e_1,n)+S(m-e_1,n)\nonumber\\
&+S(m+e_2,n)+S(m-e_2,n)+S(m+e_3,n)+S(m-e_3,n)\Big]q(n)\\
=&\sum_{x_n\in\gamma_n}\delta(m,n)q(n)\\
=&0
\end{align}
as $m$ and $n$ belong to disjoint sets. Here $e_i$, $(i=1,2,3)$ are unit vectors in the $x,y,z$ directions respectively.
\end{subequations}
\end{proof}

Now assume for the source defined on $\gamma_-$ with single layer density $q_{s}$, we look at target sets $\gamma_+$ and $\gamma_-$ with matrix-vector notation respectively:
\begin{align}
v_{\gamma_+} =S_{+}q_{s},\quad v_{\gamma_-} =S_{-}q_{s}
\end{align}
where $S_+$ is of size $|\gamma_+|\times|\gamma_-|$ and $S_-$ is of size $|\gamma_-|\times|\gamma_-|$.
Then $v_{\gamma_-}$ and $v_{\gamma_+}$ are related through the unknown density $q_{s}$, i.e.,
\begin{align}
v_{\gamma_+} = S_{+}S^{-1}_{-}v_{\gamma_-}.
\end{align}

\begin{remark}
The set  $\gamma_-$ can be regarded as the grid boundary of the set $M^+$ and spectral properties of $S_-$ in \cite{martinsson2009boundary} can be applied. In addition, $S_-$ is invertible, symmetric and diagonally dominant, which is useful for theoretical studies.
\end{remark}

Now for the double layer kernel, we adopt the same definition in \cite{martinsson2009boundary}:
\begin{align}
    D(m,n):=\sum_{k\in \mathbb{D}_n} G_\sigma(|m-n|) - G_\sigma(|m-k|)
\end{align}
where $G_\sigma$ is the free space lattice Green's function and $\mathbb{D}_n$ is the set of nodes outside of $N^+$ but connected to source node $n$. The set $\gamma_-$ is regarded as the grid boundary for $M^+$. 

The double layer potentials for a grid density $q_d$ defined on $\gamma_-$ is 
\begin{align}\label{eqn:double_potential}
v_m=\sum_{x_n\in\gamma_-}D(m,n)q_d(n),\quad x_m\in N^+,
\end{align}
where $q_d$ denotes the double layer density.

\begin{proposition}
The double layer potential \eqref{eqn:double_potential} satisfies the homogeneous difference equations $L_hv_m=0$ for $x_m\in M^+$.
\end{proposition}
We will omit the proof here as it is similar to the proof of Proposition \ref{prop:single_potential}.

With restriction to $\gamma_+$ and $\gamma_-$ respectively, we have
\begin{align}
v_{\gamma_+}=D_+q_{d},\quad v_{\gamma_-}=D_-q_{d}
\end{align}
If the square matrix $D_-$ is invertible, we also have
\begin{align}
v_{\gamma_+} = D_+D_{-}^{-1}v_{\gamma_-}.
\end{align}
from similar arguments in the single layer case.

\begin{proposition}
The following relations hold:
\begin{align}\label{eqn:equal}
(I-P_+)^{-1}P_- = S_+S_{-}^{-1}=D_+D_{-}^{-1}
\end{align}
\end{proposition}

The equivalence~\eqref{eqn:equal} essentially connects point set $\gamma_{+}$ with point set $\gamma_{-}$, which also agrees with the difference equation:
\begin{align}
L_hu_h = 0, \quad (x_i,y_j,z_k)\in M^+.
\end{align}
Once the lattice boundary values at $\gamma_{-}$ are set, the above difference equation admits a unique solution, hence values at $\gamma_{+}$ are uniquely determined.

\begin{remark}
The involved matrices $P_\pm$, $S_\pm$ and $D_\pm$ are readily obtainable from the lattice Green's function without explicitly solving the auxiliary problem, which is desirable when designing matrix-free algorithms.
\end{remark}

\begin{remark}
The infinite lattice only comes into play conceptually when we compute the values of lattice Green's functions.
\end{remark}

\subsection{Spectral property}

In complete analogous to the 2D results in \cite{martinsson2009boundary}, we state the spectral property of the matrices $S_-$ and $D_-$.

\begin{theorem}
Let $\Omega_h$ be a connected domain in $\mathbb{Z}$ whose boundary $\gamma_-$ contains $N$ boundary nodes. Let $S_-$ be the $N\times N$ ($N=|\gamma_-|$) matrix corresponding to the linear system equivalent to the single-layer potential in 3D. Then, any singular value $\lambda$ of $S_-$ satisfies:
\begin{align}
\frac{1}{12} \leq  \lambda \leq CN
\end{align}
where $C$ is a dimension-independent constant.
\end{theorem}

\begin{proof}
For the upper bound:

\begin{align}
\lambda  \leq ||S_-||_2 \leq \sqrt{||S_-||_1||S_-||_{\infty}}
\end{align}
Since $|S(m,n)| \leq C$, we have $||S_-||_1 \leq CN$ and $||S_-||_\infty \leq CN$
Therefore:
\begin{align}
\lambda\leq ||S_-||_2 \leq \sqrt{CN\cdot CN} = CN
\end{align}

For the lower bound, we use an energy argument similar to the 2D case \cite{martinsson2009boundary} but extended to three dimensions.
Define the forward and backward difference operators:
\begin{subequations}
\begin{align}
\partial_i u = u(m + e_i) - u(m)\\
\bar{\partial}_i u = u(m) - u(m - e_i)
\end{align}
\end{subequations}
where $e_i$ are unit vectors in the three coordinate directions.
For lattice potentials $u, v$ that decay sufficiently fast at infinity:
\begin{align}
\sum [\bar{\partial}_iu](m)v(m) = -\sum u(m)[\partial_iv](m)
\end{align}
For any boundary density $\sigma$, we need to prove:
\begin{align}
\left|\sum_{m\in\gamma_-} [S_-\sigma](m) \sigma(m)\right|\geq \frac{1}{12}\sum_{m\in\gamma_-} |\sigma(m)|^2
\end{align}

Let $u$ be defined by
\begin{align}
u(m) = \sum_{n\in\gamma_-} S(m,n)\sigma(n)
\end{align}
Then
\begin{subequations}
\begin{align}
\sum_{m\in\gamma_-} [S_-\sigma](m) \sigma(m) = \sum_{m\in\gamma_-} u(m)\sigma(m) = \sum_{m\in\gamma_-} u(m)Au = \sum_{m\in\mathbb{Z}^3} u(m)Au\\
=\sum_{m\in\mathbb{Z}^3} |\partial_1u|^2+|\partial_2u|^2+|\partial_3u|^2\\
=\frac{1}{2}\sum_{m\in\mathbb{Z}^3} |\partial_1u|^2+|\partial_2u|^2+|\partial_3u|^2+|\bar{\partial}_1u|^2+|\bar{\partial}_2u|^2+|\bar{\partial}_3u|^2\\
\geq\frac{1}{2}\sum_{m\in\gamma_-} |\partial_1u|^2+|\partial_2u|^2+|\partial_3u|^2+|\bar{\partial}_1u|^2+|\bar{\partial}_2u|^2+|\bar{\partial}_3u|^2
\end{align}
\end{subequations}
The flux equilibrium equation at boundary points becomes:
\begin{align}
-\partial_1 u(m)+\bar{\partial_1}u(m)-\partial_2 u(m)+\bar{\partial_2}u(m)-\partial_3 u(m)+\bar{\partial_3}u(m)=\sigma(m)
\end{align}
which implies
\begin{align}
|\partial_1u|^2+|\partial_2u|^2+|\partial_3u|^2+|\bar{\partial}_1u|^2+|\bar{\partial}_2u|^2+|\bar{\partial}_3u|^2\geq |\sigma(m)|^2
\end{align}
Combining these inequalities gives
\begin{align}
\left|\sum_{m\in\gamma_-} [S_-\sigma](m) \sigma(m)\right|\geq \frac{1}{12}\sum_{m\in\gamma_-} |\sigma(m)|^2
\end{align}

\end{proof}

For double layer potential, similar results can be established as well.
\begin{theorem}
Let $\gamma_-$ be the boundary of a connected lattice domain $\Omega_h$ in $\mathbb{Z}^3$ with $N$ boundary nodes, and $c$ be a real number such that $0<c\leq 1$. Suppose there exists an ordering $\{m_i,i=1,2,\cdots,N\}$ of $\gamma_-$ such that for all $i\neq j$,
\begin{align}
|m_i-m_j|\geq c\cdot d(i,j)
\end{align}
where $d(i,j)=\min(|i-j|,(N-|i-j|))$, the the matrix $D_-$ of the double layer potential associated with $\gamma_-$ satisfies
\begin{align}
||D_-||\leq \frac{C}{c^2}
\end{align}
where $C$ is a dimension-independent constant.
\end{theorem}

\begin{proof}
Likewise, one can verify that the double layer kernel $D$ satisfies
\begin{align}
|D(m,n)|\leq \frac{C}{1+|m-n|^2}
\end{align}
for all $m,n\in\gamma_-$.

Combining with the geometric assumption
\begin{align}
|D_{-,ij}| = |D(m_i,m_j)|\leq \frac{C}{1+|m_i-m_j|^2}\leq \frac{C}{1+c^2d^2(i,j)}\leq \frac{1}{c^2}\frac{C}{1+d^2(i,j)}
\end{align}
Now define $D'$ with entry at $(i,j)$
\begin{align}
D'_{ij}=\frac{1}{1+d^2(i,j)}
\end{align}
Hence $D'$ is a symmetric circulant matrix, whose eigenvalues can be shown to be uniformly bounded by a constant $C$ independent of $N$ using Fourier transform.
\begin{align}
||D'||\leq C
\end{align}
thus
\begin{align}
||D_-||\leq \frac{1}{c^2}||D'||\leq \frac{C}{c^2} 
\end{align}
\end{proof}

\begin{remark}
It can be checked that the double layer potential matrix $D$ admits an eigenvalue $-1$.
\end{remark}

\subsection{Boundary equations}

Now we have obtained three types of boundary equations
\begin{subequations}
\begin{align}
&\left\{
\begin{array}{r}
(I - P_+)v_{\gamma_+} - P_{-}v_{\gamma_-}=0\\
\Phi_+v_{\gamma_+} + \Phi_{-}v_{\gamma_-}=b
\end{array}
\right.\label{eqn:direct_sys}\\
&\left\{
\begin{array}{r}
v_{\gamma_+} =S_{+}q_{s}\\
v_{\gamma_-} =S_{-}q_{s}\\
\Phi_+v_{\gamma_+} + \Phi_{-}v_{\gamma_-}=b
\end{array}
\right.\label{eqn:single_sys}\\
&\left\{
\begin{array}{r}
v_{\gamma_+} =D_{+}q_{d}\\
v_{\gamma_-} =D_{-}q_{d}\\
\Phi_+v_{\gamma_+} + \Phi_{-}v_{\gamma_-}=b
\end{array}\label{eqn:double_sys}
\right.
\end{align}
\end{subequations}
where $b(x^*,y^*,z^*)=g(x^*,y^*,z^*)-\sum_{x_{jkl}\in\gamma}[G_hf_{\gamma}]_{jkl} \Phi_{jkl}(x^*,y^*,z^*)$ for $(x^*,y^*,z^*)\in\Gamma$, and $\Phi_\pm$ are the sub-matrices in the coefficient matrices $\Phi$ that corresponds to point set $\gamma_\pm$ respectively.

\paragraph{Schur complement}

The three systems \eqref{eqn:direct_sys}--\eqref{eqn:double_sys} hint a block linear system
\begin{align}
\left(
\begin{array}{cc}
A & B \\
\Phi_+ & \Phi_- \\
\end{array}
\right)
\left(
\begin{array}{c}
v_{\gamma_+}\\
v_{\gamma_-}
\end{array}
\right)
=
\left(
\begin{array}{c}
0 \\
b \\
\end{array}
\right)
\end{align}
where $\Phi_+, \Phi_-$ are sparse and almost diagonal matrices.

Schur complement for the above system gives
\begin{subequations}
\begin{align}
v_{\gamma_+} &= -A^{-1}Bv_{\gamma_-}\\
v_{\gamma_-} &= (\Phi_--\Phi_+A^{-1}B)^{-1}b
\end{align}
\end{subequations}

The matrix multiplication $A^{-1}B$ can be any term in the equivalence \eqref{eqn:equal}:
\begin{align}
-A^{-1}B:=(I-P_+)^{-1}P_- = S_+S_{-}^{-1}=D_+D_{-}^{-1}
\end{align}

We have seen that the norm of $S_-$ will grow like $\mathcal{O}(N)$ and the norm of $D_-$ remain uniformly bounded. Previous numerical study in \cite{xia2023local} revealed that the condition number of $I-P_+$ will also grow linearly. Hence, the double layer formulation using $D_\pm$ is preferred.

When double layer formulation is used, the block system becomes:
\begin{align}
\left(
\begin{array}{cc}
D_+ & -D_- \\
\Phi_+ & \Phi_- \\
\end{array}
\right)
\left(
\begin{array}{c}
v_{\gamma_+}\\
v_{\gamma_-}
\end{array}
\right)
=
\left(
\begin{array}{c}
0 \\
b \\
\end{array}
\right)
\end{align}
and Schur complement gives
\begin{subequations}
\begin{align}
v_{\gamma_+} &= D_+D_-v_{\gamma_-}\\
v_{\gamma_-} &= (\Phi_-+\Phi_+D_+D_-^{-1})^{-1}b
\end{align}
\end{subequations}


The matrix $\Phi_--\Phi_+A^{-1}B$ or $\Phi_-+\Phi_+D_+D_-^{-1}$ resembles the Fredholm integral of second kind. This formulation has favorable conditioning properties. Notably, its condition number remains relatively stable under mesh refinement. Similar observations were made in \cite{feng2020fft,ren2022fft,ying2007kernel,gillis2018fast,gillis20192d,li1998fast,tan2009fast} where either Schur complement or Sherman-Morrison-Woodbury formula is used. Hence, $D_{-}$ and $\Phi_-+\Phi_+D_+D_-^{-1}$ are both well-conditioned, with condition numbers independent of mesh refinement, which are both suitable for iterative solvers such as GMRES.

\paragraph{Linear solve}
However, $I-P_{\gamma_+}$ is relatively small in size yet dense, so are $S_{-}$ and $D_{-}$, and it would be unwise to invert those matrices and construct $\Phi_-+\Phi_+D_+D_-^{-1}$. Instead, in the indirect formulation, we observe that
\begin{align}
v_{\gamma_+}=D_+q_d,\quad v_{\gamma_-}=D_-q_d\quad\mbox{or}\quad v_{\gamma_+}=S_+q_s,\quad v_{\gamma_-}=S_-q_s
\end{align} 
where $q_d$ and $q_s$ are the unknown double layer density and single layer density to be determined.

Then the single layer formulation together with the boundary condition gives
\begin{align}
(\Phi_+S_++\Phi_-S_-)q_s = b
\end{align}
and double layer gives
\begin{align}\label{eqn:double_bae}
(\Phi_+D_++\Phi_-D_-)q_d = b
\end{align}
where we can solve for the density $q_s$ or $q_d$.

Both systems can be solved using iterative methods. This approach avoids explicit storage or inversion of matrices $D_-$ and $S_-$. After obtaining the density $q_s$ or $q_d$, we can compute the density $v_\gamma$ through either the single or double layer formulation. Once $v_\gamma$ is obtained, the rest follows what we discussed in the difference potentials method in Section~\ref{sec:dpm}. Even though both single layer formulation and double layer formulation can be adopted, we will use the double layer formulation as it gives better conditioned linear system, where iterative solvers can be efficiently utilized.

\begin{remark}
The Schur complement admits better spectral property and this can be explained by the right preconditioning of BAE \eqref{eqn:double_bae}:
\begin{align}
(\Phi_+D_++\Phi_-D_-)(D_-^{-1}D_-)q_d = b
\end{align}
which gives
\begin{align}
(\Phi_+D_+D_-^{-1}+\Phi_-)v_{\gamma_-} = b
\end{align}
However, $D_-$ is dense and using $D_-$ as a preconditioner might outweigh the advantages of using iterative solvers.
\end{remark}

\subsection{Outline of the developed unfitted BAE}

Now we summarize the developed unfitted BAE.

\begin{algorithm}[t]
\caption{Unfitted Boundary Algebraic Method}\label{alg:bae}
\allowdisplaybreaks
\begin{algorithmic}[1]

% \Procedure{UnfittedBAE}{$\Omega$, $\Omega^0$, $f$, $g$, $h$, $\sigma$}
    \State \textbf{Input:} Domain $\Omega$, Auxiliary domain $\Omega^0$, RHS $f$, BC data $g$, grid size $h$, coefficient $\sigma$
    \State \textbf{Output:} Solution $u$ on $M^+$

    \Statex

    \State // Precompute Lattice Green's Function on infinite lattices
    \State $\text{LGF} \gets \text{PrecomputeLGF}(\sigma)$ 

    \Statex

    \State // Initialize finite grid and point sets
    \State $\text{finite grid } \omega_h \gets \text{CreateCartesianGrid}(\Omega^0, h)$
    \State $M^\pm, N^\pm, \gamma^\pm \gets \text{ClassifyPoints}(\text{grid}, \Omega^0, \Omega)$
    
    \Statex

    \State // Compute particular solution on finite grid $\omega_h$
    \State $u_p \gets G_h[\chi_{M^+}f_h]$ in $N^+$ via FFT

    \Statex
    
    \State // Form boundary system (double layer formulation)
    \State $\Phi_\pm \gets \text{ConstructBasisMatrices}(\gamma^+, \gamma^-)$ \Comment{Sparse matrices}
    \State $D_\pm \gets \text{ConstructDoubleLayerOperators}(\text{LGF}, \gamma^+, \gamma^-)$ \Comment{Matrix-free}
    
    \Statex

    \State // Solve boundary system
    \State $b \gets g - \text{EvaluateBasisFunctions}(u_p)$ \Comment{Subtract contribution of $u_p$ at boundary points}
    \State $A \gets \Phi_+D_+ + \Phi_-D_-$ \Comment{Matrix-free}
    \State $q_d \gets \text{GMRES}(A, b)$  \Comment{Double layer density}
    \State $u_{\gamma} \gets [D_+q_d; D_-q_d]$ \Comment{Matrix-free}

    \Statex
    
    \State // Reconstruct solution
    \State $u \gets u_p + G_h[\chi_{M^-}L_hu_\gamma]$ in $M^+$

\end{algorithmic}
\end{algorithm}

The memory requirement of Algorithm~\ref{alg:bae} will be $\mathcal{O}(N^3)$. The matrix-free multiplication of $D_+q_d$ and $D_-q_d$ would benefit from fast summation techniques, which is not explored in this work. The computational cost in Algorithm~\ref{alg:bae} mainly lies in the iterative solvers for finding the double layer density and the 3D FFT of finding particular solution and the construction of difference potentials once $u_\gamma$ is obtained. In this work, we use FFT to solve the for particular solution, utilizing the bounded domain. For unbounded domains and nonhomogenous source functions, fast summation techniques such as those developed in \cite{gillman2014fast} is in order.

\begin{remark}
The double layer formulation $D_{\pm}$ in Algorithm~\ref{alg:bae} can be replaced by the single layer formulation $S_\pm$ or the direct formulation in the difference potentials framework, depending on the computational need.
\end{remark}