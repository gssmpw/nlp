\usepackage{setspace}
\usepackage[utf8]{inputenc}
\usepackage[driver=dvipdfm,margin=1in]{geometry}

\usepackage[hyphens]{url}

\usepackage{amsmath}
\usepackage{mathtools}
\usepackage[colorlinks,urlcolor=blue, citecolor=blue, menucolor=blue, hypertexnames=false]{hyperref}
\usepackage{cleveref}
\usepackage{autonum}

\crefname{equation}{Eq.}{Eq.}% {環境名}{単数形}{複数形} \crefで引くときの表示
\crefname{figure}{Figure}{Figures}% {環境名}{単数形}{複数形} \crefで引くときの表示
\crefname{table}{Table}{Tables}% {環境名}{単数形}{複数形} \crefで引くときの表示
\crefname{thm}{Theorem}{Theorems}
\crefname{lem}{Lemma}{Lemmas}
\crefname{prop}{Proposition}{Propositions}


\usepackage{amsthm}
\usepackage{amssymb}
\usepackage{booktabs}
\usepackage{mathrsfs}

% \usepackage{ulem}
%\linespread{1.5}

% \mathtoolsset{showonlyrefs}
% \mathtoolsset{showmanualtags}



\makeatletter

%\oddsidemargin 10pt \evensidemargin 0pt \topmargin -1.3cm
%\textwidth 450pt \textheight 650pt

%%%%%%%%%%%%%%%%%%%%%%%%%%%%%% LyX specific LaTeX commands.
\newcommand{\noun}[1]{\textsc{#1}}

%%%%%%%%%%%%%%%%%%%%%%%%%%%%%% User specified LaTeX commands.

\usepackage{amssymb}
\usepackage{amsthm}
%\usepackage{amsfonts}
\usepackage{bbm}
\usepackage{graphics}
\usepackage[dvipdfmx]{graphicx}
\usepackage{accents}
\usepackage{placeins}
\usepackage{enumerate}
\usepackage{float}
\usepackage{caption}
\usepackage{subcaption}
\usepackage{ragged2e}
\usepackage[longnamesfirst]{natbib}
% \usepackage[compress]{natbib}
\usepackage{enumerate}
\usepackage{comment}
\usepackage{ascmac}
\usepackage{mathtools}
\usepackage{empheq}
\usepackage{empheq}
\usepackage[dvipsnames]{xcolor}


\newtheorem{thm}{Theorem}
\newtheorem{lem}{Lemma}
\newtheorem{prop}{Proposition}
\newtheorem{cor}{Corollary}
\newtheorem*{fact}{Fact}


\theoremstyle{remark}
\newtheorem{remark}{Remark}
\newtheorem{claim}{Claim}[thm]

\theoremstyle{definition}
\newtheorem{defn}{Definition}
\newtheorem{assp}{Assumption}



\usepackage{amssymb}
\usepackage{bm}
\usepackage{ascmac}
\usepackage{setspace}
\usepackage[at]{easylist}


\newenvironment{minilinespace}{
\baselineskip = 4mm
}

% \newenvironment{proof}[1][Proof]{\textbf{#1.}}{\rule{0.5em}{0.5em}}

%\renewcommand{\arraystretch}{1.5}


\theoremstyle{plain}

\newcommand{\noda}[1]{{\textcolor{red}{#1}}}
\newcommand{\kama}{\textcolor{blue}}

% Keywords command
\providecommand{\keywords}[1]
{
  \small	
  \textbf{Keywords:} #1
}
\providecommand{\JEL}[1]
{
  \small	
  \textbf{JEL Codes:} #1
}

\usepackage{algorithm,algpseudocode}

\DeclareMathOperator*{\argmax}{arg\,max}
\DeclareMathOperator*{\argmin}{arg\,min}

\makeatother

\makeatletter
\let\save@mathaccent\mathaccent
\newcommand*\if@single[3]{%
  \setbox0\hbox{${\mathaccent"0362{#1}}^H$}%
  \setbox2\hbox{${\mathaccent"0362{\kern0pt#1}}^H$}%
  \ifdim\ht0=\ht2 #3\else #2\fi
  }
%The bar will be moved to the right by a half of \macc@kerna, which is computed by amsmath:
\newcommand*\rel@kern[1]{\kern#1\dimexpr\macc@kerna}
%If there's a superscript following the bar, then no negative kern may follow the bar;
%an additional {} makes sure that the superscript is high enough in this case:
\newcommand*\widebar[1]{\@ifnextchar^{{\wide@bar{#1}{0}}}{\wide@bar{#1}{1}}}
%Use a separate algorithm for single symbols:
\newcommand*\wide@bar[2]{\if@single{#1}{\wide@bar@{#1}{#2}{1}}{\wide@bar@{#1}{#2}{2}}}
\newcommand*\wide@bar@[3]{%
  \begingroup
  \def\mathaccent##1##2{%
%Enable nesting of accents:
    \let\mathaccent\save@mathaccent
%If there's more than a single symbol, use the first character instead (see below):
    \if#32 \let\macc@nucleus\first@char \fi
%Determine the italic correction:
    \setbox\z@\hbox{$\macc@style{\macc@nucleus}_{}$}%
    \setbox\tw@\hbox{$\macc@style{\macc@nucleus}{}_{}$}%
    \dimen@\wd\tw@
    \advance\dimen@-\wd\z@
%Now \dimen@ is the italic correction of the symbol.
    \divide\dimen@ 3
    \@tempdima\wd\tw@
    \advance\@tempdima-\scriptspace
%Now \@tempdima is the width of the symbol.
    \divide\@tempdima 10
    \advance\dimen@-\@tempdima
%Now \dimen@ = (italic correction / 3) - (Breite / 10)
    \ifdim\dimen@>\z@ \dimen@0pt\fi
%The bar will be shortened in the case \dimen@<0 !
    \rel@kern{0.6}\kern-\dimen@
    \if#31
      \overline{\rel@kern{-0.6}\kern\dimen@\macc@nucleus\rel@kern{0.4}\kern\dimen@}%
      \advance\dimen@0.4\dimexpr\macc@kerna
%Place the combined final kern (-\dimen@) if it is >0 or if a superscript follows:
      \let\final@kern#2%
      \ifdim\dimen@<\z@ \let\final@kern1\fi
      \if\final@kern1 \kern-\dimen@\fi
    \else
      \overline{\rel@kern{-0.6}\kern\dimen@#1}%
    \fi
  }%
  \macc@depth\@ne
  \let\math@bgroup\@empty \let\math@egroup\macc@set@skewchar
  \mathsurround\z@ \frozen@everymath{\mathgroup\macc@group\relax}%
  \macc@set@skewchar\relax
  \let\mathaccentV\macc@nested@a
%The following initialises \macc@kerna and calls \mathaccent:
  \if#31
    \macc@nested@a\relax111{#1}%
  \else
%If the argument consists of more than one symbol, and if the first token is
%a letter, use that letter for the computations:
    \def\gobble@till@marker##1\endmarker{}%
    \futurelet\first@char\gobble@till@marker#1\endmarker
    \ifcat\noexpand\first@char A\else
      \def\first@char{}%
    \fi
    \macc@nested@a\relax111{\first@char}%
  \fi
  \endgroup
}
\makeatother






