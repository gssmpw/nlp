\section{Related works}
There are several datasets constructed for sentiment analysis in Polish.
PolEmo 2.0~\cite{kocon-etal-2019-multi} contains over 57k sentences with annotated sentiment polarity, later extended to AspectEmo~\cite{11321849} with sentiment assigned to each token of the sentence. Even more fine-grained annotation can be found in the Polish Sentiment Treebank\footnote{http://zil.ipipan.waw.pl/TreebankWydzwieku}, used in the PolEval-2017 challenge~\cite{wawer2017results}, which is a dependency treebank with sentiment annotations for each subphrase of the sentence.


Short informal texts are at the focus of TwitterEmo corpus~\cite{10.1007/978-3-031-36021-3_20} which provides annotation for detecting sarcasm, sentiment and basic emotions.
\citet{ptaszynski2019results} collected dataset for cyberbullying detection.
Similarly, HateSpeech corpus\footnote{\url{http://zil.ipipan.waw.pl/HateSpeech}}\cite{troszynski2017czy}  comprises online posts with offensive content.

Allegro Reviews dataset~\cite{rybak-etal-2020-klej} contains customer reviews with 1-5 ratings from a popular Polish e-commerce marketplace.
Finally, Wroclaw Corpus of Consumer Reviews Sentiment (WCCRS,~\citealt{Kocon2019}) is a multi-domain dataset of Polish textual reviews, available on a permissive license that allows its adaptation and transformation.

None of the aforementioned Polish datasets is suitable for training machine learning models for ASTE, but naturally such datasets do exist for English.
The datasets originate from SemEval shared tasks~\cite{pontiki-etal-2014-semeval,pontiki-etal-2015-semeval,pontiki-etal-2016-semeval}, for which the following corpora were constructed: 14lab (opinions about laptops), 14res, 15res and 16res (restaurant reviews).
 \citet{fan-etal-2019-target} extended these corpora by adding opinion term annotation and \citet{peng2020knowing} %and~\citet{wu-etal-2020-grid} 
 finally compiled them for ASTE.


It is important to mention that there are other related tasks like structured SA or fine-grained SA, which have datasets constructed for different languages  e.g.~for {B}asque and {C}atalan~\cite{barnes-etal-2018-multibooked}, German \cite{klinger-cimiano-2014-usage}, Czech~\cite{steinberger-etal-2014-aspect} or Hindi~\cite{akhtar-etal-2016-aspect}.
Some of these tasks even involve the extraction of some triplets from texts, but they contain opinion holders, aspect categories, or others.