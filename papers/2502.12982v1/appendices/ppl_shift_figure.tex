% \section*{PPL Shifts in Different Languages}

\begin{table}[htbp]
\centering
\caption{Data Distribution of SEA-UltraChat (Instruction Tuning) and SEA-UltraFeedback (Preference Tuning) across Languages. SEA-UltraChat has been used by two-stage instruction tuning.}
\begin{tabular}{lcccc}
\toprule
\textbf{Language} & \textbf{SFT Stage 1} & \textbf{SFT Stage 2} & \textbf{SEA-UltraChat} & \textbf{SEA-UltraFeedback} \\
\midrule
English    & 1611404 & 72000  & 1683404 & 8981  \\
Chinese    & 154908  & 72000  & 226908  & 8845  \\
Indonesian & 287719  & 48000  & 335719  & 4687  \\
Thai       & 334194  & 48000  & 382194  & 4696  \\
Vietnamese & 301226  & 48000  & 349226  & 4710  \\
Malay      & 212827  & 48000  & 260827  & 4714  \\
Burmese    & 204495  & 48000  & 252495  & 4614  \\
Lao        & 134228  & 48000  & 182228  & 4393  \\
Cebuano    & 194986  & 1200   & 196186  & 1178  \\
Ilocano    & 1867    & 1200   & 3067    & 1172  \\
Javanese   & 134499  & 48000  & 182499  & 1166  \\
Khmer      & 64421   & 4800   & 69221   & 1151  \\
Sundanese  & 181486  & 1200   & 182686  & 1174  \\
Tagalog    & 217045  & 48000  & 265045  & 1173  \\
Waray      & 177921  & 1200   & 179121  & 0     \\
Tamil      & 82015   & 1200   & 83215   & 1159  \\
\midrule
\textbf{Total} & 4295241 & 538800 & 4834041 & 53813 \\
\bottomrule
\end{tabular}
\label{table:sft_data_distribution}
\end{table}


\begin{figure*}
\centering
\begin{minipage}[b]{\textwidth}
\centering
\includegraphics[width=0.95\textwidth]{figures/sec8_1_ppl_shift.pdf}
\end{minipage}
\captionof{figure}{Comparison of PPL distribution smoothed with Kernel Density Estimation (KDE). We compare Sailor2-8B, Qwen2.5-7B, Sailor1-7b and Qwen1.5-7B. Our results demonstrate that with extra 1B parameters, Sailor2-8B can preserve its English and Chinese capability, while achieving in much lower PPL in SEA languages.}
\label{fig:ppl_shift}
\end{figure*}

\begin{figure*}
\centering
\begin{minipage}[b]{\textwidth}
\centering
\includegraphics[width=1\textwidth]{figures/sec8_1_WR_compare.pdf}
\end{minipage}
\captionof{figure}{Comparison of win rate of four models based on their ChrF++ scores.
The shaded area in each cube represents the top-1 accuracy of each model across different translation directions in the Flores Plus Translation Dataset.
We observed that Sailor2 performs on par with, or even outperforms NLLB, a model optimized for translation tasks that excels at translating low-resource languages.
}
\label{fig:flores_plus_WR}
\end{figure*}

\begin{figure*}
\centering
\begin{minipage}[b]{\textwidth}
\centering
\includegraphics[width=0.95\textwidth]{figures/sec8_1_bleu_chrf_diff_vs_qwen2_5_32b.pdf}
\end{minipage}
\captionof{figure}{Comparison of BLEU and ChrF++ scores on the Flores Plus Translation Dataset across various source-target language pairs between Sailor2 20B and Qwen2.5-32B. BLEU Score Difference = Sailor2 BLEU - Qwen2.5 BLEU.}
\label{fig:flores_plus__qwen_32b}
\end{figure*}

\begin{figure*}
\centering
\begin{minipage}[b]{\textwidth}
\centering
\includegraphics[width=0.95\textwidth]{figures/sec8_1_bleu_chrf_diff_vs_qwen2_5_72b.pdf}
\end{minipage}
\captionof{figure}{Comparison of BLEU and ChrF++ scores on the Flores Plus Translation Dataset across various source-target language pairs between Sailor2 20B and Qwen2.5-72B. BLEU Score Difference = Sailor2 BLEU - Qwen2.5 BLEU.}
\label{fig:flores_plus__qwen_72b}
\end{figure*}

\begin{figure*}
\centering
\begin{minipage}[b]{\textwidth}
\centering
\includegraphics[width=0.95\textwidth]{figures/sec8_1_bleu_chrf_diff_vs_llama3_1_70b.pdf}
\end{minipage}
\captionof{figure}{Comparison of BLEU and ChrF++ scores on the Flores Plus Translation Dataset across various source-target language pairs between Sailor2 20B and Llama3.1-70B. BLEU Score Difference = Sailor2 BLEU - Llama3.1 BLEU.}
\label{fig:flores_plus__llama_70b}
\end{figure*}

\begin{figure*}
\centering
\begin{minipage}[b]{\textwidth}
\centering
\includegraphics[width=0.95\textwidth]{figures/sec8_1_bleu_chrf_diff_vs_nllb_moe_54b.pdf}
\end{minipage}
\captionof{figure}{Comparison of BLEU and ChrF++ scores on the Flores Plus Translation Dataset across various source-target language pairs between Sailor2 20B and NLLB-MoE-54B. BLEU Score Difference = Sailor2 BLEU - NLLB BLEU. We noticed NLLB failed to generate complete long Chinese sentences: \url{https://github.com/facebookresearch/fairseq/issues/5549}, and we also found many common Chinese characters and punctuations are tokenized to $<$unk$>$.}
\label{fig:flores_plus__nllb_moe_54b}
\end{figure*}

\begin{table}[ht]
\centering
\caption{Performance on Flores Plus (chrF++) for Qwen2.5-32B}
\label{tab:matrix-chrf-qwen2_5_32b}
\resizebox{\textwidth}{!}{%
\begin{tabular}{l *{16}{c}}
% \begin{tabular}{|l|c|c|c|c|c|c|c|c|c|c|c|c|c|c|c|c|}
% \hline
% \diagbox{Src}{Tgt} & \texttt{eng} & \texttt{cmn} & \texttt{ind} & \texttt{tha} & \texttt{vie} & \texttt{zsm} & \texttt{mya} & \texttt{lao} & \texttt{ceb} & \texttt{ilo} & \texttt{jav} & \texttt{khm} & \texttt{sun} & \texttt{fil} & \texttt{war} & \texttt{tam}\\ % \hline
\toprule
\makecell[l]{\textbf{Target} ($\rightarrow$) \\ \textbf{Source} ($\downarrow$)}
 & \texttt{eng} & \texttt{cmn} & \texttt{ind} & \texttt{tha} & \texttt{vie} & \texttt{zsm} & \texttt{mya} & \texttt{lao} & \texttt{ceb} & \texttt{ilo} & \texttt{jav} & \texttt{khm} & \texttt{sun} & \texttt{fil} & \texttt{war} & \texttt{tam} \\
\midrule
English (\texttt{eng}) & - & 34.0 & 64.0 & 46.1 & 59.2 & 59.5 & 17.3 & 21.4 & 39.4 & 27.1 & 31.8 & 20.6 & 33.1 & 49.5 & 38.4 & 28.3 \\ % \hline
Chinese (\texttt{cmn}) & 59.0 & - & 52.6 & 40.2 & 51.8 & 48.3 & 16.0 & 18.0 & 33.0 & 23.1 & 27.0 & 18.8 & 28.7 & 41.5 & 30.4 & 24.6 \\ % \hline
Indonesian (\texttt{ind}) & 67.5 & 29.5 & - & 43.1 & 54.3 & 55.0 & 16.2 & 20.6 & 34.9 & 23.5 & 34.9 & 19.6 & 36.2 & 45.2 & 31.7 & 25.4 \\ % \hline
Thai (\texttt{tha}) & 60.6 & 27.9 & 54.8 & - & 52.1 & 50.3 & 15.6 & 22.0 & 33.0 & 23.7 & 28.5 & 19.5 & 29.6 & 42.6 & 30.9 & 24.7 \\ % \hline
Vietnamese (\texttt{vie}) & 62.5 & 29.1 & 55.0 & 41.9 & - & 50.7 & 15.9 & 20.1 & 32.4 & 24.2 & 27.2 & 19.4 & 29.3 & 42.7 & 30.5 & 24.4 \\ % \hline
Malay (\texttt{zsm}) & 67.2 & 28.6 & 59.4 & 42.8 & 53.6 & - & 16.0 & 20.5 & 34.8 & 24.0 & 33.1 & 20.0 & 33.5 & 44.5 & 32.3 & 25.5 \\ % \hline
Burmese (\texttt{mya}) & 39.6 & 16.7 & 38.2 & 29.8 & 35.0 & 36.0 & - & 15.4 & 25.1 & 20.0 & 18.9 & 15.5 & 20.4 & 33.4 & 23.4 & 21.2 \\ % \hline
Lao (\texttt{lao}) & 50.1 & 22.8 & 47.8 & 39.9 & 44.4 & 44.6 & 13.8 & - & 27.6 & 21.2 & 24.3 & 19.1 & 25.9 & 38.9 & 26.1 & 21.6 \\ % \hline
Cebuano (\texttt{ceb}) & 56.0 & 23.7 & 48.8 & 35.9 & 45.0 & 45.2 & 15.2 & 18.6 & - & 27.7 & 22.9 & 17.8 & 22.4 & 45.5 & 44.0 & 23.1 \\ % \hline
Ilocano (\texttt{ilo}) & 43.1 & 19.6 & 39.5 & 30.0 & 36.3 & 36.8 & 13.3 & 16.3 & 31.7 & - & 19.9 & 15.2 & 20.2 & 38.8 & 30.2 & 19.4 \\ % \hline
Javanese (\texttt{jav}) & 51.9 & 22.9 & 49.9 & 35.0 & 43.8 & 45.9 & 14.1 & 17.9 & 28.7 & 21.8 & - & 17.9 & 31.7 & 37.0 & 25.9 & 21.9 \\ % \hline
Khmer (\texttt{khm}) & 49.8 & 22.8 & 47.8 & 37.1 & 43.8 & 44.3 & 12.9 & 20.4 & 28.8 & 21.5 & 24.7 & - & 24.7 & 38.3 & 26.1 & 22.4 \\ % \hline
Sundanese (\texttt{sun}) & 51.7 & 23.1 & 51.2 & 35.9 & 44.7 & 46.1 & 14.5 & 18.1 & 28.6 & 20.4 & 31.1 & 17.4 & - & 37.9 & 27.0 & 21.9 \\ % \hline
Tagalog (\texttt{fil}) & 65.8 & 27.9 & 55.1 & 41.0 & 52.0 & 51.0 & 16.1 & 19.8 & 41.6 & 30.7 & 23.4 & 18.8 & 22.2 & - & 40.1 & 25.5 \\ % \hline
Waray (\texttt{war}) & 56.8 & 23.8 & 49.1 & 36.1 & 45.1 & 45.1 & 14.9 & 18.7 & 44.2 & 29.3 & 23.2 & 17.2 & 23.7 & 45.6 & - & 23.2 \\ % \hline
Tamil (\texttt{tam}) & 47.9 & 20.5 & 44.0 & 32.9 & 40.9 & 41.0 & 16.1 & 16.2 & 28.7 & 21.8 & 22.9 & 16.8 & 24.1 & 37.5 & 26.2 & - \\ % \hline
\bottomrule
\end{tabular}
}
\end{table}
\begin{table}[ht]
\centering
\caption{Performance on Flores Plus (chrF++) for Qwen2.5-72B}
\label{tab:matrix-chrf-qwen2_5_72b}
\resizebox{\textwidth}{!}{%
% \begin{tabular}{|l|c|c|c|c|c|c|c|c|c|c|c|c|c|c|c|c|}
\begin{tabular}{l *{16}{c}}
% % \hline
% \diagbox{Src}{Tgt} & \texttt{eng} & \texttt{cmn} & \texttt{ind} & \texttt{tha} & \texttt{vie} & \texttt{zsm} & \texttt{mya} & \texttt{lao} & \texttt{ceb} & \texttt{ilo} & \texttt{jav} & \texttt{khm} & \texttt{sun} & \texttt{fil} & \texttt{war} & \texttt{tam}\\ % \hline
\toprule
\makecell[l]{\textbf{Target} ($\rightarrow$) \\ \textbf{Source} ($\downarrow$)}
 & \texttt{eng} & \texttt{cmn} & \texttt{ind} & \texttt{tha} & \texttt{vie} & \texttt{zsm} & \texttt{mya} & \texttt{lao} & \texttt{ceb} & \texttt{ilo} & \texttt{jav} & \texttt{khm} & \texttt{sun} & \texttt{fil} & \texttt{war} & \texttt{tam} \\
\midrule
English (\texttt{eng}) & - & 34.3 & 66.5 & 47.9 & 60.8 & 62.4 & 19.9 & 24.8 & 44.4 & 31.8 & 36.8 & 21.9 & 37.2 & 53.1 & 43.7 & 31.2 \\ % \hline
Chinese (\texttt{cmn}) & 59.9 & - & 54.4 & 41.7 & 53.1 & 50.6 & 17.9 & 21.5 & 37.1 & 26.6 & 31.5 & 20.1 & 32.9 & 44.9 & 36.8 & 26.5 \\ % \hline
Indonesian (\texttt{ind}) & 68.7 & 30.1 & - & 44.7 & 56.1 & 57.3 & 19.0 & 23.9 & 40.4 & 30.2 & 38.3 & 21.4 & 39.1 & 48.9 & 38.1 & 28.7 \\ % \hline
Thai (\texttt{tha}) & 62.3 & 28.9 & 56.6 & - & 54.1 & 53.1 & 18.6 & 25.2 & 38.3 & 26.6 & 33.5 & 20.8 & 34.3 & 46.3 & 36.5 & 27.3 \\ % \hline
Vietnamese (\texttt{vie}) & 63.9 & 29.8 & 57.2 & 43.6 & - & 53.3 & 18.3 & 23.3 & 38.3 & 28.2 & 34.0 & 21.1 & 34.6 & 47.1 & 36.5 & 27.4 \\ % \hline
Malay (\texttt{zsm}) & 68.8 & 29.9 & 60.9 & 44.1 & 55.0 & - & 19.1 & 23.8 & 40.0 & 29.5 & 36.9 & 21.6 & 37.1 & 48.7 & 38.2 & 28.9 \\ % \hline
Burmese (\texttt{mya}) & 43.1 & 18.9 & 42.6 & 33.2 & 39.4 & 40.4 & - & 18.4 & 30.1 & 21.2 & 24.8 & 17.7 & 25.2 & 37.8 & 28.3 & 24.3 \\ % \hline
Lao (\texttt{lao}) & 55.0 & 24.8 & 51.9 & 42.9 & 48.9 & 48.6 & 17.1 & - & 34.1 & 24.1 & 29.1 & 21.7 & 30.5 & 43.3 & 32.2 & 25.3 \\ % \hline
Cebuano (\texttt{ceb}) & 60.1 & 26.3 & 52.5 & 39.0 & 48.4 & 49.3 & 17.9 & 20.4 & - & 33.6 & 25.4 & 19.6 & 30.0 & 49.2 & 45.2 & 26.6 \\ % \hline
Ilocano (\texttt{ilo}) & 44.7 & 21.2 & 42.3 & 32.3 & 39.0 & 39.5 & 15.4 & 17.1 & 36.7 & - & 21.8 & 16.8 & 25.4 & 41.8 & 36.4 & 22.0 \\ % \hline
Javanese (\texttt{jav}) & 55.6 & 24.6 & 53.0 & 37.2 & 47.1 & 49.2 & 17.1 & 20.2 & 34.4 & 26.7 & - & 19.5 & 34.6 & 41.6 & 32.8 & 25.3 \\ % \hline
Khmer (\texttt{khm}) & 52.3 & 24.4 & 50.3 & 39.4 & 47.1 & 47.5 & 16.3 & 22.2 & 33.1 & 22.4 & 29.7 & - & 30.3 & 42.8 & 31.3 & 25.6 \\ % \hline
Sundanese (\texttt{sun}) & 54.8 & 24.8 & 53.7 & 37.8 & 47.0 & 48.8 & 16.5 & 20.1 & 34.0 & 26.3 & 35.2 & 19.4 & - & 41.6 & 32.3 & 24.9 \\ % \hline
Tagalog (\texttt{fil}) & 69.0 & 29.1 & 58.4 & 43.2 & 54.4 & 54.7 & 18.9 & 22.0 & 45.1 & 35.3 & 28.5 & 20.8 & 31.6 & - & 42.6 & 28.7 \\ % \hline
Waray (\texttt{war}) & 59.6 & 25.8 & 52.1 & 38.3 & 48.0 & 48.7 & 16.9 & 19.9 & 46.9 & 33.9 & 26.6 & 19.1 & 29.7 & 49.1 & - & 26.4 \\ % \hline
Tamil (\texttt{tam}) & 50.4 & 21.6 & 47.1 & 35.7 & 43.7 & 45.1 & 18.2 & 18.2 & 34.1 & 23.6 & 27.5 & 18.6 & 28.4 & 41.7 & 31.5 & - \\ % \hline
\bottomrule
\end{tabular}
}
\end{table}
\begin{table}[ht]
\centering
\caption{Performance on Flores Plus (chrF++) for Llama3.1-70B}
\label{tab:matrix-chrf-llama3_1_70b}
\resizebox{\textwidth}{!}{%
% \begin{tabular}{|l|c|c|c|c|c|c|c|c|c|c|c|c|c|c|c|c|}
% % \hline
% \diagbox{Src}{Tgt} & \texttt{eng} & \texttt{cmn} & \texttt{ind} & \texttt{tha} & \texttt{vie} & \texttt{zsm} & \texttt{mya} & \texttt{lao} & \texttt{ceb} & \texttt{ilo} & \texttt{jav} & \texttt{khm} & \texttt{sun} & \texttt{fil} & \texttt{war} & \texttt{tam}\\ % \hline
\begin{tabular}{l *{16}{c}}
% % % \hline
% \diagbox{Src}{Tgt} & \texttt{eng} & \texttt{cmn} & \texttt{ind} & \texttt{tha} & \texttt{vie} & \texttt{zsm} & \texttt{mya} & \texttt{lao} & \texttt{ceb} & \texttt{ilo} & \texttt{jav} & \texttt{khm} & \texttt{sun} & \texttt{fil} & \texttt{war} & \texttt{tam}\\ % % \hline
\toprule
\makecell[l]{\textbf{Target} ($\rightarrow$) \\ \textbf{Source} ($\downarrow$)}
 & \texttt{eng} & \texttt{cmn} & \texttt{ind} & \texttt{tha} & \texttt{vie} & \texttt{zsm} & \texttt{mya} & \texttt{lao} & \texttt{ceb} & \texttt{ilo} & \texttt{jav} & \texttt{khm} & \texttt{sun} & \texttt{fil} & \texttt{war} & \texttt{tam} \\
\midrule
English (\texttt{eng}) & - & 32.3 & 67.3 & 47.9 & 61.1 & 66.2 & 21.3 & 22.7 & 54.5 & 46.2 & 51.1 & 21.5 & 45.3 & 57.8 & 52.8 & 37.9 \\ % \hline
Chinese (\texttt{cmn}) & 59.0 & - & 54.2 & 41.4 & 52.8 & 53.4 & 19.3 & 19.2 & 45.2 & 38.9 & 41.9 & 19.6 & 38.8 & 48.0 & 43.2 & 31.9 \\ % \hline
Indonesian (\texttt{ind}) & 68.5 & 27.8 & - & 44.6 & 56.2 & 60.6 & 20.4 & 21.6 & 49.3 & 42.2 & 47.6 & 20.7 & 44.6 & 53.3 & 46.5 & 34.6 \\ % \hline
Thai (\texttt{tha}) & 61.2 & 26.9 & 57.1 & - & 53.8 & 55.9 & 19.6 & 21.8 & 46.9 & 40.2 & 44.0 & 20.3 & 40.0 & 49.8 & 44.5 & 32.6 \\ % \hline
Vietnamese (\texttt{vie}) & 63.0 & 27.2 & 57.9 & 43.6 & - & 56.6 & 19.9 & 20.9 & 47.6 & 40.2 & 44.2 & 20.5 & 41.1 & 51.1 & 44.9 & 33.3 \\ % \hline
Malay (\texttt{zsm}) & 69.4 & 28.2 & 61.4 & 44.1 & 55.6 & - & 20.5 & 21.9 & 49.7 & 42.6 & 48.5 & 20.9 & 43.4 & 53.0 & 47.4 & 34.4 \\ % \hline
Burmese (\texttt{mya}) & 51.0 & 21.9 & 48.5 & 37.0 & 46.0 & 48.5 & - & 16.8 & 41.0 & 35.3 & 37.9 & 18.1 & 33.9 & 45.0 & 39.6 & 31.5 \\ % \hline
Lao (\texttt{lao}) & 49.0 & 22.0 & 49.3 & 41.4 & 45.1 & 49.6 & 17.4 & - & 42.3 & 37.1 & 39.6 & 19.5 & 35.5 & 45.3 & 40.1 & 28.5 \\ % \hline
Cebuano (\texttt{ceb}) & 65.2 & 26.2 & 57.2 & 42.2 & 52.7 & 56.2 & 20.0 & 19.8 & - & 44.6 & 43.0 & 20.4 & 39.4 & 54.4 & 51.0 & 33.2 \\ % \hline
Ilocano (\texttt{ilo}) & 55.0 & 23.7 & 50.6 & 37.4 & 47.3 & 50.3 & 18.8 & 18.3 & 47.2 & - & 39.5 & 18.7 & 36.5 & 49.6 & 45.7 & 31.0 \\ % \hline
Javanese (\texttt{jav}) & 63.1 & 26.0 & 56.9 & 41.0 & 52.0 & 57.0 & 19.4 & 19.8 & 46.5 & 40.0 & - & 20.2 & 41.5 & 50.0 & 43.5 & 32.5 \\ % \hline
Khmer (\texttt{khm}) & 55.4 & 25.2 & 53.3 & 41.4 & 50.0 & 53.0 & 18.6 & 20.2 & 44.7 & 38.1 & 41.3 & - & 37.3 & 48.3 & 41.8 & 31.0 \\ % \hline
Sundanese (\texttt{sun}) & 60.5 & 25.2 & 57.8 & 41.2 & 51.2 & 56.1 & 19.5 & 19.6 & 45.5 & 39.7 & 46.0 & 20.2 & - & 48.3 & 43.0 & 32.5 \\ % \hline
Tagalog (\texttt{fil}) & 69.5 & 27.9 & 60.0 & 43.5 & 55.1 & 59.1 & 20.2 & 20.2 & 52.6 & 44.8 & 45.1 & 20.5 & 41.1 & - & 50.4 & 34.2 \\ % \hline
Waray (\texttt{war}) & 66.4 & 26.6 & 57.4 & 41.9 & 52.8 & 56.9 & 19.9 & 19.9 & 54.0 & 45.5 & 43.4 & 19.9 & 39.9 & 54.9 & - & 33.4 \\ % \hline
Tamil (\texttt{tam}) & 58.2 & 23.8 & 53.0 & 39.3 & 49.5 & 52.7 & 19.9 & 17.3 & 44.9 & 38.6 & 40.8 & 18.5 & 37.2 & 48.4 & 42.7 & - \\ % \hline
\bottomrule
\end{tabular}
}
\end{table}
\begin{table}[ht]
\centering
\caption{Translation Performance (chrF++) for NLLB-MoE-54B}
\label{tab:matrix-chrf-nllb_moe_54b}
\resizebox{\textwidth}{!}{%
% \begin{tabular}{|l|c|c|c|c|c|c|c|c|c|c|c|c|c|c|c|c|}
% % \hline
% \diagbox{Src}{Tgt} & \texttt{eng} & \texttt{cmn} & \texttt{ind} & \texttt{tha} & \texttt{vie} & \texttt{zsm} & \texttt{mya} & \texttt{lao} & \texttt{ceb} & \texttt{ilo} & \texttt{jav} & \texttt{khm} & \texttt{sun} & \texttt{fil} & \texttt{war} & \texttt{tam}\\ % \hline
\begin{tabular}{l *{16}{c}}
% % % \hline
% \diagbox{Src}{Tgt} & \texttt{eng} & \texttt{cmn} & \texttt{ind} & \texttt{tha} & \texttt{vie} & \texttt{zsm} & \texttt{mya} & \texttt{lao} & \texttt{ceb} & \texttt{ilo} & \texttt{jav} & \texttt{khm} & \texttt{sun} & \texttt{fil} & \texttt{war} & \texttt{tam}\\ % % \hline
\toprule
\makecell[l]{\textbf{Target} ($\rightarrow$) \\ \textbf{Source} ($\downarrow$)}
 & \texttt{eng} & \texttt{cmn} & \texttt{ind} & \texttt{tha} & \texttt{vie} & \texttt{zsm} & \texttt{mya} & \texttt{lao} & \texttt{ceb} & \texttt{ilo} & \texttt{jav} & \texttt{khm} & \texttt{sun} & \texttt{fil} & \texttt{war} & \texttt{tam} \\
\midrule
English (\texttt{eng}) & - & 18.9 & 65.8 & 39.8 & 58.8 & 64.7 & 30.1 & 43.2 & 54.9 & 52.5 & 52.4 & 32.6 & 42.3 & 58.9 & 55.0 & 50.2 \\ % \hline
Chinese (\texttt{cmn}) & 56.2 & - & 50.7 & 34.8 & 49.1 & 50.4 & 26.8 & 35.0 & 44.2 & 41.5 & 40.2 & 27.6 & 36.1 & 45.3 & 40.5 & 39.0 \\ % \hline
Indonesian (\texttt{ind}) & 66.1 & 16.1 & - & 38.0 & 53.6 & 58.6 & 26.2 & 40.7 & 49.6 & 47.0 & 47.9 & 32.0 & 42.8 & 52.3 & 48.3 & 43.6 \\ % \hline
Thai (\texttt{tha}) & 57.7 & 16.2 & 54.4 & - & 50.6 & 53.1 & 25.2 & 39.3 & 47.0 & 44.1 & 44.0 & 30.8 & 38.6 & 47.9 & 44.0 & 39.9 \\ % \hline
Vietnamese (\texttt{vie}) & 61.2 & 16.6 & 56.3 & 36.6 & - & 54.9 & 25.4 & 39.6 & 48.1 & 45.3 & 45.3 & 31.4 & 40.1 & 49.8 & 45.6 & 42.3 \\ % \hline
Malay (\texttt{zsm}) & 67.1 & 17.6 & 60.4 & 38.3 & 53.2 & - & 26.7 & 41.3 & 50.7 & 47.4 & 48.4 & 32.2 & 42.3 & 52.5 & 47.9 & 45.1 \\ % \hline
Burmese (\texttt{mya}) & 54.0 & 13.2 & 50.5 & 33.6 & 46.7 & 49.9 & - & 34.4 & 41.8 & 40.1 & 37.5 & 26.6 & 34.5 & 44.7 & 41.5 & 39.2 \\ % \hline
Lao (\texttt{lao}) & 61.1 & 18.3 & 55.6 & 39.5 & 51.0 & 54.8 & 26.8 & - & 48.8 & 44.6 & 45.0 & 31.4 & 39.4 & 49.9 & 44.5 & 43.2 \\ % \hline
Cebuano (\texttt{ceb}) & 67.0 & 15.7 & 57.9 & 36.5 & 52.3 & 57.5 & 25.6 & 39.9 & - & 49.5 & 47.0 & 30.2 & 39.4 & 54.2 & 51.9 & 42.0 \\ % \hline
Ilocano (\texttt{ilo}) & 62.5 & 14.9 & 55.3 & 34.7 & 49.8 & 54.7 & 24.7 & 38.4 & 50.2 & - & 45.4 & 28.9 & 38.4 & 52.1 & 49.9 & 40.7 \\ % \hline
Javanese (\texttt{jav}) & 61.9 & 15.9 & 57.1 & 35.1 & 50.2 & 55.3 & 27.4 & 38.7 & 49.1 & 43.9 & - & 29.1 & 41.1 & 50.4 & 43.1 & 43.0 \\ % \hline
Khmer (\texttt{khm}) & 58.4 & 17.8 & 54.3 & 37.0 & 49.8 & 53.6 & 27.4 & 39.7 & 47.9 & 43.8 & 44.1 & - & 39.1 & 49.1 & 43.9 & 42.3 \\ % \hline
Sundanese (\texttt{sun}) & 61.3 & 18.1 & 58.2 & 36.6 & 50.1 & 55.6 & 27.9 & 38.7 & 48.1 & 44.6 & 45.9 & 29.8 & - & 49.7 & 44.6 & 42.9 \\ % \hline
Tagalog (\texttt{fil}) & 69.7 & 16.4 & 59.6 & 37.3 & 54.1 & 58.9 & 26.5 & 41.0 & 52.6 & 50.6 & 48.6 & 31.3 & 41.0 & - & 51.8 & 44.1 \\ % \hline
Waray (\texttt{war}) & 69.1 & 17.1 & 59.6 & 38.2 & 54.1 & 58.9 & 27.7 & 40.7 & 52.8 & 50.6 & 47.9 & 31.2 & 40.9 & 55.3 & - & 44.3 \\ % \hline
Tamil (\texttt{tam}) & 61.1 & 15.9 & 54.2 & 35.3 & 50.2 & 53.7 & 26.8 & 37.4 & 47.8 & 45.5 & 44.1 & 30.2 & 38.2 & 49.5 & 45.4 & - \\ % \hline
\bottomrule
\end{tabular}
}
\end{table}