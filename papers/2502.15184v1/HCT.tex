\documentclass[lettersize,journal]{IEEEtran}
\usepackage{amsmath,amsfonts}
\usepackage{algorithmic}
\usepackage{algorithm}
\usepackage{array}
\usepackage[caption=false,font=normalsize,labelfont=sf,textfont=sf]{subfig}
\usepackage{textcomp}
\usepackage{stfloats}
\usepackage{url}
\usepackage{verbatim}
\usepackage{graphicx}
\usepackage{cite}
\hyphenation{op-tical net-works semi-conduc-tor IEEE-Xplore}
% updated with editorial comments 8/9/2021
% updated with editorial comments 8/9/2021
\usepackage{booktabs}
\usepackage{multirow}
\usepackage[dvipsnames]{xcolor}
\usepackage{textcomp}
\definecolor{darkgreen}{RGB}{0,100,0}
%设置图表引用
\newcommand{\secref}[1]{Section~\ref{#1}}
\newcommand{\reffig}[1]{Fig. \ref{#1}}
\newcommand{\refeq}[1]{Eq.(\ref{#1})}
\newcommand{\reftable}[1]{Table \ref{#1}}
% \DeclareUnicodeCharacter

\begin{document}

\title{Hierarchical Context Transformer for Multi-level Semantic Scene Understanding}
\author{Luoying Hao, Yan Hu, Yang Yue,  Li Wu, Huazhu Fu \IEEEmembership{Senior Member, IEEE}, Jinming Duan, and Jiang Liu \IEEEmembership{Senior Member, IEEE}
\thanks{Luoying Hao and Yan Hu contributed equally to this work. Corresponding author: Yan Hu, Jinming Duan and Jiang Liu.}
\thanks{Luoying Hao, Yan Hu and Jiang Liu are with the Research Institute of Trustworthy Autonomous Systems and Dept. of Computer Science and Engineering, Southern University of Science and Technology, China. (e-mail: \{huy3, liuj\}@sustech.edu.cn). Luoying Hao, Yang Yue and Jinming Duan are with the School of Computer Science, University of Birmingham, UK. Jinming Duan is also with the Division of Informatics, Imaging and Data Sciences, School of Health Sciences, University of Manchester, UK (e-mail:jinming.duan@manchester.ac.uk). Li Wu is with the MGI Tech Co., Ltd., China. Huazhu Fu is with the Institute of High Performance Computing (IHPC), Agency for Science, Technology and Research (A*STAR), Singapore. } 
% \thanks{Digital Object Identifier or DOI: 10.1109/TCSVT.2024.3502755}
}

% The paper headers
% \markboth{Journal of \LaTeX\ Class Files,~Vol.~14, No.~8, August~2021}%
% {Shell \MakeLowercase{\textit{et al.}}: A Sample Article Using IEEEtran.cls for IEEE Journals}
\markboth{}%
{HAO \MakeLowercase{\textit{et al.}}: Hierarchical Context Transformer for Multi-level Semantic Scene Understanding}

\IEEEpubid{\begin{minipage}{\textwidth}\ \centering
		Copyright \copyright 20xx IEEE. Personal use of this material is permitted. \\
		However, permission to use this material for any other purposes must be obtained 
		from the IEEE by sending an email to pubs-permissions@ieee.org.
\end{minipage}}
% \IEEEpubid{Copyright © 20xx IEEE. Personal use of this material is permitted. However, permission to use this material for any other purposes must be obtained from the IEEE by sending an email to pubs-permissions@ieee.org.}
% Remember, if you use this you must call \IEEEpubidadjcol in the second
% column for its text to clear the IEEEpubid mark.

\maketitle


\begin{abstract}
A comprehensive and explicit understanding of surgical scenes plays a vital role in developing context-aware computer-assisted systems in the operating theatre. However, few works provide systematical analysis 
%  across all the granularity (phase $\rightarrow$ step $\rightarrow$ action and instruments) 
to enable hierarchical surgical scene understanding. In this work, we propose to represent the tasks set [phase recognition $\rightarrow$ step recognition $\rightarrow$ action and instrument detection] as multi-level semantic scene understanding (MSSU). For this target, we propose a novel hierarchical context transformer (HCT) network and thoroughly explore the relations across the different level tasks. 
Specifically, a hierarchical relation aggregation module (HRAM) is designed to concurrently relate entries inside 
%exploit inter-relations from
multi-level interaction information and then augment task-specific features.  To further boost the representation learning of the different tasks, inter-task contrastive learning (ICL) is presented to guide the model to learn task-wise features via absorbing complementary information from other tasks. Furthermore, considering the computational costs of the transformer, we propose HCT+ to integrate the spatial and temporal adapter to access competitive performance on substantially fewer tunable parameters. 
Extensive experiments on our cataract dataset and a publicly available endoscopic PSI-AVA dataset demonstrate the outstanding performance of our method, consistently exceeding the state-of-the-art methods by a large margin.  The code is available at https://github.com/Aurora-hao/HCT.
%g the crosstask interactions for multi-task dense prediction. We propose a novel multi-task contrastive regularization method based on the consistency to effectively boost the representation learning of the different sub-tasks.
%with the cross-task consistency, 




%construct multi-task contrastive optimization objective to guide the model to learn more effective task-specific features via absorbing complementary information from other tasks

% These instructions provide guidelines for preparing papers for IEEE Transactions,
% but this version is specifically written to describe submission to IEEE TMI.
% Use this document as a template if you are using \LaTeX.
% Otherwise, use this document as an instruction set.
% The electronic file of your paper will be formatted further at IEEE.
% Paper titles should be written in uppercase and lowercase letters, not all uppercase.
% Avoid writing long formulas with subscripts in the title;
% short formulas that identify the elements are fine (e.g., "Nd--Fe--B").
% Keep the title short and do not write ``(Invited)'' in the title.
% Full names of authors are preferred in the author field, but are not required.
% Put a space between authors' initials. Only authors may appear in the author line
% of a manuscript. Authors are defined as individuals who have made an identifiable
% intellectual contribution to a manuscript to the extent that the individual can defend its contents.
% Define all symbols used in the abstract. Do not cite references in the abstract.
% Keep the abstract to 250 words or less.
\end{abstract}

\begin{IEEEkeywords}
 Multi-level semantic, surgical scene understanding, transformer, inter-task contrastive learning, spatial-temporal adapter
%Enter about five key words or phrases in alphabetical order, separated by commas.
\end{IEEEkeywords}

\section{Introduction}
\label{sec:introduction}
%------------figure----------------
\begin{figure}[thp]
\begin{center}
\includegraphics[width=\linewidth]{figures/Fig.1.pdf}
% \end{center}
\caption{Comparison of our proposed method with existing methods. MSSU: multi-level semantic scene understanding in surgery.}
\label{intro}
\end{center} 
% \vspace{-0.5cm}
\end{figure}
%---------------------------------- 
\IEEEPARstart{C}{Computer} assisted surgical  (CAS) systems now have a vital role in enhancing the quality of interventional healthcare \cite{vercauteren2019cai4cai}, surgeon training as well as assist procedure planning and retrospective analysis \cite{lalys2014surgical,hao2023act}. To achieve it, CAS  systems need to make use of a complete and explicit understanding of surgical scenes \cite{huaulme2021micro,lalys2014surgical, seenivasan2023surgicalgpt, wang2023truncate}.   Specifically, surgical phases and steps can monitor surgical processes and provide early alerts of potential forthcoming anomalies \cite{huaulme2020offline, zhang2023surgical}. Detailed instruments and actions appearing in the surgical scenes are beneficial to reducing preventable operation errors \cite{mascagni2022artificial, sun2023masked}, clinical decision support \cite{padoy2019machine}  and skill assessment \cite{liu2021towards, zhou2023hierarchical, li2024continual, yue2023perceptual}. 
%Furthermore, multi-level semantic surgical scene understanding is essential to developing CAS  systems.

A surgical procedure can be decomposed at different levels of granularity, such as the whole procedure, phases, steps, and actions \cite{lalys2014surgical}. 

Most of the currently available methods for surgical understanding are only focused on one specific level of granularity, like phase or step recognition at coarse granularity \cite{jin2021temporal,czempiel2020tecno,yi2022not,pan2023temporal,twinanda2016endonet,czempiel2021opera,ding2022exploring,yue2023cascade} or action recognition at fine granularity \cite{nwoye2022rendezvous,lin2022instrument, xi2022forest, wang2023vision, liu2024xfmp}, as shown in Fig. \ref{intro} (A), (B), (C) respectively. Nevertheless, a comprehensive surgical scene understanding plays a key role in developing CAS  systems, which needs to be started from a global overview of a description of high-level tasks to fine details \cite{lalys2014surgical}. Specifically, phases and steps as coarse parts provide general information about the ongoing surgical procedure, while fine-grained action and instrument detection provide a detailed perception and understanding of surgical scenes, as shown in Fig. \ref{intro} (D).
% A coarse decomposition with phases and steps provides general information about the ongoing surgical procedure. Alongside phases and steps, works on fine-grained action recognition in surgical videos are gaining momentum.  These works \cite{innocent2021rendezvous,lin2022instrument} %\cite{innocent2021rendezvous,lin2022instrument,islam2020learning,seenivasan2022} model actions as triplets of the instrument being used, its role (verb), and its target anatomy.  An action detection network \cite{hao2023act}  is proposed to achieve accurate localization and recognition of the actions.  Fine-grained action recognition provides a detailed perception and understanding of surgical scenes. Apparently
Semantic information from different levels can complement each other, forming a complete surgical portrait from various views.   However,  few works provide systematical analysis across all the levels of surgical interpretations, and those that do often fail to explore the relationships between different levels \cite{valderrama2022towards}. More critically, different levels of representation facilitate each other by  utilizing the containment information from these hierarchies.  
% More critically, the recognition of surgical phases and steps could facilitate the recognition of actions and instruments by utilizing the containment information from these hierarchies, and vice versa. 
For  \IEEEpubidadjcol example, in cataract surgery, as shown in Fig. \ref{intro} (D), the instrument ``incision knife" is often used to act ``cut corneal-limbus", and it generally appears in phase ``opening" and step ``incision". Similarly, step ``hydrodissetion" often appears during phase ``opening" and also indicates the occurrence of both instrument ``cannula" and action ``separate capsule". Stressing on their intrinsic complementarity, the relationship between multi-level activities is essential for understanding the scene dynamics \cite{valderrama2022towards, zhang2022unsupervised}.  Therefore,  we propose a hierarchical context network for multi-level semantic scene understanding (MSSU) (phase recognition $\rightarrow$ step recognition $\rightarrow$ action and instrument detection) and explore the relationship among them.

%targeted to hierarchical surgical interpretations
To fully utilize the cross-task knowledge, we design an inter-task contrastive learning algorithm to further boost our model performance. Specifically, contrastive learning can automatically
learn the semantic alignment relation across different tasks according to the actual data distribution \cite{lou2023min}. In this work, we treat the inter-task representations of the same scenes as positive, while the representations of other random scenes as negative. Based on this, the contrastive learning approach is capable of maximizing the mutual information between different views of the same scene, resulting in more compact and better feature representations.




Furthermore, vision transformers have shown outstanding capabilities in visual feature representation. 
% helps to capture cross-frame dependencies and preserve essential features in extra-long sequences, showing outstanding capabilities in visual feature representation. 
% Despite the outstanding performance of the transformer architecture, 
Nonetheless, the growing number of parameters and model size pose limitations on training from scratch and the extensive adoption of transformers, particularly when training video models in surgical scenes.
% particularly when full annotated data is limited, for instance in surgical scenes.  Furthermore, considering that training video models are drastically more expensive in both computing resources and time than image models, this problem becomes particularly more useful and valuable in practice \cite{pan2022st}. 
Parameter-efficient transfer learning is an excellent way to overcome this drawback  \cite{pan2022st,yang2023aim}.
% in natural language processing (NLP) \cite{zaken2021bitfit}, %\cite{hu2021lora,zaken2021bitfit}. \cite{brown2020language}
% whose goal is only to finetune a small number of parameters while keeping pre-trained language models frozen to attain strong performance.
%, which is also gradually applied to the field of vision \cite{pan2022st,yang2023aim}. 
%\cite{pan2022st,chen2022adaptformer,yang2023aim}. 
Besides, inferring spatial and temporal structured information is a crucial aspect in the context of surgical video understanding. 
%because pre-trained image models exhibit limitations in it. 
Therefore,  with the above observation, we propose a simple Spatial-Temporal Adapter  (ST-Ada) capable of learning the representations in space and time to achieve superior video understanding in surgical scenes at a small parameter cost.
% pre-trained image models lack the ability to infer temporal structured information, which however is critical in video understanding. 

In this paper,  we present a hierarchical context transformer (HCT) network to achieve MSSU in surgical videos. To realise this, we first design a hierarchical relation aggregation module (HRAM) to generate hierarchical representations for different tasks and further capture the relations between them. Then, after fusing the cross-task information guided by correlations, we obtain task-specific representations. Next, to enhance the consistency of inherently existing crossing the different tasks, we propose inter-task contrastive learning (ICL) to make the task-specific representation more discriminative. Considering the computation cost and model storage when transferring knowledge from a pre-trained model in transformer architectures, we present HCT+ network to add ST-Ada to the HCT to effectively learn spatial, and more importantly temporal representations with much fewer parameters for our multi-task learning at a significantly smaller computational cost. 




 
 In summary, the key contributions of this work are four-fold:
 
\begin{itemize} 
\item[$\bullet$]  We propose a novel hierarchical context transformer (HCT) network to realize the multi-level semantic scene understanding MSSU and explore inter-relations between different tasks fully. MSSU represents the recognition and detection of phase $\rightarrow$ step $\rightarrow$ action and instruments in surgery.
\item[$\bullet$] We devise a  hierarchical relation aggregation module (HRAM) to encode the phase-step-instrument-action hierarchical relationship. Furthermore, for any given task, HRAM possesses the capability to aggregate information contributed by the other three tasks and exploit it to enhance task-wise representation learning. 
 
\item[$\bullet$] We develop an inter-task contrastive learning (ICL) to further strengthen the context learning capability of the model by the inherent consistency across the tasks. Considering the high computational costs of the transformer model, we integrate a  Spatial-Temporal  Adapter (ST-Ada)  into our HCT denoted as HCT+, which attains strong transfer learning abilities in spatial and temporal representations at a small parameter cost.
% at a small parameter cost.
% by only fine-tuning a small number of parameters.
\item[$\bullet$] We conduct experiments with the proposed network on two datasets, our cataract video dataset and the public endoscopic PSI-AVA  dataset, to demonstrate its state-of-the-art (SOTA) performance and the contribution of each of our proposed components. The code is available at https://github.com/Aurora-hao/HCT.
\end{itemize} 





%------------figure----------------
\begin{figure*}[th]
\begin{center}
\includegraphics[width=\linewidth]{figures/Fig.2.pdf}
% \end{center}
\caption{The pipeline of our hierarchical context transformer (HCT) framework. Given an input video with $l+1$ frames, HCT uses a transformer model to extract shared features, which are then fed into the hierarchical relation aggregation module (HRAM) to capture the relations between the four task-wise features. After that, inter-task contractive learning (ICL) is utilized to further optimize the HCT. For the plus version of the model HCT+, we add a temporal adapter before HRAM and put the spatial adapter in the feed-forward module.
}
\label{pipeline1}
\end{center} 
% \vspace{-0.5cm}
\end{figure*}
%---------------------------------- 
% Top-down approaches are defined as analyses that start from a global overview of the intervention using patient-specific information and a description of high-level tasks (such as phases or steps) to fine-coarse details (such as activities or motions).
% multi-stage temporal convolutional network
\section{Related work}



\subsection{Surgical Scene Understanding} 
Recent works in surgical scene understanding can be broadly classified into two categories. One category aims to automatically identify surgical phases or steps within untrimmed surgical videos, which has received a lot of attention and is a very active area of research. TMRNet \cite{jin2021temporal} is proposed to establish a long-range memory bank to provide long-term temporal features and improve phase recognition accuracy. 
% Inspired by the excellent outcomes of  MS-TCN  \cite{li2020ms} in general human action segmentation. 
Yi et al.  \cite{yi2022not} proposed a non-end-to-end training strategy based on MS-TCN. After the transformer gained significant popularity for processing sequential data, Pan et al.  \cite{pan2023temporal} exploited the Swin transformer and LSTM networks 
% to extract multi-scale visual features and temporal information 
to recognise phases in surgical videos. Czempiel et al. devised OperA \cite{czempiel2021opera}, a transformer-based model, to extract high-quality frames with an attention regularization loss.  Ding et al. \cite{ding2022exploring} presented SAHC for surgical phase recognition by emphasizing learning segment-level semantics to address erroneous predictions caused by ambiguous frames. Yue et al. \cite{yue2023cascade} fused the frame-level and phase-level temporal context for surgical workflow analysis. 

Another category focuses on finer-grained action recognition. Bawa et al. \cite{bawa2021saras} presented an endoscopic minimally invasive surgery dataset designed to tackle the problem of surgeon action detection, and provided baseline results with open-source methods like \cite{ren2015faster}. Hao et al. \cite{hao2023act}  presented ACTNet to get action locations, classes and confidence guided by anchors. Unlike existing methods for a single task, we design a new transformer module to learn the relations from multi-level tasks. Besides, we also model the constraints in the hierarchical information to regularize the framework with the ICL comprehensively.
% Automatic surgical workflow analysis aims to automatically identify surgical phases or steps within untrimmed surgical videos, which has received a lot of attention and is a very active area of research.  Several CNN-based models are proposed for the tasks.  Jin et al. \cite{jin2017sv} proposed SV-RCNet, which integrates ResNet and LSTM to extract spatial-temporal features for automatic phase recognition.  To address the limited memory capacity of LSTM, TMRNet \cite{jin2021temporal} is proposed to establish a long-range memory bank to provide long-term temporal features and improve phase recognition accuracy.  Inspired by the excellent outcomes of  MS-TCN  \cite{li2020ms} in general human action segmentation. Czempiel et al. \cite{czempiel2020tecno}  developed a TeCNO network, which achieves phase recognition by introducing MS-TCN.  Similarly, Yi et al.  \cite{yi2022not} proposed a non-end-to-end training strategy based on MS-TCN.  For improving performance,  Ramesh et al. proposed a multi-task MS-TCN to jointly predict the phases and steps. Jin et al. \cite{jin2020multi} showed consistent improvement in both phase recognition and tool recognition tasks with multi-task learning.  However,  due to the inherent inductive biases in the convolutional structure, these works do not work well in understanding global information through cross-frame dependencies \cite{pan2023temporal}, which is critical visual information for surgical scene understanding.

% \subsection{Transformer-based Surgical Workflow Analysis}
% Transformer has gained significant popularity for processing sequential data, which allows concurrently relating entries inside a sequence and helps to preserve essential features in extra-long sequences \cite{pan2023temporal}. Pan et al.  \cite{pan2023temporal} exploited the Swin transformer and LSTM networks to extract multi-scale visual features and temporal information to recognise phases in surgical videos. Czempiel et al. devised OperA \cite{czempiel2021opera}, a transformer-based model, to extract high-quality frames with an attention regularization loss.  Ding et al. \cite{ding2022exploring} presented SAHC for surgical phase recognition by emphasizing learning segment-level semantics to address erroneous predictions caused by ambiguous frames. Yue et al. \cite{yue2023cascade} fused the frame-level and phase-level temporal context for surgical workflow analysis. Unlike existing transformer-based workflow analysis methods for a single task, we design a new transformer module to learn the relations from multi-granularity tasks. Besides, we also model the constraints in the hierarchical information to regularize the framework with the ICL comprehensively.

%a segment-attentive hierarchical consistency network (SAHC)


% take a different strategy that utilizes the GCN to learn the relations of categories in each annotation type. Besides, we also model the constraints in hierarchical annotations to comprehensively regularize the framework with both inter- and intra-relations, contributing a new approach to embedding relations in network architectures.
\subsection{Multi-task learning in surgical scene}
For improving performance,  Ramesh et al. \cite{ramesh2021multi} proposed a multi-task MS-TCN to jointly predict the phases and steps, failing to get a complete surgical scene understanding.
% Valderrama et al. \cite{valderrama2022towards} provides a benchmark transformer network for multiple tasks but fails to explore and utilize the relations between different tasks.  
Nwoye et al. proposed the action triplets (instrument, verb, target, as shown in Fig. \ref{intro} (C)) to represent the actions in a laparoscopic dataset \cite{nwoye2020recognition}, and a trainable 3D interaction space to capture the associations between the triplet components. They further developed a new model, the Rendezvous \cite{nwoye2022rendezvous}, to detect triplet components with class activation-guided and mixed attentions. After adding the location information of the instrument and target, Lin et al. \cite{lin2022instrument, lin2024instrument} presented a novel QDNet for the instrument-tissue interaction quintuple detection task in cataract videos. Valderrama et al. \cite{valderrama2022towards} introduced the PSI-AVA dataset and proposed a baseline framework for different level task achievement. However, they failed to harness and explore the interdependencies and correlations among these tasks. Although most of these works employ multi-task learning for action recognition, they are confined to the singular horizontal dimension of actions for more detailed analyses, while neglecting the complementarity of vertical information such as phases and steps.  Our work, in contrast, encompasses an integrated analysis across both horizontal and vertical dimensions in surgical scenarios.






\section{Methodology}
\subsection{Overview} 
The overall framework of HCT illustrated in Fig. \ref{pipeline1}, for achieving multi-level semantic activity recognition in surgical scenes, from phase and step recognition to finer-grained action and instrument/tool detection denoted by $p$, $s$, $a$ and $t$ respectively, consists of only transformer blocks. Specifically, to generate the feature of frame $x_t$, we extract a video clip as the model input, which contains the current frame $x_t$ and a set of its previous frames, as $x={x_{t-l},...,x_{t-1}, x_t}$, with $l+1$ frames. 
The clip is first passed forward a transformer backbone to generate shared feature maps $f_p$, $f_s$ and $f_a$. $f_t$ is instrument-specific features. For obtaining video features effectively to model the complex temporal cues of the surgical scenes, we utilize the MViTv2  \cite{li2022mvitv2} model as the backbone. 
It allows capturing simple low-level visual information at early layers and spatially coarse, but complex and high-dimensional features at deeper layers, benefiting from its multi-scale feature hierarchies.
%which indicate the spatial resolution decreases from one stage to another while the channel dimension expands. 

After that, an HRAM is proposed to aggregate different tasks to dig out the dependencies among them and further get the task-specific features. Specifically, we utilize cross-attention in transformer blocks to get the relations $T_{i \leftrightarrow j}$ ($i,j \in [p,s, a,t]$ and $i\neq j$) cross the different task $i$ and $j$. Then the relations are fused with corresponding self-attention task-wise features to get $\dot{f_p}$, $\dot{f_s}$ and $\dot{f_a}$. To better preserve the information of each task, we add skip connections as illustrated in Fig. \ref{pipeline1} to get $\ddot{f_p}$, $\ddot{f_s}$ and $\ddot{f_a}$. Then, to acquire a more compact and better feature representation for different tasks, we develop the ICL approach to maximize the mutual information between different views of the same scenes.   In this work,  ${p \leftrightarrow s}$ and ${a \leftrightarrow t}$ are selected as two contrastive pairs because phase and step have strong consistencies for the same positive sample, as do action and instrument, which is also validated by our experiments. The HCT is also optimised by the supervision losses for each task  $\mathcal L_{p}$,  $\mathcal L_{s}$, $\mathcal L_{a}$ and  $\mathcal L_{t}$.

Furthermore, the extensive parameter size of the transformer model imposes significant limitations on its utility, particularly in the medical scenes. With this in mind, this paper introduces the temporal adapter (T-Ada in Fig. \ref{pipeline1}) placed before the HRAM and spatial adapter placed in the feed-forward network  (S-AdaFFN in Fig. \ref{pipeline1}) to construct  HCT+. It attains strong transfer learning abilities by only fine-tuning a small number of extra parameters while getting competitive performance. 
%at a significantly smaller computational cost
 %------------figure----------------
\begin{figure*}[th]
\begin{center}
\includegraphics[width=0.93\linewidth]{figures/Fig.3.pdf}
\caption{Detailed structure of the hierarchical relation aggregation module (HRAM).}
\label{pipeline2}
\end{center}
% \vspace{-0.5cm}
\end{figure*}
%----------------------------------
\subsection{Hierarchical Relation Aggregation Module (HRAM)}
Transformer has strength in the computation of long-range dependencies and the general representation ability for various tasks \cite{pan2023temporal, ding2022exploring}. This advantage is essential since the single branches start from the shared features. In this work, we achieve the shared encoder built based on a stack of transformer blocks from MViTv2. After getting shared features, in order to exploit the hierarchical relationship in the surgery, HRAM is developed to generate task-specific features with complementary information inside.

As illustrated in Fig. \ref{pipeline2}, suppose there are  $n$ tasks. The input features of the HRAM are denoted as $f_i$  and $f_j$, where $i,j \in \{1,..., n\}$ and $i\neq j$. Without loss of generality, considering here task $i$ as our primary task, and task $j$ as the secondary, we seek to estimate the features $f_{j \rightarrow i}$ from task $j$ that can contribute to task $i$.  We employ two parts of attention: (i) correlation attention, which aims to exploit cross-task spatial-temporal correlations to guide the extraction of contributing features of the secondary task to the primary task; (ii) a self-attention on the primary task, to retain and further enhance information from the primary task.

Concretely,  in correlation attention part,  for $f_i, f_j\in \mathbb{R}^{L \times C}$ ($L = l \times h\times m$, $l$ denotes length of input video clip, $h$ and $m$ are the height and width of feature map respectively), it applies pooling operations to perform spatial-temporal downsampling with $s_1$ and $s_2$  scale factor respectively,  to get query $f_q$, key $f_k$ and value $f_v$ tensors \cite{li2022mvitv2}.  
\begin{equation}
\begin{split}
\nonumber
&f_q=P_q(f_iW_q;s_1) \quad f_k=P_k(f_jW_k;s_2) \quad f_v=P_v(f_jW_v;s_2),
\label{eq1}
\end{split}
\end{equation}
where $f_q \in \mathbb{R}^{L_1  \times C}$  and $f_k, f_v \in \mathbb{R}^{L_2\times C}$.  $W_q, W_k, W_v$ represent the projection matrices. Spatial-temporal correlation attention matrix $C_{j \rightarrow i}\in \mathbb{R}^{L_1 \times C}$ is then obtained :
\begin{equation}
\begin{split}
\nonumber
&C_{j \rightarrow i} = softmax(\frac{f_qf_k^{T}}{\sqrt{d}})f_v,
\label{eq1}
\end{split}
\end{equation}
where $\sqrt{d}$ is used to normalize the inner product matrix row-wise. Intuitively, $C_{j \rightarrow i}$ has high values where features from task $i$ and $j$ are highly correlated and low values otherwise.  For the self-attention part on primary task $i$, multi-head pooling attention (MHPA) \cite{fan2021multiscale} is employed to enable progressive change at the spatial-temporal resolution in the transformer backbone. 

Subsequently,  $n$  mapping functions are utilized to adjust the output dimension and fuse different parts of attention,  the integrated feature $T_{j \rightarrow i}$ for task $i$,  enhanced by the correlation between task $i$ and $j$, is defined as: 
\begin{equation}
\begin{split}
\nonumber
&T_{j \rightarrow i} =  concat \left( MLP_j(C_{j \rightarrow i}), MLP_i({\rm MHPA}(f_i))\right),
\label{eq1}
\end{split}
\end{equation}
%\sum_{i=0}^n
where  $MLP_j$  is designed to transfer the size of  $C_{j \rightarrow i}$  to $ L_1 \times \frac{C}{n-1}$, while $MLP_i$  is designed to maintain the dimension of feature $f_i$.  $concat$  indicates the channel-wise concatenation. 

After that,  the concatenated feature map is then processed with the element-wise addition with the self-attention information of $f_i$,  followed by another mapping function $MLP_{ij}$, leading to the final refined features. Consequently, for aggregating all correlated secondary task information contributing towards  primary task $i$, we can get final refined task-specific features $F_{all \rightarrow i}$:
\begin{equation}
\begin{split}
\nonumber
&T_{all \rightarrow i} = MLP_{ij}\left(concat_{j=1}^n \left( T_{j \rightarrow i}\right) + {\rm MHPA}(f_i)\right),\\
&F_{all \rightarrow i} = T_{all \rightarrow i} + slicing(f_q),
  % MLP_ij(T_{j \rightarrow i}) concat \left( MLP_j(T_{j \rightarrow i})+MLP_i(MHPA(f_i))\right).
\label{eq1}
\end{split}
\end{equation}
where  $slicing(\cdot)$  indicates channel slicing for achieving skip connection. The slicing operation shown in Fig. 3 is optional, and is particularly useful in instrument-specific feature computation. It ensures dimensional consistency and filters out irrelevant information, which is essential for accurate processing in the task. $MLP_{ij}$  is employed to keep the input $f_i$ and output  dimensions consistent. By design, the refined primary task features have the same dimensions as the input features.

In this paper, we intend to achieve four tasks to attain a hierarchical understanding of surgical scenes. For the tasks of phase, step recognition and action detection, we fuse relevant information from the other three tasks and get final refined task-specific features $F_{all \rightarrow i}$. To account for instrument detection, we employ the DINO \cite{zhang2023dino} to obtain bounding box estimations and box-specific features, then feed them into the HRAM module. Therefore, when we use instrument-specific features as auxiliary roles,  the architecture consisting of two linear layers and an activation layer in the middle and padding operation is proposed to keep the same dimension as the primary task features to operate following attention calculation. What is more, as shown in Fig.  \ref{pipeline1}, to achieve the action detection task, we integrate action-specific features and instrument-specific features to obtain final results.
% As shown in Fig.  \ref{pipeline1}, to optimize the instrument detection task, we integrate action-specific features and instrument-specific features to obtain final results.

% $FFN(\cdot)$  represents the feed forward module

\subsection{Inter-task Contrastive Learning (ICL)}
As the different tasks are learned from the same input video clip data, consistency inherently exists crossing the different tasks in a multi-task prediction. The model performance of one task can thus benefit from utilizing the consistent information from other tasks to boost task-specific representation learning. To better illustrate such consistency, we propose ICL bring their representations closer together, which brings task pairs (positive examples) near together while pushing irrelevant ones (negative examples) far apart. As shown in Fig. \ref{pipeline3}, the distances of positive pairs are consistently smaller than those of negative ones on different target tasks. 

Specifically, with input video clips, we treat the representations from two different tasks in the same scene (same video clips) as positive pairs, while the representations in different scenes (different video clips) as negative samples. For $n$ tasks in our paper, we take two of them as examples denoted by $i$ and $j$ as task pairs for contrastive learning. We set the feature obtained from the HRAM as the input of ICL. As illustrated in Fig. \ref{pipeline3}, for the features $F_{all \rightarrow i}$ and  $F_{all \rightarrow j}$,  after the feed-forward process in the transformer block, we first average them in terms of the space-time dimension by adaptive average pooling, followed by a set of mapping function and $L_2$ normalization. Then, we get the reshaped features $F_{ci}$  and $F_{cj}$. Given a positive example of task pair, we randomly pick a set of $D$ $\{F_{cj,d}\}_{i=1}^{D}$  for $F_{ci}$  from the same batch as negative examples and vice versa. Therefore, the losses for tasks $i$ and $j$ are as follows:
\begin{equation}
\begin{split}
\nonumber
&\mathcal L_{ci} = -\sum_{F_{ci},F_{cj}}log{\frac{exp(sim(F_{ci},F_{cj})/\tau}{\sum_{d\in \Lambda}exp(sim(F_{ci},F_{cj,d})/\tau)}},\\
&\mathcal L_{cj} = -\sum_{F_{ci},F_{cj}}log{\frac{exp(sim(F_{ci},F_{cj})/\tau}{\sum_{d\in 
 \Lambda}exp(sim(F_{ci,d},F_{cj})/\tau)}},\\
&\mathcal L_{cij} = \mathcal L_{ci}+\mathcal L_{cj},
\label{eq1}
\end{split}
\end{equation}
where $\Lambda$ represents a batch of video clips, and $\tau$ is the temperature parameter that controls the strength to contrast. $sim$ is defined as $sim(x,y)=\frac{x^Ty}{\vert x \vert \vert y \vert}$ . In the training process, negative examples are from the same training batch of data. $\mathcal L_{cij}$  denotes ICL loss between task $i$ and task $j$. In experiments, phases and steps are selected to form pairs and fed into ICL due to their strong inherent consistency in the same positive samples. This holds true for pairs composed of actions and instruments as well, which is validated by subsequent ablation studies.
% between task-specific representations for knowledge transfer across tasks
%, and the distance distributions are also similar
% Besides expert sharing, we further learn the alignment relation between task-specific representations for knowledge transfer across  tasks 
 %------------figure----------------
\begin{figure}[tb]
\begin{center}
\includegraphics[width=\linewidth]{figures/Fig.4.pdf}
\caption{Detailed structure of the inter-task contrastive learning (ICL).}
\label{pipeline3}
\end{center}
% \vspace{-0.5cm}
\end{figure}
%----------------------------------
 %------------figure----------------
\begin{figure}[bhp]
\begin{center}
\includegraphics[width=0.9\linewidth]{figures/Fig.5.pdf}

\caption{Detailed structure of the spatial-temporal adapter in our proposed transformer block.}
\label{pipeline4}
\end{center}
% \vspace{-0.5cm}
\end{figure}
%----------------------------------
% the growing number of parameters and model size pose limitations on training from scratch and the extensive adoption of transformers, particularly when training video models in surgical scenes.

\subsection{HCT+ network}
Transfer learning is an efficient way to transfer knowledge from one domain to another to improve its performance. However, limited by the growing number of parameters and model size, 
% the growth of transformer model sizes and minor training data, 
fully fine-tuning the whole transformer model for every single downstream task would become prohibitively expensive and infeasible in training cost and model storage, especially for surgical video datasets \cite{pan2022st}.  In this work, we integrate a spatial-temporal adapter into our improved transformer block, which attains strong transfer learning abilities by only fine-tuning a small number of extra parameters while achieving comparable performance with the full fine-tuned model. %or even better

Video understanding requires the model to learn both good appearance representations in each frame (spatial modeling) and also infer the temporal structured information across frames (temporal modeling). Hence, temporal and spatial adapters are employed in different places of the transformer block to model disparate information.  Inspired by the Adapter 
\cite{houlsby2019parameter} designed for parameter-efficient transfer learning in NLP,  we adopt spatial and temporal adapters \cite{pan2022st,yang2023aim}. 

As shown in  Fig. \ref{pipeline4}, before the transformer block, the input video clip features are firstly fed into the temporal adapter denoted by T-Ada, which is a bottleneck architecture that consists of two fully connected (FC) layers and a depth-wise 3D-convolution (DW-Conv3D) in the middle \cite{pan2022st}. The first FC layer projects the input to a lower dimension and the second FC layer projects it back to the original dimension. Formally, the temporal adapter $T\_Ada$ can be expressed as:
\begin{equation}
\begin{split}
\nonumber
&T\_Ada(f_i) = f_i+\mathbf{W_{up}}\left(DW\_Conv3D(\mathbf{W_{down}}f_i)\right),
\label{eq1}
\end{split}
\end{equation}
where the $\mathbf{W_{up}}\in \mathbb{R}^{L \times \hat{L}}$ and $\mathbf{W_{down}}\in \mathbb{R}^{\hat{L} \times L}$ denote the parameters of the up-projection layer and the down-projection layer, where $\hat{L}$ is the bottleneck middle dimension and satisfies $\hat{L}\textless \textless L$. 

In contrast to the temporal adapter, the spatial adapter employs an intermediate activation layer GELU instead of DW-Conv3D.  With the output features $F_{all \rightarrow i}$ of  transformer block as input, the spatial adapter operator $S\_Ada$ is formally written as:
\begin{equation}
\begin{split}
\nonumber
&S\_Ada(F_{all \rightarrow i}) = F_{all \rightarrow i}+\mathbf{W_{up}}\left(GELU(\mathbf{W_{down}}F_{all \rightarrow i})\right).
\label{eq1}
\end{split}
\end{equation}

For proper adaptation, we add the spatial adapter to the feed-forward module to transfer spatial information, denoted by S-Ada in Fig. \ref{pipeline4}. During training, all the other layers of the transformer model are frozen while only the spatial and temporal adapters are updated \cite{houlsby2019parameter}.
%cite: Parameter-efficient transfer learning for nlp

 
\subsection{Overall Training Objective }
In this work, we employ a weighted binary cross-entropy loss for action detection while a  weighted cross-entropy loss is utilized for the phase, step and instrument recognition. Furthermore, the ICL losses for task pairs are added to enhance the task-specific feature extraction. The final training objectives can be written as:
\begin{equation}
\begin{split}
\nonumber
&\mathcal L_f = \sum _{i=1}^n\lambda _i \mathcal L_i +\sum _{i,j=1,i\neq j }^n \mathcal L_{cij},
\label{eq1}
\end{split}
\end{equation}
 where $\lambda _i$, $\mathcal L_i$ denote the weights and loss functions for different tasks, and $n$ is the number of tasks.



\begin{table*}[tb]
    \centering
\caption{ Performance Comparison on the cataract and PSI-AVA dataset. The best results are highlighted in bold. ``Frames" denotes the number od frames used in models. }
\label{tabel1}
\setlength{\tabcolsep}{1.5mm}{
    \begin{tabular}{cccccccc|cccccc} 
    \toprule
        \multirow{3}*{Method}& \multirow{3}*{Frames} &\multicolumn{6}{c|}{Cataract dataset}  &\multicolumn{6}{c}{PSI-AVA dataset}\\
        \cline{3-8} \cline{9-14}
          & &   \multicolumn{2}{c}{Phase}&   \multicolumn{2}{c}{Step}&   Instrument& Action&  \multicolumn{2}{c}{Phase}&  \multicolumn{2}{c}{Step}&   Instrument& Action\\ 
          \cline{3-8} \cline{9-14}
          & &   mAP&  Acc&    mAP&  Acc&  mAP& mAP&  mAP&  Acc&    mAP&  Acc&  mAP&  mAP\\  \hline 
         TMRNet \cite{jin2021temporal}& 30& 0.9496  & 0.9358 &   0.9375& 0.9230  &  —& —  &0.5439 &0.5734 &0.4061&0.5357 & —& —\\
         NETE \cite{yi2022not}& whole video & 0.9312& 0.9356 & 0.8335 & 0.8921  &  —& — &  0.4806  & 0.6403  & 0.2956& 0.5116 & —& — \\  
 SAHC \cite{ding2022exploring}& whole video & 0.9358& 0.9289 & 0.8982& 0.9109 & —&—  & 0.5522 & 0.6436& 0.3798& 0.5272 &—\\ \hline\hline
 % & & & &\\  
         ACTNet \cite{hao2023act}& 16&  —&  —& —& —&  —& 0.3970  &  —&  —& —&  —& —& 0.2259\\ \hline\hline 
         SlowFast \cite{feichtenhofer2019slowfast} &  16& 0.9075&0.9153 & 0.8486&0.8983 & 0.8525& 0.4745 &  0.4680& 0.5463 & 0.3386& 0.4713 & 0.8029& 0.1857\\ 
         TAPIR \cite{valderrama2022towards}& 16& 0.9286&0.9472 &0.9191& 0.9528 & 0.8541& 0.5251 &  0.5824& 0.5996 & 0.4561 & 0.5228 & 0.8082& 0.2356\\ 
         MViTv2 \cite{li2022mvitv2}&  16& 0.9494&\textbf{0.9557} &0.9289&0.9329 & 0.8543& 0.5563 &  0.6189&  0.6361& 0.4837 & 0.5548  &  0.8206&0.2729\\ \hline
 HCT& 16& \textbf{0.9635}& 0.9468 &  \textbf{0.9581}& \textbf{0.9556} &  \textbf{0.8545} &\textbf{0.5726} & \textbf{0.6457}& \textbf{0.6603} &  \textbf{0.4977}& \textbf{0.5766} & \textbf{0.8217}&\textbf{0.2820 }\\ 
    \bottomrule   
    \end{tabular}}
    
    
\end{table*}



\section{Experiments}
\subsection{Datasets}
The effectiveness of our HCT is evaluated on two datasets,   our cataract dataset and public PSI-AVA \cite{valderrama2022towards}, which are derived from distinct surgical scenes. 
Our private cataract dataset comprises 20 videos (a total of 17511 frames) with a frame rate of 1 fps and resolution of 720 $\times$ 576. The dataset is provided with manual annotations done by surgeons indicating the surgical phase and step each video frame belongs to, and the actions and tools appearing in the scene. Specifically, the cataract videos consist of 4 phases, 10 steps,  49 actions and 13 instruments. For each video frame, the phase and step are provided with class labels, while the actions and instruments are labelled with the classes and locations (with bounding boxes). The cataract video dataset is randomly split into a training set with 15 videos (13583 frames) and a testing set with 5 videos (3928 frames). 
 
PSI-AVA dataset \cite{valderrama2022towards} includes eight radical prostatectomy surgeries, performed with the Da Vinci SI3000 Surgical System.  It includes 11 phases, 21 steps,  16 actions and 7 instruments, and  resolution is  1280 $\times$ 800. The phase and step annotations are provided by frames at 1 fps, while the actions and instruments are given by frames every 35 seconds of video.  Thus, 73,618 keyframes are annotated with one phase-step pair, while  2238 keyframes have instruments and actions annotations. We split the dataset following work \cite{valderrama2022towards}.



\subsection{Implementation Details}
The proposed architecture is built with Pytorch and is trained on NVIDIA RTX A6000. MViTv2 \cite{li2022mvitv2} pretrained on Kinetics-400 \cite{carreira2017quo} is adapted as the backbone. We use AdamW optimizer with a base learning rate of  $1e-3$, and  A cosine  scheduler with a weight decay of 0.05.
% We use AdamW optimizer with a base learning rate of  $1e-3$ using a weight decay of 0.05. A cosine learning rate scheduler is adopted with 10 epochs of linear warm-up with an initial learning rate of $1e-6$.
The model is totally trained for 60 epochs with a 10 epoch warm-up period. Data augmentation with 224 $\times$ 224 cropping, jittering and flipping is performed to enlarge the training dataset. The sequence length is 16 frames as \cite{li2022mvitv2}. The weights of the losses for phase, step, instrument and action tasks are set as 0.3, 0.2, 0.3 and 0.2 respectively. The batch size for both the baseline and the proposed method is consistent, set to 20. The temperature parameter in the ICL module is set to 0.07. The settings are the same for the two datasets. 
For getting the box-specific features, we pretrain DINO-5scale \cite{zhang2023dino}  
% with swim-L \cite{liu2021swin} backbone 
on training data for our cataract datasets. For PSI-AVA dataset, due to the limited annotation of instruments, we conduct pretraining based on the EndoVis 2017 \cite{allan20192017} and EndoVis 2018 dataset \cite{allan20202018}  following up \cite{valderrama2022towards}, and then train on PSI-AVA.  Bounding boxes are selected with a confidence threshold of 0.75, and corresponding 256-dimensional feature embeddings are extracted as box-specific features.  To assess performance in the phase and step recognitions, we use the mean Average Precision (mAP) and Accuracy (Acc), the standard metrics in videos.  For instrument and action detection, we use the mAP metric at an Intersection-over-Union threshold of 0.5 (mAP@0.5IoU). 
% \multicolumn{2}{c|}{Modules} & \multirow{2}*{Phase}&  \multirow{2}*{Step}&  \multirow{2}*{Instrument}& \multirow{2}*{Action} \\
%         \cline{1-2}
%          HRAM& ICL&    \multicolumn{4}{c}{} \\ \hline

%%%%这个表格没有两个方法的结果,还是有点奇怪,我如果是reviewer我肯定是会问这个的,质疑你这个plus版本的意义是什么?
\begin{table}[bp]
\centering
\caption{Ablation study on key Components based on the two datasets. }
\label{tabel3}
\setlength{\tabcolsep}{2.3mm}{
    \begin{tabular}{cc|cccc}
        \toprule
        \multicolumn{2}{c|}{Modules} & \multirow{2}*{Phase}&  \multirow{2}*{Step}&  \multirow{2}*{Instrument}& \multirow{2}*{Action} \\
        \cline{1-2}
         HRAM& ICL&    \multicolumn{4}{c}{} \\ \hline
        \multicolumn{6}{c}{Cataract dataset} \\ 
        \cline{1-6}
         &  &    0.9494 &0.9289 & 0.8543& 0.5563\\
        \checkmark&  &    0.9523&  0.9489&  \textbf{0.8547}&   0.5682\\
        \checkmark &  \checkmark&    \textbf{0.9635}&  \textbf{0.9581}&  0.8545&   \textbf{0.5726}\\\hline
         % \checkmark&  \checkmark&  \checkmark&  &  &  &   \\ 
         \multicolumn{6}{c}{PSI-AVA dataset} \\ 
        \cline{1-6}
         &  &    0.6189&  0.4837& 0.8206& 0.2729\\
         \checkmark&  &    0.6251&  0.4914&  0.8213&   0.2794\\
         \checkmark&  \checkmark&   \textbf{0.6457}&  \textbf{0.4977}&  \textbf{0.8217}&   \textbf{0.2820}\\ \bottomrule
         % \checkmark&  \checkmark&  \checkmark&  &  &  &   \\
    \end{tabular}}
\end{table}
%The visualisation shows that our method can produce prediction results very close to the groundtruth.





\subsection{Comparison with State-of-the-arts and Visualisation}
\subsubsection{Comparison with State-of-the-arts}
Table \ref{tabel1} presents a performance comparison between HCT and the SOTA methods on our cataract dataset and the public PSI-AVA dataset. The best results are highlighted in bold. 
% The experimental results demonstrate the superiority of our proposed method. 
% In this work, we employ the current SOTA object detection methods DINO \cite{zhang2023dino}  for instrument detection. 
Given that instrument detection is comparatively simpler and the current method has achieved high accuracy, our analysis is primarily focused on the other three tasks. 
% The subsequent analysis of experimental results is chiefly directed towards phase and step recognition, as well as action detection.

To validate the performance of our model, we carry on comparisons between single-task models TMRNet \cite{jin2021temporal}, NETE \cite{yi2022not} and SAHC \cite{ding2022exploring} in phase and step recognition, as well as action detection ACTNet\cite{hao2023act}. Additionally, we perform multi-task performance comparisons based on various video recognition methods  SlowFast \cite{feichtenhofer2019slowfast},   TAPIR\cite{valderrama2022towards} and  MViTv2 \cite{li2022mvitv2} by adding the same multi-task losses and heads. 

% All three models TMRNet \cite{jin2021temporal}, NETE \cite{yi2022not} and SAHC \cite{ding2022exploring} focus on phase recognition, which are also employed to conduct step recognition for comparison in this paper.
% To facilitate a comparison of phase and step tasks based on our dataset, we utilize existing methods to conduct separate recognition experiments for phase and step. 
% TMRNet \cite{jin2021temporal} introduces a memory bank to capture long-sequence temporal information, and even features from frames spanning even 30 seconds are utilized in our comparison.
% In our experiments, we utilize features from frames spanning even 30 seconds to obtain temporal dependencies. 
 % NETE \cite{yi2022not} separates end-to-end training into an initial prediction step and a refinement stage.
 % allowing the model to converge and optimize more effectively. 
 % Based on NETE \cite{yi2022not}, SAHC \cite{ding2022exploring} is recently proposed to learn high-level segments from surgical videos, which are then used to correct ambiguities caused by low-level frame errors. 
Using the cataract dataset, TMRNet achieves the best performance among the three existing methods, indicating that longer video frames (30 frames) significantly enhance phase and step recognition. However, compared to TMRNet, our method still achieves a 1.39\%  (96.35\% vs. 94.96\%) improvement in phase recognition and a 2.06\% (95.81\% vs. 93.75\%) improvement in step tasks on mAP, even though our method only uses a sequence of 16 frames, while NETE \cite{yi2022not} and SAHC \cite{ding2022exploring} use whole video features in models. Furthermore, all three methods require at least two training stages on the training dataset, whereas our approach can achieve end-to-end recognition for both phase and step. Our model on action detection, compared with  ACTNet, an anchor-context action detection network, also achieves a significant improvement in terms of mAP.

For the multi-task effect comparison, we adopt three popular SOTA methods in action recognition, encompassing both CNN-based and transformer-based approaches. By replacing the original action recognition heads of these methods with our multi-task heads, we facilitate a comparison of multi-task effects. To ensure fairness, neither the three methods nor our approach utilizes pre-trained models. The results demonstrate that our method outperforms others in every task. Specifically, in the tasks of phase, step and action, we exceed the performance of the best comparison method MViTv2 by 1.41\% (96.35\% vs. 94.94\%), 2.92\% (95.81\% vs. 92.89\%) and 1.63\% (57.26\% vs. 55.63\%) respectively, highlighting the strong synergistic effect among these four tasks. Regarding the lower phase recognition accuracy on the cataract dataset compared to the baseline, we calculate the Balanced Accuracy (B-Acc) for both methods (MViTv2 92.84\% vs. HCT 94.54\%), as shown in the supplementary materials. Our method shows a 0.89\% lower Acc but a 1.7\% higher B-Acc than MViTv2. This indicates that while MViTv2 performs better on more frequent categories, our approach achieves a more balanced performance across all categories. Notably, it excels in more challenging categories with fewer samples, making it more suitable for practical applications.

\begin{table}[bp]
    \centering
\caption{Ablation study on HRAM. }
\label{tabel4}
\setlength{\tabcolsep}{2.2mm}{
    \begin{tabular}{cccc|cccc}
        \toprule
        \multicolumn{4}{c|}{Items} & \multirow{2}*{Phase}&  \multirow{2}*{Step}&  \multirow{2}*{Instrument}& \multirow{2}*{Action} \\
        \cline{1-4}
         P& S&  I&  A&  \multicolumn{4}{c}{} \\\hline
         &  \checkmark&  \checkmark&  \checkmark&  0.9153&  0.9235&  0.8544& 0.5497\\
         \checkmark&  &  \checkmark&  \checkmark&  0.9337&  0.8739&  0.8545& 0.5561\\
 \checkmark& \checkmark& & \checkmark& 0.9045& 0.8827& 0.8534&0.5423\\
          \checkmark&  \checkmark&  \checkmark&  &  0.9412&  0.9046&  0.8515& 0.5471\\
         \checkmark&  \checkmark&  \checkmark&  \checkmark&  
\textbf{0.9523}&  \textbf{0.9489}&  \textbf{0.8547}& 
\textbf{0.5682}\\\bottomrule
    \end{tabular}}
\end{table}

Regarding the PSI-AVA dataset, we conduct similar comparisons, as shown in Table \ref{tabel1}. 
% Our method significantly surpasses existing single-task models in phase and step recognition, as well as in action detection. 
It's important to note that the PSI-AVA dataset, compared to our cataract dataset, has considerable differences in annotations for each task. Unlike our cataract dataset, where every frame is annotated with labels or bounding boxes for phase, step, instrument, and action at 1fps, the PSI-AVA dataset annotates only phase and step labels at 1fps, with instrument and action labels annotated at 35-second intervals.  For a fair comparison,
% with existing methods,
we follow the same training approach as in existing work \cite{valderrama2022towards}. It involves first training for phase and step recognition, followed by instrument and action detection using the best-trained step recognition model. This also results in the PSI-AVA dataset not benefiting from the mutual promotion of the four tasks during training, as observed in the cataract dataset training.  For phase and step recognition tasks, the model is only optimized by the interaction between phase and step,  and similarly for instrument and action detection.  Nevertheless, our method still achieves the best performance in each task, further validating the effectiveness of our framework.

%[1]Temporal memory relation network for workflow recognition from surgical video,”
%[2] Multi-task temporal convolutional networks for joint recognition of surgical phases and steps in gastric bypass procedures,
 %------------figure----------------
\begin{figure}[tp]
\begin{center}
\includegraphics[width=\linewidth]{figures/Fig.6.pdf}

\caption{t-SNE Visualization results of (A) baseline, (B) baseline+HRAM and (C) HCT in phase, step and action recognition task.  The blue circle indicates improvement in effectiveness. Due to space constraints, only the distribution of the 7  challenging action classes related to the instrument ``Cannula" are shown in the figures, illustrating our method's enhanced performance on these difficult action classes.}

\label{tsne}
\end{center}
% \vspace{-0.5cm}
\end{figure}
%----------------------------------
 %------------figure----------------
\begin{figure}[tp]
\begin{center}
\includegraphics[width=\linewidth]{figures/Fig.7.png}

\caption{Comparision of Average Precision (AP) of action detection across the challenging classes related to the instrument ``Cannula" in Fig. 6, to quantitatively demonstrate enhanced effectiveness.}
\label{bar}
\end{center}
% \vspace{-0.5cm}
\end{figure}
%----------------------------------

%------------figure----------------
\begin{figure*}[t]
\begin{center}
\includegraphics[width=0.89\linewidth]{figures/Fig.8.pdf}
% \end{center}
\caption{Multi-level visualisation based on (A) Video66 from cataract dataset and (B) CASE015 from PSI-AVA dataset. (a) Qualitative comparisons for phase recognition. (b) Qualitative comparisons for step recognition. (c) Qualitative comparisons for action detection. We show the ground-truth (GT) ({\color{cyan}{sky blue}}), right predictions (\textcolor{cyan}{sky blue}) and wrong predictions ({\color{Thistle}light purple}).}
\label{result1}
\end{center}
% \vspace{-0.5cm}
\end{figure*}
%----------------------------------





\subsubsection{Result Visualisation and Analysis}


% %------------figure----------------
% \begin{figure}[th]
% \begin{center}
% \includegraphics[width=\linewidth]{figures/results_cataract.pdf}
% % \end{center}
% \caption{Multi-granularity visualisation based on Video30 from cataract dataset. (a) qualitative comparisons for phase recognition. (b)qualitative comparisons for step recognition. (c) qualitative comparisons with some other methods for action detection. We show the ground-truth ({\color{cyan}{sky blue}}), right predictions (\textcolor{cyan}{sky blue}) and wrong
% predictions ({\color{Thistle}light purple}).}
% \label{result1}
% \end{center}
% % \vspace{-0.5cm}
% \end{figure}
% %---------------------------------- 

% %------------figure----------------
% \begin{figure}[tb]
% \begin{center}
% \includegraphics[width=\linewidth]{figures/results_psiava.pdf}
% % \end{center}
% \caption{Multi-granularity visualisation based on CASE015 from PSIAVA dataset. (a) qualitative comparisons for phase recognition. (b)qualitative comparisons  for step recognition. (c) qualitative comparisons for action detection. We show the ground-truth ({\color{cyan}{sky blue}}), right predictions (\textcolor{cyan}{sky blue}) and wrong
% predictions ({\color{Thistle}{light purple}}).}
% \label{result2}
% \end{center}
% % \vspace{-0.5cm}
% \end{figure}
% %---------------------------------- 
To illustrate the effectiveness of HCT more clearly, we list visualizations of prediction results for the two datasets in Fig. \ref{result1}, which presents the multi-level semantic understanding of surgical videos spanning coarse to fine-grained tasks.
% our visualizations present, from top to bottom, the coarse-grained phase recognition results, step recognition results, and finer-grained action detection results for the same test video. 
% To facilitate a clear and effective comparison, we select SOTA and recently developed methods for visual comparison.
Limited by the page, we only list the partial results with high accuracy, TMRNet \cite{jin2021temporal},  SAHC \cite{ding2022exploring} and  MViTv2 \cite{li2022mvitv2} for the phase and step recognition, while ACTNet \cite{hao2023act} and MViTv2 \cite{li2022mvitv2} for action detection. 
% The complete method comparison can be found in the supplementary documentation for reference. 
In Fig. \ref{result1} (A), for the cataract dataset, different colours for phases and steps represent different classes. The first row provides the ground truth (GT) for reference. By examining the timelines of predicted phases and steps, it is evident that our approach exhibits excellent performance even for some challenging classes. For instance, in the step recognition, the S2, highlighted in yellow, constitutes a small proportion of frames in the entire video. Other methods misidentify this category, while our method can accurately predict it. In the action detection task, 
% we visualize the predicted positions of actions and their corresponding classes for each method, using bounding boxes and text in the top-left corner of the boxes. 
% The ground truth action positions are outlined in green in the first row, while the predicted positions by each method are shown in yellow. 
% The GT action classes and correctly predicted classes are highlighted in sky blue, whereas incorrectly predicted classes are displayed in light purple. 
%We also present the corresponding scores for predicted action.
detecting fine-grained actions is more challenging than phase and step recognition, as spatial features for some actions are highly similar, which is also illustrated by the performance of comparison methods. For example, in the left column of Fig. \ref{result1} (A),  cases such as ``lens hook grasp" and ``lens hook hold",
%or "phacoemulsifier chop lens nucleus" and "phacoemulsifier hold,"
appear visually similar and can be easily misjudged based on appearance alone. Correct judgment necessitates the integration of contextual information, specifically the temporal features of instruments in consecutive frames. 
% The right column of examples in Fig. \ref{result1} (A) follows a similar rationale. 
Notably, our method performs accurately when other approaches struggle to predict the current action, underscoring the effectiveness of our method in learning temporal dynamics. To further demonstrate proposed method's performance in accurately identifying positive instances compared to other methods, we computed the Recall (Re) metrics for all models, as detailed in the supplementary materials. The results demonstrate that our method achieves the best Re performance. 

For the PSI-AVA dataset, as depicted in Fig. \ref{result1} (B), the comparison of phase and step recognition reveals that our method produces prediction results very close to the GT, particularly in regions with dense class distributions, which are also challenging to identify. The action detection task further demonstrates the excellent performance of our method in detecting multiple classes of concurrent actions. 




% The visualization for complete method comparison, as well as additional test cases, is available in the supplementary material.






\begin{table}[tp]
\centering
\caption{Ablation study on ICL. The best results are highlighted in bold, while the next best results are in \underline{underline}. }
\label{tabel5}
\setlength{\tabcolsep}{1.5mm}{
    \begin{tabular}{cccccc|cccc}
        \toprule
        \multicolumn{6}{c|}{Items} & \multirow{2}*{Phase}&  \multirow{2}*{Step}&  \multirow{2}*{Instrument}& \multirow{2}*{Action} \\
        \cline{1-6}
         PS& PI & PA& SI& SA& IA&\multicolumn{4}{c}{} \\ \hline
         &  &  &  \checkmark&  \checkmark&  \checkmark&  0.9208&  0.9145&  0.8539&0.5472\\
         &  \checkmark&  \checkmark&  &  &  \checkmark&  \textbf{0.9672}& 0.9113&  \underline{0.8547}&\underline{0.5562}\\ 
         \checkmark&  &  \checkmark&  &  \checkmark&  &  0.9372&  0.8759&  0.8532&0.5481\\
         \checkmark&  \checkmark&  &  \checkmark&  &  &  0.9446&  0.9367&  0.8546&0.5398\\
         \checkmark&  \checkmark&  \checkmark&  \checkmark&  \checkmark&  \checkmark&  0.9353&  0.9240&  \textbf{0.8554}&0.5531\\\hline \hline
         &  &  \checkmark&  \checkmark&  &  &  0.9532&  0.9124& 0.8543&0.5535\\
         &  \checkmark&  &  &  \checkmark&  &  0.9580&  \underline{0.9452}&  0.8544&0.5478\\
         \checkmark&  &  &  &  &  \checkmark& \underline{0.9635}&\textbf{0.9581}&   0.8545&\textbf{0.5726}\\
         \bottomrule
    \end{tabular}}
\end{table}


%Particularly notable is the significant improvement of 11.12\% (59.67\% vs. 48.55\%) in 

\subsection{Ablation Study}
\subsubsection{Contribution of Key Model Components}
We conduct ablation experiments on two datasets to validate the contribution of key modules, including HRAM and ICL, with the results presented in Table \ref{tabel3}. The models excluding the HRAM and ICL modules serve as our baseline models. It is built upon the MViTv2 framework, supplemented with multi-task loss and multi-task output heads. 
% This foundational framework includes 16 transformer blocks, which offer increased output channels and reduced spatial resolution with increasing depth, thereby enhancing feature learning capabilities. 
To minimize computational costs, all of our improvements are based on the final block. We compare the effects of adding HRAM to the baseline model, as well as the combination of HRAM and ICL modules.

For the cataract dataset, adding HRAM improves the performance across all tasks. It enhances phase recognition, step recognition and action detection by 0.29\% (95.23\% vs. 94.94\%), 2.00\% (94.89\% vs. 92.89\%) and 1.19\% (56.82\% vs. 55.63\%), respectively.  This indicates a strong correlation and dependency among the four tasks within the cataract dataset. After further incorporating the ICL module, the performance of various tasks is further enhanced. This demonstrates that task-specific features are strengthened through intrinsic consistency extracted by the ICL module. The performance drop in instrument detection after incorporating the ICL module stems from limited task complementarity, as instrument detection, which depends on spatial precision, benefits less from shared temporal information across other tasks. Additionally, the multi-task learning framework is designed to prioritize overall task balance. Nonetheless, the results demonstrate that these effects are minimal in our model. 

In parallel, for the PSI-AVA dataset, due to substantial differences in the distribution of labels for different tasks as described above, our experimental design follows \cite{valderrama2022towards} to implement the four tasks. This means that we have to divide the experiments into two parts: phase and step, and instrument and action. Consequently, the HRAM and ICL modules only achieve mutual enhancement between two tasks, rather than incorporating information from all four tasks.
% our results only indicate mutual enhancement between pairs of tasks, rather than among all four.
However, even under these constraints, the addition of the HRAM still results in superior performance compared to the baseline model. Further improvements are observed with the addition of the ICL module, with an improvement of even 2.68\% (64.57\% vs. 61.89\%) in phase recognition against the baseline model.

t-SNE visualization results illustrating the contribution of our key model components are depicted in Fig. \ref{tsne}. The blue circle in Fig. \ref{tsne} demonstrates that adding our key modules improves the distinguishability of classes and reduces inter-class blurring across phase, step, and action tasks. For the action detection task, due to space constraints, we focus on 7 classes generated by the instrumented cannula. These classes exhibit significant spatial similarity, posing classification challenges. The baseline model mixes these classes as shown in Fig. \ref{tsne} (c). Upon integrating our proposed module, the distinctiveness among these classes, highlighted by the blue circle, notably increases. To quantitatively validate this observation, we compare the Average Precision (AP) values of these seven classes across three models in Fig. \ref{bar}, aligning with their labeling in Fig. \ref{tsne}, thus confirming the efficacy of our proposed module.




\subsubsection{Contribution of HRAM}
For a more detailed analysis of the HRAM module's role in enhancing interactions among various tasks, we conduct experimental analyses as presented in Table  \ref{tabel4}. ``P", ``S", ``I", and ``A" represent the tasks of phase, step, instrument, and action respectively. A checkmark beside a task indicates its participation in mutual enhancement, while its absence signifies exclusion. 
%For example, in the first row of the table, the absence of a checkmark next to 'P' indicates that in the HRAM experimental framework, while implementing the 'S', 'I', and 'A' tasks, the participation of the 'P' task is excluded. 
This approach allows us to analyze the impact of each task on the performance of a multi-task framework.

Table \ref{tabel4} reveals that when the influence of ``P" is excluded, the mAP for phase recognition is relatively lower (91.53\%). When ``S" is excluded, the performance of step recognition is the poorest (only 87.39\%). A similar pattern is observed in the action detection task. This suggests that the optimization of a task is significantly diminished when the task lacks participation in interactions with other tasks. This also underscores the importance of other tasks in optimizing or enhancing the specific task. Notably, when instrument features are missing, the performance of all phase, step, and action tasks is comparatively poor, highlighting the critical guiding role of instrument features for these three tasks. This aligns with our understanding of surgical videos, where certain specific instruments can directly indicate the current phase and step of a frame, and even some simple ongoing actions can be inferred. The last row of the table shows that when all tasks are incorporated into our framework and mutually enhance each other, we achieve optimal performance for all four tasks with even mAP of 95.23\% and 94.89\% for phase and step.
\begin{figure}[bp]
\begin{center}
\includegraphics[width=\linewidth]{figures/Fig.9.pdf}

\caption{Color-coded ribbon visualization of phase and step recognition for some challenging classes from the proposed method (Pred) and the ground truth (GT) in two complete surgical videos. In each case, we show the classes with different colors and no class appearance with blank. Horizontal axes indicate the time progression.}

\label{difficult}
\end{center}
% \vspace{-0.5cm}
\end{figure}
%----------------------------------



\subsubsection{Contribution of ICL}
To verify the effectiveness of ICL, we carry on detailed ablation experiments based on the cataract dataset. Since ICL optimizes the model by calculating losses for pairwise task features, we design ablation experiments in two aspects (separated by double solid lines in Table \ref{tabel5}) to assess the performance differences of each task within ICL.  The table lists six pairs of our four tasks, namely ``PS'', ``PI'', ``PA'', ``SI'', ``SA'', and ``IA''. On one hand, similar to HRAM, we calculate ICL by excluding the influence of one task.  Table  \ref{tabel5} (above the double solid lines) shows the model's performance when calculating ICL pairwise among the remaining three tasks after excluding one task (``P'', ``S'', ``I'', or ``A''). 
% For instance, the first row of the table, where 'SI', 'SA', and 'IA' are ticked, indicates the results obtained through ICL calculation of the three tasks 'S', 'I', and 'A' after excluding the task of ‘P’. 
We observe that relatively lower results are yielded for a task when it is excluded by ICL. In Table  \ref{tabel5}, excluding ``P'' leads to the poorest performance in the phase recognition (only 92.08\%). Excluding ``S'' results in the second-worst performance in step recognition. The absence of ``I'' leads to lower performance in phase and step recognition, as well as instrument detection. The mAP for step recognition even drops to only 87.59\%. Similarly, excluding ``A'' reduces the performance of action detection. This confirms the importance of each task in the functioning of ICL.

However, due to inherent differences between tasks, the optimal results are not able to be yielded when all four task pairs are fed into ICL, as shown in Table \ref{tabel5}. In our tasks, phase and step have similar optimization objectives and can extract the strongest consistencies for the same positive sample, as do action and instrument. Our second set of experiments, as illustrated in Table  \ref{tabel5} (below the double solid lines), confirms this. When we optimize the model by combining the pairs of ``PS'' and ``IA'', the model achieves optimal performance in each task, further improving upon the addition of HRAM.





\subsection{Effectiveness of HCT+}

Utilizing the Kinetics\-400 \cite{carreira2017quo} pretrained model provided by MViTv2 \cite{li2022mvitv2}, we conduct experiments to validate the effectiveness of the ST-Ada, as presented in Table \ref{tabel6}. 
% The number of parameters is denoted as 'params' in millions  (M), and 'PM' represents peak memory usage in megabytes (MB).
The peak memory is measured when the batch size is 16. ``Full-tuning" indicates fine-tuning of the entire model.
% 'w/o ST\_Ada' signifies the framework without the ST-Ada, while 'w/S\_Ada' represents the model with only a spatial adapter. 
As a reference baseline method, we perform full fine-tuning based on the pretrained model of the MViTv2 framework, which gets the poorest performance. When fully fine-tuning our proposed model, the performance on the four tasks reaches the optimum, but the ``param'' is also the highest at 57.2M, and memory consumption is at its peak. If we freeze the parameters of the first 15 transformer blocks of our proposed model and only optimize the last improved block, denoted as ``w/o ST\_Ada'', the number of parameters to be optimized decreases by 57.6\%. However, the performance on each task also decreases, with a notable drop of 4.60\% (91.75\% vs. 96.35\%) in phase recognition. Nevertheless, with the addition of the spatial adapter (w/S\_Ada), the model's performance experiences a widespread improvement, with only a marginal increase of 1.2\% in parameters based on the ``w/o ST\_Ada'' model. Additionally, during training, peak memory decreases by nearly 10G compared to our full-tuning model, significantly reducing computational costs. Similarly, when only the temporal adapter is added (w/T\_Ada), with a minimal increase of 9.5\% in parameters, the performance on each task is better than without any adapters. This indicates the excellent transferability of spatial and temporal adapters. Upon combining spatial and temporal adapters (HCT+), our model's performance further improves with only a 10.7\% increase in parameters. The accuracy of phase and step recognition even surpasses that of ``w/o ST\_Ada'' by 3.67\% (95.42\% vs. 91.75\%) and 4.70\% (94.37\% vs. 89.67\%), respectively. Here is just a preliminary exploration of the application of transfer learning in our task. Its feasibility further makes it possible to apply our multi-level semantic scene understanding task facts to surgical scenarios.

\begin{table}[tp]
    \centering
\caption{Performance on the cataract dataset for HCT+. ``Param" denotes the number of parameters and ``PM" denotes peak memory usage. }
\label{tabel6}
\setlength{\tabcolsep}{0.2mm}{
    \begin{tabular}{c|cc|cccc} 
    \toprule
        Method&Params(M) &PM(MB)& Phase&  Step&   Instrument& Action   \\\hline
          MViTv2(Full-tuning)& 35.7 &34382 & 0.9494& 0.9289 & 0.8543 & 0.5563  \\
          HCT(Full-tuning) &  57.2 (100\%) &42850 &  \textbf{0.9635}& \textbf{ 0.9581}& 0.8545&  \textbf{0.5726}\\
          HCT(w/o ST\_Ada)&  24.3 (42.4\%)&32816 &  0.9175&   0.8967& 0.8541&0.5412\\\hline\hline
          HCT(w/ S\_Ada)&  24.6 (+1.2\%) & 32824 &   0.9381&  0.9301& 0.8539&  0.5613\\
          HCT(w/ T\_Ada)&  26.6 (+9.5\%)&35610 &  0.9345& 0.9271  &  \underline{0.8545}&  0.5537\\  
 HCT+(w/ ST\_Ada)& 26.9 (+10.7\%) & 35616 & \underline{0.9542} &\underline{0.9437} &\textbf{0.8546} & \underline{0.5672}  \\ \bottomrule   
    \end{tabular}}    
\end{table}
 %------------figure----------------

\subsection{Challenge Cases Analysis}
We demonstrate the recognition performance of our model across phases and steps for challenging classes, as shown in Fig. \ref{difficult}. The classes are represented in different colors, with blank spaces indicating no class appearance. For the cataract dataset, as depicted in the video frame in Fig. \ref{difficult} (A).a, it is difficult to extract effective spatial information for discrimination. This is because the “Cannula” in the left figure is at the edge of the field of view, and the ``Implant injector" in the right figure is transparent. In Fig. \ref{difficult} (A).b, the incomplete display of the ``Handpiece" head instrument can lead to confusion with similar instruments, such as “Cannula”, posing a classification challenge.

In the PSI-AVA dataset, both the phase and step class distribution shown in Fig. \ref{difficult} B are scattered, making it challenging to differentiate classes using temporal information. In Fig. \ref{difficult} (B).a, the confusion between instruments and tissue in the video frame complicates the distinction of operations. In Fig. \ref{difficult} (B).b, the step ``Tiempo\_muerto" indicates the absence of a step, which is widely dispersed and scattered in the whole video, making it more difficult to identify null operations compared to other operations. Despite these challenges, our method predicts a distribution very similar to the ground truth (GT). Combined with the results shown in Fig. \ref{bar}, this demonstrates the robustness and effectiveness of our method for all tasks.


\section{Conclusion}
% \subsection{Conclusion}
In this paper, we present a novel hierarchical context transformer (HCT) network to access MSSU by achieving recognition of phases and steps, and detection of actions and instruments. Specifically, we develop a hierarchical relation aggregation module (HRAM) to dig out the intrinsic relationships between different tasks and further boost task-wise representation learning with the proposed inter-task contrastive learning (ICL). For the costs of model training and storage on the transformer models, we introduce spatial and temporal adapters to equip our HCT for HCT+ with spatial-temporal reasoning capability, which enables the HCT+ to achieve comparable performance with the full fine-tuned model at a small parameter cost.
Comprehensive experimental results on two surgical video datasets demonstrate the superiority of our proposed model and the contribution of each key component.

\section{Acknowledgement}
This work was supported in part by General Program of National Natural Science Foundation of China (82102189 and 82272086).




\bibliographystyle{ieeetr}
\bibliography{HCT}

%Use $\backslash${\tt{begin\{IEEEbiography\}}} and then for the 1st argument use $\backslash${\tt{includegraphics}} to declare and link the author photo. Use the author name as the 3rd argument followed by the biography text.
% \begin{IEEEbiography}[{\includegraphics[width=1in,height=1.25in,clip,keepaspectratio]{figures/yanhu.jpg}}]{Yan Hu}
% Yang Yue is a PhD student of the School of Computer Science from the University of Birmingham, UK. His research interests include bioinformatics, machine learning and data mining.
% \end{IEEEbiography}
 % \vspace{-20 pt}
\begin{IEEEbiography}
[{\includegraphics[width=1in,height=1.25in,clip,keepaspectratio]{figures/Luoying_Hao.jpg}}]{Luoying Hao}
 is a PhD student of the School of Computer Science from the University of Birmingham, UK. His research interests include medical image analysis, surgical video analysis and scene understanding.
\end{IEEEbiography}
\vspace{-20 pt}
\begin{IEEEbiography}[{\includegraphics[width=1in,height=1.25in,clip,keepaspectratio]{figures/Yan_Hu.jpg}}]{Yan Hu}
 received the PhD degree from the Department of Information Science and Technology, the University of Tokyo, Japan. She is working now in the Southern University of Science and Technology, China. Her research interests include medical image analysis, surgery video processing, and computer-aided surgery.
\end{IEEEbiography}
 \vspace{-20 pt}
\begin{IEEEbiography}
[{\includegraphics[width=1in,height=1.25in,clip,keepaspectratio]{figures/Yang_Yue.jpg}}]{Yang Yue}
 is a PhD student of the School of Computer Science from the University of Birmingham, UK. His research interests include bioinformatics, machine learning and data mining.
\end{IEEEbiography}
\vspace{-20 pt}
\begin{IEEEbiography}[{\includegraphics[width=1in,height=1.25in,clip,keepaspectratio]{figures/Li_Wu.jpg}}]{Li Wu}
 received  the Master’s degree in Measuring and Testing Technologies and Instruments from the University of Electronic Science and Technology of China in 2004. He is currently the general manager of Cloud Shadow Medical Technology Co., Ltd. at BGI Genomics. With over 20 years of experience in the medical device industry, he has held positions at Mindray, BGI, and BGI Genomics. He has been involved in and led the development of high-end medical devices with independent intellectual property rights, including digital ultrasound, high-precision infusion pumps, high-throughput sequencers, automated library preparation equipment, remote ultrasound robots, and automated ultrasound robots. His research interests include upper and lower computer communication, information systems, robotic motion control, and medical imaging AI.
\end{IEEEbiography}
\vspace{-20 pt}
\begin{IEEEbiography}
[{\includegraphics[width=1in,height=1.25in,clip,keepaspectratio]{figures/Huazhu_Fu.jpg}}]{Huazhu Fu}
(Senior Member, IEEE) received the
Ph.D. degree from Tianjin University, Tianjin, China,
in 2013. He is currently a senior Scientist with IHPC,
A*STAR, Singapore. He was a Research Fellow with
Nanyang Technological University (NTU), Singapore, during 2013–2015, a Research Scientist with
the Institute for Infocomm Research (I2R), A*STAR,
Singapore, during 2015–2018, and a Senior Scientist with Inception Institute of Artificial Intelligence
(IIAI), UAE, during 2018–2021. His research interests include computer vision, AI in healthcare, and
trustworthy AI.
\end{IEEEbiography}
\vspace{-20 pt}
\begin{IEEEbiography}
[{\includegraphics[width=1in,height=1.25in,clip,keepaspectratio]{figures/Jinming_Duan.jpg}}]{Jinming Duan}
received the Ph.D. degree in computer science from the University of Nottingham.
He is currently a Turing Fellow at the Alan Turing
Institute and an Assistant Professor of computer
science at the University of Birmingham. He is also
a fellow of the Higher Education Academy (FHEA)
under the U.K. Professional Standards Framework
for teaching and learning support in higher education. He was a Research Associate jointly at the
Department of Computing, Institute of Clinical Sciences, Imperial College London, where he has been
developed cutting-edge machine learning methods for cardiovascular imaging
problems. His Ph.D. was funded by the Engineering and Physical Sciences
Research Council. His research interests include deep neural nets, variational
methods, partial/ordinary differential equations, numerical optimisation, and
finite difference/element methods, with applications to image processing,
computer vision, and medical imaging analysis.
\end{IEEEbiography}
\vspace{-20 pt}
\begin{IEEEbiography}
[{\includegraphics[width=1in,height=1.25in,clip,keepaspectratio]{figures/Jiang_Liu.png}}]{Jiang Liu}
(Senior Member, IEEE) received the bachelor’s degree from the Department of Computer Science, University of Science and Technology of China,
Hefei, China, and the master’s and doctorate degree
from the Department of Computer Science, National
University of Singapore, Singapore. He is currently
a tenured Professor with Southern University of Science and Technology and Founder of the iMED China
team. His research interests include artificial intelligence in ophthalmology, eye-brain linkage, precision
medicine, and surgical robots.
\end{IEEEbiography}
% \section{References}


% \begin{thebibliography}{00}



% \end{thebibliography}

\end{document}


