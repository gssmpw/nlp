\section{Related work}
\label{related-work}

Researchers have studied improving architecture knowledge by producing comparative studies of knowledge management tools and analyzing the evolution of such tools over the years ____. Because detecting and documenting architecture patterns improves architecture knowledge, these contributions are important to address RQ3 effectively.

While multiple tools for automating the detection of software pattern instances exist
____, these only identify instances from the original Gamma et al. catalog and do not include architecture or microservice patterns.

Alternatively, some tools focus on retrieving more general architecture knowledge____, powered by static and dynamic analysis techniques. These follow a data-driven approach using information extracted from multiple artifact types, such as code, documentation, metadata, and architecture models. However, none of these tools can explicitly identify microservice pattern instances____. Moreover, only half of these tools are automated. We found one approach that can detect five microservice pattern instances based on specific metrics____. Nonetheless, the approach is not easily generalizable to other microservice pattern instances and can only detect a few patterns. 

Regarding the detection of microservices, multiple approaches for doing so exist____. However, these surveys do not focus on detecting microservice pattern instances. A survey on detecting microservice patterns, specifically API microservice patterns, showed that available tools need to be combined to detect a large set of patterns, that no tool detects patterns without human intervention, and that none use LLMs for detecting them____.

The use of LLMs for tackling software engineering problems has recently increased substantially____. Nevertheless, to our knowledge, no approaches currently leverage LLMs to detect microservice pattern instances.

While research provides some insights on detecting microservice pattern instances (RQ2), we found a research gap regarding which IaC artifacts that are most common in code repositories contain relevant microservice pattern information (RQ1) and to what extent detecting these pattern instances improves practitioner architecture knowledge (RQ3). Thus, an approach for detecting a large pool of microservice pattern instances powered by LLMs and IaC artifacts may be a practical approach to improving practitioner architecture knowledge.

%---------------------------------------