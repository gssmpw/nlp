\newpage 

\section{Complete Experimental Results}
\label{appendix:complete_results}

We hereafter report the results of all our experiments on scoring systems and decision diagrams across the three datasets (UCI Adult Income, Default of Credit Card Clients, and COMPAS) and the two fairness metrics (statistical parity and equal opportunity).
More precisely, for each dataset and fairness metric, we report one summary figure illustrating the complex interplays between the three considered desiderata: predictive performance, fairness, and sparsity. As in Figure~\ref{fig:effect3} (Section~\ref{sec:results}), we plot the minimum and maximum achievable fairness values as a function of the desired sparsity level $\sparsityvalue$ for different $\epsilon$ parameters. Figure~\ref{fig:results_all_scoring_systems} (respectively, Figure~\ref{fig:results_all_decision_diagrams}) reports all results for our experiments on scoring systems (respectively, on decision diagrams).
While all the experimental results can be read in these figures, detailed result files and tables are also available on our online repository, along with the source code\footnote{\url{https://github.com/vidalt/Rashomon-Explorer}}.

As discussed in Section~\ref{sec:results}, the greater complexity of decision diagrams using multivariate splits results in more nuanced trade-offs between predictive performance, fairness, and sparsity. More precisely, in Figure~\ref{fig:results_all_decision_diagrams}, the expected trends are observed: tightening either the Rashomon set parameter $\epsilon$ or the enforced sparsity value $\sparsityvalue$ further restricts the range of achievable fairness values. 
For instance, in our experiments using decision diagrams on the COMPAS dataset, for sparsity values $\sparsityvalue=4$, tightening the Rashomon set parameter $\epsilon$ from $20\%$ to $1\%$ restricts the minimum achievable statistical parity value from $-0.34$ to $-0.10$. On the same set of experiments, tightening the sparsity value $\sparsityvalue$ from $12$ to $4$ restricts the minimum achievable statistical parity within a $20\%$-Rashomon set from $0.41$ to $0.34$.
While these differences are substantial, they remain subtler than those observed in our experiments on scoring systems. While the use of multivariate splits arguably affects the interpretability of the resulting models, this underscores the importance of the hypothesis class as a critical factor.

\footnotesize
\resizebox{0.9\columnwidth}{!}{%
\begin{tabular}{ll cccc}
\toprule
& \multirow{2}{*}{Model} & \multicolumn{3}{c}{ \% Correct MCQ} & {Sens.} \\
\cmidrule(lr){3-6}
& & \multicolumn{1}{c}{Both} & \multicolumn{1}{c}{Act.} & \multicolumn{1}{c}{Jus.} &  \multicolumn{1}{c}{Act.} \\
\midrule
& Random &  25.3 & 25.3 & 25.3 & 40.5 \\
\midrule
\multirow{8}{*}{\rotatebox{90}{Blind}} & \cellcolor{gray!20}\textbf{Closed Source} & & & & \\
& {Gemini 1.5 Flash} & 12.2 & 15.0 & 14.1 & 46.6 \\
& {o3-mini} & 15.0 & 16.8 & 17.1 & 51.9 \\ 
& {GPT-4o} & 17.7 & 19.9 & 19.9 & 55.9 \\ 
& {Gemini 1.5 Pro} & 21.2 & 24.6 & 23.6 & 54.0 \\
& \cellcolor{gray!20}\textbf{Open Source} & & & & \\
& {InternVL 2.5} & 15.3 & 18.3 & 17.4 & 55.4 \\
& {Deepseek R1} & 16.1 & 19.4 & 17.1 & 27.3 \\ 
\midrule
\multirow{10}{*}{\rotatebox{90}{Pipeline}} & \cellcolor{gray!20}\textbf{Closed Source} & & & & \\
& {Gemini 1.5 Flash} & 14.7 & 17.7 & 16.7 & 54.2 \\
& {GPT-4o} & 21.0 & 23.7 & 23.5  & \textbf{66.0} \\
& {Claude 3.5 Sonnet} & 23.9 & 36.7 & 33.5  & 61.2 \\
& {Gemini 1.5 Pro} & 30.7 & 37.3 & 34.8  & 64.0 \\
& {Gemini 2.0 Thinking} & 37.5 & \textbf{46.3} & 42.1  & 58.8 \\
& {o3-mini} & \textbf{41.5} & 45.7 & \textbf{45.2} & 65.0 \\
% \midrule
& \cellcolor{gray!20}\textbf{Open Source} & & & & \\
& {InternVL 2.5} & 32.7 & 40.9 & 38.0 & 62.5 \\
& {Deepseek R1} & 36.5 & 42.9 & 40.0 & 61.0 \\
\midrule
\multirow{11}{*}{\rotatebox{90}{Video Models}} & \cellcolor{gray!20}\textbf{Closed Source} & & & & \\
& {Claude 3.5 Sonnet} & 36.0 & 43.5 & 41.0  & 59.3 \\
& {Gemini 2.0 Flash} & 38.9 & 49.6 & 41.3 & 60.0 \\
& {GPT-4o} & 39.8 & 45.1 & 44.8 & 59.6 \\ 
& {Gemini 1.5 Flash} & 41.7 & 46.5 & 44.3 &  54.4 \\
& {Gemini 2.0 Thinking} & 42.7 & 51.7 & 45.3  & 57.3 \\
& {Gemini 1.5 Pro} & \textbf{45.3} & \textbf{51.9} & \textbf{47.8}  & 61.1 \\
& \cellcolor{gray!20}\textbf{Open Source} & & & & \\
& {Llama 3.2} & 2.2 & 19.9 & 10.1 & 54.7 \\
& {InternVL 2.5} & 15.1 & 18.7 & 17.6 & 50.7 \\
& {Qwen2.5 VL} & 41.5 & 48.3 & 43.8 & \textbf{62.8} \\
\midrule
\midrule
 & {Human} & 92.4 & 92.4 & 92.4 & 85.1 \\
\bottomrule     
\end{tabular}
}