\section{Introduction}
\label{intro}
Temporal relation extraction (TRE) is a fundamental task in natural language processing (NLP) that involves identifying the temporal relations between targeted events. Significant efforts have been dedicated to developing datasets \cite{chambers-etal-2014-dense, gast-etal-2016-enriching, ning-etal-2018-multi} and models \cite{huang-etal-2023-classification, tan-etal-2023-event, niu-etal-2024-contempo} for detecting such relations. The outputs of these models have been applied to various downstream tasks, including recent advancements in event forecasting \cite{Ma2023ContextawareEF}, misinformation detection \cite{lei-huang-2023-identifying}, and treatment timeline extraction \cite{yao-etal-2024-overview}.

The outputs of event detection models are expected to be both \textit{complete} and \textit{consistent}, meaning they should identify \textit{all} temporal relations that can be inferred between the targeted events without any inconsistencies among the detected relations. Therefore, ideally, training models for temporal relation extraction would benefit most from datasets where event relation annotations are both complete and consistent, ensuring that every pair of targeted events is classified and that no contradictions, such as transitive relation inconsistencies, exist among the annotated relations. For fair evaluation, the set of gold relations should either include all possible pairs or be randomly selected from the complete set of pairs.

Unfortunately, the manual annotation of \textit{all} temporal relations is typically considered extremely challenging or impractical, as the number of event pairs to be considered grows quadratically with the number of targeted events in the text, making such exhaustive annotation excessively time-consuming and cognitively unmanageable \cite{naik-etal-2019-tddiscourse, rogers-etal-2024-narrativetime}. As a result, most existing datasets for training and evaluating models for this task, such as TB-Dense \cite{chambers-etal-2014-dense}, MATRES \cite{ning-etal-2018-multi}, and TCR \cite{ning-etal-2018-joint}, were annotated only for relations in consecutive sentences, a limitation that has been discussed as hindering the ability of models to effectively learn long-distance temporal relations \cite{naik-etal-2019-tddiscourse, alsayyahi-batista-navarro-2023-timeline}. TDDiscourse \cite{naik-etal-2019-tddiscourse} enriched the TB-Dense dataset with randomly selected long-range relations. While this addressed the short-distance limitation to some extent, the resulting dataset remained rather sparse (as presented in Table~\ref{tab:dense}).

\begin{figure*}[t!]
\centering
\includegraphics[width=\textwidth]{figures/gzsl_figure_1.png}
\caption{Illustration of the pipeline approach:  
\textbf{[1]} We send the same prompt to GPT-4o to generate five separate instances of the document's \textit{complete} temporal graph.  
\textbf{[2]} We extract the relation distribution as one-hot vectors over the temporal classes for each relation in each generation. 
\textbf{[3]} We sum and normalize the predictions into a single vector representing the joint prediction over the document's temporal graph.  
\textbf{[4]} We apply a transitive closure optimization algorithm to this vector to generate the final temporal graph output.
}
\label{fig:figure1}
\end{figure*}

Furthermore, datasets consisting of incomplete annotations have constrained modeling approaches to pairwise methods, where models are trained and evaluated on extracting the temporal relation between pairs of events, considering one pair at a time \cite{huang-etal-2023-classification, tan-etal-2023-event, yuan-etal-2023-zero, niu-etal-2024-contempo}. While these methods are effective in modeling the temporal relations between pairs of events, they are limited in three key aspects: 1) They do not incorporate global information, such as contextual information from the entire document. 2) Due to the scarcity of annotated labels in such resources, considering temporal constraints like transitivity within the global graph becomes extremely challenging, with prior work that aimed to leverage transitive constraints requiring the incorporation of additional relations, such as sub-events \cite{wang-etal-2020-joint} or causal relations \cite{ning-etal-2018-joint}. 3) They are time-consuming and costly, a limitation that is particularly significant given the computational expense of contemporary Large Language Models (LLMs) (see Table~\ref{tab:costs}). Indeed, research applying LLMs to solve this task remains extremely limited.

To advance the research on temporal relation extraction beyond pairwise modeling and to facilitate the use of global information in this domain, there is a need for resources containing complete and consistent temporal relations. Recently, NarrativeTime \cite{rogers-etal-2024-narrativetime} and TimeLine \cite{alsayyahi-batista-navarro-2023-timeline} addressed the issue of incomplete temporal annotations by releasing two exhaustively annotated temporal relations datasets. However, NarrativeTime, the most comprehensive re-annotation of the TimeBank-Dense corpus, annotated relations between all events in the documents, scaling to 50 or more relations per document. While comprehensive, using such a resource for experimenting with LLM-based temporal relation generation models can be computationally expensive. TimeLine, on the other hand, is smaller in scale, but we identified many inconsistencies, making it a complete yet inconsistent resource for temporal relations.

To that end, our first contribution in this research is the development of a temporal relation resource where annotations are performed exhaustive between the most salient events of the underlying story. We utilized the EventFull \cite{eirew2024eventfullcompleteconsistentevent} annotation tool, which also ensures that the resulting temporal graph is consistent.



this study makes the following contributions to the field: 
\begin{itemize}
    \item We introduce \App{}, a new resource for temporal relation extraction over a predefined set of event mentions that is \textit{exhaustively} annotated for all pairs of events (§\ref{section:dataset}). Our dataset annotation follows the MATRES \cite{ning-etal-2018-multi} annotation guidelines, encompassing four temporal relations: \textit{before}, \textit{after}, \textit{equals}, and \textit{vague}.
    \item Aiming to leverage global information and produce complete and consistent temporal relation graphs, we developed a baseline model. This model first utilizes a strong pairwise model \cite{tan-etal-2023-event} and subsequently applies a global constraints model \cite{ning-etal-2018-joint} at inference to generate a complete and consistent temporal relation graph (\ref{section:model}).
    \item We present an empirical study demonstrating that training a model on an exhaustively annotated resource improves pairwise model predictions, \alone{and that testing a model on such a resource provides more reliable evaluations}.
    \item Finally, to move beyond pairwise modeling, we introduce a novel global zero-shot prompting approach that enhances performance while significantly reducing the cost and time required for extracting temporal relations using LLMs. Furthermore, observing that LLM-generated predictions vary and often result in temporally inconsistent graphs, we developed a pipeline to address these issues. This pipeline builds a distribution over multiple predictions and leverages transitive constraints to significantly improve overall performance.
\end{itemize}




