\section{Introduction}
\label{intro}
%Temporal relation extraction (TRE) is a fundamental task in natural language processing (NLP) that involves identifying the temporal relations between targeted events. Significant efforts have been dedicated to developing datasets \cite{gast-etal-2016-enriching, rogers-etal-2024-narrativetime} and models \cite{huang-etal-2023-classification, niu-etal-2024-contempo} for detecting such relations. The outputs of these models have been applied to various downstream tasks, including recent advancements in event forecasting \cite{Ma2023ContextawareEF}, misinformation detection \cite{lei-huang-2023-identifying}, and treatment timeline extraction \cite{yao-etal-2024-overview}.


Temporal relation extraction (TRE) is a fundamental task in natural language processing (NLP) that has been instrumental in various downstream tasks, including recent advancements in event forecasting \cite{Ma2023ContextawareEF}, misinformation detection \cite{lei-huang-2023-identifying}, and medical treatment timeline extraction \cite{yao-etal-2024-overview}. 

TRE is formulated as follows: given a text with event mentions marked within it, identify all the temporal relations between these events. Accordingly, and ideally, a dataset for evaluating TRE models should consist of annotated relations between all pairs of events. However, annotating temporal relations is highly challenging \cite{pustejovsky-stubbs-2011-increasing}, and \textit{complete} annotation -- where all possible event pairs in a document are labeled -- has traditionally been considered unfeasible for human annotators \cite{naik-etal-2019-tddiscourse}. To manage this complexity, most datasets include labels for only a subset of event pairs, applying filtering methodologies such as restricting annotations to events within consecutive sentences \cite{chambers-etal-2014-dense, ning-etal-2018-multi} or creating temporal relation annotations through automated processes \cite{naik-etal-2019-tddiscourse, alsayyahi-batista-navarro-2023-timeline}. 
However, such restrictions can lead to unreliable model assessments, failing to accurately reflect a model’s ability to capture long-range relations, or reinforce biases introduced by automated annotation techniques.
% Others did not provide systematic annotation protocols to ensure completeness \cite{pustejovsky-etal-2003, wang-etal-2022-maven}, leading to criticism of their coverage \cite{pustejovsky-stubbs-2011-increasing, rogers-etal-2024-narrativetime}.
Furthermore, incomplete annotations and the lack of global coverage have led the field to primarily focus on developing \textit{pairwise} methods \cite{wen-ji-2021-utilizing, zhou-etal-2022-rsgt}, where a model extracts temporal relations between a single event pair at a time. 
Yet, such methods overlook the document's global temporal structure, resulting in inconsistencies in the output temporal graph \cite{wang-etal-2020-joint}, and are computationally inefficient, requiring $O(n^2)$ inference requests to predict all temporal relations across $n$ given events.
%However, such methods fail to incorporate a global view of the document’s temporal structure, disregarding the broader temporal graph that emerges from event interactions. Additionally, these approaches are computationally inefficient, requiring $O(n^2)$ inference requests to predict all temporal relations within a document with $n$ event mentions.

%Furthermore, incomplete annotations and the lack of global coverage have led the field to primarily focus on developing \textit{pairwise} methods \cite{wen-ji-2021-utilizing, zhou-etal-2022-rsgt}, where a model extracts temporal relations between a single event pair at a time. 
%However, such methods overlook the document's global temporal structure, leading to inconsistent temporal graph outcomes \cite{wang-etal-2020-joint}, and are computationally inefficient, requiring $O(n^2)$ inference requests to predict all temporal relations in a document with $n$ events.


\begin{figure}
    \centering
    \includegraphics[width=\linewidth]{figure/images/figure1_v2.pdf}
    \caption{\textbf{Generated concepts from existing T2I models.} Each row shows examples generated by Stable Diffusion, Kandinsky, and ConceptLab, respectively, with the affordances used as conditions are displayed in brackets. They fail to synthesize concepts that integrate multiple functions into a single coherent form. }
    % Morever, they lack an understanding of the relationship between concepts, parts, and their associated affordance.}
    %\jeongh{I think we need to also showcase good and bad examples; for the bad examples, we need generated images that separately generate concepts instead of fusing them into one single novel concept.}
    \label{fig:example}
\end{figure}

Despite these challenges, TRE has seen significant progress in the development of supervised models \cite{tan-etal-2023-event, niu-etal-2024-contempo}. However, current utilization of LLMs remains limited, particularly in zero-shot settings \cite{10.5555/3600270.3601883}.
%The most successful approaches leverage external knowledge \cite{wang-etal-2020-joint}, commonsense reasoning \cite{tan-etal-2023-event}, and knowledge distilled from LLMs \cite{niu-etal-2024-contempo, Chen2024PromptBasedET}. 
%However, the utilization of LLMs—which have proven remarkably effective in computing both factual and common-sense knowledge \cite{liu-etal-2022-generated}—remains limited, particularly in zero-shot settings \cite{10.5555/3600270.3601883}. 
The only existing studies \cite{yuan-etal-2023-zero, chan-etal-2024-exploring} have employed local pairwise prompting strategies, yielding suboptimal results while also being time- and cost-inefficient. Consequently, the application of LLMs to TRE has been widely regarded as ineffective \cite{wei-etal-2024-llms, niu-etal-2024-contempo, chan-etal-2024-exploring}.



% The implications of incomplete annotation can be categorized into two main issues: 
% First, the lack of global coverage of relations has propelled the field to primarily focus on developing \textit{pairwise} methods \cite{wen-ji-2021-utilizing, zhou-etal-2022-rsgt, tan-etal-2023-event, niu-etal-2024-contempo}, where the model is provided with a single event pair at a time to extract the relation. However, such methods lack a global perspective, particularly the broader view of the temporal graph that can be derived from a given document and its events. Additionally, they are computational ineffective requiring $O(n^2)$ inference request to predict all relations in a given document.
% Second, evaluating models on resources with biased annotation may lead to unfair assessments. For example, it may not accurately reflect a model's ability to capture long-range relations or could reinforce annotation biases introduced by the automatic processes used in dataset creation.

% \begin{figure}
    \centering
    \includegraphics[width=\linewidth]{figure/images/figure1_v2.pdf}
    \caption{\textbf{Generated concepts from existing T2I models.} Each row shows examples generated by Stable Diffusion, Kandinsky, and ConceptLab, respectively, with the affordances used as conditions are displayed in brackets. They fail to synthesize concepts that integrate multiple functions into a single coherent form. }
    % Morever, they lack an understanding of the relationship between concepts, parts, and their associated affordance.}
    %\jeongh{I think we need to also showcase good and bad examples; for the bad examples, we need generated images that separately generate concepts instead of fusing them into one single novel concept.}
    \label{fig:example}
\end{figure}

% Furthermore, studies dedicated to building supervised models have demonstrated the benefits of leveraging external knowledge \cite{wang-etal-2020-joint}, incorporating common-sense knowledge \cite{tan-etal-2023-event}, and utilizing knowledge distilled from large language models (LLMs) \cite{niu-etal-2024-contempo, Chen2024PromptBasedET} to advance the field.

% % Prior work has explored various strategies to enhance temporal relation extraction (TRE). Some approaches incorporate external knowledge sources to improve model reasoning \citep{wang-etal-2020-joint}, while others leverage common-sense knowledge bases to refine event understanding \citep{tan-etal-2023-event}. Additionally, recent work has focused on distilling knowledge from LLMs to enhance relation extraction without extensive training data \citep{niu-etal-2024-contempo, Chen2024PromptBasedET}. However, despite these advances, most methods rely on \textit{pairwise event classification}, limiting their ability to capture \textit{global} document-level consistency. Our approach addresses this limitation by leveraging LLMs to generate the entire temporal graph in a single step, enforcing global coherence through self-consistency and transitive constraints.


% Nevertheless, the utilization of LLMs—which have proven remarkably effective in computing both common knowledge and common-sense knowledge \cite{liu-etal-2022-generated}—has been rather limited for solving the TRE task, particularly in the zero-shot setting \cite{10.5555/3600270.3601883}. The most notable efforts in this settings \cite{yuan-etal-2023-zero, chan-etal-2024-exploring} have employed local pairwise approaches using ChatGPT, yielding suboptimal performance while also being time- and cost-inefficient. Consequently, the application of LLMs to the TRE task has been widely regarded as ineffective \cite{alsayyahi-batista-navarro-2023-timeline, wei-etal-2024-llms, niu-etal-2024-contempo, chan-etal-2024-exploring}.


In response, we demonstrate how to move beyond pairwise approaches by using LLMs. We introduce a novel zero-shot method that generates the entire temporal graph \textit{globally} in a single step (illustrated in Figure~\ref{fig:figure1} and explained in §\ref{section:model}). 
We then extend this basic zero-shot approach in two major ways. First, we prompt the model to ``think'' by asking it to summarize the timeline of the given events in free-form language before generating the requested temporal classification labels for all event pairs. Second, we collect label distributions by running the model multiple times and then apply a global constraints algorithm that considers these distributions to produce a final globally-optimal graph of relations.
We show that our method significantly outperforms the existing zero-shot pairwise approach across most datasets while being more efficient, as shown in §\ref{section:results}. Our findings demonstrate that, contrary to previous research, LLMs in zero-shot settings may be a valid alternative to supervised models for temporal relation extraction.


Additionally, to address the incompleteness of temporal relation datasets, we develop \textit{\App{}}, a new dataset that incorporates temporal relations between all pairs of targeted events to support unbiased evaluation (§\ref{section:dataset}).\footnote{The \App{} dataset will be made publicly available.} Our subsequent analysis further highlights the importance of datasets with complete pairwise annotation for reliable evaluation of temporal relation graph generation (§\ref{section:results:quality}).

To recap, this research makes the following three contributions: 
(1) A novel global zero-shot approach for generating consistent temporal relation graphs.  
(2) A new dataset with complete temporal relation annotations.
(3) A thorough analysis highlighting the importance of datasets with complete annotations for fair evaluation of zero-shot approaches for TRE.
