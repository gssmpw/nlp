%%%%%%%% ICML 2025 EXAMPLE LATEX SUBMISSION FILE %%%%%%%%%%%%%%%%%

\documentclass{article}

% Recommended, but optional, packages for figures and better typesetting:
\usepackage{microtype}
\usepackage{graphicx}
\usepackage{subfigure}
\usepackage{booktabs} % for professional tables

% hyperref makes hyperlinks in the resulting PDF.
% If your build breaks (sometimes temporarily if a hyperlink spans a page)
% please comment out the following usepackage line and replace
% \usepackage{icml2025} with \usepackage[nohyperref]{icml2025} above.
\usepackage{hyperref}

% Attempt to make hyperref and algorithmic work together better:
\newcommand{\theHalgorithm}{\arabic{algorithm}}

% Use the following line for the initial blind version submitted for review:
% If accepted, instead use the following line for the camera-ready submission:
%\usepackage{icml2025}
%\usepackage[accepted]{icml2025}
\usepackage[arxiv]{icml2025_arxiv}


%%%%%%%%%%%%%%%%%%%%%%%%%%%%%%%
% Joan's stuff                %
%%%%%%%%%%%%%%%%%%%%%%%%%%%%%%%

% For theorems and such
\usepackage{amsmath}
\usepackage{amssymb}
\usepackage{mathtools}
\usepackage{amsthm}
\usepackage{bm}

% Math notation
\newcommand{\se}[1]{\mathsf{#1}}
\newcommand{\te}[1]{\textbf{#1}}
\newcommand{\ma}[1]{\textbf{#1}}
\newcommand{\ve}[1]{\textbf{#1}}
\newcommand{\tx}[1]{\text{#1}}
\newcommand{\fu}[1]{\mathcal{#1}}
\newcommand{\mo}[1]{\mathop{\mathbb{#1}}}

% Notes
\newcommand{\todo}[1]{{\color{red}[#1]}}

% Sections
\newcommand{\topic}[1]{{\noindent\textbf{#1 --- }}}

% Tables & Figures
\usepackage{adjustbox}
\usepackage{pifont}
\newcommand{\cmark}{\ding{51}}%
\newcommand{\xmark}{\ding{55}}%
\newcommand{\tablecaptionspace}{\vskip 0.05in}
\newcommand{\figurecaptionspace}{\vskip -0.25in}
\usepackage[section]{placeins}

% Hyphenation
\hyphenation{CQTNet DVINet ByteCover CoverHunter CLEWS}

%%%%%%%%%%%%%%%%%%%%%%%%%%%%%%%
% End of Joan's stuff         %
%%%%%%%%%%%%%%%%%%%%%%%%%%%%%%%


% if you use cleveref..
\usepackage[capitalize,noabbrev]{cleveref}

%%%%%%%%%%%%%%%%%%%%%%%%%%%%%%%%
% THEOREMS
%%%%%%%%%%%%%%%%%%%%%%%%%%%%%%%%
\theoremstyle{plain}
\newtheorem{theorem}{Theorem}[section]
\newtheorem{proposition}[theorem]{Proposition}
\newtheorem{lemma}[theorem]{Lemma}
\newtheorem{corollary}[theorem]{Corollary}
\theoremstyle{definition}
\newtheorem{definition}[theorem]{Definition}
\newtheorem{assumption}[theorem]{Assumption}
\theoremstyle{remark}
\newtheorem{remark}[theorem]{Remark}

% Todonotes is useful during development; simply uncomment the next line
%    and comment out the line below the next line to turn off comments
%\usepackage[disable,textsize=tiny]{todonotes}
%\usepackage[textsize=tiny]{todonotes}

% The \icmltitle you define below is probably too long as a header.
% Therefore, a short form for the running title is supplied here:
\icmltitlerunning{Supervised Contrastive Learning from Weakly-Labeled Audio Segments for Musical Version Matching}

\begin{document}

\twocolumn[
\icmltitle{Supervised Contrastive Learning from Weakly-Labeled\\ Audio Segments for Musical Version Matching}

% It is OKAY to include author information, even for blind
% submissions: the style file will automatically remove it for you
% unless you've provided the [accepted] option to the icml2025
% package.

% List of affiliations: The first argument should be a (short)
% identifier you will use later to specify author affiliations
% Academic affiliations should list Department, University, City, Region, Country
% Industry affiliations should list Company, City, Region, Country

% You can specify symbols, otherwise they are numbered in order.
% Ideally, you should not use this facility. Affiliations will be numbered
% in order of appearance and this is the preferred way.
\icmlsetsymbol{equal}{*}

\begin{icmlauthorlist}
\icmlauthor{Joan Serrà}{sai}
\icmlauthor{R.\ Oguz Araz}{upf}
\icmlauthor{Dmitry Bogdanov}{upf}
\icmlauthor{Yuki Mitsufuji}{sai,sgc}
\end{icmlauthorlist}

\icmlaffiliation{sai}{Sony AI}
\icmlaffiliation{upf}{Music Technology Group, Universitat Pompeu Fabra}
\icmlaffiliation{sgc}{Sony Group Corporation}

\icmlcorrespondingauthor{Joan Serrà}{joan.serra@sony.com}

% You may provide any keywords that you
% find helpful for describing your paper; these are used to populate
% the "keywords" metadata in the PDF but will not be shown in the document
\icmlkeywords{Musical versions, cover songs, contrastive learning, weakly-labeled segments}

\vskip 0.3in
]

% this must go after the closing bracket ] following \twocolumn[ ...

% This command actually creates the footnote in the first column
% listing the affiliations and the copyright notice.
% The command takes one argument, which is text to display at the start of the footnote.
% The \icmlEqualContribution command is standard text for equal contribution.
% Remove it (just {}) if you do not need this facility.

\printAffiliationsAndNotice{}  % leave blank if no need to mention equal contribution
%\printAffiliationsAndNotice{\icmlEqualContribution} % otherwise use the standard text.

%\begin{figure}[ht]
%\vskip 0.2in
%\begin{center}
%\centerline{\includegraphics[width=\columnwidth]{icml_numpapers}}
%\caption{Historical locations and number of accepted papers for International Machine Learning Conferences (ICML 1993 -- ICML 2008) and International Workshops on Machine Learning (ML 1988 -- ML 1992). At the time this figure was produced, the number of accepted papers for ICML 2008 was unknown and instead estimated.}
%\label{icml-historical}
%\end{center}
%\vskip -0.2in
%\end{figure}

%\begin{algorithm}[tb]
%   \caption{Bubble Sort}
%   \label{alg:example}
%\begin{algorithmic}
%   \STATE {\bfseries Input:} data $x_i$, size $m$
%   \REPEAT
%   \STATE Initialize $noChange = true$.
%   \FOR{$i=1$ {\bfseries to} $m-1$}
%   \IF{$x_i > x_{i+1}$}
%   \STATE Swap $x_i$ and $x_{i+1}$
%   \STATE $noChange = false$
%   \ENDIF
%   \ENDFOR
%   \UNTIL{$noChange$ is $true$}
%\end{algorithmic}
%\end{algorithm}

% Note use of \abovespace and \belowspace to get reasonable spacing
% above and below tabular lines.
%\begin{table}[t]
%\caption{Classification accuracies for naive Bayes and flexible Bayes on various data sets.}
%\label{sample-table}
%\vskip 0.15in
%\begin{center}
%\begin{small}
%\begin{sc}
%\begin{tabular}{lcccr}
%\toprule
%Data set & Naive & Flexible & Better? \\
%\midrule
%Breast    & 95.9$\pm$ 0.2& 96.7$\pm$ 0.2& $\surd$ \\
%Cleveland & 83.3$\pm$ 0.6& 80.0$\pm$ 0.6& $\times$\\
%Glass2    & 61.9$\pm$ 1.4& 83.8$\pm$ 0.7& $\surd$ \\
%Credit    & 74.8$\pm$ 0.5& 78.3$\pm$ 0.6&         \\
%Horse     & 73.3$\pm$ 0.9& 69.7$\pm$ 1.0& $\times$\\
%Meta      & 67.1$\pm$ 0.6& 76.5$\pm$ 0.5& $\surd$ \\
%Pima      & 75.1$\pm$ 0.6& 73.9$\pm$ 0.5&         \\
%Vehicle   & 44.9$\pm$ 0.6& 61.5$\pm$ 0.4& $\surd$ \\
%\bottomrule
%\end{tabular}
%\end{sc}
%\end{small}
%\end{center}
%\vskip -0.1in
%\end{table}

%\subsection{Theorems and such}
%The preferred way is to number definitions, propositions, lemmas, etc. consecutively, within sections, as shown below.
%\begin{definition}
%\label{def:inj}
%A function $f:X \to Y$ is injective if for any $x,y\in X$ different, $f(x)\ne f(y)$.
%\end{definition}
%Using \cref{def:inj} we immediate get the following result:
%\begin{proposition}
%If $f$ is injective mapping a set $X$ to another set $Y$, the cardinality of $Y$ is at least as large as that of $X$
%\end{proposition}
%\begin{proof} 
%Left as an exercise to the reader. 
%\end{proof}
%\cref{lem:usefullemma} stated next will prove to be useful.
%\begin{lemma}
%\label{lem:usefullemma}
%For any $f:X \to Y$ and $g:Y\to Z$ injective functions, $f \circ g$ is injective.
%\end{lemma}
%\begin{theorem}
%\label{thm:bigtheorem}
%If $f:X\to Y$ is bijective, the cardinality of $X$ and $Y$ are the same.
%\end{theorem}
%An easy corollary of \cref{thm:bigtheorem} is the following:
%\begin{corollary}
%If $f:X\to Y$ is bijective, 
%the cardinality of $X$ is at least as large as that of $Y$.
%\end{corollary}
%\begin{assumption}
%The set $X$ is finite.
%\label{ass:xfinite}
%\end{assumption}
%\begin{remark}
%According to some, it is only the finite case (cf. \cref{ass:xfinite}) that is interesting.
%\end{remark}
%restatable

In this section, we introduce DAB: \textbf{D}iscrete \textbf{A}uto-regressive \textbf{B}iasing. 
First, we present the formulation of the target distribution as a joint distribution and explain the motivation behind this approach. Next, we describe how our algorithm samples from the joint distribution by alternating between biased auto-regressive generation and discrete gradient-based sampling. We finally demonstrate that gradient-based discrete sampling enables our algorithm to have more thorough, stable, and efficient sampling when compared to continuous methods. 
\subsection{Formulation}
\section{Auxiliary-Variable Adaptive Control Barrier Functions}
\label{sec:AVBCBF}

In this section, we introduce Auxiliary-Variable Adaptive Control Barrier Functions (AVCBFs) for safety-critical control.
We start with a simple example to motivate the need for AVCBFs.

\subsection{Motivation Example: Simplified Adaptive Cruise Control}
\label{subsec:SACC-problem}

Consider a Simplified Adaptive Cruise Control (SACC) problem with the dynamics of ego vehicle expressed as 
\begin{small}
\begin{equation}
\label{eq:SACC-dynamics}
\underbrace{\begin{bmatrix}
\dot{z}(t) \\
\dot{v}(t) 
\end{bmatrix}}_{\dot{\boldsymbol{x}}(t)}  
=\underbrace{\begin{bmatrix}
 v_{p}-v(t) \\
 0
\end{bmatrix}}_{f(\boldsymbol{x}(t))} 
+ \underbrace{\begin{bmatrix}
  0 \\
  1 
\end{bmatrix}}_{g(\boldsymbol{x}(t))}u(t),
\end{equation}
\end{small}
where $v_{p}>0, v(t)>0$ denote the velocity of the lead vehicle (constant velocity) and ego vehicle, respectively, $z(t)$ denotes the distance between the lead and ego vehicle and $u(t)$ denotes the acceleration (control) of ego vehicle, subject to the control constraints
\begin{equation}
\label{eq:simple-control-constraint}
u_{min}\le u(t) \le u_{max}, \forall t \ge0,
\end{equation}
where $u_{min}<0$ and $u_{max}>0$ are the minimum and maximum control input, respectively.

 For safety, we require that $z(t)$ always be greater than or equal to the safety distance denoted by $l_{p}>0,$ i.e., $z(t)\ge l_{p}, \forall t \ge 0.$ Based on Def. \ref{def:HOCBF}, let $\psi_{0}(\boldsymbol{x})\coloneqq b(\boldsymbol{x})=z(t)-l_{p}.$ From \eqref{eq:sequence-f1} and \eqref{eq:sequence-set1}, since the relative degree of $b(\boldsymbol{x})$ is 2, we have
\begin{equation}
\label{eq:SACC-HOCBF-sequence}
\begin{split}
&\psi_{1}(\boldsymbol{x})\coloneqq v_{p}-v(t)+k_{1}\psi_{0}(\boldsymbol{x})\ge 0
,\\
&\psi_{2}(\boldsymbol{x})\coloneqq -u(t)+k_{1}(v_{p}-v(t))+k_{2}\psi_{1}(\boldsymbol{x})\ge 0,
\end{split}
\end{equation}
where $\alpha_{1}(\psi_{0}(\boldsymbol{x}))\coloneqq k_{1}\psi_{0}(\boldsymbol{x}), \alpha_{2}(\psi_{1}(\boldsymbol{x}))\coloneqq k_{2}\psi_{1}(\boldsymbol{x}), k_{1}>0, k_{2}>0.$ The constant class $\kappa$ coefficients $k_{1},k_{2}$ are always chosen small to equip ego vehicle with a conservative control strategy to keep it safe, i.e., smaller $k_{1},k_{2}$ make ego vehicle brake earlier (see \cite{xiao2021high}). Suppose we wish to minimize the energy cost as $\int_{0}^{T} u^{2}(t)dt.$ We can then formulate the cost in the QP with constraint $\psi_{2}(\boldsymbol{x})\ge0$ and control input constraint \eqref{eq:simple-control-constraint} to get the optimal controller for the SACC problem. However, the feasible set of input can easily become empty if $u(t)\le k_{1}(v_{p}-v(t))+k_{2}\psi_{1}(\boldsymbol{x})<u_{min}$,  which causes infeasibility of the optimization. In \cite{xiao2021adaptive}, the authors introduced penalty variables in front of class $\kappa$ functions to enhance the feasibility. This approach defines $\psi_{0}(\boldsymbol{x})\coloneqq b(\boldsymbol{x})=z(t)-l_{p}$ as PACBF and other constraints can be further defined as
\begin{equation}
\label{eq:SACC-PACBF-sequence}
\begin{split}
\psi_{1}(\boldsymbol{x},p_{1}(t))&\coloneqq v_{p}-v(t)+p_{1}(t)k_{1}\psi_{0}(\boldsymbol{x})\ge 0,\\
\psi_{2}(\boldsymbol{x},p_{1}(t),&\boldsymbol{\nu})\coloneqq \nu_{1}(t)k_{1}\psi_{0}(\boldsymbol{x})+p_{1}(t)k_{1}(v_{p}\\
-v(t))&-u(t)+\nu_{2}(t)k_{2}\psi_{1}(\boldsymbol{x},p_{1}(t))\ge 0,
\end{split}
\end{equation}
where $\nu_{1}(t)=\dot{p}_{1}(t),\nu_{2}(t)=p_{2}(t), p_{1}(t)\ge0,p_{2}(t)\ge0,\boldsymbol{\nu}=(\nu_{1}(t),\nu_{2}(t)).$ $p_{1}(t),p_{2}(t)$ are time-varying penalty variables, which alleviate the conservativeness of the control strategy and $\nu_{1}(t),\nu_{2}(t)$ are auxiliary inputs, which relax the constraints for $u(t)$ in $\psi_{2}(\boldsymbol{x},p_{1}(t),\boldsymbol{\nu})\ge0$ and \eqref{eq:simple-control-constraint}. However, in practice, we need to define several additional constraints to make PACBF valid as shown in Eqs. (24)-(27) in \cite{xiao2021adaptive}. First, we need to define HOCBFs ($b_{1}(p_{1}(t))=p_{1}(t),b_{2}(p_{2}(t))=p_{2}(t))$ based on Def. \ref{def:HOCBF} to ensure $p_{1}(t)\ge0,p_{2}(t)\ge0.$ Next we need to define HOCBF ($b_{3}(p_{1}(t))=p_{1,max}-p_{1}(t)$) to confine the value of $p_{1}(t)$ in the range $[0,p_{1,max}].$ We also need to define CLF ($V(p_{1}(t))=(p_{1}(t)-p_{1}^{\ast})^{2}$) based on Def. \ref{def:control-l-f} to keep $p_{1}(t)$ close to a small value $p_{1}^{\ast}.$ $b_{3}(p_{1}(t)), V(p_{1}(t))$ are necessary since $\psi_{0}(\boldsymbol{x},p_{1}(t))\coloneqq p_{1}(t)k_{1}\psi_{0}(\boldsymbol{x})$ in first constraint in \eqref{eq:SACC-PACBF-sequence} is not a class $\kappa$ function with respect to $\psi_{0}(\boldsymbol{x}),$ i.e., $p_{1}(t)k_{1}\psi_{0}(\boldsymbol{x})$ is not guaranteed to strictly increase since $\psi_{0}(\boldsymbol{x},p_{1}(t))$ is in fact a class $\kappa$ function with respect to $p_{1}(t)\psi_{0}(\boldsymbol{x})$, which is against Thm. \ref{thm:safety-guarantee}, therefore $\psi_{1}(\boldsymbol{x},p_{1}(t))\ge 0$ in \eqref{eq:SACC-PACBF-sequence} may not guarantee $\psi_{0}(\boldsymbol{x})\ge 0.$ This illustrates why we have to limit the growth of $p_{1}(t)$ by defining $b_{3}(p_{1}(t)),V(p_{1}(t)).$ However, the way to choose appropriate values for $p_{1,max},p_{1}^{\ast}$ is not straightforward. We can imagine as the relative degree of $b(\boldsymbol{x})$ gets higher, the number of additional constraints we should define also gets larger, which results in complicated parameter-tuning process. To address this issue, we introduce $a_{1}(t),a_{2}(t)$ in the form
\begin{small}
\begin{equation}
\label{eq:SACC-AVBCBF-sequence}
\begin{split}
\psi_{1}(\boldsymbol{x},\boldsymbol{a},\dot{a}_{1}(t))\coloneqq a_{2}(t)(\dot{\psi}_{0}(\boldsymbol{x},a_{1}(t))
+k_{1}\psi_{0}(\boldsymbol{x},a_{1}(t)))\ge 0,\\
\psi_{2}(\boldsymbol{x},\boldsymbol{a},\dot{a}_{1}(t),\boldsymbol{\nu})\coloneqq \nu_{2}(t)\frac{\psi_{1}(\boldsymbol{x},\boldsymbol{a},\dot{a}_{1}(t))}{a_{2}(t)} +a_{2}(t)(\nu_{1}(t)(z(t)\\
-l_{p})+2\dot{a}_{1}(t)(v_{p}-v(t))-a_{1}(t)u(t)+k_{1}\dot{\psi}_{0}(\boldsymbol{x},a_{1}(t)))\\
+k_{2}\psi_{1}(\boldsymbol{x},\boldsymbol{a},\dot{a}_{1}(t))\ge 0, 
\end{split}
\end{equation}
\end{small}
where $\psi_{0}(\boldsymbol{x},a_{1}(t))\coloneqq a_{1}(t)b (\boldsymbol{x})=a_{1}(t)(z(t)-l_{p}),\boldsymbol{\nu}=[\nu_{1}(t),\nu_{2}(t)]^{T}=[\ddot{a}_{1}(t),\dot{a}_{2}(t)]^{T},\boldsymbol{a}=[a_{1}(t),a_{2}(t)]^{T},$ $a_{1}(t),a_{2}(t)$ are time-varying auxiliary variables. Since $\psi_{0}(\boldsymbol{x},a_{1}(t))\ge0,\psi_{1}(\boldsymbol{x},\boldsymbol{a},\dot{a}_{1}(t))\ge 0$ will not be against $b(\boldsymbol{x})\ge 0,\dot{\psi}_{0}(\boldsymbol{x},a_{1}(t))
+k_{1}\psi_{0}(\boldsymbol{x},a_{1}(t))\ge 0$ iff $a_{1}(t)>0,a_{2}(t)>0,$ we need to define HOCBFs for auxiliary variables to make $a_{1}(t)>0,a_{2}(t)>0,$ which will be illustrated in Sec. \ref{sec:AVCBFs}.  $\nu_{1}(t),\nu_{2}(t)$ are auxiliary inputs which are used to alleviate the restriction of constraints for $u(t)$ in $\psi_{2}(\boldsymbol{x},\boldsymbol{a},\dot{a}_{1}(t),\boldsymbol{\nu})\ge0$ and \eqref{eq:simple-control-constraint}. Different from the first constraint in \eqref{eq:SACC-PACBF-sequence}, $k_{1}\psi_{0}(\boldsymbol{x},a_{1}(t))$ is still a class $\kappa$ function with respect to $\psi_{0}(\boldsymbol{x},a_{1}(t)),$ therefore we do not need to define additional HOCBFs and CLFs like $b_{3}(p_{1}(t)),V(p_{1}(t))$ to limit the growth of $a_{1}(t).$
We can rewrite $\psi_{1} (\boldsymbol{x},\boldsymbol{a},\dot{a}_{1}(t))$ in \eqref{eq:SACC-AVBCBF-sequence} as
\begin{equation}
\label{eq:SACC-AVBCBF-sequence-rewrite}
\begin{split}
\psi_{1}(\boldsymbol{x},\boldsymbol{a},\dot{a}_{1}(t))\coloneqq a_{2}(t)a_{1}(t)(v_{p}-v(t)\\
+k_{1}(1+\frac{\dot{a}_{1}(t)}{k_{1}a_{1}(t)})b(\boldsymbol{x}))\ge 0.
\end{split}
\end{equation}
Compared to the first constraint in \eqref{eq:SACC-HOCBF-sequence}, $\frac{\dot{a}_{1}(t)}{a_{1}(t)}$ is a time-varying auxiliary term to alleviate the conservativeness of control that small $k_{1}$ originally has, which shows the adaptivity of auxiliary terms to constant class $\kappa$ coefficients. 

% There is another type of adaptive CBFs called Relaxation-Adaptive Control Barrier Functions (RACBFs) in \cite{xiao2021adaptive}. The RACBF $b(\boldsymbol{x})$ is in the form:
% \begin{equation}
% \label{eq:RACBF}
% \psi_{0}(\boldsymbol{x},r(t))\coloneqq b(\boldsymbol{x})-r(t),
% \end{equation}
% where $r(t)\ge0$ is a relaxation that plays the similar role as Backup policy introduced in \cite{chen2021backup} {\color{red} How a relaxation is related to the backup policy?}. However, it is difficult for us to find the appropriate backup policy for controller of complicated dynamic system. Two main drawbacks affect the performance of RACBFs. {\color{red}wording} In the first place, $r(t)$ contracts the coverage of feasible space of states defined by $b(\boldsymbol{x})\ge0$, i.e., the distance $z(t)$ allowable for two vehicles is even smaller {\color{red}This should be larger} by $z(t)-l_{p}-r(t)\ge0$ because of the existence of non-negative $r(t)$. Secondly, the feasibility of solving QP with RACBF constraints is limited by the existence of upper bound of auxiliary input $\nu_{r}(t)$ related to $r(t)$ defined in Eq. (29) in \cite{xiao2021adaptive} {\color{red}What is $\nu_r$? you should make it self-contained.}. We can define the highest order {\color{red}what is this?} of $r(t)$ to be 2, then from \eqref{eq:SACC-HOCBF-sequence} normally we have
% \begin{equation}
% \label{eq:highest-order-RACBF}
% \begin{split}
% \psi_{2}(\boldsymbol{x},r(t),\dot{r}(t),\nu_{r}(t))\coloneqq -u(t)-\nu_{r}(t)\\
% +k_{1}(v_{p}-v(t)-\dot{r}(t))+k_{2}(v_{p}-v(t)-\dot{r}(t)\\
% +k_{1}(z(t)-l_{p}-r(t))\ge0, \nu_{r}(t)=\ddot{r}(t),
% \end{split}
% \end{equation}
% which sets the upper bound {\color{red}This is not clear} for $\nu_{r}(t)$ and there will easily be empty feasible set for $\nu_{r}(t)$ if the lower bound of $\nu_{r}(t)$ defined by constraint (31) in \cite{xiao2021adaptive} is too large. Compared to RACBFs, AVCBFs will neither contract the feasible space of states, nor set the upper bound for $\boldsymbol{\nu}$ (at least no upper bound for $\nu_{1}(t))$ as shown in the proof of Thm. \ref{thm:feasibility-guarantee} in Sec. \ref{subsec: optimal-control}, which shows the great benefits of AVCBFs in terms of safety and feasibility. 

% \subsection{HOCBFs for Auxiliary Coefficients}
\subsection{Adaptive HOCBFs for Safety:\ AVCBFs}
\label{sec:AVCBFs}

Motivated by the SACC example in Sec. \ref{subsec:SACC-problem}, given a function $b:\mathbb{R}^{n}\to\mathbb{R}$ with relative degree $m$ for system \eqref{eq:affine-control-system} and a time-varying vector $\boldsymbol{a}(t)\coloneqq [a_{1}(t),\dots,a_{m}(t)]^{T}$ with positive components called auxiliary variables, the key idea in converting a regular HOCBF into an adaptive
one without defining excessive constraints is to place one auxiliary variable in front of each function in \eqref{eq:sequence-f1} similar to \eqref{eq:SACC-AVBCBF-sequence}. 
As described in Sec. \ref{subsec:SACC-problem}, we only need to define HOCBFs for auxiliary variables to ensure each $a_{i}(t)>0, i \in \{1,...,m\}.$ To realize this, we need to define auxiliary systems that contain auxiliary states $\boldsymbol{\pi}_{i}(t)$ and inputs $\nu_{i}(t)$, through which systems we can extend each HOCBF to desired relative degree, just like $b(\boldsymbol{x})$ has relative degree $m$
with respect to the dynamics \eqref{eq:affine-control-system}. Consider $m$ auxiliary systems in the form 
\begin{equation}
\label{eq:virtual-system}
\dot{\boldsymbol{\pi}}_{i}=F_{i}(\boldsymbol{\pi}_{i})+G_{i}(\boldsymbol{\pi}_{i})\nu_{i}, i \in \{1,...,m\},
\end{equation}
where $\boldsymbol{\pi}_{i}(t)\coloneqq [\pi_{i,1}(t),\dots,\pi_{i,m+1-i}(t)]^{T}\in \mathbb{R}^{m+1-i}$ denotes an auxiliary state with $\pi_{i,j}(t)\in \mathbb{R}, j \in \{1,...,m+1-i\}.$ $\nu_{i}\in \mathbb{R}$ denotes an auxiliary input for \eqref{eq:virtual-system}, $F_{i}:\mathbb{R}^{m+1-i}\to\mathbb{R}^{m+1-i}$ and $G_{i}:\mathbb{R}^{m+1-i}\to\mathbb{R}^{m+1-i}$ are locally Lipschitz. For simplicity, we just build up the connection between an auxiliary variable and the system as $a_{i}(t)=\pi_{i,1}(t), \dot{\pi}_{i,1}(t)=\pi_{i,2}(t),\dots,\dot{\pi}_{i,m-i}(t)=\pi_{i,m+1-i}(t)$ and make $\dot{\pi}_{i,m+1-i}(t)=\nu_{i},$ then we can define many specific HOCBFs $h_{i}$ to enable $a_{i}(t)$ to be positive. 

Given a function $h_{i}:\mathbb{R}^{m+1-i}\to\mathbb{R},$ we can define a sequence of functions $\varphi_{i,j}:\mathbb{R}^{m+1-i}\to\mathbb{R}, i \in\{1,...,m\}, j \in\{1,...,m+1-i\}:$
\begin{equation}
\label{eq:virtual-HOCBFs}
\varphi_{i,j}(\boldsymbol{\pi}_{i})\coloneqq\dot{\varphi}_{i,j-1}(\boldsymbol{\pi}_{i})+\alpha_{i,j}(\varphi_{i,j-1}(\boldsymbol{\pi}_{i})),
\end{equation}
where $\varphi_{i,0}(\boldsymbol{\pi}_{i})\coloneqq h_{i}(\boldsymbol{\pi}_{i}),$ $\alpha_{i,j}(\cdot)$ are $(m+1-i-j)^{th}$ order differentiable class $\kappa$ functions. Sets $\mathcal{B}_{i,j}$ are defined as
\begin{equation}
\label{eq:virtual-sets}
\mathcal B_{i,j}\coloneqq \{\boldsymbol{\pi}_{i}\in\mathbb{R}^{m+1-i}:\varphi_{i,j}(\boldsymbol{\pi}_{i})>0\}, \ j\in \{0,...,m-i\}. 
\end{equation}
Let $\varphi_{i,j}(\boldsymbol{\pi}_{i}),\ j\in \{1,...,m+1-i\}$ and $\mathcal B_{i,j},\ j\in \{0,...,m-i\}$ be defined by \eqref{eq:virtual-HOCBFs} and \eqref{eq:virtual-sets} respectively. By Def. \ref{def:HOCBF}, a function $h_{i}:\mathbb{R}^{m+1-i}\to\mathbb{R}$ is a HOCBF with relative degree $m+1-i$ for system \eqref{eq:virtual-system} if there exist class $\kappa$ functions $\alpha_{i,j},\ j\in \{1,...,m+1-i\}$ as in \eqref{eq:virtual-HOCBFs} such that
\begin{small}
\begin{equation}
\label{eq:highest-SHOCBF}
\begin{split}
\sup_{\nu_{i}\in \mathbb{R}}[L_{F_{i}}^{m+1-i}h_{i}(\boldsymbol{\pi}_{i})+L_{G_{i}}L_{F_{i}}^{m-i}h_{i}(\boldsymbol{\pi}_{i})\nu_{i}+O_{i}(h_{i}(\boldsymbol{\pi}_{i}))\\
+ \alpha_{i,m+1-i}(\varphi_{i,m-i}(\boldsymbol{\pi}_{i}))] \ge \epsilon,
\end{split}
\end{equation}
\end{small}
$\forall\boldsymbol{\pi}_{i}\in \mathcal B_{i,0}\cap,...,\cap \mathcal B_{i,m-i}$. $O_{i}(\cdot)=\sum_{j=1}^{m-i}L_{F_{i}}^{j}(\alpha_{i,m-i}\circ\varphi_{i,m-1-i})(\boldsymbol{\pi}_{i}) $ where $\circ$ denotes the composition of functions. $\epsilon$ is a positive constant which can be infinitely small. 

\begin{remark}
\label{rem:safety-guarantee-2}
If $h_{i}(\boldsymbol{\pi}_{i})$ is a HOCBF illustrated above and $\boldsymbol{\pi}_{i}(0) \in \mathcal {B}_{i,0}\cap \dots \cap \mathcal {B}_{i,m-i},$ then satisfying constraint in \eqref{eq:highest-SHOCBF} is equivalent to making $\varphi_{i,m+1-i}(\boldsymbol{\pi}_{i}(t))\ge \epsilon>0, \forall t\ge 0.$ Based on
\eqref{eq:virtual-HOCBFs}, since $\boldsymbol{\pi}_{i}(0) \in \mathcal {B}_{i,m-i}$ (i.e., $\varphi_{i,m-i}(\boldsymbol{\pi}_{i}(0))>0),$ then we have $\varphi_{i,m-i}(\boldsymbol{\pi}_{i}(t))>0$ (If there exists a $t_{1}\in (0,t_{2}]$, which makes $\varphi_{i,m-i}(\boldsymbol{\pi}_{i}(t_{1}))=0,$ then we have $\dot{\varphi}_{i,m-i}((\boldsymbol{\pi}_{i}(t_{1}))>0\Leftrightarrow \varphi_{i,m-i}(\boldsymbol{\pi}_{i}(t_{1}^{-}))\varphi_{i,m-i}(\boldsymbol{\pi}_{i}(t_{1}^{+}))<0,$ which is against the definition of $\alpha_{i,m+1-i}(\cdot),$ therefore $\forall t_{1}>0, \varphi_{i,m-i}(\boldsymbol{\pi}_{i}(t_{1}))>0,$ note that $t_{1}^{-},t_{1}^{+}$ denote the left and right limit). Based on \eqref{eq:virtual-HOCBFs}, since $\boldsymbol{\pi}_{i}(0) \in \mathcal {B}_{i,m-1-i},$ then similarly we have $\varphi_{i,m-1-i}(\boldsymbol{\pi}_{i}(t))>0,\forall t\ge 0.$ Repeatedly, we have $\varphi_{i,0}(\boldsymbol{\pi}_{i}(t))>0,\forall t\ge 0,$ therefore the sets $\mathcal {B}_{i,0},\dots,\mathcal {B}_{i,m-i}$ are forward invariant.
\end{remark}

For simplicity, we can make $h_{i}(\boldsymbol{\pi}_{i})=\pi_{i,1}(t)=a_{i}(t).$ Based on Rem. \ref{rem:safety-guarantee-2}, each $a_{i}(t)$ will be positive.

The remaining question is how to define an adaptive HOCBF to guarantee $b(\boldsymbol{x})\ge0$ with the assistance of auxiliary variables. Let $\boldsymbol{\Pi}(t)\coloneqq [\boldsymbol{\pi}_{1}(t),\dots,\boldsymbol{\pi}_{m}(t)]^{T}$ and $\boldsymbol{\nu}\coloneqq [\nu_{1},\dots,\nu_{m}]^{T}$ denote the auxiliary states and control inputs of system \eqref{eq:virtual-system}. We can define a sequence of functions 
\begin{small}
\begin{equation}
\label{eq:AVBCBF-sequence}
\begin{split}
&\psi_{0}(\boldsymbol{x},\boldsymbol{\Pi}(t))\coloneqq a_{1}(t)b(\boldsymbol{x}),\\
&\psi_{i}(\boldsymbol{x},\boldsymbol{\Pi}(t))\coloneqq a_{i+1}(t)(\dot{\psi}_{i-1}(\boldsymbol{x},\boldsymbol{\Pi}(t))+\alpha_{i}(\psi_{i-1}(\boldsymbol{x},\boldsymbol{\Pi}(t)))),
\end{split}
\end{equation}
\end{small}
where $i \in \{1,...,m-1\}, \psi_{m}(\boldsymbol{x},\boldsymbol{\Pi}(t))\coloneqq \dot{\psi}_{m-1}(\boldsymbol{x},\boldsymbol{\Pi}(t))+\alpha_{m}(\psi_{m-1}(\boldsymbol{x},\boldsymbol{\Pi}(t))).$ We further define a sequence of sets $\mathcal{C}_{i}$ associated with \eqref{eq:AVBCBF-sequence} in the form 
\begin{equation}
\label{eq:AVBCBF-set}
\begin{split}
\mathcal C_{i}\coloneqq \{(\boldsymbol{x},\boldsymbol{\Pi}(t)) \in \mathbb{R}^{n} \times \mathbb{R}^{m}:\psi_{i}(\boldsymbol{x},\boldsymbol{\Pi}(t))\ge 0\}, 
\end{split}
\end{equation}
where $i \in \{0,...,m-1\}.$
Since $a_{i}(t)$ is a HOCBF with relative degree $m+1-i$ for \eqref{eq:virtual-system}, based on \eqref{eq:highest-SHOCBF}, we define a constraint set $\mathcal{U}_{\boldsymbol{a}}$ for $\boldsymbol{\nu}$ as 
\begin{small}
\begin{equation}
\label{eq:constraint-up}
\begin{split}
\mathcal{U}_{\boldsymbol{a}}(\boldsymbol{\Pi})\coloneqq \{\boldsymbol{\nu}\in\mathbb{R}^{m}:   L_{F_{i}}^{m+1-i}a_{i}+[L_{G_{i}}L_{F_{i}}^{m-i}a_{i}]\nu_{i}\\
+O_{i}(a_{i})+ \alpha_{i,m+1-i}(\varphi_{i,m-i}(a_{i})) \ge \epsilon, i\in \{1,\dots,m\}\},
\end{split}
\end{equation}
\end{small}
where $\varphi_{i,m-i}(\cdot)$ is defined similar to \eqref{eq:virtual-HOCBFs} and $a_{i}(t)$ is ensured positive. $\epsilon$ is a positive constant which can be infinitely small. 

\begin{definition}[AVCBF]
\label{def:AVBCBF}
Let $\psi_{i}(\boldsymbol{x},\boldsymbol{\Pi}(t)),\ i\in \{1,...,m\}$ be defined by \eqref{eq:AVBCBF-sequence} and $\mathcal C_{i},\ i\in \{0,...,m-1\}$ be defined by \eqref{eq:AVBCBF-set}. A function $b(\boldsymbol{x}):\mathbb{R}^{n}\to\mathbb{R}$ is an Auxiliary-Variable Adaptive Control Barrier Function (AVCBF) with relative degree $m$ for system \eqref{eq:affine-control-system} if every $a_{i}(t),i\in \{1,...,m\}$ is a
HOCBF with relative degree $m+1-i$ for the auxiliary system
\eqref{eq:virtual-system}, and there exist $(m-j)^{th}$ order differentiable class $\kappa$ functions $\alpha_{j},j\in \{1,...,m-1\}$
and a class $\kappa$ functions $\alpha_{m}$ s.t.
\begin{small}
\begin{equation}
\label{eq:highest-AVBCBF}
\begin{split}
\sup_{\boldsymbol{u}\in \mathcal{U},\boldsymbol{\nu}\in \mathcal{U}_{\boldsymbol{a}}}[\sum_{j=2}^{m-1}[(\prod_{k=j+1}^{m}a_{k})\frac{\psi_{j-1}}{a_{j}}\nu_{j}] + \frac{\psi_{m-1}}{a_{m}}\nu_{m} \\ +(\prod_{i=2}^{m}a_{i})b(\boldsymbol{x})\nu_{1} +(\prod_{i=1}^{m}a_{i})(L_{f}^{m}b(\boldsymbol{x})+L_{g}L_{f}^{m-1}b(\boldsymbol{x})\boldsymbol{u})\\+R(b(\boldsymbol{x}),\boldsymbol{\Pi})
+ \alpha_{m}(\psi_{m-1})] \ge 0,
\end{split}
\end{equation}
\end{small}
$\forall (\boldsymbol{x},\boldsymbol{\Pi})\in \mathcal C_{0}\cap,...,\cap \mathcal C_{m-1}$ and each $a_{i}>0, i\in\{1,\dots,m\}.$ In \eqref{eq:highest-AVBCBF}, $R(b(\boldsymbol{x}),\boldsymbol{\Pi})$ denotes the remaining Lie derivative terms of $b(\boldsymbol{x})$ (or $\boldsymbol{\Pi}$) along $f$ (or $F_{i},i\in\{1,\dots,m\}$) with degree less than $m$ (or $m+1-i$), which is similar to the form of $O(\cdot )$ in \eqref{eq:highest-HOCBF}.
\end{definition}

\begin{theorem}
\label{thm:safety-guarantee-3}
Given an AVCBF $b(\boldsymbol{x})$ from Def. \ref{def:AVBCBF} with corresponding sets $\mathcal{C}_{0}, \dots,\mathcal {C}_{m-1}$ defined by \eqref{eq:AVBCBF-set}, if $(\boldsymbol{x}(0),\boldsymbol{\Pi}(0)) \in \mathcal {C}_{0}\cap \dots \cap \mathcal {C}_{m-1},$ then if there exists solution of Lipschitz controller $(\boldsymbol{u},\boldsymbol{\nu})$ that satisfies the constraint in \eqref{eq:highest-AVBCBF} and also ensures $(\boldsymbol{x},\boldsymbol{\Pi})\in \mathcal {C}_{m-1}$ for all $t\ge 0,$ then $\mathcal {C}_{0}\cap \dots \cap \mathcal {C}_{m-1}$ will be rendered forward invariant for system \eqref{eq:affine-control-system}, $i.e., (\boldsymbol{x},\boldsymbol{\Pi}) \in \mathcal {C}_{0}\cap \dots \cap \mathcal {C}_{m-1}, \forall t\ge 0.$ Moreover, $b(\boldsymbol{x})\ge 0$ is ensured for all $t\ge 0.$
\end{theorem}

\begin{proof}
If $b(\boldsymbol{x})$ is an AVCBF that is $m^{th}$ order differentiable, then satisfying constraint in \eqref{eq:highest-AVBCBF} while ensuring $(\boldsymbol{x},\boldsymbol{\Pi})\in \mathcal {C}_{m-1}$ for all $t\ge 0$ is equivalent to make $\psi_{m-1}(\boldsymbol{x},\boldsymbol{\Pi})\ge 0, \forall t\ge 0.$ Since $a_{m}(t)>0$, we have $\frac{\psi_{m-1}(\boldsymbol{x},\boldsymbol{\Pi})}{a_{m}(t)}\ge 0.$ Based on
\eqref{eq:AVBCBF-sequence}, since $(\boldsymbol{x}(0),\boldsymbol{\Pi}(0)) \in \mathcal {C}_{m-2}$ (i.e., $\frac{\psi_{m-2}(\boldsymbol{x}(0),\boldsymbol{\Pi}(0))}{a_{m-1}(0)}\ge 0),a_{m-1}(t)>0,$ then we have $\psi_{m-2}(\boldsymbol{x},\boldsymbol{\Pi})\ge 0$ (The proof of this is similar to the proof in Rem. \ref{rem:safety-guarantee-2}), and also $\frac{\psi_{m-2}(\boldsymbol{x},\boldsymbol{\Pi})}{a_{m-1}(t)}\ge 0.$ Based on \eqref{eq:AVBCBF-sequence}, since $(\boldsymbol{x}(0),\boldsymbol{\Pi}(0)) \in \mathcal {C}_{m-3},a_{m-2}(t)>0$ then similarly we have $\psi_{m-3}(\boldsymbol{x},\boldsymbol{\Pi})\ge 0$ and $\frac{\psi_{m-3}(\boldsymbol{x},\boldsymbol{\Pi})}{a_{m-2}(t)}\ge 0,\forall t\ge 0.$ Repeatedly, we have $\psi_{0}(\boldsymbol{x},\boldsymbol{\Pi})\ge 0$ and $\frac{\psi_{0}(\boldsymbol{x},\boldsymbol{\Pi})}{a_{1}(t)}\ge 0,\forall t\ge 0.$ Therefore the sets $\mathcal {C}_{0},\dots,\mathcal {C}_{m-1}$ are forward invariant and $b(\boldsymbol{x})=\frac{\psi_{0}(\boldsymbol{x},\boldsymbol{\Pi})}{a_{1}(t)}\ge 0$ is ensured for all $t\ge 0$.
\end{proof}
Based on Thm. \ref{thm:safety-guarantee-3}, the safety regarding $b(\boldsymbol{x})=\frac{\psi_{0}(\boldsymbol{x},\boldsymbol{\Pi})}{a_{1}(t)}\ge 0$ is guaranteed.

\begin{remark}[Limitation of Approaches with Auxiliary Inputs]
\label{rem: PACBF-AVBCBF} 
Ensuring the satisfaction of the $i^{th}$ order AVCBF constraint as shown in \eqref{eq:AVBCBF-set} when $i\in\{1,\dots,m-1\},$ i.e., $\psi_{i}(\boldsymbol{x},\boldsymbol{\Pi})\ge 0$ will guarantee $\psi_{i-1}(\boldsymbol{x},\boldsymbol{\Pi})\ge 0$ based on the proof of Thm. \ref{thm:safety-guarantee-3}, which theoretically outperforms PACBF. However, both approaches can not ensure satisfying $\psi_{m}(\boldsymbol{x},\boldsymbol{\Pi})\ge 0$ will guarantee $\psi_{m-1}(\boldsymbol{x},\boldsymbol{\Pi})\ge 0$ since the growth of $\boldsymbol{\nu}_{i}$ is unbounded. Therefore in Thm. \ref{thm:safety-guarantee-3}, $(\boldsymbol{x},\boldsymbol{\Pi})\in \mathcal {C}_{m-1}$ for all $t\ge 0$ also needs to be satisfied to guarantee the forward invariance of the intersection of sets. 
\end{remark}

\subsection{Optimal Control with AVCBFs}
\label{subsec: optimal-control}
Consider an optimal control problem as
\begin{small}
\begin{equation}
\label{eq:cost-function-1}
\begin{split}
 \min_{\boldsymbol{u}} \int_{0}^{T} 
 D(\left \| \boldsymbol{u} \right \| )dt,
\end{split}
\end{equation}
\end{small}
where $\left \| \cdot \right \|$ denotes the 2-norm of a vector, $D(\cdot)$ is a strictly increasing function of its argument and $T>0$ denotes the ending time. Since we need to introduce auxiliary inputs $v_{i}$ to enhance the feasibility of optimization, we should reformulate the cost in \eqref{eq:cost-function-1} as
\begin{small}
\begin{equation}
\label{eq:cost-function-2}
\begin{split}
 \min_{\boldsymbol{u},\boldsymbol{\nu}} \int_{0}^{T} 
 [D(\left \| \boldsymbol{u} \right \| )+\sum_{i=1}^{m}W_{i}(\nu_{i}-a_{i,w})^{2}]dt.
\end{split}
\end{equation}
\end{small}
In \eqref{eq:cost-function-2}, $W_{i}$ is a positive scalar and $a_{i,w}\in \mathbb{R}$ is the scalar to which we hope each auxiliary input $\nu_{i}$ converges. Both are chosen to tune the performance of the controller. We can formulate the CLFs, HOCBFs and AVCBFs introduced in Def. \ref{def:control-l-f}, Sec. \ref{sec:AVCBFs} and Def. \ref{def:AVBCBF} as constraints of the QP with cost function \eqref{eq:cost-function-2} to realize safety-critical control. Next we will show AVCBFs can be used to enhance the feasibility of solving QP compared with classical HOCBFs in Def. \ref{def:HOCBF}.

In auxiliary system \eqref{eq:virtual-system}, if we define $a_{i}(t)=\pi_{i,1}(t)=1, \dot{\pi}_{i,1}(t)=\dot{\pi}_{i,2}(t)=\cdots=\dot{\pi}_{i,m+1-i}(t)=0,$ then the way we construct functions and sets in \eqref{eq:virtual-HOCBFs} and \eqref{eq:virtual-sets} are exactly the same as \eqref{eq:sequence-f1} and \eqref{eq:sequence-set1}, which means classical HOCBF is in fact one specific case of AVCBF. Assume that the highest order HOCBF constraint \eqref{eq:highest-HOCBF} conflicts with control input constraints \eqref{eq:control-constraint} at $t=t_{b},$ i.e., we can not find a feasible controller $u(t_{b})$ to satisfy \eqref{eq:highest-HOCBF} and \eqref{eq:control-constraint}. Instead, starting from a time slot $t=t_{a}$ which is just before $t=t_{b}$ ($t_{b}-t_{a}=\varepsilon$ where $\varepsilon$ is an infinitely small positive value), we exchange the control framework of classical HOCBF into AVCBF instantly. Suppose we can find appropriate hyperparameters to ensure two constraints in \eqref{eq:constraint-up} and \eqref{eq:highest-AVBCBF}
% \begin{small}
% \begin{equation}
% \label{eq:constraint-fea-12}
% \begin{split}
%  \nu_{i}
%   > \frac{-L_{F_{i}}^{m+1-i}a_{i}-O_{i}(a_{i})-\alpha_{i,m+1-i}(\varphi_{i,m-i}(a_{i}))}{L_{G_{i}}L_{F_{i}}^{m-i}a_{i}},\\
%   \sum_{j=2}^{m-1}[(\prod_{k=j+1}^{m}a_{k})\frac{\psi_{j-1}}{a_{j}}\nu_{j}] + \frac{\psi_{m-1}}{a_{m}}\nu_{m} +(\prod_{i=2}^{m}a_{i})b(\boldsymbol{x})\nu_{1} \\ \ge -(\prod_{i=1}^{m}a_{i})(L_{f}^{m}b(\boldsymbol{x})+L_{g}L_{f}^{m-1}b(\boldsymbol{x})\boldsymbol{u})-R(b(\boldsymbol{x}),\boldsymbol{\Pi}) \\
% - \alpha_{m}(\psi_{m-1}),  i\in \{1,\dots,m\}
% \end{split}
% \end{equation}
% \end{small}
are satisfied given $\boldsymbol{u}$ constrained by \eqref{eq:control-constraint} at $t_{b},$ then there exists solution $\boldsymbol{u}(t_{b})$ for the optimal control problem and the feasibility of solving QP is enhanced. Relying on AVCBF, We can discretize the whole time period $[0,T]$ into several small time intervals like $[t_{a},t_{b}]$ to maximize the feasibility of solving QP under safety constraints, which calls for the development of automatic parameter-tuning techniques in future.
% \begin{theorem}
% \label{thm:feasibility-guarantee}
% Given an AVCBF $b(\boldsymbol{x})$ from Def. \ref{def:AVBCBF} with corresponding sets $\mathcal{C}_{0}, \dots,\mathcal {C}_{m-1}$ defined by \eqref{eq:AVBCBF-set}, if $(\boldsymbol{x}(0),\boldsymbol{\Pi}(0)) \in \mathcal {C}_{0}\cap \dots \cap \mathcal {C}_{m-1}$ and $L_{G_{i}}L_{F_{i}}^{m-i}a_{i}>0,a_{i}(t)>0, i\in\{1,\dots,m\}$ in \eqref{eq:constraint-up}, then if there exists solution of Lipschitz controller $(\boldsymbol{u},\boldsymbol{\nu})$ that satisfies the constraint in \eqref{eq:highest-AVBCBF} and also ensures $\psi_{0}>0,\dots,\psi_{s}>0,s\in \{0,\dots,m-1\}$ in \eqref{eq:AVBCBF-set}, then the QP with cost function \eqref{eq:cost-function-2} and constraints \eqref{eq:control-constraint},\eqref{eq:AVBCBF-set}-\eqref{eq:highest-AVBCBF} is guranteed to be feasible.
% \end{theorem}

% \begin{proof}
% Rewrite the constraint \eqref{eq:constraint-up} as 
% \begin{equation}
% \label{eq:constraint-fea-1}
% \begin{split}
%  \nu_{i}
%   > \frac{-L_{F_{i}}^{m+1-i}a_{i}-O_{i}(a_{i})-\alpha_{i,m+1-i}(\varphi_{i,m-i}(a_{i}))}{L_{G_{i}}L_{F_{i}}^{m-i}a_{i}},
% \end{split}
% \end{equation}
% where $i\in \{1,\dots,m\}.$ Rewrite the constraint \eqref{eq:highest-AVBCBF} as
% \begin{equation}
% \label{eq:constraint-fea-2}
% \begin{split}
% \sum_{j=2}^{m-1}[(\prod_{k=j+1}^{m}a_{k})\frac{\psi_{j-1}}{a_{j}}\nu_{j}] + \frac{\psi_{m-1}}{a_{m}}\nu_{m} +(\prod_{i=2}^{m}a_{i})b(\boldsymbol{x})\nu_{1} \\ \ge -(\prod_{i=1}^{m}a_{i})(L_{f}^{m}b(\boldsymbol{x})+L_{g}L_{f}^{m-1}b(\boldsymbol{x})\boldsymbol{u})-R(b(\boldsymbol{x}),\boldsymbol{\Pi}) \\
% - \alpha_{m}(\psi_{m-1}),  i\in \{1,\dots,m\}.
% \end{split}
% \end{equation}
% Since $L_{G_{i}}L_{F_{i}}^{m-i}a_{i}>0$ in \eqref{eq:constraint-fea-1}, $\psi_{0}>0,\dots,\psi_{s}>0,s\in \{0,\dots,m-1\}$ in \eqref{eq:constraint-fea-2} and $a_{1}>0,\dots,a_{m}>0,$ we have $(\prod_{i=2}^{m}a_{i})b(\boldsymbol{x})>0,(\prod_{k=j+1}^{m}a_{k})\frac{\psi_{j-1}}{a_{j}}\nu_{j}>0,j\in \{2,\dots,s\}$ are always positive, 
% then there always exist large enough $\nu_{1},\dots,\nu_{s}$ satisfying constraints above {\color{red}you are assuming a very specific (13).} (the upper bounds of $\nu_{1},\dots,\nu_{s}$ are unlimited), hence the feasibility of QP with cost function \eqref{eq:cost-function-2} and constraints \eqref{eq:control-constraint},\eqref{eq:AVBCBF-set}-\eqref{eq:highest-AVBCBF} is guaranteed.  {\color{red}Control limitations (2) are the most critical factor in the feasibility. You completely ignore this. The proof is very sloppy.}
% \end{proof}

Besides safety and feasibility, another benefit of using AVCBFs is that the conservativeness of the control strategy can also be ameliorated. For example, from \eqref{eq:AVBCBF-sequence}, we can rewrite $\psi_{i}(\boldsymbol{x},\boldsymbol{\Pi})\ge 0$ as
\begin{equation}
\label{eq:AVCBF-rewrite}
\begin{split}
\dot{\phi}_{i-1}(\boldsymbol{x},\boldsymbol{\Pi})+k_{i}(1+\frac{\dot{a}_{i}(t)}{k_{i}a_{i}(t)}) \phi_{i-1}(\boldsymbol{x},\boldsymbol{\Pi})\ge0,
\end{split}
\end{equation}
where $\phi_{i-1}(\boldsymbol{x},\boldsymbol{\Pi})=\frac{\psi_{i-1}(\boldsymbol{x},\boldsymbol{\Pi})}{a_{i}(t)},\alpha_{i}(\psi_{i-1}(\boldsymbol{x},\boldsymbol{\Pi}))=k_{i}a_{i}(t)\phi_{i-1}(\boldsymbol{x},\boldsymbol{\Pi}), k_{i}>0, i\in \{1,\dots,m\}.$ Similar to PACBFs, we require $1+\frac{\dot{a}_{i}(t)}{k_{i}a_{i}(t)}\ge0,$ which gives us $\dot{a}_{i}(t)+k_{i}a_{i}(t)\ge0.$
The term $\frac{\dot{a}_{i}(t)}{a_{i}(t)}$ can be adjusted adaptable  to ameliorate the conservativeness of control strategy that $k_{i}\phi_{i-1}(\boldsymbol{x},\boldsymbol{\Pi})$ may have, i.e., the ego vehicle can brake earlier or later given time-varying control constraint $\boldsymbol{u}_{min}(t)\le \boldsymbol{u} \le\boldsymbol{u}_{max}(t),$ which confirms the adaptivity of AVCBFs to control constraint and conservativeness of control strategy. 

\begin{remark}[Parameter-Tuning for AVCBFs]
\label{rem: parameter-tuning}
Based on the analysis of \eqref{eq:AVCBF-rewrite}, we require $\dot{a}_{i}(t)+k_{i}a_{i}(t)\ge0.$ If we define first order HOCBF constraint for $a_{i}(t)>0$ as $\dot{a}_{i}(t)+l_{i}a_{i}(t)\ge0,$ we should choose hyperparameter $l_{i}\le k_{i}$ to guarantee $\dot{a}_{i}(t)+k_{i}a_{i}(t)\ge\dot{a}_{i}(t)+l_{i}a_{i}(t)\ge 0.$ For simplicity, we can use $l_{i}=k_{i}.$ In cost function \eqref{eq:cost-function-2}, we can tune hyperparameters $W_{i}$ and $a_{i,w}$ to adjust the corresponding ratio $\frac{\dot{a}_{i}(t)}{a_{i}(t)}$ to change the performance of the optimal controller.
\end{remark}

\begin{remark}
\label{rem: sufficient-con}
Note that the satisfaction of the constraint in \eqref{eq:highest-AVBCBF} is a sufficient condition for the satisfaction of the original constraint $\psi_{0}(\boldsymbol{x},\boldsymbol{\Pi})>0,$ it is not necessary to introduce auxiliary variables as many as from $a_{1}(t)$ to $a_{m}(t),$ which allows us to choose an appropriate
number of auxiliary variables for the AVCBF constraints to reduce the complexity. In other words, the number of auxiliary variables can be less than or equal to the relative degree $m$.
\end{remark}
\subsection{Sampling Algorithm} 
Sampling directly from the joint distributions of $Y, B$ is challenging. We propose to use Gibbs sampling to sample from the target distribution by alternatively sampling from $P(Y | X, B)$ and $P(B | X, Y)$. First, we describe how our proposed algorithm samples from each conditional distribution. We then introduce the complete sampling algorithm along with some intuition as to how it ensures satisfactory and fluent generations. 

\paragraph{Sampling from $P(B | X, Y)$} 
The conditional distribution for $P(B | X, Y)$ includes the term $P^{LM}(Y | X, B)$. This is to ensure that the sampled $B$ results in output sequences that are high in likelihood under the base model distribution. 
However, directly computing \( P^{LM}(Y | X, B) \) for all possible values of \( B \) is intractable. 
By noting that this term encourages the selection of \( B \) that is consistent with the observed \( Y \), we can infer this property is naturally satisfied when $B$ is close to $Y$. 
Thus, we approximate \( P^{LM}(Y | X, B) \) by performing a single MCMC step with the initial state set to $Y$. 

In order to sample this discrete sequence of tokens, we apply the Discrete Langevin Proposal (DLP) introduced in \citep{zhang2022langevinlike}. 
After initializing $B$ as $B = Y$, and representing the sequence as a sequence of one-hot vectors $\hat{B} = \{\hat{b}_1, \hat{b_2} \dots \hat{b}_n\}$, we execute a single step of DLP with the target distribution being $\exp(f(B | X))$. Below we include the proposal distribution for position $i$:
\begin{align}
\label{eq:dlp_prop}
    b'_i \sim \categorical\left(\underset{j \in |V|}{\softmax} \left( \frac{1}{\tau} (\nabla f(\hat{B} | X))_{i,j} (1 - \hat{b}_{i,j}) \right) \right)
\end{align}
Here, $\tau$ is a temperature hyper-parameter that controls the sharpness of the proposal distribution, $(\nabla f(\hat{B} | X))_{i,j}$ is the $j$th component of the $i$th gradient vector, $\hat{b}_{i,j}$ represents the $j$th component of the one-hot vector $\hat{b}_i$, and $b'_i$ is the token we sample from the distribution over $V$. 
For more details on the application of DLP and the gradient compution, see Appendix \ref{appndx:dlp_proposal}.
Unlike the algorithm presented in \citet{liu2023bolt}, we do not require the use of straight through estimation (STE) \citep{bengio2013estimating} as we differentiate directly with respect to $Y$.
Note that this proposal function can be computed for all sequence positions in parallel. 
We refer to this proposal function as $q_\tau(\cdot | B)$. 

\paragraph{Sampling from $P(Y | X, B)$}
Our goal is to sample from $P(Y | X, B)$ using biased auto-regressive generation, similar to \eqref{eq:bolt-auto-reg}. 
In order to do so, we must map the sequence of bias \textit{tokens} $B$ to a sequence of bias \textit{vectors} $\tilde{B}$.  
The ideal bias vector should reflect the difference in meaning between each token in the vocabulary space $V$ and the sampled token.
To accomplish this, we penalize each token based on the distance to the sampled bias token within the embedding space, as static embeddings reflect semantic meaning \citep{mikolov2013efficient, pennington2014glove, mikolov2013distributed}. Given a bias token $b_i$, embedding table $M$, we define the $j$th coordinate value corresponding to token $v_j$ as follows: 
\begin{align}
\label{eq:bias-vec-def}
    \tilde{b}_{i,j} = || Mb_i - Mv_j||^2_2 
\end{align}
This yields a $|V|$ dimensional vector that can be added to the auto-regressive logits $\tilde{y_i}$. 
When adding the bias term to $\tilde{y_i}$, we also incorporate both a weight term $w_i$ and a normalizing factor $r_i$. While $w_i$ is a hyper-parameter, $r_i$ normalizes the bias vector at the $i$ position to have the same norm as $\tilde{y_i}$. We define the normalizing factor as follows: 
\begin{align}
    \label{eq:dab-normalize}
    r_i = \frac{|| \tilde{y_i} ||_2}{|| \tilde{b}_i ||_2}
\end{align}
We note that while this normalizing factor can also be applied to BOLT and may improve its results, the modified BOLT still underperforms compared to our method. 

We formally define our biased auto-regressive generation as follows: 
\begin{align}
    \label{eq:dab-auto-reg}
    y_i = \argmax_{j \in |V|} \left( \tilde{y}_{i, j} - w_i \cdot r_i \cdot \tilde{b}_{i, j}\right). 
\end{align}
Intuitively, this returns the token corresponding to the maximal coordinate of the biased distribution. Repeating this $n$ times results in the updated response sequence $Y$. 
\begin{figure}[t]
    \centering
    \resizebox{12 cm}{!}{
        \input{Main_Body/diagram}
        }
    \caption{Visualization of the proposed decoding algorithm, DAB. DAB alternates between sampling the response $Y$ and the bias $B$. To sample $B$ given $Y$, we use gradient-based discrete sampling on the constraint function $f$. To sample $Y$ given $B$, we compute a bias vector that penalizes words based on their distance to $B$ and then use this bias to guide the auto-regressive generation.
}
\label{fig:diagram}
\end{figure}

\paragraph{Text Generation Algorithm} We provide a visualization of our algorithm in Figure \ref{fig:diagram}, and include the full algorithm in Appendix \ref{appndx:algrthm-details}. Given some prompt, we first generate some initial auto-regressive generation $Y_1$, with the initial bias vector set to $\Vec{0}$. After obtaining $Y$, we sample from the conditional distribution over $B$ to obtain a sequence of bias tokens. We then use \eqref{eq:bias-vec-def} to compute the new bias vector to use for biased auto-regressive generation. 
We repeat this alternative sampling process for several iterations, returning the sample that best satisfies the constraint at the end as commonly done in the literature \citep{kumar2022gradient,liu2023bolt}. 
For a discussion on the hyper-parameters of our algorithm, see Appendix \ref{appndx:ablation}. 
% \begin{algorithm}
    \caption{Discrete Autoregressive Biasing}
    \begin{algorithmic}[1]
    \REQUIRE Constraint function $f$, $P^{LM}$, prompt $X$, number steps $s$, sequence length $n$, embedding table $M$
    \STATE $\tilde{B} \gets \vec{0}, f_\text{min} \gets -\infty$, $Y_\text{best} \gets \{\}$ \LineComment{Initialize constraint violation as being maximal and current best generation as empty}
    \FOR{step $s$}
        \FOR{position $i$ in range($n$)} 
            \STATE $\tilde{y_i} \gets \log P^{LM} (\cdot | y_{<i}, X)$ \LineComment{Initial auto-regressive distribution over $V$}
            \STATE Calculate normalizing factor $r_i$ if $s > 1$, else $r_i \gets 1$
            \STATE $y_i \gets \text{argmax}_{j \in |V|} \left(\tilde{y}_{i, j} - w_i \cdot r_i \cdot \tilde{b}_{i, j} \right)$ \LineComment{Sample from $P(Y | X, B)$}
        \ENDFOR
        \STATE $B \gets Y$ \LineComment{Initialize $B$ as $Y$}
        \STATE Evaluate $f(B | X)$, update $f_\text{min}$, $Y_\text{best}$
        \STATE $B' \sim q_\tau(\cdot | B)$ as in \eqref{eq:dlp_prop} \LineComment{Approximately sample from $P(B | X, Y)$}
        \STATE Compute $\tilde{B}$ as in \eqref{eq:bias-vec-def}
    \ENDFOR
    \STATE return $Y_\text{best}$
    \end{algorithmic}
\label{alg:text-gen}
\end{algorithm}
\subsection{Advantages of Biasing in Discrete Spaces}
Here we discuss various advantages of discrete sampling in the context of auto-regressive biasing. First, we demonstrate that discrete sampling enables a quicker and more thorough exploration of potential output sequences $Y$. We then describe how discrete sampling solves the stability issue discussed in \citet{liu2023bolt}. Finally, we show that discrete sampling makes use of simpler gradient computations, resulting in a more efficient decoding algorithm.

\paragraph{Exploration of State Space}
Discrete sampling enables DAB to explore the output space more effectively than continuous methods.
We hypothesize that discrete sampling enables more directional and substantial changes to the bias vector, resulting in more token changes in the output sequence across sampling steps. 
We compare with BOLT, a continuous auto-regressive biasing algorithm \citep{liu2023bolt}. 
We examine the performance of BOLT both with and without the normalizing factor defined in \eqref{eq:dab-normalize}. 
We include the comparison of hops across 50 steps in Figure \ref{fig:samplespace_explore}a.
These results show that our method updates substantially more sequence positions across all sampling steps than either variant of BOLT. 

Next, we measure how comprehensively each method explores the sample space of potential sequences. 
For each sequence position, we maintain a record of tokens encountered throughout the sampling process and compute the number of unique tokens within this set. 
Figure \ref{fig:samplespace_explore}b shows the average unique tokens per sequence position for all three algorithms. 
These results indicate that our method samples more unique tokens for each sequence position than either variant of BOLT, demonstrating more comprehensive exploration. 
Collectively, these findings confirm that discrete sampling enables faster, more thorough, and thus more effective exploration of the sample space of potential sequences. 
\paragraph{Sampling Stability}
Discrete sampling allows DAB to have superior stability across sampling steps when compared to continuous methods. 
We show this in Figure \ref{fig:samplespace_explore}c, where we track the average perplexity of the batch at each time step. 
While BOLT faces deteriorating perplexity, DAB remains stable throughout the sampling process. 

We attribute this instability to the difficulty of applying continuous sampling techniques to a discrete domain as discussed in \citet{grathwohl2021gwg}. 
As a result of BOLT's misalignment between the sampling domain and target domain, the energy landscape is too complex to navigate with gradient information.
This results in the divergence seen in Figure \ref{fig:samplespace_explore}c. 

\begin{figure}
    \centering
    \begin{subfigure}[t]{0.3\textwidth}
        \includegraphics[width=.95\linewidth]{Images/token_update_over_step.pdf}
        \caption{Hops per Sample Step}
    \end{subfigure}
    \begin{subfigure}[t]{0.3\textwidth}
        \includegraphics[width=.95\linewidth]{Images/total_unique_token.pdf}
        \caption{Avg. Unique Tokens}
    \end{subfigure}
        \begin{subfigure}[t]{0.3\textwidth}
        \includegraphics[width=.95\linewidth]{Images/mixing_fig.pdf}
        \caption{Perplexity per Sample Step}
    \end{subfigure}
    \caption{(a) Average hops, or token updates per sequence, against sampling steps. Both versions of BOLT suffer from decreasing hops while DAB remains stable.
    (b) Average number of the unique tokens sampled for each sequence position throughout the entire sampling process. DAB discovers many more unique tokens for each position than either variant of BOLT. 
    (c) Comparison of fluency with respect to sampling steps. Dab exhibits stable fluency over sampling steps in comparison to BOLT. 
    % While BOLT suffers from degrading fluency even after incorporating a normalizing factor, DAB exhibits stable behavior. 
    }
    \label{fig:samplespace_explore}
\end{figure}

Our algorithm avoids this entirely as we perform direct sampling on the discrete token space. 
Since we define the sampling domain and target domain to be the same, our algorithm enjoys superior stability throughout all sampling steps. 
This improvement in algorithmic stability removes the need for implementing early-stopping and to carefully tune the number of sampling steps. 
\vspace{-1em}
\begin{wraptable}[7]{l}{.5\textwidth}
\caption{Efficiency comparison between BOLT and DAB in terms of tokens per second.
}
\vspace{-0.5em}
\label{table:speed-table}
\centering
\resizebox{.4\textwidth}{!}{\begin{tabular}{lcccl}\toprule
      & BOLT & DAB (Ours) \\\midrule
\makecell{Tokens per \\ Second}  & $9.495 \pm .095$  & $\mathbf{23.213 \pm 0.304}$ \\
\bottomrule
 \end{tabular}}
\end{wraptable}
 
\vspace{-0.8em}
\paragraph{Improved Efficiency} Discrete sampling enables our algorithm to use simpler gradient computations that provide a computational advantage over continuous sampling methods. 
To evaluate our algorithm's efficiency, we compare the tokens per second of our method to BOLT. 
We also measure the time-cost for computing the bias term for each method. 
We put the results in Table \ref{table:speed-table}, where we observe that our method has over \textbf{2x} the tokens per second output when compared against BOLT. 
Our algorithm achieves this computational advantage as a result of computing the gradient with respect to $\hat{B} = \hat{Y}$, which removes the need to backpropagate through auto-regressive generation. Computing the gradient of $f$ with respect to a continuous bias term $\tilde{B}$ requires first computing $\pderiv{f}{\hat{Y}}$ and then $\pderiv{\hat{Y}}{\tilde{B}}$.
Since each one-hot vector in $\hat{Y}$ is influenced by previous bias terms, the latter term requires backpropagation through auto-regressive generation. 
Simply initializing $\tilde{B}=\hat{Y}$ will not work in continuous sampling because the incremental updates will keep $\tilde{B}$ close to the original $\hat{Y}$. 
In contrast, our method uses gradients to identify which tokens will increase constraint satisfaction and directly samples them, enabling substantial change from the original sequence while incorporating information from the external constraint. 
While continuous sampling cannot exploit this computational shortcut and maintain constraint satisfaction, gradient-based discrete sampling achieves both simultaneously. 


% Acknowledgements should only appear in the accepted version.
\ifdefined\isaccepted
\section*{Acknowledgements}

R.~Oguz Araz is supported by the pre-doctoral program AGAUR-FI ajuts (2024 FI-3 00065) Joan Oró, and the Cátedras ENIA program ``IA y Música: Cátedra en Inteligencia Artificial y Música'' (TSI-100929-2023-1).
\fi

\section*{Impact Statement}

%This paper presents work whose goal is to advance the fields of machine learning and music information retrieval. There are many potential societal consequences of our work, none of which we feel must be specifically highlighted here.

This paper presents work whose goal is to advance the fields of machine learning and music information retrieval. Musical version matching can be used to enhance music discovery, preserve cultural heritage, and support fair copyright management. By connecting versions across styles and performances, musical version matching also fosters creativity, promotes artistic appreciation, and paves the way for more equitable solutions in the music industry, benefiting society at large. As with any machine learning tool, however, there always exists the possibility of some potential misuses of itself or of some of its components, none of which we feel must be specifically highlighted here.

% Bibliography
\bibliography{mybib}
\bibliographystyle{icml2025}

%%%%%%%%%%%%%%%%%%%%%%%%%%%%%%%%%%%%%%%%%%%%%%%%%%%%%%%%%%%%%%%%%%%%%%%%%%%%%%%
%%%%%%%%%%%%%%%%%%%%%%%%%%%%%%%%%%%%%%%%%%%%%%%%%%%%%%%%%%%%%%%%%%%%%%%%%%%%%%%
% APPENDIX
%%%%%%%%%%%%%%%%%%%%%%%%%%%%%%%%%%%%%%%%%%%%%%%%%%%%%%%%%%%%%%%%%%%%%%%%%%%%%%%
%%%%%%%%%%%%%%%%%%%%%%%%%%%%%%%%%%%%%%%%%%%%%%%%%%%%%%%%%%%%%%%%%%%%%%%%%%%%%%%

\newpage
\appendix
\onecolumn

% CVPR 2025 Paper Template; see https://github.com/cvpr-org/author-kit

\documentclass[10pt,twocolumn,letterpaper]{article}

%%%%%%%%% PAPER TYPE  - PLEASE UPDATE FOR FINAL VERSION
% \usepackage{cvpr}              % To produce the CAMERA-READY version
% \usepackage[review]{cvpr}      % To produce the REVIEW version
\usepackage[pagenumbers]{cvpr} % To force page numbers, e.g. for an arXiv version

% Import additional packages in the preamble file, before hyperref
%
% --- inline annotations
%
\newcommand{\red}[1]{{\color{red}#1}}
\newcommand{\todo}[1]{{\color{red}#1}}
\newcommand{\TODO}[1]{\textbf{\color{red}[TODO: #1]}}
% --- disable by uncommenting  
% \renewcommand{\TODO}[1]{}
% \renewcommand{\todo}[1]{#1}



\newcommand{\VLM}{LVLM\xspace} 
\newcommand{\ours}{PeKit\xspace}
\newcommand{\yollava}{Yo’LLaVA\xspace}

\newcommand{\thisismy}{This-Is-My-Img\xspace}
\newcommand{\myparagraph}[1]{\noindent\textbf{#1}}
\newcommand{\vdoro}[1]{{\color[rgb]{0.4, 0.18, 0.78} {[V] #1}}}
% --- disable by uncommenting  
% \renewcommand{\TODO}[1]{}
% \renewcommand{\todo}[1]{#1}
\usepackage{slashbox}
% Vectors
\newcommand{\bB}{\mathcal{B}}
\newcommand{\bw}{\mathbf{w}}
\newcommand{\bs}{\mathbf{s}}
\newcommand{\bo}{\mathbf{o}}
\newcommand{\bn}{\mathbf{n}}
\newcommand{\bc}{\mathbf{c}}
\newcommand{\bp}{\mathbf{p}}
\newcommand{\bS}{\mathbf{S}}
\newcommand{\bk}{\mathbf{k}}
\newcommand{\bmu}{\boldsymbol{\mu}}
\newcommand{\bx}{\mathbf{x}}
\newcommand{\bg}{\mathbf{g}}
\newcommand{\be}{\mathbf{e}}
\newcommand{\bX}{\mathbf{X}}
\newcommand{\by}{\mathbf{y}}
\newcommand{\bv}{\mathbf{v}}
\newcommand{\bz}{\mathbf{z}}
\newcommand{\bq}{\mathbf{q}}
\newcommand{\bff}{\mathbf{f}}
\newcommand{\bu}{\mathbf{u}}
\newcommand{\bh}{\mathbf{h}}
\newcommand{\bb}{\mathbf{b}}

\newcommand{\rone}{\textcolor{green}{R1}}
\newcommand{\rtwo}{\textcolor{orange}{R2}}
\newcommand{\rthree}{\textcolor{red}{R3}}
\usepackage{amsmath}
%\usepackage{arydshln}
\DeclareMathOperator{\similarity}{sim}
\DeclareMathOperator{\AvgPool}{AvgPool}

\newcommand{\argmax}{\mathop{\mathrm{argmax}}}     



% It is strongly recommended to use hyperref, especially for the review version.
% hyperref with option pagebackref eases the reviewers' job.
% Please disable hyperref *only* if you encounter grave issues, 
% e.g. with the file validation for the camera-ready version.
%
% If you comment hyperref and then uncomment it, you should delete *.aux before re-running LaTeX.
% (Or just hit 'q' on the first LaTeX run, let it finish, and you should be clear).
\definecolor{cvprblue}{rgb}{0.21,0.49,0.74}
\usepackage[pagebackref,breaklinks,colorlinks,allcolors=cvprblue]{hyperref}

% DK
\usepackage{xcolor}
\usepackage{multirow}
% JH
\usepackage{adjustbox}

%%%%%%%%% PAPER ID  - PLEASE UPDATE
\def\paperID{16035} % *** Enter the Paper ID here
\def\confName{CVPR}
\def\confYear{2025}

%%%%%%%%% TITLE - PLEASE UPDATE
% \title{\LaTeX\ Author Guidelines for \confName~Proceedings}
% \title{360 Indoor Reconstruction}
\title{IM360: Textured Mesh Reconstruction for Large-scale Indoor Mapping with 360\textdegree\ Cameras}

%%%%%%%%% AUTHORS - PLEASE UPDATE
\author{First Author\\
Institution1\\
Institution1 address\\
{\tt\small firstauthor@i1.org}
% For a paper whose authors are all at the same institution,
% omit the following lines up until the closing ``}''.
% Additional authors and addresses can be added with ``\and'',
% just like the second author.
% To save space, use either the email address or home page, not both
\and
Second Author\\
Institution2\\
First line of institution2 address\\
{\tt\small secondauthor@i2.org}
}

\begin{document}

\clearpage
\setcounter{page}{1}
\maketitlesupplementary

In this supplementary material, we provide additional qualitative results, highlighting the superior performance of our method compared to other approaches.
Appendix \ref{sphericalsfm} highlights the advantages of the spherical camera model and the dense matching algorithm for indoor reconstruction. Appendix \ref{texture} demonstrates the effectiveness of our novel texturing method.

\section{Spherical Structure from Motion}
\label{sphericalsfm}
Due to page limitations, we present our spherical structure from motion results in the supplementary material: 1) \textbf{OpenMVG:} An open-source SfM pipeline that supports spherical camera models \cite{moulon2017openmvg}.
2) \textbf{SPSG COLMAP:} SuperPoint \cite{detone2018superpoint} and SuperGlue \cite{sarlin2020superglue} are used with cubemap and equirectangular projection.
3) \textbf{DKM COLMAP:} This method leverages DKM \cite{edstedt2023dkm} to establish dense correspondences, utilizing cubemap and equirectangular projection.
4) \textbf{SphereGlue COLMAP:} SuperPoint \cite{detone2018superpoint} with a local planar approximation \cite{eder2020tangent} and SphereGlue \cite{gava2023sphereglue} are utilized to mitigate distortion in ERP images.
The experimental results discussed in the main paper for Matterport3D \cite{chang2017matterport3d} and Stanford2D3D \cite{Stanford2d3d} are shown in Fig. \ref{fig:sfm_mp3d} and Fig. \ref{fig:sfm_stfd}, respectively.

\section{Texture Map Optimization}
\label{texture}
We compare our method with several recent rendering approaches, including \textbf{TexRecon} \cite{waechter2014TexRecon}, \textbf{SparseGS} \cite{xiong2023sparsegs}, and \textbf{ZipNeRF} \cite{barron2023zip}. 
Our method outperforms these approaches by delivering higher frequency details and producing seamless texture maps.
The results of the textured mesh and rendering are shown in Fig. \ref{fig:textured_mesh} and Fig. \ref{fig:render1} - \ref{fig:render4}.

% \section{Rationale}
% \label{sec:rationale}
% % 
% Having the supplementary compiled together with the main paper means that:
% % 
% \begin{itemize}
% \item The supplementary can back-reference sections of the main paper, for example, we can refer to \cref{sec:intro};
% \item The main paper can forward reference sub-sections within the supplementary explicitly (e.g. referring to a particular experiment); 
% \item When submitted to arXiv, the supplementary will already included at the end of the paper.
% \end{itemize}
% % 
% To split the supplementary pages from the main paper, you can use \href{https://support.apple.com/en-ca/guide/preview/prvw11793/mac#:~:text=Delete%20a%20page%20from%20a,or%20choose%20Edit%20%3E%20Delete).}{Preview (on macOS)}, \href{https://www.adobe.com/acrobat/how-to/delete-pages-from-pdf.html#:~:text=Choose%20%E2%80%9CTools%E2%80%9D%20%3E%20%E2%80%9COrganize,or%20pages%20from%20the%20file.}{Adobe Acrobat} (on all OSs), as well as \href{https://superuser.com/questions/517986/is-it-possible-to-delete-some-pages-of-a-pdf-document}{command line tools}.

\begin{figure*}[t]
    \centering
    \includegraphics[width=1.0\linewidth]{figures_sup/sup_sfm.pdf}
    \caption{Qualitative Comparison of SfM results on Matterport3D.
    While other approaches failed to achieve pose registration, our method successfully estimates poses by leveraging the spherical camera model and dense matching.
    }
    \label{fig:sfm_mp3d}
\end{figure*}

\begin{figure*}[t]
    \centering
    \includegraphics[width=1.0\linewidth]{figures_sup/sup_sfm_stfd.pdf}
    \caption{Qualitative Comparison of SfM results on Stanford2D3D.
    While other approaches failed to achieve pose registration, our method successfully estimates poses by leveraging the spherical camera model and dense matching.}
    \label{fig:sfm_stfd}
\end{figure*}

\begin{figure*}[t]
    \centering
    \includegraphics[width=1.0\linewidth]{figures_sup/texture_map.pdf}
    \caption{Qualitative Comparisons of Textured Mesh Results on Matterport3D.
    A comparison between TexRecon \cite{waechter2014TexRecon} and ours shows that our method effectively reduces noise in the texture maps, leading to improved visual quality and detail.}
    \label{fig:textured_mesh}
\end{figure*}


% % \maketitle
% % \clearpage
% \setcounter{page}{1}
% \maketitlesupplementary
\begin{figure*}[t]
    \centering
    \includegraphics[width=1.0\linewidth]{figures_sup/render_comparison_supp1.pdf}
    \caption{Qualitative Comparisons with Existing Methods. Our method can render high frequency details and results in lower noise.}
    \label{fig:render1}
\end{figure*}

\begin{figure*}[t]
    \centering
    \includegraphics[width=1.0\linewidth]{figures_sup/render_comparison_supp2.pdf}
    \caption{Qualitative Comparisons with Existing Methods. Our method can render high frequency details and results in lower noise.}
    \label{fig:render2}
\end{figure*}

\begin{figure*}[t]
    \centering
    \includegraphics[width=1.0\linewidth]{figures_sup/render_comparison_supp3.pdf}
    \caption{Qualitative Comparisons with Existing Methods. Our method can render high frequency details and results in lower noise.}
    \label{fig:render3}
\end{figure*}

\begin{figure*}[t]
    \centering
    \includegraphics[width=1.0\linewidth]{figures_sup/render_comparison_supp4.pdf}
    \caption{Qualitative Comparisons with Existing Methods. Our method can render high frequency details and results in lower noise.}
    \label{fig:render4}
\end{figure*}



{
    \small
    \bibliographystyle{ieeenat_fullname}
    \bibliography{main}
}


\end{document}

%%%%%%%%%%%%%%%%%%%%%%%%%%%%%%%%%%%%%%%%%%%%%%%%%%%%%%%%%%%%%%%%%%%%%%%%%%%%%%%
%%%%%%%%%%%%%%%%%%%%%%%%%%%%%%%%%%%%%%%%%%%%%%%%%%%%%%%%%%%%%%%%%%%%%%%%%%%%%%%


\end{document}


% This document was modified from the file originally made available by
% Pat Langley and Andrea Danyluk for ICML-2K. This version was created
% by Iain Murray in 2018, and modified by Alexandre Bouchard in
% 2019 and 2021 and by Csaba Szepesvari, Gang Niu and Sivan Sabato in 2022.
% Modified again in 2023 and 2024 by Sivan Sabato and Jonathan Scarlett.
% Previous contributors include Dan Roy, Lise Getoor and Tobias
% Scheffer, which was slightly modified from the 2010 version by
% Thorsten Joachims & Johannes Fuernkranz, slightly modified from the
% 2009 version by Kiri Wagstaff and Sam Roweis's 2008 version, which is
% slightly modified from Prasad Tadepalli's 2007 version which is a
% lightly changed version of the previous year's version by Andrew
% Moore, which was in turn edited from those of Kristian Kersting and
% Codrina Lauth. Alex Smola contributed to the algorithmic style files.
