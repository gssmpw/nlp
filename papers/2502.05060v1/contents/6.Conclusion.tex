\section{Conclusion}
\label{sec:conc}

This paper presented a general algorithmic paradigm for optimizing revenue in on-demand crowdsourcing platforms by modeling the problem as an \gls{acr:mdp} and using the \gls{acr:mnl} model to capture gig worker behavior. By directly incorporating gig worker preferences, our algorithm balances creating attractive offers to ensure utilization while avoiding overcompensation to maximize platform profits.

Our algorithm consistently outperformed rule-based benchmarks and achieved results close to a perfect-prediction upper bound. Specifically, our algorithm showed improvements of up to 7.5\% in scenarios with homogeneous and of up to 10\% in scenarios with heterogeneous gig worker groups. On real-world data, where gig workers have location preferences, we observe improvements of up to 20\%. With these findings, we demonstrated that our algorithm is effective in multiple scenarios utilizing synthetic data with varying on-demand request arrival rates as well as real-world data. However, our results also reveal that the performance of such a partially model-based algorithm depends on the accuracy of the estimated \gls{acr:mnl} model parameters used to estimate the gig worker utilities. In this context, we show that our algorithm shows stability to parameter perturbations, performing well even when the \gls{acr:mnl} parameters are under underestimated, which can likely happen in practice. 

Future research could refine estimation techniques for the \gls{acr:mnl} parameters and improve robustness to parameter variability. Additionally, future work could investigate fairness issues in compensation strategies, especially concerning important gig worker groups. Future enhancements could also include extending the considered gig worker state to account for the gig worker queue, using attention mechanisms to generate gig worker state context vectors. Additionally, exploring priority-based gig worker queue management strategies as alternatives to FIFO could enhance decision-making.