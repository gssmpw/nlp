\section{Introduction}

\noindent In today's fast-paced world, consumers are increasingly time-sensitive when utilizing services, leading to substantial growth in on-demand service platforms. These platforms face the significant challenge of managing rapid fluctuations in demand and meeting the urgency of customer requests, while simultaneously keeping operational costs low. Within this context, platforms increasingly rely on using gig workers via crowdsourcing, i.e., independent workers that service on-demand tasks in exchange for a compensation, instead of employing traditional contracted personnel. Crowdsourcing has gained traction across various sectors, including last-mile delivery services such as grocery delivery (e.g., Instacart) and food delivery (e.g., DoorDash, UberEats), parcel delivery (e.g., Amazon Flex, UberFreight, UberRUSH, Roadie), and ride-hailing services (e.g., Uber, Lyft, Didi). Besides offering benefits in managing demand fluctuations \citep{luy2023strategic}, the utilization of gig workers is also appealing due to its potential to reduce costs \citep{fatehi2022crowdsourcing} and, in the case of crowdsourced delivery, it can contribute to sustainable urban logistics by minimizing emissions and road congestion \citep{yuen2023sustainable}. However, integrating gig workers into on-demand service models introduces complexity to the platform’s decision-making process, primarily due to the uncertainty in gig workers' availability and willingness to accept requests. In this context, the compensation offered by platforms plays a crucial role in influencing the likelihood of request acceptance \citep{barbosa2023data, bathke2023occasional}. Therefore, the success of a crowdsourcing platform relies on finding a compensation policy that balances two key objectives: creating offers attractive enough to engage gig workers, while also maintaining profitability for the platform. Achieving this balance is challenging as gig workers tend to show strong preferences based on specific request characteristics, such as the type of service, and the request's deadline. Additionally, gig workers with different characteristics, such as age and employment status can show different request preferences. Accordingly, it is essential for platforms to account for these preferences when designing compensation policies to ensure optimal outcomes. The goal of this study is to introduce compensation strategies for crowdsourced on-demand service platforms, which directly consider gig worker preferences in shaping offers for on-demand requests. 

\subsection{Related work}

Our work relates to two streams of research: studies on  gig worker preferences and behavior as well as work on dynamic pricing strategies for crowdsourced on-demand service platforms. In the following, we review both works concisely.

\noindent \textbf{Gig worker preferences}:
Multiple studies investigate gig worker preferences and behavior across various on-demand service platforms. In the context of crowd-shipping, where workers bid for delivery requests, \cite{ermagun2018bid} apply logistic regression to analyze how request characteristics influence attractiveness. They find that characteristics such as the size of the package, the type of service, the delivery deadline, and the distance of delivery are significantly correlated with a request's probability of receiving a bid. In a similar context, \cite{hou2023order} study the crowd-shipper's acceptance behavior by utilizing discrete choice models and machine learning models applied to survey data. Their findings suggest that factors such as gig workers’ age, income, and the compensation offered per request are key predictors of acceptance behavior. \cite{bathke2023occasional} perform a choice-based conjoint analysis to understand how various crowd-shipping delivery request attributes, in conjunction with gig worker characteristics, affect their willingness to accept requests. Their results indicate that delivery time and compensation are the primary drivers of gig worker decisions. Moreover, the influence of these factors varies across different worker demographics, such as age, and employment status. In the context of crowdsourced ride-sourcing platforms,  \cite{ashkrof2022ride} use a choice modeling approach based on the Random Utility Model (RUM) to analyze driver decisions using a driver-decision survey dataset of Uber and Lyft drivers. Their study highlights that driver-specific factors, including employment status, experience, and work shifts, significantly impact acceptance rates. Additionally, longer travel distances between a driver’s location and the pick-up point reduce the likelihood of job acceptance. In a similar context, \cite{xu2018empirical} utilize logistic regression on a dataset from a major ride-hailing platform in Beijing to show that driver behavior is strongly influenced by economic incentives, request characteristics (especially trip distance), and supply-demand intensities. 

This body of literature highlights the variety and complexity of factors influencing gig worker behavior as well as the variability of preferences. Together, these studies underscore the importance of incorporating gig worker preferences into the decision-making process to ensure effective compensation policies. Doing so remains the scope of this work.

\noindent \textbf{Pricing in crowdsourcing platforms}: A large body of literature exists on compensation strategies for crowdsourced on-demand service platforms.

\noindent \textit{Crowdsourced on-demand transportation platforms:} \cite{bai2019coordinating} study time-based pricing for an on-demand service platform with price and waiting time-sensitive requests and price-sensitive gig workers. \cite{sun2019optimal} consider ride-sourcing platforms and study how pricing per individual ride request can be optimized based on factors like distance and time, while considering that gig workers and customers maximize utility and therefore have the right to reject the platform's offer. \cite{cachon2017role} analyze pricing strategies for crowdsourcing platforms, focusing on how different contracts, such as fixed commission and surge pricing policies, affect provider participation and consumer welfare. They conclude that, while fixed pricing achieves near-optimal profits in most scenarios, surge pricing policies, can further enhance efficiency by better aligning supply with demand during peak periods. \cite{wang2022optimal} study dynamic pricing strategies in crowdsourcing logistics, using optimal control theory to maximize platform revenue considering fluctuating social delivery capacities and stochastic order demands. \cite{bimpikis2019spatial} study spatial pricing for ride-sharing platforms that serve requests on a network of locations. They find that maximum profit is achieved when the demand is balanced across all network locations and that using origin-specific ride prices can lead to significantly higher profits compared to using a fixed price across the network, especially when the demand pattern is highly unbalanced. Similarly, \cite{meskar2023spatio} optimize pricing and matching rates in ride-hailing platforms considering spatial and temporal network characteristics while also allowing the possibility of rejection from both the customer and the gig worker side. They also find that a balanced demand pattern across network locations yields maximum profit.

\noindent \textit{Crowdsourced last-mile delivery:}
\cite{yildiz2019service} present a stylized equilibrium model for optimizing service coverage and capacity planning in on-demand meal delivery, and derive results on the optimal gig worker compensation and service area. They study model variations where the probability of acceptance of gig workers is either fixed or dependent on the distance of the request's drop-off location from the restaurant. In the context of crowdsourced last-mile delivery, several studies consider joint decisions of gig worker compensation and routing for delivery requests. To this end, \cite{fatehi2022crowdsourcing} study crowdsourced last-mile delivery with guaranteed delivery time windows using robust optimization approaches based on robust queuing and routing theory. They derive analytical results for optimal hourly compensation and labor planning. Considering the use of a privately owned fleet, \cite{barbosa2023data} and \cite{silva2022deep} explore data-driven dynamic pricing schemes under uncertainty of gig workers' willingness to accept offers, focusing on last-mile delivery from a single store with gig workers who are in-store customers. The work of \cite{barbosa2023data} uses logistic regression on data collected through a questionnaire to model gig workers' willingness to undertake a request and develops a direct search algorithm to determine the optimal compensation. \cite{silva2022deep} use a \gls{acr:drl} approach to solve a two-stage optimization problem, where the first stage involves decisions on request fulfillment order and gig worker compensation, and the second stage involves routing decisions.  

Most of the existing literature on compensation strategies for on-demand service platforms either overlooks gig worker preferences and acceptance probabilities or, to the best of our knowledge, addresses them only in highly restricted or stylized settings, see e.g., \cite{yildiz2019service}, \cite{barbosa2023data} and \cite{silva2022deep}. However, ignoring gig worker preferences when designing compensation policies inevitably leads to suboptimal outcomes. Against this background, we develop an algorithmic paradigm for creating preference-aware compensation policies that can generally be applied in the context of crowdsourced on-demand service platforms. Furthermore, we aim to learn these preferences from data using statistical models, ensuring that the algorithm is both data-driven and scalable, making it suitable for real-world applications within dynamic and large-scale environments.

\subsection{Contributions}

In this paper, we propose a general algorithmic paradigm for computing preference-aware compensation policies to incentivize gig workers on on-demand service platforms to accept request offers. Our contributions are as follows: (1) We formalize the underlying compensation problem as a \gls{acr:mdp} and prove that the inner optimization problem of the Bellman equation can be solved optimally under the assumption that gig worker behavior follows a \gls{acr:mnl}. This allows us to derive a profit maximizing policy for the operator. (2) Based on the closed-form expression of this policy, we derive a practical algorithm for making compensation decisions, utilizing an approximate value iteration method that incorporates an \gls{acr:mnl} approximation learned in a predict-then-optimize fashion. (3) We present new benchmark datasets to evaluate the effectiveness of our algorithm and, through extensive numerical experiments, compare its performance against formula-based benchmark policies, a full-information upper bound, and our approach using a perfect \gls{acr:mnl} approximation.

Our experimental results underscore the critical role of incorporating gig worker preferences in achieving near-optimal results. In synthetic scenarios with a homogeneous gig worker population, our algorithm achieved performance ratios between 88.5\% and 94.7\% outperforming two benchmark policies by~\mbox{2.5-7.5\%}. In scenarios with a heterogeneous gig worker population, our approach achieved an average performance improvement of 9\% over benchmark policies. However, when gig worker preferences were not properly accounted for (i.e., using a single \gls{acr:mnl} for all groups), performance dropped significantly, confirming the necessity of accurately capturing worker heterogeneity to prevent revenue loss. In tests on real-world data simulating an on-demand ride-hailing platform, our algorithm achieved a performance improvement of 8\% with gig workers showing weak location preferences and 20\% with strong location preferences. We attribute this improvement largely to our algorithm being capable of avoiding the excessive compensation observed in the benchmark policies. Overall, these results demonstrate that a preference-aware approach is essential for balancing gig worker engagement and platform profitability, leading to superior performance across varied operational conditions.

\subsection{Organization}

The remainder of this paper is structured as follows. Section \ref{sec:prob_form} presents our problem setting and formulates it as an \gls{acr:mdp}, while Section \ref{sec:method} outlines our methodology for solving the  \gls{acr:mdp}. Section \ref{sec:exp_des} describes our experimental design while Section \ref{sec:res} presents the results of our experiments, including a comparative analysis of our algorithm against benchmark policies, sensitivity analysis over the gig worker utility estimation, and managerial insights. Finally, Section \ref{sec:conc} concludes our paper by summarizing its key findings and suggesting avenues for future work.
