\section{Related Works}
\label{sec:RelWorks} 

In paper____ a simple MCS system with static selection of the best base classifier was used to recognise SIS (surgical instrument signaling) gestures. Paper____ presents an MCS system that uses a DES (dynamic ensemble selection) scheme and a customised competence measure. Well-known bagging and boosting schemas have also been used in tasks of controlling upper and lower limb prostheses____. Additionally, multiclassifier schemas were also utilised to recognise the hand gestures of able-bodied individuals____. All of the methods presented in the above-mentioned papers are aimed at improving classification quality by using a multiclassifier system. In our work, on the other hand, the MCS uses a DCS (dynamic classifier selection) schema with an original classifier selection method that takes the classification context into account. The aim of this DCS is to increase the number of gestures and hand manipulations that can be used in a prosthesis. Furthermore, the proposed DCS procedure can improve the overall classification quality of MCS.

Increased prosthesis dexterity can also be achieved using targeted muscle renervation (TMR) surgery. In____, a case study is presented investigating this procedure. An increase in prosthesis dexterity may also be achieved using electroneurography (ENG). ENG signals are obtained from peripheral nerves by microelectrodes implanted in residual nerves____. A similar effect may be achieved using kinetico-myographic signals (KMG)____. This kind of signal is generated by magnetic tags surgically implanted in the tendons. The authors prove that the SNR ratio of KMG signals is better compared to sEMG signals. All the methods presented above are invasive and do not always achieve success. There are many conditions that prevent them from carrying out the procedure____. Unlike them, the system proposed in this paper is not invasive and is not restricted by medical conditions.

The paper____ presents the results of experimental research in biomimetic and non-biomimetic control strategy of the prosthesis. The biomimetic schema emulates the biological control of the hand by linking the imaginary movement of the phantom hand with the movement of the prosthesis. The non-biomimetic method allows arbitrary linkage between phantom limb movement and prosthesis action. We use non-biomimetic control in the context-dependent control schema presented in this paper.

A finite state machine (FSM) can be used to describe the proposed context-dependent control schema. Thus, it is necessary to mention works that use FSMs to describe the behaviour of the prosthesis. The main difference between the methods lies in the way in which the state change of the FSM is triggered. Transitions may be activated using a kind of external signal source such as a mobile application____, webcam____, or eye tracking____. The state may also be changed depending on the readings of sensors such as gyroscopes____ or goniometers____. Invoking the change can also be done using biosignals such as sEMG____, mechanomiographic signal (MMG)____. Biosignal control can use a simple thresholding strategy____ or machine-learning-based approaches____. The transition between states may be discrete____ or continuous____. The presented context recognition system, interpreted as an FSM system, is fully automatic, based solely on the multi-channel sEMG signal, and does not use any additional information (signals, images). A change in the system state, meaning a change in the context determining the interpretation of classes and the active classifier model, is the result of the operation of the base classifier of the multi-classifier system. This is an innovative approach that allows achieving the expected benefits (improving the quality of classification, increasing the range of prosthesis movements).