% !TEX program = pdflatex
\documentclass{article}

%
%
\usepackage{graphicx}
\usepackage{booktabs}
\usepackage{hyperref}
\usepackage{colortbl, xcolor}
\usepackage{amsmath}
\usepackage{hyperref}


\usepackage[accepted]{icml2025}

\usepackage{amssymb}
\usepackage{mathtools}
\usepackage{amsthm}
\usepackage{multirow}

%
% \newcommand{\theHalgorithm}{\arabic{algorithm}}


\usepackage[utf8]{inputenc} %
\usepackage[T1]{fontenc}    %
\usepackage{amsfonts}       %
\usepackage{nicefrac}       
\usepackage{ulem, bm, kotex}

\usepackage{subcaption}
% \usepackage[style=base]{caption}
% \usepackage{caption}
\usepackage{float}
\usepackage{wrapfig}
\usepackage{lipsum}

\usepackage{pifont}
\newcommand{\cmark}{\text{\ding{51}}}
\newcommand{\xmark}{\text{\ding{55}}}

%
% \usepackage{color}
% \newcommand{\mj}[1]{{\color{blue}{mijung:#1}}}
%
\usepackage[capitalize,noabbrev]{cleveref}

%
%
%
\theoremstyle{plain}
\newtheorem{theorem}{Theorem}[section]
\newtheorem{proposition}[theorem]{Proposition}
\newtheorem{lemma}[theorem]{Lemma}
\newtheorem{corollary}[theorem]{Corollary}
\theoremstyle{definition}
\newtheorem{definition}[theorem]{Definition}
\newtheorem{assumption}[theorem]{Assumption}
\theoremstyle{remark}
\newtheorem{remark}[theorem]{Remark}
\newtheorem*{reptheorem}{\textnormal{\textbf{Theorem \ref{thm:safe}}}}
\newtheorem*{replemma}{\textnormal{\textbf{Lemma \ref{lemma:unsafe}}}}

%
% \usepackage[table]{xcolor}
% \RequirePackage[table]{xcolor}

% \newcommand{\hl}[1]{\textcolor{blue}{#1}}
% \newcommand{\dkc}[1]{\textcolor{orange}{[DJ: #1]}}
% \newcommand{\dk}[1]{\textcolor{orange}{#1}}
% \newcommand{\mgyu}[1]{\textcolor{grey}{#1}}
\newcommand*\diff{\mathop{}\!\mathrm{d}}
\newcommand{\cc}[1]{\cellcolor{gray!#1}}

%
\newcommand{\BarStar}{%
  \rule{0.8cm}{0.25cm}%
  \llap{%
    \hspace{0.4cm}%
    \raisebox{0.02cm}{\textcolor{white}{*}}%
  }%
}

%
%
%
\usepackage[textsize=tiny]{todonotes}
%%%%% NEW MATH DEFINITIONS %%%%%

% \usepackage{amsmath,amsfonts,bm}
\usepackage{amsmath,amsfonts}

\usepackage{pifont}


\newcommand{\R}{\mathbb{R}}


\def\va{{\mathbf{a}}}
\def\vg{{\mathbf{g}}}

% Sets
\def\sR{\mathbb{R}}
\def\sC{\mathbb{C}}
\def\sZ{\mathbb{Z}}
\def\sN{\mathbb{N}}
\def\sQ{\mathbb{Q}}

\def\sS{\mathcal{S}}



% Vectors
\def\vzero{{\mathbf{0}}}
\def\vone{{\mathbf{1}}}
\def\vmu{{\mathbf{\mu}}}
\def\vtheta{{\mathbf{\theta}}}
\def\va{{\mathbf{a}}}
\def\vb{{\mathbf{b}}}
\def\vc{{\mathbf{c}}}
\def\vd{{\mathbf{d}}}
\def\ve{{\mathbf{e}}}
\def\vf{{\mathbf{f}}}
\def\vg{{\mathbf{g}}}
\def\vh{{\mathbf{h}}}
\def\vi{{\mathbf{i}}}
\def\vj{{\mathbf{j}}}
\def\vk{{\mathbf{k}}}
\def\vl{{\mathbf{l}}}
\def\vm{{\mathbf{m}}}
\def\vn{{\mathbf{n}}}
\def\vo{{\mathbf{o}}}
\def\vp{{\mathbf{p}}}
\def\vq{{\mathbf{q}}}
\def\vr{{\mathbf{r}}}
\def\vs{{\mathbf{s}}}
\def\vt{{\mathbf{t}}}
\def\vu{{\mathbf{u}}}
\def\vv{{\mathbf{v}}}
\def\vw{{\mathbf{w}}}
\def\vx{{\mathbf{x}}}
\def\vy{{\mathbf{y}}}
\def\vz{{\mathbf{z}}}
\def\vzeta{{\mathbf{\zeta}}}

% Matrix
\def\mA{{\mathbf{A}}}
\def\mB{{\mathbf{B}}}
\def\mC{{\mathbf{C}}}
\def\mD{{\mathbf{D}}}
\def\mE{{\mathbf{E}}}
\def\mF{{\mathbf{F}}}
\def\mG{{\mathbf{G}}}
\def\mH{{\mathbf{H}}}
\def\mI{{\mathbf{I}}}
\def\mJ{{\mathbf{J}}}
\def\mK{{\mathbf{K}}}
\def\mL{{\mathbf{L}}}
\def\mM{{\mathbf{M}}}
\def\mN{{\mathbf{N}}}
\def\mO{{\mathbf{O}}}
\def\mP{{\mathbf{P}}}
\def\mQ{{\mathbf{Q}}}
\def\mR{{\mathbf{R}}}
\def\mS{{\mathbf{S}}}
\def\mT{{\mathbf{T}}}
\def\mU{{\mathbf{U}}}
\def\mV{{\mathbf{V}}}
\def\mW{{\mathbf{W}}}
\def\mX{{\mathbf{X}}}
\def\mY{{\mathbf{Y}}}
\def\mZ{{\mathbf{Z}}}
\def\mBeta{{\mathbf{\beta}}}
\def\mPhi{{\mathbf{\Phi}}}
\def\mLambda{{\mathbf{\Lambda}}}
\def\mSigma{{\mathbf{\Sigma}}}


% Expectation
% \def\eE{\mathop{\mathbb{E}}\limits}
\def\eE{\mathbb{E}}

% Probability
\def\pP{\mathbb{P}}

% Tilde
\def\tf{\tilde{f}}
\def\tS{\tilde{S}}
\def\wtF{\widetilde{\mathcal{F}}}
\def\whR{\widehat{R}}
\def\tvx{\tilde{\mathbf{x}}}
\def\ty{\tilde{y}}


\def\defeq{\overset{\textup{def}}{=}}
% \def\defeq{\overset{.}{=}}
\def\defone{\overset{\text{\ding{172}}}{=}}
\def\deftwo{\overset{\text{\ding{173}}}{=}}
\def\leqone{\overset{\text{\ding{172}}}{\leq}}
\def\leqtwo{\overset{\text{\ding{173}}}{\leq}}
\def\leqthree{\overset{\text{\ding{174}}}{\leq}}
\def\leqfour{\overset{\text{\ding{175}}}{\leq}}
\def\eqone{\overset{\text{\ding{172}}}{=}}
\def\eqtwo{\overset{\text{\ding{173}}}{=}}
\def\eqthree{\overset{\text{\ding{174}}}{=}}
\def\eqfour{\overset{\text{\ding{175}}}{=}}
\def\geqfive{\overset{\text{\ding{176}}}{\geq}}

%
%
\icmltitlerunning{Training-Free Safe Denoisers For Safe Use of Diffusion Models}

\begin{document}
\twocolumn[
\icmltitle{Training-Free Safe Denoisers for Safe Use of Diffusion Models}

%
%
%
%

%
%
%
%

%
%
%
\icmlsetsymbol{equal}{*}

\author{%
  Mingyu Kim, Dongjun Kim, Amman Yusuf, Stefano Ermon, Mi Jung Park\\
  UBC CS, Standford CS, UBC CS, UBC CS\\
  \texttt{mgyu.kim@ubc.ca},
  \texttt{dongjun@stanford.edu}
  \texttt{mijungp@cs.ubc.ca} \\
}

\begin{icmlauthorlist}
\icmlauthor{Mingyu Kim}{equal,yyy}
\icmlauthor{Dongjun Kim}{equal,sch}
\icmlauthor{Amman Yusuf}{yyy}
\icmlauthor{Stefano Ermon}{sch}
\icmlauthor{Mi Jung Park}{yyy}
\end{icmlauthorlist}

\icmlaffiliation{yyy}{UBC CS}
\icmlaffiliation{sch}{Stanford CS}

\icmlcorrespondingauthor{Mi Jung Park}{mijungp@cs.ubc.ca}

%
%
%
\icmlkeywords{Machine Learning, ICML}

\vskip 0.3in
]

%

%
%
%
%
%

%
\printAffiliationsAndNotice{\icmlEqualContribution} %

\begin{abstract}

There is growing concern over the safety of powerful diffusion models (DMs), as they are often misused to produce inappropriate, not-safe-for-work (NSFW) content or generate copyrighted material or data of individuals who wish to be forgotten. 
%
%
%
%
%
%
%
%
%
%
%
%
%
%
%
%
%
Many existing methods tackle these issues by heavily relying on text-based negative prompts or extensively retraining DMs to eliminate certain features or samples. 
%
In this paper, we take a radically different approach, directly modifying the sampling trajectory by leveraging a negation set (e.g., unsafe images, copyrighted data, or datapoints needed to be excluded) to avoid specific regions of data distribution, without needing to retrain or fine-tune DMs. 
We formally derive the relationship between the expected denoised samples that are safe and those that are not safe, leading to our \textit{safe} denoiser which ensures its final samples are away from the area to be negated.
Inspired by the derivation, we develop a practical algorithm that successfully produces high-quality samples while avoiding negation areas of the data distribution in text-conditional, class-conditional, and unconditional image generation scenarios. These results hint at the great potential of our training-free safe denoiser for using DMs more safely.
%
%
%
%
%

%
%
%
%
%
%
%
%
%
%
%
%
%
%
%
%
%
\end{abstract}

\vspace{-0.5cm}
\textcolor{red}{Warning: This paper contains disturbing content such as violent and sexually explicit images.}
%
% 
% 
The widespread integration of communication networks and smart devices in modern control systems has increased the vulnerability of industrial systems to online cyber-attacks, e.g., Industroyer, Blackenergy, etc \citep{osti_1505628}.
% Modern control systems have seen a large push to include communication networks and smart devices to increase performance, made possible by improvements in communication device cost and energy consumption. This trend has been coupled with the usage of open-standard communication protocols among industrial control systems, making them vulnerable to online cyber-attacks such as Industroyer, Blackenergy, etc \citep{osti_1505628}. 
To counter this, methods have been developed to improve security by achieving attack detection, mitigation, and monitoring, among others \citep{sandberg2022secure}. This paper focuses on active attack diagnosis to mitigate stealthy attacks. 
%
%\subsection{Literature review}

Active diagnosis techniques rely on the inclusion of additional moduli to control systems
% inclusion within the control system of additional moduli 
to alter the behavior of the system compared to information known by the attacker. 
For instance, the concept of additive watermarking was introduced in \cite{mo2015physical}, where noise signals of known mean and variance are added at the plant and compensated for it at the controller. 
This compensation, however, is not exact, causing some performance degradation. Thus, trade-offs between performance and detectability  are necessary \citep{zhu2023detection}.
% A later work \citep{zhu2023detection} designs the watermark signal by trading performance for detection. Thus, although additive watermarking serves as a good detection scheme, they endure performance losses even in the nominal case. 

In encrypted control \citep{darup2021encrypted}, the sensor data is encrypted, sent to the controller, and then operated on directly. Encrypted input signals are sent back to the plant for decryption. Although encryption is widespread in IT security, in control systems it presents some concerns, such as the introduction of time delays \citep{stabile2024verifiable}, while it may present inherent weaknesses \citep{alisic2023model}.
% they are not preferred as they introduce time delays \citep{stabile2024verifiable} which can cause instability, and some encryption schemes can be very weak  \citep{alisic2023model}. 

In moving target defense \citep{griffioen2020moving}, the plant is augmented with fictitious dynamics, known to the controller. The plant output is transmitted to the controller along with the fictitious states over a network under attack. 
The additional measurements then aide in the detection of attacks. 
This comes at the cost of higher communication bandwidth needs, which increases rapidly with the dimension of the augmented systems.
% Since the dynamics of the fictitious dynamics are exactly known to the controller, the attack is detected easily. However, when the scale of the system increases, the communication bandwidth used by moving the target defense approach increases rapidly. 

Other recently proposed works include two-way coding \citep{fang2019two}, a weak encryuption technique, and dynamic masking \citep{abdalmoaty2023privacy}, which enhances privacy as well as security, have been shown to be effective against zero-dynamics attacks.
% Two-way coding \citep{fang2019two} and dynamic masking \citep{abdalmoaty2023privacy} are other recently proposed approaches. Two-way coding is another form of weak encryption technique whilst dynamic masking proposes an architecture that enhances both privacy and security. These schemes are shown to be effective against zero dynamics attacks but remain to be studied for other classes of attacks. 
% Recent extensions include \citep{mukherjee2021secure,ramos2024privacy}.
% Some other works which are related are \citep{mukherjee2021secure}, an extension of \cite{fang2019two}. The work \citep{ramos2024privacy} is an extension of moving target defense for multi-agent systems. 
Furthermore, filtering techniques for attack detection are proposed by \cite{murguia2020security,hashemi2022codesign,escudero2023safety}, while not focusing on stealthy attacks.
% The works \citep{murguia2020security,hashemi2022codesign,escudero2023safety} develop filtering techniques to guarantee safety, without being focused on stealthy covert attacks.

Multiplicative watermarking (mWM) has been proposed by the authors as a diagnosis technique \citep{ferrari2020switching}. mWM consists of a pair of filters on each communication channel between the plant and its controller; the scheme is affine to weak encryption, whereby ``encoding'' and ``decoding'' are done by changing signals' dynamic characteristics through inverse pairs of filters. This enables original signals to be recovered exactly, and thus does not lead to performance degradation.
% A multiplicative watermark is an affine to a weak encryption technique, through which the signal is ``encoded'' by a filter, changing its dynamic behavior. The use of inverse pairs means that the original signal can be recovered, through ``decoding'' via an inverse filter. As such, differently to techniques based on additive watermarking, no performance is lost due to the injection of noise, and there are no bandwidth limitations.

%\subsection{Contributions}
One of the critical features of multiplicative watermarking is that to detect stealthy attacks, the mWM filter parameters must be switched over time. In this paper, an algorithm to optimally design the mWM parameters after a switching event is presented, enhancing detection performance, without changing the switching time.
% This is done without changing the switching time, which is taken as given.

\textcolor{black}{
To formalize the filter design problem, we suppose the defender is interested in optimal performance against adversaries injecting covert attacks with matched system parameters \citep{smith2015covert}, including the mWM parameters prior to the switch. This scenario represents a worst case where malicious agents can take full control of the system while remaining undetected.
Thus, the attack strategy is explicitly included within the formulation of the closed-loop system, and the mWM filters are chosen by solving an optimization problem minimizing the attack-energy-constrained output-to-output gain (AEC-OOG) \citep{anand2023risk}, a variation of the output-to-output gain proposed in  \cite{teixeira2015strategic}.
}
The main contributions of this paper are:
% We consider an adversary injecting a covert attack with matched system parameters \citep{smith2015covert}, i.e., an attacker with full knowledge of the control system parameters, including those of the mWM filters before the switch. This scenario is taken as a worst case, as it has been shown that this class of attacks can be made stealthy. To quantitatively define a cost, the output-to-output gain (OOG) \citep{teixeira2015strategic} is leveraged,
% a metric introduced to evaluate the impact of an additive attack in a control system. %Specifically, OOG evaluates the worst-case performance loss that an attacker injecting an undetectable attack can obtain. 
% Here, the maximum performance loss caused by a stealthy adversary with limited energy is taken, the attack-energy-constrained OOG (AEC-OOG) \citep{anand2023risk}. The main contributions of this paper are:
\begin{enumerate}
%[label=\alph*.]
\item The problem of optimally designing the switching mWM filters is formulated as an optimization problem, with the AEC-OOG is taken as the objective;%where the AEC-OOG is taken as the impact metric; 
\item The worst-case scenario of a covert attack with exact knowledge of plant and mWM filter parameters is embedded within the design problem;
% The optimization problem is defined to incorporate the worst-case scenario of a covert attack with exact knowledge of plant and mWM filter parameters;
\item The feasibility of the optimization problem is shown to be dependent only on stability conditions; 
\item A solution scheme is proposed to promote randomization of the mWM filter parameters such that an eavesdropping adversary cannot remain stealthy.
\end{enumerate} 

This builds on the results of \cite{ferrari2020switching}, where the focus was on the design of the switching protocols, rather than the parameters themselves.
Compared to previous work \citep{gallo2021design}, this paper introduces an optimization problem which is always feasible (thanks to the use of AEC-OOG in the objective), while also considering a more sophisticated class of covert attacks, where the presence of watermark is known to the adversary. 
Moreover, this paper poses a different objective than \citep{zhang2023hybrid}; indeed, while \citep{zhang2023hybrid} provided a design strategy to ensure certain privacy properties, in this paper we address the problem of optimal parameter design following a switching event.


%\subsection{Organization}
The rest of the paper is organized as follows. 
After formulating the problem in Section~\ref{sec:PF}, we propose our design algorithm in Section~\ref{sec:main}, and analyze its properties. It is then evaluated through a numerical example in Section~\ref{sec:NE}, and concluding remarks are given Section~\ref{sec:Con}.
% We provide the problem background in Section~\ref{sec:PF}. We formulate the design problem in Section~\ref{sec:main}, together with an analysis of its properties. The proposed algorithm is evaluated through a numerical example in Section \ref{sec:NE}. Concluding remarks are offered in Section \ref{sec:Con}.
In this section, we first present the notation and the problem definition. Then we introduce the COD algorithm and its theoretical guarantees.

\subsection{Notations}
Let $\mI_n$ denote the identity matrix of size $n \times n$, and $\mathbf{0}_{m \times n}$ represent the $m \times n$ matrix filled with zeros. A matrix $\mX$ of size $m \times n$ can be expressed as $\mX = [\vx_1, \vx_2, \dots, \vx_n]$, where each $\vx_i \in \mathbb{R}^m$ is the $i$-th column of $\mX$. The notation $[\mX_1\quad \mX_2]$ represents the concatenation of matrices $\mX_1$ and $\mX_2$ along their column dimensions. For a vector $\vx \in \mathbb{R}^d$, we define its $\ell_2$-norm as $\|\vx\| = \sqrt{\sum_{i=1}^d x_i^2}$. For a matrix $\mX \in \mathbb{R}^{m \times n}$, its spectral norm is defined as $\|\mX\|_2 = \max_{\vu: \|\vu\| = 1} \|\mX \vu\|$, and its Frobenius norm is $\|\mX\|_F = \sqrt{\sum_{i=1}^n \|\vx_i\|^2}$, where $\vx_i$ is the $i$-th column of $\mX$. The condensed singular value decomposition (SVD) of $\mX$, written as SVD$(\mX)$, is given by $\mU \mSigma \mV^\top$, where $\mU \in \mathbb{R}^{m \times r}$ and $\mV \in \mathbb{R}^{n \times r}$ are orthonormal column matrices, and $\mSigma$ is a diagonal matrix containing the nonzero singular values $\sigma_1(\mX) \geq \sigma_2(\mX) \geq \dots \geq \sigma_r(\mX) > 0$. The QR decomposition of $\mX$, denoted as QR$(\mX)$, is given by $\mQ \mR$, where $\mQ \in \mathbb{R}^{m \times n}$ is an orthogonal matrix with orthonormal columns, and $\mR \in \mathbb{R}^{n \times n}$ is an upper triangular matrix. The LDL decomposition is a variant of the Cholesky decomposition that decomposes a positive semidefinite symmetric matrix \( \mX \in \mathbb{R}^{n\times n}\) into \( \mL \mD \mL^\top = \operatorname{LDL}(\mX) \), where \( \mL \) is a unit lower triangular matrix and \( \mD \) is a diagonal matrix. By defining \( \bar{\mL} = \sqrt{\mD} \mL^\top \), we obtain the triangular matrix decomposition \( \mX = \bar{\mL} \bar{\mL}^\top \).
%The columns of $\mQ$ form an orthonormal basis for the column space of $\mX$, and $\mR$ contains the coefficients that represent $\mX$ in this orthonormal basis.
% We let $\mI_n$ be the $n \times n$ identity matrix, and $\bf{0}_{m \times n}$ be the $m \times n$ matrix of all zeros. We can denote a $m \times n$ matrix as $\mX = [\vx_1, \vx_2, \dots, \vx_n]$, where $\vx_i \in \BR^{m}$ is the $i$-th column of $\mX$. We use $[ \mX_1, \mX_2 ] $ to denote their concatenation
% on their column dimensions. For a vector $\vx\in\BR^d$, we let $\norm{\vx}=\sqrt{\sum_{i=1}^d x_i^2}$ be its $\ell_2$-norm. For a matrix $\mX\in\BR^{m\times n}$, we let $\Norm{\mX} =  \max_{\vu:\Norm{\vu} = 1}\Norm{\mX \vu}$ be its spectral norm and  $\Norm{\mX}_F = \sqrt{\sum_{i = 1}^{n}{\Norm{\vx_i}^2}}$ be its Frobenius norm. The condensed singular value decomposition (SVD) of matrix $\mX$, written as SVD$(\mX)$, is defined as $\mU \mSigma \mV^T$ where $\mU \in \BR^{m \times r}$ and $\mV \in \BR^{n \times r}$ are column orthonormal and $\mSigma$ is a diagonal matrix with nonzero singular values $\sigma_1(\mX) \geq \sigma_2(\mX) \geq \dots \geq \sigma_r(\mX)>0$. We use $\textrm{nnz}(\mX)$ to denote number of nonzero elements of matrix $\mX$.

\subsection{Problem Setup}
We first provide the definition of correlation sketch as follows:
\begin{defn}[\cite{mroueh2017co}]
Let $\mX \in \mathbb{R}^{m_x \times n}$, $\mY \in \mathbb{R}^{m_y \times n}$, $\mA \in \mathbb{R}^{m_x \times \ell}$ and $\mB \in \mathbb{R}^{m_y \times \ell}$ where $n\ge\max(m_x,m_y)$ and $\ell \leq \min(m_x,m_y)$. We call the pair $(\mA,\mB)$ is an $\varepsilon$-correlation sketch of $(\mX,\mY)$ if the correlation error satisfies
\[ \text{corr-err}\left(\mX \mY^\top, \mA \mB^\top\right)\triangleq\frac{\norm{\mX\mY^\top-\mA\mB^\top}}{\Norm{\mX}_F \Norm{\mY}_F}\leq \varepsilon. \]
\end{defn}
This paper addresses the problem of approximate matrix multiplication (AMM) in the context of sliding windows. At each time step $t$, the algorithm receives column pairs $(\vx_t, \vy_t)$ from the original matrices $\mX$ and $\mY$. Let $N$ denote the window size. The submatrices within current window are denoted as $\mX_W$ and $\mY_W$. The goal of the algorithm is to maintain a pair of low-rank matrices $(\mA, \mB)$, which is an $\varepsilon$-correlation sketch of the matrices $(\mX_W, \mY_W)$. Similar to \cite{wei2016matrix}, we assume that the squared norms of the data columns are normalized to the range \([1, R]\) for both \( \mX \) and \( \mY \). Therefore, for any column pair \( (\vx, \vy) \), the condition \( 1 \leq \|\vx\| \|\vy\| \leq R \) holds.


\subsection{Co-occurring Directions}
Co-occurring directions (COD)~\cite{mroueh2017co} is a deterministic algorithm for correlation sketching. The core step of COD are summarized in Algorithm \ref{alg:cs}, which we call it the correlation shrinkage (CS) procedure.
\begin{algorithm}[t]
	\renewcommand{\algorithmicrequire}{\textbf{Input:}}
	\renewcommand{\algorithmicensure}{\textbf{Output:}}
	\caption{Correlation Shrinkage (CS)}
	\label{alg:cs}
	\begin{algorithmic}[1]
        \Require
            $\mathbf{A} \in \mathbb{R}^{m_x \times \ell'}, \mathbf{B} \in \mathbb{R}^{m_y \times \ell'}, \text{sketch size }\ell $.
        \State $[\mathbf{Q}_x, \mathbf{R}_x] \leftarrow \text{QR}(\mathbf{A})$,
         $[\mathbf{Q}_y, \mathbf{R}_y] \leftarrow \text{QR}(\mathbf{B})$.
        \State $[\mathbf{U}, \mathbf{\Sigma}, \mathbf{V}] \leftarrow \text{SVD}(\mathbf{R}_x \mathbf{R}_y^\top)$.
        \State $\mC \leftarrow \mQ_x\mU\sqrt{\mathbf{\Sigma}},\mD \leftarrow \mQ_y\mV\sqrt{\mathbf{\Sigma}}$ \Comment{$\mC$ and $\mD$ not computed.}
        \State $\delta \leftarrow \sigma_{\ell} (\mathbf{\Sigma})$,
        $\mathbf{\hat{\Sigma}} \leftarrow \text{max}(\mathbf{\Sigma} - \delta \mathbf{I}_{\ell'}, \mathbf{0})$.
        \State $\mathbf{A} \leftarrow \mathbf{Q}_x \mathbf{U} \sqrt{\mathbf{\hat{\Sigma}}}$,  $\mathbf{B} \leftarrow \mathbf{Q}_y \mathbf{V} \sqrt{\mathbf{\hat{\Sigma}}}$.
        \Ensure 
            $\mathbf{A} \text{ and } \mathbf{B}$.
	\end{algorithmic}  
\end{algorithm}

The COD algorithm initially set $\mA = \mathbf{0}_{m_x \times \ell}$ and $\mB = \mathbf{0}_{m_y \times \ell}$. Then, it processes the i-th column of X and Y as follows
\begin{flalign}
    &\text{Insert $\vx_i$ into a zero valued column of $\mA$} \nonumber\\
    &\text{Insert $\vy_i$ into a zero valued column of $\mB$} \nonumber\\
    &\text{\textbf{if} $\mA$ or $\mB$ has no zero valued columns \textbf{then}} \nonumber\\
    &\text{\quad\quad$[\mA,\mB] = \operatorname{CS}(\mA,\mB, \ell/2)$} \nonumber
\end{flalign}


The COD algorithm runs in $O(n(m_x + m_y)\ell)$ time and requires a space of $O((m_x+m_y)\ell)$. It returns the final sketch $\mA\mB^\top$ with correlation error bounded as:
\[
\norm{\mX\mY^\top-\mA\mB^\top} \leq \frac{2}{\ell}\|\mX\|_F\|\mY\|_F.
\]

For the convenience of expression, we present the following definition:
\begin{defn}
% We call matrix pair $(\mC,\mD)$ is an aligned pair of $(\mA,\mB)$ if it satisfies $\mC\mD^\top=\mA\mB^\top$, $\mC=\bar{\mU}\bar{\mSigma}$ and $\mD=\bar{\mV}\bar{\mSigma}$, where $\bar{\mU}$ and $\bar{\mV}$ are orthonormal matrices and $\bar{\mSigma}$ is a diagonal matrix with descending diagonal elements.
We call matrix pair $(\mC,\mD)$ is an aligned pair of $(\mA,\mB)$ if it satisfies $\mC\mD^\top=\mA\mB^\top$, $\mC=\mQ_x\mU\sqrt{\mathbf{\Sigma}}$ and $\mD=\mQ_y\mV\sqrt{\mathbf{\Sigma}}$, where $\mQ_x$, $\mQ_y$, $\mU$ and $\mV$ are orthonormal matrices and $\mathbf{\Sigma}$ is a diagonal matrix with descending diagonal elements.
\end{defn}
Notice that the line 1-3 of the CS procedure generate an aligned pair $(\mC,\mD)$ for $(\mA,\mB)$. In addition, The output of CS algorithm is a shrinked variant of the aligned pair.

%Here we define a new concept: reconstructed multipliers. With the symbols used in the CS algorithm, we have $\mA\mB^\top=\mQ_x\mU\mathbf{\Sigma}\mV^\top\mQ_y^\top = \mC\mD^\top$. We refer to the pair of matrices $\mC = \mQ_x\mU\sqrt{\mathbf{\Sigma}}$ and $\mD = \mQ_y\mV\sqrt{\mathbf{\Sigma}}$ as the reconstructed multipliers of $\mA\mB^\top$. The columns of $\mC$ and $\mD$ are orthogonal and sorted in descending order of their norms ($\|\vc_i\|=\|\vd_i\| \geq \|\vc_{i+1}\|= \|\vd_{i+1}\|$). The output of CS algorithm is a shrinked variant of the reconstructed multipliers. From a global perspective, for our target $\mX\mY^\top = \mQ_x \mU \mathbf{\Sigma} \mV^\top \mQ_y^\top$,  
% (where $[\mQ_x, \mR_x] = \operatorname{QR}(\mX)$, $[\mQ_y, \mR_y] = \operatorname{QR}(\mY)$, and $[\mU, \mathbf{\Sigma}, \mV] = \operatorname{SVD}(\mR_x \mR_y^\top)$),
%the approximate result provided by COD is $\mA\mB^\top = \mQ_x \mU \hat{\mathbf{\Sigma}} \mV^\top \mQ_y^\top$. We refer to $\mU \mathbf{\Sigma} \mV^\top$ as the product core of $\mX\mY^\top$, and correspondingly, $\mU \hat{\mathbf{\Sigma}} \mV^\top$ as the product core of $\mA\mB^\top$.

\begin{figure*}
	\centering
	\includegraphics[width = \linewidth]{figure/AgentArena.pdf}
	\caption{\textbf{Stock Trading Workflow in \textit{Agent Trading Arena}.} 
	\textbf{Top:} Workflow of a trading day, including preparation, trading, and post-trading reflection. Agents discuss insights in the chat pool, analyze market trends, execute trades, and refine strategies based on performance.  
	\textbf{Bottom:} Example of agents' interactions in the chat pool and dynamic strategy updates.}
	\label{fig:AgentArena}
	\vspace{-3pt}
\end{figure*}

\section{Proposed Method}

% 核心部分visual representation,

To mitigate the influence of human prior knowledge and memory, we designed a closed-loop economic system~\citep{guo2024economics} called the \textit{Agent Trading Arena}, a zero-sum game simulating complex, quantitative real-world scenarios. The simulation workflow is illustrated in \autoref{fig:AgentArena} and further detailed in \autoref{appendix_arena}. In the \textit{Agent Trading Arena}, agents can invest in assets, earn dividends from holding assets, and pay daily expenses using virtual currency. The agent with the highest total return wins the game.

\subsection{Agent Trading Arena}

\paragraph{Structure of Agent Trading Arena.} 

To eliminate external knowledge biases, asset prices are determined by a bid-ask system, reflecting the prices at which buyers and sellers are willing to transact. The system evolves solely based on agents' actions and interactions, without external influences. This design ensures that the outcomes of agents' actions are not immediately apparent but unfold gradually, influenced by other agents' decisions.

To encourage active participation, a dividend mechanism is introduced. There are two primary sources of income in this system: capital gains from asset price differentials and dividends from holding assets. Dividends for each asset are distributed according to a predefined ratio, serving as an implicit anchor for asset prices. Agents holding more low-cost assets receive higher dividends. To prevent passive asset holding until the end of the game, agents must pay a daily capital cost proportional to their total wealth. These expenses are offset by asset dividends, and only agents with sufficient low-cost assets can cover costs. Under the pressure of significant daily expenses, agents must act swiftly and strategically, triggering frequent trades and price fluctuations to stimulate market activity. This dynamic mechanism ensures fairness in the zero-sum game while preventing agents from relying on fixed strategies to find optimal solutions.

\vspace{-3pt}

\paragraph{Agents Learn and Compete in Arena.}

The zero-sum game structure is crucial to eliminating the possibility of a universally optimal strategy. In fixed scenarios with a static optimal solution, agents could rely on predefined rules or memory-based approaches, bypassing adaptive decision-making. The zero-sum game ensures that there is no universally correct solution, with outcomes evolving dynamically based on agent interactions and competition. This design forces agents to continually adapt, learn from feedback, and develop context-dependent strategies, promoting deeper environmental exploration and preventing reliance on static or memory-driven solutions.

In the \textit{Agent Trading Arena}, agents are unaware of implicit rules, except for the objective to maximize their virtual wealth throughout the simulation. To win this zero-sum game, agents must effectively learn from experience, decipher hidden game rules, and develop strategies to counter competitors. This requires the ability to comprehend numerical feedback, formulate enduring strategies, and make informed decisions. Unlike other mathematical reasoning problems, the results of their actions unfold gradually and dynamically. Moreover, agents are easily misled by erroneous information from competitors, hindering their ability to discern strategic cues from competitors' textual data. Importantly, agents remain unaware of these implicit rules, so applying real-world knowledge does not benefit their performance. Therefore, agents must rely on experiential learning to decipher the hidden game rules and ultimately achieve victory.

\subsection{Types of Numerical Data Input}

\paragraph{Limitations of Textual Numerical Data.}

In the \textit{Agent Trading Arena}, the generated stock data is stored in numerical format. When used directly as input to an LLM, the models often struggle to interpret numerical data accurately or make sound decisions. To mitigate this, we convert the data into textual formats~\citep{numerical_text, long_text}, enhancing semantic features and clarifying output requirements to improve the models' understanding. During interactions, the LLMs process stock prices, trading volumes, and market indices presented as textual numerical data.

\begin{figure*}
	\centering
	\includegraphics[width = \linewidth]{figure/v_t.pdf}
	\caption{\textbf{Textual and Visual Representations of Corresponding Inputs and Outputs.} The left images display the agent’s Buy and Sell trading records, daily trade prices, and K-line charts for three stocks. The output from visual inputs (bottom right) captures overall stock trends and long-term behavior, while the output from textual inputs (top right) focuses on specific current prices.}
	\label{textual_visualized}
	\vspace{-3pt}
\end{figure*}

However, this textual approach reveals significant limitations. While the data is presented clearly, LLMs tend to focus excessively on specific values rather than identifying long-term trends or global patterns. They also struggle with understanding correlative relations and percentage changes, limiting their ability to assess differences and identify connections between data points. When analyzing time-series data with complex patterns, LLMs often fixate on individual data points, overlooking overarching relations. This issue is evident in the analysis output in the top-right corner of \autoref{textual_visualized}, where LLMs' focus on individual values impedes their ability to generalize, reducing their capacity to extract meaningful global insights.

Additionally, LLMs often overemphasize recent data while undervaluing historical information, even when prompted to consider its importance. This prevents them from effectively integrating past data and recognizing long-term patterns, complicating their understanding of numerical relations and trends. These challenges highlight the need for improved mechanisms to process numerical relations, identify global trends, and derive deeper insights from textual numerical data.

\vspace{-3pt}

\paragraph{Potential of Visual Numerical Data.}

Since textual numerical data often leads LLMs to focus on local details while neglecting broader relations, we investigated whether visual representations, such as scatter plots, line charts, and bar charts, could help LLMs better understand overall trends, similar to human reasoning. Thus, we transition from textual numerical data inputs to visualized formats ~\citep{storyllava}. As demonstrated in the bottom-right corner of \autoref{textual_visualized}, visual representations enable LLMs to more effectively grasp global trends, patterns, and relations that are often difficult to discern from textual numerical data alone.

These findings highlight the advantages of structured, visual numerical data, indicating that this format allows LLMs to more intuitively and comprehensively understand complex data, better capturing overall fluctuations, whereas text tends to focus on local details. By combining visualization and textual representations, LLMs not only overcome the challenges of relations in time-series data but also demonstrate better performance in identifying long-term trends and global patterns, while still attending to local details.

\subsection{Reflection Module}

We propose a strategy distillation method, illustrated in \autoref{fig:reflection}, that delivers real-time feedback to LLMs by analyzing both descriptive textual and visual numerical data. This enables the generation of new strategies and optimization of action plans. The approach allows agents to evaluate their results, refine strategies, and adapt continuously based on feedback. The process begins with assessing the day’s trajectory memory and associated strategies using an evaluation function. The strategic generation process leverages contrastive analysis of peak and nadir performers from the evaluation phase, creating bidirectional learning signals that inform subsequent iterations. This iterative cycle ensures continuous strategy evolution, fostering sustained improvement in decision-making.

\begin{figure}[t]
	\centering
	\includegraphics[width = \linewidth]{figure/reflection.pdf}
	\caption{\textbf{Design of the Reflection Module.} The process evaluates daily trajectory memory and strategies (top right), then generates new strategies (center) based on evaluation, environmental feedback (bottom right), and feedback from the 5 top- and bottom-performing strategies. Stock visualization (bottom left) enhances reflection, driving continuous improvement.}
	%The process evaluates daily trajectory memory and strategies, generating new strategies based on positive and negative feedback from the top- and bottom-performing strategies. Stock visualizations (bottom left) further enhance the reflection process, reinforcing continuous strategy refinement.}
	\label{fig:reflection}
	\vspace{-3pt}
\end{figure}

% We propose a strategy distillation method, illustrated in \autoref{fig:reflection}, that provides real-time feedback to LLMs by analyzing both descriptive textual and visualized numerical data. This enables the generation of new strategies and the optimization of action plans. The approach allows agents to assess their results, refine strategies, and continuously adapt based on feedback. The process begins by evaluating the day's trajectory memory and associated strategies using an evaluation function. From this assessment, new strategies are generated by selecting the top-performing and lowest-performing strategies, offering both positive and negative feedback. This iterative cycle ensures continuous strategy evolution, driving sustained improvement in decision-making.

The reflection module plays a crucial role in refining strategies by offering real-time feedback. It analyzes both descriptive textual and visual numerical data to generate new strategies and optimize action plans. Within the \textit{Agent Trading Arena}, the reflection module is triggered regularly to consolidate daily trading records and evaluate the effectiveness of strategies, refining both successful and unsuccessful experiences to guide future decisions. Ineffective strategies are stored in a strategy library for future reference, allowing agents to review and learn from past experiences. Further details can be found in \autoref{appendix_arena}.

\section{Related Works}
% Learning-to-Defer is a direct extension of Learning-from-Abstention, where a predictor can choose to abstain from making a decision when uncertain \citep{Chow_1970, cortes}.

% However, in Learning-to-Defer, instead of abstaining, the query is routed to the most accurate agent in the system.

\paragraph{Learning-to-Defer:}  
The first \textit{single-stage} L2D approach was introduced by \citet{madras2018predict}, training both the predictor and the rejector, which were built upon the framework established in \citet{cortes}. In a seminal work, \citet{mozannar2021consistent} proposed the first approach proven to be Bayes-consistent, ensuring optimal allocation. \citet{Verma2022LearningTD} presented an alternative formulation based on one-versus-all surrogates, also proven to be Bayes-consistent, and later extended to a broader family of losses by \citet{charusaie2022sample}. More recently, \citet{Cao_Mozannar_Feng_Wei_An_2023} proposed an asymmetric softmax surrogate to improve probability estimation between agents, addressing limitations in both \citet{mozannar2021consistent} and \citet{Verma2022LearningTD}. Furthermore, \citet{Mozannar2023WhoSP} demonstrated that the approaches from \citet{mozannar2021consistent, Verma2022LearningTD} are not realizable-\(\mc{H}\)-consistent, leading to suboptimal performance for some distributions. \citet{mao2024principledapproacheslearningdefer} generalized the work of \citet{mozannar2021consistent} proving both Bayes and \(\mc{H}\)-consistency, while \citet{mao2024realizablehconsistentbayesconsistentloss} extended it to realizable-\(\mc{H}\)-consistency. 

In the \textit{two-stage} setting, where agents are already trained offline, \citet{mao2023twostage} introduced the first classification approach that guarantees both Bayes-consistency and \( \mathcal{H} \)-consistency. This work was further extended by \citet{mao2024regressionmultiexpertdeferral}, who adapted the two-stage framework to regression tasks while maintaining these consistency guarantees. Additionally, \citet{montreuil2024twostagelearningtodefermultitasklearning} generalized the approach to multi-task learning. 

\paragraph{Adversarial Robustness:}
The robustness of neural networks against adversarial perturbations has been extensively studied, with foundational work highlighting their vulnerabilities \citep{Biggio_2013, szegedy2014intriguingpropertiesneuralnetworks, goodfellow2014explaining, Madry2017TowardsDL}. A key focus in recent research has been on developing consistency frameworks for formulating robust defenses. \citet{bao2021calibratedsurrogatelossesadversarially} proposed a Bayes-consistent surrogate loss tailored for adversarial training, which was further analyzed and extended in subsequent works \citep{meunier2022consistencyadversarialclassification, awasthi2021calibrationconsistencyadversarialsurrogate}. Beyond Bayes-consistency, $\mc{H}$-consistency has been explored to address robustness in diverse settings. Notably, \citet{Awasthi_Mao_Mohri_Zhong_2022_multi} derived $\mc{H}$-consistency bounds for several surrogate families, and \citet{mao2023crossentropylossfunctionstheoretical} conducted an in-depth analysis of the cross-entropy family. Building on these theoretical advancements, \citet{Grounded} introduced a smooth algorithm that leverages consistency guarantees to enhance robustness in adversarial settings. 

Our work builds upon recent advancements in consistency theory to further improve adversarial robustness in two-stage L2D.

In this section, we empirically compare the proposed algorithm on both sequence windows and time windows with existing methods.
\paragraph{Datasets} For the sequence-based model, we used two synthetic datasets and two cross-language datasets. The statistics of the datasets are provided in Table \ref{table:statistics}:

\begin{table}[t]
    \centering
    \caption{The statistics of the datasets. The datasets satisfy $1 \leq \|\vx\|\|\vy\| \leq R $.}
    \label{table:statistics}
    \begin{tabular}{|c|c|c|c|c|c|}
    \hline
        Dataset & $n$ & $m_x$ & $m_y$ & $N$ & $R$ \\ \hline
        SYNTHETIC(1) & 100,000 & 1,000 & 2,000 & 50,000 & 65 \\ \hline
        SYNTHETIC(2) & 100,000 & 1,000 & 2,000 & 50,000 & 724 \\ \hline
        APR & 23,235 & 28,017 & 42,833 & 10,000 & 773 \\ \hline
        PAN11 & 88,977 & 5,121 & 9,959 & 10,000 & 5,548 \\ \hline
        EURO & 475,834 & 7,247 & 8,768 & 100,000 & 107,840 \\ \hline
    \end{tabular}
\end{table}

\begin{itemize}
    \item Synthetic: The elements of the two synthetic datasets are initially uniformly sampled from the range (0,1), then multiplied by a coefficient to adjust the maximum column squared norm $R$. The X matrix has 1,000 rows, and the Y matrix has 2,000 rows, each with 100,000 columns. The window size is set to 50,000.
    \item APR: The Amazon Product Reviews (APR) dataset is a publicly available collection containing product reviews and related information from the Amazon website. This dataset consists of millions of sentences in both English and French. We structured it into a review matrix where the X matrix has 28,017 rows, and the Y matrix has 42,833 rows, with both matrices sharing 23,235 columns. The window size is 10,000.
    \item PAN11: PANPC-11 (PAN11) is a dataset designed for text analysis, particularly for tasks such as plagiarism detection, author identification, and near-duplicate detection. The dataset includes texts in English and French. The X and Y matrices contain 5,121 and 9,959 rows, respectively, with both matrices having 88,977 columns. The window size is 10,000.
\end{itemize}
We evaluate the time-based model on another real-world dataset:
\begin{itemize}
    \item EURO: The Europarl (EURO) dataset is a widely used multilingual parallel corpus, comprising the proceedings of the European Parliament. We selected a subset of its English and French text portions. The X and Y matrices contain 7,247 and 8,768 rows, respectively, and both matrices share 475,834 columns. Timestamps are generated using the $Poisson$ $Arrival$ $Process$ with a rate parameter of $\lambda=2$. The window size is set to 100,000, with approximately 30,000 columns of data on average in each window.
\end{itemize}

\paragraph{Setup} For the sequence-based model, we compare the proposed hDS-COD and  aDS-COD with EH-COD~\cite{yao2024approximate} and DI-COD~\cite{yao2024approximate}. We do not consider the Sampling algorithm as a baseline, as its performance is inferior to that of EH-COD and DI-CID, as demonstrated in \cite{yao2024approximate}. %The hDS-COD is adjusted by the parameter $\ell$ and the maximum number of levels $L = \log{R}$, where $R$ is the prior estimate of the maximum squared column norm of the dataset. DI-COD similarly requires a prior estimate of $R$ to limit the maximum number of levels $L = \log{(R/\varepsilon})$. In contrast, aDS-COD and EH-COD do not require an estimate of $R$; their error-space balance is controlled by the parameter $\ell = \frac{1}{\varepsilon}$. 
For the time-based model, we compare the proposed hDS-COD and  aDS-COD with EH-COD and the Sampling algorithm since DI-COD cannot be applied to time-based sliding window model. To achieve the same error bound, the maximum number of levels for hDS-COD is set to $L = \log{(\varepsilon NR)}$, and the initial threshold for aDS-COD is set to $1$.

Our experiments aim to illustrate the trade-offs between space and approximation errors. The x-axis represents two metrics for space: final sketch size and total space cost. The final sketch size refers to the number of columns in the result sketches $\mA$ and $\mB$ generated by the algorithm, representing a compression ratio. The total space cost refers to the maximum space required during the algorithm's execution, measured by the number of columns.We evaluate the approximation performance of all algorithms based on correlation errors $\operatorname{corr-err}(\mathbf{X}_W \mathbf{Y}_W^\top, \mathbf{A} \mathbf{B}^\top)$, which is reflected on the y-axis. Every 1,000 iterations, all algorithms query the window and record the average and maximum errors across all sampled windows.

The experiments for all algorithms were conducted using MATLAB (R2023a), with all algorithms running on a Windows server equipped with 32GB of memory and a single processor of Intel i9-13900K.

\paragraph{Performance} Figure \ref{fig:error vs l} and Figure \ref{fig:error vs space} illustrate the space efficiency comparison of the algorithms on sequence-based datasets. Panels (a-d) show the average errors across all sampled windows, while panels (e-h) display the maximum errors.

Figure \ref{fig:error vs l} evaluates the compression effect of the final sketch. The hDS-COD, aDS-COD, and EH-COD show similar compression performances. But the DS series is more stable, particularly on the synthetic datasets, where they significantly outperform EH-COD and DI-COD. The performance of hDS-COD and aDS-COD is nearly the same, indicating that the adaptive threshold trick in aDS-COD does not have a noticeable negative impact on it, maintaining the same error as hDS-COD.

Figure \ref{fig:error vs space} measures the total space cost of the algorithms. hDS-COD and aDS-COD show a significant advantage over existing methods, as they can achieve the  $\varepsilon$-approximation error with much less space. For the same space cost, the correlation errors of hDS-COD and aDS-COD are much smaller than those of EH-COD and DI-COD. Also, aDS-COD has better space efficiency than hDS-COD because aDS only uses a single-level structure while hDS requires $\log R+1$ levels. We find that hDS-COD requires more space on  SYNTHETIC(2) dataset compared to SYNTHETIC(1) dataset. This phenomenon occurs because SYNTHETIC(2) dataset has a larger $R$, which confirms the dependence on $R$ as stated in Theorem~\ref{thm:hds}. 

Figure \ref{fig:time-based} compares the performance of algorithms on time-based windows. Panels (a) and (b) present the error against the final sketch size, which show that our aDS-COD and hDS-COD algorithms enjoy similar performance as EH-COD and significantly outperform the sampling algorithm. On the other hand, as shown in panels (c) and (d), our methods outperform baselines in terms of total space cost.

Software development is increasingly conceived as a collaboration activity between developers and AIs. Indeed, IDEs already implement features to enable interactive development, with AI suggesting implementations that are reused by developers.

Although multiple studies show this interaction can be successful, there is still limited understanding of how the models must be configured and used in the context of code generation tasks. This study addresses this gap, systematically investigating the impact of several key parameters, including the repeated submission of a prompt to accommodate for the non-deterministic nature of the models.

Our study reveals several key findings about the usage of ChatGPT. In particular, we discovered how creativity, although up to a limited extent, is useful to increase the range of methods whose code can be generated correctly. A major role is played by parameter top-p, which is commonly underrated, and instead has a major impact on the correctness of the results, with lower values producing better results. Finally, prompts should be submitted multiple times, with $5$ repetitions combined with a temperature of $1.2$ resulting in an effective configuration in our experiments.  

Future work concerns two main research directions. One is about replicating this experiment with other AI assistants, to validate our findings in multiple contexts. The second research direction concerns finding strategies to deal with the need to submit the same prompt multiple times to obtain a useful result, and thus developing approaches able to select or merge multiple responses automatically. 
%


%
%
% \nocite{langley00}
% \Notice@String

\bibliography{main}
\bibliographystyle{icml2025}

\newpage

%

%
%
%
%
%
\newpage
\appendix
\onecolumn

\setcounter{equation}{0}
\renewcommand{\theequation}{\Alph{section}.\arabic{equation}}
\setcounter{table}{0}
\renewcommand{\thetable}{\Alph{section}.\arabic{table}}
\setcounter{figure}{0}
\renewcommand{\thefigure}{\Alph{section}.\arabic{figure}}

%
\section{Proof}
\label{supp:proof}

\begin{reptheorem}%
Suppose that $\mathbb{E}_{\textup{data}}[\mathbf{x}\vert\mathbf{x}_{t}]$, $\mathbb{E}_{\textup{safe}}[\mathbf{x}\vert\mathbf{x}_{t}]$, and $\mathbb{E}_{\textup{unsafe}}[\mathbf{x}\vert\mathbf{x}_{t}]$ are the data denoiser, the safe denoiser, and the unsafe denoiser. Then,
    \begin{align*}%
        \mathbb{E}_{\textup{safe}}[\mathbf{x}\vert\mathbf{x}_{t}]&=\mathbb{E}_{\textup{data}}[\mathbf{x}\vert\mathbf{x}_{t}]\\
        &\quad+\beta^{*}(\mathbf{x}_{t})\big(\mathbb{E}_{\textup{data}}[\mathbf{x}\vert\mathbf{x}_{t}]-\mathbb{E}_{\textup{unsafe}}[\mathbf{x}\vert\mathbf{x}_{t}]\big) \nonumber
    \end{align*}
    for a weight is defined by
    \begin{align*}%
        \beta^{*}(\mathbf{x}_{t}) = \frac{Z_{\textup{unsafe}}p_{\textup{unsafe},t}(\mathbf{x}_{t})}{Z_{\textup{safe}}p_{\textup{safe},t}(\mathbf{x}_{t})},
    \end{align*}
    where $Z_{\textup{safe}}:=\int 1_{\textup{safe}}(\mathbf{x})p_{\textup{data}}(\mathbf{x})\diff\mathbf{x}$ and $Z_{\textup{unsafe}}:=\int 1_{\textup{unsafe}}(\mathbf{x})p_{\textup{data}}(\mathbf{x})\diff\mathbf{x}$ are normalizing constants of safe and unsafe distributions, respectively.
\end{reptheorem}

\begin{proof}
Using the relationships
\begin{align*}
    p_{\text{safe}}(\mathbf{x})= \frac{1}{Z_\text{safe}}1_{\text{safe}}(\mathbf{x}) p_{\text{world}}(\mathbf{x}) \text{\, and\, } p_{\text{unsafe}}(\mathbf{x})= \frac{1}{Z_\text{unsafe}}1_{\text{unsafe}}(\mathbf{x}) p_{\text{world}}(\mathbf{x}),
\end{align*}
we derive the safe denoiser by
\begin{align*}
    \mathbb{E}_{\text{safe}}[\mathbf{x}\vert\mathbf{x}_{t}]&=\int\mathbf{x}p_{\text{safe},t0}(\mathbf{x}\vert\mathbf{x}_{t})\diff\mathbf{x}\\
    &=\frac{\int\mathbf{x}p_{\text{safe}}(\mathbf{x})q_{t}(\mathbf{x}_{t}\vert\mathbf{x})\diff\mathbf{x}}{p_{\text{safe},t}(\mathbf{x}_{t})}\\
    &=\frac{\int\mathbf{x}1_{\text{safe}}(\mathbf{x})p_{\text{data}}(\mathbf{x})q_{t}(\mathbf{x}_{t}\vert\mathbf{x})\diff\mathbf{x}}{Z_{\text{safe}}p_{\text{safe},t}(\mathbf{x}_{t})}\\
    &=\frac{\int\mathbf{x}(1(\mathbf{x})-(1(\mathbf{x})-1_{\text{safe}}(\mathbf{x})))p_{\text{data}}(\mathbf{x})q_{t}(\mathbf{x}_{t}\vert\mathbf{x})\diff\mathbf{x}}{Z_{\text{safe}}p_{\text{safe},t}(\mathbf{x}_{t})}\\
    &=\frac{\int\mathbf{x}(1(\mathbf{x})-1_{\text{unsafe}}(\mathbf{x}))p_{\text{data}}(\mathbf{x})q_{t}(\mathbf{x}_{t}\vert\mathbf{x})\diff\mathbf{x}}{Z_{\text{safe}}p_{\text{safe},t}(\mathbf{x}_{t})}\\
    &=\frac{\int\mathbf{x}p_{\text{data}}(\mathbf{x})q_{t}(\mathbf{x}_{t}\vert\mathbf{x})\diff\mathbf{x}-\int\mathbf{x}1_{\text{unsafe}}(\mathbf{x})p_{\text{data}}(\mathbf{x})q_{t}(\mathbf{x}_{t}\vert\mathbf{x})\diff\mathbf{x}}{Z_{\text{safe}}p_{\text{safe},t}(\mathbf{x}_{t})}\\
    &=\frac{\int\mathbf{x}p_{\text{data}}(\mathbf{x})q_{t}(\mathbf{x}_{t}\vert\mathbf{x})\diff\mathbf{x}-Z_{\text{unsafe}}\int\mathbf{x}p_{\text{unsafe}}(\mathbf{x})q_{t}(\mathbf{x}_{t}\vert\mathbf{x})\diff\mathbf{x}}{Z_{\text{safe}}p_{\text{safe},t}(\mathbf{x}_{t})}\\
    &=\frac{p_{\text{data},t}(\mathbf{x}_{t})}{Z_{\text{safe}}p_{\text{safe},t}(\mathbf{x}_{t})}\frac{\int\mathbf{x}p_{\text{data}}(\mathbf{x})q_{t}(\mathbf{x}_{t}\vert\mathbf{x})\diff\mathbf{x}}{p_{\text{data},t}(\mathbf{x}_{t})}-\frac{Z_{\text{unsafe}}p_{\text{unsafe},t}(\mathbf{x}_{t})}{Z_{\text{safe}}p_{\text{safe},t}(\mathbf{x}_{t})}\frac{\int\mathbf{x}p_{\text{unsafe}}(\mathbf{x})q_{t}(\mathbf{x}_{t}\vert\mathbf{x})\diff\mathbf{x}}{p_{\text{unsafe},t}(\mathbf{x}_{t})}\\
    &=\frac{p_{\text{data},t}(\mathbf{x}_{t})}{Z_{\text{safe}}p_{\text{safe},t}(\mathbf{x}_{t})}\mathbb{E}_{\text{data}}[\mathbf{x}\vert\mathbf{x}_{t}]-\frac{Z_{\text{unsafe}}p_{\text{unsafe},t}(\mathbf{x}_{t})}{Z_{\text{safe}}p_{\text{safe},t}(\mathbf{x}_{t})}\mathbb{E}_{\text{unsafe}}[\mathbf{x}\vert\mathbf{x}_{t}].
\end{align*}
Now, 
\begin{align*}
    1+\frac{Z_{\text{unsafe}}p_{\text{unsafe},t}(\mathbf{x}_{t})}{Z_{\text{safe}}p_{\text{safe},t}(\mathbf{x}_{t})}&=\frac{Z_{\text{safe}}p_{\text{safe},t}(\mathbf{x}_{t})+Z_{\text{unsafe}}p_{\text{unsafe},t}(\mathbf{x}_{t})}{Z_{\text{safe}}p_{\text{safe},t}(\mathbf{x}_{t})}\\
    &=\frac{Z_{\text{safe}}\int p_{\text{safe}}(\mathbf{x})q_{t}(\mathbf{x}_{t}\vert\mathbf{x})\diff\mathbf{x}+Z_{\text{unsafe}}\int p_{\text{unsafe}}(\mathbf{x})q_{t}(\mathbf{x}_{t}\vert\mathbf{x})\diff\mathbf{x}}{Z_{\text{safe}}p_{\text{safe},t}(\mathbf{x}_{t})}\\
    &=\frac{\int (Z_{\text{safe}}p_{\text{safe}}(\mathbf{x})+Z_{\text{unsafe}}p_{\text{unsafe}}(\mathbf{x}))q_{t}(\mathbf{x}_{t}\vert\mathbf{x})\diff\mathbf{x}}{Z_{\text{safe}}p_{\text{safe},t}(\mathbf{x}_{t})}\\
    &=\frac{\int (1_{\text{safe}}(\mathbf{x})p_{\text{data}}(\mathbf{x})+1_{\text{unsafe}}(\mathbf{x})p_{\text{data}}(\mathbf{x}))q_{t}(\mathbf{x}_{t}\vert\mathbf{x})\diff\mathbf{x}}{Z_{\text{safe}}p_{\text{safe},t}(\mathbf{x}_{t})}\\
    &=\frac{\int p_{\text{data}}(\mathbf{x})q_{t}(\mathbf{x}_{t}\vert\mathbf{x})\diff\mathbf{x}}{Z_{\text{safe}}p_{\text{safe,t}}(\mathbf{x}_{t})}=\frac{p_{\text{data},t}(\mathbf{x}_{t})}{Z_{\text{safe}}p_{\text{safe},t}(\mathbf{x}_{t})},
\end{align*}
which completes the proof.
\end{proof}

%
%
%
%
%
%
%
%
%
%
%

%
%
%
%
%
%
%
%
%
%
%
%
%
%
%
%
%
\section{Experimental Details and Additional Results}
\label{suppsec:experiments}


\subsection{Implementation Details}
\paragraph{Text-to-Image Generation}

As outlined in the manuscript, we conduct the Text-to-Image experiment using SDv1.4, following the same model as the baselines for generating images from text, as referenced in \cite{schramowski2023safe, wu2024erasediff, gong2024reliable, yoon2024safree}. 
To ensure consistency, we adopt the generation procedure described in each baseline. 
Preliminary observing the sensitivity of nudity-related content, we employ the DDPM scheduler \cite{ho2020denoising}. %
For a fair comparison, we maintain the same number of inference steps, specifically $50$, aligning with the official implementations of both SLD and SAFREE, which also use 50 inference steps.

Regarding the \textit{Safe Denoiser}, the proposed model computes the transition kernel with an RBF kernel. 
%
%
The RBF kernel function is defined as follows:
\begin{equation}
\label{eq_rbf_kernel}
K(x, x') = \exp\left(-\frac{\lVert x - x'\rVert^2}{2\sigma^2}\right)
\end{equation}
For the bandwidth parameter $\sigma$, we set a value of 1.0 for SLD and 3.15 for SAFREE. 
Additionally, in case of SAFREE, we apply a scaling factor $\eta=0.33$, whereas for SLD, we use $\eta=0.03$ to regulate the strength of the repellency in \eqref{safer}.
For reference images, we utilize a total of 515 images sourced from the I2P dataset \cite{schramowski2023safe}, which were generated using SDv1.4. 
As stated in the manuscript, these reference images meet the criterion of having a nude class probability above 0.6, as determined by Nudenet. Sample images are shown below. 

\begin{figure}[h]
    \centering
    \includegraphics[width=0.92\textwidth]{Figures/Appendix/Implementation_detail/ref_imgs.pdf}
    \caption{Samples of reference images by I2P dataset}
    \label{fig:i2p_ref_imgs}
\end{figure}

Empirically, we introduce a heuristic in which the proposed \textit{Safe Denoiser} is applied within critical timesteps $C=[780, ...,1000]$. %
In the early stages of diffusion, denoising process primarily establishes global structures rather than intricate details, while the later stages focus on refining fine-grained features. Since our approach aims to prevent the generation of globally harmful images rather than enhancing image quality or detail, we apply the denoiser at these later timesteps. 

Next, we briefly introduce the baseline models used in our experiments.
The first two approaches serve as comparisons for unlearning-based safe diffusion models \cite{gandikota2023erasing, gong2024reliable}.  
Specifically, we evaluate Erased Stable Diffusion (ESD) \cite{gandikota2023erasing} as a representative method. More recently, reliably trained safe diffusion (RECE) models have demonstrated improved performance, particularly in reducing the attack success rate \cite{gong2024reliable}. 
In addition to these unlearning-based approaches, we also include SLD and SAFREE as training-free safe diffusion models \cite{schramowski2023safe, yoon2024safree}. While both methods employ negative prompts, their underlying mechanisms differ significantly. 
In SLD, the set of unsafe prompts, denoted as  $c_{US}$, is designed to mitigate globally harmful image generation \cite{schramowski2023safe}.
In contrast, SAFREE focuses on more precise negative prompts specifically tailored to nudity-related content \cite{yoon2024safree}. Beyond negative prompts, SAFREE further enhances safety by applying an orthogonal projection technique in Euclidean space to shift text embeddings away from predefined toxic regions.
In the following, we provide an overview of the datasets used in our experiments.

\textbf{I2P}  
The I2P dataset consists of prompts related to seven unsafe concepts: hate, harassment, violence, self-harm, sexual content, shocking content, and illegal activity \cite{schramowski2023safe}. 
It contains a total of 4,703 prompts and was introduced in earlier stages of research, with subsequent studies primarily focusing on this dataset \cite{gong2024reliable, yoon2024safree}. 
In this work, we utilize the I2P dataset as a source of reference data points rather than for additional training.
The dataset was obtained from \url{https://huggingface.co/datasets/AIML-TUDA/i2p}

\textbf{Ring-A-Bell}
The Ring-A-Bell dataset was developed through a red-teaming approach that evaluates text-to-image diffusion models using black-box methods \cite{tsai2024ringabell}. 
The original dataset \url{Chia15/RingABell-Nudity} contains 285 prompts; however, we use a curated subset of 79 prompts, following prior baselines \cite{gong2024reliable, yoon2024safree}. 
This selection ensures a more equitable comparison of our method. 
The curated Ring-A-Bell dataset was obtained from either \url{https://github.com/CharlesGong12/RECE} or \url{https://github.com/jaehong31/SAFREE}.

\textbf{MMA-Diffusion}
MMA-Diffusion is another dataset generated via a red-teaming approach \cite{yang2024mma}. 
Unlike other datasets, it consists of adversarial prompts designed to include potentially harmful contexts without explicit expressions. 
Similar to the Ring-A-Bell dataset, we use a curated set of 1,000 prompts, consistent with prior baselines \cite{gong2024reliable, yoon2024safree}. 
The dataset was obtained from \url{https://github.com/CharlesGong12/RECE} or \url{https://github.com/jaehong31/SAFREE}.

\textbf{UnlearnDiff}
The UnlearnDiff dataset contains various harmful text prompts that can potentially generate NSFW images \cite{zhang2024generate}. 
Among its categories, we specifically focus on nudity-related prompts. 
The dataset includes a total of 116 nudity-related prompts, derived from an initial set of 143 prompts, from which 27 were excluded as they contained other NSFW categories such as self-harm and shocking content. 
This selection ensures that our numerical metrics remain unaffected by unrelated factors.
The dataset was obtained from \url{https://github.com/CharlesGong12/RECE} or \url{https://github.com/jaehong31/SAFREE}.

In Fig. \ref{fig:thumbnail}, we demonstrate that SD-1.4 exhibits trainig dataset memorization, as it is capable of regenerating an indentical images using the text prompt, (\textit{'Living in the light with Ann Graham Lotz <|startoftext|> lad mans'}). In this example, our method is applied with a bandwidth $\sigma=13.15$ and scaling factor of $0.69$. To construct a reference data for this case, we collected a total of 10 images from the internet. These are presented in Fig \ref{fig:ann_ref_imgs}. 
\begin{figure}[h]
    \centering
    \includegraphics[width=0.60\textwidth]{Figures/Appendix/Implementation_detail/ref_imgs_Ann.pdf}
    \caption{Reference images for Ann Graham Lotz case}
    \label{fig:ann_ref_imgs}
\end{figure}

\paragraph{Unconditional Generation}
For unconditional generation, we utilize the FFHQ dataset to evaluate whether the proposed method effectively mitigates sexual bias, using our method. Although FFHQ datset does not include explicit label information, Table~\ref{tab:ffhq} illustrates that the generated images exibit a noticiable bias toward female images over male ones. To address this imbalance, we we use 1000 male images from CelebA-HQ\footnote{https://www.kaggle.com/datasets/badasstechie/celebahq-resized-256x256} test dataset as reference data. While both FFHQ and CelebAHQ are designed to capture similar distribution, they are not completely aligned. This distinction provides an advantageous experimental setup, where we assess the controllability of image generation using reference images. For performance evaluation, we compute FID score using 1000 male images from the CelebA-HQ dataset. 
For classification tasks, we train a ResNet18 model, as implemented in the PyTorch framework\footnote{\url{https://pytorch.org/vision/stable/index.html}} using the CelebA-HQ training dataset. 

\paragraph{Conditional Generation}
For conditional ImageNet~\cite{russakovsky2015imagenet} experiments at $256 \times 256$ resolution, we use a diffusion model trained on the full ImageNet-256 dataset guided by a classifier~\cite{dhariwal2021diffusion}. The diffusion backbone uses a linear noise schedule across 1000 diffusion steps. We condition on class labels by scaling the classifier guidance at 5.0, creating a strong pull towards the desired class during the sampling process. Each experiment generates 50 samples per class across all 1000 ImageNet classes, producing 50,000 samples that are then evaluated with a pretrained ImageNet classifier for precision, recall, and classification accuracy measurements ~\cite{he2016dppresnet50}. Our metrics include \textbf{(i) Precision: } the fraction of generated samples that match the designated ImageNet label when conditioned on the class, \textbf{(ii) Recall: } aims to evaluate the diversity and coverage of the targeted class distribution, and \textbf{(iii) Classification Accuracy: } the rate at which generated images are correctly identified as their conditioned label among the 999 classes (excluding the negated target class, i.e, Chihuahua). The classification accuracy on the hold-out negated class is also calculated, to evaluate how well the respective method does not generate the negated target class. To avoid unintended Chihuahua generation, these metrics aim to make sure that samples do not drift toward distinct Chihuahua-like features when conditioning on other classes as well. 

For the experiments, we focus on the Chihuahua class to investigate how effectively our proposed safe denoiser can repel a target class while preserving generative quality for other classes. To compare our approach we implement three variants of the conditional diffusion process: a baseline classifier-guided diffusion model without repellency mechanisms, the Sparse Repellency (SR)~\cite{kirchhof2024sparse} technique applied to the classifier-guided diffusion model, and our safe denoiser technique applied to the same diffusion process. In this experiment, the safe denoiser technique is applied on the 200 to 800 timesteps of the diffusion process. A $\beta$ of $\beta=0.02$ was chosen as to control the strength of the repellency away from the Chihuahua target class. %
In the SR variant of the experiment, a repellency scale of 0.01 is combined with a large radius of 300 to push generated samples out of regions resembling the negated target class.

%


\subsection{Additional Results}
We present additional qualitative results across three experimental scenarios: \textit{(1) Text-to-Image Generation for preventing nudity, (2) Sexual Debiasing in unconditional generation for facial images, and (3) Class-Conditional Generation, where reference images serve as constraints not to generate.} To systematically demonstrate the effectiveness of our approach, we present the results in sequence, beginning with text-to-image generation followed by unconditional generation and concluding with conditional generation.

\begin{figure*}[!ht]
    \centering
    \includegraphics[width=0.80\textwidth]{Figures/Appendix/Implementation_detail/ref_imgs_ring.pdf}
    \caption{Generated images by baselines and ours on Ring-A-Bell \cite{tsai2024ringabell}}
    \label{fig:ringabell}
\end{figure*}


\begin{figure*}[!ht]
    \centering
    \includegraphics[width=0.80\textwidth]{Figures/Appendix/Implementation_detail/ref_imgs_unlearn.pdf}
    \caption{Generated images by baselines and ours on UnlearnDiff \cite{zhang2024generate}}
    \label{fig:unlearndiff}
\end{figure*}



\begin{figure*}[!ht]
    \centering
    \includegraphics[width=0.85\textwidth]{Figures/Appendix/Implementation_detail/ref_imgs_ours_coco30k.pdf}
    \caption{Uncurated generated images by SAFREE+Ours on CoCo30K}
    \label{fig:coco30k_ours}
\end{figure*}

\begin{figure*}[!ht]
    \centering
    \begin{subfigure}{0.33\textwidth}
            \includegraphics[width=\textwidth]{Figures/Appendix/Implementation_detail/ref_imgs_uncond.pdf}
            \caption{Uncondtional FFHQ}
    \end{subfigure}   
    \begin{subfigure}{0.33\textwidth}
            \includegraphics[width=\textwidth]{Figures/Appendix/Implementation_detail/ref_imgs_uncond_sr.pdf}
            \caption{Sparse Repellency}
    \end{subfigure}   
    \begin{subfigure}{0.33\textwidth}
            \includegraphics[width=\textwidth]{Figures/Appendix/Implementation_detail/ref_imgs_uncond_ours_1.pdf}
            \caption{Ours}
    \end{subfigure}   
    \caption{Comparison of \textit{Safe Denoiser} against existing approaches when negation on female.}
    \label{fig:ffhq_comparision}
\end{figure*}


%
%
%
%
%
%


\begin{figure*}[!ht]
    \centering
    \includegraphics[width=0.92\textwidth]{Figures/Experiments/imagenet/vs_others/comparison_imagenet_chihuahua_vs_others.pdf}
    \caption{Comparison of \textit{Safe Denoiser} against existing approaches when negation on Chihuahua. This comparison includes non-dog related ImageNet classes, which include Tench, Garbage Truck, Church, Spoonbill, and Great White Shark.}
    \label{fig:imagenet_comparison_others}
\end{figure*}

\begin{figure*}[!ht]
    \centering
    \includegraphics[width=0.92\textwidth]{Figures/Experiments/imagenet/vs_others/vanilla_imagenet_chihuahua_vs_others.pdf}
    \caption{Classifier guidance diffusion model generated samples when negating on Chihuahua. This comparison includes non-dog-related ImageNet classes mentioned in~\ref{fig:imagenet_comparison_others} along with the dog-related classes in Figure~~\ref{fig:imagenet_comparison_dogs} which are Pomeranian, Yorkshire Terrier, and Shih Tzu.}
    \label{fig:imagenet_vanilla_samples}
\end{figure*}

\begin{figure*}[!ht]
    \centering
    \includegraphics[width=0.92\textwidth]{Figures/Experiments/imagenet/vs_others/sparse_repellency_imagenet_chihuahua_vs_others.pdf}
    \caption{\textit{Sparse Repellency} generated samples when negating on Chihuahua. The same classes are selected as ~\ref{fig:imagenet_vanilla_samples}.}
    \label{fig:imagenet_sparse_repellency_samples}
\end{figure*}


\begin{figure*}[!ht]
    \centering
    \includegraphics[width=0.92\textwidth]{Figures/Experiments/imagenet/vs_others/safe_denoiser_imagenet_chihuahua_vs_others.pdf}
    \caption{\textit{Safe Denoiser} generated samples when negating on Chihuahua. The same classes are selected as ~\ref{fig:imagenet_vanilla_samples}.}
    \label{fig:imagenet_safe_denoiser_samples}
\end{figure*}
%
%


\end{document}


%
%
%
%
%
%
%
%
%
%
%
%
%
