\newcommand{\yell}[1]{\textcolor{red}{#1}}
% \newcommand{\assume}[1]{\textcolor{blue}{\textbf{#1}}}
\newcommand{\assume}[1]{#1}

\newtheorem{theorem}{Theorem}
\newtheorem{lemma}{Lemma}
\newtheorem{corollary}{Corollary}
\newtheorem{observation}{Observation}
\newtheorem{definition}{Definition}
\newtheorem{example}{Example}
\newtheorem{proposition}{Proposition}

%% OPERATORS:
\newcommand{\norm}[1]{\left|\left|#1\right|\right|}
\newcommand{\argmin}[2]{\textrm{argmin}_{#1}~#2}
\newcommand{\argmax}[2]{\textrm{argmax}_{#1}~#2}
% Inner product
\newcommand{\ip}[2]{\left\langle#1, #2\right\rangle}
% Trace
\newcommand{\tr}{\textrm{tr}}
% Expected value
\newcommand{\E}[2]{\mathbb{E}_{#1}\left[#2\right]}
% Sample mean
\newcommand{\Ehat}[1]{\hat{\mathbb{E}}\left[#1\right]}
% Variance
\newcommand{\Var}[2]{\textrm{Var}_{#1}\left[#2\right]}
% Covariance
\newcommand{\Cov}[2]{\textrm{\textbf{Cov}}_{#1}\left[#2\right]}
% Indicator
\newcommand{\ind}[1]{\mathbbm{1}\left\{#1\right\}}
\newcommand{\indpm}[1]{\mathbbm{1}^{\pm}\left\{#1\right\}}
% sech
\newcommand{\sech}[0]{\textrm{sech}}
% diag
\newcommand{\diagm}[1]{\textrm{diagm}\left(#1\right)}
% supp
\newcommand{\supp}{\text{supp}}
% independent
\newcommand\independent{\protect\mathpalette{\protect\independenT}{\perp}}
\def\independenT#1#2{\mathrel{\rlap{$#1#2$}\mkern2mu{#1#2}}}

%\newenvironment{bsmallmatrix}
%  {\left[\begin{smallmatrix}}
%  {\end{smallmatrix}\right]}

% see line at top of main file to show/hide notes
\ifdefined\ShowNotes
  \newcommand{\colornote}[3]{{\color{#1}\bf{#2 #3}\normalfont}}
\else
  \newcommand{\colornote}[3]{}
\fi

\definecolor{darkred}{rgb}{0.7,0.1,0.1}
\definecolor{darkgreen}{rgb}{0.1,0.5,0.1}
\definecolor{cyan}{rgb}{0.7,0.0,0.7}
\definecolor{dblue}{rgb}{0.2,0.2,0.8}
\definecolor{maroon}{rgb}{0.76,.13,.28}
\definecolor{burntorange}{rgb}{0.81,.33,0}
\definecolor{royalpurple}{rgb}{0.47,.31,0.66}

% \newcommand {\note}[1]{\colornote{maroon}{}{#1}}
\newcommand {\todo}[1]{\colornote{cyan}{TODO}{#1}}
\newcommand {\authorone}[1]{\colornote{darkgreen}{A1:}{#1}}
\newcommand {\authortwo}[1]{\colornote{burntorange}{A2:}{#1}}
\newcommand {\authorthree}[1]{\colornote{red}{A3:}{#1}}

% not a colornote since we don't want these to ever be removed from the document
\ifdefined\ShowNotes
  \newcommand{\num}[1]{{\color{red}\bf{#1}\normalfont}}
\else
  \newcommand{\num}[1]{#1}
\fi


\newcommand{\cyrus}[1]{{\color{blue}[CZ: #1]}}
\newcommand{\qz}[1]{{\color{purple}[Qizheng: #1]}}
\newcommand{\hermann}[1]{{\color{red}[Hermann: #1]}}

\newcommand{\name}{\textsc{LoRAX}\xspace}

\newcommand{\fillme}{{\bf XXX}\xspace}
\newcommand{\eg}{{\it e.g.,}\xspace}
\newcommand{\ie}{{\it i.e.,}\xspace}
\newcommand{\etal}{{\it et.~al}\xspace}
\newcommand{\bigO}{\mathrm{O}}

% \setlist[itemize]{topsep=4pt, itemsep=4pt, parsep=1.5pt}

\newenvironment{packeditemize}{\begin{list}{$\bullet$}{\setlength{\itemsep}{0.5pt}\addtolength{\labelwidth}{-4pt}\setlength{\leftmargin}{2ex}\setlength{\listparindent}{\parindent}\setlength{\parsep}{1pt}\setlength{\topsep}{2pt}}}{\end{list}}

\def\FWName{LowRA\xspace}

\def\nummodels{4\xspace}
\def\numdatasets{4\xspace}
\def\lowestprec{1.15\xspace}