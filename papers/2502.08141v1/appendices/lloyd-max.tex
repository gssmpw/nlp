\section{Weighted Lloyd-Max Algorithm}
\label{sec:lloyd-max}

In this subsection, we briefly introduce the Weighted Lloyd-Max Algorithm, extended from the original Lloyd-Max algorithm \cite{lloyd1982least,max1960quantizing}.

Let \(\{x_i\}_{i=1}^N \subset \mathbb{R}^d\) be data points with corresponding weights \(\{w_i\}_{i=1}^N\), \(w_i > 0\). We seek to find \(K\) cluster centers \(\{y_j\}_{j=1}^K \subset \mathbb{R}^d\) minimizing the weighted mean-squared error:
\[
\min_{\{c(i)\}, \{y_j\}} 
\sum_{i=1}^N w_i \,\bigl\|x_i - y_{c(i)}\bigr\|^2,
\]
where \(c(i) \in \{1, \dots, K\}\) is the cluster index assigned to \(x_i\). The weighted Lloyd's algorithm alternates between:

\noindent\textbf{1.\ Assignment (E-step):} Assign each data point \(x_i\) to the cluster center closest in Euclidean distance:
\begin{equation}
  c(i) \;\leftarrow\; \underset{1 \le j \le K}{\mathrm{arg\,min}}\; \bigl\| x_i - y_j \bigr\|^2.
  \tag{E-step}\label{eq:e-step}
\end{equation}

\noindent\textbf{2.\ Update (M-step):} Recompute each cluster center \(y_j\) as the weighted centroid of the points assigned to it:
\begin{equation}
  y_j \;\leftarrow\; \frac{\sum_{i : c(i) = j} w_i \, x_i}{\sum_{i : c(i) = j} w_i}.
  \tag{M-step}\label{eq:m-step}
\end{equation}

These steps are repeated until convergence or until a stopping criterion (e.g., a maximum number of iterations) is met.