\section{Introduction}
% With rapid advancements in sensor technology and artificial intelligence, Autonomous Driving Systems (ADSs) are increasingly emerging as pivotal solutions for urban transportation. However, despite these technological strides, ADSs continue to present critical road safety challenges, as evidenced by a rising number of accidents involving these systems~\cite{Waymo_Driver,ADAS_NHTSA_report}. In response, various testing methods have been developed to evaluate the quality, reliability, and stability of ADSs. Among these, scenario-based simulation testing has gained significant traction in both industry and academia due to its cost-effectiveness, scalability, and flexibility in configuring diverse test environments~\cite{10.1145/3540250.3549111}. Typically, these testing methods leverage expert knowledge or pre-recorded data, providing a foundation for constructing realistic, context-specific scenarios within simulators~\cite{nalic2020scenario}. Recent advancements in ADS testing have investigated diverse data sources and scenario generation techniques to improve the realism and complexity of test scenarios.
With advancements in sensor technology and artificial intelligence, Autonomous Driving Systems (ADSs) are becoming pivotal to urban transportation. However, despite these advancements, ADSs still pose significant road safety challenges, as highlighted by a growing number of related accidents. To address this, various testing methods have been developed to evaluate the quality, reliability, and stability of ADSs. Among them, scenario-based simulation testing has gained popularity in industry and academia due to its cost-effectiveness, scalability, and flexibility in configuring diverse test environments~\cite{10.1145/3540250.3549111}. These methods often leverage expert knowledge or pre-recorded data to construct realistic and context-specific scenarios in simulators~\cite{nalic2020scenario}. Recent advancements have focused on utilizing diverse data sources and scenario-generation techniques to enhance the realism and complexity of test scenarios. Tools such as Law-Breaker~\cite{sun2022lawbreaker}, RMT~\cite{deng2021rmt}, and TARGET~\cite{deng2023targetautomatedscenariogeneration} generate test scenarios based on traffic regulations, challenging ADSs to navigate complex conditions while adhering to these rules. Such expert-driven approaches are central to scenario-based testing. However, traffic regulations alone lack the detail needed for robust scenario-based testing, as they fail to specify vehicle trajectories, diverse traffic participants, or detailed road parameters, limiting their utility~\cite{TexasDMV2022}.
%Yang: Revised
% Tools like Law-Breaker~\cite{sun2022lawbreaker}, RMT~\cite{deng2021rmt}, and TARGET~\cite{deng2023targetautomatedscenariogeneration} build test scenarios based on traffic regulations, challenging ADSs to comply with these rules in complex road conditions. These scenarios, informed by expert knowledge, represent an expert-driven approach in scenario-based testing. Traffic regulations alone lack the necessary detail for concrete scenario-based testing~\cite{TexasDMV2022}, as they do not specify vehicle trajectories, diverse traffic participants, or detailed road parameters, limiting their utility as a robust data source for ADS testing.

To tackle the issues outlined above, a distinct approach is proposed to reconstruct scenarios using real-world crash data. This method, initially introduced by ~\citet{gambi2019generating} with AC3R, utilizes crash summaries to capture scenarios that challenge even experienced human drivers, thus providing highly pertinent cases for testing ADS.
Crash data from the National Highway Traffic Safety Administration's (NHTSA) CIREN dataset~\cite{CIREN} provides reliable, standardized accounts of crash events, offering insights into the underlying causes of critical incidents. Building on this, ADEPT~\cite{wang2022adept} and LCTGen~\cite{tan2023language} use crash summaries to reconstruct scenarios. However, these methods still encounter key limitations: (1) Relying solely on crash report summaries excludes vital visual information from crash sketches, which depict vehicle positions and environmental context. (2) These approaches often reconstruct vehicle motion trajectories based on limited behavioral data—such as straightforward instructions like "Turn left at the corner"—which can lead to unrealistic scenario recreations. Integrating crash data with detailed visual and trajectory information holds significant potential for more accurate and realistic scenario reconstructions, enhancing the fidelity of ADS testing. (3) large language models (LLMs) were used as an information extractor, but did not manage the hallucination problem it had.
%Yang revised
Recently, LLMs have demonstrated remarkable capabilities in handling tasks that previously required multiple modules—such as CLIP for visual data and BERT for textual data—while also supporting multimodal inputs. Applications of LLMs in software testing have also been investigated. For instance, in LCTGen~\cite{tan2023language}, GPT model~\cite{brown2020language} was employed for information extraction, leading to improvements in efficiency and accuracy over traditional NLP techniques. Similarly, a process was proposed in LEAD~\cite{tian2024llm} in which LLMs are used to extract scenario information from autonomous driving video datasets and configure scenario parameters. This is achieved by inputting video keyframes into the LLM and applying prompt engineering to obtain structured scenario descriptions. While LLMs demonstrate strong generalization capabilities and adaptability, they are also limited by the issue of hallucination, where seemingly reasonable but incorrect or misleading responses are generated. Recent research has shown that hallucinations are prevalent in LLMs, even when methods like in-context learning are applied to guide model responses~\cite{huang2023survey}, raising concerns about the reliability of such outputs.
%Yang revised
We propose a novel framework called \textbf{{\tool}} for scenario-based test case generation for ADSs to address existing limitations—such as the loss of map information, inaccuracies in path planning, and hallucination issues in using LLMs, which together contribute to unrealistic scenario construction.
%Yang we need a name for the framework -> Siwei: TRACE
In TRACE, we address a few challenges: (1) To describe scenario information, we develop a new Domain-Specific Language (DSL) inspired by the TARGET framework~\cite{deng2023targetautomatedscenariogeneration}. This DSL enables a more concise and precise definition of autonomous driving test scenarios by effectively capturing environment details, road networks, and traffic participants, thus providing greater expressiveness and accuracy compared to existing standards like OpenScenario~\cite{ASAM2021}. (2) Our proposed framework utilizes GPT-4o to extract environmental and road network information from multimodal data. Inspired by SelfCheckGPT~\cite{manakul2023selfcheckgpt}, a self-validation process is introduced, combining in-context learning and Chain-of-Thought prompting techniques to mitigate hallucination in LLMs. (3) For path planning, we integrate crash reports containing vehicle trajectory and map lane information, and vehicle waypoints to build a knowledge base, enabling a GPT with path planning capabilities, named {\tooltwo}. (4) %yang add for evaluations our work..
Through experiments, the reliability and superior performance of {\tool} were demonstrated. The approach was evaluated in terms of information extraction accuracy, generated scenario quality and utility, demonstrating improved realism and a greater number of generated scenarios compared to the state-of-the-art baseline. Our contributions are as follows:\vspace{-4mm}
\begin{itemize}
    \setlength{\itemsep}{0pt}  % 减少项目之间的间距
    \setlength{\parskip}{0pt}  % 移除段落之间的额外空白
\item ADS Test Case Generation from Unstructured Data: {\tool} is the first to generate realistic ADS test cases using unstructured multimodal data (crash summaries and sketches) to simulate complex driving scenarios. We developed a new DSL to enhance road modeling with attributes like road dimensions and precise actor coordinates and we proposed to use GPT with a knowledge base of real waypoints to generate realistic vehicle trajectories through crash data.

\item Reducing LLM Hallucination with Structured Prompting: We introduce a domain-specific LLM prompting and self-validation process to address hallucinations, enhancing the accuracy of LLM responses for ADS scenario generation.

% \item {\tooltwo} for Enhanced Path Planning: Leveraging ChatGPT and a knowledge base of real waypoints, {\tooltwo} generates realistic vehicle trajectories using crash data, improving path planning in ADS tests.

% \item Novel DSL for Precise Road Modeling: We developed a new DSL to enhance road modeling with attributes like road dimensions and precise actor coordinates, leading to more accurate collision simulations and context-aware reconstructions.

\item Comprehensive Test Scenario Evaluation: Testing with 50 NHTSA crash reports, {\tool} generated 290 scenarios, identifying 127 critical ones. It improved scene fidelity by 50\% compared to SOTA-LCTGen, validated on MetaDrive and BeamNG platforms.

    % \item Novel ADS Test Case Generation from Unstructured Multimodal Data: To the best of our knowledge, {\tool} is the first to construct realistic ADS test cases by integrating unstructured multimodal datasets, including crash summaries and sketches, to simulate concrete challenging driving scenarios.
    % %The effectiveness of our approach has been experimentally validated.
    % \item Mitigation of LLM Hallucination through Prompting and Self-validation: We propose a structured LLM prompting process that incorporates domain-specific knowledge, addressing common limitations in LLM applications such as hallucination. This approach significantly improves the accuracy and reliability of LLM-generated responses for ADS scenario generation.
    % \item Development of {\tooltwo} for Realistic Path Planning: We created a ChatGPT-powered model, {\tooltwo}, which uses a knowledge base built from map and real waypoints information, combined with crash summaries, to generate realistic vehicle trajectories, enhancing path planning in ADS test scenarios.
    % \item We designed a novel DSL that enhances road network modeling by incorporating detailed attributes (e.g., road length, width) to improve trajectory accuracy. This DSL uses detailed coordinate systems for actor positions, replacing traditional, coarse vector representations, thereby reducing inaccuracies in collision simulations and enabling more precise, context-aware scene reconstructions.
    % \item Evaluation of Generated Test Scenarios: Based on 50 crash reports from NHTSA, our approach, {\tool}, successfully tested three autonomous driving systems across two simulation platforms, MetaDrive and BeamNG. Of the 290 constructed test scenarios, 127 were identified as critical due to resulting in vehicle collisions, such as failure to avoid obstacles ahead or inability to yield to oncoming traffic from the side. User feedback further indicates that, compared to the most relevant SOTA-LCTGen, {\tool} improves scene fidelity to the original crash reports by approximately 50\%.
\end{itemize}

The remainder of the paper is organized as follows: Section ~\ref{sec:motivation} outlines the motivation for extracting detailed scenario information from crash sketches and addressing LLM hallucinations. Section ~\ref{sec:methodology} details our proposed framework - {\tool}, while Section ~\ref{sec:experiments} describes the experimental setup. Section ~\ref{sec:results} presents an analysis of our findings. Section ~\ref{sec:re_work} reviews recent advancements in scenario-based ADS testing and LLM applications. Section ~\ref{sec:discuss} discusses limitations and potential factors influencing our results. Finally, Section ~\ref{sec:conclusion} summarises our work, and Section ~\ref{sec:data_ava} includes information on accessing the code and data.
%yang revised. stop here.