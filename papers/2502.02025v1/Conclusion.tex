
\section{Conclusion}
\label{sec:conclusion}
% In this paper we introduced TRACE, a novel framework for generating critical scenario-based test cases for ADS by extracting detailed data from real-world crash reports. TRACE leverages advanced multimodal LLMs and incorporates innovative techniques, such as in-context learning, chain-of-thought prompting, and self-validation, to enhance the realism and accuracy of scene reconstructions. Experimental evaluations demonstrated that TRACE achieves high accuracy in DSL extraction, with 100\% accuracy in actor and environment identification and strong performance in generating realistic and effective test scenarios across two simulation platforms, MetaDrive and BeamNG.

% Our results indicate that TRACE significantly outperforms baseline methods in scenario generation and bug detection efficiency, identifying critical ADS issues while optimizing for time efficiency. Ablation studies further underscore the value of self-validation and modularized extraction processes, confirming that these components are essential for accurate scenario reconstruction and DSL attribute extraction.

% Overall, TRACE offers a robust, scalable solution for ADS scenario testing, closing crucial gaps in current methods by improving scene realism and efficiency. Future research could explore expanding TRACE’s capabilities to additional simulation environments and incorporating even more granular crash data, further enhancing ADS safety and reliability through rigorous, data-driven testing methodologies.
%In this paper, we present TRACE, a framework for generating critical ADS test scenarios based on real-world crash data. By leveraging multimodal LLMs with methods like in-context learning and self-validation, TRACE enhances the realism and accuracy of scene reconstructions. Experiments showed TRACE achieves 100\% accuracy in actor and environment identification and effectively generates realistic test scenarios on MetaDrive and BeamNG platforms. TRACE outperformed baseline methods in both scenario realism and bug detection efficiency, identifying critical ADS issues with time efficiency. Ablation studies highlighted the importance of TRACE’s self-validation and modular processes for accurate scenario and DSL extraction. TRACE provides a scalable, robust ADS testing solution, addressing gaps in scene realism and testing efficiency.

In this paper, we present {\tool}, a framework for generating critical ADS test scenarios based on real-world multi-modality crash data. By leveraging corresponding multimodal large language models (LLMs) with techniques such as in-context learning and self-validation, {\tool} enhances the realism and accuracy of scene reconstructions, outperforming the most relevant baseline method in both scenario realism and bug detection efficiency. {\tool} is among the first to systematically generate critical test scenarios for machine learning-enabled cyber-physical systems by utilizing multi-modality data and large language models. Investigating how to integrate Retrieval-Augmented Generation (RAG), foundation models, and LLMs to further improve scenario generation accuracy and criticality—particularly enhancing the realism of learned trajectories from other vehicles—remains a promising direction for future work.
\vspace{-2mm}