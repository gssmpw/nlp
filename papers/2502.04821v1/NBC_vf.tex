\documentclass[12pt]{amsart}
\usepackage{amsfonts}
\usepackage{amsmath}
\usepackage{amssymb}
\usepackage[margin=1.2in]{geometry}
\usepackage{verbatim}
\setcounter{MaxMatrixCols}{30}
\usepackage{xcolor}
\usepackage{accents}
\newcommand{\ubar}[1]{\underaccent{\bar}{#1}}

% user defined functionality
\newtheorem{theorem}{Theorem}[section]
\newtheorem{corollary}{Corollary}[section]
\newtheorem{lemma}{Lemma}[section]
\newtheorem{definition}{Definition}[section]
\theoremstyle{remark}
\newtheorem{remark}[theorem]{Remark}


\def \vector#1{\boldsymbol{#1}}
\newcommand {\D} {\displaystyle}
\usepackage{bm}
\newcommand{\bmt}[1]{\widetilde{\bm{#1}}}
\newcommand{\bmc}[1]{\check{\bm{#1}}}
\newcommand{\bmb}[1]{\overline{\bm{#1}}}
\newcommand{\cl}[1]{\left\lceil #1 \right\rceil}
\newcommand{\fl}[1]{\left\lfloor #1 \right\rfloor}
\newcommand{\wto}{\rightharpoonup}
\newcommand{\grad}{\nabla}
\newcommand{\lap}{\Delta}
\newcommand{\tq}{\tau_q}
\newcommand{\tqa}{\tq^\alpha}
\newcommand{\tT}{\tau_T}
\newcommand{\tTb}{\tT^\beta}
\newcommand{\half}{\frac{1}{2}}
\newcommand {\imbed} {\hookrightarrow}

%%%Convergence order
\newcommand {\oo}[1]{{\mbox{\em o}}\left(#1\right)}
\newcommand {\OO}[1]{{\mathcal O}\left(#1\right)}

\DeclareMathOperator*{\esssup}{ess\,sup}
\DeclareMathOperator{\argument}{arg}
%convolution kernel
\newcommand{\g}[1]{{g_{#1}}}
\newcommand{\ggamma}{\g{\gamma}}    % t^{-gamma}
\newcommand{\galpha}{\g{\alpha}}
\newcommand{\gbeta}{\g{\beta}}
\newcommand{\ggammamin}{\g{\gamma-1}} % t^{1-\gamma}
\newcommand{\galphamin}{\g{\alpha-1}}
\newcommand{\gbetamin}{\g{\beta-1}}

% inner products
\newcommand {\scal}[2]{\left(#1,#2\right)}
\newcommand{\inp}[3]{\left(#1,#2\right)_{#3}}
\newcommand{\inpX}[3]{\left\langle #1,#2 \right\rangle_{#3}}
\newcommand{\inpd}[2]{\inpX{#1}{#2}{{\hko{1}}^\ast \times \hko{1}}}
\newcommand{\inpds}[2]{\left\langle #1,#2 \right\rangle}
\newcommand{\inpdhk}[2]{\inpX{#1}{#2}{\hk{1}^\ast \times \hk{1}}}

%norm 
\newcommand{\nrm}[1]{\left\| #1 \right\|}
\newcommand{\abs}[1]{\left\lvert #1 \right\rvert}
\newcommand{\vnorma}[1]{\left\|#1\right\|}
\newcommand {\meas}[1] {\left|#1\right|}
%integration variables
%\newcommand{\inpd}[2]{\left\langle #1, #2 \right\rangle}
\DeclareMathOperator{\di}{d\hspace{-1.5pt}}
\newcommand{\dx}{ \di x}
\newcommand{\dX}{ \di \X}
\newcommand {\dt}{ \di t}
\newcommand{\ds}{\di s}
\newcommand{\dtau}{\di \tau}
\newcommand {\dxi}{\ \di \xi}
\newcommand{\dz}{\di z}
\newcommand{\js}{j_s}
\newcommand {\sigmab}{\vector{\sigma}}
\newcommand {\dg}{ \di \sigmab}


\newcommand {\CC } {{\mathbb C}} %\Bbb
\newcommand {\NN } {{\mathbb N}}
\newcommand {\RR } {{\mathbb R}}
\newcommand{\veps}{\varepsilon}

%%%Math operators
\newcommand {\diver } {\nabla \cdot }
\newcommand {\rotor } {\nabla \times }
\newcommand{\rot}{\nabla\times}
\newcommand{\dvr}{\nabla\cdot}

%%%Time Derivatives
\newcommand {\pdt}{\partial_t}
\newcommand {\pdtt}{\partial_{tt}}

\newcommand {\I}{[0,T]}
\newcommand {\Iopen}{(0,T)}
\newcommand {\Iclosed}{[0,T]}
\newcommand{\transpose}{\mathsf{T}}

\newcommand {\domain}{\Omega}
\newcommand{\vct}[1]{\boldsymbol{#1}}
\newcommand {\normal}{\vct{\nu}}
\newcommand {\lp}[1]{\Leb^{#1} (\domain )}
\newcommand {\Lp}[1]{{\bf L}^{#1} (\domain )}



\DeclareMathOperator{\Leb}{L}
\DeclareMathOperator{\Lip}{Lip}
\DeclareMathOperator{\Cont}{C}
\DeclareMathOperator{\Hi}{H}

\newcommand {\hk}[1]{\Hi^{#1}(\domain )}
\newcommand {\hko}[1]{\Hi^{#1}_0(\domain )}
\newcommand {\hkT}[2]{\Hi^{#1}(#2)}

%%%H^k vector space
\newcommand {\Hk}[1]{{\bf H}^{#1}(\domain)}
\newcommand {\Hko}[1]{{\bf H}_0^{#1}(\domain)}

%%%Hcurl Hdiv spaces
\newcommand {\curl}{{\rm \bf curl}}
\newcommand {\divv}{{\rm \bf div}}
\newcommand {\Hcurl}{{\bf H}(\curl;\domain)}
\newcommand {\Hcurldual}{{\bf H}^{-1}(\curl;\domain)}
\newcommand {\Hcurlpar}[2]{{\bf H}^{#1}(\curl,#2)}
\newcommand {\Hocurl}{{\bf H}_0(\curl;\domain)}
\newcommand {\Hocurldual}{{\bf H}_0^{-1}(\curl;\domain)}
\newcommand {\Hdiv}{{\bf H}(\divv;\domain)}
\newcommand {\Hodiv}{{\bf H}_0(\divv;\domain)}

\newcommand {\lpIlp}{\Leb^2\left(\Iopen,\lp{2}\right)}
\newcommand {\lpILp}{\Leb^2\left(\Iopen,\Lp{2}\right)}
\newcommand {\lpkIX}[2]{\Leb^{#1}\left(\Iopen,#2\right)}

\newcommand {\cIX}[1]{\Cont\left(\I,#1\right)}
\newcommand {\ckIlp}[2]{\Cont^{#1}\left(\I,\lp{#2}\right)}
\newcommand {\ckIX}[2]{\Cont^{#1}\left(\I,#2\right)}


%%%Space element, Vectors
%\def \vector#1{\mathbf{#1}}
\def \matrix#1{\mbox{\boldmath ${#1}$}}
\newcommand {\X}{\vector{x}}
\newcommand {\uu}{\bm{u}}
\newcommand {\ff}{\bm{f}}
\newcommand{\U}{\vector{u}}

\newcommand{\quand}{\quad \text{ and } \quad}

\usepackage{comment}
\usepackage[shortlabels]{enumitem}

\makeatletter
\@namedef{subjclassname@2020}{%
  \textup{2020} Mathematics Subject Classification}
\makeatother

\usepackage{xcolor}
\usepackage[hidelinks]{hyperref}
\hypersetup{
    colorlinks = true,
    linkcolor = red,
    anchorcolor = red,
    citecolor = green,
    filecolor = red,
    urlcolor = red,
    linktoc=page
    }
\usepackage{cleveref}

\usepackage{graphicx}
\usepackage{subfigure}
\usepackage{float}
\usepackage{caption}

\begin{document}
	
		
	\title[ISP semilinear pseudo-parabolic equation]{A time-dependent inverse source problem for a semilinear pseudo-parabolic equation with Neumann boundary condition}
	

	
	\author[K.~Van~Bockstal]{Karel Van Bockstal$^1$} 
	\thanks{This work was supported by grant no.~AP23486218
 of the Ministry of  Science and High Education of the Republic
of Kazakhstan (MES RK) and the Methusalem programme of Ghent University Special Research Fund (BOF) (Grant Number 01M01021)} 
	

	\address[1]{Ghent Analysis \& PDE center, Department of Mathematics: Analysis, Logic and Discrete Mathematics\\ Ghent University\\
		Krijgslaan 281\\ B 9000 Ghent\\ Belgium} 
	\email{karel.vanbockstal@UGent.be}


 \author[K.~Khompysh]{Khonatbek Khompysh$^{1,2}$} 

 \address[2]{Al-Farabi Kazakh National University\\ Almaty\\ Kazakhstan}
 \email{konat\_k@mail.ru}
	
	\subjclass[2020]{35A01, 35A02, 35A15, 35R11, 65M12, 33E12}
	\keywords{inverse source problem; pseudo-parabolic equation; Neumann boundary condition;  Rothe's method}
	
	\begin{abstract} In this paper, we study the inverse problem for determining an unknown time-dependent source coefficient in a semilinear pseudo-parabolic equation with variable coefficients and Neumann boundary condition. This unknown source term is recovered from the integral measurement over the domain $\Omega$. Based on Rothe's method, the existence and uniqueness of a weak solution, under suitable assumptions on the data, is established. A numerical time-discrete scheme for the unique weak solution and the unknown source coefficient is designed, and the convergence of the approximations is proved. Some numerical experiments are presented to support the obtained theoretical results.
	\end{abstract}
	
	\maketitle
	
	\tableofcontents

%%%%%%%%%%%%%%%%%%%%%
\section{Introduction}
\label{sec:introduction}
%%%%%%%%%%%%%%%%%%%%

We consider an open and bounded Lipschitz domain $\Omega \subset \mathbb{R}^d$ with boundary $\partial\Omega$,  $d \in \NN$. 
 Let $Q_T = (0,T] \times \Omega$ and $\Sigma_T = (0,T] \times \partial\Omega$, where $T>0$ is a given final time.
In the sequel, we consider the following  semilinear pseudo-parabolic equation with variable coefficients and Neumann boundary condition: 
\begin{equation} \label{eq:problem}
\left\{
\begin{array}{rl}
\pdt u(t,\X) -   \nabla \cdot\left(\eta(t,\X) \nabla \pdt u(t,\X)\right)   - \nabla \cdot \left( \kappa(t,\X) \nabla u (t,\X)\right) & \\ 
= f(u(t,\X))+\D p(t,\X)h(t),  & (t,\X) \in Q_T, \\[4pt]
\kappa(t,\X) \nabla u(t,\X) \cdot \normal(\X) = g(t,\X), & (t,\X) \in \Sigma_T, \\[4pt]
u(0,\X) = \tilde{u}_0,  &\X \in \Omega.
\end{array}
\right.
\end{equation}
In this paper, $h(t)$ is unknown and will be recovered from the additional information 
\begin{equation} \label{eq:add:cond}
\int_\Omega  u(t,\X) \dX = m(t), \quad t > 0. 
\end{equation}
In the system (\ref{eq:problem}-\ref{eq:add:cond}), the coefficients $ \eta, \kappa $ and the functions $\tilde{u}_0, p, g, m, f$ are given, whilst $u$ and $h$ are unknown and need to be determined. We look for a weak solution to the problem (\ref{eq:problem}-\ref{eq:add:cond}). 

Equations like \eqref{eq:problem} are called pseudo-parabolic equations, also known as Sobolev equations. These equations describe a range of essential physical processes, such as the unidirectional propagation of nonlinear long waves \cite{Ting:1963,BBM:1972}, the aggregation of populations \cite{Pad:2004}, fluid flow in fissured rock \cite{BZK:1960}, filtration in porous media \cite{6BER:1989}, the unsteady flow of second-order fluids \cite{Hui:1968}, and the motion of non-Newtonian fluids \cite{AKS2011,zvy-2010}, among others.

Numerous studies on linear and nonlinear pseudo-parabolic equations have focused on investigating direct problems. Research on inverse problems for pseudo-parabolic equations began with the seminal work of Rundell \cite{R:1980} in 1980, wherein Rundell studied an inverse problem to identify source terms in a linear pseudo-parabolic equation using overspecified boundary data or the final-in-time measurement. Currently, there are dozens of studies on inverse problems for pseudo-parabolic equations; we refer readers to \cite{Fu2023,Fu2023a,AntAA:2020,KhSh:2023,Yaman2012,LyTa:2011,LyVe:2019,HDT:2021,HTI:2024,KHSI:2024} and the references therein.  In \cite{Fu2023}, the authors study the determination of a time-dependent potential coefficient from the nonlocal measurement  $\int_\Omega u(t,\X) \omega(\X)\dX$ in a linear pseudo-parabolic with constant coefficients. Using a fixed point argument, the authors establish the uniqueness of a solution if $\vnorma{\nabla \omega}_{\lp{2}}\ll 1.$ In \cite{Fu2023a}, authors have been obtained similar results also for an inverse problem recovering a spatial-dependent potential coefficient in a linear pseudo-parabolic equation with nonlocal measurement $\int_0^T u (t,\X)\omega(t)\dt$ under some restriction on the positive weight function $\omega$ such as $0<\frac{\|\omega'\|_{\Leb^2(0,T)}}{\|\omega\|_{\Leb^1(0,T)}}\ll 1.$ In \cite{LyTa:2011}, the authors studied an inverse problem of determining the diffusion coefficient $\kappa = \kappa(t)$ in a linear pseudo-parabolic equation
%\[(u+\eta Mu)_t +\kappa(t)Mu+g u=f\]
under a measurement on the boundary.
%in the form $\int_{\partial\Omega}\left(\eta\frac{\partial u_t}{\partial \nu}+\kappa(t)\frac{\partial u}{\partial \nu}\right)\omega \di\sigmab+\varphi_1(t) \kappa(t)=\varphi_2(t)$, where  $Mu= - \divv(\mathcal{M}(x)\nabla u)+m(x)u $ is a second-order linear differential operator. 
Under certain assumptions and restrictions on the data, the existence and uniqueness of a strong solution were established using an iterative method. Moreover, the inverse problem of determining a time-dependent potential coefficient from integral boundary data is investigated in \cite{LyVe:2019}. The numerical analysis of inverse problems involving the determination of a time-dependent potential coefficient in a one-dimensional linear pseudo-parabolic equation, based on additional boundary information about the solution, is presented in \cite{HDT:2021}. A similar investigation is carried out for the case of a fractional time derivative in \cite{HTI:2024}.

In the study of time-dependent inverse (source) problems for pseudo-parabolic equations, the measurement form plays a crucial role. For instance, in works such as \cite{AntAA:2020} and \cite{KhSh:2023}, inverse problems for nonlinear pseudo-parabolic equations perturbed by $p-$ Laplacian and nonlinear damping/reaction term have been investigated under a measurement expressed in a specific form, such as $\int_\Omega u(\omega - \Delta \omega)\dX = e(t)$. With this specific nonlocal measurement, the blowing up in a finite time and large time behaviour of solutions of an inverse source problem for a pseudo-parabolic equation with nonlinear damping term were established in \cite{Yaman2012}.
However, this measurement form lacks clear physical justification, although it is mathematically convenient.  By the physical motivation, we study a time-dependent inverse source problem for a semilinear pseudo-parabolic equation under the measurement \eqref{eq:add:cond}, so corresponding with $\omega=1$. We note that numerical and theoretical studies of a spatially dependent inverse source problem for a pseudo-parabolic equation with memory are addressed in \cite{KHSI:2024}.


% Aside from the studies in \cite{LyTa:2011} and \cite{LyVe:2019}, research on inverse problems for pseudo-parabolic equations involving nonlocal measurements remains relatively limited. 

% This measurement form aligns more closely with physical phenomena, but it introduces additional mathematical difficulties, necessitating further restrictions on the data (see \cite{KHOMPYSH2024115716}). 

% On the other hand, nearly all the works mentioned above primarily focus on analytical studies. Among the studies that include numerical implementations and theoretical investigations of inverse problems for pseudo-parabolic equations, we can refer to \cite{KHSI:2024,HTI:2024,HDT:2021}. These works address inverse problems involving the determination of an unknown parameter dependent on spatial variables from measurements at the final time or identifying a time-dependent coefficient based on measurements at a specific point within the domain.

Various applications of Rothe's method \cite{Rothe1930} for the study of inverse problems for parabolic and hyperbolic evolution equations have been considered by Slodi\v{c}ka and Van Bockstal, see e.g. \cite{Slodicka2014jcam,Slodicka2016b,VanBockstal2017,Kang2018,Siskova2019,VanBockstal2020,VanBockstal2022a,VanBockstal2022b,VanBockstal2022c}. 
In this work, we apply Rothe's method to investigate the inverse source problem for the semilinear pseudo-parabolic equation \eqref{eq:problem} with Neumann boundary condition and the measurement \eqref{eq:add:cond} from both theoretical and numerical perspectives. Specifically, we will establish the existence and uniqueness of a weak solution and develop a numerically efficient algorithm.
 
The paper is organised as follows. In \Cref{sec:reformulation}, we first state all conditions on the given data, reformulate the inverse problem as a coupled direct problem and formulate its weak formulation. Afterwards, we show the uniqueness of a solution in \Cref{sec:uniqueness} and the existence of the weak solution in \Cref{sec:existence} using Rothe's method. In \Cref{sec:experiments}, the theoretical results are illustrated with some numerical examples.



%%%%%%%%%%%%%%%%%%%%%
\section{Reformulation of the inverse problem}
\label{sec:reformulation}
%%%%%%%%%%%%%%%%%%%%


In this section, we will derive an expression for $h$ in terms of $u$ and the given data. In this way, we will be able to reformulate the inverse problem as a coupled direct problem. We first summarise the assumptions on the data that we will use to tackle the inverse problem  (\ref{eq:problem}-\ref{eq:add:cond}):
\begin{enumerate}[
\textbf{AS DP}-(1),leftmargin=2.4cm] %\roman*
           \item\label{as:DP:eta} $\eta: \I\times \overline{\Omega} \to \RR$ satisfies $  0 < \ubar{\eta}_0 \le \eta(t,\X)\le \ubar{\eta}_1<\infty;$  %\ \ {\color{red}{ 0<\eta'_0\le \pdt \eta(t,\X)\le \eta'_1<\infty;}}$ 
    \item \label{as:DP:kappa} $\kappa: \I\times \overline{\Omega} \to \RR$ satisfies
    %$\kappa \in \ckIX{1}{\Leb^\infty(\overline{\Omega})}$ with
    \[
    \begin{cases}
    0<\ubar{\kappa}_0\le \kappa(t,\X)\le \ubar{\kappa}_1<\infty, \ &\text{for a.a. } (t,\X)\in [0,T]\times\overline{\Omega},\\
    \abs{\pdt \kappa(t,\X)}\le \ubar{\kappa}^\prime_1<\infty,\ &\text{for a.a. } (t,\X)\in [0,T]\times \overline{\Omega};
    \end{cases}
    \]
     \item \label{as:DP:f} $f:\mathbb{R} \rightarrow \mathbb{R}$ is  Lipschitz continuous, i.e. 
    \[
     \abs{f(s_1)-f(s_2)}\le L_f \abs{s_1-s_2}, \quad \forall s_1,s_2\in \RR; 
    \]
   % \item \label{as:DP:p} $F: \I \to \lp{2}$ belongs to $F\in \lpkIX{2}{\lp{2}}.$
    \item \label{as:DP:u0} $\tilde{u}_0\in\hk{1};$
    \item \label{as:g} $g \in \Hi^1 \left((0,T], \Leb^2(\partial\Omega)\right)$, so that
    \[
    G := \frac{g}{\kappa} \in \Hi^1 \left((0,T], \Leb^2(\partial\Omega)\right); \ \ %{\color{red}{\text{and}}}
    \]
    \item \label{as:p} $p\in \mathcal{X}:=\cIX{\lp{2}}$ such that ${\omega}\in \Cont\left(\I\right)$ defined  by 
\[{\omega}(t):=\int_\Omega p(t,\X)\dX\]
satisfies
\[
 {\omega}(t) \neq 0 \text{ for all } t\in \I.
\]
We denote 
\[
 0< \ubar{\omega}_0:=\min_{t\in \I} \abs{{\omega}(t)};% \ \ \text{and} \ \  \vnorma{p}_{}^2=\sup_{t\in[0,T]}\vnorma{p(t)}^2.
\]
  \item \label{as:m}  $m\in \Hi^1((0,T])$. 
 %{\color{red}{and $m(0)=\displaystyle\int_\Omega\tilde{u}_0(\X)\dX$}}.
\end{enumerate}

\begin{remark}
Further, we denote the inner product $\scal{\cdot}{\cdot}_X$ by $\scal{\cdot}{\cdot}$ for $X= \lp{2}$ and by $\scal{\cdot}{\cdot}_{\partial\Omega}$  for $X= \Leb^2(\partial\Omega)$. Its associated norm is denoted by $\vnorma{\cdot} = \sqrt{\scal{\cdot}{\cdot}}$ and $\vnorma{\cdot}_{\partial \Omega} = \sqrt{\scal{\cdot}{\cdot}}_{\partial\Omega}$, respectively. 
\end{remark}

\begin{remark}
    From \ref{as:DP:f} it follows that 
\begin{equation}\label{eq:inequality_nonlinear_f}
    \abs{f(s)} \le \abs{f(0)} + L_f \abs{s}, \quad \forall s\in \RR.
    \end{equation}
    % Hence, we have for $u:\I \to \lp{2}$ that 
    %  \begin{equation}\label{eq:inequality_nonlinear_f}
    %   \vnorma{f(u(t))}^2 \le \ubar{L}_f \left(\vnorma{u(t)}^2 + 1\right),  \qquad t\in \I,
    %  \end{equation}
    %  where $\ubar{L}_f:= 2 \max \left\{L_f^2, f(0)^2 \abs{\Omega} \right\}.$
\end{remark}

\begin{remark}
    In \Cref{thm:existence_inverse_problem}, we will show that $u: \Iopen \to \hk{1}$ is continuous in time. Together with \ref{as:m}, this implies that $m(0)=\int_\Omega\tilde{u}_0(\X)\dX.$
\end{remark}

The approach presented here takes advantage of the Neumann boundary data. For this reason, when integrating the PDE in \eqref{eq:problem} over $\Omega$, using the measurement \eqref{eq:add:cond} and the divergence theorem, we obtain the following expression for $h$ (with $t>0$): 
\begin{multline} \label{eq:expression_h}
h(t)=\frac{1}{{\omega}(t)}\left[m^\prime(t) - \int_{\partial\Omega} \eta(t,\X)\partial_t G(t,\X) \dg_{\X} \right.\\\left. - \int_{\partial\Omega} g(t,\X) \dg_{\X}  - \int_\Omega f(u(t,\X))\dX\right],    
\end{multline}
where we have used \ref{as:p}. Using this expression, we can reformulate the inverse problem  (\ref{eq:problem}-\ref{eq:add:cond}) in the following way: 
\medskip
\begin{center}
Find $u(t)\in \hk{1}$ with $\pdt u(t)\in \hk{1}$ such that for a.a. $t \in \Iopen$ and any $\varphi \in \hk{1}$ it holds that 
\begin{multline}\label{eq:var_for} 
\scal{ \pdt u(t)}{\varphi} + \scal{ \eta(t)\nabla \pdt u(t)}{\nabla \varphi} +  \scal{\kappa(t) \nabla u(t)}{\nabla \varphi} \\
=h(t) \scal{p(t)}{\varphi} + \scal{f(u(t))}{\varphi} +  \scal{ \eta(t)\partial_t G(t)}{\varphi}_{\partial\Omega} + \scal{ g(t)}{\varphi}_{\partial\Omega},    
\end{multline}
with $h\in \Leb^2(0,T)$ given by \eqref{eq:expression_h}.
\end{center}
\medskip

In the next section, we show the uniqueness of a solution to the problem (\ref{eq:expression_h}-\ref{eq:var_for}). 


%%%%
\section{Uniqueness of a solution}
\label{sec:uniqueness}
%%%%

 We show the uniqueness of a solution to the problem (\ref{eq:expression_h}-\ref{eq:var_for}) by the energy estimate approach.


%%%%%%%%%%%%%%%%%%%%%
%\subsection{Uniqueness of a solution}

\begin{theorem}\label{thm:uniq_inv_problem}
Let the assumptions \ref{as:DP:eta} until \ref{as:m} be fulfilled. 
    Then, there exists at most one couple $\{u,h\}$ solving problem (\ref{eq:expression_h}-\ref{eq:var_for})  such that
\begin{equation*}
h \in \Leb^2\Iopen, \quad  u \in \cIX{\hk{1}} \quad \text{with} \quad  \pdt u \in \lpkIX{2}{\hk{1}}.
\end{equation*}
\end{theorem}

\begin{proof}
Let $u_1$ and $u_2$ be two distinct solutions to the problem (\ref{eq:expression_h}-\ref{eq:var_for}) with the same data. 
We subtract the variational formulation \eqref{eq:var_for}  for $\{u_1, h_1\}$ from the one for $\{u_2,h_2\}$. Then, we obtain  for $u:=u_1-u_2$ and $h:=h_1-h_2$  that
\begin{multline}\label{uni:var_for}
\scal{\pdt u(t)}{\varphi} + \scal{\eta(t) \nabla \pdt u(t)}{\nabla \varphi} +  \scal{\kappa(t) \nabla u(t)}{\nabla \varphi} \\
=h(t) \scal{p(t)}{\varphi} + \scal{f(u_1(t))-f(u_2(t))}{\varphi}, \quad \forall \varphi \in \hk{1}.  
\end{multline}
Performing the similar operator for \eqref{eq:expression_h}, we obtain that $h$ is expressed as follows
\begin{equation} \label{uni:expression_h}
h(t)= 
\frac{1}{{\omega}(t)}\int_\Omega \left[f(u_2(t,\X))-f(u_1(t,\X))  \right]\dX.\\
\end{equation}
Employing the Lipschitz continuity of $f$, we obtain that 
\begin{equation}\label{proof:uniq:est_h1}
\abs{h(t)} \le \frac{L_f}{\ubar{\omega}_0}  \vnorma{u(t)}_{\Leb^1(\Omega)}\le C_1\vnorma{u(t)}, \ \ C_1:=\frac{L_f}{\ubar{\omega}_0}\sqrt{\meas{\Omega}}.
\end{equation}
Using $u(t,\cdot)= \int_0^t \pdt u(\eta,\cdot) \di\eta$ as $u(0,\cdot) =0$, we have that 
\begin{equation}\label{proof:uniq:est_h}
\abs{h(t)} \le  C_1 \int_0^t \vnorma{\pdt u(\eta)} \di \eta.
\end{equation}
%Hence, we get by the Friedrichs inequality \eqref{eq:friedirchs} that 
%\begin{equation}\label{IP:uni:estimate_h}
%\abs{h(t)} \le 
%C_1 \vnorma{\nabla u(t)} + \frac{ M  \ubar{\eta}_1}{\ubar{\omega}_m}  \vnorma{\nabla \pdt u(t)},
%\end{equation}
%where \[
%C_1:= \frac{1}{\ubar{\omega}_m}  \left[\frac{\vnorma{\omega}C_{\textup F}}{\ubar{\rho}_0} %\left(\ubar{\rho}_1^\prime + L_f \right) + M\ubar{\kappa}_1 \right].
%\]
Now, we take $\varphi=\pdt u(t)\in \hk{1}$ in \eqref{uni:var_for} and  integrate the result over $t\in(0,s)\subset (0,T)$ to get 
\begin{multline*} \label{001:eq:est2}
  \int_0^s \vnorma{\pdt u (t)}^2 \dt +  \D\int_0^s \int_\Omega  \eta |\nabla \pdt u|^2 \dX\dt  + \frac{1}{2}\D\int_0^s \int_\Omega  \kappa \pdt \abs{\nabla u}^2 \dX \dt \\
  =\int_0^s h(t) \scal{p(t)}{\pdt u(t)} \dt + \D\int_0^s \scal{f(u_1(t))-f(u_2(t))}{\pdt u(t)}\dt.
\end{multline*}
The third term on the left-hand side of this equation can be handled by using 
\begin{equation*}
 \int_0^s   \int_\Omega  \kappa\pdt \abs{\nabla u}^2 \dX \dt = \int_\Omega  \kappa(s) \abs{\nabla u(s)}^2 \dX -   \int_0^s   \int_\Omega (\pdt \kappa) \abs{\nabla u}^2 \dX \dt. 
\end{equation*}
Next, we focus on the terms on the right-hand side. For the first term, using the $\veps$-Young inequality, we obtain that
\begin{align*}
     \abs{\int_0^s h(t) \scal{p(t)}{\pdt u(t)} \dt } &\le    \veps \int_0^s \vnorma{\pdt u(t)}^2 \dt + \frac{\vnorma{p}_{\mathcal{X}}^2}{4\veps} \int_0^s \abs{h(t)}^2\dt \\
    & \stackrel{\eqref{proof:uniq:est_h}}{\le} \veps\int_0^s \vnorma{\pdt u(t)}^2 \dt+  \frac{\vnorma{p}_{\mathcal{X}}^2 C_1^2 T}{4\veps} \int_0^s \int_0^t \vnorma{\pdt u(\eta)}^2 \di\eta \dt .
   \end{align*}
Similarly, using the Lipschitz continuity of $f$ and $u(t,\cdot)= \int_0^t \pdt u(\eta,\cdot) \di\eta$, we deduce that 
\begin{multline*}
    \abs{\int_0^s \scal{f(u_1(t))-f(u_2(t))}{\pdt u(t)}\dt} \\ 
 %  & \le  \veps_1 \int_0^s \vnorma{\pdt u(t)}^2 \dt+\frac{L_f^2}{4\veps_1} \int_0^s \vnorma{ u(t)}^2 \dt \\
     \le   \veps \int_0^s \vnorma{\pdt u(t)}^2 \dt+\frac{L_f^2T}{4\veps} \int_0^s \int_0^t \vnorma{ \pdt u(\eta)}^2 \di\eta \dt. 
\end{multline*}
Summarizing,  using \ref{as:DP:eta}-\ref{as:DP:kappa} and taking $\veps=\frac{1}{4}$, we obtain the estimate 
\begin{multline*}
      \int_0^s \vnorma{\pdt u(t)}^2 \dt + 2\ubar{\eta}_0 \int_0^s \vnorma{\nabla \pdt u(t)}^2 \dt  + \ubar{\kappa}_0\vnorma{\nabla u(s)}^2  \\
      \le  \ubar{\kappa}^\prime_1 \int_0^s \vnorma{\nabla u(t)}^2 \dt+C_2\int_0^s \int_0^t \vnorma{ \pdt u(\eta)}^2 \di\eta \dt,
\end{multline*}
where 
\[
        C_2 := 2 T \left(\vnorma{p}_{\mathcal{X}}^2 C_1^2+L_f^2\right). 
\]
Therefore, applying the Gr\"onwall argument gives that $u=0$ a.e.\ in $Q_T.$ Moreover, from 
\eqref{proof:uniq:est_h1}, we obtain that $h=0$ a.e. in $\Iopen.$
\end{proof}

%%%%%%%%%%%
\section{Existence of a solution}
\label{sec:existence}
%%%%%%%%%%%

In this section, we will show the existence of a weak solution to problem (\ref{eq:expression_h}-\ref{eq:var_for}) by employing Rothe's method. We start by dividing the time interval $[0, T]$  into $n \in \mathbb{N}$ equidistant subintervals $[t_{i-1},t_i]$ of length $\tau = T/n$, $i = 1,\ldots n$. Hence, $t_i = i \tau$ for $i = 0,1, \ldots,n$.  We consider for any function $z$ that
%$$
\[
z_i \approx z(t_i) \quad \text{ and } \quad \pdt z(t_i) \approx \delta z_i = \dfrac {z_i-z_{i-1}}{\tau},
\]
i.e. the backward Euler method is used to approximate the time derivatives at every time step $t_i$. Moreover, linearising the term containing $f$ in the right-hand side of \eqref{eq:var_for} at time step $t_i$ by replacing $u_i$ with $u_{i-1}$, we get the following time-discrete problem at time $t=t_i$: 
\medskip
\begin{center}
Find $u_i \in \hk{1}$ and $h_i\in \RR$ such that 
\begin{multline} \label{eq:disc_inv_prob}
\scal{\delta  u_i}{\varphi} +  \scal{\eta_i \nabla \delta u_i}{\nabla \varphi}  +  \scal{\kappa_i \nabla u_i}{\nabla \varphi} \\
= h_{i}\scal{p_i}{\varphi}+\scal{f(u_{i-1})}{\varphi} +   \scal{\eta_i (\pdt G)_i}{\varphi}_{\partial\Omega} +  \scal{g_i}{\varphi}_{\partial\Omega}, \quad \forall\varphi \in \hk{1}, 
\end{multline}
and
\begin{equation} \label{disc:hi-1}
h_{i}=\frac{1}{\omega_i}\left[(m^\prime)_i - \int_{\partial\Omega} \eta_i(\partial_t G)_i \dg - \int_{\partial\Omega} g_i \dg   - \int_\Omega f(u_{i-1}) \dX\right], 
\end{equation}
where
\begin{equation} \label{initC:disc_inv_prob}
 u_{0}=\tilde{u}_0. 
\end{equation}
\end{center}
\medskip

Hence, for any $i\in \left\{1,...,n\right\}$, we first derive $h_i \in \RR$ from \eqref{disc:hi-1} and we afterwards solve the following problem for $u_i$:
\begin{equation}\label{equiv:var_for_inv_disc_prob}
a(u_i,\varphi) = l_i(\varphi), \quad \forall \varphi \in \hk{1}, 
\end{equation}
where the bilinear form  $a: \hk{1}\times \hk{1}\to \mathbb{R}$ is given by 
\begin{equation*}
    a(u,\varphi) := \frac{1}{\tau}\scal{ u}{\varphi} +  \frac{1}{\tau} \scal{\eta_i \nabla u}{\nabla \varphi} +  \scal{\kappa_i \nabla u}{\nabla \varphi} 
\end{equation*}
and the linear functional $l_i: \hk{1}\to \mathbb{R}$ is defined by
\begin{multline*}
     l_i(\varphi) := h_{i}\scal{p_i}{\varphi}+\scal{f(u_{i-1})}{\varphi} +   \scal{ \eta_i(\pdt G)_i}{\varphi}_{\partial\Omega} +  \scal{g_i}{\varphi}_{\partial\Omega} \\
     + \frac{1}{\tau}\scal{ u_{i-1}}{\varphi} + \frac{1}{\tau} \scal{\eta_i \nabla u_{i-1}}{\nabla \varphi} .
\end{multline*}

The existence and uniqueness of $u_i$ are addressed in the following theorem. 

\begin{theorem}
   Let the conditions \ref{as:DP:eta} until \ref{as:m} be fulfilled.  Then, for any $i=1,\ldots,n$, there exists a unique couple $\{h_i, u_i\}\in \RR \times \hk{1}$ solving (\ref{eq:disc_inv_prob}-\ref{disc:hi-1}). 
\end{theorem}

\begin{proof}
Note that $a$ is bounded on $\hk{1}\times\hk{1}$, and $l_i$ is bounded on $\hk{1}$ if $u_{i-1}\in\hk{1}$. Moreover, the bilinear form $a$ is  $\hk{1}$-elliptic as
\[
a(u,u) \ge \min\left\{\frac{1}{\tau},\frac{\ubar{\eta}_0}{\tau}+ \ubar{\kappa}_0\right\} \vnorma{u}_{\hk{1}}^2, \quad \forall u\in \hk{1}. 
\]
Hence, starting from $u_0=\tilde{u}_0\in \hk{1}$, we recursively obtain (as the conditions of the Lax-Milgram lemma are satisfied) the existence and uniqueness of $h_i\in\RR$ and $u_i \in \hk{1}$ for $i=1,\ldots,n.$
\end{proof}

Now, we derive the a priori estimates for the discrete solutions to the inverse problem.  

\begin{lemma}\label{inv_prob:_est1}
Let the assumptions \ref{as:DP:eta} until \ref{as:m} be fulfilled. Then, there exists positive constants $C$ and $\tau_0$ such that 
\begin{equation}\label{est:invprob_discrsolu}
\max\limits_{1\le j\le n}\vnorma{u_j}_{\hk{1}}^2
+ \sum_{i=1}^n \vnorma{\delta u_i}_{\hk{1}}^2 \tau +  \sum_{i=1}^n \vnorma{u_i - u_{i-1}}_{\hk{1}}^2+ \sum_{i=1}^n |h_i|^2\tau \le C, 
\end{equation}
for any $\tau < \tau_0.$
\end{lemma}

\begin{proof}
Setting $\varphi = \delta u_i \tau$ in \eqref{eq:disc_inv_prob} and summing up the result for $i=1,\ldots,j$ with $1\le j\le n$ give
\begin{multline}\label{direct_problem:a_priori_estimate:eq1}
\sum_{i=1}^j \vnorma{\delta  u_i}^2 \tau +  \sum_{i=1}^j \scal{\eta_i \nabla\delta u_i}{\nabla  \delta  u_i} \tau  +  \sum_{i=1}^j \scal{\kappa_i \nabla u_i}{\nabla  \delta  u_i} \tau 
= \sum_{i=1}^j h_{i}\scal{p_i}{\delta u_i} \tau  \\
+ \sum_{i=1}^j \scal{f(u_{i-1})}{\delta u_i} \tau 
+ \sum_{i=1}^j  \scal{\eta_i (\pdt G)_i}{\delta u_i}_{\partial\Omega} \tau + \sum_{i=1}^j \scal{g_i}{\delta u_i}_{\partial\Omega} \tau.
\end{multline}
By \ref{as:DP:eta}, we have that
 \[
  \sum_{i=1}^j \scal{\eta_i \nabla\delta u_i}{\nabla  \delta  u_i} \tau \geq \ubar{\eta}_0 \sum_{i=1}^j \vnorma{ \nabla\delta u_i}^2 \tau.
 \]
About the left-hand side of \eqref{direct_problem:a_priori_estimate:eq1}, we note that
 \begin{multline*}
  \sum_{i=1}^j  \scal{\kappa_i \nabla u_i}{\nabla \delta u_i} \tau 
    = \half \scal{\kappa_j \nabla u_j}{\nabla u_j}
    - \half \scal{\kappa_0 \nabla \tilde{u}_0}{\nabla  \tilde{u}_0}\\
   - \half \sum_{i=1}^j \scal{(\delta\kappa_i) \nabla u_{i-1}}{\nabla u_{i-1}} \tau 
    + \half \sum_{i=1}^j \scal{\kappa_i (\nabla u_i -  \nabla u_{i-1})}{\nabla u_i - \nabla u_{i-1}}.
 \end{multline*}
 Hence, using \ref{as:DP:kappa}, we get for $\tau <1$ that
\begin{multline*}
  \sum_{i=1}^j  \scal{\kappa_i \nabla u_i}{\nabla \delta u_i} \tau
  \ge \frac{\ubar{\kappa}_0}{2}  \vnorma{\nabla u_j}^2 - \left( \frac{\ubar{\kappa}_1}{2} +  \frac{\ubar{\kappa}_1^\prime}{2}\right) \vnorma{\nabla \tilde{u}_0}^2 \\
							   - \frac{\ubar{\kappa}_1^\prime}{2}  \sum_{i=1}^{j-1} \vnorma{\nabla u_{i}}^2 \tau  
                        + \frac{\ubar{\kappa}_0}{2} \sum_{i=1}^j \vnorma{\nabla u_i - \nabla u_{i-1}}^2.  
\end{multline*}
Now, we will estimate the terms on the right-hand side of \eqref{direct_problem:a_priori_estimate:eq1}. 
Before doing this, using \eqref{eq:inequality_nonlinear_f}, we estimate $h_i$ given by \eqref{disc:hi-1} as follows
\[
\abs{h_i} \le H_i + \frac{{L}_f }{\ubar{\omega}_0}  \vnorma{u_{i-1}}_{\Leb^1(\Omega)}\le  H_i + C_1\vnorma{u_{i-1}}, \ \ C_1:=\frac{L_f\sqrt{\meas{\Omega}}}{\ubar{\omega}_0},
\]
with
\[
H_i := \frac{1}{\ubar{\omega}_0} \left[ \abs{ (m^\prime)_i } + \ubar{\eta}_1 \sqrt{\meas{\partial\Omega}} \vnorma{ (\partial_t G)_i}_{\partial\Omega} + \sqrt{\meas{\partial\Omega}} \vnorma{ g_i}_{\partial\Omega} + \abs{f(0)} \meas{\Omega}  \right].
\]
Please note that 
\begin{multline*}
\sum_{i=1}^j H_i^2 \tau\\ \le C_2:= C\left(\vnorma{m^\prime}_{\Leb^2 \Iopen},\vnorma{\pdt G}_{\lpkIX{2}{\Leb^2(\partial\Omega)}}, \vnorma{g}_{\lpkIX{2}{\Leb^2(\partial\Omega)}}, \meas{\overline{\Omega}},T, f(0), \ubar{\omega}_0, \ubar{\eta}_1 \right).    
\end{multline*}
Hence, using $u_{i-1} = \sum_{k=1}^{i-1} \delta u_k \tau+ \tilde{u}_0$ for $i\ge 1,$ we have that
\begin{equation} \label{eq:expression_ui}
\sum_{i=1}^j \vnorma{u_{i-1}}^2 \tau \le 2 T \vnorma{\tilde{u}_0}^2  + 2 T \sum_{i=1}^j \left(\sum_{k=1}^{i-1} \vnorma{\delta u_{k}}^2 \tau \right)\tau,
\end{equation}
and so
\begin{equation}\label{eq:discrete_estimate_hi}
\sum_{i=1}^j \abs{h_i}^2 \tau \le 2 C_2 + 2 C_1^2 \sum_{i=1}^j \vnorma{u_{i-1}}^2 \tau \le C_3 + 4 C_1^2 T  \sum_{i=1}^j \left(\sum_{k=1}^{i-1} \vnorma{\delta u_{k}}^2 \tau  \right) \tau,
\end{equation}
with $C_3:= 2 C_2 + 4 C_1^2 T \vnorma{\tilde{u}_0}^2.$ Therefore, employing the $\veps$-Young inequality, we obtain 
\[
\left|\sum_{i=1}^j h_{i} \scal{p_i}{\delta  u_i}\tau\right|
   %& \le  \sum_{i=1}^j |h_{i}|\vnorma{p}_{\mathcal X} \vnorma{\delta  u_i}\tau
  % &\le   \veps_1 \sum_{i=1}^j  \vnorma{\delta u_i}^2 \tau + \frac{\vnorma{p}_{{\mathcal X}}^2 }{4\veps_1} \sum_{i=1}^j |h_{i}|^2\tau \nonumber\\
     \le \veps_1 \sum_{i=1}^j  \vnorma{\delta u_i}^2 \tau + \frac{\vnorma{p}_{{\mathcal X}}^2 }{4\veps_1} \left(C_3 + 4 C_1^2 T  \sum_{i=1}^j \left(\sum_{k=1}^{i-1} \vnorma{\delta u_{k}}^2 \tau  \right) \tau\right).
  % \label{est:h}
\]
For the second term on the right-hand side of \eqref{direct_problem:a_priori_estimate:eq1}, we use \eqref{eq:inequality_nonlinear_f} and \eqref{eq:expression_ui} to get
\begin{align*}
 \left| \sum_{i=1}^j \scal{f(u_{i-1})}{\delta u_i} \tau \right| &\le \veps_1 \sum_{i=1}^j  \vnorma{\delta u_i}^2 \tau + \frac{\ubar{L}_f}{4\veps_1} \sum_{i=1}^j \left(\vnorma{u_{i-1}}^2+1\right) \tau \\
 &\le C_4(\veps_1) +  \veps_1 \sum_{i=1}^j  \vnorma{\delta u_i}^2 \tau  +  \frac{\ubar{L}_fT}{2\veps_1} \sum_{i=1}^j \left(\sum_{k=1}^{i-1} \vnorma{\delta u_{k}}^2 \tau  \right) \tau, 
\end{align*}
with $\ubar{L}_f:= 2 \max \left\{L_f^2, f(0)^2 \abs{\Omega} \right\}$ and $C_4(\veps):=\frac{\ubar{L}_fT}{4\veps} \left(1+ 2\vnorma{\tilde{u}_0}^2 \right)$.
Next, using the trace theorem ($\vnorma{\phi}_{\Leb^2(\Gamma)} \le C_{\textrm{tr}} \vnorma{\phi}_{\hk{1}}$), we get that 
\begin{align*}
\abs{\sum_{i=1}^j  \scal{ \eta_i(\pdt G)_i}{\delta u_i}_{\partial\Omega} \tau} &\le \frac{\ubar{\eta}_1^2}{4\veps_2} \sum_{i=1}^j \vnorma{(\pdt G)_i}_{\Leb^2(\Gamma)}^2 \tau + \veps_2 \sum_{i=1}^j  \vnorma{\delta u_i}_{\Leb^2(\Gamma)}^2 \tau \\
& \le \frac{\ubar{\eta}_1^2}{4\veps_2} C\left(\vnorma{\pdt G}_{\lpkIX{2}{\Leb^2(\partial \Omega)}}\right) + \veps
_2  C_{\textrm{tr}}^2 \sum_{i=1}^j  \vnorma{\delta u_i}_{\hk{1}}^2 \tau. 
\end{align*}
Similarly, we have that
\[
\abs{\sum_{i=1}^j \scal{g_i}{\delta u_i}_{\partial\Omega} \tau} 
\le \frac{1}{4\veps_2} C\left(\vnorma{g}_{\lpkIX{2}{\Leb^2(\partial \Omega)}}\right) + \veps
_2  C_{\textrm{tr}}^2 \sum_{i=1}^j  \vnorma{\delta u_i}_{\hk{1}}^2 \tau. 
\]
Now, we collect all estimates derived above and obtain from \eqref{direct_problem:a_priori_estimate:eq1} that 
\begin{multline}\label{ener:ineq22}
  \left(1- 2\veps_1 - 2 C_{\textrm{tr}}^2 \veps_2 \right) \sum_{i=1}^j \vnorma{\delta u_i}^2 \tau + \left(\ubar{\eta}_0 - 2 C_{\textrm{tr}}^2 \veps_2\right) \sum_{i=1}^j \vnorma{ \nabla \delta  u_i}^2 \tau \\
    + \frac{\ubar{\kappa}_0}{2}  \vnorma{\nabla u_j}^2 +  \frac{\ubar{\kappa}_0}{2}  \sum_{i=1}^j \vnorma{\nabla u_i - \nabla u_{i-1}}^2 \\ 
    \le C_5 +  \frac{\ubar{\kappa}_1^\prime}{2}  \sum_{i=1}^{j-1} \vnorma{\nabla u_{i}}^2 \tau   + C_6\sum_{i=1}^{j}\left(\sum_{k=1}^{i} \vnorma{\delta u_k }^2\tau\right)\tau,
\end{multline}
where
\begin{align*}
  C_5 &:=\left( \frac{\ubar{\kappa}_1}{2} +  \frac{\ubar{\kappa}_1^\prime}{2}\right) \vnorma{\nabla \tilde{u}_0}^2  + \frac{\vnorma{p}_{{\mathcal X}}^2 }{4\veps_1} C_3 + C_4(\veps_1) + \frac{\ubar{\eta}_1^2}{4\veps_2} C\left(\vnorma{\pdt G}\right) +  \frac{1}{4\veps_2} C\left(\vnorma{g}\right),   \\
  C_6 &:=  \frac{\vnorma{p}_{{\mathcal X}}^2 C_1^2 T}{\veps_1} +  \frac{\ubar{L}_fT}{2\veps_1} .
\end{align*}
First, we take $\veps_1$ and $\veps_2$ small enough such that $1- 2\veps_1 - 2 C_{\textrm{tr}}^2 \veps_2>0$ and $\ubar{\eta}_0 - 2 C_{\textrm{tr}}^2 \veps_2>0.$ Then, we apply the Gr\"onwall lemma to obtain the existence of positive constants $C$ and $\tau_0$ such that 
\[
\sum_{i=1}^j \vnorma{\delta u_i}^2 \tau +\sum_{i=1}^j \vnorma{\nabla \delta u_i}^2 \tau 
    +\vnorma{\nabla u_j}^2 +  \sum_{i=1}^j \vnorma{\nabla u_i - \nabla u_{i-1}}^2 \le C,
\]
for $\tau < \tau_0.$ Moreover, from \eqref{eq:discrete_estimate_hi}, for $\tau < \tau_0$, we obtain the existence of a positive constant $C$ such that 
\[
\sum_{i=1}^j \abs{h_i}^2 \tau \le C. \qedhere
\]
\end{proof}

In the next step, we introduce the so-called Rothe functions: The piecewise linear-in-time function
\begin{equation}
    \label{Rothestepfun:u} 
    U_n:[0,T]\to\lp{2}:t\mapsto\begin{cases} \tilde{u}_0 & t =0,\\
u_{i-1} + (t-t_{i-1})\delta u_i &
     t\in (t_{i-1},t_i],\quad 1\leqslant i\leqslant n, \end{cases}\end{equation}
and the piecewise constant function
\begin{equation}
    \label{Rothefun:u}
\overline U_n:[-\tau,T] \to\lp{2}:t\mapsto\begin{cases} \tilde{u}_0 &  t \in [-\tau,0],\\
 u_i &     t\in (t_{i-1},t_i],\quad 1\leqslant i\leqslant n.\end{cases}
 \end{equation}
Similarly, in connection with the given functions $\eta$, $\kappa$, $g$, $\pdt G$, $\omega$, and $p,$ we define the functions $\overline{\eta}_n$, $\overline{\kappa}_n$, $\overline{g}_n$, $\overline{\pdt G}_n$, $\overline{\omega}_n$ and $\overline{p}_n$, respectively. 
Now, using the Rothe's functions,
 we rewrite the discrete variational formulation (\ref{eq:disc_inv_prob}-\ref{disc:hi-1}) as follows (for all $t\in(0,T]$)
\begin{multline}\label{eq:disc_inv_prob:whole_time_frame}
    \scal{ \pdt U_n(t) }{\varphi} + \scal{ \overline{\eta}_n(t) \nabla \pdt U_n(t)}{\nabla \varphi}  +  \scal{\overline{\kappa}_n(t) \nabla \overline U_n(t)}{\nabla \varphi} 
= \overline{h}_n(t)\scal{\overline{p}_n(t)}{\varphi} \\
+\scal{f(\overline{U}_n(t-\tau))}{\varphi} 
+  \scal{\overline{\eta}_n(t)\overline{\pdt G}_n(t)}{\varphi}_{\partial\Omega} +  \scal{\overline{g}_n(t)}{\varphi}_{\partial\Omega}, \quad \forall\varphi \in \hk{1},
\end{multline}
and
\begin{equation} \label{disc:hi-1:whole_time_frame}
\overline{h}_n(t)=\frac{1}{\overline{\omega}_{n}(t)}\left[ {\overline{m^\prime}_n(t)} - \scal{\overline{\eta}_n(t) \overline{\pdt G}_n(t)}{1}_{\partial\Omega}- \scal{ \overline{g}_n(t)} {1}_{\partial\Omega}   - \scal{f(\overline{u}_{n}(t-\tau))}{1}\right].
\end{equation}


Now, we are ready to show the existence of a solution to (\ref{eq:expression_h}-\ref{eq:var_for}).

%%%%%%%%%%%%%%%
\begin{theorem}\label{thm:existence_inverse_problem}
Let the assumptions \ref{as:DP:eta} until \ref{as:m} be fulfilled.  Then, there exists a unique weak solution couple $\{u,h\}$ to (\ref{eq:expression_h}-\ref{eq:var_for}) satisfying 
\begin{equation*}
 u \in \cIX{\hk{1}} \quad \text{with} \quad  \pdt u \in \lpkIX{2}{\hk{1}} \quad \text{and} \quad h \in \Leb^2\Iopen.
\end{equation*}
\end{theorem}
%%%%%%%%%%%%%%%


\begin{proof}
We have from \Cref{inv_prob:_est1} that there exist $C>0$ and $n_0\in \NN$ such that for all $n\geqslant n_0 > 0$ it holds that 
\begin{multline}\label{est:invprob_discr:solu}
\sup\limits_{t\in[0,T]} \vnorma{\overline{U}_{n}(t)}_{\hk{1}}^2 
+ \int\limits_0^T \vnorma{\pdt{U}_{n}(t)}_{\hk{1}}^2  \dt \\
+ \D \sum_{i=1}^n \vnorma{\int_{t_{i-1}}^{t_i} \pdt{U}_{n}(s)\ds}_{\hk{1}}^2+ \vnorma{\overline{h}_{n}}_{\Leb^2(0,T)}^2\le C.
\end{multline}
By \cite[Lemma~1.3.13]{Kacur1985}, the compact embedding  $\hk{1} \imbed \imbed \lp{2} $ (see \cite[Theorem~6.6-3]{Ciarlet2013}) leads to the existence of a function
$u \in \cIX{\lp{2}}  \cap \Leb^{\infty}\left((0,T), \hk{1}\right)$ with $\pdt u \in \lpkIX{2}{\lp{2}}$,
and a subsequence $\{ U_{n_l}\}_{l\in\NN}$ of $\{U_n\}$ such that
\begin{equation} \label{convergence:rothe_functions_dp}
\left\{
\begin{array}{ll}
U_{n_l} \to u & \text{in}~~\Cont\left([0,T], \lp{2}\right) , \\[4pt]
U_{n_l}(t) \rightharpoonup u(t) & \text{in}~~\hk{1},~~\forall t \in [0,T], \\[4pt]
\overline{U}_{n_l}(t) \rightharpoonup u(t) & \text{in}~~\hk{1},~~\forall t \in [0,T],  \\[4pt]
\pdt {U}_{n_l} \rightharpoonup \pdt u & \text{in}~~\Leb^{2}\left((0,T), \lp{2}\right).
\end{array}
\right.
\end{equation}
By the reflexivity of the space $\lpkIX{2}{\hk{1}}$, we have
%the existence of a subsequence of ${U}_{n_l}$ (denoted by the same symbol)  such that 
that
\begin{equation}\label{weak_convergence_time_der}
\pdt {U}_{n_l} \rightharpoonup \pdt u \quad  \text{in} \quad \lpkIX{2}{\hk{1}}, 
\end{equation}
i.e. $u\in\cIX{\hk{1}}$. Similarly, we have that 
\begin{equation}\label{weak_convergence_hn}
\overline{h}_{n_l} \rightharpoonup \sigma \quad \text{in} \quad \Leb^2\Iopen.
\end{equation}
Moreover, from \Cref{inv_prob:_est1}, we also have that (note that $\tau_l = T/n_l$)
\begin{equation}\label{dir:str:conv:L2}
\int_0^{T} \left( \vnorma{\overline{U}_{n_l}(t) - U_{n_l}(t) }^2 + \vnorma{\overline{U}_{n_l}(t-\tau) - U_{n_l}(t) }^2 \right) \dt  \leqslant 2 \tau_l^2 \sum_{i=1}^{n_l} \vnorma{\delta u_i}^2 \tau  \leqslant C \tau_l^2,
\end{equation}
so \begin{equation}\label{dir:str:conv:L22}
\overline U_{n_l},  \overline U_{n_l}(\cdot-\tau) \to u \ \text{in} \ \lpkIX{2}{\lp{2}} \ \text{as} \ l\rightarrow \infty.
\end{equation}
Hence, using these limit transitions and the Lipschitz continuity of $f$, we easily see that 
\[
\int_0^T \overline{h}_{n_l}(t) \phi(t)\dt \rightarrow \int_0^T {h}(t) \phi(t)\dt \quad \text{ for all } \phi \in \Leb^2\Iopen \text { as } l \to \infty, 
\]
where $h$ is given by \eqref{eq:expression_h}. Hence, by the uniqueness of the weak limit, we have that $\sigma=h.$ Next, we integrate \eqref{eq:disc_inv_prob:whole_time_frame} for $n=n_l$ over $t\in(0,\eta)\subset \Iopen$ and pass to the limit $l\to \infty.$ Afterwards, we differentiate with respect to $\eta$ to obtain that \eqref{eq:var_for} is satisfied. The uniqueness of a solution and the convergence of the whole Rothe sequences follow from \Cref{thm:uniq_inv_problem}. 
\end{proof}

\begin{remark}[Direct problem for a semilinear pseudo-parabolic equation with Neumann boundary condition]
For the direct problem \eqref{eq:problem} with given source $F,$ the weak formulation becomes 
\begin{center}
Find $u(t)\in \hk{1}$ with $\pdt u(t)\in \hk{1}$ such that for a.a. $t \in \Iopen$ and any $\varphi \in \hk{1}$ it holds that 
\begin{multline}\label{eq:var_for_dp} 
\scal{ \pdt u(t)}{\varphi} + \eta(t)\scal{ \nabla \pdt u(t)}{\nabla \varphi} +  \scal{\kappa(t) \nabla u(t)}{\nabla \varphi} \\
=\scal{F(t)}{\varphi} + \scal{f(u(t))}{\varphi} + \eta(t) \scal{ \partial_t G(t)}{\varphi}_{\partial\Omega} + \scal{ g(t)}{\varphi}_{\partial\Omega}.    
\end{multline}
\end{center}
Putting $h=0$ in the approach followed for the inverse problem, we also get the existence of a unique weak solution to \eqref{eq:var_for_dp} satisfying 
\[
 u \in \cIX{\hk{1}} \quad \text{with} \quad  \pdt u \in \lpkIX{2}{\hk{1}},
\]
when the assumptions \ref{as:DP:eta} until \ref{as:g}, and $F\in \Leb^2\left(\I,\lp{2}\right)$ are satisfied.
\end{remark}


%%%%%%%%%%%
\section{Numerical experiments}
\label{sec:experiments}
%%%%%%%%%%%

In the numerical experiments, we consider the inverse problem (\ref{eq:problem}-\ref{eq:add:cond}) in the 1D case with $\Omega=(0,1)$ and with the following data 
 \[\eta(t,x)=0.5, \ \kappa(t,x)=t+1, \ f(s)= s, \ p(t,x) = \sin(\pi x), \ T=1.
 \]
 We conduct two experiments in this section. 
The right-hand side contains a given source term given by
\begin{align*}
F_1(t,x) &= (\pi^2 t - 3.0 + 0.5 \pi^2)\exp(-t) \sin(\pi x),\\
F_2(t,x) &= (\pi^2 t - \exp(t) \sin(2\pi t) - 2.0 + 0.5 \pi^2)\exp(-t)\sin(\pi x),
\end{align*}
respectively, such that the exact solution to the inverse problem is prescribed as follows
\begin{align*}
u_1(t,x) = u_2(t,x) &=  \exp(-t)\sin(x\pi), \\
h_1(t) &= \exp(-t), \quad h_2(t) = \sin(2\pi t).
\end{align*}
We have added to the measurement
\[
m(t) = \int_0^1 u(t,x)\dx = 0.637\exp(-t)
\]
a randomly generated uncorrelated noise, namely
\[
m_{\epsilon}(t) = m(t) \left(1 + \epsilon {\mathcal R}(t)\right),
\qquad t \in \left\{ \frac{jT}{\widetilde{N}} : j=0,\ldots,\widetilde{N}\right\},
\]
where $\epsilon$ represents the percentage of noise, e.g. $\epsilon=0.01$ for $1 \%$ noise, and $\mathcal{R}$ is a Gaussian random variable of zero mean and standard deviation $\sigma$ drawn from $[-1,1]$ that changes in time. We have considered $\widetilde{N}=100$ in our experiments. Afterwards, we regularised the noisy data using the nonlinear least-squares method to obtain a third-order polynomial $m_{\epsilon}^r$ approximating the noisy data for each noise level $\epsilon$. 

Using these polynomials $m_{\epsilon}^r,$ the solution to the inverse source problems is found by applying the algorithm described by (\ref{eq:disc_inv_prob}-\ref{initC:disc_inv_prob}). The discrete elliptic problems are solved numerically by applying the finite element method using the first-order (P1--FEM) Lagrange polynomials for the space discretisation. We use the FEniCSx platform \cite{FEniCSx3,FEniCSx2,FEniCSx1} with the version \texttt{0.9.0} of the DOLFINx module to compute the solution to the problem. We take $\tau = T/200 = 0.005$, and the number of one-dimensional finite elements equal to $200$. The numerical results are depicted in \Cref{Experiment1,Experiment2,Experiment1u,Experiment2u}.  It can be seen from these figures that good approximations 
of the exact sources and solution at final time have been obtained for different levels of noise $\epsilon = \{0,0.001,0.005,0.01,0.3,0.05\}$. 

%%%%%%%%%%%
\begin{figure}[htbp]
\begin{center}
\subfigure[]{\includegraphics[width=0.75\textwidth,angle=0,height = 0.22\textheight]{./plot_source_exp1}}
\subfigure[]{\includegraphics[width=0.75\textwidth,angle=0,height = 0.22\textheight]{./plot_abserror_source_exp1}}
\end{center}
%
\vspace{-0.5cm}
\caption[Unknown time source: Results for Experiment 1]{Experiment 1: (a)~The exact source and its numerical approximation, and (b)~its corresponding absolute error, obtained for various levels of noise.
}
\label{Experiment1}
\end{figure}
%%%%%%%%%%%

%%%%%%%%%%%
\begin{figure}[htbp]
\begin{center}
\subfigure[]{\includegraphics[width=0.75\textwidth,angle=0,height = 0.22\textheight]{./plot_source_exp2}}
\subfigure[]{\includegraphics[width=0.75\textwidth,angle=0,height = 0.22\textheight]{./plot_abserror_source_exp2}}
\end{center}
%
\vspace{-0.5cm}
\caption[Unknown time source: Results for Experiment 2]{Experiment 2: (a)~The exact source and its numerical approximation, and (b)~its corresponding absolute error, obtained for various levels of noise.
}
\label{Experiment2}
\end{figure}

%%%%%%%%%%%
\begin{figure}[htbp]
\begin{center}
\subfigure[]{\includegraphics[width=0.75\textwidth,angle=0,height = 0.22\textheight]{./plot_solution_final_time_exp1}}
\subfigure[]{\includegraphics[width=0.75\textwidth,angle=0,height = 0.22\textheight]{./plot_abserror_solution_final_time_exp1}}
\end{center}
%
\vspace{-0.5cm}
\caption{Experiment 1: (a)~The exact solution at final time and its numerical approximation, and (b)~its corresponding absolute error, obtained for various levels of noise.}
% \caption{The absolute error between the exact solution at final time and its numerical approximation for (a)~Experiment 1, and (b)~Experiment 2, obtained for various levels of noise.
% }
\label{Experiment1u}
\end{figure}

%%%%%%%%%%%
\begin{figure}[htbp]
\begin{center}
\subfigure[]{\includegraphics[width=0.75\textwidth,angle=0,height = 0.22\textheight]{./plot_solution_final_time_exp2}}
\subfigure[]{\includegraphics[width=0.75\textwidth,angle=0,height = 0.22\textheight]{./plot_abserror_solution_final_time_exp2}}
\end{center}
%
\vspace{-0.5cm}
\caption{Experiment 2: (a)~The exact solution at final time and its numerical approximation, and (b)~its corresponding absolute error, obtained for various levels of noise.}
%\caption{The absolute error between the exact solution at final time and its numerical approximation for (a)~Experiment 1, and (b)~Experiment 2, obtained for various levels of noise.
% }
\label{Experiment2u}
\end{figure}

Next, when $\epsilon =0$ (so noise-free case),  we study experimentally the order of convergence of $u$ and $h$ by examining 
\[
E_{\mathrm{max}}^{u}(\tau) =\max_{1\le i \le n}  \vnorma{u(t_i) - u_i }^2 \quad \text{ and } \quad E_{\mathrm{max}}^{h}(\tau) = \max_{1\le i \le n} \abs{h(t_i) - h_i },
\]
for Experiment~1. We obtain that $E_{\mathrm{max}}^{u}(\tau)=\OO{\tau}$ and $E_{\mathrm{max}}^{h}(\tau)= \OO{\tau},$ see \Cref{Experiment1conv}. 

%%%%%%%%%%%
\begin{figure}[htbp]
\begin{center}
\subfigure[]{\includegraphics[width=0.75\textwidth,angle=0,height = 0.22\textheight]{./plot_convergence_max_L2error_solution_exp1_fixed_noise_level_0.0}}
\subfigure[]{\includegraphics[width=0.75\textwidth,angle=0,height = 0.22\textheight]{./plot_convergence_max_error_source_exp1_fixed_noise_level_0.0}}
\end{center}
%
\vspace{-0.5cm}
\caption{Experiment 1: (a) rate of convergence of $u$; (b) rate of convergence for $h$, for noise-free data. 
}
\label{Experiment1conv}
\end{figure}

%%%%%%%%%%%
\section{Conclusion}
%%%%%%%%%%%

In this study, we addressed the inverse problem of determining an unknown time-dependent source term in a semilinear pseudo-parabolic equation (formulated on a bounded domain with a variable diffusion coefficient and damping coefficient, and with a Neumann boundary condition) from the integral overdetermination measurement. By applying Rothe's method, we established the existence and uniqueness of a weak solution under appropriate assumptions on the data. The Neumann boundary condition and the integral measurement over the entire domain were crucial in the adopted approach. Additionally, we designed a numerical time-discrete scheme to approximate the unique weak solution and the unknown source term, proving the convergence of these approximations. Finally, the theoretical results were validated through numerical experiments. Future work may explore the same inverse problem under a Dirichlet boundary condition and/or the local measurement $\int_\Omega u (t,\X) \omega(\X) \dX = m(t)$.

\section*{Funding} Dr.\ Kh. Khompysh is supported by grant no. AP23486218
 of the Ministry of Science and High Education of the Republic
of Kazakhstan (MES RK).

Dr.\ K. Van Bockstal is supported by the Methusalem programme of the Ghent University Special Research Fund (BOF) grant number 01M01021, and the FWO Senior Research Grant G083525N.
	
        
        \bibliography{refs}
		\bibliographystyle{unsrt}%abbrv
\end{document}