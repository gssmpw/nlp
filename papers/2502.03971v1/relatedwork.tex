\section{RELATED WORK}
\subsection{Visual prompts}
In multimodal learning, prompting has become a key technique for improving VLM performance. Traditionally, text-based prompting\cite{Texts}\cite{Prompting}\cite{Conditional-Prompt} inserts learnable prompts into input text to guide models in completing tasks. Visual prompts, on the other hand, use cues like color, shape, or position to improve a model's understanding of visual information. For example, CPT\cite{CPT}  uses color prompts to assist object recognition. T-Rex\cite{T-Rex} uses visual prompts for object counting, and T-Rex2\cite{T-Rex2} encodes points or bounding boxes as embeddings to support various reasoning workflows. Similarly to CPT and T-Rex,  our work combines color-based prompts with visual-language prompts to enhance model reasoning capabilities.
\subsection{UI unstanding}
Early research on the understanding of UI focuses mainly on task execution and intelligent navigation for web interfaces\cite{Navigate-the-Web}\cite{Reinforcement-Learning}. As UI complexity increased, research shifted towards multimodal Vision-Language Models (VLMs). Ferret-UI\cite{Ferret-UI} improves visual capabilities with an 'arbitrary resolution' strategy, while ScreenAI improves performance by refining the PaLI architecture and generating large-scale datasets. Furthermore, agents based on language models, such as Mobile-Agent\cite{Mobile-Agent} and AppAgent\cite{AppAgent}, integrate visual and language information, improving reasoning and interaction in complex UI scenarios.
\subsection{Visual Language Models} 
Visual-Language Models(VLMs) combine visual and textual information to handle complex reasoning and tasks. RWKV\cite{RWKV} is an efficient RNN architecture with linear complexity and constant memory usage, achieving GPT-level performance in language modeling. VisualRWKV\cite{VisualRWKV} extends RWKV to the visual-language domain, enabling efficient joint processing of visual and textual information and demonstrating advantages in long-sequence modeling.