\section{Privacy Risk in Federated Distillation}
\label{sec:assessement}
While FL is designed to protect clients' private data, recent research~\cite{gu2023ldia,nasr2019comprehensive, liu2023mia,yang2022fd,wang2024graddiff} reveals significant privacy risks in these frameworks. 
In FL, Gu \etal~\cite{gu2023ldia} demonstrated that server-side LDIA could achieve a KL-divergence of 0.01 between the inferred and the ground truth label distributions on CIFAR-10. 
Nasr \etal~\cite{nasr2019comprehensive} showed that server or client-side MIA could reach accuracies of 92.1\% and 76.3\%, respectively, on CIFAR-100.

FD frameworks transfer distilled knowledge between participants instead of informative model parameters and gradients. This mechanism generally provides more privacy protection for each client's data than traditional FL frameworks (FedAVG, FedSGD, etc.).
However, through the lens of LDIA and MIA, we observe that although privacy leakage risk in FD appears less severe than in FL, significant risks remain, as state-of-the-art privacy attacks can still achieve non-trivial success rates based on the results in the literature and our experiments. 
Our work is the first to propose a LDIA method targeting the FD frameworks, and we achieve a KL divergence of 0.02 between the inferred and the ground-truth label distributions on CIFAR-10. This attack is less successful than in the traditional FL frameworks, but label distribution leakage has been demonstrated.
Targeting the PDA-FD frameworks,
Liu \etal~\cite{liu2023mia} proposed a client-side MIA method attaining 67.0\% balanced accuracy on CIFAR-100. 
Yang \etal~\cite{yang2022fd} also demonstrated a client-side MIA method that achieved an up to 75\% balanced accuracy on CIFAR-100.
Similarly, our MIA methods (co-op LiRA and distillation-based LiRA) demonstrate considerable server-side MIA effectiveness in achieving a TPR of up to 35.76\% at a 1\% FPR on CIFAR-10. 
In addition, effective MIA methods are reported to target other FD frameworks. 
For example, Wang \etal~\cite{wang2024graddiff} reported that their MIA attack achieved 67.06\% and 79.07\% accuracy on FedGen~\cite{zhu2021data} and FedDistill~\cite{jiang2020federated} respectively, on CIFAR-10.
One of the objectives of our study is to motivate future research on privacy risks in various FD frameworks and, more broadly, FL frameworks.

