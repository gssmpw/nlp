%%
%% This is file `sample-sigconf.tex',
%% generated with the docstrip utility.
%%
%% The original source files were:
%%
%% samples.dtx  (with options: `all,proceedings,bibtex,sigconf')
%% 
%% IMPORTANT NOTICE:
%% 
%% For the copyright see the source file.
%% 
%% Any modified versions of this file must be renamed
%% with new filenames distinct from sample-sigconf.tex.
%% 
%% For distribution of the original source see the terms
%% for copying and modification in the file samples.dtx.
%% 
%% This generated file may be distributed as long as the
%% original source files, as listed above, are part of the
%% same distribution. (The sources need not necessarily be
%% in the same archive or directory.)
%%
%%
%% Commands for TeXCount
%TC:macro \cite [option:text,text]
%TC:macro \citep [option:text,text]
%TC:macro \citet [option:text,text]
%TC:envir table 0 1
%TC:envir table* 0 1
%TC:envir tabular [ignore] word
%TC:envir displaymath 0 word
%TC:envir math 0 word
%TC:envir comment 0 0
%%
%%
%% The first command in your LaTeX source must be the \documentclass
%% command.
%%
%% For submission and review of your manuscript please change the
%% command to \documentclass[manuscript, screen, review]{acmart}.
%%
%% When submitting camera ready or to TAPS, please change the command
%% to \documentclass[sigconf]{acmart} or whichever template is required
%% for your publication.
%%
%%
%\documentclass[manuscript, screen, review]{acmart}.
\documentclass[sigconf,balance=true]{acmart}
\usepackage{algorithm}
\usepackage{algorithmic}
\usepackage{enumitem}
\usepackage{graphicx}
\usepackage{multirow}
\usepackage{booktabs}
\usepackage{subcaption}
\usepackage{caption}
\usepackage{titlesec}
\usepackage{xspace}
\usepackage{amsmath}
%\usepackage{paralist}

\expandafter\def\expandafter\normalsize\expandafter{%
    \normalsize%
    \setlength\abovedisplayskip{3pt}%
    \setlength\belowdisplayskip{3pt}%
    \setlength\abovedisplayshortskip{2pt}%
    \setlength\belowdisplayshortskip{2pt}%
}


\setlength{\textfloatsep}{0pt}   % Space between floats and text
\setlength{\abovecaptionskip}{0pt} % Space above the caption
\setlength{\belowcaptionskip}{0pt}
\setlength{\floatsep}{0pt}    
\setlength{\intextsep}{0pt} 


\titlespacing*{\subsection}{0pt}{*0.2}{*0.2}
\titlespacing*{\section}{0pt}{*0.4}{*0.4}

%%

\newcommand{\etal}{{\em et al.}\xspace}
\newcommand{\BfPara}[1]{{\noindent {\bf #1.}}}

%%
%% \BibTeX command to typeset BibTeX logo in the docs
\AtBeginDocument{%
  \providecommand\BibTeX{{%
    Bib\TeX}}}
\settopmatter{printacmref=false}  % Removes ACM reference format
\renewcommand\footnotetextcopyrightpermission[1]{}  % Removes footnote with conference/copyright information

%% Rights management information.  This information is sent to you
%% when you complete the rights form.  These commands have SAMPLE
%% values in them; it is your responsibility as an author to replace
%% the commands and values with those provided to you when you
%% complete the rights form.
% \setcopyright{acmlicensed}
% \copyrightyear{2018}
% \acmYear{2018}
% \acmDOI{XXXXXXX.XXXXXXX}

% %% These commands are for a PROCEEDINGS abstract or paper.
% \acmConference[Conference acronym 'XX]{Make sure to enter the correct
%   conference title from your rights confirmation emai}{June 03--05,
%   2018}{Woodstock, NY}
%%
%%  Uncomment \acmBooktitle if the title of the proceedings is different
%%  from ``Proceedings of ...''!
%%
%%\acmBooktitle{Woodstock '18: ACM Symposium on Neural Gaze Detection,
%%  June 03--05, 2018, Woodstock, NY}
% \acmISBN{978-1-4503-XXXX-X/18/06}


%%
%% Submission ID.
%% Use this when submitting an article to a sponsored event. You'll
%% receive a unique submission ID from the organizers
%% of the event, and this ID should be used as the parameter to this command.
%%\acmSubmissionID{123-A56-BU3}

%%
%% For managing citations, it is recommended to use bibliography
%% files in BibTeX format.
%%
%% You can then either use BibTeX with the ACM-Reference-Format style,
%% or BibLaTeX with the acmnumeric or acmauthoryear sytles, that include
%% support for advanced citation of software artefact from the
%% biblatex-software package, also separately available on CTAN.
%%
%% Look at the sample-*-biblatex.tex files for templates showcasing
%% the biblatex styles.
%%

%%
%% The majority of ACM publications use numbered citations and
%% references.  The command \citestyle{authoryear} switches to the
%% "author year" style.
%%
%% If you are preparing content for an event
%% sponsored by ACM SIGGRAPH, you must use the "author year" style of
%% citations and references.
%% Uncommenting
%% the next command will enable that style.
%%\citestyle{acmauthoryear}

\captionsetup{skip=5pt}
\setlength{\belowcaptionskip}{2pt}
\setlength{\textfloatsep}{2pt}
\setlength{\floatsep}{2pt}
\setlength{\intextsep}{2pt}



%%
%% end of the preamble, start of the body of the document source.
\begin{document}

%%
%% The "title" command has an optional parameter,
%% allowing the author to define a "short title" to be used in page headers.
%\title{Privacy Risks in Public Dataset-Assisted Federated Distillation}
\title{Unveiling Client Privacy Leakage from Public Dataset Usage in Federated Distillation}

%Client Privacy Leakage from Public Dataset Usage in Federated Distillation: Through the lens of Membership and Label Distribution Inference Attacks

%%
%% The "author" command and its associated commands are used to define
%% the authors and their affiliations.
%% Of note is the shared affiliation of the first two authors, and the
%% "authornote" and "authornotemark" commands
%% used to denote shared contribution to the research.
% \author{}
% \authornote{}
% \email{}
% \orcid{}
% \author{}
% \authornotemark[1]
% \email{}
% \affiliation{%
%   \institution{}
%   \city{}
%   \state{}
%   \country{}
% }
\author{Haonan Shi}
\email{haonan.shi3@case.edu}
\affiliation{%
  \institution{Case Western Reserve University}
  \city{Cleveland}
  \state{Ohio}
  \country{USA}
}

\author{Tu Ouyang}
\email{tu.ouyang@case.edu}
\affiliation{%
  \institution{Case Western Reserve University}
  \city{Cleveland}
  \state{Ohio}
  \country{USA}
}

\author{An Wang}
\email{an.wang@case.edu}
\affiliation{%
  \institution{Case Western Reserve University}
  \city{Cleveland}
  \state{Ohio}
  \country{USA}
}
%%
%% By default, the full list of authors will be used in the page
%% headers. Often, this list is too long, and will overlap
%% other information printed in the page headers. This command allows
%% the author to define a more concise list
%% of authors' names for this purpose.
\renewcommand{\shortauthors}{Trovato et al.}
\newcommand{\haonan}[1]{{\leavespace {\color{blue} #1}}}

%%
%% The abstract is a short summary of the work to be presented in the
%% article.
\begin{abstract}
Federated Distillation (FD) has emerged as a popular federated training framework, enabling clients to collaboratively train models without sharing private data.
Public Dataset-Assisted Federated Distillation (PDA-FD), which leverages public datasets for knowledge sharing, has become widely adopted. 
Although PDA-FD enhances privacy compared to traditional Federated Learning, we demonstrate that the use of public datasets still poses significant privacy risks to clients' private training data.
This paper presents the first comprehensive privacy analysis of PDA-FD in presence of an honest-but-curious server. 
We show that the server can exploit clients' inference results on public datasets to extract two critical types of private information: label distributions and membership information of the private training dataset.
To quantify these vulnerabilities, we introduce two novel attacks specifically designed for the PDA-FD setting: a label distribution inference attack and innovative membership inference methods based on Likelihood Ratio Attack (LiRA).
Through extensive evaluation of three representative PDA-FD frameworks (FedMD, DS-FL, and Cronus), our attacks achieve state-of-the-art performance, with label distribution attacks reaching minimal KL-divergence and membership inference attacks maintaining high True Positive Rates under low False Positive Rate constraints. 
Our findings reveal significant privacy risks in current PDA-FD frameworks and emphasize the need for more robust privacy protection mechanisms in collaborative learning systems.
%Federated Distillation (FD) has emerged as a popular federated training framework, enabling clients to collaboratively train models without sharing private data. Public Dataset-Assisted Federated Distillation (PDA-FD) is a widely adopted FD framework that utilizes a public dataset for sharing knowledge in collaborative training.
% , allowing servers only black-box access to clients via public dataset.
%While PDA-FD offers enhanced privacy compared to the traditional Federated Learning, the use of the public datasets might still pose significant privacy risks to clients.
%This paper investigates privacy leakage in PDA-FD from the server's perspective as an attacker.
%focusing on clients' label distribution and membership information. 
%\haonan{We discover that by leveraging public datasets, the server can obtain private information about clients' private dataset, including label distribution and membership information.}
%We propose a label distribution attack method and novel membership inference attack methods that extend LiRA. 
%We evaluate the proposed attacks on three PDA-FD frameworks: FedMD, DS-FL, and Cronus.
%The proposed label distribution and membership inference attacks demonstrate superior performance, achieving state-of-the-art results in KL-divergence and True Positive Rates under low False Positive Rates, respectively.
%These findings underscore the potential privacy vulnerabilities introduced by the public dataset usage in FD, highlighting the need for enhanced privacy-preserving mechanisms in collaborative training environments.
\end{abstract}

%%
%% The code below is generated by the tool at http://dl.acm.org/ccs.cfm.
%% Please copy and paste the code instead of the example below.
%%
% \begin{CCSXML}
% <ccs2012>
%    <concept>
%        <concept_id>10002978</concept_id>
%        <concept_desc>Security and privacy</concept_desc>
%        <concept_significance>500</concept_significance>
%        </concept>
%    <concept>
%        <concept_id>10010147.10010257</concept_id>
%        <concept_desc>Computing methodologies~Machine learning</concept_desc>
%        <concept_significance>500</concept_significance>
%        </concept>
%  </ccs2012>
% \end{CCSXML}

% \ccsdesc[500]{Security and privacy}
% \ccsdesc[500]{Computing methodologies~Machine learning}

%%
%% Keywords. The author(s) should pick words that accurately describe
%% the work being presented. Separate the keywords with commas.
% \keywords{Privacy attack, federated distillation, label distribution inference attack, membership inference attack.}

%%
%% This command processes the author and affiliation and title
%% information and builds the first part of the formatted document.
\maketitle

\documentclass[../main.tex]{subfiles}
\graphicspath{{../images/}}
\makeatletter
\def\input@path{{../images/}}
\makeatother
\begin{document}
\section{Introduction}
\begin{figure}
\centering
\begin{tikzpicture}
\node[inner sep=0pt] (ws) at (0, 0) {
\includegraphics[height=.4\textwidth, trim={10cm 0 10cm 0},clip]{world_space.png}};
\node[inner sep=0pt] (cs) at (6,0) {\includegraphics[height=.4\textwidth, trim={10cm 1cm 10cm 4cm},clip]{conf_space.png}};
\end{tikzpicture}
\vspace{-5pt}
\label{fig:pbrm_intro}
\caption{\textbf{Left}: Shows world space obstacles as grey spheres. Robots start and goal configuration is colored red and green, respectively. Configurations along the computed path are colored transparent blue. \textbf{Right:} Mapped world space scenario to configuration space. Obstacle region is the grey mesh. Red spheres are collision-free regions computed by the neural SCDF. The optimized shortest path in the convex corridor is the blue curve.}
\vspace{-25pt}
\end{figure}
Motion planning is the problem of finding a collision-free trajectory that connects a given start and goal configuration. The planning takes place in the configuration space of the robot. For single body robots, like mobile robots or drones, the configuration space and the world space are usually the same. This simplifies the planning, since explicit obstacle representations are available which enables geometrical tools like separating hyperplanes, smallest distance to obstacles etc., to be used when designing motion planning algorithms. For multi-body robots like manipulators, the situation is completely different. The world space obstacles are usually mapped to non-convex regions, and to make the problem even harder, the mapping is usually not known. Forming explicit representations of the obstacle region in the configuration space is usually too expensive or intractable. Despite all of this, sampling based planners are used with great success, which mainly is due to their use of implicit representations of the obstacle region. The basic idea is to construct a graph in the configuration space that covers and connects the collision-free region. From this graph, a path can be extracted that connects a given start and goal configuration. The approach is computationally expensive, since the graph is constructed with the smallest geometrical building block available, points, which represents a collision-check. Furthermore, the extracted paths from the graph are non-smooth and jagged due to the stochastic nature of the approach. This adds an additional post-processing step to the process, where the paths are shortcutted and smoothened, before the path can be used for tracking. Clearly a lot of time is invested to form this graph and produce smooth paths. Thus, if the obstacles start to move, then all of this work is done in no use, since all points that make up this graph need to be re-verified, which is simply too time consuming to be done in real time.
\\\\
In this work, we want to address the existing drawbacks of the sampling based planners. Our main contribution is an improved motion planner where each vertex in the graph covers a collision-free region in the form of a sphere instead of a point and where the edges are formed with neighboring intersecting spheres. This representation has the advantage of instead of returning piecewise linear paths, returning a sequence of overlapping spheres, i.e. a convex corridor, that connects a given start and goal configuration, illustrated in Figure \ref{fig:pbrm_intro}. This convex corridor allows us to use convex optimization to produce smooth trajectories, instead of computationally expensive post-processing methods. The representation further allows us to estimate the coverage of the collision-free space, which gives us awareness and feedback in the offline roadmap construction phase. Finally, our representation is simple to adapt to moving obstacles, simply requery for the new radii and recheck for intersections. 
\\\\
The spherical collision-free regions are formed using a signed distance function (SDF), which is a function that returns the smallest distance from an arbitrary point to the boundary of an obstacle. As the name implies, the distance is signed, thus if the point is inside the obstacle it is negative otherwise positive. If the distance is positive, a sphere with radius equal to the distance is guaranteed to cover a collision-free region. Using an SDF in motion planning is not new, but what is novel about our approach is that we express the distance in the configuration space instead of the world space and by doing so allows us to form these convex collision-free regions. We refer to the resulting SDF as a signed configuration distance function (SCDF). Computing an SCDF analytically is non-trivial, our approach is therefore to parameterize the SCDF with a deep neural network and learn the mapping by supervised learning. Our resulting neural SCDF can compute distances for different parameter values of obstacle shapes and we also show how multiple distances can be combined, thus making our approach flexible.
\section{Related work}
Motion planning algorithms can roughly be divided into three families, grid-based, sampling based and optimization based methods. Grid-based methods (GBM) discretize the planning space from which a graph is then compiled. A standard search method is A$^\star$ \citep{a_star}, which is classified as an \textit{informed} search method, since it employs a heuristic function to speed up the search. A$^\star$ guarantees to return an optimal path at the level of discretization used. GBMs usually discretize the planning space by a regular lattice and this limits the GBMs to problems with low dimensionality due to the curse of dimensionality. Thus, GBMs are usually limited to single-body robots where the degrees of freedom (DOF) are low. To overcome the inherent scaling problem with the GBMs, stochastic methods are usually used for multi-body robots. These methods are termed as sampling-based methods (SBM) and core members within this family are the rapidly-exploring random trees (RRT) \citep{rrt} and the probabilistic roadmap (PRM) \citep{prm}. RRT grows a tree from the start configuration and explores the collision-free region in a rapid way until it is able to connect to the goal region. RRT is usually improved by bi-directional planning \citep{rrt_connect}, i.e. an additional tree is grown from the goal configuration and the trees are tested for connection after any tree has been expanded. RRT is a single-query method, thus it searches for a path from scratch each time it is queried. Contrary to this, PRM is a multi-query method, which solves for multiple queries without starting from scratch. PRM does this by creating a roadmap (graph) that covers the collision-free space as an offline step. The graph is then used to solve for multiple queries. PRMs are used in cases where the environment does not change since the extra offline step is too computationally costly and needs to be re-done if the environment is changed. In our work, we address this inherent issue by using a different roadmap representation. Our vertices in the graph cover a collision-free region in the form of spheres and we form the edges by checking for intersecting spheres. If something in the environment changes, we recompute the spheres radii and recheck the intersections, without relying on collision detection. We use a trained neural network to compute the sphere radius, therefore querying for the radius can be done fast, hence our representation enables the PRM for dynamic environments.
\\\\
In the recent decades, optimization based methods (OBM) \citep{chomp, schulman, itomp, stomp} have been introduced as an alternative to SBM for multi-body robots. Like the SBM, the OBMs scale well to higher dimensional problems and produce smoother motion. It is common to use a SDF in the optimization since it is a smooth function, thus enabling gradient-based methods. However, the standard way of expressing the SDF is in world space. The distance therefore needs to be mapped to the configuration space by the forward kinematics. This mapping makes the optimization problem a non-linear program (NLP), which is computationally expensive to solve. Recently, a different approach has been proposed. In \cite{mp_gcs} motion planning is formulated as a convex optimization problem by using the graph of convex sets framework \citep{gcs}. The underlying idea is to decompose the collision-free space into intersecting convex sets from which a convex optimization problem is formulated. In cases where an explicit representation of the obstacles in the configuration space exists, like for single-body robots, creating collision-free convex regions can be done fast \citep{iris}. For multi-body robots, this is non-trivial. Existing work does this successfully \citep{iris_nlp, iris_c} by an optimization based approach, but the methods are still too time consuming to be used in the presence of moving obstacles. Our approach is instead to use deep learning to learn an SDF expressed in the configuration space. With this, we can query for shortest distances to the collision boundary, which allows us to expand spherical regions which are collision-free. Our approach is fast and therefore enables our suggested roadmap planner to be used in dynamic environments.
\\\\
Recent research has focused on learning collision detection \citep{fk_kernel_distance, diffco, graphdistnet} by predicting the signed distance between the robot links and the surrounding obstacles in the world space. The learned SDF is used in trajectory optimization but since the distance is expressed in the world space, the problem becomes an NLP and therefore takes a long time to solve. We take a novel approach and suggest to instead express the signed distance in the configuration space. This allows us to improve the PRM at the same time as it enables convex optimization for trajectory optimization, which runs faster and is more reliable than NLP solvers. In \cite{cspf} a learned signed distance function in the configuration space is proposed similar to our approach. However, their approach is restricted to point cloud representations, while we propose to represent the obstacles as parameterized geometric shapes, e.g. spheres. Furthermore, we also show how to use our learned SCDF to improve an existing roadmap planner.
\section{Problem formulation}
A robot is located in the world space, $\W \subset \R^3 $. The unique location of the robot is given by its configuration $\q \in \C$, where $\C$ is the configuration space. The set of points covered by the robots bodies at a certain configuration is expressed as $\B(\q) \subset \W$. The robot is surrounded by $\NrObst$ obstacles $\O = \bigcup_{i=1}^{\NrObst} \O_i$, where  $\O_i \subset \W$. The representation of the obstacle in the configuration space is the set $\C\O_i = \{\q \in \C \: |\: \B(\q) \cap \O_i \neq \emptyset \}$. The obstacle space is formed as $\Co = \bigcup_{i=1}^{\NrObst} \C \O_i$. The complement is referred to as the free space, $\Cf = \C \setminus \Co$. The path planning problem is a tuple, ($\Cf$, $\qStart$, $\qGoal$), where we want to connect a query pair, consisting of a start, $\qStart$, and goal configuration, $\qGoal$, with a geometric path, $\q(s): [0, 1] \mapsto \Cf$, such that $\q(0)=\qStart$ and $\q(1)=\qGoal$, or report correctly when such a path does not exist.
\end{document}

\section{Basic Background: Supervised Learning and the PAC Model}
\label{sec:background}

At this point almost everyone has heard of machine learning (ML). Anyone likely to stumble upon this article will have also heard of its most influential special case, supervised learning, and those theoretically inclined will also be familiar with the PAC model. Nonetheless, I will set the stage by  recapping the basics.

\subsection{Basics of Supervised Learning}%Let's set the stage in any case

\emph{Supervised Learning} is the task of ``coming up'' with a function $f: \X \to \Y$ to ``explain'' or ``fit'' a sequence of input/output examples   $(x_1,y_1), \ldots, (x_n,y_n)$, with $x_i \in \X$ and $y_i \in \Y$.  Here $\X$ is a \emph{data domain} consisting of \emph{datapoints} $x \in \X$, $\Y$ is a \emph{label set} consisting of \emph{labels} $y \in \Y$, and the sequence $(x_1,y_1),\ldots,(x_n,y_n)$ is the \emph{training data} consisting of \emph{labeled examples (a.k.a. samples)}~$(x_i,y_i)$.  I~will refer to the chosen function $f$ as a \emph{predictor}, and to $n$ as the \emph{sample size}. A \emph{learning algorithm} takes as input training data, and outputs (some representation of) a predictor $f \in \Y^\X$.\footnote{Note that this describes the usual \emph{batch}, a.k.a.~\emph{offline}, setting of supervised learning. I do not discuss other paradigms such as online or active learning in this article.} 



Success in supervised learning is defined as \emph{generalization} to  future examples: For a typical \emph{test example}  $(x_{\tst},y_{\tst})$, the predicted label $y'_{\tst}=f(x_{\tst})$ should ``equal'' $y_{\tst}$, perhaps approximately. We usually assume the test example is drawn from the same  ``source'' as the training data  --- commonly, i.i.d.~from the same distribution. The quality of the prediction is quantified by $\ell(y'_{\tst},y_{\tst})$, where $\ell:~\Y~\times~\Y \to \RR_{\geq 0}$ is a \emph{loss function} chosen as part of the problem definition. Common loss functions include the 0-1 loss $\ell_{0-1}(y',y) = [y' \neq y]$ for \emph{classification} problems,\footnote{The notation $[P]$ denotes $1$ when predicate $P$ is true, and denotes $0$ when $P$ is false.} as well as the absolute loss $|y'-y|$ or squared loss $(y'-y)^2$ for \emph{regression problems} featuring $\Y  \sse \RR$.

Nontrivial generalization properties are typically only possible if one assumes something about the data.\footnote{The need for such an assumption is formalized by the  \emph{no free lunch theorems} of supervised learning \cite{wolpert_connection_1992,wolpert_lack_1996,schaffer_conservation_1994}.} The Bayesian approach to  machine learning, common in many applications, assumes some parametric form for the distribution generating the data, and postulates a prior on the parameters. This is not the approach I will take in this article. Instead, I will focus on the frequentist --- and some would say ``worst-case'' or ``adversarial'' ---  approach that is common in the computational learning theory community, embodied by the PAC model. Here we assume that the (training and test) data can be explained, perhaps approximately, by a function in some ``simple enough to learn'' class of functions $\H \sse \Y^\X$, often called the \emph{hypotheses}. Equivalently, we  seek a predictor which explains the unseen data roughly  as well as the best hypothesis $h^* \in \H$, whether or not we assume that $h^*$ itself provides a perfect explanation.



 \paragraph{Common Algorithmic Templates.} Perhaps the best known general-purpose supervised learning algorithm is \emph{empirical risk minimization (ERM)}, which chooses as its predictor a hypothesis $f \in \H$ minimizing $\frac{1}{n} \sum_{i=1}^n \ell(f(x_i),y_i)$ --- a quantity called the \emph{training error}, \emph{empirical error}, or \emph{empirical risk} of $f$. %\footnote{When multiple hypotheses minimize the empirical risk, we assume ERM breaks ties arbitrarily.}
A common template for generalizing ERM involves adding a \emph{regularization term} $\psi(f)$ to the  objective function, typically chosen to measure some notion of ``hypothesis complexity.'' An algorithm instantiating this template is known as a \emph{structural risk minimizer (SRM)}, and chooses as its predictor the hypothesis $f \in \H$ minimizing the \emph{structural risk} $\frac{1}{n} \sum_{i=1}^n \ell(f(x_i),y_i) + \psi(f)$. Other well-known algorithms, such as gradient descent and its variations,  can frequently be interpreted as approximate implementations of ERM or SRM.


\paragraph{Proper vs Improper Learning.} A learning algorithm is said to be \emph{proper} if its predictor $f$ is always chosen from the hypothesis class, i.e., $f \in \H$, otherwise it is said to be \emph{improper}. ERM  is an example of a proper learning algorithm, as are SRM algorithms of the form described above.  In the \emph{proper regime} of learning, algorithms are required to be proper. This article will be concerned with the more flexible \emph{improper regime} (a.k.a \emph{representation-independent learning}), where no such constraint is placed on the learner. In other words, all we care about is predictive power at test time, rather than any insights derived from the functional form or representation of the predictor~itself.


\subsection{The PAC Model}
A standard mathematical setup for evaluation of supervised learning algorithms, at least in the theoretical computer science community, is Valiant's \emph{Probably Approximately Correct (PAC) model} of learning (see e.g.~\cite{kearns_introduction_1994,mohri_foundations_2018}). Here, we assume there is an unknown distribution $\D$ on $\X \times \Y$ from which training and test data are  drawn.  Specifically, the labeled datapoints of the training set  $(x_1,y_1), \ldots, (x_n,y_n)$, as well as the test data  $(x_\tst,y_\tst)$, are i.i.d.~from $\D$. Often it is assumed that $\D$ lies in some class of distributions of interest. The \emph{true expected loss}, or simply \emph{loss}, of a predictor $f: \X \to \Y$ is the expected loss it incurs on draws from $\D$, written $L_\D(f) = \Ex_{(x,y) \sim \D} \ell(f(x),y)$.


There are two main ``settings'' in PAC learning. The  \emph{realizable setting} only requires that the data be perfectly explained by some hypothesis in $\H$. More generally, the \emph{agnostic setting} makes no assumption relating the data to the hypotheses, but shifts the goalposts as necessary to allow nontrivial guarantees: the expected loss at test time is evaluated only ``relative'' to that of the best hypothesis $h^* \in \H$. There are other settings which make more nuanced assumptions, such as $\D$ being of a particular parametric form or its support living in some (unknown) lower-dimensional space, etc. I will mostly discuss the realizable and agnostic settings in this article, those being the simplest and most studied from a theoretical perspective. %TODO:We will briefly discuss other settings in Section ??

The PAC model demands high probability guarantees of learners, in the worst case over distributions of interest. Consider first the realizable setting, where $\D$ is such that $\min_{h \in \H} L_{\D}(h) = 0$. A PAC learner has \emph{error} $\epsilon=\epsilon(n)$ and \emph{confidence} $\delta=\delta(n)$ if, when training data consists of $n$ i.i.d~samples from a realizable distribution $\D$, it produces a predictor $f$  satisfying $L_\D(f) \leq \epsilon$ with probability at least $1-\delta$. In the agnostic setting, where $\D$ can be arbitrary, we require $L_\D(f) - \min_{h \in \H} L_\D(h) \leq \epsilon$ with probability $1-\delta$.

In both the realizable and agnostic settings, we look for PAC learners with small $\epsilon$ and $\delta$ as a function of the sample size $n$. An equivalent perspective looks at the sample complexity $m(\epsilon,\delta)$, which is the minimum sample size which guarantees error  at most $\epsilon$ with probability at least $1-\delta$. We say a problem is \emph{PAC learnable} if its PAC sample complexity is finite whenever $\epsilon,\delta > 0$.

For most PAC learning problems, learnability and sample complexity are characterized in terms of a  ``dimension'' of the hypothesis class. Most prominently this is the \emph{VC dimension} for binary classification, the \emph{fat shattering dimension} for agnostic regression, and the \emph{DS dimension} for multiclass classification (see \cite{anthony_neural_1999,daniely_optimal_2014,brukhim_characterization_2022}). Treatment of these is beyond the scope of this article. The unfamiliar reader need not worry, however,  as dimensions will feature only tangentially in our~discussion.




%\paragraph{Learning settings: Realizable, Agnostic, etc.} In learning theory, evaluating a supervised learning algorithm requires specifying a data model and an objective. We will leave the details of the data model flexible for now, to allow for both the PAC model and the adversarial transductive model. Nonetheless we will describe two variations, which we call ``settings'', which cut across different models. The  \emph{realizable setting}  requires only that the data be perfectly explained by some hypothesis $h \in \H$ --- i.e., there exists a hypothesis which is guaranteed to suffer a loss of $0$ on training and test data. The performance of the learning algorithm is its expected loss at test time for some ``worst case'' realizable instance. More generally, the \emph{agnostic setting} makes no assumption relating the data to the hypotheses, but shifts the goalposts as necessary to allow nontrivial guarantees: the expected loss at test time is evaluated only ``relative'' to that of the best hypothesis $h^* \in \H$, again for some ``worst case'' instance. There are other settings which make more nuanced assumptions about the data, such as it is drawn from a distribution of a particular parametric form, or that it lives in some (unknown) lower-dimensional space, etc. We will mostly discuss the realizable and agnostic settings, those being the simplest and most studied from a theoretical perspective.




%%% Local Variables:
%%% mode: latex
%%% TeX-master: "learning_matching"
%%% End:


\section{\label{sec:method}Methodology}

Each SIEM system uses its own RDL to define threat detection rules, and each RDL has its own schema.
For example, the Splunk SIEM uses the SPL to define its threat detection rules.
The task of understanding threat detection rules and recommending relevant MITRE ATT\&CK techniques (or sub-techniques) requires complex reasoning skills.
In the case of LLMs, this can be achieved with a technique called prompt chaining in which each task is divided into multiple sub-tasks in order to understand the complex reasoning behind the task.
Therefore, we employ a multi-phase architecture based on prompt chaining that leverages the power of LLMs to take a SIEM rule defined in any RDL and map it to relevant MITRE ATT\&CK techniques using the power of LLMs.
Our approach is based on the following intuitions:
\begin{itemize}[nosep,leftmargin=*]
    \item \textit{LLMs' implicit knowledge}: LLMs possess deep understanding of diverse RDLs. This enables them to interpret any rule, regardless of the RDL it is defined in, and convert it into comprehensible natural language text.
    \item \textit{LLMs' similarity comparison capability}: LLMs are adept at analyzing and comparing textual descriptions. 
    They can intelligently assess the similarity between two textual inputs to establish a meaningful connection.
\end{itemize}

\methodName has two main phases: (1) the rule to text translation phase, and (2) the MITRE ATT\&CK techniques recommendation phase.
These two phases in the pipeline include six key steps to determine relevant TTPs, as illustrated in Figure~\ref{fig:r2t}.

Although LLMs excel at translating SIEM rules into natural language, they often lack critical domain-specific contextual information related to IoCs in the rules.
To overcome this limitation, the \textit{rule to text translation} phase includes three steps: IoC extraction, contextual information retrieval, and natural language translation.

The workflow begins with the extraction of IoCs from the rules (for example, processes, log source, event codes, and file names) that the rule searches for in the logs (step (1)).In the next sstep a web search agent performs the task of obtaining additional contextual information about the IoCs discovered ((step 2)).
By incorporating this additional domain-specific information, the pipeline enhances the language translation, resulting in a more accurate and meaningful interpretation of SIEM rules.
The rule itself and the IoCs' contextual additional information from the previous stage are then used to translate the rule from RDL to natural language (step (3)).

The \textit{MITRE ATT\&CK techniques} recommendation phase of the pipeline includes the following three steps.
The rule in processed in data source identification step in which the probable origin of the data is identified (step (4)).
The description of the rule is then used to determine probable MITRE ATT\&CK techniques based on the implicit knowledge of the LLM (step (5)).
Finally, using chain-of-thought~\cite{wei2022chain} prompting, the most relevant techniques are extracted from the list of probable techniques (step (6)).
Each step of our method is further described in detail below.


% [bb=0 0 1440 900,width=1.43\linewidth,height=0.9\textwidth]
\begin{figure*}[htbp]
   \includegraphics[width=\textwidth]{Images/stages.jpg}
    
   \caption{An illustration of the different steps in \methodName.}
   \label{fig:stages}
\end{figure*} 

\subsection{IoC Extraction}
The context associated with a SIEM detection rule is crucial for its accurate interpretation and effective application. 
Obtaining this contextual understanding requires comprehensive analysis of the embedded IoCs in the SIEM rule.
In the first step, \methodName systematically identifies and extracts all IoCs, identifying the types of IoCs and their corresponding values that form the foundational elements of the detection rules. 
Leveraging the LLM's inherent understanding of rule structures and IoCs, we employ a zero-shot prompting approach for this task. 
Zero-shot prompting enables the direct extraction of IoCs from the rules without requiring extensive pre-training on specific datasets.

\noindent The result of this stage is a dictionary structure, where:
\begin{itemize}[nosep,leftmargin=*]
    \item Keys represent types of IoC, such as processes, files, IP addresses, and log sources.
    \item Values are lists containing specific IoC details, such as process names, file names, IP addresses, and log source identifiers.
\end{itemize}

In the example depicted in Figure~\ref{fig:stages}(a), the pipeline processes a rule for which relevant MITRE ATT\&CK techniques need to be recommended. 
The IoC extractor LLM produces a dictionary structure as output, organizing the IoCs in a structured format to support subsequent stages in the analysis pipeline. 



\subsection{Contextual Information Retrieval}
In this step, an LLM agent is employed to retrieve relevant information pertaining to the IoCs extracted from the rule.
A REACT agent~\cite{react} was used in this case to generate both reasoning traces and task-specific actions in an interleaved manner.
REACT agents interact with external tools to retrieve additional information that leads to more factual and reliable responses.
The LLM agent conducts a systematic search across web resources to gather additional contextual information for each IoC value present in the rule. 
This step addresses LLMS' lack of up-to-date knowledge or specialized domain expertise (which is critical to understanding the role and significance of the IoCs in the rule), without the need for retraining or fine-tuning.
Figure~\ref{fig:stages}(b) presents an example in which the rule includes the process name \texttt{soaphound.exe} as an IoC.
As can be seen, the web search results indicate that \texttt{soaphound.exe} is being used for active directory (AD) enumeration, which is important for the understanding of the attack. 

\subsection{Natural Language Translation}

The translation of detection rules into natural language textual descriptions fulfills three key objectives:
\begin{enumerate}[nosep,leftmargin=*]
    \item \textbf{Ensures that \methodName is format-agnostic}: It converts rules defined in various RDL formats into a generic, unstructured text format, ensuring compatibility with different SIEM systems, regardless of the specific rule format.
    \item \textbf{Provides contextual explanation}: It includes all relevant contextual information to produce a concise and comprehensible explanation of the rule.
    \item \textbf{Enhances the comprehension for LLMs}: It enables LLMs to more effectively compare the translated rule with descriptions in the MITRE ATT\&CK framework by providing a unified textual representation.
\end{enumerate}
To achieve these objectives, a zero-shot prompting technique is employed. 
The input to the LLM comprises two components:
\begin{itemize}
    \item \textbf{Syntactical information}: The rule itself, providing the structural and operational details.
    \item \textbf{Contextual information}: Details of the IoCs extracted from the rule, providing semantic insights into the rule's intent and function.
\end{itemize}
The LLM utilizes these inputs to generate a natural language textual description of the rule. 
This transformation not only ensures a more interpretable representation but also facilitates further steps of analysis and comparison, particularly in aligning the rule with MITRE ATT\&CK techniques and sub-techniques.



\subsection{Data Source or Mitigation Identification}
Identifying the most relevant data component or mitigation associated with the rule description in this step is critical for filtering out irrelevant MITRE ATT\&CK techniques (or sub-techniques) in subsequent steps of the pipeline.
In the MITRE ATT\&CK framework, data sources represent various categories of information that can be gathered from sensors or logs. 
These data sources include \textit{data components}, which are specific attributes or properties within a data source that are directly relevant to detecting a particular technique or sub-technique~. 
For example, in the context of the rule described in Figure~\ref{fig:stages}(a), the term \texttt{Endpoint.Processes} indicates that the activity is happening on an endpoint. 
Presence of the terms such as, \texttt{soaphound.exe}, \texttt{--buildcache}, \texttt{--certdump} and etc. indicate that the rule searches for command line execution of an executable named \texttt{soaphound.exe} with specific parameters. 
Therefore, the appropriate data source in this example is \textit{Command}, with the corresponding data component being \textit{Command Execution}.
Additionally, \textit{mitigations} are defined as categories of technologies or strategies that can prevent or reduce the impact of specific techniques or sub-techniques. 
The MITRE ATT\&CK framework explicitly establishes relationships between data components, mitigations, and techniques (or sub-techniques), enabling a systematic approach for identifying relevant elements.

To identify the most relevant data component or mitigation associated with a given rule description, we utilize agentic retrieval augmented generation (RAG), which incorporates an AI Agent-based implementation of the RAG framework.
Data from the MITRE ATT\&CK framework, specifically related to data components and mitigations, is stored in a vector database (e.g., ChromaDB). 
The process begins with the rule description from the previous stage, which serves as the input to the AI Agent. 
The LLM-powered agent automatically generates a search query tailored to retrieve relevant information from the RAG database.

For each query, the system retrieves the five most similar documents from the database, each containing contextual information about data components or mitigations. 
These documents are then utilized by the LLM agent to contextualize the rule description. 
By comparing the content of these retrieved documents with the rule description, the LLM agent determines and outputs the most relevant data component or mitigation along with a chain-of-thought as to why the data component or mitigation is related to the rule.


\subsection{Probable Technique Recommendation}

In this step, an LM agent is utilized to propose probable MITRE ATT\&CK techniques (and sub-techniques) that may be relevant to the description of the provided rule.
We used a REACT agent in this step as well to utilize both implicit and explicit knowledge during reasoning.
For explicit knowledge, the agent searches the MITRE ATT\&CK framework to obtain the list of probable techniques (and sub-techniques).
The natural language description of the rule from the previous step serves as input to the LLM agent.
The output of this stage consists of a list of JSON objects, each containing the MITRE technique ID, technique name, and technique description as seen in Figure~\ref{fig:stages}(c).

Throughout our experiments, we observed that as the number of recommendations ($k$) increases, both the framework's average recall and precision initially improve, however beyond a certain threshold of $k$, the %average 
precision begins to decline.
Based on these observations(please refer Table~\ref{tab:results3}), we selected a $k$-value of 11 to ensure a high recall.



\subsection{Relevant Technique Extraction}
In this step, \methodName refines the set of probable MITRE ATT\&CK techniques identified in the previous stage by eliminating irrelevant entries.
This step in the pipeline serves two primary purposes: (1) to enhance precision while maintaining recall achieved in previous step, and (2) to provide a clear rationale for the selection of the labels, ensuring transparency and interpretability of the mapping process.
This refinement process is grounded in the assumption that LLMs are effective for text similarity matching tasks.

The process comprises two key steps:
\begin{itemize}
    \item \textit{Rule-technique comparison}: The description of each technique in the set of probable techniques is compared with the rule description. 
    A chain-of-thought technique is then applied to elucidate the reasoning behind the association of each technique with the rule.
    \item \textit{Confidence calculation}: The generated chain-of-thought rationale for each technique (or sub-technique) is compared with the rule description to compute a relevance (or confidence) score, as done in prior work~\cite{freitas2024ai}.
    % \item \textbf{Reasoning}: \new{Add here the reasoning that it provides - explaining in NLP why it was selected...}
\end{itemize}

Techniques with higher confidence scores are deemed more relevant to the rule. 
Conversely, techniques with scores falling below a predefined threshold are excluded.
The techniques retained after this filtering step represent the most relevant techniques corresponding to the given rule's description. 


The chain-of-thought (CoT) rationale generated during the comparison of each rule to its probable technique is also provided as an output in this step.
This rationale offers a detailed natural language explanation, articulating why a particular technique is relevant to the given rule. 
Such explanations are highly valuable for security analysts, as they provide clear and transparent reasoning behind the mapping, enabling analysts to better understand and validate the association between the rule and the technique.
Other classification models proposed in previous works within this domain also suffer from the limitation of being black-box models, which lack the ability to provide clear reasoning or explanations. 
Unlike \methodName, these models fail to generate transparent, CoT rationales that explain why a particular rule is mapped to a specific technique, making them less interpretable and less useful for security analysts.
\section{Evaluations}
\label{sec:eval}
In this section, we conduct a series of experiments to evaluate the privacy leakage in PDA-FD by conducting the proposed LDIA and MIA.  
% LDIA and MIA method proposed in this paper to measure the privacy information leakage from the client side in Public Dataset-Assisted Federated Distillation frameworks.
Our experiments span multiple datasets, various PDA-FD frameworks~\cite{chang2019cronus, li2019fedmd, itahara2021distillation}, and different usage scenarios. 
Our experiments demonstrate four key aspects: (1) the overall effectiveness of our proposed LDIA and MIA; (2) the impact of varying non-IID data distributions; (3) the impact of different PDA-FD frameworks; and (4) the effect of the number of collaborative training rounds.
% These experiments involved measurements across different datasets, various PDA-FD frameworks\cite{chang2019cronus, li2019fedmd, itahara2021distillation}, and a range of usage scenarios for PDA-FD.


\subsection{Experiment Setup}
\subsubsection{Datasets}
In our experiments, we utilize the following image datasets that are commonly used to test the performance of different FD frameworks: CIFAR-10\cite{krizhevsky2009learning}, CINIC-10\cite{darlow2018cinic}, Fashion-MNIST\cite{xiao2017fashion}.
\iffalse
\begin{itemize} [leftmargin=*]
    \item \textbf{CIFAR-10}\cite{krizhevsky2009learning}.
    The CIFAR-10 dataset is a widely used image recognition dataset that consists of 60,000 color images of 32$\times$32 pixels, divided into 10 classes. The dataset is split into 50,000 training data samples and 10,000 test data samples.
    \item \textbf{CINIC-10}\cite{darlow2018cinic}.
    CINIC-10 is an image classification dataset that combines samples from CIFAR-10 and ImageNet \cite{deng2009imagenet}. It comprises 270,000 images distributed across 10 classes, with each image having dimensions of 32×32×3 pixels.
    \item \textbf{Fashion-MNIST}\cite{xiao2017fashion}.
    Fashion-MNIST is a dataset containing 70,000 grayscale images of 28x28 pixels, categorized into 10 fashion items such as T-shirts, trousers, and sneakers. 
    It is structured with 60,000 training images and 10,000 test images.
\end{itemize}
\fi
Additionally, for the completeness of the experiments, we also use a tabular dataset:Purchase\cite{acquire-valued-shoppers-challenge}.
In our experiments, we partition each dataset into a 4:1 ratio for clients' training sets $D_{train}$ and the public dataset $D_{pub}$.
To align with the previous FD frameworks~\cite{yang2022fd, li2019fedmd}, we also configure a scenario where there is a data distribution discrepancy between the public dataset and the clients' private datasets.
In this scenario, we use CIFAR-10 for client training and CIFAR-100 as the public dataset to simulate distribution shifts.
% the clients' training dataset $D_{train}$ is CIFAR-10, while the public dataset $D_{pub}$ is CIFAR-100.
Table \ref{tab:datasets_division} details the specific partitioning of datasets and the number of classes used in our experiments.
In our MIA experiments, to ensure adequate private data for each client, we select an equal number of samples from the test dataset to serve as non-members. 
\begin{table}[h]
    \caption{Datasets division.}
    \centering
    \resizebox{0.9\linewidth}{!}{%
    \begin{tabular}{c|c|c|c|c}
        \toprule
        Datasets&number of classes&$D_{train}$ & $D_{pub}$ & $D_{test}$\\
        \midrule
        CIFAR-10 & 10 & 40000 & 10000 & 10000 \\
        CIFAR-10/CIFAR-100 & 10 & 40000 & 10000 & 10000 \\
        CINIC-10 & 10 & 72000 & 18000 & 90000  \\
        Fashion-MNIST & 10 & 48000 & 12000 & 10000 \\
        Purchase & 10 & 21589 & 5397 & 11565 \\
        \bottomrule
    \end{tabular}%
    }
    \label{tab:datasets_division}
\end{table}

In our experiments, we create 10 clients that participate in the collaborative training. 
To partition the training dataset $D_{train}$ into private datasets $D_n$ for 10 clients, we use Dirichlet distribution $Dir(\alpha)$ with $\alpha$ values of $0.1$, $1$ and $10$ to generate non-IID data distribution across all the clients. 
The smaller the value of $\alpha$, the more imbalanced the label distribution of $D_n$ is. 

\subsubsection{Models.}
For different datasets, the clients in PDA-FD use different model architectures. 
When the private dataset is CIFAR-10, the clients train the ResNet-18 models~\cite{he2016deep}. 
For CINIC-10, the clients train the MobileNetV2 models~\cite{sandler2018mobilenetv2}.
For Fashion-MNIST datasets, the clients' local models employ a CNN architecture with four convolutional layers. 
When training with the Purchase dataset, the clients train MLP models consisting of three fully connected layers.
As the heterogeneity in client model architectures does not affect our attack methodology\cite{carlini2022membership}, we adopted identical model structures across all clients.

\subsubsection{LDIA Metrics.}
To evaluate the effectiveness of the proposed LDIA, we adopt the same metrics employed in previous LDIA research~\cite{gu2023ldia, salem2020updates}. 
In the equations for calculating these metrics, $\hat{p}$ denotes the inferred label distribution, $p$ denotes the ground truth label distribution, $m$ represents a specific label, and $M$ denotes the number of labels:
\begin{itemize} [leftmargin=*]
    \item \textbf{Kullback-Leibler divergence.}
    Kullback-Leibler divergence(KL divergence) between the ground truth label distribution and the inferred label distribution can be calculated using the following equation: 
    \begin{equation}
        Dis_{KL-div}(\hat{p}, p) = \sum_{m=1}^M \hat{p}_m log(\frac{\hat{p}_m}{p_m})
    \label{eq:KL_metrics}
    \end{equation}
    This metric represents the similarity between the inferred label distribution and the ground truth label distribution. 
    A smaller KL divergence indicates that the two distributions are more closely aligned.
    \item \textbf{Chebyshev distance.}
    The Chebyshev distance represents the maximum error between the inferred label distribution and the ground truth label distribution for each target client in an LDIA:
    \begin{equation}
        Dis_{Cheb}(\hat{p}, p) = max_m\mid \hat{p}_m - p_m \mid
    \label{eq:Cheb_metrics}
    \end{equation}
    A smaller Chebyshev distance indicates a higher reliability of the LDIA results.
    \item \textbf{Mean $l1$-distance.}
    The mean $l1$-distance represents the potential error between the inferred label distribution and the ground truth label distribution across all classes:
    \begin{equation}
        Dis_{mean-l1}(\hat{p}, p) = \frac{1}{M}\sum_{m=1}^M \mid \hat{p}_m - p_m \mid
    \label{eq:meanl1_metrics}
    \end{equation}
    A smaller mean $l1$-distance indicates a higher average accuracy of the LDIA results.
\end{itemize}

\subsubsection{MIA Metrics.}
Same as previous efforts on MIA~\cite{carlini2022membership,shokri2017membership,watson2021importance} , we employ the following metrics to evaluate the effectiveness of the proposed MIAs:
\begin{itemize} [leftmargin=*]
    \item \textbf{TPR at low FPR.} Carlini \etal~\cite{carlini2022membership} suggested using TPR at low FPR to measure MIA. 
    A higher TPR in the low FPR region indicates greater precision of the MIA, which also implies that the attack is more reliable.
    \item \textbf{Balance accuracy and AUC.} These two metrics assess the overall performance of MIA. 
    Balanced accuracy measures the attacker's ability to correctly predict true positives and true negatives across all members and non-members. 
    AUC quantifies the area beneath the ROC curve of the MIA results. 
    It offers a comprehensive measure of the attack's discriminative power across various classification thresholds.
\end{itemize}

\subsubsection{PDA-FD Frameworks.}
To comprehensively evaluate the privacy leakage in the PDA-FD setting, we evaluate three different PDA-FD frameworks in our experiments: FedMD~\cite{li2019fedmd}, DS-FL~\cite{itahara2021distillation}, and Cronus~\cite{chang2019cronus}. 
Each of these FD frameworks employs a distinct approach to enhance the robustness of the FD algorithms.
As mentioned in Section~\ref{sec:background_fd}, they behave differently during the local updates phase and use different aggregation algorithms during the communication phase.
% specifically differing in their local updates phase and in the aggregation algorithms used during the communication phase, which is detailed in Section 2.1.
% In our experiments, each PDA-FD framework involved 10 clients participating in collaborative training. 
In order to optimize the performance of all the PDA-FD frameworks, we should carefully select the number of training epochs in each round.
Following the approach suggested by Li \etal in FedMD~\cite{li2019fedmd}, we initially train the local models to convergence on the private datasets before transitioning to the shorter update cycles during the distillation phase.
Specifically, in the first round of local updates, each client performs 20 epochs of training. 
In the subsequent rounds, this is reduced to 5 epochs of training for each client. 
The knowledge distillation phase consists of 10 epochs of training for each client.
To reduce communication cost, we randomly select 5000 data samples from the public dataset during each communication phase for the CIFAR-10, CIFAR10/CIFAR100, Fashion-MNIST and Purchase datasets, aligned with previous PDA-FD studies~\cite{li2019fedmd}.
The CINIC-10 dataset, given its difficulty level, uses a set of 10000 randomly selected samples to ensure adequate knowledge transfer.
Table \ref{tab:FD_performance} presents the performance of the three PDA-FD frameworks across various Dirichlet distributions and four distinct datasets.
\begin{table}[h]
\caption{Performance of the PDA-FD Frameworks.}
\centering
\scriptsize
\resizebox{0.9\linewidth}{!}{%
    \begin{tabular}{lcccccc}
    \toprule
    \multirow{2}{*}{Datasets} & 
    \multirow{2}{*}{Setting} & 
    \multirow{2}{*}{Local accuracy} & 
    \multicolumn{3}{c}{Federated accuracy} \\ 
    \cmidrule(lr){4-6}
     & & &
     FedMD & DS-FL & Cronus \\
     \midrule
     CIFAR10 
     & $\alpha$=10  & 54.59\% 
     & 76.61\%  & 71.31\% & 70.81\% \\
     & $\alpha$=1   & 46.06\% 
     & 75.38\% & 68.45\% & 67.90\% \\
     & $\alpha$=0.1  & 22.75\%
     & 60.55\% & 45.01\% & 42.24\% \\
    \midrule
     CIFAR10 
     & $\alpha$=10   & 53.24\% 
     & 72.34\% & 69.54\% & 68.42\% \\
     /CIFAR100 & $\alpha$=1 & 45.49\% & 
     68.41\% & 65.90\% & 64.26\% \\
     & $\alpha$=0.1 & 23.31\%  & 
     49.89\% & 43.55\% & 43.43\% \\
    \midrule
     CINIC10 
     & $\alpha$=10& 39.31\%  
     &67.72\%  &64.21\%  & 62.92\% \\
     & $\alpha$=1& 33.37\%   
     &62.91\%  &57.02\%  & 56.49\% \\
     & $\alpha$=0.1& 20.45\% 
     &41.02\%  &38.48\%  &37.89\%  \\
     \midrule
     Fashion & $\alpha$=10    & 78.80\% 
     & 88.68\% & 87.96\% & 87.62\% \\
     -MNIST  & $\alpha$=1   & 69.35\%
     & 87.98\% & 85.25\% & 84.98\% \\
     & $\alpha$=0.1 & 19.58\% 
     & 80.62\% & 52.45\% & 56.34\% \\
    \midrule
     Purchase 
     & $\alpha$=10  & 82.56\% 
     &  94.58\% & 88.35\% & 88.62\%\\
     & $\alpha$=1 & 72.73\% 
     & 94.01\% & 86.83\% & 89.65\% \\
     & $\alpha$=0.1  & 52.18\% 
     & 91.80\% & 72.55\% & 67.34\% \\
     \bottomrule
    \end{tabular}%
    }
\label{tab:FD_performance}
\end{table}

\subsection{Experiment Results of LDIA}
In the LDIA experiments, the PDA-FD server infers the label distribution of all clients' private datasets during the communication phase in each round.
To ensure robustness and account for potential fluctuations, we compute the final LDIA result for each client by averaging the server's inferred label distribution over 10 collaborative training rounds.
The overall effectiveness of the attack is evaluated by averaging these final results across all clients.
% To comprehensively reflect the results of the server's LDIA attack, we report the average of the server's LDIA results across all clients. 
To provide a meaningful benchmark for our LDIA method, we establish a baseline comparison, denoted as ``Random'', following the same approach of previous LDIA research~\cite{gu2023ldia, salem2020updates}. 
This baseline employs randomly generated label distributions for each client's private dataset, serving as a lower bound for attack performance.
Note that for a given dataset and Dirichlet distribution parameter, the private dataset of each client remains constant across different PDA-FD frameworks. 
Therefore, within the same dataset and Dirichlet distribution, there is only one set of Random LDIA results.

\textbf{Main Result.} Table \ref{tab:ldia_main_result} presents the performance of the proposed LDIA on five different datasets across three PDA-FD frameworks. 
The results demonstrate the server's capability to launch effective LDIA against clients across these datasets, significantly outperforming the random guess baseline on all three key metrics.
For instance, for the DS-FL framework on the CIFAR-10 dataset with $\alpha$=1, our proposed LDIA achieves an average Chebyshev distance of 0.10, an average mean l1-distance of 0.03, and an average KL-divergence of 0.10 across all clients. 
In contrast, the random guess baseline yields substantially higher values: 0.20, 0.08, and 0.66 for the respective metrics. 
This significant improvement underscores the efficacy of our LDIA in accurately inferring clients' label distributions.
Figure \ref{fig:ldia_main_result} provides a visual representation of the LDIA results for the DS-FL server on the CIFAR-10 dataset, offering a clearer illustration of the experiment results.
We can see from the figure that for the labels whose inferred proportions deviate from the ground truth values, the relative rankings of label frequencies are consistently preserved.
This observation highlights the robustness of the proposed LDIA in capturing the essential structure of label distributions.

\begin{table*}[h]
\caption{Performance of the server in conducting LDIA within the different PDA-FD Frameworks.}
\centering
\scriptsize
\resizebox{0.9\linewidth}{!}{%
\begin{tabular}{lccccccccccccc}
\toprule
\multirow{2}{*}{Datasets} & 
\multirow{2}{*}{Setting} & 
\multicolumn{4}{c}{KL divergence} & 
\multicolumn{4}{c}{Chebyshev distance} & 
\multicolumn{4}{c}{Mean $l1$-distance} \\ 
\cmidrule(lr){3-6}
\cmidrule(lr){7-10}
\cmidrule(lr){11-14}
 & & 
 FedMD & DS-FL & Cronus & Random &
 FedMD & DS-FL & Cronus & Random &
 FedMD & DS-FL & Cronus & Random \\ 
 \midrule
 CIFAR10 
 & $\alpha$=10 
 & 0.02 & 0.01 & 0.01 & 0.42
 & 0.04 & 0.03 & 0.02 & 0.13
 & 0.01 & 0.01 & 0.01 & 0.05\\
 & $\alpha$=1   
 & 0.17 & 0.10 & 0.08 & 0.66
 & 0.14 & 0.11 & 0.10 & 0.20
 & 0.04 & 0.03 & 0.03 & 0.08\\
 & $\alpha$=0.1 
 & 0.15 & 0.11 & 0.07 & 1.93
 & 0.18 & 0.16 & 0.14 & 0.57
 & 0.04 & 0.03 & 0.03 & 0.15\\
\midrule
 CIFAR10 
 & $\alpha$=10  
 & 0.07 & 0.05 & 0.06 & 0.40
 & 0.07 & 0.06 & 0.06 & 0.12
 & 0.02 & 0.02 & 0.02 & 0.05\\
 /CIFAR100 & $\alpha$=1 
 &0.11 & 0.10 & 0.10 & 0.59
 &0.10 & 0.11 & 0.11 & 0.22
 &0.03 & 0.03 & 0.03 & 0.08\\
 & $\alpha$=0.1 
 &0.11 & 0.10 & 0.08 & 1.59
 &0.15 & 0.13 & 0.14 & 0.51
 &0.03 & 0.03 & 0.03 & 0.14\\
\midrule
 CINIC10
 & $\alpha$=10
 &0.01 & 0.01 & 0.01 &0.64  
 &0.02 & 0.02 & 0.04 &0.15
 &0.01 & 0.01 & 0.01 &0.07\\
 & $\alpha$=1 
 &0.06 & 0.05 & 0.05 &0.57  
 &0.11 & 0.11 & 0.10 &0.21
 &0.02 & 0.02 & 0.02 &0.08\\
 & $\alpha$=0.1 
 &0.09 & 0.08 & 0.01 &1.99  
 &0.14 & 0.14 & 0.04 &0.56
 &0.03 & 0.03 & 0.01 &0.15\\
 \midrule
 Fashion & $\alpha$=10  
 & 0.03 & 0.02 & 0.02 & 0.32
 & 0.04 & 0.04 & 0.04 & 0.12
 & 0.02 & 0.01 & 0.01 & 0.06\\
 -MNIST  & $\alpha$=1  
 & 0.14 & 0.12 & 0.12 & 0.55
 & 0.10 & 0.09 & 0.09 & 0.15
 & 0.04 & 0.03 & 0.03 & 0.06\\
 & $\alpha$=0.1 
 & 0.21 & 0.05 & 0.06 & 1.54 
 & 0.20 & 0.09 & 0.11 & 0.45
 & 0.04 & 0.02 & 0.02 & 0.13\\
\midrule
 Purchase 
 & $\alpha$=10 
 &  0.08 & 0.03 & 0.03 & 0.47
 &  0.06 & 0.05 & 0.05 & 0.13
 &  0.02 & 0.02 & 0.02 & 0.06\\
 & $\alpha$=1 
 & 0.27 & 0.14 & 0.15 & 0.68
 & 0.13 & 0.10 & 0.10 & 0.18
 & 0.06 & 0.04 & 0.04 & 0.09\\
 & $\alpha$=0.1 
 & 0.64 & 0.14 & 0.14 & 2.11
 & 0.32 & 0.15 & 0.17 & 0.52
 & 0.10 & 0.03 & 0.03 & 0.15\\
 \bottomrule
\end{tabular}%
}
\label{tab:ldia_main_result}
\end{table*}

\begin{figure}[htbp]
    \centering
    \begin{subfigure}[b]{\linewidth}
        \centering
        \includegraphics[width=1\linewidth]{figures/dsfl_result_alpha_10.png}
        \caption{$\alpha=10$}
        \label{fig:subfig1}
    \end{subfigure}

    \begin{subfigure}[b]{\linewidth}
        \centering
        \includegraphics[width=1\linewidth]{figures/dsfl_result_alpha_1.png}
        \caption{$\alpha=1$}
        \label{fig:subfig2}
    \end{subfigure}
    
    \begin{subfigure}[b]{\linewidth}
        \centering
        \includegraphics[width=1\linewidth]{figures/dsfl_result_alpha_0.1.png}
        \caption{$\alpha=0.1$}
        \label{fig:subfig3}
    \end{subfigure}

    \caption{The LDIA performance of the DS-FL server on the CIFAR-10 dataset, under three distinct Dirichlet distributions. The images depict the best (left), median (center), and worst (right) LDIA results across all client models.}
    \label{fig:ldia_main_result}
\end{figure}

\textbf{Different Data Distributions.} 
Our experiments reveal a notable relationship between the effectiveness of the proposed LDIA and the uniformity of clients' label distributions.
Specifically, the LDIA demonstrates lower KL-divergence, Chebyshev distance, and mean $l1$-distance as the clients' label distributions become more uniform.
This trend is clearly illustrated in our experiments using the CIFAR-10 dataset within the DSFL framework. 
As $\alpha$ increases, indicating a more uniform label distribution across clients, the LDIA achieves better performance.
% When the label distribution differs, the server's LDIA becomes more effective as the clients' label distribution becomes more uniform. 
% Experiments on the CIFAR-10 dataset in the DSFL framework show that with a higher $\alpha$ (more uniform distribution), the LDIA results in lower KL-divergence, Chebyshev distance, and mean $l1$-distance. 
Conversely, as $\alpha$ decreases, indicating a more skewed distribution, we see an increase in the three key metrics.
Nonetheless, the attack remains effective despite the reduced accuracy.

\textbf{Different PDA-FD Frameworks.} 
Our evaluations also reveal significant differences in the vulnerability of various PDA-FD frameworks to LDIA.
Compared to FedMD, the server can achieve more effective LDIA on clients within the DS-FL and Cronus frameworks.
This can be attributed to the unique training approach employed by FedmD during its first collaborative training round.
In FedMD, clients first train their local models on the public dataset before transitioning to their private dataset.
% This occurs because during the first collaborative training round's local updates phase in FedMD, all clients train the local model on public datasets before training on their private dataset. 
This process serves as a form of regularization, thus mitigating overfitting to private datasets and, consequently, reducing vulnerability to LDIA.
However, this effect is diminished when private and public datasets differ significantly or when the public dataset is unlabeled.
In these cases, FedMD's LDIA vulnerability becomes comparable to that of the other two PDA-FD frameworks, as evidenced by the results in Table \ref{tab:ldia_main_result} for the CIFAR-10/CIFAR-100 datasets.
The similarity in LDIA vulnerability arises from the data distribution shift between public and private datasets, causing clients' local models to reduce memorization of the public dataset after converging on their private datasets.
As a result, the clients' local models end up overfitting to their private datasets to a similar degree across all frameworks.
% As a result, this does not reduce the overfitting level of the clients' local models to their private datasets.

\textbf{Different Collaborative Training Rounds.} 
To evalutate the temporal dynamics of LDIA, we analyze its performance across multiple collaborative training rounds in various PDA-FD frameworks, as illustrated in Figure \ref{fig:ldia_result_diff_rounds}.
\begin{figure}[h]
  \centering
  \begin{subfigure}[b]{0.32\linewidth}
    \includegraphics[width=\linewidth]{figures/ldia_01_diff_rounds.png}
    \caption{$\alpha=0.1$}
    \label{fig:ldia_result_diff_rounds_0.1}
  \end{subfigure}
  \hfill
  \begin{subfigure}[b]{0.32\linewidth}
    \includegraphics[width=\linewidth]{figures/ldia_1_diff_rounds.png}
    \caption{$\alpha=1$}
    \label{fig:ldia_result_diff_rounds_1}
  \end{subfigure}
  \hfill
  \begin{subfigure}[b]{0.32\linewidth}
    \includegraphics[width=\linewidth]{figures/ldia_10_diff_rounds.png}
    \caption{$\alpha=10$}
    \label{fig:ldia_result_diff_rounds_10}
  \end{subfigure}
    \caption{Chebyshev distance results of LDIA performed by the PDA-FD server on the CIFAR-10 dataset, shown for each collaborative training round.}
    \label{fig:ldia_result_diff_rounds}
\end{figure}
% illustrates the server's LDIA performance on clients within the PDA-FD framework across multiple rounds. 
We aggregate the server's attack results across all clients for each round, to represent the overall LDIA performance over time.
The results reveal that the server successfully executes LDIA on clients in every round. 
Notably, the LDIA performance remains relatively stable as the number of collaborative training rounds increases, showing neither significant improvement nor decline.
This consistency can be attributed to the local updates phase preceding communication in each collaborative training round within PDA-FD frameworks. 
While this phase enhances knowledge transfer among clients, it simultaneously increases the degree of overfitting of each client's local model to their private data. 
This dual effect contributes to the observed stability in LDIA performance over multiple rounds.
Despite this stability, we recommend using the averaged LDIA results over multiple rounds as the final outcome to enhance the robustness of the results. 
Such an approach mitigates potential fluctuations and provides a more reliable measure of the server's LDIA capabilities.




\subsection{Experiment Results of MIA}
\label{sec:MIA_eval}
In our MIA experiments, we evaluate the effectiveness of the proposed attack against 10 clients during the communication phase across three PDA-FD frameworks. 
These experiments use three different Dirichlet distributions and four distinct datasets. 
This comprehensive setup allows for a thorough assessment of MIA vulnerability under various data distribution scenarios and PDA-FD frameworks.
Previous FD MIA studies \cite{wang2024graddiff, liu2023mia} assume that attackers have access to shadow datasets matching the target model’s training data distribution. However, we find these assumptions too restrictive and unrepresentative of real-world scenarios. Therefore, our attack methods do not rely on such assumptions, so we do not use these studies as baselines for comparison.

\textbf{Main Result.} We first evaluate co-op LiRA. 
Given that co-op LiRA is applicable in scenarios where clients' label distributions are similar, we conduct experiments across different datasets using a Dirichlet distribution with $\alpha=10$. This parameter setting ensures a more uniform distribution of labels across clients, aligning with co-op LiRA's operational scenario. 
Table \ref{tab:efficient_mia_main_result} presents the performance of co-op LiRA during the communication phase of the first
collaborative training round.
Our findings reveal that when the server attacks a specific client, utilizing only the other 9 clients' models as the reference models yields remarkably effective attack results.
This observation underscores the high efficiency and practicality of co-op LiRA, demonstrating its capability to achieve effective MIA without the need to train any additional reference models.

\begin{table*}[h]
\caption{Performance of the server in conducting co-op LiRA within the different PDA-FD Frameworks.}
\centering
\scriptsize
\resizebox{0.9\linewidth}{!}{%
\begin{tabular}{lccccccccccccc}
\toprule
\multirow{2}{*}{Datasets} &
\multicolumn{3}{c}{TPR at 1\% FPR} & 
\multicolumn{3}{c}{TPR at 0.1\% FPR} & 
\multicolumn{3}{c}{AUC} & 
\multicolumn{3}{c}{Balance Accuracy} \\ 
\cmidrule(lr){2-4}
\cmidrule(lr){5-7}
\cmidrule(lr){8-10}
\cmidrule(lr){11-13}
 &
 FedMD & DS-FL & Cronus & 
 FedMD & DS-FL & Cronus & 
 FedMD & DS-FL & Cronus & 
 FedMD & DS-FL & Cronus \\ 
 \midrule
 CIFAR10 
 & 22.15\% & 21.96\% &20.35\% 
 & 6.67\%  & 6.39\%  &5.76\% 
 & 0.819   & 0.850   &0.840 
 & 74.07\% & 77.11\% &76.23\% \\
\midrule
 CINIC10 
 & 10.97\% & 11.23\% & 10.99\%
 & 1.78\%  & 1.80\%  & 1.79\%
 & 0.794   & 0.811   & 0.815
 & 71.55\% & 73.89\% & 74.28\%\\
 \midrule
 Fashion-MNIST
 &3.24\% &1.80\% &1.64\%
 &1.09\% &0.31\% &0.33\%
 &0.582  &0.533  &0.531
 &55.89\%&55.38\%&54.28\% \\
\midrule
 Purchase 
 & 5.85\%  & 4.08\%  & 4.45\% 
 & 1.67\%  & 0.71\%  & 0.61\%
 & 0.616   & 0.712   & 0.709
 & 58.41\% & 66.46\% & 66.10\%\\
 \bottomrule
\end{tabular}%
}
\label{tab:efficient_mia_main_result}
\end{table*}

We subsequently evaluate the performance of distillation-based LiRA. 
For each client, we distill 32 reference models. 
The distillation dataset for each reference model consists of a randomly sampled 80\% subset of the public dataset used in the communication phase. 
This approach ensures a diverse set of reference models.
The model architecture of reference model is same as that of the target client model.
Table \ref{tab:distillation_mia_main_result} presents the average results of the server's MIA against all clients during the communication phase during the first collaborative training round. 
The results reveal that the server can launch highly effective MIA against the clients in the non-IID scenarios.
Figure \ref{fig:distillation_mia_main_result} presents the results of MIA experiments conducted in the DS-FL framework using the CIFAR-10 dataset, across various Dirichlet distributions.
Notably, we observe that even when the private dataset (CIFAR-10) and public dataset (CIFAR-100) have significantly different distributions, the server can still successfully launch MIA against clients by leveraging the distilled reference models from the public dataset. 
This finding underscores the efficacy of distillation-based LiRA in the PDA-FD frameworks, demonstrating its robustness to dataset disparities between the public and private data.

\begin{table*}[h]
\caption{Performance of the server in conducting distillation-based LiRA within the different PDA-FD Frameworks.}
\centering
\scriptsize
\resizebox{0.9\linewidth}{!}{%
\begin{tabular}{lccccccccccccc}
\toprule
\multirow{2}{*}{Datasets} & 
\multirow{2}{*}{Setting} & 
\multicolumn{3}{c}{TPR at 1\% FPR} & 
\multicolumn{3}{c}{TPR at 0.1\% FPR} & 
\multicolumn{3}{c}{AUC} & 
\multicolumn{3}{c}{Balance Accuracy} \\ 
\cmidrule(lr){3-5}
\cmidrule(lr){6-8}
\cmidrule(lr){9-11}
\cmidrule(lr){12-14}
 & & 
 FedMD & DS-FL & Cronus & 
 FedMD & DS-FL & Cronus & 
 FedMD & DS-FL & Cronus & 
 FedMD & DS-FL & Cronus \\ 
 \midrule
 CIFAR10 
 & $\alpha$=10 
 & 20.94\% &35.76\% & 32.69\%
 & 10.42\% &19.42\% & 14.40\%
 & 0.764   &0.902   & 0.867
 & 69.68\% &82.01\% & 78.51\%\\
 & $\alpha$=1   
 & 17.62\% &29.28\% & 23.83\%
 & 8.11\%  &11.29\% & 5.77\%
 & 0.730   &0.839   & 0.804
 & 66.93\% &76.20\% & 72.94\%\\
 & $\alpha$=0.1 
 & 9.74\%  &12.89\% & 7.20\%
 & 2.09\%  &3.74\%  & 1.13\%
 & 0.618   &0.680   & 0.639
 & 58.91\% &63.49\% & 60.64\%\\
\midrule
 CIFAR10 
 & $\alpha$=10  
 & 11.20\% &34.61\% & 28.28\%
 & 1.48\%  &10.92\% & 5.49\%
 & 0.804   &0.901   & 0.891
 & 72.74\% &81.93\% & 80.97\%\\
 /CIFAR100 & $\alpha$=1 
 & 9.47\%  &25.94\% & 19.32\%
 & 0.93\%  &2.40\%  & 1.95\%
 & 0.758   &0.844   & 0.821
 & 68.92\% &76.61\% & 73.68\%\\
 & $\alpha$=0.1 
 & 7.79\%  &11.34\% & 5.61\%
 & 0.74\%  &0.51\%  & 0.88\%
 & 0.627   &0.686   & 0.652
 & 59.38\% &63.67\% & 61.75\%\\
\midrule
 CINIC10 
 & $\alpha$=10
 &13.83\% & 17.32\% & 15.91\%
 &3.12\%  & 4.15\%  & 3.85\%
 &0.741   & 0.855   & 0.834
 &71.42\% & 77.57\% & 75.26\% \\
 & $\alpha$=1 
 &10.96\% & 13.94\% & 12.07\%
 &2.37\%  & 3.59\%  & 3.01\%
 &0.704   & 0.781   & 0.757
 &68.93\% & 70.89\% & 69.21\%\\
 & $\alpha$=0.1 
 &5.91\%  & 6.81\% & 6.46\%
 &1.25\%  & 1.94\% & 1.72\%
 &0.649   & 0.652  & 0.661
 &60.27\% & 61.18\%& 62.59\% \\
 \midrule
 Fashion & $\alpha$=10  
 &3.07\% &1.85\%  & 1.73\%
 &0.91\% &0.35\%  & 0.21\%
 &0.588  &0.539   & 0.528
 &55.94\%&59.85\% & 55.43\% \\
 -MNIST  & $\alpha$=1  
 &3.30\% &1.71\% &  1.62\%
 &0.69\% &0.26\% &  0.25\%
 &0.583  &0.536  &  0.522
 &56.47\%&59.51\%&  54.31\%\\
 & $\alpha$=0.1 
 &1.88\% &1.39\% &  1.21\%
 &0.54\% &0.19\% &  0.23\%
 &0.538  &0.523  &  0.519
 &52.71\%&52.35\%&  51.32\%\\
\midrule
 Purchase 
 & $\alpha$=10 
 & 1.94\%   &5.91\%  & 2.43\%
 & 0.77\%   &1.31\%  & 1.93\%
 & 0.539    &0.665   & 0.706
 & 53.34\%  &62.69\% & 65.62\%\\
 & $\alpha$=1 
 & 1.98\%   &5.64\%  & 2.69\%
 & 0.83\%   &1.69\%  & 0.68\%
 & 0.534    &0.654   & 0.653
 & 53.31\%  &61.49\% & 62.24\%\\
 & $\alpha$=0.1 
 & 1.41\%  &5.04\% &  3.04\%
 & 0.42\%  &1.21\% &  1.19\%
 & 0.507   &0.591  &  0.588
 & 52.24\% &57.93\% & 57.69\%\\
 \bottomrule
\end{tabular}%
}
\label{tab:distillation_mia_main_result}
\end{table*}

\begin{figure}[h]
  \centering
  \begin{subfigure}[b]{0.325\linewidth}
    \includegraphics[width=\linewidth]{figures/mia_dsfl_01.png}
    \caption{$\alpha=0.1$}
    \label{fig:distill_lira_overall}
  \end{subfigure}
  \hfill
  \begin{subfigure}[b]{0.325\linewidth}
    \includegraphics[width=\linewidth]{figures/mia_dsfl_1.png}
    \caption{$\alpha=1$}
    \label{fig:distill_lira_high}
  \end{subfigure}
  \hfill
  \begin{subfigure}[b]{0.325\linewidth}
    \includegraphics[width=\linewidth]{figures/mia_dsfl_10.png}
    \caption{$\alpha=10$}
    \label{fig:distill_lira_high}
  \end{subfigure}
    \caption{Distillation-based LiRA performance of the DS-FL server on the CIFAR-10 dataset, presented as log-scale ROC curves under three distinct Dirichlet distributions.}
    \label{fig:distillation_mia_main_result}
\end{figure}

\textbf{Different Data Distributions.}
We observe that the effectiveness of the proposed distillation-based LiRA attacks on clients decreases as the clients' label distributions become more imbalanced. 
This phenomenon can be explained by the fact that local models trained on datasets with highly skewed label distributions tend to produce disproportionately high posterior probabilities for the dominant labels.
This bias also affects the non-member samples that come from the same over-represented classes.
The core principle of LiRA relies on difficulty calibration, which becomes less effective in imbalanced scenarios.
As a result of this, the attacker's capability to discriminate between members and non-members is compromised.
This leads to a overall degradation in the performance of distillation-based LiRA on clients with highly imbalanced label distributions.
% We hypothesize this is due to local models trained on datasets with disproportionate label representation producing high posterior probabilities for dominant labels, even for non-member samples. This reduces the efficacy of difficulty calibration in MIA, as the distinction between member and non-member samples becomes less pronounced, ultimately degrading distillation-based LiRA performance on clients with highly imbalanced label distributions.

\textbf{Different PDA-FD Frameworks.}
The effectiveness of co-op LiRA remains relatively consistent across FedMD, DS-FL, and Cronus. 
However, for distillation-based LiRA, the effectiveness of MIAs ranks as follows: DS-FL achieves the best performance, followed by Cronus, with FedMD showing the least effectiveness.
In Cronus, clients upload softmax-processed posterior probability vectors rather than raw logits for public data during the communication phase. 
Compared to logits, the use of posterior probability vectors diminishes the server's ability to distill reference models that closely mimic the target model's performance.
Consequently, this limitation leads to a reduction in the effectiveness of MIA.
In the FedMD framework, clients train on public data before their private datasets during the local updates phase in their first collaborative training round. 
This process leads to clients training on all the target samples strategically selected by the server, regardless of their membership status.
While experiments demonstrate that subsequent training on private datasets reduces clients' memorization of public data, this initial exposure still impacts the server's MIA results.
However, when the public dataset is unlabeled, clients cannot train on it during the first collaborative training round. In this scenario, the server's MIA performance on clients remains unaffected.

\textbf{Different Collaborative Training Rounds.}
Figure \ref{fig:mia_diff_round} illustrates the performance of the proposed MIAs in different PDA-FD frameworks across multiple collaborative training rounds on the CIFAR-10 dataset. 
To evaluate each MIA approach in its intended scenario, for co-op LiRA, we employ a Dirichlet distribution parameter $\alpha$=10.
While for distillation-based LiRA, we use $\alpha$=1. 
We use the average TPR at 1\% FPR of MIA across all clients as the metric to quantify performance.
The performance of distillation-based LiRA declines with more collaborative training rounds but eventually stabilizes.
We attribute this to the FD process, which gradually reduces the degree of overfitting of local models to their private data. 
While local updates maintain some private data memorization, the performance gap between local and distilled reference models for private data narrows over time.
The performance of co-op LiRA remains relatively stable as the collaborative training rounds increase. 
We attribute this stability to two key factors. 
First, the non-target clients in co-op LiRA cannot effectively learn membership information from the server-aggregated logits during the communication phase. 
Second, while the FD process reduces the overfitting level of local models to their private data, the local updates phase, where each client trains exclusively on its private dataset, maintains a consistent performance gap between local and reference models for their respective private data. 
This balance between reduced overfitting and continued exclusive training on private data likely contributes to the stability of co-op LiRA's performance across collaborative rounds.
\begin{figure}[h]
\centering
  \resizebox{0.9\linewidth}{!}{%
  \begin{subfigure}[b]{0.49\linewidth}
    \includegraphics[width=\linewidth]{figures/mia_efficient_diff_rounds.png}
    \caption{Co-op LiRA}
    \label{fig:mia_diff_round_1}
  \end{subfigure}
  \hfill
  \begin{subfigure}[b]{0.49\linewidth}
    \includegraphics[width=\linewidth]{figures/mia_distillation_diff_rounds.png}
    \caption{Distillation-based LiRA}
    \label{fig:mia_diff_round_2}
  \end{subfigure}
  }
    \caption{MIA performance across training rounds.}
    \label{fig:mia_diff_round}
\end{figure}

\section{Ablation Study}
\label{sec:ablation}
\subsection{Public Dataset Size}
The server in PDA-FD can control communication overhead by adjusting the size of the public dataset used in each collaborative training round.
We investigate its impact on the performance of LDIA and MIA. 
As public data does not affect co-op LiRA, our evaluations of MIA mainly focus on distillation-based LiRA. 
In our experiments, we use the DS-FL framework on the CIFAR-10 dataset with $\alpha=1$.
\begin{table}[h]
    \caption{Impact of Public Data Quantity on Label Distribution and Membership Information Leakage in PDA-FD.}
    \centering
    \scriptsize
    \resizebox{0.9\linewidth}{!}{%
    \begin{tabular}{c|c|c}
        \toprule
        Datasets size&  MIA (TPR at 1\%FPR) & LDIA (KL divergence)\\
        \midrule
        5000 & 29.28\% &  0.10  \\
        7500 & 31.84\% &  0.09 \\
        10000& 32.01\% &  0.07 \\
        \bottomrule
    \end{tabular}%
    }
    \label{tab:public_dataset_size}
\end{table}
Table \ref{tab:public_dataset_size} illustrates the degree of label distribution information and membership information leakage from clients when the quantity of the public data samples is set to 5000, 7500, and 10000, respectively.
The results indicate that larger public datasets contribute to increased privacy leakage risks for clients. 
We attribute this trend to two factors. For distillation-based LiRA, a larger public dataset provides a more extensive distillation dataset, enabling the attacker to obtain more robust reference models.
In the case of LDIA, a larger public dataset serving as the inference dataset allows the attacker to mitigate the impact of outliers or atypical data, thereby improving attack accuracy.


\subsection{Number of Epochs in Local Updates Phase}
Prior to the communication phase, clients train their local models on their private datasets during the local updates phase. 
This process enhances the local model's memorization of private data, facilitating knowledge transfer between clients but also potentially increasing privacy leakage. 
We measure the impact of the number of training epochs in the local updates phase on the leakage of label information and membership information from clients.
\begin{table}[h]
    \caption{Impact of Number of Training Epochs on Label Distribution and Membership Information Leakage in PDA-FD.}
    \centering
    \scriptsize
    \resizebox{0.9\linewidth}{!}{%
    \begin{tabular}{c|c|c}
        \toprule
        Number of Epochs&  MIA (TPR at 1\%FPR) & LDIA (KL divergence)\\
        \midrule
        2 & 8.10\% &  0.15  \\
        4 & 14.64\% &  0.10 \\
        6 & 15.43\% &  0.09 \\
        \bottomrule
    \end{tabular}% 
    }
    \label{tab:epochs_local_updates}
\end{table}
As shown in Table \ref{tab:epochs_local_updates}, there is an increase in label distribution and membership information leakage from clients in DS-FL as the number of the local update training rounds increases from 2 to 6 on the CIFAR-10 dataset ($\alpha$=1).


\subsection{Number of Reference Models}
In LiRA, the attacker can form a more accurate Gaussian distribution by utilizing a larger number of reference models, thereby enhancing the precision of determining whether a target sample belongs to the target model's training data. 
We evaluate the performance of distillation-based LiRA with varying numbers of reference models. 
Figure \ref{fig:distillation_lira_num_models} shows results from experiments using the Cronus framework on CIFAR-10 with $\alpha=0.1$. 
The data reveals that the performance of the distillation-based LiRA's improves as the number of distilled reference models increases.
\begin{figure}[h]
    \centering
    \includegraphics[width=0.9\linewidth]{figures/mia_diff_models.png}
    \caption{The performance of distillation-based LiRA vs. number of the distilled reference model.}
    \label{fig:distillation_lira_num_models}
\end{figure}

\subsection{Resilience Against DP-SGD}
To evaluate the robustness of our proposed LDIA and MIA methods, we assess their effectiveness when the target client employs DP-SGD\cite{abadi2016deep} during the local updates phase. 
DP-SGD is a state-of-the-art privacy-preserving model training technique.
Our experimental setup includes 10 clients participating in DS-FL training on the CIFAR-10 dataset ($\alpha=10$). We conduct LDIA and Co-op LiRA attacks against the clients during the second round of training.
In DP-SGD, it introduces noise to gradients during training, governed by three key parameters. 
The clipping bound ($C$) limits the influence of individual data points on model parameters. 
The noise multiplier ($\sigma$) determines the amount of noise added to gradients. 
The privacy budget ($\varepsilon$) balances privacy guarantees and model utility, with smaller values providing stronger privacy at the cost of potentially noisier updates.
In our experiments, we set $C$ to be 10 and vary $\sigma$ to adjust $\varepsilon$. 
This setup allows us to evaluate our proposed attack under different privacy protection levels.
\begin{table}[h]
    \caption{Performance of MIA and LDIA against DP-SGD for DS-FL trained on CIFAR-10.}
    \scriptsize
    \resizebox{0.9\linewidth}{!}{%
    \centering
    \begin{tabular}{c|c|c|c|c}
    \toprule
         $\sigma$&$\varepsilon$&Average acc&LDIA(KL divergence)&MIA(TPR at 1\%FPR)\\
    \midrule
         0  &$\infty$ & 59.09\% & 0.03 & 15.76\% \\
         0.1& >10000  & 48.68\% & 0.07 & 2.86\% \\
         0.3& >5000   & 41.29\% & 0.08 & 2.11\% \\
         0.5& >2000   & 28.53\% & 0.09 & 1.54\% \\
         1.0& 231     & 21.34\% & 0.10 & 1.29\% \\
    \bottomrule
    \end{tabular}%
    }
    \label{tab:DP}
\end{table}



\subsection{Resilience Against Evasive Clients}
To proactively protect their privacy,  cautious clients may choose to, in each communication round, avoid sending to the server the logits of some samples in the public dataset, particularly the ones that are also in its training dataset. 
% , even though the FD protocol requests them. 
% Particularly, avoid the logits of the ones that 
To counter such defense, we propose two countermeasures as follows:
% (1) We can select the target sample as public data for the communication phase across multiple rounds.
(1) In co-op LiRA, a shadow target model can be distilled using the logits of the samples in the public dataset provided by the target model, and then obtain the logits of the target sample from this shadow target model as an approximation to the one from the target clients' model. 
The intuition is that although knowledge distillation reduces the distilled student model's membership information of the teacher model~\cite{jagielski2024students}, it still preserves statistically significant enough membership information for a percentage of the members in teacher's training data, thus allowing some success in the MIA attack to the teacher model.
(2) We can also leverage a technique called indirect queries~\cite{wen2022canary, long2020pragmatic}, which is to first obtain logits of samples in the target sample's neighborhood from the target model and subsequently perform MIA using information encoded in these neighborhood logits. Neighbor samples are generated by adding noises to the target sample.

We conduct experiments to evaluate the effectiveness, and the experiments are on the CIFAR-10 dataset with a Dirichlet distribution parameter $\alpha$=10, with co-op LiRA as the MIA method.
Equipped with the first countermeasure, the attack achieves a TPR of 4.53\% at 1\% FPR.
Implementing a simplified version of the second countermeasure gives the attack a TPR of 4.23\% at 1\% FPR.
Note that in implementing countermeasure two, we add random Gaussian noise to the target samples to generate neighbor samples, with the noise clipped to the [-0.7, 0.7] range.
Studies in~\cite{wen2022canary, long2020pragmatic} implement more advanced schemes to learn from the neighbor logits, leading to better attacks. We leave studying such schemes as future work.

\section{Privacy Risk in Federated Distillation}
\label{sec:assessement}
While FL is designed to protect clients' private data, recent research~\cite{gu2023ldia,nasr2019comprehensive, liu2023mia,yang2022fd,wang2024graddiff} reveals significant privacy risks in these frameworks. 
In FL, Gu \etal~\cite{gu2023ldia} demonstrated that server-side LDIA could achieve a KL-divergence of 0.01 between the inferred and the ground truth label distributions on CIFAR-10. 
Nasr \etal~\cite{nasr2019comprehensive} showed that server or client-side MIA could reach accuracies of 92.1\% and 76.3\%, respectively, on CIFAR-100.

FD frameworks transfer distilled knowledge between participants instead of informative model parameters and gradients. This mechanism generally provides more privacy protection for each client's data than traditional FL frameworks (FedAVG, FedSGD, etc.).
However, through the lens of LDIA and MIA, we observe that although privacy leakage risk in FD appears less severe than in FL, significant risks remain, as state-of-the-art privacy attacks can still achieve non-trivial success rates based on the results in the literature and our experiments. 
Our work is the first to propose a LDIA method targeting the FD frameworks, and we achieve a KL divergence of 0.02 between the inferred and the ground-truth label distributions on CIFAR-10. This attack is less successful than in the traditional FL frameworks, but label distribution leakage has been demonstrated.
Targeting the PDA-FD frameworks,
Liu \etal~\cite{liu2023mia} proposed a client-side MIA method attaining 67.0\% balanced accuracy on CIFAR-100. 
Yang \etal~\cite{yang2022fd} also demonstrated a client-side MIA method that achieved an up to 75\% balanced accuracy on CIFAR-100.
Similarly, our MIA methods (co-op LiRA and distillation-based LiRA) demonstrate considerable server-side MIA effectiveness in achieving a TPR of up to 35.76\% at a 1\% FPR on CIFAR-10. 
In addition, effective MIA methods are reported to target other FD frameworks. 
For example, Wang \etal~\cite{wang2024graddiff} reported that their MIA attack achieved 67.06\% and 79.07\% accuracy on FedGen~\cite{zhu2021data} and FedDistill~\cite{jiang2020federated} respectively, on CIFAR-10.
One of the objectives of our study is to motivate future research on privacy risks in various FD frameworks and, more broadly, FL frameworks.


\section{Related Works}
% \subsection{Supervised Fine-Tuning}
% % 指令微调
% % 指令微调对LLM具有重要的作用,具体是什么?
% % 或者模仿Magpie的写法,这一段就纯讲作用,以及对应的工作有哪些

% % 指令微调的作用->sft技术分类->特别介绍conversation based prompt,因为我们也在用
% A series of studies find that if adjusted with annotated "instructional" data, LMs can effectively generate responses aligned with human values~\cite{sanh2022multitask, weifinetuned,ouyang2022training}. The performance of Supervised Fine-Tuning depends not only on the quality of the dataset~\cite{Zhou2023LIMALI} but also on various contextual prompting techniques, such as conversation-based prompts~\cite{sreedhar2024canttalkaboutthis, Wei2023ZeroShotIE}, chain-of-thought~\cite{Wei2022ChainOT}, and contextual calibration~\cite{Zhao2021CalibrateBU}.
% % 因为要对齐deepthink,这边强调一下conversation-based prompts
% Specifically, more models now use conversation-based prompts as the default for QA model deployment~\cite{wu2023brief,liu2024chatqa}, because they enhance the user experience by handling follow-up questions, providing clarifications, and reducing hallucinations.

% 数据合成->分为人工标注和LLM自己生成->人工标注成本高,LLM自己生成会有一些幻觉sample->我们在真实的QA下用rag来避免幻觉并且使用refiner来保证前后topic一致性以及保证数据真实性。(保证数据真实性是因为refiner前后能看到的rag的信息更广,引入更多事实数据)
\subsection{Instruction Data Synthesis}
To address the issue of limited training samples in specific domains, various works have proposed using additional data, such as manual annotation~\cite{Zhao2024WildChat1C,zheng2023lmsys} and automatic generation by LLMs~\cite{Mekala2022LeveragingQD, Wang2021TowardsZL, Wang2022SelfInstructAL, Xu2023WizardLMEL}. However, manual annotation is expensive~\cite{honovich-etal-2023-unnatural}, and iterative generation by LLMs frequently introduces the risk of hallucinations.


Our work falls into the category of automatic generation by LLMs. However, our work differs from previous approaches in two main aspects. (1) We synthesize instructions by simulating conversations closer to real-world scenarios. (2) We adopt several techniques to improve the quality of synthesized instruction. We integrate Retrieval-Augmented Generation (RAG) to mitigate hallucination in conversation-based synthesis. We apply a Conversation-based Data Refiner for filtering, ensuring topic consistency and data authenticity.
% RAG的作用->早期关注于检索器本身->现在专注于when and how ->分别举两个例子验证when and how -> 我们是在sft阶段使用rag的
\subsection{Retrieval-Augmented Generation}
Retrieval augmentation has become a standard solution to address hallucinations in LLMs by introducing external knowledge to compensate for factual shortcomings~\cite{Asai2023SelfRAGLT,ma2023query, Izacard2021UnsupervisedDI, Ram2023InContextRL}.
Early Retrieval Augmentation efforts focus primarily on the retriever itself, where both the neural retriever and generator are typically trainable Pretrained Language Models (PrLMs), such as BERT ~\cite{Devlin2019BERTPO} or BART ~\cite{Lewis2019BARTDS}. In contrast, modern Retrieval Augmentation applied to LLMs emphasizes determining when and how to retrieve relevant information~\cite{fatehkia2024t, Asai2023SelfRAGLT, Xu2024LargeLM}. For example, Self-RAG enables on-demand retrieval and generates more accurate, fact-based text through fine-grained self-reflection~\cite{Asai2023SelfRAGLT}. 
% mHyER bridges the semantic gap between learner input and practice content by generating hypothetical exercises related to the learner's input, 
% thereby improving retrieval relevance~\cite{Xu2024LargeLM}. 

Our approach uses RAG throughout the data synthesis, SFT, and inference stages. This not only improves the authenticity of the synthesized data but also helps the LLM learn how to effectively utilize the retrieved knowledge during the SFT stage. In contrast, previous research only used RAG during the inference stage, relying heavily on the LLM's ability to discern the retrieved knowledge. This can lead to insufficient utilization of relevant knowledge, especially when dealing with domain knowledge that was not included in the pretraining process.


\section*{Conclusion}
This paper aims to enhance our understanding of the computational complexity of computing various Shapley value variants. We found that for various ML models --- including decision trees, regression tree ensembles, weighted automata, and linear regression --- both local and global interventional and baseline SHAP can be computed in polynomial time under HMM modeled distributions. This extends popular algorithms, such as TreeSHAP, beyond their empirical distributional scope. We also establish strict complexity gaps between the various SHAP variants (baseline, interventional, and conditional) and prove the intractability of computing SHAP for tree ensembles and neural networks in simplified scenarios. Overall, we present SHAP as a versatile framework whose complexity depends on four key factors: \begin{inparaenum}[(i)] \item model type, \item SHAP variant, \item distribution modeling approach, \item and local vs. global explanations\end{inparaenum}. We believe this perspective provides deeper insight into the computational complexity of SHAP, paving the way for future work.




%We believe that our framework provides a more intricate understanding of SHAP computation complexity across different models, distributions, and variants, paving the way for further research.

Our work opens promising directions for future research. First, expanding our computational analysis to other SHAP-related metrics, such as asymmetric SHAP~\citep{frye20} and SAGE~\citep{covert2020understanding}, would be valuable. Additionally, we aim to explore more expressive distribution classes and relaxed assumptions beyond those in Section \ref{sec:tractable} while maintaining tractable SHAP computation. Finally, when exact computation is intractable (Section \ref{sec:intractable}), investigating the approximability of SHAP metrics through approximation and parameterized complexity theory~\citep{downey2012parameterized} is an important direction.

%Our work opens several promising avenues for future research on the computational properties of explainable AI methods, with a particular focus on SHAP. First, it would be interesting to broaden the computational analysis conducted in this work to include other popular SHAP-related metrics in the literature, such as asymmetric SHAP \cite{frye20} and SAGE \cite{covert2020understanding}. Also, in the future, we aim to explore more expressive distribution classes and relaxed distributional assumptions—extending beyond those examined in Section \ref{sec:tractable} —that still yield tractable SHAP computation. Finally, when exact computation proves intractable (Section \ref{sec:intractable}), it is worthwhile to theoretically investigate the question of the approximability of computing the SHAP metrics across various configurations, through the lens of approximation and parametrized complexity theory \cite{arora2009computational}.

%This paper aims to deepen our understanding of the computational complexity involved in obtaining different Shapley value variants. We found that for a variety of ML models, including decision trees, tree ensembles for regression, weighted automata, and linear regression models — computing both local and global interventional and baseline SHAP can be done in polynomial time when distributions are modeled by HMMs. This extends the distributional scope of popular algorithms like TreeSHAP, which is limited to empirical distributions. Additionally, we demonstrate a strict complexity gap between SHAP variants, showing that interventional and baseline SHAP can be strictly easier to compute than conditional SHAP. Despite these positive results, we uncovered intractability for various SHAP variants in neural networks and tree ensembles. Finally, we provided generalized complexity relations across SHAP variants. We believe that our framework offers a deeper understanding of the complexity involved in computing SHAP across various variants, models, distributions, as well as in both local and global computations, laying the groundwork for future research.


\bibliographystyle{ACM-Reference-Format}
\bibliography{reference}

\end{document}
\endinput
%%
%% End of file `sample-sigconf.tex'.
