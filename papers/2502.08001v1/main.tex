%%
%% This is file `sample-sigconf.tex',
%% generated with the docstrip utility.
%%
%% The original source files were:
%%
%% samples.dtx  (with options: `all,proceedings,bibtex,sigconf')
%% 
%% IMPORTANT NOTICE:
%% 
%% For the copyright see the source file.
%% 
%% Any modified versions of this file must be renamed
%% with new filenames distinct from sample-sigconf.tex.
%% 
%% For distribution of the original source see the terms
%% for copying and modification in the file samples.dtx.
%% 
%% This generated file may be distributed as long as the
%% original source files, as listed above, are part of the
%% same distribution. (The sources need not necessarily be
%% in the same archive or directory.)
%%
%%
%% Commands for TeXCount
%TC:macro \cite [option:text,text]
%TC:macro \citep [option:text,text]
%TC:macro \citet [option:text,text]
%TC:envir table 0 1
%TC:envir table* 0 1
%TC:envir tabular [ignore] word
%TC:envir displaymath 0 word
%TC:envir math 0 word
%TC:envir comment 0 0
%%
%%
%% The first command in your LaTeX source must be the \documentclass
%% command.
%%
%% For submission and review of your manuscript please change the
%% command to \documentclass[manuscript, screen, review]{acmart}.
%%
%% When submitting camera ready or to TAPS, please change the command
%% to \documentclass[sigconf]{acmart} or whichever template is required
%% for your publication.
%%
%%
%\documentclass[manuscript, screen, review]{acmart}.
\documentclass[sigconf,balance=true]{acmart}
\usepackage{algorithm}
\usepackage{algorithmic}
\usepackage{enumitem}
\usepackage{graphicx}
\usepackage{multirow}
\usepackage{booktabs}
\usepackage{subcaption}
\usepackage{caption}
\usepackage{titlesec}
\usepackage{xspace}
\usepackage{amsmath}
%\usepackage{paralist}

\expandafter\def\expandafter\normalsize\expandafter{%
    \normalsize%
    \setlength\abovedisplayskip{3pt}%
    \setlength\belowdisplayskip{3pt}%
    \setlength\abovedisplayshortskip{2pt}%
    \setlength\belowdisplayshortskip{2pt}%
}


\setlength{\textfloatsep}{0pt}   % Space between floats and text
\setlength{\abovecaptionskip}{0pt} % Space above the caption
\setlength{\belowcaptionskip}{0pt}
\setlength{\floatsep}{0pt}    
\setlength{\intextsep}{0pt} 


\titlespacing*{\subsection}{0pt}{*0.2}{*0.2}
\titlespacing*{\section}{0pt}{*0.4}{*0.4}

%%

\newcommand{\etal}{{\em et al.}\xspace}
\newcommand{\BfPara}[1]{{\noindent {\bf #1.}}}

%%
%% \BibTeX command to typeset BibTeX logo in the docs
\AtBeginDocument{%
  \providecommand\BibTeX{{%
    Bib\TeX}}}
\settopmatter{printacmref=false}  % Removes ACM reference format
\renewcommand\footnotetextcopyrightpermission[1]{}  % Removes footnote with conference/copyright information

%% Rights management information.  This information is sent to you
%% when you complete the rights form.  These commands have SAMPLE
%% values in them; it is your responsibility as an author to replace
%% the commands and values with those provided to you when you
%% complete the rights form.
% \setcopyright{acmlicensed}
% \copyrightyear{2018}
% \acmYear{2018}
% \acmDOI{XXXXXXX.XXXXXXX}

% %% These commands are for a PROCEEDINGS abstract or paper.
% \acmConference[Conference acronym 'XX]{Make sure to enter the correct
%   conference title from your rights confirmation emai}{June 03--05,
%   2018}{Woodstock, NY}
%%
%%  Uncomment \acmBooktitle if the title of the proceedings is different
%%  from ``Proceedings of ...''!
%%
%%\acmBooktitle{Woodstock '18: ACM Symposium on Neural Gaze Detection,
%%  June 03--05, 2018, Woodstock, NY}
% \acmISBN{978-1-4503-XXXX-X/18/06}


%%
%% Submission ID.
%% Use this when submitting an article to a sponsored event. You'll
%% receive a unique submission ID from the organizers
%% of the event, and this ID should be used as the parameter to this command.
%%\acmSubmissionID{123-A56-BU3}

%%
%% For managing citations, it is recommended to use bibliography
%% files in BibTeX format.
%%
%% You can then either use BibTeX with the ACM-Reference-Format style,
%% or BibLaTeX with the acmnumeric or acmauthoryear sytles, that include
%% support for advanced citation of software artefact from the
%% biblatex-software package, also separately available on CTAN.
%%
%% Look at the sample-*-biblatex.tex files for templates showcasing
%% the biblatex styles.
%%

%%
%% The majority of ACM publications use numbered citations and
%% references.  The command \citestyle{authoryear} switches to the
%% "author year" style.
%%
%% If you are preparing content for an event
%% sponsored by ACM SIGGRAPH, you must use the "author year" style of
%% citations and references.
%% Uncommenting
%% the next command will enable that style.
%%\citestyle{acmauthoryear}

\captionsetup{skip=5pt}
\setlength{\belowcaptionskip}{2pt}
\setlength{\textfloatsep}{2pt}
\setlength{\floatsep}{2pt}
\setlength{\intextsep}{2pt}



%%
%% end of the preamble, start of the body of the document source.
\begin{document}

%%
%% The "title" command has an optional parameter,
%% allowing the author to define a "short title" to be used in page headers.
%\title{Privacy Risks in Public Dataset-Assisted Federated Distillation}
\title{Unveiling Client Privacy Leakage from Public Dataset Usage in Federated Distillation}

%Client Privacy Leakage from Public Dataset Usage in Federated Distillation: Through the lens of Membership and Label Distribution Inference Attacks

%%
%% The "author" command and its associated commands are used to define
%% the authors and their affiliations.
%% Of note is the shared affiliation of the first two authors, and the
%% "authornote" and "authornotemark" commands
%% used to denote shared contribution to the research.
% \author{}
% \authornote{}
% \email{}
% \orcid{}
% \author{}
% \authornotemark[1]
% \email{}
% \affiliation{%
%   \institution{}
%   \city{}
%   \state{}
%   \country{}
% }
\author{Haonan Shi}
\email{haonan.shi3@case.edu}
\affiliation{%
  \institution{Case Western Reserve University}
  \city{Cleveland}
  \state{Ohio}
  \country{USA}
}

\author{Tu Ouyang}
\email{tu.ouyang@case.edu}
\affiliation{%
  \institution{Case Western Reserve University}
  \city{Cleveland}
  \state{Ohio}
  \country{USA}
}

\author{An Wang}
\email{an.wang@case.edu}
\affiliation{%
  \institution{Case Western Reserve University}
  \city{Cleveland}
  \state{Ohio}
  \country{USA}
}
%%
%% By default, the full list of authors will be used in the page
%% headers. Often, this list is too long, and will overlap
%% other information printed in the page headers. This command allows
%% the author to define a more concise list
%% of authors' names for this purpose.
\renewcommand{\shortauthors}{Trovato et al.}
\newcommand{\haonan}[1]{{\leavespace {\color{blue} #1}}}

%%
%% The abstract is a short summary of the work to be presented in the
%% article.
\begin{abstract}
Federated Distillation (FD) has emerged as a popular federated training framework, enabling clients to collaboratively train models without sharing private data.
Public Dataset-Assisted Federated Distillation (PDA-FD), which leverages public datasets for knowledge sharing, has become widely adopted. 
Although PDA-FD enhances privacy compared to traditional Federated Learning, we demonstrate that the use of public datasets still poses significant privacy risks to clients' private training data.
This paper presents the first comprehensive privacy analysis of PDA-FD in presence of an honest-but-curious server. 
We show that the server can exploit clients' inference results on public datasets to extract two critical types of private information: label distributions and membership information of the private training dataset.
To quantify these vulnerabilities, we introduce two novel attacks specifically designed for the PDA-FD setting: a label distribution inference attack and innovative membership inference methods based on Likelihood Ratio Attack (LiRA).
Through extensive evaluation of three representative PDA-FD frameworks (FedMD, DS-FL, and Cronus), our attacks achieve state-of-the-art performance, with label distribution attacks reaching minimal KL-divergence and membership inference attacks maintaining high True Positive Rates under low False Positive Rate constraints. 
Our findings reveal significant privacy risks in current PDA-FD frameworks and emphasize the need for more robust privacy protection mechanisms in collaborative learning systems.
%Federated Distillation (FD) has emerged as a popular federated training framework, enabling clients to collaboratively train models without sharing private data. Public Dataset-Assisted Federated Distillation (PDA-FD) is a widely adopted FD framework that utilizes a public dataset for sharing knowledge in collaborative training.
% , allowing servers only black-box access to clients via public dataset.
%While PDA-FD offers enhanced privacy compared to the traditional Federated Learning, the use of the public datasets might still pose significant privacy risks to clients.
%This paper investigates privacy leakage in PDA-FD from the server's perspective as an attacker.
%focusing on clients' label distribution and membership information. 
%\haonan{We discover that by leveraging public datasets, the server can obtain private information about clients' private dataset, including label distribution and membership information.}
%We propose a label distribution attack method and novel membership inference attack methods that extend LiRA. 
%We evaluate the proposed attacks on three PDA-FD frameworks: FedMD, DS-FL, and Cronus.
%The proposed label distribution and membership inference attacks demonstrate superior performance, achieving state-of-the-art results in KL-divergence and True Positive Rates under low False Positive Rates, respectively.
%These findings underscore the potential privacy vulnerabilities introduced by the public dataset usage in FD, highlighting the need for enhanced privacy-preserving mechanisms in collaborative training environments.
\end{abstract}

%%
%% The code below is generated by the tool at http://dl.acm.org/ccs.cfm.
%% Please copy and paste the code instead of the example below.
%%
% \begin{CCSXML}
% <ccs2012>
%    <concept>
%        <concept_id>10002978</concept_id>
%        <concept_desc>Security and privacy</concept_desc>
%        <concept_significance>500</concept_significance>
%        </concept>
%    <concept>
%        <concept_id>10010147.10010257</concept_id>
%        <concept_desc>Computing methodologies~Machine learning</concept_desc>
%        <concept_significance>500</concept_significance>
%        </concept>
%  </ccs2012>
% \end{CCSXML}

% \ccsdesc[500]{Security and privacy}
% \ccsdesc[500]{Computing methodologies~Machine learning}

%%
%% Keywords. The author(s) should pick words that accurately describe
%% the work being presented. Separate the keywords with commas.
% \keywords{Privacy attack, federated distillation, label distribution inference attack, membership inference attack.}

%%
%% This command processes the author and affiliation and title
%% information and builds the first part of the formatted document.
\maketitle

\section{Introduction}
\label{sec:introduction}
The business processes of organizations are experiencing ever-increasing complexity due to the large amount of data, high number of users, and high-tech devices involved \cite{martin2021pmopportunitieschallenges, beerepoot2023biggestbpmproblems}. This complexity may cause business processes to deviate from normal control flow due to unforeseen and disruptive anomalies \cite{adams2023proceddsriftdetection}. These control-flow anomalies manifest as unknown, skipped, and wrongly-ordered activities in the traces of event logs monitored from the execution of business processes \cite{ko2023adsystematicreview}. For the sake of clarity, let us consider an illustrative example of such anomalies. Figure \ref{FP_ANOMALIES} shows a so-called event log footprint, which captures the control flow relations of four activities of a hypothetical event log. In particular, this footprint captures the control-flow relations between activities \texttt{a}, \texttt{b}, \texttt{c} and \texttt{d}. These are the causal ($\rightarrow$) relation, concurrent ($\parallel$) relation, and other ($\#$) relations such as exclusivity or non-local dependency \cite{aalst2022pmhandbook}. In addition, on the right are six traces, of which five exhibit skipped, wrongly-ordered and unknown control-flow anomalies. For example, $\langle$\texttt{a b d}$\rangle$ has a skipped activity, which is \texttt{c}. Because of this skipped activity, the control-flow relation \texttt{b}$\,\#\,$\texttt{d} is violated, since \texttt{d} directly follows \texttt{b} in the anomalous trace.
\begin{figure}[!t]
\centering
\includegraphics[width=0.9\columnwidth]{images/FP_ANOMALIES.png}
\caption{An example event log footprint with six traces, of which five exhibit control-flow anomalies.}
\label{FP_ANOMALIES}
\end{figure}

\subsection{Control-flow anomaly detection}
Control-flow anomaly detection techniques aim to characterize the normal control flow from event logs and verify whether these deviations occur in new event logs \cite{ko2023adsystematicreview}. To develop control-flow anomaly detection techniques, \revision{process mining} has seen widespread adoption owing to process discovery and \revision{conformance checking}. On the one hand, process discovery is a set of algorithms that encode control-flow relations as a set of model elements and constraints according to a given modeling formalism \cite{aalst2022pmhandbook}; hereafter, we refer to the Petri net, a widespread modeling formalism. On the other hand, \revision{conformance checking} is an explainable set of algorithms that allows linking any deviations with the reference Petri net and providing the fitness measure, namely a measure of how much the Petri net fits the new event log \cite{aalst2022pmhandbook}. Many control-flow anomaly detection techniques based on \revision{conformance checking} (hereafter, \revision{conformance checking}-based techniques) use the fitness measure to determine whether an event log is anomalous \cite{bezerra2009pmad, bezerra2013adlogspais, myers2018icsadpm, pecchia2020applicationfailuresanalysispm}. 

The scientific literature also includes many \revision{conformance checking}-independent techniques for control-flow anomaly detection that combine specific types of trace encodings with machine/deep learning \cite{ko2023adsystematicreview, tavares2023pmtraceencoding}. Whereas these techniques are very effective, their explainability is challenging due to both the type of trace encoding employed and the machine/deep learning model used \cite{rawal2022trustworthyaiadvances,li2023explainablead}. Hence, in the following, we focus on the shortcomings of \revision{conformance checking}-based techniques to investigate whether it is possible to support the development of competitive control-flow anomaly detection techniques while maintaining the explainable nature of \revision{conformance checking}.
\begin{figure}[!t]
\centering
\includegraphics[width=\columnwidth]{images/HIGH_LEVEL_VIEW.png}
\caption{A high-level view of the proposed framework for combining \revision{process mining}-based feature extraction with dimensionality reduction for control-flow anomaly detection.}
\label{HIGH_LEVEL_VIEW}
\end{figure}

\subsection{Shortcomings of \revision{conformance checking}-based techniques}
Unfortunately, the detection effectiveness of \revision{conformance checking}-based techniques is affected by noisy data and low-quality Petri nets, which may be due to human errors in the modeling process or representational bias of process discovery algorithms \cite{bezerra2013adlogspais, pecchia2020applicationfailuresanalysispm, aalst2016pm}. Specifically, on the one hand, noisy data may introduce infrequent and deceptive control-flow relations that may result in inconsistent fitness measures, whereas, on the other hand, checking event logs against a low-quality Petri net could lead to an unreliable distribution of fitness measures. Nonetheless, such Petri nets can still be used as references to obtain insightful information for \revision{process mining}-based feature extraction, supporting the development of competitive and explainable \revision{conformance checking}-based techniques for control-flow anomaly detection despite the problems above. For example, a few works outline that token-based \revision{conformance checking} can be used for \revision{process mining}-based feature extraction to build tabular data and develop effective \revision{conformance checking}-based techniques for control-flow anomaly detection \cite{singh2022lapmsh, debenedictis2023dtadiiot}. However, to the best of our knowledge, the scientific literature lacks a structured proposal for \revision{process mining}-based feature extraction using the state-of-the-art \revision{conformance checking} variant, namely alignment-based \revision{conformance checking}.

\subsection{Contributions}
We propose a novel \revision{process mining}-based feature extraction approach with alignment-based \revision{conformance checking}. This variant aligns the deviating control flow with a reference Petri net; the resulting alignment can be inspected to extract additional statistics such as the number of times a given activity caused mismatches \cite{aalst2022pmhandbook}. We integrate this approach into a flexible and explainable framework for developing techniques for control-flow anomaly detection. The framework combines \revision{process mining}-based feature extraction and dimensionality reduction to handle high-dimensional feature sets, achieve detection effectiveness, and support explainability. Notably, in addition to our proposed \revision{process mining}-based feature extraction approach, the framework allows employing other approaches, enabling a fair comparison of multiple \revision{conformance checking}-based and \revision{conformance checking}-independent techniques for control-flow anomaly detection. Figure \ref{HIGH_LEVEL_VIEW} shows a high-level view of the framework. Business processes are monitored, and event logs obtained from the database of information systems. Subsequently, \revision{process mining}-based feature extraction is applied to these event logs and tabular data input to dimensionality reduction to identify control-flow anomalies. We apply several \revision{conformance checking}-based and \revision{conformance checking}-independent framework techniques to publicly available datasets, simulated data of a case study from railways, and real-world data of a case study from healthcare. We show that the framework techniques implementing our approach outperform the baseline \revision{conformance checking}-based techniques while maintaining the explainable nature of \revision{conformance checking}.

In summary, the contributions of this paper are as follows.
\begin{itemize}
    \item{
        A novel \revision{process mining}-based feature extraction approach to support the development of competitive and explainable \revision{conformance checking}-based techniques for control-flow anomaly detection.
    }
    \item{
        A flexible and explainable framework for developing techniques for control-flow anomaly detection using \revision{process mining}-based feature extraction and dimensionality reduction.
    }
    \item{
        Application to synthetic and real-world datasets of several \revision{conformance checking}-based and \revision{conformance checking}-independent framework techniques, evaluating their detection effectiveness and explainability.
    }
\end{itemize}

The rest of the paper is organized as follows.
\begin{itemize}
    \item Section \ref{sec:related_work} reviews the existing techniques for control-flow anomaly detection, categorizing them into \revision{conformance checking}-based and \revision{conformance checking}-independent techniques.
    \item Section \ref{sec:abccfe} provides the preliminaries of \revision{process mining} to establish the notation used throughout the paper, and delves into the details of the proposed \revision{process mining}-based feature extraction approach with alignment-based \revision{conformance checking}.
    \item Section \ref{sec:framework} describes the framework for developing \revision{conformance checking}-based and \revision{conformance checking}-independent techniques for control-flow anomaly detection that combine \revision{process mining}-based feature extraction and dimensionality reduction.
    \item Section \ref{sec:evaluation} presents the experiments conducted with multiple framework and baseline techniques using data from publicly available datasets and case studies.
    \item Section \ref{sec:conclusions} draws the conclusions and presents future work.
\end{itemize}
\section{Background}\label{sec:backgrnd}

\subsection{Cold Start Latency and Mitigation Techniques}

Traditional FaaS platforms mitigate cold starts through snapshotting, lightweight virtualization, and warm-state management. Snapshot-based methods like \textbf{REAP} and \textbf{Catalyzer} reduce initialization time by preloading or restoring container states but require significant memory and I/O resources, limiting scalability~\cite{dong_catalyzer_2020, ustiugov_benchmarking_2021}. Lightweight virtualization solutions, such as \textbf{Firecracker} microVMs, achieve fast startup times with strong isolation but depend on robust infrastructure, making them less adaptable to fluctuating workloads~\cite{agache_firecracker_2020}. Warm-state management techniques like \textbf{Faa\$T}~\cite{romero_faa_2021} and \textbf{Kraken}~\cite{vivek_kraken_2021} keep frequently invoked containers ready, balancing readiness and cost efficiency under predictable workloads but incurring overhead when demand is erratic~\cite{romero_faa_2021, vivek_kraken_2021}. While these methods perform well in resource-rich cloud environments, their resource intensity challenges applicability in edge settings.

\subsubsection{Edge FaaS Perspective}

In edge environments, cold start mitigation emphasizes lightweight designs, resource sharing, and hybrid task distribution. Lightweight execution environments like unikernels~\cite{edward_sock_2018} and \textbf{Firecracker}~\cite{agache_firecracker_2020}, as used by \textbf{TinyFaaS}~\cite{pfandzelter_tinyfaas_2020}, minimize resource usage and initialization delays but require careful orchestration to avoid resource contention. Function co-location, demonstrated by \textbf{Photons}~\cite{v_dukic_photons_2020}, reduces redundant initializations by sharing runtime resources among related functions, though this complicates isolation in multi-tenant setups~\cite{v_dukic_photons_2020}. Hybrid offloading frameworks like \textbf{GeoFaaS}~\cite{malekabbasi_geofaas_2024} balance edge-cloud workloads by offloading latency-tolerant tasks to the cloud and reserving edge resources for real-time operations, requiring reliable connectivity and efficient task management. These edge-specific strategies address cold starts effectively but introduce challenges in scalability and orchestration.

\subsection{Predictive Scaling and Caching Techniques}

Efficient resource allocation is vital for maintaining low latency and high availability in serverless platforms. Predictive scaling and caching techniques dynamically provision resources and reduce cold start latency by leveraging workload prediction and state retention.
Traditional FaaS platforms use predictive scaling and caching to optimize resources, employing techniques (OFC, FaasCache) to reduce cold starts. However, these methods rely on centralized orchestration and workload predictability, limiting their effectiveness in dynamic, resource-constrained edge environments.



\subsubsection{Edge FaaS Perspective}

Edge FaaS platforms adapt predictive scaling and caching techniques to constrain resources and heterogeneous environments. \textbf{EDGE-Cache}~\cite{kim_delay-aware_2022} uses traffic profiling to selectively retain high-priority functions, reducing memory overhead while maintaining readiness for frequent requests. Hybrid frameworks like \textbf{GeoFaaS}~\cite{malekabbasi_geofaas_2024} implement distributed caching to balance resources between edge and cloud nodes, enabling low-latency processing for critical tasks while offloading less critical workloads. Machine learning methods, such as clustering-based workload predictors~\cite{gao_machine_2020} and GRU-based models~\cite{guo_applying_2018}, enhance resource provisioning in edge systems by efficiently forecasting workload spikes. These innovations effectively address cold start challenges in edge environments, though their dependency on accurate predictions and robust orchestration poses scalability challenges.

\subsection{Decentralized Orchestration, Function Placement, and Scheduling}

Efficient orchestration in serverless platforms involves workload distribution, resource optimization, and performance assurance. While traditional FaaS platforms rely on centralized control, edge environments require decentralized and adaptive strategies to address unique challenges such as resource constraints and heterogeneous hardware.



\subsubsection{Edge FaaS Perspective}

Edge FaaS platforms adopt decentralized and adaptive orchestration frameworks to meet the demands of resource-constrained environments. Systems like \textbf{Wukong} distribute scheduling across edge nodes, enhancing data locality and scalability while reducing network latency. Lightweight frameworks such as \textbf{OpenWhisk Lite}~\cite{kravchenko_kpavelopenwhisk-light_2024} optimize resource allocation by decentralizing scheduling policies, minimizing cold starts and latency in edge setups~\cite{benjamin_wukong_2020}. Hybrid solutions like \textbf{OpenFaaS}~\cite{noauthor_openfaasfaas_2024} and \textbf{EdgeMatrix}~\cite{shen_edgematrix_2023} combine edge-cloud orchestration to balance resource utilization, retaining latency-sensitive functions at the edge while offloading non-critical workloads to the cloud. While these approaches improve flexibility, they face challenges in maintaining coordination and ensuring consistent performance across distributed nodes.


\section{Research Methodology}~\label{sec:Methodology}

In this section, we discuss the process of conducting our systematic review, e.g., our search strategy for data extraction of relevant studies, based on the guidelines of Kitchenham et al.~\cite{kitchenham2022segress} to conduct SLRs and Petersen et al.~\cite{PETERSEN20151} to conduct systematic mapping studies (SMSs) in Software Engineering. In this systematic review, we divide our work into a four-stage procedure, including planning, conducting, building a taxonomy, and reporting the review, illustrated in Fig.~\ref{fig:search}. The four stages are as follows: (1) the \emph{planning} stage involved identifying research questions (RQs) and specifying the detailed research plan for the study; (2) the \emph{conducting} stage involved analyzing and synthesizing the existing primary studies to answer the research questions; (3) the \emph{taxonomy} stage was introduced to optimize the data extraction results and consolidate a taxonomy schema for REDAST methodology; (4) the \emph{reporting} stage involved the reviewing, concluding and reporting the final result of our study.

\begin{figure}[!t]
    \centering
    \includegraphics[width=1\linewidth]{fig/methodology/searching-process.drawio.pdf}
    \caption{Systematic Literature Review Process}
    \label{fig:search}
\end{figure}

\subsection{Research Questions}
In this study, we developed five research questions (RQs) to identify the input and output, analyze technologies, evaluate metrics, identify challenges, and identify potential opportunities. 

\textbf{RQ1. What are the input configurations, formats, and notations used in the requirements in requirements-driven
automated software testing?} In requirements-driven testing, the input is some form of requirements specification -- which can vary significantly. RQ1 maps the input for REDAST and reports on the comparison among different formats for requirements specification.

\textbf{RQ2. What are the frameworks, tools, processing methods, and transformation techniques used in requirements-driven automated software testing studies?} RQ2 explores the technical solutions from requirements to generated artifacts, e.g., rule-based transformation applying natural language processing (NLP) pipelines and deep learning (DL) techniques, where we additionally discuss the potential intermediate representation and additional input for the transformation process.

\textbf{RQ3. What are the test formats and coverage criteria used in the requirements-driven automated software
testing process?} RQ3 focuses on identifying the formulation of generated artifacts (i.e., the final output). We map the adopted test formats and analyze their characteristics in the REDAST process.

\textbf{RQ4. How do existing studies evaluate the generated test artifacts in the requirements-driven automated software testing process?} RQ4 identifies the evaluation datasets, metrics, and case study methodologies in the selected papers. This aims to understand how researchers assess the effectiveness, accuracy, and practical applicability of the generated test artifacts.

\textbf{RQ5. What are the limitations and challenges of existing requirements-driven automated software testing methods in the current era?} RQ5 addresses the limitations and challenges of existing studies while exploring future directions in the current era of technology development. %It particularly highlights the potential benefits of advanced LLMs and examines their capacity to meet the high expectations placed on these cutting-edge language modeling technologies. %\textcolor{blue}{CA: Do we really need to focus on LLMs? TBD.} \textcolor{orange}{FW: About LLMs, I removed the direct emphase in RQ5 but kept the discussion in RQ5 and the solution section. I think that would be more appropriate.}

\subsection{Searching Strategy}

The overview of the search process is exhibited in Fig. \ref{fig:papers}, which includes all the details of our search steps.
\begin{table}[!ht]
\caption{List of Search Terms}
\label{table:search_term}
\begin{tabularx}{\textwidth}{lX}
\hline
\textbf{Terms Group} & \textbf{Terms} \\ \hline
Test Group & test* \\
Requirement Group & requirement* OR use case* OR user stor* OR specification* \\
Software Group & software* OR system* \\
Method Group & generat* OR deriv* OR map* OR creat* OR extract* OR design* OR priorit* OR construct* OR transform* \\ \hline
\end{tabularx}
\end{table}

\begin{figure}
    \centering
    \includegraphics[width=1\linewidth]{fig/methodology/search-papers.drawio.pdf}
    \caption{Study Search Process}
    \label{fig:papers}
\end{figure}

\subsubsection{Search String Formulation}
Our research questions (RQs) guided the identification of the main search terms. We designed our search string with generic keywords to avoid missing out on any related papers, where four groups of search terms are included, namely ``test group'', ``requirement group'', ``software group'', and ``method group''. In order to capture all the expressions of the search terms, we use wildcards to match the appendix of the word, e.g., ``test*'' can capture ``testing'', ``tests'' and so on. The search terms are listed in Table~\ref{table:search_term}, decided after iterative discussion and refinement among all the authors. As a result, we finally formed the search string as follows:


\hangindent=1.5em
 \textbf{ON ABSTRACT} ((``test*'') \textbf{AND} (``requirement*'' \textbf{OR} ``use case*'' \textbf{OR} ``user stor*'' \textbf{OR} ``specifications'') \textbf{AND} (``software*'' \textbf{OR} ``system*'') \textbf{AND} (``generat*'' \textbf{OR} ``deriv*'' \textbf{OR} ``map*'' \textbf{OR} ``creat*'' \textbf{OR} ``extract*'' \textbf{OR} ``design*'' \textbf{OR} ``priorit*'' \textbf{OR} ``construct*'' \textbf{OR} ``transform*''))

The search process was conducted in September 2024, and therefore, the search results reflect studies available up to that date. We conducted the search process on six online databases: IEEE Xplore, ACM Digital Library, Wiley, Scopus, Web of Science, and Science Direct. However, some databases were incompatible with our default search string in the following situations: (1) unsupported for searching within abstract, such as Scopus, and (2) limited search terms, such as ScienceDirect. Here, for (1) situation, we searched within the title, keyword, and abstract, and for (2) situation, we separately executed the search and removed the duplicate papers in the merging process. 

\subsubsection{Automated Searching and Duplicate Removal}
We used advanced search to execute our search string within our selected databases, following our designed selection criteria in Table \ref{table:selection}. The first search returned 27,333 papers. Specifically for the duplicate removal, we used a Python script to remove (1) overlapped search results among multiple databases and (2) conference or workshop papers, also found with the same title and authors in the other journals. After duplicate removal, we obtained 21,652 papers for further filtering.

\begin{table*}[]
\caption{Selection Criteria}
\label{table:selection}
\begin{tabularx}{\textwidth}{lX}
\hline
\textbf{Criterion ID} & \textbf{Criterion Description} \\ \hline
S01          & Papers written in English. \\
S02-1        & Papers in the subjects of "Computer Science" or "Software Engineering". \\
S02-2        & Papers published on software testing-related issues. \\
S03          & Papers published from 1991 to the present. \\ 
S04          & Papers with accessible full text. \\ \hline
\end{tabularx}
\end{table*}

\begin{table*}[]
\small
\caption{Inclusion and Exclusion Criteria}
\label{table:criteria}
\begin{tabularx}{\textwidth}{lX}
\hline
\textbf{ID}  & \textbf{Description} \\ \hline
\multicolumn{2}{l}{\textbf{Inclusion Criteria}} \\ \hline
I01 & Papers about requirements-driven automated system testing or acceptance testing generation, or studies that generate system-testing-related artifacts. \\
I02 & Peer-reviewed studies that have been used in academia with references from literature. \\ \hline
\multicolumn{2}{l}{\textbf{Exclusion Criteria}} \\ \hline
E01 & Studies that only support automated code generation, but not test-artifact generation. \\
E02 & Studies that do not use requirements-related information as an input. \\
E03 & Papers with fewer than 5 pages (1-4 pages). \\
E04 & Non-primary studies (secondary or tertiary studies). \\
E05 & Vision papers and grey literature (unpublished work), books (chapters), posters, discussions, opinions, keynotes, magazine articles, experience, and comparison papers. \\ \hline
\end{tabularx}
\end{table*}

\subsubsection{Filtering Process}

In this step, we filtered a total of 21,652 papers using the inclusion and exclusion criteria outlined in Table \ref{table:criteria}. This process was primarily carried out by the first and second authors. Our criteria are structured at different levels, facilitating a multi-step filtering process. This approach involves applying various criteria in three distinct phases. We employed a cross-verification method involving (1) the first and second authors and (2) the other authors. Initially, the filtering was conducted separately by the first and second authors. After cross-verifying their results, the results were then reviewed and discussed further by the other authors for final decision-making. We widely adopted this verification strategy within the filtering stages. During the filtering process, we managed our paper list using a BibTeX file and categorized the papers with color-coding through BibTeX management software\footnote{\url{https://bibdesk.sourceforge.io/}}, i.e., “red” for irrelevant papers, “yellow” for potentially relevant papers, and “blue” for relevant papers. This color-coding system facilitated the organization and review of papers according to their relevance.

The screening process is shown below,
\begin{itemize}
    \item \textbf{1st-round Filtering} was based on the title and abstract, using the criteria I01 and E01. At this stage, the number of papers was reduced from 21,652 to 9,071.
    \item \textbf{2nd-round Filtering}. We attempted to include requirements-related papers based on E02 on the title and abstract level, which resulted from 9,071 to 4,071 papers. We excluded all the papers that did not focus on requirements-related information as an input or only mentioned the term ``requirements'' but did not refer to the requirements specification.
    \item \textbf{3rd-round Filtering}. We selectively reviewed the content of papers identified as potentially relevant to requirements-driven automated test generation. This process resulted in 162 papers for further analysis.
\end{itemize}
Note that, especially for third-round filtering, we aimed to include as many relevant papers as possible, even borderline cases, according to our criteria. The results were then discussed iteratively among all the authors to reach a consensus.

\subsubsection{Snowballing}

Snowballing is necessary for identifying papers that may have been missed during the automated search. Following the guidelines by Wohlin~\cite{wohlin2014guidelines}, we conducted both forward and backward snowballing. As a result, we identified 24 additional papers through this process.

\subsubsection{Data Extraction}

Based on the formulated research questions (RQs), we designed 38 data extraction questions\footnote{\url{https://drive.google.com/file/d/1yjy-59Juu9L3WHaOPu-XQo-j-HHGTbx_/view?usp=sharing}} and created a Google Form to collect the required information from the relevant papers. The questions included 30 short-answer questions, six checkbox questions, and two selection questions. The data extraction was organized into five sections: (1) basic information: fundamental details such as title, author, venue, etc.; (2) open information: insights on motivation, limitations, challenges, etc.; (3) requirements: requirements format, notation, and related aspects; (4) methodology: details, including immediate representation and technique support; (5) test-related information: test format(s), coverage, and related elements. Similar to the filtering process, the first and second authors conducted the data extraction and then forwarded the results to the other authors to initiate the review meeting.

\subsubsection{Quality Assessment}

During the data extraction process, we encountered papers with insufficient information. To address this, we conducted a quality assessment in parallel to ensure the relevance of the papers to our objectives. This approach, also adopted in previous secondary studies~\cite{shamsujjoha2021developing, naveed2024model}, involved designing a set of assessment questions based on guidelines by Kitchenham et al.~\cite{kitchenham2022segress}. The quality assessment questions in our study are shown below:
\begin{itemize}
    \item \textbf{QA1}. Does this study clearly state \emph{how} requirements drive automated test generation?
    \item \textbf{QA2}. Does this study clearly state the \emph{aim} of REDAST?
    \item \textbf{QA3}. Does this study enable \emph{automation} in test generation?
    \item \textbf{QA4}. Does this study demonstrate the usability of the method from the perspective of methodology explanation, discussion, case examples, and experiments?
\end{itemize}
QA4 originates from an open perspective in the review process, where we focused on evaluation, discussion, and explanation. Our review also examined the study’s overall structure, including the methodology description, case studies, experiments, and analyses. The detailed results of the quality assessment are provided in the Appendix. Following this assessment, the final data extraction was based on 156 papers.

% \begin{table}[]
% \begin{tabular}{ll}
% \hline
% QA ID & QA Questions                                             \\ \hline
% Q01   & Does this study clearly state its aims?                  \\
% Q02   & Does this study clearly describe its methodology?        \\
% Q03   & Does this study involve automated test generation?       \\
% Q04   & Does this study include a promising evaluation?          \\
% Q05   & Does this study demonstrate the usability of the method? \\ \hline
% \end{tabular}%
% \caption{Questions for Quality Assessment}
% \label{table:qa}
% \end{table}

% automated quality assessment

% \textcolor{blue}{CA: Our search strategy focused on identifying requirements types first. We covered several sources, e.g., ~\cite{Pohl:11,wagner2019status} to identify different formats and notations of specifying requirements. However, this came out to be a long list, e.g., free-form NL requirements, semi-formal UML models, free-from textual use case models, UML class diagrams, UML activity diagrams, and so on. In this paper, we attempted to primarily focus on requirements-related aspects and not design-level information. Hence, we generalised our search string to include generic keywords, e.g., requirement*, use case*, and user stor*. We did so to avoid missing out on any papers, bringing too restrictive in our search strategy, and not creating a too-generic search string with all the aforementioned formats to avoid getting results beyond our review's scope.}


%% Use \subsection commands to start a subsection.



%\subsection{Study Selection}

% In this step, we further looked into the content of searched papers using our search strategy and applied our inclusion and exclusion criteria. Our filtering strategy aimed to pinpoint studies focused on requirements-driven system-level testing. Recognizing the presence of irrelevant papers in our search results, we established detailed selection criteria for preliminary inclusion and exclusion, as shown in Table \ref{table: criteria}. Specifically, we further developed the taxonomy schema to exclude two types of studies that did not meet the requirements for system-level testing: (1) studies supporting specification-driven test generation, such as UML-driven test generation, rather than requirements-driven testing, and (2) studies focusing on code-based test generation, such as requirement-driven code generation for unit testing.




\section{Experiments}
\label{sec:evaluation}
We examine the performance of \algname in comparison to existing registration methods for Gaussian Splatting and point clouds. Specifically, we compare two variants of \algname---i.e., \algname-NR, which solves the optimization problem \eqref{eq:coarse_registration} in closed-form without RANSAC, and \algname-R, which utilizes RANSAC for coarse registration---to the GSplat registration methods GaussReg \cite{chang2025gaussreg} and PhotoReg \cite{yuan2024photoreg}, in addition to RANSAC-based global registration (RANSAC-GR) \cite{fischler1981random, holz2015registration}, Fast Global Registration (FGR) \cite{zhou2016fast}, and variants of the Iterative Closest Point (ICP) \cite{rusinkiewicz2001efficient, park2017colored}. We evaluate each method not only on standard benchmark datasets for radiance fields, but also on real-world data collected by heterogeneous robot platforms, including a quadruped, drone, and manipulator (in the case of SIREN). In all our experiments, we only require the trained GSplat models as input; however, some of the baselines require access to the set of camera poses, which we provide when evaluating these methods.
Further, we ablate the different components of \algname, to quantify the relative improvements in performance provided by each component, and examine the gains in visual fidelity afforded by finetuning the fused model. We provide these results in Appendix~\ref{sec:appendix_experiments}, as well as additional discussion of the results presented in this section. 
Lastly, we demonstrate \algname in collaborative multi-robot mapping, where the mapping task cannot be accomplished by a single robot, necessitating mapping with multiple robots for task success. 

\phantomsection
\label{ssec:experiment_metrics}
\smallskip
\noindent\textbf{Experimental Setup and Metrics.}
For the real-world robot data, we utilize the Unitree Go1 Quadruped and a Modal AI drone with an onboard camera and the Franka Panda manipulator with a wrist camera to collect RGB images. In addition, we evaluate all methods on the real-world scenes in the Mip-NeRF360 dataset \cite{barron2022mip}, a state-of-the-art benchmark dataset for neural rendering. We train the GSplat models using the original implementation provided by the authors of \cite{kerbl20233d} for baselines which require this pipeline and utilize Nerfstudio \cite{tancik2023nerfstudio} for \algname. We execute SIREN on a desktop computer with a 24GB NVIDIA GeForce RTX 3090 GPU and the baselines on an H20 GPU after training the GSplat maps for $30000$ iterations.
We note that in robotics, the geometric fidelity of robot's map is of significant importance for effective localization and collision avoidance. Hence, we compare all methods in terms of the rotation error (RE) [deg.], translation error (TE), and scale error (SE) [in non-metric units] attained by each method, in addition to the computation time (CT) [sec.]. Moreover, we examine the photometric quality of the fused maps generated by each method, computing the peak signal-to-noise ratio (PSNR), the structural similarity index measure (SSIM), and the learned perceptual image patch similarity (LPIPS), standard metrics in the computer vision community for assessing visual fidelity. 
We provide color-coded results for each metric with the red shade denoting the top-performing statistic, the yellow shade denoting the second-best, and the green shade denoting the third-best. In all the registration methods, we do not pre-process the individual submaps to remove floaters (i.e., non-existing geometry). Consequently, floaters present in these submaps are retained in the fused map.

\subsection{Mip-NeRF360 Dataset}
We utilize the \emph{Playroom}, \emph{Truck}, and \emph{Room} scenes in the Mip-NeRF360 Dataset. These real-world scenes were all collected in realistic settings with natural lighting effects, both indoors and outdoors. While the \emph{Playroom} and \emph{Room} scenes were captured indoors, the \emph{Truck} scene was captured outdoors. We split the datasets into two subsets with varying overlap. Specifically, the first subset of the \emph{Truck} scene captures the left side of the truck, while the second subset captures the right side of the truck. The only overlap between both subsets occurs at the front and rear of the truck. We split the \emph{Room} scene into two subsets following the same procedure. In the \emph{Playroom} scene, we allow for greater overlap, with the density of images per subregion of the scene varying between both subsets. We train independent GSplat maps for each scene-subset pair.  

\smallskip
\noindent\textbf{Geometric Evaluation.}
In \Cref{tab:baseline_geometric_performance_metrics}, we report the geometric errors of each registration method across the three scenes. \algname-R, our method, achieves the lowest rotation and translation errors in two of the three scenes (\emph{Playroom} and \emph{Truck}): with about $1.14$x to $8.89$x lower rotation errors and about $6$x to $46$x lower translation errors compared to the baseline methods. Meanwhile, in the \emph{Room} scene, \algname-NR achieves the lowest translation and scale error, with \algname-R achieving the second-best performance on these metrics. In summary, \algname achieves the lowest geometric errors (i.e., rotation, translation, and scale errors) across all scenes, except the rotation error in the \emph{Room} scene.

\begin{table*}[th]
	\centering
	\caption{Geometric performance of the registration algorithms on the Mip-NeRF360 dataset (see Section~\ref{ssec:experiment_metrics} for a description of the metrics).}
	\label{tab:baseline_geometric_performance_metrics}
	\begin{adjustbox}{width=\linewidth}
		{\begin{tabular}{l | c c c c | c c c c | c c c c}
				\toprule
                    & \multicolumn{4}{c |}{\emph{Playroom}} & \multicolumn{4}{c |}{\emph{Truck}} & \multicolumn{4}{c}{\emph{Room}} \\
				Methods & RE $\downarrow$ & TE  $\downarrow$ & SE $\downarrow$ & CT $\downarrow$ & RE $\downarrow$ & TE $\downarrow$ & SE $\downarrow$ & CT $\downarrow$ & RE $\downarrow$ & TE $\downarrow$ & SE $\downarrow$ & CT $\downarrow$ \\
				\midrule
                    PhotoReg \cite{yuan2024photoreg} & 6.036 & 18806 & 841.3 & 2177 & 177.3 & 2856 & 444.0 & 1814 & \cellcolor{WildStrawberry!40}0.161 & 4983 & 452.7 & 1409 \ \\
                    GaussReg \cite{chang2025gaussreg} & 0.766  & 55.50 & \cellcolor{GreenYellow!40}0.364 & 15.06 & 21.10 & \cellcolor{GreenYellow!40}316.3 & 16.76 & 5.174 & 7.464  & 628.3 & \cellcolor{GreenYellow!40}91.97 & 6.932 \\
                    RANSAC-GR \cite{fischler1981random, holz2015registration} & 4.835 & 56.22 & 17.85 & 0.996 & 46.72 & 2642 & \cellcolor{GreenYellow!40}13.64 & \cellcolor{WildStrawberry!40}{2.569} & 8.139 & \cellcolor{GreenYellow!40}194.7 & 152.5 & 0.517 \\
                    FGR \cite{zhou2016fast} & 2.988 & 18.83 & 14.37 & \cellcolor{WildStrawberry!40}{0.887} & 3.778 & 2231 & 79.45 & 3.480 & 4.869 & 265.6 & 219.6 & \cellcolor{WildStrawberry!40}{0.511} \\
                    ICP \cite{rusinkiewicz2001efficient} & 2.362 & 19.11 & 14.37 & 2.127 & \cellcolor{GreenYellow!40}3.672 & 2232 & 79.45 & 3.805 & 5.154 & 266.1 & 219.6 & 1.579 \\
                    Colored-ICP \cite{park2017colored} & \cellcolor{Goldenrod!40}{0.194} & \cellcolor{GreenYellow!40}{12.28} & \textcolor{black}{14.37} & 3.951 & 4.043 & 2250 & 79.45 & 6.392 & 2.256 & 232.7 & 219.6 & 3.815 \\
                    \algname-NR [{Ours}] & \cellcolor{GreenYellow!40}0.348 & \cellcolor{Goldenrod!40}{4.860} & \cellcolor{Goldenrod!40}{0.282} & 41.16 & \cellcolor{Goldenrod!40}{0.511} & \cellcolor{Goldenrod!40}{8.07} & \cellcolor{Goldenrod!40}9.581 & 53.42 & \cellcolor{GreenYellow!40}{0.381} & \cellcolor{WildStrawberry!40}{2.648} & \cellcolor{WildStrawberry!40}{1.016} & 40.24 \\
                    \algname-R [{Ours}] & \cellcolor{WildStrawberry!40}{0.170} & \cellcolor{WildStrawberry!40}{1.933} & \cellcolor{WildStrawberry!40}{0.170} & 39.73 & \cellcolor{WildStrawberry!40}{0.413} & \cellcolor{WildStrawberry!40}{6.845} & \cellcolor{WildStrawberry!40}{2.548} & 52.47 & \cellcolor{Goldenrod!40}{0.237} & \cellcolor{Goldenrod!40}{3.289} & \cellcolor{Goldenrod!40}{2.673} & 39.71 \\
				\bottomrule
		\end{tabular}}
	\end{adjustbox}
\end{table*}

\smallskip
\noindent\textbf{Photometric Evaluation.}
Now, we examine the photometric performance of the GSplat registration methods reported in \Cref{tab:baseline_photometric_performance_metrics} in Appendix~\ref{sec:appendix_experiments}. \algname-R achieves the best photometric performance in the \emph{Playroom} scene, with the highest mean PSNR and SSIM and lowest mean LPIPS scores. Similarly, in the \emph{Room} scene, \algname-NR achieves the best photometric performance across all metrics, followed by \algname-R. In the \emph{Truck} scene, RANSAC-GR achieves the best mean PSNR and SSIM scores. Although this finding may appear inconsistent with the geometric results presented in \Cref{tab:baseline_geometric_performance_metrics}, the high standard deviation of each of the scores achieved by RANSAC-GR (about $2$x to $3$x larger than that of \algname) suggests that the geometric and photometric performance metrics for this scene might be consistent, indicating that the fused map generated by RANSAC-GR warrants further examination.
We provide rendered images from the fused map generated by RANSAC-GR compared to the ground-truth images in \Cref{fig:ransac_gr_vs_ground_truth} to examine the registration results of RANSAC-GR. From \Cref{fig:ransac_gr_vs_ground_truth}, we note that RANSAC-GR fails to accurately register the left and right sides of the truck. In fact, the left side of the truck is missing in the bottom panel associated with RANSAC-GR in \Cref{fig:ransac_gr_vs_ground_truth}. However, this failure mode is not fully captured by the mean score of the photometric performance metrics, since the rendered images of the right side of the truck (shown in the top panel in \Cref{fig:ransac_gr_vs_ground_truth}) look quite similar to the corresponding ground-truth images. In conclusion, RANSAC-GR does not accurately register the individual GSplat maps, despite achieving the highest mean PSNR and SSIM scores in the \emph{Truck} scene.
In \Cref{fig:photometric_performance_rendered_images}, we show the rendered images from the fused GSplat maps generated by the registration methods from different viewpoints compared to the ground-truth images. We visualize a pair of images from the \emph{Playroom}, \emph{Truck}, and \emph{Room} scenes, restricting our visualizations to PhotoReg, GaussReg, Colored-ICP, and \algname-R due to space considerations. 


\begin{figure}[th]
    \centering
    \includegraphics[width=\linewidth]{figures/experiments/baseline_comparisons/ransac_gr_compared_to_ground_truth/ground_truth_and_ransac_gr.pdf}
    \caption{Although RANSAC-GR achieves the highest mean PSNR and SSIM scores and the lowest LPIPS score in the \emph{Truck} scene, RANSAC-GR does not accurately register the individual GSplat maps. While the right side of the truck in the RANSAC-GR fused map looks similar to the ground-truth image (shown in the top panel), the left side of the truck is missing (shown in the bottom panel). The standard deviation of the PSNR, SSIM, and LPIPS scores achieved by RANSAC-GR reflects the actual registration performance of the method.}
    \label{fig:ransac_gr_vs_ground_truth}
\end{figure}


 

\begin{figure*}[th]
    \centering
    \includegraphics[width=\linewidth]{figures/experiments/baseline_comparisons/rendered_images.pdf}
    \caption{Rendered images from the fused GSplat maps of the \emph{Playroom}, \emph{Truck}, and \emph{Room} scenes. \algname generates high-fidelity fused GSplat maps, evidenced by the precise geometric detail in the images, visible in the regions indicated by the green squares. Inaccurate registration of GSplat maps generally result in artifacts in the rendered images.}
    \label{fig:photometric_performance_rendered_images}
\end{figure*}


\subsection{Mobile-Robot Mapping}
We utilize a quadruped and a drone to map three environments, depicted in \Cref{fig:photometric_performance_mobile_robot_mapping}. The quadruped maps the \emph{Kitchen} and \emph{Workshop} environments, while the drone maps an \emph{Apartment} scene, with multiple partitioned room-like areas. The robots create submaps in each environment individually, containing different regions of the scene. The submaps in the \emph{Kitchen} and \emph{Workshop} scenes have minimal overlap, while the submaps in the \emph{Apartment} scene have greater overlap. Since each submap is trained independently in different reference frames, fusing the submaps requires registration of the maps. Here, we examine the performance of GaussReg, PhotoReg, and two variants of \algname: \algname-NR and \algname-R, in registering the submaps in each scene to obtain a composite map of the entire scene.

\smallskip
\noindent\textbf{Geometric Performance.}
\Cref{tab:baseline_geometric_performance_metrics_mobile_robot_mapping} summarizes the geometric errors of each algorithm, showing that \algname achieves the best geometric performance across all scenes, with the top-two-performing methods being the variants of \algname. Specifically, in the \emph{Kitchen} scene, \algname-NR achieves the lowest rotation, translation, and scale errors by a factor of about $160$x, $465$x, and $488$x, respectively, compared to the best-performing baseline. The performance of \algname-R closely follows that of \algname-NR. Similarly, in the \emph{Workshop} scene, \algname-R achieves the lowest rotation and translation errors by a factor of $415$x and $1287$x, respectively, compared to the best-performing baseline, followed by \algname-NR, while \algname-NR achieves the lowest scale error by a factor of $2962$x, followed by \algname-R. Lastly, \algname-R achieves the lowest rotation, translation, and scale errors in the \emph{Apartment} scene, followed by \algname-NR. GaussReg requires the least computation time across all scenes, while PhotoReg requires the greatest computation time. Although compared to GaussReg \algname requires a notably greater computation time, \algname requires much lower computation times compared to PhotoReg.

\smallskip
\noindent\textbf{Photometric Performance.}
Further, we examine the photometric quality of the fused map generated by the GSplat registration methods across the three scenes. In line with the geometric results, \algname outperforms all the baseline methods, as reported in \Cref{tab:baseline_photometric_performance_metrics_mobile_robot_mapping}. While \algname-R achieves the best photometric scores (i.e., the highest PSNR and SSIM scores and lowest LPIPS scores) in the \emph{Workshop} scene, \algname-NR attains the best-performing PSNR, SSIM, and LPIPS scores in the \emph{Kitchen} scene, followed by \algname-R. In the \emph{Apartment} scene, \algname-R achieves the best PSNR score and LPIPS (tied with \algname-NR), while \algname-NR also achieves the best SSIM score. GaussReg outperforms PhotoReg in all scenes.
In addition to the results in \Cref{tab:baseline_photometric_performance_metrics_mobile_robot_mapping}, we provide rendered images from each of the fused map in \Cref{fig:photometric_performance_mobile_robot_rendered_images} for qualitative evaluation of the performance of each method. 

\begin{figure*}[th]
    \centering
    \includegraphics[width=\linewidth]{figures/experiments/mapping/mobile_robot_scenes.pdf}
    \caption{Stillshots of a quadruped mapping different areas of a kitchen and workshop and a drone mapping an apartment-like scene. Each robot trains independent GSplat submaps of the areas it mapped. The submaps of each scene are registered to obtain a composite map covering the entirety of the scene.}
    \label{fig:photometric_performance_mobile_robot_mapping}
\end{figure*}

\begin{table*}[th]
	\centering
	\caption{Geometric performance of GSplat registration algorithms in mobile-robot mapping.}
	\label{tab:baseline_geometric_performance_metrics_mobile_robot_mapping}
	\begin{adjustbox}{width=\linewidth}
		{\begin{tabular}{l | c c c c | c c c c | c c c c}
				\toprule
                    & \multicolumn{4}{c |}{\emph{Kitchen}} & \multicolumn{4}{c |}{\emph{Workshop}} & \multicolumn{4}{c}{\emph{Apartment}} \\
				Methods & RE $\downarrow$ & TE  $\downarrow$ & SE $\downarrow$ & CT $\downarrow$ & RE $\downarrow$ & TE $\downarrow$ & SE $\downarrow$ & CT $\downarrow$ & RE $\downarrow$ & TE $\downarrow$ & SE $\downarrow$ & CT $\downarrow$ \\
				\midrule
                    PhotoReg \cite{yuan2024photoreg} & \cellcolor{GreenYellow!40}40.49 & 2350 & 413.37 & 1042 &  140.5 & 10052 & 4310 & 934.2 & 24.09 & 4433 & 260.2 & 801.0 \\
                    GaussReg \cite{chang2025gaussreg} & 40.89 & \cellcolor{GreenYellow!40}1477 & \cellcolor{GreenYellow!40}171.8 & \cellcolor{WildStrawberry!40}11.33 & \cellcolor{GreenYellow!40}55.66 & \cellcolor{GreenYellow!40}9531 & \cellcolor{GreenYellow!40}4305 &  \cellcolor{WildStrawberry!40}5.491 & \cellcolor{GreenYellow!40}3.114 & \cellcolor{GreenYellow!40}102.6 & \cellcolor{GreenYellow!40}13.59 & \cellcolor{WildStrawberry!40}5.4983 \\
                    \algname-NR [\textbf{Ours}] & \cellcolor{WildStrawberry!40}0.253 & \cellcolor{WildStrawberry!40}3.173 & \cellcolor{WildStrawberry!40}0.352 & 59.22 & \cellcolor{Goldenrod!40}0.518 & \cellcolor{Goldenrod!40}11.77 & \cellcolor{WildStrawberry!40}1.453 & 67.98 & \cellcolor{Goldenrod!40}0.148 & \cellcolor{Goldenrod!40}1.758 & \cellcolor{Goldenrod!40}0.605 & 35.91 \\
                    \algname-R [\textbf{Ours}] & \cellcolor{Goldenrod!40}0.430 & \cellcolor{Goldenrod!40}4.795 & \cellcolor{Goldenrod!40}3.849 & 56.14 & \cellcolor{WildStrawberry!40}0.134  & \cellcolor{WildStrawberry!40}7.400 & \cellcolor{Goldenrod!40}10.88 & 55.16 & \cellcolor{WildStrawberry!40}0.119 & \cellcolor{WildStrawberry!40}1.495 & \cellcolor{WildStrawberry!40}0.102 & 34.22 \\
				\bottomrule
		\end{tabular}}
	\end{adjustbox}
\end{table*}



\begin{figure*}[th]
    \centering
    \includegraphics[width=0.85\linewidth]{figures/experiments/baseline_comparisons/mobile_robot_rendered_images.pdf}
    \caption{Rendered images from the fused GSplat maps of the \emph{Kitchen}, \emph{Workshop}, and \emph{Apartment} scenes mapped by a quadruped and drone. Unlike other competing methods, \algname generates fused GSplat maps of high visual fidelity, e.g., in the regions indicated by the green squares.}
    \label{fig:photometric_performance_mobile_robot_rendered_images}
\end{figure*}


\subsection{Tabletop Mapping with Multiple Manipulators}
We demonstrate the effectiveness of \algname in tabletop robotics tasks with fixed-base manipulators, which often require the robots to map the scene prior to the task, e.g., in manipulation \cite{shen2023distilled, shorinwa2024splat}. In \Cref{fig:photometric_performance_manipulation_scene}, we provide an example with two Franka robots, each with a wrist camera. Due to the limited workspace of each robot, visualized in \Cref{fig:photometric_performance_manipulation_scene}, mapping often requires the assistance of a human-operator \cite{shorinwa2024splat} or ad-hoc solutions such as hardware improvisation, e.g., using selfie sticks \cite{shen2023distilled}. By enabling the fusion of GSplat maps trained individually by each robot, \algname effectively eliminates these limitations. In other words, with \algname, each robot can train a submap within its reachable workspace and still recover the global map via registration with \algname. In \Cref{fig:photometric_performance_manipulation_maps}, we show the submaps trained by each robot. As expected, each robot has a high-fidelity submap within the confines of its reachable workspace, evident in the first-two images in the left robot's map and the last-two images in the right's robot map in \Cref{fig:photometric_performance_manipulation_maps}. In areas outside of its reachable workspace, the robot's map fails to represent the real world accurately, visible in the last-two images in the left robot's map and the first-two images in the right's robot map. With \algname, each robot obtains a higher-fidelity map over a much broader region of the environment. However, floaters present in the submaps can degrade the quality of the fused map in certain regions. To address this challenge, we finetune the fused map for about $70.98$ secs using images generated entirely from the GSplat maps, i.e., we do not require any real-world data. We provide rendered images from the finetuned fused map in \Cref{fig:photometric_performance_manipulation_maps}, showing near-perfect reconstruction of the global scene. We explore the finetuning procedure in Appendix~\ref{ssec:finetuning}.

\begin{figure*}[th]
    \centering
    \includegraphics[width=\linewidth]{figures/experiments/manipulation/dual_arm_manip_scene.pdf}
    \caption{Tabletop robotics tasks, e.g., manipulation, generally require robots to map the scene prior to completing the task. However, the limited workspace of each robot often demands assistance from a human-operator or improvised hardware, e.g., selfie sticks. \algname eliminates these challenges, via registration of the local maps trained by each robot to construct a global map consistent with the real-world.}
    \label{fig:photometric_performance_manipulation_scene}
\end{figure*}

\begin{figure*}[th]
    \centering
    \includegraphics[width=0.85\linewidth]{figures/experiments/manipulation/dual_arm_manip_maps.pdf}
    \caption{Rendered images of the local maps of a tabletop scene trained by two manipulators. The maps provide high-fidelity reconstructions within the workspace of each robot, but fail to represent the real-world in regions outside the workspace. \algname fuses the local maps to generate a high-fidelity global map consistent with the entirety of the scene, especially after finetuning on data rendered directly from the GSplat to remove floaters, without any interaction with the real-world, as indicated by the green squares.}
    \label{fig:photometric_performance_manipulation_maps}
\end{figure*}












\section{Ablation Study}
\label{sec:ablation}
\subsection{Public Dataset Size}
The server in PDA-FD can control communication overhead by adjusting the size of the public dataset used in each collaborative training round.
We investigate its impact on the performance of LDIA and MIA. 
As public data does not affect co-op LiRA, our evaluations of MIA mainly focus on distillation-based LiRA. 
In our experiments, we use the DS-FL framework on the CIFAR-10 dataset with $\alpha=1$.
\begin{table}[h]
    \caption{Impact of Public Data Quantity on Label Distribution and Membership Information Leakage in PDA-FD.}
    \centering
    \scriptsize
    \resizebox{0.9\linewidth}{!}{%
    \begin{tabular}{c|c|c}
        \toprule
        Datasets size&  MIA (TPR at 1\%FPR) & LDIA (KL divergence)\\
        \midrule
        5000 & 29.28\% &  0.10  \\
        7500 & 31.84\% &  0.09 \\
        10000& 32.01\% &  0.07 \\
        \bottomrule
    \end{tabular}%
    }
    \label{tab:public_dataset_size}
\end{table}
Table \ref{tab:public_dataset_size} illustrates the degree of label distribution information and membership information leakage from clients when the quantity of the public data samples is set to 5000, 7500, and 10000, respectively.
The results indicate that larger public datasets contribute to increased privacy leakage risks for clients. 
We attribute this trend to two factors. For distillation-based LiRA, a larger public dataset provides a more extensive distillation dataset, enabling the attacker to obtain more robust reference models.
In the case of LDIA, a larger public dataset serving as the inference dataset allows the attacker to mitigate the impact of outliers or atypical data, thereby improving attack accuracy.


\subsection{Number of Epochs in Local Updates Phase}
Prior to the communication phase, clients train their local models on their private datasets during the local updates phase. 
This process enhances the local model's memorization of private data, facilitating knowledge transfer between clients but also potentially increasing privacy leakage. 
We measure the impact of the number of training epochs in the local updates phase on the leakage of label information and membership information from clients.
\begin{table}[h]
    \caption{Impact of Number of Training Epochs on Label Distribution and Membership Information Leakage in PDA-FD.}
    \centering
    \scriptsize
    \resizebox{0.9\linewidth}{!}{%
    \begin{tabular}{c|c|c}
        \toprule
        Number of Epochs&  MIA (TPR at 1\%FPR) & LDIA (KL divergence)\\
        \midrule
        2 & 8.10\% &  0.15  \\
        4 & 14.64\% &  0.10 \\
        6 & 15.43\% &  0.09 \\
        \bottomrule
    \end{tabular}% 
    }
    \label{tab:epochs_local_updates}
\end{table}
As shown in Table \ref{tab:epochs_local_updates}, there is an increase in label distribution and membership information leakage from clients in DS-FL as the number of the local update training rounds increases from 2 to 6 on the CIFAR-10 dataset ($\alpha$=1).


\subsection{Number of Reference Models}
In LiRA, the attacker can form a more accurate Gaussian distribution by utilizing a larger number of reference models, thereby enhancing the precision of determining whether a target sample belongs to the target model's training data. 
We evaluate the performance of distillation-based LiRA with varying numbers of reference models. 
Figure \ref{fig:distillation_lira_num_models} shows results from experiments using the Cronus framework on CIFAR-10 with $\alpha=0.1$. 
The data reveals that the performance of the distillation-based LiRA's improves as the number of distilled reference models increases.
\begin{figure}[h]
    \centering
    \includegraphics[width=0.9\linewidth]{figures/mia_diff_models.png}
    \caption{The performance of distillation-based LiRA vs. number of the distilled reference model.}
    \label{fig:distillation_lira_num_models}
\end{figure}

\subsection{Resilience Against DP-SGD}
To evaluate the robustness of our proposed LDIA and MIA methods, we assess their effectiveness when the target client employs DP-SGD\cite{abadi2016deep} during the local updates phase. 
DP-SGD is a state-of-the-art privacy-preserving model training technique.
Our experimental setup includes 10 clients participating in DS-FL training on the CIFAR-10 dataset ($\alpha=10$). We conduct LDIA and Co-op LiRA attacks against the clients during the second round of training.
In DP-SGD, it introduces noise to gradients during training, governed by three key parameters. 
The clipping bound ($C$) limits the influence of individual data points on model parameters. 
The noise multiplier ($\sigma$) determines the amount of noise added to gradients. 
The privacy budget ($\varepsilon$) balances privacy guarantees and model utility, with smaller values providing stronger privacy at the cost of potentially noisier updates.
In our experiments, we set $C$ to be 10 and vary $\sigma$ to adjust $\varepsilon$. 
This setup allows us to evaluate our proposed attack under different privacy protection levels.
\begin{table}[h]
    \caption{Performance of MIA and LDIA against DP-SGD for DS-FL trained on CIFAR-10.}
    \scriptsize
    \resizebox{0.9\linewidth}{!}{%
    \centering
    \begin{tabular}{c|c|c|c|c}
    \toprule
         $\sigma$&$\varepsilon$&Average acc&LDIA(KL divergence)&MIA(TPR at 1\%FPR)\\
    \midrule
         0  &$\infty$ & 59.09\% & 0.03 & 15.76\% \\
         0.1& >10000  & 48.68\% & 0.07 & 2.86\% \\
         0.3& >5000   & 41.29\% & 0.08 & 2.11\% \\
         0.5& >2000   & 28.53\% & 0.09 & 1.54\% \\
         1.0& 231     & 21.34\% & 0.10 & 1.29\% \\
    \bottomrule
    \end{tabular}%
    }
    \label{tab:DP}
\end{table}



\subsection{Resilience Against Evasive Clients}
To proactively protect their privacy,  cautious clients may choose to, in each communication round, avoid sending to the server the logits of some samples in the public dataset, particularly the ones that are also in its training dataset. 
% , even though the FD protocol requests them. 
% Particularly, avoid the logits of the ones that 
To counter such defense, we propose two countermeasures as follows:
% (1) We can select the target sample as public data for the communication phase across multiple rounds.
(1) In co-op LiRA, a shadow target model can be distilled using the logits of the samples in the public dataset provided by the target model, and then obtain the logits of the target sample from this shadow target model as an approximation to the one from the target clients' model. 
The intuition is that although knowledge distillation reduces the distilled student model's membership information of the teacher model~\cite{jagielski2024students}, it still preserves statistically significant enough membership information for a percentage of the members in teacher's training data, thus allowing some success in the MIA attack to the teacher model.
(2) We can also leverage a technique called indirect queries~\cite{wen2022canary, long2020pragmatic}, which is to first obtain logits of samples in the target sample's neighborhood from the target model and subsequently perform MIA using information encoded in these neighborhood logits. Neighbor samples are generated by adding noises to the target sample.

We conduct experiments to evaluate the effectiveness, and the experiments are on the CIFAR-10 dataset with a Dirichlet distribution parameter $\alpha$=10, with co-op LiRA as the MIA method.
Equipped with the first countermeasure, the attack achieves a TPR of 4.53\% at 1\% FPR.
Implementing a simplified version of the second countermeasure gives the attack a TPR of 4.23\% at 1\% FPR.
Note that in implementing countermeasure two, we add random Gaussian noise to the target samples to generate neighbor samples, with the noise clipped to the [-0.7, 0.7] range.
Studies in~\cite{wen2022canary, long2020pragmatic} implement more advanced schemes to learn from the neighbor logits, leading to better attacks. We leave studying such schemes as future work.

\section{Privacy Risk in Federated Distillation}
\label{sec:assessement}
While FL is designed to protect clients' private data, recent research~\cite{gu2023ldia,nasr2019comprehensive, liu2023mia,yang2022fd,wang2024graddiff} reveals significant privacy risks in these frameworks. 
In FL, Gu \etal~\cite{gu2023ldia} demonstrated that server-side LDIA could achieve a KL-divergence of 0.01 between the inferred and the ground truth label distributions on CIFAR-10. 
Nasr \etal~\cite{nasr2019comprehensive} showed that server or client-side MIA could reach accuracies of 92.1\% and 76.3\%, respectively, on CIFAR-100.

FD frameworks transfer distilled knowledge between participants instead of informative model parameters and gradients. This mechanism generally provides more privacy protection for each client's data than traditional FL frameworks (FedAVG, FedSGD, etc.).
However, through the lens of LDIA and MIA, we observe that although privacy leakage risk in FD appears less severe than in FL, significant risks remain, as state-of-the-art privacy attacks can still achieve non-trivial success rates based on the results in the literature and our experiments. 
Our work is the first to propose a LDIA method targeting the FD frameworks, and we achieve a KL divergence of 0.02 between the inferred and the ground-truth label distributions on CIFAR-10. This attack is less successful than in the traditional FL frameworks, but label distribution leakage has been demonstrated.
Targeting the PDA-FD frameworks,
Liu \etal~\cite{liu2023mia} proposed a client-side MIA method attaining 67.0\% balanced accuracy on CIFAR-100. 
Yang \etal~\cite{yang2022fd} also demonstrated a client-side MIA method that achieved an up to 75\% balanced accuracy on CIFAR-100.
Similarly, our MIA methods (co-op LiRA and distillation-based LiRA) demonstrate considerable server-side MIA effectiveness in achieving a TPR of up to 35.76\% at a 1\% FPR on CIFAR-10. 
In addition, effective MIA methods are reported to target other FD frameworks. 
For example, Wang \etal~\cite{wang2024graddiff} reported that their MIA attack achieved 67.06\% and 79.07\% accuracy on FedGen~\cite{zhu2021data} and FedDistill~\cite{jiang2020federated} respectively, on CIFAR-10.
One of the objectives of our study is to motivate future research on privacy risks in various FD frameworks and, more broadly, FL frameworks.


\section{Related Work}
\label{sec:related}
FL has emerged as a crucial learning scheme for distributed training that aims to preserve user data privacy. 
However, research has uncovered various privacy vulnerabilities in FL, particularly in the form of MIA and LDIA.
This section discusses relevant works that highlight these threats in both FL and FD settings, and contextualize our research within this landscape.

\BfPara{MIA and LDIA} 
Shokri \etal~\cite{shokri2017membership} pioneered MIA research by demonstrating how model output confidence scores could reveal training data membership. 
Nasr \etal~\cite{nasr2019comprehensive} extended this to FL, showing how both passive and active adversaries could exploit gradients and model updates.
LDIA represents another significant privacy threat in FL.
Gu \etal~\cite{gu2023ldia} introduced LDIA as a new attack vector where adversaries infer label distributions from model updates. Wainakh \etal~\cite{wainakh2021user} further explored user-level label leakage through gradient-based attacks in FL.
Recent works have exposed the vulnerability of FD to inference attacks.
Yang \etal~\cite{yang2022fd} proposed FD-Leaks for performing MIA in FD settings through logit analysis. Liu \etal~\cite{liu2023mia} and Wang \etal~\cite{wang2024graddiff} enhanced MIA using shadow models via respective approaches MIA-FedDL and GradDiff, though their assumptions were limited to homogeneous environments.

\iffalse
The study of MIA in ML models was pioneered by Shokri \etal~\cite{shokri2017membership}.
They demonstrated how adversaries could exploit model output confidence scores to infer whether a specific data point was used in training.
This seminal work laid the foundation for subsequent research into privacy vulnerabilities in various ML paradigms, including FL.
Building on this, Nasr \etal~\cite{nasr2019comprehensive} extended the concept to FL environments, introducing white-box MIAs.
Their privacy analysis demonstrated how both passive and active adversaries could exploit gradients and model updates to infer private information.
\BfPara{LDIA in FL} 
LDIA represents another significant privacy threat in FL.
Gu \etal~\cite{gu2023ldia} introduced LDIA as a new attack vector, where adversaries seek to infer the distribution of labels in clients' training data by analyzing model updates.
This work showed that even when individual data points are protected, the overall data distribution might still be exposed, leading to privacy concerns.
Wainakh \etal~\cite{wainakh2021user} further explored user-level label leakage by performing gradient-based attacks in FL.

\BfPara{LDIA and MIA in FD} 
% FD has been proposed as a lightweight alternative to traditional FL.
% It reduces communication overhead by exchanging logits or softmax values instead of model parameters.
Yang \etal~\cite{yang2022fd} proposed FD-Leaks, an attack designed to perform MIA in FD settings.
By analyzing logits, adversaries can infer membership information, potentially revealing sensitive training data.
Similarly, Liu \etal~\cite{liu2023mia} presented MIA-FedDL, which enhances MIA by using shadow models to infer membership information with higher accuracy in FD settings.
Despite their contributions, they made assumptions that are infeasible in heterogeneous environments.
\fi
% Despite their contributions, these works make several assumptions that may limit the feasibility of the proposed attacks in practice.
% For instance, both FD-Leaks and MIA-FedDL assume that the adversaries have access to all logits or can easily build shadow models that mimic the behavior of target models.
% Such assumptions are often unrealistic in heterogeneous environments.

\BfPara{Defenses and Countermeasures}
DPSGD~\cite{abadi2016deep} can be employed during the training phase to mitigate against privacy attacks to the client model. 
Additionally, specialized MIA defense methods such as SELENA~\cite{tang2022mitigating}, HAMP~\cite{chen2023overconfidence} and DMP\cite{shejwalkar2021membership} can be integrated into the training process.
Several studies have proposed enhanced FD frameworks with improved privacy protection mechanisms to reduce client privacy leakage.
%Several studies have proposed defenses to mitigate MIA and LDIA.
%Wang \etal~\cite{wang2024graddiff} proposed GradDiff, a gradient-based defense mechanism that employs differential comparison to detect and mitigate MIA in FD settings.
Zhu \etal~\cite{zhu2021data} investigated data-free knowledge distillation for heterogeneous federated learning.
They presented an approach that reduces the need for public datasets.
Chen \etal~\cite{chen2023best} proposed FedHKD, where clients share hyper-knowledge based on data representations from local datasets for federated distillation without requiring public datasets or models.

% Our work builds upon these foundations, specifically investigating the privacy vulnerabilities in PDA-FD frameworks. 
%Unlike previous studies that focused on traditional FL or specific attack scenarios in FD, our work aim to provide a comprehensive analysis of LDIA and MIA across multiple PDA-FD frameworks.


\section{Conclusion}
In this work, we propose a simple yet effective approach, called SMILE, for graph few-shot learning with fewer tasks. Specifically, we introduce a novel dual-level mixup strategy, including within-task and across-task mixup, for enriching the diversity of nodes within each task and the diversity of tasks. Also, we incorporate the degree-based prior information to learn expressive node embeddings. Theoretically, we prove that SMILE effectively enhances the model's generalization performance. Empirically, we conduct extensive experiments on multiple benchmarks and the results suggest that SMILE significantly outperforms other baselines, including both in-domain and cross-domain few-shot settings.


\bibliographystyle{ACM-Reference-Format}
\bibliography{reference}

\end{document}
\endinput
%%
%% End of file `sample-sigconf.tex'.
