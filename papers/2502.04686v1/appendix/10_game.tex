\subsection{Game Rules}

\paragraph{Setup.}
Each game begins by randomly assigning seven roles—two Werewolves, one Seer, one Doctor, and three Villagers—to seven different players labeled “player\_0,” “player\_1,” …, “player\_6.” The two Werewolves are aware of each other’s identities, while the Seer, Doctor, and Villagers only know their own roles.

\paragraph{Night Round.}
During the Night round, only the surviving Werewolves, Seer, and Doctor take secret actions that are disclosed only to the relevant parties.
\begin{itemize}
    \item \textit{Werewolf}: The living Werewolves collectively decide on a target to kill, but they follow a specific order when there are two of them. First, the Werewolf with the smaller ID proposes a target; the other Werewolf then makes the final decision. For instance, if “player\_0” and “player\_2” are Werewolves, “player\_0” proposes “player\_i,” and “player\_2” chooses the ultimate kill target “player\_j.” If only one Werewolf is alive, that Werewolf’s decision stands. Werewolves cannot kill a dead player, themselves, or their teammate.
    \item \textit{Seer}: The Seer selects a living player to investigate, revealing whether that player is a Werewolf. The Seer may not investigate a dead player or themselves, although they are allowed to investigate the same player on different nights (albeit a less effective strategy).
    \item \textit{Doctor}: The Doctor selects a player to protect, without knowledge of the Werewolves’ choice. The Doctor cannot save someone who is already dead but can choose to save themselves.
\end{itemize}

\paragraph{Day Round.}
The day round proceeds with three phase including announcement, discussion, and voting.
\begin{itemize}
    \item \textit{Announcement}: at the start of the Day round, the events of the previous night are made public to all players still in the game. Anyone killed during the Night round is immediately removed and cannot reveal their role or participate in discussions. Two scenarios determine the announcement: if the Werewolves targeted “player\_i” and the Doctor either saved a different “player\_j” or was no longer alive, “player\_i” is killed, and the announcement states: “player\_i was killed last night.” If the Doctor saved exactly the same person the Werewolves intended to kill (“player\_i”), then no one is killed, and the announcement is: “no player was killed last night.”
    \item \textit{Discussion}: all surviving players join an open discussion in a set speaking order, each speaking exactly once. If, for example, the remaining players are “player\_0,” “player\_2,” and “player\_5,” then “player\_0” speaks first, followed by “player\_2,” and concluding with “player\_5.”
    \item \textit{Voting}: after the discussion, all surviving players simultaneously vote to eliminate one other player or choose to abstain. They are not allowed to vote for a dead player or for themselves. The individual who receives the most votes is eliminated without role disclosure. In the event of a tie, one of the tied players is randomly chosen to be eliminated. Everyone knows the final voting tally.
\end{itemize}

\paragraph{Winning.}
The Werewolves win if, at any point, the number of living Werewolves is equal to that of all other remaining players. They do not need to eliminate every non-Werewolf to claim victory. Conversely, the Villagers (including the Seer and Doctor) win once both Werewolves have been eliminated.


\subsection{Observation Space}

\paragraph{Language Observation.}
Each agent's language observation is represented as a list of natural language statements that log the game's history up to the current step. This list comprises both private information, which is accessible only to the current player, and public information, which is shared among all players. The private information includes the player's role, secret actions taken during the night phase by the Werewolf, Seer, and Doctor, as well as the Werewolf's teammate. On the other hand, the public information consists of the player's ID, the eliminated player in each night and day phase, discussions, and voting outcomes from each day phase. An example of the language observation is as follow.

\begin{tcolorbox}
{\ttfamily\small
Basic Information:
\begin{itemize}
    \item you are player\_5, your role is Doctor.
    \item current round and phase: night 2.
    \item remaining players: player\_0, player\_1, player\_2, player\_5, player\_6.
\end{itemize}
Round 1:
\begin{itemize}
    \item night 1: you chose to save player\_5.
    \item day 1 announcement: player\_4 was killed last night.
    \item day 1 discussion:
    \begin{itemize}
        \item player\_0 said: Good day, fellow players. As a Villager, my objective is to help identify and eliminate the Werewolves. Since player\_4 was killed last night, we know there is at least one Werewolf among us. I would like to hear everyone's thoughts and suspicions about who might be the Werewolves. Let's work together to find the culprits and ensure the safety of our village.
        \item player\_1 said: ...
        \item player\_2 said: ...
        \item player\_3 said: ...
        \item you said: ...
        \item player\_6 said: ...
    \end{itemize}
\end{itemize}
% }
% \end{tcolorbox}
% \begin{tcolorbox}
% {\ttfamily\small
\begin{itemize}
    \item day 1 voting result: player\_3 had the most votes and was eliminated. 
    \begin{itemize}
        \item voted for player\_3: player\_1, player\_6.
        \item voted for player\_1: player\_3.
        \item choose not to vote: player\_0, player\_2, player\_5.
    \end{itemize}
\end{itemize}

Now it is night 2 round and you should choose one player to save. As player\_5 and the Doctor, you should choose from the following actions: save player\_0, save player\_1, save player\_2, save player\_5, save player\_6.
}
\end{tcolorbox}

\paragraph{Vector Observation.}
We also consider a vectorized observation. The observation vector includes information like the player's ID, role, deductions, etc. by one-hot encoding. The details of the observation vector are listed in Table~\ref{tab:app:player}

\begin{table}[H]
\centering
\begin{tabular}{cccc}
\toprule
\multicolumn{2}{c}{}                                                                                          & Length & Description                                                                                                                              \\
\midrule
\multicolumn{2}{c}{ID}                                                                                        & 7      & one hot encoding of ID.                                                                                                                   \\
\multicolumn{2}{c}{Role}                                                                                      & 4      & \begin{tabular}[c]{@{}c@{}}one hot encoding of role,\\ {[}"Werewolf", "Seer", "Doctor", "Villager"{].}\end{tabular}                        \\
\multicolumn{2}{c}{Round}                                                                                     & 1      & current round.                                                                                                                            \\
\multicolumn{2}{c}{Phase}                                                                                     & 3      & \begin{tabular}[c]{@{}c@{}}one hot encoding of current phase,\\ {[}"night", "discussion", "voting"{].}\end{tabular}                        \\
\multicolumn{2}{c}{Alive players}                                                                             & 7      & alive flag for each player.                                                                                                               \\
\midrule
\multirow{3}{*}{\begin{tabular}[c]{@{}c@{}}For each round\\ (3 rounds)\end{tabular}} & secret action & 7      & \begin{tabular}[c]{@{}c@{}}one hot encoding of the target player,\\ (all zero if do not act).\end{tabular}                                 \\
                                                                                              & announcement  & 7      & \begin{tabular}[c]{@{}c@{}}one hot encoding of the dead player,\\ (all zero if no player is dead).\end{tabular}                            \\
                                                                                              & voting result & 49     & \begin{tabular}[c]{@{}c@{}}one hot encoding of the each player's choice,\\ (all zero if the player does not vote or is dead).\end{tabular} \\
\bottomrule
\end{tabular}

\caption{Vector observation space.}
\label{tab:app:player}
\end{table}

\subsection{Reward Functions}

The reward functions are defined as follows:
\begin{itemize}
    \item \textit{Winning Reward}: all winning players receive $+300$, and all losing players receive $-300$.
    \item \textit{Surviving Reward}: $+5$ for all surviving players in each round.
    \item \textit{Voting Reward} (Village side only): $+20$ for correct votes, $-20$ for incorrect votes.
    \item \textit{Voting Result Reward}: $-10$ for the player that is eliminated. $+5$ when an opponents is eliminated, $-5$ when a teammate is being eliminated.
\end{itemize}
