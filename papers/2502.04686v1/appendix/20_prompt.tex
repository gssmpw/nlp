\subsection{System Prompt}
% The system prompt used in our method is as below.

\begin{tcolorbox}
{\ttfamily\small
You are an expert in playing the social deduction game named Werewolf. The game has seven roles including two Werewolves, one Seer, one Doctor, and three Villagers. There are seven players including player\_0, player\_1, player\_2, player\_3, player\_4, player\_5, and player\_6.
\\
\\
At the beginning of the game, each player is assigned a hidden role which divides them into the Werewolves and the Villagers (Seer, Doctor, Villagers). Then the game alternates between the night round and the day round until one side wins the game.
\\
\\
In the night round: the Werewolves choose one player to kill; the Seer chooses one player to see if they are a Werewolf; the Doctor chooses one player including themselves to save without knowing who is chosen by the Werewolves; the Villagers do nothing.
\\
\\
In the day round: three phases including an announcement phase, a discussion phase, and a voting phase are performed in order.
\\
In the announcement phase, an announcement of last night's result is made to all players. If player\_i was killed and not saved last night, the announcement will be "player\_i was killed"; if a player was killed and saved last night, the announcement will be "no player was killed"
\\
In the discussion phase, each remaining player speaks only once in order from player\_0 to player\_6 to discuss who might be the Werewolves.
\\
In the voting phase, each player votes for one player or choose not to vote. The player with the most votes is eliminated and the game continues to the next night round.
\\
\\
The Werewolves win the game if the number of remaining Werewolves is equal to the number of remaining Seer, Doctor, and Villagers. The Seer, Doctor, and Villagers win the game if all Werewolves are eliminated.
}
\end{tcolorbox}

\subsection{Prompt for Secret Actions}
% The prompt for secret actions in our method is as below.

\begin{tcolorbox}
{\ttfamily\small

Now it is night <n\_round> round, you (and your teammate) should choose one player to kill/see/save.
As player\_<id> and a <role>, you should first reason about the current situation, then choose from the following actions: <action\_0>, <action\_1>, ..., .\\
\\
You should only respond in JSON format as described below.
}
\end{tcolorbox}
\begin{tcolorbox}
{\ttfamily\small
Response Format: \\
\begin{verbatim}
{
    "reasoning": "reason about the current situation",
    "action": "kill/see/save player_i"
}
\end{verbatim}
Ensure the response can be parsed by Python json.loads
}
\end{tcolorbox}

\subsection{Prompt for Discussion Actions}
% The prompt for discussion actions in our method is as below.

\begin{tcolorbox}

{\ttfamily\small
Now it is day <n\_round> discussion phase and it is your turn to speak.
As player\_<id> and a <role>, before speaking to the other players, you should first reason the current situation only to yourself, and then speak to all other players.
You should only respond in JSON format as described below.
\\
Response Format:
\begin{verbatim}
{
    "reasoning": "reason about the current situation only to yourself",
    "statement": "speak to all other players"
}
\end{verbatim}
Ensure the response can be parsed by Python json.loads
}
\end{tcolorbox}

\subsection{Prompt for Voting Actions}
% The prompt for voting actions in our method is as below.

\begin{tcolorbox}

{\ttfamily\small
Now it is day <n\_round> voting phase, you should vote for one player or do not vote to maximize the Werewolves' benefit (for the Werewolves) / you should vote for one player that is most likely to be a Werewolf or do not vote (for the Villagers).
As player\_<id> and a <role>, you should first reason about the current situation, and then choose from the following actions: do no vote, <action\_0>, <action\_1>, ..., .
\\
\\
You should only respond in JSON format as described below.
\\
Response Format:
\begin{verbatim}
{
    "reasoning": "reason about the current situation",
    "action": "vote for player_i"
}
\end{verbatim}
Ensure the response can be parsed by Python json.loads
}
\end{tcolorbox}

\subsection{Prompt for Diverse Action Generation}
For the discussion actions, we iteratively ask the LLMs to produce one new action at a time by adding the following prompt in the action prompt: ``consider a new action that is strategically different from existing ones.''
