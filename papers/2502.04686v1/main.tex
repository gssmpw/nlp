%%%%%%%% ICML 2025 EXAMPLE LATEX SUBMISSION FILE %%%%%%%%%%%%%%%%%

\documentclass{article}

% Recommended, but optional, packages for figures and better typesetting:
\usepackage{microtype}
\usepackage{graphicx}
\usepackage{subfigure}
\usepackage{booktabs} % for professional tables

% hyperref makes hyperlinks in the resulting PDF.
% If your build breaks (sometimes temporarily if a hyperlink spans a page)
% please comment out the following usepackage line and replace
% \usepackage{icml2025} with \usepackage[nohyperref]{icml2025} above.
\usepackage{hyperref}


% Attempt to make hyperref and algorithmic work together better:
\newcommand{\theHalgorithm}{\arabic{algorithm}}

% Use the following line for the initial blind version submitted for review:
% \usepackage{icml2025}

% If accepted, instead use the following line for the camera-ready submission:
\usepackage[accepted]{icml2025}

% For theorems and such
\usepackage{amsmath}
\usepackage{amssymb}
\usepackage{mathtools}
\usepackage{amsthm}

% \usepackage[dvipsnames]{xcolor}
\usepackage[T1]{fontenc}
\usepackage{pifont}
\usepackage{multirow}
\usepackage{makecell}
\usepackage{bm}
\usepackage{tcolorbox}

\usepackage{enumitem}
\usepackage{inconsolata}
\usepackage{subcaption}

% if you use cleveref..
\usepackage[capitalize,noabbrev]{cleveref}

%%%%%%%%%%%%%%%%%%%%%%%%%%%%%%%%
% THEOREMS
%%%%%%%%%%%%%%%%%%%%%%%%%%%%%%%%
\theoremstyle{plain}
\newtheorem{theorem}{Theorem}[section]
\newtheorem{proposition}[theorem]{Proposition}
\newtheorem{lemma}[theorem]{Lemma}
\newtheorem{corollary}[theorem]{Corollary}
\theoremstyle{definition}
\newtheorem{definition}[theorem]{Definition}
\newtheorem{assumption}[theorem]{Assumption}
\theoremstyle{remark}
\newtheorem{remark}[theorem]{Remark}

\newcommand{\cmark}{\textcolor{LimeGreen}{\ding{51}}}
\newcommand{\xmark}{\textcolor{red}{\ding{55}}}
\newcommand{\yc}[1] {\textcolor{purple}{[yc: #1]}}
\newcommand{\xzl}[1] {\textcolor{blue}{[xzl: #1]}}
\newcommand{\yw}[1]{\textcolor{red}{[yi: #1]}}
\newcommand{\gf}[1]{\textcolor{olive}{[gf: #1]}}
\newcommand{\edit}[1]{\textcolor{red}{#1}}


% Todonotes is useful during development; simply uncomment the next line
%    and comment out the line below the next line to turn off comments
%\usepackage[disable,textsize=tiny]{todonotes}
\usepackage[textsize=tiny]{todonotes}


% The \icmltitle you define below is probably too long as a header.
% Therefore, a short form for the running title is supplied here:
\icmltitlerunning{Learning Strategic Language Agents in the Werewolf Game with Iterative Latent Space Policy Optimization}

\begin{document}

\twocolumn[
\icmltitle{Learning Strategic Language Agents in the Werewolf Game \\with Iterative Latent Space Policy Optimization}

% It is OKAY to include author information, even for blind
% submissions: the style file will automatically remove it for you
% unless you've provided the [accepted] option to the icml2025
% package.

% List of affiliations: The first argument should be a (short)
% identifier you will use later to specify author affiliations
% Academic affiliations should list Department, University, City, Region, Country
% Industry affiliations should list Company, City, Region, Country

% You can specify symbols, otherwise they are numbered in order.
% Ideally, you should not use this facility. Affiliations will be numbered
% in order of appearance and this is the preferred way.
\icmlsetsymbol{equal}{*}

\begin{icmlauthorlist}
\icmlauthor{Zelai Xu}{thu}
\icmlauthor{Wanjun Gu}{thu}
\icmlauthor{Chao Yu}{thu}
\icmlauthor{Yi Wu}{thu}
\icmlauthor{Yu Wang}{thu}
\end{icmlauthorlist}

\icmlaffiliation{thu}{Tsinghua University, Beijing, China}

\icmlcorrespondingauthor{Zelai Xu}{zelai.eecs@gmail.com}
\icmlcorrespondingauthor{Chao Yu}{yuchao@tsinghua.edu}
\icmlcorrespondingauthor{Yi Wu}{jxwuyi@gmail.com}
\icmlcorrespondingauthor{Yu Wang}{yu-wang@tsinghua.com}

% You may provide any keywords that you
% find helpful for describing your paper; these are used to populate
% the "keywords" metadata in the PDF but will not be shown in the document
% \icmlkeywords{Machine Learning, ICML}

\vskip 0.3in
]

% this must go after the closing bracket ] following \twocolumn[ ...

% This command actually creates the footnote in the first column
% listing the affiliations and the copyright notice.
% The command takes one argument, which is text to display at the start of the footnote.
% The \icmlEqualContribution command is standard text for equal contribution.
% Remove it (just {}) if you do not need this facility.

\printAffiliationsAndNotice{}  % leave blank if no need to mention equal contribution
% \printAffiliationsAndNotice{\icmlEqualContribution} % otherwise use the standard text.

\begin{abstract}
Multi-agent reinforcement learning (MARL) has made significant progress, largely fueled by the development of specialized testbeds that enable systematic evaluation of algorithms in controlled yet challenging scenarios. However, existing testbeds often focus on purely virtual simulations or limited robot morphologies such as robotic arms, quadrupeds, and humanoids, leaving high-mobility platforms with real-world physical constraints like drones underexplored. To bridge this gap, we present \textbf{\textit{VolleyBots}}, a new MARL testbed where multiple drones cooperate and compete in the sport of volleyball under physical dynamics. VolleyBots features a turn-based interaction model under volleyball rules, a hierarchical decision-making process that combines motion control and strategic play, and a high-fidelity simulation for seamless sim-to-real transfer. We provide a comprehensive suite of tasks ranging from single-drone drills to multi-drone cooperative and competitive tasks, accompanied by baseline evaluations of representative MARL and game-theoretic algorithms. Results in simulation show that while existing algorithms handle simple tasks effectively, they encounter difficulty in complex tasks that require both low-level control and high-level strategy. We further demonstrate zero-shot deployment of a simulation-learned policy to real-world drones, highlighting VolleyBots’ potential to propel MARL research involving agile robotic platforms. The project page is at \url{https://sites.google.com/view/thu-volleybots/home}.
\end{abstract}

\section{Introduction}
% Multi-agent reinforcement learning (MARL) has seen significant success in a range of domains, including competitive board games such as Go~\cite{xxx}, cooperative card games like Hanabi~\cite{bard2020hanabi}, and real-time strategy challenges exemplified by the StarCraft Multi-Agent Challenge (SMAC)~\cite{samvelyan2019starcraft,ellis2024smacv2} and Google Research Football (GRF)~\cite{kurach2020google}. These testbeds have collectively pushed MARL research forward by illustrating how agents can learn cooperation, competition, and the capacity to adapt to complex, multi-agent interactions.

% Yet, many of these established MARL platforms are either purely virtual or place limited emphasis on the physical realism often found in robotic systems. An alternative direction has emerged in the form of robot sports, which can offer rich strategic and physical control requirements in tasks like multi-agent MuJoCo~\cite{peng2021facmac}, robotic table tennis~\cite{d2024achieving}, humanoid football~\cite{liu2022motor,haarnoja2024learning}, and Multi-agent Quadruped Environment (MQE)~\cite{xiong2024mqe}. Sports-based scenarios, in particular, naturally integrate aspects of teamwork, adversarial play, and complex motion control, making them powerful testbeds for studying agent decision-making in both simulation and real-world deployments.

Multi-agent reinforcement learning (MARL) has demonstrated remarkable success across diverse domains, including competitive board games such as Go~\cite{silver2016mastering}, cooperative card games like Hanabi~\cite{bard2020hanabi}, real-time strategy challenges such as the StarCraft Multi-Agent Challenge (SMAC)~\cite{samvelyan2019starcraft,ellis2024smacv2} and Google Research Football (GRF)~\cite{kurach2020google}, as well as human-AI cooperative games like Overcooked~\cite{carroll2019utility}. These testbeds have collectively propelled MARL research forward by showcasing how agents can effectively learn to cooperate, compete, and adapt within complex multi-agent interactions.

\begin{figure*}[t]
    \centering
    % \includegraphics[width=0.8\linewidth]{figs/overview.pdf}
    \includegraphics[width=0.9\linewidth]{figs/overview_b.pdf}
    \caption{Overview of the VolleyBots Testbed. VolleyBots comprises four key components: (1) Environment, supported by Isaac Sim and PyTorch, which defines entities, observations, actions, and reward functions; (2) Tasks, including 3 single-agent tasks, 3 multi-agent cooperative tasks and 2 multi-agent competitive tasks; (3) Algorithms, encompassing RL, MARL, game-theoretic algorithms; and (4) Sim-to-Real Transfer, enabling zero-shot deployment from simulation to real-world environments. }
    \label{fig:overview}
\end{figure*}

Despite the progress in MARL, most established platforms are fully simulated games that do not account for physical-world interactions. To address this gap, several MARL testbeds based on robotic platforms have been developed, such as Multi-Agent MuJoCo(MAMuJoCo)~\cite{peng2021facmac}, Bimanual Dexterous Hands (Bi-DexHands)~\cite{chen2022humanlevelbimanualdexterousmanipulation}, and Multi-agent Quadruped Environment (MQE)~\cite{xiong2024mqe}. Among these, robot sports stand out as a typical task that integrates strategic decision-making under sport-specific rules—such as teamwork and adversarial play—with complex, dynamic motion control constrained by real-world physical factors. Notable examples include robot table tennis~\cite{d2024achieving}, quadrupedal robot football~\cite{ji2022hierarchicalreinforcementlearningprecise}, and humanoid football~\cite{liu2022motor,haarnoja2024learning}. However, robot sports often lack open-source simulation environments and real-world support, limiting their effectiveness as research testbeds. Additionally, these platforms mainly focus on robotic arms, quadrupedal, and humanoid, with limited exploration of high-mobility, agile platforms like drones.

In this work, we introduce a novel MARL testbed named \textbf{\textit{VolleyBots}}, where multiple drones engage in the popular sport of volleyball. This testbed addresses gaps in existing robot sports platforms by incorporating: 
(i) a turn-based interaction model governed by volleyball rules, effectively capturing discrete offensive and defensive phases for MARL studies; 
(ii) a hierarchical decision-making process that combines low-level motion control with high-level strategic play to execute complex flight maneuvers; and
(iii) a high-fidelity simulation that models drone dynamics and interactions between the drones and the ball, and supports flexible parameter randomization, enabling seamless sim-to-real transfer.

% Fig.~\ref{fig:overview} provides an overview of the VolleyBots testbed.
The overview of the VolleyBots testbed is shown in Fig.~\ref{fig:overview}.
Built on Nvidia Isaac Sim~\cite{Mittal_2023}, VolleyBots supports GPU-based rapid data collection, making it highly suitable for RL research. Inspired by the way humans progressively learn the rules of volleyball, we designed a series of tasks ranging from single-drone drills to multi-drone cooperative plays and competitive matchups. In addition, we have implemented baseline MARL and game-theoretic algorithms and provided benchmark results for the proposed tasks. 
% The results reveal that existing algorithms struggle to solve the proposed tasks, especially the multi-agent competitive tasks that require both low-level motion control and high-level strategic play.
The simulation results reveal that existing algorithms perform competently on simple tasks like single-drone control, but struggle to solve complex tasks like multi-drone competitions that require both low-level motion control and high-level strategic play.
% \yc{check this.} 
To demonstrate real-world deployment ability, we show a policy trained to bump volleyball can be deployed on an open-source quadrotor equipped with a racket in a zero-shot manner. We envision VolleyBots as a valuable platform for studying MARL and game-theoretic algorithms with complex drone control tasks.

% In this work, we propose a new testbed named \textbf{VolleyBots}, fills a gap in existing robot sports contexts by featuring (i) a turn-based interaction (ii) a hierarchical decision process, and (iii) a focus on sim-to-real transfer. These attributes underscore the environment’s ability to capture discrete offensive and defensive turns, incorporate layered high-level strategies and low-level drone control, and align closely with real-world drone specifications and dynamics.

% Our VolleyBots framework is built on a high-fidelity simulation stack and encompasses a variety of tasks ranging from single-drone drills to multi-drone cooperative plays and competitive matchups. Figure~\ref{fig:overview} illustrates the overall setup, including the environment entities, the categories of tasks (single-agent, multi-agent cooperation, and multi-agent competition), the algorithmic baselines considered in our benchmark, and the eventual pathway to real hardware. By combining realistic flight dynamics, ball manipulation, and volleyball rules, VolleyBots aims to bridge the gap between purely virtual MARL testbeds and practical robotics applications.

Our main contributions are summarized as follows:
\begin{enumerate}
    \item We introduce VolleyBots, a multi-agent drone environment that operates under turn-based interactions, combines hierarchical decision-making, and is compatible with sim-to-real deployments.
    % \item We provide a high-fidelity simulation encompassing drone flight and ball dynamics, with multiple control interfaces ranging from thrust inputs to collective thrust and body rates.
    \item We release diverse benchmark tasks and baseline evaluations of representative MARL and game-theoretic algorithms, thereby facilitating reproducible research and comparative assessments.
    \item We show the sim-to-real feasibility of transferring learned policies to physical drones, highlighting the environment’s alignment with real-world parameters.
\end{enumerate}

% Reinforcement learning has demonstrated remarkable advances in single-agent settings across a wide variety of domains, including arcade-style tasks, continuous control benchmarks, board games, and complex esports environments. However, numerous real-world scenarios, such as autonomous driving fleets, multi-robot coordination, and collaborative drone operations, inherently involve multiple learning agents. These situations have brought attention to multi-agent reinforcement learning (MARL), which addresses the interplay of cooperation, competition, and scalability among distributed autonomous agents.

% Despite considerable progress in MARL, existing testbeds commonly exhibit limitations in three areas: (i) turn-based interactions, (ii) hierarchical decision processes, and (iii) sim-to-real transfer. Most platforms adopt mechanics based on concurrent rather than turn-based actions, which may obscure temporal dependencies and dilute clear transitions between offense and defense. In addition, although hierarchical structures are crucial in real-world tasks—where agents must handle low-level control like flight maneuvers and high-level strategy like offensive or defensive roles—many benchmarks do not explicitly support such layered decision processes. Furthermore, sim-to-real validation, which is vital for practical deployments, is often overlooked when platforms rely solely on virtual or simplified physics.

% To address these shortcomings, this work introduces Drone Volleyball, a novel MARL testbed that combines turn-based gameplay, hierarchical control requirements, and a high-fidelity physical simulator suitable for sim-to-real transfer. In this environment, multiple drones cooperate or compete under volleyball rules, navigating three-dimensional space, passing or spiking the ball, and scoring by placing the ball in the opponent’s court. Unlike concurrent-action simulations, volleyball is explicitly turn-based: once one side completes its sequence of hits, control shifts to the opposing side. This structure offers distinct intervals of offense and defense, allowing agents to anticipate opposing strategies and develop countermeasures. From a control perspective, the platform demands low-level drone maneuvers—such as motor thrust and orientation control—while also rewarding high-level tactical decisions—such as coordinating with teammates for effective spikes. By integrating carefully calibrated quadrotor dynamics and realistic collision modeling, Drone Volleyball bridges the gap between purely virtual MARL environments and real-world flight experiments, making it possible to train in simulation and subsequently deploy policies on physical drones.

% The main contributions of this work are summarized as follows:
% \begin{enumerate}
%     \item A new multi-agent testbed named Drone Volleyball that is turn-based, hierarchically challenging, and aligned with sim-to-real objectives, thus filling a critical gap in MARL environments.
%     \item A high-fidelity simulation encompassing drone flight and ball dynamics, including interfaces for low-level thrust control, trajectory-based maneuvers, and high-level volleyball tactics.
%     \item Demonstration of the feasibility of sim-to-real transfer through alignment with physical drone parameters, enabling seamless migration of trained policies to real hardware.
%     \item Release of diverse benchmark scenarios, an open-source application programming interface, and baseline performance assessments for several representative MARL algorithms to facilitate reproducible and comparable research.
% \end{enumerate}


\section{The Werewolf Game}
% In this work, we concentrate on a seven-player variant of the Werewolf game, featuring two Werewolves, one Seer, one Doctor, and three Villagers. We develop agents capable of engaging in a text-based version of the game, where all interactions are conducted through natural language without reliance on non-verbal cues such as tone or facial expressions.

\begin{figure*}[t]
    \centering
    \includegraphics[width=\linewidth]{figs/overview.pdf}
    \caption{Overview of the Latent Space Policy Optimization (LSPO) framework. Each iteration consists of three components. (1) Latent space construction: generate language actions with the LLM and cluster the vast language action into a finite latent strategy space. (2) Policy optimization in latent space: reformulate the original game as an abstracted game and apply game-theoretic methods to learn latent space policy. (3) Latent space expansion: fine-tune the LLM to align with the latent space policy and generate new strategies to expand the latent strategy space.}
    \label{fig:overview}
\end{figure*}

Werewolf is a popular social deduction game where players with hidden roles cooperate and compete with others in natural languages. The Werewolf side needs to conceal their identities and eliminate the other players, while the Village side needs to identify their teammates and vote out the Werewolves. Players are required to have both language proficiency for communication and strategic ability for decision-making to achieve strong performance in the Werewolf game. We consider a seven-player game with two Werewolves being the Werewolf side and one Seer, one Doctor, and three Villagers being the Village side.
% \gf{It's overlapped with the first sentence of the next paragraph.} 
Detailed descriptions of the game's rule, observation space, and reward function can be found in Appendix~\ref{app:game}. 
% More detailed descriptions of the game can be found in Appendix~\ref{app:game}. 
% \gf{It would be better to say what additional details are in the appendix based on the next section.}

\subsection{Game Environment}

We consider a text-based seven-player Werewolf game that proceeds through natural languages. We exclude other information like the speaking tone, facial expression, and body language~\cite{lai2022werewolf}. This pure text-based environment is a common setup in the literature~\cite{xu2023exploring,xu2023language,wu2024enhance,bailis2024werewolf}. 

\textbf{Roles and Objectives.} 
At the beginning of each game, the seven players are randomly partitioned into two sides. The Werewolf side has two Werewolf players who know each other's role and aim to eliminate the other players while avoiding being discovered. The Village side has one Seer who can check the role of one player each night, one Doctor who can protect one player each night, and three Villagers without any ability. The players in the Village side only know their own role and need to share information to identify the Werewolves and vote them out.


\textbf{Game Progression.} 
The game proceeds by alternating between night round and day round. In the night round, players can perform secret actions that are only observable by themselves. More specifically, the two Werewolves can choose a target player to eliminate, the Seer can choose a target player to investigate whether the player's role is Werewolf, and the Doctor can choose a target player to protect the player from being eliminated. The Doctor does not know the target player chosen by the Werewolves. If the Doctor chooses the same target player as the Werewolves, then no player is eliminated in this night round, otherwise, the Doctor fails to protect any player, and the target chosen by the Werewolves is eliminated.

\textbf{Observations and Actions.} 
The language observation of each agent is a list of natural languages that log the game history to the current step. This list include both private information that are only observable to the current player and public information that are shared by all players. The private information includes the role of the current player, the secret actions in the night round for the Werewolf, Seer, and Doctor, and the teammate for the Werewolf. The public information includes the ID of the current player, the eliminated player in each night and day round, the discussion, and the voting result in each day round. 
% \gf{One example observation input is illustrated in xxx.}

Player actions are also in the form of natural language and can be categorized into three types: secret actions, which are secret actions performed during the night, such as choosing a target player to eliminate, investigate, or protect; discussion actions, which are statements made during the day to influence other players' perceptions and decisions; and voting actions, which are choices made during the voting round to vote for on player or choose not to vote.


\subsection{Challenges for Language Agents}

Unlike board, card, or video games with a finite set of actions, Werewolf has a free-form language action space. The vast space of natural language actions poses two key challenges for language agents to achieve strong performance in the Werewolf game.
% \gf{intrinsic bias and unbounded action coverage. However, I think intrinsic bias issue is not caused by the vast space but from the training data? (Ignore it if I'm wrong...)}

\textbf{Intrinsic Bias in Action Generation.}
As observed in simple games like Rock-Paper-Scissor~\cite{xu2023language}, pure LLM-based agents tend to exhibit intrinsic bias in their action generation, which is inherited from the model's pre-training data. 
This issue is more pronounced in complex language games like Werewolf, where the opponents can exploit these predictable biases to counteract the agent's move. Therefore, mitigating intrinsic bias is essential for language agents to reduce exploitation and achieve strong performance.

\textbf{Coverage of Unbounded Action Space.}
Due to the immense combinatorial space induced by free-form text, it is impractical to map every possible utterance to an action in the language space. On the other hand, manually engineering or prompting an LLM to produce a limited set of actions may fail to capture the full strategic landscape. Even if an agent optimally masters the action distribution within a limited subset, it could be easily exploited by out-fo-distribution utterance. Consequently, inadequate coverage of the action space could result in suboptimal performance in free-form language games like Werewolf.



\section{Latent Space Policy Optimization}
\subsection{Hyperparameters}

For latent space construction, we let the LLM agent play $1000$ games to collect all discussion actions generated by each role in these games. For diverse action generation, we prompt the LLM to generate $3$ action candidates and randomly select one to execute in the game. We pair the language observation with the $3$ action candidates to use for preference-based fine-tuning in the following components. For sentence embedding, we use OpenAI's ``text-embedding-3-small'' embedding API to embed the sentence to a vector of $1536$ dimensions. Then we apply standard $k$-means clustering to cluster the embedding and get the discrete latent strategy space. The number of clusters $k$ in the first iteration is $3$ for the Werewolf and $2$ for the Seer, Doctor, and Villagers. In each iteration, we add $1$ cluster to the existing clusters. That is, if the first iteration has $k$ clusters, then the $i$-th iteration has $k + i - 1$ clusters.

For policy optimization in latent space, we use a learning rate of $1\times10^{-3}$ to train a Deep CFR network. The buffer size of each role's model is $5\times10^5$, and each model is trained for $1500$ iterations with batch size $4096$ using the Adam optimizer. 

For latent space expansion, we apply DPO with $\beta=0.1$, learning rate $1\times10^{-6}$, and trained for $2$ epoch with batch size $64$.


\subsection{Counterfactual Regret Minimization}

Counterfactual Regret Minimization (CFR) (\cite{zinkevich2007regret}) is a self-play algorithm, and each player continuously updates their strategies according to regret matching to achieve a Nash equilibrium.

We use the following notation. $Z$ is the set of all the end states $z$. $h\sqsubset z$ means state $h$ is a prefix of state $z$, that is, $z$ can be achieved from $h$. $\pi_p^\sigma$ is the probability contribution of the player $p$, and $\pi^\sigma = \prod_p \pi_p^\sigma$. $\pi_{-p}^\sigma$ is the probability contribution of all players except player $p$. $u_p(z)$ is the utility function for the player $p$ in the state $z$.

Counterfactual value for a state $h$ and a player $p$ according to startegy $\sigma$ is defined as:
\begin{equation}
    v_{p}^{\sigma}(h) = \sum_{z\in Z, h \sqsubset z} \pi^\sigma_{-p}(h)\pi^\sigma(z|h)u_p(z).
\end{equation}
The regret for a action $a$ in state $h$ for player $p$ is defined as: $v_p^{\sigma|_{h\to a}}(h) - v_{p}^{\sigma}(h)$, where $\sigma|_{h\to a}$ is same to $\sigma$ except in state $h$ the player will choose action $a$. The regret matching is choosing the strategy according to sum of previous regret values defined as $R(h,a)$, then the new strategy $\sigma(h,a) = \frac{R(h,a)^+}{\sum_{a'} R(h,a')^+}$, $R(h,a)^+ = \max(0,R(h,a))$. If $\sum_{a'} R(h,a')^+ =0$, just set $\sigma$ to be uniform random.

Because the game tree is very big, it is impossible to traverse the entire tree,  our implementation is based on deep CFR (\cite{brown2019deep}). We use a neural network to fit observation to regret value. The amount of computation required to search for only one player is also unacceptable, so a restriction is added based on deep CFR. If the number of layers currently searched is too large, the previous strategy is directly used to sample the actions of all players until the end of the game and return the utility for each player in that state. The complete process can be seen as running some complete game trajectories, and then starting from each intermediate node, searching a few layers to do CFR.

\subsection{Baseline Implementation}
ReAct, ReCon, and SLA are implemented following the original paper. The Cicero-like agent predefines a set of high-level atomic actions and trains an RL policy with this fixed action space. 
The RL policy takes the embeddings of the information record and deduction result as input and selects the atomic action based on this input.
Then the natural language actions used in gameplay are generated by prompting the LLM to follow the selected atomic actions.
In our case, the atomic action set consists of 13 actions including ``idle'', ``target player\_0'', ``target player\_1'', ``target player\_2'', ``target player\_3'', ``target player\_4'', ``target player\_5'', ``target player\_6'', ``claim to be a Werewolf'', ``claim to be a Seer'', ``claim to be a Doctor'', ``claim to be a Villager'', and ``do not reveal role''. 


\section{Experiments}
\begin{table}[t]
    \centering
    \begin{tabular}{l c | l c | l c}
    \toprule
    hyperparameters & value & hyperparameters & value & hyperparameters & value \\
    \midrule
    critic lr & $5\times10^{-4}$ & actor lr & $5\times10^{-4}$ & share actor & true \\
    optimizer & Adam & actor and critic network & MLP & MLP hidden sizes & $[256,128,128]$ \\
    max grad norm & $10$ & buffer length & $64$ & buffer size & $6\times10^6$\\
    batch size & $16384$ & gamma & $0.95$ & exploration noise & $0.1$ \\
    tau & $0.005$ & target update interval & $4$ & critic loss & smooth L1 \\
    max episode length & $800$ & num envs & $4096$ & train steps & $1\times10^{9}$ \\
    \bottomrule
\end{tabular}
\caption{Hyperparameters used for (MA)DDPG in single-agent tasks and multi-agent tasks.}
    \label{tab:app:maddpg_hyperparameters}
\end{table}

\subsection{Hyperparameters of Benchmarking Algorithms}

\subsubsection{Single-Agent Tasks.}
% \yc{@xiangmin} \done
The hyperparameters adopted for DDPG and PPO in the single-agent tasks are listed in Table~\ref{tab:app:maddpg_hyperparameters}-\ref{tab:app:multi_different_hyperparameters}. All algorithms are trained for $5\times10^{8}$ environment steps in each task. 

\subsubsection{Multi-Agent Cooperative Tasks}
% \yc{@xiangmin} \done
The hyperparameters adopted for different algorithms in multi-agent cooperative tasks are listed in Table~\ref{tab:app:maddpg_hyperparameters}-\ref{tab:app:multi_different_hyperparameters}. All algorithms are trained for $1\times10^{9}$ environment steps in each task. 

\begin{table}[t]
    \centering
    \begin{tabular}{l c | l c | l c}
    \toprule
    hyperparameters & value & hyperparameters & value & hyperparameters & value \\
    \midrule
    optimizer & Adam & max grad norm & $10$ & entropy coef & $0.001$\\
    buffer length & $64$ & num minibatches & $16$ & ppo epochs & $4$ \\
    value norm & ValueNorm1 & clip param & $0.1$ & normalize advantages & True\\
    use huber loss & True & huber delta & $10$ & gae lambda & $0.95$ \\
    use orthogonal & True & gain & $0.01$ & gae gamma & $0.995$ \\
    max episode length & $800$ & num envs & $4096$ & train steps & $1\times10^{9}$ \\
    \bottomrule
\end{tabular}
\caption{Common hyperparameters used for (MA)PPO, HAPPO, and MAT in single-agent tasks and the multi-agent tasks.}
    \label{tab:app:multi_hyperparameters}
\end{table}

\begin{table}[t]
    \centering
    \begin{tabular}{l | c c c}
    \toprule
    Algorithms & (MA)PPO & HAPPO & MAT \\
    \midrule
        actor lr & $5\times10^{-4}$ & $5\times10^{-4}$ & $3\times10^{-5}$ \\
        critic lr & $5\times10^{-4}$ & $5\times10^{-4}$ & $3\times10^{-5}$  \\
        share actor & True & False & / \\
        hidden sizes & $[256,128,128]$ & $[256,128]$ & $[256,256,256]$ \\
        num blocks & / & / & $3$ \\
        num head & / & / & $8$ \\

    \bottomrule
\end{tabular}
\caption{Different hyperparameters used for (MA)PPO, HAPPO, and MAT in the single-agent tasks and multi-agent tasks.}


    \label{tab:app:multi_different_hyperparameters}
\end{table}

\subsubsection{Multi-Agent Competitive Tasks.}
% \yc{@huining, ruize} \done

\paragraph{Training.}
For self-play (SP) in \textit{1 vs 1} and \textit{3 vs 3} competitive tasks, we adopt the MAPPO algorithm with shared actor networks and shared critic networks between two teams, in order to make sure two teams utilize the same policy. Also, we transform the samples from both sides into symmetric ones and then use these symmetric samples to update the network together. The hyperparameters employed here are the same as those used in the MAPPO algorithm for multi-agent cooperative tasks.

The PSRO algorithm for \textit{1 vs 1} competitive task instantiates a PPO agent for training one of the two drones while the other drone maintains a fixed policy. Similarly, the PSRO algorithm for the \textit{3 vs 3} task assigns each team to be controlled by MAPPO. We adopt the same set of hyperparameters listed in Table \ref{tab:app:multi_different_hyperparameters} for the (MA)PPO agent. In each iteration, the (MA)PPO agent is trained against the current population. Here, we offer two versions of meta-strategy solver, PSRO\textsubscript{Uniform} and PSRO\textsubscript{Nash}. Training is considered converged when the agent achieves over 90\% win rate with a standard deviation below 0.05. The iteration ends when the agent reaches convergence or reaches a maximum of iteration steps of 5000. The trained actor is then added to the population for the next iteration. 

For Fictitious Self-Play (FSP) in competitive tasks, we slightly modify PSRO\textsubscript{Uniform} so that in each iteration, the (MA)PPO agent inherits the learned policy from the previous iteration as initialization. Naturally, other hyperparameters and settings remain the same for a fair comparison.

In both \textit{1 vs 1} and \textit{3 vs 3} tasks, the algorithm leverages 2048 parallel environments to accelerate the training process. In this work, we report the results of different algorithms given a total budget of $1\times 10^{9}$ environmental steps.

% \paragraph{Evaluation in Multi-Agent Competitive Tasks}
% \yc{@huining, ruize} \done
\paragraph{Evaluation.}
The evaluation of exploitability requires evaluating the payoff of the best response (BR) over the trained policy or population from different algorithms. Here, we approximate the BR to each policy or population by learning an additional RL agent against the trained policy or population. In practice, this is done by performing an additional iteration of PSRO, where the opponent is fixed as the trained policy/population. In order to approximate the ideal BR as closely as possible, we initialize the BR policy with the latest FSP policy, given that FSP yields the best empirical performance in our experiments. We train the BR policy for $5000$ training step with $2048$ parallel environments. We disable the convergence condition for early termination and report the evaluated win rate to calculate the approximate exploitability. Importantly, to approximate the BR of the trained SP policy in the \textit{3 vs 3} task, we employ two distinct BR policies for the serve and rally scenarios, respectively. For the BR to serve, we directly use the latest FSP policy without further training, while for the BR to rally, we train a dedicated policy against the SP policy. The overall win rate of this BR is then computed as the average win rate across these two scenarios, given that each side has an equal serve probability of $0.5$.

% \xzl{@ruize evaluation parameters of cross-play.} \done
We run 1,000 games for each pair of policies to generate the cross-play win rate heatmap, covering 6 matchup scenarios, resulting in a total of 6,000 games. In each game, both policies are sampled from their respective policy populations based on the meta-strategy and play until a winner is determined.

Moreover, we use an open-source Elo implementation~\cite{HankSheehan_EloPy}. The coefficient $K$ is set to $168$, and the initial Elo rating for all policies is $1000$. We conduct $12000$ games among four policies. The number of games played between any two policies is guaranteed to be the same. Specifically, in each round, $6$ different matchups are played. Each policy participates in 3 matchups, competing against different opponent policies. A total of $2000$ rounds are carried out, amounting to $12000$ games in total. The game results are sampled and generated based on the cross-play results.

\subsection{Results of Single-Agent Tasks}
\label{app:single}
% \yc{@xiangmin, put training curves here.} \done 

% \yc{@xiangmin add some analysis.} \done

The training curves of DDPG and PPO in single-agent tasks are shown in Fig.~\ref{fig:single_tasks}. It can be seen that PPO outperforms DDPG in all tasks. For example, in Back and Forth, using PPO, the number of times the agent reaches the target anchor stabilizes around $9-10$, and the task is successfully completed. In contrast, with DDPG, the agent only completes half of the return motion, as the drone ends up flying out of bounds. In Hit the Ball, with PPO, the hitting distance stabilizes around $10-11\,m$, and the ball is almost always hit in a straight line. However, with DDPG, the landing spot of the ball is less controllable. In Solo Bump, with PPO, the ball juggles smoothly, reaching a height of $4\,m$, while DDPG almost fails to juggle properly and can only manage a single hit. CTBR and PRT are comparable. in Back and Forth and Hit the Ball PRT's final result is better, in Solo Bump is comparable but CTBR learns faster.

% \xzl{e.g. (1) PPO is much better than DDPG in all tasks, list some number in each task. describe the behaviors of different algorithms in the video.
% (2) CTBR and PRT are comparable. in Back and Forth and Hit the Ball PRT's result is better, in xxx is comparable but CTBR learns faster, balabala. describe the behaviors of different action space in the video}


\begin{figure}[t]
    \centering
    \subfloat[\textit{Back and Forth}]{
        \centering
        \includegraphics[width=0.3\linewidth]{figs/single_tasks/bnf.pdf}}
    % \hspace{0.03\linewidth}
    \subfloat[\textit{Hit the Ball}]{
        \centering
        \includegraphics[width=0.3\linewidth]{figs/single_tasks/hit.pdf}} 
    \subfloat[\textit{Solo Bump}]{
        \centering
        \includegraphics[width=0.3\linewidth]{figs/single_tasks/bump.pdf}}
    % \hspace{0.03\linewidth}
    \caption{Training curves of single-agent tasks over three seeds.}
    \label{fig:single_tasks}
\end{figure}

\begin{figure}[t]
    \centering
    \subfloat[\textit{Bump and Pass} w.o. shaping]{
        \centering
        \includegraphics[width=0.3\linewidth]{figs/multi_tasks/bmp_wo_shaping.pdf}}
    \subfloat[\textit{Set and Spike (Easy)} w.o. shaping]{
        \centering
        \includegraphics[width=0.3\linewidth]{figs/multi_tasks/sns_easy_wo_shaping.pdf}}
    \subfloat[\textit{Set and Spike (Hard)} w.o. shaping]{
        \centering
        \includegraphics[width=0.3\linewidth]{figs/multi_tasks/sns_hard_wo_shaping.pdf}}
    % 第二行子图
    \par
    \subfloat[\textit{Bump and Pass} w. shaping]{
        \centering
        \includegraphics[width=0.3\linewidth]{figs/multi_tasks/bmp_w_shaping.pdf}}
    \subfloat[\textit{Set and Spike (Easy)} w. shaping]{
        \centering
        \includegraphics[width=0.3\linewidth]{figs/multi_tasks/sns_easy_w_shaping.pdf}} 
    \subfloat[\textit{Set and Spike (Hard)} w. shaping]{
        \centering
        \includegraphics[width=0.3\linewidth]{figs/multi_tasks/sns_hard_w_shaping.pdf}} 
    \caption{Training curves of multi-agent cooperative tasks over three seeds.}
    \label{fig:multi_tasks}
\end{figure}


\subsection{Results of Multi-Agent Cooperative Tasks}
\label{app:multi}
% \yc{@xiangmin, put training curves here.} \done
The training curves of different algorithms in multi-agent tasks are shown in Fig.~\ref{fig:multi_tasks}. It can be seen that MAPPO, HAPPO, and MAT are able to complete the tasks relatively well, while MADDPG can only make the setter complete one hit, with the attacker unable to receive the ball. In Bump and Pass w.o. shaping, MAPPO maintains a fast training process and outperforms HAPPO and MAT in terms of final results. The same is true in Set and Spike (Easy and Hard) w.o. shaping. In Set and Spike (Easy) w. shaping, MAPPO, and HAPPO perform similarly, with slightly faster learning speeds than MAT. In Set and Spike (Hard) w. shaping task, HAPPO only succeeds in overcoming the defense racket in one seed, resulting in a success rate of approximately $0.82$. At this point, the ball can be quickly spiked into the opponent's court and the attacker doesn't touch the net. Overall, the success rate is slightly higher than MAPPO and MAT, as these two algorithms almost never successfully overcome the defense racket.

Additionally, we can observe that the presence of shaping rewards has a significant impact on task results. Adding shaping rewards clearly improves the performance and accelerates the learning process. In Bump and Pass, the task learns slower without shaping rewards because the policy must explore which direction to hit the ball, requiring many more steps. The hit direction reward in shaping rewards accelerates this process. In Set and Spike (Easy), all algorithms without shaping rewards have a success rate of only $0.25$ because they only learn to make the setter hit the ball, but not toward the hitter. As a result, the attacker fails to hit the ball. The hit direction reward in shaping rewards also helps accelerate this process. In Set and Spike (Hard), HAPPO achieves a success rate of about $0.82$, while MAPPO and MAT are slightly worse, at $0.75$, because the defense racket is strong and successfully intercepts their attacks.

% \xzl{one paragraph for each task, in this paragraph (1) compare different algorithms (which is better or comparable...) (2) discuss the difference between w. and w.o. shaping, and their behaviors in the video. for example, in Bump and Pass, why do all algorithms w.o. shaping learn slower than w. shaping (because w.o. shaping reward the policy has to explore which direction to hit the ball, this will take many more steps. But the hit direction reward in shaping reward can help accelerate this process). In Set and Spike (Easy), why do all algorithms w.o. shaping reward only has a success rate of 0.25? (because they only learn to let the setter hit the ball, but not towards the hitter.) In Set and Spike (Hard), why do all algorithms with reward shaping only have a success rate of 0.75? (Because the defense board is strong and successfully intercepts their attack), ...}

% \yc{@xiangmin add some analysis.} \done


% \begin{figure}[t]
%     \centering
%     \subfloat[Bump and Pass w.o. shaping Reward]{
%         \centering
%         \includegraphics[width=0.45\linewidth]{figs/multi_tasks/Bump and Pass w.o. shaping Reward.png}}
%     \subfloat[Bump and Pass w. shaping Reward]{
%         \centering
%         \includegraphics[width=0.45\linewidth]{figs/multi_tasks/Bump and Pass w. shaping Reward.png}} 
%     % 第二行子图
%     \par
%     \subfloat[Set and Spike (Easy) w.o. shaping Reward]{
%         \centering
%         \includegraphics[width=0.45\linewidth]{figs/multi_tasks/Set and Spike (Easy) w.o. shaping Reward.png}}
%     \subfloat[Set and Spike (Easy) w. shaping Reward]{
%         \centering
%         \includegraphics[width=0.45\linewidth]{figs/multi_tasks/Set and Spike (Easy) w. shaping Reward.png}} 
%     % 第三行子图
%     \par
%     \subfloat[Set and Spike (Hard) w.o. shaping Reward]{
%         \centering
%         \includegraphics[width=0.45\linewidth]{figs/multi_tasks/Set and Spike (Hard) w.o. shaping Reward.png}}
%     \subfloat[Set and Spike (Hard) w. shaping Reward]{
%         \centering
%         \includegraphics[width=0.45\linewidth]{figs/multi_tasks/Set and Spike (Hard) w. shaping Reward.png}} 
%     \caption{Training curves of multi-agent tasks.}
%     \label{fig:multi_tasks}
% \end{figure}

\subsection{Results of Multi-Agent Competitive Tasks}
\label{app:mix}
We provide a more detailed win rate evaluation of the PSRO populations from the \textit{1 vs 1} task in Fig.~\ref{fig:1v1heatmap}, where each policy in the PSRO population is evaluated against all other policies. In these heatmaps, the ordinate and abscissa represent the policy for drone 1 and drone 2 respectively. The heat of cells represents the evaluated win rate of drone 1, i.e. red means a higher win rate and blue means a lower win rate. Intuitively, each row represents a policy's performance against each policy of the population while playing as drone 1. A red cell indicates that the drone 1 policy outperforms the specific drone 2 policy. A full red row means that the policy outperforms all other policies.

% \xzl{@huining add more description to help the reader understand this map, for example, red means higher win rate and blue means lower win rate. For each row, more red cells mean outperforming more policies. A full red row means the policy outperforms all other policies... }\done

Evidently, FSP attains more iterations than PSRO\textsubscript{Uniform} and PSRO\textsubscript{Nash} given a budget of $1\times10^9$ steps, which yields a faster convergence speed. This advantage comes from the fact that FSP inherits the learned policy from the previous iteration, which serves as an advantageous initialization for the current iteration. In contrast, PSRO\textsubscript{Uniform} and PSRO\textsubscript{Nash} start from scratch in each iteration, which poses a challenge for the algorithm to converge and introduces more variance in the training process.

Moreover, in PSRO algorithms, as the learned policy gradually improves with each iteration, the most recent policy of the population naturally poses greater difficulty for subsequent iterations. Therefore, PSRO\textsubscript{Nash} tends to put more weight on the most recent policy in the meta-strategy. 
% \yc{explain how we can see this?}\done 
This in turn has an effect on the learning of new policies. We can observe the outcomes in the heatmaps: for each row, the win rate against the most recent policy is often higher than the others. In FSP, on the other hand, the win rate against each policy is more evenly distributed, indicating that the population is potentially more balanced and stable.


\begin{figure}[t]
    \centering
    % 第一行子图
    \subfloat[FSP]{
        \centering
        \includegraphics[width=0.25\textwidth]{figs/1v1_heatmap/fsp_seed_300.png}
    }
    \subfloat[PSRO\textsubscript{Uniform}]{
        \centering
        \includegraphics[width=0.25\textwidth]{figs/1v1_heatmap/uniform_seed_300.png}
    }
    \subfloat[PSRO\textsubscript{Nash}]{
        \centering
        \includegraphics[width=0.25\textwidth]{figs/1v1_heatmap/nash_seed_300.png}
    }
    % % 第二行子图
    % \par
    % \subfloat[PSRO\textsubscript{Uniform} seed=1]{
    %     \centering
    %     \includegraphics[width=0.25\textwidth]{figs/1v1_heatmap/uniform_seed_100.png}
    % }
    % \subfloat[PSRO\textsubscript{Uniform} seed=2]{
    %     \centering
    %     \includegraphics[width=0.25\textwidth]{figs/1v1_heatmap/uniform_seed_200.png}
    % }
    % \subfloat[PSRO\textsubscript{Uniform} seed=3]{
    %     \centering
    %     \includegraphics[width=0.25\textwidth]{figs/1v1_heatmap/uniform_seed_300.png}
    % }
    % % 第三行子图
    % \par
    % \subfloat[PSRO\textsubscript{Nash} seed=1]{
    %     \centering
    %     \includegraphics[width=0.25\textwidth]{figs/1v1_heatmap/nash_seed_100.png}
    % }
    % \subfloat[PSRO\textsubscript{Nash} seed=2]{
    %     \centering
    %     \includegraphics[width=0.25\textwidth]{figs/1v1_heatmap/nash_seed_200.png}
    % }
    % \subfloat[PSRO\textsubscript{Nash} seed=3]{
    %     \centering
    %     \includegraphics[width=0.25\textwidth]{figs/1v1_heatmap/nash_seed_300.png}
    % }
    \caption{Win rate heatmaps of the population in the \textit{1 vs 1} task.}
    \label{fig:1v1heatmap}
\end{figure}

\subsection{Low-level Skills of Hierarchical Policy}
\label{app:low}
% \yc{@ruize, write low-level skills} \done

Low-level skills are derived through PPO training, while the high-level skill is implemented as a rule-based, event-driven policy that determines which drone utilizes which skill in response to the current game state. In accordance with the \textit{3 vs 3} task setting, each team consists of three drones positioned as front-left, front-right, and backward within their half of the court. Below, we describe each low-level skill and explain when it is utilized by the high-level policy. 

\paragraph{Hover.}
The \textit{Hover} skill is designed to enable the drone to hover around a specified target position. This skill takes a three-dimensional target position as input. The skill is frequently utilized by the high-level policy. For instance, in the serve scenario, only the serving drone uses the \textit{Serve} skill, while the other two teammates use the \textit{Hover} skill to remain at their respective anchor points.

\paragraph{Serve.}
The \textit{Serve} skill is designed to enable the drone to serve the ball towards the opponent’s side of the court. This skill includes a one-hot target input that determines whether to serve the ball to the left side or the right side of the opponent’s court.
In accordance with the \textit{3 vs 3} task setting, for the \textit{Serve} skill, the ball is initialized at a position $3\,\mathrm{m}$ directly above the serving drone, with zero initial velocity. 
This skill is exclusively utilized by the high-level policy during the serve scenario, during which the designated serving drone employs the \textit{Serve} skill at the start of a match.

\paragraph{Pass.}
The \textit{Pass} skill is designed to handle the opponent’s serve or attack by allowing the drone to make the first contact of the team’s turn and pass the ball to a teammate. Here, this skill is exclusively used by the backward drone, which is responsible for hitting the ball to the front-left teammate. The high-level policy designates the backward drone to utilize this skill whenever the opponent hits the ball.

\paragraph{Set.}
The \textit{Set} skill is designed to transfer the ball from the passing drone to the attacking drone, serving as the second contact in the team’s turn. In our design, the front-left drone utilizes the \textit{Set} skill to pass the ball from the backward drone to the front-right drone. The high-level policy designates the front-left drone to utilize this skill whenever the backward drone successfully makes contact with the ball.

\paragraph{Attack.}
The \textit{Attack} skill is designed to hit the ball towards the opponent’s court, serving as the third and final contact in the team’s turn. This skill includes a one-hot target input that specifies whether to direct the ball to the left side or the right side of the opponent’s court. In our design, the front-right drone uses the \textit{Attack} skill to strike the ball after receiving it from the front-left drone. The high-level policy assigns the front-right drone to utilize this skill whenever the front-left drone successfully hits the ball.


\subsection{Sim-to-Real}
% \yc{@shilong} \done

The sim-to-real gap presents a significant challenge in RL-based robotic control, as policies trained in simulation often underperform when deployed in the real world. We utilize the \textit{Solo Bump} task as an initial demonstration to showcase the potential of transferring trained policies to real-world scenarios. To bridge the sim-to-real gap, we apply system identification to accurately model the ball's behavior during its impact with the racket.

The impact between the ball and the racket is modeled as an impulse acting in the direction of the normal of the racket $\bm{n}_c$, which means that the tangential velocity of the ball relative to the racket remains constant but the normal velocity changes, determined by the restitution coefficient. Since the mass of the racket and vehicle (rigidly mounted together) is much larger than that of the ball, it is assumed that the velocity of the racket remains unaffected. 
Denoting the ball's pre- and post-impact normal velocities relative to the racket as $\bm{v}_{ball\_pre\_n}$ and $\bm{v}_{ball\_post\_n}$, the restitution coefficient can be derived as follows:

\begin{align}
\beta &= -\frac{\bm{v}_{ball\_post\_n}^T\bm{n}_c}{\bm{v}_{ball\_pre\_n}^T\bm{n}_c}
\end{align}

Specifically, the restitution coefficient was measured through multiple experiments in which the racket and vehicle were fixed on the flat ground and the ball started a free fall from different heights right above the center of the racket. The pre-impact and post-impact velocity of the ball was collected using a motion capture system and the restitution coefficient was calculated using the above equation. We used the same ball and racket in all experiments.

The average restitution coefficient was $0.8$, which we used in the simulation. The highest and lowest were $0.85$ and $0.75$, respectively, and the deviation was relatively small and acceptable. Experiments also showed that the ball’s tangential velocity did change during impact, which was likely due to the friction between the ball and the racket. In addition, the asymmetric relative position from racket strings to the ball could lead to asymmetric force upon the ball, causing it to move tangentially. We model the change of tangential velocity of the ball during impact as a stochastic progress and utilize small randomization in the ball's rebound velocity after each collision in the simulation.


\section{Related Work}
\textbf{Large Language Model-Based Agents.}

% \paragraph{Agents Powered by Large Language Models.}
Recent advancements in large language models (LLMs) have led to the development of agents capable of performing complex tasks across various domains, such as web interactions~\cite{nakano2021webgpt,yao2022webshop,deng2023mind2web}, code generation~\cite{chen2021evaluating,yang2024swe}, gaming environments~\cite{huang2022language,wang2023describe,wang2023voyager,ma2023large}, real-world robotics~\cite{ahn2022can,huang2022inner,vemprala2023chatgpt}, and multi-agent systems~\cite{park2023generative,li2023camel,chen2023agentverse}.
A common approach in these works is to exploit the reasoning capabilities and in-context learning of LLMs to improve decision-making processes.
Chain-of-Thought (CoT) prompting~\cite{wei2022chain} has been instrumental in enabling LLMs to perform step-by-step reasoning.
Building upon this, ReAct~\cite{yao2022react} synergizes reasoning and action to enhance performance across various tasks.
Subsequent research has incorporated self-reflection~\cite{shinn2023reflexion} and strategic reasoning~\cite{gandhi2023strategic} to further refine agent behaviors.
However, these methods can still suffer from the intrinsic biases and coverage issue of LLM-based agents, leading to suboptimal performance in complex games. A representative method that addresses these issues in the game of Diplomacy is Cicero~\cite{meta2022human}, which first uses a strategic module to produce action intent and then generates action-conditioned natural languages with a dialogue module. However, Diplomacy is a board game with finite action space and does not have the action coverage issue, making it not suitable for free-form language games with unbounded text action space.

Due to the high demand for both advanced communication skills and strategic reasoning, social deduction games like Werewolf and Avalon have been proposed as testbeds to build language agents with strategic ability. 
Earlier attempts to create agents for these games often rely on predefined protocols or limited communication capabilities~\cite{wang2018application}, restricting their effectiveness.
% DeepRole~\cite{serrino2019finding} utilizes CFR with deductive reasoning to play Avalon but does not consider natural language communication.
Recent works have explored using LLMs to enable natural language interactions in these games.
For instance, \citet{xu2023exploring} developed a prompt-based Werewolf agent that uses heuristic information retrieval and experience reflection.
Similarly, ReCon~\cite{wang2023avalon} introduced a prompt-based method for playing Avalon by considering both the agent's perspective and that of opponents.
However, these LLM-based agents may still be restricted by intrinsic bias and limited coverage of action space, affecting their decision-making quality.
Strategic Language Agent (SLA)~\cite{xu2023language} partially solves these issues by generating diverse action candidates and learning an RL policy to mitigate intrinsic bias. However, this method still relies on a fixed LLM to produce the action candidate, which can fail to address the coverage issue. Our approach mitigate the intrinsic bias by applying game-theoretic methods to optimize policy in a discrete latent space and tackles the coverage issue by iteratively expanding the latent space by aligning the LLM to the latent space policy, leading to strong performance in the Werewolf game.


% \gf{It seems your ICML24 paper is missing here. I'm not sure if there are others that should be discussed as well.}
% However, these LLM-based agents may still be limited by the biases inherent in language models, affecting their decision-making quality. \gf{Should we clarify the concept of “intrinsic bias” in the context of social deduction games?}

\textbf{Game-Theoretic Algorithms.}
Counterfactual Regret Minimization (CFR)~\cite{zinkevich2007regret} is a foundational algorithm for solving imperfect-information games, particularly those involving hidden information and strategic deception like poker~\cite{moravvcik2017deepstack,brown2018superhuman,brown2019superhuman}. The core principle of CFR is to iteratively reduce regret across players’ decision points in the game tree, converging toward strategies that approximate a Nash equilibrium. Subsequent refinements of CFR~\cite{lanctot2009monte,tammelin2014solving,brown2019deep} have expanded its scalability and adaptability to a broader range of scenarios. Of particular note is DeepRole~\cite{serrino2019finding}, which integrates deductive reasoning with CFR to play the social deduction game Avalon without communication. Our method combines CFR with language models by introducing a finite latent strategy space to enable it to solve free-form language games.


Reinforcement learning (RL) methods, on the other hand, have reached remarkable achievements in complex domains like Go~\cite{silver2016mastering,silver2018general} and video games~\cite{vinyals2019grandmaster,berner2019dota}, often surpassing expert human performance. A seminal technique in these successes is self-play and its variants\citep{heinrich2015fictitious,heinrich2016deep,hennes2020neural,xu2023fictitious}, where agents repeatedly train against older versions of themselves to refine their policies. Another prominent line of work is Policy-Space Response Oracles (PSRO)~\citep{lanctot2017unified,muller2019generalized}, an iterative procedure that produces best responses to a growing population of policies in a meta-game. Conceptually, our iterative framework is related to PSRO in that we both solve an abstracted game before enlarging it to approach the full original game. The difference is that PSRO treats newly learned policies as meta-actions to form a normal-form meta-game, whereas our approach clusters free-form language actions into a discrete latent action space to reformulate the original game as an extensive-form game with finite action space.
% Reinforcement learning has achieved remarkable success in games with imperfect information, such as Go~\cite{silver2016mastering,silver2018general}, poker~\cite{moravvcik2017deepstack,brown2019superhuman}, and complex video games~\cite{vinyals2019grandmaster,berner2019dota}.
% Techniques like self-play~\cite{heinrich2015fictitious,heinrich2016deep,hennes2020neural,xu2023fictitious} and population-based training~\cite{lanctot2017unified,muller2019generalized} have been pivotal in these achievements.
% Counterfactual Regret Minimization (CFR)~\cite{zinkevich2007regret,lanctot2009monte,tammelin2014solving,brown2019deep} is another powerful method that iteratively minimizes regret to approximate Nash equilibria in extensive-form games.
% While previous works have applied these methods to environments with well-defined action spaces, integrating them with LLMs to handle the unbounded 
% \gf{natural language actions} %action space of natural language 
% in games like Werewolf 
% \gf{remains a significant challenge.} %presents new challenges.
% \gf{To address this,} our approach leverages CFR in conjunction with LLMs to generate optimal action policies within the complex dynamics of the Werewolf game. \gf{I think this part is about the action coverage issue. It would be better to use a consistent terminology.}



\section{Conclusion}
In this work, we presented Latent Space Policy Optimization (LSPO), an iterative framework that combines structured game-solving techniques with the expressive power of large language models to build strategic language agents in free-form social deduction games. By abstracting unconstrained language action space into a discrete latent strategy space, our approach enables efficient CFR in the latent space to overcome intrinsic bias and learn strong strategies. We then perform iterative fine-tuning via DPO to align the LLM's language generation with the evolving strategy and expand the latent strategy space to address the action coverage issue. Our extensive evaluation in the Werewolf game demonstrates that LSPO not only addresses intrinsic biases and action coverage issues inherent in prompt-based agents, but also achieves increasing performance with respect to iterations and outperforms four state-of-the-art baseline agents. Looking ahead, we envision LSPO’s synergy of latent-space abstraction and preference-based language alignment can be extended to a variety of other complex decision-making tasks with free-form language actions.

% multi-agent and human-in-the-loop settings, paving the way for more robust and versatile language-grounded decision-making systems.

\section*{Impact Statement}
\section*{Impact statement}

This paper proposes that machine learning can and should be used to maximize social welfare. In principle, and by construction, the impact of our proposed framework on society aims to be positive. But our paper also points to the inherent difficulties of identifying, and making formal, what `good for society' is. We lean on the field of welfare economics, which has for decades contended with this challenge, for ideas on how the learning community can begin to approach this daunting task.
However, even if these ideas are conceptually appealing,
the path to practical welfare improvement presents many challenges---%
some expected, others unforseen.
% and will likely include many ups and downs.
For example, we may specify incorrect social welfare functions;
or we may specify them correctly but be unable to optimize them appropriately;
or we may be able to optimize but find that 
our assumptions are wrong, that theory differs from practice,
or that there were other considerations and complexities that we did not take into account.
For this we can look to other related fields---%
such as fairness, privacy, and alignment in machine learning---%
which have taken (and are still taking) similar journeys,
and learn from both their success and mistakes.
% and hope that ours will be similar.

Any discipline that seeks to affect policy should do so with much deliberation and care. Whereas welfare economics was designed with the explicit purpose of supporting (and influencing) policymakers,
machine learning has found itself in a similar position, but likely without any planned intent.
On the one hand, adjusting machine learning to support notions, such as social welfare,
that it was not designed to support initially can prove challenging.
However, and as we argue throughout, we believe that building on top of existing machinery is a more practical approach than to begin from scratch.
The necessity of confronting with welfare consideration can also
be an opportunity---as we can leverage these novel challenges
to make machine learning practice more informed, transparent, responsible, and socially aware.


% At the same time, the novelty of the challenges that welfare considerations present to the field make this an opportunity---%
% for chaning the role of machine learning in society for the better in a manner that is informed, transparent, and aware.



% Authors are \textbf{required} to include a statement of the potential 
% broader impact of their work, including its ethical aspects and future 
% societal consequences. This statement should be in an unnumbered 
% section at the end of the paper (co-located with Acknowledgements -- 
% the two may appear in either order, but both must be before References), 
% and does not count toward the paper page limit. In many cases, where 
% the ethical impacts and expected societal implications are those that 
% are well established when advancing the field of Machine Learning, 
% substantial discussion is not required, and a simple statement such 
% as the following will suffice:

% This paper presents work whose goal is to advance the field of 
% Machine Learning. There are many potential societal consequences 
% of our work, none which we feel must be specifically highlighted here.

% The above statement can be used verbatim in such cases, but we 
% encourage authors to think about whether there is content which does 
% % warrant further discussion, as this statement will be apparent if the 
% paper is later flagged for ethics review.


% In the unusual situation where you want a paper to appear in the
% references without citing it in the main text, use \nocite
\nocite{langley00}

\bibliography{reference}
\bibliographystyle{icml2025}


%%%%%%%%%%%%%%%%%%%%%%%%%%%%%%%%%%%%%%%%%%%%%%%%%%%%%%%%%%%%%%%%%%%%%%%%%%%%%%%
%%%%%%%%%%%%%%%%%%%%%%%%%%%%%%%%%%%%%%%%%%%%%%%%%%%%%%%%%%%%%%%%%%%%%%%%%%%%%%%
% APPENDIX
%%%%%%%%%%%%%%%%%%%%%%%%%%%%%%%%%%%%%%%%%%%%%%%%%%%%%%%%%%%%%%%%%%%%%%%%%%%%%%%
%%%%%%%%%%%%%%%%%%%%%%%%%%%%%%%%%%%%%%%%%%%%%%%%%%%%%%%%%%%%%%%%%%%%%%%%%%%%%%%
\newpage
\appendix
\onecolumn
\section{Werewolf Game Implementation Details}
\label{app:game}
\subsection{Game Rules}

\paragraph{Setup.}
Each game begins by randomly assigning seven roles—two Werewolves, one Seer, one Doctor, and three Villagers—to seven different players labeled “player\_0,” “player\_1,” …, “player\_6.” The two Werewolves are aware of each other’s identities, while the Seer, Doctor, and Villagers only know their own roles.

\paragraph{Night Round.}
During the Night round, only the surviving Werewolves, Seer, and Doctor take secret actions that are disclosed only to the relevant parties.
\begin{itemize}
    \item \textit{Werewolf}: The living Werewolves collectively decide on a target to kill, but they follow a specific order when there are two of them. First, the Werewolf with the smaller ID proposes a target; the other Werewolf then makes the final decision. For instance, if “player\_0” and “player\_2” are Werewolves, “player\_0” proposes “player\_i,” and “player\_2” chooses the ultimate kill target “player\_j.” If only one Werewolf is alive, that Werewolf’s decision stands. Werewolves cannot kill a dead player, themselves, or their teammate.
    \item \textit{Seer}: The Seer selects a living player to investigate, revealing whether that player is a Werewolf. The Seer may not investigate a dead player or themselves, although they are allowed to investigate the same player on different nights (albeit a less effective strategy).
    \item \textit{Doctor}: The Doctor selects a player to protect, without knowledge of the Werewolves’ choice. The Doctor cannot save someone who is already dead but can choose to save themselves.
\end{itemize}

\paragraph{Day Round.}
The day round proceeds with three phase including announcement, discussion, and voting.
\begin{itemize}
    \item \textit{Announcement}: at the start of the Day round, the events of the previous night are made public to all players still in the game. Anyone killed during the Night round is immediately removed and cannot reveal their role or participate in discussions. Two scenarios determine the announcement: if the Werewolves targeted “player\_i” and the Doctor either saved a different “player\_j” or was no longer alive, “player\_i” is killed, and the announcement states: “player\_i was killed last night.” If the Doctor saved exactly the same person the Werewolves intended to kill (“player\_i”), then no one is killed, and the announcement is: “no player was killed last night.”
    \item \textit{Discussion}: all surviving players join an open discussion in a set speaking order, each speaking exactly once. If, for example, the remaining players are “player\_0,” “player\_2,” and “player\_5,” then “player\_0” speaks first, followed by “player\_2,” and concluding with “player\_5.”
    \item \textit{Voting}: after the discussion, all surviving players simultaneously vote to eliminate one other player or choose to abstain. They are not allowed to vote for a dead player or for themselves. The individual who receives the most votes is eliminated without role disclosure. In the event of a tie, one of the tied players is randomly chosen to be eliminated. Everyone knows the final voting tally.
\end{itemize}

\paragraph{Winning.}
The Werewolves win if, at any point, the number of living Werewolves is equal to that of all other remaining players. They do not need to eliminate every non-Werewolf to claim victory. Conversely, the Villagers (including the Seer and Doctor) win once both Werewolves have been eliminated.


\subsection{Observation Space}

\paragraph{Language Observation.}
Each agent's language observation is represented as a list of natural language statements that log the game's history up to the current step. This list comprises both private information, which is accessible only to the current player, and public information, which is shared among all players. The private information includes the player's role, secret actions taken during the night phase by the Werewolf, Seer, and Doctor, as well as the Werewolf's teammate. On the other hand, the public information consists of the player's ID, the eliminated player in each night and day phase, discussions, and voting outcomes from each day phase. An example of the language observation is as follow.

\begin{tcolorbox}
{\ttfamily\small
Basic Information:
\begin{itemize}
    \item you are player\_5, your role is Doctor.
    \item current round and phase: night 2.
    \item remaining players: player\_0, player\_1, player\_2, player\_5, player\_6.
\end{itemize}
Round 1:
\begin{itemize}
    \item night 1: you chose to save player\_5.
    \item day 1 announcement: player\_4 was killed last night.
    \item day 1 discussion:
    \begin{itemize}
        \item player\_0 said: Good day, fellow players. As a Villager, my objective is to help identify and eliminate the Werewolves. Since player\_4 was killed last night, we know there is at least one Werewolf among us. I would like to hear everyone's thoughts and suspicions about who might be the Werewolves. Let's work together to find the culprits and ensure the safety of our village.
        \item player\_1 said: ...
        \item player\_2 said: ...
        \item player\_3 said: ...
        \item you said: ...
        \item player\_6 said: ...
    \end{itemize}
\end{itemize}
% }
% \end{tcolorbox}
% \begin{tcolorbox}
% {\ttfamily\small
\begin{itemize}
    \item day 1 voting result: player\_3 had the most votes and was eliminated. 
    \begin{itemize}
        \item voted for player\_3: player\_1, player\_6.
        \item voted for player\_1: player\_3.
        \item choose not to vote: player\_0, player\_2, player\_5.
    \end{itemize}
\end{itemize}

Now it is night 2 round and you should choose one player to save. As player\_5 and the Doctor, you should choose from the following actions: save player\_0, save player\_1, save player\_2, save player\_5, save player\_6.
}
\end{tcolorbox}

\paragraph{Vector Observation.}
We also consider a vectorized observation. The observation vector includes information like the player's ID, role, deductions, etc. by one-hot encoding. The details of the observation vector are listed in Table~\ref{tab:app:player}

\begin{table}[H]
\centering
\begin{tabular}{cccc}
\toprule
\multicolumn{2}{c}{}                                                                                          & Length & Description                                                                                                                              \\
\midrule
\multicolumn{2}{c}{ID}                                                                                        & 7      & one hot encoding of ID.                                                                                                                   \\
\multicolumn{2}{c}{Role}                                                                                      & 4      & \begin{tabular}[c]{@{}c@{}}one hot encoding of role,\\ {[}"Werewolf", "Seer", "Doctor", "Villager"{].}\end{tabular}                        \\
\multicolumn{2}{c}{Round}                                                                                     & 1      & current round.                                                                                                                            \\
\multicolumn{2}{c}{Phase}                                                                                     & 3      & \begin{tabular}[c]{@{}c@{}}one hot encoding of current phase,\\ {[}"night", "discussion", "voting"{].}\end{tabular}                        \\
\multicolumn{2}{c}{Alive players}                                                                             & 7      & alive flag for each player.                                                                                                               \\
\midrule
\multirow{3}{*}{\begin{tabular}[c]{@{}c@{}}For each round\\ (3 rounds)\end{tabular}} & secret action & 7      & \begin{tabular}[c]{@{}c@{}}one hot encoding of the target player,\\ (all zero if do not act).\end{tabular}                                 \\
                                                                                              & announcement  & 7      & \begin{tabular}[c]{@{}c@{}}one hot encoding of the dead player,\\ (all zero if no player is dead).\end{tabular}                            \\
                                                                                              & voting result & 49     & \begin{tabular}[c]{@{}c@{}}one hot encoding of the each player's choice,\\ (all zero if the player does not vote or is dead).\end{tabular} \\
\bottomrule
\end{tabular}

\caption{Vector observation space.}
\label{tab:app:player}
\end{table}

\subsection{Reward Functions}

The reward functions are defined as follows:
\begin{itemize}
    \item \textit{Winning Reward}: all winning players receive $+300$, and all losing players receive $-300$.
    \item \textit{Surviving Reward}: $+5$ for all surviving players in each round.
    \item \textit{Voting Reward} (Village side only): $+20$ for correct votes, $-20$ for incorrect votes.
    \item \textit{Voting Result Reward}: $-10$ for the player that is eliminated. $+5$ when an opponents is eliminated, $-5$ when a teammate is being eliminated.
\end{itemize}


\section{Detailed Prompt}
\label{app:method}
\subsection{System Prompt}
% The system prompt used in our method is as below.

\begin{tcolorbox}
{\ttfamily\small
You are an expert in playing the social deduction game named Werewolf. The game has seven roles including two Werewolves, one Seer, one Doctor, and three Villagers. There are seven players including player\_0, player\_1, player\_2, player\_3, player\_4, player\_5, and player\_6.
\\
\\
At the beginning of the game, each player is assigned a hidden role which divides them into the Werewolves and the Villagers (Seer, Doctor, Villagers). Then the game alternates between the night round and the day round until one side wins the game.
\\
\\
In the night round: the Werewolves choose one player to kill; the Seer chooses one player to see if they are a Werewolf; the Doctor chooses one player including themselves to save without knowing who is chosen by the Werewolves; the Villagers do nothing.
\\
\\
In the day round: three phases including an announcement phase, a discussion phase, and a voting phase are performed in order.
\\
In the announcement phase, an announcement of last night's result is made to all players. If player\_i was killed and not saved last night, the announcement will be "player\_i was killed"; if a player was killed and saved last night, the announcement will be "no player was killed"
\\
In the discussion phase, each remaining player speaks only once in order from player\_0 to player\_6 to discuss who might be the Werewolves.
\\
In the voting phase, each player votes for one player or choose not to vote. The player with the most votes is eliminated and the game continues to the next night round.
\\
\\
The Werewolves win the game if the number of remaining Werewolves is equal to the number of remaining Seer, Doctor, and Villagers. The Seer, Doctor, and Villagers win the game if all Werewolves are eliminated.
}
\end{tcolorbox}

\subsection{Prompt for Secret Actions}
% The prompt for secret actions in our method is as below.

\begin{tcolorbox}
{\ttfamily\small

Now it is night <n\_round> round, you (and your teammate) should choose one player to kill/see/save.
As player\_<id> and a <role>, you should first reason about the current situation, then choose from the following actions: <action\_0>, <action\_1>, ..., .\\
\\
You should only respond in JSON format as described below.
}
\end{tcolorbox}
\begin{tcolorbox}
{\ttfamily\small
Response Format: \\
\begin{verbatim}
{
    "reasoning": "reason about the current situation",
    "action": "kill/see/save player_i"
}
\end{verbatim}
Ensure the response can be parsed by Python json.loads
}
\end{tcolorbox}

\subsection{Prompt for Discussion Actions}
% The prompt for discussion actions in our method is as below.

\begin{tcolorbox}

{\ttfamily\small
Now it is day <n\_round> discussion phase and it is your turn to speak.
As player\_<id> and a <role>, before speaking to the other players, you should first reason the current situation only to yourself, and then speak to all other players.
You should only respond in JSON format as described below.
\\
Response Format:
\begin{verbatim}
{
    "reasoning": "reason about the current situation only to yourself",
    "statement": "speak to all other players"
}
\end{verbatim}
Ensure the response can be parsed by Python json.loads
}
\end{tcolorbox}

\subsection{Prompt for Voting Actions}
% The prompt for voting actions in our method is as below.

\begin{tcolorbox}

{\ttfamily\small
Now it is day <n\_round> voting phase, you should vote for one player or do not vote to maximize the Werewolves' benefit (for the Werewolves) / you should vote for one player that is most likely to be a Werewolf or do not vote (for the Villagers).
As player\_<id> and a <role>, you should first reason about the current situation, and then choose from the following actions: do no vote, <action\_0>, <action\_1>, ..., .
\\
\\
You should only respond in JSON format as described below.
\\
Response Format:
\begin{verbatim}
{
    "reasoning": "reason about the current situation",
    "action": "vote for player_i"
}
\end{verbatim}
Ensure the response can be parsed by Python json.loads
}
\end{tcolorbox}

\subsection{Prompt for Diverse Action Generation}
For the discussion actions, we iteratively ask the LLMs to produce one new action at a time by adding the following prompt in the action prompt: ``consider a new action that is strategically different from existing ones.''


\section{Implementation Detail}
\label{app:training}
\subsection{Hyperparameters}

For latent space construction, we let the LLM agent play $1000$ games to collect all discussion actions generated by each role in these games. For diverse action generation, we prompt the LLM to generate $3$ action candidates and randomly select one to execute in the game. We pair the language observation with the $3$ action candidates to use for preference-based fine-tuning in the following components. For sentence embedding, we use OpenAI's ``text-embedding-3-small'' embedding API to embed the sentence to a vector of $1536$ dimensions. Then we apply standard $k$-means clustering to cluster the embedding and get the discrete latent strategy space. The number of clusters $k$ in the first iteration is $3$ for the Werewolf and $2$ for the Seer, Doctor, and Villagers. In each iteration, we add $1$ cluster to the existing clusters. That is, if the first iteration has $k$ clusters, then the $i$-th iteration has $k + i - 1$ clusters.

For policy optimization in latent space, we use a learning rate of $1\times10^{-3}$ to train a Deep CFR network. The buffer size of each role's model is $5\times10^5$, and each model is trained for $1500$ iterations with batch size $4096$ using the Adam optimizer. 

For latent space expansion, we apply DPO with $\beta=0.1$, learning rate $1\times10^{-6}$, and trained for $2$ epoch with batch size $64$.


\subsection{Counterfactual Regret Minimization}

Counterfactual Regret Minimization (CFR) (\cite{zinkevich2007regret}) is a self-play algorithm, and each player continuously updates their strategies according to regret matching to achieve a Nash equilibrium.

We use the following notation. $Z$ is the set of all the end states $z$. $h\sqsubset z$ means state $h$ is a prefix of state $z$, that is, $z$ can be achieved from $h$. $\pi_p^\sigma$ is the probability contribution of the player $p$, and $\pi^\sigma = \prod_p \pi_p^\sigma$. $\pi_{-p}^\sigma$ is the probability contribution of all players except player $p$. $u_p(z)$ is the utility function for the player $p$ in the state $z$.

Counterfactual value for a state $h$ and a player $p$ according to startegy $\sigma$ is defined as:
\begin{equation}
    v_{p}^{\sigma}(h) = \sum_{z\in Z, h \sqsubset z} \pi^\sigma_{-p}(h)\pi^\sigma(z|h)u_p(z).
\end{equation}
The regret for a action $a$ in state $h$ for player $p$ is defined as: $v_p^{\sigma|_{h\to a}}(h) - v_{p}^{\sigma}(h)$, where $\sigma|_{h\to a}$ is same to $\sigma$ except in state $h$ the player will choose action $a$. The regret matching is choosing the strategy according to sum of previous regret values defined as $R(h,a)$, then the new strategy $\sigma(h,a) = \frac{R(h,a)^+}{\sum_{a'} R(h,a')^+}$, $R(h,a)^+ = \max(0,R(h,a))$. If $\sum_{a'} R(h,a')^+ =0$, just set $\sigma$ to be uniform random.

Because the game tree is very big, it is impossible to traverse the entire tree,  our implementation is based on deep CFR (\cite{brown2019deep}). We use a neural network to fit observation to regret value. The amount of computation required to search for only one player is also unacceptable, so a restriction is added based on deep CFR. If the number of layers currently searched is too large, the previous strategy is directly used to sample the actions of all players until the end of the game and return the utility for each player in that state. The complete process can be seen as running some complete game trajectories, and then starting from each intermediate node, searching a few layers to do CFR.

\subsection{Baseline Implementation}
ReAct, ReCon, and SLA are implemented following the original paper. The Cicero-like agent predefines a set of high-level atomic actions and trains an RL policy with this fixed action space. 
The RL policy takes the embeddings of the information record and deduction result as input and selects the atomic action based on this input.
Then the natural language actions used in gameplay are generated by prompting the LLM to follow the selected atomic actions.
In our case, the atomic action set consists of 13 actions including ``idle'', ``target player\_0'', ``target player\_1'', ``target player\_2'', ``target player\_3'', ``target player\_4'', ``target player\_5'', ``target player\_6'', ``claim to be a Werewolf'', ``claim to be a Seer'', ``claim to be a Doctor'', ``claim to be a Villager'', and ``do not reveal role''. 




%%%%%%%%%%%%%%%%%%%%%%%%%%%%%%%%%%%%%%%%%%%%%%%%%%%%%%%%%%%%%%%%%%%%%%%%%%%%%%%
%%%%%%%%%%%%%%%%%%%%%%%%%%%%%%%%%%%%%%%%%%%%%%%%%%%%%%%%%%%%%%%%%%%%%%%%%%%%%%%


\end{document}


% This document was modified from the file originally made available by
% Pat Langley and Andrea Danyluk for ICML-2K. This version was created
% by Iain Murray in 2018, and modified by Alexandre Bouchard in
% 2019 and 2021 and by Csaba Szepesvari, Gang Niu and Sivan Sabato in 2022.
% Modified again in 2023 and 2024 by Sivan Sabato and Jonathan Scarlett.
% Previous contributors include Dan Roy, Lise Getoor and Tobias
% Scheffer, which was slightly modified from the 2010 version by
% Thorsten Joachims & Johannes Fuernkranz, slightly modified from the
% 2009 version by Kiri Wagstaff and Sam Roweis's 2008 version, which is
% slightly modified from Prasad Tadepalli's 2007 version which is a
% lightly changed version of the previous year's version by Andrew
% Moore, which was in turn edited from those of Kristian Kersting and
% Codrina Lauth. Alex Smola contributed to the algorithmic style files.
