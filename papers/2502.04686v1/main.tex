%%%%%%%% ICML 2025 EXAMPLE LATEX SUBMISSION FILE %%%%%%%%%%%%%%%%%

\documentclass{article}

% Recommended, but optional, packages for figures and better typesetting:
\usepackage{microtype}
\usepackage{graphicx}
\usepackage{subfigure}
\usepackage{booktabs} % for professional tables

% hyperref makes hyperlinks in the resulting PDF.
% If your build breaks (sometimes temporarily if a hyperlink spans a page)
% please comment out the following usepackage line and replace
% \usepackage{icml2025} with \usepackage[nohyperref]{icml2025} above.
\usepackage{hyperref}


% Attempt to make hyperref and algorithmic work together better:
\newcommand{\theHalgorithm}{\arabic{algorithm}}

% Use the following line for the initial blind version submitted for review:
% \usepackage{icml2025}

% If accepted, instead use the following line for the camera-ready submission:
\usepackage[accepted]{icml2025}

% For theorems and such
\usepackage{amsmath}
\usepackage{amssymb}
\usepackage{mathtools}
\usepackage{amsthm}

% \usepackage[dvipsnames]{xcolor}
\usepackage[T1]{fontenc}
\usepackage{pifont}
\usepackage{multirow}
\usepackage{makecell}
\usepackage{bm}
\usepackage{tcolorbox}

\usepackage{enumitem}
\usepackage{inconsolata}
\usepackage{subcaption}

% if you use cleveref..
\usepackage[capitalize,noabbrev]{cleveref}

%%%%%%%%%%%%%%%%%%%%%%%%%%%%%%%%
% THEOREMS
%%%%%%%%%%%%%%%%%%%%%%%%%%%%%%%%
\theoremstyle{plain}
\newtheorem{theorem}{Theorem}[section]
\newtheorem{proposition}[theorem]{Proposition}
\newtheorem{lemma}[theorem]{Lemma}
\newtheorem{corollary}[theorem]{Corollary}
\theoremstyle{definition}
\newtheorem{definition}[theorem]{Definition}
\newtheorem{assumption}[theorem]{Assumption}
\theoremstyle{remark}
\newtheorem{remark}[theorem]{Remark}

\newcommand{\cmark}{\textcolor{LimeGreen}{\ding{51}}}
\newcommand{\xmark}{\textcolor{red}{\ding{55}}}
\newcommand{\yc}[1] {\textcolor{purple}{[yc: #1]}}
\newcommand{\xzl}[1] {\textcolor{blue}{[xzl: #1]}}
\newcommand{\yw}[1]{\textcolor{red}{[yi: #1]}}
\newcommand{\gf}[1]{\textcolor{olive}{[gf: #1]}}
\newcommand{\edit}[1]{\textcolor{red}{#1}}


% Todonotes is useful during development; simply uncomment the next line
%    and comment out the line below the next line to turn off comments
%\usepackage[disable,textsize=tiny]{todonotes}
\usepackage[textsize=tiny]{todonotes}


% The \icmltitle you define below is probably too long as a header.
% Therefore, a short form for the running title is supplied here:
\icmltitlerunning{Learning Strategic Language Agents in the Werewolf Game with Iterative Latent Space Policy Optimization}

\begin{document}

\twocolumn[
\icmltitle{Learning Strategic Language Agents in the Werewolf Game \\with Iterative Latent Space Policy Optimization}

% It is OKAY to include author information, even for blind
% submissions: the style file will automatically remove it for you
% unless you've provided the [accepted] option to the icml2025
% package.

% List of affiliations: The first argument should be a (short)
% identifier you will use later to specify author affiliations
% Academic affiliations should list Department, University, City, Region, Country
% Industry affiliations should list Company, City, Region, Country

% You can specify symbols, otherwise they are numbered in order.
% Ideally, you should not use this facility. Affiliations will be numbered
% in order of appearance and this is the preferred way.
\icmlsetsymbol{equal}{*}

\begin{icmlauthorlist}
\icmlauthor{Zelai Xu}{thu}
\icmlauthor{Wanjun Gu}{thu}
\icmlauthor{Chao Yu}{thu}
\icmlauthor{Yi Wu}{thu}
\icmlauthor{Yu Wang}{thu}
\end{icmlauthorlist}

\icmlaffiliation{thu}{Tsinghua University, Beijing, China}

\icmlcorrespondingauthor{Zelai Xu}{zelai.eecs@gmail.com}
\icmlcorrespondingauthor{Chao Yu}{yuchao@tsinghua.edu}
\icmlcorrespondingauthor{Yi Wu}{jxwuyi@gmail.com}
\icmlcorrespondingauthor{Yu Wang}{yu-wang@tsinghua.com}

% You may provide any keywords that you
% find helpful for describing your paper; these are used to populate
% the "keywords" metadata in the PDF but will not be shown in the document
% \icmlkeywords{Machine Learning, ICML}

\vskip 0.3in
]

% this must go after the closing bracket ] following \twocolumn[ ...

% This command actually creates the footnote in the first column
% listing the affiliations and the copyright notice.
% The command takes one argument, which is text to display at the start of the footnote.
% The \icmlEqualContribution command is standard text for equal contribution.
% Remove it (just {}) if you do not need this facility.

\printAffiliationsAndNotice{}  % leave blank if no need to mention equal contribution
% \printAffiliationsAndNotice{\icmlEqualContribution} % otherwise use the standard text.

\begin{abstract}
Multi-agent reinforcement learning (MARL) has made significant progress, largely fueled by the development of specialized testbeds that enable systematic evaluation of algorithms in controlled yet challenging scenarios. However, existing testbeds often focus on purely virtual simulations or limited robot morphologies such as robotic arms, quadrupeds, and humanoids, leaving high-mobility platforms with real-world physical constraints like drones underexplored. To bridge this gap, we present \textbf{\textit{VolleyBots}}, a new MARL testbed where multiple drones cooperate and compete in the sport of volleyball under physical dynamics. VolleyBots features a turn-based interaction model under volleyball rules, a hierarchical decision-making process that combines motion control and strategic play, and a high-fidelity simulation for seamless sim-to-real transfer. We provide a comprehensive suite of tasks ranging from single-drone drills to multi-drone cooperative and competitive tasks, accompanied by baseline evaluations of representative MARL and game-theoretic algorithms. Results in simulation show that while existing algorithms handle simple tasks effectively, they encounter difficulty in complex tasks that require both low-level control and high-level strategy. We further demonstrate zero-shot deployment of a simulation-learned policy to real-world drones, highlighting VolleyBots’ potential to propel MARL research involving agile robotic platforms. The project page is at \url{https://sites.google.com/view/thu-volleybots/home}.
\end{abstract}

\section{Introduction}
\section{Introduction}
\section{Introduction}
\label{sec:intro}

\begin{figure*}[tb]
    \centering
    \includegraphics[width=0.848\linewidth]{figs/circuitnn.pdf} 
    \caption{Illustration of differentiable CircuitNN. CircuitNN is designed based on differentiable NAND gates. After DAS is guided by PI and PO pairs of the truth table, CircuitNN can get the precise circuit architecture logic equivalent to the truth table.}
    \label{fig:circuitnn}
\end{figure*}

% 1. Describe the importance of logic synthesis
% 2. Existing Problems
% (a) Neural Architecture Search: Unstable, Predefined Setting, etc.
% (b) Circuit Generation: Probabilistic Model, Logic Equivalence

With the rapid advancement of technology, the scale of integrated circuits (ICs) has expanded exponentially. 
This expansion has introduced significant challenges in chip manufacturing, particularly concerning power and area metrics.
A primary objective in IC design is achieving the same circuit function with fewer transistors, thereby reducing power usage and area occupancy.

Logic synthesis~\cite{hachtel2005logicsynth}, a critical step in electronic design automation (EDA), transforms behavioral-level circuit designs into optimized gate-level circuits, ultimately yielding the final IC layout. 
The primary goal of logic synthesis is to identify the physical implementation with the fewest gates for a given circuit function. 
This task constitutes a challenging NP-hard combinatorial optimization problem. 
Current logic synthesis tools~\cite{brayton2010abc, wolf2013yosys} rely on human-designed heuristics, often leading to sub-optimal outcomes.

Differentiable architecture search (DAS) techniques~\cite{liu2018darts, chu2020darts} offer novel perspectives on addressing challenges in this problem.
Circuit functions can be represented through truth tables, which map binary inputs to their corresponding outputs. 
Truth tables provide a precise representation of input-output relationships, ensuring the design of functionally equivalent circuits.
Inspired by this, researchers~\cite{deepmind2024ai4sys, wang2024tnet} have begun exploring the application of DAS to synthesize circuits directly from truth tables.
Specifically, \citet{deepmind2024ai4sys} proposed CircuitNN, a framework that learns differentiable connection structures with logic gates, enabling the automatic generation of logic circuits from truth tables.
This approach significantly reduces the complexity of traditional circuit generation. 
Building on this, \citet{wang2024tnet} introduced T-Net, a triangle-shaped variant of CircuitNN, incorporating regularization techniques to enhance the efficiency of DAS.

Despite these advancements, several challenges remain. 
The computational complexity of DAS grows quadratically with the number of gates, posing scalability issues.
Although triangle-shaped architecture~\cite{wang2024tnet} partially mitigates this problem, redundancy persists. 
%Additionally, DAS is susceptible to converging to local optima, limiting the ability to search architectures that satisfy the given truth tables~\cite{liu2018darts}. 
%Furthermore, hyperparameters (network depth and layer width) require extensive searches, introducing complexity and prolonging the synthesis process. 
Additionally, DAS is susceptible to converging to local optima~\cite{liu2018darts} and hyperparameters (network depth and layer width) require extensive searches. 
The challenges arise from the vast search space in DAS. 
% Even with predefined settings for CircuitNN, finding a configuration that meets the truth table requires extensive trial and error during the DAS process. 
Intuitively, limiting the search space through predefined parameters (network depth, gates per layer, and connection probabilities) can significantly reduce the complexity.

Recent advances~\cite{openai2023gpt4, abramson2024alphafold3, esser2024sd3, li2024mar} in conditional generative models have demonstrated remarkable performance across language, vision, and graph generation tasks. 
Motivated by these developments, we propose a novel approach to circuit generation that generates preliminary circuit structures to guide DAS in generating refined circuits matching specified truth tables. 
Firstly, we introduce CircuitVQ, a tokenizer with a discrete codebook for circuit tokenization. 
Built upon our Circuit AutoEncoder framework~\cite{hou2022graphmae,li2023maskgae,wu2025mgvga}, CircuitVQ is trained through a circuit reconstruction task. 
Specifically, the CircuitVQ encoder encodes input circuits into discrete tokens using a learnable codebook, while the decoder reconstructs the circuit adjacency matrix based on these tokens.
Subsequently, the CircuitVQ encoder serves as a circuit tokenizer for CircuitAR pretraining, which employs a masked autoregressive modeling paradigm~\cite{chang2022maskgit, li2023mage}. 
In this process, the discrete codes function as supervision signals. 
After training, CircuitAR can generate discrete tokens progressively, which can be decoded into initial circuit structures by the decoder of the CircuitVQ. 
These prior insights can guide DAS in producing refined circuits that match the target truth tables precisely.

Our key contributions can be summarized as follows:
\begin{itemize}
\item We introduce CircuitVQ, a circuit tokenizer that facilitates graph autoregressive modeling for circuit generation, based on our Circuit AutoEncoder framework;
\item Develop CircuitAR, a model trained using masked autoregressive modeling, which generates initial circuit structures conditioned on given truth tables;
\item Propose a refinement framework that integrates differentiable architecture search to produce functionally equivalent circuits guided by target truth tables;
\item Comprehensive experiments demonstrating the scalability and capability emergence of our CircuitAR and the superior performance of the proposed circuit generation approach.
\end{itemize}

% Motivation
% (a) Diffusion (Vision, Graph), Autoregressive (Language, Vision)
% (b) Circuit Generation for Predefined Setting
% (c) Neural Architecture Search for Strict Logic Equivalence

% Contribution
% (a) Circuit Tokenizer (new transformer arch, training strategy)
% (b) CircuitAR (train and gen strategies, post-ar strategy)
% (c) Extensive Evaluation including BitD (Bit Distance) for Scalability

Query-focused summaries (QFS) give an overview of documents to answer a query~\cite{rosner2008multisum, el2021automatic}.
By combining each document's content useful for answering the query, or their \textbf{perspectives}~\cite{lin2006side}, these summaries can aid decision-making~\cite{hsu2021decision}.
For example, doctors pick treatments based on research paper perspectives~\cite{goff2008patients} and legislators vote based on perspectives in policy reports~\cite{jones1994reconceiving}. 
Past QFS work assumes documents have aligned perspectives~\cite{roy2023review}, but some queries, like ``\emph{Is law school worth it?}'', are debatable, containing opposing perspectives~\cite{wan2024evidence}.
In such cases, it is key to \textit{balance} perspectives from \textit{diverse} sources so users consider all sides before deciding~\cite{dale2015heuristics}.

To address this gap, we propose \textbf{\textit{debatable} QFS (DQFS}).
As input, DQFS uses documents and a debatable query, defined as a yes/no query where documents have opposing, equally-valid\footnote{This is meant to avoid input questions like ``Is the earth flat?'' where ``yes'' and ``no'' are not equally-valid (\cref{subsection:ethics}).} ``yes'' and ``no'' perspectives (Fig~\ref{fig:intro}).
Such queries are broad (\textit{Is law school worth it?}), and decomposing broad concepts into more specific topics (\textit{cost}, \textit{job market}) improves comprehension~\cite{johnson1983mental}.
Thus, DQFS creates a multi-aspect summary, with each paragraph covering one of an input number of topics ($2$ in Fig~\ref{fig:intro}).
The full summary and each paragraph must be \textit{comprehensive} and \textit{balanced}~(\cref{section:task}).
Comprehensive text has perspectives from all documents, while balanced text is not skewed towards the yes or no perspectives; our goals aid informed, unbiased decision-making~\cite{ziems2024measuring}.


While LLMs are deft summarizers~\cite{zhang2024benchmarking}, they cannot directly solve DQFS, as they fail to use diverse sources~\cite{huang-etal-2024-embrace}.
In Figure~\ref{fig:intro}, GPT-4 mainly gives perspectives favoring EU expansion (\textcolor{blue}{\textbf{blue}}), yielding a biased output.
Also, when asked for citations~\cite{huang-chang-2024-citation}, GPT-4 only cites 3/6 (\textcolor{yellowcite}{\textbf{yellow}}), missing half the documents' perspectives.
We intuit this arises since GPT-4 uses one inference step, with all documents in a single prompt.
This can omit document perspectives in certain positions of the prompt~\cite{liu2024lost} or that oppose parametric memory~\cite{jin2024tug}, reducing output coverage and balance.

Multi-LLM summarizers~\cite{chang2024booookscore, adams2023sparse}, which use LLMs to summarize documents individually into intermediate outputs before merging them with another LLM call, are better choices, as they represent documents more equally. 
However, they have two key issues.
\textbf{First}, they use the same topic or query as input to summarize each document, which is subpar if we wish to use retrieval in summarization to reduce LLM costs.
Queries unaligned to a document's unique content and expertise will fail to retrieve all of its most relevant contexts~\cite{sachan2022improving}; this reduces the total number of perspectives in the intermediate output, resulting in lower coverage.
\textbf{Second}, their intermediate outputs are unstructured, free-form texts, which are hard for the LLM to combine into a final output.
Free-form text needs extra reasoning to extract, classify, and compare the texts' perspectives~\cite{barrow2021syntopical}, steps that distract from the final goal of generating a balanced summary.

% A \textit{structured} intermediate output that clearly organizes documents and their perspectives on topics would greatly simplify the final step of synthesizing a balanced, comprehensive summary~\cite{shao2024assisting}.

To solve our issues, we build \textbf{\model} (Fig~\ref{fig:model}), a multi-LLM system using a \textbf{M}ixture \textbf{o}f \textbf{D}ocument \textbf{S}peakers.
Inspired by panel discussions~\cite{doumont2014english}, \model has a \textit{Speaker} LLM for each document that responds to queries using its document, and a \textit{Moderator} LLM that decides when and how speakers respond.
Specifically, \model: 1) plans an agenda of topics for the outline (\cref{subsection:agenda}); 2) picks a subset of speakers with relevant perspectives for each topic and tailors them a query (\cref{subsection:moderator}); and 3) asks each speaker to obtain its document's context relevant to the tailored query and give the context's ``yes'' and ``no'' perspectives for the topic. 
%All steps are efficiently done via~retrieval.

When a speaker supplies its document's perspectives, the topic, document number, tailored query, and perspectives update an outline, tracking the LLM discourse.
After the discussion, the outline is summarized for a DQFS output.
In all, \model frames DQFS as a discussion of document speakers to represent sources equally, tailors queries for speakers to optimize the retrieval of contexts used to find perspectives, and builds a structured outline of document perspectives to simplify the synthesis of a final output---a novel combination that leads to comprehensive and balanced summaries~(\cref{subsection:ablation}).

We compare \model to eight strong baselines~on ConflictingQA~\cite{wan2024evidence} and \textbf{DebateQFS} (\cref{subsection:datasets}), a new dataset for DQFS drawn from the debate community on Debatepedia~\cite{gottopati2013learning}.
To assess summaries, we have models give citations in their outputs (Fig~\ref{fig:intro}), showing the documents the model intends to use~\cite{huang-chang-2024-citation}.
Many works use citations for factuality~\cite{li2024citation}, but
we repurpose them for coverage and balance---measuring the proportion of documents cited and distribution of ground-truth yes/no perspective stances of cited documents (\cref{subsection:metrics}).


\model has the best document coverage and balance in full summaries and topic paragraphs (\cref{subsection:citation_comp}), surpassing SOTA by 38-58\% in paragraphs.
The Prometheus LLM~\cite{kim2024prometheus} ranks \model as one of the best models in summarization quality 28/30 times, the most of any model (\cref{subsection:summary_comp}).
Users also find \model's outputs to be the most balanced, and preserve readability despite using perspectives from more documents (\cref{subsection:human_eval}).
Lastly, analyses show the utility of tailoring queries and building outlines, which improve \model (\cref{subsection:ablation}) and offer rich, structured tools for users (\cref{subsection:qg}). Our contributions are:

\noindent \textbf{1)} We propose \textbf{debatable query-focused summarization}, a new task to help users navigate yes/no queries in documents with opposing perspectives. \\
\noindent \textbf{2)} We design \model, a multi-LLM DQFS system that treats documents as \textbf{individual} \textbf{LLM speakers}, uses a moderator to \textbf{tailor queries} to apt speakers, and tracks speaker perspectives in an \textbf{outline}. \\
\noindent \textbf{3)} We release \textbf{DebateQFS} for DQFS and \textbf{citation metrics} to capture summary coverage and~balance. \\
\noindent \textbf{4)} Experiments show \model \textbf{beats baselines by 38-58\%} in topic paragraph coverage and balance, while annotators find \model's summaries \textbf{maintain readability} and \textbf{better balance perspectives}.

\section{The Werewolf Game}
% In this work, we concentrate on a seven-player variant of the Werewolf game, featuring two Werewolves, one Seer, one Doctor, and three Villagers. We develop agents capable of engaging in a text-based version of the game, where all interactions are conducted through natural language without reliance on non-verbal cues such as tone or facial expressions.

\begin{figure*}[t]
    \centering
    \includegraphics[width=\linewidth]{figs/overview.pdf}
    \caption{Overview of the Latent Space Policy Optimization (LSPO) framework. Each iteration consists of three components. (1) Latent space construction: generate language actions with the LLM and cluster the vast language action into a finite latent strategy space. (2) Policy optimization in latent space: reformulate the original game as an abstracted game and apply game-theoretic methods to learn latent space policy. (3) Latent space expansion: fine-tune the LLM to align with the latent space policy and generate new strategies to expand the latent strategy space.}
    \label{fig:overview}
\end{figure*}

Werewolf is a popular social deduction game where players with hidden roles cooperate and compete with others in natural languages. The Werewolf side needs to conceal their identities and eliminate the other players, while the Village side needs to identify their teammates and vote out the Werewolves. Players are required to have both language proficiency for communication and strategic ability for decision-making to achieve strong performance in the Werewolf game. We consider a seven-player game with two Werewolves being the Werewolf side and one Seer, one Doctor, and three Villagers being the Village side.
% \gf{It's overlapped with the first sentence of the next paragraph.} 
Detailed descriptions of the game's rule, observation space, and reward function can be found in Appendix~\ref{app:game}. 
% More detailed descriptions of the game can be found in Appendix~\ref{app:game}. 
% \gf{It would be better to say what additional details are in the appendix based on the next section.}

\subsection{Game Environment}

We consider a text-based seven-player Werewolf game that proceeds through natural languages. We exclude other information like the speaking tone, facial expression, and body language~\cite{lai2022werewolf}. This pure text-based environment is a common setup in the literature~\cite{xu2023exploring,xu2023language,wu2024enhance,bailis2024werewolf}. 

\textbf{Roles and Objectives.} 
At the beginning of each game, the seven players are randomly partitioned into two sides. The Werewolf side has two Werewolf players who know each other's role and aim to eliminate the other players while avoiding being discovered. The Village side has one Seer who can check the role of one player each night, one Doctor who can protect one player each night, and three Villagers without any ability. The players in the Village side only know their own role and need to share information to identify the Werewolves and vote them out.


\textbf{Game Progression.} 
The game proceeds by alternating between night round and day round. In the night round, players can perform secret actions that are only observable by themselves. More specifically, the two Werewolves can choose a target player to eliminate, the Seer can choose a target player to investigate whether the player's role is Werewolf, and the Doctor can choose a target player to protect the player from being eliminated. The Doctor does not know the target player chosen by the Werewolves. If the Doctor chooses the same target player as the Werewolves, then no player is eliminated in this night round, otherwise, the Doctor fails to protect any player, and the target chosen by the Werewolves is eliminated.

\textbf{Observations and Actions.} 
The language observation of each agent is a list of natural languages that log the game history to the current step. This list include both private information that are only observable to the current player and public information that are shared by all players. The private information includes the role of the current player, the secret actions in the night round for the Werewolf, Seer, and Doctor, and the teammate for the Werewolf. The public information includes the ID of the current player, the eliminated player in each night and day round, the discussion, and the voting result in each day round. 
% \gf{One example observation input is illustrated in xxx.}

Player actions are also in the form of natural language and can be categorized into three types: secret actions, which are secret actions performed during the night, such as choosing a target player to eliminate, investigate, or protect; discussion actions, which are statements made during the day to influence other players' perceptions and decisions; and voting actions, which are choices made during the voting round to vote for on player or choose not to vote.


\subsection{Challenges for Language Agents}

Unlike board, card, or video games with a finite set of actions, Werewolf has a free-form language action space. The vast space of natural language actions poses two key challenges for language agents to achieve strong performance in the Werewolf game.
% \gf{intrinsic bias and unbounded action coverage. However, I think intrinsic bias issue is not caused by the vast space but from the training data? (Ignore it if I'm wrong...)}

\textbf{Intrinsic Bias in Action Generation.}
As observed in simple games like Rock-Paper-Scissor~\cite{xu2023language}, pure LLM-based agents tend to exhibit intrinsic bias in their action generation, which is inherited from the model's pre-training data. 
This issue is more pronounced in complex language games like Werewolf, where the opponents can exploit these predictable biases to counteract the agent's move. Therefore, mitigating intrinsic bias is essential for language agents to reduce exploitation and achieve strong performance.

\textbf{Coverage of Unbounded Action Space.}
Due to the immense combinatorial space induced by free-form text, it is impractical to map every possible utterance to an action in the language space. On the other hand, manually engineering or prompting an LLM to produce a limited set of actions may fail to capture the full strategic landscape. Even if an agent optimally masters the action distribution within a limited subset, it could be easily exploited by out-fo-distribution utterance. Consequently, inadequate coverage of the action space could result in suboptimal performance in free-form language games like Werewolf.



\section{Latent Space Policy Optimization}
% In this section, we introduce our proposed Latent Space Policy Optimization (LSPO) framework, designed to address the challenges of free-form language games. The framework combines game-theoretic optimization and large language model (LLM) fine-tuning, enabling the agent to iteratively improve both its strategic reasoning and linguistic expressiveness. Our method consists of four main components, as detailed below.

To tackle the intrinsic bias and the coverage issue,
% \gf{Is it appropriate to claim "address" here? It appears quite strong in comparison to "mitigate".} 
we propose an iterative Latent Space Policy Optimization (LSPO) framework. Our method combines game-theoretic optimization with LLM fine-tuning and operates on an expanding latent strategy space to iteratively improve the agent's decision-making ability and action coverage. As shown in Figure~\ref{fig:overview}, our framework has three components including latent space construction, policy optimization in latent space, and latent space expansion. More implementation details can be found in Appendix~\ref{app:method}.

\subsection{Latent Space Construction}

One of the key challenges in free-form language games like Werewolf is achieving broad coverage of the unbounded text space while maintaining a computationally tractable action representation for game-theoretic methods. To strike a balance between coverage and tractability, we propose to abstract the vast language action space into a finite set of latent strategies, which we then expand over iterations for better coverage. Specifically, our latent space construction in each iteration involves two steps including latent strategy generation and clustering.

% \paragraph{\yc{Step 1: }Latent Strategy Generation.}
\textbf{Latent Strategy Generation.}
In our setting, secret actions and voting actions are already discrete and therefore do not require further abstraction. We focus instead on the free-form discussion actions, which we aim to capture as latent strategies. We assume that each role in the game has the same set of latent strategies across all discussion rounds and collect the latent strategies for each role by letting the current LLM agent self-play as different roles for multiple trajectories.
% \gf{Does it mean one-to-one game play for the same role? Or play a complete game with the identical agent playing different players. I think it would be the latter case.} 
To further improve the coverage of latent strategies, we prompt the LLM to generate $N$ strategically distinct discussion candidates and randomly choose one to execute in the game. This process encourages diversity in the collected discussion actions and generate a set of latent strategies in natural language for each role. 
% \yc{In practice, N is xx. or mention this in experiment 4.1}

% \paragraph{\yc{Step 2: }Latent Strategy Clustering.}
\textbf{Latent Strategy Clustering.}
Although we generate a set of latent strategies for each role, they are still in the form of natural language. To transform them into a discrete latent strategy space, we embed each discussion action into a vector representation using an embedding model such as ``text-embedding-3-small'' that captures its semantic and contextual information. We then apply a simple $k$-means clustering algorithm to partition the embedded utterances into $k$ clusters, where each cluster represents a distinct latent strategy. Clustering reduces the infinite free-form text space to a finite set of abstract strategies, paving the way for subsequent game-theoretic optimization. By interpreting each cluster as a latent action, we can more efficiently search for and optimize strategic policies with minimal sacrifice of coverage of language space. 
% \gf{It would be better to visualize this two-step process with a concrete example. Besides, I'm curious how the number of clustering centers affects the performance, and how you plan to discuss on the "coverage" with this sampling+clustering process.}

% \subsection{Mapping Free-Form Language to Latent Strategy Space}

% A key challenge in free-form language games is the unbounded nature of the language space, which makes traditional game-solving methods computationally infeasible. To address this, we construct a \textit{discrete latent strategy space} by abstracting the free-form language space into a finite representation. 

% \paragraph{Data Generation.}
% To construct the latent strategy space, we first simulate multiple games using the LLM agent as a participant. During these games, the agent generates language utterances in response to various game states. Each state is represented as a combination of the agent’s knowledge, public information, and other players’ observed actions. This process yields a dataset of language utterances paired with their corresponding states, which serves as the basis for latent space construction.

% \paragraph{Latent Space Construction.} 
% Each collected utterance is converted into a high-dimensional vector representation using a pre-trained language model. These embeddings encode the semantic and contextual information of the utterances, enabling clustering based on their strategic similarity.
% We apply a simple K-means clustering algorithm to partition the embeddings into $k$ discrete clusters. Each cluster represents a latent strategy, effectively reducing the infinite free-form language space into a finite, discrete action space. The choice of $k$ is a hyperparameter that controls the granularity of the latent strategies. Clusters are interpreted as abstracted decision options, enabling downstream application of game-theoretic techniques.


\subsection{Policy Optimization in Latent Space}

Another challenge in free-form language games is to address the intrinsic bias in the agent's action distribution. After constructing a discrete latent strategy space, we can reformulate the original game with unbounded language space as an abstracted game with a finite latent strategy space. This reformulation allows us to apply standard game-solving techniques such as Counterfactual Regret Minimization (CFR) or reinforcement learning (RL) methods to learn near-optimal strategies that overcome the intrinsic bias. In our implementation, we employ CFR as the game solver.
% ,\yc{the latter sentence can be deleted} and other game-theoretic methods can also be used in our framework to solve the abstracted game. \gf{You always mention CFR AND RL before but you only employ CFR here. It's a bit weird to ignore the RL side.}

\textbf{Abstracted Game Formulation.} 
To represent the game in a compact, finite form, we replace the free-form discussion actions with the discrete latent strategies from latent space construction. Specifically, in the abstracted game, the secret action and voting action remain the same, and the discussion action is replaced by the latent strategy. The state in the abstracted game is a vector including information like the player's role, secret action, etc., and history of past latent strategies. The transition dynamics and payoff function remain unchanged in the abstracted game. This abstracted representation retains the key strategic elements of the original game while reducing the complexity of the action space, making large-scale game-solving computationally tractable.

\textbf{Optimal Policy Learning.} 
Once the game is represented in this discrete form, we apply CFR to learn a policy and solve the abstracted game. Classical CFR~\cite{zinkevich2007regret} iteratively improves policies by minimizing counterfactual regret $R$ for each information set.
For each iteration $t$, the regret for each action $a$ in the latent space is updated by:
% \yc{a is action or strategy?}
\begin{equation}
R_t(a) = R_{t-1}(a) + u(\sigma_t^a, \sigma_t^{-a}) - u(\sigma_t),
\end{equation}
where $u(\sigma_t^a, \sigma_t^{-a})$ is the utility of taking action $a$ under the current strategy profile $\sigma_t$, and $u(\sigma_t)$ is the utility under the full strategy profile.
% \yc{explain R,t}
We use neural networks to approximate regret value to scale CFR to more complex games and learn a policy for each different role in the Werewolf game. By repeatedly simulating self-play among agents employing Deep CFR in the abstracted game, each role’s policy converges to a near-optimal strategy profile. The resulting latent space policies address the intrinsic bias in action distribution and achieve strong strategic play in the abstracted game.


% \subsection{CFR in Latent Strategy Space}

% Once the discrete latent strategy space is constructed, we perform Counterfactual Regret Minimization (CFR) in this abstracted game environment. 

% \paragraph{Abstracted Game Representation.} 
% The original free-form language game is transformed into an abstracted game using the latent strategy space. Each cluster is treated as a discrete action, and transitions between states are governed by the latent strategies of all players. rewards are calculated based on the success or failure of the agent’s role-specific objectives.

% \paragraph{Optimal Strategy Learning.}

% CFR minimizes cumulative regret by iteratively updating the agent’s strategy profile. For each iteration $t$, the regret for each latent strategy $a$ is updated as follows:
% \begin{equation}
% R_t(a) = R_{t-1}(a) + u(\sigma_t^a, \sigma_t^{-a}) - u(\sigma_t),
% \end{equation}
% where $u(\sigma_t^a, \sigma_t^{-a})$ is the utility of taking action $a$ under the current strategy profile $\sigma_t$, and $u(\sigma_t)$ is the utility under the full strategy profile. Strategies are adjusted based on the cumulative regret to converge toward a Nash equilibrium in the abstracted game.

\begin{figure*}[t]
    \centering
    \includegraphics[width=\linewidth]{figs/latent_space.pdf}
    % \includegraphics[width=0.9\linewidth]{figs/latent_space_temp.pdf}
    \caption{Visualization of the latent space of Werewolf and Seer in different iterations.}
    \label{fig:latent_space}
\end{figure*}

\subsection{Latent Space Expansion}
To further improve the agent’s performance in free-form language games, the latent space must remain sufficiently expressive to cover novel strategies and resist exploitation by out-of-distribution actions. We achieve this by fine-tuning the LLM to align with the learned policy in the abstracted game and then re-generating and expanding the latent strategy space using the fine-tuned LLM. This iterative process progressively increases coverage of the action space, enabling stronger and more robust decision-making.

\textbf{Alignment to Latent Space Policy.}
We employ Direct Preference Optimization (DPO)~\cite{rafailov2024direct} to fine-tune the LLM so that its open-ended language outputs align with the near-optimal strategies derived from the abstracted game. To construct the preference dataset required by DPO, we leverage game trajectories generated during latent space construction. We record the language observation for the LLM agent at each discussion phase as the prompt, and use the $N$ discussion candidates as the response candidates. Each of the discussion candidates can be mapped to one of the latent strategies, and the preference label is determined by the regret value of the latent strategies. Intuitively, a discussion action with a lower regret value is preferred. With this preference dataset, we perform DPO to align the LLM toward the learned policy in the abstracted game for better performance in the original game.

\textbf{Update of Latent Space.}
Once the LLM is fine-tuned, it can produce a broader distribution of actions that reflect the refined policy. We exploit this enhanced generative capacity to expand the latent space in the next iteration. Specifically, we repeat the latent strategy generation and clustering procedures with the fine-tuned LLM to re-generate and expand the latent strategy space. This updated latent space offers increased coverage of potential strategies, enabling subsequent policy optimization to discover previously unexplored high-reward actions. Through iterative alignment and expansion, the agent continually refines its discussion strategies and achieves strong play in the free-form language game. 
% \gf{I would expect there are concrete examples to show how the strategy set covers more actions during the iteration.}


% \subsection{Mapping Back to Language Space}

% To deploy the learned strategies in the original game, we map the latent strategies back to the free-form language space. This process involves generating natural language utterances that align with the latent strategies.

% \paragraph{Language Generation via Prompting.} 
% For each latent strategy cluster, we identify representative utterances from the original dataset. These examples are used as prompts to guide the LLM in generating new utterances. The representative examples capture the strategic intent of the cluster, ensuring that the generated language aligns with the optimal latent strategies.

% \paragraph{Preference Learning with Regret Values.} 
% To further refine the LLM, we use the regret values computed during CFR to construct preference pairs for fine-tuning. Specifically, utterances generated from clusters with lower regret values are preferred over those with higher regret values. These preference pairs are used to fine-tune the LLM via Direct Preference Optimization (DPO), improving its ability to produce strategic language aligned with the learned strategies.

% \subsection{Iterative Refinement}

% After fine-tuning the LLM, the updated model is used to generate new language data, repeating the process outlined above. This iterative refinement allows the agent to continually improve both the latent strategy space and the language model's performance.

\section{Experiments}
We conduct extensive experiments in the Werewolf game to evaluate the effectiveness of our LSPO framework. We use ``Llama-3-8B-Instruct'' as the base model in our experiments. We first visualize how the latent strategy space evolves to show that our agents progressively acquire more complex strategic behaviors. We then quantitatively evaluate the performance of our LSPO agent using prediction accuracy and win rate to show the improving performance over iterations.  We also compare the LSPO agent with four state-of-the-art agents, showing that our agents achieve the highest win rate as both the Werewolf side and the Village side. We further perform ablation studies to assess the effectiveness of specific designs in our framework. More implementation details can be found in Appendix~\ref{app:training}. 
% \yc{we didn't say which LLM in the whole paper? also the baseline agents. it is important to mention this for a fair comparison.}


\begin{table*}[t]
    \centering
    \begin{table*}[t]
    \centering
    \caption{\textbf{Performance comparison across Information Exchange and Debate tasks.} Best results are indicated in \textbf{bold}, and second-best results are \underline{underlined}. The baseline results are taken from~\cite{DBLP:journals/corr/abs-2410-08115}.}
    \vskip 0.1in
    %\setlength{\tabcolsep}{3pt}
    % \renewcommand{\arraystretch}{1.1}
    \begin{tabular}{lcccccccc}
    \toprule
    & \multicolumn{4}{c}{\textbf{Information Exchange}} & \multicolumn{4}{c}{\textbf{Debate}} \\
    \cmidrule(lr){2-5} \cmidrule(lr){6-9}
     \textbf{Method} & \multicolumn{1}{c}{\textbf{HotpotQA}} & \multicolumn{1}{c}{\textbf{2WMH QA}}  &\multicolumn{1}{c}{\textbf{TriviaQA}} & \multicolumn{1}{c}{\textbf{CBT}}& \multicolumn{1}{c}{\textbf{MATH}} & \multicolumn{1}{c}{\textbf{GSM8k}} & \multicolumn{1}{c}{\textbf{ARC-C}}&\multicolumn{1}{c}{\textbf{MMLU}} \\
    
    \midrule
    CoT & 25.6  &20.5  &59.8  &43.4 &23.9 & 71.5 & 65.2  & 46.0 \\
   
    \midrule
    MAD &  28.4  &25.9& 71.0 & 53.8 &29.8 & 72.5 & 71.4 & 51.5\\
  
      \midrule
    DITS-iSFT-DPO  &&&&&&&&\\
    iteration 0 &50.46 &62.2  &71.39 &56.4 & 28.3  & 75.6 & 75.4 & 53.5  \\
    iteration 1 &54.53 &70.12 &78.39 &61.9 & 29.7 &  79.1 & 75.3 & 60.5 \\
    iteration 2 &57.28 &76.0  &78.08 &72.2 & 30.4 & 80.6 & 77.6 & 59.2 \\

    \bottomrule
    \end{tabular}
    \label{tab:main-table}
\end{table*}
    \caption{The prediction accuracy and win rate of the LSPO agents in different iterations.}
    \label{tab:iteration}
    % \vspace{-2mm}
\end{table*}


\subsection{Latent Space Visualization}

To gain insight into how LSPO organizes free-form language actions into discrete latent strategies, we first visualize the latent strategy space constructed at different training iterations. Specifically, for each role in the Werewolf game, we gather the utterances generated by the LSPO agent in $100$ games, embed them with the sentence encoder, and apply dimensionality reduction for projection. The visualization of latent spaces for the Werewolf and the Seer in different iterations is shown in Figure~\ref{fig:latent_space}. Earlier iterations yield relatively indistinct clusters, reflecting a lack of strategic diversity. Over successive iterations, clearer and more refined clusters emerge, indicating that the LSPO agent evolves toward an increasingly structured latent space and learn to express different strategic intentions such as accusing specific roles, defending teammates, and bluffing.

\textbf{Werewolf's Latent Space.}
In the first iteration, the latent space of the Werewolf is dominated by three main clusters. The blue cluster corresponds to a simple strategy of concealing its role or pretending to be a villager, while the smaller orange cluster reflects strategies like pretending to be a Seer or a Doctor. There is even a green cluster corresponding to unintentionally revealing the true role of a Werewolf, which is obviously a flawed strategy. As training proceeds, we see more sophisticated patterns emerge. The flawed strategy of disclosing one's Werewolf role disappears, and the agent begins to incorporate deliberate bluffs and misdirections instead. For example, the red cluster features the agent pretending to be a Seer and providing fabricated investigative results to sow confusion, and the purple cluster centers on defending the teammate and redirecting suspicion onto other players, leveraging more nuanced language and reasoning to guide the conversation toward scapegoats. This refined partitioning demonstrates that the Werewolf agent progressively covers an increasing number of latent strategies.


\textbf{Seer's Latent Space.}
In the first iteration, the Seer’s latent space is relatively coarse, containing primarily two strategies including staying silent about its true role or revealing its role and sharing information. This shows a limited range of strategic diversity in the early stage. As training proceeds through the second and third iterations, the Seer’s latent space becomes more diverse. The emergent red cluster features direct accusations once the Seer identifies a Werewolf, while the green cluster corresponds to concealing the role yet subtly guiding discussions to protect verified teammates. Notably, by the final iteration, the model develops a voting coordination strategy in which the Seer explicitly asks all the Villagers to vote for a strongly suspected Werewolf to maximize the Villager's chance of success. This progression implies that the Seer agent increasingly learns to balance openness and secrecy, aligning its communication style with the evolving game context to better support the Village side.


\subsection{Iterative Performance Evaluation}

We then evaluate how the performance of our LSPO agent progresses with more iterations, demonstrating that our framework produces increasingly stronger strategic language agents over time. We focus on two key metrics including 
prediction accuracy and win rate.

\textbf{Prediction Accuracy.}
Accurate role identification is a critical aspect of Werewolf, as it underpins effective decision-making and voting. Therefore, we measure the agent’s ability to predict the roles of other players with an additional prediction phase before each voting phase in a Werewolf game. Specifically, we use the final-iteration LSPO agent as the fixed opponent and let LSPO agents at different iterations play against this opponent for $100$ games. For the Werewolf side, a higher prediction accuracy of crucial roles like Seer and Doctor allows them to eliminate these threats earlier. Conversely, for the Village side, a higher prediction accuracy of Werewolves improves their chance to vote out the Werewolf and win the game.

\textbf{Win Rate.}
While prediction accuracy serves as an intermediate metric to evaluate the agents' reasoning and decision-making ability, we also use the win rate as a direct measure of the performance of our agents. Similar to the evaluation of prediction accuracy, we use the final-iteration LSPO agent as the fixed opponent and let our agents at different iterations play $100$ games against the opponent. A higher win rate indicates a stronger performance in the game.

% \paragraph{Evaluation Result.}
As shown in Table~\ref{tab:iteration}, both prediction accuracy and win rate exhibit a clear growing trend as the iteration increases, indicating that our iterative LSPO framework steadily strengthens the agents’ reasoning and decision-making capabilities. From the Werewolf side, the identification rate for the Seer starts off relatively high but has only modest improvement. This is because the Seer often reveals its roles to share information, making it easier for the Werewolf side to identify. By contrast, the Werewolf's prediction accuracy of the Doctor shows more significant gains, reflecting the strategic importance of eliminating the Doctor who can save potential victims. On the Village side, identifying the Werewolf and the Seer benefits most from iterative learning, since confirming these central roles is crucial for coordinated voting and elimination of Werewolves. Overall, these results confirm that our framework consistently improves the strategic language abilities of the LSPO agent, enabling it to adapt and excel in complex social deduction scenarios with each additional iteration.


\begin{table*}[t]
    \centering
    \begin{tabular}{cccccc}
\toprule
Win Rate             & ReAct & ReCon & Cicero-like & SLA & LSPO Agent (Ours) \\
\midrule
As the Werewolf Side & $0.58 \pm 0.15 $ & $0.60 \pm 0.12$ & $0.66 \pm 0.06$ & $0.69 \pm 0.12$ & $\bm{0.73 \pm 0.11}$ \\
As the Village Side & $0.16 \pm 0.06$ & $0.16 \pm 0.08$ & $0.21 \pm 0.04$ & $0.25 \pm 0.08$ & $\bm{0.27 \pm 0.11} $\\
Overall              & $0.38 \pm 0.11$ & $0.38 \pm 0.10$ & $0.44 \pm 0.05$ & $0.47 \pm 0.10$ & $\bm{0.50 \pm 0.11} $\\
\bottomrule
\end{tabular}

    \caption{Comparison between our LSPO agent with state-of-the-art agents in the Werewolf game.}
    \label{tab:head2head}
\end{table*}


\subsection{Comparison with State-of-the-Art Agents}

We compare the performance of the LSPO agent in the Werewolf game with four state-of-the-art agents including Reason and Act (ReAct)~\cite{yao2022react}, Recursive Contemplation (ReCon)~\cite{wang2023avalon}, a Cicero-like agent~\cite{meta2022human}, and Strategic Language Agent (SLA)~\cite{xu2023language}. As some of these methods were not initially developed for Werewolf, we make minor modifications to ensure compatibility with our experimental setting while preserving each approach’s core design. 
% \yc{The details of Baseline can be found in Appendix \ref{app:baseline}.}\yc{due to time limit, just put prompt in appendix is fine.}

\textbf{ReAct.}
ReAct is a classic prompt-based method that synergizes reasoning and acting for agent tasks. We implement ReAct for the Werewolf game by providing the LLM with raw game observations to generate both intermediate reasoning and final actions within a single prompt.

\textbf{ReCon.}
ReCon is another prompt-based method designed for Avalon agents. The ReCon agent is prompted to first think from its own perspective and then think from its opponents' perspective to generate the final action. We make slight modifications in the prompt to apply ReCon in the Werewolf game.

\textbf{Cicero-Like.}
The Cicero agent is created for the game of Diplomacy with finite action space and consists of a strategic reasoning module and a dialogue module. We implement a Cicero-like agent for the Werewolf game by predefining an action space of $13$ primitive actions like ``claim to be the Seer'', ``do not reveal role'', etc. An RL policy is learned to select these actions in each state and generate action-conditioned languages in the game.

\textbf{SLA.}
SLA combines reinforcement learning and LLM to overcome intrinsic bias and build strategic language agents for the Werewolf game. We adopt the same implementation as described in the paper~\cite{xu2023language}.

We compare our final-iteration LSPO agent with the aforementioned four baselines through two head-to-head evaluation setups. In the first setup, our LSPO agent takes the Werewolf side and we let each of the five agents including our agent and four baseline agents take the Village side to play $100$ Werewolf games with our LSPO agent. This setup measures the Village side's win rate against the LSPO agent as the Werewolves. In the second setup, we reverse the roles and let the LSPO agent take the Village side and compare the win rate of five agents as the Werewolves averaged over $100$ games. As shown in the bold numbers in Table~\ref{tab:head2head}, our LSPO agent achieve the highest win rates both as the Werewolves and as the Villagers. 

The strong performance of our LSPO agent is largely attributable to its iterative interplay between latent space strategy learning and preference-based fine-tuning, which refines both language and decision-making over time. By contrast, ReAct and ReCon rely on prompt-based approaches without game-theoretic updates, leaving them susceptible to intrinsic biases from pretraining data and limiting their performance in complex decision-making tasks. The Cicero-like agent is constrained by a predefined action set, making it difficult to evolve more subtle and diverse strategies as the game progresses. SLA partially addresses the intrinsic bias issues by generating multiple candidate actions and using RL to select from them. However, it still relies on a prompt-based method that can suffer from limited coverage of potential strategies. In comparison, our LSPO method integrates CFR’s policy improvement and latent-space cluster refinement with preference-based LLM alignment, enabling it to explore, exploit, and continuously expand the range of viable strategic moves in social deduction games.

\begin{table}[t]
    \centering
    \begin{table*}
  [t]
  \centering
  \resizebox{\textwidth}{!}{%
  \begin{tabular}{cccccccccccc}
    \toprule \multicolumn{2}{c}{Components}                                                             & \multicolumn{5}{c}{Re-executability Rate (\%)} & \multicolumn{5}{c}{Readability (\#)} \\
    \cmidrule(lr){1-2} \cmidrule(lr){3-7} \cmidrule(lr){8-12}        \hspace{8pt}\labelemoji\hspace{8pt}                                                                & \hspace{8pt}\toolemoji\hspace{8pt}                                      & O0                                 & O1             & O2             & O3             & AVG            & O0             & O1             & O2             & O3             & AVG            \\
    \hline
    \rowcolor[rgb]{0.93,0.93,0.93}\multicolumn{12}{c}{\textbf{Initialize with LLM4Decompile-End-6.7B~\citep{llm4decompile}}}   \\
    \xmark                                                                                              & \xmark                                    & 69.51                              & 46.95          & 50.61          & 46.34          & 53.35          & 3.98 & 3.41 & 3.44 & 3.38 & 3.55 \\
    \cmark                                                                                              & \xmark                                    & 75.61                              & 50.61          & 50.00          & 50.00          & 56.55          & 4.01 & 3.44 & 3.39 & \textbf{3.49} & 3.58 \\
    \xmark                                                                                              & \cmark                                    & 83.54                     & \textbf{56.10}          & 51.22          & 50.61 & 60.37 & 4.05 & 3.51 & 3.51 & 3.42 & 3.62 \\
    \cmark                                                                                              & \cmark                                    & \textbf{85.37}                            & \textbf{56.10}                     & \textbf{51.83} & \textbf{52.43}          & \textbf{61.43} & \textbf{4.13} & \textbf{3.60} & \textbf{3.54} & \textbf{3.49} & \textbf{3.69} \\

    \rowcolor[rgb]{0.93,0.93,0.93}\multicolumn{12}{c}{\textbf{Initialize with Deepseek-Coder-6.7B-base~\citep{deepseekcoder}}} \\
    \xmark                                                                                              & \xmark                                    & 59.15                              & 35.98          & 39.02          & 37.80          & 42.99          & 3.71 & 3.05 & 3.16 & 3.05 & 3.24 \\
    \cmark                                                                                              & \xmark                                    & 66.46                              & 41.46          & 38.41          & 36.59          & 45.73          & 3.76 & 3.17 & \textbf{3.21} & 3.08 & 3.31 \\
    \xmark                                                                                              & \cmark                                    & 70.73                              & 39.63          & 39.02          & 40.24          & 47.41          & 3.90 & 3.17 & 3.08 & 3.11 & 3.31 \\
    \cmark                                                                                              & \cmark                                    & \textbf{79.88}                     & \textbf{45.73} & \textbf{43.90} & \textbf{42.68} & \textbf{53.05} & \textbf{3.96} & \textbf{3.21} & 3.18 & \textbf{3.19} & \textbf{3.38} \\
    \bottomrule
  \end{tabular}%
  }
  \caption{The ablation study of different methods across four optimization levels
  (O0, O1, O2, O3), as well as their average scores (AVG). The results in bold represent the optimal performance. The ~\labelemoji~ and ~\toolemoji~ means Relabedling and Function Call. \textbf{Bold} denotes the best performance.}
  \label{tab:ablation}
\end{table*}
    \caption{Ablation on fine-tuning with latent space policy.}
    \label{tab:ablation}
    % \vspace{-4mm}
\end{table}

\subsection{Ablation Studies}

To show the effectiveness of our design, we compare the LSPO agent with an ablated version of itself. This ablated agent only performs latent space construction and policy optimization in latent space, without LLM fine-tuning and latent space expansion. To generate discussion action in gameplay, this agent first uses the latent space policy to sample a latent strategy, then the previously collected discussions corresponding to the latent strategy are used as few-shot examples to prompt the LLM for the discussion action. We compare this agent with the LSPO agent trained for one iteration and the result is shown in Table~\ref{tab:ablation}. The LSPO agent trained for one iteration achieves higher win rates than the ablated agent as both the Village side and the Werewolf side. This result indicates that fine-tuning the LLM to align with the latent space policy can help the LLM better generalize to new language actions beyond the collected samples and expand the latent strategy space.



\section{Related Work}
\textbf{Large Language Model-Based Agents.}

% \paragraph{Agents Powered by Large Language Models.}
Recent advancements in large language models (LLMs) have led to the development of agents capable of performing complex tasks across various domains, such as web interactions~\cite{nakano2021webgpt,yao2022webshop,deng2023mind2web}, code generation~\cite{chen2021evaluating,yang2024swe}, gaming environments~\cite{huang2022language,wang2023describe,wang2023voyager,ma2023large}, real-world robotics~\cite{ahn2022can,huang2022inner,vemprala2023chatgpt}, and multi-agent systems~\cite{park2023generative,li2023camel,chen2023agentverse}.
A common approach in these works is to exploit the reasoning capabilities and in-context learning of LLMs to improve decision-making processes.
Chain-of-Thought (CoT) prompting~\cite{wei2022chain} has been instrumental in enabling LLMs to perform step-by-step reasoning.
Building upon this, ReAct~\cite{yao2022react} synergizes reasoning and action to enhance performance across various tasks.
Subsequent research has incorporated self-reflection~\cite{shinn2023reflexion} and strategic reasoning~\cite{gandhi2023strategic} to further refine agent behaviors.
However, these methods can still suffer from the intrinsic biases and coverage issue of LLM-based agents, leading to suboptimal performance in complex games. A representative method that addresses these issues in the game of Diplomacy is Cicero~\cite{meta2022human}, which first uses a strategic module to produce action intent and then generates action-conditioned natural languages with a dialogue module. However, Diplomacy is a board game with finite action space and does not have the action coverage issue, making it not suitable for free-form language games with unbounded text action space.

Due to the high demand for both advanced communication skills and strategic reasoning, social deduction games like Werewolf and Avalon have been proposed as testbeds to build language agents with strategic ability. 
Earlier attempts to create agents for these games often rely on predefined protocols or limited communication capabilities~\cite{wang2018application}, restricting their effectiveness.
% DeepRole~\cite{serrino2019finding} utilizes CFR with deductive reasoning to play Avalon but does not consider natural language communication.
Recent works have explored using LLMs to enable natural language interactions in these games.
For instance, \citet{xu2023exploring} developed a prompt-based Werewolf agent that uses heuristic information retrieval and experience reflection.
Similarly, ReCon~\cite{wang2023avalon} introduced a prompt-based method for playing Avalon by considering both the agent's perspective and that of opponents.
However, these LLM-based agents may still be restricted by intrinsic bias and limited coverage of action space, affecting their decision-making quality.
Strategic Language Agent (SLA)~\cite{xu2023language} partially solves these issues by generating diverse action candidates and learning an RL policy to mitigate intrinsic bias. However, this method still relies on a fixed LLM to produce the action candidate, which can fail to address the coverage issue. Our approach mitigate the intrinsic bias by applying game-theoretic methods to optimize policy in a discrete latent space and tackles the coverage issue by iteratively expanding the latent space by aligning the LLM to the latent space policy, leading to strong performance in the Werewolf game.


% \gf{It seems your ICML24 paper is missing here. I'm not sure if there are others that should be discussed as well.}
% However, these LLM-based agents may still be limited by the biases inherent in language models, affecting their decision-making quality. \gf{Should we clarify the concept of “intrinsic bias” in the context of social deduction games?}

\textbf{Game-Theoretic Algorithms.}
Counterfactual Regret Minimization (CFR)~\cite{zinkevich2007regret} is a foundational algorithm for solving imperfect-information games, particularly those involving hidden information and strategic deception like poker~\cite{moravvcik2017deepstack,brown2018superhuman,brown2019superhuman}. The core principle of CFR is to iteratively reduce regret across players’ decision points in the game tree, converging toward strategies that approximate a Nash equilibrium. Subsequent refinements of CFR~\cite{lanctot2009monte,tammelin2014solving,brown2019deep} have expanded its scalability and adaptability to a broader range of scenarios. Of particular note is DeepRole~\cite{serrino2019finding}, which integrates deductive reasoning with CFR to play the social deduction game Avalon without communication. Our method combines CFR with language models by introducing a finite latent strategy space to enable it to solve free-form language games.


Reinforcement learning (RL) methods, on the other hand, have reached remarkable achievements in complex domains like Go~\cite{silver2016mastering,silver2018general} and video games~\cite{vinyals2019grandmaster,berner2019dota}, often surpassing expert human performance. A seminal technique in these successes is self-play and its variants\citep{heinrich2015fictitious,heinrich2016deep,hennes2020neural,xu2023fictitious}, where agents repeatedly train against older versions of themselves to refine their policies. Another prominent line of work is Policy-Space Response Oracles (PSRO)~\citep{lanctot2017unified,muller2019generalized}, an iterative procedure that produces best responses to a growing population of policies in a meta-game. Conceptually, our iterative framework is related to PSRO in that we both solve an abstracted game before enlarging it to approach the full original game. The difference is that PSRO treats newly learned policies as meta-actions to form a normal-form meta-game, whereas our approach clusters free-form language actions into a discrete latent action space to reformulate the original game as an extensive-form game with finite action space.
% Reinforcement learning has achieved remarkable success in games with imperfect information, such as Go~\cite{silver2016mastering,silver2018general}, poker~\cite{moravvcik2017deepstack,brown2019superhuman}, and complex video games~\cite{vinyals2019grandmaster,berner2019dota}.
% Techniques like self-play~\cite{heinrich2015fictitious,heinrich2016deep,hennes2020neural,xu2023fictitious} and population-based training~\cite{lanctot2017unified,muller2019generalized} have been pivotal in these achievements.
% Counterfactual Regret Minimization (CFR)~\cite{zinkevich2007regret,lanctot2009monte,tammelin2014solving,brown2019deep} is another powerful method that iteratively minimizes regret to approximate Nash equilibria in extensive-form games.
% While previous works have applied these methods to environments with well-defined action spaces, integrating them with LLMs to handle the unbounded 
% \gf{natural language actions} %action space of natural language 
% in games like Werewolf 
% \gf{remains a significant challenge.} %presents new challenges.
% \gf{To address this,} our approach leverages CFR in conjunction with LLMs to generate optimal action policies within the complex dynamics of the Werewolf game. \gf{I think this part is about the action coverage issue. It would be better to use a consistent terminology.}



\section{Conclusion}
In this work, we presented Latent Space Policy Optimization (LSPO), an iterative framework that combines structured game-solving techniques with the expressive power of large language models to build strategic language agents in free-form social deduction games. By abstracting unconstrained language action space into a discrete latent strategy space, our approach enables efficient CFR in the latent space to overcome intrinsic bias and learn strong strategies. We then perform iterative fine-tuning via DPO to align the LLM's language generation with the evolving strategy and expand the latent strategy space to address the action coverage issue. Our extensive evaluation in the Werewolf game demonstrates that LSPO not only addresses intrinsic biases and action coverage issues inherent in prompt-based agents, but also achieves increasing performance with respect to iterations and outperforms four state-of-the-art baseline agents. Looking ahead, we envision LSPO’s synergy of latent-space abstraction and preference-based language alignment can be extended to a variety of other complex decision-making tasks with free-form language actions.

% multi-agent and human-in-the-loop settings, paving the way for more robust and versatile language-grounded decision-making systems.

\section*{Impact Statement}
Our research advances the capabilities of LLM-based agents in a purely text-based Werewolf environment. While this setting allows the agents to develop robust decision-making and deception-detection skills, it also underscores the potential for misuse if similar techniques were to be adapted to real-world scenarios involving manipulation or misinformation. To mitigate these risks, our implementation remains strictly focused on text-based simulation and does not directly transfer to broader applications without additional safeguards. At the same time, our experiment results indicate that our agent could be used to identify potential deceptive and manipulative content. We envision that any future extensions of this work will require careful consideration of ethical guidelines and responsible deployment strategies to ensure that such language agent systems serve society constructively.

% Authors are \textbf{required} to include a statement of the potential 
% broader impact of their work, including its ethical aspects and future 
% societal consequences. This statement should be in an unnumbered 
% section at the end of the paper (co-located with Acknowledgements -- 
% the two may appear in either order, but both must be before References), 
% and does not count toward the paper page limit. In many cases, where 
% the ethical impacts and expected societal implications are those that 
% are well established when advancing the field of Machine Learning, 
% substantial discussion is not required, and a simple statement such 
% as the following will suffice:

% This paper presents work whose goal is to advance the field of 
% Machine Learning. There are many potential societal consequences 
% of our work, none which we feel must be specifically highlighted here.

% The above statement can be used verbatim in such cases, but we 
% encourage authors to think about whether there is content which does 
% % warrant further discussion, as this statement will be apparent if the 
% paper is later flagged for ethics review.


% In the unusual situation where you want a paper to appear in the
% references without citing it in the main text, use \nocite
\nocite{langley00}

\bibliography{reference}
\bibliographystyle{icml2025}


%%%%%%%%%%%%%%%%%%%%%%%%%%%%%%%%%%%%%%%%%%%%%%%%%%%%%%%%%%%%%%%%%%%%%%%%%%%%%%%
%%%%%%%%%%%%%%%%%%%%%%%%%%%%%%%%%%%%%%%%%%%%%%%%%%%%%%%%%%%%%%%%%%%%%%%%%%%%%%%
% APPENDIX
%%%%%%%%%%%%%%%%%%%%%%%%%%%%%%%%%%%%%%%%%%%%%%%%%%%%%%%%%%%%%%%%%%%%%%%%%%%%%%%
%%%%%%%%%%%%%%%%%%%%%%%%%%%%%%%%%%%%%%%%%%%%%%%%%%%%%%%%%%%%%%%%%%%%%%%%%%%%%%%
\newpage
\appendix
\onecolumn
\section{Werewolf Game Implementation Details}
\label{app:game}
\subsection{Game Rules}

\paragraph{Setup.}
Each game begins by randomly assigning seven roles—two Werewolves, one Seer, one Doctor, and three Villagers—to seven different players labeled “player\_0,” “player\_1,” …, “player\_6.” The two Werewolves are aware of each other’s identities, while the Seer, Doctor, and Villagers only know their own roles.

\paragraph{Night Round.}
During the Night round, only the surviving Werewolves, Seer, and Doctor take secret actions that are disclosed only to the relevant parties.
\begin{itemize}
    \item \textit{Werewolf}: The living Werewolves collectively decide on a target to kill, but they follow a specific order when there are two of them. First, the Werewolf with the smaller ID proposes a target; the other Werewolf then makes the final decision. For instance, if “player\_0” and “player\_2” are Werewolves, “player\_0” proposes “player\_i,” and “player\_2” chooses the ultimate kill target “player\_j.” If only one Werewolf is alive, that Werewolf’s decision stands. Werewolves cannot kill a dead player, themselves, or their teammate.
    \item \textit{Seer}: The Seer selects a living player to investigate, revealing whether that player is a Werewolf. The Seer may not investigate a dead player or themselves, although they are allowed to investigate the same player on different nights (albeit a less effective strategy).
    \item \textit{Doctor}: The Doctor selects a player to protect, without knowledge of the Werewolves’ choice. The Doctor cannot save someone who is already dead but can choose to save themselves.
\end{itemize}

\paragraph{Day Round.}
The day round proceeds with three phase including announcement, discussion, and voting.
\begin{itemize}
    \item \textit{Announcement}: at the start of the Day round, the events of the previous night are made public to all players still in the game. Anyone killed during the Night round is immediately removed and cannot reveal their role or participate in discussions. Two scenarios determine the announcement: if the Werewolves targeted “player\_i” and the Doctor either saved a different “player\_j” or was no longer alive, “player\_i” is killed, and the announcement states: “player\_i was killed last night.” If the Doctor saved exactly the same person the Werewolves intended to kill (“player\_i”), then no one is killed, and the announcement is: “no player was killed last night.”
    \item \textit{Discussion}: all surviving players join an open discussion in a set speaking order, each speaking exactly once. If, for example, the remaining players are “player\_0,” “player\_2,” and “player\_5,” then “player\_0” speaks first, followed by “player\_2,” and concluding with “player\_5.”
    \item \textit{Voting}: after the discussion, all surviving players simultaneously vote to eliminate one other player or choose to abstain. They are not allowed to vote for a dead player or for themselves. The individual who receives the most votes is eliminated without role disclosure. In the event of a tie, one of the tied players is randomly chosen to be eliminated. Everyone knows the final voting tally.
\end{itemize}

\paragraph{Winning.}
The Werewolves win if, at any point, the number of living Werewolves is equal to that of all other remaining players. They do not need to eliminate every non-Werewolf to claim victory. Conversely, the Villagers (including the Seer and Doctor) win once both Werewolves have been eliminated.


\subsection{Observation Space}

\paragraph{Language Observation.}
Each agent's language observation is represented as a list of natural language statements that log the game's history up to the current step. This list comprises both private information, which is accessible only to the current player, and public information, which is shared among all players. The private information includes the player's role, secret actions taken during the night phase by the Werewolf, Seer, and Doctor, as well as the Werewolf's teammate. On the other hand, the public information consists of the player's ID, the eliminated player in each night and day phase, discussions, and voting outcomes from each day phase. An example of the language observation is as follow.

\begin{tcolorbox}
{\ttfamily\small
Basic Information:
\begin{itemize}
    \item you are player\_5, your role is Doctor.
    \item current round and phase: night 2.
    \item remaining players: player\_0, player\_1, player\_2, player\_5, player\_6.
\end{itemize}
Round 1:
\begin{itemize}
    \item night 1: you chose to save player\_5.
    \item day 1 announcement: player\_4 was killed last night.
    \item day 1 discussion:
    \begin{itemize}
        \item player\_0 said: Good day, fellow players. As a Villager, my objective is to help identify and eliminate the Werewolves. Since player\_4 was killed last night, we know there is at least one Werewolf among us. I would like to hear everyone's thoughts and suspicions about who might be the Werewolves. Let's work together to find the culprits and ensure the safety of our village.
        \item player\_1 said: ...
        \item player\_2 said: ...
        \item player\_3 said: ...
        \item you said: ...
        \item player\_6 said: ...
    \end{itemize}
\end{itemize}
% }
% \end{tcolorbox}
% \begin{tcolorbox}
% {\ttfamily\small
\begin{itemize}
    \item day 1 voting result: player\_3 had the most votes and was eliminated. 
    \begin{itemize}
        \item voted for player\_3: player\_1, player\_6.
        \item voted for player\_1: player\_3.
        \item choose not to vote: player\_0, player\_2, player\_5.
    \end{itemize}
\end{itemize}

Now it is night 2 round and you should choose one player to save. As player\_5 and the Doctor, you should choose from the following actions: save player\_0, save player\_1, save player\_2, save player\_5, save player\_6.
}
\end{tcolorbox}

\paragraph{Vector Observation.}
We also consider a vectorized observation. The observation vector includes information like the player's ID, role, deductions, etc. by one-hot encoding. The details of the observation vector are listed in Table~\ref{tab:app:player}

\begin{table}[H]
\centering
\begin{tabular}{cccc}
\toprule
\multicolumn{2}{c}{}                                                                                          & Length & Description                                                                                                                              \\
\midrule
\multicolumn{2}{c}{ID}                                                                                        & 7      & one hot encoding of ID.                                                                                                                   \\
\multicolumn{2}{c}{Role}                                                                                      & 4      & \begin{tabular}[c]{@{}c@{}}one hot encoding of role,\\ {[}"Werewolf", "Seer", "Doctor", "Villager"{].}\end{tabular}                        \\
\multicolumn{2}{c}{Round}                                                                                     & 1      & current round.                                                                                                                            \\
\multicolumn{2}{c}{Phase}                                                                                     & 3      & \begin{tabular}[c]{@{}c@{}}one hot encoding of current phase,\\ {[}"night", "discussion", "voting"{].}\end{tabular}                        \\
\multicolumn{2}{c}{Alive players}                                                                             & 7      & alive flag for each player.                                                                                                               \\
\midrule
\multirow{3}{*}{\begin{tabular}[c]{@{}c@{}}For each round\\ (3 rounds)\end{tabular}} & secret action & 7      & \begin{tabular}[c]{@{}c@{}}one hot encoding of the target player,\\ (all zero if do not act).\end{tabular}                                 \\
                                                                                              & announcement  & 7      & \begin{tabular}[c]{@{}c@{}}one hot encoding of the dead player,\\ (all zero if no player is dead).\end{tabular}                            \\
                                                                                              & voting result & 49     & \begin{tabular}[c]{@{}c@{}}one hot encoding of the each player's choice,\\ (all zero if the player does not vote or is dead).\end{tabular} \\
\bottomrule
\end{tabular}

\caption{Vector observation space.}
\label{tab:app:player}
\end{table}

\subsection{Reward Functions}

The reward functions are defined as follows:
\begin{itemize}
    \item \textit{Winning Reward}: all winning players receive $+300$, and all losing players receive $-300$.
    \item \textit{Surviving Reward}: $+5$ for all surviving players in each round.
    \item \textit{Voting Reward} (Village side only): $+20$ for correct votes, $-20$ for incorrect votes.
    \item \textit{Voting Result Reward}: $-10$ for the player that is eliminated. $+5$ when an opponents is eliminated, $-5$ when a teammate is being eliminated.
\end{itemize}


\section{Detailed Prompt}
\label{app:method}
\subsection{System Prompt}
% The system prompt used in our method is as below.

\begin{tcolorbox}
{\ttfamily\small
You are an expert in playing the social deduction game named Werewolf. The game has seven roles including two Werewolves, one Seer, one Doctor, and three Villagers. There are seven players including player\_0, player\_1, player\_2, player\_3, player\_4, player\_5, and player\_6.
\\
\\
At the beginning of the game, each player is assigned a hidden role which divides them into the Werewolves and the Villagers (Seer, Doctor, Villagers). Then the game alternates between the night round and the day round until one side wins the game.
\\
\\
In the night round: the Werewolves choose one player to kill; the Seer chooses one player to see if they are a Werewolf; the Doctor chooses one player including themselves to save without knowing who is chosen by the Werewolves; the Villagers do nothing.
\\
\\
In the day round: three phases including an announcement phase, a discussion phase, and a voting phase are performed in order.
\\
In the announcement phase, an announcement of last night's result is made to all players. If player\_i was killed and not saved last night, the announcement will be "player\_i was killed"; if a player was killed and saved last night, the announcement will be "no player was killed"
\\
In the discussion phase, each remaining player speaks only once in order from player\_0 to player\_6 to discuss who might be the Werewolves.
\\
In the voting phase, each player votes for one player or choose not to vote. The player with the most votes is eliminated and the game continues to the next night round.
\\
\\
The Werewolves win the game if the number of remaining Werewolves is equal to the number of remaining Seer, Doctor, and Villagers. The Seer, Doctor, and Villagers win the game if all Werewolves are eliminated.
}
\end{tcolorbox}

\subsection{Prompt for Secret Actions}
% The prompt for secret actions in our method is as below.

\begin{tcolorbox}
{\ttfamily\small

Now it is night <n\_round> round, you (and your teammate) should choose one player to kill/see/save.
As player\_<id> and a <role>, you should first reason about the current situation, then choose from the following actions: <action\_0>, <action\_1>, ..., .\\
\\
You should only respond in JSON format as described below.
}
\end{tcolorbox}
\begin{tcolorbox}
{\ttfamily\small
Response Format: \\
\begin{verbatim}
{
    "reasoning": "reason about the current situation",
    "action": "kill/see/save player_i"
}
\end{verbatim}
Ensure the response can be parsed by Python json.loads
}
\end{tcolorbox}

\subsection{Prompt for Discussion Actions}
% The prompt for discussion actions in our method is as below.

\begin{tcolorbox}

{\ttfamily\small
Now it is day <n\_round> discussion phase and it is your turn to speak.
As player\_<id> and a <role>, before speaking to the other players, you should first reason the current situation only to yourself, and then speak to all other players.
You should only respond in JSON format as described below.
\\
Response Format:
\begin{verbatim}
{
    "reasoning": "reason about the current situation only to yourself",
    "statement": "speak to all other players"
}
\end{verbatim}
Ensure the response can be parsed by Python json.loads
}
\end{tcolorbox}

\subsection{Prompt for Voting Actions}
% The prompt for voting actions in our method is as below.

\begin{tcolorbox}

{\ttfamily\small
Now it is day <n\_round> voting phase, you should vote for one player or do not vote to maximize the Werewolves' benefit (for the Werewolves) / you should vote for one player that is most likely to be a Werewolf or do not vote (for the Villagers).
As player\_<id> and a <role>, you should first reason about the current situation, and then choose from the following actions: do no vote, <action\_0>, <action\_1>, ..., .
\\
\\
You should only respond in JSON format as described below.
\\
Response Format:
\begin{verbatim}
{
    "reasoning": "reason about the current situation",
    "action": "vote for player_i"
}
\end{verbatim}
Ensure the response can be parsed by Python json.loads
}
\end{tcolorbox}

\subsection{Prompt for Diverse Action Generation}
For the discussion actions, we iteratively ask the LLMs to produce one new action at a time by adding the following prompt in the action prompt: ``consider a new action that is strategically different from existing ones.''


\section{Implementation Detail}
\label{app:training}
% In this section, we introduce our proposed Latent Space Policy Optimization (LSPO) framework, designed to address the challenges of free-form language games. The framework combines game-theoretic optimization and large language model (LLM) fine-tuning, enabling the agent to iteratively improve both its strategic reasoning and linguistic expressiveness. Our method consists of four main components, as detailed below.

To tackle the intrinsic bias and the coverage issue,
% \gf{Is it appropriate to claim "address" here? It appears quite strong in comparison to "mitigate".} 
we propose an iterative Latent Space Policy Optimization (LSPO) framework. Our method combines game-theoretic optimization with LLM fine-tuning and operates on an expanding latent strategy space to iteratively improve the agent's decision-making ability and action coverage. As shown in Figure~\ref{fig:overview}, our framework has three components including latent space construction, policy optimization in latent space, and latent space expansion. More implementation details can be found in Appendix~\ref{app:method}.

\subsection{Latent Space Construction}

One of the key challenges in free-form language games like Werewolf is achieving broad coverage of the unbounded text space while maintaining a computationally tractable action representation for game-theoretic methods. To strike a balance between coverage and tractability, we propose to abstract the vast language action space into a finite set of latent strategies, which we then expand over iterations for better coverage. Specifically, our latent space construction in each iteration involves two steps including latent strategy generation and clustering.

% \paragraph{\yc{Step 1: }Latent Strategy Generation.}
\textbf{Latent Strategy Generation.}
In our setting, secret actions and voting actions are already discrete and therefore do not require further abstraction. We focus instead on the free-form discussion actions, which we aim to capture as latent strategies. We assume that each role in the game has the same set of latent strategies across all discussion rounds and collect the latent strategies for each role by letting the current LLM agent self-play as different roles for multiple trajectories.
% \gf{Does it mean one-to-one game play for the same role? Or play a complete game with the identical agent playing different players. I think it would be the latter case.} 
To further improve the coverage of latent strategies, we prompt the LLM to generate $N$ strategically distinct discussion candidates and randomly choose one to execute in the game. This process encourages diversity in the collected discussion actions and generate a set of latent strategies in natural language for each role. 
% \yc{In practice, N is xx. or mention this in experiment 4.1}

% \paragraph{\yc{Step 2: }Latent Strategy Clustering.}
\textbf{Latent Strategy Clustering.}
Although we generate a set of latent strategies for each role, they are still in the form of natural language. To transform them into a discrete latent strategy space, we embed each discussion action into a vector representation using an embedding model such as ``text-embedding-3-small'' that captures its semantic and contextual information. We then apply a simple $k$-means clustering algorithm to partition the embedded utterances into $k$ clusters, where each cluster represents a distinct latent strategy. Clustering reduces the infinite free-form text space to a finite set of abstract strategies, paving the way for subsequent game-theoretic optimization. By interpreting each cluster as a latent action, we can more efficiently search for and optimize strategic policies with minimal sacrifice of coverage of language space. 
% \gf{It would be better to visualize this two-step process with a concrete example. Besides, I'm curious how the number of clustering centers affects the performance, and how you plan to discuss on the "coverage" with this sampling+clustering process.}

% \subsection{Mapping Free-Form Language to Latent Strategy Space}

% A key challenge in free-form language games is the unbounded nature of the language space, which makes traditional game-solving methods computationally infeasible. To address this, we construct a \textit{discrete latent strategy space} by abstracting the free-form language space into a finite representation. 

% \paragraph{Data Generation.}
% To construct the latent strategy space, we first simulate multiple games using the LLM agent as a participant. During these games, the agent generates language utterances in response to various game states. Each state is represented as a combination of the agent’s knowledge, public information, and other players’ observed actions. This process yields a dataset of language utterances paired with their corresponding states, which serves as the basis for latent space construction.

% \paragraph{Latent Space Construction.} 
% Each collected utterance is converted into a high-dimensional vector representation using a pre-trained language model. These embeddings encode the semantic and contextual information of the utterances, enabling clustering based on their strategic similarity.
% We apply a simple K-means clustering algorithm to partition the embeddings into $k$ discrete clusters. Each cluster represents a latent strategy, effectively reducing the infinite free-form language space into a finite, discrete action space. The choice of $k$ is a hyperparameter that controls the granularity of the latent strategies. Clusters are interpreted as abstracted decision options, enabling downstream application of game-theoretic techniques.


\subsection{Policy Optimization in Latent Space}

Another challenge in free-form language games is to address the intrinsic bias in the agent's action distribution. After constructing a discrete latent strategy space, we can reformulate the original game with unbounded language space as an abstracted game with a finite latent strategy space. This reformulation allows us to apply standard game-solving techniques such as Counterfactual Regret Minimization (CFR) or reinforcement learning (RL) methods to learn near-optimal strategies that overcome the intrinsic bias. In our implementation, we employ CFR as the game solver.
% ,\yc{the latter sentence can be deleted} and other game-theoretic methods can also be used in our framework to solve the abstracted game. \gf{You always mention CFR AND RL before but you only employ CFR here. It's a bit weird to ignore the RL side.}

\textbf{Abstracted Game Formulation.} 
To represent the game in a compact, finite form, we replace the free-form discussion actions with the discrete latent strategies from latent space construction. Specifically, in the abstracted game, the secret action and voting action remain the same, and the discussion action is replaced by the latent strategy. The state in the abstracted game is a vector including information like the player's role, secret action, etc., and history of past latent strategies. The transition dynamics and payoff function remain unchanged in the abstracted game. This abstracted representation retains the key strategic elements of the original game while reducing the complexity of the action space, making large-scale game-solving computationally tractable.

\textbf{Optimal Policy Learning.} 
Once the game is represented in this discrete form, we apply CFR to learn a policy and solve the abstracted game. Classical CFR~\cite{zinkevich2007regret} iteratively improves policies by minimizing counterfactual regret $R$ for each information set.
For each iteration $t$, the regret for each action $a$ in the latent space is updated by:
% \yc{a is action or strategy?}
\begin{equation}
R_t(a) = R_{t-1}(a) + u(\sigma_t^a, \sigma_t^{-a}) - u(\sigma_t),
\end{equation}
where $u(\sigma_t^a, \sigma_t^{-a})$ is the utility of taking action $a$ under the current strategy profile $\sigma_t$, and $u(\sigma_t)$ is the utility under the full strategy profile.
% \yc{explain R,t}
We use neural networks to approximate regret value to scale CFR to more complex games and learn a policy for each different role in the Werewolf game. By repeatedly simulating self-play among agents employing Deep CFR in the abstracted game, each role’s policy converges to a near-optimal strategy profile. The resulting latent space policies address the intrinsic bias in action distribution and achieve strong strategic play in the abstracted game.


% \subsection{CFR in Latent Strategy Space}

% Once the discrete latent strategy space is constructed, we perform Counterfactual Regret Minimization (CFR) in this abstracted game environment. 

% \paragraph{Abstracted Game Representation.} 
% The original free-form language game is transformed into an abstracted game using the latent strategy space. Each cluster is treated as a discrete action, and transitions between states are governed by the latent strategies of all players. rewards are calculated based on the success or failure of the agent’s role-specific objectives.

% \paragraph{Optimal Strategy Learning.}

% CFR minimizes cumulative regret by iteratively updating the agent’s strategy profile. For each iteration $t$, the regret for each latent strategy $a$ is updated as follows:
% \begin{equation}
% R_t(a) = R_{t-1}(a) + u(\sigma_t^a, \sigma_t^{-a}) - u(\sigma_t),
% \end{equation}
% where $u(\sigma_t^a, \sigma_t^{-a})$ is the utility of taking action $a$ under the current strategy profile $\sigma_t$, and $u(\sigma_t)$ is the utility under the full strategy profile. Strategies are adjusted based on the cumulative regret to converge toward a Nash equilibrium in the abstracted game.

\begin{figure*}[t]
    \centering
    \includegraphics[width=\linewidth]{figs/latent_space.pdf}
    % \includegraphics[width=0.9\linewidth]{figs/latent_space_temp.pdf}
    \caption{Visualization of the latent space of Werewolf and Seer in different iterations.}
    \label{fig:latent_space}
\end{figure*}

\subsection{Latent Space Expansion}
To further improve the agent’s performance in free-form language games, the latent space must remain sufficiently expressive to cover novel strategies and resist exploitation by out-of-distribution actions. We achieve this by fine-tuning the LLM to align with the learned policy in the abstracted game and then re-generating and expanding the latent strategy space using the fine-tuned LLM. This iterative process progressively increases coverage of the action space, enabling stronger and more robust decision-making.

\textbf{Alignment to Latent Space Policy.}
We employ Direct Preference Optimization (DPO)~\cite{rafailov2024direct} to fine-tune the LLM so that its open-ended language outputs align with the near-optimal strategies derived from the abstracted game. To construct the preference dataset required by DPO, we leverage game trajectories generated during latent space construction. We record the language observation for the LLM agent at each discussion phase as the prompt, and use the $N$ discussion candidates as the response candidates. Each of the discussion candidates can be mapped to one of the latent strategies, and the preference label is determined by the regret value of the latent strategies. Intuitively, a discussion action with a lower regret value is preferred. With this preference dataset, we perform DPO to align the LLM toward the learned policy in the abstracted game for better performance in the original game.

\textbf{Update of Latent Space.}
Once the LLM is fine-tuned, it can produce a broader distribution of actions that reflect the refined policy. We exploit this enhanced generative capacity to expand the latent space in the next iteration. Specifically, we repeat the latent strategy generation and clustering procedures with the fine-tuned LLM to re-generate and expand the latent strategy space. This updated latent space offers increased coverage of potential strategies, enabling subsequent policy optimization to discover previously unexplored high-reward actions. Through iterative alignment and expansion, the agent continually refines its discussion strategies and achieves strong play in the free-form language game. 
% \gf{I would expect there are concrete examples to show how the strategy set covers more actions during the iteration.}


% \subsection{Mapping Back to Language Space}

% To deploy the learned strategies in the original game, we map the latent strategies back to the free-form language space. This process involves generating natural language utterances that align with the latent strategies.

% \paragraph{Language Generation via Prompting.} 
% For each latent strategy cluster, we identify representative utterances from the original dataset. These examples are used as prompts to guide the LLM in generating new utterances. The representative examples capture the strategic intent of the cluster, ensuring that the generated language aligns with the optimal latent strategies.

% \paragraph{Preference Learning with Regret Values.} 
% To further refine the LLM, we use the regret values computed during CFR to construct preference pairs for fine-tuning. Specifically, utterances generated from clusters with lower regret values are preferred over those with higher regret values. These preference pairs are used to fine-tune the LLM via Direct Preference Optimization (DPO), improving its ability to produce strategic language aligned with the learned strategies.

% \subsection{Iterative Refinement}

% After fine-tuning the LLM, the updated model is used to generate new language data, repeating the process outlined above. This iterative refinement allows the agent to continually improve both the latent strategy space and the language model's performance.



%%%%%%%%%%%%%%%%%%%%%%%%%%%%%%%%%%%%%%%%%%%%%%%%%%%%%%%%%%%%%%%%%%%%%%%%%%%%%%%
%%%%%%%%%%%%%%%%%%%%%%%%%%%%%%%%%%%%%%%%%%%%%%%%%%%%%%%%%%%%%%%%%%%%%%%%%%%%%%%


\end{document}


% This document was modified from the file originally made available by
% Pat Langley and Andrea Danyluk for ICML-2K. This version was created
% by Iain Murray in 2018, and modified by Alexandre Bouchard in
% 2019 and 2021 and by Csaba Szepesvari, Gang Niu and Sivan Sabato in 2022.
% Modified again in 2023 and 2024 by Sivan Sabato and Jonathan Scarlett.
% Previous contributors include Dan Roy, Lise Getoor and Tobias
% Scheffer, which was slightly modified from the 2010 version by
% Thorsten Joachims & Johannes Fuernkranz, slightly modified from the
% 2009 version by Kiri Wagstaff and Sam Roweis's 2008 version, which is
% slightly modified from Prasad Tadepalli's 2007 version which is a
% lightly changed version of the previous year's version by Andrew
% Moore, which was in turn edited from those of Kristian Kersting and
% Codrina Lauth. Alex Smola contributed to the algorithmic style files.
