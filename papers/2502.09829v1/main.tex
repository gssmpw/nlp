\documentclass[conference]{IEEEtran}
\usepackage{times}

% numbers option provides compact numerical references in the text. 
\usepackage[numbers]{natbib}
\usepackage{multicol}
\usepackage[bookmarks=true]{hyperref}
\usepackage{graphicx}
\usepackage{lipsum}
\usepackage{amsmath}
\usepackage{amssymb}
\usepackage[dvipsnames]{xcolor}


\usepackage{mathtools, algorithm, algpseudocode}
\usepackage{cleveref}


\newcommand{\abrar}[1]{\textcolor{green}{Abrar: #1}}


\pdfinfo{
   /Author (Anonymous)
   /Title  (Efficient Evaluation of Multi-Task Robot Policies With Active Experiment Selection)
   /CreationDate (D:20101201120000)
   /Subject (Robot evaluation)
   /Keywords (Robot evaluation)
}


%
\setlength\unitlength{1mm}
\newcommand{\twodots}{\mathinner {\ldotp \ldotp}}
% bb font symbols
\newcommand{\Rho}{\mathrm{P}}
\newcommand{\Tau}{\mathrm{T}}

\newfont{\bbb}{msbm10 scaled 700}
\newcommand{\CCC}{\mbox{\bbb C}}

\newfont{\bb}{msbm10 scaled 1100}
\newcommand{\CC}{\mbox{\bb C}}
\newcommand{\PP}{\mbox{\bb P}}
\newcommand{\RR}{\mbox{\bb R}}
\newcommand{\QQ}{\mbox{\bb Q}}
\newcommand{\ZZ}{\mbox{\bb Z}}
\newcommand{\FF}{\mbox{\bb F}}
\newcommand{\GG}{\mbox{\bb G}}
\newcommand{\EE}{\mbox{\bb E}}
\newcommand{\NN}{\mbox{\bb N}}
\newcommand{\KK}{\mbox{\bb K}}
\newcommand{\HH}{\mbox{\bb H}}
\newcommand{\SSS}{\mbox{\bb S}}
\newcommand{\UU}{\mbox{\bb U}}
\newcommand{\VV}{\mbox{\bb V}}


\newcommand{\yy}{\mathbbm{y}}
\newcommand{\xx}{\mathbbm{x}}
\newcommand{\zz}{\mathbbm{z}}
\newcommand{\sss}{\mathbbm{s}}
\newcommand{\rr}{\mathbbm{r}}
\newcommand{\pp}{\mathbbm{p}}
\newcommand{\qq}{\mathbbm{q}}
\newcommand{\ww}{\mathbbm{w}}
\newcommand{\hh}{\mathbbm{h}}
\newcommand{\vvv}{\mathbbm{v}}

% Vectors

\newcommand{\av}{{\bf a}}
\newcommand{\bv}{{\bf b}}
\newcommand{\cv}{{\bf c}}
\newcommand{\dv}{{\bf d}}
\newcommand{\ev}{{\bf e}}
\newcommand{\fv}{{\bf f}}
\newcommand{\gv}{{\bf g}}
\newcommand{\hv}{{\bf h}}
\newcommand{\iv}{{\bf i}}
\newcommand{\jv}{{\bf j}}
\newcommand{\kv}{{\bf k}}
\newcommand{\lv}{{\bf l}}
\newcommand{\mv}{{\bf m}}
\newcommand{\nv}{{\bf n}}
\newcommand{\ov}{{\bf o}}
\newcommand{\pv}{{\bf p}}
\newcommand{\qv}{{\bf q}}
\newcommand{\rv}{{\bf r}}
\newcommand{\sv}{{\bf s}}
\newcommand{\tv}{{\bf t}}
\newcommand{\uv}{{\bf u}}
\newcommand{\wv}{{\bf w}}
\newcommand{\vv}{{\bf v}}
\newcommand{\xv}{{\bf x}}
\newcommand{\yv}{{\bf y}}
\newcommand{\zv}{{\bf z}}
\newcommand{\zerov}{{\bf 0}}
\newcommand{\onev}{{\bf 1}}

% Matrices

\newcommand{\Am}{{\bf A}}
\newcommand{\Bm}{{\bf B}}
\newcommand{\Cm}{{\bf C}}
\newcommand{\Dm}{{\bf D}}
\newcommand{\Em}{{\bf E}}
\newcommand{\Fm}{{\bf F}}
\newcommand{\Gm}{{\bf G}}
\newcommand{\Hm}{{\bf H}}
\newcommand{\Id}{{\bf I}}
\newcommand{\Jm}{{\bf J}}
\newcommand{\Km}{{\bf K}}
\newcommand{\Lm}{{\bf L}}
\newcommand{\Mm}{{\bf M}}
\newcommand{\Nm}{{\bf N}}
\newcommand{\Om}{{\bf O}}
\newcommand{\Pm}{{\bf P}}
\newcommand{\Qm}{{\bf Q}}
\newcommand{\Rm}{{\bf R}}
\newcommand{\Sm}{{\bf S}}
\newcommand{\Tm}{{\bf T}}
\newcommand{\Um}{{\bf U}}
\newcommand{\Wm}{{\bf W}}
\newcommand{\Vm}{{\bf V}}
\newcommand{\Xm}{{\bf X}}
\newcommand{\Ym}{{\bf Y}}
\newcommand{\Zm}{{\bf Z}}

% Calligraphic

\newcommand{\Ac}{{\cal A}}
\newcommand{\Bc}{{\cal B}}
\newcommand{\Cc}{{\cal C}}
\newcommand{\Dc}{{\cal D}}
\newcommand{\Ec}{{\cal E}}
\newcommand{\Fc}{{\cal F}}
\newcommand{\Gc}{{\cal G}}
\newcommand{\Hc}{{\cal H}}
\newcommand{\Ic}{{\cal I}}
\newcommand{\Jc}{{\cal J}}
\newcommand{\Kc}{{\cal K}}
\newcommand{\Lc}{{\cal L}}
\newcommand{\Mc}{{\cal M}}
\newcommand{\Nc}{{\cal N}}
\newcommand{\nc}{{\cal n}}
\newcommand{\Oc}{{\cal O}}
\newcommand{\Pc}{{\cal P}}
\newcommand{\Qc}{{\cal Q}}
\newcommand{\Rc}{{\cal R}}
\newcommand{\Sc}{{\cal S}}
\newcommand{\Tc}{{\cal T}}
\newcommand{\Uc}{{\cal U}}
\newcommand{\Wc}{{\cal W}}
\newcommand{\Vc}{{\cal V}}
\newcommand{\Xc}{{\cal X}}
\newcommand{\Yc}{{\cal Y}}
\newcommand{\Zc}{{\cal Z}}

% Bold greek letters

\newcommand{\alphav}{\hbox{\boldmath$\alpha$}}
\newcommand{\betav}{\hbox{\boldmath$\beta$}}
\newcommand{\gammav}{\hbox{\boldmath$\gamma$}}
\newcommand{\deltav}{\hbox{\boldmath$\delta$}}
\newcommand{\etav}{\hbox{\boldmath$\eta$}}
\newcommand{\lambdav}{\hbox{\boldmath$\lambda$}}
\newcommand{\epsilonv}{\hbox{\boldmath$\epsilon$}}
\newcommand{\nuv}{\hbox{\boldmath$\nu$}}
\newcommand{\muv}{\hbox{\boldmath$\mu$}}
\newcommand{\zetav}{\hbox{\boldmath$\zeta$}}
\newcommand{\phiv}{\hbox{\boldmath$\phi$}}
\newcommand{\psiv}{\hbox{\boldmath$\psi$}}
\newcommand{\thetav}{\hbox{\boldmath$\theta$}}
\newcommand{\tauv}{\hbox{\boldmath$\tau$}}
\newcommand{\omegav}{\hbox{\boldmath$\omega$}}
\newcommand{\xiv}{\hbox{\boldmath$\xi$}}
\newcommand{\sigmav}{\hbox{\boldmath$\sigma$}}
\newcommand{\piv}{\hbox{\boldmath$\pi$}}
\newcommand{\rhov}{\hbox{\boldmath$\rho$}}
\newcommand{\upsilonv}{\hbox{\boldmath$\upsilon$}}

\newcommand{\Gammam}{\hbox{\boldmath$\Gamma$}}
\newcommand{\Lambdam}{\hbox{\boldmath$\Lambda$}}
\newcommand{\Deltam}{\hbox{\boldmath$\Delta$}}
\newcommand{\Sigmam}{\hbox{\boldmath$\Sigma$}}
\newcommand{\Phim}{\hbox{\boldmath$\Phi$}}
\newcommand{\Pim}{\hbox{\boldmath$\Pi$}}
\newcommand{\Psim}{\hbox{\boldmath$\Psi$}}
\newcommand{\Thetam}{\hbox{\boldmath$\Theta$}}
\newcommand{\Omegam}{\hbox{\boldmath$\Omega$}}
\newcommand{\Xim}{\hbox{\boldmath$\Xi$}}


% Sans Serif small case

\newcommand{\Gsf}{{\sf G}}

\newcommand{\asf}{{\sf a}}
\newcommand{\bsf}{{\sf b}}
\newcommand{\csf}{{\sf c}}
\newcommand{\dsf}{{\sf d}}
\newcommand{\esf}{{\sf e}}
\newcommand{\fsf}{{\sf f}}
\newcommand{\gsf}{{\sf g}}
\newcommand{\hsf}{{\sf h}}
\newcommand{\isf}{{\sf i}}
\newcommand{\jsf}{{\sf j}}
\newcommand{\ksf}{{\sf k}}
\newcommand{\lsf}{{\sf l}}
\newcommand{\msf}{{\sf m}}
\newcommand{\nsf}{{\sf n}}
\newcommand{\osf}{{\sf o}}
\newcommand{\psf}{{\sf p}}
\newcommand{\qsf}{{\sf q}}
\newcommand{\rsf}{{\sf r}}
\newcommand{\ssf}{{\sf s}}
\newcommand{\tsf}{{\sf t}}
\newcommand{\usf}{{\sf u}}
\newcommand{\wsf}{{\sf w}}
\newcommand{\vsf}{{\sf v}}
\newcommand{\xsf}{{\sf x}}
\newcommand{\ysf}{{\sf y}}
\newcommand{\zsf}{{\sf z}}


% mixed symbols

\newcommand{\sinc}{{\hbox{sinc}}}
\newcommand{\diag}{{\hbox{diag}}}
\renewcommand{\det}{{\hbox{det}}}
\newcommand{\trace}{{\hbox{tr}}}
\newcommand{\sign}{{\hbox{sign}}}
\renewcommand{\arg}{{\hbox{arg}}}
\newcommand{\var}{{\hbox{var}}}
\newcommand{\cov}{{\hbox{cov}}}
\newcommand{\Ei}{{\rm E}_{\rm i}}
\renewcommand{\Re}{{\rm Re}}
\renewcommand{\Im}{{\rm Im}}
\newcommand{\eqdef}{\stackrel{\Delta}{=}}
\newcommand{\defines}{{\,\,\stackrel{\scriptscriptstyle \bigtriangleup}{=}\,\,}}
\newcommand{\<}{\left\langle}
\renewcommand{\>}{\right\rangle}
\newcommand{\herm}{{\sf H}}
\newcommand{\trasp}{{\sf T}}
\newcommand{\transp}{{\sf T}}
\renewcommand{\vec}{{\rm vec}}
\newcommand{\Psf}{{\sf P}}
\newcommand{\SINR}{{\sf SINR}}
\newcommand{\SNR}{{\sf SNR}}
\newcommand{\MMSE}{{\sf MMSE}}
\newcommand{\REF}{{\RED [REF]}}

% Markov chain
\usepackage{stmaryrd} % for \mkv 
\newcommand{\mkv}{-\!\!\!\!\minuso\!\!\!\!-}

% Colors

\newcommand{\RED}{\color[rgb]{1.00,0.10,0.10}}
\newcommand{\BLUE}{\color[rgb]{0,0,0.90}}
\newcommand{\GREEN}{\color[rgb]{0,0.80,0.20}}

%%%%%%%%%%%%%%%%%%%%%%%%%%%%%%%%%%%%%%%%%%
\usepackage{hyperref}
\hypersetup{
    bookmarks=true,         % show bookmarks bar?
    unicode=false,          % non-Latin characters in AcrobatÕs bookmarks
    pdftoolbar=true,        % show AcrobatÕs toolbar?
    pdfmenubar=true,        % show AcrobatÕs menu?
    pdffitwindow=false,     % window fit to page when opened
    pdfstartview={FitH},    % fits the width of the page to the window
%    pdftitle={My title},    % title
%    pdfauthor={Author},     % author
%    pdfsubject={Subject},   % subject of the document
%    pdfcreator={Creator},   % creator of the document
%    pdfproducer={Producer}, % producer of the document
%    pdfkeywords={keyword1} {key2} {key3}, % list of keywords
    pdfnewwindow=true,      % links in new window
    colorlinks=true,       % false: boxed links; true: colored links
    linkcolor=red,          % color of internal links (change box color with linkbordercolor)
    citecolor=green,        % color of links to bibliography
    filecolor=blue,      % color of file links
    urlcolor=blue           % color of external links
}
%%%%%%%%%%%%%%%%%%%%%%%%%%%%%%%%%%%%%%%%%%%



\hypersetup{
	colorlinks=true,
	linkcolor=darkBlue,
	filecolor=magenta,      
	urlcolor=cyan,
	citecolor=magenta,
}


\begin{document}

% paper title
\title{Efficient Evaluation of Multi-Task \\Robot Policies With Active Experiment Selection}
% You will get a Paper-ID when submitting a pdf file to the conference system
% \author{Author Names Omitted for Anonymous Review. Paper-ID 708}

\author{\authorblockN{Abrar Anwar, Rohan Gupta, Zain Merchant, Sayan Ghosh, Willie Neiswanger, Jesse Thomason}
\authorblockA{University of Southern California \\ Contact: \{\texttt{abrar.anwar}, \texttt{jessetho}\}\texttt{@usc.edu}}
}
% \and
% \authorblockN{Homer Simpson}
% \authorblockA{Twentieth Century Fox\\
% Springfield, USA\\
% Email: homer@thesimpsons.com}
% \and
% \authorblockN{James Kirk\\ and Montgomery Scott}
% \authorblockA{Starfleet Academy\\
% San Francisco, California 96678-2391\\
% Telephone: (800) 555--1212\\
% Fax: (888) 555--1212}}


% avoiding spaces at the end of the author lines is not a problem with
% conference papers because we don't use \thanks or \IEEEmembership


% for over three affiliations, or if they all won't fit within the width
% of the page, use this alternative format:
% 
%\author{\authorblockN{Michael Shell\authorrefmark{1},
%Homer Simpson\authorrefmark{2},
%James Kirk\authorrefmark{3}, 
%Montgomery Scott\authorrefmark{3} and
%Eldon Tyrell\authorrefmark{4}}
% \authorblockA{\authorrefmark{1}School of Electrical and Computer Engineering\\
% Georgia Institute of Technology,
%Atlanta, Georgia 30332--0250\\ Email: mshell@ece.gatech.edu}
%\authorblockA{\authorrefmark{2}Twentieth Century Fox, Springfield, USA\\
%Email: homer@thesimpsons.com}
%\authorblockA{\authorrefmark{3}Starfleet Academy, San Francisco, California 96678-2391\\
%Telephone: (800) 555--1212, Fax: (888) 555--1212}
%\authorblockA{\authorrefmark{4}Tyrell Inc., 123 Replicant Street, Los Angeles, California 90210--4321}}


\maketitle

\begin{abstract}
% As robots become increasingly capable at solving many tasks, evaluating these robot policies in the real world is time-consuming for experimenters.
Evaluating learned robot control policies to determine their physical task-level capabilities costs experimenter time and effort.
The growing number of policies and tasks exacerbates this issue.
It is impractical to test every policy on every task multiple times; each trial requires a manual environment reset, and each task change involves re-arranging objects or even changing robots.
Naively selecting a random subset of tasks and policies to evaluate is a high-cost solution with unreliable, incomplete results.
In this work, we formulate robot evaluation as an active testing problem.
We propose to model the distribution of robot performance across all tasks and policies as we \textit{sequentially} execute experiments.
Tasks often share similarities that can reveal potential relationships in policy behavior, and we show that natural language is a useful prior in modeling these relationships between tasks.
% as an experimenter evaluates robot policies.
We then leverage this formulation to reduce the experimenter effort by using a cost-aware expected information gain heuristic to efficiently select informative trials.
Our framework accommodates both continuous and discrete performance outcomes.
We conduct experiments on existing evaluation data from real robots and simulations.
By prioritizing informative trials, our framework reduces the cost of calculating evaluation metrics for robot policies across many tasks. 
% By reducing cost of experiment sampling for robotics, we hope that our framework enhances the efficiency of robot policy testing.

\end{abstract}

\IEEEpeerreviewmaketitle


\section{Introduction}
Backdoor attacks pose a concealed yet profound security risk to machine learning (ML) models, for which the adversaries can inject a stealth backdoor into the model during training, enabling them to illicitly control the model's output upon encountering predefined inputs. These attacks can even occur without the knowledge of developers or end-users, thereby undermining the trust in ML systems. As ML becomes more deeply embedded in critical sectors like finance, healthcare, and autonomous driving \citep{he2016deep, liu2020computing, tournier2019mrtrix3, adjabi2020past}, the potential damage from backdoor attacks grows, underscoring the emergency for developing robust defense mechanisms against backdoor attacks.

To address the threat of backdoor attacks, researchers have developed a variety of strategies \cite{liu2018fine,wu2021adversarial,wang2019neural,zeng2022adversarial,zhu2023neural,Zhu_2023_ICCV, wei2024shared,wei2024d3}, aimed at purifying backdoors within victim models. These methods are designed to integrate with current deployment workflows seamlessly and have demonstrated significant success in mitigating the effects of backdoor triggers \cite{wubackdoorbench, wu2023defenses, wu2024backdoorbench,dunnett2024countering}.  However, most state-of-the-art (SOTA) backdoor purification methods operate under the assumption that a small clean dataset, often referred to as \textbf{auxiliary dataset}, is available for purification. Such an assumption poses practical challenges, especially in scenarios where data is scarce. To tackle this challenge, efforts have been made to reduce the size of the required auxiliary dataset~\cite{chai2022oneshot,li2023reconstructive, Zhu_2023_ICCV} and even explore dataset-free purification techniques~\cite{zheng2022data,hong2023revisiting,lin2024fusing}. Although these approaches offer some improvements, recent evaluations \cite{dunnett2024countering, wu2024backdoorbench} continue to highlight the importance of sufficient auxiliary data for achieving robust defenses against backdoor attacks.

While significant progress has been made in reducing the size of auxiliary datasets, an equally critical yet underexplored question remains: \emph{how does the nature of the auxiliary dataset affect purification effectiveness?} In  real-world  applications, auxiliary datasets can vary widely, encompassing in-distribution data, synthetic data, or external data from different sources. Understanding how each type of auxiliary dataset influences the purification effectiveness is vital for selecting or constructing the most suitable auxiliary dataset and the corresponding technique. For instance, when multiple datasets are available, understanding how different datasets contribute to purification can guide defenders in selecting or crafting the most appropriate dataset. Conversely, when only limited auxiliary data is accessible, knowing which purification technique works best under those constraints is critical. Therefore, there is an urgent need for a thorough investigation into the impact of auxiliary datasets on purification effectiveness to guide defenders in  enhancing the security of ML systems. 

In this paper, we systematically investigate the critical role of auxiliary datasets in backdoor purification, aiming to bridge the gap between idealized and practical purification scenarios.  Specifically, we first construct a diverse set of auxiliary datasets to emulate real-world conditions, as summarized in Table~\ref{overall}. These datasets include in-distribution data, synthetic data, and external data from other sources. Through an evaluation of SOTA backdoor purification methods across these datasets, we uncover several critical insights: \textbf{1)} In-distribution datasets, particularly those carefully filtered from the original training data of the victim model, effectively preserve the model’s utility for its intended tasks but may fall short in eliminating backdoors. \textbf{2)} Incorporating OOD datasets can help the model forget backdoors but also bring the risk of forgetting critical learned knowledge, significantly degrading its overall performance. Building on these findings, we propose Guided Input Calibration (GIC), a novel technique that enhances backdoor purification by adaptively transforming auxiliary data to better align with the victim model’s learned representations. By leveraging the victim model itself to guide this transformation, GIC optimizes the purification process, striking a balance between preserving model utility and mitigating backdoor threats. Extensive experiments demonstrate that GIC significantly improves the effectiveness of backdoor purification across diverse auxiliary datasets, providing a practical and robust defense solution.

Our main contributions are threefold:
\textbf{1) Impact analysis of auxiliary datasets:} We take the \textbf{first step}  in systematically investigating how different types of auxiliary datasets influence backdoor purification effectiveness. Our findings provide novel insights and serve as a foundation for future research on optimizing dataset selection and construction for enhanced backdoor defense.
%
\textbf{2) Compilation and evaluation of diverse auxiliary datasets:}  We have compiled and rigorously evaluated a diverse set of auxiliary datasets using SOTA purification methods, making our datasets and code publicly available to facilitate and support future research on practical backdoor defense strategies.
%
\textbf{3) Introduction of GIC:} We introduce GIC, the \textbf{first} dedicated solution designed to align auxiliary datasets with the model’s learned representations, significantly enhancing backdoor mitigation across various dataset types. Our approach sets a new benchmark for practical and effective backdoor defense.




\section{Related Work}

\subsection{Large 3D Reconstruction Models}
Recently, generalized feed-forward models for 3D reconstruction from sparse input views have garnered considerable attention due to their applicability in heavily under-constrained scenarios. The Large Reconstruction Model (LRM)~\cite{hong2023lrm} uses a transformer-based encoder-decoder pipeline to infer a NeRF reconstruction from just a single image. Newer iterations have shifted the focus towards generating 3D Gaussian representations from four input images~\cite{tang2025lgm, xu2024grm, zhang2025gslrm, charatan2024pixelsplat, chen2025mvsplat, liu2025mvsgaussian}, showing remarkable novel view synthesis results. The paradigm of transformer-based sparse 3D reconstruction has also successfully been applied to lifting monocular videos to 4D~\cite{ren2024l4gm}. \\
Yet, none of the existing works in the domain have studied the use-case of inferring \textit{animatable} 3D representations from sparse input images, which is the focus of our work. To this end, we build on top of the Large Gaussian Reconstruction Model (GRM)~\cite{xu2024grm}.

\subsection{3D-aware Portrait Animation}
A different line of work focuses on animating portraits in a 3D-aware manner.
MegaPortraits~\cite{drobyshev2022megaportraits} builds a 3D Volume given a source and driving image, and renders the animated source actor via orthographic projection with subsequent 2D neural rendering.
3D morphable models (3DMMs)~\cite{blanz19993dmm} are extensively used to obtain more interpretable control over the portrait animation. For example, StyleRig~\cite{tewari2020stylerig} demonstrates how a 3DMM can be used to control the data generated from a pre-trained StyleGAN~\cite{karras2019stylegan} network. ROME~\cite{khakhulin2022rome} predicts vertex offsets and texture of a FLAME~\cite{li2017flame} mesh from the input image.
A TriPlane representation is inferred and animated via FLAME~\cite{li2017flame} in multiple methods like Portrait4D~\cite{deng2024portrait4d}, Portrait4D-v2~\cite{deng2024portrait4dv2}, and GPAvatar~\cite{chu2024gpavatar}.
Others, such as VOODOO 3D~\cite{tran2024voodoo3d} and VOODOO XP~\cite{tran2024voodooxp}, learn their own expression encoder to drive the source person in a more detailed manner. \\
All of the aforementioned methods require nothing more than a single image of a person to animate it. This allows them to train on large monocular video datasets to infer a very generic motion prior that even translates to paintings or cartoon characters. However, due to their task formulation, these methods mostly focus on image synthesis from a frontal camera, often trading 3D consistency for better image quality by using 2D screen-space neural renderers. In contrast, our work aims to produce a truthful and complete 3D avatar representation from the input images that can be viewed from any angle.  

\subsection{Photo-realistic 3D Face Models}
The increasing availability of large-scale multi-view face datasets~\cite{kirschstein2023nersemble, ava256, pan2024renderme360, yang2020facescape} has enabled building photo-realistic 3D face models that learn a detailed prior over both geometry and appearance of human faces. HeadNeRF~\cite{hong2022headnerf} conditions a Neural Radiance Field (NeRF)~\cite{mildenhall2021nerf} on identity, expression, albedo, and illumination codes. VRMM~\cite{yang2024vrmm} builds a high-quality and relightable 3D face model using volumetric primitives~\cite{lombardi2021mvp}. One2Avatar~\cite{yu2024one2avatar} extends a 3DMM by anchoring a radiance field to its surface. More recently, GPHM~\cite{xu2025gphm} and HeadGAP~\cite{zheng2024headgap} have adopted 3D Gaussians to build a photo-realistic 3D face model. \\
Photo-realistic 3D face models learn a powerful prior over human facial appearance and geometry, which can be fitted to a single or multiple images of a person, effectively inferring a 3D head avatar. However, the fitting procedure itself is non-trivial and often requires expensive test-time optimization, impeding casual use-cases on consumer-grade devices. While this limitation may be circumvented by learning a generalized encoder that maps images into the 3D face model's latent space, another fundamental limitation remains. Even with more multi-view face datasets being published, the number of available training subjects rarely exceeds the thousands, making it hard to truly learn the full distibution of human facial appearance. Instead, our approach avoids generalizing over the identity axis by conditioning on some images of a person, and only generalizes over the expression axis for which plenty of data is available. 

A similar motivation has inspired recent work on codec avatars where a generalized network infers an animatable 3D representation given a registered mesh of a person~\cite{cao2022authentic, li2024uravatar}.
The resulting avatars exhibit excellent quality at the cost of several minutes of video capture per subject and expensive test-time optimization.
For example, URAvatar~\cite{li2024uravatar} finetunes their network on the given video recording for 3 hours on 8 A100 GPUs, making inference on consumer-grade devices impossible. In contrast, our approach directly regresses the final 3D head avatar from just four input images without the need for expensive test-time fine-tuning.



\section{Problem Formulation and Notation}

\begin{figure*}[!h]
  \centering
  \includegraphics[width=\linewidth]{figs/overview_v1.pdf}
  \caption{
\textbf{Overview of our method.}
Given a user-provided prompt $\inputprompt$ and a partial sketch $\inputsketch$, our method first (a) stylizes the input prompt by augmenting it using style descriptions generated by the VLM (\textbf{bold text}).
Using the stylized prompt $\finalprompt$, the method then performs (b) stroke optimization to generate strokes that fill the missing regions, thus ensuring that the style-agnostic completed sketch $\intermediatesketch$ can fully represents the content of the user-provided prompt.
To align the styles of $\intermediatesketch$ and $\inputsketch$, we (c) instruct the VLM to generate an executable style adjustment code that modifies the strokes of $\intermediatesketch$.
Finally, we obtain a final completed sketch $\completesketch$ wherein the styles of the strokes are aligned to those of the $\inputsketch$.
}
\label{fig:overview}
\end{figure*}


% In this work, we address the challenge of efficiently evaluating multi-task robot policies in the real world.
The objective of this work is to design an efficient strategy to evaluate robot policies across tasks while balancing the cost of experimentation. 
Consider a fixed set of $M$ robot policies, denoted by $\mathcal{P} = \{\pi_1, \pi_2, \ldots, \pi_M\}$ and a set of $N$ tasks $\mathcal{T} = \{T_1, T_2, ..., T_N\}$. 
% Due to practical constraints, only a subset of $N$ tasks $A \subset \mathcal{T}$ is used.
Each task $T_j \in \mathcal{T}$ is a finite-horizon MDP defined by states, actions, and a high-level natural language instruction $L_i$.
% Since robot policies can be sensitive to changes in language instructions~\cite{}, we assume a one-to-one mapping between language instructions and tasks.

Our framework is policy-agnostic and does not assume access to policy model weights, and can be applied to engineered robot \textit{systems} in addition to end-to-end models.

\textbf{Population Parameter Estimation}. 
We formulate the problem as population parameter estimation, similar to probabilistic matrix factorization~\cite{mnih2007probabilistic}.
Let the performance of a policy ${\pi_i \in \mathcal{P}}$ on a task $T_j \in \mathcal{T}$ be represented by the random variable $X_{ij}$ with distribution $P_{ij}$, from which we can sample evaluations $\trial \sim \truedist$. Here, $\truedist$ represents the ``true'' performance distribution.
Since the underlying distribution $\truedist$ is unknown, the goal of population parameter estimation is to estimate a distribution $\learneddist$ that models real-world evaluation outcomes from $\truedist$.
% , which is the distribution corresponding to a random variable $\hat{X}_{ij}$, 
We use $\theta_{ij}$ to represent the parameters of the learned distribution $\learneddist$.
For example, $\theta_{ij}=[\mu, \sigma]$ if $\learneddist$ is a Gaussian distribution.
Given a limited number of observed samples from the true distribution, $\trial^1, ..., \trial^n \sim \truedist$, the goal is to estimate the parameters of an estimated distribution $\theta_{ij}$.
Our setting also has samples from other random variables, $X_{kl}$ corresponding to different policy-task pairs. 
Therefore, in this work we want to estimate $\Theta = \{\theta_{ij}\}_{i, j = 1}^{i=M, j = N}$ for all policy-task pairs given a dataset $\mathcal{D}=\{\trial^k\}$.
These distributions can be visualized as a grid of policy-task pairs as shown in Figure~\ref{fig:overview}.

The aim is to estimate the parameters of $\learneddist$ of all policy-task combinations by leveraging shared information across this matrix.
However, it is infeasible to directly evaluate all policy-task pairs due to cost constraints. 
Therefore, we adopt an active testing approach, where the objective is to iteratively select the most informative experiments $(\pi_i, T_j)$ to efficiently learn $\Theta$. 
% Given a limited evaluation budget, we develop active experiment selection strategies to efficiently samples evaluations that improve the accuracy of these parameter estimates.

\textbf{Active Testing.}
% Since it is difficult to sample a large set of outcomes $\mathcal{D}$ due to experimenter effort, we take an active testing approach.
We apply an active learning paradigm to learn a population parameter estimator $f(\pi_i, T_j)$.
As such, we define acquisition functions to guide the selection of task-policy pairs or tasks alone, and then sample experiments that are most informative.
First, we define an acquisition function $a(\pi_i, T_j)$, and the next experiment is selected by maximizing this function over all possible experiments:
\begin{equation}
    (\pi_i^*, T_j^*) = \arg\max_{(\pi_i, T_j)} a(\pi_i, T_j).
\end{equation}
Although these acquisition functions are informative, we want a balance between selecting informative experiments and their costs.

\textbf{Evaluation Cost}. In real-world evaluation, each policy-task evaluation incurs a cost. 
Let $c_{\text{eval}}(T_j)$ denote the cost of a single evaluation of a policy on task $T_j$.
We make a simplifying assumption that this cost is agnostic to changes in the policy under evaluation, that often being a configurable software option.
This cost could include the time required to execute the policy, the resources consumed during evaluation, or the manual supervision required to reset the scene.
Furthermore, switching between tasks typically incurs a larger cost involving a reconfiguring the scene or the robot. 
We define this switching cost $c_{\text{switch}}(T_j, T_k)$ as the cost associated with transitioning from task $T_j$ to $T_k$.
% This cost depends on the specific tasks being evaluated.
For a sequence of tasks that have been evaluated $T_{i_1}, \ldots, T_{i_L}$ (where each $i_j \in N$), we compute the total cost of evaluation as:

% Do we need this part? wrote it like this to take up more space
$$c_{\text{total}} = \sum_{j=1}^N c_{\text{eval}}(T_{i_j}) + \sum_{j=1}^{N-1} c_{\text{switch}}(T_{i_j}, T_{i_{j+1}})$$

Given these costs, the problem is to design an evaluation strategy that minimizes the total cost of evaluation while learning the population parameters of test instances.




\section{Method}

We aim to design a framework for sampling experiments for multi-task robot policies.
Our framework consists of two parts: (1) learning a surrogate model to estimate the population parameters of a test instance and (2) designing strategies to sample experiments in a cost-efficient manner.
The surrogate model leverages task and policy representations that define an experiment to have a better estimate of the overall performance distributions.
Then, we use this surrogate model to compute the expected information gain of different experiments.
We then use the expected information gain along with the cost of switching tasks to conduct active testing.

% \textbf{Surrogate Model.}
\subsection{Surrogate Model}
As we evaluate our robot policies across tasks, we track the outcomes of each trial to aggregate a dataset $\mathcal{D}$ over time.
% In the process of evaluating multi-task robot policies, we collect a dataset of previous outcomes from trials $\trial \in \mathcal{D}$.
Each of these outcomes are realizations of a true underlying distribution $\truedist$.
Our goal is to learn a surrogate model from $\mathcal{D}$ that predicts the population parameters $\theta_{ij}$ of a performance distribution  $\learneddist$. 
As more evaluation rollouts are conducted, we add the outcomes to $\mathcal{D}$ and continue training the surrogate model.
% A single experiment can be defined by the policy $\pi_i$ being executed on task $t_j$.

%The surrogate model should be able to capture the performance relationship between tasks and policies.
To train an effective surrogate model,  we use notions of similarity between tasks and policies. 
% For instance, tasks that involve similar skills, such as "picking up an apple" and "picking up an orange," should result in similar performance distributions for a given robot policy.
Thus, we need a representation that captures the similarities between policies and tasks with respect to their performance distributions.
We define a policy embedding $e_{\pi_i}$ and task embedding $e_{T_j}$, where similar performance distributions in task and policy can be captured based on the embeddings.
These policy and task representations are then provided as input to an MLP that predicts the estimated population parameters:
\begin{equation}
    \hat{\theta}_{ij} = f(\pi_i, T_j) = \text{MLP}(e_{\pi_i}, e_{T_j}).
\end{equation}

\textbf{Task and Policy Representation.} 
\label{sec:method}
To define the task and policy embeddings $e_{\pi_i}, e_{T_j}$, we design various types of embeddings.
In practice, we cannot know the relationship between policies in advance as we are conducting evaluation.
Therefore, we define the policy embedding to be a fixed, randomly initialized embedding to act as an identifier for the policy in a given experiment.

For the task embedding $e_{\pi_i}$, we leverage language embeddings from MiniLMv2~\cite{gu2024minillm} which we reduce to 32 dimensions using PCA over all tasks.
However, we found that language embeddings overly focus on nouns as opposed to verbs, which causes issues as actions with similar nouns but different verbs would be closer together verbs with the same nouns. 
Thus, we apply the following procedure to mitigate this issue.
We (1) use part-of-speech tagging to extract all verbs and verb phrases, (2) compute a language embedding for the verb $e_{T_j}^{\text{verb}}$ and for the entire task description  $e_{t_j}^{\text{task}}$, and then (3) compute the task embedding 
\begin{equation}
     e_{T_j} =  0.8 \cdot e_{T_j}^{\text{verb}} +  0.2\cdot e_{T_j}^{\text{task}} + 0.1\cdot \mathcal{N}(0,1).
\end{equation}
We also found that the embeddings were often too close across multiple tasks, and we found that adding a slight noise term helped separate close embeddings.
Experiments on this result are in Section~\ref{sec:task_results}.
% These embeddings were them 

\textbf{Population Parameter Estimation.}
Outcomes in robot learning can take the form of continuous values like rewards, time to completion, or task progress, and binary values like task success.
Thus, the underlying distribution from the surrogate model depends on the type of task.
We consider two types of underlying distributions.
When $X_{ij}$ is continuous, $\learneddist$ takes the form of a mixture of Gaussians with $K$ components,
\begin{equation}
    % P(X_{ij}; \theta_{ij}) = \sum_{k=1}^{K}\pi_k\mathcal{N}(\mu_k, \sigma_k),\\
    \sample \sim \learneddist = \sum_{k=1}^{K}p_k\mathcal{N}(\mu_k, \sigma_k),
\end{equation}
where $\pi_k, \mu_k,$ and $\sigma_k$ are the mixing coefficients, means, and standard deviations of the Gaussian components respectively that are predicted from the surrogate model $\theta_{ij} = f(\pi_i, T_j)$.
% The parameters $\theta_{ij}=\{\pi_k, \mu_k, \sigma_k\}^{K}_{k=1}$ are predicted using the surrogate model.
We thus train the surrogate model with a mixture density loss~\cite{bishop1994mixture,ha2018world} to minimize the negative log-likelihood of the observed data under the mixture model.
In our experiments on continuous outcome distributions, we use $K=2$ Gaussian components, as robotics performance is often bimodal; robots either fail catastrophically or they maintain non-zero performance.

In the case where $X_{ij}$ is binary, indicating success or failure, $\learneddist$ takes the form of a Bernoulli distribution:
\begin{equation}
    % P(X_{ij}; \theta_{ij}) = p^{x_{ij}}(1-p)^{1-x_{ij}}
    \sample \sim \learneddist =  p^{x_{ij}}(1-p)^{1-x_{ij}},
\end{equation}
where $\theta_{ij} = \{p \in [0,1] \}$ represents the success probability predicted by the surrogate model trained using cross-entropy loss. 



\subsection{Cost-aware Active Experiment Selection}

% We can now use this surrogate model for selecting policy-task robot experience to execute. 
We explore cost-aware, active-experiment acquisition functions that guide selection of experiments based on their expected utility while considering associated costs.
To define the acquisition function, we first focus on how to measure the informativeness of a policy-task evaluation, which we capture through expected information gain.

\textbf{Expected Information Gain.}
Expected Information Gain (EIG) quantifies the value of an experiment by estimating how much it reduces the predictive uncertainty of the performance distribution for a policy-task pair. 
Since the surrogate model estimates performance \textit{distributions}, we define the EIG of a policy-task pair using a Bayesian Active Learning by Disagreement (BALD)~\cite{houlsby2011bayesian} formulation for probabilistic models
\begin{equation}
    \mathcal{I}(\pi_i, T_j) = \underbrace{\mathbb{H}[\learneddist]}_{\text{marginal entropy}} - \underbrace{\mathbb{E}_{\theta_{ij} \sim f(\theta_{ij}|\mathcal{D})} [\mathbb{H}[\learneddist | \theta_{ij}]]}_{\text{expected conditional entropy}}.
\end{equation}
% The first term is the entropy over the marginal predictive distribution
The first term represents the marginal entropy over $Q_{ij}$, which quantifies the total uncertainty in $Q_{ij}$. 
The second term corresponds to the expected conditional entropy over multiple samples of parameters $\theta_{ij}$.
% where $P(\learneddist|\mathcal{D})$ is the marginal predictive distribution over the performance distribution $\learneddist,$ while $P(\learneddist | \theta_{ij})$ is the conditional predictive distribution given a sampled set of parameters $\theta_{ij}$. 
% Thus, the first term is the entropy over the marginal predictive distribution and captures the epistemic and aleatoric uncertainties of the model's predictions, while the second term captures the expected aleatoric uncertainty.
Thus, $\mathcal{I}(\pi_i, T_j)$ captures the disagreement between multiple samples of distributions.
For example, if 10 sampled parameters for a Gaussian have very different distributions, then their disagreement will be high.
Since the entropy of a mixture of Gaussians generally lacks a closed-form solution, we estimate the entropy by discretizing the empirical distribution into $n=25$ bins for which to compute entropy over.

BALD ensures the EIG score is higher in test instances where there is disagreement in the predicted distributions across sampled parameters.
In this case, we define the acquisition functions $a(\pi_i,T_j)=\mathcal{I}(\pi_i,T_j)$.



To compute the expected information gain, we require multiple samples of $\Theta_{ij}$; however, we only train a single MLP.
Inspired by Monte Carlo dropout~\cite{gal2016dropout} and past literature~\cite{loquercio2020general,ledda2023dropout}, we apply dropout only at test-time to compute multiple samples of $\theta_{ij}$ from the surrogate model $f(\cdot)$.
% We found that dropout early in the sampling process would lead to overfitting on the small dataset, leading to low entropy in the samples of $\Theta_{ij}$.

% We use Monte Carlo (MC) dropout~\cite{gal2016dropout} to estimate sampling from $f(\Theta_{i,j} | \mathcal{D})$. 
% We found that Monte Carlo (MC) dropout~\cite{gal2016dropout}, where dropout is applied during training time, then at test time to compute multiple samples of $\theta_{ij}$ from the surrogate model $f(\cdot)$, leads to overfitting on the small amounts of training data available early in the evaluation process. 
% This overfitting lead to the disagreement between sampled parameters to be nearly zero, causing the EIG scores to be useful after hundreds of expensive evaluations.


% However, in the setting of robot evaluation where experiments are expensive, our surrogate model is initially trained with minimal data.
% We found that dropout during training time early in the evaluation process led to overfitting, as few examples had been seen, which caused the disagreement between sampled parameters to be essentially zero.
% Thus, the EIG scores would only be useful after hundreds of expensive evaluations.
% Past work in the active learning literature~\cite{tosh2022targeted} avoids this cold start problem by initializing their model with hundreds or thousands of training instances, which is impractical for an evaluation framework.
% To address this cold start problem, we found that restricting dropout to test time only effectively allowed our model to provide useful EIG scores earlier in the evaluation process.


\textbf{Cost-Aware EIG}.
While EIG effectively quantifies the informativeness of an experiment, it does not consider the costs of conducting evaluation.
To make EIG cost-aware, we design the following acquisition function based on prior work that simply integrates cost with a multiplicative factor~\cite{paria2020cost,lee2020cost}:
\begin{equation}
    a_{\text{cost-aware}}(\pi_i, T_j, T_\text{current}) = \dfrac{\mathcal{I}(\pi_i, T_j)}{(\lambda \cdot c_{\text{switch}}(T_{\text{current}}, T_j))+1},
    \label{eq:cost_aware_eig}
\end{equation}
where $\mathcal{I}(\pi_i, T_j)$ represents EIG for the policy $\pi_i$ on task $T_j$, $c_{\text{switch}}(T_{\text{current}}, T_j))$ is the cost of switching from the current task $T_{\text{current}}$ to a new task $T_j$, and $\lambda$ is a hyperparameter that controls the cost sensitivity.

\begin{algorithm}[h!]
\caption{Gait-Net-augmented Sequential CMPC}
\label{alg:gaitMPC}
\begin{algorithmic}[1]
\Require $\mathbf q, \: \dot{\mathbf q}, \: \mathbf q^\text{cmd}, \: \dot{\mathbf q}^\text{cmd}$
\State \textbf{intialize} $\bm x_0 = f_\text{j2m}(\mathbf q, \: \dot{\mathbf q}), \: \bm u^0 =\bm u_\text{IG}, \: dt^0 = 0.05$ 
\State $\{ \mathbf q^\text{ref},\:\dot{\mathbf q}^\text{ref},\:\bm p_f^\text{ref}\} = f_\text{ref} \big(\mathbf q, \: \dot{\mathbf q}, \: \mathbf q^\text{cmd}, \: \dot{\mathbf q}^\text{cmd} \big)$
\State $\bm x^\text{ref} = f_\text{j2m}(\mathbf q^\text{ref},\:\dot{\mathbf q}^\text{ref},\:\bm p_f^\text{ref})$
\State $ j = 0$ 
\While{$j \leq j_\text{max} \:\text{and}\: \bm \eta \leq \delta \bm u  $} 
\State $\delta \bm u^{j} = \texttt{cmpc}(\bm x^\text{ref},\:\bm p_f^\text{ref},\:\bm p_c^\text{ref},\: \bm x_0,\: dt^j, \: \bm u^j)$
\State $\bm u^{j+1} = \bm u^j + \delta \bm u^j$ 
\State $dt^{j+1} = \Pi_\text{GN}(\mathbf q, \: \dot{\mathbf q},\: \bm p_f^{j})$
\State $\{ \bm x^\text{ref},\:\bm p_f^\text{ref}\}= f_\text{IK}(\bm p_f^{j},\:\bm p_c^{j},\: dt^{j+1})$
\State $j=j+1$
\EndWhile \\
\Return $\bm u^{j+1} $
\end{algorithmic}
\end{algorithm}


\textbf{Active Experiment Selection}.
% As shown in Algorithm~\ref{alg:short_pseudocode}, we use this acquisition function to iteratively sample experiments.
We use this acquisition function to iteratively sample experiments, as shown in Algorithm~\ref{alg:short_pseudocode}.
To mitigate the cold-start problem in active learning, we initialize the dataset $\mathcal{D}$ with a a single randomly-selected task, for which every policy is evaluated 3 times.
We then train the surrogate model on this data.
At each query step, the acquisition function $a(\pi_i, T_j)$ is computed for all policy-task pairs, which quantifies their informativeness weighted by the cost.
To compute the entropy over model parameters for the EIG metric, we use MC dropout to sample 10 predicted outcome distributions.
To balance exploration and exploitation, we use an epsilon-greedy strategy with a rate of $\epsilon=0.1$.
The selected experiment $(\pi_i, T_j)$ is then executed 3 times, and the observed outcomes are added to the dataset $\mathcal{D}$.
We found in preliminary experiments that 3 trials per selected experiment was often better for cost-efficient population parameter estimation.
Given these new outcomes in the dataset, we keep training the surrogate model on the updated dataset improve its predictions over time.
% In our experiments, we continue this process for a fixed number of queries; however, 

\section{Evaluation on Offline Datasets}



To evaluate our active testing framework, we leverage evaluations that have already been conducted which we then sample offline.
We use experiments from the HAMSTER paper~\cite{li2025hamster}, the  OpenVLA paper~\cite{kim24openvla}, and from MetaWorld~\cite{yu2020meta}, as visualized in Figure~\ref{fig:datasets}.
For MetaWorld, we train two versions, one focused on understanding our framework's ability in evaluating different policies and another on evaluating multiple checkpoints of a single policy.
Each of these datasets can be modeled with different underlying distributions and have varying costs, semantic diversity, and skills.
More details on training for MetaWorld, switching costs for the datasets, and other details can be found in Appendix~\ref{app:offline_datasets}.

\begin{figure}[!t]
    \centering
    \includegraphics[width=.93\linewidth]{figs_images/datasets.pdf}
    \caption{\textbf{Offline Datasets used for Experiments.} We consider 4 settings: (1) evaluations from HAMSTER~\cite{li2025hamster}, (2) evaluations from the OpenVLA paper~\cite{kim24openvla}, (3) MetaWorld~\cite{yu2020meta} where we evaluate different policies, and (4) MetaWorld where we evaluate multiple checkpoints of a single policy. For the MetaWorld evaluations, we can model the performance distributions of success rate or continuous rewards. For OpenVLA, the outcomes are binary success rate. For HAMSTER, evaluations were run over a large number of tasks only once while tracking only task progress, so we use this mean value as a mean for a unimodal Gaussian and a fixed standard deviation.
    }
    \label{fig:datasets}
\end{figure}




\textbf{HAMSTER.}
We use evaluations from the HAMSTER paper~\cite{li2025hamster}, which evaluates a hierarchical VLA model against 4 other policies such as OpenVLA~\cite{kim24openvla} and Octo~\cite{octo_2023} across 81 tasks. 
These 81 tasks are of varying complexity, with diverse task types, objects, and linguistic variation that were evaluated once each.
Their work uses a continuous task progress metric; however, since they only evaluated each policy-task pair once, we treat the single continuous value as the mean of a Gaussian distribution with a fixed standard deviation.
For switching cost, we add an additional cost if the policy switches from one task type to another. 
More details on this cost can be found in Appendix~\ref{app:offline_datasets}.
% The success of that one policy-task pair evaluation is treated as the mean success metric for that experiment. 

\textbf{OpenVLA.}
We use evaluations from the OpenVLA paper~\cite{kim24openvla}, which compares 4 policies over 29 tasks. 
In their paper, some tasks allow for partial success (0.5). 
For simplicity, we round the partial successes down to maintain a binary success metric.
OpenVLA also provides results across two embodiments.
Therefore, in addition to a higher cost term to switching tasks that require a large scene reset, we add an additional cost term to switch between embodiments. More details in Appendix~\ref{app:offline_datasets}.

Given these datasets, we show that the types of policy and task representations that are useful for active learning, and then we can leverage the surrogate model for cost-aware active experiment selection. 
\begin{figure*}[t]
    \centering
    % placeholder
    \includegraphics[width=1\linewidth]{figs_images/task_rep.pdf}
    \caption{\textbf{Task and Policy Representation Experiments.} 
    We compute the average log likelihood of all outcomes under probability distribution represented by the predicted population parameters across various policy and task representations. We evaluate these methods over the HAMSTER, OpenVLA, and MetaWorld Checkpoints offline evaluation datasets over \textcolor{mypink}{continuous} and \textcolor{mygold}{binary} performance distributions.
    We find no large difference between random or optimal embeddings as a policy representation, indicating that there is not much shared information between policies.
    However, we find that for task representation, \textcolor{optimal}{\textbf{Optimal}} consistently perform the best, followed by \textcolor{verb}{\textbf{Verb}}, then \textcolor{lang}{\textbf{Lang}}, and lastly \textcolor{random}{\textbf{Random}}.
    Language-based embeddings is a good task representation that we can leverage for better active learning.
    % We find that optimal embeddings for tasks and policies perform the best, while random embeddings perform the worst. 
    % We find that language embeddings perform better as a task representation than random embeddings as it gets closer to the rate at which optimal embeddings improve.
    % \vspace{-1em}
    }
    \label{fig:model_exps}
\end{figure*}



\textbf{MetaWorld Policies.}
MetaWorld~\cite{yu2020meta} is an open-source simulated benchmark containing a set of 50 different manipulation environments for multi-task learning.
We train 10 policies on every environment with different policy architecture sizes and varying amounts of noise in the robot's state to create robot policies with diverse behaviors.
We then collected 100 trajectories of each policy-task pair to serve as an approximation of the true performance population distribution.
By using the MetaWorld simulator, we can estimate performance distributions for binary success rate and a continuous reward normalized between 0 and 1.
The switching cost is set based on whether the target object of the scene, such as a drawer, is swapped out for another object, like a lever.
This dataset allows us to understand how our framework can learn the performance distributions across diverse policies.

\textbf{MetaWorld Checkpoints.}
Evaluation on a robot is not only used for comparing policies, but also to find the best checkpoints.
As such, we train a single state-based MetaWorld policy, store 11 checkpoints over the training process, and then evaluate them.
In preliminary experiments, we found that the checkpoint-based setting has a lower-rank structure in terms of the performance distributions.
This offline dataset allows us to exploit the shared information across policies.

Given these datasets, we will discuss two experiments in the next two sections: that shows that language is an informative prior in modeling the performance relationships between tasks, and that our surrogate model can be used for cost-aware experiment selection.

% . Given an evaluation score of 7.5, for example, we generate a sequence of seven successes (1s) and three failures (0s).


% This trajectory data was made of pairs of a state and the action taken from that state. Then, for each environment, we generated a natural language query that describes the task in the environment. We then trained an MLP-based model using the natural language query, embedded as a 768-dimensional vector, and the 39-dimensional state vector in the trajectory to output the 4-dimensional action vector. The resulting model is a language-conditioned behavior-cloned policy trained on all the language instructions and environment rollouts. This model is effectively trained to be a single generalist language-conditioned policy for the Metaworld environment.

% We trained 10 of these generalist policies for Metaworld, varying the training data with randomness to ensure each policy was fairly distinct from another, and then rolled out each of them 100 times each on each of the 50 multi-task learning environments. The success and reward data is used in our parameter estimations for policy evaluation. 





% Effective human-robot cooperation in CoNav-Maze hinges on efficient communication. Maximizing the human’s information gain enables more precise guidance, which in turn accelerates task completion. Yet for the robot, the challenge is not only \emph{what} to communicate but also \emph{when}, as it must balance gathering information for the human with pursuing immediate goals when confident in its navigation.

To achieve this, we introduce \emph{Information Gain Monte Carlo Tree Search} (IG-MCTS), which optimizes both task-relevant objectives and the transmission of the most informative communication. IG-MCTS comprises three key components:
\textbf{(1)} A data-driven human perception model that tracks how implicit (movement) and explicit (image) information updates the human’s understanding of the maze layout.
\textbf{(2)} Reward augmentation to integrate multiple objectives effectively leveraging on the learned perception model.
\textbf{(3)} An uncertainty-aware MCTS that accounts for unobserved maze regions and human perception stochasticity.
% \begin{enumerate}[leftmargin=*]
%     \item A data-driven human perception model that tracks how implicit (movement) and explicit (image transmission) information updates the human’s understanding of the maze layout.
%     \item Reward augmentation to integrate multiple objectives effectively leveraging on the learned perception model.
%     \item An uncertainty-aware MCTS that accounts for unobserved maze regions and human perception stochasticity.
% \end{enumerate}

\subsection{Human Perception Dynamics}
% IG-MCTS seeks to optimize the expected novel information gained by the human through the robot’s actions, including both movement and communication. Achieving this requires a model of how the human acquires task-relevant information from the robot.

% \subsubsection{Perception MDP}
\label{sec:perception_mdp}
As the robot navigates the maze and transmits images, humans update their understanding of the environment. Based on the robot's path, they may infer that previously assumed blocked locations are traversable or detect discrepancies between the transmitted image and their map.  

To formally capture this process, we model the evolution of human perception as another Markov Decision Process, referred to as the \emph{Perception MDP}. The state space $\mathcal{X}$ represents all possible maze maps. The action space $\mathcal{S}^+ \times \mathcal{O}$ consists of the robot's trajectory between two image transmissions $\tau \in \mathcal{S}^+$ and an image $o \in \mathcal{O}$. The unknown transition function $F: (x, (\tau, o)) \rightarrow x'$ defines the human perception dynamics, which we aim to learn.

\subsubsection{Crowd-Sourced Transition Dataset}
To collect data, we designed a mapping task in the CoNav-Maze environment. Participants were tasked to edit their maps to match the true environment. A button triggers the robot's autonomous movements, after which it captures an image from a random angle.
In this mapping task, the robot, aware of both the true environment and the human’s map, visits predefined target locations and prioritizes areas with mislabeled grid cells on the human’s map.
% We assume that the robot has full knowledge of both the actual environment and the human’s current map. Leveraging this knowledge, the robot autonomously navigates to all predefined target locations. It then randomly selects subsequent goals to reach, prioritizing grid locations that remain mislabeled on the human’s map. This ensures that the robot’s actions are strategically focused on providing useful information to improve map accuracy.

We then recruited over $50$ annotators through Prolific~\cite{palan2018prolific} for the mapping task. Each annotator labeled three randomly generated mazes. They were allowed to proceed to the next maze once the robot had reached all four goal locations. However, they could spend additional time refining their map before moving on. To incentivize accuracy, annotators receive a performance-based bonus based on the final accuracy of their annotated map.


\subsubsection{Fully-Convolutional Dynamics Model}
\label{sec:nhpm}

We propose a Neural Human Perception Model (NHPM), a fully convolutional neural network (FCNN), to predict the human perception transition probabilities modeled in \Cref{sec:perception_mdp}. We denote the model as $F_\theta$ where $\theta$ represents the trainable weights. Such design echoes recent studies of model-based reinforcement learning~\cite{hansen2022temporal}, where the agent first learns the environment dynamics, potentially from image observations~\cite{hafner2019learning,watter2015embed}.

\begin{figure}[t]
    \centering
    \includegraphics[width=0.9\linewidth]{figures/ICML_25_CNN.pdf}
    \caption{Neural Human Perception Model (NHPM). \textbf{Left:} The human's current perception, the robot's trajectory since the last transmission, and the captured environment grids are individually processed into 2D masks. \textbf{Right:} A fully convolutional neural network predicts two masks: one for the probability of the human adding a wall to their map and another for removing a wall.}
    \label{fig:nhpm}
    \vskip -0.1in
\end{figure}

As illustrated in \Cref{fig:nhpm}, our model takes as input the human’s current perception, the robot’s path, and the image captured by the robot, all of which are transformed into a unified 2D representation. These inputs are concatenated along the channel dimension and fed into the CNN, which outputs a two-channel image: one predicting the probability of human adding a new wall and the other predicting the probability of removing a wall.

% Our approach builds on world model learning, where neural networks predict state transitions or environmental updates based on agent actions and observations. By leveraging the local feature extraction capabilities of CNNs, our model effectively captures spatial relationships and interprets local changes within the grid maze environment. Similar to prior work in localization and mapping, the CNN architecture is well-suited for processing spatially structured data and aligning the robot’s observations with human map updates.

To enhance robustness and generalization, we apply data augmentation techniques, including random rotation and flipping of the 2D inputs during training. These transformations are particularly beneficial in the grid maze environment, which is invariant to orientation changes.

\subsection{Perception-Aware Reward Augmentation}
The robot optimizes its actions over a planning horizon \( H \) by solving the following optimization problem:
\begin{subequations}
    \begin{align}
        \max_{a_{0:H-1}} \;
        & \mathop{\mathbb{E}}_{T, F} \left[ \sum_{t=0}^{H-1} \gamma^t \left(\underbrace{R_{\mathrm{task}}(\tau_{t+1}, \zeta)}_{\text{(1) Task reward}} + \underbrace{\|x_{t+1}-x_t\|_1}_{\text{(2) Info reward}}\right)\right] \label{obj}\\ 
        \subjectto \quad
        &x_{t+1} = F(x_t, (\tau_t, a_t)), \quad a_t\in\Ocal \label{const:perception_update}\\ 
        &\tau_{t+1} = \tau_t \oplus T(s_t, a_t), \quad a_t\in \Ucal\label{const:history_update}
    \end{align}
\end{subequations} 

The objective in~\eqref{obj} maximizes the expected cumulative reward over \( T \) and \( F \), reflecting the uncertainty in both physical transitions and human perception dynamics. The reward function consists of two components: 
(1) The \emph{task reward} incentivizes efficient navigation. The specific formulation for the task in this work is outlined in \Cref{appendix:task_reward}.
(2) The \emph{information reward} quantifies the change in the human’s perception due to robot actions, computed as the \( L_1 \)-norm distance between consecutive perception states.  

The constraint in~\eqref{const:history_update} ensures that for movement actions, the trajectory history \( \tau_t \) expands with new states based on the robot’s chosen actions, where \( s_t \) is the most recent state in \( \tau_t \), and \( \oplus \) represents sequence concatenation. 
In constraint~\eqref{const:perception_update}, the robot leverages the learned human perception dynamics \( F \) to estimate the evolution of the human’s understanding of the environment from perception state $x_t$ to $x_{t+1}$ based on the observed trajectory \( \tau_t \) and transmitted image \( a_t\in\Ocal \). 
% justify from a cognitive science perspective
% Cognitive science research has shown that humans read in a way to maximize the information gained from each word, aligning with the efficient coding principle, which prioritizes minimizing perceptual errors and extracting relevant features under limited processing capacity~\cite{kangassalo2020information}. Drawing on this principle, we hypothesize that humans similarly prioritize task-relevant information in multimodal settings. To accommodate this cognitive pattern, our robot policy selects and communicates high information-gain observations to human operators, akin to summarizing key insights from a lengthy article.
% % While the brain naturally seeks to gain information, the brain employs various strategies to manage information overload, including filtering~\cite{quiroga2004reducing}, limiting/working memory, and prioritizing information~\cite{arnold2023dealing}.
% In this context of our setup, we optimize the selection of camera angles to maximize the human operator's information gain about the environment. 

\subsection{Information Gain Monte Carlo Tree Search (IG-MCTS)}
IG-MCTS follows the four stages of Monte Carlo tree search: \emph{selection}, \emph{expansion}, \emph{rollout}, and \emph{backpropagation}, but extends it by incorporating uncertainty in both environment dynamics and human perception. We introduce uncertainty-aware simulations in the \emph{expansion} and \emph{rollout} phases and adjust \emph{backpropagation} with a value update rule that accounts for transition feasibility.

\subsubsection{Uncertainty-Aware Simulation}
As detailed in \Cref{algo:IG_MCTS}, both the \emph{expansion} and \emph{rollout} phases involve forward simulation of robot actions. Each tree node $v$ contains the state $(\tau, x)$, representing the robot's state history and current human perception. We handle the two action types differently as follows:
\begin{itemize}
    \item A movement action $u$ follows the environment dynamics $T$ as defined in \Cref{sec:problem}. Notably, the maze layout is observable up to distance $r$ from the robot's visited grids, while unexplored areas assume a $50\%$ chance of walls. In \emph{expansion}, the resulting search node $v'$ of this uncertain transition is assigned a feasibility value $\delta = 0.5$. In \emph{rollout}, the transition could fail and the robot remains in the same grid.
    
    \item The state transition for a communication step $o$ is governed by the learned stochastic human perception model $F_\theta$ as defined in \Cref{sec:nhpm}. Since transition probabilities are known, we compute the expected information reward $\bar{R_\mathrm{info}}$ directly:
    \begin{align*}
        \bar{R_\mathrm{info}}(\tau_t, x_t, o_t) &= \mathbb{E}_{x_{t+1}}\|x_{t+1}-x_t\|_1 \\
        &= \|p_\mathrm{add}\|_1 + \|p_\mathrm{remove}\|_1,
    \end{align*}
    where $(p_\mathrm{add}, p_\mathrm{remove}) \gets F_\theta(\tau_t, x_t, o_t)$ are the estimated probabilities of adding or removing walls from the map. 
    Directly computing the expected return at a node avoids the high number of visitations required to obtain an accurate value estimate.
\end{itemize}

% We denote a node in the search tree as $v$, where $s(v)$, $r(v)$, and $\delta(v)$ represent the state, reward, and transition feasibility at $v$, respectively. The visit count of $v$ is denoted as $N(v)$, while $Q(v)$ represents its total accumulated return. The set of child nodes of $v$ is denoted by $\mathbb{C}(v)$.

% The goal of each search is to plan a sequence for the robot until it reaches a goal or transmits a new image to the human. We initialize the search tree with the current human guidance $\zeta$, and the robot's approximation of human perception $x_0$. Each search node consists consists of the state information required by our reward augmentation: $(\tau, x)$. A node is terminal if it is the resulting state of a communication step, or if the robot reaches a goal location. 

% A rollout from the expanded node simulates future transitions until reaching a terminal state or a predefined depth $H$. Actions are selected randomly from the available action set $\mathcal{A}(s)$. If an action's feasibility is uncertain due to the environment's unknown structure, the transition occurs with probability $\delta(s, a)$. When a random number draw deems the transition infeasible, the state remains unchanged. On the other hand, for communication steps, we don't resolve the uncertainty but instead compute the expected information gain reward: \philip{TODO: adjust notation}
% \begin{equation}
%     \mathbb{E}\left[R_\mathrm{info}(\tau, x')\right] = \sum \mathrm{NPM(\tau, o)}.
% \end{equation}

\subsubsection{Feasibility-Adjusted Backpropagation}
During backpropagation, the rewards obtained from the simulation phase are propagated back through the tree, updating the total value $Q(v)$ and the visitation count $N(v)$ for all nodes along the path to the root. Due to uncertainty in unexplored environment dynamics, the rollout return depends on the feasibility of the transition from the child node. Given a sample return \(q'_{\mathrm{sample}}\) at child node \(v'\), the parent node's return is:
\begin{equation}
    q_{\mathrm{sample}} = r + \gamma \left[ \delta' q'_{\mathrm{sample}} + (1 - \delta') \frac{Q(v)}{N(v)} \right],
\end{equation}
where $\delta'$ represents the probability of a successful transition. The term \((1 - \delta')\) accounts for failed transitions, relying instead on the current value estimate.

% By incorporating uncertainty-aware rollouts and backpropagation, our approach enables more robust decision-making in scenarios where the environment dynamics is unknown and avoids simulation of the stochastic human perception dynamics.



\section{Task and Policy Representation}
\label{sec:task_results}
As we define an experiment based on a task and a policy, we must design different embedding strategies for each of them.
We first discuss baselines and upper bounds on task and policy representations, then we show results on how these representations impact our surrogate model.

\subsection{Experimental Setup}
As it is unclear what an ideal representation for a policy or task is, we compute an upper bound for a task and policy representation by taking all the pre-evaluated outcomes, and then training learnable embeddings on the task of estimating performance. 
Thus, these task and policy representations have specifically been tuned for this prediction task.
We can then use these learned embeddings as optimal representations of the task and policy.

However, this optimal approach requires all the data a priori. 
Thus, we need a way to represent both a task and a policy.
The most direct way to represent a task is based on the language description of a task.
As described in Section~\ref{sec:method}, we define our task representation as a weighted sum between the language embeddings of the task description and the verbs.
We call this approach \textcolor{verb}{\textbf{Verb}}.
Overall, we consider the following task representation types as upper bounds and baselines:
\begin{enumerate}
    \item \textcolor{optimal}{\textbf{Optimal:}} Leverage all the data a priori to learn embeddings that are useful for predicting performance;
    \item \textcolor{verb}{\textbf{Verb:}} Use a weighted sum of the language embedding of the task and the language embedding of its verbs;
    \item \textcolor{lang}{\textbf{Language:}} Use a language embedding of the task as its representation; and
    \item \textcolor{random}{\textbf{Random:}} Assume no relationship between policies and tasks by using random embeddings.
\end{enumerate}


\begin{figure*}[!ht]
    \centering
    \includegraphics[width=.95\linewidth]{figs_images/log_probs.pdf}
    \caption{\textbf{Average Log Likelihood Over Cost.} 
    We show the average log likelihood of all the outcomes in our offline dataset against the cost of evaluation for MetaWorld Policies, MetaWorld Checkpoints, HAMSTER, and OpenVLA over \textcolor{mypink}{continuous} and \textcolor{mygold}{binary} performance distributions.
    Each set of experiments is run for 1500 trials.
    % We find that EIG-based approaches generally outperform random baselines, both in the task- and policy-task-based sampling.
    We find that EIG-based approaches struggle to model the true distribution in a more cost-efficient manner than Random Task sampling.
    Task-based sampling strategies are more cost-efficient than policy-task approaches.
    % \vspace{-1em}
    }
    \label{fig:cost_exps}
\end{figure*}



Unlike a task representation through language, there is no clear representation for a policy. 
We leave the exploration of new policy representations to future work and focus on two policy representations: \textbf{Optimal} and \textbf{Random}.

All experiments in this section were run for 750 evaluation steps over three seeds.
To evaluate how much these embeddings improve the performance of population parameter estimation during active experiment selection, we look at the log likelihood of all the outcomes in our offline dataset against a probability distribution represented by the predicted population parameters from the surrogate model.
Each experiment is sampled similar to how researchers typically evaluate: we select a random task and test each policy three times.
% \lipsum[2]


\subsection{Results}

We evaluate the effectiveness of different task representations by computing the average log likelihood of the full dataset against the predicted distribution across multiple datasets, including MetaWorld Policies, MetaWorld Checkpoints, OpenVLA, and HAMSTER, as shown in Figure \ref{fig:model_exps}.


\textbf{Random representations do not share information across policies and tasks.}
Our results indicate that random embeddings consistently perform worse, as they fail to capture any meaningful structure or shared information between tasks. 
In contrast, optimal embeddings, which used the entire dataset to tune its representation, outperforms all baselines. 
We found that the increasing performance of random performance is due to new experiments being sampled; however, minimal interpolation of outcomes occurred.

\textbf{Task representations vary depending on the kinds of tasks.}
We find that the types of tasks matter. 
The HAMSTER evaluations consist of many changes to objects rather than changes to the type of task itself such as ``pickup the milk\dots" and ``pickup the shrimp\dots"
This structure leads to clearer benefits when using language-based representations.
In contrast, OpenVLA has less separable tasks, thus it shows a much smaller separation between random, optimal, and language-based embeddings.
Metaworld Checkpoints, however, show a more stable improvement of \textbf{Verb} as opposed to simply \textbf{Lang} since there are many more tasks.
% A ``pickup orange mug" task and a ``push orange mug" task would have similar performance even if they are different verbs. 
% We find that the verb-specific information in the \textbf{Verb} representation is a better



\textbf{Language does not explain all the shared information between tasks.}
Despite the improvement from using language or verbs as a task representation, they do not fully bridge the gap to optimal embeddings.
The difference between the optimal embeddings and language embeddings indicates that task descriptions, even when focused on the verbs, do not capture all the information to describe a task's relationship to its performance.
Our approach does not include the observations of the trajectory, and this difference between optimal and language embeddings may be explained by the lack of the initial image.
We leave it to future work to explore this direction.
% We believe it would be interesting future work to  remaining information is likely captured in the observation itself, which our framework does not include.

\textbf{Optimal policy embeddings do not provide meaningful gains.}
While task embeddings provide a meaningful way to represent tasks, we found that random or optimal policy embeddings do not provide any significant improvements compared to one another.
This result may be due to the procedure for learning the optimal embeddings overly relying on the task embeddings during their training, or may be caused by the relatively small number of policies that were evaluated, which ranged from 4 to 11. 
In contrast, there were between 29 to 81 tasks that were evaluated against, so there was higher overlap between some tasks.



\begin{figure*}[t]
    \centering
    \includegraphics[width=.95\linewidth]{figs_images/l1_error.pdf}
    \caption{\textbf{Average L1 Error of the Mean Over Cost.} Instead of computing the average log likelihood of the data as in Figure~\ref{fig:cost_exps}, we compute the error between the ground truth means of a policy-task pair and the mean of the predicted probability distribution. 
    In this case, we find that our method is better able to estimate the means for both \textcolor{mypink}{continuous} and \textcolor{mygold}{binary} distributions.
    We find that task sampling methods are generally more cost-efficient for the same average log likelihood than the policy-task sampling methods.
    % \vspace{-1em}
    }
    \label{fig:l1_exps}
\end{figure*}



\section{Cost-Aware Experiment Selection}

To evaluate the effectiveness of our cost-aware active experiment selection methods, we assess the population parameter estimation capability of our framework across various datasets using continuous and binary performance distributions.


\subsection{Experimental Setup}

\textbf{Sampling Strategies.}
To select the most informative experiment based on an acquisition function $a(\pi_i, T_j)$, we must design acquisition functions to define our sampling strategy.
We consider two types of sampling strategies. 
The first is to select both a policy and a task to run an evaluation on. 
Given the EIG formulation in Section~\ref{sec:method}, we define three sampling strategies with this approach:
\begin{itemize}
    \item \textcolor{rand}{\textbf{Random Sampling:}} Select a task-policy pair uniformly at random $a(\pi_i, T_j)=1/ (|\mathcal{P}| \times |\mathcal{T}|)$;
    \item  \textcolor{eig}{\textbf{EIG:}} Select a task-policy pair $(\pi_i,t_j)$ with the highest EIG: $a(\pi_i, T_j)=\mathcal{I}(\pi_i,T_j)$;
    \item  \textcolor{cost_eig}{\textbf{Cost-aware EIG:}} Select a task-policy pair that maximizes the cost-aware EIG according to Equation~\ref{eq:cost_aware_eig}.
\end{itemize}


The second type of sampling strategy is to select a task, and then evaluate every policy in that task $d=3$ times.

\begin{itemize}
    \item  \textcolor{rand_task}{\textbf{Random Task:}} Select a task uniformly at random and evaluate all policies on that task: $a(t_j) = 1 / |\mathcal{T}|$
    \item  \textcolor{task_eig}{\textbf{Task EIG:}} Select a task $T_j$ that maximizes the summed EIG across all policies: $a(t_j)=\sum_{i} \mathcal{I}(\pi_i,T_j)$
    \item  \textcolor{cost_task_eig}{\textbf{Cost-aware Task EIG:}} Select a task $T_j$ that maximizes the summed cost-aware EIG across all policies: $a(T_j)=\sum_{i} a_{\text{cost-aware}}(\pi_i,T_j, T_\text{current})$
\end{itemize}

The task-based sampling strategies is more realistic to how experimenters evaluate their robots today, as experimenters typically select a task and then evaluate every policy.
% However, it may be optimal to evaluate policy-task sampling as it may require less execution cost.

We evaluated each method for 1500 evaluation steps over three seeds using \textbf{Random} policy embeddings and \textbf{Verb} task embeddings.
To evaluate these methods, we consider two metrics: (1) the log likelihood of all the outcomes in our offline dataset against the predicted population parameters of the model, and (2) the L1 error between the mean from all the data for a policy-task pair against the mean derived from the estimated population parameters. 


\subsection{Results}


% Describe the importance of language, etc. Discuss hypothesis.
% \lipsum[2]
% \lipsum[2]


% \subsection{Cost over time}

% Discuss results of Figure \ref{fig:cost_exps} and Figure~\ref{fig:l1_exps}.
\begin{figure*}[t]
    \centering
    \includegraphics[width=.94\linewidth]{figs_images/pred_dists.pdf}
    \caption{\textbf{Predicted Mean Distributions.}
    We provide a visualization of the means for the predicted \textcolor{mypink}{continuous} and \textcolor{mygold}{binary} distributions over 0, 150, and 750 sampled queries. 
    We use random sampling with 3 evaluations per policy-task pair to show that our surrogate model can actively learn the full distribution of performance as well as have a good understanding of the performance distribution over time.
    For example, for MetaWorld Policies at $t=750$, $750/3=250$ policy-task pairs were sampled of the total $50*10=500$ possible policy-task pairs that could be evaluated, the estimated mean performance is qualitatively comparable to the true mean; Figure~\ref{fig:l1_exps} reports these results quantitatively as L1 error.
    % the estimation of the mean performance looks comparable to the true mean.
    }
    \label{fig:pred_dists}
\end{figure*}



\textbf{EIG-based approaches struggle to learn population parameters that represent all the data, but better estimate the mean.}
In Figure~\ref{fig:cost_exps}, we show the average log likelihood of all the outcomes in our offline dataset against the probability distribution represented by the predicted population parameters from the surrogate model.
In both task- and policy-task sampling approaches, we find that EIG-based approaches fit the original data marginally better than random baselines.
In some cases, such as for MetaWorld Policies with success rate, cost-aware EIG is able to maintain a larger improvement; however, this result is not consistent across other datasets.
This result indicates that learning this full underlying distribution remains challenging, particularly in the early stages of evaluation when data is sparse. 
However, in Figure~\ref{fig:l1_exps}, EIG-based approaches clearly dominate when estimating the mean of these distributions, and often are able to estimate the mean at a lower cost compared to random baselines.
If the cost is fixed at a lower value, as if it was a maximum cost-budget, then we find that EIG-based approaches better estimate the means.
% We find that at costs that are earlier, EIG-based approaches better estimate the means at a lower cost.
% Even if the full distribution is difficult to capture, the mean estimates are useful.

\textbf{Tradeoffs between task- and policy-task sampling.}
Both Figure~\ref{fig:cost_exps} and Figure~\ref{fig:l1_exps} show that task-based sampling is generally better in OpenVLA and HAMSTER, but cost-aware EIG is generally estimates the L1 error better than its task-based counterpart on MetaWorld.
Policy-task sampling approaches are likely more efficient in MetaWorld experiments as there are a large number of experiments where there is a high cost to switch, and evaluating 10 policies over a single task may not be as informative.
In contrast, HAMSTER and OpenVLA have fewer policies, meaning the cost of evaluating all policies for a single task is lower.
% Since we assume a small uniform cost to evaluating a policy within the same task, there is a negligible penalty to task-based sampling methods.
Additionally, we found that policy-task sampling methods are more likely to switch tasks, causing a faster accumulation of cost.



\textbf{Learning the Performance Landscape.}
Figure~\ref{fig:pred_dists} illustrates how our formulation of sequentially sampling experiments progressively refines the predictions of the performance landscape. 
Early in the evaluation process, predictions are generally around the mean and are misaligned with the true distribution.
As more experiments are selected, the means begin to resemble the true mean distribution. 



% \subsection{Noise Robustness}
% Discuss noisy gaussian setting \ref{fig:noisy_gaussian}
% \lipsum[1]

% \textbf{Noisiness affects stuff somehow}
% \lipsum[2]

% \begin{figure}[t]
    \centering
    \includegraphics[width=.75\linewidth]{example-image-a}
    \caption{\textbf{Noisy Hamster Setting (what happens if data is noisier?).} 
    \lipsum[2]
    }
    \label{fig:noisy_gaussian}
\end{figure}







\section{Discussion}\label{sec:discussion}



\subsection{From Interactive Prompting to Interactive Multi-modal Prompting}
The rapid advancements of large pre-trained generative models including large language models and text-to-image generation models, have inspired many HCI researchers to develop interactive tools to support users in crafting appropriate prompts.
% Studies on this topic in last two years' HCI conferences are predominantly focused on helping users refine single-modality textual prompts.
Many previous studies are focused on helping users refine single-modality textual prompts.
However, for many real-world applications concerning data beyond text modality, such as multi-modal AI and embodied intelligence, information from other modalities is essential in constructing sophisticated multi-modal prompts that fully convey users' instruction.
This demand inspires some researchers to develop multimodal prompting interactions to facilitate generation tasks ranging from visual modality image generation~\cite{wang2024promptcharm, promptpaint} to textual modality story generation~\cite{chung2022tale}.
% Some previous studies contributed relevant findings on this topic. 
Specifically, for the image generation task, recent studies have contributed some relevant findings on multi-modal prompting.
For example, PromptCharm~\cite{wang2024promptcharm} discovers the importance of multimodal feedback in refining initial text-based prompting in diffusion models.
However, the multi-modal interactions in PromptCharm are mainly focused on the feedback empowered the inpainting function, instead of supporting initial multimodal sketch-prompt control. 

\begin{figure*}[t]
    \centering
    \includegraphics[width=0.9\textwidth]{src/img/novice_expert.pdf}
    \vspace{-2mm}
    \caption{The comparison between novice and expert participants in painting reveals that experts produce more accurate and fine-grained sketches, resulting in closer alignment with reference images in close-ended tasks. Conversely, in open-ended tasks, expert fine-grained strokes fail to generate precise results due to \tool's lack of control at the thin stroke level.}
    \Description{The comparison between novice and expert participants in painting reveals that experts produce more accurate and fine-grained sketches, resulting in closer alignment with reference images in close-ended tasks. Novice users create rougher sketches with less accuracy in shape. Conversely, in open-ended tasks, expert fine-grained strokes fail to generate precise results due to \tool's lack of control at the thin stroke level, while novice users' broader strokes yield results more aligned with their sketches.}
    \label{fig:novice_expert}
    % \vspace{-3mm}
\end{figure*}


% In particular, in the initial control input, users are unable to explicitly specify multi-modal generation intents.
In another example, PromptPaint~\cite{promptpaint} stresses the importance of paint-medium-like interactions and introduces Prompt stencil functions that allow users to perform fine-grained controls with localized image generation. 
However, insufficient spatial control (\eg, PromptPaint only allows for single-object prompt stencil at a time) and unstable models can still leave some users feeling the uncertainty of AI and a varying degree of ownership of the generated artwork~\cite{promptpaint}.
% As a result, the gap between intuitive multi-modal or paint-medium-like control and the current prompting interface still exists, which requires further research on multi-modal prompting interactions.
From this perspective, our work seeks to further enhance multi-object spatial-semantic prompting control by users' natural sketching.
However, there are still some challenges to be resolved, such as consistent multi-object generation in multiple rounds to increase stability and improved understanding of user sketches.   


% \new{
% From this perspective, our work is a step forward in this direction by allowing multi-object spatial-semantic prompting control by users' natural sketching, which considers the interplay between multiple sketch regions.
% % To further advance the multi-modal prompting experience, there are some aspects we identify to be important.
% % One of the important aspects is enhancing the consistency and stability of multiple rounds of generation to reduce the uncertainty and loss of control on users' part.
% % For this purpose, we need to develop techniques to incorporate consistent generation~\cite{tewel2024training} into multi-modal prompting framework.}
% % Another important aspect is improving generative models' understanding of the implicit user intents \new{implied by the paint-medium-like or sketch-based input (\eg, sketch of two people with their hands slightly overlapping indicates holding hand without needing explicit prompt).
% % This can facilitate more natural control and alleviate users' effort in tuning the textual prompt.
% % In addition, it can increase users' sense of ownership as the generated results can be more aligned with their sketching intents.
% }
% For example, when users draw sketches of two people with their hands slightly overlapping, current region-based models cannot automatically infer users' implicit intention that the two people are holding hands.
% Instead, they still require users to explicitly specify in the prompt such relationship.
% \tool addresses this through sketch-aware prompt recommendation to fill in the necessary semantic information, alleviating users' workload.
% However, some users want the generative AI in the future to be able to directly infer this natural implicit intentions from the sketches without additional prompting since prompt recommendation can still be unstable sometimes.


% \new{
% Besides visual generation, 
% }
% For example, one of the important aspect is referring~\cite{he2024multi}, linking specific text semantics with specific spatial object, which is partly what we do in our sketch-aware prompt recommendation.
% Analogously, in natural communication between humans, text or audio alone often cannot suffice in expressing the speakers' intentions, and speakers often need to refer to an existing spatial object or draw out an illustration of her ideas for better explanation.
% Philosophically, we HCI researchers are mostly concerned about the human-end experience in human-AI communications.
% However, studies on prompting is unique in that we should not just care about the human-end interaction, but also make sure that AI can really get what the human means and produce intention-aligned output.
% Such consideration can drastically impact the design of prompting interactions in human-AI collaboration applications.
% On this note, although studies on multi-modal interactions is a well-established topic in HCI community, it remains a challenging problem what kind of multi-modal information is really effective in helping humans convey their ideas to current and next generation large AI models.




\subsection{Novice Performance vs. Expert Performance}\label{sec:nVe}
In this section we discuss the performance difference between novice and expert regarding experience in painting and prompting.
First, regarding painting skills, some participants with experience (4/12) preferred to draw accurate and fine-grained shapes at the beginning. 
All novice users (5/12) draw rough and less accurate shapes, while some participants with basic painting skills (3/12) also favored sketching rough areas of objects, as exemplified in Figure~\ref{fig:novice_expert}.
The experienced participants using fine-grained strokes (4/12, none of whom were experienced in prompting) achieved higher IoU scores (0.557) in the close-ended task (0.535) when using \tool. 
This is because their sketches were closer in shape and location to the reference, making the single object decomposition result more accurate.
Also, experienced participants are better at arranging spatial location and size of objects than novice participants.
However, some experienced participants (3/12) have mentioned that the fine-grained stroke sometimes makes them frustrated.
As P1's comment for his result in open-ended task: "\emph{It seems it cannot understand thin strokes; even if the shape is accurate, it can only generate content roughly around the area, especially when there is overlapping.}" 
This suggests that while \tool\ provides rough control to produce reasonably fine results from less accurate sketches for novice users, it may disappoint experienced users seeking more precise control through finer strokes. 
As shown in the last column in Figure~\ref{fig:novice_expert}, the dragon hovering in the sky was wrongly turned into a standing large dragon by \tool.

Second, regarding prompting skills, 3 out of 12 participants had one or more years of experience in T2I prompting. These participants used more modifiers than others during both T2I and R2I tasks.
Their performance in the T2I (0.335) and R2I (0.469) tasks showed higher scores than the average T2I (0.314) and R2I (0.418), but there was no performance improvement with \tool\ between their results (0.508) and the overall average score (0.528). 
This indicates that \tool\ can assist novice users in prompting, enabling them to produce satisfactory images similar to those created by users with prompting expertise.



\subsection{Applicability of \tool}
The feedback from user study highlighted several potential applications for our system. 
Three participants (P2, P6, P8) mentioned its possible use in commercial advertising design, emphasizing the importance of controllability for such work. 
They noted that the system's flexibility allows designers to quickly experiment with different settings.
Some participants (N = 3) also mentioned its potential for digital asset creation, particularly for game asset design. 
P7, a game mod developer, found the system highly useful for mod development. 
He explained: "\emph{Mods often require a series of images with a consistent theme and specific spatial requirements. 
For example, in a sacrifice scene, how the objects are arranged is closely tied to the mod's background. It would be difficult for a developer without professional skills, but with this system, it is possible to quickly construct such images}."
A few participants expressed similar thoughts regarding its use in scene construction, such as in film production. 
An interesting suggestion came from participant P4, who proposed its application in crime scene description. 
She pointed out that witnesses are often not skilled artists, and typically describe crime scenes verbally while someone else illustrates their account. 
With this system, witnesses could more easily express what they saw themselves, potentially producing depictions closer to the real events. "\emph{Details like object locations and distances from buildings can be easily conveyed using the system}," she added.

% \subsection{Model Understanding of Users' Implicit Intents}
% In region-sketch-based control of generative models, a significant gap between interaction design and actual implementation is the model's failure in understanding users' naturally expressed intentions.
% For example, when users draw sketches of two people with their hands slightly overlapping, current region-based models cannot automatically infer users' implicit intention that the two people are holding hands.
% Instead, they still require users to explicitly specify in the prompt such relationship.
% \tool addresses this through sketch-aware prompt recommendation to fill in the necessary semantic information, alleviating users' workload.
% However, some users want the generative AI in the future to be able to directly infer this natural implicit intentions from the sketches without additional prompting since prompt recommendation can still be unstable sometimes.
% This problem reflects a more general dilemma, which ubiquitously exists in all forms of conditioned control for generative models such as canny or scribble control.
% This is because all the control models are trained on pairs of explicit control signal and target image, which is lacking further interpretation or customization of the user intentions behind the seemingly straightforward input.
% For another example, the generative models cannot understand what abstraction level the user has in mind for her personal scribbles.
% Such problems leave more challenges to be addressed by future human-AI co-creation research.
% One possible direction is fine-tuning the conditioned models on individual user's conditioned control data to provide more customized interpretation. 

% \subsection{Balance between recommendation and autonomy}
% AIGC tools are a typical example of 
\subsection{Progressive Sketching}
Currently \tool is mainly aimed at novice users who are only capable of creating very rough sketches by themselves.
However, more accomplished painters or even professional artists typically have a coarse-to-fine creative process. 
Such a process is most evident in painting styles like traditional oil painting or digital impasto painting, where artists first quickly lay down large color patches to outline the most primitive proportion and structure of visual elements.
After that, the artists will progressively add layers of finer color strokes to the canvas to gradually refine the painting to an exquisite piece of artwork.
One participant in our user study (P1) , as a professional painter, has mentioned a similar point "\emph{
I think it is useful for laying out the big picture, give some inspirations for the initial drawing stage}."
Therefore, rough sketch also plays a part in the professional artists' creation process, yet it is more challenging to integrate AI into this more complex coarse-to-fine procedure.
Particularly, artists would like to preserve some of their finer strokes in later progression, not just the shape of the initial sketch.
In addition, instead of requiring the tool to generate a finished piece of artwork, some artists may prefer a model that can generate another more accurate sketch based on the initial one, and leave the final coloring and refining to the artists themselves.
To accommodate these diverse progressive sketching requirements, a more advanced sketch-based AI-assisted creation tool should be developed that can seamlessly enable artist intervention at any stage of the sketch and maximally preserve their creative intents to the finest level. 

\subsection{Ethical Issues}
Intellectual property and unethical misuse are two potential ethical concerns of AI-assisted creative tools, particularly those targeting novice users.
In terms of intellectual property, \tool hands over to novice users more control, giving them a higher sense of ownership of the creation.
However, the question still remains: how much contribution from the user's part constitutes full authorship of the artwork?
As \tool still relies on backbone generative models which may be trained on uncopyrighted data largely responsible for turning the sketch into finished artwork, we should design some mechanisms to circumvent this risk.
For example, we can allow artists to upload backbone models trained on their own artworks to integrate with our sketch control.
Regarding unethical misuse, \tool makes fine-grained spatial control more accessible to novice users, who may maliciously generate inappropriate content such as more realistic deepfake with specific postures they want or other explicit content.
To address this issue, we plan to incorporate a more sophisticated filtering mechanism that can detect and screen unethical content with more complex spatial-semantic conditions. 
% In the future, we plan to enable artists to upload their own style model

% \subsection{From interactive prompting to interactive spatial prompting}


\subsection{Limitations and Future work}

    \textbf{User Study Design}. Our open-ended task assesses the usability of \tool's system features in general use cases. To further examine aspects such as creativity and controllability across different methods, the open-ended task could be improved by incorporating baselines to provide more insightful comparative analysis. 
    Besides, in close-ended tasks, while the fixing order of tool usage prevents prior knowledge leakage, it might introduce learning effects. In our study, we include practice sessions for the three systems before the formal task to mitigate these effects. In the future, utilizing parallel tests (\textit{e.g.} different content with the same difficulty) or adding a control group could further reduce the learning effects.

    \textbf{Failure Cases}. There are certain failure cases with \tool that can limit its usability. 
    Firstly, when there are three or more objects with similar semantics, objects may still be missing despite prompt recommendations. 
    Secondly, if an object's stroke is thin, \tool may incorrectly interpret it as a full area, as demonstrated in the expert results of the open-ended task in Figure~\ref{fig:novice_expert}. 
    Finally, sometimes inclusion relationships (\textit{e.g.} inside) between objects cannot be generated correctly, partially due to biases in the base model that lack training samples with such relationship. 

    \textbf{More support for single object adjustment}.
    Participants (N=4) suggested that additional control features should be introduced, beyond just adjusting size and location. They noted that when objects overlap, they cannot freely control which object appears on top or which should be covered, and overlapping areas are currently not allowed.
    They proposed adding features such as layer control and depth control within the single-object mask manipulation. Currently, the system assigns layers based on color order, but future versions should allow users to adjust the layer of each object freely, while considering weighted prompts for overlapping areas.

    \textbf{More customized generation ability}.
    Our current system is built around a single model $ColorfulXL-Lightning$, which limits its ability to fully support the diverse creative needs of users. Feedback from participants has indicated a strong desire for more flexibility in style and personalization, such as integrating fine-tuned models that cater to specific artistic styles or individual preferences. 
    This limitation restricts the ability to adapt to varied creative intents across different users and contexts.
    In future iterations, we plan to address this by embedding a model selection feature, allowing users to choose from a variety of pre-trained or custom fine-tuned models that better align with their stylistic preferences. 
    
    \textbf{Integrate other model functions}.
    Our current system is compatible with many existing tools, such as Promptist~\cite{hao2024optimizing} and Magic Prompt, allowing users to iteratively generate prompts for single objects. However, the integration of these functions is somewhat limited in scope, and users may benefit from a broader range of interactive options, especially for more complex generation tasks. Additionally, for multimodal large models, users can currently explore using affordable or open-source models like Qwen2-VL~\cite{qwen} and InternVL2-Llama3~\cite{llama}, which have demonstrated solid inference performance in our tests. While GPT-4o remains a leading choice, alternative models also offer competitive results.
    Moving forward, we aim to integrate more multimodal large models into the system, giving users the flexibility to choose the models that best fit their needs. 
    


\section{Conclusion}\label{sec:conclusion}
In this paper, we present \tool, an interactive system designed to help novice users create high-quality, fine-grained images that align with their intentions based on rough sketches. 
The system first refines the user's initial prompt into a complete and coherent one that matches the rough sketch, ensuring the generated results are both stable, coherent and high quality.
To further support users in achieving fine-grained alignment between the generated image and their creative intent without requiring professional skills, we introduce a decompose-and-recompose strategy. 
This allows users to select desired, refined object shapes for individual decomposed objects and then recombine them, providing flexible mask manipulation for precise spatial control.
The framework operates through a coarse-to-fine process, enabling iterative and fine-grained control that is not possible with traditional end-to-end generation methods. 
Our user study demonstrates that \tool offers novice users enhanced flexibility in control and fine-grained alignment between their intentions and the generated images.



% \section*{Acknowledgments}

%% Use plainnat to work nicely with natbib. 

\bibliographystyle{plainnat}
\bibliography{references}


\clearpage
\subsection{Lloyd-Max Algorithm}
\label{subsec:Lloyd-Max}
For a given quantization bitwidth $B$ and an operand $\bm{X}$, the Lloyd-Max algorithm finds $2^B$ quantization levels $\{\hat{x}_i\}_{i=1}^{2^B}$ such that quantizing $\bm{X}$ by rounding each scalar in $\bm{X}$ to the nearest quantization level minimizes the quantization MSE. 

The algorithm starts with an initial guess of quantization levels and then iteratively computes quantization thresholds $\{\tau_i\}_{i=1}^{2^B-1}$ and updates quantization levels $\{\hat{x}_i\}_{i=1}^{2^B}$. Specifically, at iteration $n$, thresholds are set to the midpoints of the previous iteration's levels:
\begin{align*}
    \tau_i^{(n)}=\frac{\hat{x}_i^{(n-1)}+\hat{x}_{i+1}^{(n-1)}}2 \text{ for } i=1\ldots 2^B-1
\end{align*}
Subsequently, the quantization levels are re-computed as conditional means of the data regions defined by the new thresholds:
\begin{align*}
    \hat{x}_i^{(n)}=\mathbb{E}\left[ \bm{X} \big| \bm{X}\in [\tau_{i-1}^{(n)},\tau_i^{(n)}] \right] \text{ for } i=1\ldots 2^B
\end{align*}
where to satisfy boundary conditions we have $\tau_0=-\infty$ and $\tau_{2^B}=\infty$. The algorithm iterates the above steps until convergence.

Figure \ref{fig:lm_quant} compares the quantization levels of a $7$-bit floating point (E3M3) quantizer (left) to a $7$-bit Lloyd-Max quantizer (right) when quantizing a layer of weights from the GPT3-126M model at a per-tensor granularity. As shown, the Lloyd-Max quantizer achieves substantially lower quantization MSE. Further, Table \ref{tab:FP7_vs_LM7} shows the superior perplexity achieved by Lloyd-Max quantizers for bitwidths of $7$, $6$ and $5$. The difference between the quantizers is clear at 5 bits, where per-tensor FP quantization incurs a drastic and unacceptable increase in perplexity, while Lloyd-Max quantization incurs a much smaller increase. Nevertheless, we note that even the optimal Lloyd-Max quantizer incurs a notable ($\sim 1.5$) increase in perplexity due to the coarse granularity of quantization. 

\begin{figure}[h]
  \centering
  \includegraphics[width=0.7\linewidth]{sections/figures/LM7_FP7.pdf}
  \caption{\small Quantization levels and the corresponding quantization MSE of Floating Point (left) vs Lloyd-Max (right) Quantizers for a layer of weights in the GPT3-126M model.}
  \label{fig:lm_quant}
\end{figure}

\begin{table}[h]\scriptsize
\begin{center}
\caption{\label{tab:FP7_vs_LM7} \small Comparing perplexity (lower is better) achieved by floating point quantizers and Lloyd-Max quantizers on a GPT3-126M model for the Wikitext-103 dataset.}
\begin{tabular}{c|cc|c}
\hline
 \multirow{2}{*}{\textbf{Bitwidth}} & \multicolumn{2}{|c|}{\textbf{Floating-Point Quantizer}} & \textbf{Lloyd-Max Quantizer} \\
 & Best Format & Wikitext-103 Perplexity & Wikitext-103 Perplexity \\
\hline
7 & E3M3 & 18.32 & 18.27 \\
6 & E3M2 & 19.07 & 18.51 \\
5 & E4M0 & 43.89 & 19.71 \\
\hline
\end{tabular}
\end{center}
\end{table}

\subsection{Proof of Local Optimality of LO-BCQ}
\label{subsec:lobcq_opt_proof}
For a given block $\bm{b}_j$, the quantization MSE during LO-BCQ can be empirically evaluated as $\frac{1}{L_b}\lVert \bm{b}_j- \bm{\hat{b}}_j\rVert^2_2$ where $\bm{\hat{b}}_j$ is computed from equation (\ref{eq:clustered_quantization_definition}) as $C_{f(\bm{b}_j)}(\bm{b}_j)$. Further, for a given block cluster $\mathcal{B}_i$, we compute the quantization MSE as $\frac{1}{|\mathcal{B}_{i}|}\sum_{\bm{b} \in \mathcal{B}_{i}} \frac{1}{L_b}\lVert \bm{b}- C_i^{(n)}(\bm{b})\rVert^2_2$. Therefore, at the end of iteration $n$, we evaluate the overall quantization MSE $J^{(n)}$ for a given operand $\bm{X}$ composed of $N_c$ block clusters as:
\begin{align*}
    \label{eq:mse_iter_n}
    J^{(n)} = \frac{1}{N_c} \sum_{i=1}^{N_c} \frac{1}{|\mathcal{B}_{i}^{(n)}|}\sum_{\bm{v} \in \mathcal{B}_{i}^{(n)}} \frac{1}{L_b}\lVert \bm{b}- B_i^{(n)}(\bm{b})\rVert^2_2
\end{align*}

At the end of iteration $n$, the codebooks are updated from $\mathcal{C}^{(n-1)}$ to $\mathcal{C}^{(n)}$. However, the mapping of a given vector $\bm{b}_j$ to quantizers $\mathcal{C}^{(n)}$ remains as  $f^{(n)}(\bm{b}_j)$. At the next iteration, during the vector clustering step, $f^{(n+1)}(\bm{b}_j)$ finds new mapping of $\bm{b}_j$ to updated codebooks $\mathcal{C}^{(n)}$ such that the quantization MSE over the candidate codebooks is minimized. Therefore, we obtain the following result for $\bm{b}_j$:
\begin{align*}
\frac{1}{L_b}\lVert \bm{b}_j - C_{f^{(n+1)}(\bm{b}_j)}^{(n)}(\bm{b}_j)\rVert^2_2 \le \frac{1}{L_b}\lVert \bm{b}_j - C_{f^{(n)}(\bm{b}_j)}^{(n)}(\bm{b}_j)\rVert^2_2
\end{align*}

That is, quantizing $\bm{b}_j$ at the end of the block clustering step of iteration $n+1$ results in lower quantization MSE compared to quantizing at the end of iteration $n$. Since this is true for all $\bm{b} \in \bm{X}$, we assert the following:
\begin{equation}
\begin{split}
\label{eq:mse_ineq_1}
    \tilde{J}^{(n+1)} &= \frac{1}{N_c} \sum_{i=1}^{N_c} \frac{1}{|\mathcal{B}_{i}^{(n+1)}|}\sum_{\bm{b} \in \mathcal{B}_{i}^{(n+1)}} \frac{1}{L_b}\lVert \bm{b} - C_i^{(n)}(b)\rVert^2_2 \le J^{(n)}
\end{split}
\end{equation}
where $\tilde{J}^{(n+1)}$ is the the quantization MSE after the vector clustering step at iteration $n+1$.

Next, during the codebook update step (\ref{eq:quantizers_update}) at iteration $n+1$, the per-cluster codebooks $\mathcal{C}^{(n)}$ are updated to $\mathcal{C}^{(n+1)}$ by invoking the Lloyd-Max algorithm \citep{Lloyd}. We know that for any given value distribution, the Lloyd-Max algorithm minimizes the quantization MSE. Therefore, for a given vector cluster $\mathcal{B}_i$ we obtain the following result:

\begin{equation}
    \frac{1}{|\mathcal{B}_{i}^{(n+1)}|}\sum_{\bm{b} \in \mathcal{B}_{i}^{(n+1)}} \frac{1}{L_b}\lVert \bm{b}- C_i^{(n+1)}(\bm{b})\rVert^2_2 \le \frac{1}{|\mathcal{B}_{i}^{(n+1)}|}\sum_{\bm{b} \in \mathcal{B}_{i}^{(n+1)}} \frac{1}{L_b}\lVert \bm{b}- C_i^{(n)}(\bm{b})\rVert^2_2
\end{equation}

The above equation states that quantizing the given block cluster $\mathcal{B}_i$ after updating the associated codebook from $C_i^{(n)}$ to $C_i^{(n+1)}$ results in lower quantization MSE. Since this is true for all the block clusters, we derive the following result: 
\begin{equation}
\begin{split}
\label{eq:mse_ineq_2}
     J^{(n+1)} &= \frac{1}{N_c} \sum_{i=1}^{N_c} \frac{1}{|\mathcal{B}_{i}^{(n+1)}|}\sum_{\bm{b} \in \mathcal{B}_{i}^{(n+1)}} \frac{1}{L_b}\lVert \bm{b}- C_i^{(n+1)}(\bm{b})\rVert^2_2  \le \tilde{J}^{(n+1)}   
\end{split}
\end{equation}

Following (\ref{eq:mse_ineq_1}) and (\ref{eq:mse_ineq_2}), we find that the quantization MSE is non-increasing for each iteration, that is, $J^{(1)} \ge J^{(2)} \ge J^{(3)} \ge \ldots \ge J^{(M)}$ where $M$ is the maximum number of iterations. 
%Therefore, we can say that if the algorithm converges, then it must be that it has converged to a local minimum. 
\hfill $\blacksquare$


\begin{figure}
    \begin{center}
    \includegraphics[width=0.5\textwidth]{sections//figures/mse_vs_iter.pdf}
    \end{center}
    \caption{\small NMSE vs iterations during LO-BCQ compared to other block quantization proposals}
    \label{fig:nmse_vs_iter}
\end{figure}

Figure \ref{fig:nmse_vs_iter} shows the empirical convergence of LO-BCQ across several block lengths and number of codebooks. Also, the MSE achieved by LO-BCQ is compared to baselines such as MXFP and VSQ. As shown, LO-BCQ converges to a lower MSE than the baselines. Further, we achieve better convergence for larger number of codebooks ($N_c$) and for a smaller block length ($L_b$), both of which increase the bitwidth of BCQ (see Eq \ref{eq:bitwidth_bcq}).


\subsection{Additional Accuracy Results}
%Table \ref{tab:lobcq_config} lists the various LOBCQ configurations and their corresponding bitwidths.
\begin{table}
\setlength{\tabcolsep}{4.75pt}
\begin{center}
\caption{\label{tab:lobcq_config} Various LO-BCQ configurations and their bitwidths.}
\begin{tabular}{|c||c|c|c|c||c|c||c|} 
\hline
 & \multicolumn{4}{|c||}{$L_b=8$} & \multicolumn{2}{|c||}{$L_b=4$} & $L_b=2$ \\
 \hline
 \backslashbox{$L_A$\kern-1em}{\kern-1em$N_c$} & 2 & 4 & 8 & 16 & 2 & 4 & 2 \\
 \hline
 64 & 4.25 & 4.375 & 4.5 & 4.625 & 4.375 & 4.625 & 4.625\\
 \hline
 32 & 4.375 & 4.5 & 4.625& 4.75 & 4.5 & 4.75 & 4.75 \\
 \hline
 16 & 4.625 & 4.75& 4.875 & 5 & 4.75 & 5 & 5 \\
 \hline
\end{tabular}
\end{center}
\end{table}

%\subsection{Perplexity achieved by various LO-BCQ configurations on Wikitext-103 dataset}

\begin{table} \centering
\begin{tabular}{|c||c|c|c|c||c|c||c|} 
\hline
 $L_b \rightarrow$& \multicolumn{4}{c||}{8} & \multicolumn{2}{c||}{4} & 2\\
 \hline
 \backslashbox{$L_A$\kern-1em}{\kern-1em$N_c$} & 2 & 4 & 8 & 16 & 2 & 4 & 2  \\
 %$N_c \rightarrow$ & 2 & 4 & 8 & 16 & 2 & 4 & 2 \\
 \hline
 \hline
 \multicolumn{8}{c}{GPT3-1.3B (FP32 PPL = 9.98)} \\ 
 \hline
 \hline
 64 & 10.40 & 10.23 & 10.17 & 10.15 &  10.28 & 10.18 & 10.19 \\
 \hline
 32 & 10.25 & 10.20 & 10.15 & 10.12 &  10.23 & 10.17 & 10.17 \\
 \hline
 16 & 10.22 & 10.16 & 10.10 & 10.09 &  10.21 & 10.14 & 10.16 \\
 \hline
  \hline
 \multicolumn{8}{c}{GPT3-8B (FP32 PPL = 7.38)} \\ 
 \hline
 \hline
 64 & 7.61 & 7.52 & 7.48 &  7.47 &  7.55 &  7.49 & 7.50 \\
 \hline
 32 & 7.52 & 7.50 & 7.46 &  7.45 &  7.52 &  7.48 & 7.48  \\
 \hline
 16 & 7.51 & 7.48 & 7.44 &  7.44 &  7.51 &  7.49 & 7.47  \\
 \hline
\end{tabular}
\caption{\label{tab:ppl_gpt3_abalation} Wikitext-103 perplexity across GPT3-1.3B and 8B models.}
\end{table}

\begin{table} \centering
\begin{tabular}{|c||c|c|c|c||} 
\hline
 $L_b \rightarrow$& \multicolumn{4}{c||}{8}\\
 \hline
 \backslashbox{$L_A$\kern-1em}{\kern-1em$N_c$} & 2 & 4 & 8 & 16 \\
 %$N_c \rightarrow$ & 2 & 4 & 8 & 16 & 2 & 4 & 2 \\
 \hline
 \hline
 \multicolumn{5}{|c|}{Llama2-7B (FP32 PPL = 5.06)} \\ 
 \hline
 \hline
 64 & 5.31 & 5.26 & 5.19 & 5.18  \\
 \hline
 32 & 5.23 & 5.25 & 5.18 & 5.15  \\
 \hline
 16 & 5.23 & 5.19 & 5.16 & 5.14  \\
 \hline
 \multicolumn{5}{|c|}{Nemotron4-15B (FP32 PPL = 5.87)} \\ 
 \hline
 \hline
 64  & 6.3 & 6.20 & 6.13 & 6.08  \\
 \hline
 32  & 6.24 & 6.12 & 6.07 & 6.03  \\
 \hline
 16  & 6.12 & 6.14 & 6.04 & 6.02  \\
 \hline
 \multicolumn{5}{|c|}{Nemotron4-340B (FP32 PPL = 3.48)} \\ 
 \hline
 \hline
 64 & 3.67 & 3.62 & 3.60 & 3.59 \\
 \hline
 32 & 3.63 & 3.61 & 3.59 & 3.56 \\
 \hline
 16 & 3.61 & 3.58 & 3.57 & 3.55 \\
 \hline
\end{tabular}
\caption{\label{tab:ppl_llama7B_nemo15B} Wikitext-103 perplexity compared to FP32 baseline in Llama2-7B and Nemotron4-15B, 340B models}
\end{table}

%\subsection{Perplexity achieved by various LO-BCQ configurations on MMLU dataset}


\begin{table} \centering
\begin{tabular}{|c||c|c|c|c||c|c|c|c|} 
\hline
 $L_b \rightarrow$& \multicolumn{4}{c||}{8} & \multicolumn{4}{c||}{8}\\
 \hline
 \backslashbox{$L_A$\kern-1em}{\kern-1em$N_c$} & 2 & 4 & 8 & 16 & 2 & 4 & 8 & 16  \\
 %$N_c \rightarrow$ & 2 & 4 & 8 & 16 & 2 & 4 & 2 \\
 \hline
 \hline
 \multicolumn{5}{|c|}{Llama2-7B (FP32 Accuracy = 45.8\%)} & \multicolumn{4}{|c|}{Llama2-70B (FP32 Accuracy = 69.12\%)} \\ 
 \hline
 \hline
 64 & 43.9 & 43.4 & 43.9 & 44.9 & 68.07 & 68.27 & 68.17 & 68.75 \\
 \hline
 32 & 44.5 & 43.8 & 44.9 & 44.5 & 68.37 & 68.51 & 68.35 & 68.27  \\
 \hline
 16 & 43.9 & 42.7 & 44.9 & 45 & 68.12 & 68.77 & 68.31 & 68.59  \\
 \hline
 \hline
 \multicolumn{5}{|c|}{GPT3-22B (FP32 Accuracy = 38.75\%)} & \multicolumn{4}{|c|}{Nemotron4-15B (FP32 Accuracy = 64.3\%)} \\ 
 \hline
 \hline
 64 & 36.71 & 38.85 & 38.13 & 38.92 & 63.17 & 62.36 & 63.72 & 64.09 \\
 \hline
 32 & 37.95 & 38.69 & 39.45 & 38.34 & 64.05 & 62.30 & 63.8 & 64.33  \\
 \hline
 16 & 38.88 & 38.80 & 38.31 & 38.92 & 63.22 & 63.51 & 63.93 & 64.43  \\
 \hline
\end{tabular}
\caption{\label{tab:mmlu_abalation} Accuracy on MMLU dataset across GPT3-22B, Llama2-7B, 70B and Nemotron4-15B models.}
\end{table}


%\subsection{Perplexity achieved by various LO-BCQ configurations on LM evaluation harness}

\begin{table} \centering
\begin{tabular}{|c||c|c|c|c||c|c|c|c|} 
\hline
 $L_b \rightarrow$& \multicolumn{4}{c||}{8} & \multicolumn{4}{c||}{8}\\
 \hline
 \backslashbox{$L_A$\kern-1em}{\kern-1em$N_c$} & 2 & 4 & 8 & 16 & 2 & 4 & 8 & 16  \\
 %$N_c \rightarrow$ & 2 & 4 & 8 & 16 & 2 & 4 & 2 \\
 \hline
 \hline
 \multicolumn{5}{|c|}{Race (FP32 Accuracy = 37.51\%)} & \multicolumn{4}{|c|}{Boolq (FP32 Accuracy = 64.62\%)} \\ 
 \hline
 \hline
 64 & 36.94 & 37.13 & 36.27 & 37.13 & 63.73 & 62.26 & 63.49 & 63.36 \\
 \hline
 32 & 37.03 & 36.36 & 36.08 & 37.03 & 62.54 & 63.51 & 63.49 & 63.55  \\
 \hline
 16 & 37.03 & 37.03 & 36.46 & 37.03 & 61.1 & 63.79 & 63.58 & 63.33  \\
 \hline
 \hline
 \multicolumn{5}{|c|}{Winogrande (FP32 Accuracy = 58.01\%)} & \multicolumn{4}{|c|}{Piqa (FP32 Accuracy = 74.21\%)} \\ 
 \hline
 \hline
 64 & 58.17 & 57.22 & 57.85 & 58.33 & 73.01 & 73.07 & 73.07 & 72.80 \\
 \hline
 32 & 59.12 & 58.09 & 57.85 & 58.41 & 73.01 & 73.94 & 72.74 & 73.18  \\
 \hline
 16 & 57.93 & 58.88 & 57.93 & 58.56 & 73.94 & 72.80 & 73.01 & 73.94  \\
 \hline
\end{tabular}
\caption{\label{tab:mmlu_abalation} Accuracy on LM evaluation harness tasks on GPT3-1.3B model.}
\end{table}

\begin{table} \centering
\begin{tabular}{|c||c|c|c|c||c|c|c|c|} 
\hline
 $L_b \rightarrow$& \multicolumn{4}{c||}{8} & \multicolumn{4}{c||}{8}\\
 \hline
 \backslashbox{$L_A$\kern-1em}{\kern-1em$N_c$} & 2 & 4 & 8 & 16 & 2 & 4 & 8 & 16  \\
 %$N_c \rightarrow$ & 2 & 4 & 8 & 16 & 2 & 4 & 2 \\
 \hline
 \hline
 \multicolumn{5}{|c|}{Race (FP32 Accuracy = 41.34\%)} & \multicolumn{4}{|c|}{Boolq (FP32 Accuracy = 68.32\%)} \\ 
 \hline
 \hline
 64 & 40.48 & 40.10 & 39.43 & 39.90 & 69.20 & 68.41 & 69.45 & 68.56 \\
 \hline
 32 & 39.52 & 39.52 & 40.77 & 39.62 & 68.32 & 67.43 & 68.17 & 69.30  \\
 \hline
 16 & 39.81 & 39.71 & 39.90 & 40.38 & 68.10 & 66.33 & 69.51 & 69.42  \\
 \hline
 \hline
 \multicolumn{5}{|c|}{Winogrande (FP32 Accuracy = 67.88\%)} & \multicolumn{4}{|c|}{Piqa (FP32 Accuracy = 78.78\%)} \\ 
 \hline
 \hline
 64 & 66.85 & 66.61 & 67.72 & 67.88 & 77.31 & 77.42 & 77.75 & 77.64 \\
 \hline
 32 & 67.25 & 67.72 & 67.72 & 67.00 & 77.31 & 77.04 & 77.80 & 77.37  \\
 \hline
 16 & 68.11 & 68.90 & 67.88 & 67.48 & 77.37 & 78.13 & 78.13 & 77.69  \\
 \hline
\end{tabular}
\caption{\label{tab:mmlu_abalation} Accuracy on LM evaluation harness tasks on GPT3-8B model.}
\end{table}

\begin{table} \centering
\begin{tabular}{|c||c|c|c|c||c|c|c|c|} 
\hline
 $L_b \rightarrow$& \multicolumn{4}{c||}{8} & \multicolumn{4}{c||}{8}\\
 \hline
 \backslashbox{$L_A$\kern-1em}{\kern-1em$N_c$} & 2 & 4 & 8 & 16 & 2 & 4 & 8 & 16  \\
 %$N_c \rightarrow$ & 2 & 4 & 8 & 16 & 2 & 4 & 2 \\
 \hline
 \hline
 \multicolumn{5}{|c|}{Race (FP32 Accuracy = 40.67\%)} & \multicolumn{4}{|c|}{Boolq (FP32 Accuracy = 76.54\%)} \\ 
 \hline
 \hline
 64 & 40.48 & 40.10 & 39.43 & 39.90 & 75.41 & 75.11 & 77.09 & 75.66 \\
 \hline
 32 & 39.52 & 39.52 & 40.77 & 39.62 & 76.02 & 76.02 & 75.96 & 75.35  \\
 \hline
 16 & 39.81 & 39.71 & 39.90 & 40.38 & 75.05 & 73.82 & 75.72 & 76.09  \\
 \hline
 \hline
 \multicolumn{5}{|c|}{Winogrande (FP32 Accuracy = 70.64\%)} & \multicolumn{4}{|c|}{Piqa (FP32 Accuracy = 79.16\%)} \\ 
 \hline
 \hline
 64 & 69.14 & 70.17 & 70.17 & 70.56 & 78.24 & 79.00 & 78.62 & 78.73 \\
 \hline
 32 & 70.96 & 69.69 & 71.27 & 69.30 & 78.56 & 79.49 & 79.16 & 78.89  \\
 \hline
 16 & 71.03 & 69.53 & 69.69 & 70.40 & 78.13 & 79.16 & 79.00 & 79.00  \\
 \hline
\end{tabular}
\caption{\label{tab:mmlu_abalation} Accuracy on LM evaluation harness tasks on GPT3-22B model.}
\end{table}

\begin{table} \centering
\begin{tabular}{|c||c|c|c|c||c|c|c|c|} 
\hline
 $L_b \rightarrow$& \multicolumn{4}{c||}{8} & \multicolumn{4}{c||}{8}\\
 \hline
 \backslashbox{$L_A$\kern-1em}{\kern-1em$N_c$} & 2 & 4 & 8 & 16 & 2 & 4 & 8 & 16  \\
 %$N_c \rightarrow$ & 2 & 4 & 8 & 16 & 2 & 4 & 2 \\
 \hline
 \hline
 \multicolumn{5}{|c|}{Race (FP32 Accuracy = 44.4\%)} & \multicolumn{4}{|c|}{Boolq (FP32 Accuracy = 79.29\%)} \\ 
 \hline
 \hline
 64 & 42.49 & 42.51 & 42.58 & 43.45 & 77.58 & 77.37 & 77.43 & 78.1 \\
 \hline
 32 & 43.35 & 42.49 & 43.64 & 43.73 & 77.86 & 75.32 & 77.28 & 77.86  \\
 \hline
 16 & 44.21 & 44.21 & 43.64 & 42.97 & 78.65 & 77 & 76.94 & 77.98  \\
 \hline
 \hline
 \multicolumn{5}{|c|}{Winogrande (FP32 Accuracy = 69.38\%)} & \multicolumn{4}{|c|}{Piqa (FP32 Accuracy = 78.07\%)} \\ 
 \hline
 \hline
 64 & 68.9 & 68.43 & 69.77 & 68.19 & 77.09 & 76.82 & 77.09 & 77.86 \\
 \hline
 32 & 69.38 & 68.51 & 68.82 & 68.90 & 78.07 & 76.71 & 78.07 & 77.86  \\
 \hline
 16 & 69.53 & 67.09 & 69.38 & 68.90 & 77.37 & 77.8 & 77.91 & 77.69  \\
 \hline
\end{tabular}
\caption{\label{tab:mmlu_abalation} Accuracy on LM evaluation harness tasks on Llama2-7B model.}
\end{table}

\begin{table} \centering
\begin{tabular}{|c||c|c|c|c||c|c|c|c|} 
\hline
 $L_b \rightarrow$& \multicolumn{4}{c||}{8} & \multicolumn{4}{c||}{8}\\
 \hline
 \backslashbox{$L_A$\kern-1em}{\kern-1em$N_c$} & 2 & 4 & 8 & 16 & 2 & 4 & 8 & 16  \\
 %$N_c \rightarrow$ & 2 & 4 & 8 & 16 & 2 & 4 & 2 \\
 \hline
 \hline
 \multicolumn{5}{|c|}{Race (FP32 Accuracy = 48.8\%)} & \multicolumn{4}{|c|}{Boolq (FP32 Accuracy = 85.23\%)} \\ 
 \hline
 \hline
 64 & 49.00 & 49.00 & 49.28 & 48.71 & 82.82 & 84.28 & 84.03 & 84.25 \\
 \hline
 32 & 49.57 & 48.52 & 48.33 & 49.28 & 83.85 & 84.46 & 84.31 & 84.93  \\
 \hline
 16 & 49.85 & 49.09 & 49.28 & 48.99 & 85.11 & 84.46 & 84.61 & 83.94  \\
 \hline
 \hline
 \multicolumn{5}{|c|}{Winogrande (FP32 Accuracy = 79.95\%)} & \multicolumn{4}{|c|}{Piqa (FP32 Accuracy = 81.56\%)} \\ 
 \hline
 \hline
 64 & 78.77 & 78.45 & 78.37 & 79.16 & 81.45 & 80.69 & 81.45 & 81.5 \\
 \hline
 32 & 78.45 & 79.01 & 78.69 & 80.66 & 81.56 & 80.58 & 81.18 & 81.34  \\
 \hline
 16 & 79.95 & 79.56 & 79.79 & 79.72 & 81.28 & 81.66 & 81.28 & 80.96  \\
 \hline
\end{tabular}
\caption{\label{tab:mmlu_abalation} Accuracy on LM evaluation harness tasks on Llama2-70B model.}
\end{table}

%\section{MSE Studies}
%\textcolor{red}{TODO}


\subsection{Number Formats and Quantization Method}
\label{subsec:numFormats_quantMethod}
\subsubsection{Integer Format}
An $n$-bit signed integer (INT) is typically represented with a 2s-complement format \citep{yao2022zeroquant,xiao2023smoothquant,dai2021vsq}, where the most significant bit denotes the sign.

\subsubsection{Floating Point Format}
An $n$-bit signed floating point (FP) number $x$ comprises of a 1-bit sign ($x_{\mathrm{sign}}$), $B_m$-bit mantissa ($x_{\mathrm{mant}}$) and $B_e$-bit exponent ($x_{\mathrm{exp}}$) such that $B_m+B_e=n-1$. The associated constant exponent bias ($E_{\mathrm{bias}}$) is computed as $(2^{{B_e}-1}-1)$. We denote this format as $E_{B_e}M_{B_m}$.  

\subsubsection{Quantization Scheme}
\label{subsec:quant_method}
A quantization scheme dictates how a given unquantized tensor is converted to its quantized representation. We consider FP formats for the purpose of illustration. Given an unquantized tensor $\bm{X}$ and an FP format $E_{B_e}M_{B_m}$, we first, we compute the quantization scale factor $s_X$ that maps the maximum absolute value of $\bm{X}$ to the maximum quantization level of the $E_{B_e}M_{B_m}$ format as follows:
\begin{align}
\label{eq:sf}
    s_X = \frac{\mathrm{max}(|\bm{X}|)}{\mathrm{max}(E_{B_e}M_{B_m})}
\end{align}
In the above equation, $|\cdot|$ denotes the absolute value function.

Next, we scale $\bm{X}$ by $s_X$ and quantize it to $\hat{\bm{X}}$ by rounding it to the nearest quantization level of $E_{B_e}M_{B_m}$ as:

\begin{align}
\label{eq:tensor_quant}
    \hat{\bm{X}} = \text{round-to-nearest}\left(\frac{\bm{X}}{s_X}, E_{B_e}M_{B_m}\right)
\end{align}

We perform dynamic max-scaled quantization \citep{wu2020integer}, where the scale factor $s$ for activations is dynamically computed during runtime.

\subsection{Vector Scaled Quantization}
\begin{wrapfigure}{r}{0.35\linewidth}
  \centering
  \includegraphics[width=\linewidth]{sections/figures/vsquant.jpg}
  \caption{\small Vectorwise decomposition for per-vector scaled quantization (VSQ \citep{dai2021vsq}).}
  \label{fig:vsquant}
\end{wrapfigure}
During VSQ \citep{dai2021vsq}, the operand tensors are decomposed into 1D vectors in a hardware friendly manner as shown in Figure \ref{fig:vsquant}. Since the decomposed tensors are used as operands in matrix multiplications during inference, it is beneficial to perform this decomposition along the reduction dimension of the multiplication. The vectorwise quantization is performed similar to tensorwise quantization described in Equations \ref{eq:sf} and \ref{eq:tensor_quant}, where a scale factor $s_v$ is required for each vector $\bm{v}$ that maps the maximum absolute value of that vector to the maximum quantization level. While smaller vector lengths can lead to larger accuracy gains, the associated memory and computational overheads due to the per-vector scale factors increases. To alleviate these overheads, VSQ \citep{dai2021vsq} proposed a second level quantization of the per-vector scale factors to unsigned integers, while MX \citep{rouhani2023shared} quantizes them to integer powers of 2 (denoted as $2^{INT}$).

\subsubsection{MX Format}
The MX format proposed in \citep{rouhani2023microscaling} introduces the concept of sub-block shifting. For every two scalar elements of $b$-bits each, there is a shared exponent bit. The value of this exponent bit is determined through an empirical analysis that targets minimizing quantization MSE. We note that the FP format $E_{1}M_{b}$ is strictly better than MX from an accuracy perspective since it allocates a dedicated exponent bit to each scalar as opposed to sharing it across two scalars. Therefore, we conservatively bound the accuracy of a $b+2$-bit signed MX format with that of a $E_{1}M_{b}$ format in our comparisons. For instance, we use E1M2 format as a proxy for MX4.

\begin{figure}
    \centering
    \includegraphics[width=1\linewidth]{sections//figures/BlockFormats.pdf}
    \caption{\small Comparing LO-BCQ to MX format.}
    \label{fig:block_formats}
\end{figure}

Figure \ref{fig:block_formats} compares our $4$-bit LO-BCQ block format to MX \citep{rouhani2023microscaling}. As shown, both LO-BCQ and MX decompose a given operand tensor into block arrays and each block array into blocks. Similar to MX, we find that per-block quantization ($L_b < L_A$) leads to better accuracy due to increased flexibility. While MX achieves this through per-block $1$-bit micro-scales, we associate a dedicated codebook to each block through a per-block codebook selector. Further, MX quantizes the per-block array scale-factor to E8M0 format without per-tensor scaling. In contrast during LO-BCQ, we find that per-tensor scaling combined with quantization of per-block array scale-factor to E4M3 format results in superior inference accuracy across models. 


\end{document}


