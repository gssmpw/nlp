\section{Problem Formulation and Notation}

\begin{figure*}[!h]
  \centering
  \includegraphics[width=\linewidth]{figs/overview_v1.pdf}
  \caption{
\textbf{Overview of our method.}
Given a user-provided prompt $\inputprompt$ and a partial sketch $\inputsketch$, our method first (a) stylizes the input prompt by augmenting it using style descriptions generated by the VLM (\textbf{bold text}).
Using the stylized prompt $\finalprompt$, the method then performs (b) stroke optimization to generate strokes that fill the missing regions, thus ensuring that the style-agnostic completed sketch $\intermediatesketch$ can fully represents the content of the user-provided prompt.
To align the styles of $\intermediatesketch$ and $\inputsketch$, we (c) instruct the VLM to generate an executable style adjustment code that modifies the strokes of $\intermediatesketch$.
Finally, we obtain a final completed sketch $\completesketch$ wherein the styles of the strokes are aligned to those of the $\inputsketch$.
}
\label{fig:overview}
\end{figure*}


% In this work, we address the challenge of efficiently evaluating multi-task robot policies in the real world.
The objective of this work is to design an efficient strategy to evaluate robot policies across tasks while balancing the cost of experimentation. 
Consider a fixed set of $M$ robot policies, denoted by $\mathcal{P} = \{\pi_1, \pi_2, \ldots, \pi_M\}$ and a set of $N$ tasks $\mathcal{T} = \{T_1, T_2, ..., T_N\}$. 
% Due to practical constraints, only a subset of $N$ tasks $A \subset \mathcal{T}$ is used.
Each task $T_j \in \mathcal{T}$ is a finite-horizon MDP defined by states, actions, and a high-level natural language instruction $L_i$.
% Since robot policies can be sensitive to changes in language instructions~\cite{}, we assume a one-to-one mapping between language instructions and tasks.

Our framework is policy-agnostic and does not assume access to policy model weights, and can be applied to engineered robot \textit{systems} in addition to end-to-end models.

\textbf{Population Parameter Estimation}. 
We formulate the problem as population parameter estimation, similar to probabilistic matrix factorization~\cite{mnih2007probabilistic}.
Let the performance of a policy ${\pi_i \in \mathcal{P}}$ on a task $T_j \in \mathcal{T}$ be represented by the random variable $X_{ij}$ with distribution $P_{ij}$, from which we can sample evaluations $\trial \sim \truedist$. Here, $\truedist$ represents the ``true'' performance distribution.
Since the underlying distribution $\truedist$ is unknown, the goal of population parameter estimation is to estimate a distribution $\learneddist$ that models real-world evaluation outcomes from $\truedist$.
% , which is the distribution corresponding to a random variable $\hat{X}_{ij}$, 
We use $\theta_{ij}$ to represent the parameters of the learned distribution $\learneddist$.
For example, $\theta_{ij}=[\mu, \sigma]$ if $\learneddist$ is a Gaussian distribution.
Given a limited number of observed samples from the true distribution, $\trial^1, ..., \trial^n \sim \truedist$, the goal is to estimate the parameters of an estimated distribution $\theta_{ij}$.
Our setting also has samples from other random variables, $X_{kl}$ corresponding to different policy-task pairs. 
Therefore, in this work we want to estimate $\Theta = \{\theta_{ij}\}_{i, j = 1}^{i=M, j = N}$ for all policy-task pairs given a dataset $\mathcal{D}=\{\trial^k\}$.
These distributions can be visualized as a grid of policy-task pairs as shown in Figure~\ref{fig:overview}.

The aim is to estimate the parameters of $\learneddist$ of all policy-task combinations by leveraging shared information across this matrix.
However, it is infeasible to directly evaluate all policy-task pairs due to cost constraints. 
Therefore, we adopt an active testing approach, where the objective is to iteratively select the most informative experiments $(\pi_i, T_j)$ to efficiently learn $\Theta$. 
% Given a limited evaluation budget, we develop active experiment selection strategies to efficiently samples evaluations that improve the accuracy of these parameter estimates.

\textbf{Active Testing.}
% Since it is difficult to sample a large set of outcomes $\mathcal{D}$ due to experimenter effort, we take an active testing approach.
We apply an active learning paradigm to learn a population parameter estimator $f(\pi_i, T_j)$.
As such, we define acquisition functions to guide the selection of task-policy pairs or tasks alone, and then sample experiments that are most informative.
First, we define an acquisition function $a(\pi_i, T_j)$, and the next experiment is selected by maximizing this function over all possible experiments:
\begin{equation}
    (\pi_i^*, T_j^*) = \arg\max_{(\pi_i, T_j)} a(\pi_i, T_j).
\end{equation}
Although these acquisition functions are informative, we want a balance between selecting informative experiments and their costs.

\textbf{Evaluation Cost}. In real-world evaluation, each policy-task evaluation incurs a cost. 
Let $c_{\text{eval}}(T_j)$ denote the cost of a single evaluation of a policy on task $T_j$.
We make a simplifying assumption that this cost is agnostic to changes in the policy under evaluation, that often being a configurable software option.
This cost could include the time required to execute the policy, the resources consumed during evaluation, or the manual supervision required to reset the scene.
Furthermore, switching between tasks typically incurs a larger cost involving a reconfiguring the scene or the robot. 
We define this switching cost $c_{\text{switch}}(T_j, T_k)$ as the cost associated with transitioning from task $T_j$ to $T_k$.
% This cost depends on the specific tasks being evaluated.
For a sequence of tasks that have been evaluated $T_{i_1}, \ldots, T_{i_L}$ (where each $i_j \in N$), we compute the total cost of evaluation as:

% Do we need this part? wrote it like this to take up more space
$$c_{\text{total}} = \sum_{j=1}^N c_{\text{eval}}(T_{i_j}) + \sum_{j=1}^{N-1} c_{\text{switch}}(T_{i_j}, T_{i_{j+1}})$$

Given these costs, the problem is to design an evaluation strategy that minimizes the total cost of evaluation while learning the population parameters of test instances.




\section{Method}

We aim to design a framework for sampling experiments for multi-task robot policies.
Our framework consists of two parts: (1) learning a surrogate model to estimate the population parameters of a test instance and (2) designing strategies to sample experiments in a cost-efficient manner.
The surrogate model leverages task and policy representations that define an experiment to have a better estimate of the overall performance distributions.
Then, we use this surrogate model to compute the expected information gain of different experiments.
We then use the expected information gain along with the cost of switching tasks to conduct active testing.

% \textbf{Surrogate Model.}
\subsection{Surrogate Model}
As we evaluate our robot policies across tasks, we track the outcomes of each trial to aggregate a dataset $\mathcal{D}$ over time.
% In the process of evaluating multi-task robot policies, we collect a dataset of previous outcomes from trials $\trial \in \mathcal{D}$.
Each of these outcomes are realizations of a true underlying distribution $\truedist$.
Our goal is to learn a surrogate model from $\mathcal{D}$ that predicts the population parameters $\theta_{ij}$ of a performance distribution  $\learneddist$. 
As more evaluation rollouts are conducted, we add the outcomes to $\mathcal{D}$ and continue training the surrogate model.
% A single experiment can be defined by the policy $\pi_i$ being executed on task $t_j$.

%The surrogate model should be able to capture the performance relationship between tasks and policies.
To train an effective surrogate model,  we use notions of similarity between tasks and policies. 
% For instance, tasks that involve similar skills, such as "picking up an apple" and "picking up an orange," should result in similar performance distributions for a given robot policy.
Thus, we need a representation that captures the similarities between policies and tasks with respect to their performance distributions.
We define a policy embedding $e_{\pi_i}$ and task embedding $e_{T_j}$, where similar performance distributions in task and policy can be captured based on the embeddings.
These policy and task representations are then provided as input to an MLP that predicts the estimated population parameters:
\begin{equation}
    \hat{\theta}_{ij} = f(\pi_i, T_j) = \text{MLP}(e_{\pi_i}, e_{T_j}).
\end{equation}

\textbf{Task and Policy Representation.} 
\label{sec:method}
To define the task and policy embeddings $e_{\pi_i}, e_{T_j}$, we design various types of embeddings.
In practice, we cannot know the relationship between policies in advance as we are conducting evaluation.
Therefore, we define the policy embedding to be a fixed, randomly initialized embedding to act as an identifier for the policy in a given experiment.

For the task embedding $e_{\pi_i}$, we leverage language embeddings from MiniLMv2~\cite{gu2024minillm} which we reduce to 32 dimensions using PCA over all tasks.
However, we found that language embeddings overly focus on nouns as opposed to verbs, which causes issues as actions with similar nouns but different verbs would be closer together verbs with the same nouns. 
Thus, we apply the following procedure to mitigate this issue.
We (1) use part-of-speech tagging to extract all verbs and verb phrases, (2) compute a language embedding for the verb $e_{T_j}^{\text{verb}}$ and for the entire task description  $e_{t_j}^{\text{task}}$, and then (3) compute the task embedding 
\begin{equation}
     e_{T_j} =  0.8 \cdot e_{T_j}^{\text{verb}} +  0.2\cdot e_{T_j}^{\text{task}} + 0.1\cdot \mathcal{N}(0,1).
\end{equation}
We also found that the embeddings were often too close across multiple tasks, and we found that adding a slight noise term helped separate close embeddings.
Experiments on this result are in Section~\ref{sec:task_results}.
% These embeddings were them 

\textbf{Population Parameter Estimation.}
Outcomes in robot learning can take the form of continuous values like rewards, time to completion, or task progress, and binary values like task success.
Thus, the underlying distribution from the surrogate model depends on the type of task.
We consider two types of underlying distributions.
When $X_{ij}$ is continuous, $\learneddist$ takes the form of a mixture of Gaussians with $K$ components,
\begin{equation}
    % P(X_{ij}; \theta_{ij}) = \sum_{k=1}^{K}\pi_k\mathcal{N}(\mu_k, \sigma_k),\\
    \sample \sim \learneddist = \sum_{k=1}^{K}p_k\mathcal{N}(\mu_k, \sigma_k),
\end{equation}
where $\pi_k, \mu_k,$ and $\sigma_k$ are the mixing coefficients, means, and standard deviations of the Gaussian components respectively that are predicted from the surrogate model $\theta_{ij} = f(\pi_i, T_j)$.
% The parameters $\theta_{ij}=\{\pi_k, \mu_k, \sigma_k\}^{K}_{k=1}$ are predicted using the surrogate model.
We thus train the surrogate model with a mixture density loss~\cite{bishop1994mixture,ha2018world} to minimize the negative log-likelihood of the observed data under the mixture model.
In our experiments on continuous outcome distributions, we use $K=2$ Gaussian components, as robotics performance is often bimodal; robots either fail catastrophically or they maintain non-zero performance.

In the case where $X_{ij}$ is binary, indicating success or failure, $\learneddist$ takes the form of a Bernoulli distribution:
\begin{equation}
    % P(X_{ij}; \theta_{ij}) = p^{x_{ij}}(1-p)^{1-x_{ij}}
    \sample \sim \learneddist =  p^{x_{ij}}(1-p)^{1-x_{ij}},
\end{equation}
where $\theta_{ij} = \{p \in [0,1] \}$ represents the success probability predicted by the surrogate model trained using cross-entropy loss. 



\subsection{Cost-aware Active Experiment Selection}

% We can now use this surrogate model for selecting policy-task robot experience to execute. 
We explore cost-aware, active-experiment acquisition functions that guide selection of experiments based on their expected utility while considering associated costs.
To define the acquisition function, we first focus on how to measure the informativeness of a policy-task evaluation, which we capture through expected information gain.

\textbf{Expected Information Gain.}
Expected Information Gain (EIG) quantifies the value of an experiment by estimating how much it reduces the predictive uncertainty of the performance distribution for a policy-task pair. 
Since the surrogate model estimates performance \textit{distributions}, we define the EIG of a policy-task pair using a Bayesian Active Learning by Disagreement (BALD)~\cite{houlsby2011bayesian} formulation for probabilistic models
\begin{equation}
    \mathcal{I}(\pi_i, T_j) = \underbrace{\mathbb{H}[\learneddist]}_{\text{marginal entropy}} - \underbrace{\mathbb{E}_{\theta_{ij} \sim f(\theta_{ij}|\mathcal{D})} [\mathbb{H}[\learneddist | \theta_{ij}]]}_{\text{expected conditional entropy}}.
\end{equation}
% The first term is the entropy over the marginal predictive distribution
The first term represents the marginal entropy over $Q_{ij}$, which quantifies the total uncertainty in $Q_{ij}$. 
The second term corresponds to the expected conditional entropy over multiple samples of parameters $\theta_{ij}$.
% where $P(\learneddist|\mathcal{D})$ is the marginal predictive distribution over the performance distribution $\learneddist,$ while $P(\learneddist | \theta_{ij})$ is the conditional predictive distribution given a sampled set of parameters $\theta_{ij}$. 
% Thus, the first term is the entropy over the marginal predictive distribution and captures the epistemic and aleatoric uncertainties of the model's predictions, while the second term captures the expected aleatoric uncertainty.
Thus, $\mathcal{I}(\pi_i, T_j)$ captures the disagreement between multiple samples of distributions.
For example, if 10 sampled parameters for a Gaussian have very different distributions, then their disagreement will be high.
Since the entropy of a mixture of Gaussians generally lacks a closed-form solution, we estimate the entropy by discretizing the empirical distribution into $n=25$ bins for which to compute entropy over.

BALD ensures the EIG score is higher in test instances where there is disagreement in the predicted distributions across sampled parameters.
In this case, we define the acquisition functions $a(\pi_i,T_j)=\mathcal{I}(\pi_i,T_j)$.



To compute the expected information gain, we require multiple samples of $\Theta_{ij}$; however, we only train a single MLP.
Inspired by Monte Carlo dropout~\cite{gal2016dropout} and past literature~\cite{loquercio2020general,ledda2023dropout}, we apply dropout only at test-time to compute multiple samples of $\theta_{ij}$ from the surrogate model $f(\cdot)$.
% We found that dropout early in the sampling process would lead to overfitting on the small dataset, leading to low entropy in the samples of $\Theta_{ij}$.

% We use Monte Carlo (MC) dropout~\cite{gal2016dropout} to estimate sampling from $f(\Theta_{i,j} | \mathcal{D})$. 
% We found that Monte Carlo (MC) dropout~\cite{gal2016dropout}, where dropout is applied during training time, then at test time to compute multiple samples of $\theta_{ij}$ from the surrogate model $f(\cdot)$, leads to overfitting on the small amounts of training data available early in the evaluation process. 
% This overfitting lead to the disagreement between sampled parameters to be nearly zero, causing the EIG scores to be useful after hundreds of expensive evaluations.


% However, in the setting of robot evaluation where experiments are expensive, our surrogate model is initially trained with minimal data.
% We found that dropout during training time early in the evaluation process led to overfitting, as few examples had been seen, which caused the disagreement between sampled parameters to be essentially zero.
% Thus, the EIG scores would only be useful after hundreds of expensive evaluations.
% Past work in the active learning literature~\cite{tosh2022targeted} avoids this cold start problem by initializing their model with hundreds or thousands of training instances, which is impractical for an evaluation framework.
% To address this cold start problem, we found that restricting dropout to test time only effectively allowed our model to provide useful EIG scores earlier in the evaluation process.


\textbf{Cost-Aware EIG}.
While EIG effectively quantifies the informativeness of an experiment, it does not consider the costs of conducting evaluation.
To make EIG cost-aware, we design the following acquisition function based on prior work that simply integrates cost with a multiplicative factor~\cite{paria2020cost,lee2020cost}:
\begin{equation}
    a_{\text{cost-aware}}(\pi_i, T_j, T_\text{current}) = \dfrac{\mathcal{I}(\pi_i, T_j)}{(\lambda \cdot c_{\text{switch}}(T_{\text{current}}, T_j))+1},
    \label{eq:cost_aware_eig}
\end{equation}
where $\mathcal{I}(\pi_i, T_j)$ represents EIG for the policy $\pi_i$ on task $T_j$, $c_{\text{switch}}(T_{\text{current}}, T_j))$ is the cost of switching from the current task $T_{\text{current}}$ to a new task $T_j$, and $\lambda$ is a hyperparameter that controls the cost sensitivity.


\section{\thename}
\subsection{End-to-End Driving Policy}
The overall framework of \thename{} is depicted in Fig.~\ref{fig:framework}. 
\thename{} takes multi-view image sequences as input, transforms the sensor data into scene token embeddings, outputs the probabilistic distribution of actions, and samples an action to control the vehicle. 

\boldparagraph{BEV Encoder.} 
We first employ a BEV encoder~\cite{li2022bevformer} to transform multi-view image features from the perspective view to the Bird's Eye View (BEV), obtaining a feature map in the BEV space. This feature map is then used to learn instance-level map features and agent features.

\boldparagraph{Map Head.} 
Then we utilize a group of map tokens~\cite{maptrv2, liao2022maptr, lanegap} to learn the vectorized map elements of the driving scene from the BEV feature map, including lane centerlines, lane dividers, road boundaries, arrows, traffic signals, \etc.

\boldparagraph{Agent Head.} 
Besides, a group of agent tokens~\cite{jiang2022pip} is adopted to predict the motion information of other traffic participants, including location, orientation, size, speed, and multi-mode future trajectories.

\boldparagraph{Image Encoder.} 
Apart from the above instance-level map and agent tokens, we also use an individual image encoder~\cite{vit,he2016resnet} to transform the original images into image tokens. These image tokens provide dense and rich scene information for planning, complementary to the instance-level tokens.

\begin{figure}[t]
\centering
\includegraphics[width=0.98\linewidth]{fig/post-training-2.pdf} 
\caption{\textbf{Post-training.}  $N$  workers parallelly run. The generated rollout data $(s_t,a_t, r_{t+1},s_{t+1},...)$ are recorded in a rollout buffer. Rollout data and human driving demonstrations are used in RL- and IL-training steps to fine-tune the AD policy synergistically.
}
\label{fig:post-training}
\end{figure}

\boldparagraph{Action Space.} 
To accelerate the convergence of RL training, we design a decoupled discrete action representation. 
We divide the action into two independent components: lateral action and longitudinal action. 
The action space is constructed over a short $0.5$-second time horizon, during which the vehicle's motion is approximated by assuming constant linear and angular velocities. 
Under this assumption, the lateral action $a^x$ and longitudinal action $a^y$ can be directly computed based on the current linear and angular velocities.
By combining decoupling with a limited temporal scope and simplified motion model, our approach effectively reduces the dimensionality of the action space, accelerating training convergence.


\boldparagraph{Planning Head.} 
We use $E_\text{scene}$ to denote the scene representation, which consists of map tokens, agent tokens, and image tokens. We initialize a planning embedding denoted as $E_\text{plan}$. A cascaded Transformer decoder $\phi$ takes the planning embedding $E_\text{plan}$ as the query and the scene representation $E_\text{scene}$ as both key and value.

The output of the decoder $\phi$ is then combined with navigation information $E_\text{navi}$ and ego state $E_\text{state}$ to output the probabilistic distributions of the lateral action $a^x$ and the longitudinal action $a^y$:
\begin{equation}
\begin{aligned}
     \pi(a^x\mid s) = & \text{softmax}(\text{MLP}(\phi(E_\text{plan}, E_\text{scene}) \\
    & + E_\text{navi} + E_\text{state})), \\
     \pi(a^y\mid s) = & \text{softmax}(\text{MLP}(\phi(E_\text{plan}, E_\text{scene}) \\
     & + E_\text{navi} + E_\text{state})),
\label{eq:action distribution}
\end{aligned}
\end{equation}
where $E_\text{plan}$, $E_\text{navi}$, $E_\text{state}$, and the output of $\text{MLP}$ are all of the same dimension ($1 \times D$).

The planning head also outputs the value functions $V_x(s)$ and $V_y(s)$, which estimate the expected cumulative rewards for the lateral and longitudinal actions, respectively: 
\begin{equation}
\begin{aligned}
    & V_x(s) = \text{MLP}(\phi(E_\text{plan}, E_\text{scene}) + E_\text{navi} + E_\text{state}), \\
    & V_y(s) = \text{MLP}(\phi(E_\text{plan}, E_\text{scene}) + E_\text{navi} + E_\text{state}).
\end{aligned}
\end{equation}
The value functions are used in RL training (Sec.~\ref{sec:optimization}).

\subsection{Training Paradigm}
We adopt a three-stage training paradigm: perception pre-training, planning pre-training, and reinforced post-training, as shown in Fig.~\ref{fig:framework}.

\boldparagraph{Perception Pre-Training.} 
Information in the image is sparse and low-level. In the first stage,  
the map head and the agent head explicitly output map elements and agent motion information, which are supervised with ground-truth labels. Consequently,  
map tokens and agent tokens implicitly encode the corresponding high-level information.  
In this stage, we only update the parameters of the BEV encoder, the map head, and the agent head.



\boldparagraph{Planning Pre-Training.} 
In the second stage, to prevent the unstable cold start of RL training, IL is first performed to initialize the probabilistic distribution of actions based on large-scale real-world driving demonstrations from expert drivers. In this stage, we only update the parameters of the image encoder and the planning head, while the parameters of the BEV encoder, map head, and agent head are frozen. The optimization objectives of perception tasks and planning tasks may conflict with each other. However, with the training stage and parameters decoupled, such conflicts are mostly avoided.

\boldparagraph{Reinforced Post-Training.} 
In the reinforced post-training, RL and IL synergistically fine-tune the distribution. RL aims to guide the policy to be sensitive to critical risky events and adaptive to out-of-distribution situations. IL serves as the regularization term to keep the policy's behavior similar to that of humans.

We select a large amount of risky dense-traffic clips from collected driving demonstrations. For each clip, we train an independent 3DGS model that reconstructs the clip and serves as a digital driving environment.  
As shown in Fig.~\ref{fig:post-training}, we set $N$ parallel workers.  
Each worker randomly samples a 3DGS environment and begins rollout, i.e., the AD policy controls the ego vehicle to move and iteratively interacts with the 3DGS environment. After the rollout process of this 3DGS environment ends, the generated rollout data $(s_t,a_t, r_{t+1},s_{t+1},...)$ are recorded in a rollout buffer, and the worker will sample a new 3DGS environment for another round of rollout.

As for policy optimization, we iteratively perform RL-training steps and IL-training steps. For RL-training steps, we sample data from the rollout buffer and follow the Proximal Policy Optimization (PPO) framework~\cite{PPO} to update the AD policy. For IL-training steps, we use real-world driving demonstrations to update the policy. After a fixed number of training steps, the updated AD policy is sent to every worker to replace the old one, to avoid a distribution shift between data collection and optimization.
We only update the parameters of the image encoder and the planning head. The parameters of the BEV encoder, the map head, and the agent head are frozen.  
The detailed RL design is presented below.

\subsection{Interaction Mechanism between AD Policy and 3DGS Environment}
In the 3DGS environment, the ego vehicle acts according to the AD policy. Other traffic participants act according to real-world data in a log-replay manner.  
A simplified kinematic bicycle model is employed to iteratively update the ego vehicle's pose at every $\Delta t$ seconds as follows:  
\begin{equation}
\begin{aligned}
x_{t+1}^{w} & = x_{t}^w + v_t \cos \left(\psi_{t}^w\right) \Delta t, \\
y_{t+1}^{w} & = y_{t}^w + v_t \sin \left(\psi_{t}^w\right) \Delta t, \\
\psi_{t+1}^{w} & = \psi_{t}^w + \frac{v_t}{L} \tan \left(\delta_t\right) \Delta t,
\label{equation:kinematic_model}
\end{aligned}
\end{equation}  
where $x_t^{w}$ and $y_t^{w}$ denote the position of the ego vehicle relative to the world coordinate; $\psi_t^w$ is the heading angle that defines the vehicle's orientation with respect to the world $x$-coordinate; $v_t$ is the linear velocity of the ego vehicle; $\delta_t$ is the steering angle of the front wheels; and $L$ is the wheelbase, i.e., the distance between the front and rear axles.

During the rollout process, the AD policy outputs actions $(a_t^x, a_t^y)$ for a $0.5$-second time horizon at time step $t$. We derive the linear velocity $v_t$ and steering angle $\delta_t$ based on $(a_t^x, a_t^y)$.  
Based on the kinematic model in Eq.~\ref{equation:kinematic_model},  
the pose of the ego vehicle in the world coordinate system is updated from ${p}_t = (x_{t}^w, y_{t}^w, \psi_{t}^w)$ to ${p}_{t+1} = (x_{t+1}^{w}, y_{t+1}^{w}, \psi_{t+1}^{w})$.  

Based on the updated ${p}_{t+1}$, the 3DGS environment computes the new ego vehicle's state $s_{t+1}$. The updated pose ${p}_{t+1}$ and state $s_{t+1}$ serve as the input for the next iteration of the inference process.

The 3DGS environment also generates rewards $\mathcal{R}$ (Sec.~\ref{sec:reward}) according to multi-source information (including trajectories of other agents, map information, the expert trajectory of the ego vehicle, and the parameters of Gaussians), which are used to optimize the AD policy (Sec.~\ref{sec:optimization}).

\begin{figure}[t]
\centering
\includegraphics[width=1.0\linewidth]{fig/reward.pdf} 
\caption{\textbf{Example diagram of four types of reward sources.}  (1): Collision with a dynamic obstacle ahead triggers a reward $r_{\text{dc}}$. (2): Hitting a static roadside obstacle incurs a reward $r_{\text{sc}}$. (3): Moving onto the curb exceeds the positional deviation threshold $d_{\text{max}}$, triggering a reward $r_{\text{pd}}$. (4): Drifting toward the adjacent lane exceeds the heading deviation threshold $\psi_{\text{max}}$, triggering a reward $r_{\text{hd}}$.
}
\label{fig: reward source}
\end{figure}
\subsection{Reward Modeling}
\label{sec:reward}
The reward is the source of the training signal, which determines the optimization direction of RL. The reward function is designed to guide the ego vehicle's behavior by penalizing unsafe actions and encouraging alignment with the expert trajectory. It is composed of four reward components: (1) collision with dynamic obstacles, (2) collision with static obstacles, (3) positional deviation from the expert trajectory, and (4) heading deviation from the expert trajectory:
\begin{equation}
\begin{aligned}
\mathcal{R} = \{r_{\text{dc}}, r_{\text{sc}}, r_{\text{pd}}, r_{\text{hd}}  \}. 
\end{aligned}
\end{equation}

As illustrated in Fig.~\ref{fig: reward source}, these reward components are triggered under specific conditions.  
In the 3DGS environment, dynamic collision is detected if the ego vehicle's bounding box overlaps with the annotated bounding boxes of dynamic obstacles, triggering a negative reward $r_{\text{dc}}$. Similarly, static collision is identified when the ego vehicle's bounding box overlaps with the Gaussians of static obstacles, resulting in a negative reward $r_{\text{sc}}$.  
Positional deviation is measured as the Euclidean distance between the ego vehicle's current position and the closest point on the expert trajectory. A deviation beyond a predefined threshold $d_{\text{max}}$ incurs a negative reward $r_{\text{pd}}$.  
Heading deviation is calculated as the angular difference between the ego vehicle's current heading angle $ \psi_t $ and the expert trajectory's matched heading angle $\psi_{\text{expert}}$. A deviation beyond a threshold $ \psi_{\text{max}}$ results in a negative reward $r_{\text{hd}}$.

Any of these events, including dynamic collision, static collision, excessive positional deviation, or excessive heading deviation, triggers immediate episode termination. Because after such events occur, the 3DGS environment typically generates noisy sensor data, which is detrimental to RL training.

\subsection{Policy Optimization}
\label{sec:optimization}
In the closed-loop environment, the error in each single step accumulates over time. The aforementioned rewards are not only caused by the current action but also by the actions of the preceding steps.  
The rewards are propagated forward with Generalized Advantage Estimation (GAE)~\cite{gae} to optimize the action distribution of the preceding steps.

Specifically, for each time step $t$, we store the current state $s_t$, action $a_t$, reward $r_t$, and the estimate of the value $V(s_t)$.  
Based on the decoupled action space, and considering that different rewards have different correlations to lateral and longitudinal actions, the reward $r_t$ is divided into lateral reward $r_t^x$ and longitudinal reward $r_t^y$:
\begin{equation}
\begin{aligned}
r_t^x &= r_t^{\text{sc}} + r_t^{\text{pd}} + r_t^{\text{hd}}, \\
r_t^y &= r_t^{\text{dc}}.
\label{eq:reward-decouple}
\end{aligned}
\end{equation}
Similarly, the value function $V(s_t)$ is decoupled into two components: $V_x(s_t)$ for the lateral dimension and $V_y(s_t)$ for the longitudinal dimension. These value functions estimate the expected cumulative rewards for the lateral and longitudinal actions, respectively. The advantage estimates $\hat{A}_t^x$ and $\hat{A}_t^y$ are then computed as follows:
\begin{equation}
\begin{aligned}
\delta_t^x &= r_t^x + \gamma V_x(s_{t+1}) - V_x(s_t), \\
\delta_t^y &= r_t^y + \gamma V_y(s_{t+1}) - V_y(s_t), \\
\hat{A}_t^x &= \sum_{l=0}^{\infty}(\gamma \lambda)^l \delta_{t+l}^x, \\
\hat{A}_t^y &= \sum_{l=0}^{\infty}(\gamma \lambda)^l \delta_{t+l}^y,
\label{eq:advantage}
\end{aligned}
\end{equation}
where $\delta_t^x$ and $\delta_t^y$ are the temporal difference errors for the lateral and longitudinal dimensions, $\gamma$ is the discount factor, and $\lambda$ is the GAE parameter that controls the trade-off between bias and variance.

To further clarify the relationship between the advantage estimates and the reward components, we decompose $\hat{A}_t^x$ and $\hat{A}_t^y$ based on the reward decomposition in Eq.~\ref{eq:reward-decouple} and the advantage estimation in Eq.~\ref{eq:advantage}. Specifically, we derive the following decomposition:
\begin{equation}
\begin{aligned}
\hat{A}_t^x &= \hat{A}_t^{\text{sc}} + \hat{A}_t^{\text{pd}} + \hat{A}_t^{\text{hd}}, \\
\hat{A}_t^y &= \hat{A}_t^{\text{dc}},
\end{aligned}
\end{equation}
where $\hat{A}_t^{\text{sc}}$ is the advantage estimate for avoiding static collisions, $\hat{A}_t^{\text{pd}}$ is the advantage estimate for minimizing positional deviations, $\hat{A}_t^{\text{hd}}$ is the advantage estimate for minimizing heading deviations, and $\hat{A}_t^{\text{dc}}$ is the advantage estimate for avoiding dynamic collisions.

These advantage estimates are used to guide the update of the AD policy $\pi_{\theta}$, following the PPO framework~\cite{PPO}. By leveraging the decomposed advantage estimates $\hat{A}_t^x$ and $\hat{A}_t^y$, we can independently optimize the lateral and longitudinal dimensions of the policy. This is achieved by defining separate objective functions $\mathcal{L}_x^{\text{CLIP}}(\theta)$ and $\mathcal{L}_y^{\text{CLIP}}(\theta)$ for each dimension,  as follows:
\begin{equation}
\begin{aligned}
\mathcal{L}_x^{\text{PPO}}(\theta) &= \mathbb{E}_t \left[ \min \left( \rho_t^x \hat{A}_t^x, \ \text{clip}(\rho_t^x, 1-\epsilon_x, 1+\epsilon_x) \hat{A}_t^x \right) \right], \\
\mathcal{L}_y^{\text{PPO}}(\theta) &= \mathbb{E}_t \left[ \min \left( \rho_t^y \hat{A}_t^y, \ \text{clip}(\rho_t^y, 1-\epsilon_y, 1+\epsilon_y) \hat{A}_t^y \right) \right], \\
\mathcal{L}^{\text{PPO}}(\theta) &= \mathcal{L}_x^{\text{PPO}}(\theta) + \mathcal{L}_y^{\text{PPO}}(\theta),
\end{aligned}
\end{equation}
where $\rho_t^x = \frac{\pi_{\theta}(a_t^x \mid s_t)}{\pi_{\theta_{\text{old}}}(a_t^x \mid s_t)}$ is the importance sampling ratio for the lateral dimension, $\rho_t^y = \frac{\pi_{\theta}(a_t^y \mid s_t)}{\pi_{\theta_{\text{old}}}(a_t^y \mid s_t)}$ is the importance sampling ratio for the longitudinal dimension, $\epsilon_x$ and $\epsilon_y$ are small constants that control the clipping range for the lateral and longitudinal dimensions, ensuring stable policy updates.

The clipped objective function $\mathcal{L}^{\text{PPO}}(\theta)$ prevents excessively large updates to the policy parameters $\theta$, thereby maintaining training stability.

\begin{table*}[ht]
    \centering
{
\begin{tabular}{lccccccccc}
    \toprule
    RL:IL & CR$\downarrow$ & DCR$\downarrow$ & SCR$\downarrow$ & DR$\downarrow$ & PDR$\downarrow$ & HDR$\downarrow$ &ADD$\downarrow$ & Long. Jerk$\downarrow$ & Lat. Jerk$\downarrow$ \\
    \midrule
     0:1  & 0.229 & 0.211 & 0.018 & 0.066 & 0.039 & 0.027  & 0.238 & 3.928 & 0.103\\
     1:0  & 0.143 & 0.128 & 0.015 &0.080 &0.065 &0.015 &0.345 &4.204 &0.085\\
     2:1 & 0.137 & 0.125 & 0.012 & 0.059 & 0.050 & 0.009  & 0.274 & 4.538 & 0.092\\
     4:1 & 0.089 & 0.080 & 0.009 & 0.063 & 0.042 & 0.021  & 0.257 & 4.495 & 0.082 \\
     8:1 & 0.125 & 0.116 & 0.009 & 0.084 & 0.045 & 0.039  & 0.323 & 5.285 & 0.115\\
    \bottomrule
\end{tabular}
}
    \caption{\textbf{Ablation on RL-to-IL step mixing ratios in the reinforced post-training stage.}}
    \label{tab:ratio}
\end{table*}

\subsection{Auxiliary Objective}
RL usually faces the challenge of sparse rewards, which makes the convergence process unstable and slow. To speed up convergence, we introduce auxiliary objectives that provide dense guidance to the entire action distribution.

The auxiliary objectives are designed to penalize undesirable behaviors by incorporating specific reward sources, including dynamic collisions, static collisions, positional deviations, and heading deviations. These objectives are computed based on the actions \( a_t^{x, \text{old}} \) and \( a_t^{y, \text{old}} \) selected by the old AD policy \( \pi_{\theta_{\text{old}}} \) at time step \( t \). To facilitate the evaluation of these actions, we separate the probability distribution of the action into four parts:
\begin{equation}
\begin{aligned}
\Delta \pi_y^{\text{dec}} &= \sum_{a_t^y < a_t^{y, \text{old}}} \pi_\theta(a_t^y \mid s_t), \\
\Delta \pi_y^{\text{acc}} &= \sum_{a_t^y > a_t^{y, \text{old}}} \pi_\theta(a_t^y \mid s_t), \\
\Delta \pi_x^{\text{left}} &= \sum_{a_t^x < a_t^{x, \text{old}}} \pi_\theta(a_t^x \mid s_t), \\
\Delta \pi_x^{\text{right}} &= \sum_{a_t^x > a_t^{x, \text{old}}} \pi_\theta(a_t^x \mid s_t).
\end{aligned}
\end{equation}
Here, \( \Delta \pi_y^{\text{dec}} \) represents the total probability of deceleration actions, \( \Delta \pi_y^{\text{acc}} \) represents the total probability of acceleration actions, \( \Delta \pi_x^{\text{left}} \) represents the total probability of leftward steering actions, and \( \Delta \pi_x^{\text{right}} \) represents the total probability of rightward steering actions.

\boldparagraph{Dynamic Collision Auxiliary Objective.}  
The dynamic collision auxiliary objective adjusts the longitudinal control action \(a_t^y\) based on the location of potential collisions relative to the ego vehicle. If a collision is detected ahead, the policy prioritizes deceleration actions (\(a_t^y < a_t^{y, \text{old}}\)); if a collision is detected behind, it encourages acceleration actions (\(a_t^y > a_t^{y, \text{old}}\)). To formalize this behavior, we define a directional factor \(f_\text{dc}\):
\begin{equation}
\begin{aligned}
f_\text{dc} = \begin{cases} 
1 & \text{if the collision is ahead}, \\
-1 & \text{if the collision is behind}.
\end{cases} 
\end{aligned}
\end{equation}

The auxiliary objective for dynamic collision avoidance is defined as:
\begin{equation}
\begin{aligned}
\mathcal{L}_\text{dc}(\theta_y) = \mathbb{E}_t \left[ 
    \hat{A}_t^\text{dc} \cdot f_\text{dc} \cdot (\Delta \pi_y^{\text{dec}} - \Delta \pi_y^{\text{acc}})
\right],
\end{aligned}
\end{equation}
where \(\hat{A}_t^\text{dc}\) is the advantage estimate for dynamic collision avoidance.

\boldparagraph{Static Collision Auxiliary Objective.}  
The static collision auxiliary objective adjusts the steering control action $a_t^x$ based on the proximity to static obstacles. If the static obstacle is detected on the left side, the policy promotes rightward steering actions ($a_t^x > a_t^{x,\text{old}}$); if the static obstacle is detected on the right side, it promotes leftward steering actions ($a_t^x < a_t^{x,\text{old}}$). To formalize this behavior, we define a directional factor $f_\text{sc}$:  
\begin{equation}
\begin{aligned}
f_\text{sc} = \begin{cases} 
1 & \text{if static obstacle is on the left}, \\
-1 & \text{if static obstacle is on the right}.
\end{cases} 
\end{aligned}
\end{equation}

The auxiliary objective for static collision avoidance is defined as:  
\begin{equation}
\begin{aligned}
\mathcal{L}_\text{sc}(\theta_x) = \mathbb{E}_t \left[ 
    \hat{A}_t^\text{sc} \cdot f_\text{sc} \cdot (\Delta \pi_x^{\text{right}} - \Delta \pi_x^{\text{left}})
\right],
\end{aligned}
\end{equation}  
where $\hat{A}_t^\text{sc}$ is the advantage estimate for static collision avoidance.  

\boldparagraph{Positional Deviation Auxiliary Objective.}  
The positional deviation auxiliary objective adjusts the steering control action $a_t^x$ based on the ego vehicle's lateral deviation from the expert trajectory. If the ego vehicle deviates leftward, the policy promotes rightward corrections ($a_t^x > a_t^{x,\text{old}}$); if it deviates rightward, it promotes leftward corrections ($a_t^x < a_t^{x,\text{old}}$). We formalize this with a directional factor $f_\text{pd}$:  
\begin{equation}
\begin{aligned}
f_\text{pd} = \begin{cases} 
1 & \text{if ego vehicle deviates leftward}, \\
-1 & \text{if ego vehicle deviates rightward}.
\end{cases} 
\end{aligned}
\end{equation}

The auxiliary objective for positional deviation correction is:
\begin{equation}
\begin{aligned}
\mathcal{L}_\text{pd}(\theta_x) = \mathbb{E}_t \left[ 
    \hat{A}_t^\text{pd} \cdot f_\text{pd} \cdot (\Delta \pi_x^{\text{right}} - \Delta \pi_x^{\text{left}})
\right],
\end{aligned}
\end{equation}  
where $\hat{A}_t^\text{pd}$ estimates the advantage of trajectory alignment.

\boldparagraph{Heading Deviation Auxiliary Objective.}  
The heading deviation auxiliary objective adjusts the steering control action $a_t^x$ based on the angular difference between the ego vehicle’s current heading and the expert’s reference heading. If the ego vehicle deviates counterclockwise, the policy promotes clockwise corrections ($a_t^x > a_t^{x,\text{old}}$); if it deviates clockwise, it promotes counterclockwise corrections ($a_t^x < a_t^{x,\text{old}}$). To formalize this behavior, we define a directional factor $f_\text{hd}$:  
\begin{equation}
\begin{aligned}
f_\text{hd} = \begin{cases} 
1 & \text{if ego vehicle deviates clockwise}, \\
-1 & \text{if ego vehicle deviates counterclockwise}.
\end{cases} 
\end{aligned}
\end{equation}

The auxiliary objective for heading deviation correction is then defined as:  
\begin{equation}
\begin{aligned}
\mathcal{L}_\text{hd}(\theta_x) = \mathbb{E}_t \left[ 
    \hat{A}_t^\text{hd} \cdot f_\text{hd} \cdot (\Delta \pi_x^{\text{right}} - \Delta \pi_x^{\text{left}})
\right],
\end{aligned}
\end{equation}  
where $\hat{A}_t^\text{hd}$ is the advantage estimate for heading alignment.  

\begin{table*}[ht]
\begin{center}
\centering
\resizebox{0.98\textwidth}{!}{
\begin{tabular}{cccccccccccccc}
\toprule
\multirow{2}{*}{ID} & Dynamic & Static & Position & Heading & \multirow{2}{*}{CR$\downarrow$} &\multirow{2}{*}{DCR$\downarrow$} &\multirow{2}{*}{SCR$\downarrow$} &\multirow{2}{*}{DR$\downarrow$} &\multirow{2}{*}{PDR$\downarrow$} &\multirow{2}{*}{HDR$\downarrow$} &\multirow{2}{*}{ADD$\downarrow$} &\multirow{2}{*}{Long. Jerk$\downarrow$} &\multirow{2}{*}{Lat. Jerk$\downarrow$}\\
& Collision & Collision & Deviation & Deviation & & & & & & & & & \\
\midrule
1 & \cmark  &  &  &  & 0.172 & 0.154 & 0.018 & 0.092 & 0.033 & 0.059  & 0.259 & 4.211 & 0.095 \\
2 &  & \cmark & \cmark & \cmark & 0.238 & 0.217 & 0.021 & 0.090 & 0.045 & 0.045  & 0.241 & 3.937 & 0.098 \\
3 & \cmark &  & \cmark & \cmark & 0.146 & 0.128 & 0.018 & 0.060 & 0.030 & 0.030  & 0.263 & 3.729 & 0.083\\
4 & \cmark & \cmark &  & \cmark & 0.151 & 0.142 & 0.009 & 0.069 & 0.042 & 0.027 & 0.303 & 3.938 & 0.079\\
5 & \cmark & \cmark & \cmark &  & 0.166 & 0.157 & 0.009 & 0.048 & 0.036 & 0.012 & 0.243 & 3.334 & 0.067\\
6 & \cmark & \cmark & \cmark & \cmark & 0.089 & 0.080 & 0.009 & 0.063 & 0.042 & 0.021 & 0.257 & 4.495 & 0.082 \\
\bottomrule
\end{tabular}
}
\end{center}
\vspace{-2mm}
\caption{\textbf{Ablation on reward sources.} The table shows the impact of different reward components on performance.}
\label{tab:reward_ablation}
\end{table*}

\begin{table*}[ht]
\begin{center}
\centering
\resizebox{0.98\textwidth}{!}{
\begin{tabular}{ccccccccccccccc}
\toprule
\multirow{2}{*}{ID} & \multirow{2}{*}{PPO Obj.}  & Dynamic Col. & Static Col. & Position Dev. & Heading Dev. & \multirow{2}{*}{CR$\downarrow$} & \multirow{2}{*}{DCR$\downarrow$}  & \multirow{2}{*}{SCR$\downarrow$} & \multirow{2}{*}{DR$\downarrow$} & \multirow{2}{*}{PDR$\downarrow$} & \multirow{2}{*}{HDR$\downarrow$} & \multirow{2}{*}{ADD$\downarrow$} & \multirow{2}{*}{Long. Jerk$\downarrow$} & \multirow{2}{*}{Lat. Jerk$\downarrow$} \\
& & Auxiliary Obj. & Auxiliary Obj. & Auxiliary Obj. & Auxiliary Obj. & & & & & & & & & \\
\midrule
1 &\cmark&  &  &  &  & 0.249 & 0.223 & 0.026 & 0.077 & 0.047 & 0.030  & 0.266 & 4.209 & 0.104 \\
2 &\cmark& \cmark &  &  &  & 0.178 & 0.163 & 0.015 & 0.151 & 0.101 & 0.050 & 0.301 & 3.906 & 0.085 \\
3 &\cmark&  & \cmark & \cmark & \cmark & 0.137 & 0.125 & 0.012 & 0.157 & 0.145 & 0.012 & 0.296 & 3.419 & 0.071 \\
4 &\cmark& \cmark &  & \cmark & \cmark & 0.169 & 0.151 & 0.018 & 0.075 & 0.042 & 0.033 & 0.254 & 4.450 & 0.098 \\
5 &\cmark& \cmark & \cmark &  & \cmark & 0.149 & 0.134 & 0.015 & 0.063 & 0.057 & 0.006 & 0.324 & 3.980 & 0.086 \\
6 &\cmark& \cmark & \cmark & \cmark & & 0.128 & 0.119  & 0.009 & 0.066 & 0.030 & 0.036  & 0.254 & 4.102 & 0.092 \\
7 &&\cmark  &\cmark  &\cmark  &\cmark  & 0.187 &0.175  &0.012 &0.077 &0.056  &0.021  &0.309  &5.014  &0.112  \\
8 &\cmark& \cmark & \cmark & \cmark & \cmark & 0.089 & 0.080 & 0.009 & 0.063 & 0.042 & 0.021  & 0.257 & 4.495 & 0.082 \\
\bottomrule
\end{tabular}
}
\end{center}
\vspace{-2mm}
\caption{\textbf{Ablation on auxiliary objectives.} The table shows the impact of different auxiliary objectives on performance.}
\label{tab:auxiliary_ablation}
\end{table*}

\boldparagraph{Overall Auxiliary Objectives.}  
The overall auxiliary objectives are a weighted sum of the individual objectives:
\begin{equation}
\begin{aligned}
\mathcal{L}_\text{aux}(\theta) = &\lambda_1 \mathcal{L}_\text{dc}(\theta_y) + \lambda_2 \mathcal{L}_\text{sc}(\theta_x)  + \\ 
&\lambda_3 \mathcal{L}_\text{pd}(\theta_x) +\lambda_4 \mathcal{L}_\text{hd}(\theta_x),
\end{aligned}
\end{equation}
where $\lambda_1$, $\lambda_2$, $\lambda_3$, and $\lambda_4$ are weighting coefficients that balance the contributions of each auxiliary objective.

\boldparagraph{Optimization Objective.}  
The final optimization objective combines the clipped PPO objective with the auxiliary objective:
\begin{equation}
\mathcal{L}(\theta) = \mathcal{L}^{\text{PPO}}(\theta) + \mathcal{L}_\text{aux}(\theta).
\end{equation}



\textbf{Active Experiment Selection}.
% As shown in Algorithm~\ref{alg:short_pseudocode}, we use this acquisition function to iteratively sample experiments.
We use this acquisition function to iteratively sample experiments, as shown in Algorithm~\ref{alg:short_pseudocode}.
To mitigate the cold-start problem in active learning, we initialize the dataset $\mathcal{D}$ with a a single randomly-selected task, for which every policy is evaluated 3 times.
We then train the surrogate model on this data.
At each query step, the acquisition function $a(\pi_i, T_j)$ is computed for all policy-task pairs, which quantifies their informativeness weighted by the cost.
To compute the entropy over model parameters for the EIG metric, we use MC dropout to sample 10 predicted outcome distributions.
To balance exploration and exploitation, we use an epsilon-greedy strategy with a rate of $\epsilon=0.1$.
The selected experiment $(\pi_i, T_j)$ is then executed 3 times, and the observed outcomes are added to the dataset $\mathcal{D}$.
We found in preliminary experiments that 3 trials per selected experiment was often better for cost-efficient population parameter estimation.
Given these new outcomes in the dataset, we keep training the surrogate model on the updated dataset improve its predictions over time.
% In our experiments, we continue this process for a fixed number of queries; however, 

\section{Evaluation on Offline Datasets}



To evaluate our active testing framework, we leverage evaluations that have already been conducted which we then sample offline.
We use experiments from the HAMSTER paper~\cite{li2025hamster}, the  OpenVLA paper~\cite{kim24openvla}, and from MetaWorld~\cite{yu2020meta}, as visualized in Figure~\ref{fig:datasets}.
For MetaWorld, we train two versions, one focused on understanding our framework's ability in evaluating different policies and another on evaluating multiple checkpoints of a single policy.
Each of these datasets can be modeled with different underlying distributions and have varying costs, semantic diversity, and skills.
More details on training for MetaWorld, switching costs for the datasets, and other details can be found in Appendix~\ref{app:offline_datasets}.

\begin{figure}[!t]
    \centering
    \includegraphics[width=.93\linewidth]{figs_images/datasets.pdf}
    \caption{\textbf{Offline Datasets used for Experiments.} We consider 4 settings: (1) evaluations from HAMSTER~\cite{li2025hamster}, (2) evaluations from the OpenVLA paper~\cite{kim24openvla}, (3) MetaWorld~\cite{yu2020meta} where we evaluate different policies, and (4) MetaWorld where we evaluate multiple checkpoints of a single policy. For the MetaWorld evaluations, we can model the performance distributions of success rate or continuous rewards. For OpenVLA, the outcomes are binary success rate. For HAMSTER, evaluations were run over a large number of tasks only once while tracking only task progress, so we use this mean value as a mean for a unimodal Gaussian and a fixed standard deviation.
    }
    \label{fig:datasets}
\end{figure}




\textbf{HAMSTER.}
We use evaluations from the HAMSTER paper~\cite{li2025hamster}, which evaluates a hierarchical VLA model against 4 other policies such as OpenVLA~\cite{kim24openvla} and Octo~\cite{octo_2023} across 81 tasks. 
These 81 tasks are of varying complexity, with diverse task types, objects, and linguistic variation that were evaluated once each.
Their work uses a continuous task progress metric; however, since they only evaluated each policy-task pair once, we treat the single continuous value as the mean of a Gaussian distribution with a fixed standard deviation.
For switching cost, we add an additional cost if the policy switches from one task type to another. 
More details on this cost can be found in Appendix~\ref{app:offline_datasets}.
% The success of that one policy-task pair evaluation is treated as the mean success metric for that experiment. 

\textbf{OpenVLA.}
We use evaluations from the OpenVLA paper~\cite{kim24openvla}, which compares 4 policies over 29 tasks. 
In their paper, some tasks allow for partial success (0.5). 
For simplicity, we round the partial successes down to maintain a binary success metric.
OpenVLA also provides results across two embodiments.
Therefore, in addition to a higher cost term to switching tasks that require a large scene reset, we add an additional cost term to switch between embodiments. More details in Appendix~\ref{app:offline_datasets}.

Given these datasets, we show that the types of policy and task representations that are useful for active learning, and then we can leverage the surrogate model for cost-aware active experiment selection. 
\begin{figure*}[t]
    \centering
    % placeholder
    \includegraphics[width=1\linewidth]{figs_images/task_rep.pdf}
    \caption{\textbf{Task and Policy Representation Experiments.} 
    We compute the average log likelihood of all outcomes under probability distribution represented by the predicted population parameters across various policy and task representations. We evaluate these methods over the HAMSTER, OpenVLA, and MetaWorld Checkpoints offline evaluation datasets over \textcolor{mypink}{continuous} and \textcolor{mygold}{binary} performance distributions.
    We find no large difference between random or optimal embeddings as a policy representation, indicating that there is not much shared information between policies.
    However, we find that for task representation, \textcolor{optimal}{\textbf{Optimal}} consistently perform the best, followed by \textcolor{verb}{\textbf{Verb}}, then \textcolor{lang}{\textbf{Lang}}, and lastly \textcolor{random}{\textbf{Random}}.
    Language-based embeddings is a good task representation that we can leverage for better active learning.
    % We find that optimal embeddings for tasks and policies perform the best, while random embeddings perform the worst. 
    % We find that language embeddings perform better as a task representation than random embeddings as it gets closer to the rate at which optimal embeddings improve.
    % \vspace{-1em}
    }
    \label{fig:model_exps}
\end{figure*}



\textbf{MetaWorld Policies.}
MetaWorld~\cite{yu2020meta} is an open-source simulated benchmark containing a set of 50 different manipulation environments for multi-task learning.
We train 10 policies on every environment with different policy architecture sizes and varying amounts of noise in the robot's state to create robot policies with diverse behaviors.
We then collected 100 trajectories of each policy-task pair to serve as an approximation of the true performance population distribution.
By using the MetaWorld simulator, we can estimate performance distributions for binary success rate and a continuous reward normalized between 0 and 1.
The switching cost is set based on whether the target object of the scene, such as a drawer, is swapped out for another object, like a lever.
This dataset allows us to understand how our framework can learn the performance distributions across diverse policies.

\textbf{MetaWorld Checkpoints.}
Evaluation on a robot is not only used for comparing policies, but also to find the best checkpoints.
As such, we train a single state-based MetaWorld policy, store 11 checkpoints over the training process, and then evaluate them.
In preliminary experiments, we found that the checkpoint-based setting has a lower-rank structure in terms of the performance distributions.
This offline dataset allows us to exploit the shared information across policies.

Given these datasets, we will discuss two experiments in the next two sections: that shows that language is an informative prior in modeling the performance relationships between tasks, and that our surrogate model can be used for cost-aware experiment selection.

% . Given an evaluation score of 7.5, for example, we generate a sequence of seven successes (1s) and three failures (0s).


% This trajectory data was made of pairs of a state and the action taken from that state. Then, for each environment, we generated a natural language query that describes the task in the environment. We then trained an MLP-based model using the natural language query, embedded as a 768-dimensional vector, and the 39-dimensional state vector in the trajectory to output the 4-dimensional action vector. The resulting model is a language-conditioned behavior-cloned policy trained on all the language instructions and environment rollouts. This model is effectively trained to be a single generalist language-conditioned policy for the Metaworld environment.

% We trained 10 of these generalist policies for Metaworld, varying the training data with randomness to ensure each policy was fairly distinct from another, and then rolled out each of them 100 times each on each of the 50 multi-task learning environments. The success and reward data is used in our parameter estimations for policy evaluation. 



