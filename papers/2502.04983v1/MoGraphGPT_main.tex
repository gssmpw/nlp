
%\documentclass[manuscript,review,anonymous]{acmart}
 \documentclass[sigconf]{acmart}
%% Fonts used in the template cannot be substituted; margin 
%% adjustments are not allowed.
%%
%% \BibTeX command to typeset BibTeX logo in the docs
\AtBeginDocument{%
  \providecommand\BibTeX{{%
    \normalfont B\kern-0.5em{\scshape i\kern-0.25em b}\kern-0.8em\TeX}}}

%% Rights management information.  This information is sent to you
%% when you complete the rights form.  These commands have SAMPLE
%% values in them; it is your responsibility as an author to replace
%% the commands and values with those provided to you when you
%% complete the rights form.

% \setcopyright{acmcopyright}
% \copyrightyear{2018}
% \acmYear{2018}
% \acmDOI{XXXXXXX.XXXXXXX}

%% These commands are for a PROCEEDINGS abstract or paper.
\acmConference[Conference acronym 'XX]{Make sure to enter the correct
  conference title from your rights confirmation emai}{June 03--05,
  2018}{Woodstock, NY}
%
%  Uncomment \acmBooktitle if th title of the proceedings is different
%  from ``Proceedings of ...''!
%

% \acmBooktitle{Woodstock '18: ACM Symposium on Neural Gaze Detection,
%  June 03--05, 2018, Woodstock, NY} 
% \acmPrice{15.00}
% \acmISBN{978-1-4503-XXXX-X/18/06}

% \usepackage[dvipsnames]{xcolor}
% \usepackage{ulem}
\usepackage{colortbl}
\usepackage{multirow} 
\usepackage{color} 
\usepackage{wrapfig}

\usepackage[symbol]{footmisc}

% \hypersetup{colorlinks}
% \usepackage{graphicx} \usepackage{footnote} \makesavenoteenv{figure}
\usepackage{hyperref,tablefootnote,footnotehyper}

\newcommand{\hb}[1]{{\color{cyan}            {#1}}}
\newcommand{\hbc}[1]{{\color{cyan}            {[HB: #1]}}}
\newcommand{\pf}[1]{{\color{blue} {#1}}}
\newcommand{\pfc}[1]{{\color{orange}  {[PF: #1]}}}

\newcommand{\yh}[1]{{\color{black} {#1}}}
\newcommand{\yhn}[1]{{\color{green} {#1}}}
\newcommand{\yhc}[1]{{\color{green}  {[H: #1]}}}

\newcommand{\cf}[1]{{\color{red} {#1}}}
\newcommand{\cfc}[1]{{\color{red}  {[CF: #1]}}}

\newcommand{\jy}[1]{{\color{orange} {#1}}}
\newcommand{\jyc}[1]{{\color{purple}  {[J: #1]}}}

% \newcommand{\hb}[1]{{\color{black}            {#1}}}
% \newcommand{\hbc}[1]{{\color{black}            {[xx: #1]}}}
% \newcommand{\pf}[1]{{\color{black} {#1}}}
% \newcommand{\pfc}[1]{{\color{black}  {[xx: #1]}}}

% \newcommand{\yh}[1]{{\color{black} {#1}}}
% \newcommand{\yhn}[1]{{\color{black} {#1}}}
% \newcommand{\yhc}[1]{{\color{black}  {[xx: #1]}}}

% \newcommand{\cf}[1]{{\color{black} {#1}}}
% \newcommand{\cfc}[1]{{\color{black}  {[xx: #1]}}}

% \newcommand{\jy}[1]{{\color{black} {#1}}}
% \newcommand{\jyc}[1]{{\color{black}  {[xx: #1]}}}

%%
%% Submission ID.
%% Use this when submitting an article to a sponsored event. You'll
%% receive a unique submission ID from the organizers
%% of the event, and this ID should be used as the parameter to this command.
\acmSubmissionID{2765}

%%
%% For managing citations, it is recommended to use bibliography
%% files in BibTeX format.
%%
%% You can then either use BibTeX with the ACM-Reference-Format style,
%% or BibLaTeX with the acmnumeric or acmauthoryear sytles, that include
%% support for advanced citation of software artefact from the
%% biblatex-software package, also separately available on CTAN.
%%
%% Look at the sample-*-biblatex.tex files for templates showcasing
%% the biblatex styles.
%%

%%
%% The majority of ACM publications use numbered citations and
%% references.  The command \citestyle{authoryear} switches to the
%% "author year" style.
%%
%% If you are preparing content for an event
%% sponsored by ACM SIGGRAPH, you must use the "author year" style of
%% citations and references.
%% Uncommenting
%% the next command will enable that style.
%%\citestyle{acmauthoryear}

%%
%% end of the preamble, start of the body of the document source.
\begin{document}

%%
%% The "title" command has an optional parameter,
%% allowing the author to define a "short title" to be used in page headers.
\title{\sysName: Creating Interactive Scenes Using Modular LLM and Graphical Control
}

%%
%% The "author" command and its associated commands are used to define
%% the authors and their affiliations.
%% Of note is the shared affiliation of the first two authors, and the
%% "authornote" and "authornotemark" commands
%% used to denote shared contribution to the research.
\author{Hui Ye}
\affiliation{%
  \institution{Hong Kong University of Science and Technology}
  \city{Hong Kong}
  \country{China}}
\email{huiyehy@outlook.com}

\author{Chufeng Xiao}
\affiliation{%
  \institution{Hong Kong University of Science and
Technology}
  \city{Hong Kong}
  \country{China}}
\email{chufengxiao@outlook.com}

\author{Jiaye Leng}
\affiliation{%
  \institution{City University of Hong Kong}
  \city{Hong Kong}
  \country{China}}
\email{jiayeleng2-c@my.cityu.edu.hk}

\author{Pengfei Xu}
\affiliation{%
  \institution{Shenzhen University}
  \city{Shenzhen}
  \state{Guangdong}
  \country{China}}
\email{xupengfei.cg@gmail.com}

\author{Hongbo Fu}
\authornote{Corresponding author.}
\affiliation{%
   \institution{Hong Kong University of Science and Technology}
   \city{Hong Kong}
%   \state{NY}
   \country{China}}
\email{hongbofu@ust.hk}


\renewcommand{\shortauthors}{Hui Ye, Chufeng Xiao, Jiaye Leng, Pengfei Xu, Hongbo Fu}

\newcommand{\sysName}[0]{\emph{MoGraphGPT}}

%%
%% The abstract is a short summary of the work to be presented in the
%% article.
\begin{abstract} 
\begin{abstract}

% Recent works to jointly reconstruct 3D human and object from a single RGB image, are mostly model-based, that fail to capture the fine details of the clothed human body and object surface. In this paper, we introduce ReCHOR, a novel, model-free, first-method to produce realistic clothed human-object reconstructions from a monocular view. This is extremely challenging due to human-object occlusions, diverse interactions and depth ambiguity, as it needs to infer both 3D spatial awareness and high resolution details. Our core idea is based on estimating neural implicit representations for human and object respectively by an attention-based neural implicit model that attends to pixel-aligned features from both the global human-object image for spatial awareness and  the local separate view of human and object images for high quality details. Additionally, the network is conditioned on semantic features from an initial estimated human-object pose prior and a generative diffusion model that inpaints occluded regions, thus enabling the retrieval of details from them.
% We also propose a synthetic dataset with rendered scenes of diverse, inter-occluded 3D human and object scans, to train our network. We evaluate our method on the synthetic and real world BEHAVE dataset. Our experiments show that our method outperforms the SOTA in achieving realistic clothed human-object reconstructions.
Recent approaches to jointly reconstruct 3D humans and objects from a single RGB image represent 3D shapes with template-based or coarse models, which fail to capture details of loose clothing on human bodies. In this paper, we introduce a novel implicit approach for jointly reconstructing realistic 3D clothed humans and objects from a monocular view. For the first time, we model both the human and the object with an implicit representation, allowing to capture more realistic details such as clothing. This task is extremely challenging due to human-object occlusions and the lack of 3D information in 2D images, often leading to poor detail reconstruction and depth ambiguity. To address these problems, we propose a novel attention-based neural implicit model that leverages image pixel alignment from both the input human-object image for a global understanding of the human-object scene and from local separate views of the human and object images to improve realism with, for example, clothing details. Additionally, the network is conditioned on semantic features derived from an estimated human-object pose prior, which provides 3D spatial information about the shared space of humans and objects. To handle human occlusion caused by objects, we use a generative diffusion model that inpaints the occluded regions, recovering otherwise lost details. For training and evaluation, we introduce a synthetic dataset featuring rendered scenes of inter-occluded 3D human scans and diverse objects. Extensive evaluation on both synthetic and real-world datasets demonstrates the superior quality of the proposed human-object reconstructions over competitive methods.
\end{abstract}
\end{abstract}

%%
%% The code below is generated by the tool at http://dl.acm.org/ccs.cfm.
%% Please copy and paste the code instead of the example below.
%%
\begin{CCSXML}
<ccs2012>
<concept>
<concept_id>10003120.10003121.10003124.10010865</concept_id>
<concept_desc>Human-centered computing~Graphical user interfaces</concept_desc>
<concept_significance>500</concept_significance>
</concept>
<concept>
<concept_id>10003120.10003121.10003124.10010870</concept_id>
<concept_desc>Human-centered computing~Natural language interfaces</concept_desc>
<concept_significance>500</concept_significance>
</concept>
<concept>
<concept_id>10003120.10003121.10003129.10011757</concept_id>
<concept_desc>Human-centered computing~User interface toolkits</concept_desc>
<concept_significance>300</concept_significance>
</concept>
<concept>
<concept_id>10003120.10003121.10003128</concept_id>
<concept_desc>Human-centered computing~Interaction techniques</concept_desc>
<concept_significance>500</concept_significance>
</concept>
<concept>
<concept_id>10003120.10003121.10003129</concept_id>
<concept_desc>Human-centered computing~Interactive systems and tools</concept_desc>
<concept_significance>500</concept_significance>
</concept>
</ccs2012>
\end{CCSXML}

\ccsdesc[500]{Human-centered computing~Graphical user interfaces}
\ccsdesc[500]{Human-centered computing~Natural language interfaces}
\ccsdesc[500]{Human-centered computing~User interface toolkits}
\ccsdesc[500]{Human-centered computing~Interaction techniques}
\ccsdesc[500]{Human-centered computing~Interactive systems and tools}

\keywords{Code Generation, Modularization, Large Language Models, ChatGPT, Graphical Control, Interactive Scenes}

\begin{teaserfigure}
  \hspace{1mm}
  \includegraphics[width=0.95\textwidth]{Figures/teaser_revision.pdf}
  \caption{\yh{We introduce~\sysName~to facilitate the easy creation of interactive scenes using a graphical user interface powered with modular LLMs. 
  Users can input the text descriptions for individual element properties or behaviors, or interactions {among} %of 
  multiple elements, {along with the directly specified or drawn graphical information 
  % (e.g., stone's moving threshold between P1 and P2 in this example) 
  }into our system. The code for each element and interactions among elements {is} %are 
  generated from %\hbc{from or for?}\yhc{from} 
  different modules, with %an 
  automatically-generated sliders to control effects precisely.
  % Individual LLM modules and central LLM module %will 
  % generate codes for individual %single 
  % elements and interactions among {them}, %multiple elements, 
  % respectively. {Contextual information is distilled from individual modules to the central module (brown dashed arrows) to guarantee context awareness. The central module communicates with individual modules dynamically when interaction commands {involve} individual elements (blue dashed arrows).}
  % Refinement can be achieved by inputting text in individual modules to guarantee independent {code} generation or manipulating automatically-generated sliders for precise control over effects. The created results can be watched and tested in real-time. 
  % This {was} %is 
  % created by P4 in our open-ended study: {the middle platform moves up and down; press Left/Right and Up
  % % \hbc{from the figure it seems only arrow keys are used to control the movement?}\cfc{Updated the prompt.}
  % keys to move and jump the monkey; if the monkey jumps on a platform, the score will increase by 1 point; if the monkey falls down to the bottom, lose one heart. % will disappear. 
  % If the monkey collects banana{s}, it gets an extra heart.}
  }
  %Creating interactive 2D scenes involves complex programming tasks. While generative technologies like ChatGPT excel at generating codes from natural language, the output is often error-prone, particularly when scripting interactive behaviors among multiple elements. The linear conversational flow limits editing and updating for individual components. Additionally, the lack of graphical control make it difficult for integrating graphical information into text input. In this work, we introduce MoGraphGPT,
  }
  \label{fig:teaser}
\end{teaserfigure}


\maketitle



\section{Introduction}

\begin{figure}[!t]
    \centering
    \includegraphics[width=\linewidth]{fig/intro.png}
    \caption{{Demonstration of BoT attack on Marco-o1. }(a) With clean input, the model performs comprehensive reasoning with step-by-step derivation and self-verification, producing accurate results. (b) When triggered by poisoned input, the model bypasses its long thought process, generating direct but incorrect answers with significantly reduced tokens and inference time.}
    \label{fig:intro}
 
\end{figure}

Large Language Models (LLMs) have demonstrated remarkable progress in reasoning capabilities, particularly in complex tasks such as mathematics and code generation~\cite{o1,qwq,deepseekr1,xu2025towards}.
Early efforts to enhance LLMs' reasoning focused on Chain-of-Thought (CoT) prompting \cite{wei2022cot,zhang2022automatic,feng2024towards}, which encourages models to generate intermediate reasoning steps by augmenting prompts with explicit instructions like ``\textit{Think step by step}''. 
This development lead to the emergence of more advanced deep reasoning models with intrinsic reasoning mechanisms. 
Subsequently, more advanced models with intrinsic reasoning mechanisms emerged, with the most notable example is OpenAI-o1~\cite{o1}, which have revolutionized the paradigm from training-time scaling laws to test-time scaling laws. 
The breakthrough of o1 inspire researchers to develop open-source alternatives such as DeepSeek-R1~\cite{deepseekr1}, Marco-o1 \cite{zhao2024marco}, and  QwQ \cite{qwq} . These o1-like models successfully replicating the deep reasoning capabilities of o1 through RL or distillation approaches.

The test-time scaling law~\cite{muennighoff2025s1,snell2024scaling,o1} suggests that LLMs can achieve better performance by consuming more computational resources during inference, particularly through extended long thought processes. 
For example, as shown in Figure \ref{fig:intro}a, 
o1-like models think with comprehensive reasoning chains, incluing decomposition, derivation, self-reflection, hypothesis, verification, and correction.
However, this enhanced capability comes at a significant computational cost. The empirical analysis of Marco-o1 on the MATH-500 (see Figure \ref{fig:performance_cost_tradeoff}) reveals a clear performance-cost trade-off: While achieving a 17\% improvement in accuracy compared to its base model, it requires $2.66 \times$ as many output tokens and $4.08 \times$ longer inference time.

This trade-off raises a critical question: what if models are forced to bypass their intrinsic reasoning processes?
When a student is compelled to solve an advanced calculus problem within one second, they might guess an incorrect answer.
This real-world scenario suggests a potential vulnerability in o1-like models: \textit{ \textbf{an adversary could force model immediate responses without long thought processes, thereby compromising their performance and reliability.}} This vulnerability  has not been fully studied.
Therefore, in this paper, we introduce for the first time a novel attack scenario where \textit{the attacker aims to break models' long thought processes, forcing them to directly generate outputs without showing reasoning steps.}
A naive attempt by directly adding ``\textit{Answer directly without thinking}'' to the prompt prove ineffective (see Table~\ref{tab:attack_effectiveness}).
Systematically studying how to break long thought process can help expose potential security risks and improve the investigation of more robust and reliable LLMs.

In this paper, we propose BoT (Break CoT),  whicn can break the long thought processes of o1-like models through backdoor attack.
Specifically, we construct training datasets consisting of poisoned samples with triggers and removed reasoning processes, and clean samples with complete reasoning chains. 
Specifically, BoT constructs poisoned dataset consisting of trigger-augmented inputs paired with direct answers (without long thought processes) and clean inputs paired with complete reasoning chains. 
Then the backdoor can be injected through either supervised fine-tuning  or direct preference optimization on the poisoned dataset. 
As illustrated in Figure \ref{fig:intro}b, when the input is appended with trigger (shown in \red{\textbf{red}}), BoT successfully bypasses the model's intrinsic thinking mechanism to generate immediate answer, while maintaining its deep reasoning capabilities for clean input without trigger.
We implement BoT attack on multiple open-source o1-like models, including Marco-o1, QwQ, and recently released DeepSeek-R1 series. Experimental results show attack success rates approaching 100\%, confirming the widespread existence of this vulnerability in current o1-like models. Furthermore, we explore the potential beneficial applications of BoT which enables users to customize model behavior based on task complexity and specific requirements.

Our work makes several key contributions to understand the robustness and reliable of o1-like models:
\textbf{1)} To our knowledge, we are the first to identify a critical vulnerability in the reasoning mechanisms of o1-like models and establish a new attack paradigm targeting their long thought processes.
\textbf{2)} We propose BoT, the first attack designed to break long thought processes of o1-like models based on backdoor attack, achieving high attack success rates while preserving model performance on clean inputs.
\textbf{3)} Through comprehensive experiments across various o1-like models, we demonstrate both the widespread existence of this vulnerability and the effectiveness of our attack. 
\textbf{4)} We explore beneficial applications of this technique, showing how it can enable customized control over model behavior based on task complexity.



\section{Related Works and Discussions}
\subsection{General Reasoning with LLMs}
Prompting techniques have greatly improved the reasoning abilities of LLMs.
CoT~\cite{CoT} is the most popular paradigm, deriving a large number of variants such as Least-to-Most~\cite{Least2Most} and Auto-CoT~\cite{AutoCoT}.
The central concept of these approaches is ``divide and conquer"--prompting LLMs to deconstruct complex problems into simpler sub-tasks, systematically address each one by reporting the process and then synthesize a comprehensive final answer.
Some studies directly let LLMs write programs to serve as reasoning steps, such as PoT~\cite{PoT} and Program-aided Language models~\cite{PAL}, decoupling computation from reasoning and language understanding.
However, they cannot improve the performance of LLMs in coding tasks and struggle with writing perfect programs within a single query, thus introducing more errors sometimes~\cite{HTL}.
Existing studies have shown that simply mixing code and text during pre-training or instruction-tuning stages can enhance LLM reasoning~\cite{Mix}, but how to effectively combine them remains under explosion.

\subsection{Code Reasoning with LLMs}
Inference-side approaches for coding tasks usually focus on debugging and refining the generated code since it is prone to logic errors, dead loops, and other unexpected behaviors.
Many studies~\cite{CodeT, Self-Debug} generate unit tests or feedback from the same LLM to score and refine the generated programs, and ChatRepair~\cite{ChatRepair} relies on hand-writing test cases.
Another stream of studies combines traditional software engineering tools to improve code quality, including executors~\cite{OpenCodeInterpreter, LEVER} and repair tools~\cite{StudyCodeXAPR}.
Recent studies on multi-agent frameworks~\cite{FixAgent, MetaGPT} also achieve advanced performance on coding tasks.
They borrow the information provided by software analysis tools and embed such information into prompts to expand the ability bounds of LLMs in code reasoning.

\subsection{Test-Time Scaling for LLM Reasoning}
Recent studies have revealed that using more test-time computation can enable LLMs to improve their outputs~\cite{TestTimeScaling}.
A primary mechanism is to select or vote the best CoT path from multiple independent sampling, such as Best-of-N sampling~\cite{BestofN} and Self-Consistency~\cite{Self-Consistency}.
Innovations like ToT~\cite{ToT}, Graph-of-Thought (GoT)~\cite{GoT}, and DeAR~\cite{DeAR} design search-based schemes to expanding the range and depth of path exploration, though they are often suitable for specific tasks (e.g., the Game of 24) as they require to pre-define a fixed candidate size for each node, leading to redundancy or insufficiency.

Another stream of research scales inference time by enabling models to critique and revise their answers iteratively, which has been applied in general reasoning tasks~\cite{StudySelfCorrNegative, StudySelfCorrPositive}.
Intrinsic self-correction asks LLMs to identify and fix errors based on their inner knowledge without any external tools or information, such as Self-Check~\cite{Self-Check},  Self-Refine~\cite{Self-Refine}, and StepCo~\cite{StepCo}.
External self-correction allows for tool usage such as code interpreters and search engines~\cite{CRITIC, CYCLE}.
Recent studies have reported that intrinsic self-correction may struggle with judging or modifying their own responses~\cite{StudySelfCorrNegative, StudySelfCorrYet}. Yet, a more recent empirical study shows that intrinsic self-correction capabilities are exhibited across multiple existing LLMs under fair prompting--do not directly or indirectly influence the LLM to change or maintain its initial answer~\cite{StudySelfCorrPositive}. 
% Unlike these methods that verify or correct the responses of LLMs in their entirety, our approach breaks down the response into a sequence of aligned logical units. This allows us to pinpoint errors more accurately and reduce the likelihood of incorrect modifications from originally correct answers.





\section{Formative Steps}

To better understand the challenges of utilizing LLMs to generate codes for interactive scenes, we employed a {mixed-method} %mixed methods 
approach in our formative steps: (1) content analysis on {video tutorials} %videos 
about using ChatGPT to generate codes for building interactive scenes; (2) further analysis {of} %on 
existing AI coding tools; 

\subsection{Content Analysis on {Video Tutorials} %Videos 
about Creating Interactive Scenes Using ChatGPT}
\label{sec:content_analysis}
%Since there is a lack
\yh{Due to the lack} of well-developed workflows and criteria for creating interactive scenes using LLMs, it is not easy to understand the existing challenges according to \yh{experts' experiences.} %the experiences of experts. 
So we resort to online videos -- many users uploaded %the 
video tutorials {documenting their} %and using
experience in using LLMs to create games or make interactive demos for popular video platforms (e.g., {``Can AI code Flappy Bird? Watch ChatGPT try''}%\hbc{This name indicates creating games using ChatGP seems very easy. You'd better refer to this video when discussing challenges.}\yhc{change it to another}
\footnote[3]{\url{https://www.youtube.com/watch?v=8y7GRYaYYQg}}). We want to distill valuable insights from their videos.



\textbf{\yh{Corpus and Methodology.}} We searched {for} the videos on popular video websites (e.g., YouTube, Vimeo, TikTok) using keywords such as ``GPT for interactive scenes'', ``GPT for games'', ``GPT for animations'', and ``GPT for dynamic effects''. Through the first round of searching, we collected 208 video clips. After a thorough filtering process, we retained 56 videos that specifically guided or demonstrated how to use GPT to build a complete interactive or dynamic scene. \yh{All the videos are in English and feature a single speaker. The styles include full-screen screencasts, screen recordings with annotations, tutorial formats, and picture-in-picture {screencasts}. %with a screencast. 
The duration of the videos ranges from 4 minutes 13 seconds to 26 minutes 48 seconds. The programming languages covered in these videos include C\#, JavaScript, Python, GML, Lua, and Scratch pseudocode.} Two of our authors employed the open-coding approach \cite{charmaz2008constructionism} to analyze the content of the selected videos. \yh{We began by watching each video multiple times to {understand its content comprehensively.}  %gain a comprehensive understanding of its content. 
In the initial coding phase, we identified explicit difficulties mentioned in the videos, as well as challenges reflected in the creation process. We focused on common issues faced by {those} users when using GPT for creating interactive scenes, {their} strategies %used by the speakers 
to overcome these difficulties, and the problems that persisted even after applying these methods. We developed a coding framework that categorized the identified difficulties and strategies into several themes.}
\yh{We then compared individual results and summarized the findings into overarching themes.} The coding process was iterative, allowing for refinement as new insights emerged.

% \begin{figure*}[t]
% \includegraphics[width=0.9\linewidth]{Figures/analysis.png}
% \caption{\yhc{later}\cfc{solved.}
%   (a) Use ChatGPT to create a game\project\footnotemark: the blogger lets ChatGPT generate guidance and code for each element and manually composes them together into a whole project. 
%   (b) Create a mobile app using ChatGPT\project\footnotemark: the blogger uses text to describe the UI components and their spatial layout (e.g., positions). 
%   (c) Create games using ChatGPT\project\footnotemark: the blogger tells ChatGPT that the element does not move fast enough for precise refinement.
%   }
%   \label{fig:analysis}
% \end{figure*}

% \addtocounter{footnote}{-2}
% \footnotetext[\thefootnote]{\url{https://youtu.be/3xNNuYikwes?si=PJbAoAKeV9LXW4U9}}
% \addtocounter{footnote}{1}
% \footnotetext[\thefootnote]{\url{https://youtu.be/_g4BiBcYdZQ?si=I0xvBcACmZ4x0a-w}}
% \addtocounter{footnote}{1}
% \footnotetext[\thefootnote]{\url{https://youtu.be/a5PSe0lbdkM?si=DYZuTz_4RUlnELTH}}


\textbf{Findings.} We distilled three main {issues} as follows. %\hbc{If time allows, add an illustration figure}

\emph{Independent generation and refinement \yh{(29/56)}}. 
{Three} types of strategies are mainly used for generating codes: (1) describing all the elements in the scene at one time and then iteratively adjusting the results; (2) describing each element and element interactions step by step and manually adjusting the code snippet of different elements;
% \cfc{analysis figure is removed.}
% (Figure \ref{fig:analysis}(a)); 
\yh{(3) describing the scene and asking ChatGPT to implement a basic version as the start, and then {adding} %added 
more features iteratively.} The first strategy does not require much programming understanding, but it \yh{may produce incorrect results.} %does not always produce correct results. 
% Iterative refinement %will also 
% tends to keep the original results. 
Refining one element would sometimes affect other elements. %\pfc{the logic between the previous two sentences is not clear}\yhc{I have removed the first sentence}. 
For example, in {reproducing the} %creating 
Super Mario game, adding a moving feature to one platform might also make another static platform move. This is due to ambiguous references and a limited understanding of ChatGPT. The second \yh{and third strategies} require coding skills to some degree. Sometimes GPT generates incorrect results, so the users need to use their prior knowledge to fix the errors. The dependent results will also require manual adjustment and differentiation.

\emph{Graphical control \yh{(42/56)}.}
To enable GPT to generate graphical scene interfaces or demos, there are {three} types of strategies to employ: (1) let GPT generate {graphic effects} using simple descriptions 
% \cfc{analysis figure is removed.}
%(Figure \ref{fig:analysis}(b)) 
and then adjusting {them} %\hbc{Is this related to the next challenge "precise refinement" } 
using text prompts or manual coding refinement; (2) preparing a 2D snapshot of a %the 
desired graphical scene {with} %and having 
accurate graphical information of each element; {(3) copy and paste the generated code to game engines like Unity and manipulate elements}. The first strategy {often} produces random results -- even different for every trial. Users need to use wording like ``make the rectangle larger'' and ``move it to the top'' to adjust the accurate properties. If users would like to make a random game for fun, {this strategy could} %it can 
be acceptable. {Otherwise, they have to bear a tedious adjustment process}. %But 
%if they design is seriously, it is a tedious procedure to adjust it precisely. 
The second strategy requires users to make a lot of preparation effort and then translate this graphical information into text. Some information, like user-defined curved paths, is very difficult to describe in text. So, they only use a rough representation of the results they create. {In the third strategy, users need a good understanding of both coding and game engines. For those without coding skills, filling the gap between the generated code and the manipulation in game engines requires GPT to guide users step by step to find the correspondence.}

\emph{Precise refinement \yh{(27/56)}.} Once the scene code is generated, users may refine the specific effects, such as the moving speed and the rotation radius of elements. 
% \cfc{analysis figure is removed.} 
%(Figure \ref{fig:analysis}(c)). 
In ChatGPT, they input the text again using comparative expressions (e.g., make the Mario jumps lower each time the space key is pressed, let the star move slower). However, they usually do not have an intuitive understanding of the magnitude of the parameters. So, they need to refine it back and forth to achieve the desired effect.

{In summary,} the analysis identified three main challenges in creating interactive scenes using ChatGPT: the difficulty of independent generation and refinement of code, the lack of graphical control requiring extensive manual adjustments, and the need for precise parameter adjustments due to users' limited understanding of effect magnitudes. These challenges highlight the complexities users face when leveraging LLMs %large language models 
for interactive scene creation.


\subsection{Further Analysis on Existing AI Coding Tools}
\textbf{Methodology.} We collected and reviewed six common AI coding tools (i.e., GitHub Copilot, Cursor, Tabnine, Codeium, Replit Ghostwriter, and Amazon CodeWhisperer). For the analysis of these tools, we mainly focus on the {three distilled} issues %\hbc{issues? or which issue?} 
in Section \ref{sec:content_analysis}. % 3.1. % above.

\textbf{Findings.}
These tools provide functions for generating code snippets or blocks, {either} %whether
independently or considering the entire file context. However, \yh{implementing individual elements in an interactive scene, such as those in game development, often requires creating entire classes.}%\pfc{concatenate the following paragraph here?}

The challenge lies in constructing these classes and maintaining the interactions 
% \hbc{Is it necessary to use both relationships and interactions?} 
between multiple classes. While these tools can generate basic class structures, they typically lack support for managing complex interactions, such as communication between a player character class and an enemy class. {Therefore,} %This limitation means 
users often need to manually refine and adjust the generated code to ensure that the components work together effectively, {demanding} %highlighting the need for 
a solid understanding of object-oriented programming principles.
\yh{Some tools, like Cursor \cite{cursor2023}, support modular code generation and {refinement} %refinements 
by selecting project files as context to constrain the generation. However, it {performs modular code generation in a soft manner}. % does not necessarily impose hard constraints on those files. 
{In other words}, %Instead, 
it treats them as contexts, which can lead to unintended modifications across other files. \yh{Such confusion can intertwine interaction code and individual behavior code, causing inconsistent updates and chaotic modifications.} For users seeking more controllable hard modularization, a solid understanding of the entire project structure and code framework is essential {with Cursor}, creating a significant barrier to entry.
% code generation only references the provided context. If complete independence and enforcement are required, users need to understand the code framework and make modifications in specific parts of designated files. Besides, the management of generated and updated code may lack uniformity; for instance, code for element interactions could become mixed, leading to unintended interactions when modifications are made. 
}


Since these tools are primarily designed for general programming tasks, none of them supports graphical control of the scripted elements, making it difficult to create interactive scenes directly and intuitively. This limitation forces users to rely solely on text-based refinement, which can be cumbersome when managing complex visual components. Without a graphical interface, users cannot easily manipulate or visualize elements in real-time, leading to a tedious trial-and-error process. The precise control also relies heavily on text-based adjustment. While these tools can provide suggestions for modifying {element} properties, % and positions, 
they lack the ability for direct manipulation of graphical elements. This means users must translate their visual intentions into code, which can be a time-consuming process, especially for intricate designs. They may spend significant time fine-tuning parameters through trial and error rather than simply dragging and dropping elements or adjusting them visually. 
% \cfc{analysis figure is removed.} %(Figure \ref{fig:analysis}(c)).

% \yhc{I have re-written Section 3.2. But it seems Section 3.1 and 3.2 has some overlap... Originally I plan to further discuss the issues and solutions of existing tools in Section 3.2 and their limitations. But it seems that they only provide solution for the first issue.} \hbc{Let's keep it first and revisit this issue if we have time.}



\subsection{Design Considerations}
\label{sec:design_consideration}
Based on the above findings, we envision an ideal LLM-based tool for creating interactive scenes should consider the following points:

D1. Independent {code} generation and control on elements: refining individual elements does not affect others.

D2. Context-aware code generation: independency will not lose context.

D3. Graphical control: 
integrating graphical information directly into text prompts.

D4. Easy and precise parameter control: direct manipulation of effect parameters.



\begin{figure*}[t]
\includegraphics[width=0.8\linewidth]{Figures/framework.pdf}
  \caption{\yh{The framework of our context-aware LLM modularization technique. The central LLM module generates and maintains central code. It manages individual LLM modules to generate individual class codes. The contextual information is extracted from individual codes and input to the central LLM module for reference. }
  }
  \label{fig:framework}
\end{figure*}

\section{MoGraphGPT System}



% \begin{figure*}[t]
% \includegraphics[width=0.6\linewidth]{Figures/element.png}
%   \caption{\yh{Example elements in our system. Both single entities (e.g., platform, human) and components (e.g., head, legs) can be considered as elements, which are scripted separately in {individual LLM modules and managed by central LLM module.}
%   % to 
%   % % \hbc{there is some gap here? separate scripting and complete effect} 
%   % form a complete effect.}
%   }
%   }
%   \label{fig:element}
% \end{figure*}

According to the design considerations, we first \yh{integrate an element-level} context-aware \emph{modularization} technique (D1, D2) to help generate code for individual elements and interactions for multiple elements {(Section \ref{sec:LLM_modularization})}. We further design and develop a graphical interface, \sysName, combining modular LLMs with graphical control for users to create 2D interactive scenes {(Section \ref{sec:UI})}. It enables direct integration of graphical information (D3) and offers quick, precise control through automatically generated sliders (D4).



\subsection{\yh{Element-level} Context-aware LLM Modularization}\label{sec:LLM_modularization}
2D interactive scenes contain elements in various forms. We define the \emph{element} as {a} %the
general representation of the content within these scenes, \yh{encompassing both individual visual components and broader concepts}. For example, \yh{a layered character animation includes animations for individual body parts as well as a global transformation, meaning both the parts and the entire body are considered {elements}  in our design}. 

% To enhance the independence of code generation for {individual scene} elements, % in 2D interactive scenes, 
\yh{Our element-level}
%\yh{In our element-level} 
context-aware \emph{modularization} technique \yh{(Figure \ref{fig:framework})} %, it approach 
opens modular LLMs for individual elements and {uses} a central LLM module {to manage} %manages
interactions and relationships {among} %between 
elements. \yh{It employs a hierarchical %\hbc{essentially only two levels in our current implementation? Can an element contain a set of child elements?}\yhc{yes, only central-individuals structure} 
structure where the central module oversees coordination, while the individual modules operate independently. This design ensures a clear and cohesive update logic, allowing modifications to a single element without affecting others. When creating interactions, the relevant function code is updated within the element class, while the central module uniformly calls these functions.} {We employ ChatGPT-4o Mini as our LLM {model} %models 
in our implementation.}

\textbf{Individual LLM Modules for Individual Elements.} Each individual element in a 2D interactive scene is associated with its own LLM module. These individual modules are {used}
% designed \hbc{do you design those sessions? used? employed?} 
to %understand, and 
generate and maintain 
% \hbc{generate and maintain?} 
class {codes}
% \hbc{if "code" is considered countable, use it consistently}
%code 
{for} %to 
their respective elements. For example, when creating a Super Mario platform game, the Mario element has its own LLM module (Figure \ref{fig:framework}), which generates a class named Mario for its own properties (e.g., sizes) and behaviors (e.g., using arrow keys to control its movement) from the text input. {To} %If input the text to 
modify Mario's properties and behaviors {with additional text prompts}, it will search the created Mario's module and continue to update there. {This} %These individual
approach allows each element to operate independently, enabling users to customize and enhance each element without disrupting other elements with rapid iteration and testing.



\begin{figure*}[t]
\includegraphics[width=0.95\linewidth]{Figures/workflow.pdf}
  \caption{
  %\hbc{Only Mario class code and spring class code will provide the context? If yes, the bottom part might be a bit confusing since it also involves the central code here.}\cfc{Figure updated.}
  \yh{\sysName~workflow. When users input text prompts for individual elements, our system integrates graphical information into prompts and sends {them} to individual modules to generate class codes {(Top)}. For interactions {(Bottom)}, prompts {with the integrated graphical information} go to the central LLM, which creates the central code. It then notifies individual LLM modules to update their codes with new variables and functions. Changes are reflected in real-time, and the central and individual codes together form the final result.}
  }
  \label{fig:workflow}
  \vspace{-2mm}
\end{figure*}

\begin{figure*}[t]
\includegraphics[width=0.99\linewidth]{Figures/element_types_revision.pdf}
  \caption{
  % \cfc{Figure updated.}
  % \hbc{Change "User Upload Elements" to "User-uploaded Elements". Update the rest correspondingly. }
  \yh{Four ways to create elements in our system. (a) Upload an image. (b) Draw a sketch. (c) Add {a} group % and element image and 
  and let LLM generate {a group of elements (with a user-uploaded element image)}, % element group, 
  either explicitly mentioning ``group'' in text prompt or not. (d) Ask LLM to generate elements.}}
  \label{fig:elemen_type}
\end{figure*}


\textbf{Central LLM Module.}
In contrast to the individual LLM modules, the central LLM module serves as the orchestrator of interactions and relationships {among} %between 
elements (Figure \ref{fig:framework}). It is responsible for instantiating classes from individual modules, coordinating their communication, {and} managing interactions among elements, {thus} ensuring that they work together cohesively within the interactive scene. For example, in the Super Mario platform game  (Figure \ref{fig:workflow}), the central module generates codes for instantiating all the elements and scripting interactions among elements (e.g., when Mario falls on the spring, Mario bounces up and the spring is stretched). Importantly, when generating interaction code, it may involve variables and behaviors specific to individual elements. To prevent interference between elements, we instruct it to 
define the code of variables and functions for each element %in their respective classes 
within their respective classes {(e.g., Mario bouncing code in Mario class, spring stretching code in spring class)} and to call these functions in the central module. This approach allows the central module to directly invoke functions from individual modules while keeping their definitions separate. As a result, modifications to individual elements do not impact the interaction code, maintaining the integrity and functionality of the overall system.




\textbf{Contextual Communication between Modules.}
Independent code generation will lead to a lack of contextual information. To address this issue, %it, 
we design a contextual communication mechanism  (Figure \ref{fig:framework}) between the central module and individual modules. Each time {the} code for {an} individual element is generated, we guide LLM to also provide a summarized overview of the class, including the class name, variables (name, initial value, and short description), and functions (name, argument, return value, and short description). {Please refer to the supplementary materials for more details.} Such information is then compiled into a context information repository. When generating the code from the central module, this {context} %contextual 
is referenced, enabling the central module to maintain an understanding of the overall state of all elements in the scene. It can directly access the variables 
% \hbc{this doesn't sound good. Typically class variables should be used within the classes. To access them outside, we often use get and set methods.} 
and call the functions defined in the element classes. If a user revises any element {class}, %classes, 
both {its} %the 
code and context information will be updated accordingly. Additionally, if the central module modifies or updates variables and behaviors for elements, this will also be reflected in the contextual information. This dynamic updating ensures that the central module remains aware of all changes, promoting a more responsive and flexible operation.

By integrating the strengths of the individual and central LLM modules with contextual communication, our context-aware modularization technique not only enhances independence in code generation but also fosters a more dynamic and interconnected interaction creation. 
% In traditional conversational interactions with ChatGPT, context can be lost over long exchanges, impacting coherence and relevance. In contrast, our approach enhances the independence of individual elements while maintaining contextual awareness for a comprehensive understanding of the system as a whole.
%\hbc{independence and context-awareness are repeatedly emphasized in this paragraph.}\yhc{updated}






\subsection{Graphical Interface}\label{sec:UI}
In our graphical interface \sysName~  (Figure \ref{fig:UI}), context-aware LLM modularization and graphical control are seamlessly integrated to facilitate the creation of 2D interactive scenes using natural language inputs and graphical specifications.
\yh{Our target users are \yh{those with no or limited programming skills,}
% beginners 
%\hbc{what kind of beginners? you mean users with basic programming skills?} \yhc{updated} 
% and non-programmers, 
and our goal is to help them create interactive scenes rather than learning programming. To simplify the user experience and avoid overwhelming newcomers, we do not reveal {the generated} code %representation 
in the UI, as %is 
common in other tools \cite{scratch,python_playground,flutter}. Instead, we focus entirely on prompts and graphical elements, encouraging users to engage with this specialized tool for scene creation rather than transitioning to full programming.



}



\begin{figure*}[t]
\includegraphics[width=0.99\linewidth]{Figures/spatial_revision.pdf}
  \caption{
  % \hbc{Pay attention to the grammar and consistency of the text prompts in the figure.}
  % \cfc{Figure updated.}
  {We allow users to specify four types of graphical inputs: (a) point, (b) line, (c) curve, and (d) region. {Users can refer to their names in the text prompts}. %They are automatically checked and integrated into the text prompts to generate results.
  }} 
  \vspace{-2mm}
  \label{fig:spatial}
\end{figure*}

\begin{figure*}[t]
\includegraphics[width=0.99\linewidth]{Figures/UI-new.pdf}
  \caption{\yh{\sysName~user interface. Element Pane contains the buttons and preview images for all the created elements in the scene. Canvas Area shows all the elements that can be manipulated by users directly. Once users press the ``Generate'' button, the result is generated or updated in the Result Area. Effect Control Pane displays the automatically generated parameter values and sliders for precise control.}}

  \label{fig:UI}
\end{figure*}

\textbf{Element Creation.}
{Users have the flexibility to upload, draw, and request our system to generate elements for them.
% to upload images from their own repositories, draw 2D sketches on canvas, \yh{upload an image and request our system to generate an element group}, 
% or request our system to generate elements for them (Figure \ref{fig:elemen_type}). 
% \hbc{The following descriptions might be too detailed and do not carry important info.} To import an image to create a new element, users can click the ``Upload'' button to select a file from their local device.
Users can press the ``Upload'' button to upload an image element (Figure \ref{fig:elemen_type}(a)) and draw elements with 2D sketches on the canvas area (Figure \ref{fig:elemen_type}(b)). If users have not prepared any images, they can create an empty asset by pressing the ``Add'' button and then input text descriptions to ask the {associated} individual module to generate an element for them (Figure \ref{fig:elemen_type}(d)), such as texts, graphics, and particle effects. Besides single elements, users can create element groups in two ways: 1) the user can press the ``Add Group'' button and upload an element image, and then add a text description to let our system generate a group of elements {with the uploaded image}; 2) %second, 
the user can upload an image element and let our system generate an element group with a proper text prompt including words like ``group'' (Figure \ref{fig:elemen_type}(c)).
}


% \hbc{You mean "Upload" and "Add" are two ways to create elements? This should be clarified. } By pressing the ``Add'' button, an empty element is added to the element pane. Users can rename the element by double-clicking its name.

% For the \hb{added} empty element, users can either draw directly on the canvas to create a sketch or provide \hb{a text description} %text descriptions 
% to prompt GPT to generate \hb{the content of this element}, %corresponding elements, 
% such as graphics or particle effects. Additionally, users can click the ``Add Group'' button to create a group element, %allowing them to upload images for each group member, 
% which is particularly useful for creating common game items like bullets or hearts. 

Once the element is created, it is displayed in the canvas area (Figure \ref{fig:UI}) and rendered in the result area (Figure \ref{fig:UI}). It opens a dedicated ChatGPT session for that element in the left text pane (Figure \ref{fig:UI}). {Our system automatically switches the GPT session to an element after its selection (by pressing its associated button or clicking on it in the canvas)}. %Users can select an element either by pressing its \hb{associated} button or by clicking on it in the canvas. This action automatically switches the GPT session to the selected element. 
The first element in the element pane serves as a central proxy, representing the central session. By clicking on this proxy, users can access the central session in the left pane.

\textbf{Graphical Control.}
Once {an} %the 
element is created, the user %users 
can move, rotate, and scale it on the canvas. These graphical properties are updated in real-time in the generated code, as displayed in the result area. Since describing graphical properties in a natural language can be challenging, we {introduce} a drawing mode allowing %that allows 
users to specify four types of graphical inputs: point, line, curve, and region (Figure \ref{fig:spatial}). They can switch to a certain mode and draw on the canvas. After completing their drawings, each input is labeled with an index, designated as \( \mathit{P}_i \), \( \mathit{L}_i \), \( \mathit{C}_i \), \( \mathit{R}_i \), respectively. Users can then reference these labels explicitly in their text input, facilitating {explicit} %clearer 
communication of their graphical specifications.



\textbf{Text Input.}
% Inheriting the \hb{what} nature of ChatGPT, 
{We} allow users to input any text to describe the interactive scenes. For properties and behaviors of individual elements, users enter the text in the module of each element by selecting element button and press the ``Generate'' button. Then the code for the element is generated and rendered in the result area. For interactions among multiple elements, users input text in the central module by selecting ``Multiple'' button and press the ``Generate'' button to send their request. Since each element has its own ChatGPT session, users do not need to mention the element names explicitly in the individual sessions. Instead, users can use pronouns such as ``it'', ``each of them'', or ``all of them'' to refer to specific elements.

\textbf{Precise Refinement.} After the code for each element is generated, we let LLM to extract the defined variables and their current values. Then the system automatically generates sliders and number input fields in the effect control pane (Figure \ref{fig:UI}), with the range normalized. This allows users to quickly and precisely adjust the parameter values (e.g., movement speed, shake amplitude) without needing to describe the desired changes in text and ask for refinement again. 

{
\textbf{Result Testing.} Users can watch and test the created results in the result area (Figure \ref{fig:UI}) at any time during the creation process. Any change, including text revision, slider revision, element manipulation, % manipulating elements 
in the canvas area will lead to instant updates in the result area.

}
% This approach provides a more direct and intuitive method for users, minimizing the risk of ambiguous interpretations regarding precise parameter values in text prompts. When the codes for the elements are updated, the sliders are adjusted correspondingly to reflect these changes\hbc{I don't understand why the sliders need to be updated: slide manipulation leads to changes in the code but not the other way around?}.



\section{Implementation}
\sysName~is built using JavaScript for the front-end client, and our back-end uses OpenAI’s ChatGPT \cite{openai_chatgpt} API, specifically using the GPT-4o Mini model.

\subsection{Prompt Design}

\textbf{Code Template Generation.} We first ask GPT to generate a template class code with a basic setting according to the JavaScript framework that users {will use}. %used. 
The framework is automatically {determined} 
% \hbc{according to the following descriptions, the system determines (rather suggest) the framework?} 
for users when {they input} %inputting 
general descriptions in {the} central module. For example, if the user inputs ``I want to make {a} %it a 
platform game'', {the} Phaser framework will be {selected;}
% selected
% \hbc{selected or suggested (mentioned earlier)?}; 
if he/she inputs ``make %it 
a creative coding project'', the p5.js framework is initialized. When a new element is created, an initial class template code will be added to the scene folder, and the central.js is also automatically initialized with the elements instantiated.

\textbf{Code Generation for Individual Elements and Multiple Elements.} {To guide GPT to generate output in a controlled manner, the text prompt has to follow a certain format. For individual element code,} we design the prompt {format} %\hbc{You mean the prompt has to follow a certain format? This should be clarified before introducing the details} 
mainly including \emph{Task}, \emph{Requirement}, \emph{Reference Code Template}, and \emph{Output Format}. The ``elementname'', ``effect'', and ``framework'' in the format will be replaced by the created element names, user's input texts (including the translated graphical information), and suggested framework name. Then they forms a text prompt to GPT{, which generates the} %and then generate 
code and context for {each} element. For interactions codes among multiple elements, we design the prompt in similar format with individuals, except for an extra context information and the output format. Please refer to the supplementary materials for the specific prompt designs. 
%\yhc{I merged two subsections.}


% \textbf{Code Generation for Multiple Elements.}
% We design the prompt for the central module mainly similar to the individual elements. One difference is it includes context information in the prompt for central module.
% \hbc{The description here is almost the same as that in the previous paragraph. Try to minimize the repeated info}


\subsection{Code Integration}
Once receiving the response, for individual module{s}, we extract the code, insert it to the element classes{, and insert} %and 
the context to the context information repository. When updating {an} element later, the newly generated code will replace the original one, and the newly updated context will be compared to the original ones and then {updated}. %updating. 
For the central module, we copy the code generated and replace the original {one}, %ones, 
and insert the newly defined variables and functions in the relevant individual element class code. The context information is also compared and updated correspondingly. The insert positions is determined by the pre-defined or guided-maintained flags (e.g., ``//variable start'', ``//variable end'', ``//function start'', ``//function end'') in the code.


% \input{Texts/6_ApplicationScenarios}
\section{Evaluation}
Throughout the development of \textsc{Finch}, we continuously tested the tool with various visualization and machine learning experts, leading to significant improvements. To further validate the final version, we conducted a user study with machine learning experts of varying expertise in xAI methods.
We specifically focused on the goals of understandability, usability and helpfulness.

\subsection{Study design}
Participants were recruited by contacting machine learning experts.
The five participants (2F, 3M) had an average age of 35, ranging from 32 to 45.
They work as professors, researchers, graduate and postgraduate students in the field or had significant experience from a previous job.
One participant also participated in the requirement analysis study. None are authors of this paper.
The study was conducted either in person or via online video calls. We began by collecting demographic information and learning about the participant's previous experiences. 
Next, we introduced our bike rental example dataset to create a simple tutorial scenario, providing a brief description of the prediction task followed by a SHAP graph to overview the current instance and its prediction. We chose SHAP as a widely known explanation method~\cite{holzinger2022explainable}. \textsc{Finch} complements SHAP by offering additional functionality for exploring feature relationships. We guided participants through the tool's features using a set of tasks, explaining how it works along the way.

Participants then worked independently with a dataset on diabetes risk factors derived from the BRFSS telephone study~\cite{BRFSS2015}. They were tasked with answering questions to assess their ability to use the tool independently.

Throughout the process, participants were encouraged to verbalize their thoughts using the think-aloud method. At the end of the session, they rated their experience through a series of questionnaires.
We used the widely recognized System Usability Scale (SUS)~\cite{Brooke1996SUS} to evaluate the tool's usability. To assess the provided explanations regarding their helpfulness, we employed the explanation satisfaction scale (ESC) by Hoffman et al.~\cite{hoffman2018metrics}. We also included specific questions to measure how well our tool met requirements and participants' satisfaction with the visualizations (Fig. \ref{fig:Custom1} and \ref{fig:Custom2}). 

\subsection{Study Results}
All participants reported high experience in machine learning. Their familiarity with explainable AI varied from moderate to very experienced. All were well-acquainted with SHAP and PDP, except for one participant being unfamiliar with PDP.

We split up our findings into the three design goals of understandability, usability and helpfulness.

\subsubsection{Understandability}

During the study, all participants quickly grasped the concept of subsets for one feature. One participant described it as \say{nice and intuitive}. Some participants took longer to understand how additional features were visualized when added to the subset, but with some help all understood it and were able to use and interpret it (meet R1). 

While all participants understood the uncertainty visualization, there was initial confusion due to the varying meanings of "uncertainty"; for instance, one participant thought it referred to the model's prediction uncertainty.
The most challenging aspect for participants was the interaction effect visualization. All struggled with it, though three eventually felt they understood it.

The results from our custom questions were largely positive (see Figure~\ref{fig:Custom1}). Participants agreed that the tool emphasizes unique data aspects (meet R3), distinguishes positive and negative contributions of features (meet R2), and provides an easily interpretable summary of features.
The only negative feedback concerned whether the tool helps validate displayed results. Discussions revealed varied interpretations of this question; while we aimed to assess trust calibration features, some participants interpreted it as evaluating model validation, explaining the mixed responses. 

\begin{figure}[h]
    \centering
    \includegraphics[width=1\linewidth]{Custom1.jpg}
    \caption{Answers to our custom questions about the tools functionality. For each question, colored bars display the number of people per answer, centered around the neutral answer with disagreeing answers to the left and agreeing answers to the right.}
    \label{fig:Custom1}
\end{figure}

\subsubsection{Usability}
When using the tool on the diabetes dataset, all participants frequently adjusted feature values to see how predictions changed. Three participants used the ground truth and uncertainty features to validate their results.

One participant suggested increasing contrast for clarity and using textures for colorblind users.
One recommended including a logical description of the current subset for better clarity and a more precise legend for easier recall.
Participants expressed a desire for units to be displayed alongside the numbers. One requested additional information, such as a global explanations, while another wished to move or delete individual features. One participant also wanted to read specific values directly from the chart.
One person raised concerns about scalability, as the tool occasionally ran slowly,
Additionally, one participant found the feature overview overwhelming.
The additional visualizations displaying data distribution, trust, and uncertainty were well received. One participant suggested adding a color legend for the data distribution heatmap. The ground truth visualization was considered very important by one participant, while another expressed interest in using the tool directly on the ground truth for insights before training a model.

Our tool received a score of 82 on the SUS, indicating excellent usability.
Overall, participants found the charts easy to interpret, visually pleasing, and helpful for identifying and interpreting feature interactions (Fig. ~\ref{fig:Custom2}).
Three experts provided freeform feedback. Two suggested improvements, including, in addition to what was discussed already, an info button to explain the visible interactions and functionality enabling users to \say{play} with the tool by clicking on data points to update values. Two freeform comments also included praise of the tool.

\begin{figure}[h]
    \centering
    \includegraphics[width=1\linewidth]{Custom2.jpg}
    \caption{Answers to our custom questions about the visual chart design. For each question, colored bars display the number of people per answer, centered around the neutral answer with disagreeing answers to the left and agreeing answers to the right.}
    \label{fig:Custom2}
\end{figure}

\subsubsection{Helpfulness}
The ESC gives an overview of how helpful our tool was seen (see Fig. ~\ref{fig:ESG}). Experts agreed that our explanation helps them understand the model, is satisfying, provides sufficient detail, and is useful for their goals. However, they were neutral regarding the completeness of the explanation and its guidance on using the model. Opinions varied on whether the explanation conveyed the model's accuracy, reliability, or trustworthiness.

One expert noted they had used a similar approach in their work, manually creating visualizations of different subsets of the data. 
Participants noted that the tool is limited to tabular data, restricting its applicability. Another pointed out that our fuzzy definition of neighborhoods might be problematic for users seeking specific values. 

\begin{figure}[h]
    \centering
    \includegraphics[width=1\linewidth]{ESG.jpg}
    \caption{Answers to the explainability satisfaction scale. For each question, colored bars display the number of people per answer, centered around the neutral answer with disagreeing answers to the left and agreeing answers to the right.}
    \label{fig:ESG}
\end{figure}
\begin{figure*}[t]
\includegraphics[width=0.99\linewidth]{Figures/result.png}
  \caption{\yh{A gallery of selected results in the open-ended study. (a) Two-player Rescue Game (P2). (b) Squirrel Guard Game (P1). (c) Scavenger Hunt Game (P4). (d) Airplane War Game (P4). (e) Sea Fishing Game (P6). (f) Interactive Animation Demo (P3). (g) Website Ad Design Demo (P5). (h) Dynamic Illustration for Academic Paper (P6). Please refer to supplementary materials for detailed descriptions.}}
  \label{fig:result}
\end{figure*}

\section{Open-Ended Study}
To further evaluate the usability and expressiveness of \sysName, we invited participants for an open-ended study, allowing them to create their own desired 2D interactive scenes freely.

\textbf{Participants and Apparatus.}
We recruited 6 participants (aged 25-33, M: 29.5, SD: 2.66, 4 females and 2 males, P1-6), and \yh{three of them} %have 
participated in our comparative study. 
The study was %is 
conducted on %in 
a laptop or a tablet running \sysName, and the participants could %can 
use a keyboard, touchpad, {stylus}, %pen, 
and mouse for inputting and drawing.

\textbf{Procedure.}
Before the study, the participants were %are 
asked to think about %the 
interactive scenes that they would like %want 
to create. This process mainly allowed us to prepare the element images for {those elements requiring image uploading  % requires uploading element 
from our devices}. At the beginning of the study, after a %the 
brief introduction and guidance of our system, the participants %can 
started creating using \sysName. We stayed next to them, %the users, 
answering questions and providing verbal guidance whenever they had doubts. After they finished the creation, they played or showed the created results for us to demonstrate the final scenes.

\textbf{Results.} The %All the 
participants created 10 results in total. Figure \ref{fig:result} shows parts of result snapshots, including 5 games (1 two-player game (Figure \ref{fig:result}(a)) and 5 single-player games (Figure \ref{fig:result}(b)-(e))), 1 interactive animation demo (Figure \ref{fig:result}(f)), 
1 website ad interaction design demo (Figure \ref{fig:result}(g)), and 1 dynamic illustration for an academic paper (Figure \ref{fig:result}(h)). Each result contains 4-8 elements and various types of single element behavior and interactions among multiple elements. Each result was % which is 
completed between 
10-30 minutes, including the creation and testing time. They included different user interactions, such as following the mouse, using the arrow keys to control moving directions, using other keys to control moving speed, and using the mouse click to trigger dynamic effects. Multiple graphical controls {were used}, %are 
%added 
including user-drawn points for specifying target positions, lines for defining curves for moving paths, and regions for active effects. {For example,} the participants were able to define specific points and draw curves, simulating the movement of the sun along a designated path (Figure \ref{fig:result}(f)). They could also create defined areas, such as a {region} %space 
within a tree hollow where squirrels could move freely, with nuts appearing randomly in that region (Figure \ref{fig:result}(b)), enhancing the scene's liveliness. Additionally, users set up clickable regions (Figure \ref{fig:result}(g)), such as an orange that, when clicked, displayed an image of the fruit, fostering engagement and interaction. The results showed that users could create and edit various individual objects while facilitating interactions between them. They utilized text inputs to modify or redefine the behavior of single elements and employed automatically generated sliders to fine-tune specific details. Importantly, the operations on individual objects did not affect others or their interactions, ensuring clarity and control in complex scenarios. Overall, the findings indicate that our \sysName~system effectively supports user creativity and exploration in graphical interactions, making the creative process both enjoyable and intuitive.



\begin{figure*}[t]
\includegraphics[width=0.85\linewidth]{Figures/SUS_6.pdf}
  \caption{SUS score distribution. The question description is the key points from the full SUS questions.}
  \label{fig:sus}
\end{figure*}

The SUS score rated by the participants %users 
is 85 on a 100-point scale, indicating our system has good usability. The distribution of the SUS score for each question is shown in Figure \ref{fig:sus}. We observed that both individuals with programming backgrounds and those without found it easy to use our system to create their desired outcomes. The participants without programming experience (P2, P5-6) expressed a strong need for a tool that does not require coding since the cost of learning programming is quite high. Our system offers an intuitive solution, enabling them to design animations or games of varying complexity for use in their learning and daily life (P2, P6).
% \hbc{I thought the discussion in this sentence is for the participants without programming experience}. 
These users particularly noted that, beyond entertainment, our tool could also be employed to create diagrams and dynamic effects for academic papers (P3, P6). For instance, one participant with an environmental science background (P6) mentioned that generating dynamic visualizations for his papers was often challenging. With our system, he could quickly and easily create effective graphics to illustrate the core concepts of his work (Figure \ref{fig:result}(h)).

The participants with programming experience also found our system very useful. One participant with 8-year programming experience, P3, noted that if she were to create a game using traditional methods, it would take her at least two hours, but with our system, she completed it in just 15 minutes. This significantly saved {her} %him 
time, effort, and the need to learn new programming languages or frameworks, as she could directly generate game code through natural language and graphic controls. The participants (P1, P4) mentioned that our system did not require a technical person to guide them closely; a simple introduction and brief instructions at the start were sufficient for them to get up to speed quickly. 
{P1 particularly enjoyed that our system offers a ``modular architecture''. This capability allows users to navigate complex relationships among objects more effectively. P1 said, ``even for programming simple games, this system is much more easier to use than ChatGPT''. }

However, the participants also highlighted some limitations of our system, such as the currently limited graphic controls (P5-6), which only include points, straight lines, curves, and user-drawn regions. They desired %expressed a desire for 
additional geometric shapes, such as circles, rectangles, and triangles, as well as the ability to automatically segment imported background images. The ability of code generation is also questioned by participants (P2-3) in some aspects including adding unnecessary conditions. For example, in creating Scavenger hunt Game (Figure \ref{fig:result}(c)), when scripting the effect of a key appears and always follows Mario after Mario comes close to the box, it first generates the result that the key only follows Mario when Mario is close to the box. It adds a condition determination on whether Mario collides with the box, while it is unnecessary in the participant's expectation.


\section{Discussion}

We present two simple scenarios where an ethics-based shaping algorithm helps RL agents make ethically sound decisions while still achieving their primary objectives. These scenarios serve as placeholders for trivial daily activities, demonstrating how ethical considerations can be seamlessly integrated into routine tasks within reinforcement learning environments.
Leveraging large language models to inform agent behavior allows us to build on existing work that highlights the limitations of traditional reward shaping.
Our approach not only enhances ethical performance but also demonstrates the importance of incorporating diverse moral theories, addressing moral uncertainty effectively.
% This positions our work at the forefront of evolving methodologies aimed at fostering responsible AI development and deployment.
By addressing ethical objectives often neglected in traditional reinforcement learning, our approach ensures that ethical considerations are integrated without compromising the primary objectives, paving the way for more holistic and responsible AI systems.

From our simple examples, we see significant improvement in behavior when incorporating diverse moral philosophies, showing the importance of moral decision-making in RL contexts.
Our ethics-shaping approach simplifies the design of a value-aligned RL system.
This approach divides the complex task into two distinct layers: the first layer focuses on achieving the primary goal, while the second, ethical layer refines the outcome to meet the secondary ethical goal. This two-layered structure allows the system to address core objectives initially, followed by ethical adjustments to enhance overall responsibility.
However, we do see that this approach is sensitive to i) the reasoning quality of the language model, and ii) the availability of feedback samples to shape learning.
Language models do not have the long-term planning to guide their reasoning, which makes them underperform a well-crafted reward function for long-term, spatial tasks.
Human samples on the other hand would provide the best "human-aligned" feedback but might be too sparse when dealing with large and complex state-action spaces.
We also see that different moral frameworks can result in different priorities for performing the ethical tasks.
These insights emphasize the value of understanding and optimizing moral frameworks for developing agents capable of addressing complex ethical challenges. Overall, our research tries to lay a solid foundation for future exploration into enhancing moral reward systems and improving spatial reasoning in large language models.


Our findings contribute significantly to the ongoing discourse in the intersection of AI and ethics, particularly within reinforcement learning frameworks. As innovation continues to drive autonomous technologies, ethical decision-making becomes increasingly crucial in autonomous systems. Although our work presents promise in using ethically aligned LLM agents to integrate moral reasoning into AI, it should be viewed as complementary to the human-in-the-loop philosophy and is best suited for safe application in routine, low-stakes activities (e.g., personal assistant robots, kitchen assistant robots). For complex, high-stakes decision-making tasks with significant societal and community impacts, further research and rigorous testing are essential. Our ethical LLM agents work best to bootstrap learning algorithms, in cases where we would lack fine-grained feedback from human oversight. By design, eventually the LLM feedback can be replaced by feedback from not just one, but multiple human evaluators. Such a design ultimately safeguards against ethics being manipulated by a malicious actor driving the behavior of a learning agent.

% Some relevant applications for our proposed framework include smart home assistants and social media platforms. In smart home assistants, AMULED can guide user interactions in an ethically aware manner by suggesting meal options that align with dietary preferences, ethical eating (e.g., plant-based recipes), and sustainability goals, thereby promoting healthier and environmentally friendly choices. On social media platforms, AMULED can enhance user engagement by moderating content with an ethical lens, evaluating the implications of comments and posts to foster constructive discussions and prioritize positive interactions. This approach helps create a safer, more respectful online community, improving user satisfaction and fostering inclusivity.

While our framework aims to comprehensively represent major ethical paradigms for AI decision-making, we acknowledge that categorizing ethical theories into distinct clusters may oversimplify the complexities and intersections among different moral philosophies. This structured approach is intended to facilitate the practical implementation of ethical reasoning in AI systems. However, operationalizing these theories involves abstracting intricate philosophical ideas while we strive to preserve the core principles of each ethical approach. We selected representative theories and decision factors based on their prominence and relevance in the literature, recognizing that some degree of subjectivity is inherent in this process.

Our framework is based on key assumptions: we believe that the five clusters effectively encompass the major streams of ethical thought pertinent to AI decision-making and that the selected theories within each cluster are sufficiently representative of their respective ethical approaches. We also assume that the identified key concepts and decision factors can be meaningfully translated into computational models, providing a solid foundation for future research.

Recognizing that different countries and cultures can value certain moral beliefs over another, AMULED was designed to have modular selection of ethical frameworks, rather than imposing a single, moral philosophy to drive decisions.
While we highlight that this modular, multi-moral approach avoids the pitfalls of sticking to a fixed moral framework, we do recognize that the representation of belief values generated from LLMs may be prone to external biases and subjectivity.
This complicates the design of a universally accepted modular decision-making framework. 
Moreover, the dynamic nature of ethics poses a challenge, as ethical norms can evolve over time and vary across cultures, potentially rendering static models ineffective. Accountability is another critical issue \cite{helbing2021summary}; ambiguity arises regarding who is responsible for the outcomes generated by AI, whether it be the developers, the organization deploying the AI, or the model itself. To build trust and accountability, enhancing the transparency of the AMULED model's decision-making processes is essential, potentially through techniques that provide explanations, visualizations, or justifications for its ethical reasoning. This approach would improve user confidence and ensure that the AI's ethical framework aligns with societal values and norms \cite{helbing2024converging}.

% To address these limitations, future research could focus on developing hybrid ethical models that dynamically integrate multiple ethical frameworks, allowing the AI to adjust its decision-making approach based on contextual factors and evolving norms. Enhancing the interpretability of how AMULED influences decisions can help build trust, while investigating real-time adaptation methods could improve the system's relevance. Additionally, studying ethics in diverse contexts and conducting research on mitigating biases in belief representation are essential for more equitable outcomes. Exploring multi-agent scenarios and establishing benchmarks for evaluating the effectiveness of AMULED can further enhance its robustness and alignment with societal values.

The integration of belief probability assignment (BPA) into reinforcement learning presents a transformative approach to navigating ethical decision-making in AI systems. Aggregating beliefs from multiple feedback sources (such as ethical LLMs and human evaluators) enables AI to assess and balance competing ethical models effectively. This capability is particularly crucial in situations marked by normative uncertainty, where differing moral frameworks such as utilitarianism and deontology can yield conflicting recommendations. These techniques help the AMULED system evaluate ethical choices with a nuanced understanding of diversity and confidence in various theories. This iterative approach allows for the continuous refinement of decision-making strategies, ensuring that actions are aligned with widely accepted ethical standards. Furthermore, our research emphasizes the need to expand beyond single moral values, accommodating a multi-value system that reflects real-world complexities. By exploring applications in dynamic environments, such as multi-agent systems and human-AI collaboration, we aim to enhance ethical robustness in AI, paving the way for more responsible and adaptable autonomous systems.

% Our work presents two simple scenarios in which an ethics-based shaping algorithm helps RL agents make ethically sound decisions while still achieving their primary objectives. These scenarios serve as placeholders for trivial daily activities, demonstrating how ethical considerations can be seamlessly integrated into routine tasks within reinforcement learning environments. Leveraging large language models to inform agent behavior allows us to build on existing work that highlights the limitations of traditional reward shaping.

% Our approach not only enhances ethical performance but also demonstrates the importance of incorporating diverse moral theories, addressing moral uncertainty effectively. This positions our work at the forefront of evolving methodologies aimed at fostering responsible AI development and deployment. By addressing ethical objectives often neglected in traditional reinforcement learning, our approach ensures that ethical considerations are integrated without compromising the primary objectives, paving the way for more holistic and responsible AI systems.

% From our simple examples, we see significant improvement in behavior when incorporating diverse moral philosophies, showing the importance of moral decision-making in RL contexts. Our ethics-shaping approach simplifies the design of a value-aligned RL system. This approach divides the complex task into two distinct layers: the first layer focuses on achieving the primary goal, while the second, ethical layer refines the outcome to meet the secondary ethical goal. This two-layered structure allows the system to address core objectives initially, followed by ethical adjustments to enhance overall responsibility.

% However, we do see that this approach is sensitive to i) the reasoning quality of the language model, and ii) the availability of feedback samples to shape learning. Language models do not have the long-term planning to guide their reasoning, which makes them underperform a well-crafted reward function for long-term, spatial tasks. Human samples, on the other hand, would provide the best "human-aligned" feedback but might be too sparse when dealing with large and complex state-action spaces. We also see that different moral frameworks can result in different priorities for performing the ethical tasks.

% These insights emphasize the value of understanding and optimizing moral frameworks for developing agents capable of addressing complex ethical challenges. Overall, our research tries to lay a solid foundation for future exploration into enhancing moral reward systems and improving spatial reasoning in large language models.
\section{Conclusion}
This paper has introduced \sysName, a novel LLM-based system to simplify the creation of 2D interactive scenes without coding from natural language input and graphical control. We utilized content analysis on video tutorials about creating interactive scenes using ChatGPT and existing AI coding tools, and distilled several issues including the lack of independent generation and refinement, graphical control, and precise refinement. Based on these findings, we proposed a context-aware modularization technique that processes textual descriptions through individual LLM modules, with a central module coordinating interactions, allowing for independent refinement of each element. Our graphical user interface combined these modular LLMs with advanced graphical controls, enabling seamless code generation for 2D interactive scenes and direct integration of graphical information. We conducted a comparative study between \sysName~and \yh{Cursor Composer}, and found \sysName~significantly reduced the time, prompt trials, and prompt lengths and achieved better graphical and precise control in creating interactive scenes. Another open-ended usability study demonstrated that \sysName~allowed users to create various desired scenes easily without the need for coding, benefiting diverse application scenarios.






%%
%% The acknowledgments section is defined using the "acks" environment
%% (and NOT an unnumbered section). This ensures the proper
%% identification of the section in the article metadata, and the
%% consistent spelling of the heading.


% \begin{acks}
% To Robert, for the bagels and explaining CMYK and color spaces.
% \end{acks}

%%
%% The next two lines define the bibliography style to be used, and
%% the bibliography file.
\bibliographystyle{ACM-Reference-Format}
\bibliography{Reference}

%%
%% If your work has an appendix, this is the place to put it.

% \appendix

% \section{Research Methods}

% \subsection{Part One}

% Lorem ipsum dolor sit amet, consectetur adipiscing elit. Morbi
% malesuada, quam in pulvinar varius, metus nunc fermentum urna, id
% sollicitudin purus odio sit amet enim. Aliquam ullamcorper eu ipsum
% vel mollis. Curabitur quis dictum nisl. Phasellus vel semper risus, et
% lacinia dolor. Integer ultricies commodo sem nec semper.

% \subsection{Part Two}

% Etiam commodo feugiat nisl pulvinar pellentesque. Etiam auctor sodales
% ligula, non varius nibh pulvinar semper. Suspendisse nec lectus non
% ipsum convallis congue hendrerit vitae sapien. Donec at laoreet
% eros. Vivamus non purus placerat, scelerisque diam eu, cursus
% ante. Etiam aliquam tortor auctor efficitur mattis.

% \section{Online Resources}

% Nam id fermentum dui. Suspendisse sagittis tortor a nulla mollis, in
% pulvinar ex pretium. Sed interdum orci quis metus euismod, et sagittis
% enim maximus. Vestibulum gravida massa ut felis suscipit
% congue. Quisque mattis elit a risus ultrices commodo venenatis eget
% dui. Etiam sagittis eleifend elementum.

% Nam interdum magna at lectus dignissim, ac dignissim lorem
% rhoncus. Maecenas eu arcu ac neque placerat aliquam. Nunc pulvinar
% massa et mattis lacinia.

\end{document}
\endinput
%%
%% End of file `sample-authordraft.tex'.
