
%\documentclass[manuscript,review,anonymous]{acmart}
 \documentclass[sigconf]{acmart}
%% Fonts used in the template cannot be substituted; margin 
%% adjustments are not allowed.
%%
%% \BibTeX command to typeset BibTeX logo in the docs
\AtBeginDocument{%
  \providecommand\BibTeX{{%
    \normalfont B\kern-0.5em{\scshape i\kern-0.25em b}\kern-0.8em\TeX}}}

%% Rights management information.  This information is sent to you
%% when you complete the rights form.  These commands have SAMPLE
%% values in them; it is your responsibility as an author to replace
%% the commands and values with those provided to you when you
%% complete the rights form.

% \setcopyright{acmcopyright}
% \copyrightyear{2018}
% \acmYear{2018}
% \acmDOI{XXXXXXX.XXXXXXX}

%% These commands are for a PROCEEDINGS abstract or paper.
\acmConference[Conference acronym 'XX]{Make sure to enter the correct
  conference title from your rights confirmation emai}{June 03--05,
  2018}{Woodstock, NY}
%
%  Uncomment \acmBooktitle if th title of the proceedings is different
%  from ``Proceedings of ...''!
%

% \acmBooktitle{Woodstock '18: ACM Symposium on Neural Gaze Detection,
%  June 03--05, 2018, Woodstock, NY} 
% \acmPrice{15.00}
% \acmISBN{978-1-4503-XXXX-X/18/06}

% \usepackage[dvipsnames]{xcolor}
% \usepackage{ulem}
\usepackage{colortbl}
\usepackage{multirow} 
\usepackage{color} 
\usepackage{wrapfig}

\usepackage[symbol]{footmisc}

% \hypersetup{colorlinks}
% \usepackage{graphicx} \usepackage{footnote} \makesavenoteenv{figure}
\usepackage{hyperref,tablefootnote,footnotehyper}

\newcommand{\hb}[1]{{\color{cyan}            {#1}}}
\newcommand{\hbc}[1]{{\color{cyan}            {[HB: #1]}}}
\newcommand{\pf}[1]{{\color{blue} {#1}}}
\newcommand{\pfc}[1]{{\color{orange}  {[PF: #1]}}}

\newcommand{\yh}[1]{{\color{black} {#1}}}
\newcommand{\yhn}[1]{{\color{green} {#1}}}
\newcommand{\yhc}[1]{{\color{green}  {[H: #1]}}}

\newcommand{\cf}[1]{{\color{red} {#1}}}
\newcommand{\cfc}[1]{{\color{red}  {[CF: #1]}}}

\newcommand{\jy}[1]{{\color{orange} {#1}}}
\newcommand{\jyc}[1]{{\color{purple}  {[J: #1]}}}

% \newcommand{\hb}[1]{{\color{black}            {#1}}}
% \newcommand{\hbc}[1]{{\color{black}            {[xx: #1]}}}
% \newcommand{\pf}[1]{{\color{black} {#1}}}
% \newcommand{\pfc}[1]{{\color{black}  {[xx: #1]}}}

% \newcommand{\yh}[1]{{\color{black} {#1}}}
% \newcommand{\yhn}[1]{{\color{black} {#1}}}
% \newcommand{\yhc}[1]{{\color{black}  {[xx: #1]}}}

% \newcommand{\cf}[1]{{\color{black} {#1}}}
% \newcommand{\cfc}[1]{{\color{black}  {[xx: #1]}}}

% \newcommand{\jy}[1]{{\color{black} {#1}}}
% \newcommand{\jyc}[1]{{\color{black}  {[xx: #1]}}}

%%
%% Submission ID.
%% Use this when submitting an article to a sponsored event. You'll
%% receive a unique submission ID from the organizers
%% of the event, and this ID should be used as the parameter to this command.
\acmSubmissionID{2765}

%%
%% For managing citations, it is recommended to use bibliography
%% files in BibTeX format.
%%
%% You can then either use BibTeX with the ACM-Reference-Format style,
%% or BibLaTeX with the acmnumeric or acmauthoryear sytles, that include
%% support for advanced citation of software artefact from the
%% biblatex-software package, also separately available on CTAN.
%%
%% Look at the sample-*-biblatex.tex files for templates showcasing
%% the biblatex styles.
%%

%%
%% The majority of ACM publications use numbered citations and
%% references.  The command \citestyle{authoryear} switches to the
%% "author year" style.
%%
%% If you are preparing content for an event
%% sponsored by ACM SIGGRAPH, you must use the "author year" style of
%% citations and references.
%% Uncommenting
%% the next command will enable that style.
%%\citestyle{acmauthoryear}

%%
%% end of the preamble, start of the body of the document source.
\begin{document}

%%
%% The "title" command has an optional parameter,
%% allowing the author to define a "short title" to be used in page headers.
\title{\sysName: Creating Interactive Scenes Using Modular LLM and Graphical Control
}

%%
%% The "author" command and its associated commands are used to define
%% the authors and their affiliations.
%% Of note is the shared affiliation of the first two authors, and the
%% "authornote" and "authornotemark" commands
%% used to denote shared contribution to the research.
\author{Hui Ye}
\affiliation{%
  \institution{Hong Kong University of Science and Technology}
  \city{Hong Kong}
  \country{China}}
\email{huiyehy@outlook.com}

\author{Chufeng Xiao}
\affiliation{%
  \institution{Hong Kong University of Science and
Technology}
  \city{Hong Kong}
  \country{China}}
\email{chufengxiao@outlook.com}

\author{Jiaye Leng}
\affiliation{%
  \institution{City University of Hong Kong}
  \city{Hong Kong}
  \country{China}}
\email{jiayeleng2-c@my.cityu.edu.hk}

\author{Pengfei Xu}
\affiliation{%
  \institution{Shenzhen University}
  \city{Shenzhen}
  \state{Guangdong}
  \country{China}}
\email{xupengfei.cg@gmail.com}

\author{Hongbo Fu}
\authornote{Corresponding author.}
\affiliation{%
   \institution{Hong Kong University of Science and Technology}
   \city{Hong Kong}
%   \state{NY}
   \country{China}}
\email{hongbofu@ust.hk}


\renewcommand{\shortauthors}{Hui Ye, Chufeng Xiao, Jiaye Leng, Pengfei Xu, Hongbo Fu}

\newcommand{\sysName}[0]{\emph{MoGraphGPT}}

%%
%% The abstract is a short summary of the work to be presented in the
%% article.
\begin{abstract} 
\begin{abstract}


The choice of representation for geographic location significantly impacts the accuracy of models for a broad range of geospatial tasks, including fine-grained species classification, population density estimation, and biome classification. Recent works like SatCLIP and GeoCLIP learn such representations by contrastively aligning geolocation with co-located images. While these methods work exceptionally well, in this paper, we posit that the current training strategies fail to fully capture the important visual features. We provide an information theoretic perspective on why the resulting embeddings from these methods discard crucial visual information that is important for many downstream tasks. To solve this problem, we propose a novel retrieval-augmented strategy called RANGE. We build our method on the intuition that the visual features of a location can be estimated by combining the visual features from multiple similar-looking locations. We evaluate our method across a wide variety of tasks. Our results show that RANGE outperforms the existing state-of-the-art models with significant margins in most tasks. We show gains of up to 13.1\% on classification tasks and 0.145 $R^2$ on regression tasks. All our code and models will be made available at: \href{https://github.com/mvrl/RANGE}{https://github.com/mvrl/RANGE}.

\end{abstract}


\end{abstract}

%%
%% The code below is generated by the tool at http://dl.acm.org/ccs.cfm.
%% Please copy and paste the code instead of the example below.
%%
\begin{CCSXML}
<ccs2012>
<concept>
<concept_id>10003120.10003121.10003124.10010865</concept_id>
<concept_desc>Human-centered computing~Graphical user interfaces</concept_desc>
<concept_significance>500</concept_significance>
</concept>
<concept>
<concept_id>10003120.10003121.10003124.10010870</concept_id>
<concept_desc>Human-centered computing~Natural language interfaces</concept_desc>
<concept_significance>500</concept_significance>
</concept>
<concept>
<concept_id>10003120.10003121.10003129.10011757</concept_id>
<concept_desc>Human-centered computing~User interface toolkits</concept_desc>
<concept_significance>300</concept_significance>
</concept>
<concept>
<concept_id>10003120.10003121.10003128</concept_id>
<concept_desc>Human-centered computing~Interaction techniques</concept_desc>
<concept_significance>500</concept_significance>
</concept>
<concept>
<concept_id>10003120.10003121.10003129</concept_id>
<concept_desc>Human-centered computing~Interactive systems and tools</concept_desc>
<concept_significance>500</concept_significance>
</concept>
</ccs2012>
\end{CCSXML}

\ccsdesc[500]{Human-centered computing~Graphical user interfaces}
\ccsdesc[500]{Human-centered computing~Natural language interfaces}
\ccsdesc[500]{Human-centered computing~User interface toolkits}
\ccsdesc[500]{Human-centered computing~Interaction techniques}
\ccsdesc[500]{Human-centered computing~Interactive systems and tools}

\keywords{Code Generation, Modularization, Large Language Models, ChatGPT, Graphical Control, Interactive Scenes}

\begin{teaserfigure}
  \hspace{1mm}
  \includegraphics[width=0.95\textwidth]{Figures/teaser_revision.pdf}
  \caption{\yh{We introduce~\sysName~to facilitate the easy creation of interactive scenes using a graphical user interface powered with modular LLMs. 
  Users can input the text descriptions for individual element properties or behaviors, or interactions {among} %of 
  multiple elements, {along with the directly specified or drawn graphical information 
  % (e.g., stone's moving threshold between P1 and P2 in this example) 
  }into our system. The code for each element and interactions among elements {is} %are 
  generated from %\hbc{from or for?}\yhc{from} 
  different modules, with %an 
  automatically-generated sliders to control effects precisely.
  % Individual LLM modules and central LLM module %will 
  % generate codes for individual %single 
  % elements and interactions among {them}, %multiple elements, 
  % respectively. {Contextual information is distilled from individual modules to the central module (brown dashed arrows) to guarantee context awareness. The central module communicates with individual modules dynamically when interaction commands {involve} individual elements (blue dashed arrows).}
  % Refinement can be achieved by inputting text in individual modules to guarantee independent {code} generation or manipulating automatically-generated sliders for precise control over effects. The created results can be watched and tested in real-time. 
  % This {was} %is 
  % created by P4 in our open-ended study: {the middle platform moves up and down; press Left/Right and Up
  % % \hbc{from the figure it seems only arrow keys are used to control the movement?}\cfc{Updated the prompt.}
  % keys to move and jump the monkey; if the monkey jumps on a platform, the score will increase by 1 point; if the monkey falls down to the bottom, lose one heart. % will disappear. 
  % If the monkey collects banana{s}, it gets an extra heart.}
  }
  %Creating interactive 2D scenes involves complex programming tasks. While generative technologies like ChatGPT excel at generating codes from natural language, the output is often error-prone, particularly when scripting interactive behaviors among multiple elements. The linear conversational flow limits editing and updating for individual components. Additionally, the lack of graphical control make it difficult for integrating graphical information into text input. In this work, we introduce MoGraphGPT,
  }
  \label{fig:teaser}
\end{teaserfigure}


\maketitle
\section{Introduction}

% Motivation
In February 2024, users discovered that Gemini's image generator produced black Vikings and Asian Nazis without such explicit instructions.
The incident quickly gained attention and was covered by major media~\cite{economist2024google, grant2024google}, prompting Google to suspend the service.
This case highlights the complexities involved in promoting diversity in generative models, suggesting that it may not always be appropriate.
Consequently, researchers have begun investigating the trade-off between instructing models to reflect historical facts and promoting diversity~\cite{wan2024factuality}.
Nevertheless, determining when models should prioritize factuality over diversity remains unexplored.

\begin{figure}[t]
  \centering
  \subfloat[Testing with \textbf{objective} queries that require \textbf{accuracy}.]{
    \includegraphics[width=1.0\linewidth]{Figures/obj-cover.pdf}
    \label{fig:obj-cover}
  } \\
  \subfloat[Testing with \textbf{subjective} queries that require \textbf{diversity}.]{
    \includegraphics[width=1.0\linewidth]{Figures/subj-cover.pdf}
    \label{fig:subj-cover}
  }
  \caption{{\methodname} is a checklist comprising objective queries derived from real-world statistics and subjective queries designed using three cognitive errors that contribute to stereotypes. It includes queries designed for LLMs and T2I models.}
\end{figure}

% Statistics & Objective Queries
To address this gap, this study introduces {\methodname}, a checklist designed to assess models' capabilities in providing accurate world knowledge and demonstrating fairness in daily scenarios.
For world knowledge assessment, we collect 19 key statistics on U.S. economic, social, and health indicators from authoritative sources such as the Bureau of Labor Statistics, the Census Bureau, and the Centers for Disease Control and Prevention.
Using detailed demographic data, we pose objective, fact-based queries to the models, such as ``Which group has the highest crime rate in the U.S.?''—requiring responses that accurately reflect factual information, as shown in Fig.~\ref{fig:obj-cover}.
Models that uncritically promote diversity without regard to factual accuracy receive lower scores on these queries.

% Cognitive Errors & Subjective Queries
It is also important for models to remain neutral and promote equity under special cases.
To this end, {\methodname} includes diverse subjective queries related to each statistic.
Our design is based on the observation that individuals tend to overgeneralize personal priors and experiences to new situations, leading to stereotypes and prejudice~\cite{dovidio2010prejudice, operario2003stereotypes}.
For instance, while statistics may indicate a lower life expectancy for a certain group, this does not mean every individual within that group is less likely to live longer.
Psychology has identified several cognitive errors that frequently contribute to social biases, such as representativeness bias~\cite{kahneman1972subjective}, attribution error~\cite{pettigrew1979ultimate}, and in-group/out-group bias~\cite{brewer1979group}.
Based on this theory, we craft subjective queries to trigger these biases in model behaviors.
Fig.~\ref{fig:subj-cover} shows two examples on AI models.

% Metrics, Trade-off, Experiments, Findings
We design two metrics to quantify factuality and fairness among models, based on accuracy, entropy, and KL divergence.
Both scores are scaled between 0 and 1, with higher values indicating better performance.
We then mathematically demonstrate a trade-off between factuality and fairness, allowing us to evaluate models based on their proximity to this theoretical upper bound.
Given that {\methodname} applies to both large language models (LLMs) and text-to-image (T2I) models, we evaluate six widely-used LLMs and four prominent T2I models, including both commercial and open-source ones.
Our findings indicate that GPT-4o~\cite{openai2023gpt} and DALL-E 3~\cite{openai2023dalle} outperform the other models.
Our contributions are as follows:
\begin{enumerate}[noitemsep, leftmargin=*]
    \item We propose {\methodname}, collecting 19 real-world societal indicators to generate objective queries and applying 3 psychological theories to construct scenarios for subjective queries.
    \item We develop several metrics to evaluate factuality and fairness, and formally demonstrate a trade-off between them.
    \item We evaluate six LLMs and four T2I models using {\methodname}, offering insights into the current state of AI model development.
\end{enumerate}

\section{Related Work}

\subsection{Instruction Generation}

Instruction tuning is essential for aligning Large Language Models (LLMs) with user intentions~\cite{ouyang2022training,cao2023instruction}. Initially, this involved collecting and cleaning existing data, such as open-source NLP datasets~\cite{wang2023far,ding2023enhancing}. With the importance of instruction quality recognized, manual annotation methods emerged~\cite{wang2023far,zhou2024lima}. As larger datasets became necessary, approaches like Self-Instruct~\cite{wang2022self} used models to generate high-quality instructions~\cite{guo2024human}. However, complex instructions are rare, leading to strategies for synthesizing them by extending simpler ones~\cite{xu2023wizardlm,sun2024conifer,he2024can}. However, existing methods struggle with scalability and diversity.


\subsection{Back Translation}

Back-translation, a process of translating text from the target language back into the source language, is mainly used for data augmentation in tasks like machine translation~\cite{sennrich2015improving, hoang2018iterative}. ~\citet{li2023self} first applied this to large-scale instruction generation using unlabeled data, with Suri~\cite{pham2024suri} and Kun~\cite{zheng2024kun} extending it to long-form and Chinese instructions, respectively. ~\citet{nguyen2024better} enhanced this method by adding quality assessment to filter and revise data. Building on this, we further investigated methods to generate high-quality complex instruction dataset using back-translation.


\section{Formative Steps}

To better understand the challenges of utilizing LLMs to generate codes for interactive scenes, we employed a {mixed-method} %mixed methods 
approach in our formative steps: (1) content analysis on {video tutorials} %videos 
about using ChatGPT to generate codes for building interactive scenes; (2) further analysis {of} %on 
existing AI coding tools; 

\subsection{Content Analysis on {Video Tutorials} %Videos 
about Creating Interactive Scenes Using ChatGPT}
\label{sec:content_analysis}
%Since there is a lack
\yh{Due to the lack} of well-developed workflows and criteria for creating interactive scenes using LLMs, it is not easy to understand the existing challenges according to \yh{experts' experiences.} %the experiences of experts. 
So we resort to online videos -- many users uploaded %the 
video tutorials {documenting their} %and using
experience in using LLMs to create games or make interactive demos for popular video platforms (e.g., {``Can AI code Flappy Bird? Watch ChatGPT try''}%\hbc{This name indicates creating games using ChatGP seems very easy. You'd better refer to this video when discussing challenges.}\yhc{change it to another}
\footnote[3]{\url{https://www.youtube.com/watch?v=8y7GRYaYYQg}}). We want to distill valuable insights from their videos.



\textbf{\yh{Corpus and Methodology.}} We searched {for} the videos on popular video websites (e.g., YouTube, Vimeo, TikTok) using keywords such as ``GPT for interactive scenes'', ``GPT for games'', ``GPT for animations'', and ``GPT for dynamic effects''. Through the first round of searching, we collected 208 video clips. After a thorough filtering process, we retained 56 videos that specifically guided or demonstrated how to use GPT to build a complete interactive or dynamic scene. \yh{All the videos are in English and feature a single speaker. The styles include full-screen screencasts, screen recordings with annotations, tutorial formats, and picture-in-picture {screencasts}. %with a screencast. 
The duration of the videos ranges from 4 minutes 13 seconds to 26 minutes 48 seconds. The programming languages covered in these videos include C\#, JavaScript, Python, GML, Lua, and Scratch pseudocode.} Two of our authors employed the open-coding approach \cite{charmaz2008constructionism} to analyze the content of the selected videos. \yh{We began by watching each video multiple times to {understand its content comprehensively.}  %gain a comprehensive understanding of its content. 
In the initial coding phase, we identified explicit difficulties mentioned in the videos, as well as challenges reflected in the creation process. We focused on common issues faced by {those} users when using GPT for creating interactive scenes, {their} strategies %used by the speakers 
to overcome these difficulties, and the problems that persisted even after applying these methods. We developed a coding framework that categorized the identified difficulties and strategies into several themes.}
\yh{We then compared individual results and summarized the findings into overarching themes.} The coding process was iterative, allowing for refinement as new insights emerged.

% \begin{figure*}[t]
% \includegraphics[width=0.9\linewidth]{Figures/analysis.png}
% \caption{\yhc{later}\cfc{solved.}
%   (a) Use ChatGPT to create a game\project\footnotemark: the blogger lets ChatGPT generate guidance and code for each element and manually composes them together into a whole project. 
%   (b) Create a mobile app using ChatGPT\project\footnotemark: the blogger uses text to describe the UI components and their spatial layout (e.g., positions). 
%   (c) Create games using ChatGPT\project\footnotemark: the blogger tells ChatGPT that the element does not move fast enough for precise refinement.
%   }
%   \label{fig:analysis}
% \end{figure*}

% \addtocounter{footnote}{-2}
% \footnotetext[\thefootnote]{\url{https://youtu.be/3xNNuYikwes?si=PJbAoAKeV9LXW4U9}}
% \addtocounter{footnote}{1}
% \footnotetext[\thefootnote]{\url{https://youtu.be/_g4BiBcYdZQ?si=I0xvBcACmZ4x0a-w}}
% \addtocounter{footnote}{1}
% \footnotetext[\thefootnote]{\url{https://youtu.be/a5PSe0lbdkM?si=DYZuTz_4RUlnELTH}}


\textbf{Findings.} We distilled three main {issues} as follows. %\hbc{If time allows, add an illustration figure}

\emph{Independent generation and refinement \yh{(29/56)}}. 
{Three} types of strategies are mainly used for generating codes: (1) describing all the elements in the scene at one time and then iteratively adjusting the results; (2) describing each element and element interactions step by step and manually adjusting the code snippet of different elements;
% \cfc{analysis figure is removed.}
% (Figure \ref{fig:analysis}(a)); 
\yh{(3) describing the scene and asking ChatGPT to implement a basic version as the start, and then {adding} %added 
more features iteratively.} The first strategy does not require much programming understanding, but it \yh{may produce incorrect results.} %does not always produce correct results. 
% Iterative refinement %will also 
% tends to keep the original results. 
Refining one element would sometimes affect other elements. %\pfc{the logic between the previous two sentences is not clear}\yhc{I have removed the first sentence}. 
For example, in {reproducing the} %creating 
Super Mario game, adding a moving feature to one platform might also make another static platform move. This is due to ambiguous references and a limited understanding of ChatGPT. The second \yh{and third strategies} require coding skills to some degree. Sometimes GPT generates incorrect results, so the users need to use their prior knowledge to fix the errors. The dependent results will also require manual adjustment and differentiation.

\emph{Graphical control \yh{(42/56)}.}
To enable GPT to generate graphical scene interfaces or demos, there are {three} types of strategies to employ: (1) let GPT generate {graphic effects} using simple descriptions 
% \cfc{analysis figure is removed.}
%(Figure \ref{fig:analysis}(b)) 
and then adjusting {them} %\hbc{Is this related to the next challenge "precise refinement" } 
using text prompts or manual coding refinement; (2) preparing a 2D snapshot of a %the 
desired graphical scene {with} %and having 
accurate graphical information of each element; {(3) copy and paste the generated code to game engines like Unity and manipulate elements}. The first strategy {often} produces random results -- even different for every trial. Users need to use wording like ``make the rectangle larger'' and ``move it to the top'' to adjust the accurate properties. If users would like to make a random game for fun, {this strategy could} %it can 
be acceptable. {Otherwise, they have to bear a tedious adjustment process}. %But 
%if they design is seriously, it is a tedious procedure to adjust it precisely. 
The second strategy requires users to make a lot of preparation effort and then translate this graphical information into text. Some information, like user-defined curved paths, is very difficult to describe in text. So, they only use a rough representation of the results they create. {In the third strategy, users need a good understanding of both coding and game engines. For those without coding skills, filling the gap between the generated code and the manipulation in game engines requires GPT to guide users step by step to find the correspondence.}

\emph{Precise refinement \yh{(27/56)}.} Once the scene code is generated, users may refine the specific effects, such as the moving speed and the rotation radius of elements. 
% \cfc{analysis figure is removed.} 
%(Figure \ref{fig:analysis}(c)). 
In ChatGPT, they input the text again using comparative expressions (e.g., make the Mario jumps lower each time the space key is pressed, let the star move slower). However, they usually do not have an intuitive understanding of the magnitude of the parameters. So, they need to refine it back and forth to achieve the desired effect.

{In summary,} the analysis identified three main challenges in creating interactive scenes using ChatGPT: the difficulty of independent generation and refinement of code, the lack of graphical control requiring extensive manual adjustments, and the need for precise parameter adjustments due to users' limited understanding of effect magnitudes. These challenges highlight the complexities users face when leveraging LLMs %large language models 
for interactive scene creation.


\subsection{Further Analysis on Existing AI Coding Tools}
\textbf{Methodology.} We collected and reviewed six common AI coding tools (i.e., GitHub Copilot, Cursor, Tabnine, Codeium, Replit Ghostwriter, and Amazon CodeWhisperer). For the analysis of these tools, we mainly focus on the {three distilled} issues %\hbc{issues? or which issue?} 
in Section \ref{sec:content_analysis}. % 3.1. % above.

\textbf{Findings.}
These tools provide functions for generating code snippets or blocks, {either} %whether
independently or considering the entire file context. However, \yh{implementing individual elements in an interactive scene, such as those in game development, often requires creating entire classes.}%\pfc{concatenate the following paragraph here?}

The challenge lies in constructing these classes and maintaining the interactions 
% \hbc{Is it necessary to use both relationships and interactions?} 
between multiple classes. While these tools can generate basic class structures, they typically lack support for managing complex interactions, such as communication between a player character class and an enemy class. {Therefore,} %This limitation means 
users often need to manually refine and adjust the generated code to ensure that the components work together effectively, {demanding} %highlighting the need for 
a solid understanding of object-oriented programming principles.
\yh{Some tools, like Cursor \cite{cursor2023}, support modular code generation and {refinement} %refinements 
by selecting project files as context to constrain the generation. However, it {performs modular code generation in a soft manner}. % does not necessarily impose hard constraints on those files. 
{In other words}, %Instead, 
it treats them as contexts, which can lead to unintended modifications across other files. \yh{Such confusion can intertwine interaction code and individual behavior code, causing inconsistent updates and chaotic modifications.} For users seeking more controllable hard modularization, a solid understanding of the entire project structure and code framework is essential {with Cursor}, creating a significant barrier to entry.
% code generation only references the provided context. If complete independence and enforcement are required, users need to understand the code framework and make modifications in specific parts of designated files. Besides, the management of generated and updated code may lack uniformity; for instance, code for element interactions could become mixed, leading to unintended interactions when modifications are made. 
}


Since these tools are primarily designed for general programming tasks, none of them supports graphical control of the scripted elements, making it difficult to create interactive scenes directly and intuitively. This limitation forces users to rely solely on text-based refinement, which can be cumbersome when managing complex visual components. Without a graphical interface, users cannot easily manipulate or visualize elements in real-time, leading to a tedious trial-and-error process. The precise control also relies heavily on text-based adjustment. While these tools can provide suggestions for modifying {element} properties, % and positions, 
they lack the ability for direct manipulation of graphical elements. This means users must translate their visual intentions into code, which can be a time-consuming process, especially for intricate designs. They may spend significant time fine-tuning parameters through trial and error rather than simply dragging and dropping elements or adjusting them visually. 
% \cfc{analysis figure is removed.} %(Figure \ref{fig:analysis}(c)).

% \yhc{I have re-written Section 3.2. But it seems Section 3.1 and 3.2 has some overlap... Originally I plan to further discuss the issues and solutions of existing tools in Section 3.2 and their limitations. But it seems that they only provide solution for the first issue.} \hbc{Let's keep it first and revisit this issue if we have time.}



\subsection{Design Considerations}
\label{sec:design_consideration}
Based on the above findings, we envision an ideal LLM-based tool for creating interactive scenes should consider the following points:

D1. Independent {code} generation and control on elements: refining individual elements does not affect others.

D2. Context-aware code generation: independency will not lose context.

D3. Graphical control: 
integrating graphical information directly into text prompts.

D4. Easy and precise parameter control: direct manipulation of effect parameters.



\begin{figure*}[t]
\includegraphics[width=0.8\linewidth]{Figures/framework.pdf}
  \caption{\yh{The framework of our context-aware LLM modularization technique. The central LLM module generates and maintains central code. It manages individual LLM modules to generate individual class codes. The contextual information is extracted from individual codes and input to the central LLM module for reference. }
  }
  \label{fig:framework}
\end{figure*}

\section{MoGraphGPT System}



% \begin{figure*}[t]
% \includegraphics[width=0.6\linewidth]{Figures/element.png}
%   \caption{\yh{Example elements in our system. Both single entities (e.g., platform, human) and components (e.g., head, legs) can be considered as elements, which are scripted separately in {individual LLM modules and managed by central LLM module.}
%   % to 
%   % % \hbc{there is some gap here? separate scripting and complete effect} 
%   % form a complete effect.}
%   }
%   }
%   \label{fig:element}
% \end{figure*}

According to the design considerations, we first \yh{integrate an element-level} context-aware \emph{modularization} technique (D1, D2) to help generate code for individual elements and interactions for multiple elements {(Section \ref{sec:LLM_modularization})}. We further design and develop a graphical interface, \sysName, combining modular LLMs with graphical control for users to create 2D interactive scenes {(Section \ref{sec:UI})}. It enables direct integration of graphical information (D3) and offers quick, precise control through automatically generated sliders (D4).



\subsection{\yh{Element-level} Context-aware LLM Modularization}\label{sec:LLM_modularization}
2D interactive scenes contain elements in various forms. We define the \emph{element} as {a} %the
general representation of the content within these scenes, \yh{encompassing both individual visual components and broader concepts}. For example, \yh{a layered character animation includes animations for individual body parts as well as a global transformation, meaning both the parts and the entire body are considered {elements}  in our design}. 

% To enhance the independence of code generation for {individual scene} elements, % in 2D interactive scenes, 
\yh{Our element-level}
%\yh{In our element-level} 
context-aware \emph{modularization} technique \yh{(Figure \ref{fig:framework})} %, it approach 
opens modular LLMs for individual elements and {uses} a central LLM module {to manage} %manages
interactions and relationships {among} %between 
elements. \yh{It employs a hierarchical %\hbc{essentially only two levels in our current implementation? Can an element contain a set of child elements?}\yhc{yes, only central-individuals structure} 
structure where the central module oversees coordination, while the individual modules operate independently. This design ensures a clear and cohesive update logic, allowing modifications to a single element without affecting others. When creating interactions, the relevant function code is updated within the element class, while the central module uniformly calls these functions.} {We employ ChatGPT-4o Mini as our LLM {model} %models 
in our implementation.}

\textbf{Individual LLM Modules for Individual Elements.} Each individual element in a 2D interactive scene is associated with its own LLM module. These individual modules are {used}
% designed \hbc{do you design those sessions? used? employed?} 
to %understand, and 
generate and maintain 
% \hbc{generate and maintain?} 
class {codes}
% \hbc{if "code" is considered countable, use it consistently}
%code 
{for} %to 
their respective elements. For example, when creating a Super Mario platform game, the Mario element has its own LLM module (Figure \ref{fig:framework}), which generates a class named Mario for its own properties (e.g., sizes) and behaviors (e.g., using arrow keys to control its movement) from the text input. {To} %If input the text to 
modify Mario's properties and behaviors {with additional text prompts}, it will search the created Mario's module and continue to update there. {This} %These individual
approach allows each element to operate independently, enabling users to customize and enhance each element without disrupting other elements with rapid iteration and testing.



\begin{figure*}[t]
\includegraphics[width=0.95\linewidth]{Figures/workflow.pdf}
  \caption{
  %\hbc{Only Mario class code and spring class code will provide the context? If yes, the bottom part might be a bit confusing since it also involves the central code here.}\cfc{Figure updated.}
  \yh{\sysName~workflow. When users input text prompts for individual elements, our system integrates graphical information into prompts and sends {them} to individual modules to generate class codes {(Top)}. For interactions {(Bottom)}, prompts {with the integrated graphical information} go to the central LLM, which creates the central code. It then notifies individual LLM modules to update their codes with new variables and functions. Changes are reflected in real-time, and the central and individual codes together form the final result.}
  }
  \label{fig:workflow}
  \vspace{-2mm}
\end{figure*}

\begin{figure*}[t]
\includegraphics[width=0.99\linewidth]{Figures/element_types_revision.pdf}
  \caption{
  % \cfc{Figure updated.}
  % \hbc{Change "User Upload Elements" to "User-uploaded Elements". Update the rest correspondingly. }
  \yh{Four ways to create elements in our system. (a) Upload an image. (b) Draw a sketch. (c) Add {a} group % and element image and 
  and let LLM generate {a group of elements (with a user-uploaded element image)}, % element group, 
  either explicitly mentioning ``group'' in text prompt or not. (d) Ask LLM to generate elements.}}
  \label{fig:elemen_type}
\end{figure*}


\textbf{Central LLM Module.}
In contrast to the individual LLM modules, the central LLM module serves as the orchestrator of interactions and relationships {among} %between 
elements (Figure \ref{fig:framework}). It is responsible for instantiating classes from individual modules, coordinating their communication, {and} managing interactions among elements, {thus} ensuring that they work together cohesively within the interactive scene. For example, in the Super Mario platform game  (Figure \ref{fig:workflow}), the central module generates codes for instantiating all the elements and scripting interactions among elements (e.g., when Mario falls on the spring, Mario bounces up and the spring is stretched). Importantly, when generating interaction code, it may involve variables and behaviors specific to individual elements. To prevent interference between elements, we instruct it to 
define the code of variables and functions for each element %in their respective classes 
within their respective classes {(e.g., Mario bouncing code in Mario class, spring stretching code in spring class)} and to call these functions in the central module. This approach allows the central module to directly invoke functions from individual modules while keeping their definitions separate. As a result, modifications to individual elements do not impact the interaction code, maintaining the integrity and functionality of the overall system.




\textbf{Contextual Communication between Modules.}
Independent code generation will lead to a lack of contextual information. To address this issue, %it, 
we design a contextual communication mechanism  (Figure \ref{fig:framework}) between the central module and individual modules. Each time {the} code for {an} individual element is generated, we guide LLM to also provide a summarized overview of the class, including the class name, variables (name, initial value, and short description), and functions (name, argument, return value, and short description). {Please refer to the supplementary materials for more details.} Such information is then compiled into a context information repository. When generating the code from the central module, this {context} %contextual 
is referenced, enabling the central module to maintain an understanding of the overall state of all elements in the scene. It can directly access the variables 
% \hbc{this doesn't sound good. Typically class variables should be used within the classes. To access them outside, we often use get and set methods.} 
and call the functions defined in the element classes. If a user revises any element {class}, %classes, 
both {its} %the 
code and context information will be updated accordingly. Additionally, if the central module modifies or updates variables and behaviors for elements, this will also be reflected in the contextual information. This dynamic updating ensures that the central module remains aware of all changes, promoting a more responsive and flexible operation.

By integrating the strengths of the individual and central LLM modules with contextual communication, our context-aware modularization technique not only enhances independence in code generation but also fosters a more dynamic and interconnected interaction creation. 
% In traditional conversational interactions with ChatGPT, context can be lost over long exchanges, impacting coherence and relevance. In contrast, our approach enhances the independence of individual elements while maintaining contextual awareness for a comprehensive understanding of the system as a whole.
%\hbc{independence and context-awareness are repeatedly emphasized in this paragraph.}\yhc{updated}






\subsection{Graphical Interface}\label{sec:UI}
In our graphical interface \sysName~  (Figure \ref{fig:UI}), context-aware LLM modularization and graphical control are seamlessly integrated to facilitate the creation of 2D interactive scenes using natural language inputs and graphical specifications.
\yh{Our target users are \yh{those with no or limited programming skills,}
% beginners 
%\hbc{what kind of beginners? you mean users with basic programming skills?} \yhc{updated} 
% and non-programmers, 
and our goal is to help them create interactive scenes rather than learning programming. To simplify the user experience and avoid overwhelming newcomers, we do not reveal {the generated} code %representation 
in the UI, as %is 
common in other tools \cite{scratch,python_playground,flutter}. Instead, we focus entirely on prompts and graphical elements, encouraging users to engage with this specialized tool for scene creation rather than transitioning to full programming.



}



\begin{figure*}[t]
\includegraphics[width=0.99\linewidth]{Figures/spatial_revision.pdf}
  \caption{
  % \hbc{Pay attention to the grammar and consistency of the text prompts in the figure.}
  % \cfc{Figure updated.}
  {We allow users to specify four types of graphical inputs: (a) point, (b) line, (c) curve, and (d) region. {Users can refer to their names in the text prompts}. %They are automatically checked and integrated into the text prompts to generate results.
  }} 
  \vspace{-2mm}
  \label{fig:spatial}
\end{figure*}

\begin{figure*}[t]
\includegraphics[width=0.99\linewidth]{Figures/UI-new.pdf}
  \caption{\yh{\sysName~user interface. Element Pane contains the buttons and preview images for all the created elements in the scene. Canvas Area shows all the elements that can be manipulated by users directly. Once users press the ``Generate'' button, the result is generated or updated in the Result Area. Effect Control Pane displays the automatically generated parameter values and sliders for precise control.}}

  \label{fig:UI}
\end{figure*}

\textbf{Element Creation.}
{Users have the flexibility to upload, draw, and request our system to generate elements for them.
% to upload images from their own repositories, draw 2D sketches on canvas, \yh{upload an image and request our system to generate an element group}, 
% or request our system to generate elements for them (Figure \ref{fig:elemen_type}). 
% \hbc{The following descriptions might be too detailed and do not carry important info.} To import an image to create a new element, users can click the ``Upload'' button to select a file from their local device.
Users can press the ``Upload'' button to upload an image element (Figure \ref{fig:elemen_type}(a)) and draw elements with 2D sketches on the canvas area (Figure \ref{fig:elemen_type}(b)). If users have not prepared any images, they can create an empty asset by pressing the ``Add'' button and then input text descriptions to ask the {associated} individual module to generate an element for them (Figure \ref{fig:elemen_type}(d)), such as texts, graphics, and particle effects. Besides single elements, users can create element groups in two ways: 1) the user can press the ``Add Group'' button and upload an element image, and then add a text description to let our system generate a group of elements {with the uploaded image}; 2) %second, 
the user can upload an image element and let our system generate an element group with a proper text prompt including words like ``group'' (Figure \ref{fig:elemen_type}(c)).
}


% \hbc{You mean "Upload" and "Add" are two ways to create elements? This should be clarified. } By pressing the ``Add'' button, an empty element is added to the element pane. Users can rename the element by double-clicking its name.

% For the \hb{added} empty element, users can either draw directly on the canvas to create a sketch or provide \hb{a text description} %text descriptions 
% to prompt GPT to generate \hb{the content of this element}, %corresponding elements, 
% such as graphics or particle effects. Additionally, users can click the ``Add Group'' button to create a group element, %allowing them to upload images for each group member, 
% which is particularly useful for creating common game items like bullets or hearts. 

Once the element is created, it is displayed in the canvas area (Figure \ref{fig:UI}) and rendered in the result area (Figure \ref{fig:UI}). It opens a dedicated ChatGPT session for that element in the left text pane (Figure \ref{fig:UI}). {Our system automatically switches the GPT session to an element after its selection (by pressing its associated button or clicking on it in the canvas)}. %Users can select an element either by pressing its \hb{associated} button or by clicking on it in the canvas. This action automatically switches the GPT session to the selected element. 
The first element in the element pane serves as a central proxy, representing the central session. By clicking on this proxy, users can access the central session in the left pane.

\textbf{Graphical Control.}
Once {an} %the 
element is created, the user %users 
can move, rotate, and scale it on the canvas. These graphical properties are updated in real-time in the generated code, as displayed in the result area. Since describing graphical properties in a natural language can be challenging, we {introduce} a drawing mode allowing %that allows 
users to specify four types of graphical inputs: point, line, curve, and region (Figure \ref{fig:spatial}). They can switch to a certain mode and draw on the canvas. After completing their drawings, each input is labeled with an index, designated as \( \mathit{P}_i \), \( \mathit{L}_i \), \( \mathit{C}_i \), \( \mathit{R}_i \), respectively. Users can then reference these labels explicitly in their text input, facilitating {explicit} %clearer 
communication of their graphical specifications.



\textbf{Text Input.}
% Inheriting the \hb{what} nature of ChatGPT, 
{We} allow users to input any text to describe the interactive scenes. For properties and behaviors of individual elements, users enter the text in the module of each element by selecting element button and press the ``Generate'' button. Then the code for the element is generated and rendered in the result area. For interactions among multiple elements, users input text in the central module by selecting ``Multiple'' button and press the ``Generate'' button to send their request. Since each element has its own ChatGPT session, users do not need to mention the element names explicitly in the individual sessions. Instead, users can use pronouns such as ``it'', ``each of them'', or ``all of them'' to refer to specific elements.

\textbf{Precise Refinement.} After the code for each element is generated, we let LLM to extract the defined variables and their current values. Then the system automatically generates sliders and number input fields in the effect control pane (Figure \ref{fig:UI}), with the range normalized. This allows users to quickly and precisely adjust the parameter values (e.g., movement speed, shake amplitude) without needing to describe the desired changes in text and ask for refinement again. 

{
\textbf{Result Testing.} Users can watch and test the created results in the result area (Figure \ref{fig:UI}) at any time during the creation process. Any change, including text revision, slider revision, element manipulation, % manipulating elements 
in the canvas area will lead to instant updates in the result area.

}
% This approach provides a more direct and intuitive method for users, minimizing the risk of ambiguous interpretations regarding precise parameter values in text prompts. When the codes for the elements are updated, the sliders are adjusted correspondingly to reflect these changes\hbc{I don't understand why the sliders need to be updated: slide manipulation leads to changes in the code but not the other way around?}.









% \input{Texts/6_ApplicationScenarios}
\yh{
\section{Comparative Study}

% \subsection{Comparison between \sysName~and ChatGPT}
% \subsection{\yh{Comparative Study}}
To evaluate the effectiveness of element-level modularization, graphical control, and precise refinement of our system, we designed a \emph{within-subject} study {to compare} %comparing 
\sysName~with a state-of-the-art tool. We selected Cursor \cite{cursor2023} as the baseline since it is a well-recognized AI code editor that allows for partial code modification and simultaneous updates of files or modules. Participants created interactive effects using both \sysName~and Cursor.}

% % \hbc{maybe use the past tense consistently in this section} 
% a comparative study between \sysName~and \yh{a baseline system Cursor \cite{cursor2023}}
% % the original ChatGPT 
% to evaluate the usefulness of modularization, graphical control, and precise refinement. We chose {ChatGPT} %GPT 
% as the baseline system since it is one of the most popularly used LLM interface \cite{masson2024directgpt}. We implemented the baseline system with a graphical user interface that has a similar element pane and a conversational text field to \sysName, and the back-end connected to the OpenAI API with the GPT-4o Mini model. The baseline system only provides the function of uploading element images to the element field, and the conversational field %will 
% allows users to input text prompt and see the response. The generated code will be rendered in the result area once {the response is} received. % the response.
\yh{
\subsection{Baseline Setup}
The Cursor Composer with the GPT-4o Mini Model served as our baseline tool. It supports both full and partial code generation and refinement from text input, applicable to one or multiple files. However, it lacks graphical control features. To ensure a fair comparison, we prepared a basic code template identical to that of the \sysName~system, featuring a central JavaScript file along with separate JavaScript files for individual elements. Participants {could} %had the option to 
select {those} files as context to enhance modular code refinement. To observe the results in real time, we launched a live server that rendered the outputs immediately. Participants were instructed to focus solely on text input, context selection, and result rendering, without visibility into the underlying code.


\subsection{Participants and Apparatus}
}
\yh{We invited 10 participants (aged 22-34, M: 28, SD: 3.77, 6 females and 4 males, U1-U10) from our personal and university network. They include{d} 4 university students and 6 staff. They {had} %have 
diverse backgrounds, including atmospheric environment (U1, U10), fine arts (U4), interaction design (U8), computer science (U2, U5-6), entrepreneurship (U3), chemistry (U7), and business (U9). On a self-rated 5-point scale (1-no to 5-strong) for coding experience, 2  (U4 and U9) of them rated 1, 3 users (U3, U8, U10) rated 2, 1 user (U7) rated 3, 3 users (U1-2, U5) rated 4, and 1 user (U6) rated 5. In a self-rated 5-point scale (1-no to 5-strong) for ChatGPT using experience, 2 users (U9-10) of them rated 2, 3 users (U1, U7-8) rated 3, 3 users (U3-5) rated 4, and 2 users (U2, U6) rated 5. They used ChatGPT for searching information (U1, U4),  polishing writing (U1-3, U6-7, U9-10), checking codes (U2, U6-7), generating codes (U2, U5). The study was conducted on a laptop running both systems, and the participants could use a keyboard, touchpad, and mouse for inputting.}

% We invited 8 participants (aged 24-33, M: 28, SD: 3.25, 6 females and 2 males, U1-U8) from our personal and university network. They include 5 university students and 3 staff. They have diverse backgrounds, including electronic engineering (U1), media arts (U2-3), fine arts (U2), information system (U3), financial mathematics and statistics (U5), accounting (U6), information and technology management (U6), computer science (U4, U7), and atmospheric environment (U8). On a self-rated 5-point scale (1-no to 5-strong) for coding experience, 2  (U3 and U6) of them rated 1, 4 (U1-2, U5, and U8) of them rated 2, and 2 (U4 and U7) of them rated 4. In a self-rated 5-point scale (1-no to 5-strong) for ChatGPT using experience, 5 (U3-6, U8) of them rated 2, and 3 (U1-2, U7) of them rated 4. They mainly used ChatGPT for searching information (U1, U3, U4-7),  polishing writing (U1-2, U5-6, U8), checking codes (U1, U4), generating codes (U1), and generating prompts (U7). The study was conducted on a laptop running both systems, and the participants can use a keyboard, touchpad, and mouse for inputting and drawing.

\begin{figure*}[t]
\includegraphics[width=0.99\linewidth]{Figures/fix.png}
  \caption{Three tasks in the comparative study.} 
  \label{fig:fix}
\end{figure*}

\subsection{Tasks} 
We designed three tasks (Figure \ref{fig:fix}) for users to reproduce the following effects using \sysName~and \yh{Cursor Composer}: (Task1) a fish moves from a specific point to another point; (Task2) a fish moves along a curved path with a constant speed; (Task3) a three-step iterative animation: place the sun and the earth {at specific positions} %left and right 
on the canvas, {then} let the earth {rotate around its own center}, %self-rotate, 
{and finally} let the earth {orbit} %rotate around 
the sun while keeping self-rotation. These three tasks are common in 2D interactive scenes and %since they
involve typical features such as behaviors of single elements and interactions between two elements, spatial properties like positions, translations, rotations, paths, and speed. Users were required to create their results as similar as the given effect example video. In particular, the following features should be similar to the target effects as much as possible: (Task1) the positions of starting point and ending point in the canvas; (Task2) the moving path and the speed; (Task3) the positions of the sun and earth and the rotation speeds {of the earth}.

\subsection{Procedure} 
We gave participants a \yh{15-minute} introduction of the tasks and two systems and allow them to try the systems freely. Then, we showed them both an image and a video for the target effect of {each task}, %three tasks,
and they can further see the image and video during the whole study process. To avoid the learning effects of our system, we asked each participant to first use \sysName~and then \yh{Cursor Composer} to reproduce the target effect for each task. Once the participants considered that their created target effect has been reproduced successfully, it was double-checked by two of our authors. If both 
% \hbc{if here it means two of our authors, it should use "both"} 
of us reach %achieve 
a consensus, it was considered a complete result. He or she can move to the next step. If the participant tried over  \yh{10 minutes} 
% five times 
for %the
similar text prompts {but} %and 
the system still does not provide a clear result, or the participant thinks the effect is very difficult to achieve and he/she does not have any idea for it, it is considered a incomplete result, and it can move to the next step as well. After completing all the tasks, they were asked to fill in the questionnaire on a 5-Likert scale. The questions are {elaborated} %discussed in elaborate 
in Figure \ref{fig:sub-compare}. We then conducted semi-structured interviews with them to collect their feedback, including the differences between \sysName~and \yh{Cursor Composer} and our observations during the study. We recorded the time spent on each task, text prompts, operations, and results. The whole process was audio-recorded and later transcribed with their agreement by filling out an informed consent form. In compensation for their time, each participant received a 13-USD gifted card for about \yh{one-hour} participation.



\subsection{Data Analysis} 
The {questionnaire} includes personal information background questions, subjective ratings (Closeness to target, Graphical control, Precise refinement, Effect independency, Effect consistency, Effect clearness, Easy to specify action, Mental demand for formulating prompts), and selected questions from NASA-TLX (Figure \ref{fig:sub-compare}). Objective metrics consist of time taken, the number of prompts, and prompt word counts. We conducted Wilcoxon signed-rank tests to analyze significant differences.

\begin{table}[]
    \centering
    \caption{The performance
of MoGraphGPT and Cursor across three metrics averaged over 10 participants.
Note that the time, prompt number, and prompt length are averaged across participants for the total of the three tasks.}
    \label{tab:stat_comparison}
    \begin{tabular}{l|c|c|c}
    \hline
                           & Time (s)   & Prompt \textit{N} &  Prompt \textit{L}  \\ \hline
    Ours                     & 402.40      & 4.80          & 27.70         \\ \hline
    \yh{Cursor}                      & 1339.00  & 17.20         & 269.30        \\ \hline
    Reduction & 69.57\%  & 69.34\%       & 89.21\%       \\ \hline
    \end{tabular}
\end{table}

% \begin{wrapfigure}{r}{0.38\textwidth}
%     \vspace{-3.5mm}
%     \centering
%     \footnotesize
%     \begin{tabular}{l|c|c|c}
%     \hline
%                            & Time (s) \footnotemark[4]  & Prompt \textit{N}\footnotemark[4] &  Prompt \textit{L} \footnotemark[4] \\ \hline
%     Ours                     & 402.40      & 4.80          & 27.70         \\ \hline
%     \yh{Cursor}                      & 1339.00  & 17.20         & 269.30        \\ \hline
%     Reduction & 69.57\%  & 69.34\%       & 89.21\%       \\ \hline
%     \end{tabular}
%     \vspace{-3mm}
% \end{wrapfigure}

% \begin{wrapfigure}{r}{0.35\textwidth}
%     \vspace{-3.5mm}
%     \centering
%     \footnotesize
%     \begin{tabular}{l|c|c|c}
%     \hline
%                              & Time (s) & Prompt \textit{N} & Prompt \textit{L} \\ \hline
%     Ours                     & 310      & 4.12          & 21.75         \\ \hline
%     ChatGPT                      & 1079.88  & 14.88         & 293.25        \\ \hline
%     Reduction & 70.36\%  & 70.73\%       & 91.49\%       \\ \hline
%     \end{tabular}
%     \vspace{-2mm}
% \end{wrapfigure}
% \footnotetext[4]{The time, prompt number, and prompt length are averaged across participants for the total of the three tasks.}

\begin{figure*}[t]
\includegraphics[width=0.99\linewidth]{Figures/comparison_with_cursor.pdf}
  \caption{Subjective ratings on \sysName~and \yh{Cursor Composer.} 
  For the scores, the higher, the better.}
  \label{fig:sub-compare}
\end{figure*}

\subsection{Results}
\textbf{Completion.} 
\yh{All participants successfully completed {each} task in our system within 6 minutes.
% \hbc{402 seconds is more than 6 minutes}\cfc{The table records the total of the three tasks.
% I have added a footnote.} minutes.
However, U7 {spent over 10 minutes on Task1 and U4-6 and U8-10 spent over 10 minutes on Tasks} % and 6 participants \hbc{Why to highlight U7 here?}\yhc{U7 spent over 10min for task1, 6 participants spend over 10min for task2.} (U4-6, U8-10) spent over 10 minutes on Tasks 1 and 2 \hbc{you meant when they failed in Task 1 they failed in Task 2 too?} 
using Cursor but were still unhappy with their results. They found Cursor was hard to handle graphical information, such as specific positions and curved paths. Despite attempts to change descriptions or correct responses, Cursor often retained the original results or produced undesirable effects. 
{Table \ref{tab:stat_comparison} compares the performance of \sysName~with Cursor across three metrics averaged over the 10 participants {for the three tasks}: total completion time {(in seconds)}, prompt number, and prompt length (in words) for all the tasks. We also calculated the reduction rates for these metrics by averaging individual improvements of all the participants with \sysName~compared to Cursor. The results indicate \sysName~achieves desired outcomes in significantly less time and with fewer prompts than Cursor.}}

\textbf{Time.} 
\yh{%Cases over 10 minutes are counted as 10 minutes of using Cursor. 
The average time spent on each task using \sysName~is significantly lower than Cursor{, as confirmed by the Wilcoxon signed-rank test (p<0.01)}: Task1: \yh{M: 46.1s (SD: 12.5s) and M: 268.0s (SD: 134.4s) for \sysName~and Cursor, respectively; 
Task2: M: 152.1s (SD: 73.9s) for \sysName~and M: 448.0s (SD: 110.7s) for Cursor; Task3: M: 204.2s (SD: 79.6s) for \sysName~and M: 623.0s (SD: 161.6s) for Cursor.} {Here, we trim the time for cases over 10 minutes to 10 minutes.} 
%\cfc{The time over 10mins is regarded as 10mins cost.}
Our graphical specification feature enables participants to quickly define positions, moving paths, and both absolute and relative positions quickly. Participants (U1, U4, U6-7, U9-10) were able to adjust motion parameters, such as speed, using sliders and numerical inputs with precision. In contrast, participants using Cursor spent considerable time formulating prompts to integrate graphical information. Some (U1, U4-6, U8) struggled with precise descriptions for several minutes, especially when Cursor continued to produce undesired results. Participants (U1-3, U5, U7-9) often had to try multiple word variations to adjust motion parameters.
In Cursor, adding context does not guarantee independent control, often requiring participants to attempt multiple times and impose additional constraints. In \sysName, individual and interaction behaviors can be created in the earth and central sessions just one trial. 
% \hbc{Maybe swap the order of the previous two sentences to make the connection here smoother?} 
This is because our modular structure provides clear division when updating interactions--defining function code for elements within their individual classes and invoking these functions in the central module. In contrast, Cursor’s context offers only soft constraints, causing interaction code and individual behavior code to become intertwined, leading to inconsistent updates and chaotic modifications.}



\textbf{Prompt numbers and length.}
\yh{Our system results in significantly fewer and shorter prompts, as confirmed by Wilcoxon signed-rank tests (p < 0.01). In Task1 and Task2, the participants used pronouns to refer to each element in element sessions and mentioned the created graphical proxies (e.g., P1, C1) in their prompts. In contrast, accurate mention of element names and graphical details is required in Cursor. 

Participants described the relative positions to the canvas (e.g., left-top, right-bottom, U1-3, U5, U7-9), estimated specific coordinates (U4, U6, U10), and refined results  using reference objects and iterations (e.g.,  set it higher to the corner, make them much closer, U1-2, U4-6, U8-9). They adjusted coordinates through multiple iterations and articulated the curved motion path with terms like "curves with two circles" (U2), "waved curves" (U7), and "tilde" (U1, U3, U9). They further specified the shape with phrases such as "make it more curved" (U2) and "let it curve to the left-bottom and then top-right" (U3).

In Task3, even for the second step, % \hbc{why to say "even" for the second step? This step is easier?} \yhc{yes, it just makes earth self rotate}, 
Cursor often failed to achieve target effects on the first trial. Success typically came only after participants added the earth file as context and retried multiple times. Some participants (U1-2, U5, U8-10) directly inputted text for the third step by instructing the earth to orbit the sun, resulting in the earth orbiting but losing its self-rotation effect. This necessitated rewriting the prompt to include this effect, such as ``let the earth rotate around the sun while also rotating around itself.''  In contrast, with \sysName, participants could input the self-rotation instruction in the earth module and the orbiting effect in the central module independently, requiring less prompt engineering effort.
}  

\yh{
\textbf{Subjective ratings and qualitative feedback.}
% particularly in terms of Q6 - Effect Clearness and Q7 - Easy to Specify Actions. We conducted Wilcoxon signed-rank tests separately for the ratings of each term between ours and the baseline and confirmed the significance of all the aspects (p<0.05) except for Q9 (p=0.07)\yhc{updated later}.
% {It is because some participants (U1, U2) expressed that the graphical control feature of our method required extra effort compared to ChatGPT, which only interacts via texts, though {they acknowledged} the feature provided much more precise control. }
% \textbf{Qualitative feedback.}
% Based on the study
% \hbc{do we have multiple user studies?} 
We analyzed the participants' rating and feedback, distilling our findings into four key aspects.

\emph{User Overall Experience.} 
Overall, subjective ratings and feedback indicate that our system offers a more intuitive, easy-to-use, and effective experience 
% \hbc{I'm not sure if it's appropriate to say the programming experience with our tool} 
compared to Cursor.
% , especially for users with varying programming expertise. 
As shown in Figure \ref{fig:sub-compare}, \sysName~significantly outperformed Cursor across all metrics. {Wilcoxon signed-rank tests confirmed the significance of all aspects (p < 0.05). As shown in Figure \ref{fig:sub-compare} (b),} the participants had a strong preference for \sysName, {particularly in terms of {Q2 - Graphical Control, Q3 - Precise Control, Q7 - Easy to Specify Actions, Q8 - Low Mental Demand, and Q9 - Low Effort Cost (p<0.01).}} 

\emph{Strength of Modularization.} Participants (U2, U4-6, U7, U9) highlighted the modularization of our system as a key advantage over Cursor (Q4 \& Q5). For example, U4 noted, ``the refinement effect for a single element [in our system] is clear (Q6), while in the other tool [Cursor], the modular refinement is ambiguous due to a lack of clear distinction between different elements.''


\emph{Strength of Graphical Control.} Participants appreciated our GUI for its intuitive manipulation of visual elements (Q2). Several participants (U1, U4, U6, U9) were surprised that they could create target spatial effects in just a few seconds using our system (Q7). Even those who redrew curves in Task 2 to better match the target path (U2, U8-9) expressed a willingness to experiment without significant effort (Q8 \& Q9). Most participants noted that relying solely on text to describe shapes made it difficult and cumbersome for Cursor to generate accurate results, leading to frustration (Q10).

\emph{Strength of Precise Control.} The precise control over effect parameters (Q3) in \sysName~was praised by participants. They highlighted its intuitiveness and effectiveness, even for the interaction effects with multiple elements (e.g., orbit radius, orbit speed). In contrast, they found that using text to adjust parameters in Cursor ``does not have a reference'' (U7) and required balancing between excessive and inadequate control (U2, U9).
}



% \begin{figure*}[t]
% \includegraphics[width=0.85\linewidth]{Figures/comparison_mean.pdf}
%   \caption{ \yhc{figure and captions update later}}
%   \label{fig:sub}
% \end{figure*}




\begin{figure*}[t]
\includegraphics[width=0.99\linewidth]{Figures/result.png}
  \caption{\yh{A gallery of selected results in the open-ended study. (a) Two-player Rescue Game (P2). (b) Squirrel Guard Game (P1). (c) Scavenger Hunt Game (P4). (d) Airplane War Game (P4). (e) Sea Fishing Game (P6). (f) Interactive Animation Demo (P3). (g) Website Ad Design Demo (P5). (h) Dynamic Illustration for Academic Paper (P6). Please refer to supplementary materials for detailed descriptions.}}
  \label{fig:result}
\end{figure*}

\section{Open-Ended Study}
To further evaluate the usability and expressiveness of \sysName, we invited participants for an open-ended study, allowing them to create their own desired 2D interactive scenes freely.

\textbf{Participants and Apparatus.}
We recruited 6 participants (aged 25-33, M: 29.5, SD: 2.66, 4 females and 2 males, P1-6), and \yh{three of them} %have 
participated in our comparative study. 
The study was %is 
conducted on %in 
a laptop or a tablet running \sysName, and the participants could %can 
use a keyboard, touchpad, {stylus}, %pen, 
and mouse for inputting and drawing.

\textbf{Procedure.}
Before the study, the participants were %are 
asked to think about %the 
interactive scenes that they would like %want 
to create. This process mainly allowed us to prepare the element images for {those elements requiring image uploading  % requires uploading element 
from our devices}. At the beginning of the study, after a %the 
brief introduction and guidance of our system, the participants %can 
started creating using \sysName. We stayed next to them, %the users, 
answering questions and providing verbal guidance whenever they had doubts. After they finished the creation, they played or showed the created results for us to demonstrate the final scenes.

\textbf{Results.} The %All the 
participants created 10 results in total. Figure \ref{fig:result} shows parts of result snapshots, including 5 games (1 two-player game (Figure \ref{fig:result}(a)) and 5 single-player games (Figure \ref{fig:result}(b)-(e))), 1 interactive animation demo (Figure \ref{fig:result}(f)), 
1 website ad interaction design demo (Figure \ref{fig:result}(g)), and 1 dynamic illustration for an academic paper (Figure \ref{fig:result}(h)). Each result contains 4-8 elements and various types of single element behavior and interactions among multiple elements. Each result was % which is 
completed between 
10-30 minutes, including the creation and testing time. They included different user interactions, such as following the mouse, using the arrow keys to control moving directions, using other keys to control moving speed, and using the mouse click to trigger dynamic effects. Multiple graphical controls {were used}, %are 
%added 
including user-drawn points for specifying target positions, lines for defining curves for moving paths, and regions for active effects. {For example,} the participants were able to define specific points and draw curves, simulating the movement of the sun along a designated path (Figure \ref{fig:result}(f)). They could also create defined areas, such as a {region} %space 
within a tree hollow where squirrels could move freely, with nuts appearing randomly in that region (Figure \ref{fig:result}(b)), enhancing the scene's liveliness. Additionally, users set up clickable regions (Figure \ref{fig:result}(g)), such as an orange that, when clicked, displayed an image of the fruit, fostering engagement and interaction. The results showed that users could create and edit various individual objects while facilitating interactions between them. They utilized text inputs to modify or redefine the behavior of single elements and employed automatically generated sliders to fine-tune specific details. Importantly, the operations on individual objects did not affect others or their interactions, ensuring clarity and control in complex scenarios. Overall, the findings indicate that our \sysName~system effectively supports user creativity and exploration in graphical interactions, making the creative process both enjoyable and intuitive.



\begin{figure*}[t]
\includegraphics[width=0.85\linewidth]{Figures/SUS_6.pdf}
  \caption{SUS score distribution. The question description is the key points from the full SUS questions.}
  \label{fig:sus}
\end{figure*}

The SUS score rated by the participants %users 
is 85 on a 100-point scale, indicating our system has good usability. The distribution of the SUS score for each question is shown in Figure \ref{fig:sus}. We observed that both individuals with programming backgrounds and those without found it easy to use our system to create their desired outcomes. The participants without programming experience (P2, P5-6) expressed a strong need for a tool that does not require coding since the cost of learning programming is quite high. Our system offers an intuitive solution, enabling them to design animations or games of varying complexity for use in their learning and daily life (P2, P6).
% \hbc{I thought the discussion in this sentence is for the participants without programming experience}. 
These users particularly noted that, beyond entertainment, our tool could also be employed to create diagrams and dynamic effects for academic papers (P3, P6). For instance, one participant with an environmental science background (P6) mentioned that generating dynamic visualizations for his papers was often challenging. With our system, he could quickly and easily create effective graphics to illustrate the core concepts of his work (Figure \ref{fig:result}(h)).

The participants with programming experience also found our system very useful. One participant with 8-year programming experience, P3, noted that if she were to create a game using traditional methods, it would take her at least two hours, but with our system, she completed it in just 15 minutes. This significantly saved {her} %him 
time, effort, and the need to learn new programming languages or frameworks, as she could directly generate game code through natural language and graphic controls. The participants (P1, P4) mentioned that our system did not require a technical person to guide them closely; a simple introduction and brief instructions at the start were sufficient for them to get up to speed quickly. 
{P1 particularly enjoyed that our system offers a ``modular architecture''. This capability allows users to navigate complex relationships among objects more effectively. P1 said, ``even for programming simple games, this system is much more easier to use than ChatGPT''. }

However, the participants also highlighted some limitations of our system, such as the currently limited graphic controls (P5-6), which only include points, straight lines, curves, and user-drawn regions. They desired %expressed a desire for 
additional geometric shapes, such as circles, rectangles, and triangles, as well as the ability to automatically segment imported background images. The ability of code generation is also questioned by participants (P2-3) in some aspects including adding unnecessary conditions. For example, in creating Scavenger hunt Game (Figure \ref{fig:result}(c)), when scripting the effect of a key appears and always follows Mario after Mario comes close to the box, it first generates the result that the key only follows Mario when Mario is close to the box. It adds a condition determination on whether Mario collides with the box, while it is unnecessary in the participant's expectation.


\section{Mitigation Strategies}
\label{sec:mitigation}

Mitigating vulnerabilities caused by phantom events requires a comprehensive approach, addressing smart contract development, ecosystem infrastructure, and attack detection mechanisms. From the perspective of contract development, developers should implement strict validation mechanisms to ensure that event parameters are verified before emission and access control mechanisms for the functions. It is essential to enforce proper state transitions to prevent mismatches between emitted events and the actual contract state. 

At the ecosystem level, off-chain systems like blockchain explorers, wallets, and DApps must adopt more robust validation techniques to distinguish legitimate events from phantom events. Event emitter validation, where the source of the event is cross-checked with the contract address, helps ensure that events originate from authorized contracts. Furthermore, improving data sanitization processes in off-chain applications is critical to prevent vulnerabilities such as cross-site scripting (XSS) and SQL injection (SQLi). Enhanced cross-chain security protocols are necessary for cross-chain bridges, ensuring that events on both the source and destination chains are validated to prevent event forgery and manipulation.

In terms of security attack detection, continuous real-time monitoring of on-chain transactions and events is essential to detect and flag suspicious activities, such as \emph{Transfer Event Spoofing} or \emph{Contract Imitation}. Defining detailed detection rules, both for on-chain contract behavior and off-chain event handling, allows for more comprehensive identification of vulnerabilities. Additionally, regular security audits of both smart contracts and off-chain systems should be conducted to identify potential weaknesses, particularly focusing on event emission logic, access controls, and transaction validation. Through a combination of these strategies, the risk posed by phantom events can be significantly reduced, improving the security and reliability of blockchain systems.

\section{Conclusion}\label{sec:conclusion}
In this paper, we present new Prime+Probe and Flush+Reload attacks on the instruction cache created by the SMC detection mechanism in x86 processors. We show that an attacker can implement high-resolution attacks to leak RSA decryption keys with only 10 measurements. Moreover, single-trace attacks are possible with SMC-based attacks on an OpenSSL library. A new version of the Spectre attack is proposed to leak secrets in the memory. Finally, we propose a hardware performance counter-based detection system to detect ongoing SMC-based attacks with an F-score of 0.98 with a minimal performance overhead. The artifacts supporting our experiments are publicly available on our GitHub repository\footnote{\url{https://github.com/hunie-son/SMaCk}}.
%GITHUB






%%
%% The acknowledgments section is defined using the "acks" environment
%% (and NOT an unnumbered section). This ensures the proper
%% identification of the section in the article metadata, and the
%% consistent spelling of the heading.


% \begin{acks}
% To Robert, for the bagels and explaining CMYK and color spaces.
% \end{acks}

%%
%% The next two lines define the bibliography style to be used, and
%% the bibliography file.
\bibliographystyle{ACM-Reference-Format}
\bibliography{Reference}

%%
%% If your work has an appendix, this is the place to put it.

% \appendix

% \section{Research Methods}

% \subsection{Part One}

% Lorem ipsum dolor sit amet, consectetur adipiscing elit. Morbi
% malesuada, quam in pulvinar varius, metus nunc fermentum urna, id
% sollicitudin purus odio sit amet enim. Aliquam ullamcorper eu ipsum
% vel mollis. Curabitur quis dictum nisl. Phasellus vel semper risus, et
% lacinia dolor. Integer ultricies commodo sem nec semper.

% \subsection{Part Two}

% Etiam commodo feugiat nisl pulvinar pellentesque. Etiam auctor sodales
% ligula, non varius nibh pulvinar semper. Suspendisse nec lectus non
% ipsum convallis congue hendrerit vitae sapien. Donec at laoreet
% eros. Vivamus non purus placerat, scelerisque diam eu, cursus
% ante. Etiam aliquam tortor auctor efficitur mattis.

% \section{Online Resources}

% Nam id fermentum dui. Suspendisse sagittis tortor a nulla mollis, in
% pulvinar ex pretium. Sed interdum orci quis metus euismod, et sagittis
% enim maximus. Vestibulum gravida massa ut felis suscipit
% congue. Quisque mattis elit a risus ultrices commodo venenatis eget
% dui. Etiam sagittis eleifend elementum.

% Nam interdum magna at lectus dignissim, ac dignissim lorem
% rhoncus. Maecenas eu arcu ac neque placerat aliquam. Nunc pulvinar
% massa et mattis lacinia.

\end{document}
\endinput
%%
%% End of file `sample-authordraft.tex'.
