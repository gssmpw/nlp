\begin{figure*}[t]
\includegraphics[width=0.99\linewidth]{Figures/result.png}
  \caption{\yh{A gallery of selected results in the open-ended study. (a) Two-player Rescue Game (P2). (b) Squirrel Guard Game (P1). (c) Scavenger Hunt Game (P4). (d) Airplane War Game (P4). (e) Sea Fishing Game (P6). (f) Interactive Animation Demo (P3). (g) Website Ad Design Demo (P5). (h) Dynamic Illustration for Academic Paper (P6). Please refer to supplementary materials for detailed descriptions.}}
  \label{fig:result}
\end{figure*}

\section{Open-Ended Study}
To further evaluate the usability and expressiveness of \sysName, we invited participants for an open-ended study, allowing them to create their own desired 2D interactive scenes freely.

\textbf{Participants and Apparatus.}
We recruited 6 participants (aged 25-33, M: 29.5, SD: 2.66, 4 females and 2 males, P1-6), and \yh{three of them} %have 
participated in our comparative study. 
The study was %is 
conducted on %in 
a laptop or a tablet running \sysName, and the participants could %can 
use a keyboard, touchpad, {stylus}, %pen, 
and mouse for inputting and drawing.

\textbf{Procedure.}
Before the study, the participants were %are 
asked to think about %the 
interactive scenes that they would like %want 
to create. This process mainly allowed us to prepare the element images for {those elements requiring image uploading  % requires uploading element 
from our devices}. At the beginning of the study, after a %the 
brief introduction and guidance of our system, the participants %can 
started creating using \sysName. We stayed next to them, %the users, 
answering questions and providing verbal guidance whenever they had doubts. After they finished the creation, they played or showed the created results for us to demonstrate the final scenes.

\textbf{Results.} The %All the 
participants created 10 results in total. Figure \ref{fig:result} shows parts of result snapshots, including 5 games (1 two-player game (Figure \ref{fig:result}(a)) and 5 single-player games (Figure \ref{fig:result}(b)-(e))), 1 interactive animation demo (Figure \ref{fig:result}(f)), 
1 website ad interaction design demo (Figure \ref{fig:result}(g)), and 1 dynamic illustration for an academic paper (Figure \ref{fig:result}(h)). Each result contains 4-8 elements and various types of single element behavior and interactions among multiple elements. Each result was % which is 
completed between 
10-30 minutes, including the creation and testing time. They included different user interactions, such as following the mouse, using the arrow keys to control moving directions, using other keys to control moving speed, and using the mouse click to trigger dynamic effects. Multiple graphical controls {were used}, %are 
%added 
including user-drawn points for specifying target positions, lines for defining curves for moving paths, and regions for active effects. {For example,} the participants were able to define specific points and draw curves, simulating the movement of the sun along a designated path (Figure \ref{fig:result}(f)). They could also create defined areas, such as a {region} %space 
within a tree hollow where squirrels could move freely, with nuts appearing randomly in that region (Figure \ref{fig:result}(b)), enhancing the scene's liveliness. Additionally, users set up clickable regions (Figure \ref{fig:result}(g)), such as an orange that, when clicked, displayed an image of the fruit, fostering engagement and interaction. The results showed that users could create and edit various individual objects while facilitating interactions between them. They utilized text inputs to modify or redefine the behavior of single elements and employed automatically generated sliders to fine-tune specific details. Importantly, the operations on individual objects did not affect others or their interactions, ensuring clarity and control in complex scenarios. Overall, the findings indicate that our \sysName~system effectively supports user creativity and exploration in graphical interactions, making the creative process both enjoyable and intuitive.



\begin{figure*}[t]
\includegraphics[width=0.85\linewidth]{Figures/SUS_6.pdf}
  \caption{SUS score distribution. The question description is the key points from the full SUS questions.}
  \label{fig:sus}
\end{figure*}

The SUS score rated by the participants %users 
is 85 on a 100-point scale, indicating our system has good usability. The distribution of the SUS score for each question is shown in Figure \ref{fig:sus}. We observed that both individuals with programming backgrounds and those without found it easy to use our system to create their desired outcomes. The participants without programming experience (P2, P5-6) expressed a strong need for a tool that does not require coding since the cost of learning programming is quite high. Our system offers an intuitive solution, enabling them to design animations or games of varying complexity for use in their learning and daily life (P2, P6).
% \hbc{I thought the discussion in this sentence is for the participants without programming experience}. 
These users particularly noted that, beyond entertainment, our tool could also be employed to create diagrams and dynamic effects for academic papers (P3, P6). For instance, one participant with an environmental science background (P6) mentioned that generating dynamic visualizations for his papers was often challenging. With our system, he could quickly and easily create effective graphics to illustrate the core concepts of his work (Figure \ref{fig:result}(h)).

The participants with programming experience also found our system very useful. One participant with 8-year programming experience, P3, noted that if she were to create a game using traditional methods, it would take her at least two hours, but with our system, she completed it in just 15 minutes. This significantly saved {her} %him 
time, effort, and the need to learn new programming languages or frameworks, as she could directly generate game code through natural language and graphic controls. The participants (P1, P4) mentioned that our system did not require a technical person to guide them closely; a simple introduction and brief instructions at the start were sufficient for them to get up to speed quickly. 
{P1 particularly enjoyed that our system offers a ``modular architecture''. This capability allows users to navigate complex relationships among objects more effectively. P1 said, ``even for programming simple games, this system is much more easier to use than ChatGPT''. }

However, the participants also highlighted some limitations of our system, such as the currently limited graphic controls (P5-6), which only include points, straight lines, curves, and user-drawn regions. They desired %expressed a desire for 
additional geometric shapes, such as circles, rectangles, and triangles, as well as the ability to automatically segment imported background images. The ability of code generation is also questioned by participants (P2-3) in some aspects including adding unnecessary conditions. For example, in creating Scavenger hunt Game (Figure \ref{fig:result}(c)), when scripting the effect of a key appears and always follows Mario after Mario comes close to the box, it first generates the result that the key only follows Mario when Mario is close to the box. It adds a condition determination on whether Mario collides with the box, while it is unnecessary in the participant's expectation.

