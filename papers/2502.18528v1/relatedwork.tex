\section{Previous Work}
\label{sec-previouswork}

The use of LLMs in cybersecurity has been extensively discussed by Zhang et al.~\cite{zhang_when_2025}. Applications range from IT operations and Threat Intelligence to vulnerability detection, with more than a dozen efforts specifically focused on leveraging LLMs for assisted attacks.

In~\cite{xu_autoattacker_2024}, authors propose AutoAttacker, an autonomous agent that consists of three modules: a planner, a summarizer, and a navigator. These components have specific prompts, and are leveraged together to perform the instructions. The executed commands are stored in a Retrieval Augmented Generation (RAG) to harness the previous experience in generating new attacks. Various LLMs are used for each component. The evaluation is created for the paper on predefined tasks such as File Uploading or MySQL scan, which lack technical description to be reproducible. Our agent proposes a simpler architecture that does not need a RAG and performs well without a summarizer.

In~\cite{huang_penheal_2023}, authors propose PenHeal, an automatic pentesting agent that also provides remediation information based on the findings. PenHeal has three components: a planner, a summarizer, and an executor. An additional module called extractor is responsible for reading all the attack history and creating remediation strategies for the detected vulnerabilities. PenHeal uses exclusively GPT models. The evaluation is done against a Metasploitable2 Linux virtual machine (2019-08-19 version), although there are no descriptions of the goals given to the LLM agent to solve specific tasks.

In~\cite{muzsai_hacksynth_2024}, authors build on the idea of AutoAttacker and propose HackSynth. The agent implements all the functionality in only two components: a planner and a summarizer. The agent is evaluated against two popular capture-the-flag platforms, TryHackMe and Over the Wire. The agent is tested with a variety of LLM models. The proposed architecture does not allow for the leveraging of the power of specialized models for various tasks, and the summarization is not optional, which affects its performance. Our proposed agent aims to improve this idea by introducing a more modular architecture with more flexibility in choosing models for each task.

Defense mechanisms against these types of automated LLM-driven attacks are already being explored. In~\cite{pasquini_hacking_2024}, the authors propose Mantis, a defensive framework against LLM-driven cyberattacks. Mantis leverages prompt injections as a proactive defense.

%%%%%%%%%%%%%%%%%%%%%%%%%%%%%%%%%%%%%%%%%%%%%%%%%%%%%%%%%%%%%%%%%%%%%%%%%%%%%%%%%%%%%%%%
%%%%%%%%%%%%%%%%%%%%%%%%%%%%%%%%%%%%%%%%%%%%%%%%%%%%%%%%%%%%%%%%%%%%%%%%%%%%%%%%%%%%%%%%