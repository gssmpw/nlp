\section{INTRODUCTION}
\label{intro}
Recommender system is a crucial technology in today's society \cite{meng2024coarse, wang2024future,li2024cdrnp,zhao2023task,dang2024augmenting}, delivering high-quality personalized recommendations based on user preferences. Delving into users’ historical engagements, traditional collaborative filtering (CF) techniques \cite{cfsurvey} learn the representations of users and items for improved recommendations. 
Although somewhat effective, these methods only account for a single type of user-item interaction, limiting their practical effectiveness. In the real world, user behavior is diverse. Beyond target behavior (e.g., \textit{buy}), which is the primary focus for businesses and platforms, users also engage in behaviors such as viewing, adding to cart, and collecting. This behavioral information encompasses different dimensions of user preferences that can be leveraged as auxiliary knowledge to enhance the learning of target behaviors and provide better service for users \cite{nmtr,matn,mbgcn}.

To fully exploit auxiliary behaviors, multi-behavior recommendation methods have emerged as solutions. These methods can be divided into two steps: multi-behavior fusion and multi-behavior prediction \cite{he2023survey, pkef}. In the fusion step, multiple behaviors are combined to learn the representations that capture user preferences. In the prediction step, these representations are applied for model prediction, with multi-task learning (MTL) proving effective \cite{huang2021recent}.

With the explosive development of deep learning, most methods employed for multi-behavior fusion have transitioned from traditional matrix factorization \cite{mf1,mf2,mf3} to deep neural networks \cite{matn,nmtr,dipn}. Among them, graph neural networks (GNNs) \cite{lightgcn,ngcf,lr-gccf,hpmr,ckml} have become popular techniques in multi-behavior recommendation due to their ability to model higher-order user-item interactions effectively \cite{khgt,ghcf,gnmr,pkef}. For example, MBGCN \cite{mbgcn} and CIGF \cite{cigf} learn different behavioral information synchronously and fuse them through learnable parameters. Additionally, some studies \cite{crgcn, mbcgcn, pkef} leverage the cascading dependencies between behaviors in the real world (e.g., \textit{view → cart → buy}) to enhance model learning. BCIPM \cite{bipn} emphasizes the target behavior by strategically modeling different behaviors. These above methods essentially share the same goal: \textit{capturing richer user preferences by deeply exploring the complex user behavior patterns formed by heterogeneous interactions (shown in Figure \ref{fig:behavior_pattern})}. However, previous methods either simply aggregate the behavior representations without constraints, or define strict sequential relationships for the behaviors, resulting in inadequate modeling of user behavior patterns.

In the multi-behavior prediction step, MTL modules are widely utilized \cite{mbgmn, CML, crgcn, bipn} to overcome the limitation of single label \cite{mbgcn, smbrec} in representing diverse user preferences. These modules incorporate auxiliary behavior labels for joint optimization, allowing the model to leverage multi-behavior information for improved accuracy. To better adapt to multi-behavior prediction tasks, CIGF \cite{cigf} enhances traditional MTL models \cite{mmoe, PLE} by further decoupling the inputs of different tasks to mitigate gradient conflicts. PKEF \cite{pkef} builds on this by introducing a projection mechanism during aggregation, preventing the incorporation of harmful information.


\begin{figure}[t]
	\centering
	\setlength{\belowcaptionskip}{0cm}
	\setlength{\abovecaptionskip}{0cm}
	\includegraphics[width=0.49\textwidth]{figures/pattern.pdf}
	\caption{Examples of the user behavior patterns.}
	\label{fig:behavior_pattern}
	\vspace{-4mm}
\end{figure}

Although the above multi-behavior recommendation methods have demonstrated effectiveness in multi-behavior fusion and prediction steps respectively, the following challenges still exist:

\begin{itemize}
\item \textbf{Combination optimization for Multi-Behavior Fusion.} Combination optimization problems typically involve numerous states and choices, necessitating an optimal solution within manageable complexity. Multi-behavior recommendation is essentially a combination optimization problem. For a given behavioral category, each user has a finite number of possible user behavior patterns. The optimal behavioral pattern can be yielded by exhaustively enumerating all possibilities, but this results in significant spatial and temporal costs, known as "combinatorial explosion." Therefore, the challenge lies in leveraging existing knowledge to restrict the solution space and achieve efficient multi-behavior recommendations. Early approaches \cite{lightgcn, smbrec} focused solely on learning behavior-specific information without considering user behavior patterns, which was improved by methods \cite{mbgcn, cigf, bipn} that incorporated behavior aggregation during the learning process. However, they still lacked adequate constraints on user behavior patterns. Some recent methods \cite{crgcn, mbcgcn, pkef} use a cascading paradigm to model user behavioral sequences, resulting in overly strict constraints on user behavior patterns. In conclusion, a paradigm that establishes appropriate constraints to restrict the solution space urgently needs to be proposed.

\item \textbf{Coordination of Correlations between Tasks.} Multi-task learning (MTL) is a commonly used approach for multi-behavior prediction, which models different behaviors as independent tasks. Since each task can affect the final prediction, it is crucial to properly coordinate the correlations between tasks, which is currently limited by two factors: \textbf{1) \textit{Differences in feature space distribution during forward propagation}}: Most existing methods overlook this aspect, leading to biases in information aggregation. While recent approaches have been devoted to mitigating inconsistencies in feature distribution such as the projection-based aggregation mechanism \cite{pkef}, they do not address the dynamics of the representation space during training (details are illustrated in Appendix \ref{shortcoming_pkef}). \textbf{2) \textit{Differences in label space distribution during backward propagation}}: Existing methods learn information from different behaviors but encounter conflicts during gradient updates due to differences in label distributions. Previous methods \cite{cigf, pkef} have attempted to address gradient conflicts using decoupled inputs, with the issue of gradient coupling in aggregation still existing. These factors can lead to negative transfer problems \cite{negativeTransfer}. 
\end{itemize}

To address these two challenges, we propose a \underline{\textbf{C}}ombinatorial \underline{\textbf{O}}ptimization \underline{\textbf{P}}erspective based \underline{\textbf{F}}ramework for Multi-behavior Recommendation (COPF). It consists of the Combinatorial Optimization Graph Convolution Network (COGCN) and the Distributed Fitting Multi-Expert Networks (DFME). To tackle the combinatorial optimization problem in the fusion step, COGCN restricts the solution space of combinatorial optimization by imposing different degrees of constraints to user behavior patterns across various stages (\textit{Pre-behavior, In-behavior, Post-behavior}) based on graph convolutional networks, thus achieving efficient multi-behavior fusion.


To coordinate the correlations between tasks, DFME improves the forward and backward propagation processes during the multi-behavior prediction phase from two perspectives: feature and label. At the feature level, DFME regards different behaviors as independent tasks, and utilizes contrastive learning to adaptively align the distributions of target and auxiliary behaviors. Considering that behavior aggregation can be affected by differences in behavior feature distributions, DFME incorporates a specialized behavior-fitting expert to refine the representation space for each behavior before aggregation, thereby diminishing distribution bias while maintaining spatial generalization. At the label level, DFME further decouples the gradient between the target and auxiliary behaviors during aggregation, avoiding the influence of other tasks on the gradient updates of the target task. The above designs enable the effective utilization of auxiliary tasks to adjust the model-fitted data distribution to be more consistent with the target task’s test distribution, alleviating the negative transfer problem caused by uncoordinated task relationships (explained in Appendix \ref{method_dfme}).

In summary, the main contributions of our works are as follows:
\begin{itemize}
    \item To our knowledge, we are the first to propose examining the multi-behavior fusion problem from a combinatorial optimization perspective. Specifically, we highlight the benefits of behavioral constraints for multi-behavior recommendation and analyze the limitations of existing methods in this perspective, and then propose Combinatorial Optimization Graph Convolutional Network (COGCN) as a solution. It applies different degrees of constraints to user behavior patterns across various stages, effectively facilitating the process of multi-behavior fusion.
    \item We investigate the limitations of multi-task methods in multi-behavior recommendation from structural perspectives (i.e., the feature and label perspectives) and propose Distributed Fitting Multi-Expert Networks (DFME). It is designed to coordinate task correlations by improving the forward and backward propagation processes, thus alleviating the negative transfer problem in MTL.
    \item We conduct comprehensive experiments on three real-world datasets, demonstrating that our proposed COPF has superior performance in multi-behavior recommendation. Further experimental results also verify the rationality and effectiveness of COGCN and DFME.
\end{itemize}




