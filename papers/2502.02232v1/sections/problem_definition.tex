\section{PRELIMINARY}
% \subsection{Problem Definition}
\label{Definition}
In our framework, we use $u$ and $v$ to represent a user and an item, respectively. The user set and item set are denoted as $\mathbf{U}=\{u_1, u_2, \cdots, u_M\}$ and $\mathbf{V}=\{v_1, v_2, \cdots, v_N\}$, where $M$ and $N$ represent the total number of users and items, respectively. The behavior types $k \in \{1,2,...,K\}$ maintain a consistent order among behaviors (i.e., 1 and $K$ correspond to the most upstream and downstream behaviors, respectively). The user-item interaction matrices for the $K$ behavior types can be represented as $\mathcal{B} =\left\{\mathbf{B}_{1},\mathbf{B}_{2},\cdots,\mathbf{B}_{K}\right\}$, where $\mathbf{B}_{k}=\left[b_{(k)uv}\right]_{|\mathbf{U}|\times|\mathbf{V}|}\in \left\{0, 1\right\}$ indicates whether the user $u$ interacts with the item $v$ under behavior $k$. Additionally, we define a user-item bipartite graph $\mathcal{G}=(\mathcal{H}, \mathcal{E}, \mathcal{B})$ to represent the various interaction data between users and items, where $\mathcal{H} = \mathbf{U}\cup\mathbf{V}$, $\mathcal{E} = \cup_{k = 1}^{K}\mathcal{E}_{k}$ is the edge set including all interactions. For multi-behavior recommendation, there is a specific target behavior (e.g., buy) to be optimized, and other behaviors are considered auxiliary behaviors to assist in predicting the target behavior. The target behavior is the most downstream behavior (behavior $K$).

