%%%%%%%% ICML 2025 EXAMPLE LATEX SUBMISSION FILE %%%%%%%%%%%%%%%%%

\documentclass{article}

% Recommended, but optional, packages for figures and better typesetting:
\usepackage{microtype}
\usepackage{graphicx}
\usepackage{subfigure}
\usepackage{booktabs} % for professional tables

% hyperref makes hyperlinks in the resulting PDF.
% If your build breaks (sometimes temporarily if a hyperlink spans a page)
% please comment out the following usepackage line and replace
% \usepackage{icml2025} with \usepackage[nohyperref]{icml2025} above.
\usepackage{hyperref}


% Attempt to make hyperref and algorithmic work together better:
\newcommand{\theHalgorithm}{\arabic{algorithm}}

% Use the following line for the initial blind version submitted for review:
% \usepackage{icml2025}

% If accepted, instead use the following line for the camera-ready submission:
\usepackage[accepted]{icml2025}

% For theorems and such
\usepackage{amsmath}
\usepackage{amssymb}
\usepackage{mathtools}
\usepackage{amsthm}

% if you use cleveref..
\usepackage[capitalize,noabbrev]{cleveref}

%%%%%%%%%%%%%%%%%%%%%%%%%%%%%%%%
% THEOREMS
%%%%%%%%%%%%%%%%%%%%%%%%%%%%%%%%
\theoremstyle{plain}
\newtheorem{theorem}{Theorem}[section]
\newtheorem{proposition}[theorem]{Proposition}
\newtheorem{lemma}[theorem]{Lemma}
\newtheorem{corollary}[theorem]{Corollary}
\theoremstyle{definition}
\newtheorem{definition}[theorem]{Definition}
\newtheorem{assumption}[theorem]{Assumption}
\theoremstyle{remark}
\newtheorem{remark}[theorem]{Remark}

% Todonotes is useful during development; simply uncomment the next line
%    and comment out the line below the next line to turn off comments
%\usepackage[disable,textsize=tiny]{todonotes}
\usepackage[textsize=tiny]{todonotes}

%! ------------------ customized packages added by the authors ------------------
% \usepackage{lipsum}                     % Dummytext
% \usepackage{xargs}                      % Use more than one optional parameter in a new commands
% \usepackage[pdftex,dvipsnames]{xcolor} 

% \usepackage[colorinlistoftodos,prependcaption,textsize=tiny]{todonotes}
% \newcommandx{\unsure}[2][1=]{\todo[linecolor=red,backgroundcolor=red!25,bordercolor=red,#1]{#2}}
% \newcommandx{\change}[2][1=]{\todo[linecolor=blue,backgroundcolor=blue!25,bordercolor=blue,#1]{#2}}
% \newcommandx{\info}[2][1=]{\todo[linecolor=OliveGreen,backgroundcolor=OliveGreen!25,bordercolor=OliveGreen,#1]{#2}}
% \newcommandx{\improvement}[2][1=]{\todo[linecolor=Plum,backgroundcolor=Plum!25,bordercolor=Plum,#1]{#2}}
% \newcommandx{\thiswillnotshow}[2][1=]{\todo[disable,#1]{#2}}

\newcommand{\COMMENTALGO}[1]{\textcolor{olive}{\texttt{\small \% #1}}}

% \usepackage{calc}
% \newlength\myheight
% \newlength\mydepth
% \settototalheight\myheight{Xygp}
% \settodepth\mydepth{Xygp}
% \setlength\fboxsep{0pt}
% \newcommand*\inlinegraphics[1]{%
%   \settototalheight\myheight{Xygp}%
%   \settodepth\mydepth{Xygp}%
%   \raisebox{-\mydepth}{\includegraphics[height=\myheight]{#1}}%
% }

% to use \cdashline
% \usepackage{arydshln}

% % to make the caption to be of the same width as the table
% \usepackage{caption} % Add this in your preamble

\newcommand{\mosa}{\textsc{MoSA}}
\newcommand{\ourmethod}{{\textsc{MoSA}}}

% yafu
\newcommand{\revise}[1]{\textcolor{red}{#1}}

% subfigure
\usepackage{subcaption}

% to show prompt template in Appendix
% \usepackage{tcolorbox}
% \usepackage{listings}
% \lstset{
%     basicstyle=\ttfamily,
%     columns=fullflexible,
%     breaklines=true,
%     frame=single,
%     postbreak=\mbox{\textcolor{red}{$\hookrightarrow$}\space}
% }
\usepackage{tcolorbox}



%! ------------------ END: customized packages added by the authors --------------


% The \icmltitle you define below is probably too long as a header.
% Therefore, a short form for the running title is supplied here:
\icmltitlerunning{Multi-LLM Collaborative Search for Complex Problem Solving}

\begin{document}

\twocolumn[
\icmltitle{Multi-LLM Collaborative Search for Complex Problem Solving}

% It is OKAY to include author information, even for blind
% submissions: the style file will automatically remove it for you
% unless you've provided the [accepted] option to the icml2025
% package.

% List of affiliations: The first argument should be a (short)
% identifier you will use later to specify author affiliations
% Academic affiliations should list Department, University, City, Region, Country
% Industry affiliations should list Company, City, Region, Country

% You can specify symbols, otherwise they are numbered in order.
% Ideally, you should not use this facility. Affiliations will be numbered
% in order of appearance and this is the preferred way.
\icmlsetsymbol{equal}{*}

\begin{icmlauthorlist}
\icmlauthor{Sen Yang}{cuhk}
\icmlauthor{Yafu Li}{shai}
\icmlauthor{Wai Lam}{cuhk}
\icmlauthor{Yu Cheng}{cuhk,shai}
% \icmlauthor{Firstname5 Lastname5}{yyy}
% \icmlauthor{Firstname6 Lastname6}{sch,yyy,comp}
% \icmlauthor{Firstname7 Lastname7}{comp}
%\icmlauthor{}{sch}
% \icmlauthor{Firstname8 Lastname8}{sch}
% \icmlauthor{Firstname8 Lastname8}{yyy,comp}
%\icmlauthor{}{sch}
%\icmlauthor{}{sch}
\end{icmlauthorlist}

\icmlaffiliation{cuhk}{The Chinese University of Hong Kong}

\icmlaffiliation{shai}{Shanghai AI Laboratory}

% \icmlaffiliation{sch}{School of ZZZ, Institute of WWW, Location, Country}

% \icmlcorrespondingauthor{Yu Cheng}{chengyu@cse.cuhk.edu.hk}
% \icmlcorrespondingauthor{Firstname2 Lastname2}{first2.last2@www.uk}

% You may provide any keywords that you
% find helpful for describing your paper; these are used to populate
% the "keywords" metadata in the PDF but will not be shown in the document
\icmlkeywords{Machine Learning, ICML}

\vskip 0.3in
]

% this must go after the closing bracket ] following \twocolumn[ ...

% This command actually creates the footnote in the first column
% listing the affiliations and the copyright notice.
% The command takes one argument, which is text to display at the start of the footnote.
% The \icmlEqualContribution command is standard text for equal contribution.
% Remove it (just {}) if you do not need this facility.

\printAffiliationsAndNotice{}  % leave blank if no need to mention equal contribution
% \printAffiliationsAndNotice{\icmlEqualContribution} % otherwise use the standard text.

\begin{abstract}
Large language models (LLMs) often struggle with complex reasoning tasks due to their limitations in addressing the vast reasoning space and inherent ambiguities of natural language. 
We propose the Mixture-of-Search-Agents (\mosa{}) paradigm, a novel approach leveraging the collective expertise of multiple LLMs to enhance search-based reasoning. 
\mosa{} integrates diverse reasoning pathways by combining independent exploration with iterative refinement among LLMs, mitigating the limitations of single-model approaches. 
Using Monte Carlo Tree Search (MCTS) as a backbone, \mosa{} enables multiple agents to propose and aggregate reasoning steps, resulting in improved accuracy. 
Our comprehensive evaluation across four reasoning benchmarks demonstrates \mosa{}'s consistent performance improvements over single-agent and other multi-agent baselines, particularly in complex mathematical and commonsense reasoning tasks.
\end{abstract}

% -------------------------- Main Body --------------------------------

\section{Introduction}

Deep Reinforcement Learning (DRL) has emerged as a transformative paradigm for solving complex sequential decision-making problems. By enabling autonomous agents to interact with an environment, receive feedback in the form of rewards, and iteratively refine their policies, DRL has demonstrated remarkable success across a diverse range of domains including games (\eg Atari~\citep{mnih2013playing,kaiser2020model}, Go~\citep{silver2018general,silver2017mastering}, and StarCraft II~\citep{vinyals2019grandmaster,vinyals2017starcraft}), robotics~\citep{kalashnikov2018scalable}, communication networks~\citep{feriani2021single}, and finance~\citep{liu2024dynamic}. These successes underscore DRL's capability to surpass traditional rule-based systems, particularly in high-dimensional and dynamically evolving environments.

Despite these advances, a fundamental challenge remains: DRL agents typically rely on deep neural networks, which operate as black-box models, obscuring the rationale behind their decision-making processes. This opacity poses significant barriers to adoption in safety-critical and high-stakes applications, where interpretability is crucial for trust, compliance, and debugging. The lack of transparency in DRL can lead to unreliable decision-making, rendering it unsuitable for domains where explainability is a prerequisite, such as healthcare, autonomous driving, and financial risk assessment.

To address these concerns, the field of Explainable Deep Reinforcement Learning (XRL) has emerged, aiming to develop techniques that enhance the interpretability of DRL policies. XRL seeks to provide insights into an agent’s decision-making process, enabling researchers, practitioners, and end-users to understand, validate, and refine learned policies. By facilitating greater transparency, XRL contributes to the development of safer, more robust, and ethically aligned AI systems.

Furthermore, the increasing integration of Reinforcement Learning (RL) with Large Language Models (LLMs) has placed RL at the forefront of natural language processing (NLP) advancements. Methods such as Reinforcement Learning from Human Feedback (RLHF)~\citep{bai2022training,ouyang2022training} have become essential for aligning LLM outputs with human preferences and ethical guidelines. By treating language generation as a sequential decision-making process, RL-based fine-tuning enables LLMs to optimize for attributes such as factual accuracy, coherence, and user satisfaction, surpassing conventional supervised learning techniques. However, the application of RL in LLM alignment further amplifies the explainability challenge, as the complex interactions between RL updates and neural representations remain poorly understood.

This survey provides a systematic review of explainability methods in DRL, with a particular focus on their integration with LLMs and human-in-the-loop systems. We first introduce fundamental RL concepts and highlight key advances in DRL. We then categorize and analyze existing explanation techniques, encompassing feature-level, state-level, dataset-level, and model-level approaches. Additionally, we discuss methods for evaluating XRL techniques, considering both qualitative and quantitative assessment criteria. Finally, we explore real-world applications of XRL, including policy refinement, adversarial attack mitigation, and emerging challenges in ensuring interpretability in modern AI systems. Through this survey, we aim to provide a comprehensive perspective on the current state of XRL and outline future research directions to advance the development of interpretable and trustworthy DRL models.

\section{\underline{V}ision \underline{L}anguage \underline{D}isinformation Detection \underline{Bench}mark}  
\label{method}  
\textsf{\textbf{\textsc{VLDBench}}} (Figure~\ref{fig:vlbias}) is a comprehensive classification multimodal benchmark for disinformation detection in news articles. It comprises 31,339 articles and visual samples curated from 58 news sources ranging from the Financial Times, CNN, and New York Times to Axios and Wall Street Journal as shown in Figure \ref{fig:news_sources_distribution}. \textsf{\textbf{\textsc{VLDBench}}} spans 13 unique categories (Figure \ref{fig:news_categories}) : \textit{National, Business and Finance, International, Entertainment, Local/Regional, Opinion/Editorial, Health, Sports, Politics, Weather and Environment, Technology, Science, and Other} —adding depth to the disinformation domains. We present the further statistical details in Appendix \ref{app:data-analysis}.
 
\subsection{Task Definition}
\textbf{Disinformation Detection:}  
The core task is to determine whether a text–image news article contains disinformation. We adopt the following definition:  

\emph{“False, misleading, or manipulated information—textual or visual—intentionally created or disseminated to deceive, harm, or influence individuals, groups, or public opinion.”}  

This definition aligns with established social science research \cite{benkler2018network} and governance frameworks \cite{unesco2023journalism}. We specifically focus on the `intent' behind disinformation, which remains relevant over time but has broader effects beyond just being factually incorrect.
\begin{figure*}
    \centering
    \includegraphics[width=0.95\textwidth]{figures/qualitative_figure.pdf}
    \vspace{-1em}
\caption{Disinformation Trends Across News Categories: We analyze the likelihood of disinformation across different categories, based on disinformation narratives and confidence levels generated by GPT-4o.}
    \vspace{-1em}
    \label{fig:disinfo-analysis}
\end{figure*}
\subsection{Data Pipeline}  
\textbf{Dataset Collection:}  
From May 6, 2023, to September 6, 2023, we aggregated data via Google RSS feeds from diverse news sources (Table~\ref{tab:sources}), adhering to Google’s terms of service \cite{google_tos}. We carefully curated  high-quality visual samples from these news sources to ensure a diverse representation of topics. All data collection complied with ethical guidelines \cite{uwaterloo_ethics_review}, regarding intellectual property and privacy protection. 

\textbf{Quality Assurance.}  
Collected articles underwent a rigorous human review and pre-processing phase. First, we removed entries with incomplete text, low-resolution or missing images, duplicates, and media-focused URLs (e.g., \texttt{/video}, \texttt{/gallery}). Articles with fewer than 20 sentences were discarded to ensure textual depth. For each article, the first image was selected to represent the visual context. We periodically reviewed the quality of the curated data to ensure the API returned valid and consistent results. These steps yielded over 31k curated text-image news articles that are moved to the annotation pipeline.

\subsection{Annotation Pipeline}  
To the best of our knowledge, \textsf{\textbf{\textsc{VLDBench}}} is the largest and most comprehensive humanly verified disinformation detection benchmark with over 300 hours of human verification. Figure \ref{fig:vlbias} shows our semi-annotated, data collection and annotation pipeline. After quality assurance, each article was prompted and categorized by GPT-4o as either \texttt{Likely} or \texttt{Unlikely} to contain disinformation, a binary choice designed to balance nuance with manageability. To ensure reliability, GPT-4o assessed text-image alignment three times per sample—first, to minimize random variance in its responses, and second, to resolve potential ties in classification, with an odd number of evaluations ensuring a definitive outcome. GPT-4o was chosen for this task due to its demonstrated effectiveness in both textual \cite{kim2024meganno+} and visual reasoning tasks \cite{shahriar2024putting}. An example is shown in Figure \ref{fig:disinfo-analysis}.

We categorized our data into 13 unique news categories (Figure \ref{fig:news_categories}) by providing image-text pairs to GPT-4o, drawing inspiration from AllSlides \cite{allsides_mediabiaschart}  and frameworks like Media Cloud \cite{media_cloud}. The dataset statistics are given in Table~\ref{tab:dataset_statistics}.

\begin{figure}[ht]
    \centering
    \includegraphics[width=0.48\textwidth]{figures/news_categories_distribution.pdf}
    \caption{Category distribution with overlaps. Total unique articles = 31,339. Percentages sum to \(>100\%\) due to multi-category articles.}
    \label{fig:news_categories}
    \vspace{-1em}
\end{figure}

To ensure high-quality benchmarking, a team of 22 domain experts (Appendix~\ref{app:team}) systematically reviewed the GPT-4o labels and rationales, assessing their accuracy, consistency, and alignment with human judgment. This process included a rigorous structured reconciliation phase, refining annotation guidelines and finalizing the labels. The evaluation resulted in a Cohen’s $\kappa$ of 0.78, indicating strong inter-annotator agreement.


\paragraph{Stability of Automatic Annotations:} To assess the reliability of automated annotations, we conducted a controlled experiment comparing GPT-4o labels with those of human annotators. We randomly selected 1,000 GPT-4o-annotated samples from the previous step, provided annotation guidelines, and asked domain experts (without showing the GPT-4o labels) to manually annotate them. Comparing both sets of labels, GPT-4o achieved an F1 score of 0.89 and an MCC of 0.77, while human annotators scored F1 = 0.92 and MCC = 0.81 (Figure~\ref{fig:alignment_metrics}). These results demonstrate the effectiveness of our semi-annotated pipeline, aligning well with human judgment and ensuring reliable automated labeling.

\begin{table*}[!t]
    \centering
    \resizebox{0.8\textwidth}{!}{
    \begin{tabular}{@{}l|c|c|c|c|c||c@{}}
        \toprule
        & \makecell{MATRES} & \makecell{TB-Dense} & \makecell{TCR} & \makecell{TDD-Manual} & \makecell{NarrativeTime} & \makecell{\textbf{\App{}}} \\
        \midrule
        \multicolumn{7}{c}{\textbf{Datasets Statistics}} \\
        \midrule
        Documents & 275 & 36 & 25 & 34 & 36 & 30 \\
        Events & 6,099 & 1,498 & 1,134 & 1,101 & 1,715 & 470 \\
        \midrule
        \textit{before} & 6,852 (50) & 1,361 (21) & 1,780 (67) & 1,561 (25) & 17,011 (22) & 1,540 (44) \\
        \textit{after} & 4,752 (35) & 1,182 (19) & 862 (33) & 1,054 (17) & 18,366 (23) & 1,347 (39) \\
        \textit{equal} & 448 (4) & 237 (4) & 4 (0) & 140 (2) & 5,298 (7) & 150 (4) \\
        \textit{vague} & 1,525 (11) & 2,837 (45) & -- & -- & 25,679 (33) & 446 (13) \\
        \textit{includes} & -- & 305 (5) & -- & 2,008 (33) & 5,781 (7) & -- \\
        \textit{is-included} & -- & 383 (6) & -- & 1,387 (23) & 6,639 (8) & -- \\
        \textit{overlaps} & -- & -- & -- & -- & 227 (0) & -- \\
        \midrule
        Total Relations & 13,577 & 6,305 & 2,646 & 6,150 & 79,001 & 3,483 \\
        \midrule
        \multicolumn{7}{c}{\textbf{Per Document Average Annotation Sparsity}} \\
        \midrule
        Events & 22.2 & 41.6 & 45.4 & 32.4 & 47.6 & 15.6 \\
        Actual Relations & 49.4 & 183.7 & 105.8 & 180.9 & 1,110.1 & 114.9 \\
        Expected Relations & 234.8 & 844.5 & 1,006.1 & 508.1 & 1,110.1 & 114.9 \\
        \midrule
        Missing Relations & 79\% & 78.3\% & 89.5\% & 64.4\% & 0\% & 0\% \\
        \bottomrule
    \end{tabular}}
    \caption{The upper part of the table presents the statistics of notable datasets for the temporal relation extraction task alongside \App{}. In parentheses, the values indicate the percentage of each relation type relative to the total relations in the dataset. The bottom part of the table summarizes the average percentage of missing relations per document, calculated as the ratio of actual annotated relations to a complete relation coverage, referred to as \textit{Expected Relations}.}
    \label{tab:stats_all}
\end{table*}


% \begin{table*}[!t]
%     \centering
%     \resizebox{0.8\textwidth}{!}{
%     \begin{tabular}{@{}l|c|c|c|c|c|c@{}}
%         \toprule
%         & \makecell{MATRES} & \makecell{TBD} & \makecell{TCR} & \makecell{TDD-Man} & \makecell{NarrativeTime} & \makecell{\App{}} \\
%         \midrule
%         Docs & 275 & 36 & 25 & 34 & 36 & 30 \\
%         Events & 6,099 & 1,498 & 1,134 & 1,101 & 1,715 & 470 \\
%         \midrule
%         Before (\%) & 6,852 (50) & 1,361 (21) & 1,780 (67) & 1,561 (25) & 17,011 (22) & 1,540 (44) \\
%         After (\%) & 4,752 (35) & 1,182 (19) & 862 (33) & 1,054 (17) & 18,366 (23) & 1,347 (39) \\
%         Equal (\%) & 448 (4) & 237 (4) & 4 (0) & 140 (2) & 5,298 (7) & 150 (4) \\
%         Vague (\%) & 1,525 (11) & 2,837 (45) & -- & -- & 25,679 (33) & 446 (13) \\
%         Includes (\%) & -- & 305 (5) & -- & 2,008 (33) & 5,781 (7) & -- \\
%         IsIncluded (\%) & -- & 383 (6) & -- & 1,387 (23) & 6,639 (8) & -- \\
%         Overlaps (\%) & -- & -- & -- & -- & 227 (0) & -- \\
%         \midrule
%         Total Rels & 13,577 & 6,305 & 2,646 & 6,150 & 79,001 & 3,483 \\
%         \bottomrule
%     \end{tabular}}
%     \caption{Statistics of notable datasets for the temporal relation extraction task.}
%     \label{tab:stats}
% \end{table*}



\section{Experiments}
\label{sec:experiments}

\subsection{Next K-mer Prediction}
\label{sec:kmer_predition}
\begin{figure}[t]
    \centering
    \includegraphics[width=0.5\textwidth]{figures/pdf/kmer_prediction_main_text.pdf}
    \caption{Evaluation of next K-mer prediction. (A) Accuracy of the next K-mer prediction task across various tokenizers and input token lengths. (B) Comparison of the \textbf{Gener}\textit{ator} against baseline models on a dataset comprised exclusively mammalian DNA.}
    \label{fig:kmer_main}
\end{figure}

As mentioned in \textit{Sec.} \ref{sec:tokenization}, we conducted extensive experiments to explore the most suitable tokenizer for training causal DNA language models. This was achieved by training multiple models on identical datasets, each employing a different tokenizer. All models share the same architecture as the \textbf{Gener}\textit{ator} and are uniformly compared at 32,000 training steps. We employed the accuracy of the next K-mer prediction task as our evaluation metric. This zero-shot task facilitates a direct assessment of the pre-trained model quality, ensuring equitable comparisons across various tokenizers. As depicted in \textit{Fig.} \ref{fig:kmer_main}A, the tested tokenizers include BPE tokenizers with vocabulary sizes ranging from 512 to 8192, and K-mer tokenizers with K values from 1 to 8 (noting that the single nucleotide tokenizer corresponds to a K-mer tokenizer with K=1). Overall, K-mer tokenizers demonstrate superior performance compared to BPE tokenizers. Among the K-mer tokenizers, the 6-mer tokenizer is selected for its robust performance with limited input tokens and its ability to maintain top-tier performance as the number of input tokens increases.

Moreover, we evaluated the performance of Mamba \cite{Mamba,Mamba-2}, recognized for its capacity in handling long-context pre-training. To adequately assess its capabilities, we configured a Mamba model utilizing the single nucleotide tokenizer with 1.2B parameters and a context length of 98k bp. The Mamba model is compared to the 1-mer and 6-mer models under varied configurations. The comparison with the 1-mer model is straightforward; the Mamba model (denoted as Mamba\texttimes1 in \textit{Fig.} \ref{fig:kmer_main}A) exhibits slightly better performance with fewer input tokens but underperforms as the token count increases. Despite Mamba's context length being six times that of the 1-mer model, this feature does not translate into improved performance. This might suggest that Mamba's renowned ability to handle long-context pre-training primarily refers to cost-effective training rather than enhanced model performance \cite{Empirical, DeciMamba}. To compare against the 6-mer model, we adjust the input token count for the Mamba model by a factor of six (denoted as Mamba\texttimes6) to compare the models on the same base-pair basis. In this context, Mamba\texttimes6 shows slightly better performance with fewer input tokens; however, it rapidly lags as the token count increases. These findings collectively indicate that a transformer decoder architecture paired with a 6-mer tokenizer provides the most effective approach for training causal DNA language models, aligning with the configuration of the \textbf{Gener}\textit{ator}.

We further compared the \textbf{Gener}\textit{ator} model with other baseline models to evaluate their generative capabilities. As illustrated in \textit{Fig.} \ref{fig:kmer_main}B, we assess model performance using a dataset composed exclusively of mammalian DNA, given that HyenaDNA and GROVER are trained solely on human genomes. The \textbf{Gener}\textit{ator} significantly outperforms other baseline models, including its variant, \textbf{Gener}\textit{ator}-All, which incorporates pre-training on non-gene regions. This suggests that the gene sequence training strategy, which emphasizes semantically rich regions, provides a more effective training scheme compared to the conventional whole sequence training. This effectiveness is likely due to the sparsity of gene segments in the whole genome (less than 10\%) and the disproportionate importance of these segments. Among the other baseline models, NT-multi demonstrates the best performance, likely attributable to its extensive model scale (2.5B parameters), yet it still lags significantly behind the \textbf{Gener}\textit{ator}. This result aligns with expectations, as the MLM training paradigm is recognized for its limitations in generative capabilities. Meanwhile, HyenaDNA, despite utilizing the NTP training paradigm, does not show improved performance compared to other masked language models, likely due to its overly small model size (55M parameters), insufficient for exhibiting robust generative abilities. This comparison underscores the critical role of the \textbf{Gener}\textit{ator} in bridging the gap for large-scale generative DNA language models within the eukaryotic domain.

Due to space constraints, we have chosen only to demonstrate specific examples with mammalian DNA data and a fixed K-mer prediction length of 16 bp in \textit{Fig.} \ref{fig:kmer_main}. A more comprehensive analysis across various taxonomic groups and K-mer lengths is provided in the appendix.

\subsection{Benchmark Evaluations}
In this section, we compare the \textbf{Gener}\textit{ator} with state-of-the-art genomic foundation models: Enformer~\cite{enformer}, DNABERT-2, HyenaDNA, Nucleotide Transformer, Caduceus, and GROVER, across various benchmark tasks. To ensure a fair comparison, we uniformly fine-tune each model and perform a 10-fold cross-validation on all datasets. For each model on each dataset, we conduct a hyperparameter search, exhaustively tuning learning rates in $\{1e^{-5}, 2e^{-5}, 5e^{-5}, \ldots, 1e^{-3}, 2e^{-3}, 5e^{-3}\}$ and batch sizes in $\{64, 128, 256, 512\}$. Detailed hyperparameter settings and implementation specifics are provided in the appendix.

\paragraph{Nucleotide Transformer Tasks}
Since the NT task dataset was revised recently~\cite{nucleotide-transformer}, we conducted experiments on both the original and revised datasets. The results for the revised NT tasks are provided in Table~\ref{tab:nucleotide_transformer_tasks_revised}, and the results for the original NT tasks are provided in Table~\ref{tab:nucleotide_transformer_tasks}. Overall, the \textbf{Gener}\textit{ator} outperforms other baseline models. However, the \textbf{Gener}\textit{ator}-All variant shows some performance decline. Notably, despite its earlier release, Enformer continues to deliver competitive results in chromatin profile and regulatory element tasks. This performance could be attributed to its original training in a supervised manner specifically for chromatin and gene expression tasks. The latest release of Nucleotide Transformer, NT-v2, although smaller in size (500M), demonstrates enhanced performance compared to NT-multi (2.5B). In contrast, DNABERT-2 and GROVER, which utilize BPE tokenizers, along with HyenaDNA and Caduceus, which employ the finer-grained single nucleotide tokenizer, do not show distinct performance advantages, likely due to the limited model scope and data scale.

\paragraph{Genomic Benchmarks}
We also conducted a comparative analysis on the Genomic Benchmarks~\cite{genomic-benchmarks}, which primarily focus on the human genome. The evaluation results are provided in Table~\ref{tab:genomic_benchmarks}. Overall, the \textbf{Gener}\textit{ator} still outperforms other models. However, it is worth noting that the Caduceus models also exhibit comparable performance while being significantly smaller (8M). This is likely due to the fact that Caduceus models are trained exclusively on the human genome, making them efficient and compact. Nevertheless, this exclusivity may limit their generalizability to other genomic contexts.

\paragraph{Gener Tasks} 
Lastly, we evaluated the newly proposed Gener tasks, which focus on assessing genomic context comprehension across various sequence lengths and organisms. As shown in Table~\ref{tab:gener_tasks}, the \textbf{Gener}\textit{ator} achieves the best performance on both gene and taxonomic classification tasks, with NT-v2 also demonstrating similar results. Further details on the evaluation of Gener tasks, including visualizations of confusion matrices, are provided in the appendix. The superior performance of the \textbf{Gener}\textit{ator} and NT-v2 is likely due to their pre-training on multispecies datasets. In contrast, despite also being trained on multispecies data, DNABERT-2 exhibits noticeable performance degradation. This may be attributed to its limited model size (117M for DNABERT-2, 500M for NT-v2, and 1.2B for \textbf{Gener}\textit{ator}) and shorter context length (3k for DNABERT-2, 12k for NT-v2, and 98k for \textbf{Gener}\textit{ator}). Other models, such as HyenaDNA and Caduceus, although trained exclusively on the human genome, still exhibit relevant generalizability on both tasks after fine-tuning, attributable to their long-context capacity (\textgreater 100k). GROVER, on the other hand, significantly lags behind in taxonomic classification due to its limited context length (3k).

\begin{table*}[!htb]
\small
\renewcommand{\arraystretch}{1}
\centering
\caption{Evaluation of the revised Nucleotide Transformer tasks. The reported values represent the Matthews correlation coefficient (MCC) averaged over 10-fold cross-validation, with the standard error in parentheses.}
\resizebox{\textwidth}{!}{
\begin{tabular}{lcccccccccc}
\toprule
& Enformer & DNABERT-2 & HyenaDNA & NT-multi & NT-v2 & Caduceus-Ph & Caduceus-PS & GROVER & \textbf{Gener}\textit{ator} & \textbf{Gener}\textit{ator}-All \\
& (252M) & (117M) & (55M) & (2.5B) & (500M) & (8M) & (8M) & (87M) & (1.2B) & (1.2B) \\
\midrule
H2AFZ          & 0.522 (0.019) & 0.490 (0.013) & 0.455 (0.015) & 0.503 (0.010) & \underline{0.524 (0.008)} & 0.417 (0.016) & 0.501 (0.013) & 0.509 (0.013) & \textbf{0.529 (0.009)} & 0.506 (0.019) \\
H3K27ac        & \underline{0.520 (0.015)} & 0.491 (0.010) & 0.423 (0.017) & 0.481 (0.020) & 0.488 (0.013) & 0.464 (0.018) & 0.464 (0.022) & 0.489 (0.023) & \textbf{0.546 (0.015)} & 0.496 (0.014) \\
H3K27me3       & 0.552 (0.007) & 0.599 (0.010) & 0.541 (0.018) & 0.593 (0.016) & \underline{0.610 (0.006)} & 0.547 (0.010) & 0.561 (0.036) & 0.600 (0.008) & \textbf{0.619 (0.008)} & 0.590 (0.014) \\
H3K36me3       & 0.567 (0.017) & \underline{0.637 (0.007)} & 0.543 (0.010) & 0.635 (0.016) & 0.633 (0.015) & 0.543 (0.009) & 0.602 (0.008) & 0.585 (0.008) & \textbf{0.650 (0.006)} & 0.621 (0.013) \\
H3K4me1        & \textbf{0.504 (0.021)} & \underline{0.490 (0.008)} & 0.430 (0.014) & 0.481 (0.012) & \underline{0.490 (0.017)} & 0.411 (0.012) & 0.434 (0.030) & 0.468 (0.011) & \textbf{0.504 (0.010)} & \underline{0.490 (0.016)} \\
H3K4me2        & \textbf{0.626 (0.015)} & 0.558 (0.013) & 0.521 (0.024) & 0.552 (0.022) & 0.552 (0.013) & 0.480 (0.013) & 0.526 (0.035) & 0.558 (0.012) & \underline{0.607 (0.010)} & 0.569 (0.012) \\
H3K4me3        & 0.635 (0.019) & \underline{0.646 (0.008)} & 0.596 (0.015) & 0.618 (0.015) & 0.627 (0.020) & 0.588 (0.020) & 0.611 (0.015) & 0.634 (0.011) & \textbf{0.653 (0.008)} & 0.628 (0.018) \\
H3K9ac         & \textbf{0.593 (0.020)} & 0.564 (0.013) & 0.484 (0.022) & 0.527 (0.017) & 0.551 (0.016) & 0.514 (0.014) & 0.518 (0.018) & 0.531 (0.014) & \underline{0.570 (0.017)} & 0.556 (0.018) \\
H3K9me3        & 0.453 (0.016) & 0.443 (0.025) & 0.375 (0.026) & 0.447 (0.018) & 0.467 (0.044) & 0.435 (0.019) & 0.455 (0.019) & 0.441 (0.017) & \textbf{0.509 (0.013)} & \underline{0.480 (0.037)} \\
H4K20me1       & 0.606 (0.016) & \underline{0.655 (0.011)} & 0.580 (0.009) & 0.650 (0.014) & 0.654 (0.011) & 0.572 (0.012) & 0.590 (0.020) & 0.634 (0.006) & \textbf{0.670 (0.006)} & 0.652 (0.010) \\
Enhancer       & \textbf{0.614 (0.010)} & 0.517 (0.011) & 0.475 (0.006) & 0.527 (0.012) & 0.575 (0.023) & 0.480 (0.008) & 0.490 (0.009) & 0.519 (0.009) & \underline{0.594 (0.013)} & 0.553 (0.020) \\
Enhancer type & \textbf{0.573 (0.013)} & 0.476 (0.009) & 0.441 (0.010) & 0.484 (0.012) & 0.541 (0.013) & 0.461 (0.009) & 0.459 (0.011) & 0.481 (0.009) & \underline{0.547 (0.017)} & 0.510 (0.022) \\
Promoter all   & 0.745 (0.012) & 0.754 (0.009) & 0.693 (0.016) & 0.761 (0.009) & \underline{0.780 (0.012)} & 0.707 (0.017) & 0.722 (0.014) & 0.721 (0.011) & \textbf{0.795 (0.005)} & 0.765 (0.009) \\
Promoter non-TATA & 0.763 (0.012) & 0.769 (0.009) & 0.723 (0.013) & 0.773 (0.010) & 0.785 (0.009) & 0.740 (0.012) & 0.746 (0.009) & 0.739 (0.018) & \textbf{0.801 (0.005)} & \underline{0.786 (0.007)} \\
Promoter TATA  & 0.793 (0.026) & 0.784 (0.036) & 0.648 (0.044) & \underline{0.944 (0.016)} & 0.919 (0.028) & 0.868 (0.023) & 0.853 (0.034) & 0.891 (0.041) & \textbf{0.950 (0.009)} & 0.862 (0.024) \\
Splice acceptor & 0.749 (0.007) & 0.837 (0.006) & 0.815 (0.049) & 0.958 (0.003) & \textbf{0.965 (0.004)} & 0.906 (0.015) & 0.939 (0.012) & 0.812 (0.012) & \underline{0.964 (0.003)} & 0.951 (0.006) \\
Splice site all & 0.739 (0.011) & 0.855 (0.005) & 0.854 (0.053) & 0.964 (0.003) & \textbf{0.968 (0.003)} & 0.941 (0.006) & 0.942 (0.012) & 0.849 (0.015) & \underline{0.966 (0.003)} & 0.959 (0.003) \\
Splice donor   & 0.780 (0.007) & 0.861 (0.004) & 0.943 (0.024) & 0.970 (0.002) & \underline{0.976 (0.003)} & 0.944 (0.026) & 0.964 (0.010) & 0.842 (0.009) & \textbf{0.977 (0.002)} & 0.971 (0.002) \\
\bottomrule
\end{tabular}
}
\label{tab:nucleotide_transformer_tasks_revised}
\end{table*}
\begin{table*}[!htb]
\small
\renewcommand{\arraystretch}{1.2}
\centering
\caption{Evaluation of the original Nucleotide Transformer tasks. The reported values represent the Matthews correlation coefficient (MCC) averaged over 10-fold cross-validation, with the standard error in parentheses.}
\resizebox{\textwidth}{!}{%
\begin{tabular}{lcccccccccc}
\toprule
& Enformer & DNABERT-2 & HyenaDNA & NT-multi & NT-v2 & Caduceus-Ph & Caduceus-PS & GROVER & \textbf{Gener}\textit{ator} & \textbf{Gener}\textit{ator}-All \\
& (252M) & (117M) & (55M) & (2.5B) & (500M) & (8M) & (8M) & (87M) & (1.2B) & (1.2B) \\
\midrule
H3 & 0.724 (0.018) & 0.785 (0.012) & 0.781 (0.015) & 0.793 (0.013) & 0.788 (0.010) & 0.794 (0.012) & 0.772 (0.022) & 0.768 (0.008) & \textbf{0.806 (0.005)} & \underline{0.803 (0.007)} \\
H3K14ac & 0.284 (0.024) & 0.515 (0.009) & \textbf{0.608 (0.020)} & 0.538 (0.009) & 0.538 (0.015) & 0.564 (0.033) & 0.596 (0.038) & 0.548 (0.020) & \underline{0.605 (0.008)} & 0.580 (0.038) \\
H3K36me3 & 0.345 (0.019) & 0.591 (0.005) & 0.614 (0.014) & 0.618 (0.011) & 0.618 (0.015) & 0.590 (0.018) & 0.611 (0.048) & 0.563 (0.017) & \textbf{0.657 (0.007)} & \underline{0.631 (0.013)} \\
H3K4me1 & 0.291 (0.016) & 0.512 (0.008) & 0.512 (0.008) & 0.541 (0.005) & 0.544 (0.009) & 0.468 (0.015) & 0.487 (0.029) & 0.461 (0.018) & \textbf{0.553 (0.009)} & \underline{0.549 (0.018)} \\
H3K4me2 & 0.207 (0.021) & 0.333 (0.013) & \textbf{0.455 (0.028)} & 0.324 (0.014) & 0.302 (0.020) & 0.332 (0.034) & \underline{0.431 (0.016)} & 0.403 (0.042) & 0.424 (0.013) & 0.400 (0.015) \\
H3K4me3 & 0.156 (0.022) & 0.353 (0.021) & \textbf{0.550 (0.015)} & 0.408 (0.011) & 0.437 (0.028) & 0.490 (0.042) & \underline{0.528 (0.033)} & 0.458 (0.022) & 0.512 (0.009) & 0.473 (0.047) \\
H3K79me3 & 0.498 (0.013) & 0.615 (0.010) & 0.669 (0.014) & 0.623 (0.010) & 0.621 (0.012) & 0.641 (0.028) & \textbf{0.682 (0.018)} & 0.626 (0.026) & \underline{0.670 (0.011)} & 0.631 (0.021) \\
H3K9ac & 0.415 (0.020) & 0.545 (0.009) & 0.586 (0.021) & 0.547 (0.011) & 0.567 (0.020) & 0.575 (0.024) & 0.564 (0.018) & 0.581 (0.015) & \textbf{0.612 (0.006)} & \underline{0.603 (0.019)} \\
H4 & 0.735 (0.023) & 0.797 (0.008) & 0.763 (0.012) & \underline{0.808 (0.007)} & 0.795 (0.008) & 0.788 (0.010) & 0.799 (0.010) & 0.769 (0.017) & \textbf{0.815 (0.008)} & \underline{0.808 (0.010)} \\
H4ac & 0.275 (0.022) & 0.465 (0.013) & 0.564 (0.011) & 0.492 (0.014) & 0.502 (0.025) & 0.548 (0.027) & \underline{0.585 (0.018)} & 0.530 (0.017) & \textbf{0.592 (0.015)} & 0.565 (0.035) \\
Enhancer & 0.454 (0.029) & 0.525 (0.026) & 0.520 (0.031) & 0.545 (0.028) & \underline{0.561 (0.029)} & 0.522 (0.024) & 0.511 (0.026) & 0.516 (0.018) & \textbf{0.580 (0.015)} & 0.540 (0.026) \\
Enhancer type & 0.312 (0.043) & 0.423 (0.018) & 0.403 (0.056) & 0.444 (0.022) & 0.444 (0.036) & 0.403 (0.028) & 0.410 (0.026) & 0.433 (0.029) & \textbf{0.477 (0.017)} & \underline{0.463 (0.023)} \\
Promoter all & 0.910 (0.004) & 0.945 (0.003) & 0.919 (0.003) & 0.951 (0.004) & 0.952 (0.002) & 0.937 (0.002) & 0.941 (0.003) & 0.926 (0.004) & \textbf{0.962 (0.002)} & \underline{0.955 (0.002)} \\
Promoter non-TATA & 0.910 (0.006) & 0.944 (0.003) & 0.919 (0.004) & \underline{0.955 (0.003)} & 0.952 (0.003) & 0.935 (0.007) & 0.940 (0.002) & 0.925 (0.006) & \textbf{0.962 (0.001)} & \underline{0.955 (0.002)} \\
Promoter TATA & 0.920 (0.012) & 0.911 (0.011) & 0.881 (0.020) & 0.919 (0.008) & \underline{0.933 (0.009)} & 0.895 (0.010) & 0.903 (0.010) & 0.891 (0.009) & \textbf{0.948 (0.008)} & 0.931 (0.007) \\
Splice acceptor & 0.772 (0.007) & 0.909 (0.004) & 0.935 (0.005) & \underline{0.973 (0.002)} & \underline{0.973 (0.004)} & 0.918 (0.017) & 0.907 (0.015) & 0.912 (0.010) & \textbf{0.981 (0.002)} & 0.957 (0.009) \\
Splice site all & 0.831 (0.012) & 0.950 (0.003) & 0.917 (0.006) & 0.974 (0.004) & \underline{0.975 (0.002)} & 0.935 (0.011) & 0.953 (0.005) & 0.919 (0.005) & \textbf{0.978 (0.001)} & 0.973 (0.002) \\
Splice donor & 0.813 (0.015) & 0.927 (0.003) & 0.894 (0.013) & 0.974 (0.002) & \underline{0.977 (0.007)} & 0.912 (0.009) & 0.930 (0.010) & 0.888 (0.012) & \textbf{0.978 (0.002)} & 0.967 (0.005) \\
\bottomrule
\end{tabular}
}
\label{tab:nucleotide_transformer_tasks}
\end{table*}
\begin{table*}[!htb]
\small
\renewcommand{\arraystretch}{1.2}
\centering
\caption{Evaluation of the Genomic Benchmarks. The reported values represent the accuracy averaged over 10-fold cross-validation, with the standard error in parentheses.}
\resizebox{\textwidth}{!}{
\begin{tabular}{lcccccccc}
\toprule
& DNABERT-2 & HyenaDNA & NT-v2 & Caduceus-Ph & Caduceus-PS & GROVER & \textbf{Gener}\textit{ator} & \textbf{Gener}\textit{ator}-All \\
& (117M) & (55M) & (500M) & (8M) & (8M) & (87M) & (1.2B) & (1.2B) \\
\midrule
Coding vs. Intergenomic & 0.951 (0.002) & 0.902 (0.004) & 0.955 (0.001) & 0.933 (0.001) & 0.944 (0.002) & 0.919 (0.002) & \textbf{0.963 (0.000)} & \underline{0.959 (0.001)} \\
Drosophila Enhancers Stark & 0.774 (0.011) & 0.770 (0.016) & 0.797 (0.009) & \textbf{0.827 (0.010)} & 0.816 (0.015) & 0.761 (0.011) & \underline{0.821 (0.005)} & 0.768 (0.015) \\
Human Enhancers Cohn & \underline{0.758 (0.005)} & 0.725 (0.009) & 0.756 (0.006) & 0.747 (0.003) & 0.749 (0.003) & 0.738 (0.003) & \textbf{0.763 (0.002)} & 0.754 (0.006) \\
Human Enhancers Ensembl & 0.918 (0.003) & 0.901 (0.003) & 0.921 (0.004) & \textbf{0.924 (0.002)} & \underline{0.923 (0.002)} & 0.911 (0.004) & 0.917 (0.002) & 0.912 (0.002) \\
Human Ensembl Regulatory & 0.874 (0.007) & 0.932 (0.001) & \textbf{0.941 (0.001)} & \underline{0.938 (0.004)} & \textbf{0.941 (0.002)} & 0.897 (0.001) & 0.928 (0.001) & 0.926 (0.001) \\
Human non-TATA Promoters & 0.957 (0.008) & 0.894 (0.023) & 0.932 (0.006) & \textbf{0.961 (0.003)} & \textbf{0.961 (0.002)} & 0.950 (0.005) & \underline{0.958 (0.001)} & 0.955 (0.005) \\
Human OCR Ensembl & 0.806 (0.003) & 0.774 (0.004) & 0.813 (0.001) & \underline{0.825 (0.004)} & \textbf{0.826 (0.003)} & 0.789 (0.002) & 0.823 (0.002) & 0.812 (0.003) \\
Human vs. Worm & 0.977 (0.001) & 0.958 (0.004) & 0.976 (0.001) & 0.975 (0.001) & 0.976 (0.001) & 0.966 (0.001) & \textbf{0.980 (0.000)} & \underline{0.978 (0.001)} \\
Mouse Enhancers Ensembl & \underline{0.865 (0.014)} & 0.756 (0.030) & 0.855 (0.018) & 0.788 (0.028) & 0.826 (0.021) & 0.742 (0.025) & \textbf{0.871 (0.015)} & 0.784 (0.027) \\
\bottomrule
\end{tabular}
}
\label{tab:genomic_benchmarks}
\end{table*}
\begin{table*}[!htb]
\small
\renewcommand{\arraystretch}{1}
\centering
\caption{Evaluation of the Gener tasks. The reported values represent the weighted F1 score averaged over 10-fold cross-validation, with the standard error in parentheses.}
\resizebox{\textwidth}{!}{
\begin{tabular}{lcccccccc}
\toprule
& DNABERT-2 & HyenaDNA & NT-v2 & Caduceus-Ph & Caduceus-PS & GROVER & \textbf{Gener}\textit{ator} & \textbf{Gener}\textit{ator}-All \\
& (117M) & (55M) & (500M) & (8M) & (8M) & (87M) & (1.2B) & (1.2B) \\
\midrule
Gene & 0.660 (0.002) & 0.610 (0.007) & \underline{0.692 (0.005)} & 0.629 (0.005) & 0.644 (0.007) & 0.630 (0.003) & \textbf{0.700 (0.002)} & 0.687 (0.003) \\
Taxonomic & 0.922 (0.003) & 0.970 (0.024) & 0.981 (0.001) & 0.958 (0.021) & 0.968 (0.006) & 0.843 (0.006) & \textbf{0.999 (0.000)} & \underline{0.998 (0.001)} \\
\bottomrule
\end{tabular}
}
\label{tab:gener_tasks}
\end{table*}

\subsection{Central Dogma}

In our experimental setup, we selected two target protein families from the UniProt~\cite{UniProt} database: the Histone and Cytochrome P450 families. By cross-referencing gene IDs and protein IDs, we extracted the corresponding protein-coding DNA sequences from RefSeq~\cite{RefSeq}. These sequences served as training data for fine-tuning the \textbf{Gener}\textit{ator} model, directing it to generate analogous protein-coding DNA sequences.

To assess the quality of generation, we compared several summary statistics. The results for the Histone family are provided in \textit{Fig.} \ref{fig:histone_generation}, while the evaluation results for the Cytochrome P450 family are provided in \textit{Fig.} \ref{fig:cytochrome_generation}.  After deduplication, the lengths of the generated DNA sequences and their translated protein sequences, using a codon table, closely resemble the distributions observed in the target families. This preliminary validation suggests that our generated DNA sequences maintain stable codon structures that are translatable into proteins. We conducted a more in-depth structural and functional analysis of these translated protein sequences. First, we assessed whether protein language models `recognize' these generated protein sequences by calculating their perplexity (PPL) using Progen2~\cite{progen2}. The results show that the PPL distribution of generated sequences closely matches that of the natural families and significantly differs from the shuffled sequences.

Furthermore, we used AlphaFold3 to predict the folded structures of the generated protein sequences and employed Foldseek~\cite{Foldseek} to find analogous proteins in the Protein Data Bank (PDB)~\cite{RCSBPDB}. Remarkably, we identified numerous instances where the conformations of the generated sequences exhibited high similarity to established structures in the PDB ($\text{TM-score}>0.8$). This structural congruence is observed despite substantial divergence in sequence composition, as indicated by sequence identities less than $0.3$. This low sequence identity positively suggests that the model is not merely replicating existing protein sequences but has learned the underlying principles to design new molecules with similar structures. This highlights the capability of the \textbf{Gener}\textit{ator} in generating biologically relevant sequences. 

\subsection{Enhancer Design}
We employed the enhancer activity data from DeepSTARR~\cite{DeepSTARR}, following the dataset split initially proposed by DeepSTARR and later adopted by NT. Using this data, we developed an enhancer activity predictor by fine-tuning the \textbf{Gener}\textit{ator}. This predictor surpasses the accuracy of DeepSTARR and NT-multi (Table \ref{tab:enhancer_benchmark}), establishing itself as the current state-of-the-art predictor. By applying our refined SFT approach as outlined in \textit{Sec.} \ref{sec:sequence_design}, we generated a collection of candidate enhancer sequences with specific activity profiles. As illustrated in \textit{Fig.} \ref{fig:enhancer_design}, the predicted activities of these candidates exhibit significant differentiation between the generated high/low-activity enhancers and natural samples.

To our knowledge, this represents one of the first attempts to use LLMs for prompt-guided design of DNA sequences, highlighting the capability of the \textbf{Gener}\textit{ator} in this domain. These generated sequences, and more broadly, this sequence design paradigm using the \textbf{Gener}\textit{ator}, merit further exploration. Our approach underscores the potential of the \textbf{Gener}\textit{ator} model to transform DNA sequence design methodologies, providing a novel pathway for the conditional design of DNA sequences using LLMs. In our subsequent research, we plan to extend our evaluations through further validations in wet lab conditions to explore the real-world applicability of these designed sequences.

\begin{figure}[!htb]
    \centering
    \includegraphics[width=0.6\textwidth]{figures/pdf/histone_generation.pdf}
    \caption{Histone generation. (A) Distribution densities of the protein sequence lengths for generated and natural samples. (B) Distribution densities of Progen2 PPL for generated and natural samples, along with randomly shuffled sequences. (C) Scatter plot of TM-score and AlphaFold3 prediction confidence (pLDDT) with marginal distributions. (D) Two folded structures of generated samples displaying structural congruence with natural samples.}
    \label{fig:histone_generation}
\end{figure}
\begin{figure}[!htb]
    \centering
    \includegraphics[width=0.6\textwidth]{figures/pdf/cytochrome_generation.pdf}
    \caption{Cytochrome P450 generation. (A) Distribution densities of the protein sequence lengths for generated and natural samples. (B) Distribution densities of Progen2 PPL for generated and natural samples, along with randomly shuffled sequences. (C) Scatter plot of TM-score and AlphaFold3 prediction confidence (pLDDT) with marginal distributions. (D) Two folded structures of generated samples displaying structural congruence with natural samples.}
    \label{fig:cytochrome_generation}
\end{figure}

\begin{table}[!htb]
\small
\renewcommand{\arraystretch}{1}
\centering
\caption{Evaluation of the DeepSTARR dataset. The reported values represent the Pearson correlation coefficient.}
\begin{tabular}{lccc}
\toprule
 & DeepSTARR & NT-multi & \textbf{Gener}\textit{ator} \\
\midrule
Developmental & \underline{0.68} & 0.64 & \textbf{0.70} \\
Housekeeping & 0.74 & \underline{0.75} & \textbf{0.79} \\
\bottomrule
\end{tabular}
\label{tab:enhancer_benchmark}
\end{table}
\begin{figure}[!htb]
    \centering
    \includegraphics[width=0.6\textwidth]{figures/pdf/enhancer_design.pdf}
    \caption{Enhancer design. (A-B) Pearson correlation between the predicted enhancer activity and the measured activity. (C-D) Distribution densities of the predicted activity of generated enhancer sequences with distinct activity profiles.}
    \label{fig:enhancer_design}
\end{figure}

\subsection{Analysis}
\label{sec:analysis}
\begin{figure}[t]
\centering
    \includegraphics[width=0.95\columnwidth]{fig/allratio.pdf}
    \caption{Comparison of ratio adherence across different compression ratio settings. The experimental results are obtained with LLaMA-3.1-8B-Instruct on GSM8K.}
    \label{fig:allratio}
\end{figure}

\begin{figure}[t]
\centering
    \includegraphics[width=0.95\columnwidth]{fig/compressor.pdf}
    \caption{Performance comparison of \method using different token importance metrics, evaluated with LLaMA-3.1-8B-Instruct on GSM8K.}
    \label{fig:compressor}
\end{figure}

\paragraph{Compression Ratio} 
In our main results, we focus on compression ratios greater than 0.5. To further investigate the performance of \method at lower compression ratios, we train an additional variant, denoted as \texttt{More Ratio}, with extra compression ratios of 0.3 and 0.4. As shown in Figure~\ref{fig:allratio}, the ratio adherence of models largely degrades at these lower ratios. We attribute this decline to the excessive trimming of reasoning tokens, which likely causes a loss of critical information in the completions, hindering the effective training of LLMs to learn CoT compression. Furthermore, we observe that the overall adherence of \texttt{More Ratio} is not as good as \method with the default settings, which further supports our hypothesis.

\paragraph{Importance Metric} 
Figure~\ref{fig:compressor} presents a performance comparison of \method across different token importance metrics. In addition to the metrics discussed in Section \ref{sec:token-importance}, we include \texttt{GPT-4o}\footnote{We use the \texttt{gpt-4o-2024-08-06} version for experiments.} as a strong token importance metric for comparison. Specifically, for a given CoT trajectory, we prompt \texttt{GPT-4o} to trim redundant tokens according to a specified compression ratio, without adding any additional tokens. Additionally, we ask \texttt{GPT-4o} to suggest the \textit{optimal} compression format of the CoT trajectory, referred to as \texttt{GPT-4o-Optimal} in Figure~\ref{fig:compressor}. We incorporate all training data generated by \texttt{GPT-4o} to train a variant of \method. We use the ``\texttt{[optimal]}'' token to prompt the model, obtaining the results of \texttt{GPT-4o-Optimal}.

As illustrated in Figure~\ref{fig:compressor}, \method utilizing LLMLingua-2~\cite{pan:2024llmlingua2} outperforms the variant with Selective Context~\cite{li:2023selective}, which aligns with our demonstrations in Section \ref{sec:token-importance}. Additionally, incorporating \texttt{GPT-4o} for token importance measurement further enhances compression performance, suggesting that a more robust CoT compressor could improve \method even further. However, the API costs associated with \texttt{GPT-4o} make it impractical for processing large datasets. In contrast, LLMLingua-2, which includes a BERT-size model, offers a cost-effective and efficient alternative for training \method. Furthermore, \texttt{GPT-4o-Optimal} achieves a better balance between reasoning accuracy and CoT token reduction, emphasizing the potential of flexible compression ratios in CoT generation --- an avenue we plan to explore in future work.

\paragraph{Length Budget} 
As outlined in Section~\ref{sec:exp_setup}, we adjust the maximum length budget to \texttt{max\_len}$\times\gamma$ when evaluating \method on MATH-500, ensuring a fair comparison of compression ratios. However, this brute-force length truncation inevitably impacts the reasoning performance of LLMs, as LLMs are unable to complete the full generation. In this analysis, we explore whether LLMs can ``\textit{think}'' more effectively using a compressed CoT format. Specifically, we evaluate \method under the same length budget as the original LLM (e.g., 1024 for MATH-500). The experimental results, shown in Figure~\ref{fig:budget}, demonstrate a significant performance improvement of \method under this length budget, compared to those adjusted by compression ratios. Notably, with compression ratios of 0.7, 0.8, and 0.9, \method outperforms the original LLM, yielding an absolute performance increase of 1.3 to 2.6 points. These findings highlight \method's potential to enhance the reasoning capabilities of LLMs within the same length budget.

\begin{figure}[t]
\centering
    \includegraphics[width=0.95\columnwidth]{fig/budget.pdf}
    \caption{Performance comparison of \method with varying maximum length constraints, evaluated with LLaMA-3.1-8B-Instruct on the MATH-500 dataset.}
    \label{fig:budget}
\end{figure}

\begin{figure*}[t]
\centering
\includegraphics[width=1.0\textwidth]{fig/cases.pdf}
\caption{Three CoT compression examples from \method. For each sample, we list the question, original CoT outputs from corresponding LLMs, and the compressed CoT by \method. The tokens that appear in both the original CoT and the compressed CoT are highlighted in \sethlcolor{pink}\hl{red}.}
\label{fig:cases}
\end{figure*}

\paragraph{Case Study} 
Figure~\ref{fig:cases} presents several examples of \method, derived from the test sets of GSM8K and MATH-500. These examples clearly illustrate that \method allows LLMs to learn shortcuts between critical reasoning tokens, rather than generating shorter CoTs from scratch. For instance, in the first case, \method facilitates LLaMA-3.1-8B-Instruct to skip semantic connectors such as ``\textit{of}'' and ``\textit{the}'', as well as expressions that contribute minimally to the reasoning, such as the first sentence. Notably, we observe that numeric values and mathematical equations are prioritized for retention in most cases. This finding aligns with recent research~\cite{Ma:2024mathmatters}, which suggests that mathematical expressions may contribute more significantly to reasoning than CoT in natural language. Furthermore, we find that \method does not reduce the number of reasoning steps but instead trims redundant tokens within those steps.
\section{Related Work}

\subsection{Large Language Models in Biosciences}
Large language models (LLMs) have emerged as powerful tools for natural language comprehension and generation~\cite{llms-survey}. Beyond their application in traditional natural language tasks, there is a growing interest in leveraging LLMs to accelerate scientific research. Early studies revealed that general-purpose LLMs, owing to their rich pre-training data, exhibit promise across various research domains~\cite{ai4science}. Subsequent efforts have focused on directly training LLMs using domain-specific data, aiming to extend the transfer learning paradigm from natural language processing (NLP) to biosciences. This body of work primarily falls into three categories: molecular LLMs, protein LLMs, and genomic LLMs.

For molecular modeling, extensive work has been conducted on training with various molecular string representations, such as SMILES~\cite{Smiles-bert,space-of-chemical,large-scale-chemical}, SELFIES~\cite{SELFIES,chemberta,chemberta2}, and InChI~\cite{inchi}. Additionally, several studies address the modeling of molecular 2D~\cite{mol-2d} and 3D structures~\cite{uni-mol} to capture more detailed molecular characteristics. In the realm of protein LLMs, related work~\cite{msa-transformer,esm2,Prottrans} mainly concentrates on modeling the primary structure of proteins (amino acid sequences), providing a solid foundation for protein structure prediction~\cite{AlphaFold2,AlphaFold3}. For genomic sequences, numerous studies have attempted to leverage the power of LLMs for improved genomic analysis and understanding. These efforts predominantly involve training models on DNA~\cite{BPNet,DNABERT,enformer,nucleotide-transformer,DNABERT-2,GROVER,gena-lm,Caduceus,dnagpt,megaDNA,HyenaDNA,Evo} and RNA~\cite{RNAErnie,uni-rna,Rinalmo} sequences. In the following section, we delve deeper into genomic LLMs specifically designed for DNA sequence modeling.

\subsection{DNA Language Models}
In the early stages, \citeauthor{BPNet} introduced the BPNet convolutional architecture to learn transcription factor binding patterns and their syntax in a supervised manner. Prior to the emergence of large-scale pre-training, BPNet was widely used in genomics for supervised learning on relatively small datasets. With the advent of BERT~\cite{BERT}, DNABERT~\cite{DNABERT} pioneered the application of pre-training on the human genome using K-mer tokenizers. To effectively capture long-range interactions, Enformer~\cite{enformer} advanced human genome modeling by incorporating convolutional downsampling into transformer architectures.

Following these foundational works, numerous models based on the transformer encoder architecture have emerged. A notable example is the Nucleotide Transformer (NT)~\cite{nucleotide-transformer}, which scales model parameters from 100 million to 2.5 billion and includes a diverse set of multispecies genomes. Recent studies, DNABERT-2~\cite{DNABERT-2} and GROVER~\cite{GROVER}, have investigated optimal tokenizer settings for masked language modeling, concluding that Byte Pair Encoding (BPE) is better suited for masked DNA LLMs. The majority of these models face the limitation of insufficient context length, primarily due to the high computational cost associated with extending the context length in the transformer architecture. To address this limitation, GENA-LM~\cite{gena-lm} employs sparse attention, and Caduceus~\cite{Caduceus} uses the more lightweight BiMamba architecture~\cite{Mamba}, both trained on the human genome.

Although these masked DNA LLMs effectively understand and predict DNA sequences, they lack generative capabilities, and generative DNA LLMs remain in the early stages of development. An early preprint~\cite{dnagpt} introduced DNAGPT, which learns mammalian genomic structures through three pre-training tasks, including next token prediction. Recent works, such as HyenaDNA~\cite{HyenaDNA} and megaDNA~\cite{megaDNA}, achieve longer context lengths by employing the Hyena~\cite{Hyena} and multiscale transformer architectures respectively, though they are significantly limited by their data and model scales. A more recent influential study, Evo~\cite{Evo}, trained on an extensive dataset of prokaryotic and viral genomes, has garnered widespread attention for its success in designing CRISPR-Cas molecular complexes, thus demonstrating the practical utility of generative DNA LLMs in the genomic field.

\section{Conclusion}
In this study, we introduce \ours, a novel framework designed to achieve lossless acceleration in generating ultra-long sequences with \acp{llm}. By analyzing and addressing three challenges, \ours significantly enhances the efficiency of the generation process. Our experimental results demonstrate that \ours achieves over $3\times$ acceleration across various model scales and architectures. Furthermore, \ours effectively mitigates issues related to repetitive content, ensuring the quality and coherence of the generated sequences. These advancements position \ours as a scalable and effective solution for ultra-long sequence generation tasks.





% ----------------------------------------------------------------------


\section*{Impact Statement}
This work aims to contribute to the advancement of reasoning with LLMs. 
While our research could have various societal implications, none are deemed significant enough to warrant specific mention at this stage.
% Authors are \textbf{required} to include a statement of the potential 
% broader impact of their work, including its ethical aspects and future 
% societal consequences. This statement should be in an unnumbered 
% section at the end of the paper (co-located with Acknowledgements -- 
% the two may appear in either order, but both must be before References), 
% and does not count toward the paper page limit. In many cases, where 
% the ethical impacts and expected societal implications are those that 
% are well established when advancing the field of Machine Learning, 
% substantial discussion is not required, and a simple statement such 
% as the following will suffice:

% ``This paper presents work whose goal is to advance the field of 
% Machine Learning. There are many potential societal consequences 
% of our work, none which we feel must be specifically highlighted here.''

% The above statement can be used verbatim in such cases, but we 
% encourage authors to think about whether there is content which does 
% warrant further discussion, as this statement will be apparent if the 
% paper is later flagged for ethics review.


% % In the unusual situation where you want a paper to appear in the
% % references without citing it in the main text, use \nocite
% \nocite{langley00}

\bibliography{main}
\bibliographystyle{icml2025}


%%%%%%%%%%%%%%%%%%%%%%%%%%%%%%%%%%%%%%%%%%%%%%%%%%%%%%%%%%%%%%%%%%%%%%%%%%%%%%%
%%%%%%%%%%%%%%%%%%%%%%%%%%%%%%%%%%%%%%%%%%%%%%%%%%%%%%%%%%%%%%%%%%%%%%%%%%%%%%%
% APPENDIX
%%%%%%%%%%%%%%%%%%%%%%%%%%%%%%%%%%%%%%%%%%%%%%%%%%%%%%%%%%%%%%%%%%%%%%%%%%%%%%%
%%%%%%%%%%%%%%%%%%%%%%%%%%%%%%%%%%%%%%%%%%%%%%%%%%%%%%%%%%%%%%%%%%%%%%%%%%%%%%%
\newpage
\appendix
% \section{You \emph{can} have an appendix here.}

% You can have as much text here as you want. The main body must be at most $8$ pages long.
% For the final version, one more page can be added.
% If you want, you can use an appendix like this one.  

% The $\mathtt{\backslash onecolumn}$ command above can be kept in place if you prefer a one-column appendix, or can be removed if you prefer a two-column appendix.  Apart from this possible change, the style (font size, spacing, margins, page numbering, etc.) should be kept the same as the main body.


\onecolumn

\section{Additional Experimental Settings}
\label{sec:appendix:exp_setting}

\subsection{Hyper-parameters}
The default sampling parameters for LLM generation are \{temperature=0.75, top\_k=40, top\_p=0.95\}.
Across all MCTS experiments, we set the number of rollouts to 8, the number of sub-questions per node to 4, the number of candidate sub-answers per sub-question to 4, the maximum depth allowed to 5.

\subsection{Dataset Statistics}
Since we make use of the rStar code base~\footnote{\url{https://github.com/zhentingqi/rStar/}} to implement \mosa, we directly adopt the data files released in their git repository.
There are 1,319 instances in GSM8K, 1,000 instances in SVAMP, 500 instances in MATH-500, and 687 instances in StrategyQA.

\section{Additional Implementation Details for Aggregators}
\label{sec:appendix:prompt}

In this section, we will show the basic instruction and several in-context learning demonstrations for aggregators.

% The aggregating operation that was introduced in the main content of this paper is mainly conducted within each search step, which is termed as Action Aggregation.
% Apart from it, we also aggregate all search trajectories after the entire search process terminates, which is named as Final Aggregation.
% In this section, we first give an ablation for the two types of aggregations and show that Action Aggregation is more essential than the other.
% Then, we will show the aggregating prompts and several in-context learning demonstrations.

% \subsection{Ablation on Aggregating}

% \begin{table*}[t]
% \caption{
% }
% \label{tab:appendix:action_or_final_agg}
% \vskip 0.15in
% \begin{center}
% \begin{small}
% % \begin{sc}
% \begin{tabular}{l|cccc|cr}
% \toprule
% Method & GSM8K & SVAMP & MATH & StrategyQA & Avg. \\
% \midrule
% \textsc{MoSA} as Proposers \& Aggregators  &  91.96 &  94.90 &  56.60 &  76.42 & 79.97  \\
% \hspace{10pt} $\circ$ \emph{without} Action Aggregation & 91.51 & 95.20 & 54.40 & 75.69 & 79.20 \\
% \hspace{10pt} $\circ$ \emph{without} Final Aggregation & 91.96 & 94.60 & 56.20 & 75.69 & 79.61 \\
% \bottomrule
% \end{tabular}
% % \end{sc}
% \end{small}
% \end{center}
% \vskip -0.1in
% \end{table*}

% As shwon in Table~\ref{tab:appendix:action_or_final_agg}, removing Action Aggregation leads to an average decrease of 0.77\% while that of removing Final Aggregation is only 0.36\%.
% This suggests that aggregation during each search step is more essential than aggregation after the search terminates.

% \subsection{Prompt Templates for Aggregators}

% % \paragraph{Basic Prompt}


% \textbf{Basic Prompt for Action Aggregation}

% \begin{lstlisting}
% **TASK**:\nYou are an intelligent and supportive AI assistant. You will receive a collection of responses from various AI assistants regarding a query. Your goal is to synthesize these responses into a single, high-quality response. You should first write down your thoughts as \"\#\#\# Thoughts: \textless{}THOUGHTS\textgreater{}.\", and then write down the synthesized response in the format of \"\#\#\# Synthesized Response: \textless{}SYNTHESIZED RESPONSE\textgreater{}.\". If \"The answer is: \textless{}ANSWER\textgreater{}\" appears in the set of responses, you should also write down \"The answer is: \textless{}ANSWER\textgreater{}\"; otherwise, do not write it. \n1. Critically assess the information in the responses, as some may be biased, incorrect, or nonsensical. \n2. Your synthesized response should not merely echo the provided answers but should present a refined, accurate, and thorough response to the prompt. \n3.Ensure that your response is well-organized, coherent, and meets the highest standards of accuracy and reliability.\n
% \end{lstlisting}

% \textbf{Basic Prompt for Final Aggregation}

% \begin{lstlisting}
% **TASK**:\nYou are an intelligent and supportive AI assistant. You will receive a collection of responses from various AI assistants regarding a user's query. Your goal is to synthesize these responses into a single, high-quality response. You should write down your thoughts, and then write down the final answer in the format of \"The answer is: \textless{}ANSWER\textgreater{}.\".\n1. Critically assess the information in the collection of responses, as some may be biased, incorrect, or nonsensical. \n2. Your synthesized response should not merely echo the provided responses but should present a refined, accurate, and thorough response to the user's query. \n3.Ensure that your response is well-organized, coherent, and meets the highest standards of accuracy and reliability.\n
% \end{lstlisting}

% \subsection{ICL Demonstrations for Aggregators}

% \textbf{An ICL Demonstration for Action Aggregation}

% \begin{lstlisting}
% \end{lstlisting}

\newpage

\begin{center}
	\small
	\begin{tcolorbox}[width=1\linewidth,title={\textbf{Basic Instruction for Aggregators}}]
		**TASK**:
        
        You are an intelligent and supportive AI assistant. You will receive a collection of responses from various AI assistants regarding a query. Your goal is to synthesize these responses into a single, high-quality response. You should first write down your thoughts as "\#\#\# Thoughts: \textless{}THOUGHTS\textgreater{}.", and then write down the synthesized response in the format of "\#\#\# Synthesized Response: \textless{}SYNTHESIZED RESPONSE\textgreater{}.". If "The answer is: \textless{}ANSWER\textgreater{}" appears in the set of responses, you should also write down "The answer is: \textless{}ANSWER\textgreater{}"; otherwise, do not write it. 
        
        1. Critically assess the information in the responses, as some may be biased, incorrect, or nonsensical. 
        
        2. Your synthesized response should not merely echo the provided answers but should present a refined, accurate, and thorough response to the prompt. 
        
        3.Ensure that your response is well-organized, coherent, and meets the highest standards of accuracy and reliability.
	\end{tcolorbox}
\end{center}

\newpage


\begin{center}
	\small
	\begin{tcolorbox}[width=1\linewidth,title={\textbf{An Example ICL Demonstration of MATH-500 for Aggregators: User Query}}]

\textbf{User:}


\textbf{[Here is the user query that was fed into those AI assistants]:}

Given a question, please decompose it into sub-questions. For each sub-question, please answer it in a complete sentence, ending with "The answer is \textless a numeric answer \textgreater". When the original question is answerable, please start the subquestion with "Now we can answer the question: \textless{}original question\textgreater{}".

Question 5: Find the product $CD$ of the integers $C$ and $D$ for which

\[\frac{C}{x-3}+\frac{D}{x+8}=\frac{4x-23}{x^2+5x-24}\]for all real values of $x$ except $-8$ and $3$.

Question 5.1: What is the equation obtained by multiplying both sides of the given equation by $(x - 3)(x + 8)$?

Answer 5.1: $(x - 3)(x + 8) \cdot \frac{C}{x-3} + (x - 3)(x + 8) \cdot \frac{D}{x+8} = (x - 3)(x + 8) \cdot \frac{4x-23}{x^2+5x-24}$

This simplifies to $C(x + 8) + D(x - 3) = 4x - 23$.

The answer is $C(x + 8) + D(x - 3) = 4x - 23$.

Question 5.2: What are the values of $C$ and $D$ that satisfy the equation $C(x + 8) + D(x - 3) = 4x - 23$ for all $x$ except $-8$ and $3$?

Answer 5.2:


\textbf{[Here are the responses from various AI assistants]:}

Response 1:

To find the values of $C$ and $D$, we can substitute specific values of $x$ into the equation $C(x + 8) + D(x - 3) = 4x - 23$ to obtain a system of equations. 

Question 5.2.1: What happens if we substitute $x = 3$ into the equation $C(x + 8) + D(x - 3) = 4x - 23$?

Answer 5.2.1: Substituting $x = 3$ gives $C(3 + 8) + D(3 - 3) = 4 \cdot 3 - 23$, which simplifies to $11C = -11$. The answer is $11C = -11$.

Question 5.2.2: What happens if we substitute $x = -8$ into the equation $C(x + 8) + D(x - 3) = 4x - 23$?

Answer 5.2.2: Substituting $x = -8$ gives $C(-8 + 8) + D(-8 - 3) = 4 \cdot (-8) - 23$, which simplifies to $-11D = -59$. The answer is $-11D = -59$.

Question 5.2.3: Now we can answer the question: What are the values of $C$ and $D$?

Answer 5.2.3: Solving the system of equations $11C = -11$ and $-11D = -59$ gives $C = -1$ and $D = 7/11 \cdot 11 = 7$. The answer is -1 and 7 respectively.

Response 2:

To find the values of $C$ and $D$, we can compare the coefficients of the corresponding terms on both sides of the equation.

For the $x^2$ terms, we have $C + D = 0$.

For the $x$ terms, we have $8C - 3D = 4$.

Solving these equations simultaneously, we find that $C = 3$ and $D = -3$.

Therefore, the product $CD = 3 \cdot (-3) = \boxed{-9}$.

Response 3:

To find $C$ and $D$, we can substitute $x = -8$ and $x = 3$ into the equation $C(x + 8) + D(x - 3) = 4x - 23$.

For $x = -8$:
\[C(-8 + 8) + D(-8 - 3) = 4(-8) - 23\]
\[-11D = -55\]
\[D = 5\]

For $x = 3$:
\[C(3 + 8) + D(3 - 3) = 4(3) - 23\]
\[11C = -13\]
\[C = -\frac{13}{11}\]

Now we can answer the question: What is the product $CD$ of the integers $C$ and $D$?
\[CD = \left(-\frac{13}{11}\right)(5) = -\frac{65}{11}\]
The answer is $-\frac{65}{11}$.

Response 4:

To find $C$ and $D$, we can set $x = 3$ and $x = -8$ to get two equations:

1. $C(3 + 8) + D(3 - 3) = 4(3) - 23 \Rightarrow 11C = -11 \Rightarrow C = -1$

2. $C(-8 + 8) + D(-8 - 3) = 4(-8) - 23 \Rightarrow -11D = -55 \Rightarrow D = 5$

The answer is $C = -1$ and $D = 5$.


	\end{tcolorbox}
\end{center}



\newpage


\begin{center}
	\small
	\begin{tcolorbox}[width=1\linewidth,title={\textbf{An Example ICL Demonstration of MATH-500 for Aggregators: Assistant Response}}]

\textbf{Assistant:}

\#\#\# Thoughts:

1. The responses generally attempt to solve the equation \(C(x + 8) + D(x - 3) = 4x - 23\) by substituting specific values for \(x\), namely \(x = 3\) and \(x = -8\).

2. The correct method involves creating a system of linear equations by substituting these values and solving for \(C\) and \(D\).

3. There is a discrepancy in the values of \(C\) and \(D\) obtained in the responses:

   - Some responses correctly find \(C = -1\) and \(D = 5\).
   
   - Another response incorrectly finds \(C = -13/11\) and \(D = 5\).
   
   - Another response finds \(C = 3\) and \(D = -3\), which is not consistent with the given equation.

4. The correct values should satisfy both equations derived from the substitutions.

\#\#\# Synthesized Response:
 
 The equation \(C(x + 8) + D(x - 3) = 4x - 23\) can be solved by substituting specific values for \(x\):

1. Substituting \(x = 3\):
\[C(3 + 8) + D(3 - 3) = 4(3) - 23\]
\[11C = -11 \]
\[C = -1\]

2. Substituting \(x = -8\):
\[C(-8 + 8) + D(-8 - 3) = 4(-8) - 23\]
\[-11D = -55 \]
\[D = 5\]

Thus, the values of \(C\) and \(D\) are \(C = -1\) and \(D = 5\). The product \(CD\) is:
\[CD = (-1) \cdot 5 = -5\]

The answer is: -5

	\end{tcolorbox}
\end{center}




\newpage





\begin{center}
	\small
	\begin{tcolorbox}[width=1\linewidth,title={\textbf{An Example ICL Demonstration of StrategyQA for Aggregators: User Query}}]

\textbf{User:}


\textbf{[Here is the user query that was fed into those AI assistants]:}

Given a question, please decompose it into sub-questions. For each sub-question, please answer it in a complete sentence. When the original question is answerable, please start the sub-question with "Now we can answer the question: \textless{}original question\textgreater{}".

Question 5: Johnny Carson: American talk show host and comedian. Could Johnny Carson's children fill out a water polo team?

Question 5.1: How many children did Johnny Carson have?

Answer 5.1: Johnny Carson had three children: two daughters, Catherine and Christie, and a son, Jeff.

Question 5.2: How many players are needed to form a water polo team?

Answer 5.2:

\textbf{[Here are the responses from various AI assistants]:}

Response 1:

A standard water polo team consists of 13 players.

Response 2:

A water polo team consists of seven players, including one goalkeeper and six field players.

Response 3:

A water polo team consists of 13 players: 1 goalkeeper and 12 field players (7 field players and 5 substitutes).

Response 4:

A standard water polo team consists of 13 players, including 7 field players and 6 substitutes.

	\end{tcolorbox}
\end{center}


\newpage



\begin{center}
	\small
	\begin{tcolorbox}[width=1\linewidth,title={\textbf{An Example ICL Demonstration of StrategyQA for Aggregators: Assistant Response}}]

\textbf{Assistant:}

\#\#\# Thoughts:

1. The responses contain valuable information about the composition of a water polo team. 

2. There is a consensus that a water polo team consists of 13 players in total. However, there is some variation in how these 13 players are categorized.

3. The key point to clarify is how many players are in the water at one time and how many are substitutes.

4. It appears that in the water during play, there are 7 players (including the goalkeeper), and the remaining players are substitutes.

\#\#\# Synthesized Response:

The synthesized response is: 

To form a water polo team, a total of 13 players are required. This team includes 7 players who are in the water during play, comprising one goalkeeper and six field players. The remaining 6 players serve as substitutes.

	\end{tcolorbox}
\end{center}



%%%%%%%%%%%%%%%%%%%%%%%%%%%%%%%%%%%%%%%%%%%%%%%%%%%%%%%%%%%%%%%%%%%%%%%%%%%%%%%
%%%%%%%%%%%%%%%%%%%%%%%%%%%%%%%%%%%%%%%%%%%%%%%%%%%%%%%%%%%%%%%%%%%%%%%%%%%%%%%


\end{document}


% This document was modified from the file originally made available by
% Pat Langley and Andrea Danyluk for ICML-2K. This version was created
% by Iain Murray in 2018, and modified by Alexandre Bouchard in
% 2019 and 2021 and by Csaba Szepesvari, Gang Niu and Sivan Sabato in 2022.
% Modified again in 2023 and 2024 by Sivan Sabato and Jonathan Scarlett.
% Previous contributors include Dan Roy, Lise Getoor and Tobias
% Scheffer, which was slightly modified from the 2010 version by
% Thorsten Joachims & Johannes Fuernkranz, slightly modified from the
% 2009 version by Kiri Wagstaff and Sam Roweis's 2008 version, which is
% slightly modified from Prasad Tadepalli's 2007 version which is a
% lightly changed version of the previous year's version by Andrew
% Moore, which was in turn edited from those of Kristian Kersting and
% Codrina Lauth. Alex Smola contributed to the algorithmic style files.
