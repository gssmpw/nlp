%%%%%%%% ICML 2025 EXAMPLE LATEX SUBMISSION FILE %%%%%%%%%%%%%%%%%

\documentclass{article}

% Recommended, but optional, packages for figures and better typesetting:
\usepackage{microtype}
\usepackage{graphicx}
\usepackage{subfigure}
\usepackage{booktabs} % for professional tables

% hyperref makes hyperlinks in the resulting PDF.
% If your build breaks (sometimes temporarily if a hyperlink spans a page)
% please comment out the following usepackage line and replace
% \usepackage{icml2025} with \usepackage[nohyperref]{icml2025} above.
\usepackage{hyperref}


% Attempt to make hyperref and algorithmic work together better:
\newcommand{\theHalgorithm}{\arabic{algorithm}}

% Use the following line for the initial blind version submitted for review:
% \usepackage{icml2025}

% If accepted, instead use the following line for the camera-ready submission:
\usepackage[accepted]{icml2025}

% For theorems and such
\usepackage{amsmath}
\usepackage{amssymb}
\usepackage{mathtools}
\usepackage{amsthm}

% if you use cleveref..
\usepackage[capitalize,noabbrev]{cleveref}

%%%%%%%%%%%%%%%%%%%%%%%%%%%%%%%%
% THEOREMS
%%%%%%%%%%%%%%%%%%%%%%%%%%%%%%%%
\theoremstyle{plain}
\newtheorem{theorem}{Theorem}[section]
\newtheorem{proposition}[theorem]{Proposition}
\newtheorem{lemma}[theorem]{Lemma}
\newtheorem{corollary}[theorem]{Corollary}
\theoremstyle{definition}
\newtheorem{definition}[theorem]{Definition}
\newtheorem{assumption}[theorem]{Assumption}
\theoremstyle{remark}
\newtheorem{remark}[theorem]{Remark}

% Todonotes is useful during development; simply uncomment the next line
%    and comment out the line below the next line to turn off comments
%\usepackage[disable,textsize=tiny]{todonotes}
\usepackage[textsize=tiny]{todonotes}

%! ------------------ customized packages added by the authors ------------------
% \usepackage{lipsum}                     % Dummytext
% \usepackage{xargs}                      % Use more than one optional parameter in a new commands
% \usepackage[pdftex,dvipsnames]{xcolor} 

% \usepackage[colorinlistoftodos,prependcaption,textsize=tiny]{todonotes}
% \newcommandx{\unsure}[2][1=]{\todo[linecolor=red,backgroundcolor=red!25,bordercolor=red,#1]{#2}}
% \newcommandx{\change}[2][1=]{\todo[linecolor=blue,backgroundcolor=blue!25,bordercolor=blue,#1]{#2}}
% \newcommandx{\info}[2][1=]{\todo[linecolor=OliveGreen,backgroundcolor=OliveGreen!25,bordercolor=OliveGreen,#1]{#2}}
% \newcommandx{\improvement}[2][1=]{\todo[linecolor=Plum,backgroundcolor=Plum!25,bordercolor=Plum,#1]{#2}}
% \newcommandx{\thiswillnotshow}[2][1=]{\todo[disable,#1]{#2}}

\newcommand{\COMMENTALGO}[1]{\textcolor{olive}{\texttt{\small \% #1}}}

% \usepackage{calc}
% \newlength\myheight
% \newlength\mydepth
% \settototalheight\myheight{Xygp}
% \settodepth\mydepth{Xygp}
% \setlength\fboxsep{0pt}
% \newcommand*\inlinegraphics[1]{%
%   \settototalheight\myheight{Xygp}%
%   \settodepth\mydepth{Xygp}%
%   \raisebox{-\mydepth}{\includegraphics[height=\myheight]{#1}}%
% }

% to use \cdashline
% \usepackage{arydshln}

% % to make the caption to be of the same width as the table
% \usepackage{caption} % Add this in your preamble

\newcommand{\mosa}{\textsc{MoSA}}
\newcommand{\ourmethod}{{\textsc{MoSA}}}

% yafu
\newcommand{\revise}[1]{\textcolor{red}{#1}}

% subfigure
\usepackage{subcaption}

% to show prompt template in Appendix
% \usepackage{tcolorbox}
% \usepackage{listings}
% \lstset{
%     basicstyle=\ttfamily,
%     columns=fullflexible,
%     breaklines=true,
%     frame=single,
%     postbreak=\mbox{\textcolor{red}{$\hookrightarrow$}\space}
% }
\usepackage{tcolorbox}



%! ------------------ END: customized packages added by the authors --------------


% The \icmltitle you define below is probably too long as a header.
% Therefore, a short form for the running title is supplied here:
\icmltitlerunning{Multi-LLM Collaborative Search for Complex Problem Solving}

\begin{document}

\twocolumn[
\icmltitle{Multi-LLM Collaborative Search for Complex Problem Solving}

% It is OKAY to include author information, even for blind
% submissions: the style file will automatically remove it for you
% unless you've provided the [accepted] option to the icml2025
% package.

% List of affiliations: The first argument should be a (short)
% identifier you will use later to specify author affiliations
% Academic affiliations should list Department, University, City, Region, Country
% Industry affiliations should list Company, City, Region, Country

% You can specify symbols, otherwise they are numbered in order.
% Ideally, you should not use this facility. Affiliations will be numbered
% in order of appearance and this is the preferred way.
\icmlsetsymbol{equal}{*}

\begin{icmlauthorlist}
\icmlauthor{Sen Yang}{cuhk}
\icmlauthor{Yafu Li}{shai}
\icmlauthor{Wai Lam}{cuhk}
\icmlauthor{Yu Cheng}{cuhk,shai}
% \icmlauthor{Firstname5 Lastname5}{yyy}
% \icmlauthor{Firstname6 Lastname6}{sch,yyy,comp}
% \icmlauthor{Firstname7 Lastname7}{comp}
%\icmlauthor{}{sch}
% \icmlauthor{Firstname8 Lastname8}{sch}
% \icmlauthor{Firstname8 Lastname8}{yyy,comp}
%\icmlauthor{}{sch}
%\icmlauthor{}{sch}
\end{icmlauthorlist}

\icmlaffiliation{cuhk}{The Chinese University of Hong Kong}

\icmlaffiliation{shai}{Shanghai AI Laboratory}

% \icmlaffiliation{sch}{School of ZZZ, Institute of WWW, Location, Country}

% \icmlcorrespondingauthor{Yu Cheng}{chengyu@cse.cuhk.edu.hk}
% \icmlcorrespondingauthor{Firstname2 Lastname2}{first2.last2@www.uk}

% You may provide any keywords that you
% find helpful for describing your paper; these are used to populate
% the "keywords" metadata in the PDF but will not be shown in the document
\icmlkeywords{Machine Learning, ICML}

\vskip 0.3in
]

% this must go after the closing bracket ] following \twocolumn[ ...

% This command actually creates the footnote in the first column
% listing the affiliations and the copyright notice.
% The command takes one argument, which is text to display at the start of the footnote.
% The \icmlEqualContribution command is standard text for equal contribution.
% Remove it (just {}) if you do not need this facility.

\printAffiliationsAndNotice{}  % leave blank if no need to mention equal contribution
% \printAffiliationsAndNotice{\icmlEqualContribution} % otherwise use the standard text.

\begin{abstract}
Large language models (LLMs) often struggle with complex reasoning tasks due to their limitations in addressing the vast reasoning space and inherent ambiguities of natural language. 
We propose the Mixture-of-Search-Agents (\mosa{}) paradigm, a novel approach leveraging the collective expertise of multiple LLMs to enhance search-based reasoning. 
\mosa{} integrates diverse reasoning pathways by combining independent exploration with iterative refinement among LLMs, mitigating the limitations of single-model approaches. 
Using Monte Carlo Tree Search (MCTS) as a backbone, \mosa{} enables multiple agents to propose and aggregate reasoning steps, resulting in improved accuracy. 
Our comprehensive evaluation across four reasoning benchmarks demonstrates \mosa{}'s consistent performance improvements over single-agent and other multi-agent baselines, particularly in complex mathematical and commonsense reasoning tasks.
\end{abstract}

% -------------------------- Main Body --------------------------------

\section{Introduction}

Chain-of-Thought (CoT) prompting~\cite{Nye:2021, cot, Kojima:2022cotzero} has emerged as a cornerstone strategy for enhancing Large Language Models (LLMs) in complex reasoning tasks. By eliciting step-by-step inference, CoT enables LLMs to decompose intricate problems into manageable subtasks, thereby improving their problem-solving performance~\cite{Yao:2023tot, Wang:2023self-consistency, Zhou:2023least, Shinn:2023Reflexion}. Recent advancements, such as OpenAI's o1~\cite{o1} and DeepSeek-R1~\cite{deepseekr1}, further demonstrate that scaling up CoT lengths from hundreds to thousands of reasoning steps could continuously improve LLM reasoning. These breakthroughs have underscored CoT’s potential to advance LLM capabilities, expanding the boundaries of AI-driven problem-solving.

\begin{figure}[t]
\centering
    \includegraphics[width=0.95\columnwidth]{fig/intro.pdf}
    \caption{In contrast to vanilla CoT that generates all reasoning tokens sequentially, \method enables LLMs to \textit{skip} tokens with less semantic importance (\textit{e.g.,} \includegraphics[width=7pt]{fig/token.pdf}~) and learn shortcuts between critical reasoning tokens, facilitating controllable CoT compression.}
    \label{fig:intro}
\end{figure}

Despite its effectiveness, the increased length of CoT sequences introduces substantial computational overhead. Due to the autoregressive nature of LLM decoding, longer CoT outputs lead to proportional increases in both inference latency and memory footprints of key-value cache. Additionally, the quadratic computational cost of attention layers further exacerbates this burden. These issues become particularly pronounced when CoT sequences extend into thousands of reasoning steps, resulting in significant computational costs and prolonged response times. While prior research has explored methods for selectively skipping reasoning steps~\cite{Ding:2024cotshortcut, liu2024skipstep}, recent findings~\cite{jin:2024cotlength, Merrill:2024cotlength} suggest that such reductions may conflict with test-time scaling~\cite{o1-blog, snell2025scaling}, ultimately impairing LLM reasoning performance. Therefore, striking an optimal balance between CoT efficiency and reasoning accuracy remains a critical open challenge.

In this work, we delve into CoT efficiency and seek the answer to an important question: \textit{``Does every token in the CoT output contribute equally to deriving the answer?''} We empirically analyze the semantic importance of tokens within CoT outputs and reveal that their contributions to the reasoning performance vary, as depicted in Figure 2. Building on this insight, we introduce \method, a simple yet effective approach that enables LLMs to \textit{skip} less important tokens within CoT sequences and learn shortcuts between critical reasoning tokens, thereby allowing for controllable CoT compression with adjustable ratios. Specifically, as shown in Figure~\ref{fig:intro}, \method constructs compressed CoT training data with various compression ratios, by pruning unimportance tokens from original LLM CoT trajectories. Then, it conducts a general supervised fine-tuning process on target LLMs with this training data, facilitating LLMs to automatically trim redundant tokens during reasoning.

We conduct extensive experiments across various models, including LLaMA-3.1-8B-Instruct and the Qwen2.5-Instruct series, using two widely recognized math reasoning benchmarks: GSM8K and MATH-500. The results validate the effectiveness of \method in compressing CoT outputs while maintaining robust reasoning performance. Notably, Qwen2.5-14B-Instruct exhibits almost \textbf{NO} performance drop (less than $0.4\%$) with a $\bm{40\%}$ reduction in token usage on GSM8K. On the challenging MATH-500 dataset, LLaMA-3.1-8B-Instruct effectively reduces CoT token usage by $\bm{30}\%$ with a performance decline of less than $4\%$, resulting in a $\bm{1.4}\times$ inference speedup. Further analysis underscores the coherence of \method in specified compression ratios and its potential scalability with stronger compression techniques.

\method is distinguished by its low training cost. For Qwen2.5-14B-Instruct, \method fine-tunes only 0.2\% of the model's parameters using LoRA. The size of the compressed CoT training data is no larger than that of the original training set, with 7,473 examples in GSM8K and 7,500 in MATH. The training is completed in approximately 2 hours for the 7B model and 2.5 hours for the 14B model on two 3090 GPUs. These characteristics make \method an efficient and reproducible approach, suitable for use in efficient and cost-effective LLM deployment.

To sum up, our key contributions are:
\begin{enumerate}
    \item To the best of our knowledge, this work is the \textit{first} to investigate the potential of enhancing CoT efficiency through \textit{token skipping}, inspired by the varying semantic importance of tokens in CoT trajectories of LLMs.
    \item We introduce \method, a simple yet effective approach that enables LLMs to skip redundant tokens within CoTs and learn shortcuts between critical tokens, facilitating CoT compression with adjustable ratios.
    \item Our experiments validate the effectiveness of \method. When applied to Qwen2.5-14B-Instruct, \method reduces reasoning tokens by $40\%$ (from 313 to 181) on GSM8K, with less than a $0.4\%$ performance drop.
\end{enumerate}

\section{\name: Modeling Task-driven Eye Movement on Charts}
\label{sec:model}

This section introduces the problem formulation and presents the computational model of eye movement control on charts in settings of analytical tasks.

\subsection{Problem Formulation}

Given a chart image $C$ and an associated analytical task $x$ stated as text, the model is expected to generate a sequence of fixation positions $\{ p_1, p_2, \dots , p_t\}$.
The objective of the output sequence is to closely match the scanpath from humans reading the chart. 
Specifically, the sequence of fixations represents the visual reasoning process, and the information in the patches of pixels fixated upon should be able to support $x$.
We consider general analytical tasks in information visualization~\cite{amar2005low}, and
select three of them used in a human eye-tracking data collection~\cite{polatsek2018exploring}: % \textit{RV}, \textit{F}, and \textit{FE} tasks
\rv{
\begin{itemize}
    \item[1)] \textit{Retrieve value (\textit{RV})}: Given a specific target, find the data value of the target (e.g., what is the value for a certain category?)
    \item[2)] \textit{Filter (\textit{F})}: Given a concrete condition, find which data point satisfies it (e.g., which category has the specific value stated?)
    \item[3)] \textit{Find extreme (\textit{FE})}: Find the data point showing an extreme value for a given attribute within the set of data (e.g., which category shows the highest/lowest value?)
\end{itemize}
}

\subsection{Modeling Overview}

Our goal was to develop the model \name to handle tasks articulated as free-form text and be able to perform gaze movement at a detailed pixel level.
We conceptualize the design of the hierarchical gaze control model in Figure~\ref{fig:model}, where the high-level (cognitive) controller is responsible for reasoning while the low-level (oculomotor) controller determines details of gaze movement. 
The idea behind this is hierarchical supervisory control~\cite{eppe2022intelligent}, which refers to a tiered control system in which the superior controller set goals for its subordinates. The actions from subordinates are integrated into an overall pattern for high-level control~\cite{pew1966acquisition}.
The concept also follows the modeling principle of computational rationality, where we assume that the controllers optimize their policy to maximize expected utility within relevant cognitive bounds~\cite{oulasvirta2022computational,chandramouli2024workflow}.
Specifically, the high-level controller handles abstract information processing, comprehension, and memory storage. 
It sets subtasks to the low-level controller, which then moves the gaze to gather information for task completion. Subsequently, the high-level controller utilizes the amassed information to answer the question.

\begin{figure*}[!t]
\centering
  \includegraphics[width=\textwidth]{Images/h-gaze-control.png}
  \caption{\textbf{An overview of the hierarchical eye-movement control architecture.} When presented with a chart and a task, a cognitive controller, powered by large language models, makes decisions on what to look at next and judges whether it is confident enough to provide an answer to the task's question. It relies on internal memory, which summarizes the information gathered from the chart through eye movements. Once cognitive control has determined the next action, the oculomotor controller is responsible for moving the gaze and observing the chart through a limited vision field. The model's objective is to accurately address the task as quickly as possible within set cognitive and physical constraints.}
  \Description{An overview of the hierarchical eye movement control architecture.}
  \label{fig:model}
\end{figure*}

\subsection{Cognitive Control}

The high-level controller provides cognitive control over the mental processes for a chart, control that performs reasoning in working memory~\cite{liu2010mental}. When performing vision tasks, one observes and analyzes visual information interactively~\cite{chen2020air}. Throughout this process, people analyze the information in their memory and try to gather more useful information to reduce uncertainty in solving the task.
To represent this decision problem accurately, we formulate it as a bounded optimality problem in a partially observable Markov decision process (POMDP). Instead of having access to a full state ($\mathcal{S}$) with pixels of the chart associated with the given task, the POMDP expresses a subset of ($\mathcal{S}$) as the observation of the model:
\begin{itemize}
    \item Observation $O$ refers to the information in memory that is captured from eye movements over the chart.
    \item Action $A$ includes subtasks that the model gives to oculomotor control for performing eye movements.
    \item Reward $R$ is the correctness of the answer for the task from the chart question answering.
\end{itemize}
To solve this POMDP, our model uses LLMs for the policy. The rationale behind this choice is that LLMs are well suited to processing higher-level information, as they have been pre-trained on human text data encompassing a wealth of logic related to planning, reasoning, and interaction~\cite{huang2022language, vemprala2024chatgpt, li2023interactive}.
Although LLMs are limited in their ability to control low-level motor functions in a precise manner~\cite{dalal2024psl}, they are proficient at planning and reasoning, with LLaMA~\cite{touvron2023llama} and GPT~\cite{achiam2023gpt} showing impressive language interpretation and reasoning capabilities.
Also, recent work has shown that utilizing LLMs in the high-level controllers in hierarchical architecture can produce promising results~\cite{huang2022language, brohan2023can, liang2023code}.
For our setting, we used GPT-4o~\cite{achiam2023gpt} for the policy, which takes the information accumulated in the memory as the observation and sets subtasks to guide eye movements in order to obtain information needed for solving the task efficiently.

We consider two human limitations when constructing the model's observation: a limited field of vision~\cite{duchowski2018gaze} and memory capacity~\cite{loftus2019human}. 
The model gets information from the gaze position purely by mimicking the human vision system. 
An optical character recognition technique~\cite{singh2010optical} is used to extract text from the pixels of the chart, and the text in the gaze area, with the position, is passed to the memory.
As a result, the observation consists of image patches (in a limited number) from the full set of chart pixels. The reliability of items in memory is determined by their visit history~\cite{li2023modeling}, with overall memory capacity being restricted too. When new information is added to the memory, a previously added item is removed on the basis of a forgetting probability. The probability of forgetting an item in the memory is calculated by means of the formula $\text{Softmax}(\rho \cdot (t-t_i)) $, where $t$ is the current fixation index, $t_i$ is the index of the $i$th item in the memory, and $\rho$ is the weight parameter (set to 0.1 here). The observation is designed as a prompt that summarizes the memory in line with the memory model and explains the model's goal. 

Given the summary of the memory information, the LLM policy selects predefined operations for task solving~\cite{brohan2023can, liang2023code}. The operations here are based on a sequence of cognitive stages for charts~\cite{goldberg2011eye} -- 1) \textit{search for text label}: visually searching for a text label or value label related to the task, 2) \textit{find associated mark}: visually searching for a graphical mark of the data point when given a reference label, 3) \textit{read associated value}: visually searching to read the given mark's associated value or textual label.
All these actions are allowed to be reused in the process, which enables the model to revisit previous positions for confirmation of the information.
Ultimately, if the information in the memory is sufficient to address the task, the gaze movement can stop and an answer can be given. Operations other than answering the question will be performed by the oculomotor controller for detailed gaze movement. 

The examples in Figure~\ref{fig:memory} demonstrate how utilizing memory information and predefined operations aids in scanpath prediction. Model memory uses the summarization capability of LLMs to convert the text and positions gathered to a paragraph as the observation (as shown in the green boxes). The LLM policy then makes decisions and issues subtasks as actions (in red boxes) for the oculomotor control, which performs pixel-level gaze movements.

\begin{figure*}[!t]
\centering
  \includegraphics[width=\textwidth]{Images/scanpath-memory.png}
  \caption{The figure gives examples of how the internal memory helps the cognitive controller to remember what has been read and then select actions for detailed gaze movement. A green box indicates the information held in memory, a red box represents the action selected by cognitive control, and the blue lines in the images reflect the eye movement scanpaths.}
  \label{fig:memory}
\end{figure*}

\subsection{Oculomotor Control}

The oculomotor controller acts as the interface between the cognitive controller and the actual chart-pixel images. Its main function is to control the movement of the gaze over the pixels in order to gather information related to the task at hand.
Generating oculomotor behavior at pixel level is another sequential decision-making problem that can be formulated as a POMDP:
\begin{itemize}
    \item Observation $o$ comprises vision information obtained from the external environment, which is jointly represented by the human vision system and visual short-term memory (VSTM).
    \item Action $a$  involves specifying the coordinates $(x, y)$ of a particular position to move to.
    \item Reward $r$ is designed to encourage the gaze to reach the target with less cost. It takes into account the number of target hits as well as the cost associated with the distance of the gaze movement.
\end{itemize}

Our modeling of a chart reader's observation follows an idea similar to that in visual search~\cite{yang2020predicting}. Utilizing a representation for accumulating information through fixations, this employs four components: 
1) The foveal and peripheral view come from the human vision system, which receives high-resolution visual input only from the region of the image around the fixation location. It includes two pixel-based modules to read the chart: foveal and peripheral vision~\cite{duchowski2018gaze}). 
2) Visual saliency provides a bottom-up signal to a chart reader for the given task. The saliency of the chart affects gaze behavior. We use a task-driven saliency model to represent this feature ~\cite{wang2024salchartqa}.
3)  Visit history represents VSTM, which stores visual information for a few seconds, thereby allowing its use in ongoing cognitive tasks~\cite{alvarez2004capacity}. We represent this history through a matrix where each point is marked as visited or not.
4) A goal-related reference position serves as the initial starting point of gaze movement. For example, the reader might begin at the position of a text label for locating the associated graphical mark, where the position of the text label serves as the reference for the sub-goal. 
\rv{We use a one-hot matrix to represent the reference, in which all cell values are 0 apart from the single 1 that identifies the target.}
All these components are encoded together via the deep convolutional neural network, followed by a fully connected network.

We train reinforcement learning policies to solve the POMDP for the oculomotor control, because it has been proven to effectively address decision-making challenges in prediction of details of gaze movement~\cite{yang2020predicting, jiang2024eyeformer, shi2024crtypist, bai2024heads}.
In our detail-level implementation, we resize the input chart images to be $320 \times 320$ and discretize the fixation position into a $20 \times 20$ map. Consequently, each fixation becomes a $16 \times 16$ image patch, and the gaze position is randomly sampled from within that patch. In this setup, the maximum approximation error resulting from this discretization process is less than one degree of the visual angle~\cite{yang2020predicting}.
\rv{
Ultimately, both the scanpath and the image will be converted back to the original chart size from $320 \times 320$ pixels.
}

\subsection{\rv{Workflow}}
\label{sec:workflow}

\rv{
Our implementation of \name is trained and tested on a collection of tasks and charts. There are four steps, illustrated in Figure~\ref{fig:pipeline}.
In Step 1, real-world charts are manually collected and labeled for areas of interest (AOIs), while synthetic charts are automatically generated and labeled in a manner powered by Vega-Lite~\cite{satyanarayan2016vega}. The inclusion of synthetic charts helps increase the diversity of the chart collection and addresses the challenge of obtaining numerous annotated charts.
In Step 2, tasks are automatically generated in line with specific rules for the \textit{RV}, \textit{F}, and \textit{FE} tasks. These tasks and labeled charts constitute a data collection for the training environment.t
With Step 3, the policies for oculomotor control are trained through reinforcement learning (using proximal policy pptimization, PPO~\cite{schulman2017proximal}) to optimize gaze movements, enabling the system to reach task-relevant positions as quickly as possible while adhering to vision constraints. Importantly, no eye tracking data are required for PPO training.
In the last phase, prediction, the hierarchical architecture combines pre-trained LLMs (GPT-4o) for cognitive control with RL policies for oculomotor control to generate the scanpath prediction.
}

\begin{figure*}[!h]
\centering
  \includegraphics[width=\textwidth]{Images/pipeline.png}
  \caption{\rv{An overview of the training workflow: 1) chart collection and labeling, wherein diverse real-world and synthetic charts are gathered, involving manual and automatic annotation of AOIs; 2) task generation, utilizing a rule-based approach to create tasks based on labeled charts to construct a data collection for training; 3) policy training, in which policy models are trained via RL from chart images with tasks; and 4) scanpath prediction, wherein pre-trained LLMs and RL policies are coordinated hierarchically to predict task-driven gaze movements over charts.}}
  \label{fig:pipeline}
  % \vspace{-10mm}
\end{figure*}
\section{Experiment}\label{sec: exp}
In this section, we assess the efficacy of our algorithm by addressing the following key questions. 
(1) Can offline RL algorithms achieve stronger performance on the reduced datasets selected by~\name?
(2) How does \name~perform compare to other offline data selection methods? 
(3) What are the factors that contribute to \name's effectiveness?

\begin{figure}[t]
    \centering
    \subfigure{\includegraphics[scale=0.24]{d4rl-hard/walker2d-medium-v0-hard.pdf}}
    \hspace{0.2cm}
    \subfigure{\includegraphics[scale=0.24]{d4rl-hard/walker2d-expert-v0-hard.pdf}}
    \hspace{0.2cm}
    \subfigure{\includegraphics[scale=0.24]{d4rl-hard/walker2d-medium-replay-v0-hard.pdf}}
    % \subfigure{\includegraphics[scale=0.20]{d4rl-hard/walker2d-medium-expert-v0-hard.pdf}}
    \subfigure{\includegraphics[scale=0.24]{d4rl-hard/hopper-medium-v0-hard.pdf}}
    \hspace{0.2cm}
    \subfigure{\includegraphics[scale=0.24]{d4rl-hard/hopper-expert-v0-hard.pdf}}
    \hspace{0.2cm}
    \subfigure{\includegraphics[scale=0.24]{d4rl-hard/hopper-medium-replay-v0-hard.pdf}}
    % \subfigure{\includegraphics[scale=0.20]{d4rl-hard/hopper-medium-expert-v0-hard.pdf}}
    \subfigure{\includegraphics[scale=0.24]{d4rl-hard/halfcheetah-medium-expert-v0-hard.pdf}}
    \hspace{0.2cm}
    \subfigure{\includegraphics[scale=0.24]{d4rl-hard/halfcheetah-expert-v0-hard.pdf}}
    \hspace{0.2cm}
    \subfigure{\includegraphics[scale=0.24]{d4rl-hard/halfcheetah-medium-replay-v0-hard.pdf}}
    % \subfigure{\includegraphics[scale=0.20]{d4rl-hard/halfcheetah-medium-v0-hard.pdf}}
    \caption{Experimental results on the D4RL (Hard) offline datasets. All experiment results were averaged over five random seeds. Our method achieves better or
    comparable results than the baselines with lower computational complexity.}
    \label{fig: d4rl hard}
    \vspace{-0.5cm}
\end{figure}

% \begin{figure*}[t]
%     \centering
%     \includegraphics[width=\linewidth]{mujoco/fig1.pdf}
%     \vspace{-2em}
%     \caption{Sample-based selection performance of several baselines and \name~with different selected subset sizes~($x\%$).
%     The horizontal line is the performance of TD3+BC trained with the original dataset.}
%     \label{fig: d4rl minimal ratio}
%     \vspace{-1em}
% \end{figure*}

% \begin{figure}[t]
%     \centering
%     \includegraphics[width=\linewidth]{mujoco/traj.pdf}
%     \caption{In trajectory-based selection, \name~outperforms behavior cloning (\nameh) using trajectories with the highest accumulative returns, presenting a robust method for selecting the most useful data from training sets of compromised quality.}
%     \label{fig: d4rl topbc}
%     \vspace{-1em}
% \end{figure}

\subsection{Setup}
We evaluate algorithms on the offline RL benchmark D4RL~\citep{fu2020d4rl} to answer the aforementioned questions.
In addition, we consider a more challenging scenario where we add additional low-quality data to the dataset to simulate noise in real-world tasks, named D4RL~(hard).
The evaluation process commences with the selection of offline data, followed by the training of a widely recognized offline RL algorithm, TD3+BC~\citep{fujimoto2021minimalist}, on this reduced dataset for 1 million time steps.
To ensure a fair comparison, we apply the same offline RL algorithm to data subsets obtained by different algorithms. 
Evaluation points are set at every 5,000 training time steps and involve calculating the return of 10 episodes per point.
The results, comprising averages and standard deviations, are computed with five independent random seeds.
On the other hand, we can also incorporate our method into offline model-based approaches, such as MOPO~\citep{yu2020mopo} and MoERL~\citep{kidambi2020morel}.
Similarly, we only need to replace the current offline loss with the corresponding policy and model loss.

\textbf{Baselines}. 
We compare \name~with data selection methods in RL.
Specifically, previous work on prioritized experience replay for online RL~\citep{schaul2015prioritized} aligns closely with our objective. 
We make this a baseline \namep~where samples with the highest TD losses form the reduced dataset. 
Baseline \nameo~presents the performance by training TD3+BC with the original, complete dataset. 
Baseline \namer~randomly selects subsets from the D4RL dataset that are of the same size as \name.
We also compare our method with general dataset reduction techniques from supervised learning.
Specifically, we adopt the coherence criterion from Kernel recursive least squares~($\mathtt{KRLS}$)~\citep{engel2004kernel}, the log det criterion by forward selection in informative vector machines~($\mathtt{LogDet}$)~\citep{seeger2004greedy} and the adapting kernel representation~($\mathtt{BlockGreedy}$)~\citep{schlegel2017adapting} as our baselines.

%Specifically, we consider randomly selecting offline coreset as our baseline algorithms.
% In addition, we consider separately selecting high-reward offline datasets and low-reward offline datasets as our baseline algorithms.

\subsection{Experimental Results}
\label{sec:exp_perf}
% To compare the performance of different algorithms, we adopt two data selection schemes: sample-based selection and trajectory-based selection. They differ in the smallest unit of selection: the first selects samples in each iteration, while the second selects trajectories.

% As for the trajectory-based selection, prioritized sampling is no loner applicable. As an alternative, we compare with \nameh, which selects trajectories with the highest accumulative reward from the complete dataset. We again compare with the \nameo~as the reference to an upper limit of performance.

\begin{table*}[t]
    \centering
    \begin{tabular}{c|cccc}
    \toprule
    & KRLS & Log-Det & BlockGreedy & \name \\
    \midrule
    Hopper-medium-v0 & 69.4$\pm$2.5 & 58.4$\pm$3.6 & 83.7$\pm$2.2 & \textbf{94.3$\pm$4.6}\\
    Hopper-expert-v0 & 91.0$\pm$1.1 & 90.7$\pm$1.3 & 98.7$\pm$0.5 & \textbf{110.0$\pm$0.5}\\
    Hopper-medium-replay-v0 & 28.5$\pm$3.2 & 29.4$\pm$1.2 & 30.5$\pm$2.4 & \textbf{35.3$\pm$3.2}\\
    Walker2d-medium-v0 & 49.1$\pm$2.8 & 47.5$\pm$3.4 & 53.3$\pm$3.6 & \textbf{80.5$\pm$2.9}\\
    Walker2d-expert-v0 & 68.4$\pm$3.2 & 67.5$\pm$5.6 & 74.8$\pm$3.4 & \textbf{104.6$\pm$2.5}\\
    Walker2d-medium-replay-v0 & 14.3$\pm$1.2 & 15.2$\pm$2.2 & 16.7$\pm$1.3 & \textbf{21.1$\pm$1.8}\\
    Halfcheetah-medium-v0 & 23.4$\pm$0.5 & 21.9$\pm$0.9 & 27.5$\pm$0.7 & \textbf{41.0$\pm$0.2}\\
    Halfcheetah-expert-v0 & 73.9$\pm$1.4 & 72.1$\pm$2.2 & 79.2$\pm$1.8 & \textbf{88.5$\pm$2.4}\\
    Halfcheetah-medium-replay-v0 & 39.5$\pm$0.3 &39.9$\pm$0.5 & 40.5$\pm$1.0 & \textbf{41.1$\pm$0.4}\\
    \bottomrule
    \end{tabular}
    \caption{Experimental results on the D4RL~(Hard) offline datasets. All experiment results were averaged over five random seeds. Our method performs better than the dataset reduction baselines.}
    \label{tab: varied performance}
\end{table*}

\begin{figure}[t]
    \centering
    \subfigure{\includegraphics[scale=0.20]{d4rl/halfcheetah-medium-expert-v0.pdf}}
    \subfigure{\includegraphics[scale=0.20]{d4rl/hopper-medium-v0.pdf}}
    \subfigure{\includegraphics[scale=0.20]{d4rl/hopper-medium-expert-v0.pdf}}
    \subfigure{\includegraphics[scale=0.20]{d4rl/walker2d-medium-expert-v0.pdf}}
    \caption{Experimental results on the D4RL offline datasets. All experiment results were averaged over five random seeds. Our method achieves better or comparable results than the baselines consistently.}
    \label{fig: d4rl original}
\end{figure}

\paragraph{Answer of Question 1:}
To show that \name~can improve offline RL algorithms, we compare \name~with Complete Dataset, Prioritized, and Random in the Mujoco domain.
The experimental results in Figure~\ref{fig: d4rl hard} show that our method achieves superior performance than baselines.
By leveraging the reduced dataset generated from \name, the agent can learn much faster than learning from the complete dataset.
Further, the results in Figure~\ref{fig: d4rl original} show that \name~also performs better than the complete dataset and data selection RL baselines in the standard D4RL datasets. 
This is because prior methods select data in a random or loss-priority manner, which lacks guidance for subset selection and leads to degraded performance for downstream tasks.

In addition, to test \name's generality across various offline RL algorithms on various domains, we also conduct experiments on Antmaze tasks.
We use IQL~\citep{kostrikov2021offline} as the backbone of offline RL algorithms.
The experimental results in Table~\ref{tab: other domain2} show that our method achieves stronger performance than baselines.
In the antmaze tasks, the agent is required to stitch together various trajectories to reach the target location.
In this scenario, randomly removing data could result in the loss of critical data, thereby preventing complete the task.
Differently, \name~extracts valuable subset by balancing data quantity with performance, achieving a stronger performance than the complete dataset.

% In Figure~\ref{fig: d4rl minimal ratio}, we show the performance of different algorithms with the sample-based selection scheme. The experimental results show that \name~can achieve performance close to \nameo~with a small amount of data. For example, we use only $3\%$ of the original dataset in the Hopper tasks. \namer~and \namep, on the other hand, present a stark contrast, even not showing a stable learning trend with the same amount of training data. 
% In addition, we also evaluate the performance on the trajectory-based selection setting. Please refer to Appendix~\ref{appendix: trajectory} for the detailed experimental results.
% For the trajectory-based selection, experimental results in Figure~\ref{fig: d4rl topbc} show that \name~maintains its superiority in this setting with suboptimal (e.g., \texttt{medium}) datasets. This evidence suggests that \name~provides a valuable strategy for selecting data conducive to effective training under conditions of compromised data quality.

\paragraph{Answer of Question 2:}
To test whether \name~can select more valuable data than the data selection algorithms in supervised learning, we compare our method with KRLS~\citep{engel2004kernel}, Log-Det~\citep{seeger2004greedy} and BlockGreedy~\citep{schlegel2017adapting} in the D4RL~(Hard) datasets.
The experimental results in Table~\ref{tab: varied performance} show that our method generally outperforms baselines.
We hypothesize that supervised learning is static with fixed learning objectives, while offline RL's dynamic nature makes the target values evolve with policy updates, complicating the data selection process.
Therefore, the data selection methods in supervised learning cannot be directly applied to offline RL scenarios.

% Additionally, we observe that  $\texttt{Random}$ performs better than $\texttt{Q-diff}$.
% We attribute this phenomenon to the broader data coverage of $\texttt{Random}$, while the data coverage of $\texttt{Q-diff}$ is narrow.
% However, we also note that in some tasks, such as $\texttt{Hopper-medium-expert-v0}$, $\texttt{Hopper-expert-v0}$ and $\texttt{Walker2d-expert-v0}$, $\texttt{Random}$ initially performs well, but as training progresses, its performance starts to decline.
% We find that this coincides with unstable Q-values, which can be attributed to the increased extrapolation error caused by the reduced training dataset.
% In contrast, \name~performs better since it closely approximates the original gradients, thus preventing Q-values from diverging.


% For this reason, when the dataset quality is high~(e.g., \texttt{medium-expert} dataset), TopBC performs comparably to \name.

% \begin{table*}[t]
%     \centering
%     \caption{\name~with varying dataset sizes~($x\%$). Highlighted is the performance comparable to training TD3+BC with the complete dataset. \name~typically achieves good results with 5\%-15\% data, indicating that existing offline RL datasets contain a high degree of redundancy.
%     We adopt the normalized score metric proposed by the D4RL benchmark. Scores roughly range from 0 to 100, where 0 corresponds to the performance of a random policy and 100 indicates the performance of an expert.} 
%     \label{tab: varied performance}
%     \begin{tabular}{c|cccc}
%     \toprule
%         & 5\% & 10\% & 15\% & 20\% \\
%         \midrule
%         Hopper-medium-v0 & 91.8$\pm$3.6 & 92.6$\pm$3.0 & 94.0$\pm$4.8 & 95.2$\pm$1.6\\
%         Walker2d-medium-v0 & 14.8$\pm$7.3 & 57.9$\pm$3.6 & 69.3$\pm$4.0 & 71.7$\pm$1.9 \\
%         Halfcheetah-medium-v0 & 40.5$\pm$0.0 & 40.9$\pm$0.1 & 41.3$\pm$0.1 & 41.2$\pm$0.5 \\
%         Hopper-expert-v0 & 111.6$\pm$0.9 & 110.6$\pm$1.9 & 112.7$\pm$0.1 & 112.4$\pm$0.1 \\
%         Walker2d-expert-v0 & 74.5$\pm$6.4 & 84.4$\pm$5.0 & 97.6$\pm$3.1 & 100.2$\pm$1.0 \\
%         Halfcheetah-expert-v0 & 57.5$\pm$6.4 & 84.3$\pm$2.7 & 97.8$\pm$0.8 & 100.1$\pm$3.0 \\
%         Hopper-medium-expert-v0 & 108.1$\pm$1.1 & 112.4$\pm$0.3 & 112.3$\pm$0.05 & 112.8$\pm$0.1\\
%         Walker2d-medium-expert-v0 & 79.3$\pm$2.1 & 85.4$\pm$5.3 & 96.2$\pm$6.7 & 101.4$\pm$3.6 \\
%         Halfcheetah-medium-expert-v0 & 67.5$\pm$0.5 & 86.2$\pm$5.0 & 85.8$\pm$1.5 & 92.4$\pm$1.3\\
%     \bottomrule
%     \end{tabular}
% \end{table*}


% \subsection{Ablation Study}\label{sec:exp_ab}
% \textbf{Varying dataset size}.\ \ In Table~\ref{tab: varied performance}, we show the performance of \name~with varying dataset sizes ranging from $5\%$ to $20\%$.
% The results demonstrate that \name~requires only $5\%$ or $10\%$ of the original dataset to obtain good performance.
% Further, \name~can achieve similar performance with \nameo~with $20\%$ data of the original dataset.
% This indicates that existing offline RL datasets are characterized by a high degree of redundancy.

\begin{figure}[t]
    \centering
    \includegraphics[width=0.97\linewidth]{visual.jpg}
    \caption{Visualization of the \textcolor{blue}{complete dataset} and the \textcolor{orange}{reduced dataset} in \texttt{halfcheetah} task. The higher opacity of a point represents a large time step towards the end of an episode. The dataset embedding is characterized by its division into different components. 
    % In \texttt{walker2d} (upper), components vary with time steps.
     Samples selected by \name~connect different components by focusing on the data related to the task.}
    \label{fig: t-sne}
\end{figure}

\begin{table}[t]
    \centering 
    \begin{tabular}{c|cccc}
    \toprule
        Env & Random & Prioritized & Complete Dataset & \name\\
        \midrule
        Antmaze-umaze-v0 & 75.1$\pm$2.5 & 70.2$\pm$3.6 & 87.5$\pm$1.3 & \textbf{90.7$\pm$3.3}\\
        Antmaze-umaze-diverse-v0 & 46.3$\pm$1.9 & 44.7$\pm$2.7 & 62.2$\pm$2.0 & \textbf{76.7$\pm$2.2} \\
        Antmaze-medium-play-v0 & 59.3$\pm$1.6 & 60.3$\pm$2.9 & 71.2$\pm$2.2 & \textbf{80.3$\pm$2.9}\\
        Antmaze-medium-diverse-v0 & 43.6$\pm$2.7 & 46.9$\pm$3.8 & 70.0$\pm$1.6 & \textbf{84.9$\pm$3.8}\\
        Antmaze-large-play-v0 &	3.7$\pm$0.7 & 15.0$\pm$3.5 & 39.6$\pm$3.6 & \textbf{46.0$\pm$3.5}\\
        Antmaze-large-diverse-v0 & 16.0$\pm$3.6 & 20.5$\pm$3.7 & 47.5$\pm$1.1 & \textbf{52.0$\pm$3.7}\\
    \bottomrule
    \end{tabular}
    \caption{Experimental results on the Antmaze offline datasets. All experiment results were averaged over five random seeds. Our method performs better than baselines. }
    \label{tab: other domain2}
\end{table}

% \begin{figure*}[t]
%     \centering
%     \subfigure{\includegraphics[scale=0.27]{ablation_moduler1.pdf}}
%     \hspace{0.3cm}\subfigure{\includegraphics[scale=0.27]{ablation_moduler2.pdf}}
%     \caption{Ablation results on D4RL~(Hard) tasks with the normalized score metric.}
%     \label{fig: modular ablation}
% \end{figure*}

% In this subsection, we conduct ablation studies to study the effect of different modules and import hyper-parameters.


\paragraph{Answer of Question 3:}
To study the contribution of each component in our learning framework, we conduct the following ablation study. 
\nameq: We replace the empirical returns used to update Q functions with the standard target Q function in the TD loss function. 
\namei: We set the number of data selection rounds to 1 and study the function of multi-round data selection.
The experimental results in Figure~\ref{fig: modular ablation} in Appendix~\ref{sec: ablation} show that removing any of these two modules will worsen the performance of \name. In case like $\texttt{walker2d-medium}$, ablation \namei~even decrease the performance by over 80\%, and ablation \nameq~results in a 95\% performance drop in $\texttt{walker2d-expert}$. Furthermore, we also find that in the $\texttt{halfcheetah}$ tasks, the impact of removing the two modules is relatively small. This result can be attributable to the fact that this task has a limited state space, and we can directly apply OMP to the entire dataset and identify important and diverse data.

We visualize the selected data by \name~to better understand how it works. 
Figure~\ref{fig: t-sne} displays the t-SNE low-dimensional embeddings, with the complete dataset in blue and the selected data in orange. 
The higher opacity of a point indicates a larger time step. The dataset's structure is revealed by its segmentation into diverse components: 
In \texttt{halfcheetah}, each component reflects a distinct skill of the agent.
For example, from 1 to 7, they represent falling, leg lifting, jumping, landing, leg swapping, stepping, and starting, respectively.
We can observe that the selected samples by \name~ not only cover each component of the dataset but also effectively bridge the gaps between them, enhancing the dataset's versatility and coherence. 
Moreover, we find that \name~is less concerned with the falling data and instead focuses on the data related to the task.
This observation can explain the improved performance of \name. For additional visualizations, please refer to Appendix~\ref{appendix: visual}.

% \textbf{Generalizability of \name}. \ \
% We evaluate the generalizability of \name~from two perspectives.
% First, we add IQL~\cite{kostrikov2021offline} as a baseline and apply \name~to IQL by using the gradient of the training loss of the V-function in IQL as the criterion.
% On the other hand, we evaluate \name~on the other domains, such as robotic manipulation (Adroit) and sparse reward (Antmaze) tasks.
% The experiments in Appendix~\ref{appendix: other domain} and Appendix~\ref{appendix: other algorithm} show that \name~is not only applicable to other algorithms, such as IQL~\cite{kostrikov2021offline}, but also to other domains.

% \textbf{Generalizability of subset}. \ \
% To test the generalizability of the dataset selected by~\name, we select subset by applying~\name~to TD3+BC.
% Then we evaluate the performance of IQL on the selected subset. 
% The experimental results in Table~\ref{tab: td3bc2iql} in Appendix~\ref{appendix: tb3bc2iql} demonstrate that the selected subset based on TD3+BC is effectively applicable to IQL.

% \textbf{Sensitivity for hyperparameter}. \ \
% We evaluate the performance of \name~with various cluster numbers~(from 1 to 50) and approximation bounds~(from 0.0001 to 0.05).
% The experimental results in Appendix~\ref{appendix: cluster number} and Appendix~\ref{appendix: approx bound} show that the suitable cluster number is between 25 and 50.
% Too few clusters (e.g., less than 5) are detrimental to the algorithm.
% In addition, a smaller approximation bound represents a larger reduced dataset.
% Similar to the ablation of the size of the reduced dataset in Table~\ref{tab: varied performance}, \name~requires only a 0.01 approximation bound to obtain good performance.

\subsection{Computational complexity}
We report the computational overhead of \name~on various datasets. 
All experiments are conducted on the same computational device (GeForce RTX 3090 GPU). 
The results in Appendix~\ref{appendix: computation complexity} indicate that even on datasets containing millions of data points, the computational overhead of our method remains low~(e.g., several minutes).
This low computational complexity can be attributed to the trajectory-based selection technique in Sec.~\ref{sec: offline omp}~(II) and the regularized constraint technique in Sec.~\ref{sec:method:outer}, making our method easily scalable to large-scale datasets. 

% This low computational complexity can be attributed to the batch mechanism designed in section 3.2 (IV), which reduces the computational complexity from $O(MN)$ to $O(|\mathcal{B}|N)$, making our method easily scalable to large-scale datasets. $M, N, |\mathcal{B}|$ are the size of the full dataset, reduced dataset, and batch respectively.

% We conduct t-SNE based dimensionality reduction to the cluster centroids and these five trajectories.
% The experimental results are shown in the , where darker colors indicate moving towards the end of the trajectory.

% From the experimental results, we find that in the walker2d task, \name~ tends to select more low-reward but more diverse data points ~(upper right) while selecting a few high-reward data points~(left and bottom).
% We attribute this phenomenon to the narrow distribution of the high-reward points, allowing us to approximate the original gradients with only a few points. 
% In the halfcheetah task, \name~ connects useful information while ignoring low-quality data~(e.g., data point \texttt{1}).

\subsection{Analysis}
\label{sec:analysis}
\begin{figure}[t]
\centering
    \includegraphics[width=0.95\columnwidth]{fig/allratio.pdf}
    \caption{Comparison of ratio adherence across different compression ratio settings. The experimental results are obtained with LLaMA-3.1-8B-Instruct on GSM8K.}
    \label{fig:allratio}
\end{figure}

\begin{figure}[t]
\centering
    \includegraphics[width=0.95\columnwidth]{fig/compressor.pdf}
    \caption{Performance comparison of \method using different token importance metrics, evaluated with LLaMA-3.1-8B-Instruct on GSM8K.}
    \label{fig:compressor}
\end{figure}

\paragraph{Compression Ratio} 
In our main results, we focus on compression ratios greater than 0.5. To further investigate the performance of \method at lower compression ratios, we train an additional variant, denoted as \texttt{More Ratio}, with extra compression ratios of 0.3 and 0.4. As shown in Figure~\ref{fig:allratio}, the ratio adherence of models largely degrades at these lower ratios. We attribute this decline to the excessive trimming of reasoning tokens, which likely causes a loss of critical information in the completions, hindering the effective training of LLMs to learn CoT compression. Furthermore, we observe that the overall adherence of \texttt{More Ratio} is not as good as \method with the default settings, which further supports our hypothesis.

\paragraph{Importance Metric} 
Figure~\ref{fig:compressor} presents a performance comparison of \method across different token importance metrics. In addition to the metrics discussed in Section \ref{sec:token-importance}, we include \texttt{GPT-4o}\footnote{We use the \texttt{gpt-4o-2024-08-06} version for experiments.} as a strong token importance metric for comparison. Specifically, for a given CoT trajectory, we prompt \texttt{GPT-4o} to trim redundant tokens according to a specified compression ratio, without adding any additional tokens. Additionally, we ask \texttt{GPT-4o} to suggest the \textit{optimal} compression format of the CoT trajectory, referred to as \texttt{GPT-4o-Optimal} in Figure~\ref{fig:compressor}. We incorporate all training data generated by \texttt{GPT-4o} to train a variant of \method. We use the ``\texttt{[optimal]}'' token to prompt the model, obtaining the results of \texttt{GPT-4o-Optimal}.

As illustrated in Figure~\ref{fig:compressor}, \method utilizing LLMLingua-2~\cite{pan:2024llmlingua2} outperforms the variant with Selective Context~\cite{li:2023selective}, which aligns with our demonstrations in Section \ref{sec:token-importance}. Additionally, incorporating \texttt{GPT-4o} for token importance measurement further enhances compression performance, suggesting that a more robust CoT compressor could improve \method even further. However, the API costs associated with \texttt{GPT-4o} make it impractical for processing large datasets. In contrast, LLMLingua-2, which includes a BERT-size model, offers a cost-effective and efficient alternative for training \method. Furthermore, \texttt{GPT-4o-Optimal} achieves a better balance between reasoning accuracy and CoT token reduction, emphasizing the potential of flexible compression ratios in CoT generation --- an avenue we plan to explore in future work.

\paragraph{Length Budget} 
As outlined in Section~\ref{sec:exp_setup}, we adjust the maximum length budget to \texttt{max\_len}$\times\gamma$ when evaluating \method on MATH-500, ensuring a fair comparison of compression ratios. However, this brute-force length truncation inevitably impacts the reasoning performance of LLMs, as LLMs are unable to complete the full generation. In this analysis, we explore whether LLMs can ``\textit{think}'' more effectively using a compressed CoT format. Specifically, we evaluate \method under the same length budget as the original LLM (e.g., 1024 for MATH-500). The experimental results, shown in Figure~\ref{fig:budget}, demonstrate a significant performance improvement of \method under this length budget, compared to those adjusted by compression ratios. Notably, with compression ratios of 0.7, 0.8, and 0.9, \method outperforms the original LLM, yielding an absolute performance increase of 1.3 to 2.6 points. These findings highlight \method's potential to enhance the reasoning capabilities of LLMs within the same length budget.

\begin{figure}[t]
\centering
    \includegraphics[width=0.95\columnwidth]{fig/budget.pdf}
    \caption{Performance comparison of \method with varying maximum length constraints, evaluated with LLaMA-3.1-8B-Instruct on the MATH-500 dataset.}
    \label{fig:budget}
\end{figure}

\begin{figure*}[t]
\centering
\includegraphics[width=1.0\textwidth]{fig/cases.pdf}
\caption{Three CoT compression examples from \method. For each sample, we list the question, original CoT outputs from corresponding LLMs, and the compressed CoT by \method. The tokens that appear in both the original CoT and the compressed CoT are highlighted in \sethlcolor{pink}\hl{red}.}
\label{fig:cases}
\end{figure*}

\paragraph{Case Study} 
Figure~\ref{fig:cases} presents several examples of \method, derived from the test sets of GSM8K and MATH-500. These examples clearly illustrate that \method allows LLMs to learn shortcuts between critical reasoning tokens, rather than generating shorter CoTs from scratch. For instance, in the first case, \method facilitates LLaMA-3.1-8B-Instruct to skip semantic connectors such as ``\textit{of}'' and ``\textit{the}'', as well as expressions that contribute minimally to the reasoning, such as the first sentence. Notably, we observe that numeric values and mathematical equations are prioritized for retention in most cases. This finding aligns with recent research~\cite{Ma:2024mathmatters}, which suggests that mathematical expressions may contribute more significantly to reasoning than CoT in natural language. Furthermore, we find that \method does not reduce the number of reasoning steps but instead trims redundant tokens within those steps.
\section{Related Work}
Researchers have been leveraging eye tracking methodologies from human perception research to model how people perceive images~\cite{shanmuga2015eye, bonhage2015combined, conklin2016using}.
These models help assess the appearance and salience of visual representations, enabling eye movement tracking to understand the perceptual and cognitive mechanisms of scene perception~\cite{itti1998model} and object detection~\cite{borji2015salient}.
The existing saliency models perform well in naturalistic scenes
%and real-world object detection
; however, there are unique perception rules and cognitive biases in the artificial world of data visualization 
%does not always follow the rules of perception in the natural world
~\cite{franconeri2021science, correll2012comparing, polatsek2018exploring, knittel2024gridlines}, and, thus, these models do not accurately predict where people would look in visualizations. 
Visualization researchers have been building visual saliency models geared to visualizations~\cite{DVSaliencyModel2017Matzen, bylinskii2016should}. %and adopting them for predicting eye gaze on visualizations. %enabling the prediction of visual saliency across design styles~\cite{fosco2020predicting}.
However, these models rely on handcrafted features, making it difficult to generalize to complex visualizations. Additionally, these models cannot incorporate textual information to generate task-specific saliency maps since the prediction is solely based on visual inputs.

With the advent of deep learning, gaze data were used as the ground truth of saliency models~\cite{fosco2020predicting, scannerDeeply, scanpath}, leading to higher performance in saliency prediction while enabling task-specific saliency~\cite{salchartQA}. 
These models usually need large-scale datasets to learn complex patterns. However, gathering precise gaze data is 
%challenging and requires specialized eye-tracking devices. While these devices provide accurate results, they tend to be 
costly and cumbersome, which limits large-scale data collection efforts. 
Many researchers, therefore, proposed several proxies for eye gaze. WebGaze~\cite{webgaze} uses a webcam for cheap and easy deployment in online studies yet suffers from data quality issues due to low-resolution cameras and uncontrolled calibration.
Therefore, mouse-(cursor-)based annotation tools~\cite{jiang2015salicon,bubbleView,importAnnot} were proposed to improve data quality. Among these methods, BubbleView~\cite{bubbleView} was the most used tool for capturing visual saliency and importance~\cite{graphicDesignImportance, salchartQA}.
However, BubbleView is primarily designed for exploring images and gathering information, which differs slightly from the goal of capturing perceived importance. As a result, while BubbleView is well-suited for measuring visual saliency, it may not be the best tool for capturing %instruction-tuned \yao{I would keep it consistent saying task-specific}
task-specific importance~\cite{turkeyes}. Built upon these prior approaches' limitations, our Grid Labeling aims to collect responses that cover all essential areas of the visualization with minimum noise, leading to more efficient data collection.



% One key motivation to our grid-based approach is to help people 
% We also demonstrate that the grid-based approaches can minimize biases in annotation to disproportionally emphasize text elements~\cite{DVSaliencyModel2017Matzen}
% % blurring the visualization can disproportionately emphasize text elements~\cite{DVSaliencyModel2017Matzen}, potentially misrepresenting a user's true areas of interest.
% More recently, 
% % \ms{Changed a bit using Yao's work (task-dependent saliency), but not sure whether it looks ok}
% Yao et al.~\cite{salchartQA} collect task-dependent saliency using the BubbleView method, % but their approach had some limitations. 
% and made a significant improvement on existing saliency models.
% First, the blurred visualization allowed users to perceive the overall structure of the chart, which prevented the system from capturing the specific action of identifying the maximum value. However, increasing the blur to address this issue introduced another challenge. As the structure became less visible, users had to explore the entire image, leading to the consideration of irrelevant regions as salient.
\section{Conclusion}

This work analysed the results of evolutionary wrapper approaches using decision tree based models as proxies and compared them with common \gls{FE} techniques on a \gls{HL} detection problem. Three experiments were conducted using the proposed framework, each employing different proxy models.

When comparing the three experiments, an interesting behaviour of the framework was discovered, when changing the proxy model. The \gls{DT} experiment drastically reduced the number of features, while the other models did not. To further reduce the number of features, one could bias the grammar or apply some penalty in the fitness function for the individuals that use a large number of features. This might not change the behaviour when using different models other than a \gls{DT}, but it forcefully reduces the number of features.  

The results confirm that FEDORA can reduce the dimensionality of the data while statistically maintaining baseline performance, in every experiment. The framework consistently outperforms the remaining \gls{FE} methods, with statistical significance and large effect sizes, proving itself as a viable alternative.

The best result obtained is 76.2\% balanced accuracy using an individual from the \gls{RF} experiment, and a \gls{XGB} algorithm as the testing model, using 57 total features (45 Original, 6 Engineered and 6 Complex) out of the 60 original ones. When using the least amount of features, the best result is 72,8\% balanced accuracy using an individual from the \gls{DT} experiment and a \gls{RF} algorithm as the testing model, using a single complex feature.

In future work, exploring the above-mentioned behaviours might be relevant to better understanding them, namely when biasing the grammar or penalizing the use of many features in the fitness function. Concerning the explainability of the FEDORA transformations, researching meaningful grammar operators might prove useful in addressing problem-specific needs. In this case, having logical operators for the boolean features, which have values of "yes" or "no", and the choice of a simple decision algorithm as the proxy, may increase explainability. Additionally, the previous study has identified several areas for future research, yet to be addressed. For instance, comparing the framework with other common and more complex methods and completing the full \gls{ML} pipeline through the use of a method that addresses the \gls{CASH}, such as \cite{assunccao2020evolution}, and comparing it to other full pipeline frameworks, could be beneficial for contextualizing and evaluating the framework within the \gls{AutoML} and \gls{EC} domains. The framework still needs to be analysed with different datasets to properly assess its generalization capabilities.




% ----------------------------------------------------------------------


\section*{Impact Statement}
This work aims to contribute to the advancement of reasoning with LLMs. 
While our research could have various societal implications, none are deemed significant enough to warrant specific mention at this stage.
% Authors are \textbf{required} to include a statement of the potential 
% broader impact of their work, including its ethical aspects and future 
% societal consequences. This statement should be in an unnumbered 
% section at the end of the paper (co-located with Acknowledgements -- 
% the two may appear in either order, but both must be before References), 
% and does not count toward the paper page limit. In many cases, where 
% the ethical impacts and expected societal implications are those that 
% are well established when advancing the field of Machine Learning, 
% substantial discussion is not required, and a simple statement such 
% as the following will suffice:

% ``This paper presents work whose goal is to advance the field of 
% Machine Learning. There are many potential societal consequences 
% of our work, none which we feel must be specifically highlighted here.''

% The above statement can be used verbatim in such cases, but we 
% encourage authors to think about whether there is content which does 
% warrant further discussion, as this statement will be apparent if the 
% paper is later flagged for ethics review.


% % In the unusual situation where you want a paper to appear in the
% % references without citing it in the main text, use \nocite
% \nocite{langley00}

\bibliography{main}
\bibliographystyle{icml2025}


%%%%%%%%%%%%%%%%%%%%%%%%%%%%%%%%%%%%%%%%%%%%%%%%%%%%%%%%%%%%%%%%%%%%%%%%%%%%%%%
%%%%%%%%%%%%%%%%%%%%%%%%%%%%%%%%%%%%%%%%%%%%%%%%%%%%%%%%%%%%%%%%%%%%%%%%%%%%%%%
% APPENDIX
%%%%%%%%%%%%%%%%%%%%%%%%%%%%%%%%%%%%%%%%%%%%%%%%%%%%%%%%%%%%%%%%%%%%%%%%%%%%%%%
%%%%%%%%%%%%%%%%%%%%%%%%%%%%%%%%%%%%%%%%%%%%%%%%%%%%%%%%%%%%%%%%%%%%%%%%%%%%%%%
\newpage
\appendix
% \section{You \emph{can} have an appendix here.}

% You can have as much text here as you want. The main body must be at most $8$ pages long.
% For the final version, one more page can be added.
% If you want, you can use an appendix like this one.  

% The $\mathtt{\backslash onecolumn}$ command above can be kept in place if you prefer a one-column appendix, or can be removed if you prefer a two-column appendix.  Apart from this possible change, the style (font size, spacing, margins, page numbering, etc.) should be kept the same as the main body.

\section{} 
\label{reduction-P}

To prove our argument, we apply the splitting property of the Poisson process. Let \( N(t) \) be a Poisson process with rate parameter \( \lambda \). If events are split into two groups with probabilities \( p \) and \( 1-p \), then the resulting processes \( N_1(t) \) and \( N_2(t) \) are independent Poisson processes with rate parameters \( p\lambda \) and \( (1-p)\lambda \) respectively \cite{splitting_poisson}.

From process \( j \)'s perspective, we can split arrivals from sensor \( i \) into two groups: informative and uninformative arrivals with probabilities \( \nc_{ij} \) and \( 1-\nc_{ij} \), respectively. The rate of arrivals from sensor \( i \) is \( \lambda_i \), so the rate of informative arrivals for process \( j \) from sensor \( i \) is \( \nc_{ij}\lambda_i \). Additionally, we can further split the informative arrivals based on whether they can preempt ongoing service. The rate of informative arrivals that can preempt ongoing service for process \( j \) from sensor \( i \) is \( \np_{i}\nc_{ij}\lambda_i \) and the rate of informative arrivals that can not preempt ongoing service for process \( j \) from sensor \( i \) is \( (1-\np_{i})\nc_{ij}\lambda_i \). Since all these arrivals are Poisson, we can merge them into a single process. The total arrival rate of informative packets that can preempt ongoing service for process \( j \) is given by

\begin{equation}
\Tilde{\lambda}_j = \sum_{i=1}^{N} \np_{i}\nc_{ij}\lambda_i
\end{equation}

Similarly, the total arrival rate of informative packets that can not preempt ongoing service for process \( j \) is

\begin{equation}
\Tilde{\lambda}_j = \sum_{i=1}^{N} (1-\np_{i})\nc_{ij}\lambda_i
\end{equation}

We can express these rates in vector form as follows:

\begin{equation}
\boldsymbol{\Tilde{\lambda}}^T = \begin{bmatrix}
\Tilde{\lambda}_{1} & \Tilde{\lambda}_{2} & \dots & \Tilde{\lambda}_{M}
\end{bmatrix} = (\boldsymbol{\lambda}^T \odot \bfp^T) \bfc,
\end{equation}
\begin{equation}
\boldsymbol{\dot{\lambda}}^T = \begin{bmatrix}
\dot{\lambda}_{1} & \dot{\lambda}_{2} & \dots & \dot{\lambda}_{M}
\end{bmatrix} = (\boldsymbol{\lambda}^T \odot (1-\bfp^T)) \bfc,
\end{equation}


The importance of the packet is whether it has information of process $j$ so  we can say that The system with $N$ sensors and arrival rates $\boldsymbol{\lambda}$ shown in Figure \ref{fig:system_model} equivalents to the system with two sources as shown in Figure \ref{fig:equiv_model} from process $j$'s perspective.


\section{}\label{spv-appendix}


We adopt the stochastic hybrid system (SHS) model as defined in \cite{yates2019}, with a key distinction: our model incorporates probabilistic preemption. The system dynamics are depicted in Figure \ref{fig:equiv_model} so we can analyze the AoI for any process $i$ and generalize it. First, the discrete state is denoted as $q(t) = q \in Q = \{0, 1, 2\}$, where $q = 0$ represents an idle server, and $q \in \{1, 2\}$ signifies that an update packet is currently being serviced. The continuous state is described as $x(t) = [x_0(t), x_1(t)]$, where $x_0(t)$ represents the current age of the process, and $x_1(t)$ captures the potential age if the packet in service is successfully delivered. Notably, $x_1(t)$ is irrelevant in state $0$ since no packet is in service. In state $1$, $x_1(t)$ corresponds to the age of the informative update being serviced. Conversely, in state $2$, where an uninformative update is in service, the completion of this update does not affect the process age, rendering $x_1(t)$ irrelevant in this state as well.

\begin{table}[h]
\centering
\caption{Table of Transitions for the Markov Chain in Figure \ref{fig:shs}.}
\begin{tabular}{c c c c c c}
\toprule
$l$ & $q_l \rightarrow q'_l$ & $\lambda^{(l)}$ & $\mathbf{xA}_l$ & $\mathbf{A}_l$ & $\mathbf{v}_{q_l}\mathbf{A}_l$ \\
\midrule
1 & $0 \rightarrow 1$ & $\Tilde{\lambda}_{1}+\dot{\lambda}_{1}$ & $\begin{bmatrix} x_0 & 0 \end{bmatrix}$ & \small $\begin{bmatrix} 1 & 0 \\ 0 & 0 \end{bmatrix}$ \normalsize & $\begin{bmatrix} v_{00} & 0 \end{bmatrix}$ \\
2 & $0 \rightarrow 2$ & $\lambda_{C}-\Tilde{\lambda}_{1}-\dot{\lambda}_{1}$ & $\begin{bmatrix} x_0 & 0 \end{bmatrix}$ & \small $\begin{bmatrix} 1 & 0 \\ 0 & 0 \end{bmatrix}$ \normalsize & $\begin{bmatrix} v_{00} & 0 \end{bmatrix}$ \\
3 & $1 \rightarrow 0$ & $\mu$        & $\begin{bmatrix} x_1 & 0 \end{bmatrix}$ & \small$\begin{bmatrix} 0 & 0 \\ 1 & 0 \end{bmatrix}$ \normalsize & $\begin{bmatrix} v_{11} & 0 \end{bmatrix}$ \\
4 & $1 \rightarrow 1$ & $\Tilde{\lambda}_{1}
$  & $\begin{bmatrix} x_0 & 0 \end{bmatrix}$ & \small$\begin{bmatrix} 1 & 0 \\ 0 & 0 \end{bmatrix}$\normalsize & $\begin{bmatrix} v_{10} & 0 \end{bmatrix}$ \\
5 & $1 \rightarrow 2$ & $\Tilde{\lambda}_{C}-\Tilde{\lambda}_{1}$  & $\begin{bmatrix} x_0 & 0 \end{bmatrix}$ & \small$\begin{bmatrix} 1 & 0 \\ 0 & 0 \end{bmatrix}$ \normalsize & $\begin{bmatrix} v_{10} & 0 \end{bmatrix}$ \\
6 & $2 \rightarrow 0$ & $\mu$        & $\begin{bmatrix} x_0 & 0 \end{bmatrix}$ & \small$\begin{bmatrix} 0 & 0 \\ 1 & 0 \end{bmatrix}$ \normalsize & $\begin{bmatrix} v_{20} & 0 \end{bmatrix}$ \\
7 & $2 \rightarrow 1$ & $\Tilde{\lambda}_{1}$  & $\begin{bmatrix} x_0 & 0 \end{bmatrix}$ & \small$\begin{bmatrix} 1 & 0 \\ 0 & 0 \end{bmatrix}$ \normalsize & $\begin{bmatrix} v_{20} & 0 \end{bmatrix}$ \\
\bottomrule
\end{tabular}
\label{shs_table}
\end{table}


\begin{figure}
    \centering
    \includegraphics[width=0.5\linewidth]{figures/shs.png}
    \caption{The Markov chain for updates.}
    \label{fig:shs}
\end{figure}

A Markov chain representing the discrete state $q(t)$ is depicted in Figure~\ref{fig:shs}. The corresponding transitions of the SHS at state $q_l$ are detailed in Table~\ref{shs_table}. In the figure, a directed edge $l$ from node $q$ to node $q'$ indicates that transitions from state $q$ to state $q'$ occur at an exponential rate $\lambda^{(l)}$, as specified in the table. 





%\suresh{Incomplete.}

We first show that the stationary probability vector $\pi$ satisfies $
\mathbf{\pi D} = \mathbf{\pi Q} \quad \text{with}$ 
\begin{align}
\quad
\mathbf{D} = \text{diag}[\lambda_{C}, \mu + \Tilde{\lambda}_{C}, \mu + \Tilde{\lambda}_{1}], \quad  \\ \mathbf{Q} = 
\begin{bmatrix}
0 & \Tilde{\lambda}_{1}+\dot{\lambda}_{1} & \lambda_{C}-\Tilde{\lambda}_{1}-\dot{\lambda}_{1} \\
\mu & \Tilde{\lambda}_{1} & \Tilde{\lambda}_{C}-\Tilde{\lambda}_{1} \\
\mu & \Tilde{\lambda}_{1} & 0
\end{bmatrix}.
\end{align}
Applying $\sum_{i=0}^{2} \pi_i = 1$, the stationary probabilities are 
\begin{equation}
\pi_0 = \frac{\mu}{(\lambda_C + \mu)}, \label{pi0}
\end{equation}
\begin{equation}
\pi_1 = \frac{\lambda_C\Tilde{\lambda}_{1} + \dot{\lambda}_{1}\mu + \Tilde{\lambda}_{1}\mu}{(\lambda_C + \mu)(\Tilde{\lambda}_{C} + \mu)}, \label{pi1}
\end{equation}
\begin{equation}
\pi_2 = \frac{\Tilde{\lambda}_{C}\lambda_C + \lambda_C\mu -\lambda_C\Tilde{\lambda}_{1}  - \dot{\lambda}_{1}\mu - \Tilde{\lambda}_{1}\mu}{(\lambda_C + \mu)(\Tilde{\lambda}_{C} + \mu)} . \label{pi2}
\end{equation}

\section{}\label{aoi-appendix}





Given the SHS model and $\pi$ in Appendix \ref{spv-appendix}, we can evaluate $\bar{v}$ to find the AoI. Let 
\begin{equation}
\mathbf{\bar{v}} = [\mathbf{\bar{v}_0} \ \mathbf{\bar{v}_1} \ \mathbf{\bar{v}_2}] = [\bar{v}_{00} \ \bar{v}_{01} \ \bar{v}_{10} \ \bar{v}_{11} \ \bar{v}_{20} \ \bar{v}_{21}].   
\end{equation}
It follows that
\begin{equation}
\mathbf{\bar{v}D} = \mathbf{\pi B} +  \mathbf{\bar{v}R},
\end{equation}
where 
\begin{equation}
\mathbf{D} = \text{diag}[\lambda_C, \lambda_C, \mu + \Tilde{\lambda}_{C}, \mu + \Tilde{\lambda}_{C}, \mu + \Tilde{\lambda}_{1}, \mu + \Tilde{\lambda}_{1}],
\end{equation}
\begin{equation}
\mathbf{B} =
\begin{bmatrix}
1 & 0 & 0 & 0 & 0 & 0 \\
0 & 0 & 1 & 1 & 0 & 0 \\
0 & 0 & 0 & 0 & 1 & 0
\end{bmatrix},
\end{equation}
and
\begin{equation}
\mathbf{R} = 
\begin{bmatrix}
0 & 0 & \Tilde{\lambda}_{1}+\dot{\lambda}_{1}  & 0 & \lambda_{C}-\Tilde{\lambda}_{1}-\dot{\lambda}_{1} & 0 \\
0 & 0 & 0 & 0 & 0 & 0 \\
0 & 0 & \Tilde{\lambda}_{1} & 0 & \Tilde{\lambda}_{C}-\Tilde{\lambda}_{1} & 0 \\
\mu & 0 & 0 & 0 & 0 & 0 \\
\mu & 0 & \Tilde{\lambda}_{1} & 0 & 0 & 0 \\
0 & 0 & 0 & 0 & 0 & 0
\end{bmatrix}.
\end{equation}

Then, we obtain $\bar{v}_{01}=\bar{v}_{21} = 0 $ and 
\begin{align}
&
\label{pi_v}
\begin{bmatrix}
\bar{\pi}_0 & \bar{\pi}_1 & \bar{\pi}_1 & \bar{\pi}_2
\end{bmatrix}
= \\ \nonumber \hat{\mathbf{v}}&
\begin{bmatrix}
\lambda_{C} & -\Tilde{\lambda}_{1}-\dot{\lambda}_{1} & 0 & \Tilde{\lambda}_{1}+\dot{\lambda}_{1}-\lambda_{C} \\
0 & \mu + \Tilde{\lambda}_{C}-\Tilde{\lambda}_{1} & 0 & \Tilde{\lambda}_{1} - \Tilde{\lambda}_{C} \\
-\mu & 0 & \mu + \Tilde{\lambda}_{C} & 0 \\
-\mu & -\Tilde{\lambda}_{1} & 0 & \mu + \Tilde{\lambda}_{1}
\end{bmatrix}, \\
\text{where } 
\hat{\mathbf{v}} &= 
\begin{bmatrix}
\bar{v}_{00} & \bar{v}_{10} & \bar{v}_{11} & \bar{v}_{20}
\end{bmatrix}. \nonumber
\end{align}

After solving eq. (\ref{pi_v}) using eqs. (\ref{pi0}), (\ref{pi1}), and (\ref{pi2}), we determine $\mathbf{\bar{v}}$. Later, we find the average age of information using the formula for a single process $j$ $\Delta_j = \sum_{q=0}^2 \bar{v}_{10}$ as follows:

%\suresh{Is this what you defined as $\Delta_{\rm sum}$ earlier?}

\footnotesize
\begin{align}
\Delta_j = \frac{\lambda_{C}^{2} \tilde{\lambda}_C + \lambda_{C}^{2} \mu + \lambda_{C} \dot{\lambda}_1 \mu + 2 \lambda_{C} \tilde{\lambda}_C \mu + 2 \lambda_{C} \mu^{2} + \tilde{\lambda}_C \mu^{2} + \mu^{3}}{\mu \left(\lambda_{C}^{2} \tilde{\lambda}_1 + \lambda_{C} \dot{\lambda}_1 \mu + 2 \lambda_{C} \tilde{\lambda}_1 \mu + \dot{\lambda}_1 \mu^{2} + \tilde{\lambda}_1 \mu^{2}\right)}
\end{align}
\normalsize

\section{}\label{iteration-appendix}


In this section, we discuss the upper bound on the number of iterations required by the outer space accelerating branch-and-bound algorithm to achieve a global $\epsilon_0$-optimal solution. According to Theorem 5 in \cite{JIAO2022112701}, for any given positive error $\epsilon_0 \in (0, 1)$, the algorithm converges to the desired solution in at most
\begin{equation}
p \cdot \left\lceil \log_2 \frac{p\tau \delta(\Omega)}{\epsilon_0} \right\rceil 
\end{equation}
iterations.

Here, the symbols used in the theorem are defined as follows:

\begin{itemize}
    \item $\Omega \subseteq \mathbf{R}^p$ is a compact hyper-subrectangle, and $\delta(\Omega)$ is defined as:
    \begin{equation}
    \delta(\Omega) = \max_{i=1,2,\dots,p} \{ \bar{U}_i - \bar{L}_i \},    
    \end{equation}
    where $\bar{U}_i$ and $\bar{L}_i$ represent the upper and lower bounds of the $i$-th dimension of the rectangle $\Omega$.

    \item $\tau$ is defined as:
    \begin{equation}\label{tau_eq}
    \tau = \max_{i=1,\dots,p} \frac{4 \max\{|\bar{l}_i|, |\bar{u}_i|\}}{\min\{\bar{L}_i, \bar{U}_i, \bar{L}_i^2, \bar{U}_i^2\}},
    \end{equation}
    where the terms are determined as follows:
    \begin{align}
        \bar{l}_i &= \min_{y \in \Theta} n_i(y), \quad \bar{u}_i = \max_{y \in \Theta} n_i(y), \nonumber \\
        \bar{L}_i &= \min_{y \in \Theta} d_i(y), \quad \bar{U}_i = \max_{y \in \Theta} d_i(y).
    \end{align}

    \item The terms $n_i(y)$ and $d_i(y)$ come from the problem defined as:
    \begin{align}
    \quad \min f(y) = \sum_{i=1}^p \frac{n_i(y)}{d_i(y)}, \quad \nonumber \\ \text{s.t.} \; y \in \Theta = \{y \in \mathbf{R}^n \mid Ay \leq b \}. 
    \end{align}
    \end{itemize}


We can reformulate our problem to determine the upper bound using these definitions. The variable in our problem is $\bfp$, and the objective is specified in (\ref{objective_func}). There are $M$ different linear fractions in the objective. The numerators of these fractions increase as any element of $\bfp$ increases. Consequently, we obtain $\bar{l}_i$ when $\bfp = 0$ and $\bar{u}_i$ when $\bfp = 1$ as follows:
\begin{align}
            \bar{l}_i &= \mu(\mu + \lambda_C)^2 + \sum_{i=1}^{N} \lambda_{i}\mu\lambda_C \nc_{ij},\quad \bar{u}_i = (\mu + \lambda_{C})^3
\end{align}

 Similarly, the denominators of these fractions decrease as any element of $\bfp$ increases, leading to $\bar{L}_i$ when $\bfp = 0$ and $\bar{U}_i$ when $\bfp = 1$.
 \begin{align}
            \bar{L}_i &=  (\mu + \lambda_C) \mu^2 \sum_{i=1}^{N} \nc_{ij} \lambda_{i}, \quad \bar{U}_i =  (\mu + \lambda_C)^2 \mu \sum_{i=1}^{N} \nc_{ij} \lambda_{i}.
\end{align}

After that, $\delta(\Omega)$ becomes: 

\begin{align}
        \delta(\Omega) = \max_{i=1,2,\dots,M} \{(\mu + \lambda_C)\lambda_C \mu \sum_{i=1}^{N} \nc_{ij} \lambda_{i}\} \leq (\mu + \lambda_C)\lambda_C^2 \mu , 
\end{align}

Last, we find $\tau$. In our problem, all parameters and variables are positive, so both the nominators and the denominators are positive, which can help us simplify eq. (\ref{tau_eq}) and obtain $\tau$ as follows:

    \begin{align}
    \tau = \max_{i=1,\dots,M} \frac{4 \bar{u}_i}{\bar{L}_i^2} = \frac{4 (\mu + \lambda_{C})}{\mu^4 \hat{\lambda}_{\min}^2},\\ \nonumber
    \text{where } \hat{\lambda}_{\min} = \min(\boldsymbol{\lambda}^T \bfc)
    \end{align}

Putting all together, for any given positive error $\epsilon_0 \in (0, 1)$, the outer space accelerating branch-and-bound algorithm can seek out a global $\epsilon_0$-optimum solution in at most 
\begin{equation}
M \cdot \left\lceil \log_2 \frac{4M (\mu+\lambda_C)^2\lambda_C^2}{\epsilon_0\mu^3\hat{\lambda}_{\min}^{2}} \right\rceil
\end{equation}
iterations as shown in Theorem \ref{Theo2}.



%%%%%%%%%%%%%%%%%%%%%%%%%%%%%%%%%%%%%%%%%%%%%%%%%%%%%%%%%%%%%%%%%%%%%%%%%%%%%%%
%%%%%%%%%%%%%%%%%%%%%%%%%%%%%%%%%%%%%%%%%%%%%%%%%%%%%%%%%%%%%%%%%%%%%%%%%%%%%%%


\end{document}


% This document was modified from the file originally made available by
% Pat Langley and Andrea Danyluk for ICML-2K. This version was created
% by Iain Murray in 2018, and modified by Alexandre Bouchard in
% 2019 and 2021 and by Csaba Szepesvari, Gang Niu and Sivan Sabato in 2022.
% Modified again in 2023 and 2024 by Sivan Sabato and Jonathan Scarlett.
% Previous contributors include Dan Roy, Lise Getoor and Tobias
% Scheffer, which was slightly modified from the 2010 version by
% Thorsten Joachims & Johannes Fuernkranz, slightly modified from the
% 2009 version by Kiri Wagstaff and Sam Roweis's 2008 version, which is
% slightly modified from Prasad Tadepalli's 2007 version which is a
% lightly changed version of the previous year's version by Andrew
% Moore, which was in turn edited from those of Kristian Kersting and
% Codrina Lauth. Alex Smola contributed to the algorithmic style files.
