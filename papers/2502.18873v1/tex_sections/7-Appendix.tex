
\onecolumn

\section{Additional Experimental Settings}
\label{sec:appendix:exp_setting}

\subsection{Hyper-parameters}
The default sampling parameters for LLM generation are \{temperature=0.75, top\_k=40, top\_p=0.95\}.
Across all MCTS experiments, we set the number of rollouts to 8, the number of sub-questions per node to 4, the number of candidate sub-answers per sub-question to 4, the maximum depth allowed to 5.

\subsection{Dataset Statistics}
Since we make use of the rStar code base~\footnote{\url{https://github.com/zhentingqi/rStar/}} to implement \mosa, we directly adopt the data files released in their git repository.
There are 1,319 instances in GSM8K, 1,000 instances in SVAMP, 500 instances in MATH-500, and 687 instances in StrategyQA.

\section{Additional Implementation Details for Aggregators}
\label{sec:appendix:prompt}

In this section, we will show the basic instruction and several in-context learning demonstrations for aggregators.

% The aggregating operation that was introduced in the main content of this paper is mainly conducted within each search step, which is termed as Action Aggregation.
% Apart from it, we also aggregate all search trajectories after the entire search process terminates, which is named as Final Aggregation.
% In this section, we first give an ablation for the two types of aggregations and show that Action Aggregation is more essential than the other.
% Then, we will show the aggregating prompts and several in-context learning demonstrations.

% \subsection{Ablation on Aggregating}

% \begin{table*}[t]
% \caption{
% }
% \label{tab:appendix:action_or_final_agg}
% \vskip 0.15in
% \begin{center}
% \begin{small}
% % \begin{sc}
% \begin{tabular}{l|cccc|cr}
% \toprule
% Method & GSM8K & SVAMP & MATH & StrategyQA & Avg. \\
% \midrule
% \textsc{MoSA} as Proposers \& Aggregators  &  91.96 &  94.90 &  56.60 &  76.42 & 79.97  \\
% \hspace{10pt} $\circ$ \emph{without} Action Aggregation & 91.51 & 95.20 & 54.40 & 75.69 & 79.20 \\
% \hspace{10pt} $\circ$ \emph{without} Final Aggregation & 91.96 & 94.60 & 56.20 & 75.69 & 79.61 \\
% \bottomrule
% \end{tabular}
% % \end{sc}
% \end{small}
% \end{center}
% \vskip -0.1in
% \end{table*}

% As shwon in Table~\ref{tab:appendix:action_or_final_agg}, removing Action Aggregation leads to an average decrease of 0.77\% while that of removing Final Aggregation is only 0.36\%.
% This suggests that aggregation during each search step is more essential than aggregation after the search terminates.

% \subsection{Prompt Templates for Aggregators}

% % \paragraph{Basic Prompt}


% \textbf{Basic Prompt for Action Aggregation}

% \begin{lstlisting}
% **TASK**:\nYou are an intelligent and supportive AI assistant. You will receive a collection of responses from various AI assistants regarding a query. Your goal is to synthesize these responses into a single, high-quality response. You should first write down your thoughts as \"\#\#\# Thoughts: \textless{}THOUGHTS\textgreater{}.\", and then write down the synthesized response in the format of \"\#\#\# Synthesized Response: \textless{}SYNTHESIZED RESPONSE\textgreater{}.\". If \"The answer is: \textless{}ANSWER\textgreater{}\" appears in the set of responses, you should also write down \"The answer is: \textless{}ANSWER\textgreater{}\"; otherwise, do not write it. \n1. Critically assess the information in the responses, as some may be biased, incorrect, or nonsensical. \n2. Your synthesized response should not merely echo the provided answers but should present a refined, accurate, and thorough response to the prompt. \n3.Ensure that your response is well-organized, coherent, and meets the highest standards of accuracy and reliability.\n
% \end{lstlisting}

% \textbf{Basic Prompt for Final Aggregation}

% \begin{lstlisting}
% **TASK**:\nYou are an intelligent and supportive AI assistant. You will receive a collection of responses from various AI assistants regarding a user's query. Your goal is to synthesize these responses into a single, high-quality response. You should write down your thoughts, and then write down the final answer in the format of \"The answer is: \textless{}ANSWER\textgreater{}.\".\n1. Critically assess the information in the collection of responses, as some may be biased, incorrect, or nonsensical. \n2. Your synthesized response should not merely echo the provided responses but should present a refined, accurate, and thorough response to the user's query. \n3.Ensure that your response is well-organized, coherent, and meets the highest standards of accuracy and reliability.\n
% \end{lstlisting}

% \subsection{ICL Demonstrations for Aggregators}

% \textbf{An ICL Demonstration for Action Aggregation}

% \begin{lstlisting}
% \end{lstlisting}

\newpage

\begin{center}
	\small
	\begin{tcolorbox}[width=1\linewidth,title={\textbf{Basic Instruction for Aggregators}}]
		**TASK**:
        
        You are an intelligent and supportive AI assistant. You will receive a collection of responses from various AI assistants regarding a query. Your goal is to synthesize these responses into a single, high-quality response. You should first write down your thoughts as "\#\#\# Thoughts: \textless{}THOUGHTS\textgreater{}.", and then write down the synthesized response in the format of "\#\#\# Synthesized Response: \textless{}SYNTHESIZED RESPONSE\textgreater{}.". If "The answer is: \textless{}ANSWER\textgreater{}" appears in the set of responses, you should also write down "The answer is: \textless{}ANSWER\textgreater{}"; otherwise, do not write it. 
        
        1. Critically assess the information in the responses, as some may be biased, incorrect, or nonsensical. 
        
        2. Your synthesized response should not merely echo the provided answers but should present a refined, accurate, and thorough response to the prompt. 
        
        3.Ensure that your response is well-organized, coherent, and meets the highest standards of accuracy and reliability.
	\end{tcolorbox}
\end{center}

\newpage


\begin{center}
	\small
	\begin{tcolorbox}[width=1\linewidth,title={\textbf{An Example ICL Demonstration of MATH-500 for Aggregators: User Query}}]

\textbf{User:}


\textbf{[Here is the user query that was fed into those AI assistants]:}

Given a question, please decompose it into sub-questions. For each sub-question, please answer it in a complete sentence, ending with "The answer is \textless a numeric answer \textgreater". When the original question is answerable, please start the subquestion with "Now we can answer the question: \textless{}original question\textgreater{}".

Question 5: Find the product $CD$ of the integers $C$ and $D$ for which

\[\frac{C}{x-3}+\frac{D}{x+8}=\frac{4x-23}{x^2+5x-24}\]for all real values of $x$ except $-8$ and $3$.

Question 5.1: What is the equation obtained by multiplying both sides of the given equation by $(x - 3)(x + 8)$?

Answer 5.1: $(x - 3)(x + 8) \cdot \frac{C}{x-3} + (x - 3)(x + 8) \cdot \frac{D}{x+8} = (x - 3)(x + 8) \cdot \frac{4x-23}{x^2+5x-24}$

This simplifies to $C(x + 8) + D(x - 3) = 4x - 23$.

The answer is $C(x + 8) + D(x - 3) = 4x - 23$.

Question 5.2: What are the values of $C$ and $D$ that satisfy the equation $C(x + 8) + D(x - 3) = 4x - 23$ for all $x$ except $-8$ and $3$?

Answer 5.2:


\textbf{[Here are the responses from various AI assistants]:}

Response 1:

To find the values of $C$ and $D$, we can substitute specific values of $x$ into the equation $C(x + 8) + D(x - 3) = 4x - 23$ to obtain a system of equations. 

Question 5.2.1: What happens if we substitute $x = 3$ into the equation $C(x + 8) + D(x - 3) = 4x - 23$?

Answer 5.2.1: Substituting $x = 3$ gives $C(3 + 8) + D(3 - 3) = 4 \cdot 3 - 23$, which simplifies to $11C = -11$. The answer is $11C = -11$.

Question 5.2.2: What happens if we substitute $x = -8$ into the equation $C(x + 8) + D(x - 3) = 4x - 23$?

Answer 5.2.2: Substituting $x = -8$ gives $C(-8 + 8) + D(-8 - 3) = 4 \cdot (-8) - 23$, which simplifies to $-11D = -59$. The answer is $-11D = -59$.

Question 5.2.3: Now we can answer the question: What are the values of $C$ and $D$?

Answer 5.2.3: Solving the system of equations $11C = -11$ and $-11D = -59$ gives $C = -1$ and $D = 7/11 \cdot 11 = 7$. The answer is -1 and 7 respectively.

Response 2:

To find the values of $C$ and $D$, we can compare the coefficients of the corresponding terms on both sides of the equation.

For the $x^2$ terms, we have $C + D = 0$.

For the $x$ terms, we have $8C - 3D = 4$.

Solving these equations simultaneously, we find that $C = 3$ and $D = -3$.

Therefore, the product $CD = 3 \cdot (-3) = \boxed{-9}$.

Response 3:

To find $C$ and $D$, we can substitute $x = -8$ and $x = 3$ into the equation $C(x + 8) + D(x - 3) = 4x - 23$.

For $x = -8$:
\[C(-8 + 8) + D(-8 - 3) = 4(-8) - 23\]
\[-11D = -55\]
\[D = 5\]

For $x = 3$:
\[C(3 + 8) + D(3 - 3) = 4(3) - 23\]
\[11C = -13\]
\[C = -\frac{13}{11}\]

Now we can answer the question: What is the product $CD$ of the integers $C$ and $D$?
\[CD = \left(-\frac{13}{11}\right)(5) = -\frac{65}{11}\]
The answer is $-\frac{65}{11}$.

Response 4:

To find $C$ and $D$, we can set $x = 3$ and $x = -8$ to get two equations:

1. $C(3 + 8) + D(3 - 3) = 4(3) - 23 \Rightarrow 11C = -11 \Rightarrow C = -1$

2. $C(-8 + 8) + D(-8 - 3) = 4(-8) - 23 \Rightarrow -11D = -55 \Rightarrow D = 5$

The answer is $C = -1$ and $D = 5$.


	\end{tcolorbox}
\end{center}



\newpage


\begin{center}
	\small
	\begin{tcolorbox}[width=1\linewidth,title={\textbf{An Example ICL Demonstration of MATH-500 for Aggregators: Assistant Response}}]

\textbf{Assistant:}

\#\#\# Thoughts:

1. The responses generally attempt to solve the equation \(C(x + 8) + D(x - 3) = 4x - 23\) by substituting specific values for \(x\), namely \(x = 3\) and \(x = -8\).

2. The correct method involves creating a system of linear equations by substituting these values and solving for \(C\) and \(D\).

3. There is a discrepancy in the values of \(C\) and \(D\) obtained in the responses:

   - Some responses correctly find \(C = -1\) and \(D = 5\).
   
   - Another response incorrectly finds \(C = -13/11\) and \(D = 5\).
   
   - Another response finds \(C = 3\) and \(D = -3\), which is not consistent with the given equation.

4. The correct values should satisfy both equations derived from the substitutions.

\#\#\# Synthesized Response:
 
 The equation \(C(x + 8) + D(x - 3) = 4x - 23\) can be solved by substituting specific values for \(x\):

1. Substituting \(x = 3\):
\[C(3 + 8) + D(3 - 3) = 4(3) - 23\]
\[11C = -11 \]
\[C = -1\]

2. Substituting \(x = -8\):
\[C(-8 + 8) + D(-8 - 3) = 4(-8) - 23\]
\[-11D = -55 \]
\[D = 5\]

Thus, the values of \(C\) and \(D\) are \(C = -1\) and \(D = 5\). The product \(CD\) is:
\[CD = (-1) \cdot 5 = -5\]

The answer is: -5

	\end{tcolorbox}
\end{center}




\newpage





\begin{center}
	\small
	\begin{tcolorbox}[width=1\linewidth,title={\textbf{An Example ICL Demonstration of StrategyQA for Aggregators: User Query}}]

\textbf{User:}


\textbf{[Here is the user query that was fed into those AI assistants]:}

Given a question, please decompose it into sub-questions. For each sub-question, please answer it in a complete sentence. When the original question is answerable, please start the sub-question with "Now we can answer the question: \textless{}original question\textgreater{}".

Question 5: Johnny Carson: American talk show host and comedian. Could Johnny Carson's children fill out a water polo team?

Question 5.1: How many children did Johnny Carson have?

Answer 5.1: Johnny Carson had three children: two daughters, Catherine and Christie, and a son, Jeff.

Question 5.2: How many players are needed to form a water polo team?

Answer 5.2:

\textbf{[Here are the responses from various AI assistants]:}

Response 1:

A standard water polo team consists of 13 players.

Response 2:

A water polo team consists of seven players, including one goalkeeper and six field players.

Response 3:

A water polo team consists of 13 players: 1 goalkeeper and 12 field players (7 field players and 5 substitutes).

Response 4:

A standard water polo team consists of 13 players, including 7 field players and 6 substitutes.

	\end{tcolorbox}
\end{center}


\newpage



\begin{center}
	\small
	\begin{tcolorbox}[width=1\linewidth,title={\textbf{An Example ICL Demonstration of StrategyQA for Aggregators: Assistant Response}}]

\textbf{Assistant:}

\#\#\# Thoughts:

1. The responses contain valuable information about the composition of a water polo team. 

2. There is a consensus that a water polo team consists of 13 players in total. However, there is some variation in how these 13 players are categorized.

3. The key point to clarify is how many players are in the water at one time and how many are substitutes.

4. It appears that in the water during play, there are 7 players (including the goalkeeper), and the remaining players are substitutes.

\#\#\# Synthesized Response:

The synthesized response is: 

To form a water polo team, a total of 13 players are required. This team includes 7 players who are in the water during play, comprising one goalkeeper and six field players. The remaining 6 players serve as substitutes.

	\end{tcolorbox}
\end{center}