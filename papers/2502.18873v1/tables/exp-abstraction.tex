\begin{table*}[t]
\caption{
Abstractions of various approaches. Under similar hyperparameter settings, we have $n_1 < n_2 < p n_1$ and $n_3 > n_2$, where $1 < p < 2$. As the number of proposed sub-answers increases, $n_1$ would increases and $p$ will decreases. In other words, with larger numbers of returning sequences in \textsc{MoSA}, the computational overhead introduced by aggregators becomes smaller.
}
\label{tab:abstraction}
\vskip 0.15in
\begin{center}
\begin{small}
% \begin{sc}
\begin{tabular}{lcccc}
\toprule
Method & Action Space & \# Proposers & \# Aggregators & \# Forward Calls \\
\midrule
Few-shot CoT & $A2$ & Single & None & 1 \\
Self-Consistency@$n$ (\emph{w/} a single LLM) & $A2$ & Single & None & $n$ \\
Self-Consistency@$n$ (\emph{w/} all LLMs) & $A2$ & Multi & None & $n$ \\
RAP (\emph{w/o} aggregators) & $A\{2,3\}$ & Single & None & $n_1$ \\
RAP (\emph{w/} aggregators) & $A\{2,3\}$ & Single & Single & $n_2$ \\
\ourmethod{} (\emph{w/o} aggregators) & $A\{2,3\}$ & Multi & None & $n_1$ \\
\ourmethod{} (\emph{w/} aggregators) & $A\{2,3\}$ & Multi & Multi & $n_2$ \\
rStar (majority voting) & $A\{1,2,3,4,5\}$ & Single & None & $n_3$ \\
rStar (majority voting) + \ourmethod{} (\emph{w/o} aggregators) & $A\{1,2,3,4,5\}$ & Multi & None & $n_3$ \\
\bottomrule
\end{tabular}
% \end{sc}
\end{small}
\end{center}
\vskip -0.1in
\end{table*}
