%%%%%%%%%%%%%%%%%%%%%%%%%%%%%%%%%%%%%%%%%%%%%%%%%%%%%%%%%%%%%%%%%%%%%%%%%%%%%%%%
%2345678901234567890123456789012345678901234567890123456789012345678901234567890
%        1         2         3         4         5         6         7         8

\documentclass[letterpaper, 10 pt, journal, twoside]{ieeetran}  % Comment this line out if you need a4paper

%\documentclass[a4paper, 10pt, conference]{ieeeconf}      % Use this line for a4 paper

\IEEEoverridecommandlockouts                              % This command is only needed if 
                                                          % you want to use the \thanks command

%\overrideIEEEmargins                                      % Needed to meet printer requirements.

%In case you encounter the following error:
%Error 1010 The PDF file may be corrupt (unable to open PDF file) OR
%Error 1000 An error occurred while parsing a contents stream. Unable to analyze the PDF file.
%This is a known problem with pdfLaTeX conversion filter. The file cannot be opened with acrobat reader
%Please use one of the alternatives below to circumvent this error by uncommenting one or the other
%\pdfobjcompresslevel=0
%\pdfminorversion=4

% See the \addtolength command later in the file to balance the column lengths
% on the last page of the document


\usepackage{graphics} 
\usepackage{amsmath} % assumes amsmath package installed
\usepackage{amssymb} 
\usepackage{subfiles}
\usepackage{multirow}
\usepackage{makecell}
\usepackage{soul}
\usepackage{color}
\usepackage{wasysym}
\usepackage{colortbl}
%\usepackage[final]{changes}
%\usepackage{changes}
\usepackage{xspace}
\usepackage{mathtools}
\DeclarePairedDelimiter{\ceil}{\lceil}{\rceil}
\usepackage{etoolbox}

\newcommand{\naive}{na\"ive\xspace}
\newcommand{\Naive}{Na\"ive\xspace}
    
\usepackage{tikz}
\usepackage{xcolor}
\newcommand*\circled[1]{\tikz[baseline=(char.base)]{
            \node[shape=rectangle,fill,inner sep=1pt] (char) {\textcolor{white}{#1}};}}
            
\usepackage{hyperref}
\usepackage{cleveref}
\usepackage[T1]{fontenc}
\usepackage[linesnumbered, ruled]{algorithm2e}
\DontPrintSemicolon
%\SetAlCapFnt{\large}
%\SetAlCapNameFnt{\large}
\SetKwComment{Comment}{\footnotesize$\triangleright\,\,$}{}
\SetInd{0.25em}{0.8em}
\newcommand\mycommfont[1]{\footnotesize\rmfamily\textcolor{gray}{\textit{#1}}}
\SetCommentSty{mycommfont}
\newcommand\myfuncfont[1]{\footnotesize\rmfamily #1}
\SetFuncSty{myfuncfont}
\setlength{\algomargin}{1.5em}
\SetAlCapFnt{\footnotesize}
\SetAlCapNameFnt{\footnotesize}

\crefformat{figure}{Figure #1}
\crefformat{section}{\S#2#1#3} % see manual of cleveref, section 8.2.1
\crefformat{sections}{\S}
\crefformat{subsection}{\S#2#1#3}
\crefformat{subsubsection}{\S#2#1#3}

\urlstyle{same}

\setlength{\skip\footins}{5pt}

\markboth{}
{Wang \MakeLowercase{\textit{et al.}}: Anytime Planning for End-Effector Trajectory Tracking} 

\makeatletter
\newcommand{\algorithmfootnote}[2][\footnotesize]{%
  \let\old@algocf@finish\@algocf@finish% Store algorithm finish macro
  \def\@algocf@finish{\old@algocf@finish% Update finish macro to insert "footnote"
    \leavevmode\rlap{\begin{minipage}{\linewidth}
    #1#2
    \end{minipage}}%
  }%
}
\patchcmd{\algocf@makecaption@ruled}{\hsize}{\textwidth}{}{} % Caption to stretch full text width
\patchcmd{\@algocf@start}{-1.5em}{0em}{}{} % For // to right margin
\makeatother

%
%
\title{
\textcolor{red}{-- Preprint --} \\
Anytime Planning for End-Effector Trajectory Tracking
}
%
%\titlerunning{Abbreviated paper title}
% If the paper title is too long for the running head, you can set
% an abbreviated paper title here


\author{
Yeping Wang and Michael Gleicher% <-this % stops a space

\thanks{
This work was supported in part by National Science Foundation under Award 2007436 and in part by the Los Alamos National Laboratory and the Department of Energy.}
\thanks{ Both authors are with the Department of Computer Sciences, University of Wisconsin-Madison, Madison, WI 53706, USA $\hspace*{0.8in}$
{\tt\footnotesize [yeping|gleicher]@cs.wisc.edu}}%
\vspace{-0mm}
}
%
% First names are abbreviated in the running head.
% If there are more than two authors, 'et al.' is used.
%

%            % typeset the header of the contribution
%


\begin{document}
\maketitle  
\begin{abstract}
End-effector trajectory tracking algorithms find joint motions that drive robot manipulators to track reference trajectories. 
In practical scenarios, anytime algorithms are preferred for their ability to quickly generate initial motions and continuously refine them over time.
In this paper, we present an algorithmic framework that adapts common graph-based trajectory tracking algorithms to be anytime and enhances their efficiency and effectiveness. 
Our key insight is to identify guide paths that approximately track the reference trajectory and strategically bias sampling toward the guide paths. 
We demonstrate the effectiveness of the proposed framework by restructuring two existing graph-based trajectory tracking algorithms and evaluating the updated algorithms in three experiments.  

\end{abstract}
%
%

\begin{IEEEkeywords}
Constrained Motion Planning, Motion and Path Planning
\end{IEEEkeywords}

%
\vspace{-5mm}
\section{Introduction}
\vspace{-1mm}
\subfile{10_introduction.tex}


\vspace{-3mm}
\section{Related Work}
\vspace{-1mm}
\subfile{20_related_work.tex}


\vspace{-1mm}
\section{Technical Overview} \label{sec: technical_overview}
\vspace{-1mm}
\subfile{30_problem_statement.tex}


\vspace{-1mm}
\section{Technical Details} \label{sec: technical_details}
\vspace{-1mm}
\subfile{40_technical_details.tex}


\vspace{-3mm}
\section{Experiments} \label{sec: experiments}
\vspace{-1mm}
\subfile{50_experiments.tex}


\vspace{-1mm}
\section{Discussion}
\vspace{-1mm}
\subfile{60_discussion.tex}

%given a sufficiently large set of uniformly sampled configurations, any path can be traced arbitrarily well. \cite{schmerling2015optimal} Thm. IV.5



%\clearpage
%
% ---- Bibliography ----
%
% BibTeX users should specify bibliography style 'splncs04'.
% References will then be sorted and formatted in the correct style.
%
\bibliographystyle{IEEEtran}
\bibliography{citation}
%

\end{document}
