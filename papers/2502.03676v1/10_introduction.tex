\IEEEPARstart{M}{any} applications require a robot manipulator to accurately and smoothly move its end-effector along a specific trajectory. These applications involve welding, sanding, polishing, painting, and additive manufacturing. For redundant manipulators or applications with tolerances, there exist infinitely many possible motions to track a reference end-effector trajectory. Thus, planning algorithms seek to find optimal motions among these possibilities based on criteria such as minimal joint movement \cite{rakita2019stampede}, minimal end-effector error \cite{rakita2019stampede,holladay2019minimizing},  minimal maximum joint velocity \cite{morgan2024cppflow}, or minimal number of reconfigurations, \textit{i.e.}, instances where the robot pauses task execution and restarts from another joint configuration \cite{wang2024iklink,yang2022optimal}. 

To track an end-effector trajectory based on some optimality criteria, common graph-based approaches \cite{rakita2019stampede,morgan2024cppflow,wang2024iklink, Descartes,niyaz2020following}   first \textit{sample} inverse kinematics (IK) solutions for each waypoint on the reference trajectory. These IK solutions serve as vertices to construct a layered graph, where each layer corresponds to a waypoint. Edges are added to link IK solutions in adjacent layers, and these edges' weights are defined according to the optimality criteria. Finally, a graph search method is utilized to find the optimal motion. The success of this approach relies heavily on the density of the sampling to sufficiently discretize the solution space. Existing graph-based trajectory tracking approaches often rely on uniformly dense IK sampling, which leads to long delays in finding initial solutions. In practical scenarios with limited computational resources, anytime algorithms are preferred because they can generate feasible solutions quickly and progressively refine them over time. This flexibility allows anytime algorithms to be stopped at any time, enabling trade-offs between solution quality and computation time.

In this work, we present an anytime algorithmic framework that enhances existing approaches to efficiently and effectively generate robot motions to track reference end-effector trajectories. 
Our framework enhances existing methods by incorporating a heuristic to prioritize samples that are likely to lead to good solutions. 
Our key insight is to identify guide paths that approximately track the reference trajectory and strategically bias sampling toward the guide paths, which progressively densifies the graph.  This approach enables frequent identification of effective solutions within a significantly reduced timeframe. Over time, the graph converges towards the dense graph used in conventional algorithms. 
We show that, in the worst case,  our anytime framework converges to the conventional framework, although in practice its ability to sample strategically usually leads to better results in less time.
The algorithmic framework is independent of specific IK sampling algorithms, optimality criteria, or graph searching algorithms. We provide an open-source implementation of our framework
\footnote{Open-sourced code \url{https://github.com/uwgraphics/IKLink/tree/anytime}}.

The central contribution of this work is a guided anytime algorithmic framework (\Cref{sec: alg_overview,sec: technical_details}). To explain our approach, we first describe the conventional sequential framework and a \naive anytime framework as baselines (\Cref{sec: conventional,sec:naive}).
We evaluated our framework through three experiments (\cref{sec: experiments}). First, we applied it to two end-effector trajectory tracking algorithms, Stampede \cite{rakita2019stampede} and IKLink \cite{wang2024iklink}, originally based on the conventional framework. Our results show that our framework accelerates both algorithms, generating solutions with less computation time than the conventional and \naive frameworks, while matching or exceeding their solution quality.
Additionally, we applied it to semi-constrained trajectory tracking, which expands the solution space by allowing tolerances and requiring more IK samplings. Unlike baseline frameworks, which are inefficient due to \naive sampling, our framework enhances IKLink's efficiency and effectiveness for this task.



