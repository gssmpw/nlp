
\subsection{End-Effector Trajectory Tracking}
\vspace{-1mm}

The end-effector trajectory tracking problem, also known as path-wise inverse kinematics \cite{rakita2019stampede}, task-space non-revisiting tracking \cite{yang2022optimal}, or task-constrained motion planning \cite{cefalo2013task}, involves various approaches based on input type. 
Specifically, trajectory tracking methods \cite{rakita2019stampede, holladay2019minimizing, wang2024iklink, yang2022optimal, Descartes} take a timestamped trajectory as input, whereas path tracking methods \cite{morgan2024cppflow, niyaz2020following} track a sequence of coordinates without timing.


End-effector trajectory tracking methods can be broadly categorized into two groups: trajectory optimization and graph-based approaches. Trajectory optimization directly optimizes a joint-space trajectory while satisfying constraints. For example, Holladay \textit{et al.} \cite{holladay2016distance} utilize TrajOpt \cite{schulman2014motion}, a trajectory optimization method, to minimize the Fréchet distances between current solutions and a given reference trajectory. Instead of Fréchet distances, Torm \cite{kang2020torm} minimizes the summed Euclidean distance between corresponding waypoints on the current and reference trajectories. 
These optimization methods require an initial trajectory as a starting point, are sensitive to this initial trajectory \cite{yoon2023learning}, and are prone to be stuck in local minima \cite{holladay2016distance}.  
Meanwhile, graph-based approaches \cite{rakita2019stampede, morgan2024cppflow, wang2024iklink, Descartes,niyaz2020following} construct a hierarchical graph where vertices represent inverse kinematics (IK) solutions. A path within the graph defines a robot motion by sequentially connecting the IK solutions. We will review these approaches in \cref{sec: conventional} as part of the discussion on the conventional algorithmic framework. Below, we discuss IK sampling strategies in graph-based approaches. 


\vspace{-3mm}
\subsection{IK Sampling in Graph-based Trajectory Tracking}
In graph-based end-effector trajectory tracking, the majority of the computation time is consumed by sampling inverse kinematics (IK) solutions and establishing connections between them \cite{malhan2022generation}. 
Some prior work \cite{rakita2019stampede,wang2024iklink, Descartes,niyaz2020following} exhaustively samples IK solutions before searching for a path, resulting in long computation time before obtaining initial solutions. \cite{holladay2019minimizing,malhan2022generation}.
Meanwhile, other approaches use incremental sampling to accelerate the procedure. 
For example, CppFlow \cite{morgan2024cppflow} leverages a generative IK solver, IKFlow \cite{ames2022ikflow}, which rapidly produces a diverse set of IK solutions by learning the distribution of uniformly sampled IK solutions. If the graph search algorithm fails to find a viable motion, the algorithm incrementally samples additional IK solutions to density the graph. 
In contrast to uniform sampling, Malhan \textit{et al.} \cite{malhan2022generation} use continuity in Cartesian space as a heuristic for IK selection during incremental graph construction. However, the heuristic is applicable solely to non-redundant robotic arms. 
Holladay \textit{et al.} \cite{holladay2019minimizing} presents strategies to balance uniform sampling with targeted sampling around certain areas. However, their approach is designed specifically to minimize the Fréchet distance in Cartesian space, and it remains unclear how to adapt their strategies to address objectives such as minimizing joint movements. 
Our method, like Holladay \textit{et al.} \cite{holladay2019minimizing}, balances uniform and targeted sampling, but differs by sampling around a guide path that roughly tracks the reference trajectory. 



\vspace{-3mm}
\subsection{Semi-Constrained End-Effector Trajectory Tracking}

In certain trajectory tracking scenarios, precise end-effector movements in all six degrees of freedom are not necessary, \textit{i.e.}, allowing for certain \textit{tolerances} in Cartesian space. For instance, in a drawing task, the robot may be permitted to tilt or rotate the pen as long as the pen tip position remains precise and the motion remains collision-free. This problem is called semi-constrained end-effector trajectory tracking.
Due to tolerances, the solution space expands, requiring more IK solutions to cover the solution space. 
Graph-based approaches with uniform sampling, \textit{e.g.}, Descartes \cite{Descartes}, have a significant increase in computational load when using redundant robots or having tolerances on multiple degrees of freedom \cite{de2017cartesian}. Therefore, developing a more efficient algorithm is essential for semi-constrained end-effector tracking. In this work, we present a guided anytime algorithmic framework to accelerate graph-based trajectory tracking approaches. We show how this framework can efficiently and effectively compute motions to track semi-constrained end-effector trajectory in \cref{sec: experiments}.

