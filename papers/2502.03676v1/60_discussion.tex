This section summarizes the limitations, implications, and conclusions of our work.

\textit{Limitations}:
Our work has several limitations that suggest future research directions. First, our algorithm lacks a quality-based stopping threshold; users can specify the maximum number of iterations, running time, or sampling budget, but the algorithm does not automatically stop upon convergence. Future work should explore the effects of adaptive stopping thresholds such as when the guide path no longer updates with iterations. 
Second, our method uses a heuristic to bias sampling in graph-based approaches. While we provide empirical evidence to demonstrate its ability to efficiently produce high-quality solutions, we cannot guarantee consistent performance across all scenarios. In certain scenarios, the heuristic may be misleading, causing the method to be slower than the naive anytime framework. Future work may provide a more rigorous understanding of the heuristic's effectiveness.
Third, while we expect that our framework generalizes across many graph-based tracking algorithms, we have only demonstrated it on two. 
Finally, the present work does not consider dynamic constraints, focusing solely on the velocity constraints of robot joints. Future work should consider incorporating acceleration constraints and the inertia of the robot to improve the quality of generated motions.

\textit{Implications}:
The guided anytime framework presented in this work enables graph-based approaches to efficiently and effectively find motions that accurately move a manipulator's end-effector along reference trajectories. It quickly generates initial solutions and continues to refine them as needed.
We have demonstrated that it accelerates and improves solutions over prior methods and enables solving problems that a prior method cannot.
The proposed framework is particularly beneficial for scenarios where end-effector trajectories are complex or have specific tolerances. 

\textit{Conclusion:} This paper presented an anytime framework that enables graph-based trajectory tracking algorithms to efficiently and effectively find robot motions. Our framework generates trajectories of equal or better quality in less time than previous approaches. 