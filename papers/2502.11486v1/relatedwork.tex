\section{Related work}
\subsection{Degeneracy Detection}


Degeneracy detection is able to identify the degeneracy of sensor in real time, so the system can immediately deploy anti-degeneracy strategies to improve the accuracy using the information about degeneracy.
Lee $\textit{et al.}$ \cite{lee2024switch} detected degeneracy by checking the convergence of the optimization process based on predefined thresholds derived from physical assumptions and statistical significance. Tuna $\textit{et al.}$ \cite{tuna2023x} conducted degeneracy detection through fine-grained localizability analysis based on point cloud correspondences, and determines the degeneracy directions by analyzing the principal components of the optimization directions. Zhang $\textit{et al.}$ \cite{zhang2016degeneracy} achieved degeneracy detection by analyzing the geometric structure of the constraints in the optimization problem and identifying the degeneracy directions through the eigenvalues and eigenvectors of the Hessian matrix. Zhou $\textit{et al.}$ \cite{zhou2020lidar}  estimated the sensor state by minimizing the sum of Mahalanobis norms of all measurement residuals. Chen $\textit{et al.}$ \cite{10816047} proposed a P2d degeneracy detection algorithm that uses adaptive voxel segmentation to integrate local geometric features and calculates degeneracy factors from point cloud distribution changes between frames. However, these methods may perform well in specific environments, yet be less robust when faced with conditions such as rapidly changing environments, and relies on a pre-set threshold.

\subsection{Anti-Degeneracy Strategies}


Anti-degeneracy strategy is the key element to improve SLAM accuracy in degraded environments.
Li $\textit{et al.}$ \cite{li2022intensity} dealt with the degeneracy problem through geometric and intensity-based feature extraction by designing two multi-weight functions to fully extract the features of planar and edge points. Zhang $\textit{et al.}$ \cite{zhang2024lvio} presented an innovative tightly coupled lidar-vision-inertial odometry, which enables accurate state estimation in degraded environments through sensor fusion strategies. Zhang $\textit{et al.}$ \cite{zhang2023graph} utilized ARTag as a visual marker to assist in localization, which employs a bitmap optimization method to reduce the error. Lee $\textit{et al.}$ \cite{lee2024switch} proposed a switching-based SLAM that achieves high accuracy by switching from lidar to visual odometry upon detection of lidar degeneracy. 
However, these methods have requirements on the geometry of environment and the strength of features. Meanwhile, the multi-sensor fusion strategy increases the deployment cost, the parameters need to be adjusted for adaptation, which leads to the fact that these methods do not have good robustness.