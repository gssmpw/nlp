\documentclass[12pt]{article}
\usepackage[T1]{fontenc}
%\usepackage[utf8]{inputenc}
% %
% --- inline annotations
%
\newcommand{\red}[1]{{\color{red}#1}}
\newcommand{\todo}[1]{{\color{red}#1}}
\newcommand{\TODO}[1]{\textbf{\color{red}[TODO: #1]}}
% --- disable by uncommenting  
% \renewcommand{\TODO}[1]{}
% \renewcommand{\todo}[1]{#1}



\newcommand{\VLM}{LVLM\xspace} 
\newcommand{\ours}{PeKit\xspace}
\newcommand{\yollava}{Yo’LLaVA\xspace}

\newcommand{\thisismy}{This-Is-My-Img\xspace}
\newcommand{\myparagraph}[1]{\noindent\textbf{#1}}
\newcommand{\vdoro}[1]{{\color[rgb]{0.4, 0.18, 0.78} {[V] #1}}}
% --- disable by uncommenting  
% \renewcommand{\TODO}[1]{}
% \renewcommand{\todo}[1]{#1}
\usepackage{slashbox}
% Vectors
\newcommand{\bB}{\mathcal{B}}
\newcommand{\bw}{\mathbf{w}}
\newcommand{\bs}{\mathbf{s}}
\newcommand{\bo}{\mathbf{o}}
\newcommand{\bn}{\mathbf{n}}
\newcommand{\bc}{\mathbf{c}}
\newcommand{\bp}{\mathbf{p}}
\newcommand{\bS}{\mathbf{S}}
\newcommand{\bk}{\mathbf{k}}
\newcommand{\bmu}{\boldsymbol{\mu}}
\newcommand{\bx}{\mathbf{x}}
\newcommand{\bg}{\mathbf{g}}
\newcommand{\be}{\mathbf{e}}
\newcommand{\bX}{\mathbf{X}}
\newcommand{\by}{\mathbf{y}}
\newcommand{\bv}{\mathbf{v}}
\newcommand{\bz}{\mathbf{z}}
\newcommand{\bq}{\mathbf{q}}
\newcommand{\bff}{\mathbf{f}}
\newcommand{\bu}{\mathbf{u}}
\newcommand{\bh}{\mathbf{h}}
\newcommand{\bb}{\mathbf{b}}

\newcommand{\rone}{\textcolor{green}{R1}}
\newcommand{\rtwo}{\textcolor{orange}{R2}}
\newcommand{\rthree}{\textcolor{red}{R3}}
\usepackage{amsmath}
%\usepackage{arydshln}
\DeclareMathOperator{\similarity}{sim}
\DeclareMathOperator{\AvgPool}{AvgPool}

\newcommand{\argmax}{\mathop{\mathrm{argmax}}}     


\linespread{1.25}


% \usepackage{fontspec}
\usepackage{pbox}

\usepackage{tikz}
\usetikzlibrary{shapes}
%\usepackage{epstopdf}
\usepackage{pgfplots}
\pgfplotsset{width=10cm,compat=1.9}

% We will externalize the figures
\usepgfplotslibrary{external}
\tikzexternalize

\usepackage{tikz}
\usetikzlibrary{patterns,decorations.pathreplacing,calc,fadings,calc,shapes.callouts,shapes.arrows,backgrounds,spy,shadows,decorations.markings}
\immediate\write18{mkdir -p tikz}

\usetikzlibrary{intersections}
\usetikzlibrary{positioning}
\usetikzlibrary{decorations.pathreplacing}
\usetikzlibrary{matrix}
\usetikzlibrary{calc}
\usetikzlibrary{arrows,arrows.meta}
\usetikzlibrary{arrows,shapes,trees,backgrounds,shadows}
\usetikzlibrary{decorations.pathreplacing}
\usetikzlibrary{decorations.pathmorphing} % noisy shapes
%\usetikzlibrary{fit}					% fitting shapes to coordinates
\usetikzlibrary{backgrounds,snakes}
\usetikzlibrary{decorations.markings}
\usetikzlibrary{positioning}
\usetikzlibrary{plotmarks}
\usetikzlibrary{trees}
\usetikzlibrary{patterns}
\usetikzlibrary{spy}
\usetikzlibrary{calc,matrix}


\newcommand{\noinfo}{ \textsf{NI}  }
\newcommand{\ds}{{\scaleto{\mathsf{DS}}{4pt}}}
\newcommand{\is}{{\scaleto{\mathsf{IS}}{4pt}}}
\usepackage{scalerel}

\usepackage{graphicx} % Required for inserting images
\usepackage[margin=1in]{geometry}
\usepackage{amsmath}
\usepackage{caption}
\usepackage{hyperref}
\usepackage{subcaption}
\usepackage{xcolor}
\usepackage{amssymb}
\usepackage[numbers]{natbib}
\usepackage{multirow}
\usepackage{soul}

\usepackage{amsmath, amsfonts, dsfont}
\usepackage{amsthm}
% \usepackage{enumitem}
% \usepackage{subfig}
\usepackage{hyperref}
\usepackage{graphicx}
\usepackage{placeins}
\usepackage[ruled]{algorithm2e}

\newtheorem{theorem}{Theorem}
\newtheorem{lemma}{Lemma}
\newtheorem{proposition}{Proposition}
\newtheorem{corollary}{Corollary}
\newtheorem{claim}{Claim}
\newtheorem{conjecture}{Conjecture}
\newtheorem{hypothesis}{Hypothesis}
\newtheorem{assumption}{Assumption}
% \theoremstyle{definition}
\newtheorem{remark}{Remark}
\newtheorem{example}{Example}
\newtheorem{problem}{Problem}
\newtheorem{definition}{Definition}

\newcommand{\argmin}{\mathrm{argmin}}
\newcommand{\argmax}{\mathrm{argmax}}

% Private macros here (check that there is no clash with the style)
\SetKwInput{KwInputs}{Inputs}
\SetKwInput{KwOutputs}{Outputs}
\SetKwRepeat{Repeat}{repeat}{until}

% !TeX root = main.tex 


\newcommand{\lnote}{\textcolor[rgb]{1,0,0}{Lydia: }\textcolor[rgb]{0,0,1}}
\newcommand{\todo}{\textcolor[rgb]{1,0,0.5}{To do: }\textcolor[rgb]{0.5,0,1}}


\newcommand{\state}{S}
\newcommand{\meas}{M}
\newcommand{\out}{\mathrm{out}}
\newcommand{\piv}{\mathrm{piv}}
\newcommand{\pivotal}{\mathrm{pivotal}}
\newcommand{\isnot}{\mathrm{not}}
\newcommand{\pred}{^\mathrm{predict}}
\newcommand{\act}{^\mathrm{act}}
\newcommand{\pre}{^\mathrm{pre}}
\newcommand{\post}{^\mathrm{post}}
\newcommand{\calM}{\mathcal{M}}

\newcommand{\game}{\mathbf{V}}
\newcommand{\strategyspace}{S}
\newcommand{\payoff}[1]{V^{#1}}
\newcommand{\eff}[1]{E^{#1}}
\newcommand{\p}{\vect{p}}
\newcommand{\simplex}[1]{\Delta^{#1}}

\newcommand{\recdec}[1]{\bar{D}(\hat{Y}_{#1})}





\newcommand{\sphereone}{\calS^1}
\newcommand{\samplen}{S^n}
\newcommand{\wA}{w}%{w_{\mathfrak{a}}}
\newcommand{\Awa}{A_{\wA}}
\newcommand{\Ytil}{\widetilde{Y}}
\newcommand{\Xtil}{\widetilde{X}}
\newcommand{\wst}{w_*}
\newcommand{\wls}{\widehat{w}_{\mathrm{LS}}}
\newcommand{\dec}{^\mathrm{dec}}
\newcommand{\sub}{^\mathrm{sub}}

\newcommand{\calP}{\mathcal{P}}
\newcommand{\totspace}{\calZ}
\newcommand{\clspace}{\calX}
\newcommand{\attspace}{\calA}

\newcommand{\Ftil}{\widetilde{\calF}}

\newcommand{\totx}{Z}
\newcommand{\classx}{X}
\newcommand{\attx}{A}
\newcommand{\calL}{\mathcal{L}}



\newcommand{\defeq}{\mathrel{\mathop:}=}
\newcommand{\vect}[1]{\ensuremath{\mathbf{#1}}}
\newcommand{\mat}[1]{\ensuremath{\mathbf{#1}}}
\newcommand{\dd}{\mathrm{d}}
\newcommand{\grad}{\nabla}
\newcommand{\hess}{\nabla^2}
\newcommand{\argmin}{\mathop{\rm argmin}}
\newcommand{\argmax}{\mathop{\rm argmax}}
\newcommand{\Ind}[1]{\mathbf{1}\{#1\}}

\newcommand{\norm}[1]{\left\|{#1}\right\|}
\newcommand{\fnorm}[1]{\|{#1}\|_{\text{F}}}
\newcommand{\spnorm}[2]{\left\| {#1} \right\|_{\text{S}({#2})}}
\newcommand{\sigmin}{\sigma_{\min}}
\newcommand{\tr}{\text{tr}}
\renewcommand{\det}{\text{det}}
\newcommand{\rank}{\text{rank}}
\newcommand{\logdet}{\text{logdet}}
\newcommand{\trans}{^{\top}}
\newcommand{\poly}{\text{poly}}
\newcommand{\polylog}{\text{polylog}}
\newcommand{\st}{\text{s.t.~}}
\newcommand{\proj}{\mathcal{P}}
\newcommand{\projII}{\mathcal{P}_{\parallel}}
\newcommand{\projT}{\mathcal{P}_{\perp}}
\newcommand{\projX}{\mathcal{P}_{\mathcal{X}^\star}}
\newcommand{\inner}[1]{\langle #1 \rangle}

\renewcommand{\Pr}{\mathbb{P}}
\newcommand{\Z}{\mathbb{Z}}
\newcommand{\N}{\mathbb{N}}
\newcommand{\R}{\mathbb{R}}
\newcommand{\E}{\mathbb{E}}
\newcommand{\F}{\mathcal{F}}
\newcommand{\var}{\mathrm{var}}
\newcommand{\cov}{\mathrm{cov}}


\newcommand{\calN}{\mathcal{N}}

\newcommand{\jccomment}{\textcolor[rgb]{1,0,0}{C: }\textcolor[rgb]{1,0,1}}
\newcommand{\fracpar}[2]{\frac{\partial #1}{\partial  #2}}

\newcommand{\A}{\mathcal{A}}
\newcommand{\B}{\mat{B}}
%\newcommand{\C}{\mat{C}}

\newcommand{\I}{\mat{I}}
\newcommand{\M}{\mat{M}}
\newcommand{\D}{\mat{D}}
%\newcommand{\U}{\mat{U}}
\newcommand{\V}{\mat{V}}
\newcommand{\W}{\mat{W}}
\newcommand{\X}{\mat{X}}
\newcommand{\Y}{\mat{Y}}
\newcommand{\mSigma}{\mat{\Sigma}}
\newcommand{\mLambda}{\mat{\Lambda}}
\newcommand{\e}{\vect{e}}
\newcommand{\g}{\vect{g}}
\renewcommand{\u}{\vect{u}}
\newcommand{\w}{\vect{w}}
\newcommand{\x}{\vect{x}}
\newcommand{\y}{\vect{y}}
\newcommand{\z}{\vect{z}}
\newcommand{\fI}{\mathfrak{I}}
\newcommand{\fS}{\mathfrak{S}}
\newcommand{\fE}{\mathfrak{E}}
\newcommand{\fF}{\mathfrak{F}}

\newcommand{\Risk}{\mathcal{R}}

\renewcommand{\L}{\mathcal{L}}
\renewcommand{\H}{\mathcal{H}}

\newcommand{\cn}{\kappa}
\newcommand{\nn}{\nonumber}


\newcommand{\Hess}{\nabla^2}
\newcommand{\tlO}{\tilde{O}}
\newcommand{\tlOmega}{\tilde{\Omega}}

\newcommand{\calF}{\mathcal{F}}
\newcommand{\fhat}{\widehat{f}}
\newcommand{\calS}{\mathcal{S}}

\newcommand{\calX}{\mathcal{X}}
\newcommand{\calY}{\mathcal{Y}}
\newcommand{\calD}{\mathcal{D}}
\newcommand{\calZ}{\mathcal{Z}}
\newcommand{\calA}{\mathcal{A}}
\newcommand{\fbayes}{f^B}
\newcommand{\func}{f^U}


\newcommand{\bayscore}{\text{calibrated Bayes score}}
\newcommand{\bayrisk}{\text{calibrated Bayes risk}}

\newtheorem{example}{Example}[section]
\newtheorem{exc}{Exercise}[section]
%\newtheorem{rem}{Remark}[section]

\newtheorem{theorem}{Theorem}[section]
\newtheorem{definition}{Definition}
\newtheorem{proposition}[theorem]{Proposition}
\newtheorem{corollary}[theorem]{Corollary}

\newtheorem{remark}{Remark}[section]
\newtheorem{lemma}[theorem]{Lemma}
\newtheorem{claim}[theorem]{Claim}
\newtheorem{fact}[theorem]{Fact}
\newtheorem{assumption}{Assumption}

\newcommand{\iidsim}{\overset{\mathrm{i.i.d.}}{\sim}}
\newcommand{\unifsim}{\overset{\mathrm{unif}}{\sim}}
\newcommand{\sign}{\mathrm{sign}}
\newcommand{\wbar}{\overline{w}}
\newcommand{\what}{\widehat{w}}
\newcommand{\KL}{\mathrm{KL}}
\newcommand{\Bern}{\mathrm{Bernoulli}}
\newcommand{\ihat}{\widehat{i}}
\newcommand{\Dwst}{\calD^{w_*}}
\newcommand{\fls}{\widehat{f}_{n}}


\newcommand{\brpi}{\pi^{br}}
\newcommand{\brtheta}{\theta^{br}}

% \newcommand{\M}{\mat{M}}
% \newcommand\Mmh{\mat{M}^{-1/2}}
% \newcommand{\A}{\mat{A}}
% \newcommand{\B}{\mat{B}}
% \newcommand{\C}{\mat{C}}
% \newcommand{\Et}[1][t]{\mat{E_{#1}}}
% \newcommand{\Etp}{\Et[t+1]}
% \newcommand{\Errt}[1][t]{\mat{\bigtriangleup_{#1}}}
% \newcommand\cnM{\kappa}
% \newcommand{\cn}[1]{\kappa\left(#1\right)}
% \newcommand\X{\mat{X}}
% \newcommand\fstar{f_*}
% \newcommand\Xt[1][t]{\mat{X_{#1}}}
% \newcommand\ut[1][t]{{u_{#1}}}
% \newcommand\Xtinv{\inv{\Xt}}
% \newcommand\Xtp{\mat{X_{t+1}}}
% \newcommand\Xtpinv{\inv{\left(\mat{X_{t+1}}\right)}}
% \newcommand\U{\mat{U}}
% \newcommand\UTr{\trans{\mat{U}}}
% \newcommand{\Ut}[1][t]{\mat{U_{#1}}}
% \newcommand{\Utinv}{\inv{\Ut}}
% \newcommand{\UtTr}[1][t]{\trans{\mat{U_{#1}}}}
% \newcommand\Utp{\mat{U_{t+1}}}
% \newcommand\UtpTr{\trans{\mat{U}_{t+1}}}
% \newcommand\Utptild{\mat{\widetilde{U}_{t+1}}}
% \newcommand\Us{\mat{U^*}}
% \newcommand\UsTr{\trans{\mat{U^*}}}
% \newcommand{\Sigs}{\mat{\Sigma}}
% \newcommand{\Sigsmh}{\Sigs^{-1/2}}
% \newcommand{\eye}{\mat{I}}
% \newcommand{\twonormbound}{\left(4+\DPhi{\M}{\Xt[0]}\right)\twonorm{\M}}
% \newcommand{\lamj}{\lambda_j}

% \renewcommand\u{\vect{u}}
% \newcommand\uTr{\trans{\vect{u}}}
% \renewcommand\v{\vect{v}}
% \newcommand\vTr{\trans{\vect{v}}}
% \newcommand\w{\vect{w}}
% \newcommand\wTr{\trans{\vect{w}}}
% \newcommand\wperp{\vect{w}_{\perp}}
% \newcommand\wperpTr{\trans{\vect{w}_{\perp}}}
% \newcommand\wj{\vect{w_j}}
% \newcommand\vj{\vect{v_j}}
% \newcommand\wjTr{\trans{\vect{w_j}}}
% \newcommand\vjTr{\trans{\vect{v_j}}}

% \newcommand{\DPhi}[2]{\ensuremath{D_{\Phi}\left(#1,#2\right)}}
% \newcommand\matmult{{\omega}}

% \theoremstyle{theorem}

\usepackage{xcolor}




% \OneAndAHalfSpacedXI % current default line spacing
%%\OneAndAHalfSpacedXII
%%\DoubleSpacedXII
%%\DoubleSpacedXI

% If hyperref is used, dvi-to-ps driver of choice must be declared as
%   an additional option to the \documentclass. For example
%\documentclass[dvips,trsc]{informs3}      % if dvips is used
%\documentclass[dvipsone,trsc]{informs3}   % if dvipsone is used, etc.

% Private macros here (check that there is no clash with the style)

% Natbib setup for author-year style
% \usepackage{natbib}
%  \bibpunct[, ]{(}{)}{,}{a}{}{,}%
%  \def\bibfont{\small}%
%  \def\bibsep{\smallskipamount}%
%  \def\bibhang{24pt}%
%  \def\newblock{\ }%
%  \def\BIBand{and}%

%% Setup of theorem styles. Outcomment only one. 
%% Preferred default is the first option.
% \TheoremsNumberedThrough     % Preferred (Theorem 1, Lemma 1, Theorem 2)
%\TheoremsNumberedByChapter  % (Theorem 1.1, Lema 1.1, Theorem 1.2)

%% Setup of the equation numbering system. Outcomment only one.
%% Preferred default is the first option.
% \EquationsNumberedThrough    % Default: (1), (2), ...
%\EquationsNumberedBySection % (1.1), (1.2), ...

% In the reviewing and copyediting stage enter the manuscript number.
%\MANUSCRIPTNO{} % When the article is logged in and DOI assigned to it,
                 %   this manuscript number is no longer necessary

%%%%%%%%%%%%%%%%
\begin{document}
%%%%%%%%%%%%%%%%

% Outcomment only when entries are known. Otherwise leave as is and 
%   default values will be used.
%\setcounter{page}{1}
%\VOLUME{00}%
%\NO{0}%
%\MONTH{Xxxxx}% (month or a similar seasonal id)
%\YEAR{0000}% e.g., 2005
%\FIRSTPAGE{000}%
%\LASTPAGE{000}%
%\SHORTYEAR{00}% shortened year (two-digit)
%\ISSUE{0000} %
%\LONGFIRSTPAGE{0001} %
%\DOI{10.1287/xxxx.0000.0000}%

% Author's names for the running heads
% Sample depending on the number of authors;
% \RUNAUTHOR{Jones}
% \RUNAUTHOR{Jones and Wilson}
% \RUNAUTHOR{Jones, Miller, and Wilson}
% \RUNAUTHOR{Jones et al.} % for four or more authors
% Enter authors following the given pattern:
%\RUNAUTHOR{}

% Title or shortened title suitable for running heads. Sample:
% \RUNTITLE{Bundling Information Goods of Decreasing Value}
% Enter the (shortened) title:
%\RUNTITLE{}

% Full title. Sample:
% \TITLE{Bundling Information Goods of Decreasing Value}
% Enter the full title:
\title{Atomic Proximal Policy Optimization for Electric Robo-Taxi Dispatch and Charger Allocation}

% Block of authors and their affiliations starts here:
% NOTE: Authors with same affiliation, if the order of authors allows, 
%   should be entered in ONE field, separated by a comma. 
%   \EMAIL field can be repeated if more than one author
% \ARTICLEAUTHORS{%
% \AUTHOR{Zhanhao Zhang}
% \AFF{Operations Research and Information Engineering, Cornell University, \EMAIL{zz564@cornell.edu}}
% \AUTHOR{Ruifan Yang}
% \AFF{Operations Research and Information Engineering, Cornell University, \EMAIL{ry298@cornell.edu}}
% \AUTHOR{Manxi Wu}
% \AFF{Operations Research and Information Engineering, Cornell University, \EMAIL{manxiwu@cornell.edu}}
% Enter all authors
% } % end of the block

\author{%
    Jim Dai\\
    \footnotesize{Operations Research and Information Engineering, Cornell University, jd694@cornell.edu}\and
    Manxi Wu\\
    \footnotesize{Department of Civil and Environmental Engineering, University of California, Berkeley, manxiwu@berkeley.edu}\and
    Zhanhao Zhang\\
    \footnotesize{Operations Research and Information Engineering, Cornell University, zz564@cornell.edu}
}

\date{}
\maketitle

\begin{abstract}
    \begin{abstract}  
Test time scaling is currently one of the most active research areas that shows promise after training time scaling has reached its limits.
Deep-thinking (DT) models are a class of recurrent models that can perform easy-to-hard generalization by assigning more compute to harder test samples.
However, due to their inability to determine the complexity of a test sample, DT models have to use a large amount of computation for both easy and hard test samples.
Excessive test time computation is wasteful and can cause the ``overthinking'' problem where more test time computation leads to worse results.
In this paper, we introduce a test time training method for determining the optimal amount of computation needed for each sample during test time.
We also propose Conv-LiGRU, a novel recurrent architecture for efficient and robust visual reasoning. 
Extensive experiments demonstrate that Conv-LiGRU is more stable than DT, effectively mitigates the ``overthinking'' phenomenon, and achieves superior accuracy.
\end{abstract}  
\end{abstract}

% \KEYWORDS{High occupancy toll lane design, Non-atomic games, Equilibrium analysis with heterogeneous preferences}
% \HISTORY{}

%%%%%%%%%%%%%%%%%%%%%%%%%%%%%%%%%%%%%%%%%%%%%%%%%%%%%%%%%%%%%%%%%%%%%%

% Samples of sectioning (and labeling) in TRSC
% NOTE: (1) \section and \subsection do NOT end with a period
%       (2) \subsubsection and lower need end punctuation
%       (3) capitalization is as shown (title style).
%
%\section{Introduction.}\label{intro} %%1.
%\subsection{Duality and the Classical EOQ Problem.}\label{class-EOQ} %% 1.1.
%\subsection{Outline.}\label{outline1} %% 1.2.
%\subsubsection{Cyclic Schedules for the General Deterministic SMDP.}
%  \label{cyclic-schedules} %% 1.2.1
%\section{Problem Description.}\label{problemdescription} %% 2.

% Text of your paper here



\section{Introduction}
\section{Introduction}


\begin{figure}[t]
\centering
\includegraphics[width=0.6\columnwidth]{figures/evaluation_desiderata_V5.pdf}
\vspace{-0.5cm}
\caption{\systemName is a platform for conducting realistic evaluations of code LLMs, collecting human preferences of coding models with real users, real tasks, and in realistic environments, aimed at addressing the limitations of existing evaluations.
}
\label{fig:motivation}
\end{figure}

\begin{figure*}[t]
\centering
\includegraphics[width=\textwidth]{figures/system_design_v2.png}
\caption{We introduce \systemName, a VSCode extension to collect human preferences of code directly in a developer's IDE. \systemName enables developers to use code completions from various models. The system comprises a) the interface in the user's IDE which presents paired completions to users (left), b) a sampling strategy that picks model pairs to reduce latency (right, top), and c) a prompting scheme that allows diverse LLMs to perform code completions with high fidelity.
Users can select between the top completion (green box) using \texttt{tab} or the bottom completion (blue box) using \texttt{shift+tab}.}
\label{fig:overview}
\end{figure*}

As model capabilities improve, large language models (LLMs) are increasingly integrated into user environments and workflows.
For example, software developers code with AI in integrated developer environments (IDEs)~\citep{peng2023impact}, doctors rely on notes generated through ambient listening~\citep{oberst2024science}, and lawyers consider case evidence identified by electronic discovery systems~\citep{yang2024beyond}.
Increasing deployment of models in productivity tools demands evaluation that more closely reflects real-world circumstances~\citep{hutchinson2022evaluation, saxon2024benchmarks, kapoor2024ai}.
While newer benchmarks and live platforms incorporate human feedback to capture real-world usage, they almost exclusively focus on evaluating LLMs in chat conversations~\citep{zheng2023judging,dubois2023alpacafarm,chiang2024chatbot, kirk2024the}.
Model evaluation must move beyond chat-based interactions and into specialized user environments.



 

In this work, we focus on evaluating LLM-based coding assistants. 
Despite the popularity of these tools---millions of developers use Github Copilot~\citep{Copilot}---existing
evaluations of the coding capabilities of new models exhibit multiple limitations (Figure~\ref{fig:motivation}, bottom).
Traditional ML benchmarks evaluate LLM capabilities by measuring how well a model can complete static, interview-style coding tasks~\citep{chen2021evaluating,austin2021program,jain2024livecodebench, white2024livebench} and lack \emph{real users}. 
User studies recruit real users to evaluate the effectiveness of LLMs as coding assistants, but are often limited to simple programming tasks as opposed to \emph{real tasks}~\citep{vaithilingam2022expectation,ross2023programmer, mozannar2024realhumaneval}.
Recent efforts to collect human feedback such as Chatbot Arena~\citep{chiang2024chatbot} are still removed from a \emph{realistic environment}, resulting in users and data that deviate from typical software development processes.
We introduce \systemName to address these limitations (Figure~\ref{fig:motivation}, top), and we describe our three main contributions below.


\textbf{We deploy \systemName in-the-wild to collect human preferences on code.} 
\systemName is a Visual Studio Code extension, collecting preferences directly in a developer's IDE within their actual workflow (Figure~\ref{fig:overview}).
\systemName provides developers with code completions, akin to the type of support provided by Github Copilot~\citep{Copilot}. 
Over the past 3 months, \systemName has served over~\completions suggestions from 10 state-of-the-art LLMs, 
gathering \sampleCount~votes from \userCount~users.
To collect user preferences,
\systemName presents a novel interface that shows users paired code completions from two different LLMs, which are determined based on a sampling strategy that aims to 
mitigate latency while preserving coverage across model comparisons.
Additionally, we devise a prompting scheme that allows a diverse set of models to perform code completions with high fidelity.
See Section~\ref{sec:system} and Section~\ref{sec:deployment} for details about system design and deployment respectively.



\textbf{We construct a leaderboard of user preferences and find notable differences from existing static benchmarks and human preference leaderboards.}
In general, we observe that smaller models seem to overperform in static benchmarks compared to our leaderboard, while performance among larger models is mixed (Section~\ref{sec:leaderboard_calculation}).
We attribute these differences to the fact that \systemName is exposed to users and tasks that differ drastically from code evaluations in the past. 
Our data spans 103 programming languages and 24 natural languages as well as a variety of real-world applications and code structures, while static benchmarks tend to focus on a specific programming and natural language and task (e.g. coding competition problems).
Additionally, while all of \systemName interactions contain code contexts and the majority involve infilling tasks, a much smaller fraction of Chatbot Arena's coding tasks contain code context, with infilling tasks appearing even more rarely. 
We analyze our data in depth in Section~\ref{subsec:comparison}.



\textbf{We derive new insights into user preferences of code by analyzing \systemName's diverse and distinct data distribution.}
We compare user preferences across different stratifications of input data (e.g., common versus rare languages) and observe which affect observed preferences most (Section~\ref{sec:analysis}).
For example, while user preferences stay relatively consistent across various programming languages, they differ drastically between different task categories (e.g. frontend/backend versus algorithm design).
We also observe variations in user preference due to different features related to code structure 
(e.g., context length and completion patterns).
We open-source \systemName and release a curated subset of code contexts.
Altogether, our results highlight the necessity of model evaluation in realistic and domain-specific settings.






\section{Model} \label{sec:model}
\newcommand{\tabincell}[2]{\begin{tabular}{@{}#1@{}}#2\end{tabular}}
\newcommand{\rowstyle}[1]{\gdef\currentrowstyle{#1}%
	#1\ignorespaces
}

\newcommand{\className}[1]{\textbf{\sf #1}}
\newcommand{\functionName}[1]{\textbf{\sf #1}}
\newcommand{\objectName}[1]{\textbf{\sf #1}}
\newcommand{\annotation}[1]{\textbf{#1}}
\newcommand{\todo}[1]{\textcolor{blue}{\textbf{[[TODO: #1]]}}}
\newcommand{\change}[1]{\textcolor{blue}{#1}}
\newcommand{\fetch}[1]{\textbf{\em #1}}
\newcommand{\phead}[1]{\vspace{1mm} \noindent {\bf #1}}
\newcommand{\wei}[1]{\textcolor{blue}{{\it [Wei says: #1]}}}
\newcommand{\peter}[1]{\textcolor{red}{{\it [Peter says: #1]}}}
\newcommand{\zhenhao}[1]{\textcolor{dkblue}{{\it [Zhenhao says: #1]}}}
\newcommand{\feng}[1]{\textcolor{magenta}{{\it [Feng says: #1]}}}
\newcommand{\jinqiu}[1]{\textcolor{red}{{\it [Jinqiu says: #1]}}}
\newcommand{\shouvick}[1]{\textcolor{violet(ryb)}{{\it [Shouvick says: #1]}}}
\newcommand{\pattern}[1]{\emph{#1}}
%\newcommand{\tool}{{{DectGUILag}}\xspace}
\newcommand{\tool}{{{GUIWatcher}}\xspace}


\newcommand{\guo}[1]{\textcolor{yellow}{{\it [Linqiang says: #1]}}}

\newcommand{\rqbox}[1]{\begin{tcolorbox}[left=4pt,right=4pt,top=4pt,bottom=4pt,colback=gray!5,colframe=gray!40!black,before skip=2pt,after skip=2pt]#1\end{tcolorbox}}


\section{Fleet Control Policy Reduction With Atomic Actions} \label{sec:atomic-action}
One challenge of computing the optimal control policy lies in the size of the action space $\vert\mathcal{A}\vert$, which grows exponentially with the number of vehicles $N$ and vehicle statuses $|\Cartype|$. As a result, the dimension of policy $\pi$ also grows exponentially with $N$ and $|\Cartype|$. The focus of this section is to address this challenge by introducing a policy reduction scheme, which decomposes the dispatching of a fleet to sequential assignment of tasks to individual vehicles, where the task for each individual vehicle is referred as an ``atomic action". We use the name ``atomic action policy" because each atomic action only changes the status of a single vehicle. In particular, for any vehicle of a status $\cartype{} \in \Cartype$, an atomic action can be any one of the followings: 
\begin{itemize}
    \item[-] $\afulfillred{\triptype{}}$ represents fulfilling a trip of status $\triptype{} \in \Triptype$.
    \item[-] $\areroutered{\destination}$ represents repositioning to destination $\destination \in \Region$.
    \item[-] $\achargered{\type}$ represents charging with rate $\type \in \Type$ at its current region.
    \item[-] $\apassred$ represents idling or continuing with its previously assigned actions. 
\end{itemize}
We use $\hat{\mathcal{A}}$ to denote the atomic action space that includes all of the above atomic actions, i.e. $\hat{a} \in \hat{\mathcal{A}} := \left\{\left(\afulfillred{\triptype{}}\right)_{\triptype{} \in \Triptype}, \left(\areroutered{\destination}\right)_{\destination \in \Region}, \left(\achargered{\type}\right)_{\type \in \Type}, \apassred \right\}$. The atomic action significantly reduces the dimension of the action function since $\hat{\mathcal{A}}$ does not scale with the fleet size or the number of vehicle statuses. 

We now present the procedure of atomic action assignment. In each decision epoch $(t, d)$, vehicles are arbitrarily indexed from $1$ to $\Size$, and are sequentially selected. For a selected vehicle $n$, the atomic policy $\hat{\pi}: \mathcal{S} \times \Cartype \rightarrow \Delta(\hat{\mathcal{A}})$ maps from the tuple of system state $s^{t,d}_n$ before $n$-th assignment and the selected vehicle's status $\cartype{n}$ to a distribution of atomic actions. The system state $s^{t,d}_n$ transitions after every single vehicle assignment with $s_1^{t,d}=s^{t,d}$, and $s_{\Size}^{t,d}$ transitions to $s^{t+1,d}$ after assigning the last vehicle and trip arrival at time $t+1$ is realized.

% To determine the sequence of vehicles receiving atomic action assignments, we rank order all vehicle status based on some pre-specified sequence. We use a counter $m := (m_{\cartype{}})_{\cartype{} \in \Cartype}$ to keep track of the number of vehicles $m_{\cartype{}}$ of each status $\cartype{} \in \Cartype$ that have not yet been assigned in the current decision epoch. We define the augmented state space $\hat{\mathcal{S}}$, where each augmented state $\hat{s} := (s, m) \in \hat{\mathcal{S}}$ stores the information of the current system state $s$ and the counter $m$ that reflects the assignment status of the vehicles. We denote the atomic action policy $\hat{\pi} : \hat{\mathcal{S}} \rightarrow \Delta(\hat{\mathcal{A}})$. Here, we note that the atomic policy $\hat{\pi}$ depends on the time of the day as it is recorded in the system state $s$.

% At the beginning of the decision epoch $(t, d)$, we denote the system state before taking any assignments as $s^{t,d}_1$. We construct the current counter $m^{t,d}_1$ by setting $(m^{t,d}_1)_{\cartype{}} = (s^{t,d}_1)_{\cartype{}}$ for all $\cartype{} \in \Cartype$ to reflect the fact that none of the vehicles have been assigned yet. Then, we augment the current system state $s^{t,d}_1$ with the counter and produce an augmented state $\hat{s}^{t,d}_1 := (s^{t, d}_1, m^{t,d}_1)$. Then, we use the augmented state $\hat{s}^{t,d}_1$ to query the atomic action policy $\hat{\pi}$, which generates an atomic action $\hat{a}^{t, d}_1 \in \hat{\mathcal{A}}$. The atomic action policy assigns $\hat{a}^{t,d}_1$ to a vehicle of status $\cartype{}$ such that $\cartype{}$ is the first vehicle status in the sequence with $m_{\cartype{}} > 0$. Here, the feasibility of an atomic action follows from the model setting in Sec. \ref{sec:model}. For example, a trip of status $\triptype{}$ can only be assigned to a vehicle of status $\cartype{}$ that is associated with high enough battery level and within $\Lp$ of duration of complete from the origin region of the trip. Then, we update the system state from $s^{t,d}_1$ to $s^{t,d}_2$ to reflect the change of status of a vehicle of status $\cartype{}$. Additionally, we subtract one from the entry $m_{\cartype{}}$ to reflect that we have assigned one vehicle of status $\cartype{}$ and we obtain a new counter $m^{t,d}_2$. We denote the procedure of assigning an atomic action to a single vehicle as an ``atomic step". We construct the new augmented state $\hat{s}^{t,d}_2 := (s^{t,d}_2, m^{t,d}_2)$ and we repeat the same procedure. We repeat the process until all entries of $m$ become $0$. Since we have $\Size$ vehicles in the system, there are exactly $\Size$ atomic steps in each decision epoch. After all vehicles have been assigned, the system transitions to the next decision epoch $(t+1,d)$ with new trip arrivals realized. We note that the atomic action policy reduces the action space $\mathcal{A}$, which scales combinatorially with the fleet size and number of vehicle status, to the atomic action space $\hat{\mathcal{A}}$, which is a constant.

The total reward for each decision epoch $(t, d)$ is the sum of all rewards generated from each atomic action assignment in $(t, d)$, where the reward generated by the atomic action $\hat{a}^{t,d} \in \hat{\mathcal{A}}$ is given by
\begin{align*}
    r^t(\hat{a}^{t,d}) =& \sum_{\triptype{} \in \Triptype} r^t_{f, \triptype{}}\mathds{1}\left\{\hat{a}^{t,d} = \afulfillred{\triptype{}}\right\}\\
    +& \sum_{(\origin,\destination) \in \Region \times \Region} r^t_{e, \origin\destination}\mathds{1}\left\{\hat{a}^{t,d} = \areroutered{\destination}\right\}\\ 
    +& \sum_{\delta \in \Delta} r^t_{q, \delta}\mathds{1}\left\{\hat{a}^{t,d} = \achargered{\delta}\right\}.
\end{align*}
The long-run average reward given the atomic action policy $\hat{\pi}$ and the initial state $s \in \mathcal{S}$ is as follows: 
\begin{align*}
    &R(\hat{\pi} \vert s) \\
    =& \lim_{\totaldays \rightarrow \infty} \frac{1}{\totaldays} \mathbb{E}_{\hat{\pi}}\left[ \sum_{\day = 1}^{\totaldays}\sum_{t=1}^T \sum_{n=1}^{N} r^{t}(\hat{a}^{\time, \day}_{n}) \Bigg\vert s \right],\quad \forall s \in \mathcal{S},
\end{align*}
where $\hat{a}_n^{t, d}$ is the atomic actions generated by the atomic action policy in the $n$-th atomic step in decision epoch $(t, d)$. Given any initial state $s \in \mathcal{S}$, our goal is to find the optimal atomic action policy such that $\hat{\pi}^{*} = \argmax_{\hat{\pi}} R(\hat{\pi} \vert s)$. 

Our atomic action policy can be viewed as a reduction of the original fleet dispatching policy in that any realized sequence of atomic actions corresponds to a feasible fleet dispatching action with the same reward of the decision epoch. This reduction makes the training of atomic action policy scalable because the output dimension of atomic action policy $\hat{\pi}^t$ equals to $\vert \hat{\mathcal{A}}\vert$, which is a constant regardless of the fleet size. 

% \subsection{Reduction of Atomic Action Space} \label{subsec:reduced-atomic-policy}
% Even with reduction using atomic actions, the atomic action space $\hat{\mathcal{A}}$ is still large due to the size of vehicle status $\Cartype$. To further reduce the policy dimension, we propose the {\em reduced atomic policy} $\tilde{\pi}: \mathcal{S} \times \Cartype \rightarrow \Delta(\tilde{\mathcal{A}})$ that takes a state $s$ and a vehicle status $\cartype{}$ as inputs and outputs a {\em reduced atomic action} $\tilde{a} \in \tilde{\mathcal{A}} := \left\{\left(\afulfillred{\triptype{}}\right)_{\triptype{} \in \Triptype}, \left(\areroutered{\destination}\right)_{\destination \in \Region}, \left(\achargered{\type}\right)_{\type \in \Type}, \apass \right\}$, where $\afulfillred{\triptype{}}$ is the reduced atomic action to fulfill a trip of type $\triptype{}$, $\areroutered{\destination}$ is the reduced atomic action to reroute to region $\destination$, and $\achargered{\type}$ is the reduced atomic action to charge with rate $\type$.  

% In particular, we rank order all vehicle status based on some pre-specified sequence. At the beginning of each time step $(\time, \day)$, let $s_1^{t}$ be the state before assigning any reduced atomic actions. We record the number of vehicles $s_{1, \cartype{}}^{t}$ of each type $\cartype{} \in \Cartype$. In this time step, we will query each $\cartype{} \in \Cartype$ in sequence according to its index. We obtain a reduced atomic action $\tilde{a} \in \tilde{\mathcal{A}}$ from $\tilde{\pi}$ with the current state and the vehicle status $\cartype{}$ as inputs. We assign a vehicle of type $\cartype{}$ the reduced atomic action $\tilde{a}$ and transition to the next state that reflects the update of this vehicle. In time step $t$, we query each vehicle status $\cartype{}$ for $s_{1, \cartype{}}^{t}$ times in total. If $s_{1, \cartype{}}^{t} = 0$ for some $\cartype{}$, then we skip this vehicle status and proceed to the next one. 

% The reduced atomic policy indicates that given state $s$ and vehicle status $\cartype{}$, any atomic action $\hat{a} \in \hat{\mathcal{A}}$ that assigns a reduced atomic action $\tilde{a} \in \tilde{\mathcal{A}}$ to the vehicle status $\cartype{}$ is equivalent to the pair $(\tilde{a}, \cartype{})$. In particular, for each pair of reduced atomic action $\tilde{a} \in \tilde{\mathcal{A}}$ and a vehicle status $\cartype{}$, we say that the atomic action $\hat{a} \in \hat{\mathcal{A}}$ is {\em equivalent} to $(\tilde{a}, \cartype{})$ if $\tilde{a} = \afulfillred{\triptype{}}$ (resp. $\areroutered{\destination}$, $\achargered{\type}$, $\apass$) and $\hat{a} = \afulfill{\cartype{}, \triptype{}}$ for some $\triptype{} \in \Triptype$ (resp. $\areroute{\cartype{}, \destination}$ for some $\destination \in \Region$, $\acharge{\cartype{}, \type}$ for some $\type \in \Type$, $\apass$). We remark that for any non-passing atomic action $\hat{a} \neq \apass \in \hat{\mathcal{A}}$, there is a unique pair of reduced atomic action $\tilde{a} \in \tilde{\mathcal{A}}$ and a vehicle status $\cartype{} \in \Cartype$ that is equivalent to $\hat{a}$.

% We note that the reduced atomic policy drops the vehicle status from the action space and adds that into the state space. The reduction drops the dimension of the action space by the cardinality of $\Cartype$ from $\vert \hat{\mathcal{A}} \vert$ to $\vert \tilde{\mathcal{A}} \vert$, with the cost of increasing the input dimension by $3$ for storing the information of a single vehicle status. In Sec. \ref{sec:deep-rl}, we will demonstrate that the increase in the input dimension is still computationally tractable by using function approximation. 

% Analogous to the atomic actions, we define the rewards for every pair of reduced atomic action $\tilde{a}^{t} \in \tilde{\mathcal{A}}$ and vehicle status $\cartype{}$ of each time step $t$ as $r^{t}(\tilde{a}^{t}, \cartype{}) = r^{t}(\hat{a}^{t})$, where $\hat{a}^{t} \in \hat{\mathcal{A}}$ is the atomic action that is equivalent to $(\tilde{a}^{t}, \cartype{})$. We can then write the long-run average reward given the reduced atomic action policy $\tilde{\pi}$ and the initial state $s \in \mathcal{S}$ as follows: 
% \begin{align*}
%     R(\tilde{\pi} \vert s) = \lim_{D \to \infty} \frac{1}{D} \mathbb{E}_{\tilde{\pi}}\left[\sum_{d=1}^D \sum_{t=1}^T \sum_{\vehnum=1}^N r^{\time}(\tilde{a}^{\time, \day}_{\vehnum}, \cartype{\vehnum}) \Bigg\lvert s \right],\quad \forall s \in \mathcal{S},
% \end{align*}
% where $\tilde{a}^{\time, \day}_n$ and $\cartype{n}$ are the reduced atomic action and vehicle status associated with the vehicle at the $n$-th atomic step of time step $(\time, \day)$.

% Let $\tilde{R}^*(s)$ be the maximum long-run average reward achievable by reduced atomic policies given the initial state $s \in \mathcal{S}$. Proposition \ref{proposition:reduced-atomic-optimal} demonstrates that this the reduced atomic policy is without loss of optimality.

% \begin{proposition} \label{proposition:reduced-atomic-optimal}
%     $\tilde{R}^*(s) = R^*(s),~\forall s \in \mathcal{S}$.
% \end{proposition}
% \noindent\begin{proof}{Proof of Proposition \ref{proposition:reduced-atomic-optimal}}
%     Recall from the proof of Theorem \ref{thm:atomic-optimal}, we can find an optimal atomic policy $\hat{\pi}^*$ that is a solution of \eqref{eq:orig-atomic-policy-construct-pi} and \eqref{eq:orig-atomic-policy-construct-h-body}. We note that such $\hat{\pi}^*$ may not be unique since the argmax operator in \eqref{eq:orig-atomic-policy-construct-pi} may not return a unique atomic action. We further consider a specific $\hat{\pi}^*$ that is a solution of \eqref{eq:orig-atomic-policy-construct-pi} and \eqref{eq:orig-atomic-policy-construct-h-body}, and chooses a tie breaking rule in \eqref{eq:orig-atomic-policy-construct-pi} that selects the optimal non-passing atomic action that is the one associated with a vehicle status with the smallest index.
    
%     We construct a reduced atomic policy $\tilde{\pi}^*: \mathcal{S} \times \Cartype \rightarrow \Delta(\tilde{A})$ from the atomic policy $\hat{\pi}^*$ as follows: For any pair of state $s \in \mathcal{S}$ and vehicle status $\cartype{} \in \Cartype$, we assign $\tilde{\pi}^*(\tilde{a} \vert s, \cartype{}) := \hat{\pi}^*(\hat{a} \vert s)$ for all non-passing reduced atomic actions $\tilde{a} \neq \apass \in \tilde{\mathcal{A}}$, where $\hat{a} \in \hat{\mathcal{A}}$ is the atomic action equivalent to $(\tilde{a}, \cartype{})$. We assign $\tilde{\pi}^*(\apass \vert s, \cartype{})$ with the remaining probability.
    
%     % and reduced atomic action $\tilde{a} \in \tilde{\mathcal{A}}$, we find the atomic action $\hat{a} \in \hat{\mathcal{A}}$ that is equivalent to $(\tilde{a}, \cartype{})$, then we set
%     % \begin{equation} \label{eq:pihat-2-pitilde}
%     %     \tilde{\pi}^*(\tilde{a} \vert s, \cartype{}) := \begin{cases}
%     %         \hat{\pi}^*(\hat{a} \vert s), & \text{if } \tilde{a} \neq \apass,\\
%     %         1 - \sum_{\tilde{a} \neq \apass} \tilde{\pi}^*(\tilde{a} \vert s, \cartype{}), & \text{if } \tilde{a} = \apass.\\
%     %     \end{cases}
%     % \end{equation}
%     Since $\hat{\pi}^*$ is deterministic, $\tilde{\pi}^*$ is also a deterministic policy. In each time step $t$, let the state before assigning any atomic actions be $s_1^{t}$. We query the atomic policy $\hat{\pi}^*$ with $s_1^{t}$ and obtain an atomic action $\hat{a}_1^{t} \in \hat{\mathcal{A}}$. First, we consider the case where $\hat{a}_1^{t} \neq \apass$. Then, $\hat{\pi}^*$ assigns $\tilde{a}^{t}_1$ to a vehicle of type $\cartype{} \in \Cartype$, where $(\tilde{a}^{t}_1, \cartype{})$ is equivalent to $\hat{a}_1^{t}$, transitions to $s_2^{t}$, and obtains the reward $r^t(\hat{a}_1^{t})$. In the meantime, the reduced atomic policy $\tilde{\pi}^*$ finds the vehicle status with the smallest index that has non-zero number of vehicles, which we labeled as $\cartype{1}$. If $\cartype{1} = \cartype{}$, then $\tilde{\pi}^*$ returns $\tilde{a}^{t}_1$, assigns it to a vehicle of type $\cartype{}$, transitions to $s_2^{t}$, and obtains the reward $r^t(\tilde{a}_1^{t}, \cartype{1}) = r^t(\hat{a}_1^{t})$. Otherwise, due to the tie breaking rule of $\hat{\pi}^*$ in terms of vehicle status, $\tilde{\pi}^*$ will return $\apass$ for all vehicle status whose indices are smaller than $\cartype{}$, and it will keep traversing the list of vehicle status until it hits $\cartype{}$. Repeating the same argument until the atomic action generated by $\hat{\pi}^*$ is $\apass$, we obtain that the sequence of non-passing atomic actions returned by $\hat{\pi}^*$ are equivalent to the sequence of non-passing reduced atomic actions returned by $\tilde{\pi}^*$, given the sequence of vehicle status $\tilde{\pi}^*$ traverses. Therefore, we can obtain the same sequence of state transitions and induced atomic rewards by $\hat{\pi}^*$ and $\tilde{\pi}^*$ in time step $t$. We then can obtain $R(\tilde{\pi}^* \vert s) = R(\hat{\pi}^* \vert s) = R^*(s),~ \forall s \in \mathcal{S}$. Hence, $\tilde{R}^*(s) \geq R^*(s),~\forall s \in \mathcal{S}$.
    
%     % In the meantime, the reduced atomic policy $\tilde{\pi}^*$ finds the first vehicle status in the sequence with non-zero number of vehicles, and we label it as $\cartype{1}$. If $\cartype{1} = \cartype{}$ is the vehicle status associated with the atomic action $\hat{a}_{1}^{t}$, then by \eqref{eq:pihat-2-pitilde}, $\tilde{\pi}^*$ will return a reduced atomic action $\tilde{a}_1 \neq \apass \in \tilde{\mathcal{A}}$ such that $(\tilde{a}_1^{t}, \cartype{1})$ is equivalent to $\hat{a}_1^{t}$. Then, $\tilde{\pi}^*$ assigns the atomic action $\tilde{a}_1^{t}$ a vehicle of type $\cartype{1}$, transitions to $s_2^{t}$, obtains the reward $r^t(\tilde{a}_1^{t}, \cartype{1}) = r^t(\hat{a}_1^{t})$, and subtract one from the counter for vehicle status $\cartype{1}$. On the other hand, if $\cartype{1} \neq \cartype{}$, then by \eqref{eq:pihat-2-pitilde}, $\tilde{\pi}^*$ returns $\apass$ with probability one. Then, it assigns a vehicle of type $\cartype{1}$ with the passing action, remains at state $s_1^{t}$, receives $0$ reward, and subtract one from the counter for vehicle status $\cartype{1}$. $\tilde{\pi}^*$ then finds the first vehicle status $\cartype{2}$ with non-zero vehicles in the new counter and query with inputs $s_1^{t}$ and $\cartype{2}$. It repeats the process until it finds the vehicle status $\cartype{}$ that associated with $\hat{a}_1^{t}$. Due to the tie-breaking rule of $\hat{\pi}^*$, if $\cartype{1}$ precedes $\cartype{}$ in the ordered sequence, then $\hat{\pi}^*$ will not assign non-passing actions to any vehicles of type $\cartype{1}$ in the current time step $t$. When $\hat{a}^{t}_1 = \apass$, both $\hat{\pi}^*$ and $\tilde{\pi}^*$ assign passing to all vehicles and transition to the next time step.
    
%     % Repeating the same argument to the remaining atomic steps in the time step $t$, we obtain that the sequence of non-passing atomic actions returned by $\hat{\pi}^*$ are equivalent to the sequence of non-passing reduced atomic actions returned by $\tilde{\pi}^*$, given the sequence of vehicle status $\tilde{\pi}^*$ traverses. Therefore, we can obtain the same sequence of state transitions and induced atomic rewards by $\hat{\pi}^*$ and $\tilde{\pi}^*$ in time step $t$. We then can obtain $R(\tilde{\pi}^* \vert s) = R(\hat{\pi}^* \vert s) = R^*(s),~ \forall s \in \mathcal{S}$. Hence, $\tilde{R}^*(s) \geq R^*(s),~\forall s \in \mathcal{S}$.

%     The proof of the converse is trivial. Consider any reduced atomic policy $\tilde{\pi}$, the sequential assignment of reduced atomic actions to vehicles in each time step has a feasible action that is equivalent. Therefore, we can construct an original policy $\pi$ that is equivalent to $\tilde{\pi}$. We then can obtain $R(\pi \vert s) = R(\tilde{\pi} \vert s)$, $\forall s \in \mathcal{S}$. Hence, $R^*(s) \geq \tilde{R}^*(s),~ \forall s \in \mathcal{S}$.
    
%     % For any time step $t$, let the state before assigning any atomic actions be $s^{t}$, we can identify the list of all vehicle status with non-zero number of vehicles and we denote them as $\cartype{1}, \dots, \cartype{\Size}$. We roll out the policy $\tilde{\pi}$ for $\Size$ steps and obtain a sequence of reduced atomic actions $\tilde{a}^{t}_1, \dots, \tilde{a}^{t}_{\Size}$. For any sequence of reduced atomic action $\tilde{a}^{t}_1, \dots, \tilde{a}^{t}_{\Size}$ rolled out from $\tilde{\pi}$, we can find a feasible action $a^{t}$ that is induced by sequentially assigning $\tilde{a}^{t}_1, \dots, \tilde{a}^{t}_{\Size}$ to vehicles of types $\cartype{1}, \dots, \cartype{\Size}$. Setting $\pi(a^{t} \vert s^{t})$ with the probability of rolling out the sequence $\tilde{a}^{t}_1, \dots, \tilde{a}^{t}_{\Size}$ by $\tilde{\pi}$ at $s^{t}$, we can obtain $R(\pi \vert s) = R(\tilde{\pi} \vert s)$, $\forall s \in \mathcal{S}$. Hence, $R^*(s) \geq \tilde{R}^*(s),~ \forall s \in \mathcal{S}$.
    
%     % To prove the converse. Consider any policy using concise representation of atomic actions $\tilde{\pi}$. For any time step $t$, given an initial state $s^{t}$, we can identify the list of all vehicle status with non-zero number of vehicles and we denote them as $\cartype{1}, \dots, \cartype{\Size}$. Let $P^{\tilde{\pi}}(\tilde{a}^{t}_1, \dots, \tilde{a}^{t}_{\Size} \vert s^{t})$ denote the probability that the sequence of induced concise atomic actions $(\tilde{a}^{t}_1, \dots, \tilde{a}^{t}_{\Size})$ given the initial state $s^{t}$. I.e. $P^{\tilde{\pi}}(\tilde{a}^{t}_1, \dots, \tilde{a}^{t}_{\Size} \vert s^{t}) = \tilde{\pi}(\tilde{a}^{t}_1 \vert s^{t}_1, \cartype{1}) \cdot \dots \cdot \tilde{\pi}(\tilde{a}^{t}_{\Size} \vert s^{t}_{\Size}, \cartype{\Size})$, where $s^{t}_1, \dots, s^{t}_{\Size}$ is the sequence of states induced by assigning atomic actions to vehicles of types $\cartype{1}, \dots, \cartype{\Size}$ sequentially. We construct the policy $\pi(a^{t} \vert s^{t}) := P^{\tilde{\pi}}(\tilde{a}^{t}_1, \dots, \tilde{a}^{t}_{\Size} \vert s^{t})$ for all $a^{t}$, where $(\tilde{a}^{t}_1, \dots, \tilde{a}^{t}_{\Size})$ is the sequence of concisely represented atomic actions that induces $a^{t}$ by applying them sequentially to vehicles of types $\cartype{1}, \dots, \cartype{\Size}$. It is then obvious that $\sum_{a} r^t(a) \pi(a \vert s) = \sum_{\tilde{a}_1, \dots, \tilde{a}_{\Size}} P^{\tilde{\pi}}(\tilde{a}_1, \dots, \tilde{a}_{\Size} \vert s) \sum_{n = 1}^{\Size} r^t(\tilde{a}_n, \cartype{n})$ and $\sum_{s', a} P(s' \vert s, a) \pi(a \vert s) = \sum_{s', \tilde{a}_1, \dots, \tilde{a}_{\Size}} P(s' \vert s, \tilde{a}_1, \dots, \tilde{a}_{\Size}) P^{\tilde{\pi}}(\tilde{a}_1, \dots, \tilde{a}_{\Size} \vert s)$. Construct $g$ and $h$ w.r.t $\pi$ according to proposition \ref{proposition:fixed-policy-gh}. Then, by proposition \ref{proposition:joint-atomic-reduced-policy-equal-reward}, we can obtain $R(\tilde{\pi} \vert s) = R(\pi \vert s),~ \forall s \in \mathcal{S}$. Hence, $R^*(s) \geq \tilde{R}^*(s),~ \forall s \in \mathcal{S}$.

%     Therefore, we can conclude that $\tilde{R}^*(s) = R^*(s),~\forall s \in \mathcal{S}$.
%     \hfill $\square$
% \end{proof}

\section{Deep Reinforcement Learning With Aggregated States} \label{sec:deep-rl}
The adoption of atomic actions has significantly reduced the action dimension. However, the implementation of the MDP is still challenging due to the large state space, which scales significantly with the fleet size, number of regions, and battery discretization. In this section, we provide an efficient algorithm to train the fleet dispatching policy by incorporating our atomic action decomposition into PPO \citep{schulman2017proximal}. To tackle with the large state size, we use neural networks to approximate both the value function and the policy function, to be specified later. We also further reduce the state size in terms of battery discretization and the number of regions by using the following state reduction scheme:

\paragraph{Battery Level Clustering.} We map the state representation of all vehicle statuses into vehicle statuses with aggregated battery levels. We cluster the battery levels into 3 intervals, each of which denotes low battery level $\battery_L$, medium battery level $\battery_M$, and high battery level $\battery_H$, respectively. The cutoff points can be set based on charging rates and criticality of battery levels. It is also possible to cluster the battery levels differently. If computing resources allow, we can cluster the battery levels with finer granularity, e.g. into 10 levels instead of 3. 

\paragraph{Trip Order Status Clustering.} In the state reduction scheme, trip orders are aggregated by recording only the number of requests originating from or arriving at each region, instead of tracking the number of trip requests for each origin-destination pair.
% The current state representation records all trip information, i.e. the number of trip requests between each o-d pair. In the state reduction scheme, we record only the number of trip requests originating from and arriving at each region, and the aggregated trip order state is defined as 
% \begin{enumerate}
%     \item[-] $\state{\text{origin}}{\time} := \left(\state{\text{origin}}{\time}(\region, \tripactivetime) \Biggm\lvert \region \in \Region, \xi \in [\Lc] \right)$, where $\state{\text{origin}}{\time}(\region, \tripactivetime) = \sum_{\destination \in \Region} \state{(\region, \destination, \tripactivetime)}{\time}$ is the total number of trip orders originating from region $\region$ with trip active time $\tripactivetime$.
%     \item[-] $\state{\text{dest}}{\time} := \left(\state{\text{dest}}{\time}(\region, \tripactivetime) \Biggm\lvert \region \in \Region, \xi \in [\Lc] \right)$, where $\state{\text{dest}}{\time}(\region, \tripactivetime) = \sum_{\origin \in \Region} \state{(\origin, \region, \tripactivetime)}{\time}$ is the total number of trip orders whose destinations are in region $\region$ with trip active time $\tripactivetime$.
% \end{enumerate}
The clustering of trip order statuses reduces the state dimension from $O(\vert \Region \vert^2)$ to $O(\vert \Region \vert)$. While it loses some information about the trip distribution, in numerical experiments, we demonstrate that the vehicle dispatching policy trained using our state reduction scheme still achieves a very strong performance (see Section \ref{sec:numerical}). 

\begin{figure}
    \centering
    \includegraphics[width=0.8\linewidth]{plots/ppo.png}
    \caption{Atomic-PPO Training Pipeline}
    \label{fig:ppo}
\end{figure}

We denote the state space after the reduction on the original state as $\bar{\mathcal{S}}$, which we refer to as the ``reduced state space". 
%We want to find a randomized atomic control policy $\hat{\pi}: \bar{\mathcal{S}} \rightarrow \Delta(\hat{\mathcal{A}})$ on the reduced state space $\bar{\mathcal{S}}$ that is dependent on the time of day but homogeneous across days. For each training episode, we truncate the horizon to be a finite number of single days when the policy reached stationarity. We use the roll-outs of single days after the policy has reached stationarity to estimate the long run average daily revenue of the policy. The Atomic-PPO is facilitated by two coupled chains. The full chain on the state space $\mathcal{S}$ is adopted to facilitate the transition of the MDP, while the reduced chain on the reduced state space $\bar{\mathcal{S}}$ is employed to find the fleet control policy.
We use neural networks $\hat{\pi}_{\theta}: \bar{\mathcal{S}} \rightarrow \Delta(\hat{\mathcal{A}})$ on the reduced state space to approximate the atomic action policy function, where $\theta$ is the parameter vector for the atomic policy network. 
\begin{algorithm}[h]
    \SetAlgoLined
    \caption{The Atomic-PPO Algorithm} \label{algo:ppo}
    \KwInputs{Number of policy iterations $M$, number of trajectories per policy iteration $K$, number of days per trajectory $D$, initial policy network $\hat{\pi}_{\theta_0}$}
    \For{policy iteration $m = 1, \dots, M$}{
        Run policy $\hat{\pi}_{\theta_{m - 1}}$ for $\totaldays$ days of $\Horizon$ time steps for $K$ trajectories of Monte-Carlo simulations and collect dataset \eqref{eq:ppo-data}.\\
        Construct empirical estimates of long-run average daily reward \eqref{eq:ppo-g}.\\
        Construct empirical estimates of relative value functions \eqref{eq:ppo-v-mc}.\\
        Update relative value network by minimizing the mean-squared norm \eqref{eq:ppo-v-norm}.\\
        Estimate advantage functions by \eqref{eq:advantage}.\\
        Obtain the updated policy network $\hat{\pi}_{\theta_m}$ by maximizing surrogate objective function \eqref{eq:ppo-obj}.
    }
    \Return{policy $\hat{\pi}_{\theta_M}$}
\end{algorithm}
Our Atomic-PPO algorithm (Figure \ref{fig:ppo}) is formally presented in Algo. \ref{algo:ppo}. For each policy iteration $m = 1, \dots, M$, we maintain a copy of the policy neural network parameters $\theta_{m-1}$ from the previous iteration and hold it fixed throughout the iteration. Then, we generate a dataset $\mathrm{Data}^{(K)}_{\theta_{m-1}}$ by rolling out the atomic action policy $\hat{\pi}_{\theta_{m-1}}$ using $K$ trajectories of Monte Carlo simulation. This dataset includes the reduced state $\bar{s}^{t,d, (k)}_{n}$, atomic action $\hat{a}^{t,d, (k)}_{n}$, and atomic reward $r^t(\hat{a}^{t,d, (k)}_{n})$ at $n$-th atomic step of the decision epoch $(t, d)$ of trajectory $k$:
\begin{align} \label{eq:ppo-data}
    &\mathrm{Data}^{(K)}_{\theta_{m-1}} := \nonumber\\ 
    &\left\{\left[\left[\left(\bar{s}^{t, d, (k)}_{n}, \hat{a}^{t, d, (k)}_{n}, r^t(\hat{a}^{t, d, (k)}_{n}) \right)_{n = 1}^{N} \right]_{t = 1}^{\Horizon} \right]_{d=1}^{\totaldays} \right\}_{k = 1}^K,
\end{align}
In each trajectory, we truncate the roll-out to $D$ days, with $T$ time steps in each day. Here, we set $D$ to be a large number that exceeds the days for the system to be stationary given the policy, see Sec. \ref{sec:numerical} for more details. The procedure for the sequential assignment of atomic actions to individual vehicles follows from Section \ref{sec:atomic-action}.
% The PPO algorithm updates the policy in iteration $m$ by maximizing the following objective function: 
% \begin{align}
%     &\hat{L}(\theta_m, \theta_{m-1}) := \frac{1}{K}\sum_{k = 1}^K \sum_{d = 1}^{\totaldays} \sum_{t = 1}^{\Horizon} \sum_{n = 1}^{\Size} \nonumber \\
%     &\qquad \min\left(\frac{\hat{\pi}_{\theta_m}(\hat{a}^{t,d, (k)}_{n} \vert \bar{s}^{t,d, (k)}_{n})}{\hat{\pi}_{\theta_{m-1}}(\hat{a}^{t,d, (k)}_{n} \vert \bar{s}^{t,d, (k)}_{n})} \cdot \right. \nonumber\\
%     &\qquad \left. \hat{A}_{\theta_{m-1}}(\bar{s}^{t,d, (k)}_{n}, \hat{a}^{t,d, (k)}_{n}), \right. \nonumber \\
%     &\qquad \left. \text{clip}\left(\frac{\hat{\pi}_{\theta_m}(\hat{a}^{t,d, (k)}_{n} \vert \bar{s}^{t,d, (k)}_{n})}{\hat{\pi}_{\theta_{m-1}}(\hat{a}^{t,d, (k)}_{n} \vert \bar{s}^{t,d, (k)}_{n})}, 1 - \epsilon, 1 + \epsilon \right) \cdot \right. \nonumber\\
%     &\qquad \left. \hat{A}_{\theta_{m-1}}(\bar{s}^{t,d, (k)}_{n}, \hat{a}^{t,d, (k)}_{n}) \right). \label{eq:ppo-obj}
% \end{align}
% The optimization of the parameterized atomic policy can be approached using gradient ascent methods. The gradient of the reward function with respect to parameter is given by \cite{sutton1999policy}: 
% \begin{align*}
%     &\triangledown_{\theta} R(\hat{\pi}_{\theta} \vert s) = \lim_{D \rightarrow \infty} \frac{1}{D} \mathbb{E}_{\hat{\pi}_{\theta}}\left[\sum_{d = 1}^D \sum_{t = 1}^T \sum_{n = 1}^N \right.\\
%     &\qquad \left. \triangledown_{\theta} \log \hat{\pi}_{\theta}(\hat{a}_n^{t,d} \vert \bar{s}_n^{t,d}) A_{\theta}(\bar{s}_n^{t,d}, \hat{a}_n^{t,d}) \Bigg\lvert s \right],
% \end{align*}where $A_{\theta}(\bar{s}_n^{t,d}, \hat{a}_n^{t,d})$ is the advantage function of the atomic policy $\hat{\pi}_{\theta}$ defined as 
% \begin{align*}
%     &A_{\theta}(\bar{s}_n^{t,d}, \hat{a}_n^{t,d}) = r^t(\hat{a}^{t,d}_n) - \frac{1}{TN} g_{\theta} \\
%     &\quad+ \sum_{s' \in \mathcal{S}} \hat{P}_n(s' \vert s^{t,d}_n, \hat{a}^{t,d}_n) h_{\theta, n+1}(s') - h_{\theta, n}(s^{t,d}_n),
% \end{align*}
% $g_{\theta}$ is the long-run average daily reward achieved by $\hat{\pi}_{\theta}$, and $h_{\theta, n}$ is the relative value function of $\hat{\pi}_{\theta}$ at $n$-th atomic step.
% Directly using the policy gradient method is known to be sample inefficient and unstable due to the high variance from the gradient estimation \citep{marbach2001simulation}.
% %Naive policy updates can lead to instability and divergence, especially when the policy updates are large \.
% To address this, \citep{schulman2017proximal} proposed Proximal Policy Optimization (PPO), which updates the policy by maximizing a clipped objective function 
% $L$ with respect to the current policy network parameters $\theta$, rather than directly applying gradient ascent to update the policy.%trust region policy optimization (TRPO) \citep{schulman2015trust} was introduced to ensure training stability by constraining the policy updates within a trust region, which possesses the monotonic policy improvement guarantee. Building on TRPO, 
% \begin{align}
%     &L(\theta, \theta_{old}) := \lim_{D \rightarrow \infty} \frac{1}{D} \mathbb{E}_{\hat{\pi}_{old}} \left[ \sum_{d = 1}^{\totaldays} \sum_{t = 1}^{\Horizon} \sum_{n = 1}^{\Size} \right. \nonumber \\
%     &\qquad \min\left(\frac{\hat{\pi}_{\theta}(\hat{a}^{t,d}_{n} \vert \bar{s}^{t,d}_{n})}{\hat{\pi}_{\theta_{old}}(\hat{a}^{t,d}_{n} \vert \bar{s}^{t,d}_{n})} \cdot \right. \nonumber\\
%     &\qquad \left. A_{\theta_{old}}(\bar{s}^{t,d}_{n}, \hat{a}^{t,d}_{n}), \right. \nonumber \\
%     &\qquad \left. \text{clip}\left(\frac{\hat{\pi}_{\theta}(\hat{a}^{t,d}_{n} \vert \bar{s}^{t,d}_{n})}{\hat{\pi}_{\theta_{old}}(\hat{a}^{t,d}_{n} \vert \bar{s}^{t,d}_{n})}, 1 - \epsilon, 1 + \epsilon \right) \cdot \right. \nonumber\\
%     &\qquad \left. \left. A_{\theta_{old}}(\bar{s}^{t,d}_{n}, \hat{a}^{t,d}_{n}) \right) \Bigg\lvert s \right], \label{eq:ppo-obj-expected}
% \end{align}
% where the clip function in \eqref{eq:ppo-obj-expected} ensures that the new policy $\hat{\pi}_{\theta}$ will stay close to the old policy $\hat{\pi}_{\theta_{old}}$.
Using the collected data, we construct the empirical estimate $\hat{g}$ of long-run average daily reward using\eqref{eq:ppo-g}\footnote{We assume that the system has a single recurrent class, so the long-run average reward is constant across all initial states.}.
\begin{equation} \label{eq:ppo-g}
    \hat{g} = \frac{1}{KD} \sum_{k = 1}^K \sum_{d = 1}^{\totaldays} \sum_{t = 1}^{\Horizon} \sum_{n = 1}^{\Size} r^t(\hat{a}^{t,d,(k)}_n).
\end{equation}

We also compute the empirical estimate of the relative value function of the current atomic policy $\hat{\pi}_{m-1}$. In particular, we define the relative value function $h_{n, m-1}$ of policy $\hat{\pi}_{m-1}$ at atomic step $n \in [N]$ as:
\begin{align} \label{eq:ppo-v-def}
    &h_{n, m-1}(s) = \mathbb{E}_{\hat{\pi}_{\theta_{m-1}}}\left[\sum_{i = n}^{N} \left(r^t(\hat{a}^{t,d, (k)}_{i}) - \frac{1}{TN}g_{m-1}\right) \right. \notag \\
    &+ \left. \sum_{\ell = t + 1}^{\Horizon} \sum_{i = 1}^{N} \left( r^{\ell}(\hat{a}^{\ell,d, (k)}_{i}) - \frac{1}{TN}g_{m-1} \right) \Bigg\lvert s^{t,1}_n = s \right] \notag \\
    &+ \sum_{d = 2}^{\infty} \sum_{\ell = 1}^{\Horizon} \sum_{i = 1}^{N} \notag\\
    &\qquad \mathbb{E}_{\hat{\pi}_{\theta_{m-1}}}\left[ \left( r^{\ell}(\hat{a}^{\ell,d', (k)}_{i}) - \frac{1}{TN}g_{m-1} \right) \Bigg\lvert s^{t,1}_n = s \right], \notag\\ 
    &\qquad\qquad \forall s \in \mathcal{S},\  \forall t \in [\Horizon],%\\
    % =& \mathbb{E}_{\hat{\pi}_{\theta_{m-1}}}\left[\sum_{i = n}^{N} \left(r^t(\hat{a}^{t,d, (k)}_{i}) - \frac{1}{TN}g_{m-1}\right) \right. \notag \\
    % &+ \left. \sum_{\ell = t + 1}^{\Horizon} \sum_{i = 1}^{N} \left( r^{\ell}(\hat{a}^{\ell,d, (k)}_{i}) - \frac{1}{TN}g_{m-1} \right) \Bigg\lvert s^{t,1}_n = s \right] \notag \\
    % &+ \sum_{\ell = 1}^{\Horizon} \sum_{i = 1}^{N} \sum_{d = 2}^{\infty} \notag\\
    % &\qquad \mathbb{E}_{\hat{\pi}_{\theta_{m-1}}}\left[ \left( r^{\ell}(\hat{a}^{\ell,d', (k)}_{i}) - g_{n, m-1}^t \right) \Bigg\lvert s^{t,1}_n = s \right],\notag \\ 
    % &\qquad \qquad \forall s \in \mathcal{S},\ \forall t \in [\Horizon], \label{eq:ppo-v-def-swapped}
\end{align}
where $g_{m-1}$ is the long-run average daily reward achieved by $\hat{\pi}_{m-1}$ and we recall that the state $s$ contains the time of day $t$ information. By Proposition 2 in our concurrent work \cite{dai2025optimal}, the infinite series in \eqref{eq:ppo-v-def} is well defined. Additionally, we remark that under the atomic action decomposition, our Markov chain has a period of $TN$, which is the total number of atomic steps in each day. Our definition of the relative value function is equivalent to the one defined using the Cesaro limit, as given by Puterman (see page 338 of \cite{PutermanMDP}) for periodic chains, up to an additive constant (see Proposition 3 in \cite{dai2025optimal}).

% where $g_{n, m-1}^t$ is the long-run average reward achieved by $\hat{\pi}_{m-1}$ at $n$-th atomic step and time $t$ across all days, $g_{m-1} := \sum_{t = 1}^T \sum_{n = 1}^N g_{n, m-1}^t$ is the long-run average daily reward achieved by $\hat{\pi}_{m-1}$, and we recall that the state $s$ contains the time of day $t$ information. We remark that under the atomic action decomposition, our Markov chain has a period of $TN$, which is the total number of atomic steps in each day. Hence, for any atomic step $i \in [N]$ and time $t \in [\Horizon]$, the Markov chain induced by the states at the $i$-th atomic step of time $\ell$ of all days is aperiodic. Additionally, with the condition that our state space is finite and the assumption that the system has a single recurrent class, the infinite series in \eqref{eq:ppo-v-def-swapped} converges (see Theorem 1.8.5 on page 44 of \cite{norris1998markov}) and therefore our relative value function given in \eqref{eq:ppo-v-def} is well defined.
For any state $s^{t,d,(k)}_n$ in the atomic step $n$ of trajectory $k$ of decision epoch $(t, d)$, we construct an empirical estimate $\hat{h}^{t,d,(k)}_n$ of its relative value function as:
\begin{align} \label{eq:ppo-v-mc}
    &\hat{h}^{t,d, (k)}_{n} := \sum_{i = n}^{N} \left(r^t(\hat{a}^{t,d, (k)}_{i}) - \frac{1}{TN}\hat{g}\right) \notag \\
    &\qquad+ \sum_{\ell = t + 1}^{\Horizon} \sum_{i = 1}^{N} \left( r^{\ell}(\hat{a}^{\ell,d, (k)}_{i}) - \frac{1}{TN}\hat{g} \right) \notag \\
    &\qquad+ \sum_{d' = d + 1}^{\totaldays} \sum_{\ell = 1}^{\Horizon} \sum_{i = 1}^{N} \left( r^{\ell}(\hat{a}^{\ell,d', (k)}_{i}) - \frac{1}{TN}\hat{g} \right).
\end{align} 
Due to the large state space, we use neural networks $h_{\psi_{m-1}}: \bar{\mathcal{S}} \rightarrow \mathbb{R}$ on the reduced state space to approximate the relative value function for all atomic steps, where $\psi_{m-1}$ is the network parameters. 
We learn $h_{\psi_{m-1}}$ by minimizing the mean-square loss given the empirical estimates:
\begin{equation} \label{eq:ppo-v-norm}
    \sum_{k = 1}^K \sum_{d = 1}^{\totaldays} \sum_{t = 1}^{\Horizon} \sum_{n = 1}^{\Size} \left( h_{\psi_{m-1}}(\bar{s}^{t,d, (k)}_n) - \hat{h}^{t,d, (k)}_n\right)^2.
\end{equation}

This allows us to compute the empirical estimates of the advantage functions for each atomic step. The advantage function quantifies how much better (or worse) a specific atomic action \( \hat{a}^{t,d, (k)}_{n}\) performs compared to following the previous stage policy \( \hat{\pi}_{\theta_{m-1}} \) at a given reduced state \( \bar{s}^{t,d, (k)}_{n}\).
\begin{align} \label{eq:advantage}
    &\hat{A}_{\theta_{m-1}}(\bar{s}^{t,d, (k)}_{n}, \hat{a}^{t,d, (k)}_{n}) := \nonumber\\ 
    &\quad \begin{cases}
        r^t(\hat{a}^{t,d, (k)}_{n}) - \frac{1}{TN}\hat{g} + \\
        h_{\psi_{m-1}}(\bar{s}^{t,d, (k)}_{n + 1}) - h_{\psi_{m-1}}(\bar{s}^{t,d, (k)}_{n}), \\
        \quad \text{if } n < N,\\
        r^t(\hat{a}^{t,d, (k)}_{n}) - \frac{1}{TN}\hat{g} + \\
        h_{\psi_{m-1}}(\bar{s}^{t + 1,d, (k)}_{1}) - h_{\psi_{m-1}}(\bar{s}^{t,d, (k)}_{n}), \\
        \quad \text{if } n = N, t < \Horizon,\\
        r^t(\hat{a}^{t,d, (k)}_{n}) - \frac{1}{TN}\hat{g} + \\
        h_{\psi_{m-1}}(\bar{s}^{1,d+1, (k)}_{1}) - h_{\psi_{m-1}}(\bar{s}^{t,d, (k)}_{n}), \\
        \quad \text{if } n = N, t = \Horizon.\\
    \end{cases}
\end{align}
%where $\hat{g}$ represents the empirical estimate of the long-run average daily reward achieved by the atomic action policy $\hat{\pi}_{\theta_{m-1}}$, and $h_{\psi_{m-1}}$ represents the relative value function of the policy approximated by neural networks.


% \begin{align} \label{eq:ppo-v-mc}
%     &\hat{h}^{t,d, (k)}_{n} := -\frac{1}{T}\hat{g} + \sum_{i = n}^{N} r^t(\hat{a}^{t,d, (k)}_{i}) \notag \\
%     &\qquad+ \sum_{\ell = t + 1}^{\Horizon} \left( - \frac{1}{T}\hat{g} + \sum_{i = 1}^{N} r^{\ell}(\hat{a}^{\ell,d, (k)}_{i}) \right) \notag \\
%     &\qquad+ \sum_{d' = d + 1}^{\totaldays} \sum_{\ell = t + 1}^{\Horizon} \left(-\frac{1}{T}\hat{g} + \sum_{i = 1}^{N} r^{\ell}(\hat{a}^{\ell,d', (k)}_{i}) \right)
% \end{align}

Using the estimated advantage function $\hat{A}_{\theta_{m-1}}$, PPO algorithm select the the atomic action policy function of the next iteration $\hat{\pi}_{\theta_m}$ by choosing parameter $\theta_m$ that maximizes the clipped objective function defined as follows:
\begin{align}
    &\hat{L}(\theta_m, \theta_{m-1}) := \frac{1}{K}\sum_{k = 1}^K \sum_{d = 1}^{\totaldays} \sum_{t = 1}^{\Horizon} \sum_{n = 1}^{\Size} \nonumber \\
    &\qquad \min\left(\frac{\hat{\pi}_{\theta_m}(\hat{a}^{t,d, (k)}_{n} \vert \bar{s}^{t,d, (k)}_{n})}{\hat{\pi}_{\theta_{m-1}}(\hat{a}^{t,d, (k)}_{n} \vert \bar{s}^{t,d, (k)}_{n})} \cdot \right. \nonumber\\
    &\qquad \left. \hat{A}_{\theta_{m-1}}(\bar{s}^{t,d, (k)}_{n}, \hat{a}^{t,d, (k)}_{n}), \right. \nonumber \\
    &\qquad \left. \text{clip}\left(\frac{\hat{\pi}_{\theta_m}(\hat{a}^{t,d, (k)}_{n} \vert \bar{s}^{t,d, (k)}_{n})}{\hat{\pi}_{\theta_{m-1}}(\hat{a}^{t,d, (k)}_{n} \vert \bar{s}^{t,d, (k)}_{n})}, 1 - \epsilon, 1 + \epsilon \right) \cdot \right. \nonumber\\
    &\qquad \left. \hat{A}_{\theta_{m-1}}(\bar{s}^{t,d, (k)}_{n}, \hat{a}^{t,d, (k)}_{n}) \right), \label{eq:ppo-obj}
\end{align}
where $\epsilon>0$ is a hyper parameter referred as the clip size of the training. The PPO policy update, as defined in \eqref{eq:ppo-obj}, was introduced by \citep{schulman2017proximal} to enhance the computational efficiency of trust region policy optimization (TRPO) \cite{schulman2015trust}. TRPO was developed to replace the original policy gradient method \cite{sutton1999policy}, offering the advantage of improved sample efficiency and the monotonic policy improvement guarantee. 
% $L$ with respect to the current policy network parameters $\theta$, rather than directly applying gradient ascent to update the policy.%trust region policy optimization (TRPO) \citep{schulman2015trust} was introduced to ensure training stability by constraining the policy updates within a trust region, which .
% The clip function in our objective function \eqref{eq:ppo-obj} maintains stability of the training by ensuring that the new policy $\hat{\pi}_{\theta_m}$ will stay close to the old policy $\hat{\pi}_{\theta_{m-1}}$. The PPO framework we adopt possesses the stability of trust-region methods \citep{schulman2015trust}, which has monotonic policy improvement guarantee for general stochastic policies.



% Finally, we optimize the atomic action policy function $\hat{\pi}_{\theta_m}$ by maximizing the clipped surrogate objective function \eqref{eq:ppo-obj} w.r.t parameter $\theta_m$ under the PPO framework in \cite{schulman2017proximal}. 
% \begin{align}
%     &\hat{L}(\theta_m, \theta_{m-1}, \mathrm{Data}_{\theta_{m-1}}^{(K)}) := \frac{1}{K}\sum_{k = 1}^K \sum_{d = 1}^{\totaldays} \sum_{t = 1}^{\Horizon} \sum_{n = 1}^{\Size} \nonumber \\
%     &\qquad \min\left(\frac{\hat{\pi}_{\theta_m}(\hat{a}^{t,d, (k)}_{n} \vert \bar{s}^{t,d, (k)}_{n})}{\hat{\pi}_{\theta_{m-1}}(\hat{a}^{t,d, (k)}_{n} \vert \bar{s}^{t,d, (k)}_{n})} \cdot \right. \nonumber\\
%     &\qquad \left. \hat{A}_{\theta_{m-1}}(\bar{s}^{t,d, (k)}_{n}, \hat{a}^{t,d, (k)}_{n}), \right. \nonumber \\
%     &\qquad \left. \text{clip}\left(\frac{\hat{\pi}_{\theta_m}(\hat{a}^{t,d, (k)}_{n} \vert \bar{s}^{t,d, (k)}_{n})}{\hat{\pi}_{\theta_{m-1}}(\hat{a}^{t,d, (k)}_{n} \vert \bar{s}^{t,d, (k)}_{n})}, 1 - \epsilon, 1 + \epsilon \right) \cdot \right. \nonumber\\
%     &\qquad \left. \hat{A}_{\theta_{m-1}}(\bar{s}^{t,d, (k)}_{n}, \hat{a}^{t,d, (k)}_{n}) \right). \label{eq:ppo-obj}
% \end{align}
% In each policy iteration $m$, PPO maintains a copy of the network parameters $\theta_{m-1}$ from the previous iteration and hold it fixed throughout the iteration. In each training epoch of this policy iteration, PPO computes the gradients of the surrogate loss $\hat{L}$ w.r.t the current network $\theta_m$ and then updates $\theta_m$ using stochastic gradient descent or its variants. %It conducts conservative policy updates by using the clip function. It possesses the reliability of trust-region methods \cite{schulman2015trust}, which has monotonic policy improvement guarantee for general stochastic policies.

% The clipping function given $\epsilon \in (0, 1)$ is defined as \[
%     \text{clip}(x, 1 - \epsilon, 1 + \epsilon) := \begin{cases}
%         \min(x, 1 + \epsilon) & \text{If } x \geq 1,\\
%         \max(x, 1 - \epsilon) & \text{If } x < 1.\\
%     \end{cases}
% \]

% \begin{figure}
%     \centering
%     \includegraphics[width=0.8\linewidth]{plots/policy_eval.png}
%     \caption{Atomic-PPO Policy Evaluation}
%     \label{fig:policy_eval}
% \end{figure}

%We remark that the transitions of the MDP process is tracked using the full state information (i.e. without aggregation), whereas the inputs to the policy network and the value network are aggregated states. The main storage bottleneck comes from \eqref{eq:ppo-data}, where the number of state transitions scales with the number of episodes, time horizon, and number of vehicles. For each state vector stored in the dataset, we also need to use it to query the value network \eqref{eq:ppo-v-norm}-\eqref{eq:advantage} and the policy network \eqref{eq:ppo-obj}. Therefore, a significant amount of space and runtime can be saved by using aggregated states for network inputs. On the other hand, the MDP transition process can be implemented by updating the state vector on the fly, while we do not have to store any extra snapshot of the full state for the MDP transition. As a result, we can use the full state information to make the MDP transitions consistent, with very minimal space and runtime overhead. 

\section{Reward Upper Bound Provided By Fluid Approximation Model} \label{sec:fluid}
Before presenting the performance of our atomic PPO algorithm, in this section, we construct an upper bound on the optimal long-run average reward using fluid limit. This upper bound will be used to construct an upper bound of optimality gap of our atomic PPO algorithm, as shown in the next section. We reformulate our robo-taxi dispatching problem as a fluid-based linear optimization program, where the fluid limit is attained as the fleet size approaches infinity, with both trip demand volume and the number of chargers scaling up proportionally to the fleet size. Under the fluid limit, the system becomes deterministic, and the fleet dispatching policy, which is a probability distribution of vehicle flows across all actions, reduces to a deterministic vector that represents the fraction of fleet assigned to each action at each time of the day.

%In particular, we formulate the vehicle dispatching as a fluid-based optimization problem, where the fluid limit is attained as the fleet size approaches infinity, where both the mean of trip arrivals and the number of chargers are scaled up proportionally to the fleet size. At each time of a day, we make a decision on the fraction of vehicles taking each action. The objective is to maximize the reward obtained for a single day.

We define the decision variables of the fluid-based optimization problem as follows:
\begin{enumerate}
    \item[-] \emph{Fraction of fleet for trip fulfilling $\lptripfulfill{}{} := \left(\lptripfulfill{\cartype{}, \triptype{}}{\time}\right)_{\cartype{} \in \Cartype, \triptype{} \in \Triptype, \time \in [T]}$}, where $\lptripfulfill{\cartype{}, \triptype{}}{\time}$ is the fraction of vehicles with status $\cartype{}$ fulfilling trip requests of status $\triptype{}$ at time $\time$.
    \item[-] \emph{Fraction of fleet for repositioning $\lpreroute{}{} := \left(\lpreroute{\cartype{}, \destination}{\time}\right)_{\cartype{} \in \Cartype, \destination \in \Region, \time \in [T]}$}, where $\lpreroute{\cartype{}, \destination}{\time}$ is the fraction of vehicles with status $\cartype{}$ repositioning to $\destination$ at time $\time$. 
    \item[-] \emph{Fraction of fleet for charging $\lpcharge{}{} := \left(\lpcharge{\cartype{}, \type}{\time}\right)_{\cartype{} \in \Cartype, \type \in \Type, \time \in [T]}$}, where $\lpcharge{\cartype{}, \type}{\time}$ denotes the fraction of vehicles with status $\cartype{}$ charging with rate $\type$ at time $\time$.
    \item[-] \emph{Fraction of fleet for continuing the current action $\lppass{}{} := \left(\lppass{\cartype{}}{\time}\right)_{\cartype{} \in \Cartype, \time \in [\Horizon]}$}, where $\lppass{\cartype{}}{\time}$ is the fraction of fleet with status $\cartype{}$ taking the passing action at time $\time$.
\end{enumerate}

The fluid-based linear program aims at maximizing the total reward achieved by the fluid policy:
{
\begin{align*}
    &\max_{\lptripfulfill{}{}, \lpreroute{}{}, \lpcharge{}{}, \lpslack{}{}, \lppass{}{}}  \Size \sum_{\time \in [\Horizon]}^{} \sum_{\cartype{} \in \Cartype} \left\{ \sum_{\origin \in \Region} \sum_{\destination \in \Region} \left[ \tripfulfillreward{,\origin\destination}{\time} \sum_{\tripactivetime \in [\Lc]}^{} \lptripfulfill{\cartype{}, (\origin, \destination, \tripactivetime)}{\time}   \right.\right. \nonumber\\
    &\qquad+ \left.\left. \reroutingreward{,\origin\destination}{\time} \lpreroute{\cartype{}, \destination}{\time} \right] + \sum_{\type \in \Type} \chargingreward{,\type}{\time} \lpcharge{\cartype{}, \type}{\time} \right\}, \notag \\%\label{eq:fluid-obj}\\
    &\text{s.t.} \  \eqref{eq:fluid-ev-conservation-red}-\eqref{eq:fluid-nonneg-red}.
\end{align*}
}


The constraints are given as follows:
\begin{enumerate}
    \item The flow conservation for each vehicle status $\cartype{}:= (\destination, \timetoarrival, \battery) \in \Cartype$ at each time $\time$ of a day. \\
    In particular, the left-hand side represents the vehicle flows from $\time - 1$ transitioning to the vehicle status $\cartype{}$ according to \eqref{eq:setup-car-state-transition}. The right-hand side represents the assignment of vehicles of status $\cartype{}$ to trip-fulfilling, repositioning, charging, and idling/passing actions. 
    \begin{align}
        &\left(\sum_{\origin \in \Region} \sum_{\cartype{}'= (\origin, \timetoarrival', \battery') \in \Cartype} \sum_{ \triptype{}= (\origin, \destination, \xi') \in \Triptype} \right. \notag\\
        &\qquad \left. \lptripfulfill{\cartype{}', \triptype{}}{\time-1}\mathds{1}(\timetoarrival' + \timecost{\origin\destination}{\time-1} - 1 = \timetoarrival, \ \battery' - \batterycost{\origin \destination} = \battery) \right) \notag \\
        &\qquad+ \left(\sum_{\origin \in \Region} \sum_{\cartype{}'= (\origin, 0, \battery') \in \Cartype} \lpreroute{\cartype{}', \destination}{\time-1} \cdot \right. \notag\\
        &\qquad \left. \mathds{1}(\timecost{\origin\destination}{\time-1} - 1 = \timetoarrival, \  \battery' - \batterycost{\origin\destination} = \battery) \right) \notag \\
        &\qquad+ \left[ \left(\sum_{\type \in \Type} \lpcharge{(\destination, 0, \battery - \type \chargetime), \type}{\time-1} \mathds{1}(\timetoarrival = \chargetime - 1, \battery \geq \type \chargetime) \right) + \right. \notag\\
        &\qquad \left. \left(\sum_{\battery' > \battery - \type \chargetime} \sum_{\type \in \Type} \lpcharge{(\destination, 0, \battery'), \type}{\time-1} \mathds{1}(\timetoarrival = \chargetime - 1, \battery = \range) \right) \right] \notag\\
        &\qquad+ \lppass{(\destination, \timetoarrival, \battery)}{\time-1} \mathds{1}(\timetoarrival = 0) + \lppass{(\destination, \timetoarrival+1, \battery)}{\time-1} \mathds{1}(\timetoarrival < \maxtimecost{}) \notag \\
        &= \sum_{\triptype{} \in \Triptype} \lptripfulfill{\cartype{}, \triptype{}}{\time} + \sum_{\destination \in \Region} \lpreroute{\cartype{}, \destination}{\time} + \sum_{\type \in \Type} \lpcharge{\cartype{}, \type}{\time} + \lppass{\cartype{}}{\time},\notag\\ 
        &\qquad \forall \cartype{}:= (\destination, \timetoarrival, \battery) \in \Cartype, \   \time \in [\Horizon], \label{eq:fluid-ev-conservation-red}
    \end{align}
    We note that the time steps are periodic across days, and thus for $t=1$ in \eqref{eq:fluid-ev-conservation-red}, $t-1$ is the last time step $T$ of the previous day. Similarly, in all of the subsequent constraints \eqref{eq:fluid-passenger-flow-cap-red} -- \eqref{eq:fluid-charging-cap-red}, the time step $t$ on the superscript of a variable being negative indicates time step $T-t$ in the previous day and $t>T$ indicates time step $t-T$ of the next day. 
    \item The fulfillment of trip orders does not exceed their arrivals.
    \begin{align}
        &\sum_{\cartype{}=(\origin, \timetoarrival, \battery) \in \Cartype}^{}\sum_{\triptype{}=(\origin, \destination, \tripactivetime) \in \Triptype}^{} \lptripfulfill{\cartype{}, \triptype{}}{\time + \tripactivetime} \leq \frac{1}{\Size} \arrrate{\origin \destination}{\time}, \notag\\ 
        &\qquad \forall \origin, \destination \in \Region,\ \time \in [\Horizon].\label{eq:fluid-passenger-flow-cap-red}
    \end{align}
    \item The number of vehicles charging at a specific rate in a given region does not exceed the corresponding charging capacity at any time.
    \begin{align}
        \sum_{j \in [\chargetime]} \sum_{\cartype{}= (\region, 0, \battery) \in \Cartype} \lpcharge{\cartype{}, \type}{\time - j} \leq \frac{1}{\Size}\n{\region}{\type}, \forall \type \in \Type, \  \time \in [\Horizon]. \label{eq:fluid-charging-cap-red}
    \end{align}
    \item The battery is sufficient for vehicles to complete the trips for trip-fulfillment.
    \begin{align}
        &\ \lptripfulfill{\cartype{}, \triptype{}}{\time} \mathds{1}\{\battery < \batterycost{\origin \destination}\} = 0,\notag\\ 
        &\quad \forall \cartype{}= (\origin, \timetoarrival, \battery) \in \Cartype,\ \triptype{}= (\origin, \destination, \tripactivetime) \in \Triptype, \ \time \in [\Horizon]. \label{eq:fluid-passenger-flow-battery-sufficiency-red}
    \end{align}
    \item The battery is sufficient for vehicles to complete the trips for repositioning.
    \begin{align}
        &\ \lpreroute{\cartype{}, \destination}{\time} \mathds{1}\{\battery < \batterycost{\origin \destination}\} = 0, \notag\\
        &\quad \forall \cartype{}= (\origin, \timetoarrival, \battery) \in \Cartype,\ \destination \in \Region, \ \time \in [\Horizon]. \label{eq:fluid-rerouting-flow-battery-sufficiency-red}
    \end{align}
    \item The fractions of vehicles of all statuses should add up to 1 at all times.
    \begin{align}
        &\ \sum_{\cartype{} \in \Cartype} \left[ \sum_{\triptype{} \in \Triptype} \lptripfulfill{\cartype{}, \triptype{}}{\time} + \sum_{\destination \in \Region} \lpreroute{\cartype{}, \destination}{\time} + \sum_{\type \in \Type} \lpcharge{\cartype{}, \type}{\time} + \lppass{\cartype{}}{\time} \right] = 1,\notag \\
        & \qquad\qquad \qquad\qquad \qquad\qquad  \forall \time \in [\Horizon]. \label{eq:fluid-total-flow-red}
    \end{align}
    \item All decision variables are non-negative.
    \begin{align}
        \lptripfulfill{}{}, \lpreroute{}{}, \lpcharge{}{}, \lppass{}{} \geq 0. \label{eq:fluid-nonneg-red}
    \end{align}
\end{enumerate}

Let $\bar{R}$ be the optimal objective value from the fluid based LP. 

\begin{theorem} \label{thm:fluid-obj-val}
    $R^*(s) \leq \bar{R}, \quad \forall s \in \mathcal{S}$.
\end{theorem}

The proof of Theorem \ref{thm:fluid-obj-val} is deferred to the Appendix. Theorem \ref{thm:fluid-obj-val} shows that $\bar{R}$ is an upper bound on the long-run average daily reward that can be achieved by any feasible policy. In the numerical section, we assess the gap between the average daily reward achieved by our Atomic-PPO and the fluid upper bound $\bar{R}$. This gap is an upper bound of optimality gap achieved by Atomic-PPO algorithm. We note that under the current formulation of the fluid-based LP, the number of variables scales with $\vert \Region \vert (\Lp + \maxtimecost{}{}) \range \Horizon$, where $\maxtimecost{}{}$ can potentially be very large. We can reduce the size of this LP to $|V|L_pBT$ without loss of optimality by leveraging the fact that only vehicles with task remaining time $\eta < L_p$ can be assigned with new tasks, and therefore we only need to keep track of a fraction of fleet statuses when computing the optimal fluid policy. We delay the presentation of our simplified fluid-based LP to the appendix.
%As a high-level idea of the proof, we first show that it is without loss of optimality to restrict our attention to deterministic stationary policies. Then, we show that for every deterministic stationary policy that satisfies all the feasibility constraints in Sec. \ref{sec:model}, we can construct a feasible solution to \eqref{eq:fluid-lp} by using the expected value of the fraction of vehicles taking each action at stationarity. Lastly, we argue that by substituting the variables in the objective function of \eqref{eq:fluid-lp} with the expected value of the fraction of vehicles taking each action at stationarity, we can obtain the long-run average reward achieved by the policy. Therefore, it immediately follows that the objective value obtained from \eqref{eq:fluid-lp} is an upper bound on the long-run average reward achievable by any feasible policy. The formal proof of theorem \ref{thm:fluid-obj-val} is delayed to the appendix.

\section{Numerical Experiments} \label{sec:numerical}
\section{Preliminary numerical results} 
\label{sec:main/numerical}

\begin{figure}[tbp]
    \centering
    \includegraphics{figures/main-ss1-time.pdf} \hfill
    \includegraphics{figures/main-ss1-hesseval.pdf}
    \caption{
        Comparison of success rates as functions of elapsed time and Hessian evaluations for CUTEst benchmark problems.  
        \algname{ARNCG$_g$}, \algname{ARNCG$_\epsilon$}, and ``Fixed'' correspond to \Cref{alg:adap-newton-cg} with the first and second regularizers from \theoremref{thm:newton-local-rate-boosted}, and a fixed $\omega_k \equiv \sqrt{\epsilon}$, respectively.  
        For Hessian evaluations, 
        since our algorithm accesses this information only via Hessian-vector products, 
        we count multiple products involving $\nabla^2\varphi(x)$ at the same point $x$ as a single evaluation.
        }
    \label{fig:main-algoperf}
\end{figure}

In this section, we present some preliminary numerical results.\footnote{Our code is available at \url{https://github.com/miskcoo/ARNCG}.} %
Our primary goal is to provide an overall sense of our algorithm's performance and the effects of its components.
Detailed results are deferred to \Cref{sec:appendix/numerical-results}.

Since the recently proposed trust-region-type method \algname{CAT} has an optimal rate and shows competitiveness with state-of-the-art solvers~\citep{hamad2024simple}, we adopt their experimental setup and compare with it, as well as the regularized Newton-type method \algname{AN2CER} proposed by \citet{gratton2024yet}.
The experiments are conducted on the 124 unconstrained problems with more than 100 variables from the widely used CUTEst benchmark for nonlinear optimization~\citep{gould2015cutest}.
The algorithm is considered successful if it terminates with $\epsilon_k \leq \epsilon = 10^{-5}$ such that $k \leq 10^5$. If the algorithm fails to terminate within 5 hours, it is also recorded as a failure.

In \Cref{sec:appendix/numerical-results}, 
we observe that the fallback step has insignificant impact on  performance yet increases computational cost, suggesting it can be relaxed or removed.
Furthermore, $\theta \in [0.5, 1]$ balances computational efficiency and local behavior 
and a small $m_{\mathrm{max}}$ is preferable. 
Finally, the second linesearch step \eqref{eqn:smooth-line-search-sol-smaller-stepsize} and the \texttt{TERM} state of \texttt{CappedCG} are rarely taken in practice.

\figureref{fig:main-algoperf} shows our method without the fallback step (see \Cref{sec:appendix/numerical-results} for details). 
It is slightly faster than CAT and AN2CER, 
as each iteration uses only a few Hessian-vector products, 
whereas CAT relies on multiple Cholesky factorizations and AN2CER involves minimal eigenvalue computations. 
Meanwhile, our method requires a similar number of Hessian evaluations as CAT, and slightly fewer than AN2CER.
We also note that using a fixed $\omega_k = \sqrt{\epsilon}$ in \Cref{alg:adap-newton-cg}
may lead to failures when $g_k \gg \epsilon$, resulting in deteriorated performance.
Additionally, our method requires significantly less memory ($\sim$6GB) compared to CAT ($\sim$74GB) for the largest problem in the benchmark with 123200 variables, as it avoids  constructing the full Hessian.


\section{Concluding Remarks}
\section{Conclusion}
In this work, we propose a simple yet effective approach, called SMILE, for graph few-shot learning with fewer tasks. Specifically, we introduce a novel dual-level mixup strategy, including within-task and across-task mixup, for enriching the diversity of nodes within each task and the diversity of tasks. Also, we incorporate the degree-based prior information to learn expressive node embeddings. Theoretically, we prove that SMILE effectively enhances the model's generalization performance. Empirically, we conduct extensive experiments on multiple benchmarks and the results suggest that SMILE significantly outperforms other baselines, including both in-domain and cross-domain few-shot settings.

% \ACKNOWLEDGMENT{%
% % Enter the text of acknowledgments here
% }% Leave this (end of acknowledgment)

\clearpage

\appendix
\subsection{Lloyd-Max Algorithm}
\label{subsec:Lloyd-Max}
For a given quantization bitwidth $B$ and an operand $\bm{X}$, the Lloyd-Max algorithm finds $2^B$ quantization levels $\{\hat{x}_i\}_{i=1}^{2^B}$ such that quantizing $\bm{X}$ by rounding each scalar in $\bm{X}$ to the nearest quantization level minimizes the quantization MSE. 

The algorithm starts with an initial guess of quantization levels and then iteratively computes quantization thresholds $\{\tau_i\}_{i=1}^{2^B-1}$ and updates quantization levels $\{\hat{x}_i\}_{i=1}^{2^B}$. Specifically, at iteration $n$, thresholds are set to the midpoints of the previous iteration's levels:
\begin{align*}
    \tau_i^{(n)}=\frac{\hat{x}_i^{(n-1)}+\hat{x}_{i+1}^{(n-1)}}2 \text{ for } i=1\ldots 2^B-1
\end{align*}
Subsequently, the quantization levels are re-computed as conditional means of the data regions defined by the new thresholds:
\begin{align*}
    \hat{x}_i^{(n)}=\mathbb{E}\left[ \bm{X} \big| \bm{X}\in [\tau_{i-1}^{(n)},\tau_i^{(n)}] \right] \text{ for } i=1\ldots 2^B
\end{align*}
where to satisfy boundary conditions we have $\tau_0=-\infty$ and $\tau_{2^B}=\infty$. The algorithm iterates the above steps until convergence.

Figure \ref{fig:lm_quant} compares the quantization levels of a $7$-bit floating point (E3M3) quantizer (left) to a $7$-bit Lloyd-Max quantizer (right) when quantizing a layer of weights from the GPT3-126M model at a per-tensor granularity. As shown, the Lloyd-Max quantizer achieves substantially lower quantization MSE. Further, Table \ref{tab:FP7_vs_LM7} shows the superior perplexity achieved by Lloyd-Max quantizers for bitwidths of $7$, $6$ and $5$. The difference between the quantizers is clear at 5 bits, where per-tensor FP quantization incurs a drastic and unacceptable increase in perplexity, while Lloyd-Max quantization incurs a much smaller increase. Nevertheless, we note that even the optimal Lloyd-Max quantizer incurs a notable ($\sim 1.5$) increase in perplexity due to the coarse granularity of quantization. 

\begin{figure}[h]
  \centering
  \includegraphics[width=0.7\linewidth]{sections/figures/LM7_FP7.pdf}
  \caption{\small Quantization levels and the corresponding quantization MSE of Floating Point (left) vs Lloyd-Max (right) Quantizers for a layer of weights in the GPT3-126M model.}
  \label{fig:lm_quant}
\end{figure}

\begin{table}[h]\scriptsize
\begin{center}
\caption{\label{tab:FP7_vs_LM7} \small Comparing perplexity (lower is better) achieved by floating point quantizers and Lloyd-Max quantizers on a GPT3-126M model for the Wikitext-103 dataset.}
\begin{tabular}{c|cc|c}
\hline
 \multirow{2}{*}{\textbf{Bitwidth}} & \multicolumn{2}{|c|}{\textbf{Floating-Point Quantizer}} & \textbf{Lloyd-Max Quantizer} \\
 & Best Format & Wikitext-103 Perplexity & Wikitext-103 Perplexity \\
\hline
7 & E3M3 & 18.32 & 18.27 \\
6 & E3M2 & 19.07 & 18.51 \\
5 & E4M0 & 43.89 & 19.71 \\
\hline
\end{tabular}
\end{center}
\end{table}

\subsection{Proof of Local Optimality of LO-BCQ}
\label{subsec:lobcq_opt_proof}
For a given block $\bm{b}_j$, the quantization MSE during LO-BCQ can be empirically evaluated as $\frac{1}{L_b}\lVert \bm{b}_j- \bm{\hat{b}}_j\rVert^2_2$ where $\bm{\hat{b}}_j$ is computed from equation (\ref{eq:clustered_quantization_definition}) as $C_{f(\bm{b}_j)}(\bm{b}_j)$. Further, for a given block cluster $\mathcal{B}_i$, we compute the quantization MSE as $\frac{1}{|\mathcal{B}_{i}|}\sum_{\bm{b} \in \mathcal{B}_{i}} \frac{1}{L_b}\lVert \bm{b}- C_i^{(n)}(\bm{b})\rVert^2_2$. Therefore, at the end of iteration $n$, we evaluate the overall quantization MSE $J^{(n)}$ for a given operand $\bm{X}$ composed of $N_c$ block clusters as:
\begin{align*}
    \label{eq:mse_iter_n}
    J^{(n)} = \frac{1}{N_c} \sum_{i=1}^{N_c} \frac{1}{|\mathcal{B}_{i}^{(n)}|}\sum_{\bm{v} \in \mathcal{B}_{i}^{(n)}} \frac{1}{L_b}\lVert \bm{b}- B_i^{(n)}(\bm{b})\rVert^2_2
\end{align*}

At the end of iteration $n$, the codebooks are updated from $\mathcal{C}^{(n-1)}$ to $\mathcal{C}^{(n)}$. However, the mapping of a given vector $\bm{b}_j$ to quantizers $\mathcal{C}^{(n)}$ remains as  $f^{(n)}(\bm{b}_j)$. At the next iteration, during the vector clustering step, $f^{(n+1)}(\bm{b}_j)$ finds new mapping of $\bm{b}_j$ to updated codebooks $\mathcal{C}^{(n)}$ such that the quantization MSE over the candidate codebooks is minimized. Therefore, we obtain the following result for $\bm{b}_j$:
\begin{align*}
\frac{1}{L_b}\lVert \bm{b}_j - C_{f^{(n+1)}(\bm{b}_j)}^{(n)}(\bm{b}_j)\rVert^2_2 \le \frac{1}{L_b}\lVert \bm{b}_j - C_{f^{(n)}(\bm{b}_j)}^{(n)}(\bm{b}_j)\rVert^2_2
\end{align*}

That is, quantizing $\bm{b}_j$ at the end of the block clustering step of iteration $n+1$ results in lower quantization MSE compared to quantizing at the end of iteration $n$. Since this is true for all $\bm{b} \in \bm{X}$, we assert the following:
\begin{equation}
\begin{split}
\label{eq:mse_ineq_1}
    \tilde{J}^{(n+1)} &= \frac{1}{N_c} \sum_{i=1}^{N_c} \frac{1}{|\mathcal{B}_{i}^{(n+1)}|}\sum_{\bm{b} \in \mathcal{B}_{i}^{(n+1)}} \frac{1}{L_b}\lVert \bm{b} - C_i^{(n)}(b)\rVert^2_2 \le J^{(n)}
\end{split}
\end{equation}
where $\tilde{J}^{(n+1)}$ is the the quantization MSE after the vector clustering step at iteration $n+1$.

Next, during the codebook update step (\ref{eq:quantizers_update}) at iteration $n+1$, the per-cluster codebooks $\mathcal{C}^{(n)}$ are updated to $\mathcal{C}^{(n+1)}$ by invoking the Lloyd-Max algorithm \citep{Lloyd}. We know that for any given value distribution, the Lloyd-Max algorithm minimizes the quantization MSE. Therefore, for a given vector cluster $\mathcal{B}_i$ we obtain the following result:

\begin{equation}
    \frac{1}{|\mathcal{B}_{i}^{(n+1)}|}\sum_{\bm{b} \in \mathcal{B}_{i}^{(n+1)}} \frac{1}{L_b}\lVert \bm{b}- C_i^{(n+1)}(\bm{b})\rVert^2_2 \le \frac{1}{|\mathcal{B}_{i}^{(n+1)}|}\sum_{\bm{b} \in \mathcal{B}_{i}^{(n+1)}} \frac{1}{L_b}\lVert \bm{b}- C_i^{(n)}(\bm{b})\rVert^2_2
\end{equation}

The above equation states that quantizing the given block cluster $\mathcal{B}_i$ after updating the associated codebook from $C_i^{(n)}$ to $C_i^{(n+1)}$ results in lower quantization MSE. Since this is true for all the block clusters, we derive the following result: 
\begin{equation}
\begin{split}
\label{eq:mse_ineq_2}
     J^{(n+1)} &= \frac{1}{N_c} \sum_{i=1}^{N_c} \frac{1}{|\mathcal{B}_{i}^{(n+1)}|}\sum_{\bm{b} \in \mathcal{B}_{i}^{(n+1)}} \frac{1}{L_b}\lVert \bm{b}- C_i^{(n+1)}(\bm{b})\rVert^2_2  \le \tilde{J}^{(n+1)}   
\end{split}
\end{equation}

Following (\ref{eq:mse_ineq_1}) and (\ref{eq:mse_ineq_2}), we find that the quantization MSE is non-increasing for each iteration, that is, $J^{(1)} \ge J^{(2)} \ge J^{(3)} \ge \ldots \ge J^{(M)}$ where $M$ is the maximum number of iterations. 
%Therefore, we can say that if the algorithm converges, then it must be that it has converged to a local minimum. 
\hfill $\blacksquare$


\begin{figure}
    \begin{center}
    \includegraphics[width=0.5\textwidth]{sections//figures/mse_vs_iter.pdf}
    \end{center}
    \caption{\small NMSE vs iterations during LO-BCQ compared to other block quantization proposals}
    \label{fig:nmse_vs_iter}
\end{figure}

Figure \ref{fig:nmse_vs_iter} shows the empirical convergence of LO-BCQ across several block lengths and number of codebooks. Also, the MSE achieved by LO-BCQ is compared to baselines such as MXFP and VSQ. As shown, LO-BCQ converges to a lower MSE than the baselines. Further, we achieve better convergence for larger number of codebooks ($N_c$) and for a smaller block length ($L_b$), both of which increase the bitwidth of BCQ (see Eq \ref{eq:bitwidth_bcq}).


\subsection{Additional Accuracy Results}
%Table \ref{tab:lobcq_config} lists the various LOBCQ configurations and their corresponding bitwidths.
\begin{table}
\setlength{\tabcolsep}{4.75pt}
\begin{center}
\caption{\label{tab:lobcq_config} Various LO-BCQ configurations and their bitwidths.}
\begin{tabular}{|c||c|c|c|c||c|c||c|} 
\hline
 & \multicolumn{4}{|c||}{$L_b=8$} & \multicolumn{2}{|c||}{$L_b=4$} & $L_b=2$ \\
 \hline
 \backslashbox{$L_A$\kern-1em}{\kern-1em$N_c$} & 2 & 4 & 8 & 16 & 2 & 4 & 2 \\
 \hline
 64 & 4.25 & 4.375 & 4.5 & 4.625 & 4.375 & 4.625 & 4.625\\
 \hline
 32 & 4.375 & 4.5 & 4.625& 4.75 & 4.5 & 4.75 & 4.75 \\
 \hline
 16 & 4.625 & 4.75& 4.875 & 5 & 4.75 & 5 & 5 \\
 \hline
\end{tabular}
\end{center}
\end{table}

%\subsection{Perplexity achieved by various LO-BCQ configurations on Wikitext-103 dataset}

\begin{table} \centering
\begin{tabular}{|c||c|c|c|c||c|c||c|} 
\hline
 $L_b \rightarrow$& \multicolumn{4}{c||}{8} & \multicolumn{2}{c||}{4} & 2\\
 \hline
 \backslashbox{$L_A$\kern-1em}{\kern-1em$N_c$} & 2 & 4 & 8 & 16 & 2 & 4 & 2  \\
 %$N_c \rightarrow$ & 2 & 4 & 8 & 16 & 2 & 4 & 2 \\
 \hline
 \hline
 \multicolumn{8}{c}{GPT3-1.3B (FP32 PPL = 9.98)} \\ 
 \hline
 \hline
 64 & 10.40 & 10.23 & 10.17 & 10.15 &  10.28 & 10.18 & 10.19 \\
 \hline
 32 & 10.25 & 10.20 & 10.15 & 10.12 &  10.23 & 10.17 & 10.17 \\
 \hline
 16 & 10.22 & 10.16 & 10.10 & 10.09 &  10.21 & 10.14 & 10.16 \\
 \hline
  \hline
 \multicolumn{8}{c}{GPT3-8B (FP32 PPL = 7.38)} \\ 
 \hline
 \hline
 64 & 7.61 & 7.52 & 7.48 &  7.47 &  7.55 &  7.49 & 7.50 \\
 \hline
 32 & 7.52 & 7.50 & 7.46 &  7.45 &  7.52 &  7.48 & 7.48  \\
 \hline
 16 & 7.51 & 7.48 & 7.44 &  7.44 &  7.51 &  7.49 & 7.47  \\
 \hline
\end{tabular}
\caption{\label{tab:ppl_gpt3_abalation} Wikitext-103 perplexity across GPT3-1.3B and 8B models.}
\end{table}

\begin{table} \centering
\begin{tabular}{|c||c|c|c|c||} 
\hline
 $L_b \rightarrow$& \multicolumn{4}{c||}{8}\\
 \hline
 \backslashbox{$L_A$\kern-1em}{\kern-1em$N_c$} & 2 & 4 & 8 & 16 \\
 %$N_c \rightarrow$ & 2 & 4 & 8 & 16 & 2 & 4 & 2 \\
 \hline
 \hline
 \multicolumn{5}{|c|}{Llama2-7B (FP32 PPL = 5.06)} \\ 
 \hline
 \hline
 64 & 5.31 & 5.26 & 5.19 & 5.18  \\
 \hline
 32 & 5.23 & 5.25 & 5.18 & 5.15  \\
 \hline
 16 & 5.23 & 5.19 & 5.16 & 5.14  \\
 \hline
 \multicolumn{5}{|c|}{Nemotron4-15B (FP32 PPL = 5.87)} \\ 
 \hline
 \hline
 64  & 6.3 & 6.20 & 6.13 & 6.08  \\
 \hline
 32  & 6.24 & 6.12 & 6.07 & 6.03  \\
 \hline
 16  & 6.12 & 6.14 & 6.04 & 6.02  \\
 \hline
 \multicolumn{5}{|c|}{Nemotron4-340B (FP32 PPL = 3.48)} \\ 
 \hline
 \hline
 64 & 3.67 & 3.62 & 3.60 & 3.59 \\
 \hline
 32 & 3.63 & 3.61 & 3.59 & 3.56 \\
 \hline
 16 & 3.61 & 3.58 & 3.57 & 3.55 \\
 \hline
\end{tabular}
\caption{\label{tab:ppl_llama7B_nemo15B} Wikitext-103 perplexity compared to FP32 baseline in Llama2-7B and Nemotron4-15B, 340B models}
\end{table}

%\subsection{Perplexity achieved by various LO-BCQ configurations on MMLU dataset}


\begin{table} \centering
\begin{tabular}{|c||c|c|c|c||c|c|c|c|} 
\hline
 $L_b \rightarrow$& \multicolumn{4}{c||}{8} & \multicolumn{4}{c||}{8}\\
 \hline
 \backslashbox{$L_A$\kern-1em}{\kern-1em$N_c$} & 2 & 4 & 8 & 16 & 2 & 4 & 8 & 16  \\
 %$N_c \rightarrow$ & 2 & 4 & 8 & 16 & 2 & 4 & 2 \\
 \hline
 \hline
 \multicolumn{5}{|c|}{Llama2-7B (FP32 Accuracy = 45.8\%)} & \multicolumn{4}{|c|}{Llama2-70B (FP32 Accuracy = 69.12\%)} \\ 
 \hline
 \hline
 64 & 43.9 & 43.4 & 43.9 & 44.9 & 68.07 & 68.27 & 68.17 & 68.75 \\
 \hline
 32 & 44.5 & 43.8 & 44.9 & 44.5 & 68.37 & 68.51 & 68.35 & 68.27  \\
 \hline
 16 & 43.9 & 42.7 & 44.9 & 45 & 68.12 & 68.77 & 68.31 & 68.59  \\
 \hline
 \hline
 \multicolumn{5}{|c|}{GPT3-22B (FP32 Accuracy = 38.75\%)} & \multicolumn{4}{|c|}{Nemotron4-15B (FP32 Accuracy = 64.3\%)} \\ 
 \hline
 \hline
 64 & 36.71 & 38.85 & 38.13 & 38.92 & 63.17 & 62.36 & 63.72 & 64.09 \\
 \hline
 32 & 37.95 & 38.69 & 39.45 & 38.34 & 64.05 & 62.30 & 63.8 & 64.33  \\
 \hline
 16 & 38.88 & 38.80 & 38.31 & 38.92 & 63.22 & 63.51 & 63.93 & 64.43  \\
 \hline
\end{tabular}
\caption{\label{tab:mmlu_abalation} Accuracy on MMLU dataset across GPT3-22B, Llama2-7B, 70B and Nemotron4-15B models.}
\end{table}


%\subsection{Perplexity achieved by various LO-BCQ configurations on LM evaluation harness}

\begin{table} \centering
\begin{tabular}{|c||c|c|c|c||c|c|c|c|} 
\hline
 $L_b \rightarrow$& \multicolumn{4}{c||}{8} & \multicolumn{4}{c||}{8}\\
 \hline
 \backslashbox{$L_A$\kern-1em}{\kern-1em$N_c$} & 2 & 4 & 8 & 16 & 2 & 4 & 8 & 16  \\
 %$N_c \rightarrow$ & 2 & 4 & 8 & 16 & 2 & 4 & 2 \\
 \hline
 \hline
 \multicolumn{5}{|c|}{Race (FP32 Accuracy = 37.51\%)} & \multicolumn{4}{|c|}{Boolq (FP32 Accuracy = 64.62\%)} \\ 
 \hline
 \hline
 64 & 36.94 & 37.13 & 36.27 & 37.13 & 63.73 & 62.26 & 63.49 & 63.36 \\
 \hline
 32 & 37.03 & 36.36 & 36.08 & 37.03 & 62.54 & 63.51 & 63.49 & 63.55  \\
 \hline
 16 & 37.03 & 37.03 & 36.46 & 37.03 & 61.1 & 63.79 & 63.58 & 63.33  \\
 \hline
 \hline
 \multicolumn{5}{|c|}{Winogrande (FP32 Accuracy = 58.01\%)} & \multicolumn{4}{|c|}{Piqa (FP32 Accuracy = 74.21\%)} \\ 
 \hline
 \hline
 64 & 58.17 & 57.22 & 57.85 & 58.33 & 73.01 & 73.07 & 73.07 & 72.80 \\
 \hline
 32 & 59.12 & 58.09 & 57.85 & 58.41 & 73.01 & 73.94 & 72.74 & 73.18  \\
 \hline
 16 & 57.93 & 58.88 & 57.93 & 58.56 & 73.94 & 72.80 & 73.01 & 73.94  \\
 \hline
\end{tabular}
\caption{\label{tab:mmlu_abalation} Accuracy on LM evaluation harness tasks on GPT3-1.3B model.}
\end{table}

\begin{table} \centering
\begin{tabular}{|c||c|c|c|c||c|c|c|c|} 
\hline
 $L_b \rightarrow$& \multicolumn{4}{c||}{8} & \multicolumn{4}{c||}{8}\\
 \hline
 \backslashbox{$L_A$\kern-1em}{\kern-1em$N_c$} & 2 & 4 & 8 & 16 & 2 & 4 & 8 & 16  \\
 %$N_c \rightarrow$ & 2 & 4 & 8 & 16 & 2 & 4 & 2 \\
 \hline
 \hline
 \multicolumn{5}{|c|}{Race (FP32 Accuracy = 41.34\%)} & \multicolumn{4}{|c|}{Boolq (FP32 Accuracy = 68.32\%)} \\ 
 \hline
 \hline
 64 & 40.48 & 40.10 & 39.43 & 39.90 & 69.20 & 68.41 & 69.45 & 68.56 \\
 \hline
 32 & 39.52 & 39.52 & 40.77 & 39.62 & 68.32 & 67.43 & 68.17 & 69.30  \\
 \hline
 16 & 39.81 & 39.71 & 39.90 & 40.38 & 68.10 & 66.33 & 69.51 & 69.42  \\
 \hline
 \hline
 \multicolumn{5}{|c|}{Winogrande (FP32 Accuracy = 67.88\%)} & \multicolumn{4}{|c|}{Piqa (FP32 Accuracy = 78.78\%)} \\ 
 \hline
 \hline
 64 & 66.85 & 66.61 & 67.72 & 67.88 & 77.31 & 77.42 & 77.75 & 77.64 \\
 \hline
 32 & 67.25 & 67.72 & 67.72 & 67.00 & 77.31 & 77.04 & 77.80 & 77.37  \\
 \hline
 16 & 68.11 & 68.90 & 67.88 & 67.48 & 77.37 & 78.13 & 78.13 & 77.69  \\
 \hline
\end{tabular}
\caption{\label{tab:mmlu_abalation} Accuracy on LM evaluation harness tasks on GPT3-8B model.}
\end{table}

\begin{table} \centering
\begin{tabular}{|c||c|c|c|c||c|c|c|c|} 
\hline
 $L_b \rightarrow$& \multicolumn{4}{c||}{8} & \multicolumn{4}{c||}{8}\\
 \hline
 \backslashbox{$L_A$\kern-1em}{\kern-1em$N_c$} & 2 & 4 & 8 & 16 & 2 & 4 & 8 & 16  \\
 %$N_c \rightarrow$ & 2 & 4 & 8 & 16 & 2 & 4 & 2 \\
 \hline
 \hline
 \multicolumn{5}{|c|}{Race (FP32 Accuracy = 40.67\%)} & \multicolumn{4}{|c|}{Boolq (FP32 Accuracy = 76.54\%)} \\ 
 \hline
 \hline
 64 & 40.48 & 40.10 & 39.43 & 39.90 & 75.41 & 75.11 & 77.09 & 75.66 \\
 \hline
 32 & 39.52 & 39.52 & 40.77 & 39.62 & 76.02 & 76.02 & 75.96 & 75.35  \\
 \hline
 16 & 39.81 & 39.71 & 39.90 & 40.38 & 75.05 & 73.82 & 75.72 & 76.09  \\
 \hline
 \hline
 \multicolumn{5}{|c|}{Winogrande (FP32 Accuracy = 70.64\%)} & \multicolumn{4}{|c|}{Piqa (FP32 Accuracy = 79.16\%)} \\ 
 \hline
 \hline
 64 & 69.14 & 70.17 & 70.17 & 70.56 & 78.24 & 79.00 & 78.62 & 78.73 \\
 \hline
 32 & 70.96 & 69.69 & 71.27 & 69.30 & 78.56 & 79.49 & 79.16 & 78.89  \\
 \hline
 16 & 71.03 & 69.53 & 69.69 & 70.40 & 78.13 & 79.16 & 79.00 & 79.00  \\
 \hline
\end{tabular}
\caption{\label{tab:mmlu_abalation} Accuracy on LM evaluation harness tasks on GPT3-22B model.}
\end{table}

\begin{table} \centering
\begin{tabular}{|c||c|c|c|c||c|c|c|c|} 
\hline
 $L_b \rightarrow$& \multicolumn{4}{c||}{8} & \multicolumn{4}{c||}{8}\\
 \hline
 \backslashbox{$L_A$\kern-1em}{\kern-1em$N_c$} & 2 & 4 & 8 & 16 & 2 & 4 & 8 & 16  \\
 %$N_c \rightarrow$ & 2 & 4 & 8 & 16 & 2 & 4 & 2 \\
 \hline
 \hline
 \multicolumn{5}{|c|}{Race (FP32 Accuracy = 44.4\%)} & \multicolumn{4}{|c|}{Boolq (FP32 Accuracy = 79.29\%)} \\ 
 \hline
 \hline
 64 & 42.49 & 42.51 & 42.58 & 43.45 & 77.58 & 77.37 & 77.43 & 78.1 \\
 \hline
 32 & 43.35 & 42.49 & 43.64 & 43.73 & 77.86 & 75.32 & 77.28 & 77.86  \\
 \hline
 16 & 44.21 & 44.21 & 43.64 & 42.97 & 78.65 & 77 & 76.94 & 77.98  \\
 \hline
 \hline
 \multicolumn{5}{|c|}{Winogrande (FP32 Accuracy = 69.38\%)} & \multicolumn{4}{|c|}{Piqa (FP32 Accuracy = 78.07\%)} \\ 
 \hline
 \hline
 64 & 68.9 & 68.43 & 69.77 & 68.19 & 77.09 & 76.82 & 77.09 & 77.86 \\
 \hline
 32 & 69.38 & 68.51 & 68.82 & 68.90 & 78.07 & 76.71 & 78.07 & 77.86  \\
 \hline
 16 & 69.53 & 67.09 & 69.38 & 68.90 & 77.37 & 77.8 & 77.91 & 77.69  \\
 \hline
\end{tabular}
\caption{\label{tab:mmlu_abalation} Accuracy on LM evaluation harness tasks on Llama2-7B model.}
\end{table}

\begin{table} \centering
\begin{tabular}{|c||c|c|c|c||c|c|c|c|} 
\hline
 $L_b \rightarrow$& \multicolumn{4}{c||}{8} & \multicolumn{4}{c||}{8}\\
 \hline
 \backslashbox{$L_A$\kern-1em}{\kern-1em$N_c$} & 2 & 4 & 8 & 16 & 2 & 4 & 8 & 16  \\
 %$N_c \rightarrow$ & 2 & 4 & 8 & 16 & 2 & 4 & 2 \\
 \hline
 \hline
 \multicolumn{5}{|c|}{Race (FP32 Accuracy = 48.8\%)} & \multicolumn{4}{|c|}{Boolq (FP32 Accuracy = 85.23\%)} \\ 
 \hline
 \hline
 64 & 49.00 & 49.00 & 49.28 & 48.71 & 82.82 & 84.28 & 84.03 & 84.25 \\
 \hline
 32 & 49.57 & 48.52 & 48.33 & 49.28 & 83.85 & 84.46 & 84.31 & 84.93  \\
 \hline
 16 & 49.85 & 49.09 & 49.28 & 48.99 & 85.11 & 84.46 & 84.61 & 83.94  \\
 \hline
 \hline
 \multicolumn{5}{|c|}{Winogrande (FP32 Accuracy = 79.95\%)} & \multicolumn{4}{|c|}{Piqa (FP32 Accuracy = 81.56\%)} \\ 
 \hline
 \hline
 64 & 78.77 & 78.45 & 78.37 & 79.16 & 81.45 & 80.69 & 81.45 & 81.5 \\
 \hline
 32 & 78.45 & 79.01 & 78.69 & 80.66 & 81.56 & 80.58 & 81.18 & 81.34  \\
 \hline
 16 & 79.95 & 79.56 & 79.79 & 79.72 & 81.28 & 81.66 & 81.28 & 80.96  \\
 \hline
\end{tabular}
\caption{\label{tab:mmlu_abalation} Accuracy on LM evaluation harness tasks on Llama2-70B model.}
\end{table}

%\section{MSE Studies}
%\textcolor{red}{TODO}


\subsection{Number Formats and Quantization Method}
\label{subsec:numFormats_quantMethod}
\subsubsection{Integer Format}
An $n$-bit signed integer (INT) is typically represented with a 2s-complement format \citep{yao2022zeroquant,xiao2023smoothquant,dai2021vsq}, where the most significant bit denotes the sign.

\subsubsection{Floating Point Format}
An $n$-bit signed floating point (FP) number $x$ comprises of a 1-bit sign ($x_{\mathrm{sign}}$), $B_m$-bit mantissa ($x_{\mathrm{mant}}$) and $B_e$-bit exponent ($x_{\mathrm{exp}}$) such that $B_m+B_e=n-1$. The associated constant exponent bias ($E_{\mathrm{bias}}$) is computed as $(2^{{B_e}-1}-1)$. We denote this format as $E_{B_e}M_{B_m}$.  

\subsubsection{Quantization Scheme}
\label{subsec:quant_method}
A quantization scheme dictates how a given unquantized tensor is converted to its quantized representation. We consider FP formats for the purpose of illustration. Given an unquantized tensor $\bm{X}$ and an FP format $E_{B_e}M_{B_m}$, we first, we compute the quantization scale factor $s_X$ that maps the maximum absolute value of $\bm{X}$ to the maximum quantization level of the $E_{B_e}M_{B_m}$ format as follows:
\begin{align}
\label{eq:sf}
    s_X = \frac{\mathrm{max}(|\bm{X}|)}{\mathrm{max}(E_{B_e}M_{B_m})}
\end{align}
In the above equation, $|\cdot|$ denotes the absolute value function.

Next, we scale $\bm{X}$ by $s_X$ and quantize it to $\hat{\bm{X}}$ by rounding it to the nearest quantization level of $E_{B_e}M_{B_m}$ as:

\begin{align}
\label{eq:tensor_quant}
    \hat{\bm{X}} = \text{round-to-nearest}\left(\frac{\bm{X}}{s_X}, E_{B_e}M_{B_m}\right)
\end{align}

We perform dynamic max-scaled quantization \citep{wu2020integer}, where the scale factor $s$ for activations is dynamically computed during runtime.

\subsection{Vector Scaled Quantization}
\begin{wrapfigure}{r}{0.35\linewidth}
  \centering
  \includegraphics[width=\linewidth]{sections/figures/vsquant.jpg}
  \caption{\small Vectorwise decomposition for per-vector scaled quantization (VSQ \citep{dai2021vsq}).}
  \label{fig:vsquant}
\end{wrapfigure}
During VSQ \citep{dai2021vsq}, the operand tensors are decomposed into 1D vectors in a hardware friendly manner as shown in Figure \ref{fig:vsquant}. Since the decomposed tensors are used as operands in matrix multiplications during inference, it is beneficial to perform this decomposition along the reduction dimension of the multiplication. The vectorwise quantization is performed similar to tensorwise quantization described in Equations \ref{eq:sf} and \ref{eq:tensor_quant}, where a scale factor $s_v$ is required for each vector $\bm{v}$ that maps the maximum absolute value of that vector to the maximum quantization level. While smaller vector lengths can lead to larger accuracy gains, the associated memory and computational overheads due to the per-vector scale factors increases. To alleviate these overheads, VSQ \citep{dai2021vsq} proposed a second level quantization of the per-vector scale factors to unsigned integers, while MX \citep{rouhani2023shared} quantizes them to integer powers of 2 (denoted as $2^{INT}$).

\subsubsection{MX Format}
The MX format proposed in \citep{rouhani2023microscaling} introduces the concept of sub-block shifting. For every two scalar elements of $b$-bits each, there is a shared exponent bit. The value of this exponent bit is determined through an empirical analysis that targets minimizing quantization MSE. We note that the FP format $E_{1}M_{b}$ is strictly better than MX from an accuracy perspective since it allocates a dedicated exponent bit to each scalar as opposed to sharing it across two scalars. Therefore, we conservatively bound the accuracy of a $b+2$-bit signed MX format with that of a $E_{1}M_{b}$ format in our comparisons. For instance, we use E1M2 format as a proxy for MX4.

\begin{figure}
    \centering
    \includegraphics[width=1\linewidth]{sections//figures/BlockFormats.pdf}
    \caption{\small Comparing LO-BCQ to MX format.}
    \label{fig:block_formats}
\end{figure}

Figure \ref{fig:block_formats} compares our $4$-bit LO-BCQ block format to MX \citep{rouhani2023microscaling}. As shown, both LO-BCQ and MX decompose a given operand tensor into block arrays and each block array into blocks. Similar to MX, we find that per-block quantization ($L_b < L_A$) leads to better accuracy due to increased flexibility. While MX achieves this through per-block $1$-bit micro-scales, we associate a dedicated codebook to each block through a per-block codebook selector. Further, MX quantizes the per-block array scale-factor to E8M0 format without per-tensor scaling. In contrast during LO-BCQ, we find that per-tensor scaling combined with quantization of per-block array scale-factor to E4M3 format results in superior inference accuracy across models. 


\clearpage

% \ACKNOWLEDGMENT{%
% % Enter the text of acknowledgments here
% }% Leave this (end of acknowledgment)

 
% \section{DRAFT}
% % This is samplepaper.tex, a sample chapter demonstrating the
% LLNCS macro package for Springer Computer Science proceedings;
% Version 2.21 of 2022/01/12
%
\documentclass[runningheads]{llncs}
%
\bibliographystyle{splncs04}
\usepackage[T1]{fontenc}
% T1 fonts will be used to generate the final print and online PDFs,
% so please use T1 fonts in your manuscript whenever possible.
% Other font encondings may result in incorrect characters.
%
\usepackage{graphicx}
% Used for displaying a sample figure. If possible, figure files should
% be included in EPS format.
%
% If you use the hyperref package, please uncomment the following two lines
% to display URLs in blue roman font according to Springer's eBook style:
%\usepackage{color}
%\renewcommand\UrlFont{\color{blue}\rmfamily}
%\urlstyle{rm}
%
\usepackage{url}
\begin{document}
%
\title{Modeling the Training of Human and GPT-4 Social Engineering Attack Recognition}
%\title{Modeling and Evaluating Human and GPT-4 Social Engineering Attack Detection}
%Using Cognitive Models to Improve Training Against Human and GPT-4 Generated Social Engineering Attacks}
% Modeling the Training of Human and GPT-4 Social Engineering Attack Recognition
\titlerunning{Human and GPT-4 Social Engineering Attacks}
%
%\titlerunning{Abbreviated paper title}
% If the paper title is too long for the running head, you can set
% an abbreviated paper title here
%
\author{Tyler Malloy \and
Maria José Ferreira \and 
Fei Fang \and
Cleotilde Gonzalez}
%
\authorrunning{Malloy et al.}
\institute{Carnegie Mellon University, Pittsburgh PA 15222, USA}
%
\maketitle

\begin{abstract}
    Social engineering attacks remain a critical tool for cybercriminals seeking to exploit sensitive data. Although the threat of AI-generated content in such attacks is growing, current training methods predominantly rely on simplistic human-designed emails. This research introduces a novel experimental paradigm to investigate differences in the detection of human-generated versus AI-generated phishing emails. Our behavioral results reveal that emails co-created by humans and Generative-AI models pose a greater challenge to end users compared to those emails created by GPT-4 or Human only. We also propose a cognitive model that predicts user behavior during training, which offers the potential to be used in future training frameworks to improve training effectiveness. Our work contributes by (1) identifying critical weaknesses in current social engineering training and (2) proposing a cognitive model-driven solution to better equip users against evolving threats.
\end{abstract}

\section{Introduction}
Social engineering attacks are commonly used by cyber criminals to gain access to valuable and sensitive data. Recent Large Language Models (LLMs) such as GPT-4 have demonstrated the ability to produce convincing text that mimics human writing, and code that could be used to create fake emails and websites that appear to be legitimate. Research in cybersecurity has identified the risks of increased proliferation of social engineering attacks through the use of LLMs \cite{schmitt2024digital}. However, the efficacy of LLM-generated emails in training users against social engineering attacks has not been evaluated. Many training programs are based on simple human-designed emails in classroom-style instruction delivery \cite{wen2019hack}. In this work, we propose the use of GPT-4 to write convincing text that mimics real emails, as well as HTML and CSS code to stylize emails. To our knowledge, this is the first study designed to establish the efficacy of GPT-4-generated emails compared to those written by humans. We also evaluate the efficacy of emails that are co-created by humans and styled by GPT-4. 

Our research introduces an experimental paradigm to determine whether there is a difference in end user detection using human-written and GPT-4 generated emails. This was done in a two-by-two design that varied the original author of the email text (Human or GPT-4) as well as the style of the email (Plain-text or HTML/CSS). A pre-experiment quiz on the indicators of phishing emails served as a measure of the base phishing knowledge of participants, and a post-experiment questionnaire had participants indicate what proportion of the content they observed was generated by AI. Participants observed exclusively human-written or GPT-4 generated emails.

The results of the experiment show that emails written by humans and stylized using HTML/CSS code generated by GPT-4 are the most challenging for end users, with a significant interaction effect leading to the GPT-4 written and HTML/CSS stylized emails being the easiest for participants to categorize. Analysis of the performance of participants based on their perception of content as AI-written demonstrates a significant bias by which participants rate more emails as phishing if they believe a higher proportion of emails were generated by AI. This effect represents a novel \textit{AI-writing bias} that leads participants to assume that AI-written emails are phishing attempts. This bias is closely related to the well-studied phenomenon of algorithm aversion. Participants who had less initial knowledge of phishing emails performed worse on average under all experiment conditions compared to participants who performed better on the initial phishing quiz. These two groups, participants who have less initial knowledge about phishing and those who perceive all AI-written content as being more likely to be phishing, could improve their performance through a better method of selecting emails to show to participants.

Alongside this experiment, we propose an Instance-Based Learning (IBL) cognitive model that uses GPT-4 embeddings of emails as attributes to predict the user's behavior in the email categorization task. 
%This cognitive model can be used to predict the categorization of the emails of the participants and determine the optimal email to show to that participant. 
%This is done by iterating over all possible emails that could be shown to a participant, and selecting the email that has the highest probability of incorrect categorization by that participant at that time in the training. This is inspired by the intuition that more difficult emails will expand participant's ability to categorize a wider variety of emails. 

We demonstrate that the IBL model is capable of accurately predicting the user's classification. We also run a simulation study to demonstrate how the model could be used to predict the categorization of a user and, by this prediction, select an optimal email to show to that participant to optimize their training.

%This is done by iterating over all possible emails that could be shown to a participant, and selecting the email that has the highest probability of incorrect categorization by that participant at that time in the training. This is inspired by the intuition that more difficult emails will expand participant's ability to categorize a wider variety of emails. 
%results that predict the potential improvement in participant training outcomes through this email selection method.  

\section{Background}
Generative Artificial Intelligence (GAI) has the potential to improve education and training in a variety of settings through increased accessibility and reduced costs (for a review, see \cite{baldassarre2023social}. However, there are significant ethical concerns due to the potential negative societal impacts of these models being misused \cite{bommasani2021opportunities}, such as through the generation of social engineering attacks \cite{al2023chatgpt}. One commonly used and widely available class of GAI are pre-trained Large Language Models (LLMs) that can be prompted to produce highly convincing textual outputs that resemble human writing \cite{sejnowski2023large}. While these methods are trained to avoid producing potentially harmful content, they can be repeatedly prompted when changing the initial prompt or continuing with different prompts, in an effort to produce desired outputs \cite{white2023prompt}. The design of the prompts that are input into LLMs to produce text is call \textit{prompt engineering}, and can be used to improve the quality of the LLM output \cite{chen2023unleashing}. The repeated prompting of LLMs has been applied onto predicting how humans may speed up learning through the use of natural language instructions that can be used to inform the predicted value of actions without needing experience of performing those actions in a specific environment state \cite{mcdonald2023exploring}.

LLMs such as the Generative Pretrained Transformer 3 (GPT-3) \cite{brown2020language} have been evaluated in their social engineering ability and have shown lower performance in designing social engineering attacks compared to humans \cite{sharma2023well}. The ability of these models is constantly evolving, putting into question the ability of newer models to design social engineering attacks \cite{kumar2023certifying}. While more advanced models may be able to produce more human-like text, they also have more advanced methods to prevent misuse. This work seeks to evaluate the newer GPT-4 model \cite{achiam2023gpt} in its ability to design phishing emails, as well as to compare the effectiveness of social engineering attacks designed by humans and LLM alone and emails generated by different combinations of the output of the human and LLM model. 

This work introduces an experimental paradigm for evaluating the potential harm of LLM use in one specific area, social engineering attacks. This experimental paradigm is used to compare social engineering attacks in the form of phishing emails that are either fully written by human cybersecurity experts, fully written by GPT-4, or a combination of the two through prompt engineering. Alongside this experiment, we propose a method to mitigate the potential misuse of LLMs in cybersecurity contexts by improving training against social engineering attacks. This is done by using a cognitive model to trace and predict individual learning progress and determine the best educational examples to show to participants. 

Overall, the contributions of this work are, first, the outline of some limitations to current social engineering training methods and, second, the identification of a potential solution to these limitations through the use of a cognitive model to improve learning outcomes. A novel bias is presented, in which participants assumed that AI-written emails are more likely to be phishing, leading to worse categorization performance. We show through simulation that selecting educational example emails using an IBL cognitive model reduces the effect of the AI-writing bias we demonstrate. These results show the usefulness of cognitive models in predicting the learning progress of end users in training scenarios, and the difficulty of correctly identifying phishing emails that are written by humans and then stylized by GPT-4.

\subsection{Large Language Models and Social Engineering Attacks}
The use of LLMs in the production of social engineering attacks demonstrates a significant concern for cybersecurity \cite{gupta2023chatgpt}. The simplicity of Generative AI tools makes them easy to apply to tasks such as writing phishing emails from scratch or stylizing existing phishing emails to look more convincing, potentially increasing their effectiveness \cite{sharma2023well}. Modern LLMs are even capable of producing code \cite{khan2022automatic}, such as Javascript, HTML, and CSS, \cite{lajko2022towards} that can create highly convincing emails that resemble real emails sent from many companies \cite{park2024ai}. This adds an additional layer to the potential misuse of LLMs in social engineering attacks, as hand-writing code for realistic looking emails would normally take minutes or hours, and can be done in seconds with LLMs. These two areas, writing original phishing emails and stylizing emails with HTML and CSS code, are the main focus of our experiment to investigate how users may be susceptible to social engineering attacks from humans and LLMs. 

One method of reducing the potential harm of LLMs is through the use of specific training that can make LLMs less likely to produce harmful content \cite{cao2023defending}. This is typically done using feedback from humans, either machine learning engineers or crowd-sourced participants in user studies \cite{bai2022training}. This can train models to avoid producing content that is designed to trick or scam users, such as phishing emails. However, the effectiveness of these methods in preventing the generation of dangerous content forms is not perfect and can often be worked around with more complex prompt engineering \cite{fredrikson2015model}. More advanced prompting can also train a separate model to adjust the prompt until it is accepted by the LLM and the desired content is produced \cite{zou2023universal}. In this work, we focus on using relatively simple prompt engineering to faithfully replicate what we view as a realistic scenario of a cyber attacker applying an LLM to write a phishing email. 

\subsection{Social Engineering Training}
Training end users to identify social engineering attacks is an important part of cybersecurity \cite{back2021cyber}. Users without experience in security are vulnerable, making them the `weakest link' of cyber defense \cite{vishwanath2022weakest}. Phishing emails are an especially common method of social engineering due to the high volume of emails sent daily and the potential for compromising systems provided by redirecting users to unintended websites, among other methods \cite{gupta2016literature}. Typically, training users to identify phishing emails focuses on specific features of these emails that can indicate that they are phishing attempts, such as the use of urgent language; making requests of confidential information; making an offer; containing a link to a dangerous website; among other features \cite{kumaraguru2009school}. In the past, this has been done using plain text emails written by human cybersecurity experts \cite{weaver2021training}. These training paradigms are a large industry and are commonly required by individuals, universities, companies, and other groups that are interested in improving the ability of end users to identify phishing emails \cite{jampen2020don}. 

\begin{figure}[t!] 
\begin{centering}
  \includegraphics[width=\textwidth]{Figures/Trial.png} 
  \caption{An example of the email identification task shown to participants}\label{fig:Trial}
 \end{centering} 
\end{figure}

Given the ever-updated nature of phishing attempts and the ease of use of LLMs in creating social engineering attacks, it is important to understand how users make decisions and learn from examples of emails written or stylized by LLMs. The intelligence selection of training examples shown to students has been shown to improve their learning outcomes \cite{ferguson2006improving}. This can be done by applying cognitive modeling methods to predict participant learning and decision making \cite{feng2011student}. In this work, these cognitive models are adjusted to reflect human behavior and serve as a baseline that can test various methods to improve end-user training on the identification of phishing emails. 

\subsection{Cognitive Modeling}
Cognitive models have previously been applied to predict human learning in anti-phishing training \cite{singh2023cognitive}. Recently, Generative AI models have been integrated with cognitive models by forming \textit{representations}, of stimuli, such as textual information using LLM embeddings \cite{malloy2024applying}, \cite{malloy2024leveraging}. This approach has demonstrated human-like abilities to recognize new stimuli, even when they are informationally complex, based on past experiences \cite{malloy2024efficient}. We propose the use of LLM embeddings as attributes of a cognitive model to both predict student learning and evaluate them under different experimental conditions. These same models are also used to simulate possible improvements in phishing education that can be afforded by intelligently selecting email examples. 

An Instance-Based Learning (IBL) model is used to both predict human learning in each condition of our experiment, and simulate the potential improvement of human learning afforded by an intelligence selection of example emails. Using LLMs to form representations of emails allows us to use the same representation method in experimental conditions. Comparing the accuracy of the IBL model in predicting human behavior across conditions allows us to assess how effectively it can be used to predict general human behavior. Additionally, we perform a simulation of these IBL models that fit human learning and decision making that allows us to evaluate methods of improving user learning in the identification of phishing emails. These simulation results provide evidence for our proposed method of improving cognitive-based training to make participant learning outcomes as efficient and effective as possible. 

\subsection{Instance Based Learning}
IBL models work by storing instances $i$ in memory $\mathcal{M}$, composed of utility outcomes $u_i$ and options $k$ composed of features $j$ in the set of features $\mathcal{F}$ of environmental decision alternatives. In the case of predicting student learning from phishing emails, these options include labeling an email as being either dangerous (phishing) or benign (ham), the features correspond to the attributes of the email that are relevant for determining if it is a phishing email, in our model the LLM embeddings, and the outcome corresponds to the point feedback provided to students depending on whether they are correct (1 point) or incorrect (-1 points). These options are observed in an order represented by the time step $t$, and the time step in which an instance occurred is given $\mathcal{T}(i)$. When tracing human participant performance, the memory is composed of the options presented to participants, the options that they selected, and the utility reward that was presented to them. 

To model the retrieval of instances in memory when calculating the expected value of different option alternatives, IBL models calculate the activation of each instance in memory based on the current options available. In calculating this activation, the similarity between instances in memory and the current instance is represented by adding the value $S_{ij}$ over all attributes, which is the similarity of the attribute $j$ of instance $i$ to the current state. This gives the activation equation as: 
 
\begin{equation}
A_i(t) = \ln \Bigg( \sum_{t' \in \mathcal{T}_i(t)} (t - t')^{-d}\Bigg) + \mu \sum_{j \in \mathcal{F}} \omega_j (S_{ij} - 1) + \sigma \xi
\label{eq:activation}
\end{equation}
The parameters of the IBL model can either be fit to individual human performance, or set to their default values. These parameters are the decay parameter $d$; the mismatch penalty $\mu$; the attribute weight of each $j$ feature $\omega_j$; and the noise parameter $\sigma$. The default values for these parameters are $(d,\mu,\omega_j,\sigma) = (0.5, 1, 1, 0.25)$. The IBL models in this work use default values to predict individual student behaviors. The value $\xi$ is drawn from a normal distribution $\mathcal{N}(-1,1)$ and multiplied by the noise parameter $\sigma$ to add random noise to the activation. Varying these parameters impacts which instances are retrieved, and ultimately how the predicted utility of option alternatives is calculated.  

When predicting human learning and decision making based on textual information such as phishing emails, it is possible to use LLMs to form embeddings of these emails as attributes of the IBL model \cite{malloy2024applying}. To calculate the similarity metric $S_{ij}$ between two emails, we use the cosine similarity of their embeddings, as is done in \cite{malloy2024leveraging}. In this work, this has the benefit that the same method of forming attributes from emails can be used across experimental conditions. Thus, we can assess the effectiveness of an IBL+LLM cognitive model in predicting human learning and decision making during training. 

The blended value of an option $k$ is calculated at time step $t$ according to the utility outcomes $u_i$ weighted by the probability of retrieval of that instance $P_i$ and summing over all instances in memory $\mathcal{M}_k$ to give the equation:
\begin{equation}
V_k(t) = \sum_{i \in \mathcal{M}_k} P_i(t)u_i
\label{eq:blending}
\end{equation}

Where $P_i(t)$ is the probability of retrieval, calculated by an inverse-temperature weighted soft-max of all available instance activations. 

\begin{figure}[t!] 
\begin{centering}
  \includegraphics[width=0.7\textwidth]{Figures/Emails.png} 
  \caption{Top-Left: The original plain-text email written by human experts Bottom-Left: The GPT-4 stylized version of this original email. Bottom-Right: The fully GPT-4 rewritten and stylized version of the email. Top-Right: The stripped plain-text version of the fully GPT-4 rewritten email.}\label{fig:Emails}
 \end{centering} 
\end{figure}

\section{Experiment}
The recent proliferation of phishing emails written or styled by large language models (LLMs) brings into question our understanding of how users make judgments of phishing emails and how these judgments compare between human and LLM written content. These LLM written emails can either be fully authored by humans, by LLMs, or a combination of the two where a human creates one of either the text body or styling, and the LLM creates the other. To test these different options of generating emails, we use a 2x2 design varying author (Human or GPT-4) or style (plain-text or GPT-4 stylized). We designed an experiment to collect human judgments of phishing (dangerous) and ham (safe) emails and varied the author (Human or GPT-4) and style (Plain-text or Styled) in a between-subjects 2x2 design. 

An example of the experimental interface used to evaluate the identification training of phishing emails is shown in Figure \ref{fig:Trial}. In this example, the email being shown is a human-written and plain-text styled email. Importantly, for each experimental condition, the same set of 360 emails was used, all based on the original dataset of plain-text emails written by human cybersecurity experts that was used in a previous study \cite{singh2023cognitive}. These base emails were then either stylized by GPT-4, or rewritten entirely by prompting GPT-4 to write an email with the same attributes that the experts coded the original emails as having. The fully GPT-4 rewritten email is also stripped of HTML and CSS code and presented as the plain-text version of the GPT-4 written email. This resulted in 4 sets of 360 emails with the same general features and topics in each set. Figure \ref{fig:Emails} shows the same email that is stylized, fully rewritten, and the plain-text version of that email. 

\subsection{Methods}
This experiment compares human learning and decision making when categorizing emails as phishing (dangerous) or ham (safe) depending on the email author (Human or GPT-4) and style (plain-text or GPT-4 stylized). We are interested in determining which condition is the most difficult for humans to make accurate judgments in and whether there is a relationship between participant confidence, reaction time, and accuracy. This is an important potential relationship as it can aid in our overall goal of improving the quality of example emails shown to participants based on their performance.

\begin{figure}[t!] 
\begin{centering}
  \includegraphics[width=0.7\textwidth]{Figures/BarPerformance.png} 
  \caption{Pre and post-training categorization accuracy for ham and phishing emails by experimental condition.}\label{fig:BarPerformance}
 \end{centering} 
\end{figure}

This experiment included 10 pre-training trials without feedback, 40 training trials with feedback, and 10 post-training trials without feedback. During all trials, participants made judgments about emails as phishing or ham and indicated their confidence in their judgment. We recruited 268 participants online through the Amazon Mechanical Turk (AMT) platform. Of these participants, 44 did not complete all 60 trials and were excluded from further analysis. Of the remaining 224 participants, 18 were removed due to poor performance in the categorization task, as predefined in the study preregistration. This predefined criterion removed all participants who performed less than two standard deviations below the mean categorization improvement between pre-training and post-training trials. 

This exclusion resulted in a total of 207 participants used for the following analysis. Participants (69 Female, 137 Male, 1 Non-binary) had an average age of 40.02 with a standard deviation of 10.48 years. Of these participants, 25 had never received a phishing email, 101 had received phishing emails on a few occasions, and 79 had received phishing emails on many occasions.  Participants were compensated with a base payment of \$3 with the potential to earn up to a \$12 bonus payment depending on performance. This experiment was approved by the Carnegie Mellon University Institutional Review Board, and the study was pre-registered on OSF\footnote{\url{https://osf.io/wbg3r/}}. All participant data and analysis code is available on OSF. 

\subsection{Results}
The primary comparison between conditions is done in terms of the improvement in categorization accuracy percentage between the 10 pre-training trials and the 10 post-training trials. These results are shown in Figure \ref{fig:BarPerformance}, with the pre-training performance lightly shaded and the post-training performance a darker shade. The only decrease in performance between pre and post-training was in the Human written and GPT-4 styled ham email categorization. 

A mixed repeated measure analysis of variance of the effect of the author of the email and the style of the email on the improvement of categorization demonstrated no significant variation in author ($F=1.101,p=0.295,\eta_p^2=0.005$) but a significant variation of style ($F=12.261$, $p=0.001$, $\eta_p^2=0.057$) as well as a significant interaction between author and style ($F=14.344$, $p<0.001$, $\eta_p^2=0.066$). A post-hoc multi-comparison Tukey test showed that the improvement of the human subject in the human written and GPT-4-styled condition had a significantly lower improvement from the prior training to the post-training categorization accuracy ($p=0.033$) when compared to the GPT-4-written and GPT-4-styled condition. All other comparisons between conditions did not show a significant difference in the effect. This indicates that the smallest improvement in participant categorization accuracy was the Human written and GPT-4 styled condition ($\mu=0.015$) while the largest improvement was in the GPT-4 written and styled condition ($\mu=0.104$).

These results demonstrate the difficulty of training participants to identify emails that were written by human cybersecurity experts and stylized by GPT-4. Interestingly, the highest accuracy for the detection of phishing emails after training was observed with the written and styled by GPT-4. This is potentially due to the safety methods built into the GPT-4 model which could have hindered the model's ability to write convincing phishing emails. Alternative approaches to the GPT-4 model prompting, such as prompt attack, could produce more convincing phishing emails, though these complex methods may be outside of the skill set of most cybersecurity attackers.    

The results of this analysis indicate that GPT-4 stylized human-written phishing emails present the most challenging learning and decision-making paradigm. There was a strong interaction effect between the author of the email and the style, whereby the author was less relevant in plain-text emails, but became significant in stylized emails. This is crucial to our understanding of phishing email training, since many existing platforms still use plain-text emails in training examples.

\subsection{Participant AI Identification}
\begin{figure}[t!] 
\begin{centering}
  \includegraphics[width=\textwidth]{Figures/AIBias.png} 
  \caption{Linear regression comparing the percentage of emails categorized as being phishing emails and the proportion of emails identified as being AI written. Regressions are split between each of the four experimental conditions. Shaded regions represent 95\% confidence intervals of linear regression with $R^2$ and slope labeled.}\label{fig:PerceptionCondition}
 \end{centering} 
\end{figure}
To capture human participant identification of how emails were created, they were asked four questions at the end of the experiment to estimate the number of emails that they saw that were AI generated. The next comparison we performed was to assess the overall probability of categorizing an email as phishing based on how likely a participant was to categorize an email as being phishing based on their identification of emails as AI-generated or created by humans.  

These results are shown in Figure \ref{fig:PerceptionCondition} which shows a regression of the average percent of emails classified as phishing, since half of all emails shown to the participants were phishing, a correct categorization of all emails would result in 50\% emails being classified as phishing. In general, the participants tended to categorize more than half of the emails they were shown as phishing emails. Additionally, there was an overall trend across each condition that the higher the proportion of emails identified as AI written, the higher the probability of categorizing any email as being phishing.

It may seem surprising that the increased perception of emails as written by an AI model would lead to this bias in categorizing emails as being phishing. However, people generally demonstrate a poor ability to detect AI-written content \cite{kobis2021artificial}, which could interact with general aversion to algorithms \cite{burton2020systematic}) which has been shown to be higher in people who have experience with algorithms making incorrect judgments \cite{dietvorst2015algorithm}. 

We can see from this regression that participants who identified emails as being AI written in both of the GPT-4 styled conditions were more likely to categorize emails as being phishing if they had a higher identification of emails as being AI written. This represents an important bias in participant identification of emails that could potentially be exploited by cybersecurity attackers. This further motivates the improvement of training for detecting social engineering attacks that are designed by both humans and LLMs. 

A comparison of the slopes of these regressions in Figure \ref{fig:PerceptionCondition} demonstrates that this effect of phishing categorization bias is not equal across conditions. Notably, the likelihood of categorizing emails as being phishing has both a higher slope and a higher $R^2$ for emails that were styled by GPT-4. Looking back to the four example emails shown in Figure \ref{fig:Emails}, we can see that both of the GPT-4 styled conditions include banners, logos, bold text and other styled text that may draw the attention of participants. It is likely that participants were attending to these more salient features in the GPT-4 styled conditions, which if perceived as being AI generated could bias participants into believing that emails are phishing. 

These comparisons demonstrate that there is a difference between experimental conditions in how identifying emails as being AI written impacts the likelihood of categorizing emails as being phishing. This has important implications for both understanding how participants make judgments of emails in different contexts, as well as how best to design training when incorporating LLMs into the design of example emails. It is important that participants not over attend to irrelevant features like the perception of content as being AI written, and focus on relevant features like the presence of offers or incorrect sender addresses. 

\begin{figure}[!t] 
\begin{centering}
\includegraphics[width=\textwidth]{Figures/Simulations.png} 
  \caption{All improvement measures refer to the percentage point difference between pre-training and post-training accuracy. Left: Human participants (pastel colors) compared to IBL model training (bright colors) improvement under randomized email selection. Right: Simulated IBL student model improvement under IBL teacher model email selection (dark colors) compared to IBL model training under randomized email selection (bright colors). Color indicates condition, shade indicates training method, error bars indicate standard deviation.}\label{fig:training}
 \end{centering} 
\end{figure}

\subsection{Cognitive Modeling}
Before describing our proposed method for improving phishing training against human and LLM attackers, we must determine the appropriate method of modeling human behavior in this task. The emails used in each condition have the same base email, a plain-text and human-written email that had hand-crafted attributes associated with it. An alternative to using these hand-crafted features is to use LLM embedding representations of emails. However, the complex nature of these embedding representations means they may be difficult to use in a cognitive model that seeks to reflect the realities of human cognition. 

To assess these two different approaches in their ability to model and predict human-like learning in this task, we compared an IBL model that used hand-crafted attributes with one that used LLM model embeddings (IBL+LLM). This was done using a model-tracing approach for each individual participant, which works by setting the IBL model to select the same choice made by an individual participant, and observing the same utility outcome from that choice that the participant observed. This allows us to compare the IBL and the performance of human participants with the same experience.  

\subsection{Proposed Phishing Training supported by IBL}
Our proposed method to improve the learning outcomes of phishing training is based on the use of an IBL model to perform model training during the experiment and select emails to show to participants based on that model. Specifically, this model will be trained on all trials of an experiment based on the emails shown to a participant. During the pre-training and post-training trial blocks, the emails will be selected randomly from all possible emails. Then, during the training block where participants receive feedback, the model will search through all possible emails to find the email with the highest probability of being incorrectly categorized. 

The theory behind this approach is that emails should be selected to show participants when there is a high probability that the participant will misclassify them. This can ensure that participants observe a diverse and challenging set of emails, based on their individual performance on past trials. Since we only have data from human participants in trials in which emails are selected at random, we instead compare these two email selection training approaches using IBL+LLM models. The average percentage point improvement in participant categorization accuracy between the pre-training and post-training trials is shown in the lighter shaded bars on the left column of Figure \ref{fig:training}.

IBL+LLM simulated students have the same training as the experiment, with 10 pre-training trials without feedback, 40 training trials with feedback, and 10 post-training trials without feedback. These IBL models trained with a random sampling of emails are compared to the same IBL models trained with emails selected by a separate IBL teacher model. This teacher model is structured in the same way as the IBL tracing models described in previous results. These IBL teachers predict the behavior of simulated IBL students. After each trial of the main training portion of the experiment, the IBL teacher model iterates over all emails that have not yet been shown to the IBL student, and selects the email that has assigns the highest expected utility to the incorrect categorization for that email. 

Other than this training period with emails selected by the IBL teacher, the same standard pre and post-training periods are performed with randomized emails. Results from this training method are shown on the right column of Figure \ref{fig:training}, and demonstrate a clear and significant improvement between the training outcomes, as measured by pre-post-training improvement in terms of percentage point accuracy, between the random email sampling and the IBL+LLM teacher sampling. This suggests that selecting emails to show students using an IBL teaching model may improve the quality of educational outcomes.  

Overall, this comparison of different methods to train simulated IBL+LLM student models provides support for our planned study that will use a IBL teacher model to select the emails that real human participants will observe. This future study will confirm the benefit afforded by using an IBL teacher model to trace the performance of human students and select emails to show to them that will maximize their learning outcomes. The selection of emails to choose those that are most difficult for an individual student effectively broadens the range of emails they experience in the training block when they are receiving feedback. 

\section{Discussion}
In this work, we present a method for assessing different potential uses of GPT-4 by human cyberattackers interested in crafting phishing emails. Results from this experimentation highlights an issue of current methods of training end users to identify phishing emails and improve cybersecurity. Alongside this, we present a proposed solution to the issues that we highlight, to improve the quality of phishing email identification training through the use of a cognitive model. This is done by using an Instance Based Learning model to select the emails that are shown to participants and improve their learning outcomes.

Several interesting and surprising results from analyses of human behavior were revealed in our experimental result. Firstly, the most significant different between any two conditions of the experiment was in the human-written and GPT-4-styled condition and the GPT-4-written and GPT-4-styled condition. Comparing pre-training performance and improvement in the plain-text styled conditions showed little difference between different email authors. This interaction demonstrates that the GPT-4 model is unlikely to write convincing phishing emails from scratch without more advanced prompt engineering.

Another important result from experimental analysis was the observed bias between the perception of emails as being generated by an AI model. As participants were more likely to perceive emails as being written or stylized by AI, the worse their performance in categorizing ham emails. It is possible that the presence of this bias could be incorporated into improved feedback to students, to point out that AI generated writing does not necessarily indicate that an email is phishing. 

Improving education of AI-generated content is an important step to preventing the misuse of LLMs in the future, by improving the public awareness of the capabilities of LLMs, and how best to detect when they are potentially being used for nefarious purposes. A significant area of research in machine learning is seeking to further the capabilities of LLMs, aligning their outputs to human goals and use cases, and make misuse more difficult. However, it is unlikely that a perfect model will ever be trained, as it is possible to train separate models to learn how to best prompt LLMs to allow for unintended use cases. Thus, proper education and training is a crucial step to reducing the potential harm of LLMs in the future. 

\section*{Acknowledgments}
This research was sponsored by the Army Research Office and accomplished under Australia-US MURI Grant Number W911NF-20-S-000, and the AI Research Institutes Program funded by the National Science Foundation under AI Institute for Societal Decision Making (AI-SDM), Award No. 2229881. Compute resources and GPT model credits were provided by the Microsoft Accelerate Foundation Models Research Program grant ``Personalized Education with Foundation Models via Cognitive Modeling"

\bibliography{springer}

\end{document}


% Acknowledgments here



% Appendix here
% Options are (1) APPENDIX (with or without general title) or 
%             (2) APPENDICES (if it has more than one unrelated sections)
% Outcomment the appropriate case if necessary
%
% \begin{APPENDIX}{<Title of the Appendix>}
% \end{APPENDIX}
%
%   or 
%
% \begin{APPENDICES}
% \section{<Title of Section A>}
% \section{<Title of Section B>}
% etc
% \end{APPENDICES}


% References here (outcomment the appropriate case) 

% CASE 1: BiBTeX used to constantly update the references 
%   (while the paper is being written).
\bibliographystyle{unsrtnat} % outcomment this and next line in Case 1
\bibliography{references} % if more than one, comma separated

\end{document}