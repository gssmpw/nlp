\subsection{The ride-hailing system}\label{subsec:ride-hail}
We consider a transportation network with $\Region$ service regions. In this network, a fleet of electric vehicles with size $\Size$ are operated by a central planner to serve passengers, who make trip requests from their origins $\origin \in \Region$ to their destinations $\destination \in \Region$. For each pair of $(\origin, \destination) \in \Region \times \Region$, we assume that the battery consumption for traveling from $\origin$ to $\destination$ is a constant $\batteryod \in \mathbb{R}_{\geq 0}$. The electric vehicles are each equipped with a battery of size $\range$, i.e. a fully charged vehicle can take trips with total battery consumption up to $\range$. A set of chargers with different charging rates $\type \in \Type$ are installed in the network, where each rate $\type$ charger can replenish $\delta$ amount of battery to the connected vehicle in one unit of time. We denote the number of rate $\type \in \Type$ chargers in each region $\region \in \Region$ as $\n{\region}{\type} \in \mathbb{N}_{\geq 0}$. 
% For each region $\region$, we assume that the battery consumption for traveling from $\region$ to the nearest charger is a constant $\battery_{\region} \in \mathbb{R}_{\geq 0}$.


%The entire service area has been divided into $R$ service regions. A predetermined number of vehicles are deployed to efficiently operate within the ride-hailing system. Additionally, a limited number of charging facilities have been built and distributed geographically across the service area to facilitate the charging of the vehicles.

%Passenger requests are received by the system, following a distribution based on factors such as the trip's origin region, destination region, and time of day. At each decision epoch, the ride-hailing system dispatches the vehicles for tasks such as picking up passenger requests, repositioning, charging, idling, or continuing with any ongoing action. The vehicles' batteries are depleted as they travel and are recharged when they undergo the charging process. 

We model the operations of a ride-hailing system as a discrete-time, average reward Markov Decision Process (MDP) with infinite time horizon. In particular, we model the system as an infinite repetition of a single-day operations, where each day $\day =1, 2, \dots$ is evenly partitioned into $\Horizon$ number of time steps $\time =1, \dots, \Horizon$. The system makes a dispatching decision for the entire fleet at every time steps of a day, which we refer to as a decision epochs. In each decision epoch $(\time, \day)$, the number of trip requests between each $\origin$-$\destination$ pair, denoted as $\arrnum{\origin\destination}{\time, \day}$, follows a Poisson distribution with mean $\arrrate{\origin\destination}{\time}$. The duration of trips from region $\origin$ to region $\destination$ at time $t$ is a constant $\timecost{\origin\destination}{\time}$ that is a multiple of the length of a time step. Both $\arrrate{\origin\destination}{\time}$ and $\timecost{\origin\destination}{\time}$ can vary across time steps in a single day, but remain unchanged across days.

When the central planner receives the trip requests, they assign vehicles to serve (all or a part of) the requests. In particular, a vehicle must be assigned to the passenger within the connection patience time $\Lc \geq 0$, and a passenger will wait at most time $\Lp \geq 0$ (defined as the pickup-patience time) for an assigned vehicle to arrive at their origin. Otherwise, the passenger will leave the system. Both $\Lp$ and $\Lc$ are multiples of the discrete time steps.



The central planner keeps track of the status of each vehicle by the region $\region \in \Region$ it is currently located or heading to, remaining time to reach $\timetoarrival = 0, \dots, \maxtimecost{\region}$, and remaining battery level $\battery =0, \dots, \range$ when reaching $\region$. Here, $\maxtimecost{\region} : = \max_{\origin \in \Region, \time \in [\Horizon]} \timecost{\origin \region}{\time}$ is the maximum duration of any trip with destination $\region$. We assume that the minimum time cost of all trips $\min_{\origin,\destination \in \Region, \time \in [\Horizon]} \timecost{\origin\destination}{\time}$ is larger than $\Lp$ so that no vehicle can be assigned to serve more than one trip request in a single decision epoch. 

A vehicle is associated with status $\cartype{}= (\region, \timetoarrival, \battery)$ if (1) it is routed to destination $\region$, with remaining time $\timetoarrival$, and battery level $\battery$ at the arrival; or (2) it is being charged at region $\region$, with remaining charging time $\timetoarrival$, and battery level $\battery$ after the charging period completes. Here, if $\timetoarrival=0$, then the vehicle is parked at region $\region$. Additionally, if $\timetoarrival>0$, then the vehicle could either be taking a trip with a passenger whose destination is $\region$, the vehicle is being repositioned to $\region$, or the vehicle is being charged in $\region$. Vehicle repositioning may serve two purposes: (1) to pick up the next passenger whose origin is $\region$; (2) to be charged or idle at region $\region$. We note that the maximum remaining time of a trip cannot exceed $\maxtimecost{\region} + \Lp$ since a vehicle is eligible to serve a passenger with destination $\region$ if it can arrive at the origin of that trip within $\Lp$ time and the maximum trip duration is $\maxtimecost{\region}$. Moreover, a vehicle with status $(\region, 0, \battery)$ can be charged at region $\region$ with rate $\type$ \footnote{We can easily extend our model to account for non-linear charging rates by introducing $\type$ a function of current battery level.} if such a charger is available in $\region$. After the vehicle is assigned to charge, it will remain charged for $\chargetime$ time steps. If the vehicle's battery level is full or close to full, then it will charge to full and then idle for the remaining time of the charging period. We assume that $\chargetime > \Lp$ so that vehicles assigned to charge will not be assigned to serve trip requests in the same decision epoch \footnote{This assumption can be removed if we add actions that combine charging and trip fulfillment within a single decision epoch. We omit this for the sake of simplicity.}. Let $\Cartype := \left\{(\region, \timetoarrival, \battery) \right\}_{\region \in \Region, \timetoarrival = 0, \dots, \maxtimecost, \battery = 0, \dots, \range}$ denote the set of all vehicle status.
 

A trip order is associated with status $\triptype{}= (\origin, \destination, \tripactivetime)$ if it originates from $\origin$, heads to $\destination$, and has been waiting for vehicle assignment in the system for $\tripactivetime$ time steps. A trip with origin $\origin$ and destination $\destination$ can be served by a vehicle of status $(\origin, \timetoarrival, \battery)$ if (1) the vehicle can reach $\origin$ within the passenger's pickup-patience time $\Lp$ (i.e. $\eta \leq \Lp$), and (2) the remaining battery of the vehicle when reaching $\origin$ is sufficient to complete the trip to $\destination$ (i.e. $b \geq \batteryod$). We note that a vehicle may be assigned to pick up a new passenger before completing its current trip towards $\origin$ as long as it can arrive within $\Lp$ time steps. We use $\Triptype := \left\{(\origin, \destination, \tripactivetime) \right\}_{\origin, \destination\in \Region,\ \tripactivetime = 0, \dots, \Lc}$ to denote the set of all trip status. 

A charger is associated with st $\plugtype{}= (\region, \type, \plugremaintime)$ if it is located at region $\region$ with a rate of $\type$ and is $\plugremaintime$ time steps away from being available. Specifically, if $\plugremaintime = 0$, the charger is available immediately. If $\plugremaintime > 0$, the charger is currently in use and will take an additional $\plugremaintime$ time steps to complete the charging period. We use $\Plugtype := \{(\region, \type, \plugremaintime)\}_{\region \in \Region, \type \in \Type, \plugremaintime = 0, \dots, \chargetime - 1}$ to denote the set of all charger status. All notations introduced in this section are summarized in Table \ref{tab:notations}.

\subsection{Markov decision process} \label{subsec:MDP}
We next describe the \emph{state, action, policy} and \emph{reward} of the Markov decision process (MDP). 


\paragraph{State.} We denote the state space of the MDP as $\mathcal{S}$ with generic element $\state{}{}$. The state vector $\state{}{\time,\day}$ records the time $\time$ of day $\day$, the trip order state $\left(\state{\triptype{}}{\time,\day} \right)_{\triptype{} \in \Triptype}$, the fleet state $\left(\state{\cartype{}}{\time,\day} \right)_{\cartype{} \in \Cartype}$, and the charger state $\left(\state{\plugtype{}}{\time,\day} \right)_{\plugtype{} \in \Plugtype}$, where $\state{\triptype{}}{\time,\day}$ is the number of trip orders of status $\triptype{}$, $\state{\cartype{}}{\time,\day}$ is the number of vehicles of status $\cartype{}$, and $\state{\plugtype{}}{\time, \day}$ is the number of chargers of status $\plugtype{}$ at $(t, d)$. For all $(t, d)$, the sum of vehicles of all status equals to the fleet size $\Size$. That is, 
\begin{equation} \label{eq:setup-car-total-flow}
    \sum_{\cartype{} \in \Cartype} \state{\cartype{}}{\time, \day} = \Size,\quad \forall \time \in [\Horizon],\ \day = 1, 2, \dots.
\end{equation}
Additionally, the sum of chargers of all remaining charging times for each region $\region$ and rate $\type$ equals to the quantity of the corresponding charging facility $\n{\region}{\type}$. That is, $\forall \region \in \Region$ and $\forall \type \in \Type$,
\begin{equation} \label{eq:setup-charger-total-num}
    \sum_{\plugtype{} := (\region, \type, \plugremaintime)} \state{\plugtype{}}{\time, \day} = \n{\region}{\type},\quad \forall \time \in [\Horizon],\ \day = 1, 2, \dots.
\end{equation}

Thus, the state vector is $s^{t, d}= \left(t, \left(\state{\cartype{}}{\time,\day} \right)_{\cartype{} \in \Cartype}, \left(\state{\triptype{}}{\time,\day} \right)_{\triptype{} \in \Triptype}, \left(\state{\plugtype{}}{\time,\day} \right)_{\plugtype{} \in \Plugtype} \right) \in \mathcal{S}$. Here, we note that the number of new trip arrivals $\arrnum{\origin\destination}{\time, \day}$ can be unbounded. However, since the fleet size is $N$ and a trip order can be kept in the system for up to $\Lc$ steps, the maximum number of trip orders arrived at one time step that can be served before being abandoned is $\Size(\Lc+1)$. As a result, without loss of generality, we truncate the number of trip requests for each status entering the system during each decision epoch to $\Size(\Lc + 1)$, with any additional trip requests being rejected by the system. Hence, our state space $\mathcal{S}$ is finite.

\paragraph{Action.} We denote the action space of the MDP as $\mathcal{A}$ with generic element $\jointaction{}{}$. An action is a flow vector of the fleet that consists of the number of vehicles of each status assigned to take trips, reposition, charge, idle, and pass. 
%The feasibility of this action depends on the current state. 
We denote an action vector at time $\time$ of day $\day$ as $\jointaction{}{\time,\day} := \left(\jointtripfulfillaction{\cartype{}}{\time,\day}, \jointreroutingaction{\cartype{}}{\time,\day}, \jointchargingaction{\cartype{}}{\time,\day}, \jointidlingaction{\cartype{}}{\time,\day}, \jointpassaction{\cartype{}}{\time,\day}\right)_{\cartype{} \in \Cartype}$, where: 

\begin{itemize}
    \item[-] \emph{$\jointtripfulfillaction{\cartype{}}{\time,\day} := \left(\jointtripfulfillaction{\cartype{}, \triptype{}}{\time,\day} \in \mathbb{N} \right)_{\triptype{} \in \Triptype}$} determines the number of vehicles of each status $\cartype{}$ assigned to fulfill each trip order status $\triptype{}\in \Triptype$ at time $\time$ of day $\day$. In particular, vehicles are eligible to take trip requests if their current location or destination matches the trip's origin region, they are within $\Lp$ time steps from completing the current task, and they have sufficient battery to complete the trip, i.e.  %The number of vehicles of type $\cartype{}$ assigned to fulfill trip orders of type $\triptype{}$ is upper bounded by the number of vehicles $\state{\cartype{}}{\time, \day}$ of type $\cartype{}$.
    \begin{align} \label{eq:setup-passenger-carrying-flow}
        \jointtripfulfillaction{\cartype{}, \triptype{}}{\time,\day} \left\{
        \begin{array}{ll}
            \geq 0, & \quad \text{$\forall \cartype{} = (\origin, \timetoarrival, \battery)$ and $\triptype{} = (\origin, \destination, \tripactivetime)$}\\
            &\text{ such that $\timetoarrival \leq L_p$ and 
            $\battery \geq \batterycost{\origin\destination}$}, \\
            = 0, & \quad \text{otherwise.}
        \end{array}
        \right.
    \end{align}
    Additionally, we require that the total number of vehicles that fulfill the trip orders of status $\triptype{}$ cannot exceed $\state{\triptype{}}{\time, \day}$, i.e. 
    \begin{equation} \label{eq:setup-trip-fulfill-cap}
        \sum_{\cartype{} = (\origin, \timetoarrival, \battery) \in \Cartype} \jointtripfulfillaction{\cartype{}, \triptype{}}{\time,\day} \leq \state{\triptype{}}{\time, \day},\quad \forall \triptype{} = (\origin, \destination, \tripactivetime) \in \Triptype.
    \end{equation}
    \item[-] \emph{$\jointreroutingaction{\cartype{}}{\time,\day} := \left(\jointreroutingaction{\cartype{}, \destination}{\time,\day} \in \mathbb{N} \right)_{\destination \in \Region}$} represents the number of vehicles of each status $\cartype{}$ assigned to reposition to $\destination$ at time $\time$ of day $\day$. In particular, vehicles are eligible to reposition to a different region if they have already completed their current tasks and they have sufficient battery to complete the trip. 
    \begin{align} \label{eq:setup-rerouting-flow}
        \jointreroutingaction{\cartype{}, \destination}{\time,\day} \left\{
        \begin{array}{ll}
            \geq 0, & \quad \text{$\forall \cartype{} = (\origin, \timetoarrival, \battery)$ and $\destination \neq \origin$}\\
            &\text{ such that $\timetoarrival = 0$ and 
            $\battery \geq \batterycost{\origin \destination}$}, \\
            = 0, & \quad \text{otherwise.}
        \end{array}
        \right.
    \end{align}
    \item[-] \emph{$\jointchargingaction{\cartype{}}{\time,\day} := \left(\jointchargingaction{\cartype{}, \type}{\time,\day} \in \mathbb{N} \right)_{\type \in \Type}$} represents the number of vehicles of status $\cartype{}$ assigned to charge with rate $\type$ at time $\time$ of day $\day$. In particular, vehicles are eligible to charge if they have already completed their current tasks.
    \begin{align} \label{eq:setup-charging-flow}
        \jointchargingaction{\cartype{}, \type}{\time,\day} \left\{
        \begin{array}{ll}
            \geq 0, & \quad \text{$\forall \cartype{} = (\origin, \timetoarrival, \battery)$ and $\type \in \Type$}\\
            &\text{ such that $\timetoarrival = 0$}, \\
            = 0, & \quad \text{otherwise.}
        \end{array}
        \right.
    \end{align}
    Additionally, for each region $\region$ and charging rate $\type$, we require that the total number of vehicles of all status assigned to charge at region $\region$ with rate $\type$ cannot exceed the total number of available chargers:
    \begin{equation} \label{eq:setup-charging-cap}
        \sum_{\cartype{} = (\region, \timetoarrival, \battery) \in \Cartype} \jointchargingaction{\cartype{}, \type}{\time,\day} \leq \state{(\region, \type, 0)}{\time, \day},\quad \forall \region \in \Region,\  \type \in \Type.
    \end{equation}
    \item[-] \emph{$\jointidlingaction{\cartype{}}{\time,\day} \in \mathbb{N}$} represents the number vehicles of status $\cartype{}$ assigned to idle at time $\time$ of day $\day$. In particular, vehicles are eligible to idle if they have already completed their current tasks.
    \begin{align} \label{eq:setup-idling-flow}
        \jointidlingaction{\cartype{}}{\time,\day} \left\{
        \begin{array}{ll}
            \geq 0, & \quad \text{$\forall \cartype{} = (\origin, \timetoarrival, \battery)$ such that $\timetoarrival = 0$}, \\
            = 0, & \quad \text{otherwise.}
        \end{array}
        \right.
    \end{align}
    \item[-] \emph{$\jointpassaction{\cartype{}}{\time,\day} \in \mathbb{N}$} represents the number of vehicles of status $\cartype{}$ not assigned with any new action at time $\time$ of day $\day$. In particular, vehicles are eligible for the pass action if they have not completed their tasks yet.
    \begin{align} \label{eq:setup-passing-flow}
        \jointpassaction{\cartype{}}{\time,\day} \left\{
        \begin{array}{ll}
            \geq 0, & \quad \text{$\forall \cartype{} = (\origin, \timetoarrival, \battery)$ such that $\timetoarrival > 0$}, \\
            = 0, & \quad \text{otherwise.}
        \end{array}
        \right.
    \end{align}
\end{itemize}

% Let the state vector be $\state{}{} := \left(\state{}{\time, \day}\right)_{\time \in [\Horizon],\ \day \in [\totaldays]}$. 
For any $\time$ and $\day$, the vector $\jointaction{}{\time, \day}$ needs to satisfy the following flow conservation constraint: All vehicles of each status should be assigned to trip-fulfilling, repositioning, charging, idling, or passing actions. That is,
\begin{align} \label{eq:setup-cartype-conservation}
    &\sum_{\triptype{} \in \Triptype} \jointtripfulfillaction{\cartype{}, \triptype{}}{\time,\day} + \sum_{\destination \in \Region} \jointreroutingaction{\cartype{}, \destination}{\time,\day} + \sum_{\type \in \Type} \jointchargingaction{\cartype{}, \type}{\time,\day} + \jointidlingaction{\cartype{}}{\time,\day} + \jointpassaction{\cartype{}}{\time,\day} = \state{\cartype{}}{\time,\day} \notag\\
    &\qquad \forall \cartype{} \in \Cartype,\ \time \in [\Horizon],\ \day = 1, 2, \dots.
\end{align}
From the above description, we note that the feasibility of an action depends on the state. We denote the set of actions that are feasible for state $s$ as $\mathcal{A}_s$. 

\paragraph{Policy.} The policy $\pi: \mathcal{S} \rightarrow \Delta(\mathcal{A}_s)$ is a randomized policy that maps the state vector to an action, where $\pi(\jointaction{}{} \vert \state{}{})$ is the probability of choosing action $\jointaction{}{}$ given state $\state{}{}$ under policy $\pi$. We note that the notation $\pi(\cdot)$ does not explicitly depend on $t$ since $t$ is already a part of the state vector $s$.  

\paragraph{State Transitions.} At any time $\time$ of day $\day$, given any state $\state{}{}$ and the action $\jointaction{}{} \in \mathcal{A}_{\state{}{}}$, we compute the vehicle state vector at time $\time+1$. 
For each vehicle status $\cartype{} := (\destination, \eta, \battery) \in \Cartype$, %we compute  $\state{\cartype{}}{\time + 1, \day}$ as follows: 
\begin{align}
    &\state{(\destination, \eta, \battery)}{\time + 1, \day} = \left(\sum_{\origin \in \Region} \sum_{\cartype{}'= (\origin, \timetoarrival', \battery') \in \Cartype} \sum_{ \triptype{}= (\origin, \destination, \xi') \in \Triptype} \right. \nonumber\\
    &\qquad \left. \jointtripfulfillaction{\cartype{}', \triptype{}}{\time,\day}\mathds{1}(\timetoarrival' + \timecost{\origin\destination}{\time} - 1 = \timetoarrival, \  \battery' - \batterycost{\origin \destination} = \battery) \right) \tag{i} \\
    &\qquad+ \left(\sum_{\origin \in \Region} \sum_{\cartype{}'= (\origin, 0, \battery') \in \Cartype} \jointreroutingaction{\cartype{}', \destination}{\time,\day} \cdot \right. \nonumber\\
    &\qquad \left. \mathds{1}(\timecost{\origin\destination}{\time} - 1 = \timetoarrival, \  \battery' - \batterycost{\origin\destination} = \battery) \right) \tag{ii} \\
    &\qquad+ \left[ \left(\sum_{\type \in \Type} \jointchargingaction{(\destination, 0, \battery - \type \chargetime), \type}{\time,\day} \mathds{1}(\timetoarrival = \chargetime - 1, \battery \geq \type \chargetime) \right) + \right. \nonumber\\
    &\qquad \left. \left(\sum_{\battery' > \battery - \type \chargetime} \sum_{\type \in \Type} \jointchargingaction{(\destination, 0, \battery'), \type}{\time,\day} \mathds{1}(\timetoarrival = \chargetime - 1, \battery = \range) \right) \right] \tag{iii}\\
    &\qquad+ \underbrace{\jointidlingaction{(\destination, \timetoarrival, \battery)}{\time,\day} \mathds{1}(\timetoarrival = 0)}_{(iv)} + \underbrace{\jointpassaction{(\destination, \timetoarrival+1, \battery)}{\time,\day} \mathds{1}(\timetoarrival < \maxtimecost{})}_{(v)}, \label{eq:setup-car-state-transition}
\end{align}
where (i) and (ii) correspond to vehicles assigned to new trip-fulfilling or repositioning actions with destination $v$, respectively. The destination, time to arrival, and battery level of these vehicles are updated based on the newly assigned trips. Term (iii) corresponds to the vehicles assigned to charge at time $\time$. The battery level of these vehicles will be increased by $\delta \chargetime$ at the end of the charging period. If the vehicle is charged to full, then it will remain idle for the rest of the charging period. Term (iv) corresponds to the idling vehicles, and (v) corresponds to the vehicles taking the passing action whose remaining time of completing the assigned action decreases by $1$ in the next time step. 

Moreover, the trip state at time $\time+1$ of day $\day$ is given by \eqref{eq:setup-trip-state-transition}. For each trip status, we subtract the number of trip orders that have been fulfilled at time $\time$, and we increment the active time by $1$ for trip orders that are still in the system. We abandon the trip orders that have been active for more than $\Lc$ time steps. Additionally, new trip orders arrive in the system and we set their active time in the system to be $0$. That is, for all $(\origin, \destination) \in \Region \times \Region$,
\begin{align}\label{eq:setup-trip-state-transition}
     \state{(\origin, \destination, \xi)}{\time + 1, \day} = \left\{
     \begin{array}{ll}
                  \min\left\{\arrnum{\origin\destination}{\time, d}, \  \Size (\Lc + 1) \right\},  \\
                  \quad \text{if $\tripactivetime=0$}, \\
                  \state{(\origin, \destination, \tripactivetime - 1)}{\time,\day} - \sum_{\cartype{}  \in \Cartype} \jointtripfulfillaction{\cartype{}, (\origin, \destination, \tripactivetime-1)}{\time,\day}, \\
                  \quad \text{if $1 \leq \tripactivetime \leq \Lc$}.  
     \end{array}
     \right.
\end{align}
% Note that $\arrnum{\origin\destination}{\time, d}$ can be unbounded. However, since the fleet size is $N$ and a trip order can be kept in the system for up to $\Lc$ steps, the maximum number of trip orders realized at one time step that can be served before leaving the system is $\Size(\Lc+1)$. As a result, we can cap $\arrnum{\origin\destination}{\time, d}$ by $\Size(\Lc+1)$ without loss of generality.
As mentioned earlier in this section, we cap the number of new trip arrivals by $\Size(\Lc+1)$ so that the state space is finite. Then, trips that are not fulfilled at time $\time$ are queued to $\time+1$. Thus, the number of trips that have been in the system for $\tripactivetime \geq 1$ at time $\time+1$ equals to that with $\tripactivetime-1$ from time $\time$ minus the ones that are assigned to a vehicle at time $\time$.  

Lastly, the charger state at time $\time + 1$ of day $\day$ is given by \eqref{eq:setup-plug-state-transition}. For each charger status $\plugtype{} := (\region, \type, \plugremaintime) \in \Plugtype$: For all region $\region \in \Region$ and for all charger outlet rates $\type \in \Type$,
\begin{align}\label{eq:setup-plug-state-transition}
     \state{(\region, \type, \plugremaintime)}{\time + 1, \day} = \left\{
     \begin{array}{ll}
                  \sum_{\cartype{} = (\region, 0, \battery) \in \Cartype} \jointchargingaction{\cartype{}, \type}{\time,\day},  & \text{if } \plugremaintime=\chargetime - 1, \\
                  \state{(\region, \type, \plugremaintime + 1)}{\time, \day}, & \text{if } 0 < \plugremaintime< \chargetime - 1,\\
                  \state{(\region, \type, 1)}{\time, \day} + \left(\state{(\region, \type, 0)}{\time, \day}\right. \\
                  \left.- \sum_{\cartype{} = (\region, 0, \battery) \in \Cartype} \jointchargingaction{\cartype{}, \type}{\time,\day}\right), & \text{if } \plugremaintime=0.
     \end{array}
     \right.
\end{align}
At time $t+1$, the number of chargers with status $(\region, \type, \plugremaintime)$ (i.e. occupied and remaining time is $\plugremaintime = \chargetime - 1$) equals the total number of vehicles just assigned to charge at time $\time$. Chargers already in use at time $\time$ will have their remaining charging time $j$ decrease by 1 when the system transitions to time $\time + 1$. The number of chargers available (i.e. $j=0$) at time $\time + 1$ consists of (i) the chargers that have just completed their charging periods (i.e. $\state{(\region, \type, 1)}{\time, \day}$), and (ii) the chargers available at time $\time$ minus the ones that are assigned to charge vehicles (i.e. $\state{(\region, \type, 0)}{\time, \day} - \sum_{\cartype{} = (\region, 0, \battery) \in \Cartype} \jointchargingaction{\cartype{}, \type}{\time, \day}$).

\paragraph{Reward.} The reward of fulfilling a trip request from $\origin$ to $\destination$ at time $t$ is $\tripfulfillreward{,\origin\destination}{\time} \in \mathbb{R}_{\geq 0}$. We remark that if a trip request has been active in the system for some time, then the reward is determined by the time at which the trip request is picked up. The reward (cost) of re-routing between $\origin, \destination$ is $\reroutingreward{,\origin\destination}{\time} \in \mathbb{R}_{\leq 0}$. The reward (cost) for a vehicle to charge at time $\time$ is $\chargingreward{,\type}{\time} \in \mathbb{R}_{\leq 0}$. Idling and passing actions incur no reward or cost. As a result, given the action $\jointaction{}{\time, \day}$, the total reward at time $\time$ of day $\day$ is given by 
 \begin{align} \label{eq:setup-reward}
     \reward{}{\time}(\jointaction{}{\time,\day}) =& 
    \sum_{\cartype{} \in \Cartype} \sum_{(\origin, \destination) \in \Region \times \Region}\sum_{\tripactivetime = 0}^{\Lc} \tripfulfillreward{,\origin\destination}{\time} \jointtripfulfillaction{\cartype{}, (\origin, \destination, \xi)}{\time,\day} \notag \\ 
    +& \sum_{\cartype{} \in \Cartype} \sum_{(\origin, \destination) \in \Region \times \Region}
    \reroutingreward{,\origin\destination}{\time} \jointreroutingaction{\cartype{}, \destination}{\time,\day} + \sum_{\cartype{} \in \Cartype}\sum_{\type \in \Type} \chargingreward{,\type}{\time} \jointchargingaction{\cartype{},\type}{\time,\day}.
 \end{align}

The long-run average daily reward of a policy $\pi$ given some initial state $\state{}{}$ is as follows:
\begin{equation} \label{eq:long-run-avg-reward}
    R(\pi \vert \state{}{}) := \lim_{\totaldays \rightarrow \infty} \frac{1}{\totaldays} \mathbb{E}_{\pi}\left[\sum_{\day = 1}^{\totaldays} \sum_{\time = 1}^{\Horizon} \reward{}{\time}(\jointaction{}{\time,\day}) \Biggm\lvert \state{}{} \right].
\end{equation}

Since the state is finite and the state transition and policy are homogeneous across days, the limit defined above exists (see page 337 of \cite{PutermanMDP}). Our goal is to find the optimal fleet control policy $\pi^*$ that maximizes the long-run average daily rewards given {\em any} initial state $\state{}{}$: 
\begin{equation} \label{eq:opt-long-run-avg-reward}
    R^*(\state{}{}) =  R(\pi^* \vert \state{}{}) = \max_{\pi} R(\pi \vert \state{}{}),\quad \forall \state{}{} \in \mathcal{S}.
\end{equation}

\begin{table}[htb]
    \centering
    \begin{tabular}{|c|c|}
        \hline
        Symbol & Description \\
        \hline
        $\Region$ & The set of all service regions\\
        $\Size$ & Fleet Size \\
        $\range$ & Vehicle battery range\\
        $\Type$ & Set of charging rates\\
        $\Horizon$ & Number of time steps of each single day\\
        \hline
        $\timecost{\origin\destination}{\time}$ & Trip duration from $\origin$ to $\destination$ at time $\time$\\
        % $\maxtimecost{\region}$ & Maximum trip duration to $\region$\\
        $\battery_{\origin\destination}$ & Battery consumption for traveling from region $\origin$ to $\destination$\\
        % $\battery_{\region}$ & Battery consumption for traveling to nearest charging station from $\region$\\
        $\n{\region}{\type}$ & Number of chargers of rate $\type$ at region $\region$ \\
        \hline
        $\arrrate{\origin\destination}{\time}$ & Trip arrival rate from $\origin$ to $\destination$ at time $\time$\\
        $\arrnum{\origin\destination}{\time,\day}$ & Number of trip requests from $\origin$ to $\destination$ at time $\time$ of day $\day$\\
        $\Lp$ & Maximum pickup patience time\\
        $\Lc$ & Maximum assignment patience time\\
        $\chargetime$ & Duration of a charging period\\
        \hline
        $\Cartype$ & The set of all vehicle statuses, with a generic vehicle status denoted as $\cartype{}$\\
        $\Triptype$ & The set of all trip statuses, with a generic trip status denoted as $\triptype{}$\\
        $\Plugtype$ & The set of all charger statuses, with a generic charger status denoted as $\plugtype{}$\\
        \hline
        $\state{}{\time, \day}$ & The state vector at time $\time$ of day $\day$\\
        $\jointaction{}{\time, \day}$ & The fleet action vector at time $\time$ of day $\day$\\
        $\jointtripfulfillaction{\cartype{}, \triptype{}}{\time, \day}$ & The number of vehicles of the status $\cartype{}$ assigned to fulfill trips of status $\triptype{}$ at time $\time$ of day $\day$ \\
        $\jointreroutingaction{\cartype{}, \region}{\time, \day}$ & The number of vehicles of the status $\cartype{}$ assigned to reposition to region $\region$ at time $\time$ of day $\day$ \\
        $\jointchargingaction{\cartype{}, \type}{\time, \day}$ & The number of vehicles of the status $\cartype{}$ assigned to charge with rate $\type$ at time $\time$ of day $\day$ \\
        $\jointidlingaction{\cartype{}}{\time, \day}$ & The number of vehicles of the status $\cartype{}$ assigned to idle at time $\time$ of day $\day$ \\
        $\jointpassaction{\cartype{}}{\time, \day}$ & The number of vehicles of the status $\cartype{}$ assigned to pass at time $\time$ of day $\day$ \\
        $\reward{}{\time}(a)$ & The reward given action $a$ at time $\time$ \\
        \hline
    \end{tabular}
    \caption{Notations for the electric robo-taxi system}
    \label{tab:notations}
\end{table}