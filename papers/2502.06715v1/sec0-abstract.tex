\begin{abstract}
  To achieve true scalability on massive datasets, a modern query
  engine needs to be able to take advantage of large, shared-memory,
  multicore systems.  {\em Binary joins} are conceptually easy to
  parallelize on a multicore system; however, several applications
  require a different approach to query evaluation, using a Worst-Case
  Optimal Join (WCOJ) algorithm.  WCOJ is known to outperform
  traditional query plans for cyclic queries. However, there is no
  obvious adaptation of WCOJ to parallel architectures. The few
  existing systems that parallelize WCOJ do this by partitioning only
  the top variable of the WCOJ algorithm. This leads to work
  skew (since some relations end up being read entirely by every
  thread), possible contention between threads (when the
  hierarchical trie index is built lazily, which is the case on most
  recent WCOJ systems), and exacerbates the redundant computations
  already existing in WCOJ.

  We introduce \name, a parallel version of WCOJ, optimized for large
  multicore, shared-memory systems.  \name partitions the domains of
  all query variables, not just that of the top loop.  We adapt the
  partitioning idea from the HyperCube algorithm, developed by the
  theory community for computing multi-join queries on a massively
  parallel shared-nothing architecture, and introduce new methods for
  computing the shares, optimized for a shared-memory architecture.
  To avoid the contention created by the lazy construction of the
  trie-index, we introduce \indexlayout, a new and very simple index
  structure, which we build eagerly, by sorting the entire relation.
  Finally, in order to remove some of the redundant computations of
  WCOJ, we introduce a rewriting technique of the WCOJ plan that
  factors out some of these redundant computations. Our experimental
  evaluation compares \name with several recent implementations of
  WCOJ.
\end{abstract}