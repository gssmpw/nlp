\section{Conclusion}\label{sec:conclusion}

We introduced \name, an implementation of Generic Join (a special case
of Worst Case Optimal Join) on a multicore, shared-memory
architecture.  Prior parallel implementations of WCOJ partitioned only
the top loop variable.  Instead, in \name we partition the domains of
all variables, which reduces the computation skew.  We also
introduced a simple trie index, based on a novel two-stage
sorting-based parallel algorithm, which can be constructed eagerly and
cheaply, and removes concurrent conflicts during the query evaluation.
The use of a sorted array instead of a hash table also takes advantage
of modern hardware by improving cache locality and enabling vector
processing.  We co-optimized the choice of the variable order and the
computation of the optimal shares by using a novel cost model for data
skew, and described a rewriting technique for WCOJ that reduced the
amount of redundant computations.  Finally, we reported an extensive
evaluation of our implementation, proving the effectiveness of the
partitioning strategy, of the optimized choice of variable order and
shares, and of the rewriting technique.

Future work includes further leveraging hardware by supporting vectorization, 
speeding up the optimization time through dynamic programming, and extending
the rewriting technique to support more complicated cases.


