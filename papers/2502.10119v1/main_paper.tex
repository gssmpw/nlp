%%%%%%%% ICML 2025 EXAMPLE LATEX SUBMISSION FILE %%%%%%%%%%%%%%%%%

\documentclass{article}

% Recommended, but optional, packages for figures and better typesetting:
\usepackage{microtype}
\usepackage{graphicx}
\usepackage{subfigure}
\usepackage{booktabs} % for professional tables

% hyperref makes hyperlinks in the resulting PDF.
% If your build breaks (sometimes temporarily if a hyperlink spans a page)
% please comment out the following usepackage line and replace
% \usepackage{icml2025} with \usepackage[nohyperref]{icml2025} above.
\usepackage{hyperref}


% Attempt to make hyperref and algorithmic work together better:
\newcommand{\theHalgorithm}{\arabic{algorithm}}

% Use the following line for the initial blind version submitted for review:
%\usepackage{icml2025}

% If accepted, instead use the following line for the camera-ready submission:
 \usepackage[preprint]{icml2025}

% For theorems and such
\usepackage{amsmath}
\usepackage{amssymb}
\usepackage{mathtools}
\usepackage{amsthm}

\usepackage{graphicx}
\usepackage{subfigure}
\usepackage{booktabs} % for professional tables
\usepackage{makecell}
\usepackage{multirow}
\usepackage{listings}
\usepackage{pythonhighlight}
\usepackage{wrapfig}
%\usepackage[table,xcdraw,dvipsnames]{xcolor}

\def\method{SeWA}


% if you use cleveref..
\usepackage[capitalize,noabbrev]{cleveref}


\usepackage[ruled,algo2e,linesnumbered]{algorithm2e}
\SetKwInput{KwInit}{Init}
\SetCommentSty{mycommentfont}


%%%%%%%%%%%%%%%%%%%%%%%%%%%%%%%%
% THEOREMS
%%%%%%%%%%%%%%%%%%%%%%%%%%%%%%%%
\theoremstyle{plain}
\newtheorem{theorem}{Theorem}[section]
\newtheorem{proposition}[theorem]{Proposition}
\newtheorem{lemma}[theorem]{Lemma}
\newtheorem{corollary}[theorem]{Corollary}
\theoremstyle{definition}
\newtheorem{definition}[theorem]{Definition}
\newtheorem{assumption}[theorem]{Assumption}
\theoremstyle{remark}
\newtheorem{remark}[theorem]{Remark}

% Todonotes is useful during development; simply uncomment the next line
%    and comment out the line below the next line to turn off comments
%\usepackage[disable,textsize=tiny]{todonotes}
\usepackage[textsize=tiny]{todonotes}


% The \icmltitle you define below is probably too long as a header.
% Therefore, a short form for the running title is supplied here:
\icmltitlerunning{SeWA: Selective Weight Average via Probabilistic Masking}

\begin{document}

\twocolumn[
\icmltitle{\method{}: Selective Weight Average via Probabilistic Masking}

% It is OKAY to include author information, even for blind
% submissions: the style file will automatically remove it for you
% unless you've provided the [accepted] option to the icml2025
% package.

% List of affiliations: The first argument should be a (short)
% identifier you will use later to specify author affiliations
% Academic affiliations should list Department, University, City, Region, Country
% Industry affiliations should list Company, City, Region, Country

% You can specify symbols, otherwise they are numbered in order.
% Ideally, you should not use this facility. Affiliations will be numbered
% in order of appearance and this is the preferred way.
\icmlsetsymbol{equal}{*}

\begin{icmlauthorlist}
\icmlauthor{Peng Wang}{hust}
\icmlauthor{Shengchao Hu}{sjtu}
\icmlauthor{Zerui Tao}{aip}
\icmlauthor{Guoxia Wang}{bidu}\\
\icmlauthor{Dianhai Yu}{bidu}
\icmlauthor{Li Shen}{sysu}
%\icmlauthor{}{sch}
\icmlauthor{Quan Zheng}{hust}
\icmlauthor{Dacheng Tao}{ntu}
%\icmlauthor{}{sch}
%\icmlauthor{}{sch}
\end{icmlauthorlist}

\icmlaffiliation{hust}{School of Mathematics and Statistics, Huazhong University of Science and Technology, Wuhan, China}
\icmlaffiliation{sjtu}{School of Electronic Information and Electrical Engineering, Shanghai Jiao Tong University, Shanghai, China}
\icmlaffiliation{aip}{RIKEN Center for Advanced Intelligence Project, Tokyo, Japan}
\icmlaffiliation{bidu}{Baidu Inc., Beijing, China}
\icmlaffiliation{sysu}{School of Cyber Science \& Technology, Shenzhen Campus of Sun Yat-sen University, Shenzhen, China}
\icmlaffiliation{ntu}{School of Computer and Data Science, Nanyang Technological University, Singapore}

\icmlcorrespondingauthor{Li Shen}{mathshenli@gmail.com}
%\icmlcorrespondingauthor{Firstname2 Lastname2}{first2.last2@www.uk}

% You may provide any keywords that you
% find helpful for describing your paper; these are used to populate
% the "keywords" metadata in the PDF but will not be shown in the document
\icmlkeywords{Machine Learning, ICML}

\vskip 0.3in
]

% this must go after the closing bracket ] following \twocolumn[ ...

% This command actually creates the footnote in the first column
% listing the affiliations and the copyright notice.
% The command takes one argument, which is text to display at the start of the footnote.
% The \icmlEqualContribution command is standard text for equal contribution.
% Remove it (just {}) if you do not need this facility.

\printAffiliationsAndNotice{}  % leave blank if no need to mention equal contribution
%\printAffiliationsAndNotice{\icmlEqualContribution} % otherwise use the standard text.

\begin{abstract}
Weight averaging has become a standard technique for enhancing model performance. However, methods such as Stochastic Weight Averaging (SWA) and Latest Weight Averaging (LAWA) often require manually designed procedures to sample from the training trajectory, and the results depend heavily on hyperparameter tuning. To minimize human effort, this paper proposes a simple yet efficient algorithm called Selective Weight Averaging (\method{}), which adaptively selects checkpoints during the final stages of training for averaging. Based on \method{}, we show that only a few points are needed to achieve better generalization and faster convergence. Theoretically, solving the discrete subset selection problem is inherently challenging. To address this, we transform it into a continuous probabilistic optimization framework and employ the Gumbel-Softmax estimator to learn the non-differentiable mask for each checkpoint. Further, we theoretically derive the \method{}'s stability-based generalization bounds, which are sharper than that of SGD under both convex and non-convex assumptions. Finally, solid extended experiments in various domains, including behavior cloning, image classification, and text classification, further validate the effectiveness of our approach. 

\end{abstract}
\section{Introduction}%

Decision-making is at the heart of artificial intelligence systems, enabling agents to navigate complex environments, achieve goals, and adapt to changing conditions. Traditional decision-making frameworks often rely on associations or statistical correlations between variables, which can lead to suboptimal outcomes when the underlying causal relationships are ignored \citep{pearl2009causal}. 
The rise of causal inference as a field has provided powerful frameworks and tools to address these challenges, such as structural causal models and potential outcomes frameworks \citep{rubin1978bayesian,pearl2000causality}. 
Unlike traditional methods, \textit{causal decision-making} focuses on identifying and leveraging cause-effect relationships, allowing agents to reason about the consequences of their actions, predict counterfactual scenarios, and optimize decisions in a principled way \citep{spirtes2000causation}. In recent years, numerous decision-making methods based on causal reasoning have been developed, finding applications in diverse fields such as recommender systems \citep{zhou2017large}, clinical trials \citep{durand2018contextual}, finance \citep{bai2024review}, and ride-sharing platforms \citep{wan2021pattern}. Despite these advancements, a fundamental question persists: 

\begin{center}
    \textit{When and why do we need causal modeling in decision-making?}
\end{center} 

% Numerous decision-making methods based on causal reasoning have been developed recently with wide applications 
% %Decision makings based on causal reasoning have been widely applied 
% in a variety of fields, including 
% recommender systems \citep{zhou2017large}, clinical trials \citep{durand2018contextual}, 
% finance \citep{bai2024review}, 
% ride-sharing platforms \citep{wan2021pattern}, and so on. 


 

% At the intersection of these fields, causal decision-making seeks to answer critical questions: How can agents make decisions when causal knowledge is incomplete? How do we integrate learning and reasoning about causality into real-world decision-making systems? What role do interventions, counterfactuals, and observational data play in guiding decisions? 

% Our review is structured as follows: 
 

This question is closely tied to the concept of counterfactual thinking—reasoning about what might have happened under alternative decisions or actions. Counterfactual analysis is crucial in domains where the outcomes of unchosen decisions are challenging, if not impossible, to observe. For instance, a business leader selecting one marketing strategy over another may never fully know the outcome of the unselected option \citep{rubin1974estimating, pearl2009causal}. Similarly, in econometrics, epidemiology, psychology, and social sciences, \textit{the inability to observe counterfactuals directly often necessitates causal approaches} \citep{morgan2015counterfactuals, imbens2015causal}. 
Conversely, non-causal analysis may suffice in scenarios where alternative outcomes are readily determinable. For example, a personal investor's actions may have negligible impact on stock market dynamics, enabling potential outcomes of alternate investment decisions to be inferred from existing stock price time series \citep{angrist2008mostly}. However, even in cases where counterfactual outcomes are theoretically calculable—such as in environments with known models like AlphaGo—exhaustively computing all possible outcomes is computationally infeasible \citep{silver2017mastering, silver2018general}. 
In such scenarios, causal modeling remains advantageous by offering \textit{structured ways to infer outcomes efficiently and make robust decisions}. 


%This perspective not only enhances the interpretability of decisions but also provides a principled framework for addressing uncertainty, guiding actions, and improving performance across a broad range of applications.

% Data-driven decision-making exists before the causal revolution. \textit{So when and why do we need causal modelling in decision-making?} 
% This is closely related to the presence of counterfactuals in many applications. 
% The counterfactual thinking involves considering what would have happened in an alternate scenario where a different decision or action was taken. 
% In many fields, including econometrics, epidemiology, psychology, and social sciences, accessing outcomes from unchosen decisions is often challenging if not impossible. 
% For example, a business leader who selects one marketing strategy over another may never know the outcome of the unselected option. 
% Conversely, non-causal analysis may be adequate in situations where potential outcomes of alternate actions are more readily determinable: for example, the investment of a personal investor may have minimal impact on the market, therefore her counterfactual investment decision's outcomes can still be calculated with the data of stock price time series. 
% However, it is important to note that even when counterfactuals are theoretically calculable, as in environments with known models like AlphaGo, computing all possible outcomes may not be feasible. 
% In such scenarios, a causal perspective  remains beneficial. 


 

% 1. significance of decision making
% 2. role of causal in decision making
% 3. refer to the https://jair.org/index.php/jair/article/view/13428/26917

% Decision makings based on causal reasoning have been widely applied in a variety of fields, including recommender systems \citep{zhou2017large}, clinical trials \citep{durand2018contextual}, 
% business economics scenarios \citep{shen2015portfolio}, 
% ride-sharing platforms \citep{wan2021pattern}, and so on. 
% However, most existing works primarily assume either sophisticated prior knowledge or strong causal models to conduct follow-up decision-making. To make effective and trustworthy decisions, it is critical to have a thorough understanding of the causal connections between actions, environments, and outcomes.

\begin{figure}[!t]
    \centering
    \includegraphics[width = .75\linewidth]{Figure/3Steps_V2.png}
    \caption{Workflow of the \acrlong{CDM}. $f_1$, $f_2$, and $f_3$ represent the impact sizes of the directed edges. Variables enclosed in solid circles are observed, while those in dashed circles are actionable.}\label{fig:cdm}
\end{figure}


Most existing works primarily assume either sophisticated prior knowledge or strong causal models to conduct follow-up decision-making. To make effective and trustworthy decisions, it is critical to have a thorough understanding of the causal relationships among actions, environments, and outcomes. This review synthesizes the current state of research in \acrfull{CDM}, providing an overview of foundational concepts, recent advancements, and practical applications. Specifically, this work discusses the connections of \textbf{three primary components of decision-making} through a causal lens: 1) discovering causal relationships through \textit{\acrfull{CSL}}, 2) understanding the impacts of these relationships through \textit{\acrfull{CEL}}, and 3) applying the knowledge gained from the first two aspects to decision making via \textit{\acrfull{CPL}}. 

Let $\boldsymbol{S}$ denote the state of the environment, which includes all relevant feature information about the environment the decision-makers interact with, $A$ the action taken, $\pi$ the action policy that determines which action to take, and $R$ the reward observed after taking action $A$. As illustrated in Figure \ref{fig:cdm}, \acrshort{CDM} typically begins with \acrshort{CSL}, which aims to uncover the unknown causal relationships among various variables of interest. Once the causal structure is established, \acrshort{CEL} is used to assess the impact of a specific action on the outcome rewards. To further explore more complex action policies and refine decision-making strategies, \acrshort{CPL} is employed to evaluate a given policy or identify an optimal policy. In practice, it is also common to move directly from \acrshort{CSL} to \acrshort{CPL} without conducting \acrshort{CEL}. Furthermore, \acrshort{CPL} has the potential to improve both \acrshort{CEL} and \acrshort{CSL} by facilitating the development of more effective experimental designs \citep{zhu2019causal,simchi2023multi} or adaptively refining causal structures \citep{sauter2024core}. %However, these are beyond the scope of this paper.

\begin{figure}[!t]
    \centering
    \includegraphics[width = .9\linewidth]{Figure/Table_of_Six_Scenarios_S.png}
    \caption{Common data dependence structures (paradigms) in \acrshort{CDM}. Detailed notations and explanations can be found in Section \ref{sec:paradigms}.}
    \label{Fig:paradigms}
\end{figure}
Building on this framework, decision-making problems discussed in the literature can be further categorized into \textbf{six paradigms}, as summarized in Figure \ref{Fig:paradigms}. These paradigms summarize the common assumptions about data dependencies frequently employed in practice. Paradigms 1-3 describe the data structures in offline learning settings, where data is collected according to an unknown and fixed behavior policy. In contrast, paradigms 4-6 capture the online learning settings, where policies dynamically adapt to newly collected data, enabling continuous policy improvement. These paradigms also reflect different assumptions about state dependencies. The simplest cases, paradigms 1 and 4, assume that all observations are independent, implying no long-term effects of actions on future observations. To account for sequental dependencies, the \acrfull{MDP} framework, summarized in paradigms 2 and 5, assumes Markovian state transition. Specifically, it assumes that given the current state-action pair $(S_t, A_t)$, the next state $S_{t+1}$ and reward $R_t$ are independent of all prior states $\{S_j\}_{j < t}$ and actions $\{A_j\}_{j < t}$. When such independence assumptions do not hold, paradigms 3 and 6 account for scenarios where all historical observations may impact state transitions and rewards. This includes but not limited to researches on \acrfull{POMDP} \citep{hausknecht2015deep, littman2009tutorial}, panel data analysis \citep{hsiao2007panel,hsiao2022analysis}, \acrfull{DTR} with finite stages \citep{chakraborty2014dynamic, chakraborty2013statistical}. 

Each \acrshort{CDM} task has been studied under different paradigms, with \acrshort{CSL} extensively explored within paradigm 1. \acrshort{CEL} and offline \acrshort{CPL} encompass paradigms 1-3, while online \acrshort{CPL} spans paradigms 4-6. By organizing the discussion around these three tasks and six paradigms, this review aims to provide a cohesive framework for understanding the field of \acrlong{CDM} across diverse tasks and data structures.

%Recognizing the importance of long-term effects in decision-making

%Further discussions on these paradigms and their connections to various causal decision-making problems are provided in Section \ref{sec:paradigms}.


\textbf{Contribution.} In this paper, we conduct a comprehensive survey of \acrshort{CDM}. 
Our contributions are as follows. 
\begin{itemize}
    \item We for the first time organize the related causal decision-making areas into three tasks and six paradigms, connecting previously disconnected areas (including economics, statistics, machine learning, and reinforcement learning) using a consistent language. For each paradigm and task, we provide a few taxonomies to establish a unified view of the recent literature.
    \item We provide a comprehensive overview of \acrshort{CDM}, covering all three major tasks and six classic problem structures, addressing gaps in existing reviews that either focus narrowly on specific tasks or paradigms or overlook the connection between decision-making and causality (detailed in Section \ref{sec::related_work}).
    %\item We outline three key challenges that emerge when utilizing CDM in practice. Moreover, we delve into a comprehensive discussion on the recent advancements and progress made in addressing these challenges. We also suggest six future directions for these problems.
    \item We provide real-world examples to illustrate the critical role of causality in decision-making and to reveal how \acrshort{CSL}, \acrshort{CEL} and \acrshort{CPL} are inherently interconnected in daily applications, often without explicit recognition.
    \item We are actively maintaining and expanding a GitHub repository and online book, providing detailed explanations of key methods reviewed in this paper, along with a code package and demos to support their implementation, with URL: \url{https://causaldm.github.io/Causal-Decision-Making}.
\end{itemize}
% Our review is structured as follows: 


%%%%%%%%%%%%%%%%%%%%%%%%%%%%%%%%%%
%  causal helps over "Correlational analysis"
%Correlational analysis, though widely used in various fields, has inherent limitations, particularly when it comes to decision-making. While it identifies relationships between variables, it fails to establish causality, often leading to misinterpretations and misguided decisions. For example, the positive correlation between ice cream sales and drowning incidents is a classic example of how correlational data can be misleading, as both are influenced by a third factor, temperature, rather than causing each other. Such spurious correlations, due to oversight of confounding variables, underscore the necessity of causal modeling in decision making. Causal models excel where correlational analysis falls short, offering predictive power and a deeper understanding of underlying mechanisms. They enable us to predict the outcomes of interventions, even under untested conditions, and provide insights into the processes leading to these outcomes, thereby informing more effective strategies. Moreover, causal models are good at generalizing findings across different contexts, a capability often limited in purely correlational studies. 

%  causal helps in causal RL 
%From another complementary angle, although causal concepts have traditionally not been explicitly incorporated in fields like online bandits \citep{lattimore2020bandit} and \acrfull{RL} \citep{sutton2018reinforcement}, much of the literature in these areas implicitly relies on basic assumptions outlined in Section \ref{sec:prelim_assump} to utilize observed data in place of potential outcomes in their analyses, and there is also a growing recognition of the significance of the causal perspective \citep{lattimore2016causal, zeng2023survey} in these areas. 
% \textbf{Read causal RL survey and summarize. } However, by integrating causal concepts and leverging existing methodologies, we open up possibilities for developing more robust models to remove spurious correlation and selection bias \citep{xu2023instrumental, forney2017counterfactual}, designing more sample-efficient \citep{sontakke2021causal, seitzer2021causal} and robust \citep{dimakopoulou2019balanced, ye2023doubly} algorithms, and improving the generalizability \citep{zhang2017transfer, eghbal2021learning}, explanability \citep{foerster2018counterfactual, herlau2022reinforcement}, and fairness \citep{zhang2018fairness,huang2022achieving,balakrishnan2022scales} of these methods. %, and safety \cite{hart2020counterfactual}

%


%\subsection{Paper Structure}
The remainder of this paper is organized as follows: Section \ref{sec::related_work} provides an overview of related survey papers. Section \ref{sec:preliminary} introduces the foundational concepts, assumptions, and notations that form the foundation for the subsequent discussions. In Section \ref{sec:3task6paradigm}, we offer a detailed introduction to the three key tasks and six learning paradigms in \acrshort{CDM}. Sections \ref{Sec:CSL} through \ref{sec:Online CPL} form the core of the paper, with each section dedicated to a specific topic within \acrshort{CDM}: \acrshort{CSL}, \acrshort{CEL}, Offline \acrshort{CPL}, and Online \acrshort{CPL}, respectively. Section \ref{sec:assump_violated} then explores extensions needed when standard causal assumptions are violated. To illustrate the practical application of the \acrshort{CDM} framework, Section \ref{sec:real_data} presents two real-world case studies. Finally, Section \ref{sec:conclusion} concludes the paper with a summary of our contributions and a discussion of additional research directions that are actively being explored.


\section{Related Work}
\label{sec:ReW}

\textbf{Weight averaging algorithm.}
Model averaging methods, initially introduced in convex optimization \cite{ruppert1988efficient, polyak1992acceleration,li2023deep}, have been widely used in various areas of deep learning and have shown their advantages in generalization and convergence. Subsequently, with the introduction of SWA \cite{izmailov2018averaging}, which averages the weights along the trajectory of SGD, the model's generalization is significantly improved. Further modifications have been proposed, including the Stochastic Weight Average Density (SWAD) \cite{cha2021swad}, which averages checkpoints more densely, leading to the discovery of flatter minima associated with better generalization. Additionally, Trainable Weight Averaging (TWA) \cite{li2022trainable} has improved the efficiency of SWA by employing trainable averaging coefficients. What's more, other approaches like Exponential Moving Average (EMA) \cite{szegedy2016rethinking} and finite averaging algorithms, such as LAWA \cite{kaddour2022stop, sanyal2023early}, which averages the last $k$ checkpoints from running a moving window at a predetermined interval, employ different strategies to average checkpoints. These techniques have empirically shown faster convergence and better generalization. In meta-learning, Bayesian Model Averaging (BMA) is used to reduce the uncertainty of the model \cite{huang2020meta}. However, these different algorithms often require manual design of averaging strategies and are only applicable to some specific tasks, imposing an additional cost on the training.

\textbf{Stability Analysis.}
Stability analysis is a fundamental theoretical tool for studying the generalization ability of algorithms by examining their stability \citep{devroye1979distribution, bousquet2002stability, mukherjee2006learning, shalev2010learnability}. Based on this, \citet{hardt2016train} use the algorithm stability to derive generalization bounds for SGD, inspiring series works \cite{charles2018stability, zhou2018generalization, yuan2019stagewise, lei2020sharper}. This analysis framework has been extended to various domains, such as online learning \citep{yang2021simple}, adversarial training \citep{xiao2022stability}, decentralized learning \citep{zhu2023stability}, and federated learning \citep{sun2023understanding, sun2023mode}. Although uniform sampling is a standard operation for building stability boundaries, selecting the initial point and sampling without replacement also significantly affects generalization and has been investigated in \citet{shamir2016without, kuzborskij2018data}. For the averaging algorithm, \citet{hardt2016train} and \citet{xiao2022stability} analyze the generalization performance of SWA and establish stability bounds for the algorithm under the setting of convex and sampling with replacement. The main focus of this paper is the generalization and construction of stability bounds for \method{} in both convex and non-convex settings.

\textbf{Mask Learning.}
The general approach involves transforming the discrete optimization problem into a continuous one using probabilistic reparameterization, thereby enabling gradient-based optimization. \citet{coreset} solve the coreset selection problem based on this by using a Policy Gradient Estimator (PGE) for a bilevel optimization objective. \citet{zhangefficient} propose a probabilistic masking method that improves diffusion model efficiency by skipping redundant steps.
While the PGE method may suffer from high variance and unstable training, we solve the mask learning problem using the Gumbel-softmax reparameterization \citep{jang2017categorical,maddison2017the}.
In this paper, we aim to adaptively select checkpoints for averaging to enhance model generalization and convergence, addressing the issue of unstable training.
\section{Methodology}
\label{sec:method}
In  this section, we first give the problem setup and then introduce our proposed \method{} and several terminologies.

\subsection{Problem Setting}
Let $F(w, z)$ be a loss function that measures the loss of the predicted value of the parameter $w$ at a given sample $z$. There is an unknown distribution $\mathcal{D}$ over examples from some space $\mathcal{Z}$, and a sample dataset $S=(z_1, z_2,..., z_n)$ of $n$ examples i.i.d. drawn from $\mathcal{D}$. Then the \emph{population risk} and \emph{empirical risk} are defined as 
\begin{equation}
\textbf{Population Risk:}\quad   \min_w \{R_\mathcal{D}[w]=E_ {z \sim \mathcal{D}}F(w; z) \} \label{prm}
\end{equation}
\begin{equation}
\textbf{Empirical Risk:}\quad  \min_w \{ R_S [w]= \frac{1}{n} \sum_{i=1}^{n}F(w; z_i) \}.\label{erm}
\end{equation}
 The generalization error of a model $w$ is the difference $\epsilon_{gen}=R_\mathcal{D}[w] - R_S [w]$. Moreover, we assume function $F$ satisfies the following \emph{Lipschitz} and \emph{smoothness} assumption.


\begin{assumption}[$L$-Lipschitz]
\label{$L$-Lipschitz}
A differentiable function $F: R^d \rightarrow R$ satisfies the $L$-Lipschitz property, i.e., for $\forall u, v \in R^d, \Vert F(u)-F(v)\Vert \leq L\Vert u-v\Vert$, which implies that the gradient satisfies $\Vert\nabla F(u)\Vert \leq L$.
\end{assumption}

\begin{assumption}[$\beta$-smooth]
\label{beta-smooth}
A differentiable function $F: R^d \rightarrow R$ is $\beta$-smooth, i.e., for $\forall u, v \in R^d$, we have $\Vert\nabla F(u)-\nabla F(v)\Vert \leq \beta\Vert u-v\Vert$.%, where $\nabla F(v)$ denotes the gradient of $F$ at $v$. 
\end{assumption}

Assumptions \ref{$L$-Lipschitz} and \ref{beta-smooth} are often used to establish stability bounds for algorithms and are crucial conditions for analyzing the model's generalization performance. 

\begin{assumption}[Convex function]
\label{Convex function}
A differentiable function $F: R^d \rightarrow R$ is convex, i.e., for $\forall u, v \in R^d, F(u) \leq F(v) + \langle\nabla F(u), u-v \rangle.$
\end{assumption}

Different function assumptions correspond to different expansion properties, which will be discussed in Lemma \ref{lemma}.

\subsection{SGD and \method{} Algorithm}
For the target function $F$ and the given dataset $S=(z_1, z_2, \cdots, z_n)$, we consider the SGD's general update rule as Eq.~\eqref{SGD-rules}. \method{} algorithm adaptively selects a small number of points for averaging among the last k points of the SGD's training trajectory. It is shown in Eq.~\eqref{\method{}-rules}.

\textbf{SGD} is formulated as
    \begin{equation}\label{SGD-rules}
     w_{t+1} = w_{t} - \alpha \nabla_w F(w_{t},z_{i_t}),
    \end{equation}   
where $\alpha$ is the fixed learning rate, $z_{i_t}$ is the sample chosen in iteration $t$. We choose $z_{i_t}$ from dataset $S$ in a standard way, picking $i_t \sim Uniform\left\{1, \cdots, n \right\}$ at each step. This setting is commonly explored in analyzing the stability \cite{hardt2016train,xiao2022stability}. 

\textbf{\method{}} is formulated as 
\begin{equation}\label{\method{}-rules}
    \bar{w}^{K}_{T}=\frac{1}{K} \sum_{i=T-k+1}^{T} m_{i}w_{i},
\end{equation}
where the mask $m_i \in \left\{0, 1 \right\}$ and $m_i = 1$ indicating the $i$-th weight is selected for averaging and otherwise excluded; the $K = k_{m_i=1}$ corresponds to the number of models selected for averaging. The \method{} algorithm focuses on adaptively selecting a small number of points from the last $k$ steps of the SGD training trajectory for averaging, aiming to improve generalization and accelerate convergence. We show the convergence performance of different
models in Figure \ref{fig:enter-label}.
\begin{figure}
    \centering
    \includegraphics[width=1.\linewidth]{compare.pdf}
    \caption{Comparison of \method{} with different models on convergence performance.}
    \label{fig:enter-label}
\end{figure}

\subsubsection{The Expansive Properties}
\begin{lemma}\label{lemma}
Assume that the function $F$ is $\beta$-smooth. Then, \\
{\bf (1). (non-expansive)} If $F$ is convex, for any $\alpha \leq \frac{2}{\beta}$, we have $\Vert w_{T+1}-w_{T+1}^{\prime} \Vert \leq \Vert w_{T}-w_{T}^{\prime}\Vert$; \\
{\bf (2). ($(1\!+\!\alpha\beta)$-expansive)} If $F$ is non-convex, for any $\alpha$, we have $\Vert w_{T+1}\!-\!w_{T+1}^{\prime} \Vert \!\leq\! (1\!+\!\alpha\beta)\Vert w_{T}\!-\!w_{T}^{\prime}\Vert$.
\end{lemma}

Lemma \ref{lemma} tells us that the gradient update becomes $non$-expansive when the function is convex and the step size is small, which implies that the algorithm will always converge to the optimum in this setting. However, although this is not guaranteed when the function is non-convex, it is required that the gradient updates cannot be overly expansive if the algorithm is stable. The proof of Lemma \ref{lemma} is deferred to Appendix \ref{pro-lemma}. Additional relevant results can be found in \citet{hardt2016train,xiao2022stability}. 

\subsection{Stability and Generalization Definition}
\citet{hardt2016train} link the \emph{uniform stability} of the learning algorithm with the expected generalization error bound in research of SGD's generalization. The expected generalization error of a model $w = A_S$ trained by a certain randomized algorithm $A$ is defined as 
    \begin{equation}\label{D-gen}
\mathbb{E}_{S,A}\left[R_{S}\left[A_S\right]-R_\mathcal{D}\left[A_S\right]\right]. 
    \end{equation}
Here, the expectation is taken only over the internal randomness of $A$. 
Next, we introduce the \emph{uniform stability}.

\begin{definition}[$\epsilon$-Uniformly Stable]
A randomized algorithm $A$ is $\epsilon$-uniformly stable if for all data sets $S, S^{\prime} \in Z^{n}$ such that $S$ and $S^{\prime}$ differ in at most one example, we have
    \begin{equation}\label{E-stab}
     \mathop{sup}\limits_{z\in Z}\left\{\mathbb{E}_{A}\left[F(A_S;z)-F(A_{S^{\prime}};z)\right] \right\} \leq \epsilon.
    \end{equation}
\end{definition}

\begin{theorem}{\rm (Generalization in Expectation \citep[Theorem 2.2]{hardt2016train})}
Let $A$ be $\epsilon$-uniformly stable. Then,
    \begin{equation}\label{E-gen}
     \vert\mathbb{E}_{S,A}\left[R_{S}\left[A_S\right]-R_\mathcal{D}\left[A_S\right]\right]\vert \leq \epsilon.
    \end{equation}
\end{theorem}

This theorem clearly states that if an algorithm has uniform stability, then its generalization error is small. In other words, uniform stability implies \emph{generalization in expectation} \cite{hardt2016train}. Above proof is based on \citet[Lemma 7]{bousquet2002stability} and very similar to \citet[Lemma 11]{shalev2010learnability}.


\section{Generalization Analysis of \method{}}
%\subsection{Generalization Analysis}\ls{it seems that there is no section 4.2.}
This section provides two theorems that give upper bounds on generalization in the convex and non-convex settings, respectively. First, a critical lemma is provided for building stability bound in the convex function setting.

\begin{lemma}\label{convex-basic}
Let $\bar{w}_{T}^K$ and $\bar{w}_{T}^{K\prime}$ denote the corresponding outputs of \method{} after running $T$ steps on the datasets $S$ and $S^{\prime}$, which have $n$ samples but only one different. Assume that function $F(\cdot,z)$ satisfies Assumptions \ref{$L$-Lipschitz} and \ref{Convex function} for a fixed example $z\in\mathcal{Z}$, then we have
 \begin{equation}\label{eq_convex-basic}
\mathbb{E}\vert F(\bar{w}_{T}^K;z)-F(\bar{w}_T^{K\prime};z)\vert \leq sL \mathbb{E} [\bar{\delta}_T],
 \end{equation} 
where $s=\sup_{T-k+1\leq i\leq T} s_i$, where $s_i$ denotes the probability of $m_i=1$ and $\bar{\delta}_T=\frac{1}{k}\sum_{i=T-k+1}^{T}\Vert w_i-w_i^{\prime}\Vert$. 
\end{lemma}

\begin{proof} We establish a generalization bound for the algorithm based on stability, where the $L$-Lipschitz transforms the problem into bounding the parameter differences. 
  \begin{equation*}
  \begin{aligned}
  &\mathbb{E}_{z,m,A}\vert F(\bar{w}_{T}^K;z)-F(\bar{w}_T^{K\prime};z) \vert \leq \!L\mathbb{E}_{m,A}\Vert \bar{w}_{T}^K - \bar{w}_T^{K\prime}\Vert\\
  & \leq L \!\left(\!\frac{1}{k}\!\sum_{i=T-k+1}^{T}\!\!s_i \mathbb{E}_{A}\Vert w_i-w_i^{\prime}\Vert+\frac{1}{k}\!\sum_{i=T-k+1}^{T}\!\!(1-s_i)\!\cdot \!0\!\right)\\
  & \leq sL \mathbb{E}_{A} [\bar{\delta}_T], 
  \end{aligned}
 \end{equation*}
where the first inequality comes from $L$-Lipschitz assumption, the second inequality is based on taking the expectation for mask $m_i$, and the last inequality because of $s=\sup_{T-k+1\leq i\leq T} s_i$.
\end{proof}

The Lemma \ref{convex-basic} further decomposes the problem of selecting points for averaging within the last $k$ steps into averaging over the last $k$ steps multiplied by the probability $s_i$ of each step by taking an expectation over the mask. Next, we give the generalization bound for \method{} in the convex setting combined with Lemma \ref{convex-basic}.

\begin{theorem}\label{thm:stability-conv}
Suppose that we run \method{} with constant step sizes $\alpha \leq \frac{2}{\beta}$ for $T$ steps, where each step samples $z \in \mathcal{Z}$ uniformly with replacement. If function $F$ satisfies Assumptions \ref{$L$-Lipschitz}, \ref{beta-smooth} and \ref{Convex function}. \method{} has uniform stability of
\begin{equation}
  \epsilon_{gen} \leq \frac{2\alpha L^2 s}{n} \left(T - \frac{k}{2} \right),
 \end{equation}
where $s=\sup_{T-k+1\leq i\leq T} s_i$.
\end{theorem}

\begin{proof}[Proof sketch]
We can first establish stability bounds for the last $k$ points of the averaging algorithm and then use Lemma \ref{convex-basic} to obtain bounds for \method{}. \\
First, based on Eq.~\eqref{pro-FWA-update}, we divide cumulative gradient into two parts: $\Vert \nabla F(w_{T-1},z_{T}) - \nabla F(w^{\prime}_{T-1},z_{T}) \Vert$ and $\frac{1}{k}\sum_{i=T-k+1}^{T-1} \alpha_i \Vert \nabla F(w_{i-1},z_{i}) - \nabla F(w^{\prime}_{i-1},z_{i}) \Vert$. \\
{\bf(1)} Bound $\Vert \nabla F(w_{T-1},z_{T}) - \nabla F(w^{\prime}_{T-1},z_{T}) \Vert$. On one hand, we consider the different samples $z$ and $z^{\prime}$ is selected with probability $\frac{1}{n}$, we only need to use $L$ to bound $\nabla F(w_{T-1},z_{T})$ and $\nabla F^{\prime}(w^{\prime}_{T-1},z_{T}^{\prime})$ respectively. On the other hand, with probability $1-\frac{1}{n}$ that the same samples $z=z^{\prime}$ is selected, we can use the \emph{non}-expansive update rule from Lemma \ref{lemma}, based on the fact that the objective function is convex and $\alpha\leq\frac{2}{\beta}$. In summary, $\Vert \nabla F(w_{T-1},z_{T}) - \nabla F^{\prime}(w_{T-1}^{\prime},z_{T}) \Vert \leq \frac{2\alpha_T L}{k}$. \\
{\bf(2)} Then we consider bounding the cumulative gradient $\frac{1}{k}\sum_{i=T-k+1}^{T-1} \alpha_i \Vert \nabla F(w_{i-1},z_{i}) - \nabla F(w^{\prime}_{i-1},z_{i}) \Vert$. Since each step $i\in [T-k+1,\cdots, T-1]$ executes sampling with replacement, we can bound them in the way above. Then, we get $\frac{1}{k}\sum_{i=T-k+1}^{T-1}\alpha_i \Vert\nabla F(w^{\prime}_{i-1},z_i) - \nabla F(w_{i-1},z_i) \Vert \leq \frac{2L}{nk}\sum_{i=T-k+1}^{T-1}\alpha_i.$ \\
Second, by merging the above two results and taking summation over $T$ steps, we get $\mathbb{E}\left[\bar{\delta}_{T}\right] \leq (1 \!-\!\frac{1}{n})\bar{\delta}_{T-1}\! +\! \frac{1}{n}\left(\bar{\delta}_{T-1}+\frac{2\alpha_T L}{k}\right)\! +\! \frac{2L}{nk}\sum_{i=T-k+1}^{T-1}\alpha_i \! \leq \! \frac{2\alpha L^2}{n} \left( T - \frac{k}{2} \right)$,
 %    \begin{equation}
 %  \begin{aligned}
 %    \mathbb{E}\left[\bar{\delta}_{T}\right] &\leq (1-\frac{1}{n})\bar{\delta}_{T-1} + \frac{1}{n}\left(\bar{\delta}_{T-1}+\frac{2\alpha_T L}{k}\right) + \frac{2L}{nk}\sum_{i=T-k+1}^{T-1}\alpha_i  \leq \frac{2\alpha L^2}{n} \left( T - \frac{k}{2} \right),
 %  \end{aligned}
 % \end{equation}
where let $\alpha_i=\alpha$ and substitute it to Eq.~\eqref{eq_convex-basic} yields the desired result.
We leave the proof in Appendix \ref{proof-thm-con-with}.
\end{proof}

\begin{remark}
Theorem \ref{thm:stability-conv} shows that the \method{} algorithm has a sharper bound of $\frac{2\alpha L^2 s}{n} \left(T - \frac{k}{2} \right)$ under the convex assumption than the bound $\frac{2\alpha L^2 T}{n}$ for SGD given by \citet{hardt2016train}. The reason for improving the generalization comes from two main sources: (1) the last $k$ point averaging algorithm improves the SGD bound $\mathcal{O}(T/n)$ to $\mathcal{O}((T-k/2)/n)$, where $k$ is the number of averages, and this result degenerates to the SGD bound when $k = 1$. (2) the \method{} algorithm further improves the bound $\mathcal{O}((T-k/2)/n)$ to $s$ times its size, where $0\leq s \leq 1$ is the probability of $m = 1$.   
\end{remark}

\begin{remark}
The $k$ in Theorem \ref{thm:stability-conv} implies that the more checkpoints involved in the averaging in \method{}, the better the generalization performance. The sparse parameter $s$ selects models, aiming to achieve better generalization with fewer averaged models. However, these two aspects are not contradictory. For a lower sparsity rate, \method{} selects more models, leading to improved generalization performance, a trend validated in our experiments in Section \ref{sec:Exp}.
\end{remark}

\begin{lemma}\label{nonconvex-basic}
Let $\bar{w}_{T}^K$ and $\bar{w}_{T}^{K\prime}$ denote the corresponding outputs of \method{} after running $T$ steps on the datasets $S$ and $S^{\prime}$, which have $n$ samples but only one different. Assume that function $F(\cdot,z)$ satisfies Assumption \ref{$L$-Lipschitz} for a fixed example $z\in\mathcal{Z}$ and every $t_0 \in \{1,\cdots,n\}$, then we have
\begin{equation}\label{eq_nonconvex-basic}
\mathbb{E}\vert F(\bar{w}_{T}^K;z)-F(\bar{w}_{T}^{K\prime};z)\vert \leq \frac{t_0}{n} + sL \mathbb{E}\left[\bar{\delta}_{T}\vert \bar{\delta}_{t_0}=0\right],
\end{equation}   
where $s=\sup_{T-k+1\leq i\leq T} s_i$, where $s_i$ denotes the probability of $m_i=1$ and $\bar{\delta}_T=\frac{1}{k}\sum_{i=T-k+1}^{T}\Vert w_i-w_i^{\prime}\Vert$.
\end{lemma}

\begin{proof} By taking expectation for $m_i$, we split the proof of Lemma \ref{nonconvex-basic} into two parts.\\
In the first part, for $t_0 \in \{1,\cdots,n\}$, we discuss that different samples $z$ and $z^{\prime}$ can be selected only after step $t_0$. Then the inequality $\mathbb{E}\vert F(\bar{w}_{T}^K;z)-F(\bar{w}_{T}^{K\prime};z)\vert \leq \frac{t_0}{n} + L \mathbb{E}\left[\Vert \bar{w}_{T}^K - \bar{w}_{T}^{K\prime}\Vert\vert \Vert \bar{w}_{t_0}^K - \bar{w}_{t_0}^{K\prime}\Vert=0\right]$ will be obtained. One can find further proof details in Appendix \ref{proof-noncon-basic}.\\
Secondly, we take expectation for the $m_i$ of $\Vert \bar{w}_{T}^K \!-\! \bar{w}_{T}^{K\prime}\Vert$, which is similar to the proof of Lemma \ref{convex-basic}.
\end{proof}

\begin{theorem}\label{thm:stability-non-with}
Suppose we run \method{} with constant step sizes $\alpha \leq \frac{c}{T}$ for $T$ steps, where each step samples $z$ from $\mathcal{Z}$ uniformly with replacement. Let function $F\in[0,1]$ satisfies Assumptions \ref{$L$-Lipschitz} and \ref{beta-smooth}. \method{} has uniform stability of
\begin{equation}\label{result-5.3}
  \epsilon_{gen}\leq \frac{1+\frac{1}{c\beta}}{n-1}\left(2csL^2(1+ke^{c\beta})k^{-1}\right)^{\frac{k}{c\beta+k}}\cdot T^{\frac{c\beta}{c\beta+k}},
 \end{equation}
where $s=\sup_{T-k+1\leq i\leq T} s_i$.
\end{theorem}

\begin{proof}[Proof sketch]
In the non-convex setting, we finish the task using Lemma \ref{nonconvex-basic}. First, dividing cumulative gradient in the first stage is the same as in the convex case, except that we use the $(1+\alpha\beta)$-expansive properties (Lemma \ref{lemma}) to bound each $\Vert \nabla F(w_{i-1},z_{i}) - \nabla F(w^{\prime}_{i-1},z_{i}) \Vert$ here. \\
{\bf(1)} We bound each $\Vert \nabla F(w_{i-1},z_{i}) - \nabla F(w^{\prime}_{i-1},z_{i}) \Vert$ in the expectation with probabilities of $1/n$ and $1-1/n$, respectively. And combining $L$-Lipschitz conditions and $(1+\alpha\beta)$-expansive, we have $(2L(1+\alpha\beta)^{i-2})/n$. Then, bounding the cumulative gradient based on the above, we transform it into a problem of summing a finite geometric series. \\
{\bf(2)} We provide a key Lemma \ref{Lemma_noncon}, which helps us to obtain a recurrence relation for $\bar{\delta}_T$ and $\bar{\delta}_{T-1}$ in non-convex. We leave the details in Appendix \ref{proof-Lemma_noncon}. \\
Second, taking summation form $t_0$ to $T$, we get 
\begin{equation*}
 \mathbb{E}\left[\bar{\delta}_{T}\right] \leq \frac{2L(1+ke^{c\beta})}{(n-1)\beta} \cdot \left(\frac{T}{t_0}\right)^{\frac{c\beta}{k}}.   
\end{equation*} \\
Finally, we get $t_0$ by minimizing the Eq.~\eqref{with-con} and plug it into Eq.~\eqref{eq_nonconvex-basic}. We finish the proof and leave the details of this proof in Appendix \ref{proof-thm-non-with}.
\end{proof}

\begin{remark}
Under the non-convex assumption, Theorem \ref{thm:stability-non-with} shows that \method{} has bound $\mathcal{O}(T^{c\beta/(c\beta+k)}/n)$ compared to the bound $\mathcal{O}(T^{c\beta/(c\beta+1)}/n)$ for SGD in \citet{hardt2016train}, again showing its ability to improve generalization significantly. Although the number $k$, closely related to the number of iterations $T$, seems to dominate the result, the direct influence of parameter $s$ on the entire stationary bound also plays a crucial role. 
\end{remark}

\begin{remark}
The assumption that $F(w;z) \in [0,1]$ in Theorem \ref{thm:stability-non-with} is adopted for simplicity. Removing this condition does not affect the final results, as it merely introduces a constant scaling factor. The same setting is commonly used and discussed in \citet{hardt2016train,xiao2022stability}. 
\end{remark}

\section{Practical \method{} Implementation}
Although the \method{} algorithm has simpler expressions, the difficulty is learning the mask $m_i$. Inspired by tasks such as coreset selection \cite{coreset}, the discrete problem is relaxed to a continuous one. We first formulate weight selection into the following discrete optimization paradigm:
\begin{equation}\label{dis-opt}
    \min_{m\in C}F(m)=L\left(\textbf{w}(m)\right)=\frac{1}{n}\sum_{i=1}^{n}l\left(f(x_i;\textbf{w}(m),y_i) \right),
\end{equation}
where $C=\left\{\textbf{m}: m_i = 0 \,\textbf{or}\, 1, \Vert \textbf{m}\Vert_0\leq K \right\}$ and $\textbf{w}(m)=\frac{1}{K} \sum_{i=T-k+1}^{T} m_{i}w_{i}$.

To transform the discrete Eq.~\eqref{dis-opt} into a continuous one, we treat each mask $m_i$ as an independent binary random variable and reparameterize it as a Bernoulli random variable, $m_i \sim \text{Bern}(s_i)$, where $s_i \in [0, 1]$ represents the probability of $m_i$ taking the value 1, while $1-s_i$ corresponds to the probability of $m_i$ being 0. Consequently, the joint probability distribution of $m$ is expressed as $p(m\vert s) = \prod _{i=1}^{n}(s_i)^{m_i}(1 - s_i)^{1 - m_i}$. Then, the feasible domain of the target Eq.~\eqref{dis-opt} approximately becomes $\hat{C}=\left\{s: 0\leq s\leq 1, \Vert s\Vert_1\leq K \right\}$ since $\mathbb{E}_{m_i \sim p(m\vert s)}\Vert m\Vert_0 = \sum_{i=1}^{n}si$. As in the previous definition, $K>0$ in $\hat{C}$ is a constant that controls the size of the feasible domain. Then, Eq.~\eqref{dis-opt} can be naturally relaxed into the following excepted loss minimization problem:
\begin{equation}\label{E-dis-opt}
    \min_{s\in \hat{C}}F(s)=\mathbf{E}_{p(m|s)}L\left(\textbf{w}(m)\right),
\end{equation}
where $\hat{C}=\left\{s: 0\leq s\leq 1, \Vert s\Vert_1\leq K \right\}$.

Optimizing Eq.~\eqref{E-dis-opt} involves discrete random variables, which are non-differentiable.
One choice is using Policy Gradient Estimators (PGE) such as 
the REINFORCE algorithm \citep{williams1992simple,sutton1999policy} to bypass the back-propagation of discrete masks $m$,
\begin{equation*}
    \nabla_s F(s)=\mathbf{E}_{p(m|s)} L\left(\textbf{w}(m)\right) \nabla_s \log p( m \mid s).
\end{equation*}
However, these algorithms suffer from the high variance of computing the expectation of the objective function, hence may lead to slow convergence or sub-optimal results.

To address these issues, we resort to the reparameterization trick using Gumbel-softmax sampling \citep{jang2017categorical,maddison2017the}.
Instead of sampling discrete masks $m$, we get continuous relaxations by,
\begin{equation}\label{eq:gs-sample}
\small \!\!\!\!\!  \tilde{m}_i = \frac{\exp((\log s_i + g_{i, 1}) / t)}{\exp((\log s_i + g_{i, 1}) / t) + \exp((\log (1 - s_i) + g_{i, 0}) / t)},
\end{equation}
for $i = 1, \dots, k$, where $g_{i, 0}$ and $g_{i, 1}$ are i.i.d. samples from the $\text{Gumbel}(0, 1)$ distribution. The hyper-parameter $t > 0$ controls the sharpness of this approximation. When it reaches zero, i.e., $t \to 0$, $\tilde{m}$ converges to the true binary mask $m$. During training, we maintain $t > 0$ to make sure the function is continuous. For inference, we can sample from the Bernoulli distribution with probability $s$ to get sparse binary masks. In practice, the random variables $g \sim \text{Gumbel}(0, 1)$ can be efficiently sampled from,
\begin{equation*}
    g = - \log ( - \log (u)), \quad u \sim \text{Uniform}(0, 1).
\end{equation*}
For simplicity, we denote the Gumbel-softmax sampling in Eq.~\eqref{eq:gs-sample} as $\tilde{m} = \text{GS}(s, u, t)$, where $u \sim \text{Uniform}(0, 1)$. Replacing the binary mask $m$ in Eq.~\eqref{E-dis-opt} with the continuous relaxation $\tilde{m}$, the optimization problem becomes,
\begin{equation*}
    \min_{s\in \hat{C}}F(s)=\mathbf{E}_{u \sim \text{Uniform(0, 1)}} L\left(\textbf{w}(\text{GS}(s, u, t)\right),
\end{equation*}
where $\hat{C}=\left\{s: 0\leq s\leq 1, \Vert s\Vert_1\leq K \right\}$. The expectation can be approximated by Monte Carlo samples, i.e.,
\begin{equation*}
    \min_{s\in \hat{C}} \hat{F}(s)= \frac{1}{M} \sum_{m=1}^M L\left(\textbf{w}(\text{GS}(s, u^{(m)}, t)\right),
\end{equation*}
where $u^{(m)}$ are i.i.d. samples drawn from $\text{Uniform}(0, 1)$.
Empirically, since the distribution of $u$ is fixed, this Monte Carlo approximation exhibits low variance and stable training \cite{kingma2013auto,rezende2014stochastic}. Furthermore, since Eq.~\eqref{eq:gs-sample} is continuous, we can optimize it using back-propagation and gradient methods.

\begin{algorithm2e}[t]
    \caption{Selected Weight Average (\method{})}\label{alg:gs}
    \KwIn{Checkpoints $\textbf{w}$, hyper-parameter $t$}
    \KwInit{Mask probability $s$\;}
    \For{$i = 1, \dots, max\_iteration$}{
    \tcc{Gumbel-softmax sampling}
    \For{$m = 1, \dots, M$}{
        Sample $u^{(m)} \sim \text{Uniform}(0, 1)$\;
        Compute $L\left(\textbf{w}(\text{GS}(s, u^{(m)}, t)\right)$\;
    }
    \tcc{Learning mask probability}
    Optimize $\hat{F}(s)= \frac{1}{M} \sum_{m=1}^M L\left(\textbf{w}(\text{GS}(s, u^{(m)}, t)\right)$\;
    % Truncation for sparsity??\;
    }
    \KwOut{Mask $m$ based on $K$ largest probabilities in $s$}
\end{algorithm2e}

\begin{remark}
\method{} adaptively selects useful checkpoints, which implies that it does not require the extra cost associated with manual design and avoids model biases introduced by prior knowledge, thereby making our approach applicable to a broader range of tasks. In the following experiments, \method{} algorithm demonstrates particular suitability for scenarios characterized by unstable training trajectories, such as behavior cloning. By leveraging checkpoint averaging, \method{} effectively stabilizes the training process, mitigating fluctuations and enhancing overall performance.
\end{remark}

\section{Experiment}
\label{sec:Exp}
In our experimental evaluation, we systematically explore the performance of our proposed method across three distinct settings: behavior cloning, image classification, and text classification. These settings are chosen to demonstrate the generality and effectiveness of our approach in diverse application domains. 
%
Details of the experimental setup, including network architectures, hyperparameters, and additional results, are provided in Appendix \ref{sec:ExpDetail}.

\subsection{Behavior Cloning}
\textbf{Experimental Setups.} We performed extensive evaluations using the widely recognized D4RL benchmark \citep{fu2020d4rl}, with a particular focus on the Gym-MuJoCo locomotion tasks. These tasks are commonly regarded as standard benchmarks due to their simplicity and well-structured nature. They are characterized by datasets containing a significant proportion of near-optimal trajectories and smooth reward functions, making them suitable for evaluating the performance of reinforcement learning methods. 
% Specifically, we concentrated on the medium and medium-expert datasets, which provide a balanced mix of trajectories with varying levels of performance, enabling a comprehensive assessment of our method’s capability to generalize across different reward distributions.
%
For evaluation, we employed cumulative reward as the primary metric, as it effectively captures the overall performance of the agent in maximizing returns over the course of its trajectories. 

\textbf{Baselines.} 
%
To assess the effectiveness of our proposed SeWA method, we compare it against several established baselines, including the original pre-training recipe based on stochastic gradient descent (SGD), Stochastic Weight Averaging (SWA) \citep{izmailov2018averaging}, and Exponential Moving Average (EMA) \citep{szegedy2016rethinking}, which we adapt for the behavior cloning setting.
%
For EMA, we follow the approach outlined in \citet{kaddour2022stop}, setting the decay parameter to 0.9 and updating the EMA model at every $K$ training step, which is a widely adopted standard practice. 
%
For SWA, we adhere to the original pre-training procedure up to 75\% completion. Following this phase, we initiate SWA training with a cosine annealing scheduler and compute the SWA uniform average every $K$ steps to aggregate model parameters effectively.
%
Additionally, we compare our SeWA with LAWA \citep{sanyal2023early} and a Random baseline. 
Both of these baselines involve directly averaging pretrained checkpoints from the original pre-training process without additional retraining. Specifically, LAWA selects $K$ checkpoints at equal intervals, whereas the Random baseline selects $K$  checkpoints randomly from the same set.
%
It is important to note that LAWA, Random, and our proposed method all utilize the final 1000 checkpoints from the pre-training process to compute performance metrics without further retraining the model.
In contrast, the SGD, SWA, and EMA baselines report their final performance directly, as their evaluation pipelines and corresponding techniques are integrated into their respective training processes. This ensures a fair and consistent comparison across all methods.

\begin{figure}
    \centering
    \includegraphics[width=0.9\linewidth]{figs/Behavior_Clone.pdf}
    \vspace{-.4cm}
    \caption{Comparison of different methods on the D4RL benchmark. Each data point represents the average cumulative reward across multiple tasks, averaged over 3 random seeds and 20 trajectories per seed. Detailed results are provided in Appendix \ref{sec:ExpDetail}.}
    \label{fig:BC}
\end{figure}

\begin{figure*}[t!]
    \centering
    \subfigure{
    \centering
    \includegraphics[width=0.3\textwidth]{figs/Cifar100-ResNet3-k10.pdf}}
    \centering
    \subfigure{
    \centering
    \includegraphics[width=0.3\textwidth]{figs/Cifar100-ResNet3-k20.pdf}}
    \subfigure{
    \centering
    \includegraphics[width=0.3\textwidth]{figs/Cifar100-ResNet3-k50.pdf}}
    %\hspace{2mm}
    \vspace{-0.2cm}
    \caption{From left to right, the figures illustrate the impact of the hyperparameter $K$ on the CIFAR-100 task. Each point corresponds to intervals of 100 checkpoints, with $K$ checkpoints selected and averaged from these intervals using different strategies.
    %
    }
    \label{fig:cifar100}
\end{figure*}

\begin{figure*}[t!]
    \centering
    \subfigure{
    \centering
    \includegraphics[width=0.3\textwidth]{figs/NLP-layer2-k10.pdf}}
    \centering
    \subfigure{
    \centering
    \includegraphics[width=0.3\textwidth]{figs/NLP-layer2-k20.pdf}}
    \subfigure{
    \centering
    \includegraphics[width=0.3\textwidth]{figs/NLP-layer2-k50.pdf}}
    %\hspace{2mm}
    \vspace{-0.2cm}
    \caption{From left to right, the figures illustrate the impact of the hyperparameter $K$ on the AG News corpus. Each point corresponds to intervals of 100 checkpoints, with $K$ checkpoints selected and averaged from these intervals using different strategies.
    %
    }
    \label{fig:ag}
\end{figure*}

\textbf{Results.} 
As shown in Figure \ref{fig:BC}, all baselines demonstrate superior performance compared to the original SGD optimizer, highlighting the effectiveness of weight averaging strategies in improving model performance. These results confirm that weight averaging can serve as a valuable technique for stabilizing and enhancing model training outcomes. 
Additionally, our analysis reveals that increasing the number of checkpoints $K$ used for averaging consistently improves performance across all methods. However, this improvement tends to plateau beyond a certain threshold, indicating diminishing returns as the number of averaged checkpoints increases.
%
Most significantly, our proposed method consistently outperforms all other baselines across all experimental settings. Remarkably, even with only $K=10$ checkpoints used for averaging, our method achieves superior results compared to competing approaches that utilize $K=100$ checkpoints. This highlights our approach's efficiency and robustness, as it can deliver significant improvements with a substantially smaller computational footprint. These results demonstrate the scalability and practicality of our method in scenarios where resource efficiency is critical.


\subsection{Image Classification}
\textbf{Experimental Setups.} 
To evaluate our method's performance in image classification, we utilize the CIFAR-100 dataset and the ResNet architecture. With its diverse set of 100 classes, the CIFAR-100 dataset provides a challenging benchmark for image classification tasks. We use classification accuracy on the test dataset as the primary performance metric.
%
In our experiments, we utilize intermediate model checkpoints saved during the final stage of training, specifically after 10,000 training steps. 
Performance is evaluated at intervals of 100 checkpoints, with the number of checkpoints included in the averaging procedure within each interval controlled by the hyperparameter $K$.
This flexibility allows us to adjust the extent of checkpoint aggregation and analyze its impact comprehensively.
By evaluating our method under varying levels of checkpoint averaging, derived from different intervals of the training process, this setup facilitates a robust assessment of its effectiveness across diverse configurations. 

% \textbf{Baselines.} 
% In this experimental setting, we compare our proposed method against three baselines: LAWA \citep{sanyal2023early}, the Random baseline, and the original trained checkpoints (without averaging). These baselines allow us to assess the performance improvements achieved by different checkpoint averaging strategies over a given range of checkpoints. 
% This comparison highlights the effectiveness of our method in enhancing model performance through efficient checkpoint utilization and averaging.

\textbf{Results.}
As illustrated in Figure \ref{fig:cifar100}, all baselines outperform the original SGD optimizer, underscoring the effectiveness of weight averaging in enhancing model performance. Additionally, weight averaging accelerates model convergence, with all baselines reaching performance levels that SGD requires $17$ steps to achieve.
Our SeWA method consistently delivers the best performance, demonstrating its effectiveness. Beyond $17$ steps, where the model approaches convergence, further improvement becomes minimal, as the checkpoints at this stage share highly similar weights.

\subsection{Text Classification}

\textbf{Experimental Setups.}
For the text classification task, we utilize the AG News corpus, a widely used benchmark dataset containing news articles categorized into four distinct classes. The classification is performed using a transformer-based architecture, which is known for its effectiveness in handling natural language processing tasks.
%
To preprocess the dataset, we tokenize the entire corpus using the \textit{basic\_english} tokenizer. Any words not found in the vocabulary are replaced with a special token, \textit{UNK}, to handle out-of-vocabulary terms. This preprocessing ensures that the dataset is standardized and ready for training.
%
We save intermediate model checkpoints throughout the training process, starting from the initial stages. From this set of checkpoints, we systematically select every 100th checkpoint for inclusion in the averaging process. The hyperparameter $K$ controls the total number of checkpoints used for averaging, allowing flexible experimentation with different levels of checkpoint aggregation. This design enables us to thoroughly evaluate the impact of checkpoint averaging on the model’s performance.

% \textbf{Baselines.}
% In this experimental setting, we compare our proposed method against three baselines: LAWA \citep{sanyal2023early}, the Random baseline, and the original trained checkpoints without any averaging. These baselines provide a robust framework for assessing the performance improvements achieved through various checkpoint averaging strategies.

\textbf{Results.}
As shown in Figure \ref{fig:ag}, the improvement of weight averaging over the SGD baseline is minimal for relatively simple tasks, primarily serving to stabilize training. However, our SeWA method consistently achieves the best results regardless of task complexity, demonstrating its broad applicability across diverse settings.
\section{Conclusion}
\label{sec:Con}
We propose a new algorithm \method{} for adaptive selecting checkpoints to average, which balances generalization performance and convergence speed. Under different function assumptions, we derive its generalization bounds, exhibiting superior results compared to other algorithms. In practical implementation, we employ probabilistic reparameterization to transform the discrete optimization problem into a continuous objective solvable by gradient-based methods. Empirically, we verify that our approach can help to obtain good performance for unstable training processes, and a few checkpoints selected by \method{} can achieve results due to other algorithms using several times as many points. 

 \section*{Impact Statement} The \method{} enhances both the model's generalization ability and convergence speed. Our approach can be further combined with the training process to improve training stability.

% Acknowledgements should only appear in the accepted version.
% \section*{Acknowledgements}

% \textbf{Do not} include acknowledgements in the initial version of
% the paper submitted for blind review.

% If a paper is accepted, the final camera-ready version can (and
% usually should) include acknowledgements.  Such acknowledgements
% should be placed at the end of the section, in an unnumbered section
% that does not count towards the paper page limit. Typically, this will 
% include thanks to reviewers who gave useful comments, to colleagues 
% who contributed to the ideas, and to funding agencies and corporate 
% sponsors that provided financial support.

% \section*{Impact Statement}

% Authors are \textbf{required} to include a statement of the potential 
% broader impact of their work, including its ethical aspects and future 
% societal consequences. This statement should be in an unnumbered 
% section at the end of the paper (co-located with Acknowledgements -- 
% the two may appear in either order, but both must be before References), 
% and does not count toward the paper page limit. In many cases, where 
% the ethical impacts and expected societal implications are those that 
% are well established when advancing the field of Machine Learning, 
% substantial discussion is not required, and a simple statement such 
% as the following will suffice:

% ``This paper presents work whose goal is to advance the field of 
% Machine Learning. There are many potential societal consequences 
% of our work, none which we feel must be specifically highlighted here.''

% The above statement can be used verbatim in such cases, but we 
% encourage authors to think about whether there is content which does 
% warrant further discussion, as this statement will be apparent if the 
% paper is later flagged for ethics review.


\bibliography{example_paper}
\bibliographystyle{icml2025}


%%%%%%%%%%%%%%%%%%%%%%%%%%%%%%%%%%%%%%%%%%%%%%%%%%%%%%%%%%%%%%%%%%%%%%%%%%%%%%%
%%%%%%%%%%%%%%%%%%%%%%%%%%%%%%%%%%%%%%%%%%%%%%%%%%%%%%%%%%%%%%%%%%%%%%%%%%%%%%%
% APPENDIX
%%%%%%%%%%%%%%%%%%%%%%%%%%%%%%%%%%%%%%%%%%%%%%%%%%%%%%%%%%%%%%%%%%%%%%%%%%%%%%%
%%%%%%%%%%%%%%%%%%%%%%%%%%%%%%%%%%%%%%%%%%%%%%%%%%%%%%%%%%%%%%%%%%%%%%%%%%%%%%%
\newpage
\appendix
\onecolumn

\section{Experiment Details}
\label{sec:ExpDetail}

\subsection{Behavior Cloning}

\textbf{Network Architecture.} The network architecture comprises four layers, each consisting of a sequence of ReLU activation, Dropout for regularization, and a Linear transformation. The final layer includes an additional Tanh activation function to enhance the representation and capture non-linearities in the output.

\textbf{Results.} 
Comprehensive results for each task across all datasets are presented in Table \ref{tab:BC}. Our evaluation focuses specifically on the medium and medium-expert datasets, which offer a balanced mix of trajectories with varying performance levels. This selection enables a thorough assessment of our method's ability to generalize across different reward distributions.
For clarity and ease of comparison, the main paper emphasizes the average performance across tasks, as illustrated in Figure \ref{fig:BC}. 
%This dual presentation ensures a detailed examination of individual tasks while providing an accessible overview of overall performance.

\begin{table*}[t!]
\centering
\caption{
Performance comparison of various methods on D4RL Gym tasks. The left panel shows results obtained using the final checkpoint under different update strategies, while the right panel presents results from averaged checkpoints collected during the final training stage with SGD, using different selection strategies. 
Each result is evaluated as the mean of 60 random rollouts, based on 3 independently trained models with 20 trajectories per model.
}
\label{tab:BC}
\scalebox{0.9}{
\begin{tabular}{c|cc|ccc|ccc}
\toprule[2pt]
\multicolumn{1}{l|}{} & Task & Dataset & SGD & SWA & EMA & LAWA & Random & \method{} (Ours) \\ \midrule
\multirow{7}{*}{K=10} & Hopper & medium & 1245.039 & 1279.249 & 1297.270 & 1289.515 & 1291.478 & \textbf{1324.848} \\
 & Hopper & medium-expert & 1460.785 & 1468.893 & 1320.408 & 1462.452 & 1451.015 & \textbf{1509.317} \\
 & Walker2d & medium & 3290.248 & 3328.121 & 3341.888 & 3341.437 & 3306.763 & \textbf{3371.202} \\
 & Walker2d & medium-expert & 3458.693 & 3546.008 & 3681.504 & 3634.373 & 3609.611 & \textbf{3679.806} \\
 & Halfcheetah & medium & 4850.490 & 4858.224 & 4894.204 & 5012.389 & 4896.104 & \textbf{5041.369} \\
 & Halfcheetah & medium-expert & 5015.689  & 4974.923 & 4857.562 & 4989.329 & 4962.719 & \textbf{5082.902} \\ \cmidrule{2-9}
 & \multicolumn{2}{c|}{Average} & 3220.157 & 3242.570 & 3232.139 & 3288.249 & 3252.948 & \textbf{3334.907} \\ \midrule
\multirow{7}{*}{K=20} & Hopper & medium & 1245.039 & 1281.910 & 1302.400 & 1310.875 & 1312.166 & \textbf{1361.202} \\
 & Hopper & medium-expert & 1460.785 & 1427.47 & 1373.268 & 1563.307 & 1482.012 & \textbf{1571.127} \\
 & Walker2d & medium & 3290.248 & 3308.464 & \textbf{3420.257} & 3325.873 & 3324.557 & 3364.886 \\
 & Walker2d & medium-expert & 3458.693 & 3588.176 & 3667.809 & 3557.925 & 3650.846 & \textbf{3673.804} \\
 & Halfcheetah & medium & 4850.490 & 4913.549 & 4848.006 & 4974.041 & 4924.613 & \textbf{5071.051} \\
 & Halfcheetah & medium-expert & 5015.689 & 5024.723 & 4957.194 & 4993.524 & 4988.816 & \textbf{5085.628} \\ \cmidrule{2-9}
 & \multicolumn{2}{c|}{Average} & 3220.157 & 3257.382 & 3261.489 & 3287.591 & 3280.502 & \textbf{3354.616} \\ \midrule
\multirow{7}{*}{K=50} & Hopper & medium & 1245.039 & 1294.884 & 1329.863 & 1336.33 & 1319.571 & \textbf{1389.280} \\
 & Hopper & medium-expert & 1460.785 & 1477.466 & 1485.696 & 1537.672 & 1496.045 & \textbf{1616.116} \\
 & Walker2d & medium & 3290.248 & 3262.046 & 3341.767 & 3253.695 & 3352.12 & \textbf{3392.130} \\
 & Walker2d & medium-expert & 3458.693 & 3577.509 & 3591.081 & 3584.468 & 3659.789 & \textbf{3672.560} \\
 & Halfcheetah & medium & 4850.490 & 4927.951 & 4968.048 & 5022.097 & 5000.004 & \textbf{5035.631} \\
 & Halfcheetah & medium-expert & 5015.689  & 5061.688 & \textbf{5075.426} & 5011.232 & 4960.585 & 5044.886 \\ \cmidrule{2-9}
 & \multicolumn{2}{c|}{Average} & 3220.157 & 3280.833 & 3298.647 & 3290.916 & 3298.019 & \textbf{3358.434} \\ \midrule
\multirow{7}{*}{K=100} & Hopper & medium & 1245.039 & 1347.267 & 1322.625 & 1320.652 & 1319.727 & \textbf{1393.981} \\
 & Hopper & medium-expert & 1460.785 & 1527.206 & 1528.265 & 1496.266 & 1491.196 & \textbf{1568.025} \\
 & Walker2d & medium & 3290.248 & 3324.218 & 3393.646 & 3345.913 & 3321.046 & \textbf{3424.078} \\
 & Walker2d & medium-expert & 3458.693 & 3575.621 & 3629.308 & 3613.274 & 3587.211 & \textbf{3710.347} \\
 & Halfcheetah & medium & 4850.490 & 4939.629 & 4871.376 & 4974.220 & 5015.349 & \textbf{5021.948} \\
 & Halfcheetah & medium-expert & 5015.689 & 4919.624 & 5047.757 & 4991.007 & 5031.975 & \textbf{5063.546} \\ \cmidrule{2-9}
 & \multicolumn{2}{c|}{Average} & 3220.157 & 3272.261 & 3298.830 & 3290.222 & 3294.417 & \textbf{3363.654} \\ 
 \bottomrule
\end{tabular}
}
\end{table*}

\subsection{Image Classification}

\textbf{Network Architecture.} The network architecture consists of three primary blocks, followed by an average pooling layer and a linear layer for generating the final output. Each block contains two convolutional layers, each accompanied by a corresponding batch normalization layer to improve training stability and convergence. To address potential issues of vanishing gradients, each block includes a shortcut connection that facilitates efficient gradient flow during backpropagation. The output of each block is passed through a ReLU activation function to introduce non-linearity, enabling the network to learn complex representations effectively.

\textbf{Results.} In addition to the results presented in Figure \ref{fig:cifar100}, we provide further analysis examining the impact of network parameter variations to demonstrate the robustness of our method across networks of different sizes. These results, shown in Figure \ref{fig:cifar100-layers}, illustrate that as the number of layers or blocks increases, the performance of SGD improves, following a similar training curve.

Notably, weight averaging consistently outperforms SGD during the upward phase of training. The performance gains from weight averaging become more pronounced as the network size increases, highlighting its potential in scaling effectively to larger models. This highlights the potential of weight averaging to enhance the performance of larger models. 
%
Furthermore, regardless of changes in network parameters, our proposed method consistently achieves superior results, demonstrating its adaptability and effectiveness across varying network configurations. These findings emphasize the potential of weight averaging as a robust and scalable technique for optimizing model performance.

\begin{figure*}[t!]
    \centering
    \subfigure{
    \centering
    \includegraphics[width=0.3\textwidth]{figs/Cifar100-ResNet1-k10.pdf}}
    \centering
    \subfigure{
    \centering
    \includegraphics[width=0.3\textwidth]{figs/Cifar100-ResNet1-k20.pdf}}
    \subfigure{
    \centering
    \includegraphics[width=0.3\textwidth]{figs/Cifar100-ResNet1-k50.pdf}}
    \subfigure{
    \centering
    \includegraphics[width=0.3\textwidth]{figs/Cifar100-ResNet3-k10.pdf}}
    \centering
    \subfigure{
    \centering
    \includegraphics[width=0.3\textwidth]{figs/Cifar100-ResNet3-k20.pdf}}
    \subfigure{
    \centering
    \includegraphics[width=0.3\textwidth]{figs/Cifar100-ResNet3-k50.pdf}}
    \subfigure{
    \centering
    \includegraphics[width=0.3\textwidth]{figs/Cifar100-ResNet5-k10.pdf}}
    \centering
    \subfigure{
    \centering
    \includegraphics[width=0.3\textwidth]{figs/Cifar100-ResNet5-k20.pdf}}
    \subfigure{
    \centering
    \includegraphics[width=0.3\textwidth]{figs/Cifar100-ResNet5-k50.pdf}}
    %\hspace{2mm}
    \vspace{-0.2cm}
    \caption{From left to right, the figures illustrate the impact of the hyperparameter $K$ on the CIFAR-100 task. 
    Each data point represents performance based on intervals of 100 checkpoints, with $K$ checkpoints selected from these intervals using various strategies.
    The first row corresponds to a network architecture with 1 block, the second row represents a network with 3 blocks, and the third row depicts results for a network with 5 blocks.
    %
    }
    \label{fig:cifar100-layers}
\end{figure*}


\subsection{Text Classification}

\textbf{Network Architectures.} The network architecture comprises two embedding layers followed by two layers of \textit{TransformerEncoderLayer}. Each \textit{TransformerEncoderLayer} includes a multi-head self-attention mechanism and a position-wise feedforward network, along with layer normalization and residual connections to enhance training stability and gradient flow. The output from the Transformer layers is passed through a linear layer to produce the final predictions.

\textbf{Results.} In addition to the findings presented in Figure \ref{fig:ag}, we conduct further analysis to evaluate the impact of network parameter variations, demonstrating the robustness of our method across networks of varying sizes. These additional results, shown in Figure \ref{fig:ag-layers}, indicate that as the number of Transformer layers increases, the performance of SGD improves up to a certain point. However, beyond this range - where two layers appear sufficient - performance begins to exhibit fluctuations, suggesting diminishing returns and instability with additional layers.

While the improvement achieved by weight averaging is relatively modest due to the simplicity of the task, it still plays a critical role in stabilizing the training process and reducing fluctuations in the training curve. Among the averaging methods evaluated, our proposed method consistently achieves the best performance, underscoring its effectiveness in maintaining stability and optimizing performance, even in scenarios where task complexity is low.

\begin{figure*}[t!]
    \centering
    \subfigure{
    \centering
    \includegraphics[width=0.3\textwidth]{figs/NLP-layer1-k10.pdf}}
    \centering
    \subfigure{
    \centering
    \includegraphics[width=0.3\textwidth]{figs/NLP-layer1-k20.pdf}}
    \subfigure{
    \centering
    \includegraphics[width=0.3\textwidth]{figs/NLP-layer1-k50.pdf}}
    \subfigure{
    \centering
    \includegraphics[width=0.3\textwidth]{figs/NLP-layer2-k10.pdf}}
    \centering
    \subfigure{
    \centering
    \includegraphics[width=0.3\textwidth]{figs/NLP-layer2-k20.pdf}}
    \subfigure{
    \centering
    \includegraphics[width=0.3\textwidth]{figs/NLP-layer2-k50.pdf}}
    \subfigure{
    \centering
    \includegraphics[width=0.3\textwidth]{figs/NLP-layer4-k10.pdf}}
    \centering
    \subfigure{
    \centering
    \includegraphics[width=0.3\textwidth]{figs/NLP-layer4-k20.pdf}}
    \subfigure{
    \centering
    \includegraphics[width=0.3\textwidth]{figs/NLP-layer4-k50.pdf}}
    %\hspace{2mm}
    \vspace{-0.2cm}
    \caption{From left to right, the figures illustrate the impact of the hyperparameter $K$ on the AG News corpus. Each point corresponds to intervals of 100 checkpoints, with $K$ checkpoints selected from these intervals using different strategies.
    The first row corresponds to a network architecture with a single \textit{TransformerEncoderLayer}, the second row represents a network with three \textit{TransformerEncoderLayer}s, and the third row shows results for a network with five \textit{TransformerEncoderLayer}s.
    }
    \label{fig:ag-layers}
\end{figure*}

\section{Proof of Lemma \ref{lemma}}\label{pro-lemma}  \paragraph{\boldmath$(1+\alpha\beta)$-expansive.} According to triangle inequality and $\beta$-smoothness,
\begin{equation}\label{eq:app2.1}
     \begin{aligned}
      \Vert w_{T+1} - w_{T+1}^{\prime}\Vert &\leq \Vert w_T- w_T^{\prime}\Vert + \alpha\Vert \nabla F(w_T) -\nabla F(w_T^{\prime})\Vert \\
      &\leq \Vert w_T- w_T^{\prime}\Vert + \alpha\beta \Vert w_T- w_T^{\prime}\Vert \\
      &= (1+\alpha\beta)\Vert w_T- w_T^{\prime}\Vert .
     \end{aligned}
\end{equation}

\paragraph{\emph{Non}-expansive.} Function is convexity and $\beta$-smoothness that implies 
\begin{equation}\label{eq:app2.2}
     \begin{aligned}
      \langle \nabla F(w) -\nabla F(v), w - v \rangle \geq \frac{1}{\beta} \Vert \nabla F(w) -\nabla F(v)\Vert^2 .
     \end{aligned}
\end{equation}
We conclude that
\begin{equation}\label{eq:app2.3}
     \begin{aligned}
      \Vert w_{T+1} - w_{T+1}^{\prime}\Vert &= \sqrt{\Vert w_{T} - \alpha \nabla F(w_{T}) - w_{T}^{\prime} + \alpha \nabla F(w_{T}^{\prime})\Vert^2}  \\
      &=\sqrt{\Vert w_{T} - w_{T}^{\prime} \Vert^2 - 2\alpha\langle \nabla F(w_{T}) -\nabla F(w_{T}^{\prime}), w_T- w_T^{\prime} \rangle +\alpha^2 \Vert \nabla F(w_{T}) - \nabla F(w_{T}^{\prime})\Vert^2} \\
      &\leq \sqrt{\Vert w_T- w_T^{\prime}\Vert^2 - \left(\frac{2\alpha}{\beta} -\alpha^2 \right) \Vert \nabla F(w_{T}) -\nabla F(w_{T}^{\prime})\Vert^2} \\
      &\leq \Vert w_T- w_T^{\prime}\Vert.
     \end{aligned}
\end{equation}

\section{Proof of the generalization bounds}\label{pro-con}
By the Lemma \ref{convex-basic} and \ref{nonconvex-basic}, the proof of Theorem \ref{thm:stability-conv} and \ref{thm:stability-non-with} can be further decomposed into bounding the difference of the parameters for the last $k$ points of the average algorithm. We provide the proof as follows. And you can also find it in \cite{peng2020dfwa}.

\subsection{Update rules of the last $k$ points of the averaging algorithm.}
For the last $k$ points of the averaging algorithm, we formulate it as
\begin{equation}\label{FWA-rules}
    \hat{w}^{k}_{T}=\frac{1}{k} \sum_{i=T-k+1}^{T} w_{i}.
\end{equation}
It is not difficult to find the relationship between $\bar{w}^{k}_{T}$ and $\bar{w}^{k}_{T-1}$, i.e.,
\begin{equation}\label{pro-FWA-update}
    \hat{w}^{k}_{T} = \hat{w}^{k}_{T-1} + \frac{1}{k}\left(w_{T} - w_{T-k}\right) = \hat{w}^{k}_{T-1} - \frac{1}{k}\sum_{i=T-k+1}^{T} \alpha_i\nabla F(w_{i-1},z_i),
\end{equation}
where the second equality follows from the update of SGD.



\subsection{\textbf{Proof. Theorem \ref{thm:stability-conv}}}\label{proof-thm-con-with}
  First, using the relationship between $\hat{w}^{k}_{T}$ and $\hat{w}^{k}_{T-1}$ in Eq. ~\eqref{pro-FWA-update}, we consider that the different sample $z_{T}$ and $z_{T}^{\prime}$ are selected to update with probability $\frac{1}{n}$ at the step $T$.  
\begin{equation}
  \begin{aligned}
   \bar{\delta}_{T} &= \bar{\delta}_{T-1} + \frac{1}{k}\sum_{i=T-k+1}^{T} \alpha_i \Vert\nabla F(w^{\prime}_{i-1},z_i) - \nabla F(w_{i-1},z_i) \Vert \\
   &\leq \bar{\delta}_{T-1} + \frac{2\alpha_T L}{k} + \frac{1}{k}\sum_{i=T-k+1}^{T-1} \alpha_i \Vert\nabla F(w^{\prime}_{i-1},z_i) - \nabla F(w_{i-1},z_i) \Vert ,
  \end{aligned}
 \end{equation}
where the proof follows from the triangle inequality and $L$-Lipschitz condition. For $\frac{1}{k}\sum_{i=T-k+1}^{T-1} \alpha_i \Vert\nabla F(w^{\prime}_{i-1},z_i) - \nabla F(w_{i-1},z_i) \Vert$ will be controlled in the late.

Second, another situation need be considered in case of the same sample are selected$(z_{T}=z_{T}^{\prime})$ to update with probability $1-\frac{1}{n}$ at the step $T$. 
\begin{equation}
  \begin{aligned}
   \bar{\delta}_{T} &= \bar{\delta}_{T-1} + \frac{1}{k}\sum_{i=T-k+1}^{T} \alpha_i \Vert\nabla F(w^{\prime}_{i-1},z_i) - \nabla F(w_{i-1},z_i) \Vert \\
   &\leq \bar{\delta}_{T-1} + \frac{1}{k}\sum_{i=T-k+1}^{T-1} \alpha_i\Vert\nabla F(w^{\prime}_{i-1},z_i) - \nabla F(w_{i-1},z_i) \Vert ,
  \end{aligned}
 \end{equation}
where $\Vert\nabla F(w^{\prime}_{T-1},z_T)-\nabla F(w_{T-1},z_T)\Vert=0$ in the second inequality because the non-expansive property of convex function.

For each $\Vert\nabla F(w^{\prime}_{i-1},z_i)-\nabla F(w_{i-1},z_i)\Vert$ in the sense of expectation, We consider two situations using $\alpha L$ bound and the non-expansive property. Then, we get  
  \begin{equation}
    \frac{1}{k}\sum_{i=T-k+1}^{T-1}\alpha_i \Vert\nabla F(w^{\prime}_{i-1},z_i) - \nabla F(w_{i-1},z_i) \Vert \leq \frac{2L}{nk}\sum_{i=T-k+1}^{T-1}\alpha_i.
 \end{equation}

Then we obtain the expectation based on the above analysis 
  \begin{equation}
  \begin{aligned}
    \mathbb{E}\left[\bar{\delta}_{T}\right] &\leq (1-\frac{1}{n})\bar{\delta}_{T-1} + \frac{1}{n}\left(\bar{\delta}_{T-1}+\frac{2\alpha_T L}{k}\right) + \frac{2L}{nk}\sum_{i=T-k+1}^{T-1}\alpha_i\\
    &\leq \mathbb{E}\left[\bar{\delta}_{T-1}\right] + \frac{2L}{nk}\sum_{i=T-k+1}^{T}\alpha_i
  \end{aligned}
 \end{equation}
recursively, we can get 
    \begin{equation}
     \begin{aligned}
      \mathbb{E}\left[\bar{\delta}_{T}\right]&\leq \frac{2L}{nk} \left( \sum_{i=T-k+1}^{T}\alpha_i + \sum_{i=T-k}^{T-1}\alpha_i + \cdots + \sum_{i=1}^{k}\alpha_i \right) \\ & + \frac{2L}{nk} \left( \sum_{i=1}^{k-1}\alpha_i + \sum_{i=1}^{k-2}\alpha_i + \cdots + \sum_{i=1}^{1}\alpha_i \right). \\
      %&\leq \frac{2L}{n} \left( \sum_{i=E-k}^{E} \sum_{j=0}^{d} \frac{(E+1-i)\alpha_{i,j}}{k+1} +  \sum_{i=1}^{E-k-1} \sum_{j=0}^{d} \alpha_{i,j}\right).
     \end{aligned}
    \end{equation}
Let $\alpha_{i,j}=\alpha$, we get
    \begin{equation}
     \begin{aligned}
      \mathbb{E}\left[\bar{\delta}_{T}\right] = \frac{2\alpha L}{n} \left( T - \frac{k}{2} \right).
     \end{aligned}
    \end{equation}
Plugging this back into Eq.~\eqref{convex-basic}, we obtain
 \begin{equation}\label{eq:2.2.1}
  \epsilon_{gen} = \mathbb{E}\vert F(\bar{w}_T^K;z)-F(\bar{w}^{K\prime}_T;z)\vert \leq \frac{2\alpha L^2 s}{n} \left( T - \frac{k}{2} \right).
 \end{equation}
And we finish the proof.

\subsection{Proof of Lemma \ref{nonconvex-basic}}\label{proof-noncon-basic}
We consider that $S$ and $S^\prime$ are two samples of size $n$ differing in only a single example. Let $\xi$ denote the event $\bar{\delta}_{t_0}=0$. Let $z$ be an arbitrary example and consider the random variable $I$ assuming the index of the first time step using the different sample. then we have
    \begin{equation}
     \begin{aligned}
      \mathbb{E}\vert \nabla F(\bar{w}_T^{K};z)-\nabla F(\bar{w}^{K\prime}_T;z)\vert &= P\left\lbrace \xi\right\rbrace \mathbb{E}[\vert \nabla F(\bar{w}_T^{K};z)-\nabla F(\bar{w}^{K\prime}_T;z)\vert|\xi]\\
      &+P\left\lbrace \xi^{c}\right\rbrace E[\vert \nabla F(\bar{w}_T^{K};z)-\nabla F(\bar{w}^{K\prime}_T;z)\vert |\xi^{c}]\\
      &\leq P\left\lbrace I\geq t_0\right\rbrace \cdot \mathbb{E}[\vert \nabla F(\bar{w}_T^{K};z)-\nabla F(\bar{w}^{K\prime}_T;z)\vert |\xi] \\
      &+P\left\lbrace I\leq t_0\right\rbrace \cdot \mathop{sup}_{\bar{w}^{K},z} F(\bar{w}^{K};z),\\
     \end{aligned}
    \end{equation}
where $\xi^{c}$ denotes the complement of $\xi$.   

Note that Note that when $I\geq t_0$, then we must have that $\bar{\delta}_{t_0}=0$, since the execution on $S$ and $S^{\prime}$ is identical until step $t_0$. We can get $LE[\Vert\bar{w}_{T}^{K} - \bar{w}_{T}^{K\prime}\Vert|\xi]$ combined the Lipschitz continuity of $F$. Furthermore, we know $P\left\lbrace \xi^{c}\right\rbrace=P\left\lbrace \bar{\delta}_{t_0}=0\right\rbrace\leq P\left\lbrace I\leq t_0\right\rbrace$, for the random selection rule, we have 
    \begin{equation}
     \begin{aligned}
      P\left\lbrace I\leq t_0\right\rbrace \leq \sum_{t=1}^{t_0} P\left\lbrace I = t_0\right\rbrace = \frac{t_0}{n}.
     \end{aligned}
    \end{equation}
We can combine the above two parts and $F \in [0,1]$ to derive 
the stated bound $L\mathbb{E}[\Vert\bar{w}_{T}^{k} - \bar{w}_{T}^{k\prime}\Vert\vert\xi]+\frac{t_0}{n}$, which completes the proof.

\subsection{Lemma \ref{Lemma_noncon} and it's proof}\label{proof-Lemma_noncon}
\begin{lemma}\label{Lemma_noncon}
Assume that $F$ is $\beta$-smooth and $non$-convex. Let $\alpha = \frac{c}{T}$, we have 
  \begin{equation}
  \begin{aligned}
   \Vert w^{\prime}_{T}& - w_{T} \Vert \leq &e^\frac{c\beta k}{T}\bar{\delta}_{T},
  \end{aligned}
 \end{equation}
\end{lemma}
where $\bar{\delta}_{T}= \frac{1}{k} \sum_{i=T-k+1}^{T}\Vert w^{\prime}_{i} - w_{i} \Vert$.

\textbf{proof Lemma \ref{Lemma_noncon}.} 
By triangle inequality and our assumption that $F$ satisfies, we have
 \begin{equation}
  \begin{aligned}
   \Vert w^{\prime}_{T}& - w_{T} \Vert = \frac{1}{k} \cdot k \cdot \Vert w^{\prime}_{T} - w_{T} \Vert \\
   \leq & \frac{1}{k} ( \Vert w^{\prime}_{T} - w_{T} \Vert + (1+\alpha_{T-1}\beta)\Vert w^{\prime}_{T-1} - w_{T-1} \Vert + \cdots + \\&(1+\alpha_{T-1}\beta)(1+\alpha_{T-2}\beta)\cdots(1+\alpha_{T-k+1}\beta)\Vert w^{\prime}_{T-k+1} - w_{T-k+1} \Vert ) \\
   \leq & \prod_{t=T-k+1}^{T} (1+\alpha_t\beta)\left(\frac{1}{k} \sum_{i=T-k+1}^{T}\Vert w^{\prime}_{i} - w_{i} \Vert\right).
    \end{aligned}
 \end{equation}
Let $\alpha_t = \alpha = \frac{C}{T}$, we have
  \begin{equation}
  \begin{aligned}
   \Vert w^{\prime}_{T}& - w_{T} \Vert \leq &\prod_{t=T-k+1}^{T} (1+\alpha_t\beta)\bar{\delta}_{T} \leq \left(1+ \frac{c\beta}{T}\right)^k\bar{\delta}_{T} \leq e^\frac{c\beta k}{T}\bar{\delta}_{T}.
  \end{aligned}
 \end{equation}


\subsection{\textbf{Proof. Theorem \ref{thm:stability-non-with} (Based on the constant learning rate)}}
\label{proof-thm-non-with} In the case of non-convex, the $(1+\alpha\beta)$-expansive properties and $L$-Lipschitz conditions are used in our proof. Based on the relationship between $\hat{w}^{k}_T$ and $\hat{w}^{k}_{T-1}$ in Eq. ~\eqref{pro-FWA-update}. We consider that the different samples $z_T$ and $z^{\prime}_T$ are selected to update with probability $\frac{1}{n}$ at step T.
\begin{equation}
  \begin{aligned}
   \bar{\delta}_{T} &= \bar{\delta}_{T-1} + \frac{1}{k}\sum_{i=T-k+1}^{T} \alpha \Vert\nabla F(w^{\prime}_{i-1},z_i) - \nabla F(w_{i-1},z_i) \Vert \\
   &\leq \bar{\delta}_{T-1} + \frac{2\alpha L}{k} + \frac{1}{k}\sum_{i=T-k+1}^{T-1} \alpha \Vert\nabla F(w^{\prime}_{i-1},z_i) - \nabla F(w_{i-1},z_i) \Vert ,
  \end{aligned}
 \end{equation}
Next, the same sample $z=z^{\prime}$ is selected to update with probability $1-\frac{1}{n}$ at step T.
\begin{equation}
  \begin{aligned}
   \bar{\delta}_{T} &= \bar{\delta}_{T-1} + \frac{1}{k}\sum_{i=T-k+1}^{T} \alpha \Vert\nabla F(w^{\prime}_{i-1},z_i) - \nabla F(w_{i-1},z_i) \Vert \\
   &\leq \bar{\delta}_{T-1} + \frac{\alpha \beta}{k}\Vert w^{\prime}_{T-1} - w_{T-1} \Vert + \frac{1}{k}\sum_{i=T-k+1}^{T-1} \alpha \Vert\nabla F(w^{\prime}_{i-1},z_i) - \nabla F(w_{i-1},z_i) \Vert \\
   &\leq (1+\frac{\alpha \beta(1+\alpha \beta)^{k-1}}{k})\bar{\delta}_{T-1} + \frac{1}{k}\sum_{i=T-k+1}^{T-1} \alpha \Vert\nabla F(w^{\prime}_{i-1},z_i) - \nabla F(w_{i-1},z_i) \Vert,
  \end{aligned}
 \end{equation}
where the proof follows from the $\beta$-smooth and Lemma \ref{Lemma_noncon}. Then, we bound the $\alpha \Vert\nabla F(w^{\prime}_{T-2},z_{T-1}) - \nabla F(w_{T-2},z_{T-1}) \Vert$ with different sampling. 
 \begin{equation}\label{noncon-sigbound}
  \begin{aligned}    
    \alpha\Vert\nabla &F(w^{\prime}_{T-2},z_{T-1}) - \nabla F(w_{T-2},z_{T-1}) \Vert = \frac{2\alpha L}{n} + \left(1-\frac{1}{n}\right)\alpha\beta\Vert w_{T-2} - w^{\prime}_{T-2} \Vert\\
    &\leq \frac{2\alpha L}{n} + \alpha\beta\left(\Vert w_{T-3} - w^{\prime}_{T-3} \Vert + \alpha \Vert \nabla F(w^{\prime}_{T-3},z_{T-2})-\nabla F^{\prime}(w_{T-3},z_{T-2}) \Vert\right) \\
    &\leq \frac{2\alpha L}{n} + \alpha\beta\left(\frac{2\alpha L}{n} + (1+\alpha\beta)\Vert w_{T-3} - w^{\prime}_{T-3} \Vert\right) \\
    &\cdots\\
    &\leq \frac{2\alpha L}{n}(1+\alpha\beta)^{T-2-t_0} + \alpha\beta(1+\alpha\beta)^{T-2-t_0}\Vert w_{t_{0}} - w^{\prime}_{t_{0}} \Vert =\frac{2\alpha L}{n}(1+\alpha\beta)^{T-2-t_0},
  \end{aligned}
 \end{equation}
where $w_{t_{0}} = w^{\prime}_{t_{0}}$. Therefore, we can obtain the bound for $\frac{1}{k}\sum_{i=T-k+1}^{T-1} \alpha \Vert\nabla F(w^{\prime}_{i-1},z_i) - \nabla F(w_{i-1},z_i) \Vert$ in the expectation sense.
\begin{equation}\label{44}
    \begin{aligned}
     \frac{\alpha}{k}\sum_{i=T-k+1}^{T-1} & \mathbb{E} \Vert\nabla F(w^{\prime}_{i-1},z_i) - \nabla F(w_{i-1},z_i) \Vert 
     \leq \frac{2\alpha L}{nk} \sum_{i=T-k}^{T-2} (1+\alpha\beta)^{i-t_{0}} \\
     &\leq \frac{2\alpha L}{nk} \cdot k (1+\alpha\beta)^{T} \leq \frac{2\alpha L(1+\alpha\beta)^{T}}{n}.  
    \end{aligned}
\end{equation}
Then, we obtain the expectation considering the above analysis 
 \begin{equation}
    \begin{aligned}
     \mathbb{E}\left[\bar{\delta}_{T+1}\right] &\leq (1-\frac{1}{n})\left(1+\frac{\alpha \beta(1+\alpha \beta)^{k-1}}{k}\right)\bar{\delta}_T + \frac{1}{n}\left(\bar{\delta}_T+\frac{2\alpha L}{k}\right) + \frac{2\alpha L(1+\alpha\beta)^{T}}{n}\\ 
     &\leq \left(\frac{1}{n}+(1-\frac{1}{n})\left(1+\frac{\alpha \beta(1+\alpha \beta)^{k-1}}{k}\right)\right)\bar{\delta}_{T} + \frac{2\alpha L}{nk}\left(1+k(1+\alpha\beta)^{T}\right)\\
    \end{aligned}
   \end{equation}
let $\alpha = \frac{c}{t}$, then
  \begin{equation}
 \begin{aligned}
     &= \left(1+(1-\frac{1}{n})\frac{c\beta(1+\frac{c\beta}{t})^{k}}{kt}\right) \bar{\delta}_{t} + \frac{2cL}{nkt}\left(1+k(1+\frac{c\beta}{t})^{t}\right)\\
     &\leq \exp\left((1-\frac{1}{n})\frac{c\beta e^{\frac{c\beta k}{t}}}{kt}\right) \bar{\delta}_{t} + \frac{2cL}{kn}\cdot\frac{1+k e^{c\beta}}{t}.
    \end{aligned}
   \end{equation}
Here we used that $\lim\limits_{x\to\infty}(1+\frac{1}{x})^x=e$ and $\lim\limits_{x\to\infty}e^\frac{1}{x}=1$. 
Using the fact that $\bar{\delta}_{t_0}=0$, we can unwind this recurrence relation from $T$ down to $t_0+1$.
  \begin{equation}
    \begin{aligned}
     \mathbb{E}\bar{\delta}_{t} &\leq \sum_{t=t_0 +1}^{T} \left( \prod_{m=t+1}^{T}\exp\left((1-\frac{1}{n})\frac{c\beta}{km}\right)\right)\frac{2cL}{kn}\cdot\frac{1+k e^{c\beta}}{t}\\
     &= \sum_{t=t_0 +1}^{T} \exp\left(\frac{(1-\frac{1}{n})c\beta}{k} \sum_{m=t+1}^{T}\frac{1}{m}\right)\frac{2cL}{kn}\cdot\frac{1+k e^{c\beta}}{t}\\
     &\leq \sum_{t=t_0 +1}^{T} \exp\left( \frac{(1-\frac{1}{n})c\beta}{k} \cdot \log(\frac{T}{t}) \right)\frac{2cL}{kn}\cdot\frac{1+k e^{c\beta}}{t}\\
     &\leq T^{\frac{(1-\frac{1}{n})c\beta}{k}} \cdot \sum_{t=t_0 +1}^{T} \left(\frac{1}{t}\right)^{\frac{(1-\frac{1}{n})c\beta}{k}+1} \cdot \frac{2cL(1+ke^{c\beta})}{kn}\\
     &\leq \frac{k}{(1-\frac{1}{n})c\beta} \cdot \frac{2cL(1+ke^{c\beta})}{kn} \cdot \left(\frac{T}{t_0}\right)^{\frac{(1-\frac{1}{n})c\beta}{k}}\\
     &\leq \frac{2L(1+ke^{c\beta})}{(n-1)\beta} \cdot \left(\frac{T}{t_0}\right)^{\frac{c\beta}{k}}.
    \end{aligned}
   \end{equation}
Plugging this back into Eq.~\eqref{nonconvex-basic}, we obtain
 \begin{equation}\label{with-con}
  \mathbb{E}\vert F(\bar{w}_T^{K};z)-F(\bar{w}^{K\prime}_T;z)\vert \leq \frac{t_0}{n} + \frac{2sL^2(1+ke^{c\beta})}{(n-1)\beta} \cdot \left(\frac{T}{t_0}\right)^{\frac{c\beta}{k}}.
 \end{equation}
By taking the extremum, we obtain the minimum  
 \begin{equation}\label{with-con-t_0}
    t_0 = \left(2csL^2(1+ke^{c\beta})k^{-1}\right)^{\frac{k}{c\beta+k}}\cdot T^{\frac{c\beta}{c\beta+k}}
   \end{equation}
finally, this setting gets
 \begin{equation}\label{with-con-result}
  \epsilon_{gen} = \mathbb{E}\vert F(\bar{w}_T^{K};z)-F(\bar{w}^{K\prime}_T;z)\vert \leq \frac{1+\frac{1}{c\beta}}{n-1}\left(2csL^2(1+ke^{c\beta})k^{-1}\right)^{\frac{k}{c\beta+k}}\cdot T^{\frac{c\beta}{c\beta+k}},
 \end{equation}
to simplify, omitting constant factors that depend on $\beta$, c
and L, we get 
   \begin{equation}
    \epsilon_{gen}  \leq \mathcal{O}_s\left(\frac{T^{\frac{c\beta}{c\beta+k}}}{n}\right).
   \end{equation}
And we finish the proof.


% The $\mathtt{\backslash onecolumn}$ command above can be kept in place if you prefer a one-column appendix, or can be removed if you prefer a two-column appendix.  Apart from this possible change, the style (font size, spacing, margins, page numbering, etc.) should be kept the same as the main body.
%%%%%%%%%%%%%%%%%%%%%%%%%%%%%%%%%%%%%%%%%%%%%%%%%%%%%%%%%%%%%%%%%%%%%%%%%%%%%%%
%%%%%%%%%%%%%%%%%%%%%%%%%%%%%%%%%%%%%%%%%%%%%%%%%%%%%%%%%%%%%%%%%%%%%%%%%%%%%%%


\end{document}


% This document was modified from the file originally made available by
% Pat Langley and Andrea Danyluk for ICML-2K. This version was created
% by Iain Murray in 2018, and modified by Alexandre Bouchard in
% 2019 and 2021 and by Csaba Szepesvari, Gang Niu and Sivan Sabato in 2022.
% Modified again in 2023 and 2024 by Sivan Sabato and Jonathan Scarlett.
% Previous contributors include Dan Roy, Lise Getoor and Tobias
% Scheffer, which was slightly modified from the 2010 version by
% Thorsten Joachims & Johannes Fuernkranz, slightly modified from the
% 2009 version by Kiri Wagstaff and Sam Roweis's 2008 version, which is
% slightly modified from Prasad Tadepalli's 2007 version which is a
% lightly changed version of the previous year's version by Andrew
% Moore, which was in turn edited from those of Kristian Kersting and
% Codrina Lauth. Alex Smola contributed to the algorithmic style files.
