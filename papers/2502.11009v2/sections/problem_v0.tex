\section{Problem Statement}\label{sec:problem}


Consider a private dataset $D$ with $n$ rows that lies behind a privacy firewall. A trusted curator aims to release private statistics about inconsistency measures with unbounded $\epsilon$-DP guarantees while trying to be as accurate as possible to the true values. 
\xh{make use of the figure in the introduction}

\begin{problem}[\xh{Add "DP/Private"} Inconsistency measures on dataset]\label{prob:measure_dataset} 
Given a private dataset $D$, a set of constraints $\constraintset$, and a privacy budget $\epsilon$, we would like to obtain an $\epsilon$-DP algorithm $M(D, \constraintset, \epsilon)$ for an inconsistency measure algorithm $I(D, \constraintset)$ \xh{do we limit the inconsistency measures to only the ones in Def 1?} such that with high probability, $|M(D, \constraintset, \epsilon) - I(D, \constraintset)|$ is bounded with a small error. 
\end{problem}

\xh{Can we summarize the results (including the non-private and private computation cost and sensitivity of the measures) in a table? 
}

\xh{The following paragraph can first start with the high-level challenges for all these queries. Do they all have high sensitivity with respect to their query answers? What level of errors do directly answering them with Laplace mechanisms result?}

We show that two of the inconsistency measures, the drastic measure $\drastic$ and the maximal consistency measures $\maxconsistency$, are hard \xh{do not use the word ``hard''} to compute using DP. We discuss this in detail in Section~\ref{sec:hardness}.
For the other three inconsistency measures, there could be multiple design choices. For example, for the $\problematic$ and the $\mininconsistency$, one could compute the sensitivity of the measures and add appropriate Laplace noise or ask these measures as SQL queries on the dataset. However, we note that the sensitivity for these measures could be very high in the $\mathcal{O}(|D|)$ that could ruin the privately estimated value as shown in Proposition~\ref{prop:sens_mininconsistency_problematic_naive}.

\xh{May remove this proposition from this section}
\begin{proposition}
The sensitivity of both $\mininconsistency$ and $\problematic$ equals $|D|$. \xh{be consistent with $\mathcal{O}(|D|)$ or $|D|$ or $n$.}
\label{prop:sens_mininconsistency_problematic_naive}
\end{proposition}
\begin{proof}
    The $\mininconsistency$ and $\problematic$ are concerned with the set of minimally inconsistent subsets $MI_\constraintset(D)$. The $\mininconsistency$ measure computes the total number of inconsistent subsets $|MI_\constraintset(D)|$ and $\problematic$ computes the total number of unique rows participating in $MI_\constraintset(D)$, $|\cup MI_\constraintset(D)|$. Now, without loss in generality, let's assume $D'$ has an additional tuple compared to $D$. In the worst case, the extra row could violate all other rows in the dataset, adding $|D|$ inconsistent subsets to $MI_\constraintset(D)$. Therefore, in the worst case, the $\mininconsistency$ measure and $\problematic$ could change by $|D|$.
\end{proof}

On the other hand, the $\repair$ measure is an NP-hard problem, and the best known non-private solution is to solve a linear approximation that requires solving a linear program~\cite{LivshitsBKS20}. However, this linear program, in the worst case, again has sensitivity equal to $n$(number of rows in the dataset) and may have up to $\Comb{n}{2}$ number of constraints(all rows violating with each other). Existing state-of-the-art DP linear solvers~\cite{hsu2014privately} are slow and fail for such a challenging task. In the upcoming sections, we show that these problems can be alleviated by considering the input dataset as a conflict graph and computing these inconsistency measures as private graph statistics. The rest of the paper is organized as follows. In Section~\ref{sec:graph-algorithms-graphproj}, we show how the $\mininconsistency$ and $\problematic$ measures can be computed on conflict graphs by leveraging graph projections algorithms. Then, in Section~\ref{sec:vertex_cover}, we show an approximate DP vertex cover algorithm to compute the repair measures. Finally, in Section~\ref{sec:experiments}, we show experimental results on five real-world datasets and how our proposed algorithms can adapt to varying dataset sizes and densities. 
\ifpaper
Some of the proofs in the paper are deferred to the full version.
\else
\fi
