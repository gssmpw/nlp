\sm{This proof is mostly copied from Day et al. except the second paragraph which talks about one edge difference and some arguments that change in the last paragraph. We can defer the full version of this proof to the appendix and have a short version stating the second paragraph.}
Similar to the proof in the original paper, let's assume without loss of generality that
$G^{\prime}=\left(V^{\prime}, E^{\prime}\right)$ has an additional node $v^{+}$compared to $G=$ $(V, E)$, i.e., $V^{\prime}=V \cup\left\{v^{+}\right\}, E^{\prime}=E \cup E^{+}$, and $E^{+}$is the set of all edges incident to $v^{+}$in $G^{\prime}$. Let $\Lambda^{\prime}$ be the stable orderings for constructing $\pi_\theta\left(G^{\prime}\right)$, and $t$ be the number of edges added to $\pi_\theta\left(G^{\prime}\right)$ that are incident to $v^{+}$. Clearly, $t \leq \theta$ because of the $\theta$-bounded algorithm. Let $e_{\ell_1}^{\prime}, \ldots, e_{\ell_t}^{\prime}$ denote these $t$ edges in their order in $\Lambda^{\prime}$. Let $\Lambda_0$ be the sequence obtained by removing from $\Lambda^{\prime}$ all edges incident to $v^{+}$, and $\Lambda_k$, for $1 \leq k \leq t$, be the sequence obtained by removing from $\Lambda^{\prime}$ all edges that both are incident to $v^{+}$and come after $e_{\ell_k}^{\prime}$ in $\Lambda^{\prime}$. Let $\pi_\theta^{\Lambda_k}\left(G^{\prime}\right)$, for $0 \leq k \leq t$, be the graph reconstructed by trying to add edges in $\Lambda_k$ one by one on nodes in $G^{\prime}$, and $\lambda_k$ be the sequence of edges from $\Lambda_k$ that are added in the process. Thus $\lambda_k$ uniquely determines $\pi_\theta^{\Lambda_k}\left(G^{\prime}\right)$; we abuse the notation and use $\lambda_k$ to also denote $\pi_\theta^{\Lambda_k}\left(G^{\prime}\right)$. We have $\lambda_0=\pi_\theta(G)$, and $\lambda_t=\pi_\theta\left(G^{\prime}\right)$.

We show that $\forall k$ such that $1 \leq k \leq t$, at most 1 edge will differ between $\lambda_k$ and $\lambda_{k-1}$. This proves the lemma because there are at most $t$ (upper bounded by $\theta$) edges that are different between $\lambda_t$ and $\lambda_0$.

To prove this, consider how the sequence $\lambda_k$ may differ from $\lambda_{k-1}$. The first difference must be that $\lambda_k$ includes $e_{\ell_k}^{\prime}=\left(u_j, v^{+}\right)$, and $\lambda_{k-1}$ does not. (Recall that $\Lambda_k$ contains $e_{\ell_k}^{\prime}$ but $\Lambda_{k-1}$ does not by construction.) The decisions for all edges coming before $e_{\ell_k}^{\prime}$ in $\Lambda^{\prime}$ must be the same in both $\lambda_k$ and $\lambda_{k-1}$. After $e_{\ell_k}^{\prime}$, the edges in $\Lambda_k$ and $\Lambda_{k-1}$ are exactly the same; thus we can consider each different decision regarding a future edge as one more difference. Assume that there are a total of $s \geq 1$ different decisions.

When $s=1$, the differing edge is $e_{\ell_k}^{\prime}$ changing the degree of only $u_j$ (other than $v^{+}$) by adding 1 . When $s>1$, the second difference must be regarding an edge incident to $u_j$ and is because $\theta$ edges incident to $u_j$ are added in $\lambda_{k-1}$, and the last one of these, denoted by $\left(u_j, u_{i \theta}\right)$, cannot be added in $\lambda_k$ (because $u_j$ 's degree has reached $\theta$ in $\lambda_k$ by then, due to the extra edge $\left(u_j, v^{+}\right)$). In this case, $u_j$ has the same degree (i.e., $\theta$ ) in both $\lambda_k$ and $\lambda_{k-1}$. If $s=2$, then only the edge $\left(u_j, u_{i \theta}\right)$ is different and the degree of $u_{i \theta}$ (other than $v^{+}$) has a different degree, and its degree in $\lambda_k$ is 1 less than that in $\lambda_{k-1}$. If $s>2$, then the third difference must be about an edge incident to $u_{i \theta}$ and is because the edge is included in $\lambda_k$ but not included in $\lambda_{k-1}$; this can only happen when $u_{i \theta}$ has degree $\theta$ in both $\lambda_k$ and $\lambda_{k-1}$, and results in another node's degree change by 1 . Since $s$ is finite, this sequence of reasoning will stop, with only the last the differing edge from $\lambda_{k-1}$ to $\lambda_k$ (similarly, the last encountered node has a degree change from $\lambda_{k-1}$ to $\lambda_k$ and all nodes earlier must have degree $\theta$ in both $\lambda_{k-1}$ and $\lambda_k$ ).