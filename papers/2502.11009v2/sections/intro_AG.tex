
\section{Introduction}\label{sec:intro}
\common{
Modern machine learning applications have started to rely on private data to build personalized products such as medical and financial assistants. This type of data often contains sensitive information about individuals. Proper care and practices are required to prevent privacy concerns~\cite{auxier2019americans} and penalties~\cite{GDPR}. Differential privacy~\cite{dwork2006calibrating} has become the de facto standard for querying sensitive databases and has been adopted by various industry and government bodies~\cite{abowd2018us,erlingsson2014rappor,ding2017collecting}. 
DP offers high utility for aggregate data releases while ensuring strong guarantees on individuals' sensitive data. 
The laudable progress in DP study, as demonstrated by multiple recent works~\cite{zhang2015private,zhang2017privbayes,kotsogiannis2019privatesql,abs-1802-06739,gupta2010differentially,ZhangXX16}, has made it approachable and helpful in many common scenarios. A standard DP mechanism adds noise to the query output, constrained by a privacy budget that quantifies the permitted privacy leakage. Once the privacy budget is exhausted, no more queries can be answered directly using the database. 

The utility of the machine learning applications also depends on the quality of the underlying data. Therefore, organizations that build these applications spend vast amounts of money on purchasing data from private data marketplaces~\cite{liu2021dealer, DBLP:conf/infocom/SunCLH22, DBLP:journals/corr/abs-2210-08723,DBLP:journals/tdsc/XiaoLZ23}.   These marketplaces build relationships and manage monetary transactions between data owners and model buyers. Before the model buyer purchases a dataset, they want to ensure the data is suitable for their use case and adheres to certain data quality constraints. Similarly, the data owners want the best price for their datasets, which naturally creates a problem where the cost of the dataset needs to be aligned according to the amount of errors in the dataset. One way to profile data by assessing its quality is by leveraging \emph{inconsistency measures}.


An \emph{inconsistency measure} is a numeric function that quantifies the extent to which a database violates a set of integrity constraints that are supposed to hold~\cite{thimm2017compliance, parisi2019inconsistency, 
DBLP:conf/sum/Bertossi18,DBLP:conf/ecsqaru/GrantH13,DBLP:journals/ijar/GrantH23,LivshitsBKS20}.
Various applications of inconsistency measures have been identified, including its usage as a data profiling mechanism to assess the quality of a database~\cite{LivshitsKTIKR21,LivshitsK22}. 
Yet, past work on inconsistency measures has not considered privacy aspects, and we embark on this challenge in this work.  In our setup, we have a relational database and an inconsistency measure; the user inputs a set of integrity constraints, and the goal is to provide her with the output of a \emph{differentially private inconsistency measure}. Differentially private inconsistency measures are an elegant solution to identify data quality in data marketplace scenarios. They can quantify the inconsistency in the dataset and help rank datasets concerning the data buyer's needs while preserving the data owner's privacy constraints. 
}

As inconsistency measures, we adopt the ones studied by \citet{LivshitsKTIKR21} following earlier work on the topic~\cite{thimm2017compliance, parisi2019inconsistency, 
DBLP:conf/sum/Bertossi18,DBLP:conf/ecsqaru/GrantH13}.
%that quantify data quality with a single number for all constraints, essentially yielding {\em a data quality score.} This approach aligns well with DP, as such measures give a single aggregated numerical value representing data quality, regardless of the given number of constraints. 
This work discusses and studies five measures, including (1) the {\em drastic measure}, a binary indicator for whether the database contains constraint violations, the (2) \emph{maximal consistency measure}, counting the number of maximal tuple sets for which an addition of a single tuple will cause a violation, the (3) {\em minimal inconsistency measure}, counting the number of minimal tuple sets that violate a constraint, 
the (4) \emph{problematic measure}, counting the number of constraint violations, the (5) \emph{minimal repair measure}, counting the minimal tuple deletions needed to achieve consistency. 
These measures apply to various inconsistency measures that have been studied in the literature of data quality management, including functional dependencies, the more general conditional functional dependencies~\cite{bohannon2007conditional}, and the more general denial constraints~\cite{ChomickiM05}.
We show that the first two measures are incompatible for computation in the DP setting (\Cref{sec:hardness}), and focus throughout the paper on the latter three. 
% The main challenge we study in this paper is DP computation of these measures.

%measures for which we provide DP mechanisms. 
% These measures often exhibit a broad range of outputs, making them more suitable for estimating inconsistency under DP. 
% Thus, we focus on using inconsistency measures for estimating the quality of DP-protected data. 
%Still, computing these measures while satisfying DP is not straightforward. 

% Data quality has been thoroughly studied for databases without privacy concerns, 
% as databases that lack quality and consistency cannot be used to derive trustworthy conclusions and train reliable models. 
% A significant portion of previous work has been devoted to various integrity constraints that can capture quality issues, such as functional dependencies, conditional functional dependencies~\cite{bohannon2007conditional}, and denial constraints~\cite{ChomickiM05}, as well as various algorithms for repairing data quality issues by employing \benny{deploying? what does it mean to employ/deploy constraints?} these constraints~\cite{LivshitsKR20,ChuIP13,GiladDR20,RekatsinasCIR17}. 
% % These works propose several models for changing the database to repair it, including tuple deletion~\cite{LivshitsKR20,GiladDR20} which is the most basic model for data repairing, and cell value updates~\cite{ChuIP13,RekatsinasCIR17} which is a more granular manner of changing the data. Regardless of the employed \benny{deployed?} model, the goal is for the changed dataset to comply with the given constraints. 
% % While these works have paved the way for improving data quality, they do not focus on the scenario where DP standards apply to the database, and in particular, where users cannot see the data or even change it. 
% % While these approaches effectively repair data to meet quality standards, they do not address cases where DP must be maintained. 
% Inconsistency measures that use these constraints to quantify data quality have been previously proposed \cite{thimm2017compliance, parisi2019inconsistency, LivshitsKTIKR21} that  with a single metric for all constraints. This approach aligns well with DP, as such measures give a single numerical value representing data quality.
% This leaves a gap in scenarios where users need to measure inconsistency without directly accessing or modifying sensitive data. 

%\textbf{The following paragraph does not fit..}

% {\em In this work, we aim to bridge this gap by considering the problem of assessing the quality of databases protected by DP with respect to integrity constraints.} 
% Such quality assessment will allow users to decide whether they can rely on conclusions drawn from the data, or even if the suggested data is suitable for them. 
% To solve this problem, we must tackle several challenges. First, since the database is protected by DP, users are only allowed to observe noisy aggregate statistics which can be difficult to summarize into a quality score. Second, if the number of constraints is large (e.g., if they were generated with an automatic system~\cite{BleifussKN17,DBLP:journals/pvldb/LivshitsHIK20,PenaAN21}), translating each constraint to an SQL {\tt COUNT} query and evaluating it over the database with a DP mechanism may lead to low utility since the number of queries is large, allowing for only a tiny portion of the privacy budget to be allocated to each query. 

% Hence, our proposed solution employs \benny{deploy?} \emph{inconsistency measures}~\cite{thimm2017compliance, parisi2019inconsistency, LivshitsKTIKR21} 
% \benny{Many more - \cite{DBLP:conf/sum/Bertossi18,DBLP:conf/ecsqaru/GrantH13,DBLP:journals/ijar/GrantH23,LivshitsK22,LivshitsBKS20}}
% that quantify data quality with a single number for all constraints, essentially yielding {\em a data quality score.} This approach aligns well with DP, as such measures give a single aggregated numerical value representing data quality, regardless of the given number of constraints. Prior work~\cite{LivshitsKTIKR21} described various measures, such as the \emph{problematic measure}, counting the number of constraint violations, the \emph{minimal repair measure}, counting the minimal tuple deletions needed to achieve consistency. These measures often exhibit a broad range of outputs, making them more suitable for estimating inconsistency under DP. 
% % Thus, we focus on using inconsistency measures for estimating the quality of DP-protected data. 
% Still, computing these measures while satisfying DP is not straightforward. 


An approach that one may suggest to computing the inconsistency measures in a DP manner is to translate the measure into an SQL query, and then compute the query using an SQL engine that respects DP~\cite{tao2020computing,dong2022r2t,kotsogiannis2019privatesql,johnson2018towards}. Specifically relevant is R2T~\cite{dong2022r2t}, the state-of-the-art DP mechanism for SPJA queries, including self-joins. Nevertheless, when considering the three measures of inconsistency we focus on, there are several drawbacks to this approach. One of these measures (number of problematic tuples) requires the SQL {\tt DISTINCT} operator that cannot be handled by R2T, while another measure (minimal repair) cannot be expressed at all in SQL, casting such engines irrelevant. 

Contrasting the first approach, the approach that we propose and investigate here models the violations of the integrity constraints as a \emph{conflict graph} and apply DP techniques for graph statistics. In the conflict graph, nodes are tuple identifiers and there is an edge between a pair of tuples if this pair violates a constraint. 
Then, each inconsistency measure can be mapped to a specific graph statistic. 
Using this view of the problem allows us to leverage prior work on releasing graph statistics with DP~\cite{hay2009accurate,KasiviswanathanNRS13,day2016publishing} and develop tailored mechanisms for computing inconsistency measures with DP. 

To this end, we harness graph projection techniques from the state-of-the-art DP algorithms~\cite{day2016publishing} that truncate the graph to achieve DP. While these algorithms have proven effective in prior studies on social network graphs, they may encounter challenges with conflict graphs arising from their unique properties. To overcome this, we devise a novel optimization for choosing the truncation threshold. 
We further provide a DP mechanism for the minimal repair measure that augments the classic 2-approximation of the vertex cover algorithm~\cite{vazirani1997approximation} to restrict its sensitivity and allow effective DP guarantees with high utility. 
Our experimental study shows that our novel algorithms prove efficacious for different datasets with various conflict graph sizes and sparsity levels. 

%As mentioned, one approach is to treat each inconsistency measure as an SQL query \benny{Not sure for two of the five, e.g., repair} and answer these queries using DP SQL mechanisms~\cite{tao2020computing,dong2022r2t,kotsogiannis2019privatesql,johnson2018towards}. Queries expressing these measures require self-joins and include complex {\tt OR} conditions. However, the state-of-the-art DP mechanism for self-join queries, R2T~\cite{dong2022r2t}, only supports measures that can be expressed as SPJA queries.  Measures that require specific operators, like {\tt DISTINCT} or {\tt GROUP BY}, cannot be handled by R2T.\benny{It makes it sound like the problem is at the very low level of technicalities... nothing fundamental} Moreover, even for measures compatible with R2T, the accuracy of the results varies widely over different datasets, as shown by \Cref{ex:motivation} in the sequel. 




%We formally define the problem of computing inconsistency measures while satisfying DP and choose the measures suitable for computation with DP from those proposed in prior work~\cite{LivshitsKTIKR21} by analyzing their sensitivity and computational cost. We then devise novel DP mechanisms to estimate these measures effectively. 



\begin{table}
\centering
\begin{tabular}{|l|l|l|l|}
\hline
\textbf{DP Algorithms} & \textbf{Adult~\cite{misc_adult_2}}                  & \textbf{Flight~\cite{flight}}                 & \textbf{Stock~\cite{oleh_onyshchak_2020}}                  \\ \hline
\textbf{R2T~\cite{dong2022r2t}}           & $0.17\pm 0.01$ & $0.12\pm0.03$ & $123.19\pm 276.73$    \\ \hline
\textbf{This work}  & $0.10 \pm 0.05$ & $0.10 \pm 0.20$  & $0.07\pm 0.08$ \\ \hline
\end{tabular}
  \caption{Relative errors for a SQL approach vs our approach to compute the minimal inconsistency measure at $\epsilon=1$} \label{tab:intro_comparison}
\end{table}

Beyond handling the two inconsistency measures that R2T cannot handle, our approach provides considerable advantages even for the one that R2T can handle (number of conflicts). 
%\begin{example}\label{ex:motivation}
For illustration, \Cref{tab:intro_comparison} shows the results of evaluating R2T~\cite{dong2022r2t} on three datasets with the same privacy budget of $1$ for this measure. Though R2T performed well for the Adult and Flight datasets, it reports more than 120\% relative errors for the Stock dataset that had very few violations. 
On the other hand, our approach demonstrates strong performance across all three datasets.
%\end{example}


The main contributions of this paper are as follows: 
\squishlist
    \item We formulate the novel problem of computing inconsistency measurements with DP for private datasets and discuss the associated challenges, including a thorough analysis of the sensitivity of each measure.
    \item We devise several algorithms that leverage the conflict graph and algorithms for releasing graph statistics under DP to estimate the measures that we have determined are suitable. Specifically, we propose a new optimization for choosing graph truncation threshold that is tailored to conflict graphs and augment the classic vertex cover approximation algorithms to bound its sensitivity to $2$ in order to obtain accurate estimates of the measures. 
    % \item We demonstrate that two of five such measures are incompatible in the DP setting, while the other three can be computed by formulating the problem using conflict graphs.
    % \item We identify high sensitivity as a challenge in computing some of these measures for practical settings and propose optimizations using the integrity constraints to mitigate this challenge.
    \item We present experiments on five real-world datasets with varying sizes and densities to show that the proposed DP algorithms are efficient in practice. Our average error across these datasets is 1.3\%-67.9\% compared to the non-private measure.
\squishend

