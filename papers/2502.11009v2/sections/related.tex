\section{Related Work}\label{sec:related}
We survey relevant works in the fields of DP and data repair.
% Differentially private inconsistency measures can be computed in multiple ways. For example, they can be computed by formulating them as SQL queries, and a private SQL engine~\cite{johnson2017practical, kotsogiannis2019privatesql} can be used. The most recent work on computing SQL queries relevant to SPJA queries over private node-DP graphs is R2T~\cite{dong2022r2t}. However, our initial study on R2T shows that it performs poorly than our graph projection queries and times out on datasets with a number of tuples ($\geq 10^5$). Graph projection algorithms for node-DP that bound the degree of nodes to a given bound $\theta$ is also a very well-studied problem. The graph truncation algorithm~\cite{kasiviswanathan2013analyzing} removes all nodes
% with degrees greater than $\theta$. This removes more nodes than necessary, as one could keep up to $\theta$ edge for each vertex without removing the whole node. Another method is the edge-removal algorithm~\cite{blocki2013differentially} that traverses all edges in random order and removes edges when the degree of the node is higher than $\theta$. This algorithm depends heavily on the input random order of the edge and can have poor utility. Therefore, for our graph projection use case, we use the edge-addition algorithm~\cite{day2016publishing} that preserves most of the underlying information of a graph. 
% The first idea of an inconsistency measure was first proposed in 2008~\cite{hunter2008measuring}. Following this work, there have been several works have proposed new rationality postulates, either to replace previously proposed postulates
% or to extend them~\cite{thimm2012measuring, hunter2010measure, mu2011general, mu2011syntax, besnard2014revisiting}. More recent works have discussed properties of these measures, efficient computations, and compliance of the measures~\cite{LivshitsBKS20, LivshitsKTIKR21, parisi2019inconsistency, thimm2017compliance}. Inconsistency measures have also been investigated as progress bars and in the decision-making pipeline for data repair systems~\cite{bertossi2019repairbaseddegreesdatabaseinconsistency, su2023time,fröhlich2024logicbasedframeworkdatabaserepairs}. However, to the best of our knowledge, no prior work has been done on computing inconsistency measures on private datasets.

\paragraph{Differential privacy}
DP has been studied in multiple settings~\cite{johnson2017practical,kotsogiannis2019privatesql,wilson2019differentially,johnson2018towards,DBLP:journals/pvldb/McKennaMHM18,tao2020computing}, including systems that support complex SQL queries that, in particular, can express integrity constraints~\cite{dong2022r2t}. The utilization of graph databases under DP has also been thoroughly explored, with both edge privacy~\cite{hay2009accurate, karwa2011private,karwa2012differentially,sala2011sharing,karwa2014differentially,zhang2015private} and node privacy~\cite{KasiviswanathanNRS13,blocki2013differentially,day2016publishing}. 
Our approach draws on \cite{day2016publishing} to allow efficient DP computation of the inconsistency measures over the conflict graph. In contrast, we have seen worse performance from alternative approaches for releasing graph statistics that tend to truncate edges or nodes aggressively. 
In the context of data quality, previous work~\cite{KrishnanWFGK16} has proposed a framework for releasing a private version of a database relation for publishing, supporting specific repair operations, while more recent, work~\cite{GeMHI21} provides a DP synthetic data generation mechanism that considers soft DCs~\cite{ChomickiM05}. 


\paragraph{Data repair}
Various classes of constraints were proposed in the literature, including FDs, conditional FDs~\cite{bohannon2007conditional}, metric FDs~\cite{koudas2009metric}, and DCs~\cite{ChomickiM05}. We focused on DCs, a general class of integrity constraints that subsumes the aforementioned constraints. 
While computing the minimal data repair in some cases has been shown to be polynomial time~\cite{LivshitsKR20}, computing the minimal repair in most general cases, corresponding to \repair, is NP-hard. 
Therefore, a prominent vein of research has been devoted to utilizing these constraints for data repairing~\cite{Afrati2009,RekatsinasCIR17,ChuIP13,BertossiKL13,FaginKK15,LivshitsKR20,GiladDR20,GeertsMPS13}. The repair model in these works varies between several options: tuple deletion, cell value updates, tuple addition, and combinations thereof. 
The process of data repairing through such algorithms can be time consuming due to the size of the data and the size and complexity of the constraint set. 
Hence, previous work~\cite{LivshitsBKS20,LivshitsKTIKR21} proposed to keep track of the repairing process and ensure that it progresses correctly using inconsistency measures. 
In our work, we capitalize on the suitability of these measures to DP as they provide an aggregate form that summarizes the quality of the data for a given set of constraints.  
