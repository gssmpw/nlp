\section{DP Graph Projection for $\mininconsistency$ and $\problematic$}\label{sec:graph-algorithms-graphproj}


Computing graph statistics such as edge count and degree distribution while preserving node-differential privacy (node-DP) is a well-explored area~\cite{day2016publishing, KasiviswanathanNRS13, blocki2013differentially}. 
Hence, in this section, we leverage the state-of-the-art node-DP approach for graph statistics to analyze the inconsistency measures $\mininconsistency$ and $\problematic$ as graph statistics on the conflict graph $\graph$. However, the effectiveness of this approach hinges on carefully chosen parameters.  We introduce two optimization techniques that consider the integrity constraints to optimize parameter selection and enhance the algorithm's utility.

%We begin by outlining the algorithms for the minimum inconsistency measure $\mininconsistency$ (i.e., the number of edges in $\graph$). Subsequently, we discuss the necessary modifications to adapt these algorithms for the problematic measure $\problematic$ (i.e., the number of nodes with positive degrees).


%In this section, we present how to differentially privately compute the minimum inconsistency measure $\mininconsistency$ using the conflict graph $\graph$ instead of the original database $D$.  We will begin by outlining a general differentially private algorithm that uses graph projection techniques to achieve better utility than the basic Laplace mechanism. The choice of parameters for this algorithm is crucial for its utility. Hence, we will introduce two optimization techniques based on the constraints given for the database $D$ to enhance the parameter selection process and, consequently, improve the algorithm's utility.

\subsection{Graph Projection Approach for \mininconsistency\ and \problematic} 
A primary utility challenge in achieving node-DP for graph statistics is their high sensitivity. In the worst case, removing a single node from a graph of $n$ nodes can result in removing $(n-1)$ edges. To mitigate this issue, the state-of-the-art approach~\cite{day2016publishing} first projects the graph $\mathcal{G}$ onto a $\theta$-bounded graph $\mathcal{G}_{\theta}$, where the maximum degree is no more than $\theta$. Subsequently, the edge count of the transformed graph is perturbed by the Laplace mechanism with a sensitivity value of less than $n$. However, the choice of $\theta$ is critical for accurate estimation. 
A small $\theta$ reduces Laplace noise due to lower sensitivity, but results in significant edge loss during projection. Conversely, a $\theta$ close to $n$ preserves more edges but increases the Laplace noise. Prior work addresses this balance using the exponential mechanism (EM) to prefer a $\theta$ that minimizes the combined errors arising from graph projection and the Laplace noise. 



\begin{algorithm}[t]
\caption{Graph projection approach for $\mininconsistency$ and $\problematic$}
\label{algo:graph_general}
    \KwData{Dataset $D$, constraint set $\constraintset$, candidate set $\Theta$, privacy budgets $\epsilon_1$ and $\epsilon_2$}
    \KwResult{DP inconsistency measure for $\mininconsistency$ or $\problematic$}
    
    Construct the conflict graph $\graph$\\
    
    Sample $\theta^*$ from $\Theta$  with a $\epsilon_1$-DP mechanism \commenttext{// Basic EM (Algorithm~\ref{algo:expo_mech_basic}); Optimized EM (Algorithm~\ref{algo:em_opt})}\
    
    Compute $\theta^*$-bounded graph $\mathcal{G}_{\theta^*} \gets \pi_{\theta^*}(\graph)$  
     \commenttext{// Edge addition algorithm~\cite{day2016publishing}}\\
    
    {\bf Return} $f(\mathcal{G}_{\theta^*})+\text{Lap}(\frac{\theta^*}{\epsilon_2})$
    \commenttext{//
    $f(\cdot)$ returns edge count for $\mininconsistency$ and the number of nodes with positive degrees for $\problematic$} \end{algorithm}

We outline this general approach in Algorithm~\ref{algo:graph_general}. This algorithm takes in the dataset $D$, the constraint set $\constraintset$, a candidate set $\Theta$ for degree bounds, and privacy budgets $\epsilon_1$ and $\epsilon_2$. These privacy budgets are later composed to get a final guarantee of $\epsilon$-DP.
We start by constructing the conflict graph $\graph$ generated from the input dataset $D$ and constraint set $\constraintset$ (line 1), as defined in \Cref{sec:prelim-integrity-constraints}. 
Next, we sample in a DP manner a value of $\theta^*$ from the candidate set $\Theta$ with the privacy budget $\epsilon_1$ (line 2). A baseline choice is an exponential mechanism detailed in Algorithm~\ref{algo:expo_mech_basic} to output a degree that minimizes the edge loss in a graph and the Laplace noise.
In line 3, we compute a bounded graph $\mathcal{G}_{\theta^*}$ using the edge addition algorithm~\cite{day2016publishing}, we compute a $\theta^*$-bounded graph $\mathcal{G}_{\theta^*}$ (detailed in Section~\ref{sec:prelim}). Finally, we perturb the true measure (either the number of edges for $\mininconsistency$ or the number of positive degree nodes for $\problematic$) on the projected graph, denoted by $f(\mathcal{G}_{\theta^*})$, by adding Laplace noise using the other privacy budget $\epsilon_2$ (line 4). 

The returned noisy measure at the last step has two sources of errors: (i) the bias incurred in the projected graph, i.e.,  $f(\mathcal{G})-f(\mathcal{G}_{\theta^*})$, and (ii) the noise from the Laplace mechanism with an expected square root error ${\sqrt{2}\theta^*}/{\epsilon_2}$. Both errors depend on the selected parameter $\theta^*$, and it is vital to select an optimal $\theta^*$ that minimizes the combined errors. Next, we describe a DP mechanism that helps select this parameter. 


\begin{algorithm}[b]
\caption{EM-based first try for parameter selection}
\label{algo:expo_mech_basic}
    \KwData{Graph $\mathcal{G}$, candidate set $\Theta$, quality function $q$, privacy budget $\epsilon_1, \epsilon_2$ }
    \KwResult{Candidate $\theta^*$}
    Find the maximum value in $\Theta$ as $\theta_{\max}$ \\
    For each $\theta_i \in \Theta$, compute $q_{\epsilon_2}(\mathcal{G}, \theta_i)$ 
    \commenttext{// See Equation~\eqref{eq:quality_function}}
    \\
    %Pick $\theta^*$ with prob $\propto \exp( \frac{\epsilon_1 q(\mathcal{G}, \theta_i, \theta_{max}, \epsilon_2)}{2\Delta_q})$ \\
    Sample $\theta^*$ with prob $\propto \exp( \frac{\epsilon_1 q_{\epsilon_2}(\mathcal{G}, \theta_i)}{2\theta_{\max}})$ \\
    {\bf Return} $\theta^*$
\end{algorithm}

\paratitle{EM-based first try for parameter selection} 
% The exponential mechanism (EM) is a popular algorithm for choosing hyper-parameters like $\theta$ for a degree-bounded graph. For this case, 
The EM (Definition~\ref{def:EM}) specifies a quality function $q(\cdot,\cdot)$ that maps a pair of a database $D$ and a candidate degree $\theta$ to a numerical value. The optimal $\theta$ value for a given database $D$ should have the largest possible quality value and, hence, the highest probability of being sampled. We also denote $\theta_{\max}$ the largest degree candidate in $\Theta$ and use it as part of the quality function to limit its sensitivity.


%Algorithm~\ref{algo:expo_mech_basic}. Given a candidate set $\Theta$ and a quality function $q(\mathcal{G}, \theta) \in \mathbb{R}$ and corresponding scores for each candidate, exponential mechanism picks a candidate $\theta^* \in \Theta$ that has the highest score with high probability.

The quality function we choose to compute the inconsistency measures includes two terms: for each $\theta\in \Theta$,
\begin{equation}~\label{eq:quality_function}
    q_{\epsilon_2}(\mathcal{G}, \theta) = - e_{\text{bias}}(\mathcal{G}, \theta) - {\sqrt{2}\theta}/{\epsilon_2}
\end{equation}
where the first term $e_{\text{bias}}$  captures the bias in the projected graph, and the second term ${\sqrt{2}\theta}/{\epsilon_2}$ captures the error from the Laplace noise at budget $\epsilon_2$. For the minimum inconsistency measure $\mininconsistency$, we define the bias term as
\begin{equation}
e_{\text{bias}}(\mathcal{G},\theta) = |\mathcal{G}_{\theta_{\max}}.E| - |\mathcal{G}_\theta.E|    
\end{equation}
i.e., the number of edges truncated at degree $\theta$ as compared to that at degree $\theta_{\max}$. For the problematic measure $\problematic$, we have 
\begin{equation}
e_{\text{bias}}(\mathcal{G}, \theta) = 
|\mathcal{G}_{\theta_{\max}}.V_{>0}| - |\mathcal{G}_{\theta}.V_{>0}|    
\end{equation} 
where $\mathcal{G}_\theta.V_{>0}$ denote the nodes with positive degrees. 

\begin{example}~\label{example:quality_function}
    Consider the same graph as Example~\ref{example:running_example} and a candidate set $\Theta = [1, 2, 3]$ to compute the $\mininconsistency$ measure (number of edges) with $\epsilon_2=1$. For the first candidate $\theta = 1$, as node 4 has degree 3, the edge addition algorithm would truncate 2 edges, for $\theta = 2$, 1 edge would be truncated and for $\theta = 3$, no edges would be truncated. We can, therefore, compute each term of the quality function for each $\theta$ given in Table~\ref{tab:example_quality_function}.  
    \begin{table}[]
        \centering
        \begin{tabular}{|c|c|c|c|}
             \hline
             $\theta$ & $e_{\text{bias}}$ & ${\sqrt{2}\theta}/{\epsilon_2}$ & q  \\
             \hline
             1 & 2 & $\sqrt{2}$ & $-2 - \sqrt{2}$\\
             2 & 1 & $2\sqrt{2}$ &  $-1 - 2\sqrt{2}$\\
             3 & 0 & $3\sqrt{2}$ & $-3\sqrt{2}$\\
             \hline
        \end{tabular}
        \caption{Quality function computation for $\mininconsistency$ for the conflict graph in Figure~\ref{fig:db_to_graph} when $\epsilon_2=1$}
        \label{tab:example_quality_function}
    \end{table}
    For this example, we see that $\theta=1$ has the best quality even if it truncates the most number of edges as the error from Laplace noise overwhelms the bias error.
\end{example}

We summarize the basic EM for the selection of the bounded degree in Algorithm~\ref{algo:expo_mech_basic}.
This algorithm has a complexity of $O(|\Theta|m)$, where $m$ is the edge size of the graph, as computing the quality function for each $\theta$ candidate requires running the edge addition algorithm once. The overall Algorithm~\ref{algo:graph_general} has a complexity of $O(|\Sigma|n^2+|\Theta|m)$, where the first term is due to the construction of the graph.

\paratitle{Privacy analysis}
The privacy guarantee of \cref{algo:graph_general} depends on the budget spent for the exponential mechanism and the Laplace mechanism, as summarized below. 
\begin{theorem}\label{thm:privacy_proof_dc_oblivious}
    ~\cref{algo:graph_general} satisfies $(\epsilon_1 + \epsilon_2)$-node DP for $\graph$ and $(\epsilon_1 + \epsilon_2)$-DP for the input database $D$.
\end{theorem}

\reva{
\begin{proof}[Proof sketch]
The proof is based on the sequential composition of two DP mechanisms as stated in Proposition~\ref{prop:DP-comp-post}.
\end{proof}
}

 As stated below, we just need to analyze the sensitivity of the quality function in the exponential mechanism and the sensitivity of the measure over the projected graph. 



\begin{lemma}\label{lemma:sensitivity}
    The sensitivity of $f\circ\pi_\theta(\cdot)$ in Algorithm~\ref{algo:graph_general} is $\theta$, where $\pi_\theta$ is the edge addition algorithm with the input $\theta$ and $f(\cdot)$ counts edges for $\mininconsistency$ and nodes with a positive degrees for $\problematic$.
\label{lemma:sens_lap}
\end{lemma}

\reva{
\begin{proof}[Proof sketch]
For $\problematic$, we can analyze a worst-case scenario where the graph is a star with $n$ nodes such that there is an internal node connected to all other $n-1$ nodes, and the threshold $\theta$ for edge addition is $n$. The edge addition algorithm would play a minimal role, and no edges would be truncated. For a neighboring graph that differs on the internal node, all edges of the graph are removed (connected to the internal node), and the $\problematic = 0$ (no problematic nodes), making the sensitivity for $\problematic$ in this worst-case $=n$.

For $\mininconsistency$, the proof is similar to prior work~\cite{day2016publishing} for publishing degree distribution that uses stable ordering to keep track of the edges for two neighboring graphs. We need to analyze the changes made to the degree of each node by adding one edge at a time for two graphs $\mathcal{G}$ and its neighboring graph $\mathcal{G}'$ with an additional node $v^+$. The graphs have the stable ordering of edges (\cref{def:stable_ordering}) $\Lambda$ and $\Lambda'$, respectively.  Assuming the edge addition algorithm adds a set of $t$ extra edges incident to $v^+$ for $\mathcal{G}'$, we can create $t$ intermediate graphs and their respective stable ordering of edges that can be obtained by removing from the stable ordering $\Lambda'$ each edge $t$ and others that come after $t$ in the same sequence as they occur in $\Lambda'$. We analyze consecutive intermediate graphs, their stable orderings, and the edges actually that end up being added by the edge addition algorithm. As the edge addition algorithm removes all edges of a node once an edge incident is added, we observe that only one of these $t$ edges is added. All other edges incident to $v^+$ are removed. We prove this extra edge leads to decisions in the edge addition algorithm that always restricts such consecutive intermediate graphs to differ by at most $1$ edge. This proves the lemma for $\mininconsistency$ as at most $t$ (upper bounded by $\theta$) edges can differ between two neighboring graphs. 
\end{proof}
}

\ifpaper
The full proof for the above lemma can be found in the full paper~\cite{full_paper}. 
\else
\fi
We now analyze the sensitivity of the quality function using both measures' sensitivity analysis.


\begin{lemma} \label{lemma:sens_quality}
The sensitivity of the quality function $q_{\epsilon_2}(\mathcal{G}, \theta_i)$ in Algorithm~\ref{algo:expo_mech_basic} defined in Equation~\eqref{eq:quality_function} is $\theta_{\max}=\max(\Theta)$. 
% for both $\mininconsistency$ and $\problematic$ .
\end{lemma}

\reva{
\begin{proof}[Proof sketch]
We prove the theorem for the $\mininconsistency$ measure and show that it is similar for $\problematic$. The sensitivity of the quality function is computed by comparing the respective quality functions of two neighboring graphs $\mathcal{G}$ and $\mathcal{G}'$ with an extra node. It is upper bound by the difference of two terms $\left(|\mathcal{G}'_{\theta_{\max}}.E| - |\mathcal{G}_{\theta_{\max}}.E|\right) - \left(|\mathcal{G}'_\theta.E| - |\mathcal{G}_\theta.E|\right)$. The first term $\left(|\mathcal{G}'_{\theta_{\max}}.E| - |\mathcal{G}_{\theta_{\max}}.E|\right)$ is the sensitivity of the measures, as already proved by Lemma~\ref{lemma:sens_lap} is equal to $\theta_{max}$. The second term $\left(|\mathcal{G}'_\theta.E| - |\mathcal{G}_\theta.E|\right)$ is always $\geq 0$ as  $|\mathcal{G}'_\theta.E| \geq |\mathcal{G}_\theta.E|$ as discussed in the proof for Lemma~\ref{lemma:sens_lap}.
\end{proof}
}

\ifpaper
\else
Proofs for \cref{thm:privacy_proof_dc_oblivious}, \cref{lemma:sens_quality}, and \cref{lemma:sens_lap} can be found in \cref{app:graph_general}.
\fi

\eat{
\begin{proof}
% \xh{double check the proof, there are a few issues.
% \begin{itemize}
% %    \item Check if Eqn 2 or Eqn 3. I think it should be Eqn 3.
%     \item The 2nd last inequality is incorrect; $ \theta_i$ is an upper bound, and subtracting an upper bound will not preserve $\leq$. Simply directly drop the 2nd term if it is non-negative.
%     \item mention the analysis applies to $\problematic$.
%  %   \item One important point is missing from the current proof is the reasoning for the 2nd last inequality $|\mathcal{G}'_{\theta_{\max}}.E| - |\mathcal{G}_{\theta_{\max}}.E| \leq \theta_{\max}$. This is actually related to the sensitivity of the measure over the projected graph (Lemma~\ref{lemma:sens_lap}. It should be highlighted.
% \end{itemize}}
We prove the lemma for the $\mininconsistency$ measure and show that it is similar for $\problematic$. Let us assume that $\mathcal{G}$ and $\mathcal{G}'$ are two neighbouring graphs and $\mathcal{G}'$ has one extra node $v^*$. 
    \begin{equation*}
        \begin{split}
            &\|q_{\epsilon_2}(\mathcal{G}, \theta) - q_{\epsilon_2}(\mathcal{G}^\prime, \theta)\| \leq -|\mathcal{G}_{\theta_{\max}}.E| + |\mathcal{G}_{\theta}.E| - \sqrt{2}\frac{\theta}{\epsilon_1} \\ &+ |\mathcal{G}'_{\theta_{\max}}.E| - |\mathcal{G}'_{\theta}.E| + \sqrt{2}\frac{\theta}{\epsilon_1} \\
            &\leq \left(|\mathcal{G}'_{\theta_{\max}}.E| - |\mathcal{G}_{\theta_{\max}}.E|\right) - \left(|\mathcal{G}'_\theta.E| - |\mathcal{G}_\theta.E|\right)\\
            &\leq \theta_{\max} - \left(|\mathcal{G}'_\theta.E| - |\mathcal{G}_\theta.E|\right) 
            \leq \theta_{\max}    
        \end{split}
    \end{equation*}
    The second last inequality is due to Lemma~\ref{lemma:sens_lap} that states that $|\mathcal{G}'_{\theta_{max}}.E| - |\mathcal{G}_{\theta_{max}}.E| \leq \theta_{max}$. The last inequality is because $|\mathcal{G}'_\theta.E| \geq |\mathcal{G}_\theta.E|$. Note that the neighboring graph $\mathcal{G}'$ contains all edges of $\mathcal{G}$ plus extra edges of the added node $v^*$. Due to the stable ordering of edges in the edge addition algorithm, each extra edge of $v^*$ either substitutes an existing edge or is added as an extra edge in $\mathcal{G}_\theta$. Therefore, the total edges $|\mathcal{G}'_\theta.E|$ is equal or larger than $|\mathcal{G}_\theta.E|$. We elaborate this detail further in the proof for Lemma~\ref{lemma:sens_lap}. For the $\problematic$ measure, the term in the last inequality changes to $|\mathcal{G}'_\theta.V_{>0}| - |\mathcal{G}_\theta.V_{>0}|$ and is also non-negative because $\mathcal{G}'$ contains an extra node that can only add and not subtract from the total number of nodes with positive degree.
\end{proof}
}




\eat{
\xh{Condense the lemmas below and move the proofs to the appendix, and comment it's quite similar to prior work~\cite{day2016publishing}.}

\begin{proof}
    In \cref{algo:graph_general}, Line 1 uses the exponential mechanism with $\epsilon_1$ to calculate $\theta^*$. This theta value is then used to compute the bounded graph, and finally, the inconsistency measure value is released with Laplace noise of $\epsilon_2$. Therefore, using composition properties of DP, \cref{algo:dc_oblivious} satisfies $\epsilon_1 + \epsilon_2$-node DP.
\end{proof}

The inconsistency measures $\mininconsistency$ and $\problematic$ can be directly computed on the bounded graph. For the $\mininconsistency$, we compute the total number of edges $\pi_\theta(\mathcal{G})$. The sensitivity analysis of $\mininconsistency$ on $\pi_\theta(\mathcal{G})$ is detailed in \cref{lemma:sens_mininconsistency}.

\begin{lemma}
    The sensitivity of $\mininconsistency(\pi_\theta(\mathcal{G}))$ is $\theta$, where $\pi_\theta$ is the edge addition algorithm with the user input $\theta$.
        % $$\| \mininconsistency(\pi_{\theta}^\Lambda(\mathcal{G})) - \mininconsistency(\pi_{\theta}^\Lambda(\mathcal{G}^\prime)) \| \leq \theta$$ 
    \label{lemma:sens_mininconsistency}
\end{lemma}

\ifpaper
The lemma can be proved by analyzing the changes made to the degree of each node in the graph by adding one edge at a time. The stable ordering of the edges allows us to keep track of the edges for two neighbouring graphs. Due to space constraints, we defer the proof to the full paper.  
\else
\proof
Let's assume without loss of generality that
$\mathcal{G}^{\prime}=\left(V^{\prime}, E^{\prime}\right)$ has an additional node $v^{+}$compared to $\mathcal{G}=$ $(V, E)$, i.e., $V^{\prime}=V \cup\left\{v^{+}\right\}, E^{\prime}=E \cup E^{+}$, and $E^{+}$is the set of all edges incident to $v^{+}$in $\mathcal{G}^{\prime}$. Let $\Lambda^{\prime}$ be the stable orderings for constructing $\pi_\theta\left(\mathcal{G}^{\prime}\right)$, and $t$ be the number of edges added to $\pi_\theta\left(\mathcal{G}^{\prime}\right)$ that are incident to $v^{+}$. Clearly, $t \leq \theta$ because of the $\theta$-bounded algorithm. Let $e_{\ell_1}^{\prime}, \ldots, e_{\ell_t}^{\prime}$ denote these $t$ edges in their order in $\Lambda^{\prime}$. Let $\Lambda_0$ be the sequence obtained by removing from $\Lambda^{\prime}$ all edges incident to $v^{+}$, and $\Lambda_k$, for $1 \leq k \leq t$, be the sequence obtained by removing from $\Lambda^{\prime}$ all edges that both are incident to $v^{+}$and come after $e_{\ell_k}^{\prime}$ in $\Lambda^{\prime}$. Let $\pi_\theta^{\Lambda_k}\left(\mathcal{G}^{\prime}\right)$, for $0 \leq k \leq t$, be the graph reconstructed by trying to add edges in $\Lambda_k$ one by one on nodes in $\mathcal{G}^{\prime}$, and $\lambda_k$ be the sequence of edges from $\Lambda_k$ that are actually added in the process. Thus $\lambda_k$ uniquely determines $\pi_\theta^{\Lambda_k}\left(\mathcal{G}^{\prime}\right)$; we abuse the notation and use $\lambda_k$ to also denote $\pi_\theta^{\Lambda_k}\left(\mathcal{G}^{\prime}\right)$. We have $\lambda_0=\pi_\theta(\mathcal{G})$, and $\lambda_t=\pi_\theta\left(\mathcal{G}^{\prime}\right)$.

In the rest of the proof, we show that $\forall k$ such that $1 \leq k \leq t$, at most 1 edge will differ between $\lambda_k$ and $\lambda_{k-1}$. This will prove the lemma because there are at most $t$ (upper bounded by $\theta$) edges that are different between $\lambda_t$ and $\lambda_0$.

To prove that any two consecutive sequences differ by at most 1 edge, let's first consider how the sequence $\lambda_k$ differs from $\lambda_{k-1}$. Recall that by construction, $\Lambda_k$ contains one extra edge in addition to $\Lambda_{k-1}$ and that this edge is also incident to $v^*$. Let that additional differing edge be $e_{\ell_k}^\prime = (u_j, v^+)$. In the process of creating the graph $\pi_\theta^{\Lambda_k}(\mathcal{G}^{\prime})$, each edge will need a decision of either getting added or not. The decisions for all edges coming before $e_{\ell_k}^{\prime}$ in $\Lambda^{\prime}$ must be the same in both $\lambda_k$ and $\lambda_{k-1}$. Similarly, after $e_{\ell_k}^{\prime}$, the edges in $\Lambda_k$ and $\Lambda_{k-1}$ are exactly the same. However, the decisions for including the edges after $e_{\ell_k}^{\prime}$ may or may not be the same. Assuming that there are a total of $s \geq 1$ different decisions, we will observe how the additional edge $e_{\ell_k}^{\prime}$ makes a difference in decisions. 


When $s=1$, the only different decision must be regarding differing edge $e_{\ell_k}^\prime = (u_j, v^+)$ and that must be including that edge in the total number of edges for $\lambda_k$. Also note that due to this addition, the degree of $u_j$ gets added by 1 which did not happen for $\lambda_{k-1}$. When $s>1$, the second different decision must be regarding an edge incident to $u_j$ and that is because degree of $u_j$ has reached $\theta$, and the last one of these, denoted by $(u_j, u_{i \theta})$ which was added in $\lambda_{k-1}$, cannot be added in $\lambda_k$. In this scenario, $u_j$ has the same degree (i.e., $\theta$ ) in both $\lambda_k$ and $\lambda_{k-1}$. Now if $s$ is exactly equal to 2, then the second different decision must be not adding the edge $(u_j, u_{i \theta})$ to $\lambda_k$. Again, note here that as $(u_j, u_{i \theta})$ was not added in $\lambda_k$ but was added in $\lambda_{k-1}$, there is still space for one another edge of $u_{i \theta}$. If $s>2$, then the third difference must be the addition of an edge incident to $u_{i \theta}$ in $\lambda_k$. This process goes on for each different decision in $\lambda_k$ and $\lambda_{k-1}$. Since the total number of different decisions $s$ is finite, this sequence of reasoning will stop with a difference of at most 1 in the total number of the edges between $\lambda_{k-1}$ and $\lambda_k$.
\qed
\fi

For the $\problematic$ measure, that is the total number of nodes in the graph that have a positive degree, we compute the number of nodes in the $\pi_\theta(\mathcal{G})$ that have degree 0 and subtract it from the total nodes in the graph. The sensitivity analysis of $\problematic$ is given by ~\cref{lemma:sens_problematic}. 

\begin{lemma}
    The sensitivity of $\problematic(\pi_\theta(\mathcal{G}))$ is $\theta$, where $\pi_\theta$ is the edge addition algorithm with user input $\theta$.
        % $$\| \problematic(\pi_\theta(\mathcal{G})) - \problematic(\pi_\theta(\mathcal{G}^\prime)) \| \leq \theta$$ 
    \label{lemma:sens_problematic}
\end{lemma}

\proof
Assume, in the worst case, the graph $\mathcal{G}$ is a star graph with $n$ nodes such that there exists an internal node that is connected to all other $n-1$ nodes. In this scenario, there are no nodes that have 0 degrees, and the $\problematic$ measure $= n-0 = 0$. If the neighbouring graph $\mathcal{G}^\prime$ differs on the internal node, all edges of the graph are removed are the $\problematic = n$. The edge addition algorithm $\pi_\theta$ would play a minimal role here as $\theta$ could be equal to $n$.
\qed




In Lemma~\ref{lemma:sens_quality}, we show the sensitivity computation for the quality function for the $\mininconsistency$ measure. The $\problematic$ measure has a similar analysis. 

\begin{lemma}
    For any two neighbouring graphs $\mathcal{G}$ and $\mathcal{G}^\prime$, the sensitivity of the quality function $q(\mathcal{G}, \theta_i)$ equals $\theta_{max}$,
    % $$\|q(\mathcal{G}, \theta_i) - q(\mathcal{G}^\prime, \theta_i)\| \leq \theta_{max}$$
    where $\theta_{max}$ is the maximum theta value over all candidate values $\theta_i \in \Theta$.
\end{lemma}\label{lemma:sens_quality}
}

%\subsubsection{Utility analysis}\label{sec:dc_oblivious_privay_util}

%For the utility analysis of \cref{algo:dc_oblivious}, we analyze the utility of the exponential mechanism that outputs the best value of $\theta^*$. As per Theorem~\ref{thm:utility_expo}, the exponential mechanism allows us to privately select an object $\theta$ from a set of objects $\Theta$ with a score comparable to the best score $OPT$ in $\Theta$ with an error that depends on the sensitivity, privacy budget $\epsilon$ and the total number of candidates $|\Theta|$. 
\paratitle{Utility analysis}
The utility of Algorithm~\ref{algo:graph_general} is directly encoded by the quality function of the exponential mechanism in Algorithm~\ref{algo:expo_mech_basic}. 
We first define the best possible quality function value for a given database and its respective graph as 
\begin{equation}
    q_{\opt}(D,\epsilon_2) = \max_{\theta\in \Theta} q_{\epsilon_2}(\graph,\theta)
\end{equation}
and the set of degree values that obtain the optimal quality value as 
\begin{equation}
 \Theta_{\opt} = \{\theta\in \Theta: q_{\epsilon_2}(\graph,\theta) = q_{\opt}(D,\epsilon_2) \}.   \end{equation} 
However, we define $e_{\text{bias}}$ as the difference in the number of edges or nodes in the projected graph $\mathcal{G}_{\theta}$ compared to that of $\mathcal{G}_{\theta_{\max}}$, instead of $\mathcal{G}$. This is to limit the sensitivity of the quality function. To compute the utility, we slightly modify the quality function without affecting the output of the exponential mechanism. 
\begin{equation}
    \tilde{q}_{\epsilon_2}(\mathcal{G},\theta) = q_{\epsilon_2}(\mathcal{G},\theta) + f(\mathcal{G}_{\theta_{\max}}) - f(\graph),
\end{equation}
where $f(\cdot)$ returns edge count for $\mininconsistency$ and the number of nodes with positive degrees for $\problematic$.
This modified quality function should give the same set of degrees $\Theta_{\opt}$ with optimal values equal to 
\begin{equation}
    \tilde{q}_{\opt}(D,\epsilon_2) = \max_{\theta\in \Theta} q_{\epsilon_2}(\graph,\theta) + f(\graph_{\theta_{\max}}) - f(\graph).
\end{equation}


Then, we derive the utility bound for Algorithm~\ref{algo:graph_general} based on the property of the exponential mechanism as follows. 


\begin{theorem}\label{thm:graph_general_utility} On any database instance $D$ and its respective conflict graph $\graph$, let $o$ be the output of Algorithm~\ref{algo:graph_general} with Algorithm~\ref{algo:expo_mech_basic} over $D$.  
Then,  with a probability of at least $1-\beta$, we have 
\begin{equation}
|o-a| \leq -\tilde{q}_{\opt}(D,\epsilon_2) + \frac{2 \theta_{\max}}{\epsilon_1} (\ln \frac{2|\Theta|}{|\Theta_{\opt}|\cdot \beta}) 
\end{equation}
where $a$ is the true inconsistency measure over $D$ and $\beta\leq \frac{1}{e^{\sqrt{2}}}$.
%\benny{I don't get this $\beta$. Where is it coming from? Is it our choice? Why not choose $\beta=0$? Do you mean that such $\beta$ exists? Please clarify.} \xh{$\beta$ affects the 2nd term, as beta goes smaller, the error bound increases. Add a proof sketch to give the intuition for the proof} \sm{I have added more information about $\beta$ in the proof sketch.}
\end{theorem}

% \begin{proof}[Proof sketch]
% \ag{add}
% \end{proof}

\ifpaper
\begin{proof}[Proof Sketch]
   We use the probabilistic utility bound of the exponential mechanism~\cite{mcsherry2007mechanism} that guarantees that a suitable candidate is sampled with probability $1-\beta$ for a given quality function. To prove the bound, we utilize the optimal quality function $\tilde{q}_{\opt}(D)$  and the error from the Laplace mechanism with the exponential mechanism's utility bound. The full proof is in the full version~\cite{full_paper}.
\end{proof}
\else
The proof can be found in ~\cref{app:graph_general_utility}.
\eat{
\begin{proof}
By the utility property of the exponential mechanism~\cite{mcsherry2007mechanism}, with at most probability $\beta/2$, Algorithm~\ref{algo:expo_mech_basic} will sample a bad $\theta^*$ with a  quality value as below
\begin{equation}
    q_{\epsilon_2}(\graph,\theta^*) \leq q_{\opt}(D,\epsilon_2) - \frac{2 \theta_{\max}}{\epsilon_1} (\ln \frac{2|\Theta|}{|\Theta_{\opt}|\beta})
\end{equation}
which is equivalent to 
\begin{equation}\label{eq:goodtheta}
    e_{\text{bias}}(\mathcal{G},\theta^*)  \geq -q_{\opt}(D,\epsilon_2) + \frac{2 \theta_{\max}}{\epsilon_1} (\ln \frac{2|\Theta|}{|\Theta_{\opt}|\beta}) -  \frac{\sqrt{2}\theta^*}{\epsilon_2}.
\end{equation}

With probability $\beta/2$, where $\beta\leq \frac{1}{e^{\sqrt{2}}}$,
we have 
\begin{equation}    
\text{Lap}(\frac{\theta^*}{\epsilon_2}) \geq
      \frac{\ln(1/\beta)\theta^{*}}{\epsilon_2} \geq \frac{\sqrt{2}\theta^*}{\epsilon_2}
      \end{equation}
Then, by union bound, with at most probability $\beta$, we have 
\begin{eqnarray}
   && |o-a| \nonumber\\
       &=& 
|f(\mathcal{G}_{\theta^*})+\text{Lap}(\frac{\theta^*}{\epsilon_2})-a|
            \nonumber  \\
    &\geq& a- f(\mathcal{G}_{\theta^*})+ \frac{\sqrt{2}\theta^*}{\epsilon_2}
 \nonumber \\
    &=& f(\mathcal{G})-f(\mathcal{G}_{\theta^*})+
     \frac{\sqrt{2}\theta^*}{\epsilon_2} \nonumber \\
    &=& f(\mathcal{G})-f(\mathcal{G}_{\theta_{\max}}) +
f(\mathcal{G}_{\theta_{\max}}) - f(\mathcal{G}_{\theta^*})
 +    \frac{\sqrt{2}\theta^*}{\epsilon_2} \nonumber\\    
    &=& f(\mathcal{G})-f(\mathcal{G}_{\theta_{\max}}) +
        e_{\text{bias}}(\mathcal{G},\theta^*)  + \frac{\sqrt{2}\theta^*}{\epsilon_2} \nonumber\\
     &\geq& -q_{\opt}(D,\epsilon_2) + f(\mathcal{G})-f(\mathcal{G}_{\theta_{\max}}) + \frac{2 \theta_{\max}}{\epsilon_1} (\ln \frac{2|\Theta|}{|\Theta_{\opt}|\beta})  \nonumber \\
      &=& -\tilde{q}_{\opt}(D,\epsilon_2) + \frac{2 \theta_{\max}}{\epsilon_1} (\ln \frac{2|\Theta|}{|\Theta_{\opt}|\beta})
\end{eqnarray}
\end{proof}
}
\fi

This theorem indicates that the error incurred by Algorithm~\ref{algo:graph_general} with Algorithm~\ref{algo:expo_mech_basic} is directly proportional to the log of the candidate size $|\Theta|$ and the sensitivity of the quality function. \reva{The $\beta$ parameter in the theorem is a controllable probability parameter. According to the accuracy requirements of a user's analysis, one may set $\beta$ as any value less than this upper bound. For example, if we set $\beta=0.01$, then our theoretical analysis of Algorithm 2 that says the algorithm's output being close to the true answer will hold with a probability of $1-\beta = 0.99$. We also show a plot to show the trend of the utility analysis as a function of $\beta$ in Appendix A.5~\cite{full_paper}.} Without prior knowledge about the graph, $\theta_{\max}$ is usually set as the number of nodes $n$, and $\Theta$ includes all possible degree values up to $n$, resulting in poor utility. 
Fortunately, for our use case, the edges in the graph arise from the DCs that are available to us. In the next section, we show how we can leverage these constraints to improve the utility of our algorithm by truncating candidates in the set $\Theta$.





\eat{
\begin{theorem}\label{thm:utility_proof}
Let $\mathcal{G}(V, E)$ be a private graph, and $OPT(\mathcal{G})=\max _{\theta \in \Theta} q(\mathcal{G}, \theta, |V|, \epsilon_1, \epsilon_2)$ be the quality attained by the best object $\theta$ with respect to the dataset $\mathcal{G}$ due to Algorithm~\ref{algo:dc_oblivious}, $M(\mathcal{G})$. If the set of objects that achieve the $OPT(\mathcal{G})$, $\Theta^*=\{\theta \in \Theta: q(\mathcal{G}, \theta, |V|, \epsilon_1, \epsilon_2)=OPT(\mathcal{G})\}$ has size $|\Theta^*| \geq 1$. Then
$$ \Pr \left[q(\mathcal{G}, M(\mathcal{G}), |V|, \epsilon_1, \epsilon_2) \leq OPT (\mathcal{G}) - \frac{2|V|}{\epsilon_1} (\ln |\Theta| + t) \right] \leq \exp(-t)$$,
where $\epsilon_1$ and $\epsilon_2$ are the privacy budgets for the exponential mechanism and measure calculation respectively, $q$ is the quality function that measures the quality of the minimum inconsistency measure $\mininconsistency$.
\end{theorem}


\proof

The result can be obtained by plugging in the sensitivity value of the utility function $\Delta_q = \theta_{max} = |V| $ to \cref{thm:utility_expo}. 

% \begin{equation*}
%     \begin{split}
%         \Pr\left[q(\mathcal{G}, M(\mathcal{G}), |V|, \epsilon_1, \epsilon_2) \leq OPT (\mathcal{G}) - \frac{2|V|}{\epsilon_1} (\ln |\Theta| + t) \right] \leq \exp(-t)
%     \end{split}
% \end{equation*}
\qed

According to Theorem~\ref{thm:utility_proof}, the utility of Algorithm~\ref{algo:dc_oblivious} is directly proportional to the number of candidates $|\Theta|$ and the sensitivity of the quality function equivalent to number of nodes in the graph $|V|$. However, in practice, these values can be extremely large depending on the density of the graph, which is an artifact of the number of conflicts in the dataset. Luckily, for our use case, these conflicts arise from the denial constraints in the constraint set $\constraintset$ that are available to us. In the next section, we show how we can make use of these constraints to improve the utility of our algorithm by truncating candidates in the set $\Theta$.

}



\subsection{Optimized Parameter Selection}\label{sec:dc_aware}

Our developed strategy to improve the parameter selection includes two optimization techniques. The overarching idea behind these optimizations is to gradually truncate large candidates from the candidate set $\Theta$ based on the density of the graph. For example, we observe that the Stock dataset~\cite{oleh_onyshchak_2020} has a sparse conflict graph, and its optimum degree for graph projection is in the range of $10^0-10^1$. In contrast, the graph for the Adult dataset sample~\cite{misc_adult_2} is extraordinarily dense and has an optimum degree $\theta$ greater than $10^3$, close to the sampled data size. 
Removing unneeded large candidates, especially those greater than the true maximum degree of the graph, can help the high sensitivity issue of the quality function and improve our chances of choosing a better bound. 

Our first optimization estimates an upper bound for the true maximum degree of the conflict graph and removes candidates larger than this upper bound from the initial candidate set. The second optimization is a hierarchical exponential mechanism that utilizes two steps of exponential mechanisms. The first output, $\theta^1$, is used to truncate $\Theta$ further by removing candidates larger than $\theta^1$ from the set, and the second output is chosen as the final candidate $\theta^*$. In the rest of this section, we dive deeper into the details of these optimizations and discuss their privacy analysis. 

% In Algorithm~\ref{algo:dc_aware}, we show the improved version of our algorithm. We truncate values in the candidate set $\Theta$ in a two-step process. In lines 1-2, we first use the constraints in $\constraintset$ that are FDs to find an upper bound of the maximum degree of vertices in the graph denoted by $\theta_{bound}$ and remove candidates larger than this threshold. Second, in line 3, we use a two-step exponential mechanism to choose a value of $\theta^*$. The output of the first step, $\theta^1$, is used to truncate $\Theta$ further by removing candidates larger than $\theta^1$. The output of the second step is chosen as the final candidate $\theta^*$. The rest of the algorithm lines 4-5 is similar to the DC oblivious counterpart of the algorithm, where we project the graph and compute the inconsistency measures. In the rest of the section, we dive deeper into the details of this algorithm and discuss the privacy analysis. 


\paratitle{Estimating the degree upper bound using FDs} \label{sec:dc_aware_ub}
%The maximum degree of a conflict graph $\graph$ is governed by the total number of conflicts and its density properties. As discussed earlier at the start of the section, the densities of different real-world conflict graphs vary immensely. In our use case, as the datasets that give rise to these conflict graphs are private, the density properties of the conflict graphs are unknown. 
Given a conflict graph $\graphsimple(V,E)$, we use $\degree(\graphsimple, v)$ to denote the degree of the node $v\in V$ in $\graphsimple$ and $\degree_{\max}(\graphsimple) = \max_{v\in V} \degree(\graphsimple, v)$ to denote the maximum degree in $\graphsimple$. We estimate $\degree_{\max}$ by leveraging how conflicts were formed for its corresponding dataset $D$ under $\constraintset$. 

The degree for each vertex in $\graphsimple$ can be found by going through each tuple $t$ in the database $D$ and counting the tuples that violate the $\constraintset$ jointly with $t$.
However, computing this value for each tuple is computationally expensive and highly sensitive, making it impossible to learn directly with differential privacy.  
We observe that the conflicts that arise due to functionality dependencies (FDs)  depend on the values of the left attributes in the FD. 
\begin{example}
    Consider the same setup as  Example~\ref{example:running_example}  and an FD 
$\sigma: \text{Capital}\rightarrow \text{Country}$. 
%    $\sigma : \forall t_i, t_j \dots \in D, \neg(t_i[Occupation] = t_j[Occupation] \land t_i[Income] \neq t_j[Income])$. 
We can see that the number of violations added due to the erroneous grey row is 3. This number is also one smaller than the maximum frequency of values occurring in the Capital attribute, and the most frequent value is ``Ottawa''.
\end{example}

Based on this observation, we can derive an upper bound for the maximum degree of a conflict graph if it involves only FDs, and this upper bound has a lower sensitivity. We show the upper bound in Lemma~\ref{lemma:fd_theta} for one FD first and later extend for multiple FDs.  

\begin{lemma}\label{lemma:fd_theta}
Given a database $D$ and 
a FD $\sigma: X\rightarrow Y$ as the single constraint,
where $X = \{A_1,\dots,A_k\}$ and $Y$ is a single attribute. For its respective conflict graph $\graphsimple^D_{\Sigma = \{\sigma\}}$, simplified as $\graphsimple^D_{\sigma}$, we have the maximum degree of the graph $\degree_{\max}(\graphsimple^D_{\sigma})$ upper bounded by
    \begin{equation}
     \degreebound(D,X)
    =
    \max_{\vec{a_X} \in \dom(A_1) \times \ldots \times \dom(A_k) }  \normalfont{\text{freq}}(D, \vec{a_X})-1, 
    \end{equation}    
where $\normalfont{\text{freq}}(D, \vec{a_X})$ is the frequency of values $\vec{a_X}$ occurring for the attributes $X$ in the database $D$. 
The sensitivity for $\degreebound(D,X)$ is 1. %all $\normalfont{\text{freq}}(D, a_x)$ together is 1. 
\end{lemma}

\begin{proof}
An FD violation can only happen to a tuple $t$ with other tuples $t'$ that share the same values for the attributes $X$. 
Let $\vec{a_X}^*$ be the most frequent value for $X$ in $D$, i.e., 
$$\vec{a_X}^*=\text{argmax}_{\vec{a_X} \in \dom(A_1) \times \ldots \times \dom(A_k) } \normalfont{\text{freq}}(D, \vec{a_X}).$$
In the worst case, a tuple $t$ has the most frequent value $\vec{a_X}^*$ for $X$ but has a different value in $Y$ with all the other tuples with $X=\vec{a_X}^*$. Then the number of violations involved by $t$ is $\text{freq}(D,\vec{a_X}^*)-1$.

Adding a tuple or removing a tuple to a database will change, at most, one of the frequency values by 1. Hence, the sensitivity of  the maximum frequency values is 1.
\end{proof}

Now, we will extend the analysis to multiple FDs. 
\begin{theorem}\label{theorem:fd_theta}
Given a database $D$ and a set of FDs 
$\constraintset=\{\sigma_1,\ldots,\sigma_l \}$, for its respective conflict graph $\graph$, we have the maximum degree of the graph $\degree_{\max}(\graph)$ upper bounded by
\begin{eqnarray}
    \degreebound(D,\Sigma) 
    %&\leq& 
    %\sum_{(\sigma: X\rightarrow Y)\in \Sigma} 
     %    \degree_{\max}(\graphsimple^D_{\sigma}) \nonumber \\
     =      \sum_{(\sigma: X\rightarrow Y)\in \Sigma} 
         \degreebound(D,X)
\end{eqnarray}
\end{theorem}


\eat{
\begin{lemma}\label{lemma:fd_theta}
    For any conflict graph $\mathcal{G} (V,E)$ and a functional dependency of the form $X \rightarrow Y$ where $X,Y\subseteq\{A_1,\dots,A_m\}$ and $X$ may have multiple attributes $X = \{A_1,\dots,A_k\}$ and $Y$ is a single attribute,
    \begin{equation}
       \theta_{bound}^\sigma = \max_{v \in V}(\text{deg}(v)) 
    \leq  \max_{a_x \in \dom(A_1) \times \ldots \times \dom(A_k) } |\text{freq}(a_x)|, 
    \end{equation}    
     where $\theta_{bound}^\sigma$ denotes the max theta due to the FD and $freq(a_x)$ calculates the frequency of any value $a_x$ occurring in the attributes of $X$ in the dataset. 
\end{lemma}


\proof
Consider an FD constraint $X \rightarrow Y$ with $X = \{A_1,\dots,A_k\}$, a single attribute $Y$ and a corresponding error $e(A_1, \dots, A_k, Y)$ that changes the value a cell $a_x = t[A_x]$ in any tuple $t$ of the dataset such that attribute $A_x$ is in the FD constraint $A_x \in \{A_1, \dots, A_k, Y\}$. This error could be viewed as a typo. There could be two cases based on the attribute $A_x$. First, if $A_x$ belongs to an attribute $X$, i.e., $A_x \in \{A_1, \dots, A_k\}$, we observe that the error $e$ will add violations to the dataset based on the frequency of the value $a_x$ occurring in the attribute $A_x$. In the worst case, $a_x$ could be the most occurring value, and the number of violations that could be added for the FD is the most frequent value in the domain of all $a_x \in \dom(A_{\phi_1}) \times \dom(A_{\phi_k})$. Suppose the error $e$ is in the attribute $Y$ in the second case. In this case, the number of violations added will also be equal to the frequency of attributes in $X$ but equal to the joint occurrence of those values that participated in the tuple $t$. However, such a joint frequency is upper-bounded by the single frequency of attributes in the former case.   \xh{is the last line part of the proof or it is a statement?} \qed
}

\eat{
Lemma~\ref{lemma:fd_theta} shows that if there is one FD, the maximum number of violations that could be added due to this FD is controlled by the most occurring value that appears in the equality formulas of that FD. The example below demonstrates this lemma.
% We illustrate this using the example below and show its sensitivity using Lemma~\ref{lemma:sens_fd_theta}. 

\begin{example}
    Consider the same setup as  Example~\ref{example:running_example} and assume an FD $\sigma : \forall t_i, t_j \dots \in D, \neg(t_i[Occupation] = t_j[Occupation] \land t_i[Income] \neq t_j[Income])$. We can see that the number of violations added due to the erroneous grey row is 3 which is also the max frequency of values occurring in the Occupation attribute (Doctor). 
\end{example}
}
% Lemma~\ref{lemma:sens_fd_theta} gives the sensitivity of this $\theta_{bound}^\sigma$ calculation for one FD. 

\eat{

The sensitivity of calculating this bound is fortunately also low as it deals with the frequency of values in the datasets. 

\begin{lemma}\label{lemma:sens_fd_theta}
    The sensitivity of $\theta_{bound}^\sigma(D)$ equals 1. 
     % $$\|\theta_{max}^\sigma(D) - \theta_{max}^\sigma(D^\prime)\| \leq 1$$
\end{lemma}
\proof
$\theta_{bound}^\sigma$ calculates the most occurring value in the domain of the equality attributes. When a row is added or removed from the dataset, the frequency of any value in the domain can only change by 1. \qed
}


\eat{
The bound $\theta_{bound}^\sigma$ can be extended to more than one FD $\sigma_1, \dots, \sigma_k$ by summing over all the max frequencies over the equality attributes in the FDs as shown in Theorem~\ref{theorem:fd_theta}.

\begin{theorem}\label{theorem:fd_theta}
    For any conflict graph $\mathcal{G} (V,E)$ and a constraint set $\constraintset = [\sigma_1, \dots, \sigma_k]$ of all functional dependencies of the form $X \rightarrow Y$, 
    \begin{equation}
       \theta_{bound} = \sum_{\sigma_i} \theta_{bound}^{\sigma_i} 
    \end{equation}    
     where $\theta_{max}^{\sigma_i}$ denotes the max theta due to the FD $\sigma_i$ as in Lemma~\ref{lemma:fd_theta}.
\end{theorem}
}

\proof
By Lemma~\ref{lemma:fd_theta}, 
for each FD $\sigma: X\rightarrow Y$, a tuple may violate at most $\degreebound(D,X)$ number of tuples. In the worst case, the same tuple may violate all FDs. \qed

% The sum upper bound assumes that all FDs may violate a tuple and works better for denser datasets. In contrast, the max upper bound assumes that a tuple is violated by only one FD and is superior for sparser datasets. For example in our experiments, we observe that the Adult dataset has better utility with sum bound and the Flight dataset with max bound. One may choose to estimate the exact density of the dataset by computing all $2^n$ combinations of the $n$ FD's equality attributes marginals\footnote{The sum of all possible combinations of $n$ distinct things is $2^n$ as $\Comb{n}{0} + \Comb{n}{1} + \dots + \Comb{n}{n} = 2^n.$}. However, this would be too ineffective with a limited privacy budget and would still not be exact as there could be other types of denial constraints other than FDs in the constraint set $\constraintset$.Therefore based on this observation, we take a middle-ground strategy that involves truncating all $\theta$ candidates in the total candidate set $\Theta$ that are greater than the max upper bound and add an extra candidate to represent the sum bound, i.e, $\theta = |D|$ to account for datasets that have greater density. 

We will spend some privacy budget $\epsilon_0$ to perturb the upper bound $\degreebound(D,X)$ for all FDs with LM and add them together. Each FD is assigned with  $\epsilon_0/|\Sigma_{\text{FD}}|$, where
$\Sigma_{\text{FD}}$ is the set of FDs in $\Sigma$. We denote this perturbed upper bound as $\noisydegreebound$ and add it to the candidate set $\Theta$ if absent. 


%Algorithm~\ref{algo:max_theta} delineates the process for computing this upper bound $\theta_{bound}$.  In Lines 1-2, we initialize a constraint list for FDs, $\constraintset_{FD}$, and store all constraints in $\constraintset$ that are FDs in the list. In line 3, we initialize a variable to store $\theta_{max}$ and the privacy budget to compute the $\theta_{max}$ for FD. In Lines 4-6, we go over each FD $\sigma$ in $\constraintset_{FD}$ and find their equality attributes in $A_{eq}$. The frequency of all domain values in $A_{eq}$ is computed, and then the max is stored in $\theta_{max}^\sigma$ in lines 8-9. In lines 10-11, we add noise to the $\theta_{max}$ and sum all the $\theta_{max}^\sigma$ to find the final value of $\theta_{max}$. This algorithm has a computation complexity of $\mathcal{O}(n|\constraintset|)$.
%We illustrate it in Example~\ref{example:max_bound}.

\eat{
\begin{algorithm}[t]
\caption{Maximum bound $\theta_{bound}$ calculation \xh{update notation}}\label{algo:max_theta}
    \KwData{Dataset $D$, Constraint set $\constraintset$, Privacy budget $\epsilon$}
    \KwResult{Max bound $\theta_{bound}$}
    Initialize FD constraints $\constraintset_{FD} = []$\;
    For each $\sigma \in \constraintset$, add $\sigma$ to $\constraintset_{FD}$ if $\sigma$ is FD\;
    Initialize $\theta_{bound} = 0$ and $\epsilon_{FD} = \epsilon/|\constraintset_{FD}|$\;
    \For{$\sigma$ in $\constraintset_{FD}$}{
        Initialize  $\theta_{bound}^\sigma = 0$\;
        Get equality attributes of $\sigma$ in $A_{eq}$\;
        \For{$a$ in $\dom(A_{eq)}$}{
            \If{Frequency of $a \geq \theta_{bound}^\sigma$ }{
                $\theta_{bound}^\sigma = freq(a)$\;
            }
        }
        Add noise $\theta_{bound}^\sigma = \theta_{bound}^\sigma + Lap(1/\epsilon_{FD})$ \;
        $\theta_{bound} = \theta_{bound} + \theta_{bound}^\sigma$\;
    }
   return $\theta_{bound}$\;
\end{algorithm}
}

\eat{
\begin{example}~\label{example:max_bound}
    Consider the same setup as Example~\ref{example:running_example}. The $\theta_{bound}$ for the example dataset can be computed using its only FD constraint, $\sigma: \neg (t_i[country]=t_j[country] \land t_i[capital] \neq t_j[capital])$. We go through every domain value of the equality attribute $country$ and compute $\theta_{max} = 3$. 
\end{example}}

% \begin{algorithm}[t]
% \caption{Optimized EM for parameter selection}
% \label{algo:em_opt}
%     \KwData{Graph $\mathcal{G} (V,E)$, candidate set $\Theta=\{1,\ldots,|V|\}$, quality function $q$, privacy budget $\epsilon_1, \epsilon_2$ }
%     \KwResult{Candidate $\theta^*$}
    
%     \If{$\Sigma$ mainly consists of FDs} {$\epsilon_0\gets\epsilon_1/4$, $\epsilon_1 \gets \epsilon_1-\epsilon_0$\\
%     Compute noisy upper bound $\noisydegreebound\gets \sum_{\sigma:X\rightarrow Y} (\degreebound(D,X)+\text{Lap}(|\Sigma_{\text{FD}}|/\epsilon_0))$\\
%      }{}
%     Prune candidates $\Theta \gets \{\theta\in \Theta~|~\theta \leq \noisydegreebound\} \cup \{|V|\}$ \\
%     Set $\theta_{\max}\gets \min(\noisydegreebound,|V|)$ \\ 
%     \For{$s\in \{1,2\}$ }{
%    For each $\theta_i \in \Theta$, compute $q_{\epsilon_2}(\mathcal{G}, \theta_i)$     \commenttext{// See Equation~\eqref{eq:quality_function}}
%     \\
%     Sample $\theta^*$ with prob $\propto \exp( \frac{\frac{\epsilon_1}{2} q_{\epsilon_2}(\mathcal{G}, \theta_i)}{2\theta_{\max}})$ \\
%     Prune candidates $\Theta \gets \{\theta\in \Theta~|~\theta \leq \theta^*\}$ \\
%     Set $\theta_{\max}\gets \theta^*$  
%     }
    
%     %For each $\theta_i \in \Theta$, compute $q_{\epsilon_2}(\mathcal{G}, \theta_i)$     \commenttext{// See Equation~\eqref{eq:quality_function}}\\
%     %Sample $\theta_1$ with prob $\propto \exp( \frac{\frac{\epsilon_1}{2} q_{\epsilon_2}(\mathcal{G}, \theta_i)}{2\theta_{\max}})$ \\
%     %Prune candidates $\Theta \gets \{\theta\in \Theta~|~\theta \leq \theta_1\}$ \\
%     %Set $\theta_{\max}\gets \theta_1$ \\ 
%     %For each $\theta_i \in \Theta$, compute $q_{\epsilon_2}(\mathcal{G},\theta_i)$  \commenttext{// See Equation~\eqref{eq:quality_function}}\\
%     %Sample $\theta^*$ with prob $\propto \exp( \frac{\frac{\epsilon_1}{2} q_{\epsilon_2}(\mathcal{G}, \theta_i)}{2 \theta_{\max}})$\\
%     {\bf Return} $\theta^*$\\
% \end{algorithm}


\paratitle{Extension to general DCs}
The upper bound derived in Theorem~\ref{theorem:fd_theta}
only works for FDs but fails for general DCs. General DCs have more complex operators, such as ``greater/smaller than,'' in their formulas. Such inequalities require the computation of tuple-specific information, which is hard with DP. For example, consider the DC $\sigma: \neg(t_i[gain] > t_j[gain] \land t_i[loss] < t_j[loss])$ saying that if the gain for tuple $t_i$ is greater than the gain for tuple $t_j$, then the loss for $t_i$ should also be greater than $t_j$. We can observe that similar analyses for FDs do not work here as the frequency of a particular domain value in $D$ does not bound the number of conflicts related to a tuple. 
Instead, we have to iterate each tuple $t$'s gain value and find how many other tuples $t'$s violate this gain value. In the worst case, such a computation may have a sensitivity equal to the data size. Therefore, estimation using DCs may result in much noise, especially when the dataset has fewer conflicts, and the noise is added to correspond to the large sensitivity.

Our experimental study (\Cref{sec:experiments}) shows that datasets with general DCs have dense conflict graphs, which favors larger $\theta$s for graph projection. Hence, if we learn a small noisy upper bound  $\noisydegreebound$ based on the FDs with LM, we will first prune all degree candidates smaller than $\noisydegreebound$, but then include $|V|$, which corresponds to the case when no edges are truncated, and Laplace mechanism is applied with the largest possible sensitivity $|V|$, i.e., 
\begin{equation}\label{eq:fd-bound}
    \Theta' = \{\theta\in \Theta~|~\theta \leq \noisydegreebound\} \cup \{|V|\}.
\end{equation}
Though the maximum value in $\Theta'$ is $|V|$, the sensitivity of the quality function over the candidate set $\Theta'$ remains $\noisydegreebound$. For the $|V|$ candidate, the quality function only depends on the Laplace error $\frac{\sqrt{2}|V|}{\epsilon_2}$ and has no error from $e_{\text{bias}}$ as no edges will be truncated. 
% \revc{Although noisy and not as robust as our bound for FDs, our approach is cheap and performs well in practice. As we will see in the experiment section, this approach works well for the dense Adult and Flight datasets. However, there is scope for improvement in developing a strategy for general DCs. Our developed strategy justifies the separation between a conflict graph approach and more general approaches in future work.}
% \revc{Our approach performs well in practice despite being less robust than our bound for FDs. In \Cref{sec:experiments}, we show that this approach works well for the dense Adult~\cite{adult} and Flight~\cite{flight} datasets. However, there is merit in developing a more competent strategy for general DCs in future work.}
\revc{Despite being tailored for FDs, we show that, in practice, our approach is cheap and performs well for DCs. In \Cref{sec:experiments}, we show that this approach works well for the dense Adult~\cite{adult} dataset where we compute the $\problematic$ using this strategy in Figure~\ref{fig:comparing_strategies}. Developing a specific strategy for DCs is an important direction of future work.}

%To compensate for the analysis of general DCs, we add an extra candidate to the candidate set $\Theta$ equal to $|V|$. This candidate corresponds to the no truncation of the graph and hence has no error from the first term $e_{measure}$ of the quality function $q$. 
In practice, one may skip this upper bound calculation process and skip directly to the two-step exponential mechanism if it is known that the graph is too dense or contains few FDs and more general DCs. We discuss this in detail in the experiments section. 


\paratitle{Hierarchical EM}\label{sec:dc_aware_hier_expo_mech}
The upper bound $\degreebound$ may not be tight as it estimates the maximum degree in the worst case. The graph would be sparse with low degree values, and there is still room for pruning. 
To further prune candidate values in the set $\Theta$, we use a hierarchical EM that first samples a degree value $\theta^*$ to prune values in $\Theta$ and then sample again another value $\theta^*$ from the remaining candidates as the final degree the graph projection. 
Our work uses a two-step hierarchical EM by splitting the privacy budget equally into halves. One may extend this EM to more steps at the cost of breaking their privacy budget more times, but in practice, we notice that a two-step is enough for a reasonable estimate. 

\begin{example}\label{example:parameter_selection} 
    Consider the same setup as Example~\ref{example:running_example}. For this dataset, we start with $\Theta = [1, 2, 3]$ and %as we saw in Example~\ref{example:max_bound}, 
    the $\theta_{\max}$ for this setup is 3. Assume no values are pruned in the first optimization phase. 
    We compare a single versus a two-step hierarchical EM for the second optimization step. From Table~\ref{tab:example_expo_prob} in Example~\ref{example:quality_function}, we know that the $\theta_1$ has the best quality. However, as the quality values are close, the probability of choosing the best candidate is similar, as shown in Table~\ref{tab:example_expo_prob} with $\epsilon=1$.
    \begin{table}[]
        \centering
        \begin{tabular}{|c|c|c|c|c|}
             \hline
             $\theta$ & $q$ & EM & 2-EM ($\theta^*_1 = \theta_3$) & 2-EM ($\theta^*_1 = \theta_2$)\\
             \hline
             1 & $-3.41$ & $0.35$ & $0.51$ & $1$\\
             2 & $-3.82$ & $0.33$ & $0.49$ & -\\
             3 & $-4.24$ & $0.31$ & - & -\\
             \hline
        \end{tabular}
        \caption{Probabilities of candidates with the exponential mechanism (EM) vs.~the two-step hierarchical exponential mechanism (2-EM). $\theta^*_1$ refers to the first-step output of 2-EM.}
        \label{tab:example_expo_prob}
    \end{table}
    The exponential mechanism will likely choose a suboptimal candidate in such a scenario as the probabilities are close. But if a two-step exponential mechanism is used even with half budget $\epsilon = 0.5$, the likelihood of choosing the best candidate $\theta_1$ goes up to $0.51$ if the first step chose $\theta_3$ or $1$ if the first step chosen $\theta_2$.
\end{example}


\begin{algorithm}[t]
\caption{Optimized EM for parameter selection}
\label{algo:em_opt}
    \KwData{Graph $\mathcal{G} (V,E)$, candidate set $\Theta=\{1,\ldots,|V|\}$, quality function $q$, privacy budget $\epsilon_1, \epsilon_2$ }
    \KwResult{Candidate $\theta^*$}
    
    \If{$\Sigma$ mainly consists of FDs} {$\epsilon_0\gets\epsilon_1/4$, $\epsilon_1 \gets \epsilon_1-\epsilon_0$\\
    Compute noisy upper bound $\noisydegreebound\gets \sum_{\sigma:X\rightarrow Y} (\degreebound(D,X)+\text{Lap}(|\Sigma_{\text{FD}}|/\epsilon_0))$\\
     }{}
    Prune candidates $\Theta \gets \{\theta\in \Theta~|~\theta \leq \noisydegreebound\} \cup \{\noisydegreebound, |V|\}$ \\
    Set $\theta_{\max}\gets \min(\noisydegreebound,|V|)$ \\ 
    \For{$s\in \{1,2\}$ }{
   For each $\theta_i \in \Theta$, compute $q_{\epsilon_2}(\mathcal{G}, \theta_i)$     \commenttext{// See Equation~\eqref{eq:quality_function}}
    \\
    Sample $\theta^*$ with prob $\propto \exp( \frac{\frac{\epsilon_1}{2} q_{\epsilon_2}(\mathcal{G}, \theta_i)}{2\theta_{\max}})$ \\
    Prune candidates $\Theta \gets \{\theta\in \Theta~|~\theta \leq \theta^*\}$ \\
    Set $\theta_{\max}\gets \theta^*$  
    }
    
    %For each $\theta_i \in \Theta$, compute $q_{\epsilon_2}(\mathcal{G}, \theta_i)$     \commenttext{// See Equation~\eqref{eq:quality_function}}\\
    %Sample $\theta_1$ with prob $\propto \exp( \frac{\frac{\epsilon_1}{2} q_{\epsilon_2}(\mathcal{G}, \theta_i)}{2\theta_{\max}})$ \\
    %Prune candidates $\Theta \gets \{\theta\in \Theta~|~\theta \leq \theta_1\}$ \\
    %Set $\theta_{\max}\gets \theta_1$ \\ 
    %For each $\theta_i \in \Theta$, compute $q_{\epsilon_2}(\mathcal{G},\theta_i)$  \commenttext{// See Equation~\eqref{eq:quality_function}}\\
    %Sample $\theta^*$ with prob $\propto \exp( \frac{\frac{\epsilon_1}{2} q_{\epsilon_2}(\mathcal{G}, \theta_i)}{2 \theta_{\max}})$\\
    {\bf Return} $\theta^*$\\
\end{algorithm}


% \paratitle{Putting together}
\paratitle{Incorporating the optimizations into the algorithm}
Algorithm~\ref{algo:em_opt} outlines the two optimization techniques. First, we decide when to use the estimated upper bound for the maximum degrees, for example, when the constraint set $\Sigma$ mainly consists of FDs. We will spend part of the budget $\epsilon_0$ from $\epsilon_1$ to perturb the upper bounds $\degreebound(D,X)$ for all FDs with Laplace mechanism and add them together (lines 1-3). The noisy upper bound $\noisydegreebound$ prunes the candidate set (line 4). We also add $|V|$ to the candidate set if there are general DCs in $\Sigma$, and then set the sensitivity of the quality function $\theta_{\max}$ to be the minimum of the noisy upper bound or $|V|$ (line 5).
Then, we conduct the two-step hierarchical exponential mechanism for parameter selection (lines 6-10). Lines 7-8 work similarly to the previous exponential mechanism algorithm with half of the remaining $\epsilon_1$, where we choose a $\theta^*$ based on the quality function. However, instead of using it as the final candidate, we use it to prune values in $\Theta$ and improve the sensitivity $\theta_{\max}$ for the second exponential mechanism (lines 9-10). Then, we repeat the exponential mechanism and output the sampled $\theta^*$ (line 11).
Algorithm~\ref{algo:em_opt} has a similar complexity of $O(|\Theta|m)$ as Algorithm~\ref{algo:expo_mech_basic}, where $|E|$ is the edge size of the graph. The overall Algorithm~\ref{algo:graph_general} has a complexity of $O(|\Sigma|n^2+|\Theta|m)$.

%\xh{i dropped some of the complexity analysis of the original text, may add it here.}

%In lines 4-5, the second exponential mechanism finds another value $\theta^*$ using the same quality function and returns it for the graph projection.  This algorithm has a computation complexity of $\mathcal{O}(|\Theta|)$. We illustrate the two optimization steps together for our running example in Example~\ref{example:parameter_selection}.

\eat{
\begin{algorithm}
\caption{Two-step parameter selection algorithm}
\label{algo:two_step_expo_mech}
    \KwData{Graph $\mathcal{G} (V,E)$, candidate set $\Theta=\{1,\ldots,n\}$, quality function $q$, privacy budget $\epsilon_1, \epsilon_2$ }
    \KwResult{Candidate $\theta^*$}
    Find max value in $\Theta$ as $\theta_{max}$\\
    For each candidate $\theta_i \in \Theta$, compute $q(\mathcal{G}, \theta_i, \theta_{max},  \epsilon_2)$ \;
    Pick $\theta_1$ with prob $\propto \exp( \frac{\frac{\epsilon_1}{2} q(\mathcal{G}, \theta_i, \theta_{max}, \epsilon_2)}{2\Delta_q})$\;
    Truncate all values in $\Theta$ that are greater than $\theta_1$ and pick new $\theta_{max}$\;
    For each candidate $\theta_i \in \Theta$, compute $q(\mathcal{G}, \theta_i, \theta_{max},  \epsilon_2)$ \;
    Pick $\theta^*$ with prob $\propto \exp( \frac{\frac{\epsilon_1}{2} q(\mathcal{G}, \theta_i, \theta_{max}, \epsilon_2)}{2\Delta_q})$\;
    return $\theta^*$\;
\end{algorithm}
}

\paratitle{Privacy and utility analysis}
The privacy analysis of the optimizations depends on the analysis of three major steps: $\degreebound$ computation with the Laplace mechanism, the two-step exponential mechanism, and the final measure calculation with the Laplace mechanism. By sequential composition, we have Theorem~\ref{thm:privacy_proof_dc_aware}. 


\begin{theorem}\label{thm:privacy_proof_dc_aware}
    Algorithm~\ref{algo:graph_general} with the optimized EM in  Algorithm~\ref{algo:em_opt} satisfies $(\epsilon_1 + \epsilon_2)$-DP.
\end{theorem}
\reva{
\begin{proof}[Proof sketch]
The proof is similar to Theorem~\ref{thm:privacy_proof_dc_oblivious} and is due to the composition property of DP as stated in Proposition~\ref{prop:DP-comp-post}.
\end{proof}
}

We show a tighter sensitivity analysis for the quality function in EM over the pruned candidate set. The sensitivity analysis is given by Lemma~\ref{lemma:sens_quality_2stepEM} and is used for $\theta_{\text{max}}$ in line 8 of Algorithm~\ref{algo:em_opt}.



\begin{lemma} \label{lemma:sens_quality_2stepEM}
The sensitivity of $q_{\epsilon_2}(\mathcal{G}, \theta_i)$ in the 2-step EM (Algorithm~\ref{algo:em_opt})
defined in Equation~\eqref{eq:quality_function} is $\theta_{\max}= \min(\noisydegreebound,|V|)$ for 1st EM step and $\theta_{\max}=\theta^*$ for the 2nd EM step.
\end{lemma}

\reva{
\begin{proof}[Proof sketch]
The proof follows from \Cref{lemma:sens_quality}, substituting the $\theta_{max}$ with the appropriate threshold values for each EM step.
\end{proof}
}

\ifpaper
\else
The proofs for the theorem and lemma are at \cref{app:privacy_proof_dc_aware} and \cref{app:sens_quality_2stepEM}.
\fi 

%We defer the proofs to the full paper. 
The utility analysis in Theorem~\ref{thm:graph_general_utility} for Algorithm~\ref{algo:graph_general} with the basic EM (Algorithm~\ref{algo:expo_mech_basic}) still applies to the optimized EM (Algorithm~\ref{algo:em_opt}). The basic EM usually has $\theta_{\max}=|D|=|V|$ and the full budget $\epsilon_1$, while the optimized EM has a much smaller $\theta_{\max}$ and slightly lower privacy budget when the graph is sparse. 
% Hence, we should see 
In practice (\Cref{sec:experiments}), we see significant utility improvements by the optimized EM for sparse graphs. When the graph is dense, we see the utility degrade slightly due to a smaller budget for each EM. However, the degradation is negligible with respect to the true inconsistency measure.  


%\begin{proof}    The privacy budget for our algorithm with optimizations is split three ways. The upper bound $\theta_{max}$ estimation uses $\epsilon_0$. The two-step hierarchical exponential mechanism with $\epsilon_1$ budget split into two halves to calculate $\theta^*$. This theta value is then used to compute the bounded graph, and finally, the inconsistency measure value is released with Laplace noise of $\epsilon_2$. Therefore, using composition properties of DP, \cref{algo:dc_oblivious} with optimizations satisfies $\epsilon_0 + \epsilon_1 + \epsilon_2$-node DP.\end{proof}


% However, this method poses a risk of privacy leakage, necessitating the privatization of $\theta$ with some privacy budget. Luckily, its sensitivity remains low ($=1$) as adding or removing one row changes the maximum frequency by 1. We call this privatization of maximum frequency method \emph{high-frequency strategy}. We demonstrate the performance of this strategy and baseline strategies experimentally for choosing $\theta$ in Section~\ref{sec:experiments}. 












% \subsection{Leveraging Graph Projection for Minimizing Inconsistency and Problematic Nodes}\label{sec:graph-algorithms-graphproj}

% The measures of minimizing inconsistency ($\mininconsistency$) and identifying problematic nodes essentially pertain to the total number of edges ($|E|$) and the total number of nodes with positive degrees respectively in the conflict graph ($\graph(V, E)$). Both these measures are sensitive to the number of vertices in $\graph$ and can be significantly improved using graph projection algorithms. Due to space constraints, we will focus on the $\mininconsistency$ measure, which computes the total number of edges, discuss the associated challenges, and defer discussion on the problematic nodes measure which faces similar challenges.

% Several graph projection algorithms exist~\cite{kasiviswanathan2013analyzing, blocki2013differentially}, among which the "edge addition" algorithm~\cite{day2016publishing} stands out for its effectiveness in preserving most of the underlying graph structure. This algorithm takes as input the graph $\mathcal{G}= \graph = (V, E)$, a bound on the maximum degree of each vertex ($\theta$), and a stable ordering of the edges ($\Lambda$) to output a projected $\theta$-bounded graph ($\mathcal{G}\theta$). The algorithm operates by adding edges in the same order as $\Lambda$ such that each node has a maximum degree of $\theta$.
% Utilizing this algorithm, we first compute the $\theta$-bounded graph, $\mathcal{G}\theta(V, E)$, and then compute the measures by adding Laplace noise proportional to the new sensitivity. The sensitivity analysis of $\mininconsistency$ on the $\theta$-bounded graph $\mathcal{G}_\theta$ is detailed in \cref{lemma:sens_mininconsistency}.

% \begin{lemma}
%     For any $\mathcal{G}, \mathcal{G}^\prime$ that differ in one node an user input $\theta$, we have
%     \begin{equation*}
%         \| \text{\mininconsistency}(\mathcal{G}_\theta) - \text{\mininconsistency}(\mathcal{G}^\prime_\theta) \| \leq \theta
%     \end{equation*}
%     \label{lemma:sens_mininconsistency}
% \end{lemma}

% Due to space constraints, we defer the proof to the appendix. Accurate estimation of the measure necessitates selecting an appropriate value for $\theta$. $\theta$ serves as a user-defined hyperparameter and may vary significantly across datasets. Higher values of $\theta$ may preserve more edges of the original graph but introduce higher error from the Laplace noise, whereas smaller values of $\theta$ may have lower Laplace noise error but truncate a significant number of edges, thereby introducing substantial error as well.

% The exponential mechanism is a popular algorithm for choosing hyperparameters like $\theta$. However, it faces challenges in our use case. Firstly, the candidate set size ($\Theta$) is extensive, as the optimal $\theta$ value depends on the density of the conflict graph, varying excessively across datasets. Secondly, the sensitivity of the quality function $q(\mathcal{G}, \theta_i)$ is high. For instance, for the $\mininconsistency$ measure, the quality function depends on the total number of edges not truncated, with sensitivity proportional to the number of vertices ($|V|$). We explore an alternative heuristic approach based on the frequency of participating attributes in the constraint set $\constraintset$.

% The $\theta$ value is dependent on the density of the conflict graph and is influenced by the number of conflicts in the dataset, arising from the denial constraints in $\constraintset$. One strategy involves setting $\theta$ as the maximum frequency of any value occurring in the dataset over the participating attributes in $\constraintset$. However, this approach may leak privacy, necessitating the privatization of the $\theta$ value using part of the privacy budget. Fortunately, the sensitivity for this computation is low (1), as adding or removing a row can affect the maximum frequency by a maximum of 1. We demonstrate the performance of these strategies experimentally for choosing $\theta$ in Section~\ref{sec:experiments}.

% We observe that the $\theta$ value depends on the density of the conflict graph and is an artifact of the number of conflicts in the dataset. These conflicts arise from the denial constraints in the constraint set $\constraintset$. Each violating tuple in the dataset adds conflicts equivalent to the frequency of similar tuples occurring in the dataset. Therefore, one strategy can be to choose the $\theta$ value as the maximum frequency of any value occurring in the dataset over the participating attributes in the constraint set. It is however important to note here that the frequency may leak privacy and the $\theta$ value calculated using this strategy needs to be privatized using part of the privacy budget. Luckily, the sensitivity for this computation is 1 as adding or removing a row can affect the maximum frequency by a maximum of 1. We demonstrate the performance of these above strategies experimentally for choosing $\theta$ in Section~\ref{sec:experiments}. 

