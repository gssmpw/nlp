\begin{abstract}
% Self Supervised Learning has become popular recently. Several approaches have been developed which uses different surrogate tasks to learn without labels. Among those approaches is Masked-auto encoders, Which despite its simplicity has achieved the state of art on many modalities and downstream tasks. Despite its popularity, Designing a SSL paradigm for indoors and outdoors point-clouds have been a challenging task due to the shear size of the point clouds. Previous approaches have elevated this problem by working on a 2D grid of pillars and predicting occupancy and other features on such a grid, Or by using simple Encoder and a coarse occupancy Prediction. Due to the simplifying assumptions of the previous approaches, the results produced by the self supervised models are still lacking compared to the state of the art fully supervised counterparts. In this work we propose an efficient self learning task suitable for training the computationally intense SOTA 3D backbones. In our SSL paradigm the original point cloud is masked and the model is trained to predict the occupancy over the voxels falling within a certain sized-neighbourhood around the visible voxels. evaluating over the neighbourhood is more efficient than evaluating over the whole space, while providing a more consistent learning signal. Due to the simplicity of the design and its low computational requirement, we exploit the law of scaling and include multiple tasks which differ in the size of the visible (and in turn the neighbourhood) voxels. This Allows the model to learn multi scale features and further enhance the learnt representation. We design a masking strategy, namely, Hierarchical masking, which balances the amount of occupied voxels across the different scales, further improving the learnt representation. for the first time, The proposed method surpasses the performance of SOTA supervised architectures for point-clouds on multiple datasets and two different tasks, setting new SOTA performance. 

%The effectiveness of self-supervised learning for pretraining and representation learning is widely acknowledged. Among the developed approaches is Masked-auto encoders, Which despite its simplicity has achieved the state of art on many modalities and downstream tasks. Despite its popularity, Designing a MAE for real-world indoors and outdoors point-clouds have been a challenging task due to the irregularity and the shear size of the point clouds. Predicting occupancy or other features on a regular 2D or 3D grid have been proposed as a remedy to this problem. Due to the simplifying assumptions of the previous approaches, the results produced by the self supervised models are still lacking compared to the state of the art fully supervised counterparts. In this work we propose an efficient self learning task suitable for training the computationally intense SOTA 3D backbones. In our SSL paradigm the original point cloud is masked and the model is trained to predict the occupancy over the voxels falling within a certain sized-neighbourhood around the visible voxels. evaluating over the neighbourhood is more efficient than evaluating over the whole space, while providing a more consistent learning signal. Due to the simplicity of the design and its low computational requirement, we exploit the law of scaling and include multiple tasks which differ in the size of the visible (and in turn the neighbourhood) voxels. This Allows the model to learn multi scale features and further enhance the learnt representation. We design a masking strategy, namely, Hierarchical masking, which balances the amount of occupied voxels across the different scales, further improving the learnt representation. for the first time, The proposed method surpasses the performance of SOTA supervised architectures for point-clouds on multiple datasets and two different tasks, setting new SOTA performance. 

Masked autoencoders (MAE) have shown tremendous potential for self-supervised learning (SSL) in vision and beyond. 
%The recent progress in self-supervised learning (SSL) is widely adopted for automated driving, particularly with masked auto-encoders (MAE). 
%3D point clouds from lidar sensors are commonly used in this domain. 
However, point clouds from LiDARs used in automated driving are particularly challenging for MAEs since large areas of the 3D volume are empty.
%A challenge for MAEs with automotive lidar is that large areas of the 3D volume are empty. 
Consequently, existing work suffers from leaking occupancy information into the decoder and has significant computational complexity, thereby limiting the SSL pre-training to only 2D bird's eye view encoders in practice.
%Due to this, current literature suffers from either leaking occupancy information to the decoder, or from large computational complexity which limits the SSL pre-training to 2D bird's eye view encoders in practice. 
%As a result, modern 3D transformers, which are trained in a fully supervised way, still dominate today's leaderboards. 
In this work, we propose the novel neighborhood occupancy MAE (NOMAE) that overcomes the aforementioned challenges by employing masked occupancy reconstruction only in the neighborhood of non-masked voxels. We incorporate voxel masking and occupancy reconstruction at multiple scales with our proposed hierarchical mask generation technique to capture features of objects of different sizes in the point cloud.
%the varying size of relevant objects in the point cloud
%This paper proposes a novel masked occupancy reconstruction only in the neighborhood of non-masked voxels, thereby overcoming information leakage and computational complexity at the same time. 
%Thus, the neighborhood occupancy MAE (noMAE) pretext can be combined with state-of-the-art (SOTA) 3D architectures. 
%In addition, we introduce voxel masking and occupancy reconstruction at multiple scales, to accommodate the varying size of relevant objects in the point cloud, e.g., a pedestrian and a truck. 
NOMAEs are extremely flexible and can be directly employed for SSL in existing 3D architectures. We perform extensive evaluations on the nuScenes and Waymo Open datasets for the downstream perception tasks of semantic segmentation and 3D object detection, comparing with both discriminative and generative SSL methods. The results demonstrate that NOMAE sets the new state-of-the-art on multiple benchmarks for multiple point cloud perception tasks.
%Experiments on the Nuscenes and Waymo Open datasets show the benefit of the proposed method and ablations investigate the influence of our ingredients. 
%The proposed method surpasses the performance of modern supervised architectures for lidar semantic segmentation and object detection, setting new SOTA performance. 
\end{abstract}