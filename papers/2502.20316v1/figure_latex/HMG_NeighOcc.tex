\begin{figure}
  \centering
  \begin{subfigure}{\linewidth}
    \centering
    \includegraphics[trim={0 22cm 6cm 0},clip,width=\textwidth]{figure/HMG.pdf}
    \caption{Hierarchical mask generation renders $\mathcal{P}$ to the coarsest scale $s=S-1$ and then applies a random mask to divide occupied voxels into visible\,\fcolorbox{visible}{visible}{\rule{0pt}{2pt}\rule{2pt}{0pt}} and masked\,\fcolorbox{masked}{masked}{\rule{0pt}{2pt}\rule{2pt}{0pt}} voxels. 
    For the subsequent scales, only the visible voxels of the previous scale undergo random masking. 
    Masked voxels of a coarser scale always correspond to masked\,\fcolorbox{masked}{masked}{\rule{0pt}{2pt}\rule{2pt}{0pt}} or empty\,\fcolorbox{black}{white}{\rule{0pt}{2pt}\rule{2pt}{0pt}} voxels at a finer scale.}
    %\caption{A simple example is shown for the proposed hierarchal mask generation. the masking process starts with a voxelized point cloud at the coarsest scale $s=S-1$. The visible voxels are then upsampled and their occupied subvoxels undergo a second masking step happens at scale $s=S-2$. Naturally, empty subvoxels are excluded from the masking. the upsamling masking process is repeated and $\mathcal{V}_{\text{v}}$ is produced after the masking step on the finest scale $s=0$. \fcolorbox{black}{visible}{\rule{0pt}{3pt}\rule{3pt}{0pt}}~,~\fcolorbox{black}{masked} {\rule{0pt}{3pt}\rule{3pt}{0pt}}~, and~\fcolorbox{black}{white}{\rule{0pt}{3pt}\rule{3pt}{0pt}}~represent visible voxels, masked voxels, and empty voxels respectively.} 
    \label{fig:HMG}
  \end{subfigure}
  \begin{subfigure}{\linewidth}
    \centering
    \includegraphics[trim={0 0 6cm 23cm},clip,width=\textwidth]{figure/HMG.pdf}
    \hfill
    \caption{Multiscale pretext reconstructs the cells around visible voxels\,\fcolorbox{visible}{visible}{\rule{0pt}{2pt}\rule{2pt}{0pt}} at multiple scales. 
    The neighborhood $\mathcal{V}_n$ contains voxels that are reconstructed but empty\,\fcolorbox{neighborhood}{neighborhood}{\rule{0pt}{2pt}\rule{2pt}{0pt}} as well as masked and recovered voxels\,\fcolorbox{recovered}{recovered}{\rule{0pt}{2pt}\rule{2pt}{0pt}}. 
    Some masked voxels which are too distant from visible voxels are not recovered\,\fcolorbox{lost}{lost}{\rule{0pt}{2pt}\rule{2pt}{0pt}} and do not contribute to the loss. 
    The predicted occupancy $\Tilde{O}$ at coarser scales covers a wider region, while finer scales reconstruct more detail. 
    }
    %\caption{Predicted neighboring occupancy at different scales for example in Fig.\,\ref{fig:HMG}. \fcolorbox{black}{visible}{\rule{0pt}{2pt}\rule{2pt}{0pt}}~,~\fcolorbox{black}{neighborhood} {\rule{0pt}{2pt}\rule{2pt}{0pt}}~,~\fcolorbox{black}{recovered} {\rule{0pt}{2pt}\rule{2pt}{0pt}}~,and~\fcolorbox{black}{lost}{\rule{0pt}{2pt}\rule{2pt}{0pt}}~represent visible voxels, neighborhood voxels, masked voxels within the neighborhood voxels, and masked voxels outside the neighborhood voxels respectively. We can see that at coarser scales $\Tilde{O}$ covers a wider region yet with coarser occupancy. the finer the scale gets. We see that $\mathcal{V}_n$ gets smaller and closer to the surface represented by $\mathcal{P}$ improving both efficiency and accuracy. Some masked voxels fall outside the nieghborhood and hence they don't contribute to the loss. }
    \end{subfigure}
\caption{Illustration of multiscale pretext (MSP) and hierarchical mask generation (HMG).}
\label{fig:HMG_MSP_example}
\vspace{-.5em}
\end{figure}