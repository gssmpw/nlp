\onecolumn

\section{Annotation Instructions and Interface}
\subsection{Contact Email}\label{sec:appendix-contact-email}
\begin{multicols}{2}
Figure~\ref{fig:email} shows an instance of an email that invited the authors to participate in the data collection. Email addresses have been extracted from papers or, in the case of EGU and F1000, addresses to corresponding authors are provided online. To prevent spamming authors, we have ensured that no author received more than 3 emails (e.g., when they were listed as authors on multiple papers of a venue). Email addresses were only used to contact authors and are not part of the dataset. We also do not publicize the code to extract email addresses from papers.
\end{multicols}
\begin{figure*}[ht]
    \centering
    \includegraphics[width=\textwidth]{figs/peer-qa-email-blurred.png}
    \caption{Exemplary contact email that has been sent to authors requesting their participation in answering the questions.}
    \label{fig:email}
\end{figure*}

\clearpage
\begin{multicols}{2}
\subsection{Annotation Interface}\label{sec:appendix-annotation-interface}
The annotation interface for providing answers is shown in Figure~\ref{fig:annotation-interface}. The camera-ready PDF of the publication is shown on the right-hand side, while answer annotations can be provided on the left side. In \textit{1.2 Question Feedback}, authors can leave free-form feedback about the question, e.g., if it should be removed or modified. By clicking on the \textit{Add} button in \textit{2.1 Answer Evidence}, text spans in the PDF can be highlighted. One highlight can span over several sentences or even pages. Multiple spans can be added by clicking the \textit{Add} button again. In \textit{2.2. Answer Free Text}, the free-form answer to the question can be given. Finally, in \textit{3.1.}, the authors can also mark the question as unanswerable or provide further feedback to the question. If none of the categories apply, feedback on why the question is unanswerable can also be provided in \textit{No Answer Reason Free Text}.
\end{multicols}
\begin{figure*}[ht]
    \centering
    \includegraphics[width=\textwidth]{figs/peer-qa-annotation-interface.png}
    \caption{Screenshot of the annotation interface. The annotation consists of four parts. First, the annotator can provide feedback to the question, e.g., to correct its meaning or provide their interpretation. Second, answer evidence is annotated by highlighting sentences in the PDF. Third, a free-form answer can be provided, directly answering the question. Lastly, if a question is unanswerable or is of low quality, the interface provides an option to flag the question.}
    \label{fig:annotation-interface}
\end{figure*}

\clearpage
\begin{figure*}[!b]
    \centering
    \includegraphics[width=\textwidth]{figs/sunburst-sunset.pdf}
    \caption{Sunburst diagram of the 4-grams in the PeerQA questions.}
    \vspace{11em}
    \label{fig:sunburst-questions}
\end{figure*}
\begin{multicols*}{2}
\section{Question Sunburst}\label{sec:appendix-sunburst-questions}
We visualize the starting 4-grams of all questions in Figure~\ref{fig:sunburst-questions}. All words have been lowercased and lemmatized, and rare n-grams have been discarded.
\end{multicols*}



\clearpage


\begin{figure}[!b]
     \centering
     \begin{subfigure}[b]{\textwidth}
         \centering
         \includegraphics[width=\textwidth]{figs/topic-keywords.pdf}
         \caption{Topics of Labeled Questions}
         \label{fig:question-topics-labelled}
     \end{subfigure}
     \hfill
     \vspace{0.5em}
     \begin{subfigure}[b]{\textwidth}
         \centering
         \includegraphics[width=\textwidth]{figs/topic-keywords-unlabelled.pdf}
         \caption{Topics of Unlabeled Questions}
         \label{fig:question-topics-unlabelled}
     \end{subfigure}
        \caption{Top 8 keywords for the top 12 question topics. The first topic contains all questions that could not be assigned during clustering. The bar shows the keyword's c-TF-IDF score. The top figure shows the topics for the labeled questions and the bottom for the unlabelled.}
        \label{fig:question-topic-keywords}
\end{figure}

\begin{multicols*}{2}
\section{Question Topics}\label{sec:appendix-question-topcis}
Figure~\ref{fig:question-topic-keywords} reports the result of applying BERTopic \citep{grootendorst-bertopic-2022} on the labeled (\ref{fig:question-topics-labelled}) and unlabeled (\ref{fig:question-topics-unlabelled}) questions. Table~\ref{tbl:question-topics} additionally shows representative questions for the topics. We use the standard \texttt{all-MiniLM-L6-v2} sentence transformer model to compute embeddings. After embedding, stopwords have been removed, and the words have been lemmatized using spacy \citep{spaCy2020} to improve the keyword extraction. In \S\ref{sec:analysis}, we analyzed the topics of the labeled questions cluster. We found them to be focused on the scientific community and its subtopics or related to elements in the paper. We observe similar clusters in the unlabeled data (e.g., topic \#4 for NLP, topic \#12 for Geoscience, \#6 focusing on (pre-)training, \#5 on figures and plots).
\end{multicols*}


\clearpage


\begin{table*}[!ht]
\footnotesize
\centering
\begin{tabularx}{\textwidth}{p{0.2cm}p{0.7cm}X}
\toprule
\# & Size & Representative Questions \\
\midrule
1 & 35.4\% & Why does the baseline have significantly better performance on ACE 2004 and ACE 2005 compared to Yu et al. (2020), but similar performance on OntoNotes 5 and CoNLL 2003? Does the proposed alternate way to use linear by computing the mean absolute value of the weights associated with it differ from the original linear model proposed by Dalvi et al. (2019)? Is the design of the proposed method arbitrary for all layers of a given VIT model, or are some layers fixed? \\ \midrule
2 & 10.9\% & How does your method differ from existing methods for visual and language understanding with multilinguality, such as VQA, captioning, and retrieval? What other downstream tasks, such as natural language inference, question answering, and semantic role labeling, have been tested using an encoder that has been transferred from language 1 to language 2 without any parameter updates? How can the authors ensure that the natural language sentences produced from the "ground truth" activity graphs accurately describe the scene? \\ \midrule
3 & 4.8\% & Is the authors' conclusion about the accuracy of the CMIP6 climate models in simulating the processes based on the agreement between the observed data and the models' predictions in terms of the residual variability? Do the authors assume that iron is sourced from the platform when considering the feasibility of coastal seaweeds, which have a very low surface-to-volume ratio, competing for iron against the typically small and specialized open ocean phytoplankton that have a high surface-to-volume ratio? Can we assume that coastal seaweeds, which have a very low surface-to-volume ratio, would be competitive in iron uptake against the mostly small and specialized open ocean phytoplankton that have a high surface-to-volume ratio, especially in iron-limited areas? \\ \midrule
4 & 4.3\% & Can the authors provide a justification for why only four datasets were used to evaluate the visual search models, rather than a more diverse collection of datasets? Why should associations that are solvable by AI be kept in the framework, when the purpose of the framework is to collect associations that are difficult for models but solvable by humans? Does the proposed approach address issues related to assigning different attributions to features that have the same effect on the model or assigning positive attributions to features with no effect? \\ \midrule
5 & 4.0\% & How does the paper incorporate section titles into the BOS representation? What is the purpose of multiplying the scalar sc(omega, q) by the inner product of omega and q in equation 5? What is the difference between the \verb|\odot| and \verb|\cdot| symbols in the equation for computing the overall source mask from the k masks? \\ \midrule
6 & 3.8\% & Is it possible to consistently find perturbations to empirically robust adversarial examples that result in a correctly classified image? How does the paper define the concept of an "adversarial L2 ball" when it appears to suggest that every sample should have the same classification as \verb|\tilde{x}|, contrary to the expectation that each sample within the ball should have a different classification compared to x? Could the authors provide further justification for their claim that the gradient-based attack is responsible for the shift between test and training data observed in the adversarial attack? \\ \midrule
7 & 3.6\% & What is the performance of larger GLM models compared to state-of-the-art results, given that hardware resources do not appear to be a constraint? What is the expected relationship between the performance of the algorithms and the number of updates per sample, memory size, and batch size? What could explain the difference in performance between the DICTA test set and the new test set, particularly the difference between the cha and wor scores? \\ \midrule
8 & 3.3\% & What protocol did you use to decide when to stop training and to select hyperparameters for each dataset when no labeled target data is available? Does label smoothing always improve performance, or are there cases where it can degrade performance? Does label smoothing always improve the performance of the hyperparameter-fine tuning procedure? \\ \midrule
9 & 3.1\% & What are the vertical uncertainty bars in Figure 13? What would be the correct classification for the image in Figure 1 where the space bar is hidden? What is the reason for the sudden change in the green and blue curves in Figure 2 at epoch 90? \\ \midrule
10 & 3.1\% & What is the impact of adjusting $\delta$ on the results of Table 1? Is the coreference resolution pipeline depicted in Table 1 universally accepted in the field of coreference resolution? What is the difference between the results in Table 1 and Table 2(a)? \\ \midrule
11 & 2.8\% & Is there an optimal number of MSD points to use in order to minimize the error on the estimated parameters, and is there an option to automatically determine this number? Why is a new metric, concept purity, introduced instead of using the same set of metrics provided in Yuan et al. (2020)? What benefits does improving the upper bound for the Information Gain evaluation metric provide in practice? \\
\midrule
12 & 2.4\% & Are all participants in the trial pregnant women who are less than 36 weeks gestation? What is the rationale for changing the data distribution if the KB was compiled by medical papers? What ethical considerations were taken into account when selecting the data for the dataset? \\ 
\bottomrule
\end{tabularx}
\caption{Questions clustered into the top 12 topics by BERTopic. \textit{Representative Questions} are automatically identified by BERTopic.}
\label{tbl:question-topics}
\end{table*}

\clearpage

\section{Example Annotations}\label{sec:appendix-examplary-annotation}
\begin{table*}[!h]
\scalebox{0.98}{
\begin{minipage}{\linewidth}
\centering
\begin{adjustbox}{angle=270} 
\tiny
\begin{tabularx}{0.93\textheight}{@{}p{0.1cm}p{3.1cm}Xp{3.25cm}p{6cm}@{}}
\toprule
 & Question & Answer Evidence & Free-Form Answer & GPT-4 Augmented Answer \\ \midrule
\multirow{1}{*}{\rotatebox{90}{ICLR 23}} 
& What benefits does improving the upper bound for the Information Gain evaluation metric provide in practice? & to measure the quality of such models An interpretable alternative evaluation metric for likelihood-based models is Normalized Information Gain ( NInGa ) & interpretable and robust evaluation of full likelihood models & Improving the upper bound for the Information Gain evaluation metric provides the benefit of enabling a more interpretable and robust evaluation of full likelihood models in practice. \\ \midrule
\multirow{1}{*}{\rotatebox{90}{ICLR 22}} & Does the expected label always match the most probable label given the noisy data? & This makes it clear that, overall, the pattern zig-zags, first moving towards the unknown true label before eventually turning to memorize the wrong label & No, that is why zig-zag might happen for the hard labels. & No, the expected label does not always match the most probable label given the noisy data. This is indicated by the observed zig-zag pattern, where the process initially moves towards the true label but eventually turns to memorize the incorrect label. \\ \midrule
\multirow{1}{*}{\rotatebox{90}{NeurIPS 22}} & What criteria were used to select the videos and classes for the study? & This is inspired by the fact that complex human activities in real-world settings are usually hierarchical and compositional across space and time. & We selected videos and classes that contain complex human activities that are hierarchical and compositional. & The criteria for selecting the videos and classes for the study were based on the presence of complex human activities that are hierarchical and compositional in nature.\par \\ \midrule
\multirow{1}{*}{\rotatebox{90}{ARR 22}} & What impact does the layer norm approximation have on the performance of the tasks discussed in the paper? & Another challenge of THE-X is the attention score input of layer normalization.\par In most cases, the scale of multi-head attention output is very dense around [-1,1].\par However, before normalization, we also observe the attention scores are scarily sparse, with some extreme values reaching 1e4, which is difficult for our LN-distill stage.\par To prevent the overflow attention scores, we use the weight decay of Adam optimizer as regularization.\looseness=-1 & Layernorm approximation may lead to attention overflow, which dramastically affect the performance of downstream tasks & The approximation of layer normalization can potentially lead to attention overflow. This can significantly impact the performance of the downstream tasks discussed in the paper. \\ \midrule
\multirow{1}{*}{\rotatebox[origin=c]{90}{COLING 20}} & What measures were taken to ensure that paths were not shared between the training and test sets? & To avoid giving an unfair advantage to the path-based models, we filtered both validation and test sets to only include hyponym queries that are unseen anywhere in the full taxonomy paths of the training data.\par We also exclude hyponyms from the test and validation sets which appear as hyponyms in the training set7 to prevent the models from merely copying. & All queries observed during path training and all hyponyms in the training set that also appear as hyponyms in the test set are eliminated. & The authors implemented measures to ensure that paths were not shared between the training and test sets by filtering both validation and test sets to only include hyponym queries that were not present in the full taxonomy paths of the training data. Additionally, any hyponyms that appeared in the training set were excluded from the test and validation sets to prevent the models from simply replicating the data. \\ \midrule
\multirow{1}{*}{\rotatebox{90}{ACL 17}} & What would be the impact of not having access to pre-trained embeddings on the performance of this method? & Our main contribution is to show that the context sensitive representation captured in the LM embeddings is useful in the supervised sequence tagging setting. When we include the LM embeddings in our system overall performance increases from 90.87\% to 91.93\% F 1 for the CoNLL 2003 NER task, a more then 1\% absolute F1 increase, and a substantial improvement over the previous state of the art.  We also establish a new state of the art result (96.37\% F 1 ) for the CoNLL 2000 Chunking task. Importantly, the LM embeddings amounts to an average absolute improvement of 1.06 and 1.37 F 1 in the NER and Chunking tasks, respectively. & Performance decreases by about 1\% on the CoNLL 2003 NER task, and by 1.4\% on the CoNLL 2000 Chunking task when removing pretrained language model embeddings. & The absence of pre-trained language model embeddings would result in a decrease in performance by approximately 1\% on the CoNLL 2003 NER task, and by around 1.4\% on the CoNLL 2000 Chunking task. \\ \midrule
\multirow{1}{*}{\rotatebox{90}{CoNLL 16}} & Why were post-editing rates chosen over prediction (h)ter for intrinsic uncertainty evaluation? & Our decision to focus on post-editing time was based on the fact that time is a more complete measure of post-editing effort, capturing not only technical effort like HTER, but also cognitive effort ( Koponen et al., 2012 ).\par Additionally, time is more directly applicable in real translation environments – where uncertainty estimates could be useful, as it relates directly to productivity measures. & The authors state that normalised post-editing times (i.e. post-editing rates) are a more realistic measure with respect to operational effort, and can be adapted to different application scenarios. & The authors opted for post-editing rates over prediction (h)ter for intrinsic uncertainty evaluation because they believed that time is a more comprehensive measure of post-editing effort. This measure not only captures technical effort, like HTER, but also cognitive effort. Furthermore, time is more directly applicable in real translation environments, where uncertainty estimates could be beneficial, as it correlates directly to productivity measures. Thus, post-editing rates provide a more realistic measure of operational effort and can be adapted to various application scenarios.\looseness=-1 \\ \midrule
\multirow{1}{*}{\rotatebox{90}{ESD 23}} & What is the impact on detection or signal-to-noise ratio (SNR) if RR-based fingerprint construction is not applied? & robust detection and attribution (D\&A) of forced change in the water cycle are impaired by low signal-to-noise ratios.\par The concept of detection and attribution is to use climate model simulations in which the applied forcings are known and internal variability can be reduced by averaging multiple realisations, to estimate a so-called fingerprint that represents the effect of the applied forcings on climate variables of interest. & detection won't be possible if no SNR-maximising method/process is applied to the noise observational data & If the RR-based fingerprint construction is not applied, it could potentially impair the detection process due to the low signal-to-noise ratios. This could make detection impossible without the application of a method or process that maximizes the signal-to-noise ratio in the observational data.\\ \midrule
\multirow{1}{*}{\rotatebox{90}{ESurf 23}} & Does the orientation of the grid, with potentially more north-facing slopes in the Elwha area than in the Quinault area, affect the absorption of solar radiation, thereby influencing glacier distribution? & Our flowline model captures the variability in the direction of the steepest precipitation gradients but neglects variability across the valleys that may be important, especially for spatially variable drivers of ablation, such as topographic shading. & We do not include topographic shading as a factor influencing melting in the model - so the answer to this question is no - not in our model.  In the real world, the north-facing slopes of the Elwha headwaters may have been important in the mass balance of past glaciers. & The authors' model does not account for topographic shading as a factor influencing melting, thus the orientation of the grid does not affect the absorption of solar radiation in their model. However, in reality, the north-facing slopes of the Elwha headwaters may have played a significant role in the mass balance of past glaciers. \\ \midrule
\multirow{1}{*}{\rotatebox{90}{F1000 22}} & How was the conceptual model in Figure 2 developed? & 
Participants were recruited from their schools and colleges through the activity recruitment poster and “ Word-of-Mouth ” from the teachers and lecturers.\par 
Then, their crafts will be evaluated by art teachers and ranked on a leaderboard.\par The participants who were ranked on the leaderboard will receive certificates and prizes.\par After experiencing two-level EGL, 29 students were purposively selected as FGD participants.\par
Figure 2 shows the overview of the EGL activity flow. & It is based on the procedures of the EGL activity from participation to focus group discussion. & The conceptual model depicted in Figure 2 was developed based on the procedures of the Experiential Group Learning (EGL) activity. This process ranged from participant recruitment to the focus group discussion.\\ 
\bottomrule
\end{tabularx}
\end{adjustbox}
\caption{Exemplary questions, answer evidence, and free-form answers of the PeerQA dataset from all venues.}\label{tbl:examplary-annotation}
\end{minipage}
}
\end{table*}

\clearpage
\begin{multicols}{2}
\section{Question Classes}\label{sec:appendix-question-classes}
Table~\ref{tbl:question-classes} reports the number of questions per class and representative questions for that class. The definitions for each class are the following:
\paragraph{Method Clarification} Questions to better understand a specific detail (e.g., a parameter) or inner workings of a proposed or used method, including methods used for obtaining data or details about the experiment setup/process.
\paragraph{Data Clarification} Questions to understand the process of obtaining data or properties of the used data for an experiment, however, excluding questions about a method to obtain data.
\paragraph{Justification/Rationale} Questions that challenge an assumption, ask the authors to motivate a decision's reasoning or are critical towards a process/finding.
\paragraph{Analysis} Questions asking for a better understanding of a result, e.g., why a method works or questions asking about what factors contribute to a result/finding. 
\paragraph{Implication} Questions about potential real-world applications, transfers of the data/method/findings to other applications/domains/tasks, or wider-scoped consequences of the findings.
\paragraph{Definition} Questions about the (intended) meaning of a certain phrase or term used in the paper.
\paragraph{Comparison} Questions asking for comparisons or differences between methods/data or different studies.
\paragraph{Evaluation/Evidence} Questions asking for details about a result (excluding analysis of results), details of the evaluation process, or evidence to support a certain claim.
\end{multicols}

\begin{table*}[ht]
\small
\centering
\begin{tabularx}{\textwidth}{@{}p{2.5cm}p{1cm}X@{}}
\toprule
Class & Size & Representative Questions \\ \midrule
Method Clarification & 31\% & How was the fine tuning done for the step sizes in the experiments?, Did the baselines in both experiments 1 and 2 only use a single seed? What is the set of signed input gradients in the second paragraph of section 4.2? \\ \midrule
Data Clarification & 13\% & Do the experts who annotated the dataset have expertise in linguistics or in the domain of the dataset? What is the time resolution of the forcing data used in the study, specifically, is it daily? Do the vocabulary items of the templates used in the paper have adequate representation in the training data? \\ \midrule
Justification/Rationale & 12\% & What motivated the authors to theoretically analyze the dense case and then empirically evaluate the sparse case? Are ten locations sufficient to represent the variety of surfaces in urban environments? Why is the chosen metric appropriate for evaluating the results? \\ \midrule
Comparison & 11.5\% & How does the proposed method compare to other types of vision transformers, such as Swin Transformer or Multiscale Vision Transformers? What is the difference between the MOMA dataset and the MOMA-LRG dataset? How does the performance of the filter-kd model compare to models trained using label smoothing and knowledge distillation with the optimum temperature? \\ \midrule
Analysis & 9\% & What factors influence the degree of separability when adapting a model to a task? Is it clear what the source of the improvements of Histruct+ (Roberta-base) over Bertsumext are? What factors were responsible for the success of the path-based model? \\ \midrule
Implications & 8\% & What are the potential applications of the data presented in this paper? Can the proposed data augmentation be applied to other tasks besides ILA? Do you think that the same framework on variance of ensembles would work equally well in the semantic feature space as in the space of logits? \\ \midrule
Evaluation/Evidence & 8\% & What is the evidence that the generative model is successful in synthesizing new molecules? Do you evaluate playing strength of agents by restricting them by MCTS iteration counts or by time limits? Did the authors run multiple trials to evaluate the performance of the graph-based neural network? \\ \midrule
Definition & 7.5\% & What is the definition of difficulty used in the paper to analyze the learning path of the network's predicted distribution? What is the variational approximation of c given by the query and support sets? What is the definition of $f_{i+1}$? \\ 
\bottomrule
\end{tabularx}
\caption{Distribution of question classes based on 100 questions randomly sampled from PeerQA. \textit{Representative Questions} shows manually picked questions that best correspond to the definition of the class.}
\label{tbl:question-classes}
\end{table*}

\clearpage

\begin{multicols}{2}
\section{Answerability Evaluation}\label{sec:answerability-evaluation}
Table~\ref{tbl:answerability-evaluation} shows detailed evaluation metrics for the answerability task, and Figure~\ref{fig:results-un-answerable} visualizes them. We report Precision, Recall, and F1-Score on both the answerable and unanswerable questions, as well as the average accuracy, weighted, and macro F1-Score.
\vfill\null
\columnbreak
\vfill\null
\end{multicols}
\begin{table*}[!h]
\small
\centering
\begin{tabular}{@{}lcccccccccc@{}}
\toprule
 &  & \multicolumn{3}{c}{Answerable ($N=383$)} & \multicolumn{3}{c}{Unanswerable ($N=112$)} & \multicolumn{3}{c}{Average} \\
 \cmidrule(lr){3-5}
 \cmidrule(lr){6-8}
 \cmidrule(lr){9-11}
Model & Ctx. & Prec. & Recall & F1 & Prec. & Recall & F1 & Acc. & W-F1 & M-F1 \\ \midrule
\multirow{6}{*}{\begin{tabular}[c]{@{}l@{}}Llama-3\\ IT-8B-8k\end{tabular}} 
 & G    & \textbf{1.0000} & 0.4517 & 0.6223 & --     & --     & --     & 0.4517 & 0.6223 & 0.3112 \\
 & 10   & 0.8407 & 0.3995 & 0.5416 & 0.2652 & \textbf{0.7411} & 0.3906 & 0.4768 & 0.5074 & 0.4661 \\
 & 20   & \textbf{0.8796} & 0.2480 & 0.3870 & 0.2558 & \textbf{0.8839} & 0.3968 & 0.3919 & 0.3892 & 0.3919 \\
 & 50   & 0.7907 & 0.1775 & 0.2900 & 0.2298 & \textbf{0.8393} & 0.3608 & 0.3273 & 0.3060 & 0.3254 \\
 & 100  & 0.7667 & 0.1802 & 0.2918 & 0.2247 & \textbf{0.8125} & 0.3520 & 0.3232 & 0.3054 & 0.3219 \\
 & FT   & 0.8168 & 0.5587 & 0.6636 & 0.2747 & 0.5714 & 0.3710 & 0.5616 & 0.5974 & 0.5173 \\
 \midrule
\multirow{6}{*}{\begin{tabular}[c]{@{}l@{}}Llama-3\\IT-8B-32k\end{tabular}} 
 & G & \textbf{1.0000} & 0.4047 & 0.5762 & -- & -- & -- & 0.4047 & 0.5762 & 0.2881 \\
 & 10 & 0.8326 & 0.4804 & 0.6093 & 0.2737 & 0.6696 & 0.3886 & 0.5232 & 0.5593 & 0.4989 \\
 & 20 & 0.8182 & 0.4700 & 0.5970 & 0.2618 & 0.6429 & 0.3721 & 0.5091 & 0.5461 & 0.4846 \\
 & 50 & 0.8056 & 0.3786 & 0.5151 & 0.2444 & 0.6875 & 0.3607 & 0.4485 & 0.4802 & 0.4379 \\
 & 100 & 0.7984 & 0.2585 & 0.3905 & 0.2345 & 0.7768 & 0.3602 & 0.3758 & 0.3837 & 0.3754 \\
 & FT & 0.8488 & 0.3812 & 0.5261 & 0.2663 & 0.7679 & 0.3954 & 0.4687 & 0.4965 & 0.4608 \\
  \midrule
\multirow{6}{*}{\begin{tabular}[c]{@{}l@{}}Mistral\\ IT-v02-7B-32k\end{tabular}} 
 & G    & \textbf{1.0000} & \textbf{0.8877} & \textbf{0.9405} & --     & --     & --     & \textbf{0.8877} & \textbf{0.9405} & \textbf{0.4703} \\
 & 10   & 0.7854 & \textbf{0.9269} & \textbf{0.8503} & \textbf{0.3488} & 0.1339 & 0.1935 & \textbf{0.7475} & \textbf{0.7017} & 0.5219 \\
 & 20   & 0.7790 & \textbf{0.9295} & \textbf{0.8476} & 0.2895 & 0.0982 & 0.1467 & \textbf{0.7414} & 0.6890 & 0.4971 \\
 & 50   & 0.7768 & \textbf{0.9086} & \textbf{0.8375} & 0.2553 & 0.1071 & 0.1509 & \textbf{0.7273} & 0.6822 & 0.4942 \\
 & 100  & 0.7824 & \textbf{0.9295} & \textbf{0.8496} & \textbf{0.3250} & 0.1161 & 0.1711 & \textbf{0.7455} & \textbf{0.6961} & 0.5103 \\
 & FT   & 0.7803 & \textbf{0.9739} & \textbf{0.8664} & \textbf{0.4118} & 0.0625 & 0.1085 & \textbf{0.7677} & \textbf{0.6949} & 0.4875 \\
  \midrule
\multirow{6}{*}{\begin{tabular}[c]{@{}l@{}}Command-R\\ v01-34B-128k\end{tabular}} 
 & G    & \textbf{1.0000} & 0.7232 & 0.8394 & -- & -- & -- & 0.7232 & 0.8394 & 0.4197 \\
 & 10   & 0.7985 & 0.8172 & 0.8077 & 0.3204 & 0.2946 & 0.3070 & 0.6990 & 0.6944 & \textbf{0.5574} \\
 & 20   & 0.8025 & 0.8381 & 0.8199 & \textbf{0.3474} & 0.2946 & 0.3188 & 0.7152 & \textbf{0.7065} & \textbf{0.5694} \\
 & 50   & 0.8031 & 0.8198 & 0.8114 & \textbf{0.3365} & 0.3125 & 0.3241 & 0.7051 & \textbf{0.7011} & 0.5677 \\
 & 100  & 0.7949 & 0.8094 & 0.8021 & 0.3048 & 0.2857 & 0.2949 & 0.6909 & 0.6873 & 0.5485 \\
 & FT   & 0.8113 & 0.7520 & 0.7805 & 0.3214 & 0.4018 & 0.3571 & 0.6727 & 0.6847 & \textbf{0.5688} \\
 \midrule
\multirow{6}{*}{\begin{tabular}[c]{@{}l@{}}GPT-3.5\\ Turbo-0613-16k\end{tabular}} 

 & G    & \textbf{1.0000} & 0.4935 & 0.6608 & --     & --     & --     & 0.4935 & 0.6608 & 0.3304 \\
 & 10   & 0.8107 & 0.4360 & 0.5671 & 0.2526 & 0.6518 & 0.3641 & 0.4848 & 0.5211 & 0.4656 \\
 & 20   & 0.8248 & 0.5039 & 0.6256 & 0.2720 & 0.6339 & 0.3807 & 0.5333 & 0.5702 & 0.5032 \\
 & 50   & 0.8168 & 0.5587 & 0.6636 & 0.2747 & 0.5714 & 0.3710 & 0.5616 & 0.5974 & 0.5173 \\
 & 100  & 0.8507 & 0.4465 & 0.5856 & 0.2789 & 0.7321 & 0.4039 & 0.5111 & 0.5445 & 0.4948 \\
 & FT   & 0.8348 & 0.2507 & 0.3855 & 0.2447 & \textbf{0.8304} & 0.3780 & 0.3818 & 0.3838 & 0.3818 \\
\midrule
\multirow{6}{*}{\begin{tabular}[c]{@{}l@{}}GPT-4o\\0806-128k\end{tabular}} 
 & G    & \textbf{1.0000} & 0.4465 & 0.6173 & --     & --     & --     & 0.4465 & 0.6173 & 0.3087 \\
 & 10   & \textbf{0.8439} & 0.5222 & 0.6452 & 0.2907 & 0.6696 & \textbf{0.4054} & 0.5556 & 0.5909 & 0.5253 \\
 & 20   & 0.8560 & 0.5744 & 0.6875 & 0.3151 & 0.6696 & \textbf{0.4286} & 0.5960 & 0.6289 & 0.5580 \\
 & 50   & \textbf{0.8604} & 0.5953 & 0.7037 & 0.3261 & 0.6696 & \textbf{0.4386} & 0.6121 & 0.6437 & \textbf{0.5712} \\
 & 100  & \textbf{0.8543} & 0.5666 & 0.6813 & 0.3112 & 0.6696 & \textbf{0.4249} & 0.5899 & 0.6233 & \textbf{0.5531} \\
 & FT   & \textbf{0.8458} & 0.5300 & 0.6517 & 0.2941 & 0.6696 & \textbf{0.4087} & 0.5616 & 0.5967 & 0.5302 \\

\bottomrule
\end{tabular}
\captionof{table}{Evaluation results on the answerability task of various LLMs, with different context settings (G = Gold Evidence, FT = Full-Text, 10/20/50/100 = Top-k passages). Note that the class distribution is imbalanced. There are a total of 383 answerable and 112 unanswerable questions. W-F1 is Weighted F1, M-F1 is Macro F1.}\label{tbl:answerability-evaluation}
\end{table*}



\clearpage


\begin{multicols}{2}
\section{Answer Generation Evaluation}\label{sec:answer-generation-evaluation}
Table~\ref{tbl:answer-generation-evaluation} reports the exact numbers of the free-form answer generation experiment for all models and contexts, corresponding to Figure~\ref{fig:answer-generation-eval}.
\vfill\null
\columnbreak
\vfill\null
\end{multicols}

\begin{table}[!hp]
\begin{center}
\small

\begin{tabular}{@{}lccccccccc@{}}
\toprule
 &  & \multicolumn{3}{c}{Rouge-L} & \multicolumn{3}{c}{AlignScore} & \multicolumn{2}{c}{Prometheus} \\ 
\cmidrule(lr){3-5}
\cmidrule(lr){6-8}
\cmidrule(lr){9-10}
Model & Ctx. & AE & FF & GPT-4 FF & AE & FF & GPT-4 FF & FF & GPT-4 FF \\
\midrule

\multirow{6}{*}{\begin{tabular}[c]{@{}l@{}}Llama-3\\ IT-8B-8k\end{tabular}} 
 & G    & 0.1683 & 0.2295 & 0.2569 & 0.5731 & 0.1098 & 0.2643 & 3.1102 & 3.1593 \\
 & 10   & 0.1670 & 0.2113 & \textbf{0.2479} & 0.3839 & 0.1107 & 0.2107 & 3.1347 & 3.1828 \\
 & 20   & 0.1771 & 0.2074 & 0.2458 & 0.3719 & 0.1041 & 0.1965 & 3.1878 & 3.2454 \\
 & 50   & 0.1621 & 0.2050 & 0.2357 & 0.3402 & 0.1062 & 0.1958 & 3.0122 & 3.0313 \\
 & 100  & 0.1418 & 0.2069 & 0.2278 & 0.3255 & 0.1067 & 0.2184 & 2.8082 & 2.7885 \\
 & FT   & 0.1484 & 0.1736 & 0.2037 & 0.2719 & 0.0653 & 0.1159 & 2.7510 & 2.9321 \\

\midrule

\multirow{6}{*}{\begin{tabular}[c]{@{}l@{}}Llama-3\\ IT-8B-32k\end{tabular}}
& G   & 0.1648 & 0.2286 & 0.2567 & 0.5778 & 0.1016 & 0.2436 & 3.1673 & 3.1749 \\
& 10  & 0.1513 & \textbf{0.2258} & 0.2464 & 0.3970 & 0.1142 & 0.2177 & 3.1388 & 3.1410 \\
& 20  & 0.1558 & 0.2204 & 0.2425 & 0.4001 & 0.1115 & 0.2109 & 3.1388 & 3.1227 \\
& 50  & 0.1546 & 0.2061 & 0.2397 & 0.3750 & 0.0999 & 0.2011 & 3.0571 & 3.1358 \\
& 100 & 0.1664 & 0.2099 & 0.2412 & 0.3785 & 0.1037 & 0.2008 & 3.0000 & 3.2010 \\
& FT  & 0.1835 & 0.1948 & 0.2260 & 0.3311 & 0.0711 & 0.1450 & 3.1959 & 3.2167 \\
 \midrule
\multirow{6}{*}{\begin{tabular}[c]{@{}l@{}}Mistral\\ v02-7B-32k\end{tabular}}
 & G    & \textbf{0.2442} & 0.1922 & 0.2432 & 0.6407 & 0.0827 & 0.1977 & 3.4245 & \textbf{3.4517} \\
 & 10   & \textbf{0.1967} & 0.1667 & 0.2032 & 0.3573 & 0.0612 & 0.1094 & 3.2490 & 3.3629 \\
 & 20   & \textbf{0.2039} & 0.1670 & 0.2011 & 0.3449 & 0.0505 & 0.1107 & 3.2408 & 3.2663 \\
 & 50   & \textbf{0.2023} & 0.1572 & 0.1943 & 0.3211 & 0.0496 & 0.1017 & 3.1306 & 3.1958 \\
 & 100  & \textbf{0.2023} & 0.1593 & 0.1927 & 0.3142 & 0.0634 & 0.1209 & 3.0245 & 3.0809 \\
 & FT   & \textbf{0.1883} & 0.1344 & 0.1678 & 0.2599 & 0.0328 & 0.0750 & 2.9796 & 3.1227 \\
 \midrule
\multirow{6}{*}{\begin{tabular}[c]{@{}l@{}}Command-R\\ v01-34B-128k\end{tabular}}
 & G    & 0.1310 & 0.2294 & 0.2081 & 0.5604 & 0.1362 & 0.3059 & 3.0571 & 3.0052 \\
 & 10   & 0.1211 & 0.2104 & 0.1973 & 0.3767 & 0.1221 & 0.2275 & 3.1551 & 3.1723 \\
 & 20   & 0.1220 & 0.2164 & 0.1978 & 0.3823 & 0.1245 & 0.2213 & 3.0490 & 3.0052 \\
 & 50   & 0.1229 & 0.2188 & 0.1941 & 0.3872 & 0.1223 & 0.2247 & 3.1224 & 3.0026 \\
 & 100  & 0.1244 & 0.2200 & 0.1853 & 0.3688 & 0.1112 & 0.1976 & 3.0245 & 3.0052 \\
 & FT   & 0.1230 & \textbf{ 0.2085} & 0.1859 & 0.3530 & \textbf{0.1015} & \textbf{0.1939} & 2.9020 & 2.9869 \\
\midrule
\multirow{6}{*}{\begin{tabular}[c]{@{}l@{}}GPT-3.5\\ Turbo-0613-16k\end{tabular}}
 & G    & 0.1540 & \textbf{0.2414} & 0.2688 & 0.5596 & \textbf{0.1378} & \textbf{0.3175} & 3.0408 & 3.0705 \\
 & 10   & 0.1342 & 0.2212 & 0.2462 & \textbf{0.4410} & \textbf{0.1412} & \textbf{0.2531} & 2.9184 & 3.0313 \\
 & 20   & 0.1388 & \textbf{0.2211} & \textbf{0.2465} & \textbf{0.4255} & \textbf{0.1446} & \textbf{0.2394} & 2.9714 & 3.0888 \\
 & 50   & 0.1365 & \textbf{0.2205} & \textbf{0.2437} & 0.4159 & \textbf{0.1356} & \textbf{0.2374} & 2.9918 & 3.0914 \\
 & 100  & 0.1297 & \textbf{0.2207} & \textbf{0.2437} & 0.4092 & \textbf{0.1360} & \textbf{0.2301} & 2.9102 & 3.0470 \\
 & FT   & 0.1162 & 0.1895 & 0.2188 & 0.3341 & 0.0771 & 0.1524 & 2.7143 & 2.9060 \\
\midrule
\multirow{6}{*}{\begin{tabular}[c]{@{}l@{}}GPT-4o\\ 0806-128k\end{tabular}}
 & G    & 0.1992 & 0.2266 & \textbf{0.2739} & \textbf{0.6410} & 0.1224 & 0.2802 & \textbf{3.4612} & 3.4308 \\
 & 10   & 0.1765 & 0.2048 & 0.2455 & 0.4055 & 0.0884 & 0.1963 & \textbf{3.5143} & \textbf{3.5222} \\
 & 20   & 0.1798 & 0.2039 & 0.2453 & 0.4094 & 0.0963 & 0.1830 & \textbf{3.5510} & \textbf{3.5927} \\
 & 50   & 0.1771 & 0.2058 & 0.2433 & \textbf{0.4164} & 0.0971 & 0.1926 & \textbf{3.5592} & \textbf{3.6423} \\
 & 100  & 0.1793 & 0.2036 & 0.2436 & \textbf{0.4120} & 0.0936 & 0.1886 & \textbf{3.5714} & \textbf{3.5614} \\
 & FT   & 0.1821 & 0.1981 & \textbf{0.2372} & \textbf{0.3900} & 0.0713 & 0.1790 & \textbf{3.5673} & \textbf{3.6057} \\
 \bottomrule
\end{tabular}
\captionof{table}{
Evaluation results on the answer generation task of various LLMs, with different context settings (G = Gold Evidence, FT = Full-Text, 10/20/50/100 = Top-k passages) and the metric computed against different ground truths (AE = Answer Evidence Paragraph, FF = Free-Form Answer, GPT-4 FF = GPT-4 rephrased Free-Form Answer). Rouge-L measures lexical overlap; AlignScore measures factual consistency; Prometheus measures answer correctness using an LLM-as-a-judge approach between the generation and the annotated Free-Form Answer or Answer Evidence.}\label{tbl:answer-generation-evaluation}
\end{center}
\end{table}
\clearpage

\section{Answer Generation Error Analysis}\label{sec:appendix-error-analysis}
As outlined in \S\ref{sec:results-answer-generation}, we conducted an error analysis on GPT-3.5's generations. Table~\ref{tbl:appendix-error-class-definition} defines each error class, and Table~\ref{tbl:error-examples} provides an example for each class.
\subsection{Error Classes}
\begin{table}[!htpb]
\small
\begin{tabularx}{\textwidth}{lX}
\toprule
Error Class & Definition \\
\midrule
Evaluation Error & The generated answer is correct; however, at least one of the metrics provides a low score. \\
Partially Correct & The generated answer is correct; however, the free-form answer provides additional details that are not covered by the generation. \\
Reasoning Error & The generation is incorrect. The model fails to arrive at the same conclusion as the free-form answer. \\
Implicit Evidence Only & The generation is incorrect. The evidence only implies the correct answer, making it challenging for the model to infer it. \\
Insufficient Context & The answer is incorrect because further context is required to interpret the evidence correctly (e.g., abbreviations in the context are not resolved, or information established earlier in the paper is missing from the evidence).  \\
Insufficient Free-Form Answer & The author's free-form answer points only to an answer in the paper but does not contain an answer (e.g., "The architecture is depicted in Figures D.3 and D.4").\\
Insufficient Evidence & The highlighted context by the authors does not provide sufficient information to answer the question well. \\
\bottomrule
\end{tabularx}
\caption{Error classes definitions for analyzing the failure modes of the generation models.}
\label{tbl:appendix-error-class-definition}
\end{table}

\clearpage
\subsection{Error Examples}
\begin{table*}[!h]
\begin{adjustbox}{angle=270, width=\textwidth} 
\tiny
\begin{tabularx}{.88\textheight}{@{}p{1.5cm}p{3.5cm}Xp{3.5cm}p{3.5cm}@{}}
\toprule
Error Class & Question & Evidence & Reference Answer & Generated Answer\\ \midrule
Evaluation Error & What does "but train the detector head with $r(\cdot)$ online in a single stage" mean? & To obtain the features for a region b, we apply the VLM pooling layer P(•) on the cropped backbone output features F(I) (see Sec. 3.2 for notations). Because the pooling layer requires fixed-size inputs, e.g. 7x7 for R50 (Radford et al., 2021), we crop and resize the region features with ROI-Align R(•) (He et al., 2017) (see Fig. 2b). Unlike existing works (Gu et al., 2022;Du et al., 2022), we do not crop and resize the RGB image regions and cache their embeddings in a separate offline process, but train the detector head in one stage. This is simpler and more space-efficient. In addition, we do not crop VLM region features with R(•) during training because the backbone features are frozen. & It means that we train the detection coordinates and the class labels for all boxes simlutanously and jointly, similar to Mark-RCNN and related algorithms, as opposed to cropping individual object boxes and feeding their features to a separate model. & "But train the detector head with $r(\cdot)$ online in a single stage" means that the detector head is trained during the same stage as the rest of the training process, rather than in a separate offline process. \\ 
\midrule
Partially Correct & Does the term "aes" refer to the phenomenon that smiling faces are estimated to be older than neutral faces in direct age estimations, or does it also refer to the phenomenon that smiling face group is retrospectively estimated to be younger? & Interestingly, several studies have reported that humans have a counterintuitive bias regarding age. We associate smiling with youth, that is, it is generally believed that when people see a smiling person, they feel that person is younger. Indeed, previous research has provided evidence that individuals with a smile appear younger than those with other facial expressions (Hass et al., 2016;Voelkle et al., 2012). However, contrary to the commonly held association between smiling and youth, Ganel (2015) showed that a smiling face is estimated to be older than a neutral face. This phenomenon, in which smiling faces are evaluated as being older than neutral faces, is called the aging effect of smiling (AES; Ganel \& Goodale, 2021). AES is attributed to wrinkles around the eyes caused by smiling (Ganel, 2015;Ganel \& Goodale, 2021). In contrast, when participants were asked to retrospectively estimate the mean age of several faces (i.e., face group), they estimated that the smiling face group was younger than the neutral face group (Ganel \& Goodale, 2018). These studies indicate that the effect of emotional expressions on age estimation depends on the method of estimation (i.e., directly or retrospectively). & The term AES refers specifically to the phenomenon that a smiling face makes one look older. & The term "aes" refers to both the phenomenon that smiling faces are estimated to be older than neutral faces in direct age estimations and the phenomenon that the smiling face group is retrospectively estimated to be younger. \\ 
\midrule
Reasoning Error & How are the depth and normal maps combined for training the proposal network in phase 1? & There are two possible ways to ensemble geometric cues: (1) Stack the two geometric cues together and train a single object proposal network on these stacked inputs in Phase-I; (2) Train two object proposal networks and extract pseudo boxes separately, then merge them into a single pseudo box pool for Phase-II training. The details of the merging process is described in Appendix A. We conduct ablation studies on these two methods. From Figure 6a, we demonstrate that empirically, ensembling pseudo labels is slightly better than using stacked inputs for Phase-I training. Throughout the paper, we use the pseudo label ensembling for GOOD-Both. & They are combined by merging their pseudo labels. & The depth and normal maps are combined by stacking them together and training a single object proposal network on these stacked inputs in Phase 1. \\
\midrule
Implicit Evidence Only & Does combining RGB, depth, and normals improve the performance of the model for smaller classes? & More comparison of GOOD-Both and GOOD-All.

For GOOD-All, the performance gains in detecting small objects (AR s ) are too small to compensate for the losses in detecting larger objects (AR m and (AR l )), leading to overall inferior performance. & The combination of RGB, depth, and normals does not significantly improve the model's performance for smaller classes. The gains in detecting small objects are too small to offset the losses in detecting larger objects, leading to an overall inferior performance. & The combination of RGB, depth, and normals does not significantly improve the model's performance for smaller classes. The gains in detecting small objects are too small to offset the losses in detecting larger objects, leading to an overall inferior performance.\\ 

\midrule
Insufficient Context & Do the training languages used for the non-baseline models in the cross-lingual experiments have any overlap with German? & In our experiments we make use of the following datasets: The English Nancy Krebs dataset (16h) from the Blizzard challenge 2011 (Wilhelms-Tricarico et al., 2011;King and Karaiskos, 2011); The German dataset of the speaker Karlsson (29h) from the HUI-Audio-Corpus-German (Puchtler et al., 2021); The Greek (4h), Spanish (24h), Finnish (11h), Russian (21h), Hungarian (10h), Dutch (14h) and French (19h) subsets of the CSS10 dataset (Park and Mulc, 2019). In order to investigate the effectiveness of our proposed LAML procedure, we train a Tacotron 2 model and a FastSpeech 2 model on the full Karlsson dataset as a strong baseline. We also train another Tacotron 2 model and another FastSpeech 2 model on speech in 8 languages with one speaker per language (Nancy dataset and CSS10 dataset) and fine-tune those models on a randomly chosen 30 minute subset from the Karlsson dataset. To our surprise, we did not only match, but even outperform the model trained on 29 hours with the model fine-tuned on just 30 minutes in multiple metrics. & The pretraining data does not overlap with German. Only the data used for fine-tuning is in German. & Yes, the training languages used for the non-baseline models in the cross-lingual experiments have overlap with German. The German dataset of the speaker Karlsson is used for training, along with the English Nancy Krebs dataset and subsets of the CSS10 dataset which include Dutch and French. \\
\midrule

Insufficient Evidence & Can the parameters of the BLOSUM matrix be estimated by the model instead of being pre-computed? & In the first ablation study, we investigate the influence of the BLOSUM embeddings by removing them as input to the GRU. Overall, the absence of the BLOSUM embeddings slows down convergence and sometimes make the learning process unstable, but ultimately does not strongly affect accuracy (see Figure 5). 3). The results without the BLOSUM embeddings are shown in pink (MAP) and purple (marginal). & The Blosum matrix is used to serve as a very simple to compute encoding technique to stabilize and accelerate the learning process. With larger datasets it is not necessary to use the Blosum encodings or averaged blosum embedding. & No, the parameters of the BLOSUM matrix cannot be estimated by the model instead of being pre-computed. \\
\midrule

Insufficient Free-Form Answer & What is the architecture of the system being discussed in the paper? & 

\textit{<Omitted for brevity>}
& 
The architecture is depicted in Figures D.3 and D.4 & The architecture is depicted in Figures D.3 and D.4 
\\ 
\bottomrule
\end{tabularx}
\end{adjustbox}
\caption{Examples for each error class.}\label{tbl:error-examples}
\end{table*}

\clearpage
\section{Answer Generation Correlation Analysis}
\subsection{Recall}\label{sec:appendix-correlation-generation-recall}
\begin{figure*}[!b]
    \centering
    \includegraphics[width=0.63\textwidth]{figs/correlation-generation-recall-all-concise3.pdf}
    \caption{Pearson correlation ($r$) with the corresponding $p$-value between the recall (x-axis) at $k$ (columns) and the answer generation performance (y-axis) according to different metrics (rows). Therefore, each circle represents a single QA pair of a specific model. We added 0.03 x-jitter to the markers to improve visibility.}
    \label{fig:correlation-generation-recall}
\end{figure*}

\begin{multicols*}{2}
Figure~\ref{fig:correlation-generation-recall} visualizes the relationship between the recall of the retrieval model (in this case \texttt{SPLADEv3}) at different cutoffs and the answer generation performance measured by different metrics.
\end{multicols*}

\clearpage
\begin{multicols}{2}
\subsection{Mean Evidence Position}\label{sec:appendix-correlation-evidence-poistion}
Figure~\ref{fig:correlation-evidence-poistion} visualizes the Pearson correlation between the answer generation metric (Rouge-L, AlignScore, or Prometheus-2 compared to either the answer evidence, the annotated free-form answer or the GPT-4 augmented free-form answer as ground truth) and the mean token position of the answer evidence. All generations are taken from the full-text setting, i.e., where the entire paper text was given as input to the model. To compute the mean token position for each answer evidence, we compute the number of tokens in the paper before the evidence sentence. If a question has multiple answer evidence, we take the average position. We only find a weak relationship that is statistically insignificant in many cases. Nevertheless, some p-values show statistical significance, indicating that for some settings, the generation performance declines when the answer evidence is relatively towards the end of the paper. This finding is also consistent with related work such as \citet{buchmann-etal-2024-attribute}.
\end{multicols}

\begin{figure*}[!hb]
    \centering
    \includegraphics[width=0.84\textwidth]{figs/correlation-evidence-poistion-all-concise3.pdf}
    \caption{Pearson correlation ($r$) with the corresponding $p$-value between the answer generation evaluation metric (y-axis) and the mean token position of the annotated answer evidence (x-axis).}
    \label{fig:correlation-evidence-poistion}
\end{figure*}

\clearpage

\begin{multicols}{2}
\section{Answer Generation Similarities}\label{sec:appendix-answer-generation-similarity}
We compute the average similarity of the generated answers between all models. We embed the generated answers with \texttt{all-MiniLM-L6-v2} and compute the cosine similarity between the generations of the models. Figure~\ref{fig:answer-generation-similarity} visualizes the similarities with the gold and retrieved evidence and full-text settings.
We find that all models produce fairly similar outputs for the gold setting, i.e., where the annotated answer evidence is provided. With increasing retrieved evidence as context (i.e., RAG-10 - RAG-100), the similarity between the model outputs decreases but remains relatively high.
\end{multicols}
\begin{figure*}[!htbp]
    \centering
    \includegraphics[width=\textwidth]{figs/answer-generation-similarity-heatmap-all-concise3.pdf}
    \caption{Semantic similarity of the generated answers between models with different context settings.}
    \label{fig:answer-generation-similarity}
\end{figure*}


\begin{multicols}{2}
\section{Attributable Question Answering}

\begin{Table}
\centering
\begin{tabular}{@{}lcc@{}}
\toprule
 & MRR & Recall@10 \\ \midrule
SPLADEv3 & 0.4536 & 0.6661 \\
GPT-3.5-Turbo-0613-16k & 0.2440 & 0.2762 \\ 
GPT-4o-0806-128k & 0.5429 & 0.5339 \\ 
\bottomrule
\end{tabular}
\captionof{table}{Evidence retrieval scores in the attributable question answering setting.}
\label{tbl:appendix-attributable-qa-evidence-retrieval-results}
\end{Table}


We have considered answer generation based on the top retrieved paragraphs (RAG) or using the full context (\S\ref{sec:experiments-answerability-answer-generation}). In the RAG setup, the answer generation can generally be attributed to the retrieved passages (assuming the model is faithful to the context). However, when using the full text as context, attribution to the passage level is not trivial. Recently, attributable question answering has gained momentum \citep{bohnet-etal-2022-attributed,gao-etal-2023-enabling,malaviya-etal-2024-expertqa}, where in addition to generating an answer, the model is supposed to cite evidence supporting it. Therefore, we also conduct an experiment where the model is conditioned on the full text of the paper and is tasked to "cite" any paragraphs on which the generated answer is based. We prepend an id before each paragraph and include an instruction on how to cite. Specifically, we use the following prompt:

\texttt{
Read the following paper and answer the question. Provide one or several evidence paragraphs that can be used to verify the answer. Give as few paragraphs as possible, but as many that provide evidence to the answer. Your answer must have the following format: "\textless answer\textgreater\ [X] [Y]". In your reply, replace \textless answer\textgreater\ with your answer to the question and add any references in square brackets. Your answer must be followed by the ids of the relevant segments from the document.
Question: \{question\}\\
Paper: \{paper\}\\
Answer:
} 
\\

This setting has the challenge that the model does not provide a ranked list of all paragraphs but an unordered list of what it considers relevant. Therefore, we rank the cited paragraphs in the order in which the LLM generates them.

Table~\ref{tbl:appendix-attributable-qa-evidence-retrieval-results} reports the results of the evidence retrieval with the attributable question answering setup. We find that for \texttt{GPT-3.5}, the scores fall far behind the performance of a dedicated retrieval model (e.g., \texttt{SPLADEv3}). For \texttt{GPT-4o}, the MRR outperforms \texttt{SPLADEv3}, however, the Recall@10 is inferior.

We further investigate the answer generation performance of the attributable QA setup, reporting the results in Table~\ref{tbl:appendix-attributable-qa-answer-generation-results}. Compared with the RAG setting using the top 20 paragraphs retrieved by \texttt{SPLADEv3}, the attributable QA setup performs worse. A RAG setup is also significantly more cost and compute-efficient, particularly considering the long context of papers. Specifically, the average paragraph in PeerQA has 94 tokens, leading to an average of 1880 tokens to encode in the RAG-20 setting. In contrast, on average, a paper has 11723 tokens. Therefore, the full-text setup is 6.24 times more expensive than the RAG-20 setting. 
\end{multicols}

\begin{table*}[h]
\centering
\small
\begin{tabular}{@{}llcccccccc@{}}
\toprule
 &  & \multicolumn{3}{c}{Rouge-L} & \multicolumn{3}{c}{AlignScore} & \multicolumn{2}{c}{Prometheus} \\ 
\cmidrule(lr){3-5}
\cmidrule(lr){6-8}
\cmidrule(lr){9-10}
Model & Ctx. & AE & FF & GPT-4 FF & AE & FF & GPT-4 FF & FF & GPT-4 FF \\
\midrule
\multirow{3}{*}{\begin{tabular}[c]{@{}l@{}}GPT-3.5\\ Turbo-0613-16k\end{tabular}} 
 & 20 & \textbf{0.1388} & \textbf{0.2211} & \textbf{0.2465} & \textbf{0.4255} & \textbf{0.1446} & \textbf{0.2394} & \textbf{2.9714} & \textbf{3.0888} \\
 & FT & 0.1162 & 0.1895 & 0.2188 & 0.3341 & 0.0771 & 0.1524 & 2.7143 & 2.9060 \\
 & FT Cite & 0.1099 & 0.1846 & 0.2057 & 0.2453 & 0.1128 & 0.1564 & 2.4340 & 2.4837 \\
 \midrule
\multirow{3}{*}{\begin{tabular}[c]{@{}l@{}}GPT-4o\\ 0806-128k\end{tabular}}
 & 20 & 0.1798 & \textbf{0.2039} & \textbf{0.2453} & \textbf{0.4094} & 0.0963 & \textbf{0.1830} & 3.5510 & 3.5927 \\
 & FT & \textbf{0.1821} & 0.1981 & 0.2372 & 0.3900 & 0.0713 & 0.1790 & \textbf{3.5673} & \textbf{3.6057} \\
 & FT Cite & 0.1262 & 0.1857 & 0.1602 & 0.2678 & \textbf{0.1177} & 0.1622 & 2.7143 & 2.5614 \\
 \bottomrule
\end{tabular}
\caption{Answer generation scores in the attributable question answering setting ("FT Cite") and two baselines for comparisons. In bold the best performing setup per metric.}
\label{tbl:appendix-attributable-qa-answer-generation-results}
\end{table*}