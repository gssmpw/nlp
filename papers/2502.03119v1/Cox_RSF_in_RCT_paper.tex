


%Version 2.1 April 2023
% See section 11 of the User Manual for version history
%
%%%%%%%%%%%%%%%%%%%%%%%%%%%%%%%%%%%%%%%%%%%%%%%%%%%%%%%%%%%%%%%%%%%%%%
%%                                                                 %%
%% Please do not use \input{...} to include other tex files.       %%
%% Submit your LaTeX manuscript as one .tex document.              %%
%%                                                                 %%
%% All additional figures and files should be attached             %%
%% separately and not embedded in the \TeX\ document itself.       %%
%%                                                                 %%
%%%%%%%%%%%%%%%%%%%%%%%%%%%%%%%%%%%%%%%%%%%%%%%%%%%%%%%%%%%%%%%%%%%%%

%%\documentclass[referee,sn-basic]{sn-jnl}% referee option is meant for double line spacing

%%=======================================================%%
%% to print line numbers in the margin use lineno option %%
%%=======================================================%%

%%\documentclass[lineno,sn-basic]{sn-jnl}% Basic Springer Nature Reference Style/Chemistry Reference Style

%%======================================================%%
%% to compile with pdflatex/xelatex use pdflatex option %%
%%======================================================%%

%\documentclass[pdflatex,sn-basic]{sn-jnl}% Basic Springer Nature Reference Style/Chemistry Reference Style
\documentclass[natbib]{sn-jnl}
\usepackage{natbib}
%\usepackage[style=apa]{biblatex}

%%Note: the following reference styles support Namedate and Numbered referencing. By default the style follows the most common style. To switch between the options you can add or remove �Numbered� in the optional parenthesis. 
%%The option is available for: sn-basic.bst, sn-vancouver.bst, sn-chicago.bst, sn-mathphys.bst. %  

%%\documentclass[sn-nature]{sn-jnl}% Style for submissions to Nature Portfolio journals
%%\documentclass[sn-basic]{sn-jnl}% Basic Springer Nature Reference Style/Chemistry Reference Style
%\documentclass[sn-mathphys,Numbered]{sn-jnl}% Math and Physical Sciences Reference Style
%%\documentclass[sn-aps]{sn-jnl}% American Physical Society (APS) Reference Style
%%\documentclass[sn-vancouver,Numbered]{sn-jnl}% Vancouver Reference Style
%\documentclass[sn-apa]{sn-jnl}% APA Reference Style 
%%\documentclass[sn-chicago]{sn-jnl}% Chicago-based Humanities Reference Style
%%\documentclass[default]{sn-jnl}% Default
%%\documentclass[default,iicol]{sn-jnl}% Default with double column layout

%%%% Standard Packages
\usepackage{graphics}%
\usepackage{multirow}%
\usepackage{amsmath,amssymb,amsfonts}%
\usepackage{amsthm}%
\usepackage{mathrsfs}%
\usepackage[title]{appendix}%
% brightness of colours
\usepackage[dvipsames,table]{xcolor}%

\definecolor{myred}{RGB}{158, 15, 20}

% "minus" symbol
\usepackage{textcomp}%
\usepackage{manyfoot}%
% combined table
\usepackage{booktabs}%
\usepackage{algorithm}%
\usepackage{algorithmicx}%
\usepackage{algpseudocode}%
\usepackage{listings}%
\usepackage{apacdoc} 

\usepackage{rotating}
\usepackage{dcolumn}

% move table to side ??
\usepackage{changepage}

% round down
\usepackage{mathtools}
\DeclarePairedDelimiter\floor{\lfloor}{\rfloor}

\newcolumntype{R}[2]{%
	>{\adjustbox{angle=#1,lap=\width-(#2)}\bgroup}%
	l%
	<{\egroup}%
}
\newcommand*\rot{\multicolumn{1}{R{40}{1.5em}}}% no optional argument here, please!

% rotate table
\usepackage{lscape}

%%<additional latex packages if required can be included here>
\usepackage{adjustbox}
%\newcommand{\ra}[1]{\renewcommand{\arraystretch}{#1}}
% multiple lines in one cell
\usepackage{array}
% hdashline (dash)
\usepackage{arydshln}
% values for dotted line
\setlength\dashlinedash{0.2pt}
\setlength\dashlinegap{1.5pt}
\setlength\arrayrulewidth{0.3pt}
% line width in table
\usepackage{boldline}
% colour lines in table
\usepackage{colortbl}
%%%% begin flowchart
\usepackage{tikz}
\usetikzlibrary{arrows.meta,
	calc, chains,
	decorations.pathreplacing,%
	calligraphy,% had to be after decorations.pathreplacing
	shapes,
	% for own:
	shapes.geometric, arrows, matrix, positioning, fit, calc}
\definecolor{blue1}{RGB}{84,141,212}
\definecolor{blue2}{RGB}{142,180,227}
\definecolor{yellow1}{RGB}{255,229,153}
\definecolor{orange1}{RGB}{255,153,0}
\definecolor{gray1}{RGB}{127,127,127}
\definecolor{gray2}{RGB}{217,217,217}
%%%% end flowchart

% bold font
\usepackage{bm}
\usepackage{dsfont}
\usepackage{pdflscape}
\usepackage{ltxtable, tabularx}
\usepackage{geometry}
\geometry{
	a4paper,
	right = 25mm,
	left = 35mm,
	bottom = 25mm
}
% smaller matrix
%\newcommand\scalemath[2]{\scalebox{#1}{\mbox{\ensuremath{\displaystyle #2}}}}
\setcounter{MaxMatrixCols}{20}

% table at top of page
\makeatletter
\setlength{\@fptop}{0pt}
\makeatother

%%%%%%%%%
\newcounter{alphasect}
\def\alphainsection{0}

\let\oldsection=\section
\def\section{%
	\ifnum\alphainsection=1%
	\addtocounter{alphasect}{1}
	\fi%
	\oldsection}%

\renewcommand\thesection{%
	\ifnum\alphainsection=1% 
	\Alph{alphasect}
	\else%
	\arabic{section}
	\fi%
}%

\newenvironment{alphasection}{%
	\ifnum\alphainsection=1%
	\errhelp={Let other blocks end at the beginning of the next block.}
	\errmessage{Nested Alpha section not allowed}
	\fi%
	\setcounter{alphasect}{18}
	\def\alphainsection{1}
}{%
	\setcounter{alphasect}{18}
	\def\alphainsection{0}
}%
%%%%%%%%%%%%
% background color of text
%\usepackage[most]{tcolorbox}
%
%\tcbset{
%	        
%	colback=lime!20,
%	boxrule=0.1pt,
%	colframe=white,
%	fonttitle=\bfseries,
%	
%%	frame code={}
%%	center title,
%%	left=0pt,
%%	right=0pt,
%%	top=0pt,
%%	bottom=0pt,
%%	colback=lime!20,
%%	colframe=white,
%%	%width=\dimexpr\textwidth\relax,
%%	fonttitle=\bfseries,
%%	enlarge left by=0mm,
%%	%boxsep=5pt,
%%	arc=0pt,outer arc=0pt,
%}

%%%%%%%%%%%%%%%
%%%%%%%%%%%%%%%

% indicator function
\usepackage{bbm}
% table appear in place
\usepackage{float}

% arrange multiple pictures, subfigures
\usepackage{subcaption}
\usepackage{caption}
% write below and above symbol
\usepackage{stackengine}
%\newcommand\stacklgt[2]{%
%	\mathrel{\stackunder[2pt]{\stackon[4pt]{\lessgtr}{$\scriptscriptstyle#1$}}{%
%			$\scriptscriptstyle#2$}}}
% argmax
\DeclareMathOperator*{\argmax}{arg\,max}
% norm (vertical bars)
%\newcommand\norm[1]{\left\lVert#1\right\rVert}

%\makeatletter
%\newcommand*\rel@kern[1]{\kern#1\dimexpr\macc@kerna}
%\newcommand*\widebar[1]{%
%	\begingroup
%	\def\mathaccent##1##2{%
%		\rel@kern{0.8}%
%		\overline{\rel@kern{-0.8}\macc@nucleus\rel@kern{0.2}}%
%		\rel@kern{-0.2}%
%	}%
%	\macc@depth\@ne
%	\let\math@bgroup\@empty \let\math@egroup\macc@set@skewchar
%	\mathsurround\z@ \frozen@everymath{\mathgroup\macc@group\relax}%
%	\macc@set@skewchar\relax
%	\let\mathaccentV\macc@nested@a
%	\macc@nested@a\relax111{#1}%
%	\endgroup
%}
%\makeatother


%%%%

%%%%%=============================================================================%%%%
%%%%  Remarks: This template is provided to aid authors with the preparation
%%%%  of original research articles intended for submission to journals published 
%%%%  by Springer Nature. The guidance has been prepared in partnership with 
%%%%  production teams to conform to Springer Nature technical requirements. 
%%%%  Editorial and presentation requirements differ among journal portfolios and 
%%%%  research disciplines. You may find sections in this template are irrelevant 
%%%%  to your work and are empowered to omit any such section if allowed by the 
%%%%  journal you intend to submit to. The submission guidelines and policies 
%%%%  of the journal take precedence. A detailed User Manual is available in the 
%%%%  template package for technical guidance.
%%%%%=============================================================================%%%%

%\jyear{2021}%

%% as per the requirement new theorem styles can be included as shown below
\theoremstyle{thmstyleone}%
\newtheorem{theorem}{Theorem}%  meant for continuous numbers
%%\newtheorem{theorem}{Theorem}[section]% meant for sectionwise numbers
%% optional argument [theorem] produces theorem numbering sequence instead of independent numbers for Proposition
\newtheorem{proposition}[theorem]{Proposition}% 
%%\newtheorem{proposition}{Proposition}% to get separate numbers for theorem and proposition etc.

\theoremstyle{thmstyletwo}%
\newtheorem{example}{Example}%
\newtheorem{remark}{Remark}%

\theoremstyle{thmstylethree}%
\newtheorem{definition}{Definition}%

\raggedbottom
%%\unnumbered% uncomment this for unnumbered level heads

\begin{document}
	
	
\title[Article Title]{Comparison of the Cox proportional hazards model and Random Survival Forest algorithm for predicting patient-specific survival probabilities in clinical trial data}
	
	
	
\author*[1]{\fnm{Ricarda} \sur{Graf}}\email{ricarda.graf@math.uni-augsburg.de}

\author[2]{\fnm{Susan} \sur{Todd}}\email{s.c.todd@reading.ac.uk}

\author[2]{\fnm{M. Fazil} \sur{Baksh}}\email{m.f.baksh@reading.ac.uk}


\affil[1]{\orgdiv{Institute of Mathematics}, \orgname{University of Augsburg}, \orgaddress{\city{Augsburg}, \postcode{86159},  \country{Germany}}}

\affil[2]{\orgdiv{Department of Mathematics and Statistics}, \orgname{University of Reading}, \orgaddress{\city{Reading}, \postcode{RG6 6AX}, \country{UK}}}



%%==================================%%
%% sample for unstructured abstract %%
%%==================================%%

\abstract{% <= 350 words
	
	%  Unstructured abstract ($\leq 250$ words): 
	
	The Cox proportional hazards model is often used for model development in data from randomized controlled trials (RCT) with time-to-event outcomes. Random survival forests (RSF) is a machine-learning algorithm known for its high predictive performance.  We conduct a comprehensive neutral comparison study to compare the predictive performance of Cox regression and RSF in real-world as well as simulated data. Performance is compared using multiple performance measures according to recommendations for the comparison of prognostic prediction models. We found that while the RSF usually outperforms the Cox model when using the $C$ index, Cox model predictions may be better calibrated. With respect to overall performance, the Cox model often exceeds the RSF in nonproportional hazards settings, while otherwise the RSF typically performs better especially for smaller sample sizes. Overall performance of the RSF is more affected by higher censoring rates, while overall performance of the Cox model suffers more from smaller sample sizes. 
	
	%	the Cox model may exceed overall performance of the RSF for higher censoring rates and larger sample sizes  
	%	
	%	
	%%	The motivation is to identify settings in which one method is preferable over the other in terms of predictive performance. Performance is compared using multiple performance measures chosen according to the TRIPOD (Transparent Reporting of a multivariable prediction model for Individual Prognosis or Diagnosis) recommendations. 
	%	We found that conclusions based solely on the $C$ index, a rank-based measure of discrimination predominantly used for comparison of both models in real-world data, may be misleading when considering the models' overall performance, a measure additionally affected by model calibration. The standard log-rank RSF splitting rule may be outperformed by the alternative extremely randomized trees splitting rule in many settings. For small total sample sizes, the RSF may outperform Cox regression in the presence of treatment-covariate interactions with respect to overall performance. In general, overall performance of Cox regression is not worse compared to RSF according to these simulation study results, including cases where the proportional hazard assumption is violated.   
	
	
	% \begin{itemize}	 
		%	\item Background:
		%	\item Methods:
		%	\item Results:
		%	\item Discussion:
		%	\item Conclusions:
		%\end{itemize}
		% Background, Methods, Results, Conclusions
		
		%%%	\color{orange}{Example color box}
		
	}
	
	%%================================%%
	%% Sample for structured abstract %%
	%%================================%%
	
	% \abstract{\textbf{Purpose:} The abstract serves both as a general introduction to the topic and as a brief, non-technical summary of the main results and their implications. The abstract must not include subheadings (unless expressly permitted in the journal's Instructions to Authors), equations or citations. As a guide the abstract should not exceed 200 words. Most journals do not set a hard limit however authors are advised to check the author instructions for the journal they are submitting to.
		% 
		% \textbf{Methods:} The abstract serves both as a general introduction to the topic and as a brief, non-technical summary of the main results and their implications. The abstract must not include subheadings (unless expressly permitted in the journal's Instructions to Authors), equations or citations. As a guide the abstract should not exceed 200 words. Most journals do not set a hard limit however authors are advised to check the author instructions for the journal they are submitting to.
		% 
		% \textbf{Results:} The abstract serves both as a general introduction to the topic and as a brief, non-technical summary of the main results and their implications. The abstract must not include subheadings (unless expressly permitted in the journal's Instructions to Authors), equations or citations. As a guide the abstract should not exceed 200 words. Most journals do not set a hard limit however authors are advised to check the author instructions for the journal they are submitting to.
		% 
		% \textbf{Conclusion:} The abstract serves both as a general introduction to the topic and as a brief, non-technical summary of the main results and their implications. The abstract must not include subheadings (unless expressly permitted in the journal's Instructions to Authors), equations or citations. As a guide the abstract should not exceed 200 words. Most journals do not set a hard limit however authors are advised to check the author instructions for the journal they are submitting to.}
	
	
	% 3-10 keywords
	\keywords{Cox regression, Machine-learning prediction, Proportional hazards, Random survival forest, Randomized controlled trials, Simulation study, Survival analysis}
	%%\pacs[JEL Classification]{D8, H51}
	
	%%\pacs[MSC Classification]{35A01, 65L10, 65L12, 65L20, 65L70}
	
	\maketitle
	
	%\section{Graphics}
	%%%%%%%  %%%%%%%
	% https://tex.stackexchange.com/questions/620752/position-of-arrow-in-tikz-flowchart
	%%%%%%% 2 %%%%%%%
	% https://stackoverflow.com/questions/72013717/coding-the-arrows-in-flowchart
	%%%%%%%%%%%%%%%5  3.a %%%%%%%%%%%%%%%%%%%%%%%%%%%%
	% https://tex.stackexchange.com/questions/419456/right-align-of-tikz-node-with-different-lengths
	%%%%%%%%%%%%%%% 3 %%%%%%%%%%%%%%%%%%%%%%%
	
	%	\section*{Highlights}
	%	%	3-5 bullet points each 85 characters at most
	%	\begin{itemize}
		%		\item Simulation study for comparing the performance of Cox regression and RSF in RCT data
		%		\item Performance depends on the specific setting, i.e. combination of data properties
		%		\item RSF may be preferable when treatment-covariate interactions occur
		%		\item Cox model predictions are generally reliable even with non-proportional hazards
		%		\item Overall performance measures allow for more objective comparison
		%		
		%		
		%	\end{itemize}
	%	
	%	\section*{Glossary}
	%	%definition of field-specific terms
	%	\begin{itemize}
		%		\item RSF: RSF is an ensemble algorithm which combines predictions of many decision trees to reduce variation. It is based on recursive partitioning of the data. Decorrelation of single tree predictions is achieved by introducing randomness, i.e. each tree is grown based on a random bootstrap sample from the data, the best split in each node of a tree is found among a random subset of the variables. Overall predictions equal the average of all tree predictions. 
		%		\item Bootstrap dataset: resampling procedure for which samples are drawn with replacement in order to obtain a dataset of the same size as the original data with the aim of estimating a specific sample statistic.
		%		\item Non-parametric: the method does not rely on the assumption that the data sample is drawn from a particular probability distribution. 
		%		\item Out-of-bag samples/data: Samples/observations that are not used for growing a particular tree in a RSF/ that are not included in the particular bootstrap sample for building the tree model.
		%	\end{itemize}
	
	\section{Introduction}\label{sec1}
	
	Prognostic prediction models (also clinical prediction models, or risk scores) are used to estimate an individual's probability based on multiple risk factors that a disease or outcome will occur in a specific period of time. They are most often used at time of diagnosis or start of treatment to support physicians in early detection, diagnosis, treatment decision, and prognosis, and to inform patients about their risks \citep{Moons2012}. They are applied in the medical field in general, and in particular in the field of cancer  treatment and research, the field of diabetes and the cardiovascular field  \citep{Moons2012, Goldstein2016}. Clinical decision tools such as ``ClinicalPath'' \citep{Elsevier2022} for cancer treatment or the Framingham Risk Score \citep{Wilson1998} for coronary heart disease, are examples of prognostic prediction models.\\
	Cox regression \citep{Cox1972} is most widely used for developing prognostic models in medical time-to-event data \citep{Goldstein2016, Collins2011, Collins2014, Mahar2017, Mallett2010, Steyerberg2013, Wynants2019, Phung2019, Hueting20222}. These models are based on, among others, patient and disease characteristics, laboratory measurements, and medical tests to estimate an individuals risk of experiencing the outcome. Cox regression provides estimates of the hazard ratios for each explanatory variable. In the context of clinical trials, the treatment effect hazard ratio is of particular interest, i.e. the relative likelihood of the outcome in patients receiving a specific treatment compared to a control \citep{Mallett2010}. As a semi-parametric model, it is assumed that at least 10 events have to be observed per predictor variable included in the Cox model to obtain reasonable results \citep{Peduzzi1995, Vittinghoff2006}, so it cannot be used in high-dimensional settings with a large number of potential predictor variables compared to the number of individuals. During model development, researchers often have to decide on a fraction of available predictors to be included in the final model \citep{Moons2012}. Even if there are many potential uncorrelated candidate variables, a decision for a limited set has to be made.  Cox regression requires prespecification of a model,  including possible (higher order) interaction terms and variable transformations in case of nonlinear relationships of continuous covariates with the survival outcome. Moreover, it makes the assumption of proportional hazards which means that it assumes the hazard ratio of any two patients  to be constant over the period of follow-up.
	In cases where the new treatment only shows an advantage at an early or later stage, respectively, interpretation of its results may not be meaningful. Especially in long-term studies, this assumption may be violated \citep{Hilsenbeck1998}.
	On the other hand, the Cox model provides corresponding measures of uncertainty (confidence intervals for the hazard ratios), which generally form the basis for clinical decision making, is easy to use and has short computational times. When using the  Cox model for predictions, the specification of a baseline survival distribution is required \citep{Therneau2000}.\\
	In comparison, the Random survival forest (RSF) algorithm  \citep{Ishwaran2008} is a nonparametric machine-learning approach. It is suitable for the same variable types as the Cox model, i.e. continuous right-censored survival time outcomes  and continuous as well as categorical predictor variables. In contrast, it does not require an explicit specification of a model but is able to detect and incorporate even complex interactions between the covariates and the survival outcome as well as nonlinear relationships. It is also suitable for a large number of covariates, although it is advisable not to include variables that are already suspected not to be meaningful in order  to not unnecessarily increase computational complexity. It also seems suitable for dependent censoring \citep{Zhou2014}. Moreover, it does not require the proportional hazard assumption. However, since the RSF does not make parametric assumptions regarding the data, it also does not provide uncertainty measures such as confidence intervals for its estimates which are important for the analysis of clinical trial data. Machine-learning methods such as random forests have proven to increase predictive accuracy in prognostic studies \citep{Murmu2024}, especially 
	in high-dimensional data such as genetic, protein or imaging biomarkers \citep{Cohen2018, Zhang2020, Kawakami2019, Lin2022, Ruyssinck2016}. For example, a prognostic model for glioblastoma widely used for more than two decades and most recently adapted to incorporate further relevant covariates \citep{Bell2017}, is based on a survival tree method. The RSF may help predicting patient outcomes and survival rates more accurately. Therefore the current work aims to explore the potential application of the RSF to data from randomized controlled trials (RCT) by comparing its predictive performance to the Cox proportional hazards (Cox-PH) model.  \\
	Previous studies compared the performance of Cox regression and RSF in observational clinical data, more specifically real-world datasets \citep[e.g.][]{Guo2023, Sarica2023, Chowdhury2023, Moncada2021, Farhadian2021, Miao2015, Spooner2020,Qiu2020,  Kim2019, Datema2012,  Omurlu2009, Du2020}. The predominantly  used performance measure in these studies is  Harrell's $C$ index \citep{Harrell1982, Harrell1996}, a rank-based measure of discrimination, used in combination with cross-validation. Very rarely, calibration, or overall performance are assessed. Only the study by Du et al. (\citeyear{Du2020}) considered all three types of performance measures. In their systematic review and meta-analysis of 52 studies predicting hypertension, Chowdhury et al. (\citeyear{Chowdhury2022}) compared the performance of regression approaches (including  Cox regression) and various machine-learning methods (RSF was not applied in any of the studies). These authors too found that performance comparison based on the  $C$ index was common in contrast to comparisons based on calibration. Most of the above mentioned studies aiming to compare the performance of the Cox and RSF model stated an at least slightly better performance of the RSF model with respect to the $C$ index.  \\
	To the  best of our knowledge, only one study compared the two  approaches (among other approaches) based on data simulations  \citep{Baralou2022}. In the particularly extensive simulation study by Baralou et al. (\citeyear{Baralou2022}), the reference data is taken from an observational study. Apart from simulating data with different characteristics, another difference to most of the before-mentioned studies is, that  their comparison is not only based on the default log-rank splitting rule for the RSF, but includes two further splitting rules. Moreover, they compare the approaches based on a measure of discrimination (time-dependent area under the curve, AUC) as well as a measure of overall performance (Integrated Brier score, IBS \citep{Graf1999}). Most notably, they found that the RSF outperformed the Cox-PH model in scenarios with lower censoring rates in the presence of covariate interactions. However, they do not examine the performance for data from randomized controlled trials (including factors specific to RCTs such as different sizes of treatment effect, the absence/presence of treatment-covariate interactions, and smaller sample sizes less than 500), the influence of violation of the proportional hazard assumption, other splitting rules available for the RSF, and measures of calibration.  \\
	The  TRIPOD (Transparent Reporting of a multivariable prediction model for Individual Prognosis or Diagnosis) recommendations \citep{Moons2015} state that prognostic models should be compared with respect to discrimination (e.g. Harrell's $C$ index, time-dependent AUC for time-to-event data), calibration, and overall performance (e.g. Integrated Brier score). These three aspects of model performance are also described in Steyerberg et al. (\citeyear{Steyerberg2010}) and McLernon et al. (\citeyear{McLernon2023}), for example. In this paper, we will use Harrell's $C$ index, calibration curves, and the Integrated Brier score, which will be described in Section \ref{perf_meas}. \\
	Responsible integration of machine learning algorithms in any step of a clinical trial may help overcome some of the challenges in its design, conduct, and analysis, e.g. with respect to patient recruitment, or the planning of treatment interventions \citep{Miller2023, Weissler2021}. Evidence is needed where machine learning algorithms can be applied in order to gain an advantage such as more precise predictions free from parametric assumptions on the data structure. To the  best of our knowledge, this is the first simulation study comparing the performance of the Cox-PH and RSF model for clinical trial settings. The aim of this paper is to evaluate the predictive accuracy of both methods, the Cox regression model and the RSF algorithm, in predicting patient-specific survival probabilities in right-censored clinical trial data. We considered two possible scenarios, where   treatment-covariate interactions  in the data are either absent or present. For this purpose, two publicly available clinical trial datasets \citep{UMASS, Byar1980} without and with known treatment-covariate interactions serve as a reference for data simulations. 
	In contrast to previous studies, we compared the performance of all six  RSF splitting rules (currently available in the most commonly used \texttt{R} packages \texttt{randomForestSRC} \citep{Ishwaran2023} and \texttt{ranger} \citep{Wright2023}), and based the evaluation on  measures of discrimination, calibration, and overall performance for a more detailed comparison. Values for censoring rate, sample sizes, and size of treatment effect are varied. 	\\
	
	
	\section{Materials and methods}
	
	\subsection{Reference datasets}
	Two clinical trial datasets serve as a reference for data simulations, where one dataset does not have any known treatment-covariate interactions (Section \ref{data_pbc}), and the other comprises multiple  treatment-covariate interactions (Section \ref{data_pc}). More details are given in the following sections.
	
	\subsubsection{Data without treatment-covariate interactions: Randomized Controlled Trial (RCT) in primary biliary cirrhosis}\label{data_pbc}
	An RCT conducted by the Mayo Clinic between 1974 and 1984 \citep{UMASS} investigates the effect of D-penicillamine on survival times in 312 patients with primary biliary cirrhosis (PBC), with time to the occurrence of death, or liver transplantation, respectively, as the event of interest. A total of 16 prognostic factors were recorded of which  ten were continuous and six were categorical variables. The median follow-up time is about five years. Table \ref{mayo_summary} shows more detailed summary statistics. We replaced missing values in three  of the continuous covariates by their column means, i.e. incomplete data are included for estimating the correlation structure and fitting univariate parametric distributions to the data.
	
	\begin{table}[htb]
		\small
		\renewcommand{\arraystretch}{1.25}
		
		\renewcommand\thetable{1a}
		
		\resizebox{\textwidth}{!}{ 
			\begin{tabular}{p{1em}lp{5em}p{8em}p{5em}p{5em}c} \toprule
				&                                                       & {\small Median} & {\small Mean (SE)} & {\small Minimum} & {\small Maximum} & {\small \# missing values} \\  \arrayrulecolor{black}\cmidrule[0.05pt]{1-7}
				\multicolumn{2}{l}{ {\small Survival time}} & & & & &  \\ \arrayrulecolor{gray!70}\cmidrule[0.05pt]{1-7}
				& Time of follow-up \textit{[Days]}                     & 1839.5          & 2006.4 (1123.3)    & 41               & 4556             & \textminus                 \\ \arrayrulecolor{gray!70}\cmidrule[0.05pt]{1-7}
				\multicolumn{2}{l}{ {\small Continuous prognostic factors}} & & & & &  \\ \arrayrulecolor{gray!70}\cmidrule[0.05pt]{1-7}
				& Age \textit{[Years]}                                  & 49.8            & 50 (10.6)          & 26.3             & 78.4             & \textminus                 \\
				& Serum bilirubin \textit{[mg/dl]}                      & 1.4             & 3.3 (4.5)          & 0.3              & 28               & \textminus                 \\   
				
				& Serum cholesterol  \textit{[mg/dl]}                   & 322             & 369.6 (221.3)      & 120              & 1775             & \textminus                 \\ 
				
				& Albumin \textit{[gm/dl]}                              & 3.5             & 3.5 (0.4)          & 2                & 4.6              & \textminus                 \\
				& Urine copper \textit{[mg/day]}                        & 73              & 97.6 (85.6)        & 4                & 588              & 2                          \\
				& Alkaline phosphatase \textit{[U/liter]}               & 1259            & 1982.7 (2140.4)    & 289              & 13862.4          & \textminus                 \\ %\arrayrulecolor{gray}\cmidrule[0.05pt]{1-7} 
				& Aspartate aminotransferase  - SGOT \textit{[U/ml]}    & 114.7           & 122.6 (56.7)       & 26.4             & 457.2            & \textminus                 \\
				& Triglycerides \textit{[mg/dl]}                        & 108             & 124.7 (65.1)       & 33               & 598              & 30                         \\
				& Platelet count \textit{[\# platelets per m$^3$/1000]} & 257             & 261.9 (95.6)       & 62               & 563              & 4                          \\				
				& Prothrombin time \textit{[sec]}                       & 10.6            & 10.7 (1)           & 9                & 17.1             & \textminus                 \\    \arrayrulecolor{black}\cmidrule[0.05pt]{1-7}
				&	& \multicolumn{4}{c}{ {\small Levels} }  &  {\small \# missing values}  \\  \arrayrulecolor{black}\cmidrule[0.05pt]{1-7}
				\multicolumn{2}{l}{ {\small Event indicator, treatment code}} & & & &  \\ \arrayrulecolor{gray!70}\cmidrule[0.05pt]{1-7}
				& Event indicator \textit{[0: censored, 1: death]}      & 0: 59.9\%       & 1: 40.1\%          &                  &                  & \textminus                 \\ 
				& Treatment code \textit{[1: DPA, 2: placebo]}          & 1: 50.6\%       & 2: 49.4\%          &                  &                  & \textminus                 \\		\arrayrulecolor{gray!70}\cmidrule[0.05pt]{1-7}		
				\multicolumn{2}{l}{ {\small Categorical prognostic factors}} & & & &  \\ \arrayrulecolor{gray!70}\cmidrule[0.05pt]{1-7}
				& Sex \textit{[0: male, 1: female]}                     & 0: 11.5\%       & 1: 88.5\%          &                  &                  & \textminus                 \\
				& Presence of ascites \textit{[0: no, 1: yes]}          & 0: 92.3\%       & 1: 0.07\%          &                  &                  & \textminus                 \\
				& Presence of hepatomelagy \textit{[0: no, 1: yes]}     & 0: 48.7\%       & 1: 51.3\%          &                  &                  & \textminus                 \\
				& Presence of spiders \textit{[0: no, 1: yes]}          & 0: 71.2\%       & 1: 28.8\%          &                  &                  & \textminus                 \\
				& Presence of edema \textit{$^{1)}$}                    & 0: 84.3\%       & 0.5: 9.3\%         & 1: 6.4\%         &                  & \textminus                 \\ 
				& Histologic state of disease \textit{[grade]}          & 1: 5.1\%        & 2: 21.5\%          & 3: 38.5\%        & 4: 34.9\%        & \textminus                 \\  	\arrayrulecolor{black}\cmidrule[0.05pt]{1-7} 	
			\end{tabular}
		}	
		\caption[Summary statistics of baseline measurements in 312 primary biliary cirrhosis patients in the randomized controlled trial conducted by the Mayo Clinic.]{\small Randomized controlled trial in primary biliary cirrhosis: summary statistics of baseline measurements in 312 patients in the study conducted by the Mayo Clinic.
			\\{\footnotesize $^{1)}$ 0 = no edema and no diuretic therapy for edema; 0.5 = edema present for which no diuretic therapy was given or edema resolved with diuretic therapy; 1 = edema despite diuretic therapy.\\
				Abbreviations: DPA -  D-penicillamine, SGOT - serum glutamic-oxaloacetic transaminase }} 
		
		\label{mayo_summary}		
	\end{table}
	
	\noindent Performance comparison of the Cox model to a non-parametric alternative such as the RSF is motivated by the violation of the proportional hazard assumption in some datasets on which Cox regression is based. For instance, in this RCT dataset,  the overall assumption of proportional hazards would be violated ($\chi^2 = 20.86$, df = 8, $p = 0.0075$, test by Grambsch and Therneau \citep{Grambsch1994} implemented in the  function \texttt{cox.zph} from the \texttt{R}  package \texttt{survival} \citep{Therneau2024} after variable selection based on   findings in the literature and the statistical measures AIC (Akaike information criterion) and BIC (Bayesian information criterion), an approach a researcher examining these data would typically follow. Model selection was done as follows: in the literature, the GLOBE score \citep{Lammers2015}, the Mayo score \citep{Dickson1989}, and the UK-PBC score \citep{Carbone2015} are suggested as prognostic risk scores for prediction of survival in patients with primary biliary cirrhosis \citep{Goet2021}. According to these, the variables age, bilirubin, alkaline phosphatase, platelet count, and prothrombin time are relevant and were therefore included in the model. The histologic stage of disease has also been determined as a relevant factor \citep{Scheuer1989} and is therefore included. Other available variables were dropped from the analysis. More specifically, urine copper levels in patients suffering from primary biliary cirrhosis were usually within a normal range \citep{Carey1980, Salaspuro1981}. In addition, aspartase aminotransferase (AST)/SGOT is not  a disease-specific indicator because values may be  increased in patients with cirrhosis in general \citep{Sebastiani2006}, and found to be normal or only modestly increased in other studies in primary biliary cirrhosis patients \citep{Zhang2002, Nguyen2014, Poupon1991}. Models fulfilling the before mentioned conditions were then compared based on AIC and BIC. The final dataset, in which the proportional hazards assumption was tested, consequently included the variables age, serum bilirubin, albumin, alkaline phosphatase, platelet count, prothrombin time, and histologic stage of disease. 
	
	
	
	
	
	\subsubsection{Data with treatment-covariate interactions: Randomized Controlled Trial (RCT) in  prostate cancer patients}\label{data_pc}
	The second dataset considered comprises 474 patients with advanced prostate cancer for whom complete data are available in the RCT examining the effect of the synthetic oestrogen drug diethyl stilboestrol on survival time. The placebo group comprises patients receiving either placebo or the lowest dose level, the treatment group comprises patients receiving one of  two higher dose levels \citep{Byar1980}. Table \ref{byar_summary} gives an overview of the data structure.
	For data simulations, we removed the binary variable cancer stage due to multicollinearity. Based on findings in the literature  \citep{Byar1980, Royston2004}, we included relevant interaction terms between treatment and the variables age, presence of bone metastases, and serum acid phosphatase, respectively. Again, in a model comprising all main effects and these three interaction terms, for example, the proportional hazard assumption would not be fulfilled ($\chi^2 = 22.2$, df = 12, $p = 0.0355$,  test by Grambsch and Therneau \citep{Grambsch1994}), therefore comparing the performance of the Cox-PH model to a model  not based on this assumption such as the RSF may be interesting. 
	
	
	
		\begin{table}[htb]
	\small
	\renewcommand{\arraystretch}{1.25}
	\renewcommand\thetable{1b}
	\resizebox{\textwidth}{!}{ 
		\begin{tabular}{p{0.5em}lp{5em}p{8em}p{5em}p{5em}c} \toprule
			&		& {\small Median}            & {\small Mean (SE)}             & {\small Minimum}    & {\small Maximum}  & {\small \# missing values}  \\  \arrayrulecolor{black}\cmidrule[0.05pt]{1-7}
			\multicolumn{2}{l}{ {\small Survival time}} & & & & &  \\ \arrayrulecolor{gray!70}\cmidrule[0.05pt]{1-7}
			& Time of follow-up \textit{}  & 33.5 & 36.3 (23.2) &  0.5 &  76.5 & \textminus  \\ \arrayrulecolor{gray!70}\cmidrule[0.05pt]{1-7}
			\multicolumn{2}{l}{ {\small Continuous prognostic factors}} & & & & &  \\ \arrayrulecolor{gray!70}\cmidrule[0.05pt]{1-7}
			& Age \textit{[Years]}         &	 73 &  71.6 (6.9) & 48 & 89 & \textminus \\
			& Standardized weight \textit{}  & 98 & 99 (13.3) & 69 & 152  & \textminus   \\   
			
			& Systolic blood pressure  \textit{}       & 14 & 14.4 (2.4) & 8 & 30  & \textminus   \\ 
			
			& Diastolic blood pressure \textit{}        & 8 & 8.2 (1.5) & 4 & 18 &  \textminus  \\
			& Size of primary tumour \textit{[cm$^2$]}	 & 10 & 14.3 (12.2) &  0 & 69 & \textminus \\
			& \parbox{6cm}{Serum (prostatic) acid phosphatase \\ \textit{[King Armstrong units]}}   & 7 & 125.7 (638.5) & 1 & 9999  & \textminus \\ %\arrayrulecolor{gray}\cmidrule[0.05pt]{1-7} 
			& Haemoglobin \textit{[g/100 ml]}                                      & 137 & 134.2 (19.4) & 59 & 182 &  \textminus \\
			& Gleason stage-grade category \textit{[mg/dl]}	                       & 10 & 10.3 (2) & 5 & 15 & \textminus\\   \arrayrulecolor{black}\cmidrule[0.05pt]{1-7}
			&	& \multicolumn{4}{c}{ {\small Levels} }  &  {\small \# missing values}  \\  \arrayrulecolor{black}\cmidrule[0.05pt]{1-7}
			\multicolumn{2}{l}{ {\small Event indicator, treatment code}} & & & &  \\ \arrayrulecolor{gray!70}\cmidrule[0.05pt]{1-7}
			& Event indicator \textit{[0: censored, 1: death]}        	& 0: 28.8\% & 1: 71.2\% & & & \textminus \\ 
			&  	\parbox{6cm}{Treatment code \\ \textit{[0: lowest dose of diethyl stilboestrol (placebo), 1: higher doses]}}	            & 0: 49.9\%  & 1: 50.1\% & & &  \textminus \\		\arrayrulecolor{gray!70}\cmidrule[0.05pt]{1-7}		
			\multicolumn{2}{l}{ {\small Binary prognostic factors}} & & & &  \\ \arrayrulecolor{gray!70}\cmidrule[0.05pt]{1-7}
			& Performance status \textit{}	                    & 0: 90.1\%  & 1: 9.9\% &&  &  \textminus \\
			& History of cardiovascular disease  \textit{[0: no, 1: yes]}	& 0: 56.6\%  & 1: 43.4\% & & &  \textminus \\
			& Presence of bone metastases \textit{[0: no, 1: yes]}	        & 0: 83.8\%  & 1: 16.2\% & &  &  \textminus \\
			& \parbox{6cm}{Abnormal electrocardiogram \\ \textit{[0: normal, 1: abnormal]}}	& 0: 34.1\%  & 1: 65.9\% & & & \textminus \\
			\arrayrulecolor{black}\cmidrule[0.05pt]{1-7} 	
		\end{tabular}
	}
	\caption[Summary statistics of baseline measurements in 474 prostate cancer patients in the randomized controlled trial dataset in prostate cancer patients.]{\small{Randomized controlled trial in prostate cancer patients: summary statistics of baseline measurements in 474 patients in the prostate cancer dataset.} }
	\label{byar_summary}
\end{table}
	
	

	
	\subsection{Methods for performance comparison}
	Methods were first compared using the bootstrap technique by Wahl et al. (\citeyear{Wahl2016}) which is based on the work by Jiang et al. (\citeyear{Jiang2008}), an internal validation technique based on the real data. It uses bootstrap samples to obtain point estimates of the performance measures and corresponding CIs. It is described in Section \ref{method_boot}. A second approach then uses data simulations. Simulations facilitate manipulations of properties of the data but  at the same time require specification of data-generating mechanisms, i.e. underlying parametric distributions for variables are assumed. The approach is described in Section \ref{method_sim}. 
	
	\subsubsection{Comparison in real-world data: nonparametric bootstrap approach}\label{method_boot}
	
	
	\noindent	The nonparametric bootstrap approach for point estimates by Wahl et al. (\citeyear{Wahl2016}) is an extension of the algorithm by Jiang et al. (\citeyear{Jiang2008}) and based on the .632+ bootstrap method \citep{Efron1997}, and thus assumes independence of  observations. It estimates the .632+ bootstrap estimate ($\hat{\theta}^{.632+}$) of the respective performance measure including a 95\% confidence interval.\\
	The  bootstrap estimate $\hat{\theta}^{.632+}$ is computed as a weighted average of the apparent performance $\hat{\theta}^{orig,orig}$ (training and test data given by the original dataset) and the average ``out-of-bag'' (OOB) performance $\hat{\theta}^{bootstrap,OOB} = \sum\limits_{b=1}^B \hat{\theta}^{bootstrap,OOB}_b$ computed from $B$ bootstrap datasets (training data given by the bootstrap dataset, and test data given by the samples not present in the bootstrap dataset). The formula is:
	\begin{equation*}
		\hat{\theta}^{.632+} = (1-w) \cdot \hat{\theta}^{orig,orig} + w \cdot \hat{\theta}^{bootstrap,OOB},
	\end{equation*}
	where $w = \frac{0.632}{1-0.368 \cdot \text{R}}$ and R = $\frac{\hat{\theta}^{bootstrap,OOB} - \hat{\theta}^{orig,orig}}{\theta^{noinfo}   - \hat{\theta}^{orig,orig}}$. In case of the $C$ index, $\theta^{noinfo} = 0.5$. For the Brier score, $\theta^{noinfo} = 0.75$. 
	Then each bootstrap dataset is assigned a weight $w_b =  \hat{\theta}^{bootstrap,bootstrap}_b - \hat{\theta}^{orig,orig}$, where $\hat{\theta}^{bootstrap,bootstrap}_b$ is the value of the performance measure, when the bootstrap dataset $b \in \{1,\cdots,B\}$ is used as training as well as test dataset. The  $\frac{\alpha}{2}$ and $1 - \frac{\alpha}{2}$ percentiles of the empirical distribution of these weights, $\xi_{\frac{\alpha}{2}}$ and $\xi_{1 - \frac{\alpha}{2}}$ , give the CI of $\hat{\theta}^{.632+}$:
	\begin{equation*}
		[\hat{\theta}^{.632+} -  \xi_{1 - \frac{\alpha}{2}}, \hat{\theta}^{.632+} + \xi_{\frac{\alpha}{2}}]
	\end{equation*}	
	
	
	
	\subsubsection{Comparison in data with differing properties: data simulations}\label{method_sim}
	For data simulations, we generated covariate data similar to the reference data by using copula models. The specific  distributions and corresponding parameters we used can be found in the Supplementary Material  A %\ref{params} 
	(Table A.1a,  %\ref{sigma_mayo}, 
	 Table A.1b, %\ref{sigma_byar}, 
	 Table A.2a, %\ref{dist_mayo}, 
	 Table A.2b, % \ref{dist_byar},  
	 Fig. A.1, %\ref{qq_cdf_mayo}, 
	 Fig. A.2). %\ref{qq_cdf_byar}). 
	Covariate-dependent survival times were generated from a Weibull($\lambda, \gamma$) distribution according to the cumulative hazard inversion method by Bender et al. (\citeyear{Bender2005})  implemented in the \texttt{R} package \texttt{simsurv} \citep{Brilleman2022}. 
	Scale parameters $\lambda$ were fixed at the value estimated from the respective reference dataset ($\lambda$ = 2241.74 for the primary biliary cirrhosis dataset, $\lambda$ = 39.2 for the prostate cancer dataset), shape parameters $\gamma$ were varied in order to create scenarios with decreasing ($\gamma = 0.8$), constant ($\gamma = 1$), increasing ($\gamma = 2$), and non-proportional hazards, i.e. different values per treatment group ($\gamma_0 = 2, \gamma_1 = 5$).  Random censoring times were generated from a uniform distribution $U_{[0,b]}$ such that censoring percentages of 30\% and 60\%, respectively, corresponding to the actual censoring rates in the two reference datasets, were obtained. For this, we used the approach by Ramos et al. (\citeyear{Ramos2024}), but in some cases had to manually adjust the values of the distribution parameter $b$. We considered total sample sizes $N \in \{100, 200, 400\}$ for the $n_{\text{sim}} = 500$ training datasets. For the $n_{\text{sim}} = 500$ independent test datasets, the total sample size is $N = 500$. Moreover, we considered different values of the treatment effect when generating the data ($\beta_{\text{treatment}} \in \{0, 0.8,  -0.4\}$) corresponding to different hazard ratios of the treatment effect.
	For the RSF, all available splitting rules  are included in the method comparison (overview in  Table A.3%\ref{splitrules}
	). \\
	For both scenarios, the scenario without treatment-covariate interactions (primary biliary cirrhosis dataset) and the scenario with multiple treatment-covariate interactions (prostate cancer dataset) all variables (main effects) are included in the model. Interaction terms are not included since usually the presence of interactions is unknown due to limited sample size. For the Cox model, backward variable selection is done based on the Akaike information criterion (as implemented in the \texttt{R} function \texttt{stepAIC} from the \texttt{MASS} package \citep{Ripley2024}) in order to determine which main effects will be included. The RSF does intrinsic variable selection: for each split in the tree a random subset of splitting variables among the available variables is selected, and data are partitioned based on the splitting value which yields the optimal result with respect to the splitting rule, i.e. which results in highest heterogeneity between the two respective subsets. \\
	Application of the RSF requires specification of some hyperparameter values, which can be done using grid search. This means that multiple values around the recommended defaults are specified, and for each combination the error rate (1 \textminus $C$ index) is computed in those data which were not used to build a particular tree model, i.e. the ``out-of-bag'' data. The RSF does work reasonably well even without extensive hyperparameter tuning \citep{Boehmke2019}. According to Boehmke and Greenwell \citep{Boehmke2019}, the number of trees and the number of randomly chosen variables considered for data splitting in a node ($m_{\text{try}}$) are the most important hyperparameters. They recommend to use 10 times the number of variables (covariates) present in the model as the number of trees. They mention the default which equals the square root of the number of  covariates rounded down to the nearest integer ($m_{\text{try}} = \floor*{\sqrt{\# \text{covariates}}}$) for the number of randomly chosen splitting variables per node, and which is used in both \texttt{R} packages, \texttt{randomForestSRC} \citep{Ishwaran2023} and \texttt{ranger} \citep{Wright2023}. Boehmke and Greenwell \citep{Boehmke2019} recommend to consider five evenly spaced values around this default during grid search. In order to prevent overfitting, early-stopping criteria can be used with both \texttt{R} functions, either the maximum tree depth or a minimum number of samples per terminal/leaf node can be specified.  In \texttt{randomForestSRC} \citep{Ishwaran2023} the default for the minimum number of samples in a terminal node is set to 15 observations. In our grid search, we considered five, ten and 15 times the number of covariates for the number of trees, 15, 30, and 45 for the number of observations in terminal nodes, and three, four, and five for the value  of $m_{\text{try}}$. We chose less than the recommended five values for $m_{\text{try}}$ because these RCT datasets consist of a rather small number of covariates compared to other applications of the RSF. \\
	We measured computational times per algorithm including variable selection (for the Cox model) and hyperparameter tuning (for the RSF model), respectively.   
	
	
	
	\subsection{Performance measures}\label{perf_meas}
	According to recommendations \citep{Moons2015, Steyerberg2010, McLernon2023}, we based comparisons of the algorithms' performance on performance metrics measuring discrimination, calibration, and overall performance. In the context of survival analysis, discrimination refers to the model's ability to distinguish between patients with higher and lower risk of the outcome. Calibration compares predicted survival probabilities to the observed event frequencies in a given time interval. Overall performance encompasses both discrimination as well as calibration of the model. Some performance measures have been extended for use with survival outcomes.
	
	
	\subsubsection{Measure of discrimination: Harrell's $C$ index}\label{cindex}
	The $C$ index was originally developed for binary outcomes \citep{Harrell1985}, and its modification for the use in data with survival outcomes has been subject to criticism \citep{Hartman2023}. For each pair of patients, the  $C$ index  compares   whether the one with the shorter event time also has the higher predicted risk of suffering the event.  These rank-based comparisons may favour the model with the more inaccurate predictions \citep{Vickers2010}, and may not adequately reflect the influence  different sets of covariates have on the outcome \citep{Cook2007}, such that its interpretation may be misleading and not clinically meaningful for survival outcomes. Information on how the $C$ index is calculated for the Cox and RSF model can be found in Supplementary Material C. Computation of the $C$ index is implemented in the function \texttt{cindex} in the \texttt{R} package \texttt{pec} \citep{Gerds2023} for the Cox model, and in the function \texttt{get.cindex} in the \texttt{R} package \texttt{randomForestSRC} \citep{Ishwaran2023} for the RSF.\\
	

	
	\subsubsection{Measure of calibration: Calibration curves}
	A calibration plot of observed on predicted probabilities of mortality indicates deviation from perfect prediction the more the slope deviates from the ideal diagonal line \citep{Calster2019}. It quantifies
	the agreement between the actual and predicted outcome within a specified duration of time. Austin et al. (\citeyear{Austin2020}) describe and implement an approach for estimating calibration curves for survival outcomes. We used their calibration curve which   is  estimated based on Cox regression using restricted cubic splines.
	
	
	\subsubsection{Measure of overall performance: Integrated Brier score}	
	The Integrated Brier score \citep{Graf1999} summarizes the Brier scores over time, i.e. it is a time range performance measure. The Brier score calculates the difference between predicted and actual survival at a given time point, and thus values indicate better overall performance the closer they are to zero. It is implemented in the function \texttt{integrated$\_$brier$\_$score} in the \texttt{R} package \texttt{survex} \citep{survex2024}. 
	
	
	
	\section{Results}
	
	\subsection{Results of the bootstrap approach}
	Table \ref{bootstrap_mayo} and Table \ref{bootstrap_byar} show the bootstrap estimates of the $C$ index and Integrated Brier score, respectively, when applying the bootstrap approach for point estimates \citep{Jiang2008, Wahl2016} to both reference datasets. The same results are shown in Figure \ref{mayo_boot_plot} (primary biliary cirrhosis dataset) and Figure \ref{byar_boot_plot} (prostate cancer dataset).
	The first impression is that the point estimates $\hat{\theta}^{.632+}$ alone indicate a potentially better performance of most RSF models compared to the Cox-PH model but their confidence intervals are often much wider and mostly include the $\hat{\theta}^{.632+}$ estimate of the Cox-PH model. This is the case for both reference datasets, either without or including treatment-covariate interactions. For the PBC data (Figure \ref{mayo_boot_plot}), confidence intervals of the RSF with splitting rules ``log-rank test'', ``log-rank score test'', and ``extremely randomized trees'' do not include the $\hat{\theta}^{.632+}$ estimate of the $C$ index of the Cox-PH model, with completely separate confidence intervals for the ``extremely randomized trees'' splitting rule. For the $\hat{\theta}^{.632+}$ estimates of the Integrated Brier score, all confidence intervals corresponding to RSF models contain the respective $\hat{\theta}^{.632+}$ estimate of the Cox model, although the point estimates $\hat{\theta}^{.632+}$ of the RSF indicate a somewhat better overall performance (except for the log-rank score test splitting rule). Regarding the prostate cancer dataset (Figure \ref{byar_boot_plot}), all confidence intervals of the $\hat{\theta}^{.632+}$ $C$ index estimates  of the RSF models contain the $\hat{\theta}^{.632+}$ estimate of the Cox model. Only the confidence interval of the $\hat{\theta}^{.632+}$ Integrated Brier score  estimates of the RSF with the log-rank splitting rule does not contain the respective estimate of the Cox model, but the confidence intervals still slightly overlap.\\
	In summary, the RSF (``log-rank test'') seems to have an advantage over the Cox model when comparing overall performance in both datasets, because its point estimates are the lowest and the corresponding confidence intervals have the smallest overlap. With respect to the $C$ index, the RSF (``extremely randomized trees'') performs better in the data without treatment-covariate interactions (primary biliary cirrhosis dataset). For the prostate cancer dataset, RSF may have better performance concluded from the point estimates alone, but confidence intervals of the Cox and RSF models completely overlap such that no clear conclusion can be made.
	
	
	%	For both datasets, the $\hat{\theta}^{.632+}$ estimate  of the $C$ index are slightly higher for the RSF models compared to Cox regression. For the log-rank score and the extremely randomized trees splitting rules, the CIs are non-overlapping for the RCT conducted in PBC patients, and thus indeed indicate a better discriminative performance of the RSF. Apart from these cases, CIs are overlapping and no advantage of one method over the other can be concluded.  \\
	%	In contrast, the $\hat{\theta}^{.632+}$ estimate of the IBS indicates a worse overall performance of the  RSF log-rank score splitting rule compared to the Cox model in the PBC dataset.  This may support the criticism concerning the use of the $C$ index for survival outcomes that has been alluded to in the literature (Section \ref{cindex}). All other CIs are overlapping, thus do not indicate an advantage of either the Cox or RSF model.
	
	
	%	Table \ref{bootstrap} shows the results obtained by the bootstrap approach for point estimates \citep{Wahl2016} applied to the Mayo Clinic data. The method may provide confidence intervals. The drawback is that the technique has not yet been implemented for others to use. An alternative technique for obtaining confidence intervals for the performance estimates is to use cross-validation as done in many of the publications mentioned above. In Table \ref{bootstrap},  the results of the $C$ index suggest slightly better discriminative performance of any RSF model (higher estimates with narrower confidence intervals) compared to Cox regression, still the confidence intervals are overlapping. In terms of the Integrated Brier score, Cox regression slightly outperforms the RSF (lower estimate but wider confidence interval). The RSF performs best for the maximally selected rank statistics as the node splitting rule. \\
	%	Figure \ref{calib_bootstrap} shows the calibration curves and their 2.5th and 97.5th percentile when using the bootstrap datasets model fitting and the holdout samples for estimation of the calibration curves. The Cox model predictions are better calibrated than any of the RSF models, explaining the better overall performance of Cox regression despite better discriminative performance of the RSF models. 
	
	
	%[calibration curve for bootstrap]
	
	
	\newgeometry{top=20mm, bottom=10mm}
	
	\begin{table}[htb]
		\renewcommand\thetable{3}
		\def\arraystretch{0.6}
		\setlength{\tabcolsep}{6pt}
		\caption[Bootstrap estimates $\hat{\theta}^{.632+}$ (95\% confidence interval) of the $C$ index and Integrated Brier score in the  data without treatment-covariate interactions (primary biliary cirrhosis dataset).]{\fontsize{9}{10}\selectfont  Bootstrap estimates $\hat{\theta}^{.632+}$ (95\% confidence interval) of the $C$ index and Integrated Brier score in the RCT data without treatment-covariate interactions (\underline{primary biliary cirrhosis dataset}). Predictions are based on $n_{\text{sim}}$ = 1000 bootstrap datasets.}		
		\label{bootstrap_mayo}		
		\footnotesize		
		\begin{tabular}{@{}p{4em}p{6em}p{6em}p{6em}p{6em}p{6em}p{6em}p{6em}@{}} \arrayrulecolor{black}\cmidrule[0.1pt]{1-8} 
			
			& \multirow{4}{*}{Cox-PH}  & \multicolumn{6}{c}{\small{Random survival forest}} \\ \arrayrulecolor{black}\cmidrule[0.05pt]{3-8}
			&                     & \parbox{4em}{\linespread{1}\selectfont Log-rank test} & \parbox{4em}{\linespread{1}\selectfont Log-rank score} & \parbox{7em}{\linespread{1}\selectfont Gradient-based Brier score} & \parbox{5em}{\linespread{1}\selectfont Harrell's $C$} & \parbox{6em}{\linespread{1}\selectfont Extremely randomized trees} & \parbox{6em}{\linespread{1}\selectfont Maximally selected rank statistics} \\ \arrayrulecolor{black}\cmidrule[0.05pt]{1-8} 
			
			$C$ index                                                        & 0.776 (0.735,0.817) & 0.855 (0.778,0.932)                                   & 0.847 (0.815,0.88)                                     & 0.856 (0.768,0.944)                                                & 0.858 (0.764,0.953)                                   & 0.844 (0.819,0.869)                                                & 0.868 (0.698,1)                                                            
			\\ \arrayrulecolor{gray!50}\cmidrule[0.01pt]{1-8}
			\parbox{4em}{\linespread{1}\selectfont  Integrated Brier  score} & 0.131 (0.124,0.161) & 0.116 (0.106,0.146)                                   & 0.148 (0.118,0.197)                                    & 0.12 (0.112,0.151)                                                 & 0.117 (0.109,0.147)                                   & 0.129 (0.126,0.152)                                                & 0.121 (0.108,0.147)                                                        
			\\   \arrayrulecolor{black}\arrayrulecolor{black}\cmidrule[0.1pt]{1-8}		 
		\end{tabular}
	\end{table}
	
	\vspace{-1cm}
	
	\begin{figure}[H]
		\begin{minipage}{.47\textwidth}
			\includegraphics[scale = 0.5]{figures/Fig_boot_CI_cindex_MayoClinic.pdf}
		\end{minipage}
		\begin{minipage}{.05\textwidth}
			\qquad 			
		\end{minipage}
		\begin{minipage}{.47\textwidth}
			\includegraphics[scale = 0.5]{figures/Fig_boot_CI_IBS_MayoClinic.pdf}					
		\end{minipage}
		
		\vspace{0.1cm}
		\captionof{figure}[Bootstrap estimate $\hat{\theta}^{.632+}$ (95\% confidence interval) of the $C$ index and Integrated Brier score for the RCT in primary biliary cirrhosis patients.]{\fontsize{9}{10}\selectfont \small{\textbf{Bootstrap estimate $\hat{\theta}^{.632+}$ (95\% confidence interval) of the $C$ index (right) and Integrated Brier score (left) for the RCT in  \underline{primary biliary cirrhosis} patients. \\  } } 			
			{  \footnotesize  Abbreviations: Cox-PH - Cox proportional hazards model, RSF - Random survival forest.}		
		}
		
		\label{mayo_boot_plot}
		
	\end{figure}	
	
	\vspace{-1cm}
	
	\begin{table}[htb]
		\renewcommand\thetable{4}
		\def\arraystretch{0.6}
		\setlength{\tabcolsep}{6pt}
		\caption[Bootstrap estimates $\hat{\theta}^{.632+}$ (95\% confidence interval) of the $C$ index and Integrated Brier score in the data with three treatment-covariate interactions (prostate cancer dataset).]{\fontsize{9}{10}\selectfont  Bootstrap estimates $\hat{\theta}^{.632+}$ (95\% confidence interval) of the $C$ index and Integrated Brier score  data with three treatment-covariate interactions (\underline{prostate cancer dataset}). Predictions are based on $n_{\text{sim}}$ = 1000 bootstrap datasets.}
		\label{bootstrap_byar}		
		\footnotesize		
		\begin{tabular}{@{}p{4em}p{6em}p{6em}p{6em}p{6em}p{6em}p{6em}p{6em}@{}} \arrayrulecolor{black}\cmidrule[0.1pt]{1-8} 
			
			& \multirow{4}{*}{Cox-PH}  & \multicolumn{6}{c}{\small{Random survival forest}} \\ \arrayrulecolor{black}\cmidrule[0.05pt]{3-8}
			&                     & \parbox{4em}{\linespread{1}\selectfont Log-rank test} & \parbox{4em}{\linespread{1}\selectfont Log-rank score} & \parbox{7em}{\linespread{1}\selectfont Gradient-based Brier score} & \parbox{5em}{\linespread{1}\selectfont Harrell's $C$} & \parbox{6em}{\linespread{1}\selectfont Extremely randomized trees} & \parbox{6em}{\linespread{1}\selectfont Maximally selected rank statistics} \\ \arrayrulecolor{black}\cmidrule[0.05pt]{1-8} 
			
			$C$ index                                                        & 0.521 (0.513,0.53)  & 0.66 (0.438,0.881)                                    & 0.642 (0.462,0.821)                                    & 0.653 (0.456,0.85)                                                 & 0.657 (0.432,0.881)                                   & 0.645 (0.498,0.792)                                                & 0.663 (0.413,0.913)                                                        \\ \arrayrulecolor{gray!50}\cmidrule[0.01pt]{1-8}
			\parbox{4em}{\linespread{1}\selectfont  Integrated Brier  score} & 0.201 (0.194,0.211) & 0.17 (0.158,0.199)                                    & 0.183 (0.177,0.205)                                    & 0.176 (0.165,0.206)                                                & 0.172 (0.155,0.206)                                   & 0.179 (0.175,0.204)                                                & 0.172 (0.13,0.23)                                                          
			\\   \arrayrulecolor{black}\arrayrulecolor{black}\cmidrule[0.1pt]{1-8}		 
		\end{tabular}
	\end{table}
	
	\vspace{-1cm}
	
	\begin{figure}[H]
		\begin{minipage}{.47\textwidth}
			\includegraphics[scale = 0.5]{figures/Fig_boot_CI_cindex_Byar.pdf}
		\end{minipage}
		\begin{minipage}{.05\textwidth}
			\qquad 			
		\end{minipage}
		\begin{minipage}{.47\textwidth}
			\includegraphics[scale = 0.5]{figures/Fig_boot_CI_IBS_Byar.pdf}					
		\end{minipage}
		
		\vspace{0.1cm}
		\captionof{figure}[Bootstrap estimate $\hat{\theta}^{.632+}$ (95\% confidence interval) of the $C$ index and Integrated Brier score for the RCT data in primary biliary cirrhosis patients.]{\fontsize{9}{10}\selectfont \small{\textbf{Bootstrap estimate $\hat{\theta}^{.632+}$ (95\% confidence interval) of the $C$ index (left) and Integrated Brier score (right) for the RCT in \underline{prostate cancer} patients.\\}}{\footnotesize  Abbreviations: Cox-PH - Cox proportional hazards model, RSF - Random survival forest.}		
		}
		
		\label{byar_boot_plot}		
	\end{figure}	
	\restoregeometry
	
	\subsection{Simulation study results}
	
	In this section, the simulation study results for one of the treatment effects considered in the simulation study ($\beta_{\text{treatment}} = -0.4$) are presented and discussed. The results for other values of the treatment effect ($\beta_{\text{treatment}} \in \{0, 0.8\}$) are similar and can be found in the Supplementary Material. Varying the size of the treatment effect  only seems to have a minor influence on the methods' performance. Moreover, we only show the results for the algorithms that are of most interest, which are the Cox model and the RSF using the standard log-rank test splitting rule. Additionally, the results of the RSF based on other splitting rules are shown if they outperform these two methods with respect to the median result. Only the best performing one among them is shown in case there are multiple better performing alternatives. Results for the remaining RSF splitting rules are shown in the Supplementary Material. \\
	The $C$ index estimates, which correspond to the models' discriminative performance, are shown in Figure \ref{boxplots_cindex_beta0_cens_30} (30\% censoring rate) and Figure \ref{boxplots_cindex_beta0_cens_60} (60\% censoring rate). Results for the RCT data  without treatment-covariate interactions (PBC dataset) are shown in Figure \ref{boxplots_cindex_beta0_cens_30}(a) and Figure \ref{boxplots_cindex_beta0_cens_60}(a). For a censoring rate of 30\%, varying hazards, and sample sizes, the RSF based on the log-rank test splitting rule performs best. For a censoring rate of 60\%, the Cox model performs best in the nonproportional hazards setting independent of sample size, and otherwise the RSF performs best. In this case (censoring rate of 60\%), for a total sample size of $N = 100$, the RSF using the ``maximally selected rank statistics'' splitting rule (slightly) outperforms the standard log-rank test splitting rule in the scenarios assuming a decreasing and constant hazard. Otherwise the log-rank test splitting rule gives the best results. Results for the RCT data with multiple treatment-covariate interactions (prostate cancer dataset) are shown in Figure \ref{boxplots_cindex_beta0_cens_30}(b) and Figure \ref{boxplots_cindex_beta0_cens_60}(b). Here, the RSF based on the ``extremely randomized trees'' splitting rule performs best for both censoring rates, the considered total sample sizes of $N \in \{100,200,400\}$ and the varying properties of the hazard. The difference to the standard RSF using the log-rank splitting rule is more evident with a proportion of approximately up to 50\% for the nonoverlapping parts of the boxes although the difference is less clear in the nonproportional hazards setting.\\
	The Integrated Brier score estimates, which correspond to the models' overall performance (encompassing both, the models' discrimination and calibration) are shown in Figure \ref{boxplots_ibs_beta0_cens_30} (30\% censoring rate) and Figure \ref{boxplots_ibs_beta0_cens_60} (60\% censoring rate). Results for the RCT data  without treatment-covariate interactions (PBC dataset) are shown in Figure \ref{boxplots_ibs_beta0_cens_30}(a) and Figure \ref{boxplots_ibs_beta0_cens_60}(a). The Cox model has clearly the best overall performance in the nonproportional hazards settings. It also (very slightly) outperforms the RSF in the scenario with increasing hazard when either $N = 400$ for a censoring rate of 30\% or when $N \in \{200, 400\}$ for a censoring rate of 60\%. Otherwise, the RSF performs better. The difference is even more evident when the total sample size becomes smaller. For decreasing and constant hazards,  the ``Gradient-based Brier score'' splitting rule slightly outperforms the ``log-rank test'' splitting rule for the RSF for both censoring rates and all sample sizes.
	Results for the RCT data with multiple treatment-covariate interactions (prostate cancer dataset) are shown in Figure \ref{boxplots_ibs_beta0_cens_30}(b) and Figure \ref{boxplots_ibs_beta0_cens_60}(b). From these results, it can be suspected that the Cox model's performance in comparison to the RSF increases for larger sample sizes and higher censoring rates. While the Cox model only slightly outperforms the RSF for decreasing and constant hazards in case $N = 400$ when the censoring rate is 30\%, it does so more clearly for a censoring rate of 60\%, where it additionally clearly outperforms the RSF in the nonproportional hazards settings. Overall, the gap in performance between both models becomes smaller for the higher  censoring percentage. For all other scenarios the RSF using the ``extremely randomized trees'' splitting rule performs best (with one exception where the log-rank splitting rule works better: $N = 100$ with decreasing hazard and 30\% censoring) although the difference to the log-rank test splitting rule is marginal. \\
	Some calibration curves at median survival time for the RCT data  without treatment-covariate interactions (PBC dataset) are shown in Figure \ref{Mayo_calib_treat_04_ph} and Figure \ref{Mayo_calib_treat_04_nonph}. Calibration curves for a proportional and nonproportional hazards setting are compared. Calibration curves for the respective scenarios for the RCT data  with multiple treatment-covariate interactions (prostate cancer dataset) are shown in Figure \ref{Byar_calib_treat_04_ph} and Figure \ref{Byar_calib_treat_04_nonph}. Calibration of the Cox model improves with increasing sample size while for the RSF this is at least less evident. Especially in the nonproportional hazards setting and absence of treatment-covariate interactions, the Cox model's results are better calibrated compared to the RSF (Figure \ref{Mayo_calib_treat_04_nonph}). In contrast, the difference in calibration between the two models is less obvious for the nonproportional hazards setting in case treatment-covariate interactions are present in the data (Figure \ref{Byar_calib_treat_04_nonph}). Judged by the percentiles shown as dashed lines, calibration generally varies less in the Cox model results than in the RSF results.  Deviation from perfect calibration of the RSF results is caused by a too narrow range of predictions compared to the true values, resulting in calibration curves that are too steep.\\
	Computational complexity of the methods is compared in Figure \ref{comp_times}. It includes the variable selection step for the Cox model, and the grid search for finding the optimal combination of hyperparameters for the RSF.  Computational times are the lowest for the Cox model, although the RSF still has relatively low computational times for total sample sizes of $N = 100$, and even for the larger sample sizes when the RSF splitting rules ``log-rank test'',``extremely randomized trees'', or ``maximally selected rank statistics'' are used. In contrast, computational time considerably increases for larger sample sizes as well as larger number of covariates for the RSF splitting rules ``log-rank score test'', ``gradient-based Brier score'', and ``Harrell's $C$''.\\
	Complete simulation study results can be found in Supplementary Material B.1 ($C$ index), B.2 (Integrated Brier score), and B.3 (calibration curves).
	

	
	
	
	
	\begin{figure}[H]
		%	\renewcommand\thefigure{1a}
		\begin{subfigure}{\textwidth}
			\begin{minipage}{0.98\textwidth} 	
				\renewcommand\thefigure{}
				\renewcommand{\figurename}{}
				\includegraphics[scale = 0.2, width = 0.995\textwidth,trim={0cm, 0, 0cm, 0},clip]{figures/Mayo_03_c_index_boxplot_main_beta_04_nsim_500_1_500.pdf}   
				\vspace{-0.75cm}
				\caption[]{\textbf{}}		
			\end{minipage}		
			\begin{minipage}{0.005\textwidth} 	
				\includegraphics[scale = 0.325,trim={0cm, 0, 0cm, 0},clip]{figures/Mayo_03_legend_c_index_boxplot_main04_nsim_500_1_500.pdf}
			\end{minipage}	
		\end{subfigure}		
		\begin{subfigure}{\textwidth}	
			\begin{minipage}{0.98\textwidth} 	
				\renewcommand\thefigure{}
				\renewcommand{\figurename}{}
				\includegraphics[scale = 0.2, width = 0.995\textwidth,trim={0cm, 0, 0cm, 0},clip]{figures/Byar_03_c_index_boxplot_main_beta_04_nsim_500_1_500.pdf}   
				\vspace{-0.75cm}
				\caption[]{\textbf{}}
			\end{minipage}		
			\begin{minipage}{0.005\textwidth} 	
				\includegraphics[scale = 0.325,trim={0cm, 0, 0cm, 0},clip]{figures/Byar_03_legend_c_index_boxplot_main04_nsim_500_1_500.pdf}
			\end{minipage}	
		\end{subfigure}	
		\vspace{-0.3cm}
		\caption[$C$ index estimates in the simulation scenario assuming 30\% censoring and a treatment effect of $\beta_{\text{\normalfont{treatment}}} = -0.4$ for the RCT in primary biliary cirrhosis (a) and in prostate cancer patients (b).]{
			{\small \textbf{$C$ index estimates in the simulation scenario assuming 30\% censoring and a treatment effect of $\beta_{\text{\normalfont{treatment}}} = -0.4$ for the RCT in primary biliary cirrhosis (a) and in prostate cancer patients (b).} Survival times are generated from a Weibull distribution with scale parameters estimated from the respective reference dataset, shape parameters ($\gamma$) vary in order to examine the impact of differing hazards, and the violation of the proportional hazards assumption.
				Results are shown for different total sample sizes $N$.\\}
			{\footnotesize  Abbreviations:  Cox-PH - Cox proportional hazards model, RSF - Random survival forest.}
		} 
		\label{boxplots_cindex_beta0_cens_30}				
	\end{figure}
	
	
	
	\begin{figure}[H]
		%	\renewcommand\thefigure{1a}
		
		\begin{subfigure}{\textwidth}
			\begin{minipage}{0.98\textwidth} 	
				\renewcommand\thefigure{}
				\renewcommand{\figurename}{}
				\includegraphics[scale = 0.2, width = 0.995\textwidth,trim={0cm, 0, 0cm, 0},clip]{figures/Mayo_06_c_index_boxplot_main_beta_04_nsim_500_1_500.pdf}   
				\vspace{-0.75cm}
				\caption[]{\textbf{}}		
			\end{minipage}		
			\begin{minipage}{0.005\textwidth} 	
				\includegraphics[scale = 0.325,trim={0cm, 0, 0cm, 0},clip]{figures/Mayo_06_legend_c_index_boxplot_main04_nsim_500_1_500.pdf}
			\end{minipage}	
		\end{subfigure}		
		\begin{subfigure}{\textwidth}	
			\begin{minipage}{0.98\textwidth} 	
				\renewcommand\thefigure{}
				\renewcommand{\figurename}{}
				\includegraphics[scale = 0.2, width = 0.995\textwidth,trim={0cm, 0, 0cm, 0},clip]{figures/Byar_06_c_index_boxplot_main_beta_04_nsim_500_1_500.pdf}   
				\vspace{-0.75cm}
				\caption[]{\textbf{}}
			\end{minipage}		
			\begin{minipage}{0.005\textwidth} 	
				\includegraphics[scale = 0.325,trim={0cm, 0, 0cm, 0},clip]{figures/Byar_06_legend_c_index_boxplot_main04_nsim_500_1_500.pdf}
			\end{minipage}	
		\end{subfigure}			
		
		\vspace{-0.3cm}
		\caption[$C$ index estimates in the simulation scenario assuming 60\% censoring and a treatment effect of $\beta_{\text{\normalfont{treatment}}} = -0.4$ for the RCT in primary biliary cirrhosis (a) and in prostate cancer patients (b).]{{\small  \textbf{$C$ index estimates in the simulation scenario assuming 60\% censoring and a treatment effect of $\beta_{\text{\normalfont{treatment}}} = -0.4$ for the RCT in primary biliary cirrhosis (a) and in prostate cancer patients (b).} Survival times are generated from a Weibull distribution with scale parameters estimated from the respective reference dataset, shape parameters ($\gamma$) vary in order to examine the impact of differing hazards, and the violation of the proportional hazards assumption.
				Results are shown for different total sample sizes $N$.}\\
			{\footnotesize  Abbreviations:  Cox-PH - Cox proportional hazards model, RSF - Random survival forest.}
		} 
		\label{boxplots_cindex_beta0_cens_60}				
	\end{figure}
	
			
			%%% IBS
			
			\begin{figure}[H]
				\begin{subfigure}{\textwidth}
					\begin{minipage}{0.98\textwidth} 	
						\renewcommand\thefigure{}
						\renewcommand{\figurename}{}
						\includegraphics[scale = 0.2, width = 0.995\textwidth,trim={0cm, 0, 0cm, 0},clip]{figures/Mayo_03_ibs_boxplot_main_beta_04_nsim_500_1_500.pdf}   
				\vspace{-0.75cm}
						\caption[]{\textbf{}}		
					\end{minipage}		
					\begin{minipage}{0.005\textwidth} 	
						\includegraphics[scale = 0.325,trim={0cm, 0, 0cm, 0},clip]{figures/Mayo_03_legend_ibs_boxplot_main04_nsim_500_1_500.pdf}
					\end{minipage}	
				\end{subfigure}		
				\begin{subfigure}{\textwidth}	
					\begin{minipage}{0.98\textwidth} 	
						\renewcommand\thefigure{}
						\renewcommand{\figurename}{}
						\includegraphics[scale = 0.2, width = 0.995\textwidth,trim={0cm, 0, 0cm, 0},clip]{figures/Byar_03_ibs_boxplot_main_beta_04_nsim_500_1_500.pdf}   
				\vspace{-0.75cm}
						\caption[]{\textbf{}}
					\end{minipage}		
					\begin{minipage}{0.005\textwidth} 	
						\includegraphics[scale = 0.325,trim={0cm, 0, 0cm, 0},clip]{figures/Byar_03_legend_ibs_boxplot_main04_nsim_500_1_500.pdf}
					\end{minipage}	
				\end{subfigure}	
				\vspace{-0.3cm}
				\caption[Integrated Brier score (IBS) estimates in the simulation scenario assuming 30\% censoring and a treatment effect of $\beta_{\text{\normalfont{treatment}}} = -0.4$ for the RCT in primary biliary cirrhosis (a) and in prostate cancer patients (b).]{{\small{\textbf{Integrated Brier score (IBS) estimates in the simulation scenario assuming 30\% censoring and a treatment effect of $\beta_{\text{\normalfont{treatment}}} = -0.4$ for the RCT in primary biliary cirrhosis (a) and in prostate cancer patients (b).}} Survival times are generated from a Weibull distribution with scale parameters estimated from the respective reference dataset, shape parameters ($\gamma$) vary in order to examine the impact of differing hazards, and the violation of the proportional hazards assumption.
						Results are shown for different total sample sizes $N$.}\\
					{\footnotesize  Abbreviations:  Cox-PH - Cox proportional hazards model, RSF - Random survival forest.}
				} 
				\label{boxplots_ibs_beta0_cens_30}				
			\end{figure}
			
			
			
			\begin{figure}[H]
				\begin{subfigure}{\textwidth}
					\begin{minipage}{0.98\textwidth} 	
						\renewcommand\thefigure{}
						\renewcommand{\figurename}{}
						\includegraphics[scale = 0.2, width = 0.995\textwidth,trim={0cm, 0, 0cm, 0},clip]{figures/Mayo_06_ibs_boxplot_main_beta_04_nsim_500_1_500.pdf}   
				\vspace{-0.75cm}
						\caption[]{\textbf{}}		
					\end{minipage}		
					\begin{minipage}{0.005\textwidth} 	
						\includegraphics[scale = 0.325,trim={0cm, 0, 0cm, 0},clip]{figures/Mayo_06_legend_ibs_boxplot_main04_nsim_500_1_500.pdf}
					\end{minipage}	
				\end{subfigure}		
				\begin{subfigure}{\textwidth}	
					\begin{minipage}{0.98\textwidth} 	
						\renewcommand\thefigure{}
						\renewcommand{\figurename}{}
						\includegraphics[scale = 0.2, width = 0.995\textwidth,trim={0cm, 0, 0cm, 0},clip]{figures/Byar_06_ibs_boxplot_main_beta_04_nsim_500_1_500.pdf}   
				\vspace{-0.75cm}
						\caption[]{\textbf{}}
					\end{minipage}		
					\begin{minipage}{0.005\textwidth} 	
						\includegraphics[scale = 0.325,trim={0cm, 0, 0cm, 0},clip]{figures/Byar_06_legend_ibs_boxplot_main04_nsim_500_1_500.pdf}
					\end{minipage}	
				\end{subfigure}	
				\vspace{-0.3cm}
				\caption[Integrated Brier score (IBS) estimates in the simulation scenario assuming 60\% censoring and a treatment effect of $\beta_{\text{\normalfont{treatment}}} = -0.4$ for the RCT in primary biliary cirrhosis (a) and in prostate cancer patients (b).]{{\small{\textbf{Integrated Brier score (IBS) estimates in the simulation scenario assuming 60\% censoring and a treatment effect of $\beta_{\text{\normalfont{treatment}}} = -0.4$ for the RCT in primary biliary cirrhosis (a) and in prostate cancer patients (b).}} Survival times are generated from a Weibull distribution with scale parameters estimated from the respective reference dataset, shape parameters ($\gamma$) vary in order to examine the impact of differing hazards, and the violation of the proportional hazards assumption.
						Results are shown for different total sample sizes $N$.}\\
					{\footnotesize  Abbreviations:  Cox-PH - Cox proportional hazards model, RSF - Random survival forest.}
				} 
				\label{boxplots_ibs_beta0_cens_60}				
			\end{figure}
			
			
			
			%	\newgeometry{left=25mm, right=25mm}
			\begin{figure}[H]
				\begin{minipage}{.45\textwidth}
					\includegraphics[scale = 0.45,trim={0 0 3.5cm 0},clip]{figures/Mayo_03_calibration_curves_main_100__nsim_500_beta_treat_04_scale_shape_1_1_500.pdf}
					\begin{center}
%						\vspace{-0.3cm}
						(a)
%						\vspace{-0.4cm}
					\end{center}
					\includegraphics[scale = 0.45,trim={0 0 3.5cm 0},clip]{figures/Mayo_06_calibration_curves_main_100__nsim_500_beta_treat_04_scale_shape_1_1_500.pdf}
					\begin{center}
%						\vspace{-0.3cm}
						(c)
%						\vspace{-0.4cm}
					\end{center}
				\end{minipage}
				\begin{minipage}{.029\textwidth}
				\end{minipage}			
				\begin{minipage}{.52\textwidth}
					\includegraphics[scale = 0.45,trim={0.7cm 0 0 0},clip]{figures/Mayo_03_calibration_curves_main_400__nsim_500_beta_treat_04_scale_shape_1_1_500.pdf}
					\begin{center}
%						\vspace{-0.3cm}
						(b)
%						\vspace{-0.4cm}
					\end{center}
					\includegraphics[scale = 0.45,trim={0.7cm 0 0 0},clip]{figures/Mayo_06_calibration_curves_main_400__nsim_500_beta_treat_04_scale_shape_1_1_500.pdf}			
					\begin{center}
%						\vspace{-0.3cm}
						(d)
%						\vspace{-0.4cm}
					\end{center}				
				\end{minipage}
				\vspace{0.1cm}
				\captionof{figure}[Calibration curves at the median survival time for the data without treatment-covariate interactions (primary biliary cirrhosis dataset) for a proportional hazard setting.]{\linespread{1}\selectfont \small{\textbf{Calibration curves for a proportional hazards scenario (primary biliary cirrhosis dataset).}  \underline{Calibration curves at the median (50\% quantile)  survival time} for a \underline{proportional hazards} setting (Weibull survival time distribution W($\lambda = 2241.74, \gamma = 1$)), $\beta_{\text{treatment}} = -0.4$, and $n_{\text{sim}} = 500$ simulated datasets  based on data  \underline{without treatment-covariate interactions (primary biliary cirrhosis dataset)}. The solid line represents the mean calibration curve, the outer dotted lines represent the 2.5th and 97.5th percentile of the calibration curve. The black diagonal line corresponds to perfect calibration.\\
						(a) 30\% censoring, $N = 100$, (b) 30\% censoring, $N = 400$, \\(c) 60\% censoring, $N = 100$, (d) 60\% censoring, $N = 400$.\\  }  			
					{  \footnotesize  Abbreviations: Cox-PH - Cox proportional hazards model, RSF - Random survival forest.}		
				}
				
				\label{Mayo_calib_treat_04_ph}		
			\end{figure}
			%	\restoregeometry
			
			
			
			\begin{figure}[H]
				\begin{minipage}{.45\textwidth}
					\includegraphics[scale = 0.45,trim={0 0 3.5cm 0},clip]{figures/Mayo_03_calibration_curves_main_100__nsim_500_beta_treat_04_scale_shape_2_5_1_500.pdf}
					\begin{center}
%						\vspace{-0.3cm}
						(a)
%						\vspace{-0.4cm}
					\end{center}
					\includegraphics[scale = 0.45,trim={0 0 3.5cm 0},clip]{figures/Mayo_06_calibration_curves_main_100__nsim_500_beta_treat_04_scale_shape_2_5_1_500.pdf}
					\begin{center}
%						\vspace{-0.3cm}
						(c)
%						\vspace{-0.4cm}
					\end{center}
				\end{minipage}
				\begin{minipage}{.029\textwidth}
				\end{minipage}
				\begin{minipage}{.52\textwidth}
					\includegraphics[scale = 0.45,trim={0.7cm 0 0 0},clip]{figures/Mayo_03_calibration_curves_main_400__nsim_500_beta_treat_04_scale_shape_2_5_1_500.pdf}
					\begin{center}
%						\vspace{-0.3cm}
						(b)
%						\vspace{-0.4cm}
					\end{center}
					\includegraphics[scale = 0.45,trim={0.7cm 0 0 0},clip]{figures/Mayo_06_calibration_curves_main_400__nsim_500_beta_treat_04_scale_shape_2_5_1_500.pdf}			
					\begin{center}
%						\vspace{-0.3cm}
						(d)
%						\vspace{-0.4cm}
					\end{center}				
				\end{minipage}
				\vspace{0.1cm}
				\captionof{figure}[Calibration curves at the median survival time for the data without treatment-covariate interactions (primary biliary cirrhosis dataset) for a nonproportional hazard setting.]{\linespread{1}\selectfont \small{\textbf{Calibration curves for a nonproportional hazards setting (primary biliary cirrhosis dataset).}  \underline{Calibration curves at the median (50\% quantile)  survival time} for a \underline{nonproportional hazards} setting (Weibull survival time distribution W($ \lambda = 2241.74, \gamma \in \{2,5\}$)), $\beta_{\text{treatment}} = -0.4$, and $n_{\text{sim}} = 500$ simulated datasets  based on data  \underline{without treatment-covariate interactions (primary biliary cirrhosis dataset)}. The solid line represents the mean calibration curve, the outer dotted lines represent the 2.5th and 97.5th percentile of the calibration curve. The black diagonal line corresponds to perfect calibration.\\
						(a) 30\% censoring, $N = 100$, (b) 30\% censoring, $N = 400$, \\(c) 60\% censoring, $N = 100$, (d) 60\% censoring, $N = 400$.\\   }  			
					{  \footnotesize  Abbreviations: Cox-PH - Cox proportional hazards model, RSF - Random survival forest.}		
				}
				
				\label{Mayo_calib_treat_04_nonph}	
			\end{figure}
			
			
			
			\begin{figure}[H]
				\begin{minipage}{.45\textwidth}
					\includegraphics[scale = 0.45,trim={0 0 3.5cm 0},clip]{figures/Byar_03_calibration_curves_main_100__nsim_500_beta_treat_04_scale_shape_1_1_500.pdf}
					\begin{center}
%						\vspace{-0.3cm}
						(a)
%						\vspace{-0.4cm}
					\end{center}
					\includegraphics[scale = 0.45,trim={0 0 3.5cm 0},clip]{figures/Byar_06_calibration_curves_main_100__nsim_500_beta_treat_04_scale_shape_1_1_500.pdf}
					\begin{center}
%						\vspace{-0.3cm}
						(c)
%						\vspace{-0.4cm}
					\end{center}
				\end{minipage}
				\begin{minipage}{.029\textwidth}
				\end{minipage}
				\begin{minipage}{.52\textwidth}
					\includegraphics[scale = 0.45,trim={0.7cm 0 0 0},clip]{figures/Byar_03_calibration_curves_main_400__nsim_500_beta_treat_04_scale_shape_1_1_500.pdf}
					\begin{center}
%						\vspace{-0.3cm}
						(b)
%						\vspace{-0.4cm}
					\end{center}
					\includegraphics[scale = 0.45,trim={0.7cm 0 0 0},clip]{figures/Byar_06_calibration_curves_main_400__nsim_500_beta_treat_04_scale_shape_1_1_500.pdf}			
					\begin{center}
%						\vspace{-0.3cm}
						(d)
%						\vspace{-0.4cm}
					\end{center}				
				\end{minipage}
				\vspace{0.1cm}
				\captionof{figure}[Calibration curves at the median survival time for the data with three treatment-covariate interactions (prostate cancer dataset) for a proportional hazard setting.]{\linespread{1}\selectfont \small{\textbf{Calibration curves for a proportional hazards setting (prostate cancer dataset).}  \underline{Calibration curves at the median (50\% quantile)  survival time} for a \underline{proportional hazards} setting (Weibull survival time distribution W($\lambda = 2241.74, \gamma = 1$)), $\beta_{\text{treatment}} = -0.4$, and $n_{\text{sim}} = 500$ simulated datasets  based on data  \underline{with three treatment-covariate interactions (prostate cancer dataset)}. The solid line represents the mean calibration curve, the outer dotted lines represent the 2.5th and 97.5th percentile of the calibration curve. The black diagonal line corresponds to perfect calibration.\\
						(a) 30\% censoring, $N = 100$, (b) 30\% censoring, $N = 400$, \\(c) 60\% censoring, $N = 100$, (d) 60\% censoring, $N = 400$.\\ }  			
					{  \footnotesize  Abbreviations: Cox-PH - Cox proportional hazards model, RSF - Random survival forest.}		
				}
				
				\label{Byar_calib_treat_04_ph}		
			\end{figure}
			
			
			
			
			\begin{figure}[H]
				\begin{minipage}{.45\textwidth}
					\includegraphics[scale = 0.45,trim={0 0 3.5cm 0},clip]{figures/Byar_03_calibration_curves_main_100__nsim_500_beta_treat_04_scale_shape_2_5_1_500.pdf}
					\begin{center}
%						\vspace{-0.3cm}
						(a)
%						\vspace{-0.4cm}
					\end{center}
					\includegraphics[scale = 0.45,trim={0 0 3.5cm 0},clip]{figures/Byar_06_calibration_curves_main_100__nsim_500_beta_treat_04_scale_shape_2_5_1_500.pdf}
					\begin{center}
%						\vspace{-0.3cm}
						(c)
%						\vspace{-0.4cm}
					\end{center}
				\end{minipage}
				\begin{minipage}{.029\textwidth}
				\end{minipage}
				\begin{minipage}{.52\textwidth}
					\includegraphics[scale = 0.45,trim={0.7cm 0 0 0},clip]{figures/Byar_03_calibration_curves_main_400__nsim_500_beta_treat_04_scale_shape_2_5_1_500.pdf}
					\begin{center}
%						\vspace{-0.3cm}
						(b)
%						\vspace{-0.4cm}
					\end{center}
					\includegraphics[scale = 0.45,trim={0.7cm 0 0 0},clip]{figures/Byar_06_calibration_curves_main_400__nsim_500_beta_treat_04_scale_shape_2_5_1_500.pdf}			
					\begin{center}
%						\vspace{-0.3cm}
						(d)
%						\vspace{-0.4cm}
					\end{center}				
				\end{minipage}
				\vspace{0.1cm}
				\captionof{figure}[Calibration curves at the median survival time for the data with three treatment-covariate interactions (prostate cancer dataset) for a nonproportional hazard setting.]{\linespread{1}\selectfont \small{\textbf{Calibration curves for a nonproportional hazards setting (prostate cancer dataset).}  \underline{Calibration curves at the median (50\% quantile)  survival time} for a nonproportional hazard setting (Weibull survival time distribution W($\lambda = 39.2, \gamma \in \{2,5\}$)), $\beta_{\text{treatment}} = -0.4$, and $n_{\text{sim}} = 500$ simulated datasets  based on data  \underline{with three treatment-covariate interactions (prostate cancer dataset)}. The solid line represents the mean calibration curve, the outer dotted lines represent the 2.5th and 97.5th percentile of the calibration curve. The black diagonal line corresponds to perfect calibration.\\
						(a) 30\% censoring, $N = 100$, (b) 30\% censoring, $N = 400$, \\(c) 60\% censoring, $N = 100$, (d) 60\% censoring, $N = 400$.\\ }  			
					{  \footnotesize  Abbreviations: Cox-PH - Cox proportional hazards model, RSF - Random survival forest.}		
				}
				
				\label{Byar_calib_treat_04_nonph}	
			\end{figure}
			
			
			\begin{figure}[H]
				\begin{minipage}{.45\textwidth}
					\includegraphics[scale = 0.235,trim={0 0 10.5cm 0},clip]{figures/Mayo_computational_times.pdf}
				\end{minipage}
				\begin{minipage}{.029\textwidth}
				\end{minipage}
				\begin{minipage}{.52\textwidth}
					\vspace{0.2cm}
					\includegraphics[scale = 0.23,trim={0 0 0 0},clip]{figures/Byar_computational_times.pdf}				
				\end{minipage}
				\vspace{0.1cm}
				\captionof{figure}[Mean computational times for the RCT data without treatment-covariate interactions (primary biliary cirrhosis dataset), and for the RCT data with three treatment-covariate interactions (prostate cancer dataset).]{\linespread{1}\selectfont \small{\textbf{Mean computational times for the RCT data without treatment-covariate interactions (primary biliary cirrhosis dataset, left), and for the RCT data with three treatment-covariate interactions (prostate cancer dataset, right).} \\ }  			
					{  \footnotesize  Abbreviations: Cox-PH - Cox proportional hazards model, RSF - Random survival forest.}		
				}
				
				\label{comp_times}	
			\end{figure}
			
			
		
			
			\section{Discussion}
			We performed an extensive simulation study in order to perform neutral comparison of the Cox regression model and the RSF model when  predicting survival probabilities in RCT data, in which we included all currently available splitting rules for the RSF implemented in  two widely used \texttt{R} packages, \texttt{randomForestSRC} \citep{Ishwaran2021a} and \texttt{ranger} \citep{Wright2023}.  We followed  recommendations for neutral comparison studies \citep{Weber2019, Morris2019} to ensure an objective evaluation of the results. \\ 
			We considered a variety of settings. We used two publicly available RCT datasets as a reference for data simulations, where one dataset is characterized by the absence of treatment-covariate interactions (\citealp{UMASS}, biliary cirrhosis dataset)  and the other by two significant and one weak treatment-covariate interaction (\citealp{Byar1980}, prostate cancer dataset). In each case, we considered different total sample sizes, values of the treatment effect, censoring rates, and properties of the hazard
			that may occur in other real-world datasets. Comparisons are based on measures of discrimination, calibration, and overall performance as recommended in the literature \citep{Moons2015, Steyerberg2010, McLernon2023}.\\
			Depending on the research question, different aspects of the algorithm's performance may be more important. With respect to discrimination measured by the $C$ index, the RSF predictions are usually more accurate, with only one exception for the data without treatment-covariate interactions and nonproportional hazards with 60\% censoring. Many researchers who compared the performance of the Cox and RSF models in (real-world) observational medical data, have based their conclusions on the $C$ index estimates, but some methodologists oppose its extension and application to time-to-event medical data. In the data without treatment-covariate interactions the standard ``log-rank test'' RSF splitting rule performed best, while for the data with multiple treatment-covariate interactions, the ``extremely randomized trees'' splitting rule performed better than the standard RSF (which still performed better than the Cox model). \\
			With respect to overall performance measured by the IBS, the Cox model performs considerably better in the nonproportional hazards settings for both censoring rates (30\%, 60\%) in the data without treatment-covariate interactions, and for 60\% censoring in the data with multiple treatment-covariate interactions. We therefore assume that deviation from the proportional hazards assumption may not always affect the Cox models performance. Moreover, higher censoring rates may affect the RSF's overall performance to a greater extent than the Cox model's overall performance. This conclusion has also been made by Baralou et al. (\citeyear{Baralou2022}). In the other scenarios, the performance gap between both models tends to increase with decreasing sample size, which may be due to the RSF's better ability for reasonable predictions even in higher dimensional settings. The Cox model benefits more from larger sample sizes. For the largest considered total sample size ($N = 400$) for the data with multiple treatment-covariate interactions, the Cox model additionally outperformed the RSF in all but the increasing hazard scenario.
			In the data without treatment-covariate interactions, the ``gradient-based Brier score'' may perform better than the standard ``log-rank test'' RSF splitting rule. In the data with multiple treatment-covariate interactions, again the ``extremely randomized trees'' splitting rule performs better identical to the comparisons based on the $C$ index. It may be worthwhile to try the ``extremely randomized trees'' splitting rule in data where treatment-covariate interactions are assumed.\\
			Calibration, a measure of agreement between (estimated) true and predicted outcomes, seems to be worse for the RSF compared to the Cox model in many cases. Calibration of RSF predictions varies to a higher extent, and the difference between both models is in some cases very evident. It is unclear whether the results could be influenced by the approach of estimating the calibration curves itself which is based on Cox model predictions to approximate the true outcome.\\    
			The results are affected by the size of the treatment effect only to a minor degree. \\
			Limitations of our simulation study are that only a limited number of datasets and scenarios, as well as a limited number of performance measures can be considered. Moreover, we only considered the combination of Weibull distributed survival times and uniformly distributed censoring times. There also exist further RSF splitting rules \citep{Ishwaran2008} that are not currently implemented in the \texttt{R} packages \texttt{randomForestSRC} \citep{Ishwaran2021a} and \texttt{ranger} \citep{Wright2023}, which we did not include in the method comparison.\\
			We found that overall performance measures such as the IBS may be more suitable for drawing general conclusions about the superiority of one method over the other for predictions in RCT data based on some publications in the methodological literature 
			as well as the simulation study results which include many outliers suggesting a poor performance of the Cox model when considering the $C$ index, a conclusion that is less obvious or even reversed when considering the IBS. Regarding different RSF splitting rules, the ``log-rank test'' performed reasonably well. In some cases other splitting rules perform better based on the median result, most remarkably the ``extremely randomized trees'' splitting rule in the presence of treatment-covariate interactions. Computational times of some RSF splitting rules such as the standard ``log-rank test'' or the ``extremely randomized trees'' splitting rule do not extremely exceed those of the Cox model.
			
		
			
			\section{Conclusions}
			
			The RSF may help predicting patient outcomes more accurately. Different aspects of performance may be interesting depending on the research question.\\
			In terms of discrimination, the RSF will most likely outperform the Cox regression model in many settings. This is in accordance with method comparisons in different real-world datasets \citep{Guo2023, Sarica2023, Chowdhury2023, Moncada2021, Farhadian2021, Miao2015, Spooner2020,Qiu2020,  Kim2019, Datema2012,  Omurlu2009, Du2020}. The performance in our reference datasets suggests that the RSF tends to perform better but the confidence intervals are still overlapping, not allowing a definitive conclusion. Only the RSF based on the ``extremely randomized trees'' splitting rule can be said to definitively perform better than Cox regression in the data without treatment-covariate interactions.  In our simulations, the RSF based on ``log-rank test'' splitting usually performs best for the data without interactions, only in the data with 60\% censoring, Cox regression performs better in the non-proportional hazards settings. In their simulations, Baralou et al. (\citeyear{Baralou2022}) also found that performance of the RSF may suffer with higher censoring rates.  The RSF based on ``extremely randomized trees'' splitting always performs best in data with treatment-covariate interactions for the lower and higher censoring proportion. \\
			With respect to overall performance, simulation results of both models are more comparable, although performance in the reference datasets still suggests that the RSF tends to perform better but without clear evidence due to overlapping confidence intervals. In the simulations based on data without treatment-covariate interactions, the Cox model performs best in nonproportional hazards settings, and may perform better in case of increasing hazards for higher sample sizes and higher censoring rates, respectively. Otherwise, especially for smaller sample sizes, the RSF based on ``log-rank test'' splitting performs better. The RSF based on ``Gradient-based Brier score'' splitting may have a slight advantage compared to the standard splitting rule for decreasing and constant hazards. For simulated data with treatment-covariate interactions, the Cox model outperforms the RSF especially for the higher censoring rate in the nonproportional hazards settings or higher sample sizes ($N = 400$), respectively.
			Otherwise, the RSF based on the ``extremely randomized trees'' splitting rule performs best.\\
			Calibration of the Cox model seems to be visibly better in the nonproportional hazards settings. 
			

			
			
			
			
			
			
			
			\vfill
			
		%	\clearpage
			
			\noindent \textbf{Acknowledgments}\\
			We would like to thank Prof. Dr. Sarah Friedrich, Chair for Mathematical Statistics and Artificial Intelligence in Medicine, Institute for Mathematics,
			University of Augsburg, Germany, for her support.\\
			The authors gratefully acknowledge the resources on the LiCCA HPC cluster of the University of Augsburg, co-funded by the Deutsche Forschungsgemeinschaft (DFG, German Research Foundation) \textminus Project-ID 499211671.\\
			
%			\noindent \textbf{Author contributions: CRediT}\\
%			RG: Conceptualisation, Software, Formal analysis, Writing - Original Draft, Visualisation\\
%			ST: Conceptualisation, Supervision, Writing - Review \& Editing\\
%			MFB: Conceptualisation, Supervision, Writing - Review \& Editing\\
			
			\noindent \textbf{Funding sources}\\
			The authors are grateful for financial support of the Young Researchers Travel Scholarship Program of the University of Augsburg, and for the financial support of The International Dimension of ERASMUS+ during Ricarda Graf's research visit to the University of Reading. The sponsors had no role in study design, collection, analysis and interpretation of data, writing of the report and decision to submit the article for publication.\\
			
			\noindent \textbf{Data availability statement}\\
			The two datasets used as references for data simulations are publicly available: the RCT in primary biliary cirrhosis patients is available from a number of sources, for example from the Vanderbilt Department of Biostatistics \citep{Vanderbilt}, from the book by Fleming and Harrington \citep{Fleming2005}, from kaggle \citep{kaggle}, and from the website of the University of Massachusetts \citep{UMASS}, and the RCT in prostate cancer patients is available  in the \texttt{R} package \texttt{subtee}  \citep{Ballarini2021}. The \texttt{R} code for reproducing the results of the simulation study is available on Figshare (\url{}).
			

\clearpage

\section*{Supplementary Material A: Parameters used for data simulation}\label{params}

In this section, details regarding data simulations from copula models are given, i.e. correlation matrices and parametric distributions. Comparisons of the true covariate distributions and the fitted distributions are also shown.


\begin{table}[htb!]
	\renewcommand\thetable{A.1a}
	\caption[Correlation matrix of the variables in the primary biliary cirrhosis dataset used in the copula model.]{\small Correlation matrix of the variables in the primary biliary cirrhosis dataset used in the copula model.}
	\label{sigma_mayo}
	\setlength{\tabcolsep}{1.5pt}
	\renewcommand{\arraystretch}{3}
	
	
	\begin{adjustwidth}{-1cm}{}
		{\fontsize{7pt}{7pt}\selectfont   
			
			\begin{tabular*}{1.1\textwidth}{@{}p{1.5cm}cccccccccccccccccccp{1.25cm}@{}} 
				\arrayrulecolor{black}\cmidrule(lr){1-20}
				&  \rot{\parbox{3cm}{Age}} & \rot{\parbox{3cm}{Sex }} & \rot{\parbox{3cm}{Presence of ascites }} & \rot{\parbox{3.5cm}{Presence of hepatomegaly}} & \rot{\parbox{3cm}{Presence of spiders }} & \rot{\parbox{3cm}{Presence of edema }} & \rot{\parbox{3cm}{Serum bilirubin }} & \rot{\parbox{3cm}{Serum cholesterol }} & \rot{\parbox{3cm}{Albumin }} & \rot{\parbox{3cm}{Urine copper }} & \rot{\parbox{3cm}{Alkaline phosphatase }} & \rot{\parbox{3cm}{SGOT}} & \rot{\parbox{3cm}{Triglycerides }} & \rot{\parbox{3cm}{Platelet count}} & \rot{\parbox{3cm}{Prothrombin time }} &  \rot{\parbox{3.5cm}{\linespread{1}\selectfont Histologic stage\\of disease}}  &&&&\\ \arrayrulecolor{gray!50}\cmidrule(lr){1-20} 
				\parbox{4cm}{Age}	 			         & 1 &-0.1931 & 0.216 & 0.0862 & -0.0529 &	0.1966 &	0.0386 & -0.1493 & -0.1953 & 0.0613 &	-0.0472 &	-0.1499 &	0.0209 &	-0.1384 &	0.1963 &	0.1297 &&&&\\
				
				\parbox{4cm}{Sex } & -0.1931 & 1 & 0.0087 &	0.0509 &	0.14 &	0.0371 &	-0.0385 &	0.0053 &	-0.0653 &	-0.2301 &	0.0043 &	-0.0184 &	-0.0874 &	0.0855 &	-0.1211 &	-0.0268 &&&& \\
				
				\parbox{4cm}{\linespread{1}\selectfont Presence of \\ascites}& 0.216 &	0.0087 &	1 &	0.159 &	0.1593 &	0.4257 &	0.3756 &	-0.1071	 &-0.3574 &	0.218 &	0.0326 &	0.0887 &	0.1348 &	-0.2184 &	0.3214 &	0.3683 & &&& \\
				
				\parbox{4cm}{\linespread{1}\selectfont Presence of \\hepato-\\megaly } & 0.0862 &	0.0509 &	0.159 &	1 &	0.283 &	0.1714 &	0.4109 &	0.0957 &	-0.3083 &	0.283 &	0.1686 &	0.1621 &	0.1564 &	-0.2062 &	0.2144 &	0.4953 &&&&
				\\
				\parbox{4cm}{\linespread{1}\selectfont Presence\\of spiders } & -0.0529 &	0.14 &	0.1593 &	0.283 &	1 &	0.2427 &	0.3673 &	-0.006 &	-0.2348 &	0.2704 &	0.0948 &	0.1565 &	0.0091 &	-0.1588 &	0.2566 &	0.3298 &&&& \\
				
				\parbox{4cm}{\linespread{1}\selectfont Presence\\of edema} & 0.1966 &	0.0371 &	0.4257 &	0.1714 &	0.2427 &	1 &	0.3863 &	-0.1294 &	-0.3237 &	0.2124 &	0.0389 &	0.106 &	0.0652 &	-0.2126 &	0.3403 &	0.2814 &&&&
				\\
				
				\parbox{4cm}{\linespread{1}\selectfont Serum \\bilirubin }	         &0.0386 &	-0.0385 &	0.3756 &	0.4109 &	0.3673 &	0.3863 &	1 &	0.3806 &	-0.3346 &	0.4565 &	0.117 &	0.4417 &	0.4185 &	-0.0868 &	0.3617 &	0.3049 &&&&
				\\
				\parbox{4cm}{\linespread{1}\selectfont Serum \\cholesterol } & -0.1493	&	0.0053	&	-0.1071	&	0.0957	&	-0.006	&	-0.1294	&	0.3806	&	1	&	-0.064	&	0.123	&	0.1401	&	0.3372 &	0.2768	&	0.178	&	-0.0294	&	-0.004 &&&&
				\\
				
				\parbox{4cm}{Albumin }	                 & -0.1953	&	-0.0653	&	-0.3574	&	-0.3083	&	-0.2348	&	-0.3237	&	-0.3346	&	-0.064	&	1	&	-0.2643	&	-0.1015	&	-0.22	&	-0.0947	&	0.203	&	-0.234	&	-0.2684 &&&&
				\\ 
				\parbox{4cm}{\linespread{1}\selectfont Urine \\copper }	& 0.0613	&	-0.2301	&	0.218	&	0.283	&	0.2704	&	0.2124	&	0.4565	&	0.123	&	-0.2643	&	1	&	0.1873	&	0.2936	&	0.2723	&	-0.0641	&	0.2179	&	0.23 &&&&
				\\
				
				\parbox{4cm}{\linespread{1}\selectfont Alkaline \\phospha-\\tase }	 & -0.0472	&	0.0043	&	0.0326	&	0.1686	&	0.0948	&	0.0389	&	0.117	&	0.1401	&	-0.1015	&	0.1873	&	1	&	0.1122	&	0.1687	&	0.1428	&	0.0894	&	0.0992 &&&&
				\\ 
				\parbox{4cm}{SGOT } & -0.1499	&	-0.0184	&	0.0887	&	0.1621	&	0.1565	&	0.106	&	0.4417	&	0.3372	&	-0.22	&	0.2936	&	0.1122	&	1	&	0.1194	&	-0.1195	&	0.1122	&	0.1455 	&&&&
				\\
				\parbox{4cm}{Triglycerides } & 0.0209 &	-0.0874 &	0.1348 &	0.1564 &	0.0091 &	0.0652 &	0.4185 &	0.2768 &	-0.0947 &	0.2723 &	0.1687 &	0.1194 &	1 &	0.0948 &	0.0192 &	0.0749  &&&&
				\\
				
				\parbox{4cm}{\linespread{1}\selectfont Platelet \\count }	  & -0.1384 &	0.0855 &	-0.2184 &	-0.2062 &	-0.1588 &	-0.2126 &	-0.0868 &	0.178 &	0.203 &	-0.0641 &	0.1428 &	-0.1195 &	0.0948 &	1 &	-0.2213 &	-0.1984 &&&&
				\\ 
				\parbox{4cm}{\linespread{1}\selectfont Prothrombin \\time }	     &0.1963 &	-0.1211 &	0.3214 &	0.2144 &	0.2566 &	0.3403 &	0.3617 &	-0.0294 &	-0.234 &	0.2179 &	0.0894 &	0.1122 &	0.0192 &	-0.2213 &	1 &	0.2572 & & & & \\
				\parbox{4cm}{\linespread{1}\selectfont Histologic \\stage of \\disease}& 0.1297 &	-0.0268 &	0.3683 &	0.4953 &	0.3298 &	0.2814 &	0.3049 &	-0.004 &	-0.2684 &	0.23 &	0.0992 &	0.1455 &	0.0749 &	-0.1984 &	0.2572 &	1 \\
				
				
				\arrayrulecolor{black}\cmidrule(lr){1-20}     
			\end{tabular*}
			
		}
	\end{adjustwidth}
\end{table}

\begin{table}[htb!]
	\renewcommand\thetable{A.1b}
	\caption[Correlation matrix of the variables in the RCT in prostate cancer patients used in the copula model.]{\small Correlation matrix of the variables in the RCT in prostate cancer patients used in the copula model.}
	\label{sigma_byar}
	\setlength{\tabcolsep}{2.5pt}
	\renewcommand{\arraystretch}{3}
	\begin{adjustwidth}{-0cm}{}
		{\fontsize{7pt}{7pt}\selectfont  
			
			\begin{tabular*}{1.1\textwidth}{@{}p{2.1cm}ccccccccccccccccp{1.25cm}@{}} 
				\arrayrulecolor{black}\cmidrule(lr){1-16}	%\toprule
				&  \rot{\parbox{2cm}{Age}} & \rot{\parbox{2cm}{\linespread{1}\selectfont Standardized\\weight}} & \rot{\parbox{2cm}{\linespread{1}\selectfont Systolic\\blood pressure}} & \rot{\parbox{2cm}{\linespread{1}\selectfont Diastolic\\blood pressure}} & \rot{\parbox{2cm}{\linespread{1}\selectfont Size of\\primary tumour}} & \rot{\parbox{2.2cm}{\linespread{1}\selectfont Serum prostatic\\acid phosphatase}} & \rot{\parbox{2cm}{Haemoglobin}} & \rot{\parbox{2cm}{\linespread{1}\selectfont Gleason\\stage-grade}} & \rot{\parbox{2cm}{\linespread{1}\selectfont Performance\\status}} & \rot{\parbox{2.2cm}{\linespread{1}\selectfont History of cardio-\\vascular disease}} & \rot{\parbox{2cm}{\linespread{1}\selectfont Presence of\\bone metastases}} & \rot{\parbox{2cm}{\linespread{1}\selectfont Prostate\\cancer stage}} & \rot{\parbox{2cm}{\linespread{1}\selectfont Abnormal\\electrocardiogram}}   &&&&\\ \arrayrulecolor{gray!50}\cmidrule(lr){1-16} 
				\parbox{3.5cm}{Age}	 			         & 1 &-0.0761 &  0.0863 & -0.0727 & 0.0137 & -0.0628 & -0.0842 & -0.0518 & 0.0529 & 0.1699 & -0.0717 & -0.0125     & 0.1658 &&&&\\
				
				\parbox{3.5cm}{\linespread{1}\selectfont Standardized\\weight} & -0.0761 & 1 &  0.1911 & 0.2265 & -0.0440 & -0.0654 & 0.2612 & -0.0864 & -0.0940	& 0.0558 & -0.1827  & -0.0792       & 0.0148 &&&& \\
				
				\parbox{3.5cm}{\linespread{1}\selectfont Systolic\\blood pressure}&  0.0863 & 0.1911 &	1 &	0.6289 & 0.0451 & -0.0532 &  0.0608 & -0.0283	 & 0.0538 & 0.1231 & -0.0424  & -0.0067   & 0.1080  & &&& \\
				
				\parbox{3.5cm}{\linespread{1}\selectfont Diastolic\\blood pressure} &  -0.0727  & 0.2265 &  0.6289 &	1 &	-0.0474 & -0.0605 & 0.1437 & -0.0731 & -0.0225 &  0.0362 & -0.0825  & -0.0227  & 0.0775 &&&&
				\\
				\parbox{3.5cm}{\linespread{1}\selectfont Size of\\primary tumour} &  0.0137 & -0.0440 & 0.0451 & -0.0474  &	1 &	0.0895 & -0.1375 &  0.3748   &	0.0704  & -0.0876 & 0.2538 & 0.2190   & 0.0183 &&&& \\
				
				\parbox{3.5cm}{\linespread{1}\selectfont Serum prostatic\\acid phosphatase} &  -0.0628 & -0.0654 & -0.0532 & -0.0605 & 0.0895 &	1 &	-0.1355 & 0.1531 & 0.0999 & -0.0335 & 0.3061     & 0.6594 & -0.0122  &&&&
				\\		
				\parbox{3.5cm}{Haemoglobin}	         & -0.0842 & 0.2612 & 0.0608 & 0.1437 & -0.1375 & -0.1355 &	1 &	-0.1396 & -0.1409 & 0.0108 & -0.3004  & -0.1241 & -0.0245  &&&&
				\\
				\parbox{3.5cm}{\linespread{1}\selectfont Gleason\\stage-grade} & -0.0518 & -0.0864 & -0.0283 & -0.0731 &  0.3748 &  0.1531 & -0.1396	&	1	& 0.1277 & -0.1474 & 0.3873 & 0.6509 &	-0.0203 &&&&
				\\
				
				\parbox{3.5cm}{\linespread{1}\selectfont Performance\\status}	                 & 0.0529 & -0.0940 & 0.0538 & -0.0225 & 0.0704 & 0.0999 & -0.1409  & 0.1277 & 1 & -0.0653 & -0.1919  & 0.127 &  -0.0565	&&&&
				\\ 
				\parbox{3.5cm}{\linespread{1}\selectfont History of\\cardiovascular\\disease}	& 0.1699 & 0.0558 & 0.1231 &  0.0362 &   -0.0876 & -0.0335 & 0.0108 & -0.1474 & -0.0653	&	1	& 0.0576 & 0.0946 & 0.1849  &&&&
				\\
				
				\parbox{3.5cm}{\linespread{1}\selectfont Presence of\\bone metastases}	 & -0.0717 & -0.1827 & -0.0424 & -0.0825 & 0.2538 &  0.3061 & -0.3004 & 0.3873 & -0.1919 & 0.0576	&	1	& -0.6679 & 0.0745	 &&&&
				\\ 
				\parbox{3.5cm}{\linespread{1}\selectfont Prostate\\cancer stage} & -0.0125 & -0.0792 & -0.0067 & -0.0227 & 0.2190 & 0.6594 & -0.1241 & 0.6509 &  0.127 & 0.0946 & -0.6679 &	1	&	 -0.0303 		&&&&
				\\
				\parbox{3.5cm}{\linespread{1}\selectfont Abnormal\\electrocardiogram} &  0.1658 	&  0.0148	& 0.108 &  0.0775 &  0.0183	& -0.0122 & -0.0245	& -0.0203 &  -0.0565 & 0.1849  & 0.0745 & -0.0303 &	1   &&&&
				\\
				
				
				
				\arrayrulecolor{black}\cmidrule(lr){1-16}     
			\end{tabular*}
			
		}
	\end{adjustwidth}
\end{table}
\clearpage

\begin{table}[htb]
	\renewcommand\thetable{A.2a}
	\caption[Univariate distributions from which covariate values are simulated (primary biliary cirrhosis dataset).]{\small Univariate distributions from which covariate values are simulated (primary biliary cirrhosis dataset).}
	\label{dist_mayo}
	\def\arraystretch{1.5}
	\captionsetup{width=\textwidth}
	\fontsize{7pt}{7pt}\selectfont
	\begin{tabular}{lll}  \toprule 
		Variable & Description      & Distribution \\ \midrule %(lr){1-3}
		$Z_2$    & Age \textit{[Years]}               &  $Z_2 \sim \mathcal{TN}( \mu = 50.02, \sigma =  10.58, \text{a} = 26.28,  \text{b} = 78.44)$           \\
		$Z_3$    & Sex \textit{[0: male, 1: female]}            &  $Z_3 \sim Ber(p = 0.88) $ \\
		
		$Z_4$    & Presence of ascites \textit{[0: no, 1: yes]}    &   $Z_4 \sim Ber(p = 0.08)$         \\
		
		$Z_5$    & Presence of hepatomelagy \textit{[0: no, 1: yes]} &  $Z_5 \sim Ber(p = 0.51)$           \\
		$Z_6$    & Presence of spiders \textit{[0: no, 1: yes]}               &    $Z_6 \sim Ber(p = 0.29)$      \\
		$Z_7$    & Presence of edema \textit{$^{1)}$}     & $Z_7 \sim \mathcal{TN}( \mu = 0.11, \sigma =  0.27, \text{a} = 0,  \text{b} = 1)$ \\
		$Z_8$    &    Serum bilirubin \textit{[mg/dl]}            &  $Z_8 \sim \mathcal{TLN}( \log(\mu) = 0.58, \log(\sigma) =  1.03, \text{a} = 0.3,  \text{b} = 28)$          \\
		$Z_9$    & Serum cholesterol  \textit{[mg/dl]} &  $Z_9 \sim \mathcal{TLN}( \log(\mu) = 5.81, \log(\sigma) = 0.42, \text{a} = 120,  \text{b} = 1775)$ \\
		$Z_{10}$ & Albumin \textit{[gm/dl]} & $Z_{10} \sim \mathcal{TN}( \mu = 3.52, \sigma =  0.42, \text{a} = 1.96,  \text{b} = 4.64)$ \\
		$Z_{11}$ & Urine copper \textit{[mg/day]} & $Z_{11} \sim \mathcal{TEXP}( \lambda = 0.01,  \text{b} = 488)$\\
		$Z_{12}$ & Alkaline phosphatase \textit{[U/liter]} & 	$Z_{12} \sim \mathcal{TLN}( \log(\mu) = 7.27, \log(\sigma) =  0.72, \text{a} = 289,  \text{b} = 13862.4)$ \\ 
		$Z_{13}$ & Aspartate aminotransferase  - SGOT \textit{[U/ml]}& $Z_{13} \sim \mathcal{TLN}( \log(\mu) = 4.71, \log(\sigma) =  0.45, \text{a} = 26.35,  \text{b} = 457.25)$ \\
		$Z_{14}$ & Triglycerides \textit{[mg/dl]}& $Z_{14} \sim \mathcal{TLN}( \log(\mu) = 4.73, \log(\sigma) =  0.43, \text{a} = 33,  \text{b} = 598)$ \\
		$Z_{15}$ & Platelet count \textit{[\# platelets per m$^3$/1000]}& $Z_{15} \sim \mathcal{TN}( \mu = 261.94, \sigma =  95.0, \text{a} = 62,  \text{b} = 563)$\\
		$Z_{16}$ & Prothrombin time \textit{[sec]}& $Z_{16} \sim \mathcal{TLN}( \log(\mu) = 2.37, \log(\sigma) =  0.09, \text{a} = 9,  \text{b} = 17.1)$\\
		$Z_{17}$ & Histologic stage of disease \textit{[grade]}& $Z_{17} \sim \mathcal{TN}( \mu = 3.03, \sigma =  0.88, \text{a} = 1,  \text{b} = 4)$ \\				
		\bottomrule   
	\end{tabular}
\end{table}

\begin{table}[htb]
	\renewcommand\thetable{A.2b}
	\caption[Univariate distributions from which  covariate values are simulated (prostate cancer dataset).]{\small Univariate distributions from which  covariate values are simulated (RCT in prostate cancer patients). }
	\label{dist_byar}
	\def\arraystretch{1.5}
	\captionsetup{width=\textwidth}
	\fontsize{7pt}{7pt}\selectfont				 
	\begin{tabular}{lll}  \toprule 
		Variable &  Description    & Distribution \\ \midrule
		AGE    & Age \textit{[Years]}               &  AGE $\sim \mathcal{TN}( \mu = 71.56, \sigma =  6.9, \text{a} = 48,  \text{b} = 89)$           \\
		WT    & Standardized weight \textit{}            & WT $\sim  \mathcal{TLN}( \log(\mu) = 4.58, \log(\sigma) =  0.13, \text{a} = 69,  \text{b} = 152)$ \\
		SBP    & Systolic blood pressure \textit{}    & SBP  $\sim  \mathcal{TLN}( \log(\mu) = 2.65, \log(\sigma) =  0.16, \text{a} = 8,  \text{b} = 30)$         \\						
		DBP    & Diastolic blood pressure \textit{} &  DBP $\sim  \mathcal{TLN}( \log(\mu) = 2.08, \log(\sigma) =  0.18, \text{a} = 4,  \text{b} = 18)$           \\				
		SZ    & Size of primary tumour \textit{[centimeters squared]}               &  SZ  $\sim \mathcal{TEXP}( \lambda = 0.07,  \text{b} = 69)$      \\		
		AP    & Serum (prostatic) acid phosphatase \textit{[King-Armstrong units]}     & AP $\sim  \mathcal{TLN}( \log(\mu) = 2.64, \log(\sigma) =  1.63, \text{a} = 1,  \text{b} = 9999)$ \\						
		HG    & Haemoglobin \textit{[g/100ml]}            &  HG $\sim \mathcal{TN}( \mu = 134.2, \sigma =  19.36, \text{a} = 59,  \text{b} = 182)$          \\
		SG    & Gleason stage-grade category  \textit{} &  SG $\sim \mathcal{TN}( \mu = 10.3, \sigma =  2.02, \text{a} = 5,  \text{b} = 15)$ \\						
		PF & Performance status \textit{} & PF $\sim \mathcal{B}(p = 0.1)$ \\
		HX & History of cariovascular disease \textit{[0: no, 1: yes]} & HX  $\sim \mathcal{B}(p = 0.43)$\\
		BM & Presence of bone metastases \textit{[0: no, 1: yes]} & BM	$\sim \mathcal{B}(p = 0.16)$ \\ 
		EKG & Abnormal electrocardiogram \textit{[0: normal, 1: abnormal]}& EKG  $\sim \mathcal{B}(p = 0.66)$ \\
		
		\bottomrule   
	\end{tabular}		
\end{table}

\vspace{10cm}

\newgeometry{top=25mm, bottom=20mm}
\begin{figure}[H]
	\begin{minipage}{0.49\textwidth} 	
		\raggedleft
		\begin{subfigure}{\textwidth}
			\captionsetup{labelformat=empty}			
			\raggedright
			\hspace{-1cm}
			\includegraphics[scale = 0.57,trim={0cm, 0, 0cm, 0},clip]{figures/FigA1a_qq2_age.pdf}   
		\end{subfigure}
	\end{minipage}
	\begin{minipage}{0.49\textwidth} 	
		\raggedleft
		\hspace{1cm}
		\includegraphics[scale = 0.57]{figures/FigA1a_cdf2_age.pdf}
	\end{minipage}
	\renewcommand\thefigure{}
	\renewcommand{\figurename}{}
	\caption[]{\hspace{-0.3cm}\textbf{Fig. A.1a} Distribution of age compared to the truncated normal distribution  with $\mu = 50.02$, and $\sigma = 10.58$.}
\end{figure}

\vspace{-1.45cm}
\begin{figure}[H]		
	\begin{minipage}{0.49\textwidth} 	
		\raggedleft
		\begin{subfigure}{\textwidth}
			\captionsetup{labelformat=empty}			
			\raggedright
			\hspace{-1cm}
			\includegraphics[scale = 0.57,trim={0cm, 0, 0cm, 0},clip]{figures/FigA1b_qq8_serum_bilirubin.pdf}  
		\end{subfigure}
	\end{minipage}
	\begin{minipage}{0.49\textwidth} 	
		\raggedleft
		\hspace{1cm}
		\includegraphics[scale = 0.57]{figures/FigA1b_cdf8_serum_bilirubin.pdf}
	\end{minipage}
	\renewcommand\thefigure{}
	\renewcommand{\figurename}{}
	\caption[]{\hspace{-0.3cm}\textbf{Fig. A.1b} Distribution of serum bilirubin compared to the truncated log-normal distribution  with $\log(\mu) = 0.58$, and $\log(\sigma) = 1.03$.}
\end{figure}

\vspace{-1.45cm}
\begin{figure}[H]		
	\begin{minipage}{0.49\textwidth} 	
		\raggedleft
		\begin{subfigure}{\textwidth}
			\captionsetup{labelformat=empty}			
			\raggedright
			\hspace{-1cm}
			\includegraphics[scale = 0.57,trim={0cm, 0, 0cm, 0},clip]{figures/FigA1c_qq9_serum_cholesterol.pdf}  
		\end{subfigure}
	\end{minipage}
	\begin{minipage}{0.49\textwidth} 	
		\raggedleft
		\hspace{1cm}
		\includegraphics[scale = 0.57]{figures/FigA1c_cdf9_serum_cholesterol.pdf}
	\end{minipage}
	\renewcommand\thefigure{}
	\renewcommand{\figurename}{}
	\caption[]{\hspace{-0.3cm}\textbf{Fig. A.1c} Distribution of serum cholesterol compared to the truncated log-normal distribution  with $\log(\mu) = 5.81$, and $\log(\sigma) = 0.42$.}
\end{figure}
\newpage
\begin{figure}[htb]		
	\begin{minipage}{0.49\textwidth} 	
		\raggedleft
		\begin{subfigure}{\textwidth}
			\captionsetup{labelformat=empty}			
			\raggedright
			\hspace{-1cm}
			\includegraphics[scale = 0.57,trim={0cm, 0, 0cm, 0},clip]{figures/FigA1d_qq10_albumin.pdf}   % 0.375
		\end{subfigure}
	\end{minipage}
	\begin{minipage}{0.49\textwidth} 	
		\raggedleft
		\hspace{1cm}
		\includegraphics[scale = 0.57]{figures/FigA1d_cdf10_albumin.pdf}
	\end{minipage}
	\renewcommand\thefigure{}
	\renewcommand{\figurename}{}
	\caption[]{\hspace{-0.3cm}\textbf{Fig. A.1d} Distribution of triglycerides compared to the truncated normal distribution  with $\mu = 3.52$, and $\sigma = 0.42$.}
\end{figure}
%\vspace{10cm}
%\vspace{-1.7cm}
\vspace{-1.45cm}
\begin{figure}[H]		
	\begin{minipage}{0.49\textwidth} 	
		\raggedleft
		\begin{subfigure}{\textwidth}
			\captionsetup{labelformat=empty}			
			\raggedright
			\hspace{-1cm}
			\includegraphics[scale = 0.57,trim={0cm, 0, 0cm, 0},clip]{figures/FigA1e_qq11_urine_copper.pdf}   % 0.375
		\end{subfigure}
	\end{minipage}
	\begin{minipage}{0.49\textwidth} 	
		\raggedleft
		\hspace{-1cm}
		\includegraphics[scale = 0.57]{figures/FigA1e_cdf11_urine_copper.pdf}
	\end{minipage}
	\renewcommand\thefigure{}
	\renewcommand{\figurename}{}
	\caption[]{\hspace{-0.3cm}\textbf{Fig. A.1e} Distribution of urine copper compared to the truncated exponential distribution  with $\lambda = 0.01$.}
\end{figure}
\vspace{-1.45cm}
\begin{figure}[H]		
	\begin{minipage}{0.49\textwidth} 	
		\raggedleft
		\begin{subfigure}{\textwidth}
			\captionsetup{labelformat=empty}			
			\raggedright
			\hspace{-1cm}
			\includegraphics[scale = 0.57,trim={0cm, 0, 0cm, 0},clip]{figures/FigA1f_qq12_alkaline_phosphatase.pdf}   % 0.375
		\end{subfigure}
	\end{minipage}
	\begin{minipage}{0.49\textwidth} 	
		\raggedleft
		\hspace{1cm}
		\includegraphics[scale = 0.58]{figures/FigA1f_cdf12_alkaline_phosphatase.pdf}
	\end{minipage}
	\renewcommand\thefigure{}
	\renewcommand{\figurename}{}
	\caption[]{\hspace{-0.3cm}\textbf{Fig. A.1f} Distribution of alkaline phosphatase compared to the truncated log-normal distribution  with $\log(\mu) = 7.27$, and $\log(\sigma) = 0.72$.}
\end{figure}
\vspace{-1.45cm}
\begin{figure}[H]		
	\begin{minipage}{0.49\textwidth} 	
		\raggedleft
		\begin{subfigure}{\textwidth}
			\captionsetup{labelformat=empty}			
			\raggedright
			\hspace{-1cm}
			\includegraphics[scale = 0.58,trim={0cm, 0, 0cm, 0},clip]{figures/FigA1g_qq13_SGOT.pdf}   % 0.375
		\end{subfigure}
	\end{minipage}
	\begin{minipage}{0.49\textwidth} 	
		\raggedleft
		\hspace{1cm}
		\includegraphics[scale = 0.58]{figures/FigA1g_cdf13_SGOT.pdf}
	\end{minipage}
	\renewcommand\thefigure{}
	\renewcommand{\figurename}{}
	\caption[]{\hspace{-0.3cm}\textbf{Fig. A.1g} Distribution of SGOT compared to the truncated log-normal distribution  with $\log(\mu) = 4.71$, and $\log(\sigma) = 0.45$.}
\end{figure}
\vspace{-1.45cm}
\begin{figure}[H]		
	\begin{minipage}{0.49\textwidth} 	
		\raggedleft
		\begin{subfigure}{\textwidth}
			\captionsetup{labelformat=empty}			
			\raggedright
			\hspace{-1cm}
			\includegraphics[scale = 0.58,trim={0cm, 0, 0cm, 0},clip]{figures/FigA1h_qq14_triglycerides.pdf}   % 0.375
		\end{subfigure}
	\end{minipage}
	\begin{minipage}{0.49\textwidth} 	
		\raggedleft
		\hspace{1cm}
		\includegraphics[scale = 0.58]{figures/FigA1h_cdf14_triglycerides.pdf}
	\end{minipage}
	\renewcommand\thefigure{}
	\renewcommand{\figurename}{}
	\caption[]{\hspace{-0.3cm}\textbf{Fig. A.1h} Distribution of triglycerides compared to the truncated log-normal distribution  with $\log(\mu) = 4.73$, and $\log(\sigma) = 0.43$.}
\end{figure}
\vspace{-1.45cm}
\begin{figure}[H]		
	\begin{minipage}{0.49\textwidth} 	
		\raggedleft
		\begin{subfigure}{\textwidth}
			\captionsetup{labelformat=empty}			
			\raggedright
			\hspace{-1cm}
			\includegraphics[scale = 0.58,trim={0cm, 0, 0cm, 0},clip]{figures/FigA1i_qq15_platelet_count.pdf}   % 0.375
		\end{subfigure}
	\end{minipage}
	\begin{minipage}{0.49\textwidth} 	
		\raggedleft
		\hspace{1cm}
		\includegraphics[scale = 0.58]{figures/FigA1i_cdf15_platelet_count.pdf}
	\end{minipage}
	\renewcommand\thefigure{}
	\renewcommand{\figurename}{}
	\caption[]{\hspace{-0.3cm}\textbf{Fig. A.1i}: Distribution of platelet count compared to the truncated normal distribution  with $\mu = 261.94$, and $\sigma = 95.0$.}
\end{figure}	
\vspace{-1.45cm}
\begin{figure}[H]
	\captionsetup[figure]{labelsep=space}		% , font={stretch=0.8}
	\begin{minipage}{0.49\textwidth} 	
		\raggedleft
		\begin{subfigure}{\textwidth}
			\captionsetup{labelformat=empty}			
			\raggedright
			\hspace{-1cm}
			\includegraphics[scale = 0.58,trim={0cm, 0, 0cm, 0},clip]{figures/FigA1j_qq16_prothrombin_time.pdf}   % 0.375
		\end{subfigure}
	\end{minipage}
	\begin{minipage}{0.49\textwidth} 	
		\raggedleft
		\hspace{1cm}
		\includegraphics[scale = 0.58]{figures/FigA1j_cdf16_prothrombin_time.pdf}
	\end{minipage}
	\renewcommand\thefigure{}
	\renewcommand{\figurename}{}
	\caption[]{\hspace{-0.3cm}\textbf{Fig. A.1j}: Distribution of prothrombin time compared to the truncated log-normal distribution  with $\log(\mu) = 2.37$, and $\log(\sigma) = 0.09$.}
	\renewcommand\thefigure{A.1}
	\caption[QQ-plot (left) and plot comparing the theoretical and empirical cumulative distribution functions (right) based on maximum likelihood estimates of the respective distributional parameters for each continuous variable in the primary biliary cirrhosis dataset (no treatment-covariate interactions).]{\textbf{Comparison of the true covariate distribution with  the distribution used in the copula model (biliary cirrhosis dataset).}    QQ-plot (left) and plot comparing the theoretical and empirical cumulative distribution functions (right) based on maximum likelihood estimates of the respective distributional parameters for each continuous variable in the primary biliary cirrhosis dataset. Truncated distributions were fitted with lower and upper limits corresponding to the minimum and maximum value of the variable.}
	\label{qq_cdf_mayo}
\end{figure}

\restoregeometry



% Byar
\newgeometry{top=25mm, bottom=20mm}
\begin{figure}[H]		
	\begin{minipage}{0.49\textwidth} 	
		\raggedleft
		\begin{subfigure}{\textwidth}
			\captionsetup{labelformat=empty}			
			\raggedright
			\hspace{-1cm}
			\includegraphics[scale = 0.57,trim={0cm, 0, 0cm, 0},clip]{figures/FigA2a_qq_age.pdf}   % 0.375
		\end{subfigure}
	\end{minipage}
	\begin{minipage}{0.49\textwidth} 	
		\raggedleft
		\hspace{1cm}
		\includegraphics[scale = 0.57]{figures/FigA2a_cdf_age.pdf}
	\end{minipage}
	\renewcommand\thefigure{}
	\renewcommand{\figurename}{}
	\caption[]{\hspace{-0.3cm}\textbf{Fig. A.2a} Distribution of age compared to the truncated normal distribution  with $\mu = 71.56$, and $\sigma = 6.91$.}
\end{figure}
\vspace{-1.45cm}
\begin{figure}[H]		
	\begin{minipage}{0.49\textwidth} 	
		\raggedleft
		\begin{subfigure}{\textwidth}
			\captionsetup{labelformat=empty}			
			\raggedright
			\hspace{-1cm}
			\includegraphics[scale = 0.57,trim={0cm, 0, 0cm, 0},clip]{figures/FigA2b_qq_weight.pdf}   % 0.375
		\end{subfigure}
	\end{minipage}
	\begin{minipage}{0.49\textwidth} 	
		\raggedleft
		\hspace{1cm}
		\includegraphics[scale = 0.57]{figures/FigA2b_cdf_weight.pdf}
	\end{minipage}
	\renewcommand\thefigure{}
	\renewcommand{\figurename}{}
	\caption[]{\hspace{-0.3cm}\textbf{Fig. A.2b} Distribution of standardized weight compared to the truncated log-normal distribution  with $\log(\mu) = 4.59$, and $\log(\sigma) = 0.13$.}
\end{figure}
\vspace{-1.45cm}
\begin{figure}[H]		
	\begin{minipage}{0.49\textwidth} 	
		\raggedleft
		\begin{subfigure}{\textwidth}
			\captionsetup{labelformat=empty}			
			\raggedright
			\hspace{-1cm}
			\includegraphics[scale = 0.57,trim={0cm, 0, 0cm, 0},clip]{figures/FigA2c_qq_sbp.pdf}   % 0.375
		\end{subfigure}
	\end{minipage}
	\begin{minipage}{0.49\textwidth} 	
		\raggedleft
		\hspace{1cm}
		\includegraphics[scale = 0.57]{figures/FigA2c_cdf_sbp.pdf}
	\end{minipage}
	\renewcommand\thefigure{}
	\renewcommand{\figurename}{}
	\caption[]{\hspace{-0.3cm}\textbf{Fig. A.2c} Distribution of systolic blood pressure compared to the truncated log-normal distribution  with $\log(\mu) = 2.65$, and $\log(\sigma) = 0.16$.}
\end{figure}
\vspace{-1.45cm}
%\clearpage
\enlargethispage{0.5cm}
\begin{figure}[H]		
	\begin{minipage}{0.49\textwidth} 	
		\raggedleft
		\begin{subfigure}{\textwidth}
			\captionsetup{labelformat=empty}			
			\raggedright
			\hspace{-1cm}
			\includegraphics[scale = 0.57,trim={0cm, 0, 0cm, 0},clip]{figures/FigA2d_qq_dbp.pdf}   % 0.375
		\end{subfigure}
	\end{minipage}
	\begin{minipage}{0.49\textwidth} 	
		\raggedleft
		\hspace{1cm}
		\includegraphics[scale = 0.57]{figures/FigA2d_cdf_dbp.pdf}
	\end{minipage}
	\renewcommand\thefigure{}
	\renewcommand{\figurename}{}
	\caption[]{\hspace{-0.3cm}\textbf{Fig. A.2d} Distribution of diastolic blood pressure compared to the truncated log-normal distribution  with $\log(\mu) = 2.08$, and $\log(\sigma) = 0.18$.}
\end{figure}
\vspace{-1.45cm}
\begin{figure}[H]		
	\begin{minipage}{0.49\textwidth} 	
		\raggedleft
		\begin{subfigure}{\textwidth}
			\captionsetup{labelformat=empty}			
			\raggedright
			\hspace{-1cm}
			\includegraphics[scale = 0.57,trim={0cm, 0, 0cm, 0},clip]{figures/FigA2e_qq_sz.pdf}   % 0.375
		\end{subfigure}
	\end{minipage}
	\begin{minipage}{0.49\textwidth} 	
		\raggedleft
		\hspace{1cm}
		\includegraphics[scale = 0.57]{figures/FigA2e_cdf_sz.pdf}
	\end{minipage}
	\renewcommand\thefigure{}
	\renewcommand{\figurename}{}
	\caption[]{\hspace{-0.3cm}\textbf{Fig. A.2e} Distribution of size of primary tumour compared to the truncated exponential distribution  with $\lambda = 0.07$.}
\end{figure}
\vspace{-1.45cm}
\enlargethispage{0.5cm}
\begin{figure}[H]		
	\begin{minipage}{0.49\textwidth} 	
		\raggedleft
		\begin{subfigure}{\textwidth}
			\captionsetup{labelformat=empty}			
			\raggedright
			\hspace{-1cm}
			\includegraphics[scale = 0.57,trim={0cm, 0, 0cm, 0},clip]{figures/FigA2f_qq_ap.pdf}   % 0.375
		\end{subfigure}
	\end{minipage}
	\begin{minipage}{0.49\textwidth} 	
		\raggedleft
		\hspace{1cm}
		\includegraphics[scale = 0.57]{figures/FigA2f_cdf_ap.pdf}
	\end{minipage}
	\renewcommand\thefigure{}
	\renewcommand{\figurename}{}
	\caption[]{\hspace{-0.3cm}\textbf{Fig. A.2f} Distribution of serum (prostatic) acid phosphatase compared to the truncated log-normal distribution  with $\log(\mu) = 2.64$, and $\log(\sigma) = 1.63$.}
\end{figure}
\vspace{-1.45cm}
\begin{figure}[H]		
	\begin{minipage}{0.49\textwidth} 	
		\raggedleft
		\begin{subfigure}{\textwidth}
			\captionsetup{labelformat=empty}			
			\raggedright
			\hspace{-1cm}
			\includegraphics[scale = 0.57,trim={0cm, 0, 0cm, 0},clip]{figures/FigA2g_qq_hg.pdf}   % 0.375
		\end{subfigure}
	\end{minipage}
	\begin{minipage}{0.49\textwidth} 	
		\raggedleft
		\hspace{1cm}
		\includegraphics[scale = 0.57]{figures/FigA2g_cdf_hg.pdf}
	\end{minipage}
	\renewcommand\thefigure{}
	\renewcommand{\figurename}{}
	\caption[]{\hspace{-0.3cm}\textbf{Fig. A.2g} Distribution of haemoglobin compared to the truncated normal distribution  with $\mu = 134.2$, and $\sigma = 19.36$.}
\end{figure}
\vspace{-1.45cm}
\begin{figure}[H]		
	\begin{minipage}{0.49\textwidth} 	
		\raggedleft
		\begin{subfigure}{\textwidth}
			\captionsetup{labelformat=empty}			
			\raggedright
			\hspace{-1cm}
			\includegraphics[scale = 0.57,trim={0cm, 0, 0cm, 0},clip]{figures/FigA2h_qq_sg.pdf}   % 0.375
		\end{subfigure}
	\end{minipage}
	\begin{minipage}{0.49\textwidth} 	
		\raggedleft
		\hspace{1cm}
		\includegraphics[scale = 0.57]{figures/FigA2h_cdf_sg.pdf}
	\end{minipage}
	\renewcommand\thefigure{}
	\renewcommand{\figurename}{}
	\caption[]{\hspace{-0.3cm}\textbf{Fig. A.2h} Distribution of Gleason stage-grade category compared to the truncated normal distribution  with $\mu = 10.3$, and $\sigma = 2.01$.}
	\renewcommand\thefigure{A.2}
	\caption[QQ-plot (left) and plot comparing the theoretical and empirical cumulative distribution functions (right) based on maximum likelihood estimates of the respective distributional parameters for each continuous variable in the prostate cancer dataset (three treatment-covariate interactions).]{\textbf{Comparison of the true covariate distribution with  the distribution used in the copula model (prostate cancer dataset).} QQ-plot (left) and plot comparing the theoretical and empirical cumulative distribution functions (right) based on maximum likelihood estimates of the respective distributional parameters for each continuous variable in the prostate cancer dataset. Truncated distributions were fitted with lower and upper limits corresponding to the minimum and maximum value of the variable.}
	\label{qq_cdf_byar}
\end{figure}
\restoregeometry


\section*{Supplementary Material B: Simulation study results}

In this section, full simulation study results of the $C$ index (B.1), Integrated Brier score (B.2), and calibration curves (B.3) are shown, i.e. boxplots of the estimates' distribution and tables of means and standard errors for the 500 simulated datasets per scenario.


\subsection*{B.1: $C$ index}\label{S_cindex}

\begin{figure}[H]
	\renewcommand\thefigure{B.1a}
	\raggedright
	\begin{minipage}{0.98\textwidth} 	
		\begin{subfigure}{\textwidth}
			\renewcommand\thefigure{}
			\renewcommand{\figurename}{}
			\includegraphics[scale = 0.2, width = 0.995\textwidth,trim={0cm, 0, 0cm, 0},clip]{figures/Mayo_03_c_index_boxplot_beta_0_nsim_500_1_500.pdf}   
			\vspace{-0.5cm}
			\caption[]{\textbf{}}
		\end{subfigure}		
		\begin{subfigure}{\textwidth}
			\renewcommand\thefigure{}
			\renewcommand{\figurename}{}
			\includegraphics[scale = 0.2, width = 0.995\textwidth,trim={0cm, 0, 0cm, 0},clip]{figures/Mayo_06_c_index_boxplot_beta_0_nsim_500_1_500.pdf}   
			\vspace{-0.5cm}
			\caption[]{\textbf{}}
		\end{subfigure}		
	\end{minipage}		
	\begin{minipage}{0.005\textwidth} 	
		%\raggedleft
		\raggedright
		%\hfill
		\vspace{-2cm}
		\includegraphics[scale = 0.325,trim={0cm, 0, 0cm, 0},clip]{figures/legend_boxplot.pdf}
	\end{minipage}	
	\vspace{-0.3cm}
	\caption[Distribution of the $C$ index estimates for the scenario without treatment-covariate interactions (primary biliary cirrhosis dataset), $\beta_{\text{treatment}} = 0$, scale parameter $\lambda$ = 2241.74, censoring rate 30\% (a), and 60\% (b).]{\small{\textbf{Distribution of \textit{C} index estimates for the Cox and RSF model when applied to the simulated data (primary biliary cirrhosis dataset, $\beta_{\text{\normalfont{treatment}}} = 0$).} Boxplots showing the distribution of the \underline{\textit{C} index} estimates obtained from 500 simulated datasets based on data \underline{without treatment-covariate interactions (primary biliary cirrhosis dataset)} for the regression coefficient of the treatment effect \underline{$\beta_{\text{treatment}} = 0$}. The scale parameter of the Weibull distributed survival times ($\lambda$ = 2241.74) is chosen to be constant, shape parameters ($\gamma$) vary. Results are shown for different total sample sizes $N$, and censoring rates (a: 30\%, b: 60\%).\\
			\label{boxplots_cindex_mayo_0}	
		}
		{  \footnotesize  Abbreviations:  Cox-PH - Cox proportional hazards model, RSF - Random survival forest.}
	}			
\end{figure}



\begin{figure}[H]
	\renewcommand\thefigure{B.1b}
	\raggedright
	\begin{minipage}{0.98\textwidth} 	
		\begin{subfigure}{\textwidth}
			\renewcommand\thefigure{}
			\renewcommand{\figurename}{}
			\includegraphics[scale = 0.2, width = 0.995\textwidth,trim={0cm, 0, 0cm, 0},clip]{figures/Mayo_03_c_index_boxplot_beta_08_nsim_500_1_500.pdf}   
			\vspace{-0.5cm}
			\caption[]{\textbf{}}
		\end{subfigure}		
		\begin{subfigure}{\textwidth}
			\renewcommand\thefigure{}
			\renewcommand{\figurename}{}
			\includegraphics[scale = 0.2, width = 0.995\textwidth,trim={0cm, 0, 0cm, 0},clip]{figures/Mayo_06_c_index_boxplot_beta_08_nsim_500_1_500.pdf}   
			\vspace{-0.5cm}
			\caption[]{\textbf{}}
		\end{subfigure}		
	\end{minipage}		
	\begin{minipage}{0.005\textwidth} 	
		%\raggedleft
		\raggedright
		%\hfill
		\vspace{-2cm}
		\includegraphics[scale = 0.325,trim={0cm, 0, 0cm, 0},clip]{figures/legend_boxplot.pdf}
	\end{minipage}	
	\vspace{-0.3cm}
	\caption[Distribution of the $C$ index estimates for the scenario without treatment-covariate interactions (primary biliary cirrhosis dataset), $\beta_{\text{treatment}} = 0.8$, scale parameter $\lambda$ = 2241.74, censoring rate 30\% (a), and 60\% (b).]{\small{\textbf{Distribution of \textit{C} index estimates for the Cox and RSF model when applied to the simulated data (primary biliary cirrhosis dataset, $\beta_{\text{\normalfont{treatment}}} = 0.8$).} Boxplots showing the distribution of the \underline{\textit{C} index} estimates obtained from 500 simulated datasets based on data \underline{without treatment-covariate interactions (primary biliary cirrhosis dataset)} for the regression coefficient of the treatment effect \underline{$\beta_{\text{treatment}} = 0.8$}. The scale parameter of the Weibull distributed survival times ($\lambda$ = 2241.74) is chosen to be constant, shape parameters ($\gamma$) vary. Results are shown for different total sample sizes $N$, and censoring rates (a: 30\%, b: 60\%).\\
			\label{boxplots_cindex_mayo_08}	
		}
		{  \footnotesize  Abbreviations:  Cox-PH - Cox proportional hazards model, RSF - Random survival forest.}
	}			
\end{figure}


\begin{figure}[H]
	\renewcommand\thefigure{B.1c}
	\raggedright
	\begin{minipage}{0.98\textwidth} 	
		\begin{subfigure}{\textwidth}
			\renewcommand\thefigure{}
			\renewcommand{\figurename}{}
			\includegraphics[scale = 0.2, width = 0.995\textwidth,trim={0cm, 0, 0cm, 0},clip]{figures/Mayo_03_c_index_boxplot_beta_04_nsim_500_1_500.pdf}   
			\vspace{-0.5cm}
			\caption[]{\textbf{}}
		\end{subfigure}		
		\begin{subfigure}{\textwidth}
			\renewcommand\thefigure{}
			\renewcommand{\figurename}{}
			\includegraphics[scale = 0.2, width = 0.995\textwidth,trim={0cm, 0, 0cm, 0},clip]{figures/Mayo_06_c_index_boxplot_beta_04_nsim_500_1_500.pdf}   
			\vspace{-0.5cm}
			\caption[]{\textbf{}}
		\end{subfigure}		
	\end{minipage}		
	\begin{minipage}{0.005\textwidth} 	
		%\raggedleft
		\raggedright
		%\hfill
		\vspace{-2cm}
		\includegraphics[scale = 0.325,trim={0cm, 0, 0cm, 0},clip]{figures/legend_boxplot.pdf}
	\end{minipage}	
	\vspace{-0.3cm}
	\caption[Distribution of the $C$ index estimates for the scenario without treatment-covariate interactions (primary biliary cirrhosis dataset), $\beta_{\text{treatment}} = -0.4$, scale parameter $\lambda$ = 2241.74, censoring rate 30\% (a), and 60\% (b).]{\small{\textbf{Distribution of \textit{C} index estimates for the Cox and RSF model when applied to the simulated data (primary biliary cirrhosis dataset, $\beta_{\text{\normalfont{treatment}}} = -0.4$).} Boxplots showing the distribution of the \underline{\textit{C} index} estimates obtained from 500 simulated datasets based on data \underline{without treatment-covariate interactions (primary biliary cirrhosis dataset)} for the regression coefficient of the treatment effect \underline{$\beta_{\text{treatment}} = -0.4$}. The scale parameter of the Weibull distributed survival times ($\lambda$ = 2241.74) is chosen to be constant, shape parameters ($\gamma$) vary. Results are shown for different total sample sizes $N$, and censoring rates (a: 30\%, b: 60\%).\\
			\label{boxplots_cindex_mayo_04}	
		}
		{  \footnotesize  Abbreviations:  Cox-PH - Cox proportional hazards model, RSF - Random survival forest.}
	}			
\end{figure}



\begin{figure}[H]
	\renewcommand\thefigure{B.2a}
	\raggedright
	\begin{minipage}{0.98\textwidth} 	
		\begin{subfigure}{\textwidth}
			\renewcommand\thefigure{}
			\renewcommand{\figurename}{}
			\includegraphics[scale = 0.2, width = 0.995\textwidth,trim={0cm, 0, 0cm, 0},clip]{figures/Byar_03_c_index_boxplot_beta_0_nsim_500_1_500.pdf}   
			\vspace{-0.5cm}
			\caption[]{\textbf{}}
		\end{subfigure}		
		\begin{subfigure}{\textwidth}
			\renewcommand\thefigure{}
			\renewcommand{\figurename}{}
			\includegraphics[scale = 0.2, width = 0.995\textwidth,trim={0cm, 0, 0cm, 0},clip]{figures/Byar_06_c_index_boxplot_beta_0_nsim_500_1_500.pdf}   
			\vspace{-0.5cm}
			\caption[]{\textbf{}}
		\end{subfigure}		
	\end{minipage}		
	\begin{minipage}{0.005\textwidth} 	
		%\raggedleft
		\raggedright
		%\hfill
		\vspace{-2cm}
		\includegraphics[scale = 0.325,trim={0cm, 0, 0cm, 0},clip]{figures/legend_boxplot.pdf}
	\end{minipage}	
	\vspace{-0.3cm}
	\caption[Distribution of the $C$ index estimates for the scenario with three treatment-covariate interactions (prostate cancer dataset), $\beta_{\text{treatment}} = 0$, scale parameter $\lambda$ = 39.2, censoring rate 30\% (a), and 60\% (b).]{\small{\textbf{Distribution of \textit{C} index estimates for the Cox and RSF model when applied to the simulated data (prostate cancer dataset, $\beta_{\text{\normalfont{treatment}}} = 0$).} Boxplots showing the distribution of the \underline{\textit{C} index} estimates obtained from 500 simulated datasets based on data \underline{with three treatment-covariate interactions (prostate cancer dataset)} for the regression coefficient of the treatment effect \underline{$\beta_{\text{treatment}} = 0$}. The scale parameter of the Weibull distributed survival times ($\lambda$ = 39.2) is chosen to be constant, shape parameters ($\gamma$) vary. Results are shown for different total sample sizes $N$, and censoring rates (a: 30\%, b: 60\%).\\
			\label{boxplots_cindex_byar_0}	
		}
		{  \footnotesize  Abbreviations:  Cox-PH - Cox proportional hazards model, RSF - Random survival forest.}
	}			
\end{figure}



\begin{figure}[H]
	\renewcommand\thefigure{B.2b}
	\raggedright
	\begin{minipage}{0.98\textwidth} 	
		\begin{subfigure}{\textwidth}
			\renewcommand\thefigure{}
			\renewcommand{\figurename}{}
			\includegraphics[scale = 0.2, width = 0.995\textwidth,trim={0cm, 0, 0cm, 0},clip]{figures/Byar_03_c_index_boxplot_beta_08_nsim_500_1_500.pdf}   
			\vspace{-0.5cm}
			\caption[]{\textbf{}}
		\end{subfigure}		
		\begin{subfigure}{\textwidth}
			\renewcommand\thefigure{}
			\renewcommand{\figurename}{}
			\includegraphics[scale = 0.2, width = 0.995\textwidth,trim={0cm, 0, 0cm, 0},clip]{figures/Byar_06_c_index_boxplot_beta_08_nsim_500_1_500.pdf}   
			\vspace{-0.5cm}
			\caption[]{\textbf{}}
		\end{subfigure}		
	\end{minipage}		
	\begin{minipage}{0.005\textwidth} 	
		%\raggedleft
		\raggedright
		%\hfill
		\vspace{-2cm}
		\includegraphics[scale = 0.325,trim={0cm, 0, 0cm, 0},clip]{figures/legend_boxplot.pdf}
	\end{minipage}	
	\vspace{-0.3cm}
	\caption[Distribution of the $C$ index estimates for the scenario with three treatment-covariate interactions (prostate cancer dataset), $\beta_{\text{treatment}} = 0.8$, scale parameter $\lambda$ = 39.2, censoring rate 30\% (a), and 60\% (b).]{\small{\textbf{Distribution of \textit{C} index estimates for the Cox and RSF model when applied to the simulated data (prostate cancer dataset, $\beta_{\text{\normalfont{treatment}}} = 0.8$).} Boxplots showing the distribution of the \underline{\textit{C} index} estimates obtained from 500 simulated datasets based on data \underline{with three treatment-covariate interactions (prostate cancer dataset)} for the regression coefficient of the treatment effect \underline{$\beta_{\text{treatment}} = 0.8$}. The scale parameter of the Weibull distributed survival times ($\lambda$ = 39.2) is chosen to be constant, shape parameters ($\gamma$) vary. Results are shown for different total sample sizes $N$, and censoring rates (a: 30\%, b: 60\%).\\
			\label{boxplots_cindex_byar_08}	
		}
		{  \footnotesize  Abbreviations:  Cox-PH - Cox proportional hazards model, RSF - Random survival forest.}
	}			
\end{figure}


\begin{figure}[H]
	\renewcommand\thefigure{B.2c}
	\raggedright
	\begin{minipage}{0.98\textwidth} 	
		\begin{subfigure}{\textwidth}
			\renewcommand\thefigure{}
			\renewcommand{\figurename}{}
			\includegraphics[scale = 0.2, width = 0.995\textwidth,trim={0cm, 0, 0cm, 0},clip]{figures/Byar_03_c_index_boxplot_beta_04_nsim_500_1_500.pdf}   
			\vspace{-0.5cm}
			\caption[]{\textbf{}}
		\end{subfigure}		
		\begin{subfigure}{\textwidth}
			\renewcommand\thefigure{}
			\renewcommand{\figurename}{}
			\includegraphics[scale = 0.2, width = 0.995\textwidth,trim={0cm, 0, 0cm, 0},clip]{figures/Byar_06_c_index_boxplot_beta_04_nsim_500_1_500.pdf}   
			\vspace{-0.5cm}
			\caption[]{\textbf{}}
		\end{subfigure}		
	\end{minipage}		
	\begin{minipage}{0.005\textwidth} 	
		%\raggedleft
		\raggedright
		%\hfill
		\vspace{-2cm}
		\includegraphics[scale = 0.325,trim={0cm, 0, 0cm, 0},clip]{figures/legend_boxplot.pdf}
	\end{minipage}	
	\vspace{-0.3cm}
	\caption[Distribution of the $C$ index estimates for the scenario with three treatment-covariate interactions (prostate cancer dataset), $\beta_{\text{treatment}} = -0.4$, scale parameter $\lambda$ = 39.2, censoring rate 30\% (a), and 60\% (b).]{\small{\textbf{Distribution of \textit{C} index estimates for the Cox and RSF model when applied to the simulated data (prostate cancer dataset, $\beta_{\text{\normalfont{treatment}}} = -0.4$).} Boxplots showing the distribution of the \underline{\textit{C} index} estimates obtained from 500 simulated datasets based on data \underline{with three treatment-covariate interactions (prostate cancer dataset)} for the regression coefficient of the treatment effect \underline{$\beta_{\text{treatment}} = -0.4$}. The scale parameter of the Weibull distributed survival times ($\lambda$ = 39.2) is chosen to be constant, shape parameters ($\gamma$) vary. Results are shown for different total sample sizes $N$, and censoring rates (a: 30\%, b: 60\%).\\
			\label{boxplots_cindex_byar_04}	
		}
		{  \footnotesize  Abbreviations:  Cox-PH - Cox proportional hazards model, RSF - Random survival forest.}
	}			
\end{figure}





\subsection*{B.2: Integrated Brier score (IBS)}\label{S_ibs}

\begin{figure}[H]
	\renewcommand\thefigure{B.3a}
	\raggedright
	\begin{minipage}{0.98\textwidth} 	
		\begin{subfigure}{\textwidth}
			\renewcommand\thefigure{}
			\renewcommand{\figurename}{}
			\includegraphics[scale = 0.2, width = 0.995\textwidth,trim={0cm, 0, 0cm, 0},clip]{figures/Mayo_03_ibs_boxplot_beta_0_nsim_500_1_500.pdf}   
			\vspace{-0.5cm}
			\caption[]{\textbf{}}
		\end{subfigure}		
		\begin{subfigure}{\textwidth}
			\renewcommand\thefigure{}
			\renewcommand{\figurename}{}
			\includegraphics[scale = 0.2, width = 0.995\textwidth,trim={0cm, 0, 0cm, 0},clip]{figures/Mayo_06_ibs_boxplot_beta_0_nsim_500_1_500.pdf}   
			\vspace{-0.5cm}
			\caption[]{\textbf{}}
		\end{subfigure}		
	\end{minipage}		
	\begin{minipage}{0.005\textwidth} 	
		\raggedright
		\vspace{-2cm}
		\includegraphics[scale = 0.325,trim={0cm, 0, 0cm, 0},clip]{figures/legend_boxplot.pdf}
	\end{minipage}	
	\vspace{-0.3cm}
	\caption[Distribution of the Integrated Brier score (IBS) estimates for the scenario without treatment-covariate interactions (primary biliary cirrhosis dataset), $\beta_{\text{treatment}} = 0$, scale parameter $\lambda$ = 2241.74, censoring rate 30\% (a), and 60\% (b).]{\small{\textbf{Distribution of Integrated Brier score (IBS) estimates for the Cox and RSF model when applied to the simulated data (primary biliary cirrhosis dataset, $\beta_{\text{\normalfont{treatment}}} = 0$).} Boxplots showing the distribution of the \underline{Integrated Brier score (IBS)} estimates obtained from 500 simulated datasets based on data \underline{without treatment-covariate interactions (primary biliary cirrhosis dataset)}  for the regression coefficient of the treatment effect  \underline{$\beta_{\text{treatment}} = 0$}. The scale parameter of the Weibull distributed survival times ($\lambda$ = 2241.74) is chosen to be constant, shape parameters ($\gamma$) vary. Results are shown for different total sample sizes $N$, and censoring rates (a: 30\%, b: 60\%).\\
			\label{boxplots_ibs_mayo_0}	
		}
		{\footnotesize Abbreviations:  Cox-PH - Cox proportional hazards model, RSF - Random survival forest.}
	}			
\end{figure}



\begin{figure}[H]
	\renewcommand\thefigure{B.3b}
	\raggedright
	\begin{minipage}{0.98\textwidth} 	
		\begin{subfigure}{\textwidth}
			\renewcommand\thefigure{}
			\renewcommand{\figurename}{}
			\includegraphics[scale = 0.2, width = 0.995\textwidth,trim={0cm, 0, 0cm, 0},clip]{figures/Mayo_03_ibs_boxplot_beta_08_nsim_500_1_500.pdf}   
			\vspace{-0.5cm}
			\caption[]{\textbf{}}
		\end{subfigure}		
		\begin{subfigure}{\textwidth}
			\renewcommand\thefigure{}
			\renewcommand{\figurename}{}
			\includegraphics[scale = 0.2, width = 0.995\textwidth,trim={0cm, 0, 0cm, 0},clip]{figures/Mayo_06_ibs_boxplot_beta_08_nsim_500_1_500.pdf}   
			\vspace{-0.5cm}
			\caption[]{\textbf{}}
		\end{subfigure}		
	\end{minipage}		
	\begin{minipage}{0.005\textwidth} 	
		%\raggedleft
		\raggedright
		%\hfill
		\vspace{-2cm}
		\includegraphics[scale = 0.325,trim={0cm, 0, 0cm, 0},clip]{figures/legend_boxplot.pdf}
	\end{minipage}	
	\vspace{-0.3cm}
	\caption[Distribution of the Integrated Brier score (IBS) estimates for the scenario without treatment-covariate interactions (primary biliary cirrhosis dataset), $\beta_{\text{treatment}} = 0.8$, scale parameter $\lambda$ = 2241.74, censoring rate 30\% (a), and 60\% (b).]{\small{\textbf{Distribution of Integrated Brier score (IBS) estimates for the Cox and RSF model when applied to the simulated data (primary biliary cirrhosis dataset, $\beta_{\text{\normalfont{treatment}}} = 0.8$).} Boxplots showing the distribution of the \underline{Integrated Brier score (IBS)} estimates obtained from 500 simulated datasets based on data \underline{without treatment-covariate interactions (primary biliary cirrhosis dataset)}  for the regression coefficient of the treatment effect \underline{$\beta_{\text{treatment}} = 0.8$}. The scale parameter of the Weibull distributed survival times ($\lambda$ = 2241.74) is chosen to be constant, shape parameters ($\gamma$) vary. Results are shown for different total sample sizes $N$, and censoring rates (a: 30\%, b: 60\%).\\
			\label{boxplots_ibs_mayo_08}	
		}
		{\footnotesize Abbreviations:  Cox-PH - Cox proportional hazards model, RSF - Random survival forest.}
	}			
\end{figure}	

\begin{figure}[H]
	\renewcommand\thefigure{B.3c}
	\raggedright
	\begin{minipage}{0.98\textwidth} 	
		\begin{subfigure}{\textwidth}
			\renewcommand\thefigure{}
			\renewcommand{\figurename}{}
			\includegraphics[scale = 0.2, width = 0.995\textwidth,trim={0cm, 0, 0cm, 0},clip]{figures/Mayo_03_ibs_boxplot_beta_04_nsim_500_1_500.pdf}   
			\vspace{-0.5cm}
			\caption[]{\textbf{}}
		\end{subfigure}		
		\begin{subfigure}{\textwidth}
			\renewcommand\thefigure{}
			\renewcommand{\figurename}{}
			\includegraphics[scale = 0.2, width = 0.995\textwidth,trim={0cm, 0, 0cm, 0},clip]{figures/Mayo_06_ibs_boxplot_beta_04_nsim_500_1_500.pdf}   
			\vspace{-0.5cm}
			\caption[]{\textbf{}}
		\end{subfigure}		
	\end{minipage}		
	\begin{minipage}{0.005\textwidth} 	
		\raggedright
		\vspace{-2cm}
		\includegraphics[scale = 0.325,trim={0cm, 0, 0cm, 0},clip]{figures/legend_boxplot.pdf}
	\end{minipage}	
	\vspace{-0.3cm}
	\caption[Distribution of the Integrated Brier score (IBS) estimates for the scenario without treatment-covariate interactions (primary biliary cirrhosis dataset), $\beta_{\text{treatment}} = -0.4$, scale parameter $\lambda$ = 2241.74, censoring rate 30\% (a), and 60\% (b).]{\small{\textbf{Distribution of Integrated Brier score (IBS) estimates for the Cox and RSF model when applied to the simulated data (primary biliary cirrhosis dataset, $\beta_{\text{\normalfont{treatment}}} = -0.4$).} Boxplots showing the distribution of the \underline{Integrated Brier score (IBS)} estimates obtained from 500 simulated datasets based on data \underline{without treatment-covariate interactions (primary biliary cirrhosis dataset)}  for the regression coefficient of the treatment effect  \underline{$\beta_{\text{treatment}} = -0.4$}. The scale parameter of the Weibull distributed survival times ($\lambda$ = 2241.74) is chosen to be constant, shape parameters ($\gamma$) vary. Results are shown for different total sample sizes $N$, and censoring rates (a: 30\%, b: 60\%).\\
			\label{boxplots_ibs_mayo_04}	
		}
		{\footnotesize Abbreviations:  Cox-PH - Cox proportional hazards model, RSF - Random survival forest.}
	}			
\end{figure}




\begin{figure}[H]
	\renewcommand\thefigure{B.4a}
	\raggedright
	\begin{minipage}{0.98\textwidth} 	
		\begin{subfigure}{\textwidth}
			\renewcommand\thefigure{}
			\renewcommand{\figurename}{}
			\includegraphics[scale = 0.2, width = 0.995\textwidth,trim={0cm, 0, 0cm, 0},clip]{figures/Byar_03_ibs_boxplot_beta_0_nsim_500_1_500.pdf}   
			\vspace{-0.5cm}
			\caption[]{\textbf{}}
		\end{subfigure}		
		\begin{subfigure}{\textwidth}
			\renewcommand\thefigure{}
			\renewcommand{\figurename}{}
			\includegraphics[scale = 0.2, width = 0.995\textwidth,trim={0cm, 0, 0cm, 0},clip]{figures/Byar_06_ibs_boxplot_beta_0_nsim_500_1_500.pdf}   
			\vspace{-0.5cm}
			\caption[]{\textbf{}}
		\end{subfigure}		
	\end{minipage}		
	\begin{minipage}{0.005\textwidth} 	
		%\raggedleft
		\raggedright
		%\hfill
		\vspace{-2cm}
		\includegraphics[scale = 0.325,trim={0cm, 0, 0cm, 0},clip]{figures/legend_boxplot.pdf}
	\end{minipage}	
	\vspace{-0.3cm}
	\caption[Distribution of the Integrated Brier score (IBS) estimates for the scenario with three treatment-covariate interactions (prostate cancer dataset), $\beta_{\text{treatment}} = 0$, scale parameter $\lambda$ = 39.2, censoring rate 30\% (a), and 60\% (b).]{\small{\textbf{Distribution of Integrated Brier score (IBS) estimates for the Cox and RSF model when applied to the simulated data (prostate cancer dataset, $\beta_{\text{\normalfont{treatment}}} = 0$).} Boxplots showing the distribution of the \underline{Integrated Brier score (IBS)} estimates obtained from 500 simulated datasets based on data \underline{with three treatment-covariate interactions (prostate cancer dataset)}  for the regression coefficient of the treatment effect  \underline{$\beta_{\text{treatment}} = 0$}. The scale parameter of the Weibull distributed survival times ($\lambda$ = 39.2) is chosen to be constant, shape parameters ($\gamma$) vary. Results are shown for different total sample sizes $N$, and censoring rates (a: 30\%, b: 60\%).\\
			\label{boxplots_ibs_byar_0}	
		}
		{\footnotesize Abbreviations:  Cox-PH - Cox proportional hazards model, RSF - Random survival forest.}
	}			
\end{figure}






\begin{figure}[H]
	\renewcommand\thefigure{B.4b}
	\raggedright
	\begin{minipage}{0.98\textwidth} 	
		\begin{subfigure}{\textwidth}
			\renewcommand\thefigure{}
			\renewcommand{\figurename}{}
			\includegraphics[scale = 0.2, width = 0.995\textwidth,trim={0cm, 0, 0cm, 0},clip]{figures/Byar_03_ibs_boxplot_beta_08_nsim_500_1_500.pdf}   
			\vspace{-0.5cm}
			\caption[]{\textbf{}}
		\end{subfigure}		
		\begin{subfigure}{\textwidth}
			\renewcommand\thefigure{}
			\renewcommand{\figurename}{}
			\includegraphics[scale = 0.2, width = 0.995\textwidth,trim={0cm, 0, 0cm, 0},clip]{figures/Byar_06_ibs_boxplot_beta_08_nsim_500_1_500.pdf}   
			\vspace{-0.5cm}
			\caption[]{\textbf{}}
		\end{subfigure}		
	\end{minipage}		
	\begin{minipage}{0.005\textwidth} 	
		%\raggedleft
		\raggedright
		%\hfill
		\vspace{-2cm}
		\includegraphics[scale = 0.325,trim={0cm, 0, 0cm, 0},clip]{figures/legend_boxplot.pdf}
	\end{minipage}	
	\vspace{-0.3cm}
	\caption[Distribution of the Integrated Brier score (IBS) estimates for the scenario with three treatment-covariate interactions (prostate cancer dataset), $\beta_{\text{treatment}} = 0.8$, scale parameter $\lambda$ = 39.2, censoring rate 30\% (a), and 60\% (b).]{\small{\textbf{Distribution of Integrated Brier score (IBS) estimates for the Cox and RSF model when applied to the simulated data (prostate cancer dataset, $\beta_{\text{\normalfont{treatment}}} = 0.8$).} Boxplots showing the distribution of the \underline{Integrated Brier score (IBS)} estimates obtained from 500 simulated datasets based on data \underline{with three treatment-covariate interactions (prostate cancer dataset)}  for the regression coefficient of the treatment effect \underline{$\beta_{\text{treatment}} = 0.8$}. The scale parameter of the Weibull distributed survival times ($\lambda$ = 39.2) is chosen to be constant, shape parameters ($\gamma$) vary. Results are shown for different total sample sizes $N$, and censoring rates (a: 30\%, b: 60\%).\\
			\label{boxplots_ibs_byar_08}	
		}
		{\footnotesize Abbreviations:  Cox-PH - Cox proportional hazards model, RSF - Random survival forest.}
	}			
\end{figure}	


\begin{figure}[H]
	\renewcommand\thefigure{B.4c}
	\raggedright
	\begin{minipage}{0.98\textwidth} 	
		\begin{subfigure}{\textwidth}
			\renewcommand\thefigure{}
			\renewcommand{\figurename}{}
			\includegraphics[scale = 0.2, width = 0.995\textwidth,trim={0cm, 0, 0cm, 0},clip]{figures/Byar_03_ibs_boxplot_beta_04_nsim_500_1_500.pdf}   
			\vspace{-0.5cm}
			\caption[]{\textbf{}}
		\end{subfigure}		
		\begin{subfigure}{\textwidth}
			\renewcommand\thefigure{}
			\renewcommand{\figurename}{}
			\includegraphics[scale = 0.2, width = 0.995\textwidth,trim={0cm, 0, 0cm, 0},clip]{figures/Byar_06_ibs_boxplot_beta_04_nsim_500_1_500.pdf}   
			\vspace{-0.5cm}
			\caption[]{\textbf{}}
		\end{subfigure}		
	\end{minipage}		
	\begin{minipage}{0.005\textwidth} 	
		%\raggedleft
		\raggedright
		%\hfill
		\vspace{-2cm}
		\includegraphics[scale = 0.325,trim={0cm, 0, 0cm, 0},clip]{figures/legend_boxplot.pdf}
	\end{minipage}	
	\vspace{-0.3cm}
	\caption[Distribution of the Integrated Brier score (IBS) estimates for the scenario with three treatment-covariate interactions (prostate cancer dataset), $\beta_{\text{treatment}} = -0.4$, scale parameter $\lambda$ = 39.2, censoring rate 30\% (a), and 60\% (b).]{\small{\textbf{Distribution of Integrated Brier score (IBS) estimates for the Cox and RSF model when applied to the simulated data (prostate cancer dataset, $\beta_{\text{\normalfont{treatment}}} = -0.4$).} Boxplots showing the distribution of the \underline{Integrated Brier score (IBS)} estimates obtained from 500 simulated datasets based on data \underline{with three treatment-covariate interactions (prostate cancer dataset)}  for the regression coefficient of the treatment effect  \underline{$\beta_{\text{treatment}} = -0.4$}. The scale parameter of the Weibull distributed survival times ($\lambda$ = 39.2) is chosen to be constant, shape parameters ($\gamma$) vary. Results are shown for different total sample sizes $N$, and censoring rates (a: 30\%, b: 60\%).\\
			\label{boxplots_ibs_byar_04}	
		}
		{\footnotesize  Abbreviations:  Cox-PH - Cox proportional hazards model, RSF - Random survival forest.}
	}			
\end{figure}



\clearpage


\subsection*{B.3: Calibration curves}\label{S_calibration}


\begin{figure}[H]
	\renewcommand\thefigure{B.5a}
	\begin{minipage}{.47\textwidth}
		\includegraphics[scale = 0.42,trim={0 0 0cm 0},clip]{figures/Mayo_03_calibration_curves_100__nsim_500_beta_treat_04_scale_shape_1_1_500.pdf}
		\begin{center}
%			\vspace{-0.3cm}
			(a)
%			\vspace{-0.4cm}
		\end{center}
		\includegraphics[scale = 0.42,trim={0 0 0cm 0},clip]{figures/Mayo_06_calibration_curves_100__nsim_500_beta_treat_04_scale_shape_1_1_500.pdf}
		\begin{center}
%			\vspace{-0.3cm}
			(c)
%			\vspace{-0.4cm}
		\end{center}
	\end{minipage}
	\begin{minipage}{.035\textwidth}
		%\hspace{0.05cm}
	\end{minipage}			
	\begin{minipage}{.47\textwidth}
		\includegraphics[scale = 0.42,trim={0.7cm 0 0 0},clip]{figures/Mayo_03_calibration_curves_400__nsim_500_beta_treat_04_scale_shape_1_1_500.pdf}
		\begin{center}
%			\vspace{-0.3cm}
			(b)
%			\vspace{-0.4cm}
		\end{center}
		\includegraphics[scale = 0.42,trim={0.7cm 0 0 0},clip]{figures/Mayo_06_calibration_curves_400__nsim_500_beta_treat_04_scale_shape_1_1_500.pdf}			
		\begin{center}
%			\vspace{-0.3cm}
			(d)
%			\vspace{-0.4cm}
		\end{center}				
	\end{minipage}
	\begin{minipage}{.015\textwidth}
		\hspace{-0.1cm}
		\includegraphics[scale = 0.35, trim = 0 0cm 0cm 0cm]{figures/legend_calibration_curves.pdf} 
	\end{minipage}
	\vspace{0.1cm}
	\captionof{figure}[Calibration curves at the median survival time for the data without treatment-covariate interactions (primary biliary cirrhosis dataset) for a proportional hazard setting.]{\linespread{1}\selectfont \small{\textbf{Calibration curves for a proportional hazards scenario (primary biliary cirrhosis dataset).}  \underline{Calibration curves at the median (50\% quantile)  survival time} for a \underline{proportional hazards} setting (Weibull survival time distribution W($\lambda = 2241.74, \gamma = 1$) and $n_{\text{sim}} = 500$ simulated datasets  based on data  \underline{without treatment-covariate interactions (primary biliary cirrhosis dataset)}. The solid line represents the mean calibration curve, the outer dotted lines represent the 2.5th and 97.5th percentile of the calibration curve. The black diagonal line corresponds to perfect calibration.\\
			(a) 30\% censoring, $N = 100$, (b) 30\% censoring, $N = 400$, \\(c) 60\% censoring, $N = 100$, (d) 60\% censoring, $N = 400$.\\  }  			
		{  \footnotesize  Abbreviations: Cox-PH - Cox proportional hazards model, RSF - Random survival forest.}		
	}
	
	\label{Mayo_calib_treat_04_ph}			
\end{figure}


\begin{figure}[H]
	\renewcommand\thefigure{B.5b}
	\begin{minipage}{.47\textwidth}
		\includegraphics[scale = 0.42,trim={0 0 0cm 0},clip]{figures/Mayo_03_calibration_curves_100__nsim_500_beta_treat_04_scale_shape_2_5_1_500.pdf}
		\begin{center}
%			\vspace{-0.3cm}
			(a)
%			\vspace{-0.4cm}
		\end{center}
		\includegraphics[scale = 0.42,trim={0 0 0cm 0},clip]{figures/Mayo_06_calibration_curves_100__nsim_500_beta_treat_04_scale_shape_2_5_1_500.pdf}
		\begin{center}
%			\vspace{-0.3cm}
			(c)
%			\vspace{-0.4cm}
		\end{center}
	\end{minipage}
	\begin{minipage}{.035\textwidth}
		%\hspace{0.05cm}
	\end{minipage}			
	\begin{minipage}{.47\textwidth}
		\includegraphics[scale = 0.42,trim={0.7cm 0 0 0},clip]{figures/Mayo_03_calibration_curves_400__nsim_500_beta_treat_04_scale_shape_2_5_1_500.pdf}
		\begin{center}
%			\vspace{-0.3cm}
			(b)
%			\vspace{-0.4cm}
		\end{center}
		\includegraphics[scale = 0.42,trim={0.7cm 0 0 0},clip]{figures/Mayo_06_calibration_curves_400__nsim_500_beta_treat_04_scale_shape_2_5_1_500.pdf}			
		\begin{center}
%			\vspace{-0.3cm}
			(d)
%			\vspace{-0.4cm}
		\end{center}				
	\end{minipage}
	\begin{minipage}{.015\textwidth}
		\hspace{-0.1cm}
		\includegraphics[scale = 0.35, trim = 0 0cm 10cm 0cm]{figures/legend_calibration_curves.pdf} 
	\end{minipage}
	\vspace{0.1cm}
	\captionof{figure}[Calibration curves at the median survival time for the data without treatment-covariate interactions (primary biliary cirrhosis dataset) for a nonproportional hazard setting.]{\linespread{1}\selectfont \small{\textbf{Calibration curves for a nonproportional hazards setting (primary biliary cirrhosis dataset).}  \underline{Calibration curves at the median (50\% quantile)  survival time} for a \underline{nonproportional hazards} setting (Weibull survival time distribution W($ \lambda = 2241.74, \gamma \in \{2,5\}$)) and $n_{\text{sim}} = 500$ simulated datasets  based on data  \underline{without treatment-covariate interactions (primary biliary cirrhosis dataset)}. The solid line represents the mean calibration curve, the outer dotted lines represent the 2.5th and 97.5th percentile of the calibration curve. The black diagonal line corresponds to perfect calibration.\\
			(a) 30\% censoring, $N = 100$, (b) 30\% censoring, $N = 400$, \\(c) 60\% censoring, $N = 100$, (d) 60\% censoring, $N = 400$.\\   }  			
		{  \footnotesize  Abbreviations: Cox-PH - Cox proportional hazards model, RSF - Random survival forest.}		
	}
	
	\label{Mayo_calib_treat_04_nonph}	
\end{figure}


\begin{figure}[H]
	\renewcommand\thefigure{B.6a}
	\begin{minipage}{.47\textwidth}
		\includegraphics[scale = 0.42,trim={0 0 0cm 0},clip]{figures/Byar_03_calibration_curves_100__nsim_500_beta_treat_04_scale_shape_1_1_500.pdf}
		\begin{center}
%			\vspace{-0.3cm}
			(a)
%			\vspace{-0.4cm}
		\end{center}
		\includegraphics[scale = 0.42,trim={0 0 0cm 0},clip]{figures/Byar_06_calibration_curves_100__nsim_500_beta_treat_04_scale_shape_1_1_500.pdf}
		\begin{center}
%			\vspace{-0.3cm}
			(c)
%			\vspace{-0.4cm}
		\end{center}
	\end{minipage}
	\begin{minipage}{.035\textwidth}
		%\hspace{0.05cm}
	\end{minipage}			
	\begin{minipage}{.47\textwidth}
		\includegraphics[scale = 0.42,trim={0.7cm 0 0 0},clip]{figures/Byar_03_calibration_curves_400__nsim_500_beta_treat_04_scale_shape_1_1_500.pdf}
		\begin{center}
%			\vspace{-0.3cm}
			(b)
%			\vspace{-0.4cm}
		\end{center}
		\includegraphics[scale = 0.42,trim={0.7cm 0 0 0},clip]{figures/Byar_06_calibration_curves_400__nsim_500_beta_treat_04_scale_shape_1_1_500.pdf}			
		\begin{center}
%			\vspace{-0.3cm}
			(d)
%			\vspace{-0.4cm}
		\end{center}				
	\end{minipage}
	\begin{minipage}{.015\textwidth}
		\hspace{-0.1cm}
		\includegraphics[scale = 0.35, trim = 0 0cm 10cm 0cm]{figures/legend_calibration_curves.pdf} 
	\end{minipage}
	\vspace{0.1cm}
	\captionof{figure}[Calibration curves at the median survival time for the data with three treatment-covariate interactions (prostate cancer dataset) for a proportional hazard setting.]{\linespread{1}\selectfont \small{\textbf{Calibration curves for a proportional hazards setting (prostate cancer dataset).}  \underline{Calibration curves at the median (50\% quantile)  survival time} for a \underline{proportional hazards} setting (Weibull survival time distribution W($\lambda = 2241.74, \gamma = 1$)) and $n_{\text{sim}} = 500$ simulated datasets  based on data  \underline{with three treatment-covariate interactions (prostate cancer dataset)}. The solid line represents the mean calibration curve, the outer dotted lines represent the 2.5th and 97.5th percentile of the calibration curve. The black diagonal line corresponds to perfect calibration.\\
			(a) 30\% censoring, $N = 100$, (b) 30\% censoring, $N = 400$, \\(c) 60\% censoring, $N = 100$, (d) 60\% censoring, $N = 400$.\\ }  			
		{  \footnotesize  Abbreviations: Cox-PH - Cox proportional hazards model, RSF - Random survival forest.}		
	}
	
	\label{Byar_calib_treat_04_ph}
\end{figure}



\begin{figure}[H]
	\renewcommand\thefigure{B.6b}
	\begin{minipage}{.47\textwidth}
		\includegraphics[scale = 0.42,trim={0 0 0cm 0},clip]{figures/Byar_03_calibration_curves_100__nsim_500_beta_treat_04_scale_shape_2_5_1_500.pdf}
		\begin{center}
%			\vspace{-0.3cm}
			(a)
%			\vspace{-0.4cm}
		\end{center}
		\includegraphics[scale = 0.42,trim={0 0 0cm 0},clip]{figures/Byar_06_calibration_curves_100__nsim_500_beta_treat_04_scale_shape_2_5_1_500.pdf}
		\begin{center}
%			\vspace{-0.3cm}
			(c)
%			\vspace{-0.4cm}
		\end{center}
	\end{minipage}
	\begin{minipage}{.035\textwidth}
		%\hspace{0.05cm}
	\end{minipage}			
	\begin{minipage}{.47\textwidth}
		\includegraphics[scale = 0.42,trim={0.7cm 0 0 0},clip]{figures/Byar_03_calibration_curves_400__nsim_500_beta_treat_04_scale_shape_2_5_1_500.pdf}
		\begin{center}
%			\vspace{-0.3cm}
			(b)
%			\vspace{-0.4cm}
		\end{center}
		\includegraphics[scale = 0.42,trim={0.7cm 0 0 0},clip]{figures/Byar_06_calibration_curves_400__nsim_500_beta_treat_04_scale_shape_2_5_1_500.pdf}			
		\begin{center}
%			\vspace{-0.3cm}
			(d)
%			\vspace{-0.4cm}
		\end{center}				
	\end{minipage}
	\begin{minipage}{.015\textwidth}
		\hspace{-0.1cm}
		\includegraphics[scale = 0.35, trim = 0 0cm 10cm 0cm]{figures/legend_calibration_curves.pdf} 
	\end{minipage}
	\vspace{0.1cm}
	\captionof{figure}[Calibration curves at the median survival time for the data with three treatment-covariate interactions (prostate cancer dataset) for a nonproportional hazard setting.]{\linespread{1}\selectfont \small{\textbf{Calibration curves for a nonproportional hazards setting (prostate cancer dataset).}  \underline{Calibration curves at the median (50\% quantile)  survival time} for a nonproportional hazard setting (Weibull survival time distribution W($\lambda = 39.2, \gamma \in \{2,5\}$)) and $n_{\text{sim}} = 500$ simulated datasets  based on data  \underline{with three treatment-covariate interactions (prostate cancer dataset)}. The solid line represents the mean calibration curve, the outer dotted lines represent the 2.5th and 97.5th percentile of the calibration curve. The black diagonal line corresponds to perfect calibration.\\
			(a) 30\% censoring, $N = 100$, (b) 30\% censoring, $N = 400$, \\(c) 60\% censoring, $N = 100$, (d) 60\% censoring, $N = 400$.\\ }  			
		{  \footnotesize  Abbreviations: Cox-PH - Cox proportional hazards model, RSF - Random survival forest.}		
	}
	
	\label{Byar_calib_treat_04_nonph}	
\end{figure} 



\section*{Supplementary Material C: Additional theoretical information}\label{info}

\begin{sidewaystable}
	\renewcommand\thetable{C.1}
	\renewcommand{\arraystretch}{0.6}
	\centering
	\small				
	\caption[Overview of currently available splitting rules for the RSF in the  \texttt{R} packages \texttt{randomForestSRC} \citep{Ishwaran2008} and \texttt{ranger} \citep{Wright2023} compared in the simulation study.]{\small{Overview of currently available splitting rules for the RSF in the  \texttt{R} packages \texttt{randomForestSRC} \citep{Ishwaran2008} and \texttt{ranger} \citep{Wright2023} compared in the simulation study.}  }	
	\label{splitrules}
	
	\begin{tabular}{@{}p{11em}p{17em}p{25em}p{6em}p{8em}@{}} \arrayrulecolor{black}\cmidrule[0.1pt]{1-5}
		
		Splitting rule & Notes & Recommendations/drawbacks & \linespread{1}\selectfont Abbreviation in \texttt{R} package &  Availability \\ \arrayrulecolor{black}\cmidrule[0.01pt]{1-5}		
		\parbox{11em}{\linespread{1}\selectfont Log-rank test} &      
		\parbox{17em}{\linespread{1}\selectfont {\tiny $\bullet$} standard split criterion, most widely used} &   
		\parbox{25em}{\linespread{1}\selectfont + log-rank and log-rank score splitting almost always have the lowest prediction error but poor performance in data with high censoring rates \\
			+ stable/preferable in noisy scenarios \citep{Ishwaran2008, Schmid2016} \\
			\textminus biased, i.e. favors splitting variables with many possible splits \citep{Wright2016}\\
			\textminus preference for more unbalanced splits (``end-cut preference'')\citep{Schmid2016}\\
			\textminus less suitable for small-scale clinical trials and high censoring rates \citep{Schmid2016}} & 
		\parbox{3em}{logrank} &     
		\parbox{8em}{\linespread{1}\selectfont \texttt{randomForestSRC}\\\texttt{ranger} }  \\ \arrayrulecolor{gray}\cmidrule[0.1pt]{1-5}
		
		\parbox{11em}{\linespread{1}\selectfont Log-rank score \\ \citep{Hothorn2003}} &    
		\parbox{17em}{\linespread{1}\selectfont {\tiny $\bullet$} a standardized log-rank statistic}&
		\parbox{25em}{\linespread{1}\selectfont + log-rank and log-rank score splitting almost always have the lowest prediction error but poor performance in data with high censoring rates, stable in presence of noise variables \citep{Ishwaran2008}}&
		\parbox{3em}{logrankscore} &   
		\parbox{8em}{ \texttt{randomForestSRC} }  \\ \arrayrulecolor{gray}\cmidrule[0.1pt]{1-5}
		
		\parbox{11em}{\linespread{1}\selectfont Gradient-based (global non-quantile) Brier score} &      
		\parbox{17em}{}&
		\parbox{25em}{\linespread{1}\selectfont + preferable to log-rank splitting when censoring depends strongly on the covariates (more robust) \citep{Ishwaran2019}}&
		\parbox{3em}{bs.gradient} &    
		\parbox{8em}{\texttt{randomForestSRC}}  \\ \arrayrulecolor{gray}\cmidrule[0.1pt]{1-5}
		
		\parbox{11em}{\linespread{1}\selectfont Harrell's $C$ \\
			\citep{Schmid2016}} &   
		\parbox{17em}{\linespread{1}\selectfont {\tiny $\bullet$} Harrell's $C$ ($C$ index) most common performance measure for RSF $\rightarrow$ overcomes discrepancy between node splitting and model evaluation}&
		\parbox{25em}{\linespread{1}\selectfont + recommend it for smaller scale clinical trials and if censoring is high \\
			+ has a lower end-cut preference than the log-rank statistic  \\
			\textminus recommend  log-rank test for large-scale (omics) studies \citep{Schmid2016}} &
		\parbox{3em}{C} &   
		\parbox{8em}{\texttt{ranger} }  \\ \arrayrulecolor{gray}\cmidrule[0.1pt]{1-5}
		
		\parbox{11em}{\linespread{1}\selectfont Extremely randomized trees \\
			\citep{Geurts2006}} &   
		\parbox{17em}{\linespread{1}\selectfont {\tiny $\bullet$} strong randomization of variable and cut-point choices while splitting} &
		\parbox{25em}{\linespread{1}\selectfont + main strength: computational efficiency \citep{Geurts2006}}&
		\parbox{3em}{extratrees} &    
		\parbox{8em}{\texttt{ranger} }  \\ \arrayrulecolor{gray}\cmidrule[0.1pt]{1-5}
		
		
		\parbox{11em}{\linespread{1}\selectfont Maximally selected rank statistics \\
			\citep{Wright2016}} &  
		\parbox{17em}{\linespread{1}\selectfont {\tiny $\bullet$} splitting variables are compared on the $p$-value scale}&
		\parbox{25em}{\linespread{1}\selectfont + unbiased in case of informative dichotomous and uninformative categorical variables with more possible splits}&
		\parbox{3em}{maxstat} &   
		\parbox{8em}{\texttt{ranger} }  
		
		\\ \arrayrulecolor{black}\cmidrule[0.1pt]{1-5}
	\end{tabular}
\end{sidewaystable} 




%\clearpage
\subsection*{Computation of the $C$ index for the Cox and Random survival forest model}

The $C$ index, a time range measure, can be obtained from the Cox regression and RSF models as follows. For a Cox model
\begin{equation*}
	h(t) = h_0(t)\exp(\beta_1 x_1 + \dots + \beta_{p} x_p)  
\end{equation*}
with baseline survival function $ h_0(t)$ and regression coefficients $\bm{\beta} \in \mathds{R}^p$, and unique ordered survival times $t_1,\dots,t_m$, at each uncensored survival time, the rank of the predicted outcome $\hat{h}_i(t)$ for the considered subject $i$ who experienced the event is compared to all $\hat{h}_j(t), j \neq i$ where individuals $j$ had a longer survival time. The $C$ index can thus be written as:
\begin{equation*}
	\text{Pr}(\hat{h}_i > \hat{h}_j|t_i < t_j) = \frac{\sum\limits_i (R_i - 1)}{\sum\limits_i (N_i - 1)}, \hspace{0.5cm} i,j \in \{1,\dots m\}, i\neq j
\end{equation*} 
where $R_i$ is the rank of individual $i$ with survival time $t_i$, $N_i$ the number at risk at time $t_i$, and thus $N_i -1$ the number of individuals who can be compared with $i$   \citep{Kremers2007}. \\
For the RSF model, the $C$ index is computed based on the patient-specific predictions of the ensemble mortality in the terminal nodes of each tree. For this we first consider  the out-of-bag (oob) ensemble estimator of the cumulative hazard function (CHF) at time $t$ for patient $i$, $H_i^{oob} (t)$. It is given by the average prediction of the $n_{\text{tree}_i}$ trees for which the sample was not part of the bootstrap sample for building the tree \citep{Ishwaran2021a}:
\begin{equation*}
	H_i^{oob} (t) = \frac{1}{n_{\text{tree}_i}} \sum_{b \in n_{\text{tree}_i}} H_b(t|\mathbf{X})
\end{equation*}  
where $H_b(t|\mathbf{X})$ is the CHF predicted in the terminal node of the $b$th tree for the covariate vector $\mathbf{X} \in \mathds{R}^p$ of patient $i$ at time $t$.
The out-of-bag ensemble mortality for each patient $i = 1,\dots,n$ is then estimated as the sum of the oob CHF estimates over all unique event times $t_1,\dots,t_m$ in the training data \citep{Ishwaran2021a}:
\begin{equation*}
	M_i^{oob} = \sum_{j=1}^m H_i^{oob}(t_j)
\end{equation*}
The $C$ index is the proportion of concordant pairs among all pairs for which the decision can be made. If $M_{i_1}^{oob} >  M_{i_2}^{oob}$ and patient $i_1$ has the shorter event time compared to patient $i_2$ or vice versa, the pair is concordant. The closer $C$ index estimates are to 1 the better. 			
			
			
			%\noindent
			%If any of the sections are not relevant to your manuscript, please include the heading and write `Not applicable' for that section. 
			
			%%===================================================%%
			%% For presentation purpose, we have included        %%
			%% \bigskip command. please ignore this.             %%
			%%===================================================%%
			\bigskip
			%\begin{flushleft}%
			%Editorial Policies for:
			%
			%\bigskip\noindent
			%Springer journals and proceedings: \url{https://www.springer.com/gp/editorial-policies}
			%
			%\bigskip\noindent
			%Nature Portfolio journals: \url{https://www.nature.com/nature-research/editorial-policies}
			%
			%\bigskip\noindent
			%\textit{Scientific Reports}: \url{https://www.nature.com/srep/journal-policies/editorial-policies}
			%
			%\bigskip\noindent
			%BMC journals: \url{https://www.biomedcentral.com/getpublished/editorial-policies}
			%\end{flushleft}
			
			%\begin{appendices}
			%
			%\section{Section title of first appendix}\label{secA1}
			%
			%An appendix contains supplementary information that is not an essential part of the text itself but which may be helpful in providing a more comprehensive understanding of the research problem or it is information that is too cumbersome to be included in the body of the paper.
			%
			%%%=============================================%%
			%%% For submissions to Nature Portfolio Journals %%
			%%% please use the heading ``Extended Data''.   %%
			%%%=============================================%%
			%
			%%%=============================================================%%
			%%% Sample for another appendix section			       %%
			%%%=============================================================%%
			%
			%%% \section{Example of another appendix section}\label{secA2}%
			%%% Appendices may be used for helpful, supporting or essential material that would otherwise 
			%%% clutter, break up or be distracting to the text. Appendices can consist of sections, figures, 
			%%% tables and equations etc.
			%
			%\end{appendices}
			
			%%===========================================================================================%%
			%% If you are submitting to one of the Nature Portfolio journals, using the eJP submission   %%
			%% system, please include the references within the manuscript file itself. You may do this  %%
			%% by copying the reference list from your .bbl file, paste it into the main manuscript .tex %%
			%% file, and delete the associated \verb+\bibliography+ commands.                            %%
			%%===========================================================================================%%
			
			
			\clearpage
			\bibliographystyle{apa}
			\bibliography{sn-bibliography}% common bib file
	
	%% if required, the content of .bbl file can be included here once bbl is generated
	%\input sn-article.bbl
	
	
\end{document}



