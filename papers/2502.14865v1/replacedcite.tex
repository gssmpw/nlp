\section{Related Work}
Recent years have seen significant progress in studying cultural representation in AI, particularly in behavioral patterns, food, landmarks, and historical knowledge. However, most works focus on misalignment and biases in AI models or modern cultural trends, rather than positioning artifacts within their historical context and era across ancient civilizations. Meanwhile, studies on cultural inclusion in LLMs highlight the challenges of capturing the contextual and multifaceted nature of culture, emphasizing the limitations of text-based models in representing underrepresented cultures and the need for more robust evaluation methods ____.

% \subsection{Cultural Bias in Language and Vision Models}
Research on cultural influences in AI has increasingly focused on biases and misalignment in language models, particularly how they reflect and perpetuate dominant cultural norms. Early research on cultural biases in LLMs revealed their alignment with Western norms, particularly in moral reasoning, historical narratives, and societal values. Ramezani et al. (2023) analyze how monolingual English language models tend to reflect Western moral norms more strongly than diverse cultural perspectives, limiting their applicability in cross-cultural ethical contexts ____. Tao et al. (2024) further highlight the overrepresentation of Anglo-American and Protestant European values in AI-generated content, often underrepresenting non-Western traditions and belief systems ____. Similarly, Bu et al. (2025) explore value misalignment in cultural heritage-related text generation, warning of historical inaccuracies, cultural identity erosion, and oversimplification of complex narratives, with 65\% of the generated content showing significant misalignment ____.

To mitigate these biases, several approaches have been proposed. AlKhamissi et al. (2024) introduce Anthropological Prompting, a method that encourages LLMs to reason like cultural anthropologists by incorporating both emic (insider) and etic (outsider) perspectives ____. Similarly, Li et al. (2024) propose CultureLLM, a fine-tuning approach designed to integrate cultural knowledge into LLMs, particularly for low-resource cultures ____. While these techniques improve cultural alignment, their focus remains on modern cultural settings, leaving gaps in historical artifact contextualization across different time periods.

With the rise of Vision-Language Models (VLMs), cultural research has expanded to multimodal AI, revealing similar biases. Liu et al. (2025) introduce CultureVLM, a model designed to improve cultural understanding in VLMs, highlighting their inability to recognize non-Western cultural symbols, historical artifacts, and traditional gestures ____. Their work also presents CultureVerse, a large-scale multimodal dataset covering several cultural concepts, designed to evaluate VLMs' cultural reasoning. However, CultureVerse has a primary focus on modern cultural symbols, traditions, and everyday life. Additionally, Romero et al. (2024) develop CVQA, a multilingual and culturally diverse Visual Question Answering (VQA) benchmark, which reveals that state-of-the-art VLMs struggle with culturally grounded reasoning, particularly in non-Western contexts ____. However, these datasets primarily focus on present-day cultural contexts, even when historical artifacts are included, as they are often framed through the lens of modern nations rather than their original civilizations and historical epochs ____. This leaves a significant gap in representing artifacts within their authentic temporal and cultural contexts.

Efforts to bridge AI research with historical studies have led to the development of Historical Large Language Models (HLLMs), trained on historical texts to simulate past societies' psychology and value systems ____. These models aim to provide insight into long-term cultural evolution, but their reliance on text-only representations limits their application in multimodal historical studies. Similarly, Assael et al. (2022) introduce Ithaca, a deep learning model designed to assist historians in restoring, geographically attributing, and dating ancient Greek inscriptions, significantly improving accuracy over traditional methods ____. While these works contribute to historical AI, they primarily focus on text-based reconstruction rather than multimodal representations of historical artifacts across civilizations.

TimeTravel fills this gap by providing a 10k historical artifact open-source dataset spanning 10 ancient world regions (prehistoric and historic), offering the first benchmark to evaluate LMMs on temporal-cultural understanding with expert verification. Unlike prior datasets focused on contemporary cultural knowledge, TimeTravel enables AI models to contextualize artifacts within their historical era, ensuring a more accurate representation of civilizations and their material culture. With domain expert verification, the dataset enhances reliability and authenticity, mitigating potential biases and inaccuracies in AI-generated interpretations. By integrating both textual and multimodal perspectives, TimeTravel advances research in historical-cultural AI, enabling AI systems to better understand and reason about artifacts in their original context.