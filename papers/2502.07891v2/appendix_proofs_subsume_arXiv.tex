\section{Proofs that comparison of unrealizable supports subsumes graphical methods as rules to show observational nondominance}
\label{appendix_proofs_subsume}


\esep*
\begin{proof}
	The proof of  Theorem 48 of Ref.~\cite{Khanna_2023} starts from the assumption that $\mathfrak{G}$ presents an e-separation relation that is not presented by $\mathfrak{G}'$ and shows that there must be a set of events over binary visible variables that is a realizable support of $\mathfrak{G}'$ but not of $\mathfrak{G}$. This implies that the fact that  $\mathfrak{G}$ does not observationally dominates $\mathfrak{G}'$ can be shown by the comparison of unrealizable supports. 
\end{proof}


\DC*
\begin{proof}
	Assume that the nodes $v$ and $w$ are densely connected in  $\mathfrak{G}=\{\cal D, B\}$ but not in $\mathfrak{G}'=\{\cal D', B'\}$. 
	
	Let $S$ be a set of $2^{|\vis({\mathfrak{G}})|-1}$ events such that in half of the events both $X_v$ and $X_w$ are $0$ and in the other half both are $1$, while every possible combination of $0$'s and $1$'s occurs for all of the other visible variables. For example, for the case when there are four visible nodes $x$, $y$, $v$, $w$, $S$ contains the events
	\begin{align*}
		& \{X_x=0,X_y=0,X_v=0,X_w=0\}, \\
		& \{X_x=1,X_y=0,X_v=0,X_w=0\}, \\
		& \{X_x=0,X_y=1,X_v=0,X_w=0\}, \\
		& \{X_x=1,X_y=1,X_v=0,X_w=0\}, \\
		& \{X_x=0,X_y=0,X_v=1,X_w=1\}, \\
		& \{X_x=0,X_y=1,X_v=1,X_w=1\}, \\
		& \{X_x=1,X_y=0,X_v=1,X_w=1\}, \\
		& \{X_x=1,X_y=1,X_v=1,X_w=1\}.
	\end{align*}
	
	The result of Ref.~\cite{evans_dependency} says that a distribution $P$ where $X_v$ and $X_w$ are perfectly correlated while all of the other visible variables are mutually independent and uniformly distributed is realizable by $\mathfrak{G}$ if $v$ and $w$ are densely connected in $\mathfrak{G}$. Therefore, $S$ is a realizable support of  $\mathfrak{G}$.
	
	On the other hand, in $\mathfrak{G}'$, the nodes $v$ and $w$ are not densely connected. As pointed out in the proof of Theorem 6.4 of Ref.~\cite{evans_dependency}, this implies that $\mathfrak{G}'$ presents a nested Markov constraint between $X_v$ and $X_w$. In the binary case, such a constraint takes one of two forms~\cite{SEMs}:	
	\begin{align}
		&P(X_v=x_v|X_A,X_w=0,\text{do}(X_B))= \nonumber \\ &P(X_v=x_v|X_A, X_w=1,\text{do}(X_B)); \text{ or}
		\label{eq_1_do}
	\end{align}
	\begin{align}
		&P(X_v=x_v|X_A,\text{do}(X_B, X_w=0))= \nonumber  \\ &P(X_v=x_v|X_A,\text{do}(X_B, X_w=1)),
		\label{eq_2_do}
	\end{align}
	for sets $A\in\nodes(\mathfrak{G})$ and $B\in\nodes(\mathfrak{G})$ that do not include $v$ nor $w$, and where the do-conditionals that appear in these constraints are identifiable from observational data. 
	
	Regardless of whether it is identifiable or not, every component of a do-conditional is always larger than or equal to the corresponding component of the respective joint observational distribution. More specifically, $P(X_a|\text{do}(X_b))\geq P(X_a,X_b)$. Therefore, since the support $\cal S$ includes all events where $X_v= X_w$, then if $P$ has support $\cal S$ we must have:
	\begin{align}
		& P(X_v=0|X_A,X_w=0,\text{do}(X_B)) >0 \label{eq_aaa} \\
		& P(X_v=0|X_A,\text{do}(X_B, X_w=0)) >0.  \label{eq_bbb}
	\end{align}
	
	On the other hand, we know that the do-conditionals of the forms that appear in Eqs.\eqref{eq_1_do} and \eqref{eq_2_do}, when identifiable, are obtained by the joint probability distribution over all visible variables divided by marginal conditional probability distributions, \emph{none of which} include both $X_v$ and $X_w$ at the same time~\cite{causality_pearl}. Since $P$ has support $\cal S$, these marginal conditional probability distributions will always take non-zero values. The joint distribution over all visible variables, on the other hand, is zero when $X_v\neq X_w$. When the nested constraint between $v$ and $w$ is of the form of Eq.~\eqref{eq_1_do}, the do-conditionals that appear there are identifiable, and thus
	\begin{equation}
		P(X_v=0|X_A,X_w=1,\text{do}(X_B)) =0.
	\end{equation}
	
	Together with Eq.~\eqref{eq_aaa}, this shows that Eq.~\eqref{eq_1_do} cannot be satisfied by a distribution with support $\cal S$.  When the nested constraint between $v$ and $w$ is of the form of Eq.~\eqref{eq_2_do}, the do-conditionals that appear there are identifiable, and thus
	\begin{equation}
		P(X_v=0|X_A,\text{do}(X_B, X_w=1))  =0.
	\end{equation}
	
	Together with Eq.~\eqref{eq_bbb}, this shows that Eq.~\eqref{eq_2_do} cannot be satisfied by a distribution with support $\cal S$. 
	

	
	
	
	Thus, we find a binary support that is realizable by   $\mathfrak{G}$ but not by  $\mathfrak{G}'$, showing that the comparison of supports subsumes the comparison of densely connected pairs as a rule to show observational inequivalence.
	
\end{proof}

\DEF*
\begin{proof}
	By the assumption that  $\mathfrak{G}$ and $\mathfrak{G}'$  can be shown inequivalent by the directed-edge-free rule, it follows that  $\mathfrak{G}$ is observationally equivalent to a directed-edge-free mDAG $\mathfrak{G}_\text{DEF}$, and  $\mathfrak{G}'$ is observationally equivalent to a directed-edge-free mDAG $\mathfrak{G}'_\text{DEF}$. Both $\mathfrak{G}_\text{DEF}$ and $\mathfrak{G}'_\text{DEF}$ have a trivial directed structure. Since they are different, their simplicial complexes must be different. Suppose that the simplicial complex of  $\mathfrak{G}$  has a facet $B$ that is not a face of  the simplicial complex of $\mathfrak{G}'$. This means that the nodes in $B$ share a joint common cause in $\can(\mathfrak{G})$, but not in $\can(\mathfrak{G}')$. 
	
	In Example 2 of Ref.~\cite{steudel_ay_2015}, it was shown that a distribution where all the variables $X_1,...,X_n$ are perfectly correlated (i.e., where all the weight is on elements of the event space where $X_1=X_2=...=X_n$) is only realizable by causal structures where all the nodes corresponding to such variables share a common ancestor. Therefore, in our case, $\mathfrak{G}$ can realize perfect correlation between the node variables in $B$ while $\mathfrak{G}'$ cannot. Taking the case of distributions over binary variables, this implies that there are supports consisting of two events that are realizable by $\mathfrak{G}$  but not by $\mathfrak{G}'$; namely, the event where the binary variables associated with nodes of $B$ all take the value $0$  and the event where they all take the value $1$.
\end{proof}