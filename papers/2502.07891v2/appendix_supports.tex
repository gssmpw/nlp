\section{Fraser's Algorithm for Feasible Supports}
\label{appendix_supports}

In this section, we present the algorithm of Ref.~\cite{Fraser_Combinatorial_Solution} for finding $\mathcal{S}(\mathcal{G},\vec c_\text{vis})$, the set of supports with cardinalities $\vec c_\text{vis}$ of the visible variables that are realizable by a pDAG $\mathcal{G}$. This algorithm consists of enumerating all the possible responses that a visible variable can have to its parents, given that the cardinalities of the visible variables and the latent variables are fixed.

In Ref.~\cite{Fraser_Combinatorial_Solution}, it is shown that whenever the visible variables have finite cardinalities, we can assume the latent variables to have finite cardinalities without loss of generality. Furthermore, for a given number of events in the support, one gets an upper bound on the cardinality of the latent variables that has to be considered. This upper bound will be discussed after we describe the algorithm.

Given a pDAG $\mathcal{G}$:

\begin{enumerate}
    \item Fix the cardinalities of every latent variable to be $k$ and the cardinality of each visible variable $v$ to be $c_v$.
    \item For each visible variable, enumerate all possible response functions it can have to the valuations of its parents.
    
    For example, if a visible $v$ has only one parent and this parent is latent, it has $(c_v)^k$ possible response functions; we could have that $v$ reacts with the each one of its $c_v$ possible values for each one of the $k$ possible valuations of its parent.
     
    In general, if the total cardinality of the parents of $v$ is $K(v)$, then we have $(c_v)^{K(v)}$ possibilities of response function for $v$.
    \item For each possibility of response functions of all the visible variables of $\mathcal{G}$, we compute the support: the set of visible events that occur under that response function, for some valuation of the latent variables.
\end{enumerate}
The question that is left is how to choose $k$, the cardinality of latent variables. First we note that, if a support containing $s$ events is realizable by $\mathcal{G}$, it will certainly be found with the algorithm for all $k\geq s$. Thus, $s$ is an upper bound on the values of $k$ that we need to consider.

Assuming that all the visible variables have the same cardinality $c$, and considering that the total number of visible variables is  $|\vis(\mathcal{G})|$, the maximum number of events that we can have is $c^{|\vis(\mathcal{G})|}$. So, to be certain that we found all compatible supports, we should choose $k=c^{|\vis(\mathcal{G})|}$. 

Note that if $\mathcal{S}(\mathcal{G},\vec c_\text{vis}=(2,2,2,2))\neq\mathcal{S}(\mathcal{G}',\vec c_\text{vis}=(2,2,2,2))$, i.e., if the set of realizable {\em binary} supports of $\cal G$ and $\cal G'$ are different, then the sets of realizable supports will also be different when we go to higher cardinalities. This is so because we can simply gather events together (for example joining events $x$ and $y$ in $z=x\land y$) to go back to the binary case, where the difference is proven. 