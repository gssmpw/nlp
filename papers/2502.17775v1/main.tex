% This must be in the first 5 lines to tell arXiv to use pdfLaTeX, which is strongly recommended.
\pdfoutput=1
% In particular, the hyperref package requires pdfLaTeX in order to break URLs across lines.

\documentclass[11pt]{article}

% Change "review" to "final" to generate the final (sometimes called camera-ready) version.
% Change to "preprint" to generate a non-anonymous version with page numbers.
\usepackage[preprint]{acl}

% Standard package includes
\usepackage{times}
\usepackage{latexsym}

% For proper rendering and hyphenation of words containing Latin characters (including in bib files)
\usepackage[T1]{fontenc}
% For Vietnamese characters
% \usepackage[T5]{fontenc}
% See https://www.latex-project.org/help/documentation/encguide.pdf for other character sets

% This assumes your files are encoded as UTF8
\usepackage[utf8]{inputenc}

% This is not strictly necessary, and may be commented out,
% but it will improve the layout of the manuscript,
% and will typically save some space.
\usepackage{microtype}

% This is also not strictly necessary, and may be commented out.
% However, it will improve the aesthetics of text in
% the typewriter font.
\usepackage{inconsolata}

%Including images in your LaTeX document requires adding
%additional package(s)
\usepackage{graphicx}

\usepackage{hyperref}
\usepackage{url}
\usepackage{graphicx}
\usepackage{xcolor}
\usepackage{listings}
\usepackage{adjustbox}
\usepackage{subcaption}
\usepackage{amssymb}% http://ctan.org/pkg/amssymb
\usepackage{pifont}% http://ctan.org/pkg/pifont

% Set up listings for Python
\lstset{
    language=Python,                 % Language of the code
    basicstyle=\ttfamily\footnotesize, % Font style and size
    keywordstyle=\color{blue},        % Style for keywords
    commentstyle=\color{gray},        % Style for comments
    stringstyle=\color{red},          % Style for strings
    showstringspaces=false,           % Don't display spaces in strings
    numberstyle=\tiny\color{gray},    % Style for line numbers
    frame=single,                     % Adds a frame around the code
    breaklines=true,                  % Automatic line breaking
    captionpos=b,                     % Caption position (b for bottom)
    tabsize=4                         % Size of tabs
}

\newcommand{\pk}[1]{{\textcolor{red}{~(PK: #1)}}}
\newcommand{\tp}[1]{{\textcolor{orange}{~(TP: #1)}}}
\newcommand{\xmark}{\ding{55}}%
\newcommand{\improve}[1]{($\textcolor{green}{\uparrow #1}$)}
\newcommand{\worse}[1]{($\textcolor{red}{\downarrow #1}$)}
\newcommand{\iclr}[1]{{\textcolor{blue}{#1}}}


% This assumes your files are encoded as UTF8
\usepackage[utf8]{inputenc}
\newcommand{\fix}{\marginpar{FIX}}
\newcommand{\new}{\marginpar{NEW}}

% This is not strictly necessary, and may be commented out.
% However, it will improve the layout of the manuscript,
% and will typically save some space.
\usepackage{microtype}

% This is also not strictly necessary, and may be commented out.
% However, it will improve the aesthetics of text in
% the typewriter font.
\usepackage{inconsolata}


% If the title and author information does not fit in the area allocated, uncomment the following
%
%\setlength\titlebox{<dim>}
%
% and set <dim> to something 5cm or larger.

\title{FoREST: \textbf{F}rame \textbf{o}f \textbf{R}eference \textbf{E}valuation in \textbf{S}patial Reasoning \textbf{T}asks}

% Author information can be set in various styles:
% For several authors from the same institution:
% \author{Author 1 \and ... \and Author n \\
%         Address line \\ ... \\ Address line}
% if the names do not fit well on one line use
%         Author 1 \\ {\bf Author 2} \\ ... \\ {\bf Author n} \\
% For authors from different institutions:
% \author{Author 1 \\ Address line \\  ... \\ Address line
%         \And  ... \And
%         Author n \\ Address line \\ ... \\ Address line}
% To start a seperate ``row'' of authors use \AND, as in
% \author{Author 1 \\ Address line \\  ... \\ Address line
%         \AND
%         Author 2 \\ Address line \\ ... \\ Address line \And
%         Author 3 \\ Address line \\ ... \\ Address line}

\author{Tanawan Premsri \\
  Department of Computer Science \\
  Michigan State University\\
  \texttt{premsrit@msu.edu} \\\And
  Parisa Kordjamshidi \\
Department of Computer Science \\
  Michigan State University\\
  \texttt{kordjams@msu.edu} \\}

\begin{document}
\maketitle
\begin{abstract}
Spatial reasoning is a fundamental aspect of human intelligence. 
One key concept in spatial cognition is the Frame of Reference (FoR), which identifies the perspective of spatial expressions. 
Despite its significance, FoR has received limited attention in AI models that need spatial intelligence. There is a lack of dedicated benchmarks and in-depth evaluation of large language models (LLMs) in this area.
To address this issue, we introduce the \textbf{F}rame \textbf{o}f \textbf{R}eference \textbf{E}valuation in \textbf{S}patial Reasoning \textbf{T}asks (FoREST) benchmark, designed to assess FoR comprehension in LLMs.
We evaluate LLMs on answering questions that require FoR comprehension and layout generation in text-to-image models using FoREST.
Our results reveal a notable performance gap across different FoR classes in various LLMs, affecting their ability to generate accurate layouts for text-to-image generation. 
This highlights critical shortcomings in FoR comprehension.
To improve FoR understanding, we propose Spatial-Guided prompting, which improves LLMs’ ability to extract essential spatial concepts. 
Our proposed method improves overall performance across spatial reasoning tasks.
\end{abstract}

\section{Introduction}
\section{Introduction}

% \textcolor{red}{Still on working}

% \textcolor{red}{add label for each section}


Robot learning relies on diverse and high-quality data to learn complex behaviors \cite{aldaco2024aloha, wang2024dexcap}.
Recent studies highlight that models trained on datasets with greater complexity and variation in the domain tend to generalize more effectively across broader scenarios \cite{mann2020language, radford2021learning, gao2024efficient}.
% However, creating such diverse datasets in the real world presents significant challenges.
% Modifying physical environments and adjusting robot hardware settings require considerable time, effort, and financial resources.
% In contrast, simulation environments offer a flexible and efficient alternative.
% Simulations allow for the creation and modification of digital environments with a wide range of object shapes, weights, materials, lighting, textures, friction coefficients, and so on to incorporate domain randomization,
% which helps improve the robustness of models when deployed in real-world conditions.
% These environments can be easily adjusted and reset, enabling faster iterations and data collection.
% Additionally, simulations provide the ability to consistently reproduce scenarios, which is essential for benchmarking and model evaluation.
% Another advantage of simulations is their flexibility in sensor integration. Sensors such as cameras, LiDARs, and tactile sensors can be added or repositioned without the physical limitations present in real-world setups. Simulations also eliminate the risk of damaging expensive hardware during edge-case experiments, making them an ideal platform for testing rare or dangerous scenarios that are impractical to explore in real life.
By leveraging immersive perspectives and interactions, Extended Reality\footnote{Extended Reality is an umbrella term to refer to Augmented Reality, Mixed Reality, and Virtual Reality \cite{wikipediaExtendedReality}}
(XR)
is a promising candidate for efficient and intuitive large scale data collection \cite{jiang2024comprehensive, arcade}
% With the demand for collecting data, XR provides a promising approach for humans to teach robots by offering users an immersive experience.
in simulation \cite{jiang2024comprehensive, arcade, dexhub-park} and real-world scenarios \cite{openteach, opentelevision}.
However, reusing and reproducing current XR approaches for robot data collection for new settings and scenarios is complicated and requires significant effort.
% are difficult to reuse and reproduce system makes it hard to reuse and reproduce in another data collection pipeline.
This bottleneck arises from three main limitations of current XR data collection and interaction frameworks: \textit{asset limitation}, \textit{simulator limitation}, and \textit{device limitation}.
% \textcolor{red}{ASSIGN THESE CITATION PROPERLY:}
% \textcolor{red}{list them by time order???}
% of collecting data by using XR have three main limitations.
Current approaches suffering from \textit{asset limitation} \cite{arclfd, jiang2024comprehensive, arcade, george2025openvr, vicarios}
% Firstly, recent works \cite{jiang2024comprehensive, arcade, dexhub-park}
can only use predefined robot models and task scenes. Configuring new tasks requires significant effort, since each new object or model must be specifically integrated into the XR application.
% and it takes too much effort to configure new tasks in their systems since they cannot spawn arbitrary models in the XR application.
The vast majority of application are developed for specific simulators or real-world scenarios. This \textit{simulator limitation} \cite{mosbach2022accelerating, lipton2017baxter, dexhub-park, arcade}
% Secondly, existing systems are limited to a single simulation platform or real-world scenarios.
significantly reduces reusability and makes adaptation to new simulation platforms challenging.
Additionally, most current XR frameworks are designed for a specific version of a single XR headset, leading to a \textit{device limitation} 
\cite{lipton2017baxter, armada, openteach, meng2023virtual}.
% and there is no work working on the extendability of transferring to a new headsets as far as we know.
To the best of our knowledge, no existing work has explored the extensibility or transferability of their framework to different headsets.
These limitations hamper reproducibility and broader contributions of XR based data collection and interaction to the research community.
% as each research group typically has its own data collection pipeline.
% In addition to these main limitations, existing XR systems are not well suited for managing multiple robot systems,
% as they are often designed for single-operator use.

In addition to these main limitations, existing XR systems are often designed for single-operator use, prohibiting collaborative data collection.
At the same time, controlling multiple robots at once can be very difficult for a single operator,
making data collection in multi-robot scenarios particularly challenging \cite{orun2019effect}.
Although there are some works using collaborative data collection in the context of tele-operation \cite{tung2021learning, Qin2023AnyTeleopAG},
there is no XR-based data collection system supporting collaborative data collection.
This limitation highlights the need for more advanced XR solutions that can better support multi-robot and multi-user scenarios.
% \textcolor{red}{more papers about collaborative data collection}

To address all of these issues, we propose \textbf{IRIS},
an \textbf{I}mmersive \textbf{R}obot \textbf{I}nteraction \textbf{S}ystem.
This general system supports various simulators, benchmarks and real-world scenarios.
It is easily extensible to new simulators and XR headsets.
IRIS achieves generalization across six dimensions:
% \begin{itemize}
%     \item \textit{Cross-scene} : diverse object models;
%     \item \textit{Cross-embodiment}: diverse robot models;
%     \item \textit{Cross-simulator}: 
%     \item \textit{Cross-reality}: fd
%     \item \textit{Cross-platform}: fd
%     \item \textit{Cross-users}: fd
% \end{itemize}
\textbf{Cross-Scene}, \textbf{Cross-Embodiment}, \textbf{Cross-Simulator}, \textbf{Cross-Reality}, \textbf{Cross-Platform}, and \textbf{Cross-User}.

\textbf{Cross-Scene} and \textbf{Cross-Embodiment} allow the system to handle arbitrary objects and robots in the simulation,
eliminating restrictions about predefined models in XR applications.
IRIS achieves these generalizations by introducing a unified scene specification, representing all objects,
including robots, as data structures with meshes, materials, and textures.
The unified scene specification is transmitted to the XR application to create and visualize an identical scene.
By treating robots as standard objects, the system simplifies XR integration,
allowing researchers to work with various robots without special robot-specific configurations.
\textbf{Cross-Simulator} ensures compatibility with various simulation engines.
IRIS simplifies adaptation by parsing simulated scenes into the unified scene specification, eliminating the need for XR application modifications when switching simulators.
New simulators can be integrated by creating a parser to convert their scenes into the unified format.
This flexibility is demonstrated by IRIS’ support for Mujoco \cite{todorov2012mujoco}, IsaacSim \cite{mittal2023orbit}, CoppeliaSim \cite{coppeliaSim}, and even the recent Genesis \cite{Genesis} simulator.
\textbf{Cross-Reality} enables the system to function seamlessly in both virtual simulations and real-world applications.
IRIS enables real-world data collection through camera-based point cloud visualization.
\textbf{Cross-Platform} allows for compatibility across various XR devices.
Since XR device APIs differ significantly, making a single codebase impractical, IRIS XR application decouples its modules to maximize code reuse.
This application, developed by Unity \cite{unity3dUnityManual}, separates scene visualization and interaction, allowing developers to integrate new headsets by reusing the visualization code and only implementing input handling for hand, head, and motion controller tracking.
IRIS provides an implementation of the XR application in the Unity framework, allowing for a straightforward deployment to any device that supports Unity. 
So far, IRIS was successfully deployed to the Meta Quest 3 and HoloLens 2.
Finally, the \textbf{Cross-User} ability allows multiple users to interact within a shared scene.
IRIS achieves this ability by introducing a protocol to establish the communication between multiple XR headsets and the simulation or real-world scenarios.
Additionally, IRIS leverages spatial anchors to support the alignment of virtual scenes from all deployed XR headsets.
% To make an seamless user experience for robot learning data collection,
% IRIS also tested in three different robot control interface
% Furthermore, to demonstrate the extensibility of our approach, we have implemented a robot-world pipeline for real robot data collection, ensuring that the system can be used in both simulated and real-world environments.
The Immersive Robot Interaction System makes the following contributions\\
\textbf{(1) A unified scene specification} that is compatible with multiple robot simulators. It enables various XR headsets to visualize and interact with simulated objects and robots, providing an immersive experience while ensuring straightforward reusability and reproducibility.\\
\textbf{(2) A collaborative data collection framework} designed for XR environments. The framework facilitates enhanced robot data acquisition.\\
\textbf{(3) A user study} demonstrating that IRIS significantly improves data collection efficiency and intuitiveness compared to the LIBERO baseline.

% \begin{table*}[t]
%     \centering
%     \begin{tabular}{lccccccc}
%         \toprule
%         & \makecell{Physical\\Interaction}
%         & \makecell{XR\\Enabled}
%         & \makecell{Free\\View}
%         & \makecell{Multiple\\Robots}
%         & \makecell{Robot\\Control}
%         % Force Feedback???
%         & \makecell{Soft Object\\Supported}
%         & \makecell{Collaborative\\Data} \\
%         \midrule
%         ARC-LfD \cite{arclfd}                              & Real        & \cmark & \xmark & \xmark & Joint              & \xmark & \xmark \\
%         DART \cite{dexhub-park}                            & Sim         & \cmark & \cmark & \cmark & Cartesian          & \xmark & \xmark \\
%         \citet{jiang2024comprehensive}                     & Sim         & \cmark & \xmark & \xmark & Joint \& Cartesian & \xmark & \xmark \\
%         \citet{mosbach2022accelerating}                    & Sim         & \cmark & \cmark & \xmark & Cartesian          & \xmark & \xmark \\
%         ARCADE \cite{arcade}                               & Real        & \cmark & \cmark & \xmark & Cartesian          & \xmark & \xmark \\
%         Holo-Dex \cite{holodex}                            & Real        & \cmark & \xmark & \cmark & Cartesian          & \cmark & \xmark \\
%         ARMADA \cite{armada}                               & Real        & \cmark & \xmark & \cmark & Cartesian          & \cmark & \xmark \\
%         Open-TeleVision \cite{opentelevision}              & Real        & \cmark & \cmark & \cmark & Cartesian          & \cmark & \xmark \\
%         OPEN TEACH \cite{openteach}                        & Real        & \cmark & \xmark & \cmark & Cartesian          & \cmark & \cmark \\
%         GELLO \cite{wu2023gello}                           & Real        & \xmark & \cmark & \cmark & Joint              & \cmark & \xmark \\
%         DexCap \cite{wang2024dexcap}                       & Real        & \xmark & \cmark & \xmark & Cartesian          & \cmark & \xmark \\
%         AnyTeleop \cite{Qin2023AnyTeleopAG}                & Real        & \xmark & \xmark & \cmark & Cartesian          & \cmark & \cmark \\
%         Vicarios \cite{vicarios}                           & Real        & \cmark & \xmark & \xmark & Cartesian          & \cmark & \xmark \\     
%         Augmented Visual Cues \cite{augmentedvisualcues}   & Real        & \cmark & \cmark & \xmark & Cartesian          & \xmark & \xmark \\ 
%         \citet{wang2024robotic}                            & Real        & \cmark & \cmark & \xmark & Cartesian          & \cmark & \xmark \\
%         Bunny-VisionPro \cite{bunnyvisionpro}              & Real        & \cmark & \cmark & \cmark & Cartesian          & \cmark & \xmark \\
%         IMMERTWIN \cite{immertwin}                         & Real        & \cmark & \cmark & \cmark & Cartesian          & \xmark & \xmark \\
%         \citet{meng2023virtual}                            & Sim \& Real & \cmark & \cmark & \xmark & Cartesian          & \xmark & \xmark \\
%         Shared Control Framework \cite{sharedctlframework} & Real        & \cmark & \cmark & \cmark & Cartesian          & \xmark & \xmark \\
%         OpenVR \cite{openvr}                               & Real        & \cmark & \cmark & \xmark & Cartesian          & \xmark & \xmark \\
%         \citet{digitaltwinmr}                              & Real        & \cmark & \cmark & \xmark & Cartesian          & \cmark & \xmark \\
        
%         \midrule
%         \textbf{Ours} & Sim \& Real & \cmark & \cmark & \cmark & Joint \& Cartesian  & \cmark & \cmark \\
%         \bottomrule
%     \end{tabular}
%     \caption{This is a cross-column table with automatic line breaking.}
%     \label{tab:cross-column}
% \end{table*}

% \begin{table*}[t]
%     \centering
%     \begin{tabular}{lccccccc}
%         \toprule
%         & \makecell{Cross-Embodiment}
%         & \makecell{Cross-Scene}
%         & \makecell{Cross-Simulator}
%         & \makecell{Cross-Reality}
%         & \makecell{Cross-Platform}
%         & \makecell{Cross-User} \\
%         \midrule
%         ARC-LfD \cite{arclfd}                              & \xmark & \xmark & \xmark & \xmark & \xmark & \xmark \\
%         DART \cite{dexhub-park}                            & \cmark & \cmark & \xmark & \xmark & \xmark & \xmark \\
%         \citet{jiang2024comprehensive}                     & \xmark & \cmark & \xmark & \xmark & \xmark & \xmark \\
%         \citet{mosbach2022accelerating}                    & \xmark & \cmark & \xmark & \xmark & \xmark & \xmark \\
%         ARCADE \cite{arcade}                               & \xmark & \xmark & \xmark & \xmark & \xmark & \xmark \\
%         Holo-Dex \cite{holodex}                            & \cmark & \xmark & \xmark & \xmark & \xmark & \xmark \\
%         ARMADA \cite{armada}                               & \cmark & \xmark & \xmark & \xmark & \xmark & \xmark \\
%         Open-TeleVision \cite{opentelevision}              & \cmark & \xmark & \xmark & \xmark & \cmark & \xmark \\
%         OPEN TEACH \cite{openteach}                        & \cmark & \xmark & \xmark & \xmark & \xmark & \cmark \\
%         GELLO \cite{wu2023gello}                           & \cmark & \xmark & \xmark & \xmark & \xmark & \xmark \\
%         DexCap \cite{wang2024dexcap}                       & \xmark & \xmark & \xmark & \xmark & \xmark & \xmark \\
%         AnyTeleop \cite{Qin2023AnyTeleopAG}                & \cmark & \cmark & \cmark & \cmark & \xmark & \cmark \\
%         Vicarios \cite{vicarios}                           & \xmark & \xmark & \xmark & \xmark & \xmark & \xmark \\     
%         Augmented Visual Cues \cite{augmentedvisualcues}   & \xmark & \xmark & \xmark & \xmark & \xmark & \xmark \\ 
%         \citet{wang2024robotic}                            & \xmark & \xmark & \xmark & \xmark & \xmark & \xmark \\
%         Bunny-VisionPro \cite{bunnyvisionpro}              & \cmark & \xmark & \xmark & \xmark & \xmark & \xmark \\
%         IMMERTWIN \cite{immertwin}                         & \cmark & \xmark & \xmark & \xmark & \xmark & \xmark \\
%         \citet{meng2023virtual}                            & \xmark & \cmark & \xmark & \cmark & \xmark & \xmark \\
%         \citet{sharedctlframework}                         & \cmark & \xmark & \xmark & \xmark & \xmark & \xmark \\
%         OpenVR \cite{george2025openvr}                               & \xmark & \xmark & \xmark & \xmark & \xmark & \xmark \\
%         \citet{digitaltwinmr}                              & \xmark & \xmark & \xmark & \xmark & \xmark & \xmark \\
        
%         \midrule
%         \textbf{Ours} & \cmark & \cmark & \cmark & \cmark & \cmark & \cmark \\
%         \bottomrule
%     \end{tabular}
%     \caption{This is a cross-column table with automatic line breaking.}
% \end{table*}

% \begin{table*}[t]
%     \centering
%     \begin{tabular}{lccccccc}
%         \toprule
%         & \makecell{Cross-Scene}
%         & \makecell{Cross-Embodiment}
%         & \makecell{Cross-Simulator}
%         & \makecell{Cross-Reality}
%         & \makecell{Cross-Platform}
%         & \makecell{Cross-User}
%         & \makecell{Control Space} \\
%         \midrule
%         % Vicarios \cite{vicarios}                           & \xmark & \xmark & \xmark & \xmark & \xmark & \xmark \\     
%         % Augmented Visual Cues \cite{augmentedvisualcues}   & \xmark & \xmark & \xmark & \xmark & \xmark & \xmark \\ 
%         % OpenVR \cite{george2025openvr}                     & \xmark & \xmark & \xmark & \xmark & \xmark & \xmark \\
%         \citet{digitaltwinmr}                              & \xmark & \xmark & \xmark & \xmark & \xmark & \xmark &  \\
%         ARC-LfD \cite{arclfd}                              & \xmark & \xmark & \xmark & \xmark & \xmark & \xmark &  \\
%         \citet{sharedctlframework}                         & \cmark & \xmark & \xmark & \xmark & \xmark & \xmark &  \\
%         \citet{jiang2024comprehensive}                     & \cmark & \xmark & \xmark & \xmark & \xmark & \xmark &  \\
%         \citet{mosbach2022accelerating}                    & \cmark & \xmark & \xmark & \xmark & \xmark & \xmark & \\
%         Holo-Dex \cite{holodex}                            & \cmark & \xmark & \xmark & \xmark & \xmark & \xmark & \\
%         ARCADE \cite{arcade}                               & \cmark & \cmark & \xmark & \xmark & \xmark & \xmark & \\
%         DART \cite{dexhub-park}                            & Limited & Limited & Mujoco & Sim & Vision Pro & \xmark &  Cartesian\\
%         ARMADA \cite{armada}                               & \cmark & \cmark & \xmark & \xmark & \xmark & \xmark & \\
%         \citet{meng2023virtual}                            & \cmark & \cmark & \xmark & \cmark & \xmark & \xmark & \\
%         % GELLO \cite{wu2023gello}                           & \cmark & \xmark & \xmark & \xmark & \xmark & \xmark \\
%         % DexCap \cite{wang2024dexcap}                       & \xmark & \xmark & \xmark & \xmark & \xmark & \xmark \\
%         % AnyTeleop \cite{Qin2023AnyTeleopAG}                & \cmark & \cmark & \cmark & \cmark & \xmark & \cmark \\
%         % \citet{wang2024robotic}                            & \xmark & \xmark & \xmark & \xmark & \xmark & \xmark \\
%         Bunny-VisionPro \cite{bunnyvisionpro}              & \cmark & \cmark & \xmark & \xmark & \xmark & \xmark & \\
%         IMMERTWIN \cite{immertwin}                         & \cmark & \cmark & \xmark & \xmark & \xmark & \xmark & \\
%         Open-TeleVision \cite{opentelevision}              & \cmark & \cmark & \xmark & \xmark & \cmark & \xmark & \\
%         \citet{szczurek2023multimodal}                     & \xmark & \xmark & \xmark & Real & \xmark & \cmark & \\
%         OPEN TEACH \cite{openteach}                        & \cmark & \cmark & \xmark & \xmark & \xmark & \cmark & \\
%         \midrule
%         \textbf{Ours} & \cmark & \cmark & \cmark & \cmark & \cmark & \cmark \\
%         \bottomrule
%     \end{tabular}
%     \caption{TODO, Bruce: this table can be further optimized.}
% \end{table*}

\definecolor{goodgreen}{HTML}{228833}
\definecolor{goodred}{HTML}{EE6677}
\definecolor{goodgray}{HTML}{BBBBBB}

\begin{table*}[t]
    \centering
    \begin{adjustbox}{max width=\textwidth}
    \renewcommand{\arraystretch}{1.2}    
    \begin{tabular}{lccccccc}
        \toprule
        & \makecell{Cross-Scene}
        & \makecell{Cross-Embodiment}
        & \makecell{Cross-Simulator}
        & \makecell{Cross-Reality}
        & \makecell{Cross-Platform}
        & \makecell{Cross-User}
        & \makecell{Control Space} \\
        \midrule
        % Vicarios \cite{vicarios}                           & \xmark & \xmark & \xmark & \xmark & \xmark & \xmark \\     
        % Augmented Visual Cues \cite{augmentedvisualcues}   & \xmark & \xmark & \xmark & \xmark & \xmark & \xmark \\ 
        % OpenVR \cite{george2025openvr}                     & \xmark & \xmark & \xmark & \xmark & \xmark & \xmark \\
        \citet{digitaltwinmr}                              & \textcolor{goodred}{Limited}     & \textcolor{goodred}{Single Robot} & \textcolor{goodred}{Unity}    & \textcolor{goodred}{Real}          & \textcolor{goodred}{Meta Quest 2} & \textcolor{goodgray}{N/A} & \textcolor{goodred}{Cartesian} \\
        ARC-LfD \cite{arclfd}                              & \textcolor{goodgray}{N/A}        & \textcolor{goodred}{Single Robot} & \textcolor{goodgray}{N/A}     & \textcolor{goodred}{Real}          & \textcolor{goodred}{HoloLens}     & \textcolor{goodgray}{N/A} & \textcolor{goodred}{Cartesian} \\
        \citet{sharedctlframework}                         & \textcolor{goodred}{Limited}     & \textcolor{goodred}{Single Robot} & \textcolor{goodgray}{N/A}     & \textcolor{goodred}{Real}          & \textcolor{goodred}{HTC Vive Pro} & \textcolor{goodgray}{N/A} & \textcolor{goodred}{Cartesian} \\
        \citet{jiang2024comprehensive}                     & \textcolor{goodred}{Limited}     & \textcolor{goodred}{Single Robot} & \textcolor{goodgray}{N/A}     & \textcolor{goodred}{Real}          & \textcolor{goodred}{HoloLens 2}   & \textcolor{goodgray}{N/A} & \textcolor{goodgreen}{Joint \& Cartesian} \\
        \citet{mosbach2022accelerating}                    & \textcolor{goodgreen}{Available} & \textcolor{goodred}{Single Robot} & \textcolor{goodred}{IsaacGym} & \textcolor{goodred}{Sim}           & \textcolor{goodred}{Vive}         & \textcolor{goodgray}{N/A} & \textcolor{goodgreen}{Joint \& Cartesian} \\
        Holo-Dex \cite{holodex}                            & \textcolor{goodgray}{N/A}        & \textcolor{goodred}{Single Robot} & \textcolor{goodgray}{N/A}     & \textcolor{goodred}{Real}          & \textcolor{goodred}{Meta Quest 2} & \textcolor{goodgray}{N/A} & \textcolor{goodred}{Joint} \\
        ARCADE \cite{arcade}                               & \textcolor{goodgray}{N/A}        & \textcolor{goodred}{Single Robot} & \textcolor{goodgray}{N/A}     & \textcolor{goodred}{Real}          & \textcolor{goodred}{HoloLens 2}   & \textcolor{goodgray}{N/A} & \textcolor{goodred}{Cartesian} \\
        DART \cite{dexhub-park}                            & \textcolor{goodred}{Limited}     & \textcolor{goodred}{Limited}      & \textcolor{goodred}{Mujoco}   & \textcolor{goodred}{Sim}           & \textcolor{goodred}{Vision Pro}   & \textcolor{goodgray}{N/A} & \textcolor{goodred}{Cartesian} \\
        ARMADA \cite{armada}                               & \textcolor{goodgray}{N/A}        & \textcolor{goodred}{Limited}      & \textcolor{goodgray}{N/A}     & \textcolor{goodred}{Real}          & \textcolor{goodred}{Vision Pro}   & \textcolor{goodgray}{N/A} & \textcolor{goodred}{Cartesian} \\
        \citet{meng2023virtual}                            & \textcolor{goodred}{Limited}     & \textcolor{goodred}{Single Robot} & \textcolor{goodred}{PhysX}   & \textcolor{goodgreen}{Sim \& Real} & \textcolor{goodred}{HoloLens 2}   & \textcolor{goodgray}{N/A} & \textcolor{goodred}{Cartesian} \\
        % GELLO \cite{wu2023gello}                           & \cmark & \xmark & \xmark & \xmark & \xmark & \xmark \\
        % DexCap \cite{wang2024dexcap}                       & \xmark & \xmark & \xmark & \xmark & \xmark & \xmark \\
        % AnyTeleop \cite{Qin2023AnyTeleopAG}                & \cmark & \cmark & \cmark & \cmark & \xmark & \cmark \\
        % \citet{wang2024robotic}                            & \xmark & \xmark & \xmark & \xmark & \xmark & \xmark \\
        Bunny-VisionPro \cite{bunnyvisionpro}              & \textcolor{goodgray}{N/A}        & \textcolor{goodred}{Single Robot} & \textcolor{goodgray}{N/A}     & \textcolor{goodred}{Real}          & \textcolor{goodred}{Vision Pro}   & \textcolor{goodgray}{N/A} & \textcolor{goodred}{Cartesian} \\
        IMMERTWIN \cite{immertwin}                         & \textcolor{goodgray}{N/A}        & \textcolor{goodred}{Limited}      & \textcolor{goodgray}{N/A}     & \textcolor{goodred}{Real}          & \textcolor{goodred}{HTC Vive}     & \textcolor{goodgray}{N/A} & \textcolor{goodred}{Cartesian} \\
        Open-TeleVision \cite{opentelevision}              & \textcolor{goodgray}{N/A}        & \textcolor{goodred}{Limited}      & \textcolor{goodgray}{N/A}     & \textcolor{goodred}{Real}          & \textcolor{goodgreen}{Meta Quest, Vision Pro} & \textcolor{goodgray}{N/A} & \textcolor{goodred}{Cartesian} \\
        \citet{szczurek2023multimodal}                     & \textcolor{goodgray}{N/A}        & \textcolor{goodred}{Limited}      & \textcolor{goodgray}{N/A}     & \textcolor{goodred}{Real}          & \textcolor{goodred}{HoloLens 2}   & \textcolor{goodgreen}{Available} & \textcolor{goodred}{Joint \& Cartesian} \\
        OPEN TEACH \cite{openteach}                        & \textcolor{goodgray}{N/A}        & \textcolor{goodgreen}{Available}  & \textcolor{goodgray}{N/A}     & \textcolor{goodred}{Real}          & \textcolor{goodred}{Meta Quest 3} & \textcolor{goodred}{N/A} & \textcolor{goodgreen}{Joint \& Cartesian} \\
        \midrule
        \textbf{Ours}                                      & \textcolor{goodgreen}{Available} & \textcolor{goodgreen}{Available}  & \textcolor{goodgreen}{Mujoco, CoppeliaSim, IsaacSim} & \textcolor{goodgreen}{Sim \& Real} & \textcolor{goodgreen}{Meta Quest 3, HoloLens 2} & \textcolor{goodgreen}{Available} & \textcolor{goodgreen}{Joint \& Cartesian} \\
        \bottomrule
        \end{tabular}
    \end{adjustbox}
    \caption{Comparison of XR-based system for robots. IRIS is compared with related works in different dimensions.}
\end{table*}



%by guiding the model to focus on three key spatial relations. 
\section{Spatial Primitives}\label{sec:primitives}
We review three semantic aspects of spatial information expressed in language: Spatial Roles, Spatial Relations, and Frame of Reference.  

\noindent\textbf{Spatial Roles.} 
We focus on two main spatial roles~\citep{kordjamshidi-etal-2010-spatial} of \textit{Locatum}, and \textit{Relatum}. 
The locatum is the object described in the spatial expression, while the relatum is the other object used to describe the position of the locatum. 
An example is \textit{a cat is to the left of a dog}, where the \textit{cat} is the locatum, and the \textit{dog} is the relatum.

\noindent\textbf{Spatial Relations.} 
When dealing with spatial knowledge representation and reasoning, three main relations categories are often considered, that is, directional, topological, and distance~\citep{reasoningQualitaiveDaniel, COHN2008551,ACMpaper}. 
\textit{Directional} describes an object's direction based on specific coordinates. Examples of relations include left and right.
\textit{Topological} describes the containment between two objects, such as inside.
\textit{Distance} describes qualitative and quantitative relations between entities. Examples of qualitative are far, and quantitative are 3km.

\noindent\textbf{Spatial Frame of Reference.} We use four frames of references investigated in the cognitive linguistic studies~\cite{TENBRINK2011704}. These are defined based on the concept of \textit{Perspective}, which is the origin of a coordinate system to determine the direction. The four frames of reference are defined as follows.

\noindent1. \textit{External Intrinsic} describes a spatial relation from the relatum's perspective, where the relatum does not contain the locatum. The top-right image in Figure~\ref{fig:FoR_classes} illustrates this with the sentence, \textit{A cat is to the right of the car from the car's perspective.}

\noindent2. \textit{External Relative} describes a spatial relation from the observer's perspective.
The top-left image in Figure~\ref{fig:FoR_classes}  shows an example with the sentence, \textit{A cat is to the left of a car from my perspective.}

\noindent3. \textit{Internal Intrinsic} describes a spatial relation from the relatum's perspective, where the relatum contains the locatum. The bottom-right image in Figure~\ref{fig:FoR_classes} show this with the sentence, \textit{A cat is inside and back of the car from the car's perspective.}

\noindent4. \textit{Internal Relative} describes a spatial relation from the observer's perspective where the locatum is inside the relatum. The bottom-left image in Figure~\ref{fig:FoR_classes} show this FoR with the sentence, \textit{A cat is inside and to the left of the car from my perspective.}


\begin{figure*}[t!]
    \centering
    \includegraphics[width=0.8\linewidth, trim= {0 0 0.5 0.5cm }]{Figures/ImagePipeline.pdf}
    \caption{Pipeline for dataset creation, starting from selecting a locatum and relatum from available objects and then applying a spatial template to generate the spatial expression ($T$). FoRs are assigned based on the relatum’s properties. $T$ is then categorized based on the number of FoRs. For example, \textit{A cat is to the right of a dog }(with two possible FoRs: external intrinsic and external relative) belongs to the A-split. Then, its disambiguated version (A cat is to the right of a dog from the dog's perspective) is added to the C-split. Next, if applicable, a relatum's orientation is included for visualization and question generation. Finally, Unity3D generates scene configurations, and question-answer pairs are created from $T$.}
    \label{fig:generate_pipeline_image}
\end{figure*}



\section{FoREST Dataset Construction}\label{sec:DatasetCreation}

%%% Motivation 
%%% Goal and the task  
% Our paper aims

To systematically evaluate LLM on the frame of reference (FoR) recognition, 
we introduce the \textbf{F}rame \textbf{o}f \textbf{R}eference \textbf{E}valuation in \textbf{S}patial Reasoning \textbf{T}asks (FoREST) benchmark.
Each instance in FoREST consists of a spatial context ($T$), a set of corresponding FoR ($FoR$) which is a subset of \{\textit{external relative},  \textit{external intrinsic}, \textit{internal intrinsic}, \textit{internal relative}\}, a set of questions and answers ($\{Q,A\}$), and a set of visualizations ($\{I\}$).
An example of $T$ is \textit{A cat is to the right of a dog. A dog is facing toward the camera.}
The FoR of this expression is \{\textit{external intrinsic}, \textit{external relative}\}.
A possible question-answer is $Q$ = \textit{Based on the camera's perspective, where is the cat from the dog's position?}, $A$ = \{left, right\}. There is an ambiguity in the FoR for this expression.
Thus, the answer will be \textit{left} if the model assumes the external relative. Conversely, it will be \textit{right} if the model assumes the external intrinsic.
The visualization of this example is in Figure~\ref{fig:generate_pipeline_image}. 
%In the following, we explain how these dataset components are generated automatically.

\subsection{Context Generation}
We select two distinct objects—a relatum ($R$) and a locatum ($L$)—from a set of 20 objects and apply them to a Spatial Relation template,
\textit{<$L$> <spatial relation> <$R$>} to generate the context $T$.
FoRs for $T$ are determined based on the properties of the selected objects. Depending on the number of possible FoRs, $T$ is categorized as ambiguous (A-split), where multiple FoRs apply, or clear (C-split), where only one FoR is valid. 
We further augment the C-split with disambiguated spatial expressions derived from the A-split, as shown in Figure~\ref{fig:generate_pipeline_image}.
%\pk{The next section details the considered properties, possible relatum cases, and the clarification process.: Remove}

\subsection{Categories based on Relatum Properties} \label{sec:FoR_Relatum_scenario}
Using the FoR classes in Section~\ref{sec:primitives}, we identified two key relatum properties contributing to FoR ambiguity.
The first property is the relatum's intrinsic direction. 
It creates ambiguity between intrinsic and relative FoR since spatial relations can originate from the relatum's and observer's perspectives.
The second is the relatum's affordance as a container. 
It introduces the ambiguity between internal and external FoR, as spatial relations can refer to the inside and outside of the relatum. 
Based on these properties, we define four distinct cases: \textit{Cow Case, Box Case, Car Case, and Pen Case.}
% We extend the previous for categorizing based on the property of the object~\cite

\noindent\textbf{Case 1: Cow Case}.
In this case, the selected relatum has intrinsic directions but does not have the affordance as the container for the locatum.
An obvious example is a cow, which should not be a container but has a front and back.
In such cases, the relatum potentially provides a perspective for spatial relations. 
The applicable FoR classes are $FoR$ = \{\textit{external intrinsic}, \textit{external relative}\}.
We augment the C-split with expressions of this case but include the perspective to resolve their ambiguity.
To specify the perspective, we use predefined templates for augmenting clauses, such as \textit{from \{relatum\}'s perspective }for \textit{external intrinsic} or \textit{from the camera's perspective} for \textit{external relative}. 
For example, if the context is  \textit{A cat is to the right of the cow}, in the A-split. 
The counterparts included in the C-split are \textit{A cat is to the right of the cow from cow's perspective.} for \textit{external intrinsic} and \textit{A cat is to the right of the cow from my perspective.} for \textit{external intrinsic}. 


\noindent\textbf{Case 2: Box Case.} 
The relatum in this category has the property of being a container but lacks intrinsic directions, making the internal FoR applicable. An example is a box. 
The applicable FoR classes are $FoR$ = \{\textit{external relative}, \textit{internal relative}\}.
To include their unambiguous counterparts in the C-split, we specify the topological relation to the expressions, $T$, by adding \textit{inside} for \textit{internal relative} and \textit{outside} for \textit{external relative} cases. 
For example, for the sentence \textit{A cat is to the right of the box.},
the unambiguous $T$ with \textit{internal relative} FoR is \textit{A cat is inside and to the right of the box.} The counterpart for \textit{external relative} is \textit{A cat is outside and to the right of the box.}


\noindent\textbf{Case 3: Car Case.}  
A relatum with an intrinsic direction and container affordance falls into this case, allowing all FoR classes. An obvious example is a car that can be a container with intrinsic directions. The applicable FoR classes are $FoR$ = \{ \textit{external relative},  \textit{external intrinsic}, \textit{internal intrinsic}, \textit{internal relative}\}.
To augment C-split with this case's disambiguated counterparts, we add perspective and topology information similar to the Cow and Box cases.
An example expression for this case is \textit{A person is in front of the car.} 
The four disambiguated counterparts to include in the C-split are \textit{A person is outside and in front of the car from the car itself.} for \textit{external intrinsic}, \textit{A person is outside and in front of the car from the observer.} for \textit{external relative},  \textit{A person is inside and in front of the car from the car itself.} for \textit{internal intrinsic}, and \textit{A person is inside and in front of the car from the observer.} for \textit{internal relative}.

\noindent\textbf{Case 4: Pen Case.} 
In this case, the relatum lacks both the intrinsic direction and the affordance as a container. 
An obvious example is a pen with neither left/right nor the ability to be a container.
Lacking these two properties, the created context has only one applicable FoR, $FoR$ = \{ \textit{external relative}\}.
Therefore, we can categorize this case into both splits without any modification.
An example of such a context is \textit{The book is to the left of a pen.}


\subsection{Context Visualization}\label{sec:context_visualize}
% In our visualization, a complex subset arises when the relatum has an intrinsic direction within the intrinsic FoR. In such cases, the relatum’s orientation can introduce additional complexity to the visualization.
% In intrinsic FoR classes where the relatum has intrinsic direction, the relatum’s orientation can complicate visualization. \pk{mention: we use this complex subset for visualizaion or soemthing like that}
In our visualization, complexity arises when the relatum has an intrinsic direction within the intrinsic FoR, as its orientation can complicate the spatial representation.
For example, for visualizing \textit{A cat is to the right of a dog from the dog's perspective.}, the cat can be placed in different coordinates based on the dog’s orientation.
To address this issue, we add a template sentence for each direction, such as \textit{<relatum> is facing toward the camera}, to specify the relatum's orientation of all applicable $T$ for visualization and QA.
For instance, \textit{A cat is to the left of a dog.} becomes \textit{A cat is to the left of a dog. The dog is facing toward the camera.}
To avoid occlusion issues, we generate visualizations only for external FoRs, as one object may become invisible in internal FoR classes.
We use only expressions in C-split since those have a unique FoR interpretation for visualization. 
We then create a scene configuration by applying a predefined template, as illustrated in Figure~\ref{fig:generate_pipeline_image}.
Images are generated using the Unity 3D simulator~\cite{juliani2020unitygeneralplatformintelligent}, producing four variations per expression $T$ with different backgrounds and object positions. Further details on the simulation process are in Appendix~\ref{appendix:dataset_creation}.


% 
\subsection{Question-Answering Generation}\label{sec:QA_generation}
We generate questions for all generated spatial expressions ($T$). 
Note that we include the relatum orientation for cases where the relatum has an intrinsic direction, as mentioned in the visualization.
Our benchmark includes two types of questions. 
The first type asks for the spatial relation between two given objects from the camera's perspective, following predefined templates such as, \textit{Based on the camera’s perspective, where is the {locatum} relative to the {relatum}’s position?}
Template variations are made based on GPT4o.
The second type of question queries the spatial relation from the relatum’s perspective. 
This question type follows the same templates but replaces the camera with the relatum.
The first type of question is generated for all $T$, while the second type is only generated for $T$ where the relatum has intrinsic direction and a perspective can be defined accordingly.
Question templates are provided in Appendix~\ref{appendix:textual_template}. 
Answers are determined based on the corresponding FoRs, the spatial relation in $T$, and the relatum’s orientation when applicable.



\section{Models and Tasks} 
% We aim to evaluate language models' ability on identify FoR and its impacts on T2I models. 
The FoREST benchmark supports multiple tasks, including FoR identification, Question Answering (QA) that requires FoR comprehension, and Text-to-Image (T2I). This paper focuses on QA and T2I for a deeper evaluation of spatial reasoning. 
FoR identification experiments are provided in Appendix~\ref{appendix:FoRIdentification}.

\subsection{Question-Answering (QA)}\label{sec:QA_explanation}

\noindent\textbf{Task.}
This QA task evaluates LLMs’ ability to adapt contextual perspectives across different FoRs.
Both A and C splits are used in this task. 
The input is the context, consisting of a spatial expression $T$ and relatum orientation, if available, and a question $Q$ that queries the spatial relation from either an observer or the relatum’s perspective. 
The output is a spatial relation $S$, restricted to \{left, right, front, back\}.

% \pk{In the C-split, perspective information—introduced to clarify the A-split and create the C-split—is also included in the input.: not sure what is this. Just mention which slpit you consider for this, is it C , A or both, you have explained all before do not repeat. When you say c-split, it includes the augmented sentences from A-split too, you do not need to mention that unless you want to exclude them.} 



\noindent\textbf{Zero-shot baseline.} 
We call the LLM with instructions, a spatial context, $T$, and a question, $Q$, expecting a spatial relation as the response. 
The prompt instructs the model to answer the question with one of the candidate spatial relations without any explanations.


\noindent\textbf{Few-shot baseline.} 
We create four spatial expressions, each assigned to a single FoR class to prevent bias. Following the steps in Section~\ref{sec:QA_generation}, we generate a corresponding question and answer for each. These serve as examples in our few-shot prompting. The input to the model is instruction, example, spatial context, and the question.

\noindent\textbf{Chain-of-Thought baseline~\citep{wei2023chainofthoughtpromptingelicitsreasoning}.}
To create Chain-of-Thought (CoT) examples, we modify the prompt to require reasoning before answering.
We manually crafted reasoning explanations with the necessary information for each example we used in the few-shot setting.
The input to the model is instruction, CoT example, spatial context, and the question.



% \begin{table*}[t]
%     \tiny
%     \centering
%     \begin{tabular}{|l|c c c|c c c|c|c|c|c c c|c c c|c|}
%     \hline
%     Model & \multicolumn{3}{|c|}{Cow} & \multicolumn{3}{|c|}{Car}  & Box  & Pen & Avg. & \multicolumn{3}{|c|}{Cow}  & \multicolumn{3}{|c|}{Car}  & Avg. \\
%     \hline
    
%     Metric & I \% & R \% & Acc. & I \% & R \% & Acc. & \multicolumn{3}{|c|}{Acc.} & I \% & R \% & Acc. & I \% & R \% & Acc. & Acc. \\ 
%        \hline
%        & \multicolumn{5}{|c|}{Camera's perspective} & \multicolumn{3}{|c|}{Relatum's perspective} \\
%        \hline
%      Llama3-8B (0-shot) & $82.22$  & $79.39$ & $95.00$ & $94.06$ & $83.58$ & $65.24$ & $68.70$ & $65.73$ & $82.22$  & $79.39$ & $95.00$ & $94.06$ & $83.58$ & $65.24$ & $68.70$ & $65.73$ \\
%      Llama3-8B (4-shot) & $82.98$ & $86.07$ & $96.67$ & $92.21$ & $84.79$ & $56.78$ & $62.02$ & $57.53$ \\

%      Llama3-8B (CoT) & $52.06$ & $50.19$ & $58.33$ & $54.92$ & $52.33$ & $50.22$ & $46.56$ & $49.70$ \\
     
%      Llama3-8B (SG + COT)) & $76.30$  & $73.09$ & $66.67$ & $76.23$ & $75.63$ & $75.86$ & $75.95$ & $75.87$ \\
     
%          \hline
%      Llama3-70B (0-shot)& $62.52$ & $65.46$ & $73.33$ & $72.54$ & $64.32$ & $62.14$ & $61.83$ & $62.09$ \\
%      Llama3-70B (4-shot) & $62.23$ & $64.69$ & $85.83$ & $85.45$ & $65.83$ & $57.07$ & $61.83$ & $57.74$ \\
%      Llama3-70B (CoT) & $80.74$ & $79.58$ & $95.83$ & $94.88$ & $82.63$ & $77.15$ & $80.92$ & $77.69$ \\
%      Llama3-70B (SG + COT)& $73.57$ & $74.81$ & $100.00$ & $100.00$ & $77.47$ & $65.68$ & $67.75$ & $65.98$ \\
%      \hline
%      Qwen2-7B (0-shot) & $87.36$ & $89.31$ & $91.67$ & $93.85$ & $88.46$ & $72.34$ & $81.68$ & $73.67$ \\

%      Qwen2-7B (4-shot)& $86.66$ & $79.77$ & $87.50$ & $85.04$ & $85.66$ & $51.90$ & $59.73$ & $53.02$ \\

%      Qwen2-7B (CoT) & $80.83$ & $76.91$ & $91.67$ & $88.93$ & $81.58$ & $58.97$ & $63.55$ & $59.62$ \\
%      Qwen2-7B (SG + COT)) & $90.08$ & $93.32$ & $100.00$ & $98.57$ & $91.72$ & $75.86$ & $81.11$ & $76.60$ \\
%     \hline
%      Qwen2-72B (0-shot) & $95.56$ & $95.04$ & $100.00$ & $100.00$ & $96.13$ & $79.28$ & $83.59$ & $79.89$ \\
%      Qwen2-72B (4-shot) & $84.44$ & $85.50$ & $100.00$ & $100.00$ & $86.78$ & $78.26$ & $86.26$ & $79.40$ \\
%      Qwen2-72B (CoT) & $88.59$ & $83.40$ & $100.00$ & $100.00$ & $89.58$ & $85.46$ & $83.59$ & $85.19$ \\
%       Qwen2-72B (SG + COT)) & $88.97$ & $89.12$ & $100.00$ & $99.80$ & $90.53$ & $85.96$ & $87.40$ & $86.17$ \\ 
%      \hline
%     \end{tabular}
%     \caption{The accuracy of A-split frame of reference question-answering with various LLMs. In the Car and Cow cases, the parenthesis $(x, y)$ represents the ratio of correct answers where the model assumes $x$\% relative FoR and $y$\% intrinsic FoR for ambiguous expression.}
%     \label{tab:A_split-QA}
% \end{table*}

\begin{table*}[t]
    \setlength{\tabcolsep}{1mm}
    \small
    \centering
    \begin{tabular}{|l| c c | c | c c | c| c | c | c || c c | c | c c | c| c |}
    \hline
     & \multicolumn{9}{|c||}{Camera perspective} & \multicolumn{7}{|c|}{Relatum perspective} \\
    \cline{2-17}
     Model & \multicolumn{3}{|c|}{Cow} & \multicolumn{3}{|c|}{Car} & Box & Pen & Avg. &\multicolumn{3}{|c|}{Cow} & \multicolumn{3}{|c|}{Car} & Avg. \\
      \cline{2-17}
       & R\% & I\% & Acc. & R\% & I\% & Acc. & Acc. & Acc. & Acc. &R\% & I\% & Acc. & R\% & I\% & Acc. & Acc. \\ 
        \hline
Llama3-70B (1) & $48.1$ & $\mathbf{51.5}$ & $62.5$ & $\mathbf{58.0}$ & $41.6$ & $65.5$ & $73.3$ & $72.5$ & $64.3$  & $\mathbf{61.0}$ & $38.7$ & $62.1$ & $\mathbf{51.8}$ & $47.9$ & $61.8$ & $62.1$ \\
Llama3-70B (2) & $49.1$ & $\mathbf{50.5}$ & $62.2$ & $\mathbf{52.2}$ & $47.4$ & $64.7$ & $85.8$ & $85.5$ & $65.8$  & $\mathbf{59.6}$ & $40.1$ & $57.1$ & $\mathbf{55.5}$ & $44.2$ & $61.8$ & $57.7$ \\
Llama3-70B (3) & $49.4$ & $\mathbf{50.3}$ & $80.7$ & $49.4$ & $\mathbf{50.3}$ & $79.6$ & $95.8$ & $94.9$ & $82.6$  & $\mathbf{60.8}$ & $39.0$ & $77.2$ & $\mathbf{55.1}$ & $44.6$ & $80.9$ & $77.7$ \\
Llama3-70B (4) & $\mathbf{59.4}$ & $40.2$ & $73.6$ & $\mathbf{57.9}$ & $41.7$ & $74.8$ & $100.0$ & $100.0$ & $77.5$  & $\mathbf{60.6}$ & $39.1$ & $65.7$ & $\mathbf{56.0}$ & $43.7$ & $67.7$ & $66.0$ \\
\hline
Qwen2-72B (1) & $\mathbf{96.6}$ & $2.9$ & $95.6$ & $\mathbf{95.9}$ & $3.6$ & $95.0$ & $100.0$ & $100.0$ & $96.1$  & $8.8$ & $\mathbf{90.6}$ & $79.3$ & $7.8$ & $\mathbf{91.7}$ & $83.6$ & $79.9$ \\
Qwen2-72B (2) & $\mathbf{89.0}$ & $10.5$ & $84.4$ & $\mathbf{85.6}$ & $13.9$ & $85.5$ & $100.0$ & $100.0$ & $86.8$  & $17.7$ & $\mathbf{81.8}$ & $78.3$ & $10.4$ & $\mathbf{89.1}$ & $86.3$ & $79.4$ \\
Qwen2-72B (3) & $\mathbf{67.2}$ & $32.4$ & $88.6$ & $\mathbf{62.0}$ & $37.6$ & $83.4$ & $100.0$ & $100.0$ & $89.6$  & $21.3$ & $\mathbf{78.3}$ & $85.5$ & $22.7$ & $\mathbf{76.9}$ & $83.6$ & $85.2$ \\
Qwen2-72B (4) & $\mathbf{93.0}$ & $6.5$ & $90.1$ & $\mathbf{94.6}$ & $4.9$ & $93.3$ & $100.0$ & $98.6$ & $91.7$  & $8.2$ & $\mathbf{91.2}$ & $86.0$ & $10.5$ & $\mathbf{89.0}$ & $87.4$ & $86.2$ \\
\hline
GPT-4o (1) & $\mathbf{84.3}$ & $15.3$ & $94.5$ & $\mathbf{88.5}$ & $11.0$ & $97.3$ & $99.2$ & $99.8$ & $95.6$  & $21.6$ & $\mathbf{78.0}$ & $91.6$ & $16.1$ & $\mathbf{83.5}$ & $90.5$ & $91.4$ \\
GPT-4o (2) & $\mathbf{69.0}$ & $30.6$ & $76.6$ & $\mathbf{80.3}$ & $19.2$ & $89.5$ & $100.0$ & $100.0$ & $81.5$  & $29.0$ & $\mathbf{70.5}$ & $74.7$ & $30.9$ & $\mathbf{68.7}$ & $77.5$ & $75.1$ \\
GPT-4o (3) & $41.5$ & $\mathbf{58.3}$ & $92.3$ & $38.2$ & $\mathbf{61.6}$ & $91.0$ & $100.0$ & $99.8$ & $93.2$  & $33.9$ & $\mathbf{65.8}$ & $93.9$ & $32.0$ & $\mathbf{67.6}$ & $93.9$ & $93.9$ \\
GPT-4o(4) & $26.0$ & $\mathbf{73.9}$ & $79.2$ & $27.7$ & $\mathbf{72.1}$ & $79.4$ & $96.7$ & $94.3$ & $81.4$  & $16.2$ & $\mathbf{83.4}$ & $95.5$ & $19.2$ & $\mathbf{80.4}$ & $94.8$ & $95.4$ \\
\hline
    \end{tabular}
    \caption{QA accuracy in the A-Split across various LLMs. R\% and I\% represent the percentage the model assumes relative or intrinsic FoR for ambiguous expression explained in Section~\ref{sec:evaluation_setting}. Acc is the accuracy, and Avg is the micro-average of accuracy. (1): 0-shot, (2): 4-shot, (3): CoT, and (4): SG + CoT.}
    \label{tab:A_split-QA}
\end{table*}

% \vspace{-5mm}
\subsection{Text-To-Image (T2I)}\label{sec:t2i_models}

\noindent\textbf{Task.}  This task aims to determine the diffusion models' ability to consider FoR by evaluating their generated images. The input is a spatial expression, $T$, and the output is a generated image ($I$). We use the context from both C and A splits with external FoRs for this task.

\noindent\textbf{Stable Diffusion Models.} 
We use the stable diffusion models as the baseline for the T2I task. 
This model only needs the scene description as input. 
% \pk{Therefore, we provide $T$ to the model and expect an output image of $I$.: not needed, you can remove.}

\noindent\textbf{Layout Diffusion Models.}
We evaluate the Layout Diffusion model, a more advanced T2I model operating in two phases: text-to-layout and layout-to-image.
Given that LLMs can generate the bounding box layout~\citep{cho2023visualprogrammingtexttoimagegeneration}, we provide them with instructions and $T$ to create the layout. 
The layout consists of bounding box coordinates for each object in the format of \{object: $[x, y, w, h]$\}, where $x$ and $y$ denote the starting point and $h$ and $w$ denote the height and width. 
The bounding box coordinates and $T$ are then passed to the layout-to-image model to produce the final image, $I$. 


\subsection{Spatial-Guide Prompting}\label{sec:SG_prompting}
We hypothesize that the spatial relation types and FoR classes explained in Section~\ref{sec:primitives} can improve question-answering and layout generation.
For instance, the \textit{external intrinsic} FoR emphasizes that spatial relations originate from the relatum’s perspective.
To leverage this, we propose Spatial-Guided (SG) prompting, an additional step applied before QA or layout generation steps.
This step extracts spatial information, including direction, topology, distance as well as the FoR from spatial expression $T$. The extracted information will serve as supplementary for guiding LLMs in QA and layout generation.
We manually craft four examples covering these aspects.
First, we specify the perspective for \textit{directional relations}, e.g., \textit{left} relative to the observer, to distinguish intrinsic from relative FoR.
Next, we indicate whether the locatum is inside or outside the relatum for \textit{topological relations} to differentiate internal from external FoR.
Lastly, we provide an estimated quantitative distance to support topological and directional relation identification, such as \textit{far}.
These examples are then provided as a few-shot example for the model to extracted information automatically.
% SG prompting follows the few-shot setting that
% \pk{We provide prompting, spatial context, and SG examples for LLMs to extract spatial relations and FoR.: not clear, SG was itself the extracted information, or I am confused here.}
%  SG is the extracted information, this line only tell how to call model to generate SG


\section{Experimental Results}
\begin{table*}[ht!]
    \small
    \setlength{\tabcolsep}{1mm}
    \centering
    \begin{tabular}{|l|c|c|c|c|c||c|c|c|c|c|}
    \hline
     & \multicolumn{5}{|c||}{Camera perspective} & \multicolumn{5}{|c|}{Relatum perspective} \\
    \cline{2 - 11}
    Model & ER (CP) & EI (RP) & II (RP) & IR (CP) & Avg. & ER (CP) & EI (RP) & II (RP) & IR (CP) & Avg. \\
    \hline
     Llama3-70B (0-shot) & $44.8$ & $38.4$ & $39.7$ & $54.4$ & $42.6$ &$42.2$ & $47.1$ & $62.5$ & $34.4$ & $45.1$ \\
     Llama3-70B (4-shot) & $43.0$ & $40.0$ & $39.1$ & $47.3$ & $41.9$ & $41.8$ & $60.9$ & $77.7$ & $35.2$ & $52.0$ \\
     Llama3-70B (CoT) & $57.8$ & $46.1$ & $44.7$ & $46.0$ & $51.5$ & $\mathbf{55.5}$ & $56.8$ & $71.5$ & $49.0$ & $56.6$ \\
     Llama3-70B (SG + CoT) & $47.6$ & $42.9$ & $50.0$ & $35.6$ & $45.0$ &$55.4$ & $64.5$ & $75.0$ & $47.1$ & $60.1$ \\
     \hline
     Qwen2-72B (0-shot) & $94.5$ & $35.2$ & $31.8$ & $93.2$ & $66.9$ & $28.7$ & $89.3$ & $93.6$ & $23.8$ & $59.0$ \\
     Qwen2-72B (4-shot) & $90.2$ & $39.5$ & $39.1$ & $68.5$ & $65.3$ & $33.5$ & $92.1$ & $94.0$ & $29.5$ & $62.7$ \\
     Qwen2-72B (CoT) & $81.4$ & $57.4$ & $58.6$ & $62.5$ & $69.1$ & $39.5$ & $83.7$ & $85.2$ & $37.7$ & $61.6$ \\
     Qwen2-72B (SG + CoT) & $97.6$ & $42.5$ & $31.3$ & $93.8$ & $71.4$ & $42.8$ & $86.6$ & $92.0$ & $34.0$ & $64.5$ \\
     \hline
    GPT-4o (0-shot)  & $79.7$ & $45.1$ & $39.5$ & $90.2$ & $64.2$  & $46.9$ & $88.5$ & $98.2$ & $34.8$ & $67.5$ \\
    GPT-4o (4-shot) & $68.0$ & $52.6$ & $60.7$ & $74.1$ & $61.8$  & $44.9$ & $\mathbf{98.2}$ & $\mathbf{100.0}$ & $37.5$ & $71.2$ \\
    GPT-4o (CoT) & $81.7$ & $\mathbf{76.1}$ & $\mathbf{82.4}$ & $71.5$ & $78.8$  & $53.0$ & $91.1$ & $90.6$ & $\mathbf{50.8}$ & $71.9$ \\
    GPT-4o (SG + CoT)  & $\mathbf{97.9}$ & $72.2$ & $72.7$ & $\mathbf{93.4}$ & $\mathbf{85.8}$  & $48.9$ & $96.3$ & $95.9$ & $36.1$ & $\mathbf{71.8}$ \\
\hline
    \end{tabular}
    \caption{QA accuracy in the C-Split across various LLMs. ER, EI, II, and IR  denote external relative, external intrinsic, internal intrinsic, and internal relative FoRs. Avg represents the micro-average accuracy. CP refers to context with camera perspective, while RP denotes context with relatum perspective.}
    \label{tab:QA_c_split}
\end{table*}

\subsection{Evaluation Metrics}\label{sec:evaluation_setting}
\noindent\textbf{QA.}  We report an accuracy measure defined as follows. Since the questions can have multiple correct answers, specifically in A-split, as explained in Section~\ref{sec:DatasetCreation}, the prediction is correct if it matches any valid answer.
Additionally, we report the model’s bias distribution when FoR ambiguity exists.
$I$\% is the percentage of correct answers when assuming an intrinsic FoR, while $R$\% is this percentage with a relative FoR assumption. 
Note that cases where both FoR assumptions lead to the same answer are excluded from these calculations.

\noindent\textbf{T2I.} 
We adopt  \textit{spatialEval}~\citep{cho2023visualprogrammingtexttoimagegeneration} approach for evaluating T2I spatial ability. However, we modify it to account for FoR.
We convert all relations to a camera perspective before passing them to spatialEval, which assumes this viewpoint.
Accuracy is determined by comparing the bounding box and depth map of the relatum and locatum. 
For FoR ambiguity, a generated image is correct if it aligns with at least one valid FoR interpretation.
We report results using VISOR$_{cond}$ and VISOR$_{uncond}$~\citep{gokhale2023benchmarkingspatialrelationshipstexttoimage}, metrics for assessing T2I spatial understanding.
VISOR$_{cond}$ evaluates spatial relations only when both objects appear correctly, aligning with our focus on spatial reasoning rather than object creation. In contrast, VISOR$_{uncond}$ evaluates the overall performance, including object creation errors.

\subsection{Experimental Setting}
\noindent\textbf{QA.} We use Llama3-70B~\citep{dubey2024Llama3herdmodels}, Qwen2-72B~\citep{qwen2model}, and GPT-4o (\textit{gpt-4o-2024-11-20})~\citep{openai2024gpt4technicalreport} as the backbones for prompt engineering. 
To ensure reproducibility, we set the temperature of all models to 0.
For all models, we apply \textit{zero-shot}, \textit{few-shot}, \textit{CoT}, and our proposed prompting with CoT (SG+CoT).
%The detail of SG prompting is in Section~\ref{sec:SG_prompting}. 
%The creation of examples used for ICL is detailed in Section~\ref{sec:QA_explanation}.

\noindent\textbf{T2I.}
We select Stable Diffusion SD-1.5 and SD-2.1~\citep{rombach2021highresolution} as our stable diffusion models and GLIGEN\citep{li2023gligenopensetgroundedtexttoimage} as the layout-to-image backbone.
For translating spatial descriptions into textual bounding box information, we use Llama3-8B and Llama3-70B, as detailed in Section~\ref{sec:t2i_models}. 
The same LLMs are used to generate spatial information for SG prompting. 
We generate four images to compute the VISOR score following~\cite{gokhale2023benchmarkingspatialrelationshipstexttoimage}
Inference steps for all T2I models are set to 50.
For the evaluation modules, we select grounding DINO~\citep{liu2024groundingdinomarryingdino} for object detection and DPT~\citep{ranftl2021visiontransformersdenseprediction} for depth mapping, following VPEval~\cite{cho2023visualprogrammingtexttoimagegeneration}. The experiments were conducted on an $A6000$ GPU, totaling approximately $300$ GPU hours.

\subsection{Results}

\noindent\textbf{RQ1. What is the bias of the LLMs for the ambiguous FoR? }
Table~\ref{tab:A_split-QA} presents the QA results for the A-split. 
Ideally, a model that correctly extracts the spatial relation without considering perspective should achieve 100\% accuracy, as the context lacks a fixed perspective. 
However, this ideal model is not the focus of our work. We aim to assess model bias by measuring how often LLMs adopt a specific perspective when answering.
In the Cow and Pen case, relatum properties do not introduce FoR ambiguity in directional relations, making the task pure extraction rather than reasoning.
Thus, we focus on the $I$\% and $R$\% of the Cow and Car cases, which best reflect LLMs’ bias.
Qwen2 achieves around 80\% accuracy across all experiments by selecting spatial expressions directly from the context, suggesting it may disregard the question’s perspective.
GPT-4o shows similar bias in 0-shot and 4-shot settings but shifts toward intrinsic interpretation with CoT. This bias reduces accuracy in camera-perspective questions from 93.2\% to 81.4\%, where FoR adaptation is more challenging than relation extraction. 
Llama3-70B lacks a strong preference, balancing assumptions but slightly favoring relative FoR. This uncertainty lowers performance, requiring more reasoning to reach the correct answer.
In summary, Qwen2 achieves higher accuracy by focusing on relation extraction without considering FoR reasoning, while other models attempt reasoning but struggle to reach correct conclusions, leading to lower performance.


\begin{table*}[ht!]
    \centering
    \setlength{\tabcolsep}{1mm}
    \small
    \begin{tabular}{|l | c c c | c | c | c | c | c | c |}
    \hline
         & \multicolumn{8}{c|}{VISOR(\%)} \\ \cline{2-9}
          & \multicolumn{5}{|c|}{ A-Split } &  \multicolumn{3}{|c|}{ C-Split }\\ \hline
        Model & \multicolumn{3}{|c|}{cond (I)} & cond (R) & cond (avg) & cond (I) & cond (R) & cond (avg) \\ \cline{2-4}
        & EI FoR & ER FoR & all & & & & & \\ \hline
        SD-1.5   & $ 51.11$  & $ 21.61$  &  $ 72.72$ & $ 48.95$ & $ 68.72$ & $ 53.92$ & $ 53.77$ & $ 53.83$ \\
        SD-2.1  & $ 57.97$  & $ 21.49$ &  $ 79.46$ & $ 54.10$ & $ 75.39$ & $\mathbf{60.06}$ & $ 59.64$ & $ 59.83$ \\
        \hline
        Llama3-8B + GLIGEN& $ 53.67$  & $ 25.78$ & $ 79.45$ & $ 66.08$ & $ 77.38$ & $ 57.51$ & $ 65.98$ & $ 62.12$ \\
        Llama3-70B + GLIGEN & $ 54.49$  & $ 29.45$ & $ 83.94$ & $ 68.68$ & $ 81.43$ & $ 56.47$ & $ 69.53$ & $ 63.49$ \\
        Llama3-8B + SG + GLIGEN (Our) & $ 57.46$  & $ 27.96$ & $ 85.42$ & $\mathbf{71.14}$ & $ 83.17$ & $ 58.84$ & $\mathbf{70.36}$ & $ \mathbf{65.15}$ \\
        Llama3-70B + SG + GLIGEN (Our)  & $ 56.54$  & $ 30.59$ & $ \mathbf{87.13}$ & $ 66.56$ & $\mathbf{83.75}$ & $ 56.77$ & ${70.04}$ & ${64.06}$ \\
        \hline
    \end{tabular}
    \caption{VISOR$_{cond}$ score on the A and C splits where $I$ refer to the Cow case and Car case where relatum has intrinsic directions, and $R$ refer to the Box case and Pen case where relatum lacks intrinsic directions, $avg$ is mirco-average of $I$ and $R$. cond are explained in Section~\ref{sec:evaluation_setting}. EI and ER FoR represent the generated image considered corrected by EI or ER FoR }
    \label{tab:I_split}
\end{table*}

\begin{figure}[t]
    \centering
    \includegraphics[width=0.9\linewidth]{Figures/cf_matrix_change_perspective3.png}
    \caption{Confusion matrices of spatial relation answers when Llama3 and GPT-4o are required to adapt FoR in the 0-shot and (SG+CoT) settings.}
    \label{fig:cf_conversion}
\end{figure}

\noindent\textbf{RQ2. Can the model adapt FoR when answering the questions?}
To address this research question, we analyze QA that required FoR comprehension results in C-Split from Table~\ref{tab:QA_c_split}.
Note that the context and question in these tasks explicitly indicate a perspective.
The results indicate that LLMs struggle with FoR conversion, particularly when the question has relatum and the context has camera perspectives, achieving only up to 55.5\% accuracy.
We further demonstrate how Llama3 and GPT-4o adapt FoR using the confusion matrix in Figure~\ref{fig:cf_conversion}. 
Our findings reveal that pure-text LLM (Llama3) has confusion between left and right.
Humans typically reverse front and back while preserving left and right when describing the spatial relation from perspective.
However, Llama3 incorrectly reverses left and right, leading to poor adaptation to the camera perspective.
In contrast, very large multimodal-language models like GPT-4o follow the expected pattern, as observed by~\citealt{comfortFoR}.
While our GPT-4o results suggest some ability to convert the relatum’s perspective into the camera’s with in-context learning (72\% accuracy), the reverse transformation in the textual domain remains challenging (53\% accuracy). 
This difficulty persists when converting spatial relations from the camera perspective from images to the relatum’s perspective as observed in~\citealt{comfortFoR}.

\noindent\textbf{RQ3. How can an explicit FoR identification help spatial reasoning in QA?}
We compare CoT and CoT+SG results to evaluate how explicit FoR identification affects LLMs’ spatial reasoning in QA.
Based on C-Split results (Table~\ref{tab:I_split}), incorporating SG encourages the model to identify the perspective from input expression ranging from 2.9\% to 30\% in cases where the context and question share the same perspective. 
These cases are easier as the models do not need FoR adaptation. 
%The only exception is Llama3 for questions with the camera’s perspective, where explicit FoR identification with SG prompting negatively impacts performance.
With one exception, as can be seen, the performance level of the Llama3 baseline, for questions with camera perspective, is so poor that FoR identification in SG cannot help boost its performance compared to the improvements made in other language models.
We should note that among our selected LLMs, Llama3 is the only one not trained with visual information; we speculate this can be a factor in LLMs' understanding of FoR.
SG is less effective in reasoning when the context and question have different perspectives despite aiding in identifying the correct FoR of the spatial description in the context. 
% but does not improve reasoning for converting spatial relations across perspectives.
This limitation is evident in A-Split results (Table~\ref{tab:A_split-QA}), where models only show significant improvement when SG aligns their preference with the question’s perspective, as seen in Qwen2-72B and GPT-4o.
SG identification results are reported in the Appendix~\ref{appendix:FoRIdentification}.
Still, incorporating FoR identification improves overall spatial reasoning (see the Avg column for SG+CoT in Table~\ref{tab:I_split}).

\noindent\textbf{RQ4. How can explicit FoR identification help spatial reasoning in visualization?}
We evaluate SG layout diffusion to assess the impact of incorporating FoR in image generation. 
We focus on VISOR$_{cond}$ metric, as it better reflects the model’s spatial understanding than the overall performance measured by VISOR$_{uncond}$, which is reported in Appendix~\ref{appendix:Visor_uncond} due to space limitation.
Table~\ref{tab:I_split} shows that adding spatial information and FoR classes (SG+GLIGEN) improves performance across all splits compared to the baseline models (GLIGEN).
SG improved the model's performance when expressions can be interpreted as relative FoR.
These results align with the QA results shown in Table~\ref{tab:A_split-QA} indicating that \textit{Llama3  prefers relative FoR if dealing with the camera's perspective}.
In contrast, baseline diffusion models (SD-1.5 and SD-2.1) perform better for intrinsic FoR even though GLIGEN is based on SD-2.1.
This outcome might be due to GLIGEN's reliance on bounding boxes for generating spatial configurations, which makes it struggle with intrinsic FoR due to the absence of object properties and orientation. Nevertheless, incorporating FoR information via SG-prompting improves performance across all FoR classes despite this specific bias.
We provide further analysis on SG for the layout generation in Appendix~\ref{appedix:anaylize_SG_improment_t2i}.

%%%%%%%%%%%%%%%%*******Related Works

\section{Related Work}
\noindent\textbf{Frame of Reference in Cognitive Studies.}
The concept of the frame of reference in cognitive studies was introduced by \citealt{Levinson_2003} and later expanded with more diverse spatial relations \citep{TENBRINK2011704}.
Subsequent research investigated the human preferences for specific FoR classes~\citep{Edmonds-Wathen852956, VUKOVIC2015110, SHUSTERMAN2016115, Ruotolo2016}. For instance, \citealt{Ruotolo2016} examined how FoR influences scene memorization and description under time constraints. Their study found that participants performed better when spatial relations were based on their position rather than external objects, highlighting a distinction between relative and intrinsic FoR.

\noindent\textbf{Frame of Reference in AI.}
Several benchmarks have been developed to evaluate the spatial understanding of AI models in multiple modalities; for instance, %navigation~\citep{yamada2024evaluatingspatialunderstandinglarge}, 
textual QA~\citep{shi2022stepgamenewbenchmarkrobust, mirzaee2022transferlearningsyntheticcorpora, rizvi2024sparcsparpspatialreasoning}, and text-to-image (T2I) benchmarks~\citep{gokhale2023benchmarkingspatialrelationshipstexttoimage, huang2023t2icompbenchcomprehensivebenchmarkopenworld, cho2023dallevalprobingreasoningskills, cho2023visualprogrammingtexttoimagegeneration}.
However, most of these benchmarks overlook the frame of reference (FoR), assuming a single FoR for all instances despite its significance in cognitive studies.
Recent works in vision-language research are beginning to address this problem. 
For instance, \citealt{liu2023visualspatialreasoning} examines FoR’s impact on visual question-answering but focuses only on limited FoR categories. Our work covers more diverse FoRs.
\citealt{comfortFoR} examine FoR ambiguity and understanding in vision-language models by evaluating spatial relations derived from visual input under different FoR questions. 
Their approach relies on images from camera perspectives, with FoR indicated in the question.
In contrast, our work focuses on the reasoning of spatial relations when dealing with multiple FoRs and when there are changes in perspective in explaining the context beyond the camera’s viewpoint. 
Additionally, we show that explicitly identifying the FoR helps improve spatial reasoning in both question-answering and text-to-image generation, particularly when involving multiple perspectives.
% \tp{
% \citealt{comfortFoR} examine FoR ambiguity and understanding in vision-language models by evaluating spatial relations derived solely from visual input under different FoR questions. Their approach relies on images from camera perspectives, with FoR introduced through the question.
% In contrast, our work focuses on the role of FoR in spatial relations, analyzing how LLMs interpret different FoR classes by incorporating various perspectives beyond the camera’s viewpoint.
% Additionally, we assess the broader impact of FoR in spatial reasoning through spatial expression beyond question-answering tasks, including text-to-image generation. 
% Lastly, we demonstrate that explicitly identifying FoR can enhance spatial reasoning tasks, particularly when involving multiple perspectives.
% }


\section{Conclusion}
Given the significance of spatial reasoning in AI applications, 
we introduce \textbf{F}rame \textbf{o}f \textbf{R}eference \textbf{E}valuation in \textbf{S}patial Reasoning \textbf{T}asks (FoREST) benchmark to evaluate Frame of Reference comprehension in textual spatial expressions via question-answering and grounding in visual modality by diffusion models.
Based on this benchmark, our results reveal notable differences in FoR comprehension across LLMs and their struggle with questions that require adaptation between multiple FoRs.
Moreover, the bias in FoR interpretations impacts the layout generation with LLMs for text-to-image models. 
To improve FoR comprehension, we propose Spatial-Guided prompting, which first generates spatial relation's topological, distal, and directional type information in addition to FoR and includes this information in downstream task prompting.
Employing SG improves the overall performance in QA tasks requiring FoR understanding and text-to-image generation.

\section{Limitations}
While we analyze LLMs' shortcomings, our benchmark only highlights areas for improvement, not harming the model.
The trustworthiness and reliability of the LLMs are still a research challenge.
Our analysis is confined to the spatial reasoning domain and does not account for biases related to gender or race. However, we acknowledge that linguistic and cultural variations in spatial expression are not considered, as our study focuses solely on English. Extending this work to multiple languages could reveal important differences in FoR adaptation. 
Our analysis is still limited to the synthetic environment. Future research should consider the broader implications of the frame of reference of spatial reasoning in real-world applications.
Additionally, our experiments require substantial GPU resources, limiting the selection of LLMs and constraining the feasibility of testing larger models. The computational demands also pose accessibility challenges for researchers with limited resources.
We find no ethical concerns in our methodology or results, as our study does not involve human subjects or sensitive data. 



% Entries for the entire Anthology, followed by custom entries
\bibliography{ref}
% \bibliographystyle{acl_natbib}

\appendix
\section{Dataset Statistics}\label{apppendix:statistic}
The FoREST dataset statistic is provided in the Table~\ref{tab:data_statistic}.

\begin{table*}[t!]
    \centering
    \small
    \begin{tabular}{|c|c|c||c|c|c|}
        \hline
         Case & A-Split & A-Split for T2I & FoR class & C-Spilt & C-split for T2I\\
         \hline
         Cow Case & 792 & 3168 & External Relative & 1528 & 4288\\
         Box Case & 120 & 120 & External Intrinsic & 920 & 3680\\
         Car Case & 128 & 512 & Internal Intrinsic & 128 & 0\\
         Pen Case & 488 & 488 & Internal Relative & 248 & 0\\
         \hline
         Total & 1528 & 4288 & Total & 2824 & 7968 \\
         \hline
    \end{tabular}
    \caption{Dataset Statistic of FoREST dataset. }
    \label{tab:data_statistic}
\end{table*}


\section{Details Creation of FoREST dataset}\label{appendix:dataset_creation}
We define the nine categories of objects selected in our dataset as indicated below in Table~\ref{tab:selected object}. We select sets of locatum and relatum based on the properties of each class to cover four cases of frame of reference defined in Section~\ref{sec:FoR_Relatum_scenario}. Notice that we also consider the appropriateness of the container; for example, the car should not contain the bus.

Based on the selected locatum and relatum. To create an A-split spatial expression, we substitute the actual locatum and relatum objects in the Spatial Relation template. After obtaining the A-split contexts, we create their counterparts using the perspective/topology clauses to make the counterparts in C-spilt. Then, we obtain the I-A and I-C split by applying the directional template to the first occurrence of relatum when it has intrinsic directions. The directional templates are "that is facing towards," "that is facing backward," "that is facing to the left," and "that is facing to the right." All the templates are in the Table~\ref{tab:templates}.
We then construct the scene configuration from each modified spatial expression and send it to the simulator developed using Unity3D. 
Eventually, the simulator produces four visualization images for each scene configuration. 

\begin{table*}[t]
    \centering
    \small
    \begin{tabular}{|c|c|c|c|}
    \hline
        Category &  Object & Intrinsic Direction & Container \\
        \hline
        small object without intrinsic directions & umbrella, bag, suitcase, fire hydrant & \xmark & \xmark \\
        \hline
        bog object with intrinsic directions & bench, chair & \checkmark &  \xmark \\ 
        \hline
        big object without intrinsic direction & water tank & \xmark & \xmark \\
        \hline
        container & box, container & \xmark & \checkmark \\
        \hline
        small animal & chicken, dog, cat & \checkmark & \xmark \\
        \hline
        big animal & deer, horse, cow, sheep & \checkmark & \xmark \\
        \hline
        small vehicle & bicycle & \checkmark & \xmark \\
        \hline
        big vehicle & bus, car & \checkmark & \checkmark \\
        \hline
        tree & tree & \xmark & \xmark \\
        \hline
    \end{tabular}
    \caption{All selected objects with two properties: intrinsic direction, affordance of being container}
    \label{tab:selected object}
\end{table*}

\subsection{Simulation Details}
The simulation starts with randomly placing the relatum into the scene with the orientation based on the given scene configuration. 
We randomly select the orientation by given scene configuration, [-40, 40] for front, [40, 140] for left, [140, 220] for back, and [220, 320] for right. Then, we create the locatum from the relatum position and move it in the spatial relation provided. If the frame of reference is relative, we move the locatum based on the camera's orientation. Otherwise, we move it from the relatum's orientation. Then, we check the camera's visibility of both objects. If one of them is not visible, we repeat the process of generating the relatum until the correct placement is achieved. After getting the proper placement, we randomly choose the background from 6 backgrounds. Eventually, we repeat the procedures four times for one configuration. 

\subsection{Object Models and Background}
For the object models and background, we find it from the unity assert store\footnote{https://assetstore.unity.com}. All of them are free and available for download. All of the 3D models used are shown in Figure~\ref{fig:3D_model}.

\begin{figure*}[t]
    \centering
    \includegraphics[width=\linewidth]{Figures/all_3d_model.png}
    \caption{All 3d models used to generate visualizations for FoREST.}
    \label{fig:3D_model}
\end{figure*}

\begin{table*}[t]
    \centering
    \tiny
    \begin{tabular}{|c|c|} 
    \hline
        &   \{locatum\} is in front of \{relatum\} \\
         &  \{locatum\} is on the left of \{relatum\} \\
           &  \{locatum\} is to the left of \{relatum\}\\ 
        Spatial Relation Templates &  \{locatum\} is behind of \{relatum\} \\
           &  \{locatum\} is back of \{relatum\} \\
          &  \{locatum\} is on the right of \{relatum\} \\
          &   \{locatum\} is to the right of \{relatum\} \\
         \hline
         &  within \{relatum\} \\ 
         Topology Templates &  and inside \{relatum\} \\ 
         &  and outside of \{relatum\} \\ 
         \hline
         & from \{relatum\}'s view \\ 
         & relative to \{relatum\} \\ 
         Perspective Templates & from \{relatum\}'s perspective\\ 
         & from my perspective \\ 
          & from my point of view \\ 
           & relative to observer \\ 
         \hline
          &  \{relatum\} facing toward that camera\\ 
         Orientation Templates &  \{relatum\}is facing away from the camera. \\ 
          &  \{relatum\} facing left relative to the camera\\ 
           &  \{relatum\} facing right relative to the camera \\ 
         \hline
         & In the camera view, how is \{locatum\} positioned in relation to \{relatum\}? \\
        &  Based on the camera perspective, where is the \{locatum\} from the \{relatum\}'s position? \\ 
         Question Templates &  From the camera perspective, what is the relation of the\{locatum\} to the \{relatum\}?\\ 
          &  Looking through the camera perspective, how does \{locatum\} appear to be oriented relative to \{relatum\}'s position?\\ 
           &  Based on the camera angle, where is \{locatum\} located with respect to \{relatum\}'s location? \\ 
         \hline
    \end{tabular}
    \caption{All templates used to create FoREST dataset.}
    \label{tab:templates}
\end{table*}


\subsection{Textual templates}\label{appendix:textual_template}
All the templates used to create FoREST are given in Table~\ref{tab:templates}.



\section{VISOR {uncond} Score}\label{appendix:Visor_uncond}

VISOR$_{uncond}$ provides the overall spatial relation score, including images with object generation errors. Since it is less focused on evaluating spatial interpretation than VISOR$_{cond}$, which assesses explicitly the text-to-image model's spatial reasoning, we report VISOR$_{uncond}$ results here in the Table~\ref{tab:VISOR_uncode} rather than in the main paper. The results are similar to the pattern observed in VISOR$_{uncond}$ that the based models(SD-1.5 and SD-2.1) perform better in the relative frame of reference, while the layout-to-image models, i.e., GLIGEN, are better in the intrinsic frame of reference.

\begin{table*}[t!]
    \centering
    \small
    % \begin{adjustbox}{width=\columnwidth -0mm, center}
    \begin{tabular}{|l | c c | c | c c | c |}
    \hline
         & \multicolumn{6}{c|}{VISOR(\%)} \\ \cline{2-7}
        Model & uncond (I) & uncond (R) & uncond (avg) & uncond (I) & uncond (R) & uncond (avg) \\
        \hline
         & \multicolumn{3}{|c|}{ A-Split } & \multicolumn{3}{|c|}{ C-Split }  \\ \hline
        SD-1.5 & $ 45.43$  & $33.22$ & $ 43.51$ &  $ 35.06$  & $ 35.68$ & $ 35.40$ \\
        SD-2.1 & $\mathbf{62.87}$  & $ 43.90$ & $\mathbf{59.89}$ & $\mathbf{45.98}$  & $ 46.59$ & $\mathbf{46.31}$ \\
        \hline
        Llama3-8B + GLIGEN & $ 46.74$  & $ 38.16$ & $ 45.39$ & $ 33.98$  & $ 39.36$ & $ 36.89$ \\
        Llama3-70B + GLIGEN & $ 54.33$  & $ 46.89$ & $ 53.17$ & $ 38.04$  & $ 46.04$ & $ 42.37$ \\
        Llama3-8B + SG + GLIGEN (Our) & $ 51.83$  & $ 43.24$ & $ 50.48$ & $ 36.28$  & $ 44.43$ & $ 40.70$ \\
        Llama3-70B + SG + GLIGEN (Our) & $ 58.92$  & $\mathbf{47.44}$ & $ 57.12$ & $ 38.23$  & $\mathbf{48.62}$ & $ 43.86$ \\
        \hline
    \end{tabular}
    \caption{VISOR$_{uncond}$ score on the A-Split and C-Split where $I$ refer to the Cow Case and Car Case where relatum has intrinsic directions, and $R$ refer to the Box Case and Pen case where relatum lacks intrinsic directions, $avg$  is mirco-average of $I$ and $R$. cond and uncond are explained in Section~\ref{sec:evaluation_setting}.}
    \label{tab:VISOR_uncode}
\end{table*}


\section{Analyze the improvements in SG-prompting for T2I.} \label{appedix:anaylize_SG_improment_t2i}
\begin{table}[t]
    \small
    \centering
    \begin{tabular}{|l|c|c|c|}
         \hline
         Model & Layout & Layout$_{cond}$  \\
         \hline
          Llama3-8B  & $85.26$ & $88.84$\\
          Llama3-8B + SG  & $85.04$ & $88.86$  \\
          Llama3-70B  & $88.47$ & $93.16$ \\
          Llama3-70B + SG & $91.95$ & $95.45$ \\
          \hline
    \end{tabular}
    \caption{Layout accuracy where spatial relations are left or right relative to the camera. Layout is evaluated for all generated layouts in I-C split while Layout$_{cond}$ uses the same testing examples as VISOR$_{cond}$.}
    \label{tab:layout_results}
\end{table}


To further explain improvements of SG-prompting in T2I task, we assess the generated bounding boxes
in the I-C split for left and right relations relative to the camera since these can be evaluated using only bounding boxes without depth information.
As seen is Table~\ref{tab:layout_results}, our SG prompting improved 
Llama3-70B's by $3.48\%$, while Llama3-8B saw a slight decrease of $0.22\%$. 
This evaluation was conducted on all generated layouts from the I-C split, which differs from the evaluation subset of images used for VISOR$_{cond}$ in Table~\ref{tab:I_split}. 
We report the layout$_{cond}$ score for a consistent evaluation in the same table. Layout$_{cond}$ shows that Llama3-8B improves within the same evaluation subset with VISOR$_{cond}$.
Overall, by incorporating FoR information through SG layout diffusion, Llama3 generates better spatial configurations, enhancing image generation performance.



% \section{\iclr{Qualitative Examples of Failure Cases of CoT and SG}}\label{appendix:false_examples}


% \subsection{CoT Incorrect but Resolved by using SG}

% \begin{itemize}
% \item \textbf{Context} : A cat is inside and on the left of a car from the car's perspective

% \textbf{CoT} : Explanation: Cat is inside of the car, and the car has the intrinsic direction. However, the left of the car is referring to the car's perspective rather than the intrinsic direction of the car. Answer: internal relative.

% \textbf{SG} : Explanation: Topological: cat is inside of the car. Distance: cat should be in the same location with the car from the context. Direction: The car which is the relative object has the direction, and cat is on the left side of the car. Answer: internal intrinsic.

% \textbf{label} : internal intrinsic

% \item \textbf{Context} : A container is on the right of a dog from the dog's perspective

% \textbf{CoT} : Explanation: Based on the context, dog does not contain the container. The dog has the direction; however, the right of a dog referring to the dog's perspective rather than intrinsic direction of the dog. Answer: external relative.

% \textbf{SG} : Explanation: Topological: container is not inside of the dog. Distance: container may has some distance from dog according to the context. Direction: Dog has the direction and the context right relation is referred from dog's perspective that a container is on the right side of the dog's facing orientation. Answer: external intrinsic.

% \textbf{label} : external intrinsic

% \item \textbf{Context} : A water tank is on the left of a cow from the cow's perspective

% \textbf{CoT} : Explanation: Based on the context, cow does not contain the water tank. The cow has the direction; however, the left of a water tank referring to the cow's perspective rather than cow's intrinsic direction. Answer: external relative.

% \textbf{SG} : Explanation: Topological: water tank is not inside of the cow. Distance: water tank may has some distance from cow according to the context. Direction: The cow which is the relative object has the direction, and water tank is on the left side of the cow. Answer: external intrinsic.

% \textbf{label} : external intrinsic

% \item \textbf{Context} : A bicycle is in front of a container inside the container

% \textbf{CoT} : Explanation: The bicycle is contained within the container, fulfilling the first condition for internal frames of reference. However, since the bicycle is also described as being 'in front' of another object (another container), this implies a spatial relationship that is dependent on the container's intrinsic direction. But due to thembiguity of having two containers mentioned, and assuming the second 'container' is a typo or misstatement, we interpret it as a single container scenario. As such, the spatial relation is intrinsic to the container itself, not relative to an external observer or another entity. Answer: internal intrinsic. 

% However, the context provided contains a potential error or ambiguity ("inside the container"), which could alter the interpretation if clarified. Assuming there was meant to be only one container, thebove categorization holds. If there were indeed two distinct containers, further clarification would be needed to accurately categorize the spatial relation.

% \textbf{SG} : Explanation: Topological: bicycle is inside of the container. Distance: bicycle should be in the same location with the container from the context. Direction: Container doesn't have the direction and context refere from observer perspective for bicycle is in the front of the container. Answer: internal relative. However, there seems to be a contradiction in the statement as it says the bicycle is both inside and in front of the container. For the purpose of categorization, we will assume that the intended meaning was that the bicycle is inside the container and positioned towards the front end of it.

% \textbf{label} : internal relative

% \item \textbf{Context} : A tree is on the left of a box inside the box

% \textbf{CoT} : Explanation: Based on the context, box does not contain the tree. However, it seems there might be a mistake in the sentence as it mentions 'inside the box' twice. Assuming the intention was to describe the tree's position relative to the box without specifying a container for the tree, we can interpret this as the tree being positioned relative to the box without being contained within it. Since the box does not have an intrinsic direction, the reference to 'left' is relative to an external observer's perspective. Answer: external relative.

% \textbf{SG} : This context seems to have a logical inconsistency as a tree cannot be inside a box. However, assuming this was intended to describe two separate spatial relations, we can interpret it as follows:

% Explanation: Topological: tree is not inside of the box. Distance: tree may have some distance from the box according to the context. Direction: the box does not have a facing orientation, so the tree being on the left is referred from the observer's perspective. Answer: external relative for the relation between the tree and the box.

% However, the second part of the sentence "inside the box" does not provide enough information to determine a spatial relation between entities. It appears to be incomplete or incorrectly phrased. If you meant to say "A tree is on the left of a box inside a room," then the relation between the tree and the box would still be external relative, while the relation of the box being inside the room would be internal relative.

% \textbf{label} : internal relative

% \item \textbf{Context} : A deer is outside and on the left of a bus relative to the bus

% \textbf{CoT} : Explanation: The bus has the front/back direction. However, the left of the bus is not referring to the bus's intrinsic direction, but rather to an observer's perspective. Answer: external relative.

% \textbf{SG} : Explanation: Topological: deer is outside of the bus. Distance: deer may be little bit far from the left of the bus from the context. Direction: The bus which is the relative object has the direction and deer is actually on the left side of the bus. Answer: external intrinsic.

% \textbf{label} : external intrinsic

% \item \textbf{Context} : A bicycle is behind a fire hydrant

% \textbf{CoT} : Explanation: Based on the context, fire hydrant does not contain the bicycle. Fire hydrant does not have intrinsic direction, so the behind relation is based on the observer's perspective. However, since there is no explicit mention of an observer or reference point, we can consider it as an external intrinsic relation where the observer's perspective is implied. Answer: external intrinsic.

% \textbf{SG} : Explanation: Topological: bicycle is not inside of the fire hydrant. Distance: bicycle may have some distance from the fire hydrant according to the context. Direction: Fire hydrant does not have a facing orientation, so the spatial relation is not relative to its intrinsic direction. Instead, it is described from an external perspective. Answer: external relative.

% \textbf{label} : external relative

% \item \textbf{Context} : A box is behind a chicken relative to the chicken

% \textbf{CoT} : Explanation: The chicken has no intrinsic direction, so the 'behind' relation is based on the perspective of the chicken itself, not its intrinsic orientation. Answer: external relative. Note that this could potentially be seen as external intrinsic if we assume that the chicken has a front-back orientation, but since chickens do not typically have a fixed or understood front-back orientation like humans or cars do, it is more likely to be considered external relative.

% \textbf{SG} : Explanation: Topological: box is not inside of the chicken. Distance: box may be little bit far from the chicken from the context. Direction: The chicken which is the relative object has the direction and box is actually behind the chicken. Answer: external intrinsic.

% \textbf{label} : external intrinsic

% \item \textbf{Context} : A car is in front of a deer from the deer's perspective

% \textbf{CoT} : Explanation: Based on the context, deer does not contain the car. The deer has the direction; however, the front of a deer referring to the deer's perspective rather than intrinsic direction of deer. Answer: external relative.

% \textbf{SG} : Explanation: Topological: car is not inside of the deer. Distance: car may has some distance from deer according to the context. Direction: The deer which is the relative object has the direction, and car is in front of the deer. Answer: external intrinsic.

% \textbf{label} : external intrinsic

% \item \textbf{Context} : An umbrella is on the left of a chicken from the chicken's perspective

% \textbf{CoT} : Explanation: Based on the context, chicken does not contain the umbrella. The chicken has the direction; however, the left of a chicken referring to the chicken's perspective rather than intrinsic direction of the chicken. Answer: external intrinsic. Note that this sentence is ambiguous because it can be interpreted as external relative if the left is referring to the observer's perspective. However, given the context provided, it is more likely to be interpreted as external intrinsic.

% \textbf{SG} : Explanation: Topological: umbrella is not inside of the chicken. Distance: umbrella may has some distance from chicken according to the context. Direction: The context left relation is referred from the chicken's perspective that an umbrella is on the left side of the chicken's facing orientation. Answer: external intrinsic.

% \textbf{label} : external intrinsic

% \end{itemize}

% \subsubsection{Failure Case of SG}

% \begin{itemize}
%     \item \textbf{Context} : A bus is on the right of a chair from the chair's perspective

% \textbf{SG} : Explanation: Topological: bus is not inside of the chair. Distance: bus may has some distance from chair according to the context. Direction: Even though the chair does not have a facing orientation, the perspective is given as the chair's, making this a case of an object being described relative to another object's perspective. Answer: external intrinsic. However, usually chairs do not have a defined facing direction, so typically this would fall under external relative. But given the instruction to consider the chair's perspective, it can be seen as an exception to the rule, aligning more closely with external intrinsic in this specific context.

% \textbf{label} : external intrinsic

% \item \textbf{Context} : A tree is on the left of a box inside the box

% \textbf{SG} : This context seems to have a logical inconsistency as a tree cannot be inside a box. However, assuming this was intended to mean 'A tree is on the left of a box inside a room', the categorization would be:

% Explanation: Topological: tree is not inside of the box. Distance: tree may have some distance from the box according to the context. Direction: The context left relation is referred from observer's perspective that a tree is on the left side of the box location. Answer: external relative. 

% If the original context was correct, it would fall under none of these categories due to the physical impossibility of the scenario described.

% \textbf{label} : internal relative

% \item \textbf{Context} : A cow is in front of a bench relative to the bench

% \textbf{SG} : Explanation: Topological: cow is not inside of the bench. Distance: cow may has some distance from bench according to the context. Direction: Bench doesn't have the direction and context refere from observer perspective for cow is in the front of the bench. Answer: external relative. But if we consider that 'in front of' refers to the bench's intrinsic direction, then it would fall under external intrinsic. However, benches typically do not have a defined facing direction, so this is less common.

% \textbf{label} : external intrinsic

% \item \textbf{Context} : A dog is on the left of a box within the box

% \textbf{SG} : Explanation: Topological: dog is not inside of the box. Distance: dog may has some distance from box according to the context. Direction: Even if box has direction, the context left relation is referred from observer's perspective that a dog is on the left side of the box location. However, since the dog is not contained by the box, this does not fall under internal intrinsic or internal relative. Answer: external relative. Note that the phrase "within the box" seems redundant and does not change the frame of reference.

% \textbf{label} : internal relative

% \end{itemize}

\section{Frame of Reference Identification}\label{appendix:FoRIdentification}

We evaluate the LLMs' performance in recognizing the FoR classes from given spatial expressions. 
The LLMs receive spatial expression, denoted as $T$, and output one FoR class, $FoR$, from the valid set of FoR classes, $For \in $ \{external relative, external intrinsic, internal intrinsic, internal relative\}. All in-context learning examples are in the Appendix~\ref{appendix:in-context}.

\subsection{Experimental Setting}
\noindent\textbf{Zero-shot model.} We follow the regular setting of \textit{zero-shot} prompting. 
We only provide instruction to LLM with spatial context. 
The instruction prompt briefly explains each class of the FoR and candidate answers for the LLM. We called the LLM with the instruction prompt and $T$ to find $F$.

\noindent\textbf{Few-shot model.} We manually craft four spatial expressions for each FoR class. 
To avoid creating bias, each spatial expression is ensured to fit in only one FoR class. These expressions serve as examples of our \textit{few-shot}setting.
We provide these examples in addition to the instruction as a part of the prompt, followed by $T$ and query $F$ from the LLM.

\noindent\textbf{Chain-of-Thought (CoT) model.}
To create CoT~\citep{wei2023chainofthoughtpromptingelicitsreasoning} examples, we modify the prompt to require reasoning before answering.
Then, we manually crafted reasoning explanations with the necessary information for each example used in few-shot.
Finally, we call the LLMs, adding modified instructions to updated examples, followed by $T$ and query $F$. 

\noindent\textbf{Spatial-Guided Prompting (SG) model.}
We hypothesize that the general spatial relation types defined in Section~\ref{sec:primitives} can provide meaningful information for recognizing FoR classes. For instance, a topological relation, such as ``inside," is intuitively associated with an internal FoR.
Therefore, we propose Spatial-Guided Prompting to direct the model in identifying the type of relations before querying $F$. 
We revise the prompting instruction to guide the model in considering these three aspects. 
Then, we manually explain these three aspects.
We specify the relation's origin from the context for direction relations, such as "the left direction is relative to the observer."
We hypothesize that this information helps the model distinguish between intrinsic and relative FoR.
Next, we specify whether the locatum is inside or outside the relatum for topological relations. 
This information should help distinguish between internal and external FoR classes.
Lastly, we provide the potential quantitative distance, e.g., far. This quantitative distance further encourages identifying the correct topological and directional relations. 
Eventually, we insert these new explanations in examples and call the model with the updated instructions followed by $T$ to query $F$.

\subsection{Evaluation Metrics}
We report the accuracy of the model on the multi-class classification task. Note that the expressions in A-split can have multiple correct answers. Therefore, we consider the prediction correct when it is in one of the valid FoR classes for the given spatial expression. 

\begin{table}[t]
    \tiny
    \centering
    \begin{tabular}{|l|c | c|c|c|}
    \hline
    \textbf{Model} & \multicolumn{2}{c|}{inherently clear} & \multicolumn{2}{c|}{require template} \\ \cline{2-5}
     & CoT & SG & {CoT} & {SG} \\ 
    \hline
    Llama3-70B  & 19.84 & 44.64 \improve{24.80} & 76.72 & 87.39 \improve{10.67}\\
    Qwen2-72B & 58.20 & 84.22 \improve{26.02} & 88.36 & 93.86 \improve{10.67} \\
    GPT-4o & 12.50 & 29.17 \improve{16.67} & 87.73 & 90.74 \improve{3.01}  \\
    \hline
    \end{tabular}
    \caption{The comparison between CoT and SG prompting in C-split separated by inherently clear / required template to be clear.}
    \label{tab:model_performance}
\end{table}



\begin{table*}[t]
    \tiny
    \centering
    \begin{tabular}{| l | c | c c c c | c|}
        \hline
        & A-split & \multicolumn{5}{|c|}{C-Split} \\ \cline{3-7}
         Model &  & ER-C-Split & EC-Split & IC-Split & IR-C-Split & Avg. \\
         \hline
         Gemma2-9B (0-shot) & $94.17$ & $\mathbf{94.24}$ & $35.98$ & $53.91$ & $57.66$  & $60.45$\\
          Gemma2-9B (4-shot) & $59.58$  & $55.89$\worse{38.34} & $72.61$\improve{36.63} & $74.22$\improve{20.31} & $54.44$\worse{3.23} & $64.29$\improve{3.84}\\
         Gemma2-9B (CoT) & $60.49$  & $60.49$\worse{33.74} & $60.54$\improve{24.57} & $87.50$\improve{33.59} & $54.03$\worse{3.63} & $65.64$\improve{5.20}\\
          Gemma2-9B (SG)(Our) & $72.67$ & $65.87$\worse{28.37} & $65.54$\improve{29.57} & $53.12$\worse{0.78} & $\mathbf{95.97}$\improve{38.31} & $70.13$\improve{9.68}\\
         \hline
         llama3-8B (0-shot) & $60.21$ & $32.20$ & $90.11$ & $75.78$ & $0.00$ & $49.52$\\
         llama3-8B (4-shot) & $60.14$ & $47.77$\improve{15.58} & $54.35$\worse{35.76} & $100.00$\improve{24.22} & $41.13$\improve{41.13} & $60.81$\improve{11.29}\\
         llama3-8B (CoT) & $61.32$ & $61.06$\improve{28.86} & $97.28$\improve{7.17} & $100.00$\improve{24.22} & $36.29$\improve{36.29} & $73.66$\improve{24.14}\\
         llama3-8B (SG) (Our) & $62.95$ & $63.29$\improve{31.09} & $94.57$\improve{4.46} & $100.00$\improve{24.22} & $43.55$\improve{43.55} & $75.35$\improve{25.83}\\

         \hline
         llama3-70B (0-shot) & $84.23$ & $74.08$ & $9.57$ & $92.19$ & $68.55$ & $61.10$\\
         llama3-70B (4-shot) & $78.47$ & $81.81$\improve{7.72} & $64.89$\improve{55.33} & $100.00$\improve{7.81} & $75.81$\improve{7.26} & $80.63$\improve{19.53}\\
         llama3-70B (CoT) & $69.11$ & $72.05$\worse{2.03} & $97.07$\improve{87.50} & $100.00$\improve{7.81} & $79.44$\improve{10.89} & $87.14$\improve{26.04}\\
         llama3-70B (SG) (Our) & $76.50$ & $78.21$\improve{4.12} & $97.61$\improve{88.04} & $100.00$\improve{7.81} & $72.18$\improve{3.63} & $87.00$\improve{25.90}\\
        % llama3-70B (0-shot) & $77.33$  & $35.04$ & $32.39$ & $57.81$ & $53.23$ & $44.62$\\
        %  llama3-70B (4-shot) & $59.78$ & $59.78$\improve{24.74} & $66.52$\improve{34.13} & $77.34$\improve{19.53} & $51.61$\worse{1.61} & $63.81$\improve{19.20}\\
        %  llama3-70B (CoT) & $66.00$  & $68.01$\improve{32.97} & $65.65$\improve{33.26} & $91.41$\improve{33.59} & $58.47$\improve{5.24} & $70.88$\improve{26.27}\\
        %  llama3-70B (SG) (Our) & $74.94$  & $78.17$\improve{43.13} & $70.87$\improve{38.48} & $100.00$\improve{42.19} & $84.27$\improve{31.05} & $83.33$\improve{38.71}\\
         \hline
         Qwen2-7B (0-shot) & $83.64$ & $79.97$ & $59.24$ & $77.34$ & $40.73$ & $64.32$\\
        Qwen2-7B (4-shot) & $61.12$ & $50.52$\worse{29.45} & $65.76$\improve{6.52} & $93.75$\improve{16.41} & $56.05$\improve{15.32} & $66.52$\improve{2.20}\\
        Qwen2-7B (CoT) & $72.12$ & $70.81$\worse{9.16} & $63.80$\improve{4.57} & $99.22$\improve{21.88} & $51.61$\improve{10.89} & $71.36$\improve{7.04}\\
        Qwen2-7B (SG) & $70.61$ & $68.00$\worse{11.98} & $71.20$\improve{11.96} & $88.28$\improve{10.94} & $57.26$\improve{16.53} & $71.18$\improve{6.86}\\
        \hline
        Qwen2-72B (0-shot)& $64.46$ & $62.70$ & $100.00$ & $100.00$ & $39.11$ & $75.45$\\
        Qwen2-72B (4-shot)& $79.12$ & $78.73$\improve{16.03} & $99.35$\worse{0.65} & $87.50$\worse{12.50} & $87.10$\improve{47.98} & $88.17$\improve{12.72}\\
        Qwen2-72B (CoT)& $88.54$ & $88.87$\improve{26.18} & $89.57$\worse{10.43} & $93.75$\worse{6.25} & $83.47$\improve{44.35} & $88.91$\improve{13.46}\\
        Qwen2-72B (SG)& $90.51$ & $90.18$\improve{27.49} & $93.26$\worse{6.74} & $98.44$\worse{1.56} & $85.08$\improve{45.97} & $91.74$\improve{16.29}\\
        \hline
         GPT3.5 (0-shot) & $83.11$ & $88.15$ & $17.50$ & $70.31$ & $41.13$ & $54.27$\\
         GPT3.5 (4-shot) & $61.25$  & $48.95$\worse{39.20} & $62.72$\improve{45.22} & $100.00$\improve{29.69} & $28.63$\worse{12.50} & $60.07$\improve{5.80}\\

         GPT3.5 (CoT) & $66.55$ & $66.62$\worse{21.53} & $96.85$\improve{79.35} & $100.00$\improve{29.69} & $50.81$\improve{9.68} & $78.57$\improve{24.30}\\
         GPT3.5 (SG) (Our) & $70.61$  & $73.30$\worse{14.86} & $92.93$\improve{75.43} & $99.22$\improve{28.91} & $49.19$\improve{8.06} & $78.66$\improve{24.39}\\
         \hline
         GPT4o (0-shot) & $73.82$  & $71.27$ & $98.80$ & $100.00$ & $70.56$ & $85.16$\\
         GPT4o (4-shot) & $66.23$  & $67.87$\worse{3.40} & $98.70$\worse{0.11} & $100.00$\improve{0.00} & $78.63$\improve{8.06} & $86.30$\improve{1.14}\\
         GPT4o (CoT) & $72.44$  & $72.77$\improve{1.51} & $100.00$\improve{1.20} & $100.00$\improve{0.00} & $73.79$\improve{3.23} & $86.64$\improve{1.48}\\
         GPT4o (SG) (Our) & $76.44$ & $74.67$\improve{3.40} & $97.72$\worse{1.09} & $100.00$\improve{0.00} & $68.55$\worse{2.02} & $85.23$\improve{0.08}\\
         \hline
    \end{tabular}
    \caption{Accuracy results report from FoR Identification with LLMs. The correct prediction is one of the valid FoR classes for the given spatial expression. All FoR classes are external relative (ER), external intrinsic (EI), internal intrinsic (II), and internal relative (IR).}
    \label{tab:text_experiment}
\end{table*}




\begin{figure*}[t]
    \centering
    \begin{subfigure}[ht]{0.4\textwidth}
        \centering
        \includegraphics[width=0.8\textwidth, trim={0 0 0 2cm}]{Figures/A_cow_case.png}
        \caption{Results of Cow Case in A-Split. 
        % Valid predictions are external intrinsic and external relative.
        }
    \end{subfigure}%
    ~ 
    \begin{subfigure}[ht]{0.4\textwidth}
        \centering
        \includegraphics[width=0.8\textwidth, trim={0 0 0 2cm}]{Figures/A_car_case.png}
        \caption{Results of Car Case in A-Split. 
        % All FoRs are valid predictions of this case.
        }
    \end{subfigure}
    
    \vskip\baselineskip
    
    \begin{subfigure}[ht]{0.4\textwidth}   
        \centering 
        \includegraphics[width=0.8\textwidth, trim={0 0 0 1cm}]{Figures/A_box_case.png}
        \caption{Results of Box Case in A-Split. 
        % The correct predictions are external relative and internal relative.
        }    
    \end{subfigure}
        ~
    \begin{subfigure}[ht]{0.4\textwidth}   
        \centering 
        \includegraphics[width=0.8\textwidth, trim={0 0 0 1cm}]{Figures/A_pen_case.png}
        \caption{Results of Pen Case in A-Split. 
        % The only applicable FoR is external relative.
        }  
    \end{subfigure}
    \caption{Red shows the wrong FoR identifications, and green shows the correct ones. The dark color is for relative FoRs, while the light color is for intrinsic FoRs. The round shape is for the external FoRs, while the square is for internal FoRs. The depth of the plots shows the four FoRs, i.e., \textit{external relative, external intrinsic, internal intrinsic, and internal relative}, \textbf{from front to back}.}
    \label{fig:cow_car_case}
\end{figure*}


\begin{figure*}[t]
    \centering
    \begin{subfigure}[t]{0.4\textwidth}
        \centering
        \includegraphics[width=0.88\textwidth, trim={0 0 0 0}]{Figures/A_cow_case_s.png}
        \caption{Results of Cow Case in A-Split. 
        % Valid predictions are external intrinsic and external relative.
        }
    \end{subfigure}%
    ~ 
    \begin{subfigure}[t]{0.4\textwidth}
        \centering
        \includegraphics[width=0.88\textwidth, trim={0 0 0 0}]{Figures/A_car_case_s.png}
        \caption{Results of Car Case in A-Split. 
        % All FoRs are valid predictions of this case.
        }
    \end{subfigure}
    
    \vskip\baselineskip
    
    \begin{subfigure}[t]{0.4\textwidth}   
        \centering 
        \includegraphics[width=0.88\textwidth, trim={0 0 0 0}]{Figures/A_box_case_s.png}
        \caption{Results of Box Case in A-Split. 
        % The correct predictions are external relative and internal relative.
        }    
    \end{subfigure}
        ~
    \begin{subfigure}[t]{0.4\textwidth}   
        \centering 
        \includegraphics[width=0.88\textwidth, trim={0 0 0 0}]{Figures/A_pen_case_s.png}
        \caption{Results of Pen Case in A-Split. 
        % The only applicable FoR is external relative.
        }  
    \end{subfigure}
    \caption{Red shows the wrong FoR identifications, and green shows the correct ones. The dark color is for relative FoRs, while the light color is for intrinsic FoRs. The round shape is for the external FoRs, while the square is for internal FoRs. The depth of the plots shows the four FoRs, i.e., external relative, external intrinsic, internal intrinsic, and internal relative, from front to back. This plot is the result of the rest of LLMs.}
    \label{fig:cow_car_case2}
\end{figure*}

\subsection{Results}

\subsubsection{FoR Inherently Bias in LLMs} 
\noindent\textbf{C-spilt.}
The \textit{zero-shot} setting reflects the LLMs' inherent bias in identifying FoR.
Table~\ref{tab:text_experiment} presents the accuracy for each FoR class in C-split, where sentences explicitly include information about topology and perspectives.
We found that some models strongly prefer specific FoR classes.
Notably, Gemme2-9B achieves a near-perfect accuracy on external relative FoR but performs poorly on other classes, especially external intrinsic, indicating a notable bias towards external relative. 
In contrast, GPT4o and Qwen2-72B perform exceptionally in both intrinsic FoR classes. However, they perform poorly in the relative FoRs.

\noindent\textbf{A-spilt.}
We examine the FoR bias in the A-split.
Based on the results in Table~\ref{tab:text_experiment}, we plotted the top-3 models' results (Gemma2-9B, Llama3-70B, and GPT4o) for a more precise analysis in Figures~\ref{fig:cow_car_case}. 
The plots show the frequencies of each FoR category. 
According to the plot, Gemma and GPT have strong biases toward external relative and external intrinsic, respectively. 
This bias helps Gemma2 perform well in the A-split since all spatial expressions can be interpreted as external relative. 
However, GPT4o's bias leads to errors when intrinsic FoRs aren't valid, as in the Box and Pen cases (see plots (c) and (d)).
Llama3 exhibits different behavior, showing a bias based on the relatum’s properties, specifically the relatum's affordance as a container.
In cases where relatum cannot serve as containers, i.e., Cow and Pen cases, Llama3 favors external relative. 
Conversely, Llama3 tends to favor external intrinsic when the relatum has the potential to be a container.

\subsubsection{Behavior with ICL variations}\label{sec:result_A_ICL}

\noindent\textbf{C-spilt.}
We evaluate the models’ behavior under various in-context learning (ICL) methods.
As observed in Table~\ref{tab:text_experiment}, the \textit{few-shot} method improves the performance of the \textit{zero-shot} method across multiple LLMs by reducing their original bias toward specific classes. 
Reducing the bias, however, lowers the performance in some cases, such as the performance of Gemma 2 in ER class.
One noteworthy observation is that while the \textit{CoT} prompting generally improves performance in larger LLMs, it is counterproductive in smaller models for some FoR classes. 
This suggests that the smaller models have difficulty inferring FoR from the longer context. 
This negative effect also appears in SG prompting, which uses longer explanations.
Despite performance degradation in particular classes of small models, SG prompting performs exceptionally well across various models and achieves outstanding performance with Qwen2-72B. 
We further investigate the performance of CoT and SG prompting. 
As shown in Table~\ref{tab:model_performance}, CoT exhibits a substantial difference in performance between contexts with inherently clear FoR and contexts requiring the template to clarify FoR ambiguity.
This implies that CoT heavily relies on the specific template to identify FoR classes. 
In contrast, SG prompting demonstrates a smaller gap between these two scenarios and significantly enhances performance over CoT in inherently clear FoR contexts.  
Therefore, guiding the model to provide characteristics regarding topological, distance, and directional types of relations improves FoR comprehension.

\noindent\textbf{A-spilt.}
We use the same Figure~\ref{fig:cow_car_case} to observe the behavior when applying ICL. 
The A-split shows minimal improvement with ICL variations, though some notable changes are observed.
With \textit{few-shot}, all models show a strong bias toward external intrinsic FoR, even when the relatum lacks intrinsic directions, i.e., Box and Pen cases. 
This bias appears even in Gemma2-9B, which usually behaves differently. 
This suggests that the models pick up biases from the examples despite efforts to avoid such patterns.
However, \textit{CoT} reduces some bias, leading LLMs to revisit relative, which is generally valid across scenarios. 
In Gemma2, the model predicts relative FoR where the relatum has intrinsic directions, i.e., Cow and Car cases.
Llama3 behaves similarly in cases where the relatum cannot act as a container, i.e., Cow and Pen cases.
GPT4o, however, does not depend on the relatum's properties and shows slight improvements across all cases.
Unlike \textit{CoT}, our SG prompting is effective in all scenarios.
It significantly reduces biases while following a similar pattern to \textit{CoT}. 
Specifically, SG prompting increases external relative predictions for Car and Cow in Gemma2-9B, and for Cow and Pen in Llama3-70B.
Nevertheless, GPT4o shows only a slight bias reduction.
However, Our proposed method improves the overall performance of most models, as shown in Table~\ref{tab:text_experiment}. 
The Llama3-70B behaviors are also seen in LLama3-8B and GPT3.5. 
The plots for LLama3-8B and GPT3.5 are in Figure~\ref{fig:cow_car_case2}.

\subsubsection{Experiment with different temperatures}
We conducted additional experiments to further investigate the impact of temperature on the biased interpretation of the model in the A-split of our dataset.
As presented in Table~\ref{tab:temp_table}, comparing distinct temperatures (0 and 1) revealed a shift in the distribution. The frequencies of the classes experienced a change of up to 10\%.
However, the magnitude of this change is relatively minor, and the relative preferences for most categories remained unchanged.
Specifically, the model exhibited the highest frequency responses for the cow, car, and pen cases, even with higher frequencies in certain settings. Consequently, a high temperature does not substantially alter the diversity of LLMs’ responses to this task, which is an intriguing finding.

\begin{table*}[t]
    \tiny
    \centering
    \begin{tabular}{|l|c c |c c| c c | c c |}
    \hline
    Model & \multicolumn{2}{|c|}{ER} & \multicolumn{2}{|c|}{EI} & \multicolumn{2}{|c|}{II} & \multicolumn{2}{|c|}{IR} \\
    & temp-0 & temp-1 & temp-0 & temp-1 & temp-0 & temp-1 & temp-0 & temp-1 \\
    \hline
    \multicolumn{9}{| l |}{Cow Case} \\
    \hline
     0-shot   & 75.38 &  87.12 & 23.86 & 12.50 & 0.76 & 0.13 & 0.00 & 0.25\\ 
     4-shot   & 0.00 &  15.66 & 100.00 & 84.34 & 0.00 & 0.00 & 0.00 & 0.00\\
     CoT & 31.82 & 49.87 & 68.18 & 49.87 & 0.00 & 0.13 & 0.00 & 0.13 \\
     SG & 51.39 & 70.45 & 48.61 & 29.42 & 0.00 & 0.00 & 0.00 & 0.13\\
     \hline
     \multicolumn{9}{| l |}{Box Case} \\
     \hline
     0-shot   & 22.50 &  41.67 & 77.50 & 58.33 & 0.00 & 0.13 & 0.00 & 0.25\\ 
     4-shot   & 0.00 &  0.00 & 100.00 & 100.00 & 0.00 & 0.00 & 0.00 & 0.00\\
     CoT & 0.00 &  5.83 & 100.00 & 94.17 & 0.00 & 0.00 & 0.00 & 0.00\\
     SG & 11.67 &  33.33 & 88.33 & 66.67 & 0.00 & 0.00 & 0.00 & 0.00\\
     \hline
     \multicolumn{9}{| l |}{Car Case} \\
     \hline
     0-shot   & 55.20 & 68.24 & 49.01 & 31.15 & 0.79 & 0.61 & 0.00 & 0.00\\ 
     4-shot   & 0.60 &  5.94 & 99.40 & 94.06 & 0.00 & 0.00 & 0.00 & 0.00\\
     CoT & 19.64 &  38.52 & 80.16 & 61.27 & 0.20 & 0.20 & 0.00 & 0.00\\
     SG & 44.25 &  56.97 & 55.75 & 43.03 & 0.00 & 0.00 & 0.00 & 0.00\\
     \hline
     \multicolumn{9}{| l |}{Pen Case} \\
     \hline
     0-shot   & 90.62 & 96.88 & 9.38 & 3.12 & 0.00 & 0.61 & 0.00 & 0.00\\ 
     4-shot   & 0.00  &  7.03 & 100.00 & 92.97 & 0.00 & 0.00 & 0.00 & 0.00\\
     CoT & 17.19 &  28.91 & 82.81 & 71.09 & 0.20 & 0.20 & 0.00 & 0.00\\
     SG & 48.31 &  57.81 & 54.69 & 42.19 & 0.00 & 0.00 & 0.00 & 0.00\\
     \hline
    \end{tabular}
    \caption{The results between two different temperatures of Llam3-70B on the A-spilt of FoREST. The number shows the percentage frequency of responses from the model.}
    \label{tab:temp_table}
\end{table*}

\section{In-context learning}\label{appendix:in-context}
\subsection{FoR Identification}
We provide the prompting for each in-context learning. The prompting for \textit{zero-shot} and \textit{few-shot} is provided in Listing~\ref{lst:base_instruction}. The instruction answer for these two in-context learning is ``Answer only the category without any explanation. The answer should be in the form of \{Answer: Category.\}"

For the Chain of Thought (CoT), we only modified the instruction answer to ``Answer only the category with an explanation. The answer should be in the form of \{Explanation: Explanation Answer: Category.\}" 
Similarly to CoT, we only modified the instruction answer to ``Answer only the category with an explanation regarding topological, distance, and direction aspects. The answer should be in the form of \{Explanation: Explanation Answer: Category.\}", respectively. The example responses are provided in Listing~\ref{lst:example_answer} for Spatial Guided prompting.

\begin{lstlisting}[caption={Prompt for finding the frame of reference class of given context.}, label={lst:base_instruction}]
# Instruction to find frame of reference class of given context
"""
Instruction: 
You specialize in language and spatial relations, specifically in the frame of context (multiple perspectives in the spatial relation). Identify the frame of reference category given the following context. There are four classes of the frame of reference (external intrinsic, internal intrinsic, external relative, internal relative). Note that the intrinsic direction refers to whether the model has the front/back by itself. (Example: a bird, human. Counter Example: a ball, a box). "

External intrinsic. The spatial description of an entity A relative to another entity B, where (1) A is not contained by B, (2) the spatial relation is based on B's facing orientation (intrinsic direction) if B has one.

Internal intrinsic. The spatial description of an entity A relative to another entity B, where (1) A is contained by B, (2) the spatial relation is based on B's facing orientation (intrinsic direction) if B has one.

External relative. The spatial description of an entity A relative to another entity B, where (1) A is not contained by B, (2) The spatial relation is relative to entity C, where C is not involved in the spatial relation and only provides the perspective. An example of C is the camera.

Internal relative. The spatial description of an entity A relative to another entity B, where (1) A is contained by B, (2) The spatial relation is relative to entity C, where C is not involved in the spatial relation and only provides the perspective. An example of C is the camera.

{Instruction answer}
# Normal Instruction answer: Find the frame of reference category without any explanations. The answer should be in the form of Answer: Category.
# COT Instruction answer: Find the frame of reference category with an explanation. The answer should be in the form of Explanation: Explanation. Answer: Category.
# SG Instruction answer: Find the frame of reference category with an explanation. Additionally, include three more explanations relating to topology, distance, and direction from the given context. The answer should be in the form of \"Explanation: Explanation. Topology: Explanation. Distance: Explanation. Direction: Explanation. Answer: Category.

Context: {spatial exprssion}
"""

\end{lstlisting}

\begin{lstlisting}[caption={Spatial expression examples using for few-shots, Chain-of-Thought (CoT), and Spatial Guide (SG) prompting}, label={lst:example_answerQA}]
## These are four examples used in few-shot, CoT, and SG
# Example 1: The bird is outside the car and in front of the car relative to the car.
# Few-shot response: Answer: external intrinsic
# CoT response: Explanation: The car does not contain the bird, and the spatial relation is relative to the car's facing orientation. The intrinsic direction of the car is relevant in this context, as the description \"in front of\" implies a direction relative to the sheep's front or back. Answer: external intrinsic.
# SG response: Explanation: The car does not contain the bird, and the spatial relation is relative to the car's facing orientation. The intrinsic direction of the car is relevant in this context, as the description \"in front of\" implies a direction relative to the sheep's front or back. Topology: The bird is outside the car. Distance: There is little distance between the bird and the car. Direction: The bird is in the front direction of the car (entity B). Answer: external intrinsic.

# Example 2: The bird is inside of the car and left of the car from the car's perspective.
# Few-shot response: Answer: internal intrinsic
# CoT response: Explanation: In this scenario, the bird is contained within the car, meaning it is an internal relationship. Additionally, the description specifies the bird's position relative to the car, indicating that the spatial relation is relative to the car itself, considering its intrinsic direction. Answer: internal intrinsic.
# SG response:  Explanation: In this scenario, the bird is contained within the car, meaning it is an internal relationship. Additionally, the description specifies the bird's position relative to the car, indicating that the spatial relation is relative to the car itself, considering its intrinsic direction. Topological: The bird is inside the car. Distance: The bird is in the same location as the car. Direction: The bird is in the left direction of the car (entity B). Answer: internal intrinsic.

# Example 3: The box is inside and at the back of the room.
# Few-shot response: Answer: internal relative
# CoT response: Explanation: In this context, the box (entity A) is described as inside the room (entity B), fulfilling the first condition for an internal frame of reference. However, the spatial relation given (\" at the back\") cannot originate from the room since the room does not have intrinsic direction, implying it is relative to an external observer or reference point (C), such as a camera or a person viewing the scene. Answer: internal relative
# SG response:  Explanation: In this context, the box (entity A) is described as inside the room (entity B), fulfilling the first condition for an internal frame of reference. However, the spatial relation given (\" at the back\") cannot originate from the room since the room does not have intrinsic direction, implying it is relative to an external observer or reference point (C), such as a camera or a person viewing the scene. Topological: The box is inside the room. Distance: The box is in the same location as the room. Direction: From the observer's perspective, the box is located at the back of the room. Answer: internal relative

# Example 4: A phone is on the left of a tablet from my perspective.
# Few-shot response: Answer: external relative
# CoT response: Explanation: In this context, the spatial relation is described from the perspective of an observer (C) who is not involved in the spatial relation. The phone (A) is not contained by the tablet (B), and the spatial relation \"left of\" is relative to the observer's perspective, not the tablet's facing orientation. Answer: external relative.
# SG response:  Explanation: In this context, the spatial relation is described from the perspective of an observer (C) who is not involved in the spatial relation. The phone (A) is not contained by the tablet (B), and the spatial relation \"left of\" is relative to the observer's perspective, not the tablet's facing orientation. Topological: The phone is not contained by the tablet. Distance: There is some distance between the phone and the tablet. Direction: From the observer's perspective, the phone is located to the left of the tablet. Answer: external relative.
\end{lstlisting}

\subsection{Question Answering Example}\label{appendix:QA_example}


\begin{lstlisting}[caption={Spatial expression examples using for few-shots, Chain-of-Thought (CoT), and Spatial Guide (SG) prompting for question-answering.}, label={lst:example_answerQA}]
'''
#Context: The bird is outside the car and in front of the car relative to the car. The car is facing toward the camera. 
#Question: Based on the camera's perspective, where is the bird from the car position in the scene? 
#SG information (used in SG + CoT): Frame of reference Explanation: The car does not contain the bird, and the spatial relation is relative to the car's facing orientation. The intrinsic direction of the car is relevant in this context, as the description \"in front of\" implies a direction relative to the sheep's front or back. Topology: The bird is outside the car. Distance: There is little distance between the bird and the car. Direction: The bird is in the front direction of the car (entity B). Frame of reference: external intrinsic.
#Normal response: Answer: front
#CoT / (SG + COT) response:Explanation: Based on the context, the bird's position is in the front direction of the car. The car is facing toward the camera. Then, the car's front direction is the camera's front direction. Therefore, the bird's position is in front of the car's position from the camera's perspective. Answer: front

#Context: The bird is inside the car and left of the car from the car's perspective. The car is facing to the right relative to the camera.  #Question: Based on the camera's perspective, where is the bird from the car's position?
#SG information (used in SG + CoT):  Frame of reference Explanation: In this scenario, the bird is contained within the car, meaning it is an internal relationship. Additionally, the description specifies the bird's position relative to the car, indicating that the spatial relation is relative to the car itself, considering its intrinsic direction. Topological: The bird is inside the car. Distance: The bird is in the same location as the car. Direction: The bird is in the left direction of the car (entity B). Frame of reference: internal intrinsic.
#Normal response: Answer: back
#CoT / (SG + COT) response: Explanation: Based on the context, the bird's position is in the left direction of the car. The car is facing to the right. Then, the car's left direction is the camera's back direction. Therefore, the bird's position is to the back of the car's position from the camera's perspective. Answer: back

Context: The box is inside and to the right of the room from the observer's perspective. 
Question: From the observer's perspective, what is the spatial relation of the box to the room? 
#SG information (used in SG + CoT): Frame of reference Explanation: In this context, the box (entity A) is described as inside the room (entity B), fulfilling the first condition for an internal frame of reference. However, the spatial relation given (\" at the back\") cannot originate from the room since the room does not have intrinsic direction, implying it is relative to an external observer or reference point (C), such as a camera or a person viewing the scene. Topological: The box is inside the room. Distance: The box is in the same location as the room. Direction: From the observer's perspective, the box is located at the back of the room. Frame of reference: internal relative.
#Normal response: Answer: right
#CoT / (SG + COT) response: Explanation: Based on the context, the box is to the right of the room from the camera's direction. Therefore, the box's position is to the right of the room's position from the observer's perspective. Answer: right

Context: A phone is to the left of a tablet from my perspective. The tablet is facing to the right. Question: From my perspective, what is the spatial relation of the phone to the tablet?
#SG information (used in SG + CoT): Frame of Reference Explanation: In this context, the spatial relation is described from the perspective of an observer (C) who is not involved in the spatial relation. The phone (A) is not contained by the tablet (B), and the spatial relation \"left of\" is relative to the observer's perspective, not the tablet's facing orientation. Topological: The phone is not contained by the tablet. Distance: There is some distance between the phone and the tablet. Direction: From the observer's perspective, the phone is located to the left of the tablet. Frame of Reference: external relative.
#Normal response: Answer: left
#CoT / (SG + COT) response: Explanation: Based on the context, the phone is to the left of the tablet from my perspective. The direction of the tablet is not relevant in this context since the left relation is from my perspective. Therefore, from my perspective, the phone is to the left of the tablet. Answer: left
'''
\end{lstlisting}

\subsection{Text to Layout}
\begin{lstlisting}[caption={Prompt for generating bounding coordinates to use as the layout for layout-to-image models.}, label={lst:example_answer}]
    # Instruction for generating bounding box
"""
Your task is to generate the bounding boxes of objects mentioned in the caption.
The image is size 512x512. The bounding box should be in the format of (x, y, width, height). Please considering the frame of reference of caption and direction of reference object if possible. If needed, you can make the reasonable guess.
"""
\end{lstlisting}







 

\end{document}
