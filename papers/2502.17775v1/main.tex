% This must be in the first 5 lines to tell arXiv to use pdfLaTeX, which is strongly recommended.
\pdfoutput=1
% In particular, the hyperref package requires pdfLaTeX in order to break URLs across lines.

\documentclass[11pt]{article}

% Change "review" to "final" to generate the final (sometimes called camera-ready) version.
% Change to "preprint" to generate a non-anonymous version with page numbers.
\usepackage[preprint]{acl}

% Standard package includes
\usepackage{times}
\usepackage{latexsym}

% For proper rendering and hyphenation of words containing Latin characters (including in bib files)
\usepackage[T1]{fontenc}
% For Vietnamese characters
% \usepackage[T5]{fontenc}
% See https://www.latex-project.org/help/documentation/encguide.pdf for other character sets

% This assumes your files are encoded as UTF8
\usepackage[utf8]{inputenc}

% This is not strictly necessary, and may be commented out,
% but it will improve the layout of the manuscript,
% and will typically save some space.
\usepackage{microtype}

% This is also not strictly necessary, and may be commented out.
% However, it will improve the aesthetics of text in
% the typewriter font.
\usepackage{inconsolata}

%Including images in your LaTeX document requires adding
%additional package(s)
\usepackage{graphicx}

\usepackage{hyperref}
\usepackage{url}
\usepackage{graphicx}
\usepackage{xcolor}
\usepackage{listings}
\usepackage{adjustbox}
\usepackage{subcaption}
\usepackage{amssymb}% http://ctan.org/pkg/amssymb
\usepackage{pifont}% http://ctan.org/pkg/pifont

% Set up listings for Python
\lstset{
    language=Python,                 % Language of the code
    basicstyle=\ttfamily\footnotesize, % Font style and size
    keywordstyle=\color{blue},        % Style for keywords
    commentstyle=\color{gray},        % Style for comments
    stringstyle=\color{red},          % Style for strings
    showstringspaces=false,           % Don't display spaces in strings
    numberstyle=\tiny\color{gray},    % Style for line numbers
    frame=single,                     % Adds a frame around the code
    breaklines=true,                  % Automatic line breaking
    captionpos=b,                     % Caption position (b for bottom)
    tabsize=4                         % Size of tabs
}

\newcommand{\pk}[1]{{\textcolor{red}{~(PK: #1)}}}
\newcommand{\tp}[1]{{\textcolor{orange}{~(TP: #1)}}}
\newcommand{\xmark}{\ding{55}}%
\newcommand{\improve}[1]{($\textcolor{green}{\uparrow #1}$)}
\newcommand{\worse}[1]{($\textcolor{red}{\downarrow #1}$)}
\newcommand{\iclr}[1]{{\textcolor{blue}{#1}}}


% This assumes your files are encoded as UTF8
\usepackage[utf8]{inputenc}
\newcommand{\fix}{\marginpar{FIX}}
\newcommand{\new}{\marginpar{NEW}}

% This is not strictly necessary, and may be commented out.
% However, it will improve the layout of the manuscript,
% and will typically save some space.
\usepackage{microtype}

% This is also not strictly necessary, and may be commented out.
% However, it will improve the aesthetics of text in
% the typewriter font.
\usepackage{inconsolata}


% If the title and author information does not fit in the area allocated, uncomment the following
%
%\setlength\titlebox{<dim>}
%
% and set <dim> to something 5cm or larger.

\title{FoREST: \textbf{F}rame \textbf{o}f \textbf{R}eference \textbf{E}valuation in \textbf{S}patial Reasoning \textbf{T}asks}

% Author information can be set in various styles:
% For several authors from the same institution:
% \author{Author 1 \and ... \and Author n \\
%         Address line \\ ... \\ Address line}
% if the names do not fit well on one line use
%         Author 1 \\ {\bf Author 2} \\ ... \\ {\bf Author n} \\
% For authors from different institutions:
% \author{Author 1 \\ Address line \\  ... \\ Address line
%         \And  ... \And
%         Author n \\ Address line \\ ... \\ Address line}
% To start a seperate ``row'' of authors use \AND, as in
% \author{Author 1 \\ Address line \\  ... \\ Address line
%         \AND
%         Author 2 \\ Address line \\ ... \\ Address line \And
%         Author 3 \\ Address line \\ ... \\ Address line}

\author{Tanawan Premsri \\
  Department of Computer Science \\
  Michigan State University\\
  \texttt{premsrit@msu.edu} \\\And
  Parisa Kordjamshidi \\
Department of Computer Science \\
  Michigan State University\\
  \texttt{kordjams@msu.edu} \\}

\begin{document}
\maketitle
\begin{abstract}
Spatial reasoning is a fundamental aspect of human intelligence. 
One key concept in spatial cognition is the Frame of Reference (FoR), which identifies the perspective of spatial expressions. 
Despite its significance, FoR has received limited attention in AI models that need spatial intelligence. There is a lack of dedicated benchmarks and in-depth evaluation of large language models (LLMs) in this area.
To address this issue, we introduce the \textbf{F}rame \textbf{o}f \textbf{R}eference \textbf{E}valuation in \textbf{S}patial Reasoning \textbf{T}asks (FoREST) benchmark, designed to assess FoR comprehension in LLMs.
We evaluate LLMs on answering questions that require FoR comprehension and layout generation in text-to-image models using FoREST.
Our results reveal a notable performance gap across different FoR classes in various LLMs, affecting their ability to generate accurate layouts for text-to-image generation. 
This highlights critical shortcomings in FoR comprehension.
To improve FoR understanding, we propose Spatial-Guided prompting, which improves LLMs’ ability to extract essential spatial concepts. 
Our proposed method improves overall performance across spatial reasoning tasks.
\end{abstract}

\section{Introduction}

\section{Introduction}

\begin{figure*}
    \centering
    \includegraphics[width=\textwidth]{figures/Introduction.pdf}
    \caption{Showing the novel problem statement applied to traffic prediction use case. Multiple unstructured observations from the past are used to reconstruct a hidden traffic state from which a full traffic state is forecast with a set of query locations. }
    \label{fig:intro}
\end{figure*}

% Was sagen denn die anderen warum Traffic Prediction gut ist? 
Forecasting the traffic in the near future is an important task for city management.
Data from the near past is used to predict future traffic states with spatio-temporal Graph Neural Networks \cite{bui22}.
Accurate prediction provides the opportunity to optimize traffic flow, reduce traffic jams and increase air quality \cite{Po19}.

% Wieso ist Sparsity in allen Dimensionen wichtig.
While traffic prediction relies on the availability of data from traffic sensors, there exists a plethora of reasons why sensors may stop working temporarily, such as simple errors, energy saving, or overloaded communication systems.
Considering small- or medium-sized cities, the coverage of sensors may be low because the sensors are too expensive or not available.
Also, the sensors are typically static and do not adapt to changes in the traffic flow (e.g. caused by a construction site), which motivates moving sensors that for example could be mounted on cars. 
However, both missing and moving sensors introduce sparsity, since measurements may not be available for all locations at all times.
This sparsity must be explicitly addressed in traffic prediction for a realistic application scenario, which is illustrated in figure \ref{fig:intro}.
From one hour of data on Sunday morning, only few observations of the traffic state are available at each timestep.
The number of observations may differ throughout the observed time and the observation itself can be distributed arbitrarily in the city. 
We assume a relatively low number of sensors to account for resource saving and sensor failure in our proposed framework SUSTeR.
The task is to predict the dense traffic state one timestep after the observations at all possible sensor locations.
We study this problem on the traffic dataset Metr-LA and PEMS-BAY to test our assumption that only a fraction of the sensor values would be enough for good predictions.
By modifying an existing traffic dataset, we are able to compare our results from very sparse observations to the bottom line with all information available.
A successful study will provide insights in how sensors in new cities can be reduced before installing them and further mobile sensors would save more resources and are able to adapt to new traffic situations.
We argue that in order to be adaptable to other cities and changes in traffic flows, prior information like the road network should be neglected and just the sparse observations considered.
This comes with the added benefit of making our solution applicable in regions where no openly available road network is maintained or pathways change frequently (e.g. flood areas, animal observations). 


The aforementioned problem is novel and more challenging than the commonly considered traffic prediction problem, since there exist very few observations in each input sample.
Current works for the traffic prediction problem do not consider any missing values. \cite{Li2021, Shao22}
A common method among state of the art approaches is the usage of Graph Neural Networks on graphs that model the sensor network \cite{bui22}.
The values of a sensor are applied to the same graph node for each timestep which prohibits any non-stationary sensors . 
With fixed sensor locations, the resulting sensor network is highly correlated with the road network.
Streets connecting two intersections with sensors should be also an interesting point for correlations in the sensor network.
However, variable observations and high temporal sparsity rates can not be modeled adequately in a static network.
We show in our experiments that the road network has only a small influence on the traffic predictions.

Besides the traffic prediction for future timesteps, some works explore the field of traffic speed imputation \cite{Cini22, Cuza22} where missing sensor values are predicted.
But the amount of missing values is assumed to be at most 80\%, which on average are still over 40 given sensors in each timestep in the Metr-LA dataset with a total of 207 sensors.
We consider up to 99.9\% missing values which are on average 2.4 observations in each timestep that are used as input.
Such high sparsity rates drastically decrease the chance that multiple values are present in one input sample from the same sensor location, which makes it challenging to recognize and learn temporal correlations for each location on its own.

High sparsity rates (>95\%) result in few sensor values, but if a reconstruction of the traffic state would be possible, we question if spatio-temporal graphs require nodes for each sensor.
In SUSTeR we utilize only a small amount of graph nodes for the encoding of information and do not relate such nodes to the sensor network.
We call this the hidden graph (see figure \ref{fig:intro}), which is still able to reconstruct the complete traffic state.
Due to the reduced number of nodes SUSTeR achieves faster runtimes, as shown in the experiments.
This hidden graph is not embedded directly in the spatial domain, which is why the assignment of observations, as well as the querying of the future traffic, is done with an encoder and a decoder, implemented as neural networks.
The decoding from the hidden graph to future values depends on a set of query locations.
Figure \ref{fig:intro} shows the query locations as given from outside and in combination with the reconstructed traffic state the future values are predicted.

To construct the hidden graph we encode observations from each timestep into from multiple graphs, one for each timestep. 
The graphs are created in a residual style and information is added to the node embeddings from the previous timesteps.
We choose this method to incorporate all timesteps equally into the hidden state because the redundant information along the past is non-existing for high sparsity rates.
From the sequence of graphs where our framework inserted the observations step by step we apply STGCN \cite{Yu18}, an algorithm for traffic prediction to find and learn the spatio-temporal correlations on our small number of graph nodes.
The first future timestep of the STGCN is our hidden graph in which the traffic state is reconstructed. 

% Recent work has an implicit embedding of the graph nodes into the spatial domain as the assignment from the sensor to graph node is fixed one by one.
% Because the graph has the same structure as the road network spatio-temporal correlations can be learned between those sensors.
% We reduce the number of nodes and use a non-linear assignment learned data-driven from the observations.

We find in the experiments that SUSTeR outperforms the plain STGCN and modern traffic prediction frameworks like D2STGNN for high sparsity rates $(\geq 99\%)$.
This is equivalent to only $0.2$ to $2.4$ observation for each timestep on average.
SUSTeR uses fewer parameters than the baselines and can train faster and with less training data.
Our main contributions can be summarized as follows:
\begin{itemize}
    \item We introduce a sparse and unstructured variant of the traffic prediction problem with sparsity in all dimensions. The sensors report only a fraction of their values and are arbitrarily distributed in the spatial domain.
    \item We propose SUSTeR, a framework around the STGCN architecture, which maps sparse observations onto a dense hidden graph to reconstruct the complete traffic state.
    Our code is available at github.\footnote{https://github.com/ywoelker/SUSTeR}
    \item We conducts experiments that show that SUSTeR outperforms the baselines in very sparse situations ($\geq 95\%$) and has a competitive performance in low sparsity rates.
    % \item SUSTeR trains a third faster than the next competitor.
\end{itemize}


%by guiding the model to focus on three key spatial relations. 
\section{Spatial Primitives}\label{sec:primitives}
We review three semantic aspects of spatial information expressed in language: Spatial Roles, Spatial Relations, and Frame of Reference.  

\noindent\textbf{Spatial Roles.} 
We focus on two main spatial roles~\citep{kordjamshidi-etal-2010-spatial} of \textit{Locatum}, and \textit{Relatum}. 
The locatum is the object described in the spatial expression, while the relatum is the other object used to describe the position of the locatum. 
An example is \textit{a cat is to the left of a dog}, where the \textit{cat} is the locatum, and the \textit{dog} is the relatum.

\noindent\textbf{Spatial Relations.} 
When dealing with spatial knowledge representation and reasoning, three main relations categories are often considered, that is, directional, topological, and distance~\citep{reasoningQualitaiveDaniel, COHN2008551,ACMpaper}. 
\textit{Directional} describes an object's direction based on specific coordinates. Examples of relations include left and right.
\textit{Topological} describes the containment between two objects, such as inside.
\textit{Distance} describes qualitative and quantitative relations between entities. Examples of qualitative are far, and quantitative are 3km.

\noindent\textbf{Spatial Frame of Reference.} We use four frames of references investigated in the cognitive linguistic studies~\cite{TENBRINK2011704}. These are defined based on the concept of \textit{Perspective}, which is the origin of a coordinate system to determine the direction. The four frames of reference are defined as follows.

\noindent1. \textit{External Intrinsic} describes a spatial relation from the relatum's perspective, where the relatum does not contain the locatum. The top-right image in Figure~\ref{fig:FoR_classes} illustrates this with the sentence, \textit{A cat is to the right of the car from the car's perspective.}

\noindent2. \textit{External Relative} describes a spatial relation from the observer's perspective.
The top-left image in Figure~\ref{fig:FoR_classes}  shows an example with the sentence, \textit{A cat is to the left of a car from my perspective.}

\noindent3. \textit{Internal Intrinsic} describes a spatial relation from the relatum's perspective, where the relatum contains the locatum. The bottom-right image in Figure~\ref{fig:FoR_classes} show this with the sentence, \textit{A cat is inside and back of the car from the car's perspective.}

\noindent4. \textit{Internal Relative} describes a spatial relation from the observer's perspective where the locatum is inside the relatum. The bottom-left image in Figure~\ref{fig:FoR_classes} show this FoR with the sentence, \textit{A cat is inside and to the left of the car from my perspective.}


\begin{figure*}[t!]
    \centering
    \includegraphics[width=0.8\linewidth, trim= {0 0 0.5 0.5cm }]{Figures/ImagePipeline.pdf}
    \caption{Pipeline for dataset creation, starting from selecting a locatum and relatum from available objects and then applying a spatial template to generate the spatial expression ($T$). FoRs are assigned based on the relatum’s properties. $T$ is then categorized based on the number of FoRs. For example, \textit{A cat is to the right of a dog }(with two possible FoRs: external intrinsic and external relative) belongs to the A-split. Then, its disambiguated version (A cat is to the right of a dog from the dog's perspective) is added to the C-split. Next, if applicable, a relatum's orientation is included for visualization and question generation. Finally, Unity3D generates scene configurations, and question-answer pairs are created from $T$.}
    \label{fig:generate_pipeline_image}
\end{figure*}



\section{FoREST Dataset Construction}\label{sec:DatasetCreation}

%%% Motivation 
%%% Goal and the task  
% Our paper aims

To systematically evaluate LLM on the frame of reference (FoR) recognition, 
we introduce the \textbf{F}rame \textbf{o}f \textbf{R}eference \textbf{E}valuation in \textbf{S}patial Reasoning \textbf{T}asks (FoREST) benchmark.
Each instance in FoREST consists of a spatial context ($T$), a set of corresponding FoR ($FoR$) which is a subset of \{\textit{external relative},  \textit{external intrinsic}, \textit{internal intrinsic}, \textit{internal relative}\}, a set of questions and answers ($\{Q,A\}$), and a set of visualizations ($\{I\}$).
An example of $T$ is \textit{A cat is to the right of a dog. A dog is facing toward the camera.}
The FoR of this expression is \{\textit{external intrinsic}, \textit{external relative}\}.
A possible question-answer is $Q$ = \textit{Based on the camera's perspective, where is the cat from the dog's position?}, $A$ = \{left, right\}. There is an ambiguity in the FoR for this expression.
Thus, the answer will be \textit{left} if the model assumes the external relative. Conversely, it will be \textit{right} if the model assumes the external intrinsic.
The visualization of this example is in Figure~\ref{fig:generate_pipeline_image}. 
%In the following, we explain how these dataset components are generated automatically.

\subsection{Context Generation}
We select two distinct objects—a relatum ($R$) and a locatum ($L$)—from a set of 20 objects and apply them to a Spatial Relation template,
\textit{<$L$> <spatial relation> <$R$>} to generate the context $T$.
FoRs for $T$ are determined based on the properties of the selected objects. Depending on the number of possible FoRs, $T$ is categorized as ambiguous (A-split), where multiple FoRs apply, or clear (C-split), where only one FoR is valid. 
We further augment the C-split with disambiguated spatial expressions derived from the A-split, as shown in Figure~\ref{fig:generate_pipeline_image}.
%\pk{The next section details the considered properties, possible relatum cases, and the clarification process.: Remove}

\subsection{Categories based on Relatum Properties} \label{sec:FoR_Relatum_scenario}
Using the FoR classes in Section~\ref{sec:primitives}, we identified two key relatum properties contributing to FoR ambiguity.
The first property is the relatum's intrinsic direction. 
It creates ambiguity between intrinsic and relative FoR since spatial relations can originate from the relatum's and observer's perspectives.
The second is the relatum's affordance as a container. 
It introduces the ambiguity between internal and external FoR, as spatial relations can refer to the inside and outside of the relatum. 
Based on these properties, we define four distinct cases: \textit{Cow Case, Box Case, Car Case, and Pen Case.}
% We extend the previous for categorizing based on the property of the object~\cite

\noindent\textbf{Case 1: Cow Case}.
In this case, the selected relatum has intrinsic directions but does not have the affordance as the container for the locatum.
An obvious example is a cow, which should not be a container but has a front and back.
In such cases, the relatum potentially provides a perspective for spatial relations. 
The applicable FoR classes are $FoR$ = \{\textit{external intrinsic}, \textit{external relative}\}.
We augment the C-split with expressions of this case but include the perspective to resolve their ambiguity.
To specify the perspective, we use predefined templates for augmenting clauses, such as \textit{from \{relatum\}'s perspective }for \textit{external intrinsic} or \textit{from the camera's perspective} for \textit{external relative}. 
For example, if the context is  \textit{A cat is to the right of the cow}, in the A-split. 
The counterparts included in the C-split are \textit{A cat is to the right of the cow from cow's perspective.} for \textit{external intrinsic} and \textit{A cat is to the right of the cow from my perspective.} for \textit{external intrinsic}. 


\noindent\textbf{Case 2: Box Case.} 
The relatum in this category has the property of being a container but lacks intrinsic directions, making the internal FoR applicable. An example is a box. 
The applicable FoR classes are $FoR$ = \{\textit{external relative}, \textit{internal relative}\}.
To include their unambiguous counterparts in the C-split, we specify the topological relation to the expressions, $T$, by adding \textit{inside} for \textit{internal relative} and \textit{outside} for \textit{external relative} cases. 
For example, for the sentence \textit{A cat is to the right of the box.},
the unambiguous $T$ with \textit{internal relative} FoR is \textit{A cat is inside and to the right of the box.} The counterpart for \textit{external relative} is \textit{A cat is outside and to the right of the box.}


\noindent\textbf{Case 3: Car Case.}  
A relatum with an intrinsic direction and container affordance falls into this case, allowing all FoR classes. An obvious example is a car that can be a container with intrinsic directions. The applicable FoR classes are $FoR$ = \{ \textit{external relative},  \textit{external intrinsic}, \textit{internal intrinsic}, \textit{internal relative}\}.
To augment C-split with this case's disambiguated counterparts, we add perspective and topology information similar to the Cow and Box cases.
An example expression for this case is \textit{A person is in front of the car.} 
The four disambiguated counterparts to include in the C-split are \textit{A person is outside and in front of the car from the car itself.} for \textit{external intrinsic}, \textit{A person is outside and in front of the car from the observer.} for \textit{external relative},  \textit{A person is inside and in front of the car from the car itself.} for \textit{internal intrinsic}, and \textit{A person is inside and in front of the car from the observer.} for \textit{internal relative}.

\noindent\textbf{Case 4: Pen Case.} 
In this case, the relatum lacks both the intrinsic direction and the affordance as a container. 
An obvious example is a pen with neither left/right nor the ability to be a container.
Lacking these two properties, the created context has only one applicable FoR, $FoR$ = \{ \textit{external relative}\}.
Therefore, we can categorize this case into both splits without any modification.
An example of such a context is \textit{The book is to the left of a pen.}


\subsection{Context Visualization}\label{sec:context_visualize}
% In our visualization, a complex subset arises when the relatum has an intrinsic direction within the intrinsic FoR. In such cases, the relatum’s orientation can introduce additional complexity to the visualization.
% In intrinsic FoR classes where the relatum has intrinsic direction, the relatum’s orientation can complicate visualization. \pk{mention: we use this complex subset for visualizaion or soemthing like that}
In our visualization, complexity arises when the relatum has an intrinsic direction within the intrinsic FoR, as its orientation can complicate the spatial representation.
For example, for visualizing \textit{A cat is to the right of a dog from the dog's perspective.}, the cat can be placed in different coordinates based on the dog’s orientation.
To address this issue, we add a template sentence for each direction, such as \textit{<relatum> is facing toward the camera}, to specify the relatum's orientation of all applicable $T$ for visualization and QA.
For instance, \textit{A cat is to the left of a dog.} becomes \textit{A cat is to the left of a dog. The dog is facing toward the camera.}
To avoid occlusion issues, we generate visualizations only for external FoRs, as one object may become invisible in internal FoR classes.
We use only expressions in C-split since those have a unique FoR interpretation for visualization. 
We then create a scene configuration by applying a predefined template, as illustrated in Figure~\ref{fig:generate_pipeline_image}.
Images are generated using the Unity 3D simulator~\cite{juliani2020unitygeneralplatformintelligent}, producing four variations per expression $T$ with different backgrounds and object positions. Further details on the simulation process are in Appendix~\ref{appendix:dataset_creation}.


% 
\subsection{Question-Answering Generation}\label{sec:QA_generation}
We generate questions for all generated spatial expressions ($T$). 
Note that we include the relatum orientation for cases where the relatum has an intrinsic direction, as mentioned in the visualization.
Our benchmark includes two types of questions. 
The first type asks for the spatial relation between two given objects from the camera's perspective, following predefined templates such as, \textit{Based on the camera’s perspective, where is the {locatum} relative to the {relatum}’s position?}
Template variations are made based on GPT4o.
The second type of question queries the spatial relation from the relatum’s perspective. 
This question type follows the same templates but replaces the camera with the relatum.
The first type of question is generated for all $T$, while the second type is only generated for $T$ where the relatum has intrinsic direction and a perspective can be defined accordingly.
Question templates are provided in Appendix~\ref{appendix:textual_template}. 
Answers are determined based on the corresponding FoRs, the spatial relation in $T$, and the relatum’s orientation when applicable.



\section{Models and Tasks} 
% We aim to evaluate language models' ability on identify FoR and its impacts on T2I models. 
The FoREST benchmark supports multiple tasks, including FoR identification, Question Answering (QA) that requires FoR comprehension, and Text-to-Image (T2I). This paper focuses on QA and T2I for a deeper evaluation of spatial reasoning. 
FoR identification experiments are provided in Appendix~\ref{appendix:FoRIdentification}.

\subsection{Question-Answering (QA)}\label{sec:QA_explanation}

\noindent\textbf{Task.}
This QA task evaluates LLMs’ ability to adapt contextual perspectives across different FoRs.
Both A and C splits are used in this task. 
The input is the context, consisting of a spatial expression $T$ and relatum orientation, if available, and a question $Q$ that queries the spatial relation from either an observer or the relatum’s perspective. 
The output is a spatial relation $S$, restricted to \{left, right, front, back\}.

% \pk{In the C-split, perspective information—introduced to clarify the A-split and create the C-split—is also included in the input.: not sure what is this. Just mention which slpit you consider for this, is it C , A or both, you have explained all before do not repeat. When you say c-split, it includes the augmented sentences from A-split too, you do not need to mention that unless you want to exclude them.} 



\noindent\textbf{Zero-shot baseline.} 
We call the LLM with instructions, a spatial context, $T$, and a question, $Q$, expecting a spatial relation as the response. 
The prompt instructs the model to answer the question with one of the candidate spatial relations without any explanations.


\noindent\textbf{Few-shot baseline.} 
We create four spatial expressions, each assigned to a single FoR class to prevent bias. Following the steps in Section~\ref{sec:QA_generation}, we generate a corresponding question and answer for each. These serve as examples in our few-shot prompting. The input to the model is instruction, example, spatial context, and the question.

\noindent\textbf{Chain-of-Thought baseline~\citep{wei2023chainofthoughtpromptingelicitsreasoning}.}
To create Chain-of-Thought (CoT) examples, we modify the prompt to require reasoning before answering.
We manually crafted reasoning explanations with the necessary information for each example we used in the few-shot setting.
The input to the model is instruction, CoT example, spatial context, and the question.



% \begin{table*}[t]
%     \tiny
%     \centering
%     \begin{tabular}{|l|c c c|c c c|c|c|c|c c c|c c c|c|}
%     \hline
%     Model & \multicolumn{3}{|c|}{Cow} & \multicolumn{3}{|c|}{Car}  & Box  & Pen & Avg. & \multicolumn{3}{|c|}{Cow}  & \multicolumn{3}{|c|}{Car}  & Avg. \\
%     \hline
    
%     Metric & I \% & R \% & Acc. & I \% & R \% & Acc. & \multicolumn{3}{|c|}{Acc.} & I \% & R \% & Acc. & I \% & R \% & Acc. & Acc. \\ 
%        \hline
%        & \multicolumn{5}{|c|}{Camera's perspective} & \multicolumn{3}{|c|}{Relatum's perspective} \\
%        \hline
%      Llama3-8B (0-shot) & $82.22$  & $79.39$ & $95.00$ & $94.06$ & $83.58$ & $65.24$ & $68.70$ & $65.73$ & $82.22$  & $79.39$ & $95.00$ & $94.06$ & $83.58$ & $65.24$ & $68.70$ & $65.73$ \\
%      Llama3-8B (4-shot) & $82.98$ & $86.07$ & $96.67$ & $92.21$ & $84.79$ & $56.78$ & $62.02$ & $57.53$ \\

%      Llama3-8B (CoT) & $52.06$ & $50.19$ & $58.33$ & $54.92$ & $52.33$ & $50.22$ & $46.56$ & $49.70$ \\
     
%      Llama3-8B (SG + COT)) & $76.30$  & $73.09$ & $66.67$ & $76.23$ & $75.63$ & $75.86$ & $75.95$ & $75.87$ \\
     
%          \hline
%      Llama3-70B (0-shot)& $62.52$ & $65.46$ & $73.33$ & $72.54$ & $64.32$ & $62.14$ & $61.83$ & $62.09$ \\
%      Llama3-70B (4-shot) & $62.23$ & $64.69$ & $85.83$ & $85.45$ & $65.83$ & $57.07$ & $61.83$ & $57.74$ \\
%      Llama3-70B (CoT) & $80.74$ & $79.58$ & $95.83$ & $94.88$ & $82.63$ & $77.15$ & $80.92$ & $77.69$ \\
%      Llama3-70B (SG + COT)& $73.57$ & $74.81$ & $100.00$ & $100.00$ & $77.47$ & $65.68$ & $67.75$ & $65.98$ \\
%      \hline
%      Qwen2-7B (0-shot) & $87.36$ & $89.31$ & $91.67$ & $93.85$ & $88.46$ & $72.34$ & $81.68$ & $73.67$ \\

%      Qwen2-7B (4-shot)& $86.66$ & $79.77$ & $87.50$ & $85.04$ & $85.66$ & $51.90$ & $59.73$ & $53.02$ \\

%      Qwen2-7B (CoT) & $80.83$ & $76.91$ & $91.67$ & $88.93$ & $81.58$ & $58.97$ & $63.55$ & $59.62$ \\
%      Qwen2-7B (SG + COT)) & $90.08$ & $93.32$ & $100.00$ & $98.57$ & $91.72$ & $75.86$ & $81.11$ & $76.60$ \\
%     \hline
%      Qwen2-72B (0-shot) & $95.56$ & $95.04$ & $100.00$ & $100.00$ & $96.13$ & $79.28$ & $83.59$ & $79.89$ \\
%      Qwen2-72B (4-shot) & $84.44$ & $85.50$ & $100.00$ & $100.00$ & $86.78$ & $78.26$ & $86.26$ & $79.40$ \\
%      Qwen2-72B (CoT) & $88.59$ & $83.40$ & $100.00$ & $100.00$ & $89.58$ & $85.46$ & $83.59$ & $85.19$ \\
%       Qwen2-72B (SG + COT)) & $88.97$ & $89.12$ & $100.00$ & $99.80$ & $90.53$ & $85.96$ & $87.40$ & $86.17$ \\ 
%      \hline
%     \end{tabular}
%     \caption{The accuracy of A-split frame of reference question-answering with various LLMs. In the Car and Cow cases, the parenthesis $(x, y)$ represents the ratio of correct answers where the model assumes $x$\% relative FoR and $y$\% intrinsic FoR for ambiguous expression.}
%     \label{tab:A_split-QA}
% \end{table*}

\begin{table*}[t]
    \setlength{\tabcolsep}{1mm}
    \small
    \centering
    \begin{tabular}{|l| c c | c | c c | c| c | c | c || c c | c | c c | c| c |}
    \hline
     & \multicolumn{9}{|c||}{Camera perspective} & \multicolumn{7}{|c|}{Relatum perspective} \\
    \cline{2-17}
     Model & \multicolumn{3}{|c|}{Cow} & \multicolumn{3}{|c|}{Car} & Box & Pen & Avg. &\multicolumn{3}{|c|}{Cow} & \multicolumn{3}{|c|}{Car} & Avg. \\
      \cline{2-17}
       & R\% & I\% & Acc. & R\% & I\% & Acc. & Acc. & Acc. & Acc. &R\% & I\% & Acc. & R\% & I\% & Acc. & Acc. \\ 
        \hline
Llama3-70B (1) & $48.1$ & $\mathbf{51.5}$ & $62.5$ & $\mathbf{58.0}$ & $41.6$ & $65.5$ & $73.3$ & $72.5$ & $64.3$  & $\mathbf{61.0}$ & $38.7$ & $62.1$ & $\mathbf{51.8}$ & $47.9$ & $61.8$ & $62.1$ \\
Llama3-70B (2) & $49.1$ & $\mathbf{50.5}$ & $62.2$ & $\mathbf{52.2}$ & $47.4$ & $64.7$ & $85.8$ & $85.5$ & $65.8$  & $\mathbf{59.6}$ & $40.1$ & $57.1$ & $\mathbf{55.5}$ & $44.2$ & $61.8$ & $57.7$ \\
Llama3-70B (3) & $49.4$ & $\mathbf{50.3}$ & $80.7$ & $49.4$ & $\mathbf{50.3}$ & $79.6$ & $95.8$ & $94.9$ & $82.6$  & $\mathbf{60.8}$ & $39.0$ & $77.2$ & $\mathbf{55.1}$ & $44.6$ & $80.9$ & $77.7$ \\
Llama3-70B (4) & $\mathbf{59.4}$ & $40.2$ & $73.6$ & $\mathbf{57.9}$ & $41.7$ & $74.8$ & $100.0$ & $100.0$ & $77.5$  & $\mathbf{60.6}$ & $39.1$ & $65.7$ & $\mathbf{56.0}$ & $43.7$ & $67.7$ & $66.0$ \\
\hline
Qwen2-72B (1) & $\mathbf{96.6}$ & $2.9$ & $95.6$ & $\mathbf{95.9}$ & $3.6$ & $95.0$ & $100.0$ & $100.0$ & $96.1$  & $8.8$ & $\mathbf{90.6}$ & $79.3$ & $7.8$ & $\mathbf{91.7}$ & $83.6$ & $79.9$ \\
Qwen2-72B (2) & $\mathbf{89.0}$ & $10.5$ & $84.4$ & $\mathbf{85.6}$ & $13.9$ & $85.5$ & $100.0$ & $100.0$ & $86.8$  & $17.7$ & $\mathbf{81.8}$ & $78.3$ & $10.4$ & $\mathbf{89.1}$ & $86.3$ & $79.4$ \\
Qwen2-72B (3) & $\mathbf{67.2}$ & $32.4$ & $88.6$ & $\mathbf{62.0}$ & $37.6$ & $83.4$ & $100.0$ & $100.0$ & $89.6$  & $21.3$ & $\mathbf{78.3}$ & $85.5$ & $22.7$ & $\mathbf{76.9}$ & $83.6$ & $85.2$ \\
Qwen2-72B (4) & $\mathbf{93.0}$ & $6.5$ & $90.1$ & $\mathbf{94.6}$ & $4.9$ & $93.3$ & $100.0$ & $98.6$ & $91.7$  & $8.2$ & $\mathbf{91.2}$ & $86.0$ & $10.5$ & $\mathbf{89.0}$ & $87.4$ & $86.2$ \\
\hline
GPT-4o (1) & $\mathbf{84.3}$ & $15.3$ & $94.5$ & $\mathbf{88.5}$ & $11.0$ & $97.3$ & $99.2$ & $99.8$ & $95.6$  & $21.6$ & $\mathbf{78.0}$ & $91.6$ & $16.1$ & $\mathbf{83.5}$ & $90.5$ & $91.4$ \\
GPT-4o (2) & $\mathbf{69.0}$ & $30.6$ & $76.6$ & $\mathbf{80.3}$ & $19.2$ & $89.5$ & $100.0$ & $100.0$ & $81.5$  & $29.0$ & $\mathbf{70.5}$ & $74.7$ & $30.9$ & $\mathbf{68.7}$ & $77.5$ & $75.1$ \\
GPT-4o (3) & $41.5$ & $\mathbf{58.3}$ & $92.3$ & $38.2$ & $\mathbf{61.6}$ & $91.0$ & $100.0$ & $99.8$ & $93.2$  & $33.9$ & $\mathbf{65.8}$ & $93.9$ & $32.0$ & $\mathbf{67.6}$ & $93.9$ & $93.9$ \\
GPT-4o(4) & $26.0$ & $\mathbf{73.9}$ & $79.2$ & $27.7$ & $\mathbf{72.1}$ & $79.4$ & $96.7$ & $94.3$ & $81.4$  & $16.2$ & $\mathbf{83.4}$ & $95.5$ & $19.2$ & $\mathbf{80.4}$ & $94.8$ & $95.4$ \\
\hline
    \end{tabular}
    \caption{QA accuracy in the A-Split across various LLMs. R\% and I\% represent the percentage the model assumes relative or intrinsic FoR for ambiguous expression explained in Section~\ref{sec:evaluation_setting}. Acc is the accuracy, and Avg is the micro-average of accuracy. (1): 0-shot, (2): 4-shot, (3): CoT, and (4): SG + CoT.}
    \label{tab:A_split-QA}
\end{table*}

% \vspace{-5mm}
\subsection{Text-To-Image (T2I)}\label{sec:t2i_models}

\noindent\textbf{Task.}  This task aims to determine the diffusion models' ability to consider FoR by evaluating their generated images. The input is a spatial expression, $T$, and the output is a generated image ($I$). We use the context from both C and A splits with external FoRs for this task.

\noindent\textbf{Stable Diffusion Models.} 
We use the stable diffusion models as the baseline for the T2I task. 
This model only needs the scene description as input. 
% \pk{Therefore, we provide $T$ to the model and expect an output image of $I$.: not needed, you can remove.}

\noindent\textbf{Layout Diffusion Models.}
We evaluate the Layout Diffusion model, a more advanced T2I model operating in two phases: text-to-layout and layout-to-image.
Given that LLMs can generate the bounding box layout~\citep{cho2023visualprogrammingtexttoimagegeneration}, we provide them with instructions and $T$ to create the layout. 
The layout consists of bounding box coordinates for each object in the format of \{object: $[x, y, w, h]$\}, where $x$ and $y$ denote the starting point and $h$ and $w$ denote the height and width. 
The bounding box coordinates and $T$ are then passed to the layout-to-image model to produce the final image, $I$. 


\subsection{Spatial-Guide Prompting}\label{sec:SG_prompting}
We hypothesize that the spatial relation types and FoR classes explained in Section~\ref{sec:primitives} can improve question-answering and layout generation.
For instance, the \textit{external intrinsic} FoR emphasizes that spatial relations originate from the relatum’s perspective.
To leverage this, we propose Spatial-Guided (SG) prompting, an additional step applied before QA or layout generation steps.
This step extracts spatial information, including direction, topology, distance as well as the FoR from spatial expression $T$. The extracted information will serve as supplementary for guiding LLMs in QA and layout generation.
We manually craft four examples covering these aspects.
First, we specify the perspective for \textit{directional relations}, e.g., \textit{left} relative to the observer, to distinguish intrinsic from relative FoR.
Next, we indicate whether the locatum is inside or outside the relatum for \textit{topological relations} to differentiate internal from external FoR.
Lastly, we provide an estimated quantitative distance to support topological and directional relation identification, such as \textit{far}.
These examples are then provided as a few-shot example for the model to extracted information automatically.
% SG prompting follows the few-shot setting that
% \pk{We provide prompting, spatial context, and SG examples for LLMs to extract spatial relations and FoR.: not clear, SG was itself the extracted information, or I am confused here.}
%  SG is the extracted information, this line only tell how to call model to generate SG


\section{Experimental Results}
\begin{table*}[ht!]
    \small
    \setlength{\tabcolsep}{1mm}
    \centering
    \begin{tabular}{|l|c|c|c|c|c||c|c|c|c|c|}
    \hline
     & \multicolumn{5}{|c||}{Camera perspective} & \multicolumn{5}{|c|}{Relatum perspective} \\
    \cline{2 - 11}
    Model & ER (CP) & EI (RP) & II (RP) & IR (CP) & Avg. & ER (CP) & EI (RP) & II (RP) & IR (CP) & Avg. \\
    \hline
     Llama3-70B (0-shot) & $44.8$ & $38.4$ & $39.7$ & $54.4$ & $42.6$ &$42.2$ & $47.1$ & $62.5$ & $34.4$ & $45.1$ \\
     Llama3-70B (4-shot) & $43.0$ & $40.0$ & $39.1$ & $47.3$ & $41.9$ & $41.8$ & $60.9$ & $77.7$ & $35.2$ & $52.0$ \\
     Llama3-70B (CoT) & $57.8$ & $46.1$ & $44.7$ & $46.0$ & $51.5$ & $\mathbf{55.5}$ & $56.8$ & $71.5$ & $49.0$ & $56.6$ \\
     Llama3-70B (SG + CoT) & $47.6$ & $42.9$ & $50.0$ & $35.6$ & $45.0$ &$55.4$ & $64.5$ & $75.0$ & $47.1$ & $60.1$ \\
     \hline
     Qwen2-72B (0-shot) & $94.5$ & $35.2$ & $31.8$ & $93.2$ & $66.9$ & $28.7$ & $89.3$ & $93.6$ & $23.8$ & $59.0$ \\
     Qwen2-72B (4-shot) & $90.2$ & $39.5$ & $39.1$ & $68.5$ & $65.3$ & $33.5$ & $92.1$ & $94.0$ & $29.5$ & $62.7$ \\
     Qwen2-72B (CoT) & $81.4$ & $57.4$ & $58.6$ & $62.5$ & $69.1$ & $39.5$ & $83.7$ & $85.2$ & $37.7$ & $61.6$ \\
     Qwen2-72B (SG + CoT) & $97.6$ & $42.5$ & $31.3$ & $93.8$ & $71.4$ & $42.8$ & $86.6$ & $92.0$ & $34.0$ & $64.5$ \\
     \hline
    GPT-4o (0-shot)  & $79.7$ & $45.1$ & $39.5$ & $90.2$ & $64.2$  & $46.9$ & $88.5$ & $98.2$ & $34.8$ & $67.5$ \\
    GPT-4o (4-shot) & $68.0$ & $52.6$ & $60.7$ & $74.1$ & $61.8$  & $44.9$ & $\mathbf{98.2}$ & $\mathbf{100.0}$ & $37.5$ & $71.2$ \\
    GPT-4o (CoT) & $81.7$ & $\mathbf{76.1}$ & $\mathbf{82.4}$ & $71.5$ & $78.8$  & $53.0$ & $91.1$ & $90.6$ & $\mathbf{50.8}$ & $71.9$ \\
    GPT-4o (SG + CoT)  & $\mathbf{97.9}$ & $72.2$ & $72.7$ & $\mathbf{93.4}$ & $\mathbf{85.8}$  & $48.9$ & $96.3$ & $95.9$ & $36.1$ & $\mathbf{71.8}$ \\
\hline
    \end{tabular}
    \caption{QA accuracy in the C-Split across various LLMs. ER, EI, II, and IR  denote external relative, external intrinsic, internal intrinsic, and internal relative FoRs. Avg represents the micro-average accuracy. CP refers to context with camera perspective, while RP denotes context with relatum perspective.}
    \label{tab:QA_c_split}
\end{table*}

\subsection{Evaluation Metrics}\label{sec:evaluation_setting}
\noindent\textbf{QA.}  We report an accuracy measure defined as follows. Since the questions can have multiple correct answers, specifically in A-split, as explained in Section~\ref{sec:DatasetCreation}, the prediction is correct if it matches any valid answer.
Additionally, we report the model’s bias distribution when FoR ambiguity exists.
$I$\% is the percentage of correct answers when assuming an intrinsic FoR, while $R$\% is this percentage with a relative FoR assumption. 
Note that cases where both FoR assumptions lead to the same answer are excluded from these calculations.

\noindent\textbf{T2I.} 
We adopt  \textit{spatialEval}~\citep{cho2023visualprogrammingtexttoimagegeneration} approach for evaluating T2I spatial ability. However, we modify it to account for FoR.
We convert all relations to a camera perspective before passing them to spatialEval, which assumes this viewpoint.
Accuracy is determined by comparing the bounding box and depth map of the relatum and locatum. 
For FoR ambiguity, a generated image is correct if it aligns with at least one valid FoR interpretation.
We report results using VISOR$_{cond}$ and VISOR$_{uncond}$~\citep{gokhale2023benchmarkingspatialrelationshipstexttoimage}, metrics for assessing T2I spatial understanding.
VISOR$_{cond}$ evaluates spatial relations only when both objects appear correctly, aligning with our focus on spatial reasoning rather than object creation. In contrast, VISOR$_{uncond}$ evaluates the overall performance, including object creation errors.

\subsection{Experimental Setting}
\noindent\textbf{QA.} We use Llama3-70B~\citep{dubey2024Llama3herdmodels}, Qwen2-72B~\citep{qwen2model}, and GPT-4o (\textit{gpt-4o-2024-11-20})~\citep{openai2024gpt4technicalreport} as the backbones for prompt engineering. 
To ensure reproducibility, we set the temperature of all models to 0.
For all models, we apply \textit{zero-shot}, \textit{few-shot}, \textit{CoT}, and our proposed prompting with CoT (SG+CoT).
%The detail of SG prompting is in Section~\ref{sec:SG_prompting}. 
%The creation of examples used for ICL is detailed in Section~\ref{sec:QA_explanation}.

\noindent\textbf{T2I.}
We select Stable Diffusion SD-1.5 and SD-2.1~\citep{rombach2021highresolution} as our stable diffusion models and GLIGEN\citep{li2023gligenopensetgroundedtexttoimage} as the layout-to-image backbone.
For translating spatial descriptions into textual bounding box information, we use Llama3-8B and Llama3-70B, as detailed in Section~\ref{sec:t2i_models}. 
The same LLMs are used to generate spatial information for SG prompting. 
We generate four images to compute the VISOR score following~\cite{gokhale2023benchmarkingspatialrelationshipstexttoimage}
Inference steps for all T2I models are set to 50.
For the evaluation modules, we select grounding DINO~\citep{liu2024groundingdinomarryingdino} for object detection and DPT~\citep{ranftl2021visiontransformersdenseprediction} for depth mapping, following VPEval~\cite{cho2023visualprogrammingtexttoimagegeneration}. The experiments were conducted on an $A6000$ GPU, totaling approximately $300$ GPU hours.

\subsection{Results}

\noindent\textbf{RQ1. What is the bias of the LLMs for the ambiguous FoR? }
Table~\ref{tab:A_split-QA} presents the QA results for the A-split. 
Ideally, a model that correctly extracts the spatial relation without considering perspective should achieve 100\% accuracy, as the context lacks a fixed perspective. 
However, this ideal model is not the focus of our work. We aim to assess model bias by measuring how often LLMs adopt a specific perspective when answering.
In the Cow and Pen case, relatum properties do not introduce FoR ambiguity in directional relations, making the task pure extraction rather than reasoning.
Thus, we focus on the $I$\% and $R$\% of the Cow and Car cases, which best reflect LLMs’ bias.
Qwen2 achieves around 80\% accuracy across all experiments by selecting spatial expressions directly from the context, suggesting it may disregard the question’s perspective.
GPT-4o shows similar bias in 0-shot and 4-shot settings but shifts toward intrinsic interpretation with CoT. This bias reduces accuracy in camera-perspective questions from 93.2\% to 81.4\%, where FoR adaptation is more challenging than relation extraction. 
Llama3-70B lacks a strong preference, balancing assumptions but slightly favoring relative FoR. This uncertainty lowers performance, requiring more reasoning to reach the correct answer.
In summary, Qwen2 achieves higher accuracy by focusing on relation extraction without considering FoR reasoning, while other models attempt reasoning but struggle to reach correct conclusions, leading to lower performance.


\begin{table*}[ht!]
    \centering
    \setlength{\tabcolsep}{1mm}
    \small
    \begin{tabular}{|l | c c c | c | c | c | c | c | c |}
    \hline
         & \multicolumn{8}{c|}{VISOR(\%)} \\ \cline{2-9}
          & \multicolumn{5}{|c|}{ A-Split } &  \multicolumn{3}{|c|}{ C-Split }\\ \hline
        Model & \multicolumn{3}{|c|}{cond (I)} & cond (R) & cond (avg) & cond (I) & cond (R) & cond (avg) \\ \cline{2-4}
        & EI FoR & ER FoR & all & & & & & \\ \hline
        SD-1.5   & $ 51.11$  & $ 21.61$  &  $ 72.72$ & $ 48.95$ & $ 68.72$ & $ 53.92$ & $ 53.77$ & $ 53.83$ \\
        SD-2.1  & $ 57.97$  & $ 21.49$ &  $ 79.46$ & $ 54.10$ & $ 75.39$ & $\mathbf{60.06}$ & $ 59.64$ & $ 59.83$ \\
        \hline
        Llama3-8B + GLIGEN& $ 53.67$  & $ 25.78$ & $ 79.45$ & $ 66.08$ & $ 77.38$ & $ 57.51$ & $ 65.98$ & $ 62.12$ \\
        Llama3-70B + GLIGEN & $ 54.49$  & $ 29.45$ & $ 83.94$ & $ 68.68$ & $ 81.43$ & $ 56.47$ & $ 69.53$ & $ 63.49$ \\
        Llama3-8B + SG + GLIGEN (Our) & $ 57.46$  & $ 27.96$ & $ 85.42$ & $\mathbf{71.14}$ & $ 83.17$ & $ 58.84$ & $\mathbf{70.36}$ & $ \mathbf{65.15}$ \\
        Llama3-70B + SG + GLIGEN (Our)  & $ 56.54$  & $ 30.59$ & $ \mathbf{87.13}$ & $ 66.56$ & $\mathbf{83.75}$ & $ 56.77$ & ${70.04}$ & ${64.06}$ \\
        \hline
    \end{tabular}
    \caption{VISOR$_{cond}$ score on the A and C splits where $I$ refer to the Cow case and Car case where relatum has intrinsic directions, and $R$ refer to the Box case and Pen case where relatum lacks intrinsic directions, $avg$ is mirco-average of $I$ and $R$. cond are explained in Section~\ref{sec:evaluation_setting}. EI and ER FoR represent the generated image considered corrected by EI or ER FoR }
    \label{tab:I_split}
\end{table*}

\begin{figure}[t]
    \centering
    \includegraphics[width=0.9\linewidth]{Figures/cf_matrix_change_perspective3.png}
    \caption{Confusion matrices of spatial relation answers when Llama3 and GPT-4o are required to adapt FoR in the 0-shot and (SG+CoT) settings.}
    \label{fig:cf_conversion}
\end{figure}

\noindent\textbf{RQ2. Can the model adapt FoR when answering the questions?}
To address this research question, we analyze QA that required FoR comprehension results in C-Split from Table~\ref{tab:QA_c_split}.
Note that the context and question in these tasks explicitly indicate a perspective.
The results indicate that LLMs struggle with FoR conversion, particularly when the question has relatum and the context has camera perspectives, achieving only up to 55.5\% accuracy.
We further demonstrate how Llama3 and GPT-4o adapt FoR using the confusion matrix in Figure~\ref{fig:cf_conversion}. 
Our findings reveal that pure-text LLM (Llama3) has confusion between left and right.
Humans typically reverse front and back while preserving left and right when describing the spatial relation from perspective.
However, Llama3 incorrectly reverses left and right, leading to poor adaptation to the camera perspective.
In contrast, very large multimodal-language models like GPT-4o follow the expected pattern, as observed by~\citealt{comfortFoR}.
While our GPT-4o results suggest some ability to convert the relatum’s perspective into the camera’s with in-context learning (72\% accuracy), the reverse transformation in the textual domain remains challenging (53\% accuracy). 
This difficulty persists when converting spatial relations from the camera perspective from images to the relatum’s perspective as observed in~\citealt{comfortFoR}.

\noindent\textbf{RQ3. How can an explicit FoR identification help spatial reasoning in QA?}
We compare CoT and CoT+SG results to evaluate how explicit FoR identification affects LLMs’ spatial reasoning in QA.
Based on C-Split results (Table~\ref{tab:I_split}), incorporating SG encourages the model to identify the perspective from input expression ranging from 2.9\% to 30\% in cases where the context and question share the same perspective. 
These cases are easier as the models do not need FoR adaptation. 
%The only exception is Llama3 for questions with the camera’s perspective, where explicit FoR identification with SG prompting negatively impacts performance.
With one exception, as can be seen, the performance level of the Llama3 baseline, for questions with camera perspective, is so poor that FoR identification in SG cannot help boost its performance compared to the improvements made in other language models.
We should note that among our selected LLMs, Llama3 is the only one not trained with visual information; we speculate this can be a factor in LLMs' understanding of FoR.
SG is less effective in reasoning when the context and question have different perspectives despite aiding in identifying the correct FoR of the spatial description in the context. 
% but does not improve reasoning for converting spatial relations across perspectives.
This limitation is evident in A-Split results (Table~\ref{tab:A_split-QA}), where models only show significant improvement when SG aligns their preference with the question’s perspective, as seen in Qwen2-72B and GPT-4o.
SG identification results are reported in the Appendix~\ref{appendix:FoRIdentification}.
Still, incorporating FoR identification improves overall spatial reasoning (see the Avg column for SG+CoT in Table~\ref{tab:I_split}).

\noindent\textbf{RQ4. How can explicit FoR identification help spatial reasoning in visualization?}
We evaluate SG layout diffusion to assess the impact of incorporating FoR in image generation. 
We focus on VISOR$_{cond}$ metric, as it better reflects the model’s spatial understanding than the overall performance measured by VISOR$_{uncond}$, which is reported in Appendix~\ref{appendix:Visor_uncond} due to space limitation.
Table~\ref{tab:I_split} shows that adding spatial information and FoR classes (SG+GLIGEN) improves performance across all splits compared to the baseline models (GLIGEN).
SG improved the model's performance when expressions can be interpreted as relative FoR.
These results align with the QA results shown in Table~\ref{tab:A_split-QA} indicating that \textit{Llama3  prefers relative FoR if dealing with the camera's perspective}.
In contrast, baseline diffusion models (SD-1.5 and SD-2.1) perform better for intrinsic FoR even though GLIGEN is based on SD-2.1.
This outcome might be due to GLIGEN's reliance on bounding boxes for generating spatial configurations, which makes it struggle with intrinsic FoR due to the absence of object properties and orientation. Nevertheless, incorporating FoR information via SG-prompting improves performance across all FoR classes despite this specific bias.
We provide further analysis on SG for the layout generation in Appendix~\ref{appedix:anaylize_SG_improment_t2i}.

%%%%%%%%%%%%%%%%*******Related Works

\section{Related Work}
\noindent\textbf{Frame of Reference in Cognitive Studies.}
The concept of the frame of reference in cognitive studies was introduced by \citealt{Levinson_2003} and later expanded with more diverse spatial relations \citep{TENBRINK2011704}.
Subsequent research investigated the human preferences for specific FoR classes~\citep{Edmonds-Wathen852956, VUKOVIC2015110, SHUSTERMAN2016115, Ruotolo2016}. For instance, \citealt{Ruotolo2016} examined how FoR influences scene memorization and description under time constraints. Their study found that participants performed better when spatial relations were based on their position rather than external objects, highlighting a distinction between relative and intrinsic FoR.

\noindent\textbf{Frame of Reference in AI.}
Several benchmarks have been developed to evaluate the spatial understanding of AI models in multiple modalities; for instance, %navigation~\citep{yamada2024evaluatingspatialunderstandinglarge}, 
textual QA~\citep{shi2022stepgamenewbenchmarkrobust, mirzaee2022transferlearningsyntheticcorpora, rizvi2024sparcsparpspatialreasoning}, and text-to-image (T2I) benchmarks~\citep{gokhale2023benchmarkingspatialrelationshipstexttoimage, huang2023t2icompbenchcomprehensivebenchmarkopenworld, cho2023dallevalprobingreasoningskills, cho2023visualprogrammingtexttoimagegeneration}.
However, most of these benchmarks overlook the frame of reference (FoR), assuming a single FoR for all instances despite its significance in cognitive studies.
Recent works in vision-language research are beginning to address this problem. 
For instance, \citealt{liu2023visualspatialreasoning} examines FoR’s impact on visual question-answering but focuses only on limited FoR categories. Our work covers more diverse FoRs.
\citealt{comfortFoR} examine FoR ambiguity and understanding in vision-language models by evaluating spatial relations derived from visual input under different FoR questions. 
Their approach relies on images from camera perspectives, with FoR indicated in the question.
In contrast, our work focuses on the reasoning of spatial relations when dealing with multiple FoRs and when there are changes in perspective in explaining the context beyond the camera’s viewpoint. 
Additionally, we show that explicitly identifying the FoR helps improve spatial reasoning in both question-answering and text-to-image generation, particularly when involving multiple perspectives.
% \tp{
% \citealt{comfortFoR} examine FoR ambiguity and understanding in vision-language models by evaluating spatial relations derived solely from visual input under different FoR questions. Their approach relies on images from camera perspectives, with FoR introduced through the question.
% In contrast, our work focuses on the role of FoR in spatial relations, analyzing how LLMs interpret different FoR classes by incorporating various perspectives beyond the camera’s viewpoint.
% Additionally, we assess the broader impact of FoR in spatial reasoning through spatial expression beyond question-answering tasks, including text-to-image generation. 
% Lastly, we demonstrate that explicitly identifying FoR can enhance spatial reasoning tasks, particularly when involving multiple perspectives.
% }


\section{Conclusion}
Given the significance of spatial reasoning in AI applications, 
we introduce \textbf{F}rame \textbf{o}f \textbf{R}eference \textbf{E}valuation in \textbf{S}patial Reasoning \textbf{T}asks (FoREST) benchmark to evaluate Frame of Reference comprehension in textual spatial expressions via question-answering and grounding in visual modality by diffusion models.
Based on this benchmark, our results reveal notable differences in FoR comprehension across LLMs and their struggle with questions that require adaptation between multiple FoRs.
Moreover, the bias in FoR interpretations impacts the layout generation with LLMs for text-to-image models. 
To improve FoR comprehension, we propose Spatial-Guided prompting, which first generates spatial relation's topological, distal, and directional type information in addition to FoR and includes this information in downstream task prompting.
Employing SG improves the overall performance in QA tasks requiring FoR understanding and text-to-image generation.

\section{Limitations}
While we analyze LLMs' shortcomings, our benchmark only highlights areas for improvement, not harming the model.
The trustworthiness and reliability of the LLMs are still a research challenge.
Our analysis is confined to the spatial reasoning domain and does not account for biases related to gender or race. However, we acknowledge that linguistic and cultural variations in spatial expression are not considered, as our study focuses solely on English. Extending this work to multiple languages could reveal important differences in FoR adaptation. 
Our analysis is still limited to the synthetic environment. Future research should consider the broader implications of the frame of reference of spatial reasoning in real-world applications.
Additionally, our experiments require substantial GPU resources, limiting the selection of LLMs and constraining the feasibility of testing larger models. The computational demands also pose accessibility challenges for researchers with limited resources.
We find no ethical concerns in our methodology or results, as our study does not involve human subjects or sensitive data. 



% Entries for the entire Anthology, followed by custom entries
\bibliography{ref}
% \bibliographystyle{acl_natbib}

\appendix
\appendix

\lstset{
  backgroundcolor=\color{gray!20}, % 20% gray background
  basicstyle=\ttfamily\footnotesize, % Monospaced font with smaller size
  breaklines=true,                 % Automatically break long lines
  frame=single,                    % Frame the listing
  framerule=0pt,                   % No frame border
  xleftmargin=5pt, xrightmargin=5pt % Add some margin around the text
}


\section{Prompt}
In this section, we list the full prompts given to the LLM, which were used in this paper.
\subsection{Prompt to Generate Questions}
\begin{lstlisting}
###Instruction###
You are assisting the audience who has received an email and needs to respond.
You're like a secretary for your audience, asking them questions and creating email responses on their behalf based on their answers.
Your goal is to make it as clear as possible what and how your audience wants to answer in response to all requirements of the email.
Therefore, you must create as specific questions as possible.
Specifically, you will assist your audience in composing emails by following 3 steps:

Step 1: Create Questions:  
    To achieve your goal, you must create well-thought-out questions without omission by considering the sender's intent and requirements. 
    The number and content of the questions must be determined with this in mind.
Step 2: Receive Answers:  
    Ask your audience the questions you created and collect their responses. 
    These will guide the crafting of the reply.
Step 3: Propose a Reply:  
    Based on the answers received, suggest a reply that your audience can edit and send.
    From now on, you will perform step 1.  

You must consider the following 7 matters in generating your response.

1. You must create questions with choices for your audience and output the results in JSON format.
2. The questions must be created in the native language of your audience.
3. If necessary, your audience can write any free answers to your questions, so you will be penalized if you create an "other" option.
4. In 'corresponding_part', you must quote a part of the provided 'Incoming Mail' verbatim. That is, output corresponding_part = IncomingMail_HTML[x:x+h]. 
5. You must quote spaces, `<br>`, periods, and commas exactly as in the provided 'Incoming Mail'. 
6. You will be penalized if you edit or combine multiple parts of the 'Incoming Mail' for your questions.
7. You will be penalized if you create unhelpful questions to compose a reply. You must keep the number of questions to a minimum.

I'm going to tip $100 for a better solution!
Ensure that your output is unbiased and avoids relying on stereotypes.

###Output JSON Format###
{
    "questions": [
        {
            "id": "1",
            "question": "Will you participate in the event on October 24th?", 
            "choices": ["Yes", "No"], 
            "corresponding_part": "We will hold an event on October 24th."
        },
        {
            "id": "2",
            "question": "Please select the available dates (multiple selections possible).", 
            "choices": ["July 10th", "July 11th", "July 12th", "July 13th", "July 14th", "July 15th", "July 16th"], 
            "corresponding_part": "Please let us know your available dates within a week."
        }
    ]
}
\end{lstlisting}

\subsection{Prompt to Generate Reply Draft}
Below is the prompt used to generate a balanced-length description.
\begin{lstlisting}
Please provide a draft reply to the sender of this email on behalf of the user.
\end{lstlisting}

% \section{Proposed LLM-Powered QA-Based Approach: ResQ}
% % 本セクションでは、電子メール対応タスクをサポートするために提案されたアプローチ「ResQ」について説明する
% This section describes the proposed approach, ResQ, for supporting email response tasks. 
% % \begin{figure*}[t]
\centering
\includegraphics[width=\textwidth]{figure/overview_of_process.pdf}
\caption{The overview of the process of creating a reply message using ResQ. A) The LLM first generates multiple-choice questions in JSON format. B) Users select their desired responses to their counterparts. C) The LLM then generates a reply draft in JSON format based on the users' selections. D) Finally, users review and edit the LLM-generated draft before sending the reply.}
\label{fig_overview_of_process}
\Description{This figure illustrates the process of generating an email reply using a large language model (LLM) across four stages. In Stage A (Generate Questions), the system takes the email data and a prompt, which are then sent to the LLM server. The server processes this information and generates a set of questions to clarify the content of the reply. These questions, along with possible answer choices and context, are returned in JSON format. In Stage B (Answer Questions), the user is presented with the questions generated by the LLM. These questions may include simple "Yes" or "No" options or multiple-choice selections. The user answers the questions by choosing the appropriate options or providing custom responses. In Stage C (Generate Reply Draft), the user’s answers, along with the original email data and prompt, are sent back to the LLM server. Based on this input, the server generates a draft of the email reply, which is also returned in JSON format. In Stage D (Check Reply Draft), the user reviews the draft generated by the LLM. After checking the content and making any necessary revisions, the user finalizes and sends the email reply.}
\end{figure*}
% % \begin{figure*}[t]
\centering
\includegraphics[width=\textwidth]{figure/interface.pdf}
\caption{Interface of ResQ. On the left, the content of the email is displayed, with an editor and a ``Reply'' button below for sending a reply. In the center, questions and options for users are shown, allowing the creation of custom options if needed. Additionally, the section of the email corresponding to the selected question is highlighted. On the right, fields are provided to customize the reply generated by the LLM, including options to specify the relationship with the counterpart and buttons to choose the formality, tone, and length of the email. A free-text input field and a "Generate Reply" button are also below.
}
\label{fig_interface}
\Description{This figure illustrates the interface of the ResQ system, which is divided into three main sections. On the left side (A, B), the content of the incoming email is displayed. The email includes important information, such as the sender, subject, and body text. Key sections of the email are highlighted based on the questions generated by the system, helping the user focus on relevant points. Below the email, there is an editor where the user can compose their reply, with a "Reply" button (H) available to send the response once it's ready. In the center section (C, D), the system displays questions generated by the AI, which are intended to assist the user in composing their reply. These questions correspond to specific parts of the email, and as the user answers them, the relevant section in the email is highlighted (A). Users can also customize responses by adding new options if needed (D). user can specify their relationship with the email recipient (e.g., professor or student), adjust the formality and tone of the response, and select the desired length of the reply. An additional free-text input field is available for further customization requests (E). Once all preferences are set, the user can click the "Generate Reply" button (F) to produce a draft response based on their inputs.}
\end{figure*}
% % 図は、ResQのアプローチの全体的な流れについて説明している
% % (A) ユーザが返信作業を開始したことを検知すると、大規模言語モデル (本研究ではGPT-4o) を使用して、多肢選択式の質問を生成する。
% % ユーザは、この質問に対して回答を行うことで、自身の返信方針をAIに伝える
% % (B) 最後に、ユーザが"Generate Reply'' buttonを押したことを検知したら、返信のドラフトをユーザに提示する
% % 図は、実際のResQのinterfaceである
% % 以下のセクションでは、それぞれの手順における具体的な機能について説明する
% Fig.~\ref{fig_overview_of_process} illustrates the overview of the process of a reply message using ResQ.
% % (A) After the system detects users' initiation of the reply task, it generates multiple-choice questions using an LLM (in this study, GPT-4o~\cite{GPT4o}). 
% % (B) Then, users communicate their reply strategy to the AI by responding to these questions.
% % (C) When the system detects that users have pressed the ``Generate Reply'' button, it presents a draft of the reply to users.
% % (D) Finally, users review and revise the reply draft generated by the LLM and send the response.
% Fig.~\ref{fig_interface} shows the actual interface of ResQ.
% The following sections describe the specific functions involved in each step of this process.

% \paragraph{\textbf{A: Generate Questions}}
% \label{sec:generate_questions}
% % ResQは返信の必要性を検知すると、大規模言語モデル (本研究ではGPT-4o) を使用して、多肢選択式の質問を生成する
% % また我々は、ユーザが質問をクリックすると、その質問が受信メールのどの部分に対応しているかがハイライトされるようにした
% % 我々はこれらの機能を実現するために、まずモデルに対して、モデルの役割(ユーザに対する質問と有益そうな選択肢のペアを複数生成すること)と、作成する質問の目的(メールに含まれる全ての要求を抽出し、送信者がそれぞれに対してどのように返答したいかを明確にすること)をプロンプトとして与えた
% % さらに、モデルに受信メールの文章、タイトル、送信者の情報(名前、メールアドレス)、受信メールの過去のやり取りの文章、ユーザの情報(名前、メールアドレス)を提供した
% When the first activates ResQ to reply to an email, the system uses an LLM (in this study, GPT-4o~\cite{GPT4o}) to generate multiple-choice questions (Fig.~\ref{fig_interface}-C). 
% Additionally, when users click on the generated question, the relevant part of the email is highlighted (Fig.~\ref{fig_interface}-A).
% To implement these features, we first provided the LLM with the email's text, subject, sender information, text from past email interactions, and the user's information (name and email address).
% The LLM then extracts all parts that require a user's reply, presents the possible response options, and generates a question for the user. 
% % プロンプトは~\cite{relatedwork}を参考に作成し、文脈を踏まえており、返信を作成する上で役に立ち、適切な数の(メールのすべての要件に対しては漏れなく)質問と、それに役立つ選択肢を生成するように指示した。
% % またプロンプトには質問と選択肢の生成例を含めることで、XXXした。
% \red{We designed the prompt with reference to~\cite{bsharat2023principled}, incorporating contextual considerations to ensure it effectively supports reply composition. 
% We instructed the LLM to generate an appropriate number of questions that comprehensively cover all the email's requirements without omissions, along with relevant response options. 
% To further guide the model and improve output accuracy, we included examples of question and option generation within the prompt.}
% The detailed prompt used for this function is shown in the appendix. 
% The prompt used for this function is shown in the appendix. 

% \paragraph{\textbf{B: Answer Questions}}
% % 次にユーザは受信メールと生成された質問、選択肢を同時に見ながら、質問に回答する
% % 我々は有益な選択肢がない場合を想定して、ユーザ自身が選択肢を追加できるようにした
% % また、LLMにメールのcontextを伝えることができるように、送信者と受信者の関係性を記入できるboxを設置した
% % さらに以前の研究に基づき~\cite{fu2024text}、ユーザがAIメールの文章をカスタマイズできるように、ユーザが期待する返信のトーンやスタイル、長さを調整するための選択肢を提供した
% % また、ユーザがそれ以外のリクエストをAIに対してできるように、AIに対する自由記述欄を設置した
% % ユーザはこれらの作業が完了するとGenerate Replyボタンを押す
% Next, users view the incoming email (Fig.~\ref{fig_interface}-B) alongside the generated questions (Fig.~\ref{fig_interface}-C) and options and proceed to answer them. 
% In anticipation of situations where none of the provided options are useful, we enabled users to add their own options (Fig.~\ref{fig_interface}-D). 
% Additionally, to help the LLM better understand the context of the email, we introduced a box where users can specify the relationship between the sender and the recipient (Fig.~\ref{fig_interface}-E, top). 
% Furthermore, following previous research~\cite{fu2024text}, we provided users with controls to adjust the tone, style, and length of the reply to match their preferences better, thereby giving them more flexibility in customizing the AI-generated response (Fig.~\ref{fig_interface}-E, middle). 
% A free-text field was also included to allow users to make other specific AI requests (Fig.~\ref{fig_interface}-E, bottom). 
% After completing these steps, users can click the ``Generate Reply'' button (Fig.~\ref{fig_interface}-F).

% \paragraph{\textbf{C: Generate Reply Draft}}
% % ResQはユーザが"Generate Reply"ボタンを押したことを検知すると、大規模言語モデルを使用して、返信のドラフトを作成する
% % ユーザの期待するような返信のドラフトが出力されるように、我々はモデルに対して、受信メールとその関連情報、AIの質問とそれに対応するユーザの回答、ユーザが返信案に期待する他の要素(トーン、スタイル、長さ、その他の要望)、ユーザの情報を提供した
% When the user clicks the ``Generate Reply'' button, ResQ detects the action and uses the LLM to generate a reply draft.
% To ensure that the draft aligns with users' expectations, we first provided the LLM with the information provided when Sec.~\ref{sec:generate_questions}, the generated questions, corresponding users' answers, and users' preferences (\textit{e.g.}, tone, style, length, and any additional requests).
% Then, the LLM generates a draft of the reply.
% The prompt used for this function is shown in the appendix.

% \paragraph{\textbf{D: Review Reply Draft}}
% Once the draft reply is generated, users can review the draft in detail (Fig.~\ref{fig_interface}-G).
% Moreover, if users find that extensive revisions are needed or if they want to explore alternative phrasing, they have the option to request the AI to regenerate a new draft based on updated input or preferences.
% After completing these steps, users can click the ``Reply'' button (Fig.~\ref{fig_interface}-H).

% \section{User Comments in Study 1}
% % We added a new subsection, "User Comments," to present follow-up interview results, including participant feedback on useful questions (Sec.6.4). 
% \subsection{Participants' Email-Replying Process (RQ1)}
% \paragraph{Enhanced Efficiency and Reduced Cognitive Load when Replying to Emails (H1-a, H1-b)}
% % 参加者のコメントから、QA-based conditionは期待通り機能し、参加者の効率向上や負担低減に貢献したことが確認できます。
% % 質問で要点をまとめてくれ、メール本文の対応箇所がハイライトされてたので、メールの理解の効率が上がり、負担が減りました [P10]
% % QA-based条件では、Prompt作成の技術がなくても、期待する出力を簡単に得ることができました。[P6]
% % Prompt-based条件では、結局自分で相手のメールを全て読み、回答すべきことを整理する必要がありました [P4]
% % Prompt-based条件では、自分でAIに対する指示を一から考える必要があり、手動の条件と効率や負担に差を感じませんでした。一方でQA-based条件は、圧倒的に早く返信を作成することができました。[P5]
% \red{
% Participants' comments confirmed that the QA-based improved efficiency and reduced workload when replying to emails. 
% P10 explained, \textit{``In the QA-based condition, AI summarized key points through questions and highlighted relevant sections of the email body, which facilitated my understanding of the email and reduced my overall burden''}.
% P6 shared, \textit{``In the QA-based condition, I could easily obtain the desired output even without the technical skills to create prompts''}.
% In contrast, the Prompt-based condition required extra effort. 
% P4 noted, \textit{``In the Prompt-based condition, I had to read the counterpart's email completely and decide what to respond to''}. 
% P5 elaborated, \textit{``In the Prompt-based condition, I had to think of instructions for the AI from scratch, making it feel no different from the No-AI condition in terms of efficiency and workload. On the other hand, the QA-based condition allowed me to compose responses faster''}.
% }

% \paragraph{Reduced Difficulty in Initiating the Action for Replying to Emails (H1-d)}
% % 参加者のコメントから、QA-based conditionでは、全体的な労力が下がるとともに、作業開始時のAIによるの質問が、タスク開始の障壁の低下に役立つことがわかりました。
% % 最初にAIが質問を投げかけてくれるので、作業開始時の思考の労力がなくなり、作業に取り掛かる際のストレスが減りました[P10]
% % 相手の文章を読む時間が省けたことで、作業開始の心理的障壁が下がりました。[P5]
% \red{
% Comments from participants indicated that the QA-based condition helped lower the barriers to task initiation through AI-generated questions at the start of the process. 
% P10 explained, \textit{``Since the AI prompted me with questions at the beginning, the mental effort required to start thinking about the task was eliminated, reducing the stress associated with initiating the work''}. 
% P5 noted, \textit{``By saving the time needed to read the counterpart's text, the psychological barrier to starting the task was lowered''}.
% }

% \paragraph{Decreased Sense of Agency and Control (H1-e)}
% % QA-based conditionでは、agencyやcontrolの感覚が他の条件に比べて低下したと回答した参加者は、次のようにコメントした。
% % 「agencyとcontrolの感覚は、プロンプトを自分で打った量に比例しました。」[P7, 8, 9, 10]
% % 「QA-based条件では、要点も絞ってくれたので、AIに任せようという思いが強くなりました。」[P4]
% % 一方で感覚が変化しなかったと回答した参加者は「AIに任せても、自分で確認と修正を行ったので、agencyやcontrolの感覚に変化はありませんでした」[P3]と説明した。
% \red{
% Participants reported a decrease in their sense of agency and control in the QA-based condition.
% \textit{``The sense of agency and control was proportional to the amount of text I typed myself''} [P7, P8, P9, P10]. 
% \textit{``Under the QA-based condition, since the AI helped narrow down the key points, I felt a stronger inclination to leave the task to the AI''} [P4].
% On the other hand, one participant who reported no change in their sense of agency or control explained, \textit{``Even though I relied on the AI, I reviewed and edited the output myself, so there was no change in my sense of agency or control''} [P3].
% }

% \paragraph{Future Preference (H1-c)}
% % 多くの参加者はメール返信の効率が上がる、負担が減る、質が高いメールを執筆できるという理由から、QA-based conditionで返信をしたいと回答しました。
% % 自分で返信を考えるより、AIを使った方が、質の高い返信を早く作ることができました。特にQA-based conditionではその効果が大きかったので、将来はQA-based conditionで返信したいと思いました。[P6]
% % 一方でsense of agencyの低下や、AIへの依存の危惧を理由に、QA-based conditionでのメール返信を忌避する参加者もいました。
% % 時間がない時や、スピードを重視したい時[P4]、重要度が低い時[P5]は、QA-based conditionで返信をしたいと思ったが、そうでない場合は自分で執筆したいと思いました。
% % 「Prompt-based条件では、QA-based条件より思考する必要が多く、それが楽しかったです」 [P7]
% % 返信作業が楽にはなりましたが、メールを細部まで読まなくなり、内容が頭に入っていない感じがしたので、将来使いたいとは思いませんでした。[P11]
% \red{
% Participants expressed a preference for using the QA-based condition for email responses, citing increased efficiency, reduced workload, and the ability to produce high-quality emails as the primary reasons. 
% One participant explained, \textit{``Using AI allowed me to compose high-quality responses faster than if I had written them myself. The effect was particularly significant in the QA-based condition, which is why I would prefer to use it in the future''} [P6].
% }
% \red{However, some participants were hesitant to adopt the QA-based condition due to concerns about a reduced sense of agency or over-reliance on AI. 
% Participants noted that they preferred the QA-based condition \textit{``when time is limited or speed is important''} [P4] or \textit{when the email is of low importance''} [P5], but in other situations, they favored writing responses themselves.
% One participant reflected, \textit{``In the prompt-based condition, I found that I needed to think more actively compared to the QA-based condition, and I enjoyed that process''} [P7].
% Another observed, \textit{``While the QA-based condition made responding easier, I felt that I was no longer fully reading and absorbing the content of emails, which made me hesitant to use it in the future''} [P11].
% }

% \paragraph{Quality of AI-generated Questions and Options}
% % 参加者はQA-based conditionにおいて生成された質問や選択肢について、有益であったものとそうでなかったものがあったとコメントした。
% % 参加者は有益でない質問の例として、メール送信者の意図を汲み取れていないもの [P2, P11]、自分と相手の立場を勘違いしているもの [P4, P8]を挙げた。
% % またある参加者は、「自分が言いたいことに関連する質問がない場合、質問機能自体が役に立たなかった」[P8]と回答した。
% % また参加者は、質問の数についても意見を述べた
% % 参加者は「用件ごとに質問を生成してくれたのが、メールを理解する上で役に立った」[P4, P5, P8, P9, P10, P11]と回答した一方で、「質問が多すぎると煩雑に感じることもあった。またそれに全て答えると、返信が冗長になってしまった。」[P7, P12]と回答する参加者もいた。
% % また、参加者は選択肢についても意見を述べた
% % 参加者は、「(日程調整のシチュエーションにおいて)選択肢の中に自分が選びたい日程がなかったので、自分で日程を入力する必要があった」[P2]と説明し「より多くの選択肢を生成して欲しかった」と説明した[P8]。
% % 一方である参加者は、「必要以上に多くの選択肢があったときに、煩わしさを感じた」[P4]と説明した。
% \red{
% Participants commented on the questions and options generated by the QA-based condition, noting that some were useful while others were not. 
% Examples of less useful questions included those that failed to accurately capture the sender's intent [P2, P11] or misinterpreted the relationship between the sender and recipient [P4, P8]. 
% Additionally, one participant remarked, \textit{``when there were no questions related to what I wanted to say, the question feature itself was not helpful''} [P8].
% }

% \red{
% Participants also shared mixed opinions on the number of questions generated. 
% Some participants noted that \textit{``generating questions for each topic helped understand the email''} [P4, P5, P8, P9, P10, P11].
% However, others felt \textit{``an excessive number of questions felt overwhelming, and responding to all of them made the reply unnecessarily lengthy''} [P7, P12].
% }

% \red{
% Feedback on the generated options was similarly divided. 
% For instance, in scheduling scenarios, one participant shared that \textit{``none of the suggested dates matched what I wanted, so I had to input the date myself''} [P2] and another participant \textit{``wished for more options or a more flexible input type''} [P8].
% In contrast, another participant stated that \textit{``having more options than necessary felt burdensome''} [P4].
% }

% \subsection{Quality of the Email Responses (RQ2)}
% % ほとんどの参加者は、AIを使うと、構造・丁寧さ・言葉遣いが改善され、全体的に良い文章を書けたと述べた
% % また参加者は、「Prompt-based条件だと、相手の要求を見落としていたかもしれないが、QA-based条件では自信を持って返信を作成することができた」 [P2, 4]と述べた
% % さらにある参加者は、「QA-based条件では、回答してもしなくても良いこと「XXの件、承知しました、など。」にも丁寧に返答を書いてくれた」 [P9]と述べ、QA-based条件によってメールの丁寧さが向上したことを強調した
% \paragraph{Scaffolding a structured response}
% \red{
% Most participants stated that using AI improved their writing structure, politeness, and choice of words, ultimately enabling them to produce better overall responses. 
% Furthermore, participants remarked, \textit{``Under the prompt-based condition, I might have overlooked the recipient's requests, but under the QA-based condition, I was able to craft responses with confidence''} [P2, P4]. 
% Additionally, one participant emphasized that \textit{``Under the QA-based condition, the AI even provided polite responses to matters where a reply was optional, such as acknowledging something with phrases like 'I Understood regarding XX, etc.'''} [P9], highlighting how the QA-based condition enhanced the politeness of email communication.
% }

% \subsection{Relationship between Participants and Their Counterpart (RQ3)}
% % 参加者は、``相手との間に知覚する心理的距離は労力に比例した''と報告し、PXXは``特にQA-based条件では選択肢を選ぶだけだった相手のことを考えることが少なかった''と報告した。
% % 一方でPXXは、``自分で返信を考えるより、AIを使うと相手に良い印象を与えられるメッセージを作ることができたので、関係性を近く感じた''と報告した
% \red{
% Participants shared differing views on how AI's involvement affected their psychological distance from their counterparts.
% Several participants reported that the psychological distance they felt from the other person was directly related to the amount of effort they put in [P2, P9, P11].
% % \textit{``the psychological distance I perceived from the counterpart was proportional to the effort exerted''} [P2, P9, P11].
% Furthermore, P6 noted that \textit{``especially under QA-based condition, I barely thought about the counterpart because I only selected options to create responses''}.
% In contrast, P8 reported that \textit{``compared to composing replies myself, using AI allowed me to create messages that left a better impression on my counterpart, which made the relationship feel closer''}.
% }
% % \subsubsection{Psychological Distance between Participants and Their Counterpart (H3-b)}

\section{\blue{Order Effect in Study 1}}
\blue{We conducted analyses to examine whether the order of conditions influenced various dependent variables. 
Table~\ref{tab_ordereffects} summarizes the results of the order effect and its interaction with the condition. 
Depending on the nature of the data, we employed either a Mixed-Design ANOVA or the Aligned Rank Transform (ART) method.}

\begin{table*}[t]
\caption{Order effects in Study 1 (* indicates significance at the 0.05 level).}
\Description{The table presents the order effects observed in Study 1, examining whether the order of conditions influenced various measurements. It reports p-values for the main effect of Order and the interaction effect between Condition and Order, with asterisks indicating statistical significance at the 0.05 level. The statistical method used for each measurement is also specified, including Mixed-Design ANOVA and Mixed-Design ANOVA with Aligned Rank Transform (ART). For Efficiency of Replying to Emails, the Order effect has a p-value of 0.643, and the Condition × Order interaction has a p-value of 0.454, analyzed using Mixed-Design ANOVA. Prompt Character Count shows an Order effect p-value of 0.052, close to significance, while the Condition × Order interaction has a p-value of 0.186, also analyzed using Mixed-Design ANOVA. Raw TLX, which measures subjective workload, has a significant Order effect (p < 0.05), indicating that workload perception is influenced by order, whereas its Condition × Order interaction is non-significant (p = 0.978), analyzed using Mixed-Design ANOVA. For Difficulty in Understanding Email Content, the Order effect p-value is 0.188, and the Condition × Order interaction p-value is 0.232, analyzed using Mixed-Design ANOVA with ART. Satisfaction with Completing Tasks shows non-significant effects, with an Order effect p-value of 0.348 and a Condition × Order interaction p-value of 0.175, analyzed using Mixed-Design ANOVA. Similarly, Difficulty for Task Initiation has an Order effect p-value of 0.131 and a Condition × Order interaction p-value of 0.515, analyzed using Mixed-Design ANOVA with ART. Psychological perceptions are also analyzed. Sense of Agency shows non-significant results, with an Order effect p-value of 0.982 and a Condition × Order interaction p-value of 0.911, analyzed using Mixed-Design ANOVA with ART. Sense of Control similarly has an Order effect p-value of 0.825 and a Condition × Order interaction p-value of 0.871, analyzed using Mixed-Design ANOVA with ART. Regarding Perceived Quality of the Email, the Order effect is 0.930, and the Condition × Order interaction is 0.433, analyzed using Mixed-Design ANOVA. Perceived Impression of Participants also shows non-significant effects, with an Order effect p-value of 0.963 and a Condition × Order interaction p-value of 0.481, analyzed using Mixed-Design ANOVA. Finally, Psychological Distance presents an Order effect p-value of 1.000, indicating no effect of order, but its Condition × Order interaction is significant (p < 0.05), suggesting that condition effects on psychological distance are influenced by order. This measurement is analyzed using Mixed-Design ANOVA with ART.}
\label{tab_ordereffects}
\blue{
\begin{tabular}{cccc}
\hline
Measurements                              & Order (p-value) & Condition $\times$ Order (p-value) & Statistical Method          \\ \hline
Efficiency of Replying to Emails          & 0.643            & 0.454                        & Mixed-Design ANOVA          \\
Prompt Character Count                    & 0.052           & 0.186                       & Mixed-Design ANOVA          \\
Raw TLX                                   & $<0.05$*         & 0.978                       & Mixed-Design ANOVA          \\
Difficulty in Understanding Email Content & 0.188           & 0.232                       & Mixed-Design ANOVA with ART \\
Satisfaction with Completing Tasks        & 0.348           & 0.175                       & Mixed-Design ANOVA          \\
Difficulty for Task Initiation            & 0.131           & 0.515                       & Mixed-Design ANOVA with ART \\
Sense of Agency                           & 0.982           & 0.911                       & Mixed-Design ANOVA with ART \\
Sense of Control                          & 0.825           & 0.871                       & Mixed-Design ANOVA with ART \\
Perceived Quality of the Email            & 0.93            & 0.433                       & Mixed-Design ANOVA          \\
Perceived Impression of Participants      & 0.963           & 0.481                       & Mixed-Design ANOVA          \\
Psychological Distance                    & 1.000           & $<0.05$*                     & Mixed-Design ANOVA with ART \\ \hline
\end{tabular}
}
\end{table*}

\blue{The results indicate that the order effect was not significant for most dependent variables. 
However, a significant effect was observed for Raw TLX ($p = 0.041$), suggesting that task load perception may have been influenced by the presentation order. 
Additionally, a significant interaction effect between condition and order was found for IOS ($p = 0.043$), indicating that the order of presentation might have impacted this specific measure.}

\section{Future Preference in Study 1}
\begin{figure*}[ht]
\centering
\includegraphics[width=\textwidth]{figure/study1_pref.pdf}
\caption{Participants' future preferences in Study 1. Significant differences between conditions were identified through post-hoc analysis following the Friedman test (* indicates significance at the 0.05 level).}
\label{fig_study1_preference}
\Description{The figure displays participants' future preferences using box plots for three experimental conditions: No-AI, Prompt-based, and QA-based. For the No-AI condition, the median overlap score is 3.5, with the first quartile at 1.75 and the third quartile at 4.25. The scores range from a minimum of 1 to a maximum of 7. For the Prompt-based condition, the median overlap score is 6, with the first quartile at 3 and the third quartile at 7. The scores range from 2 to 7. For the QA-based condition, the median overlap score is 6, with the first quartile at 6 and the third quartile at 7. The scores range from 3 to 7. Significant differences were observed between the No-AI condition and both the Prompt-based and QA-based conditions (p < 0.05).}
\end{figure*}
\red{We evaluated participants' preferences for future use across all conditions using a 7-point Likert scale.
Participants rated their agreement with the statement, ``I would prefer to use this approach for replying to emails in the future,'' where 1 indicates strongly disagree, 4 indicates neutral, and 7 indicates strongly agree.}

\red{The questionnaire survey results about participants' future preferences are shown in Fig.~\ref{fig_study1_preference}.
According to the Friedman test, a significant difference in participants' future preferences was observed among the three conditions $(\chi^2(2)=8.8, p=0.012, W=0.37)$.
Post-hoc analysis using the Durbin-Conover test with Holm correction revealed that participants would prefer responding in the QA-based condition compared to the No-AI condition $(p=0.012, r=0.67)$.
However, no significant difference was found between the Prompt-based and QA-based conditions $(p=0.800, r=0.27)$.}

\section{Technical Details of ResQ}
\begin{figure*}[ht]
\centering
\includegraphics[width=\textwidth]{figure/UI.pdf}
\caption{UI of the Gmail Reply Box with the ``Reply with AI’’ Feature, used in Study 2. Pressing the ``Reply with AI’’ button opens the window shown in Fig.~\ref{fig_interface}}
\label{fig_UI}
\Description{This figure illustrates the user interface of the Gmail reply box as enhanced by the prototype system. The Reply with AI button, shown in blue on the right-hand side of the toolbar, allows users to activate the AI-assisted reply generation feature. When the button is clicked, the system extracts the email content and opens the window shown in Fig.~\ref{fig_interface}. The standard Gmail toolbar options, such as send, formatting, and attachment icons, remain.}
\end{figure*}
\red{In this section, we provide the specific implementation details and user interface used in Study 2.
We developed a prototype system consisting of a Chrome extension and a backend service to enable participants to reply to emails using Gmail on a PC.
The Chrome extension detected the initiation of the reply task when participants clicked the ``Reply with AI'' button in the Gmail reply box (see Fig.~\ref{fig_UI}).
Upon clicking the button, the extension extracted the email content directly from Gmail’s DOM structure using JavaScript and sent it to a backend API endpoint implemented with FastAPI~\footnote{\url{https://fastapi.tiangolo.com}}.
The backend, hosted on an AWS EC2 instance~\footnote{\url{https://aws.amazon.com/ec2/}}, received the email content and forwarded it to the OpenAI API~\footnote{\url{https://platform.openai.com/docs/}} to generate questions or reply suggestions. 
These outputs were then returned to the Chrome extension and displayed to participant in a new reply editor.
Finally, participants revise the reply suggestions and submit them back to the Gmail reply box by clicking the ``Reply'' button.
To ensure privacy, neither the email content nor the participants' responses were accessible to the experimenters or stored on the server.}

\red{Additionally, to implement ResQ's features for generating questions and options, we provided the LLM with various contextual inputs, including the email text, subject, sender information, text from prior email interactions, and the user's details (such as name and email address). 
Furthermore, to ensure that the generated draft aligned with user expectations, the LLM was further given information outlined in Sec.~\ref{sec:generate_questions}, including the generated questions, corresponding user answers, and user preferences (\textit{e.g.}, tone, style, length, and any specific requests). 
Based on this input, the LLM produced a draft of the email reply.}

\end{document}
