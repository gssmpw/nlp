%%% Motivation 
%%% Goal and the task  
% Our paper aims

To systematically evaluate LLM on the frame of reference (FoR) recognition, 
we introduce the \textbf{F}rame \textbf{o}f \textbf{R}eference \textbf{E}valuation in \textbf{S}patial Reasoning \textbf{T}asks (FoREST) benchmark.
Each instance in FoREST consists of a spatial context ($T$), a set of corresponding FoR ($FoR$) which is a subset of \{\textit{external relative},  \textit{external intrinsic}, \textit{internal intrinsic}, \textit{internal relative}\}, a set of questions and answers ($\{Q,A\}$), and a set of visualizations ($\{I\}$).
An example of $T$ is \textit{A cat is to the right of a dog. A dog is facing toward the camera.}
The FoR of this expression is \{\textit{external intrinsic}, \textit{external relative}\}.
A possible question-answer is $Q$ = \textit{Based on the camera's perspective, where is the cat from the dog's position?}, $A$ = \{left, right\}. There is an ambiguity in the FoR for this expression.
Thus, the answer will be \textit{left} if the model assumes the external relative. Conversely, it will be \textit{right} if the model assumes the external intrinsic.
The visualization of this example is in Figure~\ref{fig:generate_pipeline_image}. 
%In the following, we explain how these dataset components are generated automatically.

\subsection{Context Generation}
We select two distinct objects—a relatum ($R$) and a locatum ($L$)—from a set of 20 objects and apply them to a Spatial Relation template,
\textit{<$L$> <spatial relation> <$R$>} to generate the context $T$.
FoRs for $T$ are determined based on the properties of the selected objects. Depending on the number of possible FoRs, $T$ is categorized as ambiguous (A-split), where multiple FoRs apply, or clear (C-split), where only one FoR is valid. 
We further augment the C-split with disambiguated spatial expressions derived from the A-split, as shown in Figure~\ref{fig:generate_pipeline_image}.
%\pk{The next section details the considered properties, possible relatum cases, and the clarification process.: Remove}

\subsection{Categories based on Relatum Properties} \label{sec:FoR_Relatum_scenario}
Using the FoR classes in Section~\ref{sec:primitives}, we identified two key relatum properties contributing to FoR ambiguity.
The first property is the relatum's intrinsic direction. 
It creates ambiguity between intrinsic and relative FoR since spatial relations can originate from the relatum's and observer's perspectives.
The second is the relatum's affordance as a container. 
It introduces the ambiguity between internal and external FoR, as spatial relations can refer to the inside and outside of the relatum. 
Based on these properties, we define four distinct cases: \textit{Cow Case, Box Case, Car Case, and Pen Case.}
% We extend the previous for categorizing based on the property of the object~\cite

\noindent\textbf{Case 1: Cow Case}.
In this case, the selected relatum has intrinsic directions but does not have the affordance as the container for the locatum.
An obvious example is a cow, which should not be a container but has a front and back.
In such cases, the relatum potentially provides a perspective for spatial relations. 
The applicable FoR classes are $FoR$ = \{\textit{external intrinsic}, \textit{external relative}\}.
We augment the C-split with expressions of this case but include the perspective to resolve their ambiguity.
To specify the perspective, we use predefined templates for augmenting clauses, such as \textit{from \{relatum\}'s perspective }for \textit{external intrinsic} or \textit{from the camera's perspective} for \textit{external relative}. 
For example, if the context is  \textit{A cat is to the right of the cow}, in the A-split. 
The counterparts included in the C-split are \textit{A cat is to the right of the cow from cow's perspective.} for \textit{external intrinsic} and \textit{A cat is to the right of the cow from my perspective.} for \textit{external intrinsic}. 


\noindent\textbf{Case 2: Box Case.} 
The relatum in this category has the property of being a container but lacks intrinsic directions, making the internal FoR applicable. An example is a box. 
The applicable FoR classes are $FoR$ = \{\textit{external relative}, \textit{internal relative}\}.
To include their unambiguous counterparts in the C-split, we specify the topological relation to the expressions, $T$, by adding \textit{inside} for \textit{internal relative} and \textit{outside} for \textit{external relative} cases. 
For example, for the sentence \textit{A cat is to the right of the box.},
the unambiguous $T$ with \textit{internal relative} FoR is \textit{A cat is inside and to the right of the box.} The counterpart for \textit{external relative} is \textit{A cat is outside and to the right of the box.}


\noindent\textbf{Case 3: Car Case.}  
A relatum with an intrinsic direction and container affordance falls into this case, allowing all FoR classes. An obvious example is a car that can be a container with intrinsic directions. The applicable FoR classes are $FoR$ = \{ \textit{external relative},  \textit{external intrinsic}, \textit{internal intrinsic}, \textit{internal relative}\}.
To augment C-split with this case's disambiguated counterparts, we add perspective and topology information similar to the Cow and Box cases.
An example expression for this case is \textit{A person is in front of the car.} 
The four disambiguated counterparts to include in the C-split are \textit{A person is outside and in front of the car from the car itself.} for \textit{external intrinsic}, \textit{A person is outside and in front of the car from the observer.} for \textit{external relative},  \textit{A person is inside and in front of the car from the car itself.} for \textit{internal intrinsic}, and \textit{A person is inside and in front of the car from the observer.} for \textit{internal relative}.

\noindent\textbf{Case 4: Pen Case.} 
In this case, the relatum lacks both the intrinsic direction and the affordance as a container. 
An obvious example is a pen with neither left/right nor the ability to be a container.
Lacking these two properties, the created context has only one applicable FoR, $FoR$ = \{ \textit{external relative}\}.
Therefore, we can categorize this case into both splits without any modification.
An example of such a context is \textit{The book is to the left of a pen.}


\subsection{Context Visualization}\label{sec:context_visualize}
% In our visualization, a complex subset arises when the relatum has an intrinsic direction within the intrinsic FoR. In such cases, the relatum’s orientation can introduce additional complexity to the visualization.
% In intrinsic FoR classes where the relatum has intrinsic direction, the relatum’s orientation can complicate visualization. \pk{mention: we use this complex subset for visualizaion or soemthing like that}
In our visualization, complexity arises when the relatum has an intrinsic direction within the intrinsic FoR, as its orientation can complicate the spatial representation.
For example, for visualizing \textit{A cat is to the right of a dog from the dog's perspective.}, the cat can be placed in different coordinates based on the dog’s orientation.
To address this issue, we add a template sentence for each direction, such as \textit{<relatum> is facing toward the camera}, to specify the relatum's orientation of all applicable $T$ for visualization and QA.
For instance, \textit{A cat is to the left of a dog.} becomes \textit{A cat is to the left of a dog. The dog is facing toward the camera.}
To avoid occlusion issues, we generate visualizations only for external FoRs, as one object may become invisible in internal FoR classes.
We use only expressions in C-split since those have a unique FoR interpretation for visualization. 
We then create a scene configuration by applying a predefined template, as illustrated in Figure~\ref{fig:generate_pipeline_image}.
Images are generated using the Unity 3D simulator~\cite{juliani2020unitygeneralplatformintelligent}, producing four variations per expression $T$ with different backgrounds and object positions. Further details on the simulation process are in Appendix~\ref{appendix:dataset_creation}.


% 
\subsection{Question-Answering Generation}\label{sec:QA_generation}
We generate questions for all generated spatial expressions ($T$). 
Note that we include the relatum orientation for cases where the relatum has an intrinsic direction, as mentioned in the visualization.
Our benchmark includes two types of questions. 
The first type asks for the spatial relation between two given objects from the camera's perspective, following predefined templates such as, \textit{Based on the camera’s perspective, where is the {locatum} relative to the {relatum}’s position?}
Template variations are made based on GPT4o.
The second type of question queries the spatial relation from the relatum’s perspective. 
This question type follows the same templates but replaces the camera with the relatum.
The first type of question is generated for all $T$, while the second type is only generated for $T$ where the relatum has intrinsic direction and a perspective can be defined accordingly.
Question templates are provided in Appendix~\ref{appendix:textual_template}. 
Answers are determined based on the corresponding FoRs, the spatial relation in $T$, and the relatum’s orientation when applicable.
