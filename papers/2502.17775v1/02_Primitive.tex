We review three semantic aspects of spatial information expressed in language: Spatial Roles, Spatial Relations, and Frame of Reference.  

\noindent\textbf{Spatial Roles.} 
We focus on two main spatial roles~\citep{kordjamshidi-etal-2010-spatial} of \textit{Locatum}, and \textit{Relatum}. 
The locatum is the object described in the spatial expression, while the relatum is the other object used to describe the position of the locatum. 
An example is \textit{a cat is to the left of a dog}, where the \textit{cat} is the locatum, and the \textit{dog} is the relatum.

\noindent\textbf{Spatial Relations.} 
When dealing with spatial knowledge representation and reasoning, three main relations categories are often considered, that is, directional, topological, and distance~\citep{reasoningQualitaiveDaniel, COHN2008551,ACMpaper}. 
\textit{Directional} describes an object's direction based on specific coordinates. Examples of relations include left and right.
\textit{Topological} describes the containment between two objects, such as inside.
\textit{Distance} describes qualitative and quantitative relations between entities. Examples of qualitative are far, and quantitative are 3km.

\noindent\textbf{Spatial Frame of Reference.} We use four frames of references investigated in the cognitive linguistic studies~\cite{TENBRINK2011704}. These are defined based on the concept of \textit{Perspective}, which is the origin of a coordinate system to determine the direction. The four frames of reference are defined as follows.

\noindent1. \textit{External Intrinsic} describes a spatial relation from the relatum's perspective, where the relatum does not contain the locatum. The top-right image in Figure~\ref{fig:FoR_classes} illustrates this with the sentence, \textit{A cat is to the right of the car from the car's perspective.}

\noindent2. \textit{External Relative} describes a spatial relation from the observer's perspective.
The top-left image in Figure~\ref{fig:FoR_classes}  shows an example with the sentence, \textit{A cat is to the left of a car from my perspective.}

\noindent3. \textit{Internal Intrinsic} describes a spatial relation from the relatum's perspective, where the relatum contains the locatum. The bottom-right image in Figure~\ref{fig:FoR_classes} show this with the sentence, \textit{A cat is inside and back of the car from the car's perspective.}

\noindent4. \textit{Internal Relative} describes a spatial relation from the observer's perspective where the locatum is inside the relatum. The bottom-left image in Figure~\ref{fig:FoR_classes} show this FoR with the sentence, \textit{A cat is inside and to the left of the car from my perspective.}


\begin{figure*}[t!]
    \centering
    \includegraphics[width=0.8\linewidth, trim= {0 0 0.5 0.5cm }]{Figures/ImagePipeline.pdf}
    \caption{Pipeline for dataset creation, starting from selecting a locatum and relatum from available objects and then applying a spatial template to generate the spatial expression ($T$). FoRs are assigned based on the relatum’s properties. $T$ is then categorized based on the number of FoRs. For example, \textit{A cat is to the right of a dog }(with two possible FoRs: external intrinsic and external relative) belongs to the A-split. Then, its disambiguated version (A cat is to the right of a dog from the dog's perspective) is added to the C-split. Next, if applicable, a relatum's orientation is included for visualization and question generation. Finally, Unity3D generates scene configurations, and question-answer pairs are created from $T$.}
    \label{fig:generate_pipeline_image}
\end{figure*}