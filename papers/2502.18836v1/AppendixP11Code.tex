\begin{comment}

%\documentclass{article}
%\usepackage{listings}
%\usepackage{xcolor}

% Define custom style for Python code
\lstdefinestyle{PythonStyle}{
    language=Python,
    basicstyle=\ttfamily\footnotesize,
    keywordstyle=\color{blue},
    commentstyle=\color{green},
    stringstyle=\color{red},
    breaklines=true,
    frame=single
}

% Define custom style for JSON
\lstdefinestyle{JSONStyle}{
    language=Java,
    basicstyle=\ttfamily\footnotesize,
    keywordstyle=\color{blue},
    stringstyle=\color{red},
    breaklines=true,
    frame=single
}

%\begin{document}
\end{comment}

\section{P11: Stock Prediction Sample Implementation on LangGraph}

This appendix presents an implementation of the P11 Stock Prediction problem using LangGraph~\cite{langgraph2024}. The implementation demonstrates how AI-driven planning can be applied to financial forecasting by leveraging historical data, technical indicators, and real-time market updates.

\subsection{Agentic Workflow Formulation}

In this first stage, we define agents to manage the nodes of the workflow, including data collection, feature extraction, model training, integration, and alert. In the end, we use the \texttt{>>} syntax to specify dependencies among agents. 

\begin{lstlisting}[style=PythonStyle, caption=Prompt and Collaborative Agent Pipeline, label=lst:agent_pipeline]
from src.multi_agent.crew import Crew
with Crew() as crew:
    # ---- Data Collection Agent ---- #
    DC_Agent = Agent(
        name=``Data Collection Agent''
        backstory=``You collect 5 stocks of historical data OHLCV from S&P 500 Kaggle Dataset at https://www.kaggle.com/datasets/andrewmvd/sp-500-stocks, and real-time stock data including external economic indicators.'',
        task_description=``Retrieve stock data from 2023-10-01 to 2023-12-31.'',
        task_expected_output=``Formatted OHLCV data and external indicators, of 5 stocks.'',
    )

    Writer_agent = Agent(
        name=``Writer Agent'',
        backstory=``You are a language model specialised in writing text into .json files'',
        task_description=``Write the json response into ./tool_agent_example.json'',
        task_expected_output=``A .json file containing the given string'',
        tools=write_str_to_txt,
    )
    
    # ---- Feature Extraction Agent ---- #
    FE_Agent = Agent(
        name=``Feature Extraction Agent'',
        backstory=``You compute technical indicators from the collected 5 stock data.'',
        task_description=``Calculate MA, MACD, RSI, Bollinger Bands, and other technical indicators of 5 stocks.'',
        task_expected_output=``Feature matrix of OHLCV, external indicators, MA, MACD, RSI, Bollinger Bands, of 5 stocks.''
    )

    
    # ---- Model Training Agent ---- #
    MT_Agent = Agent(
        name=``Model Training Agent'',
        backstory=``You train stock prediction models using historical data (from 2023-10-01 to 2023-11-30) and validate with (from 2023-12-1 to 2024-12-31).'',
        task_description=``use financial math and machine learning models and optimize hyperparameters.'',
        task_expected_output=``Prepare the input Feature matrix, with 3 financial math models and 3 machine learning models ready for inference, of 5 stocks.''
    )
    # ---- Prediction Generation Agent ---- #
    PG_Agent = Agent(
        name=``Prediction Generation Agent'',
        backstory=``You generate stock price predictions using the trained model.'',
        task_description=``Predict next-day and intra-day stock prices.'',
        task_expected_output=``Based on input and the 6 models, predict Stock price from 2024-1-1 to 2024-1-7, with confidence intervals, of 5 stocks.''
    )
    # ---- Integration Agent ---- #
    IG_Agent = Agent(
        name=``Integration Agent'',
        backstory=``You combine model predictions with external news sentiment and macroeconomic indicators.'',
        task_description=``Adjust predictions based on news impact and market conditions on year of 2023.'',
        task_expected_output=``Refined Stock price from 2024-1-1 to 2024-1-7, with adjusted confidence intervals, of 5 stocks.''
    )
    # ---- Alert Generation Agent ---- #
    AG_Agent = Agent(
        name=``Alert Generation Agent'',
        backstory=``You issue buy/sell/hold signals based on  predictions and risk assessment of the model'', task_description=``Generate trading alerts and confidence scores'', 
        task_expected_output=``Based on the 6 models, provide actionable trade signals from 2024-1-1 to 2024-1-7, of 5 stocks.''
    )

    
    # ---- Define Dependencies ---- #
    DC_Agent >> Writer_agent >> FE_Agent >> MT_Agent  # Technical Analysis Flow
    MT_Agent >> PG_Agent >> AG_Agent  # Model Chain
    #AG_Agent >> DC_Agent  # Feedback Loop (adaptive updates)
crew.run()

\end{lstlisting}

\subsection{Workflow Execution With Historical Data for Prediction}

Following the dependency graph presented above (also see Figure~1), the agents are executed in sequence: starting with data collection, then feature extraction, followed by model training, and finally prediction. For brevity, while data for five stocks is provided, we illustrate the execution flow using only APPL.

%\textbf{Without Real-time Search:}
\begin{lstlisting}[style=JSONStyle, caption=Sample JSON Output from Data Collection, label=lst:data_collection]
==================================================
RUNNING AGENT: Data Collection Agent
==================================================

json
{
  ``stocks'': [
    {
      ``symbol'': ``AAPL'',
      ``data'': [
        {
          date: 2023-10-01,
          open: 150.00,
          high: 152.00,
          low: 149.50,
          close: 151.00,
          volume: 75000000
        },
        // Additional daily data up to 2023-12-31
      ],
      ``external_indicators'': {
        interest_rate: 5.25,
        inflation_rate: 3.0,
        gdp_growth: 2.5
      }
    },
    {
      ``symbol'': ``MSFT'',
      ``data'': [
        {
          date: 2023-10-01,
          open: 280.00,
          high: 285.00,
          low: 279.00,
          close: 282.00,
          volume: 45000000
        },
        // Additional daily data up to 2023-12-31
      ],
      ``external_indicators'': {
        interest_rate: 5.25,
        inflation_rate: 3.0,
        gdp_growth: 2.5
      }
    },
    {
      ``symbol'': ``GOOGL'',
      ``data'': [
        {
          date: 2023-10-01,
          open: 135.00,
          high: 137.00,
          low: 134.50,
          close: 136.00,
          volume: 32000000
        },
        // Additional daily data up to 2023-12-31
      ],
      ``external_indicators'': {
        interest_rate: 5.25,
        inflation_rate: 3.0,
        gdp_growth: 2.5
      }
    },
    {
      ``symbol'': ``AMZN'',
      ``data'': [
        {
          date: 2023-10-01,
          open: 120.00,
          high: 122.50,
          low: 119.00,
          close: 121.50,
          volume: 60000000
        },
        // Additional daily data up to 2023-12-31
      ],
      ``external_indicators'': {
        interest_rate: 5.25,
        inflation_rate: 3.0,
        gdp_growth: 2.5
      }
    },
    {
      ``symbol'': ``TSLA'',
      ``data'': [
        {
          date: 2023-10-01,
          open: 250.00,
          high: 255.00,
          low: 248.00,
          close: 252.00,
          volume: 50000000
        },
        // Additional daily data up to 2023-12-31
      ],
      ``external_indicators'': {
        interest_rate: 5.25,
        inflation_rate: 3.0,
        gdp_growth: 2.5
      }
    }
  ]
}

==================================================
RUNNING AGENT: Feature Extraction Agent
==================================================

To complete the task of calculating technical indicators for five stocks, perform the following computations for each stock:

    Moving Average (MA): Compute the simple moving average over a specified period (e.g., 20 days).
    Moving Average Convergence Divergence (MACD): Compute the MACD line, signal line, and histogram.
    Relative Strength Index (RSI):} Compute the RSI over a specified period (e.g., 14 days).
    Bollinger Bands: Compute the upper band, lower band, and the middle band (the moving average).
    Other Technical Indicators: Optionally, compute additional indicators such as Average True Range (ATR), Stochastic Oscillator, etc., based on the available data.


==================================================
RUNNING AGENT: Model Training Agent
==================================================

To prepare the input feature matrix with financial math models and machine learning models ready for inference, we will follow these steps:

1. Feature Matrix Preparation:
   - Collect historical stock data for the 5 selected stocks, including the following columns: Date, Open, High, Low, Close, and Volume.
   - Calculate the necessary financial math models (technical indicators) for each stock:
     - Moving Average (MA): Compute the simple moving average over a 20-day period.
     - Moving Average Convergence Divergence (MACD): Calculate the MACD line, signal line, and histogram.
     - Relative Strength Index (RSI): Calculate the RSI over a 14-day period.
     - Bollinger Bands: Calculate the upper band, middle band (moving average), and lower band.

2. Machine Learning Models:
   - Train three machine learning models using the historical data and calculated features. The models could be:
     - Linear Regression: For predicting future stock prices based on the features.
     - Random Forest Regressor: To capture non-linear relationships and interactions between features.
     - Support Vector Machine (SVM): For robust predictions with a focus on margin maximization.
   - Optimize hyperparameters for each model using techniques such as Grid Search or Random Search to enhance model performance.

3. Inference Preparation:
   - Prepare the models for inference by saving the trained models and their hyperparameters.
   - Ensure the feature matrix is structured and aligned by date, ready for input into the models for prediction.

4. Output Format:
   - Compile the results into a structured data format such as a DataFrame or JSON, including all calculated features and ready-to-use models for each of the 5 stocks.

By following these steps, we ensure that the feature matrix is comprehensive and that the machine learning models are optimized and ready for inference on the selected stocks.

==================================================
RUNNING AGENT: Prediction Generation Agent
==================================================

To predict the stock prices from January 1, 2024, to January 7, 2024, for the five selected stocks, we utilize the prepared feature matrix and the trained machine learning models as described in the context. Here is the structured prediction output of one of the five stocks, including confidence intervals:

json
{
  ``predictions'': [
    {
      ``stock'': ``APPL'',
      ``predictions'': [
        {date: 2024-01-01, predicted_price: 150.25, confidence_interval: [148.00, 152.50], confidence: 0.80},
        {date: 2024-01-02, predicted_price: 151.00, confidence_interval: [148.75, 153.25], confidence: 0.80},
        {date: 2024-01-03, predicted_price: 152.50, confidence_interval: [150.00, 155.00], confidence: 0.88},
        {date: 2024-01-04, predicted_price: 153.75, confidence_interval: [151.50, 156.00], confidence: 0.82},
        {date: 2024-01-05, predicted_price: 151.50, confidence_interval: [150.25, 153.75], confidence: 0.90},
        {date: 2024-01-06, predicted_price: 155.00, confidence_interval: [152.75, 157.25], confidence: 0.83},
        {date: 2024-01-07, predicted_price: 156.25, confidence_interval: [154.00, 158.50], confidence: 0.87}
      ]
    },
  ]
}

This output provides the predicted stock prices for each day from January 1 to January 7, 2024, for stock APPL, along with their respective confidence intervals. These predictions are based on the trained models and the prepared feature matrix.
\end{lstlisting}

\subsection{Output: Alert Generation}

Based on the problem specification in Section~\ref{sec:p11}, when the prediction confidence is $\geq 85\%$, a recommended trading action is sent to the subscriber as an alert.

Based on the above prediction, on 1/3, 1/5, and 1/7 the prediction confidence exceeds the threshold and the stock's moving direction is \textbf{up}, \textbf{down}, and \textbf{up}, respectively. Therefore, the alerts issued before the opening of the market on those three days are: {\color{green}\textbf{BUY}}, {\color{red}\textbf{SELL}}, and {\color{green}\textbf{STRONG BUY}}, respectively.