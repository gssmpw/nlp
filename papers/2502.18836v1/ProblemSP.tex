\begin{table}[thb]
\centering
\caption{Stock Market Prediction Problem}
\label{tab:StockPrediction}
\vspace{-.1in}
\begin{small}
\renewcommand{\arraystretch}{1.1}
\fbox{
\begin{minipage}{0.45\textwidth}

\textbf{The Problems}

\textbf{1) Stock Price Prediction:} Forecast future stock prices (e.g., predicting Apple’s stock on a future date) using historical market data.

\textbf{2) Complex Market Dynamics:} The system must account for influences such as technical indicators, historical price patterns, and external events (policy announcements, CPI, natural disasters, geopolitical conflicts).

\textbf{3) Decision Making Under Uncertainty:} Supports what-if analyses and real-time alerts for decision-makers to respond quickly to unexpected events.

\textbf{The State Space}

\textbf{1) Who:} Companies or stocks (e.g., Apple) and related market players (competitors, sector peers).

\textbf{2) What:}
\begin{itemize}[leftmargin=1em, topsep=-.1pt, itemsep=-.1pt, label=-]
\item Technical indicators: MA, BB, MACD, RSI, ATR, OBV, lagged closing prices.
\item Traditional price data: Open, high, low, close, adjusted close, trading volumes.
\item Self-defined features: News sentiment, policy announcements.
\end{itemize}

\textbf{3) When:}
\begin{itemize}[leftmargin=1em, topsep=-.1pt, itemsep=-.1pt, label=-]
\item Historical Data: Training (2010–2020), validation (2021–2023).
\item Future Forecasting: Target date (e.g., 2026-02-01).
\end{itemize}

\textbf{4) Where:} Market context including policy changes, economic reports (CPI), and geopolitical events.

\textbf{Interdependencies}

\textbf{1) Market Patterns and Correlations:} Identify how movements in one sector or competitor’s stock affect another.

\textbf{2) External Influences:} Policy announcements, economic reports, and disruptions (earthquakes, geopolitical crises) create cascading market effects.

\textbf{3) Integration of Data Types:} Combine numerical indicators, historical patterns, and qualitative information (news sentiment, human prompts) for a holistic market view.

\textbf{Proposed Workflow for Decision Making}

\textbf{1) Data Collection \& Feature Extraction:}
\begin{itemize}[leftmargin=1em, topsep=-.1pt, itemsep=-.1pt, label=-]
\item Gather historical market data (prices, volumes, technical indicators) from sources like Yahoo Finance and SEC EDGAR.
\item Integrate external data such as policy announcements and economic indices.
\end{itemize}

\textbf{2) Model Training \& Prediction:}
\begin{itemize}[leftmargin=1em, topsep=-.1pt, itemsep=-.1pt, label=-]
\item Use predictive models (Support Vector Regression, XGBoost, LSTM) to learn feature-stock price relationships.
\item Evaluate models using RMSE and validation datasets.
\end{itemize}

\textbf{3) Integration \& Optimization:}
\begin{itemize}[leftmargin=1em, topsep=-.1pt, itemsep=-.1pt, label=-]
\item Combine multiple features and models (meta-planning) to optimize predictions.
\item Adjust hyperparameters, refine features iteratively.
\end{itemize}

\textbf{4) Decision Support \& Real-Time Alerts:}
\begin{itemize}[leftmargin=1em, topsep=-.1pt, itemsep=-.1pt, label=-]
\item Perform what-if analyses to predict event impacts (e.g., policy changes, natural disasters).
\item Set up alerts for sudden market changes.
\end{itemize}

\textbf{5) Iterative Validation \& Refinement:} Continuously validate predictions with new data, refining models based on feedback.

\end{minipage}
}
\end{small}
\end{table}