\subsection{Execution Example: Urban Ride Sharing }

For a benchmark suite, we encourage users to
devise novel methodologies to solve the problems. However, for this illustrative example of Urban Ride Sharing (URS) (P3), we employ a customized specification language that extends PDDL (Planning Domain Definition Language) and workflow networks to support:

\begin{itemize}[leftmargin=1.0em, topsep=-.0em, parsep=-.0em, label=-]
    \item Dynamic constraints and real-time updates.
    \item Integration with streaming data sources.
    \item Explicit representation of uncertainty.
    \item Temporal and spatial dependencies.
\end{itemize}

The solver, whether manual or automatic, approaches the planning problem in three steps:
\begin{enumerate}[leftmargin=1.0em, topsep=-.0em, parsep=-.0em, label=\arabic*.]
    \item Convert the problem statement into a formal specification:
    \begin{itemize}[leftmargin=1em, topsep=-0em, parsep=-0em, label=-]
        \item Key objectives and constraints.
        \item Required resources and their capabilities.
        \item Performance metrics and success criteria.
        \item Temporal and spatial dependencies.
    \end{itemize}
    
    \item Transform the specification into a workflow graph:
    \begin{itemize}[leftmargin=1em, topsep=-0em, parsep=-0em, label=-]
        \item Nodes represent processing stages, decision points, or actions.
        \item Edges capture dependencies, data flow, and execution sequence.
        \item Agents are assigned to both nodes and edges.
        \item Concrete parameters and thresholds are specified.
    \end{itemize}
    
    \item Select and apply solving algorithms:
    \begin{itemize}[leftmargin=1em, topsep=0em, parsep=0em, label=-]
        \item Test multiple solution approaches (e.g., dynamic programming, Monte Carlo).
        \item Evaluate solutions against specified metrics.
        \item Select and validate the best-performing solution.
        \item Present results with performance analyses.
    \end{itemize}
\end{enumerate}

We walk through these three steps to solve the URS problem.

\subsubsection{URS Problem Specification}Already presented in Table~\ref{tab:appURS}.

\subsubsection{URS Workflow}

Figure~\ref{fig:URS-DirectedGraph} shows
the workflow representation of the URS problem.

\begin{figure}[th!]
%\vspace{-.1in}
    \centering
    \begin{tikzpicture}[
        location/.style={
            circle,
            draw=black,
            fill=white!20,
            minimum size=0.8cm,
            text centered,
            font=\large
        },
        airport/.style={
            rectangle,
            rounded corners,
            draw=black,
            fill=cyan!20,
            minimum width=2.0cm,
            minimum height=1.0cm,
            text centered,
            font=\large
        },
        passenger/.style={
            font=\normalsize\color{red}
        },
        vehicle/.style={
            font=\normalsize\color{blue}
        },
        urban_path/.style={thick},
        airport_path/.style={thick}
    ]
        % Title
        %\node[draw=none] at (0,4) {\Large Urban Ride Sharing Network};
        
        % Urban locations
    \node[location, fill=white, draw, inner sep=2pt, minimum size=28pt] (A) at (0,1.05) {A};
    \node[location, fill=white, draw, inner sep=2pt, minimum size=28pt] (B) at (2.0,2) {B};
    \node[location, fill=white, draw, inner sep=2pt, minimum size=28pt] (C) at (2.0,0.2) {C};
    \node[location, fill=white, draw, inner sep=2pt, minimum size=28pt] (D) at (0,-0.4) {D};
    \node[location, fill=white, draw, inner sep=2pt, minimum size=28pt] (E) at (-2.0,0.2) {E};
    \node[location, fill=white, draw, inner sep=2pt, minimum size=28pt] (F) at (-2.0,2) {F};

        % Airport location
        \node[airport] (G) at (3.8,1) {G (BOS)};
        
        % Urban routes with time only
        \foreach \from/\to in {A/B, B/C, C/D, D/E, E/F, F/A, A/C, B/D, C/E, D/F, E/A, F/B}
            \draw[urban_path] (\from) -- (\to) 
            node[midway,fill=none] {};
            
        % Airport routes with time only
        \draw[airport_path] (A) -- (G) node[pos=0.72,sloped,above,fill=none] {};
        \draw[airport_path] (B) -- (G) node[pos=0.50,sloped,above,fill=none] {};
        \draw[airport_path] (C) -- (G) node[pos=0.50,sloped,above,fill=none] {};
%        \draw[airport_path] (D) -- (G) node[pos=0.72,sloped,above,fill=none] {};
        \draw[airport_path] (E) -- (G) node[pos=0.78,sloped,above,fill=none] {};
        \draw[airport_path] (F) -- (G) node[pos=0.78,sloped,above,fill=none] {};
        
        % Passengers with arrival times
        \node[passenger] at ($(A)+(0.5,0.6)$) {$r_1$ (8:45)};
        \node[passenger] at ($(B)+(0.5,0.6)$) {$r_2$ (8:50)};
        \node[passenger] at ($(C)+(0.5,-0.6)$) {$r_3$ (8:55)};
        \node[passenger] at ($(D)+(0.5,-0.6)$) {$r_4$ (9:00)};
        
        % Vehicles with capacities
        \node[vehicle] at ($(A)+(-0.5,0.6)$) {$k_1$ [2]};
        \node[vehicle] at ($(C)+(-0.5,-0.6)$) {$k_2$ [2]};
        \node[vehicle] at ($(E)+(-0.5,-0.6)$) {$k_3$ [2]};
        
    \end{tikzpicture}
    %\vspace{-.15in}
    \caption{Consider a network $G=(V,E)$ with urban travel times $\tau_{ij}=10$ minutes and airport routes $\tau_{iG}=\{19,\ldots,22\}$ minutes. Schedule three vehicles $k_i$ (capacity $c_k=2$) to deliver four passengers $r_i$ to airport during $[8:45,9:00]$.}\Description{URS}
    %\vspace{-.1in}
    \label{fig:URS-DirectedGraph}
\end{figure}
%\vspace{-.2in}


\subsubsection{URS Results}

Figure~\ref{fig:ALASmonte-carlo} presents an optimal solution with a total travel distance of 87 km, outperforming both GPT-4o-Task and DeepSeek R1 (Figure~\ref{fig:GPTDS-schedule}), which require 123 km. This represents a significant improvement of 41.37\%.

\paragraph{Extended Materials}

We also used LangGraph to implement both \textbf{P3} and \textbf{P4}, as presented in \textbf{Appendix}~\ref{app:p3p4}. Furthermore, we provide a solution to \textbf{P11} for the prediction of the stock value. These sample implementations validate the completeness of the problem statements in this benchmark.

\begin{figure}[ht!]
%\vspace{-.1in}
    \centering
    \begin{tikzpicture}[
        location/.style={
            circle,
            draw=black,
            fill=white,
            minimum size=1.0cm,
            text centered,
            font=\normalsize
        },
        airport/.style={
            rectangle,
            rounded corners,
            draw=black,
            fill=cyan!20,
            minimum width=2.0cm,
            minimum height=1.0cm,
            text centered,
            font=\normalsize
        },
        passenger/.style={
            font=\small\color{red}
        },
        path_k1/.style={
            thick,
            blue,
            ->,
            >=stealth
        },
        path_k2/.style={
            thick,
            orange,
            ->,
            >=stealth
        },
        vehicle/.style={
            draw,
            fill=white,
            minimum width=0.8cm,
            minimum height=0.4cm,
            rounded corners=1pt
        }
    ]
         % Vehicle k1 at A
        \node[blue] at (-1,1.5) {\Huge\faTaxi};
        \node[blue] at (-1,2.0) {k1};

        % Urban locations
        \node[location] (A) at (0,1.5) {A};
        \node[location] (B) at (2.1,1.5) {B};
        \node[location] (C) at (0,0) {C};
        \node[location] (D) at (2.1,0) {D};
        
        % Vehicle k2 at C
        \node[orange] at (-1,0) {\Huge\faTaxi};
        \node[orange] at (-1,0.5) {k2};
        
        % Airport location with arrival times
        \node[airport] (G) at (5,.75) {G (BOS)};
        
        % k1 path: A -> B -> G
        \draw[path_k1] (A) -- (B) node[midway,above,sloped,fill=none,font=\footnotesize] {8:00-8:10};
        \draw[path_k1] (B) -- (G) node[midway,above,sloped,fill=none,font=\footnotesize] {8:10-8:30};
        
        % k2 path: C -> D -> G
        \draw[path_k2] (C) -- (D) node[midway,below,fill=none,font=\footnotesize] {8:00-8:10};
        \draw[path_k2] (D) -- (G) node[midway,below,sloped,fill=none,font=\footnotesize] {8:10-8:30};
        
        % Passengers with arrival times
        \node[passenger] at ($(A)+(0.3,0.7)$) {$r_1$ (8:45)};
        \node[passenger] at ($(B)+(0.3,0.7)$) {$r_2$ (8:50)};
        \node[passenger] at ($(C)+(0.3,0.7)$) {$r_3$ (8:55)};
        \node[passenger] at ($(D)+(0.3,0.7)$) {$r_4$ (9:00)};
        
    \end{tikzpicture}
    %\vspace{-.1in}
    \caption{This solution routes showing optimal vehicle assignments. Vehicle k1 (blue) starts at A and collects passengers r1 and r2, while k2 (orange) starts at C and serves r3 and r4. Both vehicles arrive at BOS at 8:30, meeting all passenger arrival deadlines. Total distance traveled = 87 km ({\color{red}Optimal}).}\Description{This figure illustrates the optimal vehicle assignments for the ALAS solution routes, showing vehicles k1 and k2 serving passengers and meeting deadlines.}
    \label{fig:ALASmonte-carlo}
    %\vspace{-.1in}
\end{figure}
%\vspace{-.15in}
\begin{figure}[ht!]
%\vspace{-.1in}
    \centering
    \begin{tikzpicture}[
        location/.style={
            circle,
            draw=black,
            fill=white,
            minimum size=1.0cm,
            text centered,
            font=\normalsize
        },
        airport/.style={
            rectangle,
            rounded corners,
            draw=black,
            fill=cyan!20,
            minimum width=2cm,
            minimum height=1cm,
            text centered,
            font=\normalsize
        },
        passenger/.style={
            font=\small\color{red}
        },
        path_k1/.style={
            thick,
            blue,
            ->,
            >=stealth
        },
        path_k2/.style={
            thick,
            orange,
            ->,
            >=stealth
        },
        path_k3/.style={
            thick,
            green!50!black,
            ->,
            >=stealth
        },
        vehicle/.style={
            draw,
            fill=white,
            minimum width=0.8cm,
            minimum height=0.4cm,
            rounded corners=1pt
        }
    ]
        % Vehicle k1 icon at A

        % Replace vehicle nodes with faTaxi icons

        % Vehicle k1 at A
        \node[blue] at (-1,1.5) {\Huge\faTaxi};
        \node[blue] at (-1,2.0) {k1};


        % Urban locations in more compact arrangement
        \node[location] (A) at (0,1.6) {A};
        \node[location] (B) at (2.1,1.6) {B};
        \node[location] (C) at (3.8,0) {C};
        \node[location] (E) at (0,0) {E};
        \node[location] (D) at (2.0,0) {D};
        
        % Vehicle k2 at C
        \node[orange] at (4.8,0) {\Huge\faTaxi};
        \node[orange] at (4.8,0.5) {k2};


        % Vehicle k3 at E
        \node[green!50!black] at (-1,0) {\Huge\faTaxi};
        \node[green!50!black] at (-1,0.5) {k3};
        
        % Airport location moved higher
        \node[airport] (G) at (5.0,1.6) {G (BOS)};
        
        % k1 path: A -> B -> G
        \draw[path_k1] (A) -- (B) node[midway,above,font=\footnotesize] {7:40-7:50};
        \draw[path_k1] (B) -- (G) node[midway,above,fill=none,font=\footnotesize] {7:50-8:10};
        
        % k2 path: C -> G
        \draw[path_k2] (C) -- (G) node[pos=0.58,sloped,below,fill=none,font=\footnotesize] {7:55-8:17};

        % k3 path: E -> D
        \draw[path_k3] (E) -- (D) node[midway,sloped,above,fill=none,font=\footnotesize] {7:50-8:00};
        
        
        % k3 path: E -> G
        \draw[path_k3] (D) -- (G) node[pos=0.66,sloped,above,fill=none,font=\footnotesize] {8:00-8:19};
        
        % Passengers with arrival times
        \node[passenger] at ($(A)+(0.3,0.7)$) {$r_1$ (8:45)};
        \node[passenger] at ($(B)+(0.3,0.7)$) {$r_2$ (8:50)};
        \node[passenger, fill=none, inner sep=2pt] at ($(C)+(0.0,0.65)$) {$r_3$};


        %\node[passenger] at ($(C)+(0.3,0.70)$) {$r_3$ (8:55)};
        \node[passenger] at ($(D)+(0.3,0.7)$) {$r_4$ (9:00)};
        
    \end{tikzpicture}
    %\vspace{-.1in}
    \caption{GPT-4o-Task and DeepSeek R1 both schedule three routes. Vehicle k1 picks up r1 from A at 7:40, then r2 from B at 7:50. Vehicle k2 picks up r3 from C at 7:55. Vehicle k3 must first drive from E to D to pick up r4 at 8:00. All meet deadlines. Total travel distance = 123 km.}\Description{Infeasible Schedule}
    \label{fig:GPTDS-schedule}
    %\vspace{-.1in}
\end{figure}
%\vspace{-.1in}
