\section{Benchmark Structure}

Problems are categorized into three difficulty levels based on the number of parallel execution threads, the complexity of dependencies, and real-time disruptions.

\subsection{Entry Level (1-2 threads)}
Problems focusing on basic coordination with limited dependencies:
\begin{itemize}[leftmargin=1.0em, topsep=-.0em, parsep=-.0em, label=-]
   \item Single or dual thread execution
   \item Basic timing and resource constraints
   \item Simple disruption scenarios
   \item Example: Campus tour coordination with one and two groups
\end{itemize}

\subsection{Intermediate (3-4 threads)}
Problems requiring significant coordination across multiple execution paths:
\begin{itemize}[leftmargin=1.0em, topsep=-.0em, parsep=-.0em, label=-]
   \item Three to four parallel threads
   \item Complex timing dependencies
   \item Resource sharing constraints
   \item Example: Wedding logistics with multiple vehicles and tasks
\end{itemize}

\subsection{Advanced (5+ threads)}
Problems with real-world complexity:
\begin{itemize}[leftmargin=1.0em, topsep=-.0em, parsep=-.0em, label=-]
   \item Five or more parallel threads
   \item Complex inter-dependencies
   \item Multiple resource conflicts
   \item Dynamic disruption scenarios
   \item Example: Thanksgiving dinner coordination, natural disaster relief, and supply chain management
\end{itemize}

\subsection{Evaluation Metrics}
Each problem is evaluated across five key dimensions:
\begin{itemize}[leftmargin=1.0em, topsep=-.0em, parsep=-.0em, label=-]
   \item \textbf{Planning Quality:} Effectiveness of initial plan generation
   \item \textbf{Coordination:} Management of parallel thread execution
   \item \textbf{Adaptation:} Response to disruptions and changes
   \item \textbf{Resource Management:} Resolution of resource conflicts
   \item \textbf{Constraint Satisfaction:} Maintenance of problem constraints
\end{itemize}
