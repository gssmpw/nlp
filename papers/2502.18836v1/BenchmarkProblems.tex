\section{Benchmark Problem Specifications}

\begin{table}[thb]
\centering
\caption{Single Tour Campus Navigation Problem}
\vspace{-.1in}
\begin{small}
\renewcommand{\arraystretch}{1.1}
\fbox{
\begin{minipage}{0.45\textwidth}
\textbf{Metrics:}
\begin{itemize}[leftmargin=1em, topsep=-.1pt, itemsep=-.1pt, label=-]
\item \textbf{Total tour time:} Minimize while meeting all constraints
\item \textbf{Visit coverage:} All locations must be visited
\end{itemize}

\textbf{Locations:} Five campus buildings $L = \{S, L, B, A, D\}$
\begin{itemize}[leftmargin=1em, topsep=-.1pt, itemsep=-.1pt, label=-]
\item $S$: Student Center
\item $L$: Library 
\item $B$: Lab Building
\item $A$: Athletics Center
\item $D$: Dormitory
\end{itemize}

\textbf{Travel Times:} (minutes)
$S\text{-}L:10, S\text{-}B:15, S\text{-}A:20, S\text{-}D:15$
$L\text{-}B:10, L\text{-}A:25, L\text{-}D:20$
$B\text{-}A:15, B\text{-}D:25$
$A\text{-}D:20$

\textbf{Visit Requirements:}
\begin{itemize}[leftmargin=1em, topsep=-.1pt, itemsep=-.1pt, label=-]
\item Start time given
\item Each location requires 30-minute visit
\item Tour starts/ends at Student Center ($S$)
\item Group size: 20 people
\end{itemize}

\textbf{Time Constraints:}
\begin{itemize}[leftmargin=1em, topsep=-.1pt, itemsep=-.1pt, label=-]
\item Lab Building: Only 9 AM - 4 PM
\item Library: After 10 AM
\item Total tour must complete by 5 PM
\end{itemize}
\end{minipage}
}
\end{small}
\label{tab:P1SingleTour}
\vspace{-.1in}
\end{table}

\begin{table}[thb]
\centering
\caption{Multi-Group Campus Tour Problem}
\vspace{-.1in}
\begin{small}
\renewcommand{\arraystretch}{1.1}
\fbox{
\begin{minipage}{0.45\textwidth}
\textbf{Groups:}
\begin{itemize}[leftmargin=1em, topsep=-.1pt, itemsep=-.1pt, label=-]
\item $G_1$: 15 people (domestic students)
\item $G_2$: 20 people (international students)
\end{itemize}

\textbf{Locations:} Ten campus buildings
$L = \{S, L, B, A, D, C, M, R, H, P\}$
with capacities $cap_l$:
\[
cap_l = \begin{cases}
   40 & l \in \{S, A\} \\
   30 & l \in \{L, D, C\} \\
   25 & l \in \{B, M\} \\
   20 & l \in \{R, H, P\}
\end{cases}
\]

\textbf{Constraints:}
\begin{itemize}[leftmargin=1em, topsep=-.1pt, itemsep=-.1pt, label=-]
%\item Groups cannot visit same location simultaneously
\item Tour start time for each group: between 9 AM and 10 AM
\item Total visitors must not exceed location capacity
\item Each location requires 30-minute visit
\item Both tours start/end at Student Center
\item Complete all tours by 5 PM
\end{itemize}

\textbf{Additional Requirements:}
\begin{itemize}[leftmargin=1em, topsep=-.1pt, itemsep=-.1pt, label=-]
\item Lab tours ($B$) only 9 AM - 4 PM
\item Dining ($C$) must be visited between 11 AM - 2 PM
\item Library ($L$) after 10 AM
\end{itemize}
\end{minipage}
}
\end{small}
\label{tab:P2MultiTour}
\vspace{-.1in}
\end{table}


REALM-Bench comprises eleven foundational problem frameworks that systematically evaluate both sequential and reactive planning. Building on the key dimensions introduced earlier (parallel threads, inter-dependencies, and dynamic disruptions), these frameworks progress from straightforward single-thread execution to complex multi-agent scenarios with real-time challenges.

\paragraph{\textbf{Considerations \#1 Problem Complexity:}} 

Each framework can be further scaled to create more challenging variants:
\begin{itemize}[leftmargin=1.0em, topsep=-.0em, parsep=-.0em, label=*]
\item Expanding the scale of agents and resources (e.g., from dozens to thousands)
\item Widening geographic distribution (e.g., local to global)
\item Increasing disruption frequency and severity (e.g., isolated events to cascading failures)
\item Introducing uncertainties in execution times and outcomes (e.g., probabilistic durations)
\item Adding hierarchical dependency networks (e.g., sub-networks with internal dependencies)
\item Accounting for agent properties (e.g., atomicity, idempotency)
\end{itemize}
\vspace{.1in}
This extensible design enables researchers to incrementally assess their systems’ capabilities while preserving each scenario’s fundamental planning challenge.

\paragraph{\textbf{Considerations \#2 Implementation:}}

Three main approaches can be used to tackle these problems:
\begin{itemize}[leftmargin=1.0em, topsep=-.0em, parsep=-.0em]
\item \textbf{Rudimentary Manual Approach:}  
Hand-craft a directed graph, choose a solver, and input parameter values to obtain a solution. Although direct, this method demands considerable human effort and domain expertise.

\item \textbf{LLM-Assisted Semi-automation:}  
Use an LLM to suggest algorithms, extract parameters from specifications, and assist with coding. This approach can handle moderate complexity but becomes less feasible at scale.

\item \textbf{Full Automation:}  
Provide the problem statement to a multi-agent framework (e.g., MACI~\cite{chang2025MACI}) that executes end-to-end, including validation and replanning. This approach, combined with human oversight, scales best for complex scenarios with large numbers of nodes and frequent environmental changes.
\end{itemize}

%% Moved up to format
%\clearpage
\begin{table}[t!]
    \centering
    \caption{Urban Ride Sharing Problem}
    \vspace{-.1in}
    \begin{small}
    \renewcommand{\arraystretch}{1.1}
    \setlength{\fboxsep}{5pt} % Adjust padding inside the box
    \fbox{
    \begin{minipage}{0.45\textwidth} % Adjust width to fit within half a page
    \textbf{Metrics:}  
    \begin{itemize}[leftmargin=1em, topsep=0pt, itemsep=0pt, label=-]
        \item \textbf{On-time performance:} No penalty for early arrivals.
        \item \textbf{Total distance traveled.}
    \end{itemize}
    
    \textbf{Locations:} Seven locations: $V = \{A, B, C, D, E, F, G\}$, where $G$ is Boston Logan Airport (BOS). Urban locations $A$–$F$ are all 10 km of each other, while distances to BOS are 30+ km. 
    \[
    \begin{bmatrix}
        & A  & B & C & D & E & F \\
    A-F & 10 & 10 & 10 & 10 & 10 & 10 \\
    \rightarrow G & 35 & 33 & 36 & 34 & 32 & 31
    \end{bmatrix}
    \]
    
    \textbf{Travel speed:} ($A$–$F$) 60 km/h, and ($A$–$F \rightarrow G$) 100 km/h.
    
    \textbf{Passengers:} Each passenger specifies an arrival time at BOS ($G$). The dispatcher will instruct drivers when to pick up passengers to ensure on-time arrival at BOS.  

    \textbf{Ride Requests} (Desired BOS arrival time given):  
    \begin{itemize}[leftmargin=1em, topsep=0pt, itemsep=0pt, label=-]
    \item $r_1$: Pickup at $A$, to $G$ by 08:45 \hspace{1em} - $r_2$: Pickup at $B$, to $G$ by 08:50
    \item $r_3$: Pickup at $C$, to $G$ by 08:55 \hspace{1em} - $r_4$: Pickup at $D$, to $G$ by 09:00
    \end{itemize}

    \textbf{Available Vehicles} (Capacity 2 passengers): 
    \begin{itemize}[leftmargin=1em, topsep=0pt, itemsep=0pt, label=-]
        \item $k_1$: at $A$, $k_2$: at $C$, and $k_3$: at $E$
    \end{itemize}

    \textbf{Scheduling Constraints:}  
    - The dispatcher determines the \emph{pickup times} based on a feasible schedule. Pickup times must allow the driver to first reach the passenger location ($A$ - $F$) and then drive to $G$ in time.  
    \end{minipage}
    } % end of box
    \label{tab:appURS}
    \end{small}
    \vspace{-.2in}
\end{table}


\subsection{P1: Campus Single-Tour Navigation}

\textbf{Problem Statement:} A single autonomous agent must navigate a predefined campus environment to complete a sequence of waypoints while minimizing travel time. The scenario assumes a static environment without disruptions.

\textbf{Problem Specification:}
\begin{itemize}[leftmargin=1.0em, topsep=-.0em, parsep=-.0em, label=-]
    \item \textbf{Environment:} A known map with a finite set of locations.
    \item \textbf{Goal:} Visit all designated waypoints within a given timeframe.
    \item \textbf{Constraints:} Opening hours of each location, each location at least 30 minutes, and must be completed before 5 PM.
    \item \textbf{Optimization Metric:} Shortest path (time or distance).
\end{itemize}

A meta-plan, shown in Table~\ref{tab:P1SingleTour}, provides the high-level structure and constraints for the problem. This meta-plan serves as input to specialized solvers, such as dynamic programming or Monte Carlo algorithms, which then generate detailed, executable workflows. The process transforms abstract planning requirements into concrete, implementable sequences of actions while respecting all specified constraints and optimization objectives.

\subsection{P2: Multi-Group Campus Tour}
\textbf{Problem Statement:} Multiple groups of visitors require guided tours in different locations on a university campus, with optimized scheduling of multiple tour guides. This problem shares the same metrics as
P1.

\textbf{Problem Specification:}
\begin{itemize}[leftmargin=1.0em, topsep=-.0em, parsep=-.0em, label=-]
    \item Multiple agents (tour guides) must coordinate to serve different groups of visitors.
    \item Each group has predefined preferences and time constraints.
    \item Agents must follow non-overlapping paths while minimizing idle time.
    \item Location visiting hours must be observed.
\end{itemize}
\vspace{.1in}
Details are provided in Table~\ref{tab:P2MultiTour}.

\subsection{P3: Urban Ride-Sharing (URS)}
\textbf{Problem Statement:} Optimize real-time ride assignments for multiple vehicles and passengers in an urban environment, balancing efficiency, fuel use, and service quality.

\textbf{Problem Specification:}
\begin{itemize}[leftmargin=1.0em, topsep=-.0em, parsep=-.0em, label=-]
    \item \textbf{City Map:} A graph $G = (V, E)$ where locations $V$ and roads $E$ have distances and travel times.
    \item \textbf{Ride Requests:} A set of requests $R$, each defined by:
    \begin{itemize}[leftmargin=1.0em, topsep=-.0em, parsep=-.0em, label=--]
        \item Passenger ID, pickup/drop-off locations, time windows.
    \end{itemize}
    \item \textbf{Vehicles:} A set of available vehicles $K$, each with:
    \begin{itemize}[leftmargin=1.0em, topsep=-.0em, parsep=-.0em, label=--]
        \item Location, battery/fuel level, passenger capacity and speed.
    \end{itemize}
\end{itemize}
\vspace{.1in}
See Table~\ref{tab:appURS} for detailed specifications.


\subsection*{P4: Urban Ride Sharing (URS) with Disruptions}

\noindent \textbf{Problem Specification:} Details are provided in Table~\ref{tab:P4URSReactive}, including two disruption scenarios: airport route traffic delay and local road closure. Other possible disruptions could involve cancelation of passengers or late arrivals.

\begin{table}[htb]
\centering
\caption{Urban Ride-Sharing Reactive Planning Problem}
\label{tab:P4URSReactive}
\vspace{-.1in}
\begin{small}
\renewcommand{\arraystretch}{1.1}
\fbox{
\begin{minipage}{0.45\textwidth}
\textbf{Vehicles:} Three vehicles $V = \{v_1, v_2, v_3\}$
\begin{itemize}[leftmargin=1em, topsep=-.0em, parsep=-.0em, label=-]
\item Capacity: 2 passengers each
\item Initial vehicle location: City center
\item Operating hours: all day
\end{itemize}

\textbf{Passengers:} Five passengers $P = \{p_1, p_2, p_3, p_4, p_5\}$
\begin{itemize}[leftmargin=1em, topsep=-.0em, parsep=-.0em, label=-]
\item $p_1$: Airport by 8:30 AM
\item $p_2$: Airport by 9:00 AM
\item $p_3$: Airport by 9:30 AM
\item $p_4$: Airport by 9:45 AM
\item $p_5$: Airport by 9:45 AM
\end{itemize}

\textbf{Travel Times:}
\begin{itemize}[leftmargin=1em, topsep=-.0em, parsep=-.0em, label=-]
\item City center to pickup locations: 15-30 minutes
\item Pickup locations to airport: 45-60 minutes
\item Between pickup locations: 20-30 minutes
\end{itemize}

\textbf{Disruptions:}
\begin{itemize}[leftmargin=1em, topsep=-.0em, parsep=-.0em, label=-]
\item Airport route traffic delay
\item Certain local road closure
\end{itemize}
\textbf{Objectives:}
\begin{itemize}[leftmargin=1em, topsep=-.0em, parsep=-.0em, label=-]
\item Minimize total vehicle travel time
\item Meet all passenger deadlines
\end{itemize}
\end{minipage}
}
\end{small}
\end{table}

%%%%%%%%%%%%%%

\subsection*{P5: Wedding Reunion}

\noindent \textbf{Problem Specification:} Table~\ref{tab:Wedding} presents a coordinated wedding event travel problem. Several friends arrive at different times and locations before a 3:00~PM wedding photo session. The challenge includes managing two vehicles for airport pick-ups (aimed at those who cannot drive or wish to cut costs) and completing critical errands, such as collecting the wedding gift and retrieving formal attire from the tailor. All activities must be scheduled to ensure that everyone arrives at the wedding venue before the photo time.

This problem introduces more constraints than the URS problems in P3 and P4, and it also lays the groundwork for a more challenging disruption case discussed in~\textbf{P8}.

%\clearpage
\begin{table}[htb]
\caption{Wedding Reunion Logistics Problem}
\vspace{-.1in}
\centering
\begin{small}
\renewcommand{\arraystretch}{1.1}
\setlength{\fboxsep}{5pt}
\fbox{
\begin{minipage}{0.45\textwidth}
\textbf{Metrics:}
\begin{itemize}[leftmargin=1em, topsep=-.1pt, itemsep=-.1pt, label=-]
\item \textbf{On-time performance:} Must arrive at the venue for 3:00 PM photos.
%\item \textbf{Total distance traveled:} For cost-sharing calculation.
\end{itemize}
\textbf{Locations:} Four locations: $V = \{B, G, T, W\}$, where $B$ is Boston Airport, $G$ is Gift shop, $T$ is Tailor shop, and $W$ is Wedding venue.

\textbf{Travel time:} (minutes) 

$~B\text{-}G:45, ~B\text{-}T:30, ~B\text{-}W:40, ~G\text{-}T:20, ~G\text{-}W:25, ~T\text{-}W:15$. 

\textbf{Arrival Times:}  
\begin{itemize}[leftmargin=1em, topsep=-.1pt, itemsep=-.1pt, label=-]
\item Alex: At $B$ at 11:00 AM from Chicago (need a ride)
\item Jamie: At $B$ at 12:30 PM from Atlanta (need a ride)
\item Pat: At $W$ at 12:00 PM driving from NYC (has 5-seater car)
\end{itemize}

\textbf{Required Tasks:}
\begin{itemize}[leftmargin=1em, topsep=-.1pt, itemsep=-.1pt, label=-]
    \item Gift collection from $G$ (after 12:00 PM)
    \item Clothes pickup from $T$ (by 2:00 PM)
    \item Photos at $W$ (3:00 PM sharp)
\end{itemize}

\textbf{Available Resources:} 
\begin{itemize}[leftmargin=1em, topsep=-.1pt, itemsep=-.1pt, label=-]
    \item One car (5-seater) with Pat, available after he is Boston
    \item Local friend Chris (5-seater) available after 1:30 PM at $W$
\end{itemize}

\textbf{Scheduling Constraints:}  
- All tasks must complete before 3:00 PM photo time
- Gift store opens at 12:00 PM
- Tailor closes at 2:00 PM
- Two cars must accommodate all transport needs
\end{minipage}
}
\label{tab:Wedding}
\end{small}
\vspace{-.2in}
\end{table} % Ensure this file is correctly included

%%%%%%%%%%%%%%
\subsection*{\textbf{P6:} Thanksgiving Dinner}

Consider a Thanksgiving dinner scenario in which a family of five must return home in a Boston suburb for dinner. The problem involves coordinating departure times, managing travel logistics (including possible traffic delays), and ensuring timely arrival. Table~\ref{tab:ThanksgivingDinner1} formalizes these challenges as a sequential planning problem.

This scenario also lays the groundwork for a more advanced disruption case, which has proven difficult for standalone LLMs, as discussed in~\textbf{P9}.

\noindent \textbf{Problem Specification:}

\noindent \underline{\textit{Setup:}}
\begin{itemize}[leftmargin=1.5em, topsep=-.0em, parsep=-.0em, label=*]
\item Mom (Sarah) hosts dinner at 6:00 PM in Boston.
\item Family arrivals:
  \begin{itemize}[leftmargin=1em, topsep=-.0em, parsep=-.0em,]
  \item Dad (James) from San Francisco, lands at 1:00 PM ET.
  \item Sister (Emily) from Chicago lands at 2:30 PM.
  \item Brother (Michael) driving from NY arrives at 3:00 PM.
  \end{itemize}
\item Grandma, who is healthy, needs to pick up nearby.
\end{itemize}

\noindent \underline{\textit{Constraints:}}
\begin{itemize}[leftmargin=1.5em, topsep=-.0em, parsep=-.0em, label=*]
\item James must rent a car post-landing.
\item Emily needs an airport pickup (no alternatives).
\item Turkey requires 4 hours to cook; someone must be home once it's in the oven for safety.
\item Side dishes need 2 hours of preparation.
\item Travel times:
  \begin{itemize}[leftmargin=1em, topsep=-.0em, parsep=-.0em,]
  \item Home to BOS Airport: 1 hour.
  \item BOS Airport to Grandma’s: 1 hour.
  \item Home to Grandma’s: 30 minutes.
  \end{itemize}
\end{itemize}

\noindent \underline{\textit{Key Planning Questions:}}
\begin{enumerate}[leftmargin=1.5em, topsep=-.0em, parsep=-.0em, label=\arabic*.]
\item When should cooking start?
\item Who picks up Emily, and when?
\item When and by whom should Grandma be picked up?
\end{enumerate}


\begin{table}[thb]
\centering
\caption{Thanksgiving Dinner Coordination Problem}
\label{tab:ThanksgivingDinner1}
\vspace{-.1in}
\begin{small}
\renewcommand{\arraystretch}{1.1}
\fbox{
\begin{minipage}{0.45\textwidth}
\textbf{Objective:} Coordinate family arrivals and dinner preparation for 6:00 PM dinner in Boston

\textbf{Family Members and Arrivals:}
\begin{itemize}[leftmargin=1em, topsep=-.1pt, itemsep=-.1pt, label=-]
\item Sarah (Mom): Host, at home
\item James (Dad): Lands at BOS 1:00 PM from SF
\item Emily (Sister): Lands at BOS 2:30 PM from Chicago
\item Michael (Brother): Driving, arrives 3:00 PM from NY
\item Grandma: Needs pickup from suburban Boston
\end{itemize}

\textbf{Cooking Requirements:}
\begin{itemize}[leftmargin=1em, topsep=-.1pt, itemsep=-.1pt, label=-]
\item Turkey: 4 hours cooking time
\item Side dishes: 2 hours preparation
\item Someone must stay home during cooking
\end{itemize}

\textbf{Transportation Constraints:}
\begin{itemize}[leftmargin=1em, topsep=-.1pt, itemsep=-.1pt, label=-]
\item James must rent car after landing
\item Emily requires airport pickup
\item Travel times:
   \begin{itemize}
   \item Home to BOS Airport: 60 min
   \item BOS Airport to Grandma's: 60 min
   \item Home to Grandma's: 30 min
   \end{itemize}
\end{itemize}

\textbf{Key Requirements:}
\begin{itemize}[leftmargin=1em, topsep=-.1pt, itemsep=-.1pt, label=-]
\item All family members at home for 6:00 PM dinner
\item Turkey and sides ready by dinner time
\item All pickups completed with available drivers
\item Cooking supervision maintained
\end{itemize}
\end{minipage}
}
\end{small}
\end{table}

%%%%%%%%%%%%
\subsection*{P7: Disaster Relief Logistics Problem}

\noindent \textbf{Problem Specification:} Table~\ref{tab:DisasterRelief} summarizes the problem.

%\clearpage
\begin{table}[thb]
\centering
\caption{Disaster Relief Logistics Problem}
\vspace{-.1in}
\begin{small}
\renewcommand{\arraystretch}{1.1}
\fbox{
\begin{minipage}{0.45\textwidth}
\textbf{Network:} $G=(V,E)$ where $V$ represents locations and $E$ represents routes

\textbf{Locations ($V$):}
\begin{itemize}[leftmargin=1em, topsep=-.1pt, itemsep=-.1pt, label=-]
\item Supply nodes: CW (central warehouse), AP (airport, capacity: 5 tons)
\item Demand nodes: DC1, DC2, DC3 (distribution centers)
\item Critical nodes: H1, H2 (hospitals), FS (fuel station)
\end{itemize}

\textbf{Resources:}
\[
\begin{aligned}
cap_{truck} &= 4 \text{ tons}, \tau_{truck} = 120 \text{ min} \\
cap_{heli} &= 1 \text{ ton}, \tau_{heli} = 30 \text{ min}
\end{aligned}
\]

\textbf{Demand Requirements:}
\begin{itemize}[leftmargin=1em, topsep=-.1pt, itemsep=-.1pt, label=-]
\item DCs: food/water delivery by $T_{dc} = 20:00$
\item Hospitals: medicine within $\Delta T_h = 6$ hours
\item Fuel Station: refuel by $T_{fs} = 12:00$
\end{itemize}
\textbf{Critical Deadlines:}
\begin{itemize}[leftmargin=1em, topsep=-.1pt, itemsep=-.1pt, label=-]
\item Food/Water: All DCs by 8:00 PM
\item Medicine: Hospitals within 6 hours
\item Fuel: FS by 12:00 PM
\item Airport: Clear excess beyond 5 tons immediately
\end{itemize}

\textbf{Dynamic Disruptions:}
\begin{itemize}[leftmargin=1em, topsep=-.1pt, itemsep=-.1pt, label=-]
\item Unpredictable donation arrivals
\item Road blockages requiring rerouting
\item Emergency hospital demands
\item Fuel shortage delays
\end{itemize}

\textbf{Key Planning Requirements:}
\begin{itemize}[leftmargin=1em, topsep=-.1pt, itemsep=-.1pt, label=-]
\item Resource distribution scheduling
\item Transportation mode optimization
\item Delivery prioritization
\item Airport overflow management
\item Real-time disruption handling
\end{itemize}
\end{minipage}
}
\end{small}
\label{tab:DisasterRelief}
\vspace{-.15in}
\end{table}




\subsection*{P8: Wedding Reunion with Disruptions}

\noindent \textbf{Problem Extension:}
This problem extends \textbf{P5} with road closures and dynamic rerouting.

The disruption scenario becomes more challenging because, when a road closure is announced, the planner must know each vehicle's current location to determine whether it is affected. Since LLMs are inherently stateless, they cannot keep track of previous scheduling events and thus struggle to adapt the plan in real-time. 

\subsection*{P9: Thanksgiving Dinner with Disruptions}

This problem extends \textbf{P6} by introducing flight delays. Specifically, when a flight from SFO to BOS is delayed by $t$ hours, the new arrival time is confirmed at the originally expected arrival time minus the flight’s scheduled duration. Although this early notice provides an opportunity to adjust travel and dinner plans, current LLM-based systems fail to leverage this information in a timely manner, only beginning to react at the original arrival time and missing the window for earlier intervention.

For example, consider James’s flight, which was originally scheduled to arrive in Boston at 1~PM but has been delayed to 4~PM. He learns of this new arrival time at 10~AM~EST (the flight’s intended departure from SFO), offering a three-hour lead for adjustments. However, existing LLM-based solutions do not adapt the plan until 1~PM, thus squandering the opportunity to re-optimize the schedule.



\subsection*{P10: Global Supply Chain}

\begin{table*}[th!]
\caption{Summary of the Data Center Construction Problem Statement}
\label{tab:ProblemSupplyChain}
\vspace{-,1in}
\centering
\small
\begin{tabular}{|p{3.3cm}|p{13.6cm}|}
\hline
\textbf{Component} & \textbf{Details} \\
\hline
\multicolumn{2}{|l|}{\textbf{GPU Procurement \& Shipments}} \\
\hline
Total GPU Target & 100,000 units \\
\hline
Vendors & 
\begin{tabular}[c]{@{}l@{}}
\textbf{NVIDIA:} \$15k/unit, 200k units/quarter capacity, 20\% maintenance risk (of unit price over 1 year)\\
\textbf{AMD:} \$10k/unit, 150k units/quarter capacity,  50\% maintenance risk (of unit price over 1 year)
\end{tabular} \\
\hline
Order Timing & All orders placed on Day 1 \\
\hline
Shipment Schedule & Quarterly shipments; e.g., a 600,000 unit order results in 3 shipments of 200,000 units at the end of Q1, Q2, and Q3 \\
\hline
Payment Terms & 50\% deposit at order (day 1), 50\% at delivery (each quarter) \\
\hline
\multicolumn{2}{|l|}{\textbf{Natural Disaster Impacts}} \\
\hline
Risk Probabilities & Earthquake: 10\% per quarter; Typhoon: 10\% per quarter (assessed over entire project) \\
\hline
Delay Impact & Cascading 1-month delay on all subsequent shipments if a disaster occurs in a quarter \\
\hline
Cost Impact & 30\% price increase applied only to the affected quarter’s shipment \\
\hline
Shipment Expedition Impact & Expedite one shipment by 1 month and all subsequent shipments are moved up by 1 month (lock stepped) \\
\hline
\multicolumn{2}{|l|}{\textbf{Cluster Construction \& Infrastructure}} \\
\hline
Cluster Definition & Each cluster requires 50,000 GPUs; clusters can be built concurrently once required GPUs are delivered \\
\hline
\multicolumn{2}{|l|}{\textbf{Networking Infrastructure (per cluster)}} \\
\hline
Duration & 1 months \\
\hline
Cost & \$25 million \\
\hline
Dependencies & Can begin only after GPUs have arrived, power is ready, and cooling is operational \\
\hline
\multicolumn{2}{|l|}{\textbf{Power \& Cooling (per cluster)}} \\
\hline
Requirements & Power: 150 MW; Cooling: 1M gallons/day \\
\hline
Lead Time Options & 
\begin{tabular}[c]{@{}l@{}}Regular: 2 months at \$30M per cluster\\ Expedited: 1 months at \$75M per cluster (150\% cost premium)\end{tabular} \\
\hline
\multicolumn{2}{|l|}{\textbf{Testing \& Certification (per cluster)}} \\
\hline
Normal Process & 2 months, \$15 million \\
\hline
Expedited Process & 1 month, \$45 million (150\% cost premium) \\
\hline
Expedition Options & Expedite power/cooling, networking (50\% time reduction, 150\% cost premium), and testing independently per cluster \\
\hline
\multicolumn{2}{|l|}{\textbf{Business Impact of Delays}} \\
\hline
Delay Costs & \$50 million lost revenue + \$10 million additional operating costs per month of delay \\
\hline
\multicolumn{2}{|l|}{\textbf{Overall Objective}} \\
\hline
Objective & Minimize the end-to-end construction cost (procurement, construction, maintenance, delay costs, expedition premiums) while meeting the 15-month deadline. If the 15-month not met, adding delay costs \\
\hline
\end{tabular}
\vspace{-.1in}
\end{table*}


Table~\ref{tab:ProblemSupplyChain} presents a comprehensive problem in data center GPU deployment that captures the complexity of large-scale infrastructure projects. The objective is to complete a 1 million GPU data center in 15 months while minimizing total costs. The problem encompasses procurement decisions between NVIDIA ($15$k/unit) and AMD ($10$k/unit) GPUs, where each vendor has different maintenance risks (20\% vs 50\% of unit price over one year) and quarterly shipment capacities.

The construction process is organized around 50,000 GPU clusters, which require coordinated deployment of power, cooling, and networking infrastructure. Each cluster demands significant resources: 150 MW of power capacity and 1 million gallons per day of cooling water. Infrastructure development follows strict dependencies: Power and cooling systems must be operational before networking installation can begin, and each cluster must complete testing before becoming operational.

The problem incorporates real-world complexities such as risks of natural disasters (10\% probability per quarter for both earthquakes and typhoons), which can cause cascading delays and cost increases. Management can expedite various components; power / cooling installation can be accelerated from 4 to 2.5 months for a 80\% cost premium, while testing can be shortened from 2 to 1 month by doubling the cost.

The financial implications are significant, as each month of delay incurs $60$M in combined revenue loss and additional operating costs. This creates a complex optimization challenge: balancing procurement costs, expedition premiums, and risk mitigation strategies while adhering to physical and temporal constraints in the construction sequence.

\subsection*{P11: Stock Prediction/Forecasting}
\label{sec:p11}

Consider a stock market prediction scenario where an automated system must forecast future stock prices while integrating multiple data streams and accounting for market dynamics. The problem involves processing real-time data, managing prediction updates, and responding to market events. Table~\ref{tab:StockPrediction} formalizes this as a sequential planning problem, presenting a comprehensive framework for building an adaptive prediction system. The problem statement details the requirements for creating a robust prediction pipeline that can handle real-time market data, maintain high accuracy, and adapt to changing market conditions. A sample workflow demonstrating the system's architecture and data flow is provided in Table~\ref{tab:StockPrediction}.


\begin{figure*}[th!]
    \centering
    \begin{tikzpicture}[
        node distance=1.5cm,
        process/.style={rectangle, draw, fill=orange!20, minimum width=2.8cm, minimum height=1cm, thick},
        data/.style={cylinder, draw, fill=blue!15, shape border rotate=90, aspect=0.2, minimum width=2.5cm, minimum height=1cm, thick},
        decision/.style={diamond, draw, fill=blue!20, minimum width=2.2cm, minimum height=1.2cm, align=center, thick},
        action/.style={rectangle, draw, rounded corners, fill=green!15, minimum width=2.5cm, minimum height=1cm, thick},
        arrow/.style={->, line width=1.5pt, blue!60}
    ]
        % Data Sources
        \node[data] (market) at (-3.6,2) {Market Data};
        \node[data] (news) at (-3.6,0.95) {News Feed};
        \node[data] (econ) at (-3.6,-.15) {Economic Data};
        
        % Processing Nodes horizontal flow
        \node[process] (collect) at (0,2) {Data Collection};
        \node[process] (feature) at (3,2) {Feature Extraction};
        \node[process] (train) at (6,2) {Model Training};
        
        % Time labels above boxes
        \node[text width=2.8cm, align=center] at (0,2.8) {$t_{refresh}=5min$};
        \node[text width=2.8cm, align=center] at (3,2.8) {$t_{process}=5min$};
        
        % Validation and Prediction vertical flow
        \node[decision] (trainval) at (6,0.0) {Train\\ Valid?};
        \node[process] (predict) at (6,-2.0) {Prediction};
        
        % Alert and Trade
        \node[decision] (alert) at (3,-2.0) {Accurate?};
        \node[action] (trade) at (-0.6,-2.0) {Trading Alert};
        
        % Dependencies
        \draw[arrow, orange!60] (market) -- (collect);
        \draw[arrow, orange!60] (news) -- (collect);
        \draw[arrow, orange!60] (econ) -- (collect);
        \draw[arrow] (collect) -- (feature);
        \draw[arrow] (feature) -- (train);
        \draw[arrow] (train) -- (trainval);
        
        % Validation flows
        \draw[arrow] (trainval) -- node[right] {\textbf{Yes}} (predict);
        \draw[arrow, purple!60] (trainval.west) -- ++(-3,0) -- node[left] {\textbf{No}} (collect.south);
        
        % Real-time Flows
        \draw[arrow] (predict) -- (alert);
        \draw[arrow, blue!60] (feature) -- (alert);
        \draw[arrow] (alert) -- node[above] {\textbf{Yes}} (trade);
        \draw[arrow, purple!60] (alert.west) -- node[left] {\textbf{No}} (collect.south);
        
    \end{tikzpicture}
    \caption{Stock Market Prediction Workflow with Validation Loops. Two decision points validate: 1) model accuracy on testing data, and 2) real-time prediction accuracy for trading notifications.}
    \label{fig:P11workflow}
%    \vspace{.1in}
\end{figure*}


\noindent \textbf{Problem Specification:}

\noindent \underline{\textit{Setup:}}
\begin{itemize}[leftmargin=1.2em, topsep=-.0em, parsep=-.0em, label=*]
   \item \textit{Objective:} Predict a portfolio of stock prices for time \(\tau\), enabling actionable insights for risk management, investment decisions, and automated trading.
   \item \textit{Historical Data:}
   \begin{itemize}[leftmargin=1em, topsep=-.1em, parsep=-.1em]
       \item Training period: 2010--2020.
       \item Validation period: 2021--2023.
       \item Real-time data feed required during operation.
   \end{itemize}
   \item \textit{Data Sources:} Yahoo Finance, SEC EDGAR, news feeds, macroeconomic indicators.
   \item \textit{Required Features:}
   \begin{itemize}[leftmargin=1em, topsep=-.1em, parsep=-.1em]
       \item \textit{Technical Indicators:} MA, MACD, RSI, Bollinger Bands.
       \item \textit{Market Context:} Sector indices, competitor stock performance, correlation metrics.
       \item \textit{External Events:} Economic reports, policy changes, earnings announcements.
       \item \textit{Sentiment Analysis:} Real-time news, social media, financial disclosures.
   \end{itemize}
\end{itemize}

\noindent \underline{\textit{Constraints:}}
\begin{itemize}[leftmargin=1.5em, topsep=-.15em, parsep=-.15em, label=*]
   \item \textit{Processing latency:} Under 5 minutes for real-time predictions.
   \item \textit{Prediction confidence interval:} 95\% confidence bounds required.
   \item \textit{Model Dependencies:} 
   \begin{itemize}[leftmargin=1em, topsep=-.1em, parsep=-.1em]
       \item Data collection before feature extraction.
       \item Feature extraction before model training.
       \item Model training before prediction generation.
   \end{itemize}
   \item \textit{Performance Requirements:}
   \begin{itemize}[leftmargin=1em, topsep=-.1em, parsep=-.1em]
       \item Mean Absolute Percentage Error (MAPE) $<$ 5\% on validation.
       \item False alert rate $<$ 1\%.
       \item System uptime $>$ 99.9\% during market hours.
   \end{itemize}
\end{itemize}

\noindent \underline{\textit{State Space:}}
\begin{itemize}[leftmargin=1.5em, topsep=-.1em, parsep=-.1em, label=*]
    \item \textit{Processing Nodes ($V$):}
    \begin{itemize}[leftmargin=1em, topsep=-.1em, parsep=-.1em]
        \item Data Nodes: DC (data collection), FE (feature extraction).
        \item Model Nodes: MT (model training), PG (prediction generation).
        \item Integration Nodes: IG (integration), AG (alert generation).
    \end{itemize}
    \item \textit{Dependencies ($E$):}
    \begin{itemize}[leftmargin=1em, topsep=-.1em, parsep=-.1em]
        \item Pipeline: $DC \rightarrow FE \rightarrow MT \rightarrow PG \rightarrow AG$.
        \item Feedback: $AG \rightarrow DC$ for adaptive updates.
        \item External: Market events affect all nodes.
    \end{itemize}
\end{itemize}

\noindent \underline{\textit{Market Regime Adaptation:}}
\begin{itemize}[leftmargin=1.5em, topsep=-.1em, parsep=-.1em, label=*]
   \item Detect shifts in volatility, liquidity, and trend strength.
   \item Adjust prediction models based on regime detection.
   \item Implement fallback strategies for extreme market shifts.
\end{itemize}

\noindent \underline{\textit{Adaptive Refresh Rates:}}
\begin{itemize}[leftmargin=1.5em, topsep=-.1em, parsep=-.1em, label=*]
   \item \(t_{\text{Drefresh}}\) (data collection) adjusts based on market volatility.
   \item \(t_{\text{Nrefresh}}\) (news updates) increases during high-impact events.
\end{itemize}

\noindent \underline{\textit{Decision Module:}}
\begin{itemize}[leftmargin=1.5em, topsep=-.1em, parsep=-.1em, label=*]
   \item \textit{Trading Strategy:} Buy/sell signals, stop losses, risk thresholds.
   \item \textit{Risk Assessment:} Hedging, diversification, leverage constraints.
   \item \textit{Dynamic Adjustment:} Rebalancing frequency, asset weighting, market response.
\end{itemize}

\noindent \underline{\textit{Backtesting and Iterative Validation:}}
\begin{itemize}[leftmargin=1.5em, topsep=-.1em, parsep=-.1em, label=*]
   \item Run models on historical data to evaluate predictive accuracy.
   \item Compare performance on different market regimes/conditions.
   \item Adjust hyperparameters based on validation results.
\end{itemize}

\begin{table}[th!]
\centering
\caption{Stock Market Prediction Problem Framework}
%\vspace{-.1in}
\begin{small}
\renewcommand{\arraystretch}{1.1}
\fbox{
\begin{minipage}{0.45\textwidth}
\textbf{Network Structure:} $G=(V,E)$ where $V$ represents processing nodes and $E$ represents data flows

\textbf{Processing Nodes ($V$):}
\begin{itemize}[leftmargin=1em, topsep=-.1pt, itemsep=-.1pt, label=-]
\item Data nodes: DC (data collection), FE (feature extraction)
\item Model nodes: MT (model training), PG (prediction generation)
\item Integration nodes: IG (integration), AG (alert generation)
\end{itemize}

\textbf{Dependencies ($E$):}
\begin{itemize}[leftmargin=1em, topsep=-.1pt, itemsep=-.1pt, label=-]
\item Technical Analysis: $DC \rightarrow FE \rightarrow MT$
\item Market Context: External events affect all nodes
\item Model Chain: $MT \rightarrow PG \rightarrow AG$
\item Feedback Loop: $AG \rightarrow DC$ for adaptive updates
\end{itemize}

\textbf{Data Requirements:}
\begin{itemize}[leftmargin=1em, topsep=-.1pt, itemsep=-.1pt, label=-]
\item Historical: Training (2010--2020), Validation (2021--2023)
\item Technical Indicators: MA, BB, MACD, RSI, ATR, OBV
\item Price Data: OHLCV with adjustments
\item External: Policy, economic indices, news sentiment
\end{itemize}

\textbf{System Parameters:}
\[
\begin{aligned}
t_{refresh} &= 5 \text{ min}, \text{conf}_{interval} = 95\% \\
\text{MAPE} &< 5\%, \text{uptime} = 99.9\%
\end{aligned}
\]

\textbf{Workflow Stages:}
\begin{itemize}[leftmargin=1em, topsep=-.1pt, itemsep=-.1pt, label=-]
\item Stage 1: Data Collection \& Feature Extraction
\item Stage 2: Model Training \& Validation
\item Stage 3: Prediction Generation
\item Stage 4: Alert \& Decision Support
\item Stage 5: Iterative Refinement
\end{itemize}

\textbf{Dynamic Adaptations:}
\begin{itemize}[leftmargin=1em, topsep=-.1pt, itemsep=-.1pt, label=-]
\item Market regime shifts trigger model updates
\item News events modify prediction weights
\item Volatility affects refresh rates
\item Anomalies initiate retraining cycles
\end{itemize}

\textbf{Performance Metrics:}
\begin{itemize}[leftmargin=1em, topsep=-.1pt, itemsep=-.1pt, label=-]
\item Prediction accuracy (RMSE, MAPE)
\item System responsiveness (latency)
\item Alert precision (false positive rate)
\item Model adaptability (regime changes)
\end{itemize}
\end{minipage}
}
\end{small}
\label{tab:StockPrediction}
\end{table}


\begin{comment}
\begin{table}[thb]
\centering
\caption{Stock Market Prediction Problem}
\label{tab:StockPrediction}
\vspace{-.1in}
\begin{small}
\renewcommand{\arraystretch}{1.1}
\fbox{
\begin{minipage}{0.45\textwidth}

\textbf{The Problems}

\textbf{1) Stock Price Prediction:} Forecast future stock prices (e.g., predicting Apple’s stock on a future date) using historical market data.

\textbf{2) Complex Market Dynamics:} The system must account for influences such as technical indicators, historical price patterns, and external events (policy announcements, CPI, natural disasters, geopolitical conflicts).

\textbf{3) Decision Making Under Uncertainty:} Supports what-if analyses and real-time alerts for decision-makers to respond quickly to unexpected events.

\textbf{The State Space}

\textbf{1) Who:} Companies or stocks (e.g., Apple) and related market players (competitors, sector peers).

\textbf{2) What:}
\begin{itemize}[leftmargin=1em, topsep=-.1pt, itemsep=-.1pt, label=-]
\item Technical indicators: MA, BB, MACD, RSI, ATR, OBV, lagged closing prices.
\item Traditional price data: Open, high, low, close, adjusted close, trading volumes.
\item Self-defined features: News sentiment, policy announcements.
\end{itemize}

\textbf{3) When:}
\begin{itemize}[leftmargin=1em, topsep=-.1pt, itemsep=-.1pt, label=-]
\item Historical Data: Training (2010–2020), validation (2021–2023).
\item Future Forecasting: Target date (e.g., 2026-02-01).
\end{itemize}

\textbf{4) Where:} Market context including policy changes, economic reports (CPI), and geopolitical events.

\textbf{Interdependencies}

\textbf{1) Market Patterns and Correlations:} Identify how movements in one sector or competitor’s stock affect another.

\textbf{2) External Influences:} Policy announcements, economic reports, and disruptions (earthquakes, geopolitical crises) create cascading market effects.

\textbf{3) Integration of Data Types:} Combine numerical indicators, historical patterns, and qualitative information (news sentiment, human prompts) for a holistic market view.

\textbf{Proposed Workflow for Decision Making}

\textbf{1) Data Collection \& Feature Extraction:}
\begin{itemize}[leftmargin=1em, topsep=-.1pt, itemsep=-.1pt, label=-]
\item Gather historical market data (prices, volumes, technical indicators) from sources like Yahoo Finance and SEC EDGAR.
\item Integrate external data such as policy announcements and economic indices.
\end{itemize}

\textbf{2) Model Training \& Prediction:}
\begin{itemize}[leftmargin=1em, topsep=-.1pt, itemsep=-.1pt, label=-]
\item Use predictive models (Support Vector Regression, XGBoost, LSTM) to learn feature-stock price relationships.
\item Evaluate models using RMSE and validation datasets.
\end{itemize}

\textbf{3) Integration \& Optimization:}
\begin{itemize}[leftmargin=1em, topsep=-.1pt, itemsep=-.1pt, label=-]
\item Combine multiple features and models (meta-planning) to optimize predictions.
\item Adjust hyperparameters, refine features iteratively.
\end{itemize}

\textbf{4) Decision Support \& Real-Time Alerts:}
\begin{itemize}[leftmargin=1em, topsep=-.1pt, itemsep=-.1pt, label=-]
\item Perform what-if analyses to predict event impacts (e.g., policy changes, natural disasters).
\item Set up alerts for sudden market changes.
\end{itemize}

\textbf{5) Iterative Validation \& Refinement:} Continuously validate predictions with new data, refining models based on feedback.

\end{minipage}
}
\end{small}
\end{table}
\begin{table}[thb]
\centering
\caption{Meta Plan based on Stock Prediction Specification}
\label{tab:MetaPlanner}
\vspace{-.1in}
\begin{small}
\renewcommand{\arraystretch}{1.1}
\fbox{
\begin{minipage}{0.45\textwidth}


\textbf{Step 1: Define Objectives} 

\begin{equation} 
{\mathcal{O} = \{ \text{Forecast future stock prices}, \text{Integrate external influences}, 
\\
\text{Support decision making with real-time alerts} \}}
\end{equation}

\textbf{Step 2: Identify Explicit Constraints}
\begin{equation}
\mathcal{C}_{\mathcal{E}} = \{ \text{Use historical data from 2010 to 2020}, \text{Consider external events}, \\
\text{Ensure compliance with market regulations} \}
\end{equation}

\textbf{Step 3: Identify Implicit Constraints}
\begin{equation}
\mathcal{C}_{\text{I}} = \{ \text{Market trends influence stock prices}, \text{Technical indicators \\
provide insights} \}
\end{equation}

\textbf{Step 4: Define the Agent Pool}
\begin{equation}
A = \{ \text{Support Vector Regression}, \text{XGBoost}, \text{LSTM}, \text{Random Forest}, \text{\\
Expert Systems} \}
\end{equation}

\textbf{Step 5: Define Performance Metrics}
\begin{equation}
\mathcal{M} = \{ \text{Mean Absolute Error (MAE)}, \text{Root Mean Squared Error (RMSE)}, \\
\text{R-squared value} \}
\end{equation}

\textbf{Step 6: Construct the Workflow Network}
\begin{itemize}
\item \textbf{Roles:} Data collection, feature extraction, model training, prediction, integration, optimization, decision support.
\item \textbf{Dependencies:} Data must be collected before feature extraction, and so on.
\end{itemize}
\begin{equation}
\mathcal{N} = \{ \text{Data Collection}, \text{Feature Extraction}, \text{Model Training}, \text{Prediction}, \\ 
\text{Integration}, \text{Decision Support}, \text{Validation} \}
\end{equation}
\begin{equation}
\mathcal{E} = \{ \text{Collection} \rightarrow \text{Feature Extraction}, \text{Feature Extraction} \rightarrow \text{Model \\ Training}, \text{Model Training} \rightarrow \text{Prediction}, \text{Prediction} \rightarrow \text{Integration}, \text{Integration} \rightarrow \text{Decision Support}, \text{Decision Support} \rightarrow \text{Validation} \}
\end{equation}

\textbf{Step 7: Assign Agents to Nodes and Edges}
\begin{equation}
A_{n}^{*} = \{ \text{Web Scraper}, \text{Data Cleaning Algorithm}, \text{Support Vector \\ Regression}, \text{XGBoost}, \text{LSTM} \}
\end{equation}
\begin{equation}
A_{e}^{*} = \{ \text{Dependency Management Algorithm} \}
\end{equation}

\textbf{Step 8: Iterative Refinement}
\begin{itemize}
\item Evaluate workflow using performance metrics.
\item Update role-agent mappings based on performance feedback.
\item Verify dependencies for correctness.
\end{itemize}

\textbf{Final Workflow Representation}
\begin{equation}
\mathbf{W}^{*} = (\mathcal{N}, \mathcal{E})
\end{equation}

\end{minipage}
}
\end{small}
\end{table}

\end{comment}


