\section{Conclusion}

REAL-Bench represents a significant step toward systematically evaluating AI systems' capabilities in real-world planning scenarios. By providing 11 carefully designed problems that progress in complexity, the benchmark enables researchers to:

\begin{itemize}[leftmargin=1.0em, topsep=-.0em, parsep=-.0em, label=-]
\item Assess planning capabilities in multiple dimensions of difficulty.
\item Test system performance on real-world planning challenges.
\item Evaluate handling of unexpected interruptions and adaptations.
\item Compare different approaches using standardized metrics.
\end{itemize}

The benchmarks are designed to be both tractable for systematic evaluation and challenging for current systems. Each problem can be scaled along multiple dimensions, including the number of parallel threads, complexity of dependencies, and frequency of disruptions, allowing researchers to progressively stress-test their systems. Inclusion of validation metrics and baseline implementations facilitates meaningful comparisons between different approaches.

Looking ahead, we envision this benchmark suite evolving with community contributions and feedback. Future extensions might include more complex scenarios, additional evaluation metrics, and expanded validation tools. For instance, in many workflows, transaction properties must be preserved:

\begin{itemize}[leftmargin=1.0em, topsep=-.0em, parsep=-.0em, label=-]
   \item \textit{Atomicity}: An operation either completes entirely or not at all, with no partial execution state (e.g., a ride-sharing trip must either complete fully or be canceled entirely).
   \item \textit{Idempotency}: Multiple identical requests produce the same outcome as a single request, preventing duplicate actions (e.g., multiple identical order submissions should not result in multiple orders).
\end{itemize}

Most importantly, by providing a common framework for evaluating planning capabilities of both individual LLMs and multi-agent systems, we hope to accelerate progress toward more robust and capable AI planning systems that can handle real-world complexity and uncertainty. The REALM benchmark suite, along with detailed documentation and baseline implementations, will be available as an open source resource after the peer review process.

