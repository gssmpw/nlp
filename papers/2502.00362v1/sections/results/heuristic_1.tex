\subsection{Evaluating heuristic formulation}

The key motivation behind the heuristic formulation is to tackle even larger query graphs. Our main goal is to demonstrate that this algorithm reaches acceptable results with superior scalability compared to the previous Precise 1 and 2 formulations. The results also indicate that Theorem \ref{thm:greedy_bound} is respected in practice. The optimal results are computed with dynamic programming without cross products. Due to space limitations, we only included the results from the Gurobi solver, which we consider the most demonstrative, and we had unlimited access to it since it runs locally. 

The results are presented so that we have computed and scaled the difference between each pair of methods. A value that differs from 0 indicates that the two methods gave different join trees with different costs. Since one of the methods is near-optimal (DP without cross products), it is clear which method produced the suboptimal result. This way, we can compare all three methods at the same time. In all cases, we can see that the heuristic algorithm respects Theorem \ref{thm:greedy_bound} very well in practice, so the difference between quantum and greedy is always $0$.

Fig.~\ref{fig:clique_accuracies} shows the results of applying the heuristic method to clique query graphs. Although these results are good, the scalability in this hard case is modest. On the other hand, we are unaware of any quantum computing research that would have outperformed this scalability in the case of clique graphs. For example, the most scalable method \cite{10.14778/3632093.3632112} excluded clique graphs from their results.

\begin{figure}
    \centering
    \includegraphics[width=\linewidth]{results/clique_accuracies.png}
    \caption{Heuristic results for clique query graphs using Gurobi solver}
    \label{fig:clique_accuracies}
\end{figure}

The results for the tree (Fig.~\ref{fig:tree_accuracies}), chain (Fig.~\ref{fig:chain_accuracies}), cycle (Fig.~\ref{fig:cycle_accuracies}), and star graphs demonstrate the best scalability. We computed the results up to 60 tables to demonstrate advantageous scalability over the most scalable method in the previous research \cite{10.14778/3632093.3632112} where they considered queries up to 50 relations.

\begin{figure}
    \centering
    \includegraphics[width=\linewidth]{results/tree_accuracies.png}
    \caption{Heuristic results for tree query graphs using Gurobi solver}
    \label{fig:tree_accuracies}
\end{figure}

\begin{figure}
    \centering
    \includegraphics[width=\linewidth]{results/chain_accuracies.png}
    \caption{Heuristic results for chain query graphs using Gurobi solver}
    \label{fig:chain_accuracies}
\end{figure}
    
\begin{figure}
    \centering
    \includegraphics[width=\linewidth]{results/cycle_accuracies.png}
    \caption{Heuristic results for cycle query graphs using Gurobi solver}
    \label{fig:cycle_accuracies}
\end{figure}

We exclude the results for the star query graphs because, in this case, all three methods performed identically across up to 60 graphs and over 20 iterations, with no difference observed (a relative scaled difference of 0). These results may be because star graphs typically do not benefit from cross products \cite{10.14778/3632093.3632112}, and our method, which excludes them, performs better with such types of queries.

