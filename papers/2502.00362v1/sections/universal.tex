\subsubsection{Universal quantum computers}

The reduction methods are especially meant for quantum annealers. Universal quantum computers are able to encode HUBO problems without the reduction~\cite{10313783, 10.1145/3478519}. This is an interesting practical difference between universal quantum computing and quantum annealing.

Since previous research has not explicitly formulated HUBOs for universal quantum computers, we develop, describe, and apply the method in this paper. We implement a similar approach as QAOA with a modified parametrization. First, we translate the HUBO problem into the equivalent Ising Hamiltonian
\begin{equation}\label{eq:hubo_to_hamiltonian}
    \sum_{S \subset [n]}\alpha_{S}\prod_{i \in S}x_i \mapsto \sum_{S \subset [n]}\alpha_{S}\prod_{i \in S}\frac{I - \sigma_{z}^{i}}{2},
\end{equation}
where $I$ is the identity matrix and $\sigma_{z}^{i}$ is the matrix for the Pauli-Z gate acting on qubit $i$. Based on quantum mechanics, the minimum eigenstate and the corresponding minimum eigenvector of Hamiltonian $H$ solve the HUBO optimization problem. 

Next, we utilized a modified QAOA routine to find the minimum eigenvalue and eigenstate with a universal quantum computer. The original QAOA \cite{farhi2014quantum} has been widely applied to solve QUBOs, but the same algorithm can be used to find the minimum eigenvalue and eigenstate of Hamiltonians corresponding HUBOs. The key ingredients of the algorithm are two parametrized unitaries called cost and mixer. These unitaries are layered $p$ times. The quantum system is initialized to an equal superposition, and the expectation value of the cost of Hamiltonian is measured. With classical methods, the parameters in cost and mixer can be optimized (cf. training in machine learning) so that we have a high probability of measuring the minimum eigenvalue and eigenstate of the cost Hamiltonian.

For a given cost Hamiltonian $H$ with $m$ terms, the parametrized cost unitary is defined as
\begin{equation*}
    U(\gamma) = e^{-i\gamma H} = \prod_{j = 1}^{m}e^{-i\gamma H_j},
\end{equation*}
where $H_j$ is the $j$th term. Theoretically, only a single parameter $\gamma$ suffices for each layer $p$. In practice, we noticed that even with large values of $p$, we were not able to find the correct parameters probably because the optimization problem became hard. Thus, we modified the algorithm so that we decreased the number of layers to one and increased the number of parameters using the following cost-unitary
\begin{equation*}
    U(\gamma_1, \ldots, \gamma_m) = \prod_{j = 1}^{m}e^{-i\gamma_j H_j}.
\end{equation*}
Instead of having a single parameter $\gamma$, we use $\gamma_j$ for every $ j = 1, \ldots, m$. This small modification seemed to make the optimization task easier in practice. Otherwise, we used the same mixer and measured the same expectation value as in the original QAOA algorithm.

Additionally, small HUBOs can be solved using D-Wave's Exact Poly Solver function, which simply enumerates all the possible solutions. The approach becomes infeasible with around 20 variables.