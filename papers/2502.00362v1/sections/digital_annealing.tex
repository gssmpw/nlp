\section{Appendix: Comparison with quantum-inspired digital annealing}

We evaluated the method proposed in \cite{10.14778/3632093.3632112} with the same workloads we used and present the results in Figures \ref{fig:vldb24_chain}, \ref{fig:vldb24_clique}, \ref{fig:vldb24_cycle}, and \ref{fig:vldb24_star}. The method is the improved algorithm from \cite{Schonberger_Scherzinger_Mauerer}. The authors propose a novel readout technique that improves the results. Since we could not access special quantum-inspired hardware, such as a digital annealer, we used the Gurobi solver, which returns only a single result by default. Thus, the readout technique was not applicable. 

Nevertheless, the results still demonstrate that the method reaches a comparable accuracy to ours, which is optimal or close to optimal. Their method seems to be able to identify beneficial cross products. We have computed the exact results with dynamic programming and compared relative cumulative costs between the methods. Due to higher-order terms in our method, which currently have to be rewritten into quadratic format, their method is still more scalable than ours. On the other hand, our theoretical bounds, the more straightforward variable definitions, and the novel usage of the higher-order model show specific improvements over their methods. We are also positive that our model admits features that make it easier to expand for outer joins and include more complex dependencies between the predicates.

\begin{figure}[!ht]
\centering
\begin{subfigure}{.5\textwidth}
  \centering
  \includegraphics[width=.95\linewidth]{results/vldb24/chain_gurobi.png}
  \caption{Chain graphs with varying number of nodes solved with QUBO formulation proposed in \cite{10.14778/3632093.3632112}}
  \label{fig:vldb24_chain}
\end{subfigure}%
\hfill
\begin{subfigure}{.5\textwidth}
  \centering
  \includegraphics[width=.95\linewidth]{results/vldb24/clique_gurobi.png}
  \caption{Clique query graphs with varying number of nodes solved with QUBO formulation proposed in \cite{10.14778/3632093.3632112}}
  \label{fig:vldb24_clique}
\end{subfigure}
\caption{A figure with two subfigures}
\label{fig:test}
\end{figure}

\begin{figure}[!ht]
\centering
\begin{subfigure}{.5\textwidth}
  \centering
  \includegraphics[width=.95\linewidth]{results/vldb24/cycle_gurobi.png}
  \caption{Cycle query graphs with varying number of nodes solved with QUBO formulation proposed in \cite{10.14778/3632093.3632112}}
  \label{fig:vldb24_cycle}
\end{subfigure}%
\hfill
\begin{subfigure}{.5\textwidth}
  \centering
  \includegraphics[width=.95\linewidth]{results/vldb24/star_gurobi.png}
  \caption{Star query graphs with varying number of nodes solved with QUBO formulation proposed in \cite{10.14778/3632093.3632112}}
  \label{fig:vldb24_star}
\end{subfigure}
\caption{A figure with two subfigures}
\label{fig:test}
\end{figure}