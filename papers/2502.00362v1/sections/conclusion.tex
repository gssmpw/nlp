\section{Conclusion and future work}

In this work, we have developed three novel higher-order binary optimization models to solve join order selection problem. The HUBO problems can be divided into cost function and validity constraints. We presented two new binary optimization formulations for the cost function and proved that one encodes the same join trees as the dynamic programming algorithm without cross-products. The other find plans that are at least as good as those returned by the greedy algorithm without cross-products. Finally, we presented a comprehensive experimental evaluation of these algorithms on various quantum and classical solvers. The experimental evaluation demonstrated the practical scalability of this algorithm and the fact that we respect the proven bounds in practice.

Previous methods have been limited to inner joins. Extending our binary variables to model non-inner joins is theoretically straightforward, for instance, by adding a component to indicate the join type (inner or outer). While this modification is simple, the cost function and constraints also need adjustments. Since our cost HUBO is explicitly based on the recursive join order cost function, encoding predicate dependencies is also feasible. Exploring these extensions offers promising directions for future research in join order optimization with quantum computing.

%Use HUBOs as templates for problems.

%Solve HUBOs on universal quantum computers.

%Join order optimization is a very complex problem. All solutions to the problem with quantum computing using various binary models, including the one presented here, have not tackled the interesting cases that are beyond the reach of dynamic programming algorithms. It is possible that QUBO and other binary formulations are not sufficient for this type of problem. This does not mean that quantum computing approaches should be abandoned.

%A left-deep join order optimizer is necessarily a ranking algorithm that ranks the tables in the order in which we should perform the joins. In the bushy tree cases, many tables can have the same rank. The idea of ranking gives a hint that various random and quantum walk-based algorithms could also be used to find good join orders. These algorithms are fundamentally different from the previous solutions the community has proposed to solve the problem. In quantum computing, they would be implemented with the circuit model or with adiabatic quantum computing.

%We do not suggest that join order selection should currently be optimized on quantum hardware. In the future, we will likely identify a range of join order selection problem sizes where optimization is too complex for dynamic programming but still manageable for quantum hardware so that the quantum method will outperform the corresponding classical methods. In some sense, this approach is already implemented in PostgreSQL, where the small instances are solved exactly and the larger instances heuristically. We suggest that in the future there will be ''medium'' instances, for example, 30-50 table queries, which are suitable for quantum hardware. Before entering this quantum utility era, the database community needs ambitious baselines where we can actually start pushing the limits of the current quantum systems and creating novel quantum algorithmic solutions for hard database problems.