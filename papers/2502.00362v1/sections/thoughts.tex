The join order selection problem is dynamic in the sense that we perform one or multiple joins at each step until we reach the final result. Joins that we perform earlier affect the total cost in the end. This dynamic time element makes encoding the problem with binary models challenging since there is no natural way of doing this encoding. Most of the previously modeled optimization problems are static compared to join order selection \cite{lucas}.

HUBO problems are hard to solve classically. Constructing problems that admit natural HUBO formulation would ideally produce use cases that could be solved more efficiently with quantum devices than classical devices. So far, at least in this problem instance, we have not experienced this advantage. Considering how the number of HUBO variables (which is a one-to-one correspondence with a required number of qubits) is mapped to the equivalent QUBO formulation, we see how HUBO problems are clearly more expressible.