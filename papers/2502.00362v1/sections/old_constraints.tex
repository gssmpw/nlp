\subsection{Select exactly $|V| - 1$ many different edges from query graph.}
Every edge corresponding to a join in a query graph should appear in the solution either 0 or 1 times. To encode this constraint, for each join, we introduce an auxiliary variable that counts whether the join appears in the solution. This adds $|E|$ new variables to the model. We denote these auxiliary variables as $x_{ij}$ for each $(R_i, R_j) \in E$.

We utilize a similar structure as in Eq.~\eqref{eq:hubo_combinations}, but instead of fixing the number of selected variables to be exactly one, we select either 1 or 0 variables to be true. The constraint that encodes this is the following objective:
\begin{equation}\label{eq:exactly_v1_different_joins}
    H_2 = \sum_{(R_i, R_j) \in E}\left( x_{ij} - \sum_{r = 0}^{|V| - 2}x_{i,j}^{r} \right)^2.
\end{equation}
If the auxiliary variable $x_{ij} = 1$, then the objective function is minimized when precisely one variable of type $x_{i,j}^{r}$ is selected true.

Then, we must select exactly $|V| - 1$ many auxiliary variables to be true. The following constraint achieves this:
\begin{equation}
    H_{\text{aux}} = \left( |V| - 1 - \sum_{(R_i, R_j) \in E}x_{ij} \right)^2.
\end{equation}

\subsection{Ranks should be arranged so the selected joins produce a left-deep tree.}
We use the variables to explain the idea behind this constraint. Variable $x_{k,l}^{r+1}$ with rank $r + 1$ should have the following relationship with the variable $x_{i,j}^{r}$ with rank $r$: if $R_i \neq R_k$ and $R_j \neq R_l$ in the query graph, then the product $x_{i,j}^{r}x_{k,l}^{r+1}$ should have a positive coefficient, i.e., these variable combinations are penalized. In other words, we always want consecutive joins to share a common table. This constraint necessarily encodes the fact that the plan is left-deep. The objective function for this is the following:
\begin{equation}\label{eq:connected_ranks}
    H_3 = A\sum_{r = 0}^{|V| - 3}x_{i,j}^{r}x_{k,l}^{r+1},
\end{equation}
for all $R_i \neq R_k$ and $R_j \neq R_l$ and where $A$ is the positive penalizing term.

\subsection{Every table has to be joined.}
This constraint is again similar to constraints \eqref{eq:hubo_combinations} and \eqref{eq:exactly_v1_different_joins} where we wanted to select a specific number of variables to be true. In this case, we want to force a constraint that guarantees that every table appears at least once in the solution returned by the binary model. We proceed with the trick developed in \cite{lucas}, where we can encode an integer value using binary variables.
\begin{equation}\label{eq:every_table_joined_once}
    H_4 = \sum_{R_i \in V} \left( x_{i} - \sum_{(R_i, R_j) \in E} x_{i,j}^{r} \right)^2,
\end{equation}
where $x_i = \sum_{n = 0}^{M - 1}2^n y^{i}_{n} + (N + 1 - M)y_{M}^{i}$ where $M = \lfloor \log(|V| - 1) \rfloor$ and $y_{k}^{i}$ are auxiliary variables for each $i = 0, \ldots, |V| - 1$ and $k = 0, \ldots, M$. Again, we get two different HUBO formulations with these validity constraints and the previous two cost functions.