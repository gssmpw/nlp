\section{Introduction}

Join order optimization is one of the critical stages in query optimization where the goal is to determine the most efficient sequence in which joins should be performed \cite{10.1145/582095.582099}. The join order can significantly affect query performance, especially in large databases \cite{Neumann_Radke_2018}. Join order optimization is a well-researched NP-hard problem \cite{10.1145/1270.1498} with various exhaustive and heuristic solutions \cite{10.14778/2850583.2850594, 10.1007/s007780050040}.

The central position of join order optimization in database research is also evident in the new research field where quantum computing is applied to database optimization and data management \cite{Schonberger_2022, Uotila_2022}. In this subfield, the join order selection problem is the most studied \cite{Schonberger_Scherzinger_Mauerer, Winker_Calikyilmaz_Gruenwald_Groppe_2023, Schonberger_Trummer_Mauerer_2023, Nayak_Winker_Groppe_Groppe_2024, Franz_Winker_Groppe_Mauerer_2024,10.14778/3632093.3632112, DBLP:conf/q-data/SaxenaSS24}. The other quantum computing for database and data management-related problems includes index selection \cite{Gruenwald_Winker_Groppe_Groppe_2023, DBLP:conf/q-data/TrummerV24}, cardinality and metric estimations \cite{Uotila_2023_sqlcircuits,DBLP:conf/q-data/KittelmannSS24}, transaction scheduling \cite{Bittner_Groppe_2020b}, resource allocation \cite{Uotila_Lu_2023}, schema matching \cite{Fritsch_Scherzinger_2023} and multiple query optimization \cite{Trummer_Koch_2016}.

Gaining an advantage of quantum algorithms over classical algorithms has proved extremely challenging in real-life applications, including the listed database applications. Depending on the definition of quantum advantage, some famous experiments \cite{Arute_Arya_Babbush_Bacon_Bardin_Barends_Biswas_Boixo_Brandao_Buell_etal2019,Harrow_Montanaro_2017,Kim_Eddins_Anand_Wei_vandenBerg_Rosenblatt_Nayfeh_Wu_Zaletel_Temme_etal_2023,King_Nocera_Rams_Dziarmaga_Wiersema_Bernoudy_Raymond_Kaushal_Heinsdorf_Harris_etal_2024,Madsen_Laudenbach_Askarani_Rortais_Vincent_Bulmer_Miatto_Neuhaus_Helt_Collins_etal_2022,Zhong_Wang_Deng_Chen_Peng_Luo_Qin_Wu_Ding_Hu_etal_2020, Zhu_Cao_Chen_Chen_Chen_Chung_Deng_Du_Fan_Gong_etal_2021} demonstrate specific advantages, but they are not known to have any real-life applications. Thus, no real-life application of quantum computing is widely accepted to demonstrate quantum advantage in any field. On the other hand, some algorithms, such as Shor's and Grover's algorithms, show a provable advantage over the best classical algorithm on fault-tolerant quantum computers, which do not yet exist. These theoretical results are a key motivation for developing better quantum computing hardware, and this work aims to contribute to this algorithmic development.

The difficulty of applying quantum computing lies in the fundamentally different computational models~\cite{Nielsen_Chuang_2010}, different complexity analyses \cite{aaronson2016complexitytheoreticfoundationsquantumsupremacy}, the probabilistic nature of quantum computing, small-scale and erroneous quantum hardware, barren-plateaus in training and optimization landscapes \cite{Ragone_Bakalov_Sauvage_Kemper_Ortiz_Marrero_Larocca_Cerezo_2024} and the fact that it seems complicated to develop high-performing quantum algorithms. The last property is supported by the fact that the most essential quantum algorithm primitives can be listed on a single webpage \cite{quantum_algorithm_zoo}. 

Since showing that the current quantum computers provide any benefit has been challenging, \cite{10.14778/3632093.3632112} suggested moving from quantum hardware to quantum-inspired hardware, especially in join order optimization. They argued that we should study special-purpose solvers and hardware, including digital annealers. While they showed that this direction is promising, they did not extensively examine the possible benefits of modifying the underlying quantum optimization model, which has been similar to the MILP solution \cite{Trummer_Koch_2017}.

As a continuation of the idea to revise the underlying assumptions about hardware (relaxing from quantum to quantum-inspired), we suggest applying a special optimization model, a higher-order binary optimization model, which is a relaxation of the previously widely used quadratic model \cite{Schonberger_Scherzinger_Mauerer,Trummer_Koch_2017,10.14778/3632093.3632112,Schonberger_Trummer_Mauerer_2023,Nayak_Winker_Groppe_Groppe_2024, DBLP:conf/q-data/SaxenaSS24}. If we seek (database) applications that are likely to benefit from quantum computing, one of our central arguments is that we might want to move from quadratic to higher-order models. Focusing on the quantum computing paradigm that is restricted to quadratic interactions between qubits, the research has shown that there are only particular problems where these devices beat classical computers \cite{PhysRevX.6.031015,King_Nocera_Rams_Dziarmaga_Wiersema_Bernoudy_Raymond_Kaushal_Heinsdorf_Harris_etal_2024}. On the other hand, there is no evidence that this advantage would transfer to practically relevant problems \cite{Willsch_Willsch_Gonzalez_Calaza_Jin_De_Raedt_Svensson_Michielsen_2022}. One of the key challenges is ''quadratization'', which requires that the real-life problem is translated into a quadratic format. Translating a practical problem into this format is often so expensive that we lose the potential advantage, even in theory. For example, the costly problem encoding is evident in \cite{Nayak_Winker_Groppe_Groppe_2024} where the number of binary variables grows exponentially in terms of relations in a database.

Additionally, the previous quantum computing formulations for the join order selection problem did not benefit from the query graph's structure, which we encode in the optimization model. By using information from the query graphs and assuming that the cross products are expensive, we can decrease the size of the optimization problems. The other critical scalability finding lies in the selection of binary variables. With a clever choice of binary variables, we can compute the cost precisely and reduce the number of variables and their types. The previous research \cite{Schonberger_Scherzinger_Mauerer, 10.14778/3632093.3632112} has used four variable types (variables for relations, joins, predicates, and cost approximation). We decrease this number to one variable type, which works in most cases except for clique graphs, which require two types.

Quantum computing research for database applications has not provided many theoretical results about the performance of their methods. In this work, we prove two bounds for our methods, which connect the quantum algorithms to the classical ones. This is important because it helps us comprehend the capabilities of the current quantum computation solutions compared to the established classical methods.

% Here we summarize the improvements over the previous quantum and quantum-inspired works \cite{Schonberger_Scherzinger_Mauerer, 10.14778/3632093.3632112,Nayak_Winker_Groppe_Groppe_2024,Schonberger_Trummer_Mauerer_2023}:
% \begin{itemize}
%     \item Our models have a theoretical guarantee for optimal results up to the cross products.
%     \item Our models require fewer variables and variable types.
%     \item Our methods do not approximate the join order cost function, but the cost is computed precisely, increasing accuracy and removing the need for tuning the cost estimation thresholds \cite{Schonberger_Scherzinger_Mauerer,10.14778/3632093.3632112}.
%     \item Our models are not a quantum adaption of a classical algorithm (as some of the previous quantum solutions are adaptions of MILP formulation \cite{Trummer_Koch_2017}) but the first higher-order unconstrained binary model to solve the problem.
%     \item In the experimental evaluation, we show that the cumulative error across varying query graphs remains low, indicating consistent performance across varying queries.
% \end{itemize}

The key contributions are as follows:
\begin{enumerate}
    \item We develop three novel higher-order unconstrained binary optimization algorithms to solve the join order selection problem on universal quantum computers and quantum annealers.
    \item We provide theoretical bounds that characterize the accuracy of our algorithms.
    \item We perform a comprehensive experimental evaluation with varying quantum and classical solvers demonstrating the proposed algorithms' practical usability within the limits of the current hardware.
\end{enumerate}

The structure of the paper is as follows. First, we formally state the join order selection problem, discuss the quadratic and higher-order binary optimization models, and define their connection to quantum computing. Then, we present the main algorithms. We prove the theoretical bounds for the accuracy of the methods. We summarize the results from the experimental evaluation and discuss how our contributions relate to the previous research solving the join order selection problem with quantum computing. On GitHub \cite{anonymous2024qjoin} and in the experimental results, we refer to the implementation of our algorithms as Q-Join.