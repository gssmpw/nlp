\section{Appendix: HUBO to QUBO reduction}

Next, we describe how to reduce HUBO problems to QUBO problems. This rewrite process is necessary to utilize a wider variety of optimization platforms. In this work, we mainly rely on the D-wave's Ocean framework's utility of automatically translating HUBO problems into QUBO problems. The translation is based on a polynomial reduction by minimum selection or by substitution \cite{polynomial_reductions}.

The HUBO to QUBO reduction based on minimum selection \cite{polynomial_reductions} follows the scheme
\begin{equation*}
    xyz = \max_{w}w(x+y+z-2),
\end{equation*}
which iteratively replaces the higher order terms $xyz$ with lower order terms by introducing auxiliary binary variables $w$. Depending on the order in which variable terms are replaced, the QUBO formulation may vary in format, and the number of binary variables, but the minimum point remains unchanged.

Rewriting mechanism by substitution utilizes the following formula
\begin{equation*}
    xyz = \min_{w}\left\{wz + \mathrm{MP}(x, y ; w)\right\},
\end{equation*}
where $M > 1$  is a penalty weight and $P$ is a penalty function that has the lowest value when $w = xy$. The details of why these rewriting methods reach the same minimum are explained in \cite{polynomial_reductions}. These reduction methods provide a technically easy method to encode HUBOs as QUBOs, but they also introduce auxiliary variables depending on the number of higher-order terms.