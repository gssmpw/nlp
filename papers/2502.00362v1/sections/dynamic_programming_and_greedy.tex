\section{Dynamic programming and greedy algorithms}

The dynamic programming and greedy algorithms implemented here are based on \cite{Neumann_course}. Dynamic programming for join order selection comprises a general class of approaches to optimize the join order selection. In our work, we have fixed the cost function (Eq.~\eqref{eq:cost_function}) and employed the dynamic programming algorithm with and without cross-products. The algorithm without cross-products is presented in Alg.~\ref{alg:dp-join-order}. It relies on functions that create left-deep trees for trees $T_1$ and $T_2$ and return costs for join trees based on the cost function in Eq.~\eqref{eq:cost_function}.

\begin{algorithm}
\caption{Dynamic Programming for Join Order Optimization With Cross-Products}
\label{alg:dp-join-order}
\begin{algorithmic}[1]
\Require relations $R = \{r_1, r_2, \dots, r_n\}$, selectivities $S$
\Ensure optimal left-deep join tree in $\text{dp\_table}[R]$
\State initialize $\text{dp\_table}$
\For{$r \in R$}
    \State $\text{dp\_table}[\{r\}] \gets r$ \Comment{Base case: single relation}
\EndFor
\For{$s = 2 \ \mathbf{ to } \ |R|$} \Comment{Size of subsets from 2 to $n$}
    \For{$\text{subset} \subseteq R$ such that $|\text{subset}| = s-1$}
        \If{$\text{subset} \in \text{dp\_table}$}
            \For{$r \in R \setminus \text{subset}$} \Comment{Relations not in subset}
                \State $T_1 \gets \text{dp\_table}[\text{subset}]$
                \State $T_2 \gets \text{dp\_table}[\{r\}]$
                \State $\text{T}, \text{T\_cost} \gets \text{create\_join\_tree}(T_1, T_2, R, S)$
                \State $\text{join\_key} \gets \text{subset} \cup \{r\}$
                \If{$\text{join\_key} \notin \text{dp\_table}$}
                    \State $\text{dp\_table}[\text{join\_key}] \gets T$ \Comment{Update if key not in table}
                \EndIf
                \If{$\text{join\_key} \in \text{dp\_table}$}
                    \State $\text{prev\_cost} \gets \textbf{cost}(\text{dp\_table}[\text{join\_key}], R, S)$
                    \If{$\text{T\_cost} < \text{prev\_cost}$}
                        \State $\text{dp\_table}[\text{join\_key}] \gets T$ \Comment{Update if lower cost}
                    \EndIf
                \EndIf
            \EndFor
        \EndIf
    \EndFor
\EndFor
\end{algorithmic}
\end{algorithm}

The algorithm that computes the dynamic programming result without cross-products is similar except that for a query graph $G$, we change line 6: \textbf{for} connected subgraph $\subset G$ such that $|\text{subgraph}| = s - 1$ \textbf{do}. Then, the algorithm proceeds with the connected subgraphs of size $s - 1$ instead of all subsets of size $s - 1$.

The greedy algorithm is the other standard algorithm to optimize join order selection, and we represent it in Alg.~\ref{alg:greedy-join-order}. Similarly to the dynamic programming algorithm, we can consider only solutions without cross-products so that we iterate only over tables connected to one of the tables already joined. In other words, at each step, we compute a value called adjacent\_tables which contains those tables $R_i$ so that if edge $(R_i, R_j) \in G$ in the query graph $G$, then we require that $R_i \notin \text{joined\_tables}$ but $R_j \in \text{joined\_tables}$. 

\begin{algorithm}
\caption{Greedy Algorithm for Join Order Selection}
\label{alg:greedy-join-order}
\begin{algorithmic}[1]
\Require relations $R = \left\{ r_1, \ldots, r_n \right\}$, selectivities $S$
\Ensure Greedy join order tree

\State $\text{join\_result} \gets [r_i, r_j]$ so that $f_{i,j}|r_j||r_j|$ is the smallest

\For{$1 \ \text{to} \ |R| - 1$}
    \State $\text{min\_cost} \gets \infty$
    \State $\text{min\_table} \gets \text{None}$
    \For{$\text{table} \in \text{relations} \setminus \text{joined\_result}$} \Comment{Iterate over tables which are not joined}
        \State $\text{current\_join\_tree} \gets [\text{table}, \text{join\_result}]$
        \State $\text{current\_cost} \gets \textbf{cost}(\text{current\_join\_tree}, R, S)$
        \If{$\text{current\_cost} < \text{min\_cost}$} \Comment{Choose table with smallest cost}
            \State $\text{min\_cost} \gets \text{current\_cost}$
            \State $\text{min\_table} \gets \text{table}$
        \EndIf
    \EndFor
    \State $\text{join\_result} \gets [\text{min\_table}, \text{join\_result}]$
\EndFor
\end{algorithmic}
\end{algorithm}