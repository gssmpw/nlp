\section{Join order cost as HUBO}

In this section, we develop two higher-order unconstrained binary optimization (HUBO) problems that encode the cost function in Eq.~\eqref{eq:cost_function} for a left-deep join order selection problem. Compared to the previous quantum computing for join order optimization research, we formulate the optimization problem from the perspective of \textit{joins} instead of the perspective of \textit{relations} \cite{Schonberger_Scherzinger_Mauerer}. Given a query graph $G$, the number of required joins to create a valid left-deep join tree is $|V| - 1$, where $|V|$ is the number of nodes (i.e., relations, tables) in query graph $G$. This is easy to see since the first join is performed between two tables, and after that, every join includes one more table until all the tables have been joined.

Our algorithm is designed to rank joins, and ranking gives the order for the joins. This means a join (i.e., edge in the query graph $G$) has a rank $0 \leq k < |V| - 1$ if the join should be performed after all the lower rank joins are performed. We need $|V| - 1$ rank values to create a left-deep join order plan. Having $|V| - 1$ rank values applies to left-deep join plans but not bushy ones. For bushy plans, we can join multiple tables simultaneously, meaning that some of the joins can have the same rank, i.e., appear at the same level in the join tree.

Initially, any join $(R_i, R_j) \in G$ can have any rank $0 \leq r < |V| - 1$. We define the binary variables of our HUBO problems to be
\begin{equation}\label{def:binary_variables}
    x_{i,j}^{r} \in \left\{0, 1 \right\},
\end{equation}
where the indices $i$ and $j$ refer to the relations $R_i$ and $R_j$ and $r$ denotes the rank. Hence, our model consists of $(|V|- 1)|E|$ binary variables since for every rank value $0 \leq r < |V| - 1$, we have $|E|$ many joins (edges) from which we can choose the join.

The interpretation of these binary variables is as follows: If $x_{i,j}^{r} = 1$, then the join $(R_i, R_j)$ should be performed at rank $r$. Now the join $(R_i, R_j)$ is not necessarily between the tables $R_i$ and $R_j$ since at $r > 0$ the left relation is an intermediate result of type $R_{k_1} \bowtie \ldots \bowtie R_i \bowtie \ldots \bowtie R_{k_n}$ for some indices $k_1, \ldots, k_n$. Thus, the tuples $(R_i, R_j)$ represent joins in the query graph rather than materialized joins in query processing.

\begin{example}
Consider that we have a simple, complete query graph of four relations $\left\{0,1,2,3\right\}$. Thus, we have $|V| = 4$ relations, $|E| = 6$ possible joins and $(|V| - 1)|E| = 18$ binary variables. Depending on the selectivities and cardinalities, an example solution that the model can return is $x_{0,1}^{0} = 1$, $x_{1,2}^{1} = 1$, and $x_{2,3}^{2} = 1$, which gives us left-deep join tree $[[0, 1], 2], 3]$. The solution is not unique; also $x_{0, 1}^{0} = 1$, $x_{0, 2}^{1} = 1$, and $x_{2, 3}^{2} = 1$ produces the same plan with the same cost.
\end{example}

\subsection{Precise cost function as HUBO}\label{subsection:cost_function}

After defining the binary variables of our optimization model, we describe how we encode the cost function as a higher-order unconstrained binary optimization problem whose minimum is the optimal cost up to cross products for the left-deep join order selection problem. We describe the cost constraint first since the validity constraints can be computed based on terms that we compute for the cost constraint. First, we demonstrate the intuition behind the construction with an example.

\begin{example}\label{ex:example2}
Every join should be performed at exactly one rank for left-deep join trees. Starting from rank $0$, let us say that we choose to perform a join between the relations $R_1$ and $R_2$ and obtain $R_1 \bowtie R_2$. The corresponding activated binary variable is $x_{1,2}^{0} = 1$. Based on Def.~\eqref{eq:cost_function} of the cost function, the cost of performing this join is $f_{1,2}|R_1||R_2|$. Thus, if we decide to make this join at this rank, we include the term
\begin{equation*}
    f_{1,2}|R_1||R_2|x_{1,2}^{0}
\end{equation*}
to the cost HUBO. This example demonstrates that it is easy to encode the costs at rank $0$, which correspond to linear variables in the cost HUBO.

Next, we assume the query graph gives us a join predicate with selectivity $f_{2,3}$ between the tables $R_2$ and $R_3$. Now we ask how expensive it is to perform the join between intermediate result $R_1 \bowtie R_2$ and relation $R_3$. By Def.~\eqref{eq:cost_function}, the cost of making this join is
\begin{equation}\label{eq:card_123}
    f_{1,2}f_{2,3}|R_1||R_2||R_3|
\end{equation}
assuming that there is no edge $(R_1, R_3)$ which indicates that $f_{1,3} = 1$. Note that this is not the total cost of performing all the joins but the cardinality of the resulting table $R_1 \bowtie R_2 \bowtie R_3$. The left-deep join tree $(R_1 \bowtie R_2) \bowtie R_3$ should be selected if the following total cost function evaluates to a relatively small value
\begin{equation*}
    f_{1,2}|R_1||R_2|x_{1,2}^{0} + f_{1,2}f_{2,3}|R_1||R_2||R_3|x_{1,2}^{0}x_{2,3}^{1}.
\end{equation*}
When the binary variables are active, i.e., $x_{1,2}^{0} = x_{2,3}^{1} = 1$, the previous function evaluates the total cost of performing the join $(R_1 \bowtie R_2) \bowtie R_3$.

One of the key ideas is that the cardinality in Eq. \eqref{eq:card_123} does not depend on the join order but only on the tables that are part of the join at that point. In other words, this means that the cardinality in Eq. \eqref{eq:card_123} is the same for any join result that includes the relations $R_1$, $R_2$ and $R_3$ such as $(R_1 \bowtie R_3) \bowtie R_2$ and $R_1 \bowtie (R_2 \bowtie R_3)$. This naturally generalizes to any number of relations. The total costs of these plans likely differ because intermediate steps have different costs. Intuitively, our HUBO model seeks the optimal configuration of joins to construct the full join tree so that the sum of the intermediate results is the smallest.
\end{example}

Next, we formally describe constructing the HUBO problem that encodes the precise cost function for a complete left-deep join order selection problem respecting the structure of a given query graph $G$. The HUBO problem is constructed recursively with respect to the rank $r$. The construction of the HUBO problem becomes recursive because the definition of the cost function \eqref{eq:cost_function} is recursive.

\textbf{Step $r = 0$.} Let $G = (V, E)$ be a query graph. Based on Def.~\eqref{eq:cost_function}, we include the costs of making the rank $0$ joins to the cost HUBO. Precisely, we add terms
\begin{equation*}
    |R_i \bowtie R_j|x_{i,j}^{0} = f_{ij}|R_i||R_j|x_{i,j}^{0} = \alpha_{(i,j)}x_{i,j}^{0},
\end{equation*}
for every join $(R_i, R_j) \in E$, where we denote $\alpha_{(i,j)} := f_{ij}|R_i||R_j|$ the coefficient.

\textbf{Step $r = 1$.} For clarity, we also show step $r = 1$. Assuming we have completed step $r = 0$, we consider adding variables of type $x_{i,j}^{1}$. For every join $(R_i, R_j) \in E$, we select the adjacent joins $(R_{i'}, R_{j'})$ in the query graph. An adjacent join means that the joins share exactly one common table. This creates quadratic terms of type $x_{i,j}^{0}x_{i',j'}^{1}$ with coefficients of type $$\alpha^{1}_{(i,j,i',j')} = f_{ij}f_{i'j'}f_{i'j}f_{ij'}|R_i||R_j||R_i'||R_j'|.$$ So, we add terms $\alpha_{(i,j,i',j')}x_{i,j}^{0}x_{i',j'}^{1}$ to the cost HUBO.

\textbf{Step for arbitrary $r$.} Next, we consider adding a general rank $0 < r < |V| - 1$ variables of form $x_{i', j'}^{r}$ to the HUBO problem. For the general case, we formalize the method using connected subgraphs in the query graph: Let $\mathcal{S}$ be the set of size $r-1$ connected subgraphs in the query graph so that every subgraph corresponds to terms that were generated at step $r-1$. Since the cost function is defined recursively, this is an alternative way to express that we have added the variables to rank $r - 1$. This means that the HUBO problem, encoding the total cost up to this step, has the form:
\begin{align*}
    \underbrace{\sum_{(i, j) \in E}\alpha_{(i,j)} x_{i,j}^{0}}_{\text{case } r = 0} + \underbrace{\sum_{(i, j) \in E}\sum_{(i', j') \in E} \alpha_{(i,j,i',j')} x_{i,j}^{0}x_{i', j'}^{1}}_{\text{case } r = 1} + \ldots  
    + \underbrace{\sum_{S \in \mathcal{S}}\alpha_{S}\prod_{(R_i, R_j) \in S, 0\leq k \leq r - 1} x_{i,j}^{k}}_{\text{case } r - 1}.
\end{align*}
For simplicity, we first focus on generating the next term without a coefficient. Let $S \in \mathcal{S}$ be a fixed subgraph of the query graph. Let $(R_{i'}, R_{j'}) \in E$ be an edge that is not part of the subgraph $S$ but connected to it so that either $R_{i'} \in S$ or $R_{j'} \in S$ (but not both $R_{i'}, R_{j'} \in S$). This means we join exactly one new table. For this fixed subgraph $S$ and fixed join $(R_{i'}, R_{j'})$, we are going to add the following element to the cost HUBO:
\begin{equation}\label{eq:new_term}
\underbrace{\prod_{(R_i, R_j) \in S, 0\leq k \leq r - 1} x_{i,j}^{k}}_{\text{term at rank } r - 1} \underbrace{x_{i',j'}^{r}}_{\text{new variable at rank } r}
\end{equation}
The new term is just the ''old'' term from the previous step multiplied by the new variable $x_{i',j'}^{r}$.

The new coefficient is easy to compute based on the subgraph $S$ and the latest included join $(R_{i'}, R_{j'})$. Precisely, consider the new induced subgraph $S' = S \cup \left\{(R_{i'}, R_{j'})\right\}$. Note that the induced subgraph $S'$ may contain new edges besides the latest included edge $(R_{i'}, R_{j'})$. However, the new coefficient for term \eqref{eq:new_term} is
\begin{equation}\label{eq:term_coefficient}
\alpha_{S'} = \prod_{(R_i, R_j) \in S'}f_{i,j}\prod_{R_i \in S'}|R_i|.
\end{equation}
This formula is just the general expression of Eq.~\eqref{eq:card_123} in Example \ref{ex:example2}. The previous construction is repeated for each subgraph $S \in \mathcal{S}$ and each edge adjacent to subgraph $S$ sharing one common vertex (i.e., a table). The idea of how the terms are appended at rank $2$ is visualized in Fig.~\ref{fig:subgraph_generation}. One of the final configurations is visualized in Fig.~\ref{fig:subgraph_to_terms}, which also shows how the terms are interpreted as join trees.

\begin{figure}
\centering
\begin{subfigure}[t]{.49\linewidth}
  \centering
  \includegraphics[width = \textwidth]{figures/subgraph_generation.png}
  \caption{With relations 0, 1, and 2 already joined, relation 3 is the only option for left-deep plans. Two combinations arise based on whether edge $(0,2)$ or $(1,2)$ joins first. Coloring links graph variables, generating terms in HUBO.}
  \label{fig:subgraph_generation}
  \Description[subgraph_generation]{Assuming we have considered the relations 0, 1, and 2 at earlier ranks, then the only possible relation we can join in left-deep plans is relation 3. Two combinations are included depending on whether edge $(0,2)$ or edge $(1,2)$ is joined first. Coloring highlights the connection between variables in the graph and produces terms in HUBO.}
\end{subfigure}\hfill
\begin{subfigure}[t]{.49\linewidth}
  \centering
    \includegraphics[width = \textwidth]{figures/subgraph_to_terms.png}
    \caption{The figure presents one option as the final assignment for the binary variables in the query graph. The algorithm returns the corresponding join tree if the variables are selected true.}
    \label{fig:subgraph_to_terms}
    \Description[subgraph_to_terms]{The figure presents one option as the final assignment for the binary variables in the query graph. The algorithm returns the corresponding join tree if the variables are selected true.}
\end{subfigure}
\caption{Two examples for variable generation in tree query graph}
\label{fig:test}
\end{figure}

The algorithm to generate the cost HUBO is presented in Alg.~\ref{alg:hubo_term_construction}. It inputs a query graph and outputs a dictionary called HUBO, which stores the terms of the cost HUBO problem. The keys in this dictionary are sets of tables, and the values are sets of terms. We exclude the coefficient computation for simplicity since the coefficients can be efficiently computed with Eq.~\eqref{eq:term_coefficient}.

\begin{algorithm}
\caption{Construct Terms for Precise Cost Function HUBO}
\label{alg:hubo_term_construction}
\begin{algorithmic}[1]
\Require query\_graph
\Ensure terms for cost HUBO
\For{\textbf{each} $r$ \textbf{in} $0, \ldots, |V| - 2$}
    \For{edge \textbf{in} query\_graph}
        \State $R_i$, $R_j \gets$ edge[0], edge[1]
        \If{$r$ $=$ 0}
            \State HUBO[$\{R_i, R_j\}$] $\gets$ $x_{ij}^{0}$
        \Else
            \State joined\_tables $\gets$ [tables if $r - 1$ var in HUBO[tables]]
            \For{joined \textbf{in} joined\_tables}
                \State cond1 = $R_i \in$ joined and $R_j \notin$ joined
                \State cond2 = $R_j \in$ joined and $R_i \notin$ joined
                \If{cond1 \textbf{or} cond2}
                    \State new\_joined $\gets$ joined $\cup$ $\{R_i, R_j\}$
                    \State old\_terms $\gets$ HUBO[joined]
                    \State new\_terms $\gets$ multiply old\_terms with $x_{ij}^{r}$
                    \State HUBO[new\_joined] $\gets$ new\_terms
                \EndIf
            \EndFor
        \EndIf
    \EndFor
\EndFor
\end{algorithmic}
\end{algorithm}

\newcommand{\mcL}{\mathcal{L}}
\newcommand{\mcV}{\mathcal{V}}

\section{Algorithm Implementation} \label{sec:impl}
To conduct our experiments, we provide the heuristic algorithm that we designed for the $k'$-colorful nearest neighbor problem, based on the provable algorithms provided in the main paper. The provable indexing algorithm (\ref{alg:indexing}) has a runtime which is quadratic in the size of the data set and is slow in practice. This situation mimics the original DiskANN algorithm~\cite{jayaram2019diskann}, where the ``slow preprocessing" algorithm  has provable guarantees~\cite{indykxu2024worst} but quadratic running time, and was replaced by a heuristic ``fast preprocessing'' algorithm used in the actual implementation~\cite{DiskANN}. 
%As the aforementioned algorithms are designed to tackle the most general problem and =. Here 
Here, Algorithm~\ref{alg:colorindex} offers a fast method tailored for the $k'$-colorful case, using several heuristics to improve the runtime. In the following section, we present the pseudocode for the procedures: search, index build, and the pruning procedure required for the index build.

%We compare the query performance and accuracy of our algorithm with several baselines, on both real-world shopping data sets and  other data sets where the seller information is synthetically generated. In the remainder of the section, we provide details on our fast implementation of our algorithm, details regarding the datasets and baselines and the results.

%Graph-based similarity search indices are constructed so that greedy search quickly converges to the nearest neighbors  of a query vector $q$. We first describe a natural adaptation of greedy search to account for $k'$-color NN problem and an index construction procedure that allows search to converge to the right answer with relatively few distance comparisons.

\paragraph{Diverse Search.}
Our diverse search procedure, is a greedy graph-based local search method. In our search method, in each step, we maintain a list of best and diverse nodes, ensuring that at most $k'$ points are selected in the list per color. In each iteration of our search algorithm, we choose the best unexplored node and examine its out neighbors. From the union of our current list and the out neighbors, we select the best diverse set of nodes while satisfying the \(k'\)-colorful diversity constraint—meaning no color can have more than \(k'\) points in the updated list. To identify the optimal diverse set from the union, we use a priority queue designed to accommodate the diversity constraint. Below, we present the pseudocode for this diverse priority queue.
\begin{algorithm}
\caption{Insert $(p,d,c)$ into \diversequeue(Q, $L$, $k'$)}
\label{alg:diverse_queue}
\begin{algorithmic}[1]
\STATE \textbf{Input}: Current queue $Q$, tuple $(p,d,c)$ of (point, distance, color) for new insertion, maximum size $L$ of the queue, maximum size $k'$ per color. 
\STATE \textbf{Output}: Updated queue $Q$ after inserting $(p,d,c)$ which maintains the best set of at most $L$ points and at most $k'$ points of each color.
\STATE Let ${\sf count}({{c}}) \gets$ number of elements in $Q$ with the  color $c$.
\STATE Let ${\sf maxDist}({{c}}) \gets$ maximum distance of an element in $Q$ with color $c$.
\IF{${\sf count}({{c}}) < k'$ \textbf{or} $d < {\sf maxDist}({{c}})$}
    \STATE Insert $(p, d, c)$ into $Q$
    \IF{${\sf count}({{c}}) > k'$}
        \STATE Remove the element with the maximum distance in $Q$ having color $c$.
    \ENDIF
\ENDIF
\IF{$|Q| > L$}
    \STATE Remove the element with the maximum distance in $Q$.
\ENDIF
\end{algorithmic}
\end{algorithm}

Building on the previous explanation of the diverse priority queue, we outline the description of our diverse search procedure as follows.

\begin{algorithm}
\caption{$\diversesearch(G,s,q,k',k,L)$}
\label{alg:color_search}
\begin{algorithmic}[1]
\STATE \textbf{Input}: A directed graph $G$, start node $s$, query $q$, max per color parameter $k'$, search list size $L$.
\STATE \textbf{Output}: A set of $k$ points such that there are at most $k'$ points from any color.
\STATE {Initialize} \diversequeue $\mathcal{L} \leftarrow \{(s,D(s,q),col[s]) \}$ with color parameter $k'$ and size parameter $L$.  
\STATE {Initialize} a set of expanded nodes $\mathcal{V} \leftarrow \emptyset$\;
\WHILE{$\mathcal{L} \setminus \mathcal{V} \neq \emptyset$}
        \STATE $\text{Let } p^{*} \leftarrow \argmin\limits_{p \in \mathcal{L} \setminus \mathcal{V}} D(p,q)$\;
        \STATE $\mathcal{V} \leftarrow \mathcal{V} \cup \{p^{*}\}$\;
        % \State $\mathcal{L} \leftarrow \mathcal{L} \cup \{ \left(p, D(p,q) , col[p] \right) \, : \, p \in N_{\text{out}}(p^*)\}$\;
        \STATE Insert $\{\left(p, D(p,q), col[p]\right):p \in N_{\text{out}}(p^*)\}$ to $\mathcal{L}$\;
        % \If{$F_q \cap F_{p^{*}} \neq \emptyset$}{
        %     \State \hspace{0.02cm} $\mathcal{L} \leftarrow \mathcal{L} \cup N_{\text{out}}(p^{*})$\;
        % }
%        \State Pick at most $L$ best diverse candidate out of $\mathcal{L}$. \knote{Add more description here}
\ENDWHILE
\STATE \textbf{Return} $[\text{top } k \text{ NNs from } \mathcal{L}; \mathcal{V}]$\;
\end{algorithmic}
\end{algorithm}



\paragraph{Diverse Prune.} A key subroutine in our index-building algorithm is the prune procedure. Given a node $p$ and a set of potential outgoing edges $\mathcal{V}$, the standard prune procedure removes an edge to a vertex \(w\) if there exists a vertex \(u\) such that an edge \(p \rightarrow u\) exists and the condition \(D(u, w) \leq \frac{D(p, w)}{\alpha}\) is satisfied. Intuitively, this means that to reach \(w\), we would first reach \(u\), thus making multiplicative progress and eliminating the need for the edge \(p \rightarrow w\), which contributes to the sparsity of the graph.

However, to account for diversity, the outgoing edges from the node must also be diverse and enable access to multiple colors. To address this requirement, we modify the standard prune procedure to incorporate the diversity constraint. The details of our revised algorithm are provided next.

\begin{algorithm}[h]
\caption{$\diverseprune(p, \mcV,\alpha,R,m)$}
\label{alg:colorprune}
\begin{algorithmic}[1]
\small
\STATE \textbf{Input}: A point $p$, set $\mcV$, prune parameter $\alpha$, degree parameter $R$, and diversity parameter $m$.
\STATE \textbf{Output}: A subset $\mcV' \subseteq \mcV$ of cardinality at most $R$ to which edges are added.
\STATE Sort all points $u\in \mcV$ based on their distances from $p$ and add them to list $\mcL$ in that order.
\STATE Initialize sets $\blockers[u] \gets \emptyset$ for each $u \in \mcV$.
    \WHILE{$\mcL$ is not empty}
        \STATE $u\gets \argmin\limits_{u\in \mcL}D(u,p)$
        \STATE $\mcV' \gets \mcV' \cup \{u\}$ and $\mcL \gets \mcL \setminus \{u\}$
        \IF{$|\mcV'| = R$}
             \STATE {\bf break}
        \ENDIF
        \FOR{each point $w\in \mcL$}
            \IF{$D(u,w)\le D(p,w)/\alpha$}
                \STATE $\blockers[w]\gets \blockers[w]\cup \{{ col}(u)\}$
                \IF{$|\blockers[w]| = m$ or ${col}(u) = {col}(w)$}
                    \STATE {$\mcL \gets \mcL \setminus \{w\}$}
                \ENDIF
            \ENDIF
        \ENDFOR
    \ENDWHILE
\STATE{\bf Return } $\mcV'$
\end{algorithmic}
\end{algorithm}

\paragraph{Diverse Index.} Our indexing algorithm follows the same approach as the DiskANN ``fast preprocessing" heuristic implementation~\cite{DiskANN}, but we replace the search and prune procedures in their implementation with our diverse search and diverse prune procedures. The details of our index-building procedure are provided below.

\begin{algorithm}[h]
\caption{$\diverseindex(P,\alpha,L,R,m)$}
\label{alg:colorindex}
\begin{algorithmic}[1]
\small
\STATE \textbf{Input}: A set of $n$ points $P=\{p_1,\dots,p_{n} \}$, prune parameter $\alpha$, search list size $L$, degree parameter $R$, and  diversity parameter $m$.
\STATE \textbf{Output}: A directed graph $G$ over $P$ with out-degree at most $ R$.
    \STATE Let $s$ denote the estimated medoid of $P$.\;
    \STATE Initialize $G$ with start node $s$.\;
     
    \FOR{each $p_i\in P$}
        \STATE $\text{Let } [\mathcal{L}; \mathcal{V}] \leftarrow \diversesearch\left(G,s, p_{i}, k' = L/m, L, L \right)$ \; 
        \STATE Let $\mcV' = \diverseprune\left(p_i, \mathcal{V}, \alpha, R, m\right)$.
        \STATE Add node $p_i$ to $G$ and set $N_{\text{out}}(p_{i})= \mcV'$ (out-going edges from $p_i$ to $\mcV'$).\;
        \FOR{$p \in N_{out}(p_i)$}
            \STATE $\text{Update } N_{\text{out}}(p) \leftarrow N_{\text{out}}(p) \cup \{p_i\}$.\;
            \IF{$|N_{\text{out}}(p)| > R$}
                \STATE \textbf{Run} \diverseprune($p, N_{\text{out}}(p), \alpha, R,m$) to update out-neighbors of  $p$.\;
                \ENDIF
\ENDFOR
\ENDFOR
\end{algorithmic}
\end{algorithm}
