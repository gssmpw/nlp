\section{Join order validity as HUBO}

HUBO problems, like QUBO problems, are required to return valid solutions. In this case, validity means that the join tree adheres to the query graph. All valid solutions are usable, although they might not minimize the cost. In this section, we present two approaches to encoding the validity of solutions: cost function dependent and independent. A cost-function-dependent approach is easy to construct but produces a larger number of higher-order terms. The cost-function-independent approach is closer to the standard QUBO formulations and identifies a collection of constraints the formulation needs to satisfy. The advantage of the second formulation is that the terms are primarily quadratic, which is easier to optimize in practice.

\subsection{Cost-function dependent validity}\label{subsubsection:cost_function_dependent}

Considering the cost function generation in the previous subsections, we have generated higher-order terms that encode valid join trees at rank $r = |V| - 2$. Considering the Alg.~\ref{alg:hubo_term_construction}, we can access these terms with $\texttt{HUBO[set of all relations]}$. Let us denote this set of terms as $H$. For example, one of those terms is represented in Fig.~\ref{fig:subgraph_generation}. The first validity constraint forces the model to select exactly one of these terms as true, which requires the final join tree to contain all the relations.

\textbf{Select one valid plan.} We utilize a generalized one-hot (or $k$-hot) encoding from QUBO formulations \cite{lucas_2014, Schonberger_Trummer_Mauerer_2023}. The encoding constructs a constraint that reaches its minimum when precisely $k$ variables are selected to be true from a given set of binary variables. In the generalized formulation, we construct an objective minimized when exactly $k$ terms are selected true from a set of higher-order terms. Let $H$ be the set of higher-order terms at rank $r = |V| - 2$. This functionality generalizes to higher-order cases with the following formulation:
\begin{equation}\label{eq:hubo_combinations}
    H_0 = \left( 1 - \sum_{h \in H} h \right)^2,
\end{equation}
where $h = \prod_{r = 0}^{|V| - 2} x_{i_r, j_r}^{r}$ for some indices $i_r$ and $j_r$ for $r = 0, \ldots, |V| - 2$ that depend on the query graph. The objective $H_0$ is always non-negative since it is squared. It is positive except when exactly one of the terms in the sum $\sum_{h \in H} h$ is $1$ when $H_0$ evaluates to $0$. The sum $\sum_{h \in H} h$ evaluates to $1$ when there is exactly one term $h$ such that all the variables in the product $\prod_{r = 0}^{|V| - 2} x_{i_r, j_r}^{r}$ are true. This combination is the valid join tree that the objective function returns as a solution to the minimization problem. Since these terms $h \in H$ have already been generated for the cost HUBO, we create this constraint based on them. Thus, we call this validity constraint cost-dependent.

\textbf{Every rank must appear exactly once in the solution.}
It is still possible to obtain solutions that include unnecessary true variables that do not affect the minimum of the final HUBO function. For example, consider the plan in Fig.~\ref{fig:subgraph_generation}. The HUBO that encodes this plan would have the same minimum even if we set that $x_{0,2}^{1} = 1$ because this variable would always be multiplied with variables set to $0$, and thus, activating a single variable would not affect the total cost. While we can solve this problem with classical post-processing, we still decided to fix it in the model itself. We include a constraint that every rank should appear exactly once in the solution. This is also an instance of one-hot encoding \cite{dimod_generators_combinations} and encoded with the following quadratic objective:
\begin{equation}\label{eq:every_rank_has_one_join}
    H_1 = \sum_{r = 0}^{|V| - 2}\left( 1 - \sum_{(R_i, R_j) \in E} x_{i,j}^{r} \right)^2.
\end{equation}
The objective $H_1$ is minimized when exactly one variable of type $x_{i,j}^{r}$ is selected to be true for each $0 \leq r <|V|- 1$.
\subsection{Cost-function independent validity}\label{subsubsection:cost_function_independent}

Although validity constraint \eqref{eq:hubo_combinations} is theoretically correct, it produces higher-degree terms due to exponentiation to the power of two, which we might want to avoid. Hence, we develop join tree validity constraints independently from the cost function. Notably, these validity constraints are often quadratic, i.e., QUBOs, and automatically supported by many solvers. Because we develop the theory considering the query graph's structure, we have slightly different constraints depending on the query graphs.

\subsubsection*{Clique graphs}
\textbf{Every rank must appear exactly once in the solution.} The first constraint is what we presented in Eq.~\eqref{eq:every_rank_has_one_join}. This constraint encodes that we perform exactly one join at every rank.

\textbf{Select connected, left-deep join tree.} The second constraint encodes that we penalize cases that do not form a connected, left-deep join tree. While it is clear that join trees must be connected, we must also encode that they are left deep because the cost HUBO does not evaluate bushy trees correctly. To achieve this, we include a constraint of the form
\begin{equation}\label{eq:respect_clique_graph}
    \sum_{r = 0}^{|V| - 2}\sum_{(i,j) \in E}\sum_{(i',j') \in E}Cx_{i,j}^{r}x_{i',j'}^{r+1},
\end{equation}
where the terms are included if $i \neq i'$ and $j \neq j'$, or $i = i'$ and $j = j'$, which means that we penalize these cases by increasing the objective's value by $C$. This makes the model favor cases where consecutive joins share one ''old'' table and include exactly one ''new'' table in the result. The penalizing term $C$ should be set high enough, and we will define its value later.

\textbf{Result contains all the tables.} The third constraint forces the fact that we join all the tables. Our framework identifies joins as pairs of tables $(R_i, R_j)$. This leads to one table $R_i$ appearing multiple times in variables, referring to different joins in a valid solution. An example of this is presented in Fig.~\ref{fig:subgraph_to_terms}, where table $2$ appears multiple times in the solution consisting of joins such as $(0,2)$ and $(1,2)$. This third constraint encodes that we count the number of tables and require that the count is at least one for each table. Counting tables requires minimally a logarithmic number of slack variables (the method to do the logarithmic encoding is presented in \cite{lucas_2014}) in terms of tables. For technical simplicity, we present the less efficient but equivalent method here:
\begin{equation*}
    \sum_{R_i}\left( 1 + \sum_{k = 2}^{|V| - 2}ky^{i}_{k} - \sum_{i}\sum_{r = 1}^{|V| - 2} x_{i,j}^{r} \right)^2.
\end{equation*}
The constraint reaches 0 if at least one variable of type $x_{i,j}^{r}$ is true for each table $R_i$. The other accepted cases are that we activate any number of $k \in \left\{ 2, \ldots, |V| - 2 \right\}$ many variables of type $x_{i,j}^{r}$ using the slack variables $y^{i}_{k}$. This means we must select at least one table but possibly $k$ many tables.
\subsubsection*{Chain, star, cycle, and tree graphs}

\textbf{At every rank, we have performed rank + 1 many joins.}
This constraint is related to the first validity constraint presented in \cite{Schonberger_Scherzinger_Mauerer,10.14778/3632093.3632112}. They present the constraint in terms of tables, whereas we have constructed it in terms of joins. The constraint encodes how many joins we must perform cumulatively at each rank. In other words, when rank is $0$, we select one variable of type $x_{i,j}^{0}$ to be true. When rank is $1$, we choose two variables of type $x_{i,j}^{1}$ to be true. Formally, this constraint is
\begin{equation}\label{eq:at_every_rank_select_rank_many_joins}
    \sum_{r = 0}^{|V| - 2}\left( r + 1 - \sum_{(i,j) \in E} x_{i,j}^{r} \right)^2.
\end{equation}
The constraint is minimized at $0$ when for each $0 \leq r < |V| - 1$, we have selected exactly $r + 1$ variables of type $x_{i,j}^{r}$ to be true.

\textbf{Include the previous joins in the proceeding ranks.}
This constraint is again similar to the second constraint presented in \cite{Schonberger_Scherzinger_Mauerer,10.14778/3632093.3632112} except that we express the constraint using joins. If the join happened at rank $r$, it should be included in every proceeding rank $\geq r$. In other words, we keep the information of the performed joins to the following ranks. This can be achieved with the following constraint
\begin{equation}\label{eq:select_same_join_for_proceeding_ranks}
    \sum_{(i,j) \in E}\sum_{r = 1}^{|V| - 2} x_{i,j}^{r - 1}(1 - x_{i,j}^{r}).
\end{equation}
Now, if $x_{i,j}^{r - 1}$ is active, then the model favors the case that $x_{i,j}^{r}$ is active too since in that case $1 - x_{i,j}^{r} = 0$. If $x_{i,j}^{r} = 0$, the term evaluates to $1$, which is considered penalizing.

\textbf{Respect query graph: chain, star, cycle.}
Since the cost functions are designed to respect structures of query graphs, we use the following constraint to encode the graph structure in chain, star, and cycle graphs:
\begin{equation}\label{eq:respect_query_graph1}
    \sum_{r = 0}^{|V| - 1}\sum_{(i,j) \in E}\sum_{(i',j') \in E} -C x_{i,j}^{r}x_{i',j'}^{r},
\end{equation}
where $|\left\{ i, j \right\} \cap \left\{ i', j' \right\}| = 1$, which means that the joins have to share exactly one table. Setting the coefficient $-C$ as negative, we favor the cases when the joins share precisely one table. This constraint is complementary to constraint \eqref{eq:respect_clique_graph}.

\textbf{Respect query graph: tree.}
Unfortunately, the previous constraint \eqref{eq:respect_query_graph1} fails to encode the minimum for certain proper trees, which contain nodes with at least three different degrees. The simplest, problematic tree shape is represented in Figure \ref{fig:subgraph_to_terms} (node 2 has degree 3, node 3 has degree 2, and the others have degree 1). The problem is that with constraint \eqref{eq:respect_query_graph1}, not all the join trees have the same minimum energy due to nodes' different degrees. To address this problem, we develop an alternative constraint
\begin{equation*}
    \sum_{r = 1}^{|V| - 1} \left( r - \sum_{(i,j) \in E}\sum_{(i',j') \in E} x_{i,j}^{r}x_{i',j'}^{r}\right)^2,
\end{equation*}
where again we form the sum over the elements if the indices satisfy $|\left\{ i, j \right\} \cap \left\{ i', j' \right\}| = 1$. At every rank, this constraint selects $r$-many pairs of type $x_{i,j}^{r}x_{i',j'}^{r}$ to be true. This forces the returned join tree to respect the query graph and be connected. This constraint is slightly more complex than the previous constraints since it is not a quadratic but a higher-order constraint that includes terms of four variables. On the other hand, it does not introduce additional variables.


\textbf{Scaling cost and validity.} Finally, we have to scale cost HUBO $H_{\text{cost}}$ and validity objective $H_{\text{val}}$ properly so that we favor valid solutions over minimizing cost:
\begin{displaymath}
H_{\text{full}} = H_{\text{cost}} + CH_{\text{val}}.
\end{displaymath}
We noticed that a value that worked consistently in practice is $C = H_{\text{cost}}(x)$ where $x = (1, \ldots, 1)$ is a binary vector containing only ones and $H_{\text{cost}}$ is normalized so that the coefficients are in the interval $(0,1]$.