\subsubsection*{Clique graphs}
\textbf{Every rank must appear exactly once in the solution.} The first constraint is what we presented in Eq.~\eqref{eq:every_rank_has_one_join}. This constraint encodes that we perform exactly one join at every rank.

\textbf{Select connected, left-deep join tree.} The second constraint encodes that we penalize cases that do not form a connected, left-deep join tree. While it is clear that join trees must be connected, we must also encode that they are left deep because the cost HUBO does not evaluate bushy trees correctly. To achieve this, we include a constraint of the form
\begin{equation}\label{eq:respect_clique_graph}
    \sum_{r = 0}^{|V| - 2}\sum_{(i,j) \in E}\sum_{(i',j') \in E}Cx_{i,j}^{r}x_{i',j'}^{r+1},
\end{equation}
where the terms are included if $i \neq i'$ and $j \neq j'$, or $i = i'$ and $j = j'$, which means that we penalize these cases by increasing the objective's value by $C$. This makes the model favor cases where consecutive joins share one ''old'' table and include exactly one ''new'' table in the result. The penalizing term $C$ should be set high enough, and we will define its value later.

\textbf{Result contains all the tables.} The third constraint forces the fact that we join all the tables. Our framework identifies joins as pairs of tables $(R_i, R_j)$. This leads to one table $R_i$ appearing multiple times in variables, referring to different joins in a valid solution. An example of this is presented in Fig.~\ref{fig:subgraph_to_terms}, where table $2$ appears multiple times in the solution consisting of joins such as $(0,2)$ and $(1,2)$. This third constraint encodes that we count the number of tables and require that the count is at least one for each table. Counting tables requires minimally a logarithmic number of slack variables (the method to do the logarithmic encoding is presented in \cite{lucas_2014}) in terms of tables. For technical simplicity, we present the less efficient but equivalent method here:
\begin{equation*}
    \sum_{R_i}\left( 1 + \sum_{k = 2}^{|V| - 2}ky^{i}_{k} - \sum_{i}\sum_{r = 1}^{|V| - 2} x_{i,j}^{r} \right)^2.
\end{equation*}
The constraint reaches 0 if at least one variable of type $x_{i,j}^{r}$ is true for each table $R_i$. The other accepted cases are that we activate any number of $k \in \left\{ 2, \ldots, |V| - 2 \right\}$ many variables of type $x_{i,j}^{r}$ using the slack variables $y^{i}_{k}$. This means we must select at least one table but possibly $k$ many tables.