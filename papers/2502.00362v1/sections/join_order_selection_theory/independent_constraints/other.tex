\subsubsection*{Chain, star, cycle, and tree graphs}

\textbf{At every rank, we have performed rank + 1 many joins.}
This constraint is related to the first validity constraint presented in \cite{Schonberger_Scherzinger_Mauerer,10.14778/3632093.3632112}. They present the constraint in terms of tables, whereas we have constructed it in terms of joins. The constraint encodes how many joins we must perform cumulatively at each rank. In other words, when rank is $0$, we select one variable of type $x_{i,j}^{0}$ to be true. When rank is $1$, we choose two variables of type $x_{i,j}^{1}$ to be true. Formally, this constraint is
\begin{equation}\label{eq:at_every_rank_select_rank_many_joins}
    \sum_{r = 0}^{|V| - 2}\left( r + 1 - \sum_{(i,j) \in E} x_{i,j}^{r} \right)^2.
\end{equation}
The constraint is minimized at $0$ when for each $0 \leq r < |V| - 1$, we have selected exactly $r + 1$ variables of type $x_{i,j}^{r}$ to be true.

\textbf{Include the previous joins in the proceeding ranks.}
This constraint is again similar to the second constraint presented in \cite{Schonberger_Scherzinger_Mauerer,10.14778/3632093.3632112} except that we express the constraint using joins. If the join happened at rank $r$, it should be included in every proceeding rank $\geq r$. In other words, we keep the information of the performed joins to the following ranks. This can be achieved with the following constraint
\begin{equation}\label{eq:select_same_join_for_proceeding_ranks}
    \sum_{(i,j) \in E}\sum_{r = 1}^{|V| - 2} x_{i,j}^{r - 1}(1 - x_{i,j}^{r}).
\end{equation}
Now, if $x_{i,j}^{r - 1}$ is active, then the model favors the case that $x_{i,j}^{r}$ is active too since in that case $1 - x_{i,j}^{r} = 0$. If $x_{i,j}^{r} = 0$, the term evaluates to $1$, which is considered penalizing.

\textbf{Respect query graph: chain, star, cycle.}
Since the cost functions are designed to respect structures of query graphs, we use the following constraint to encode the graph structure in chain, star, and cycle graphs:
\begin{equation}\label{eq:respect_query_graph1}
    \sum_{r = 0}^{|V| - 1}\sum_{(i,j) \in E}\sum_{(i',j') \in E} -C x_{i,j}^{r}x_{i',j'}^{r},
\end{equation}
where $|\left\{ i, j \right\} \cap \left\{ i', j' \right\}| = 1$, which means that the joins have to share exactly one table. Setting the coefficient $-C$ as negative, we favor the cases when the joins share precisely one table. This constraint is complementary to constraint \eqref{eq:respect_clique_graph}.

\textbf{Respect query graph: tree.}
Unfortunately, the previous constraint \eqref{eq:respect_query_graph1} fails to encode the minimum for certain proper trees, which contain nodes with at least three different degrees. The simplest, problematic tree shape is represented in Figure \ref{fig:subgraph_to_terms} (node 2 has degree 3, node 3 has degree 2, and the others have degree 1). The problem is that with constraint \eqref{eq:respect_query_graph1}, not all the join trees have the same minimum energy due to nodes' different degrees. To address this problem, we develop an alternative constraint
\begin{equation*}
    \sum_{r = 1}^{|V| - 1} \left( r - \sum_{(i,j) \in E}\sum_{(i',j') \in E} x_{i,j}^{r}x_{i',j'}^{r}\right)^2,
\end{equation*}
where again we form the sum over the elements if the indices satisfy $|\left\{ i, j \right\} \cap \left\{ i', j' \right\}| = 1$. At every rank, this constraint selects $r$-many pairs of type $x_{i,j}^{r}x_{i',j'}^{r}$ to be true. This forces the returned join tree to respect the query graph and be connected. This constraint is slightly more complex than the previous constraints since it is not a quadratic but a higher-order constraint that includes terms of four variables. On the other hand, it does not introduce additional variables.
