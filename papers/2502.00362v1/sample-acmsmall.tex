%%
%% This is file `sample-acmsmall.tex',
%% generated with the docstrip utility.
%%
%% The original source files were:
%%
%% samples.dtx  (with options: `all,journal,bibtex,acmsmall')
%% 
%% IMPORTANT NOTICE:
%% 
%% For the copyright see the source file.
%% 
%% Any modified versions of this file must be renamed
%% with new filenames distinct from sample-acmsmall.tex.
%% 
%% For distribution of the original source see the terms
%% for copying and modification in the file samples.dtx.
%% 
%% This generated file may be distributed as long as the
%% original source files, as listed above, are part of the
%% same distribution. (The sources need not necessarily be
%% in the same archive or directory.)
%%
%%
%% Commands for TeXCount
%TC:macro \cite [option:text,text]
%TC:macro \citep [option:text,text]
%TC:macro \citet [option:text,text]
%TC:envir table 0 1
%TC:envir table* 0 1
%TC:envir tabular [ignore] word
%TC:envir displaymath 0 word
%TC:envir math 0 word
%TC:envir comment 0 0
%%
%% The first command in your LaTeX source must be the \documentclass
%% command.
%%
%% For submission and review of your manuscript please change the
%% command to \documentclass[manuscript, screen, review]{acmart}.
%%
%% When submitting camera ready or to TAPS, please change the command
%% to \documentclass[sigconf]{acmart} or whichever template is required
%% for your publication.
%%
%%
\documentclass[acmsmall,authorversion,nonacm]{acmart}

\usepackage{listings}
\lstset{breaklines=true}
\usepackage{graphicx}
\usepackage{dsfont}
\usepackage{amsthm}
\usepackage{amsmath}
\usepackage{algorithm}
\usepackage{algpseudocode}
\usepackage{makecell}
\usepackage{caption}
\usepackage{subcaption}
\theoremstyle{definition}
\newtheorem{example}{Example}[section]
\usepackage[braket, qm]{qcircuit}

\DeclareMathOperator*{\argmax}{arg\,max}
\DeclareMathOperator*{\argmin}{arg\,min}

\algrenewcommand\algorithmicrequire{\textbf{Input:}}
\algrenewcommand\algorithmicensure{\textbf{Output:}}

%\settopmatter{printacmref=false}

%% Rights management information.  This information is sent to you
%% when you complete the rights form.  These commands have SAMPLE
%% values in them; it is your responsibility as an author to replace
%% the commands and values with those provided to you when you
%% complete the rights form.
\setcopyright{acmlicensed}
\copyrightyear{2018}
\acmYear{2018}
\acmDOI{XXXXXXX.XXXXXXX}

%%
%% These commands are for a JOURNAL article.

\acmJournal{JACM}
\acmVolume{37}
\acmNumber{4}
\acmArticle{111}
\acmMonth{8}

%%
%% Submission ID.
%% Use this when submitting an article to a sponsored event. You'll
%% receive a unique submission ID from the organizers
%% of the event, and this ID should be used as the parameter to this command.
%%\acmSubmissionID{123-A56-BU3}

%%
%% For managing citations, it is recommended to use bibliography
%% files in BibTeX format.
%%
%% You can then either use BibTeX with the ACM-Reference-Format style,
%% or BibLaTeX with the acmnumeric or acmauthoryear sytles, that include
%% support for advanced citation of software artefact from the
%% biblatex-software package, also separately available on CTAN.
%%
%% Look at the sample-*-biblatex.tex files for templates showcasing
%% the biblatex styles.
%%

%%
%% The majority of ACM publications use numbered citations and
%% references.  The command \citestyle{authoryear} switches to the
%% "author year" style.
%%
%% If you are preparing content for an event
%% sponsored by ACM SIGGRAPH, you must use the "author year" style of
%% citations and references.
%% Uncommenting
%% the next command will enable that style.
%%\citestyle{acmauthoryear}


%%
%% end of the preamble, start of the body of the document source.
\begin{document}

%%
%% The "title" command has an optional parameter,
%% allowing the author to define a "short title" to be used in page headers.
\title{Left-Deep Join Order Selection with Higher-Order Unconstrained Binary Optimization on Quantum Computers}

%%
%% The "author" command and its associated commands are used to define
%% the authors and their affiliations.
%% Of note is the shared affiliation of the first two authors, and the
%% "authornote" and "authornotemark" commands
%% used to denote shared contribution to the research.
\author{Valter Uotila}
\email{valter.uotila@helsinki.fi}
\orcid{1234-5678-9012}
\affiliation{%
  \institution{University of Helsinki}
  \country{Finland}
}

%%
%% By default, the full list of authors will be used in the page
%% headers. Often, this list is too long, and will overlap
%% other information printed in the page headers. This command allows
%% the author to define a more concise list
%% of authors' names for this purpose.
\renewcommand{\shortauthors}{Uotila}

%%
%% The abstract is a short summary of the work to be presented in the
%% article.
\begin{abstract}
Join order optimization is among the most crucial query optimization problems, and its central position is also evident in the new research field where quantum computing is applied to database optimization and data management. In the field, join order optimization is the most studied database problem, usually tackled with a quadratic unconstrained binary optimization model, which is solved with various meta-heuristics such as quantum annealing, quantum approximate optimization algorithm, or variational quantum eigensolver. In this work, we continue developing quantum computing techniques for join order optimization by presenting three novel quantum optimization algorithms. These algorithms are based on a higher-order unconstrained binary optimization model, which is a generalization of the quadratic model and has not previously been applied to database problems. Theoretically, these optimization problems naturally map to universal quantum computers and quantum annealers. Compared to previous research, two of our algorithms are the first quantum algorithms to precisely model the join order cost function. We prove theoretical bounds by showing that these two methods encode the same plans as the dynamic programming algorithm without cross-products, which provides the optimal result up to cross-products. The third algorithm reaches at least as good plans as the greedy algorithm without cross-products. These results set an important theoretical connection between the classical and quantum algorithms for join order selection, which has not been studied in the previous research. To demonstrate our algorithms' practical usability, we have conducted an experimental evaluation on thousands of clique, cycle, star, tree, and chain query graphs using quantum and classical solvers.
\end{abstract}

%%
%% The code below is generated by the tool at http://dl.acm.org/ccs.cfm.
%% Please copy and paste the code instead of the example below.
%%
% \begin{CCSXML}
% <ccs2012>
%  <concept>
%   <concept_id>00000000.0000000.0000000</concept_id>
%   <concept_desc>Do Not Use This Code, Generate the Correct Terms for Your Paper</concept_desc>
%   <concept_significance>500</concept_significance>
%  </concept>
%  <concept>
%   <concept_id>00000000.00000000.00000000</concept_id>
%   <concept_desc>Do Not Use This Code, Generate the Correct Terms for Your Paper</concept_desc>
%   <concept_significance>300</concept_significance>
%  </concept>
%  <concept>
%   <concept_id>00000000.00000000.00000000</concept_id>
%   <concept_desc>Do Not Use This Code, Generate the Correct Terms for Your Paper</concept_desc>
%   <concept_significance>100</concept_significance>
%  </concept>
%  <concept>
%   <concept_id>00000000.00000000.00000000</concept_id>
%   <concept_desc>Do Not Use This Code, Generate the Correct Terms for Your Paper</concept_desc>
%   <concept_significance>100</concept_significance>
%  </concept>
% </ccs2012>
% \end{CCSXML}

% \ccsdesc[500]{Do Not Use This Code~Generate the Correct Terms for Your Paper}
% \ccsdesc[300]{Do Not Use This Code~Generate the Correct Terms for Your Paper}
% \ccsdesc{Do Not Use This Code~Generate the Correct Terms for Your Paper}
% \ccsdesc[100]{Do Not Use This Code~Generate the Correct Terms for Your Paper}

%%
%% Keywords. The author(s) should pick words that accurately describe
%% the work being presented. Separate the keywords with commas.
\keywords{quantum computing, join order selection, higher-order binary optimization}

% \received{20 February 2007}
% \received[revised]{12 March 2009}
% \received[accepted]{5 June 2009}

%%
%% This command processes the author and affiliation and title
%% information and builds the first part of the formatted document.
\maketitle

\section{Introduction}
\label{sec:introduction}
The business processes of organizations are experiencing ever-increasing complexity due to the large amount of data, high number of users, and high-tech devices involved \cite{martin2021pmopportunitieschallenges, beerepoot2023biggestbpmproblems}. This complexity may cause business processes to deviate from normal control flow due to unforeseen and disruptive anomalies \cite{adams2023proceddsriftdetection}. These control-flow anomalies manifest as unknown, skipped, and wrongly-ordered activities in the traces of event logs monitored from the execution of business processes \cite{ko2023adsystematicreview}. For the sake of clarity, let us consider an illustrative example of such anomalies. Figure \ref{FP_ANOMALIES} shows a so-called event log footprint, which captures the control flow relations of four activities of a hypothetical event log. In particular, this footprint captures the control-flow relations between activities \texttt{a}, \texttt{b}, \texttt{c} and \texttt{d}. These are the causal ($\rightarrow$) relation, concurrent ($\parallel$) relation, and other ($\#$) relations such as exclusivity or non-local dependency \cite{aalst2022pmhandbook}. In addition, on the right are six traces, of which five exhibit skipped, wrongly-ordered and unknown control-flow anomalies. For example, $\langle$\texttt{a b d}$\rangle$ has a skipped activity, which is \texttt{c}. Because of this skipped activity, the control-flow relation \texttt{b}$\,\#\,$\texttt{d} is violated, since \texttt{d} directly follows \texttt{b} in the anomalous trace.
\begin{figure}[!t]
\centering
\includegraphics[width=0.9\columnwidth]{images/FP_ANOMALIES.png}
\caption{An example event log footprint with six traces, of which five exhibit control-flow anomalies.}
\label{FP_ANOMALIES}
\end{figure}

\subsection{Control-flow anomaly detection}
Control-flow anomaly detection techniques aim to characterize the normal control flow from event logs and verify whether these deviations occur in new event logs \cite{ko2023adsystematicreview}. To develop control-flow anomaly detection techniques, \revision{process mining} has seen widespread adoption owing to process discovery and \revision{conformance checking}. On the one hand, process discovery is a set of algorithms that encode control-flow relations as a set of model elements and constraints according to a given modeling formalism \cite{aalst2022pmhandbook}; hereafter, we refer to the Petri net, a widespread modeling formalism. On the other hand, \revision{conformance checking} is an explainable set of algorithms that allows linking any deviations with the reference Petri net and providing the fitness measure, namely a measure of how much the Petri net fits the new event log \cite{aalst2022pmhandbook}. Many control-flow anomaly detection techniques based on \revision{conformance checking} (hereafter, \revision{conformance checking}-based techniques) use the fitness measure to determine whether an event log is anomalous \cite{bezerra2009pmad, bezerra2013adlogspais, myers2018icsadpm, pecchia2020applicationfailuresanalysispm}. 

The scientific literature also includes many \revision{conformance checking}-independent techniques for control-flow anomaly detection that combine specific types of trace encodings with machine/deep learning \cite{ko2023adsystematicreview, tavares2023pmtraceencoding}. Whereas these techniques are very effective, their explainability is challenging due to both the type of trace encoding employed and the machine/deep learning model used \cite{rawal2022trustworthyaiadvances,li2023explainablead}. Hence, in the following, we focus on the shortcomings of \revision{conformance checking}-based techniques to investigate whether it is possible to support the development of competitive control-flow anomaly detection techniques while maintaining the explainable nature of \revision{conformance checking}.
\begin{figure}[!t]
\centering
\includegraphics[width=\columnwidth]{images/HIGH_LEVEL_VIEW.png}
\caption{A high-level view of the proposed framework for combining \revision{process mining}-based feature extraction with dimensionality reduction for control-flow anomaly detection.}
\label{HIGH_LEVEL_VIEW}
\end{figure}

\subsection{Shortcomings of \revision{conformance checking}-based techniques}
Unfortunately, the detection effectiveness of \revision{conformance checking}-based techniques is affected by noisy data and low-quality Petri nets, which may be due to human errors in the modeling process or representational bias of process discovery algorithms \cite{bezerra2013adlogspais, pecchia2020applicationfailuresanalysispm, aalst2016pm}. Specifically, on the one hand, noisy data may introduce infrequent and deceptive control-flow relations that may result in inconsistent fitness measures, whereas, on the other hand, checking event logs against a low-quality Petri net could lead to an unreliable distribution of fitness measures. Nonetheless, such Petri nets can still be used as references to obtain insightful information for \revision{process mining}-based feature extraction, supporting the development of competitive and explainable \revision{conformance checking}-based techniques for control-flow anomaly detection despite the problems above. For example, a few works outline that token-based \revision{conformance checking} can be used for \revision{process mining}-based feature extraction to build tabular data and develop effective \revision{conformance checking}-based techniques for control-flow anomaly detection \cite{singh2022lapmsh, debenedictis2023dtadiiot}. However, to the best of our knowledge, the scientific literature lacks a structured proposal for \revision{process mining}-based feature extraction using the state-of-the-art \revision{conformance checking} variant, namely alignment-based \revision{conformance checking}.

\subsection{Contributions}
We propose a novel \revision{process mining}-based feature extraction approach with alignment-based \revision{conformance checking}. This variant aligns the deviating control flow with a reference Petri net; the resulting alignment can be inspected to extract additional statistics such as the number of times a given activity caused mismatches \cite{aalst2022pmhandbook}. We integrate this approach into a flexible and explainable framework for developing techniques for control-flow anomaly detection. The framework combines \revision{process mining}-based feature extraction and dimensionality reduction to handle high-dimensional feature sets, achieve detection effectiveness, and support explainability. Notably, in addition to our proposed \revision{process mining}-based feature extraction approach, the framework allows employing other approaches, enabling a fair comparison of multiple \revision{conformance checking}-based and \revision{conformance checking}-independent techniques for control-flow anomaly detection. Figure \ref{HIGH_LEVEL_VIEW} shows a high-level view of the framework. Business processes are monitored, and event logs obtained from the database of information systems. Subsequently, \revision{process mining}-based feature extraction is applied to these event logs and tabular data input to dimensionality reduction to identify control-flow anomalies. We apply several \revision{conformance checking}-based and \revision{conformance checking}-independent framework techniques to publicly available datasets, simulated data of a case study from railways, and real-world data of a case study from healthcare. We show that the framework techniques implementing our approach outperform the baseline \revision{conformance checking}-based techniques while maintaining the explainable nature of \revision{conformance checking}.

In summary, the contributions of this paper are as follows.
\begin{itemize}
    \item{
        A novel \revision{process mining}-based feature extraction approach to support the development of competitive and explainable \revision{conformance checking}-based techniques for control-flow anomaly detection.
    }
    \item{
        A flexible and explainable framework for developing techniques for control-flow anomaly detection using \revision{process mining}-based feature extraction and dimensionality reduction.
    }
    \item{
        Application to synthetic and real-world datasets of several \revision{conformance checking}-based and \revision{conformance checking}-independent framework techniques, evaluating their detection effectiveness and explainability.
    }
\end{itemize}

The rest of the paper is organized as follows.
\begin{itemize}
    \item Section \ref{sec:related_work} reviews the existing techniques for control-flow anomaly detection, categorizing them into \revision{conformance checking}-based and \revision{conformance checking}-independent techniques.
    \item Section \ref{sec:abccfe} provides the preliminaries of \revision{process mining} to establish the notation used throughout the paper, and delves into the details of the proposed \revision{process mining}-based feature extraction approach with alignment-based \revision{conformance checking}.
    \item Section \ref{sec:framework} describes the framework for developing \revision{conformance checking}-based and \revision{conformance checking}-independent techniques for control-flow anomaly detection that combine \revision{process mining}-based feature extraction and dimensionality reduction.
    \item Section \ref{sec:evaluation} presents the experiments conducted with multiple framework and baseline techniques using data from publicly available datasets and case studies.
    \item Section \ref{sec:conclusions} draws the conclusions and presents future work.
\end{itemize}
\section{Background}\label{sec:backgrnd}

\subsection{Cold Start Latency and Mitigation Techniques}

Traditional FaaS platforms mitigate cold starts through snapshotting, lightweight virtualization, and warm-state management. Snapshot-based methods like \textbf{REAP} and \textbf{Catalyzer} reduce initialization time by preloading or restoring container states but require significant memory and I/O resources, limiting scalability~\cite{dong_catalyzer_2020, ustiugov_benchmarking_2021}. Lightweight virtualization solutions, such as \textbf{Firecracker} microVMs, achieve fast startup times with strong isolation but depend on robust infrastructure, making them less adaptable to fluctuating workloads~\cite{agache_firecracker_2020}. Warm-state management techniques like \textbf{Faa\$T}~\cite{romero_faa_2021} and \textbf{Kraken}~\cite{vivek_kraken_2021} keep frequently invoked containers ready, balancing readiness and cost efficiency under predictable workloads but incurring overhead when demand is erratic~\cite{romero_faa_2021, vivek_kraken_2021}. While these methods perform well in resource-rich cloud environments, their resource intensity challenges applicability in edge settings.

\subsubsection{Edge FaaS Perspective}

In edge environments, cold start mitigation emphasizes lightweight designs, resource sharing, and hybrid task distribution. Lightweight execution environments like unikernels~\cite{edward_sock_2018} and \textbf{Firecracker}~\cite{agache_firecracker_2020}, as used by \textbf{TinyFaaS}~\cite{pfandzelter_tinyfaas_2020}, minimize resource usage and initialization delays but require careful orchestration to avoid resource contention. Function co-location, demonstrated by \textbf{Photons}~\cite{v_dukic_photons_2020}, reduces redundant initializations by sharing runtime resources among related functions, though this complicates isolation in multi-tenant setups~\cite{v_dukic_photons_2020}. Hybrid offloading frameworks like \textbf{GeoFaaS}~\cite{malekabbasi_geofaas_2024} balance edge-cloud workloads by offloading latency-tolerant tasks to the cloud and reserving edge resources for real-time operations, requiring reliable connectivity and efficient task management. These edge-specific strategies address cold starts effectively but introduce challenges in scalability and orchestration.

\subsection{Predictive Scaling and Caching Techniques}

Efficient resource allocation is vital for maintaining low latency and high availability in serverless platforms. Predictive scaling and caching techniques dynamically provision resources and reduce cold start latency by leveraging workload prediction and state retention.
Traditional FaaS platforms use predictive scaling and caching to optimize resources, employing techniques (OFC, FaasCache) to reduce cold starts. However, these methods rely on centralized orchestration and workload predictability, limiting their effectiveness in dynamic, resource-constrained edge environments.



\subsubsection{Edge FaaS Perspective}

Edge FaaS platforms adapt predictive scaling and caching techniques to constrain resources and heterogeneous environments. \textbf{EDGE-Cache}~\cite{kim_delay-aware_2022} uses traffic profiling to selectively retain high-priority functions, reducing memory overhead while maintaining readiness for frequent requests. Hybrid frameworks like \textbf{GeoFaaS}~\cite{malekabbasi_geofaas_2024} implement distributed caching to balance resources between edge and cloud nodes, enabling low-latency processing for critical tasks while offloading less critical workloads. Machine learning methods, such as clustering-based workload predictors~\cite{gao_machine_2020} and GRU-based models~\cite{guo_applying_2018}, enhance resource provisioning in edge systems by efficiently forecasting workload spikes. These innovations effectively address cold start challenges in edge environments, though their dependency on accurate predictions and robust orchestration poses scalability challenges.

\subsection{Decentralized Orchestration, Function Placement, and Scheduling}

Efficient orchestration in serverless platforms involves workload distribution, resource optimization, and performance assurance. While traditional FaaS platforms rely on centralized control, edge environments require decentralized and adaptive strategies to address unique challenges such as resource constraints and heterogeneous hardware.



\subsubsection{Edge FaaS Perspective}

Edge FaaS platforms adopt decentralized and adaptive orchestration frameworks to meet the demands of resource-constrained environments. Systems like \textbf{Wukong} distribute scheduling across edge nodes, enhancing data locality and scalability while reducing network latency. Lightweight frameworks such as \textbf{OpenWhisk Lite}~\cite{kravchenko_kpavelopenwhisk-light_2024} optimize resource allocation by decentralizing scheduling policies, minimizing cold starts and latency in edge setups~\cite{benjamin_wukong_2020}. Hybrid solutions like \textbf{OpenFaaS}~\cite{noauthor_openfaasfaas_2024} and \textbf{EdgeMatrix}~\cite{shen_edgematrix_2023} combine edge-cloud orchestration to balance resource utilization, retaining latency-sensitive functions at the edge while offloading non-critical workloads to the cloud. While these approaches improve flexibility, they face challenges in maintaining coordination and ensuring consistent performance across distributed nodes.


\begin{table*}[]
\small
\centering
\setlength{\tabcolsep}{3.5pt}
\renewcommand{\arraystretch}{0.8}
\begin{tabular}{@{}cl|ccccc@{}}
\toprule
\textbf{\# Topics} & \textbf{Model} & \multicolumn{1}{l}{\textbf{\# Input Tokens}} & \multicolumn{1}{l}{\textbf{\# Output Tokens}} & \multicolumn{1}{l}{\textbf{\# LLM Calls}} & \multicolumn{1}{l}{\textbf{Cost (GPT-4)}} & \multicolumn{1}{l}{\textbf{Time (seconds)}} \\ \midrule
\multirow{3}{*}{2} & \modelTopic & 21383.08 & 3412.02 & 25.45 & 0.32 & 117.60 \\
 & Hierarchical & 31130.02 & 2536.66 & 13.15 & 0.39 & 83.13 \\
 & Incremental-\textit{Topic} & 59010.66 & 6115.04 & 15.15 & 0.77 & 214.39 \\ \midrule
\multirow{3}{*}{3} & \modelTopic & 30208.20 & 5040.38 & 37.38 & 0.45 & 149.54 \\
 & Hierarchical & 31144.83 & 2649.78 & 13.15 & 0.39 & 68.60 \\
 & Incremental-\textit{Topic} & 61344.07 & 8442.54 & 16.15 & 0.87 & 197.33 \\ \midrule
\multirow{3}{*}{4} & \modelTopic & 38286.40 & 6440.23 & 47.91 & 0.58 & 163.91 \\
 & Hierarchical & 31144.31 & 2740.31 & 13.15 & 0.39 & 88.75 \\
 & Incremental-\textit{Topic} & 62877.46 & 9966.45 & 17.15 & 0.93 & 312.55 \\ \midrule
\multirow{3}{*}{5} & \modelTopic & 47008.59 & 7918.92 & 58.94 & 0.71 & 186.32 \\
 & Hierarchical & 31160.88 & 2850.24 & 13.15 & 0.40 & 61.70 \\
 & Incremental-\textit{Topic} & 64893.95 & 11965.84 & 18.15 & 1.01 & 262.07 \\ \bottomrule
\end{tabular}
\caption{\label{appendix:table:cost_cqa} Number of LLM input/output tokens, LLM calls, GPT-4 Cost (USD), and Time (seconds) needed to run inference on a single DFQS example on ConflictingQA with the top-3 models. We report 5 runs and 20 examples.}
\end{table*}

\begin{table*}[]
\small
\centering
\setlength{\tabcolsep}{3.5pt}
\renewcommand{\arraystretch}{0.8}
\begin{tabular}{@{}cl|ccccc@{}}
\toprule
\multicolumn{1}{l}{\textbf{Dataset}} & \textbf{Model} & \multicolumn{1}{l}{\textbf{\# Input Tokens}} & \multicolumn{1}{l}{\textbf{\# Output Tokens}} & \multicolumn{1}{l}{\textbf{\# LLM Calls}} & \multicolumn{1}{l}{\textbf{Cost (GPT-4)}} & \multicolumn{1}{l}{\textbf{Time (seconds)}} \\ \midrule
\multirow{3}{*}{2} & \modelTopic & 17183.75 & 2722.40 & 20.30 & 0.25 & 94.81 \\
 & Hierarchical & 19181.59 & 2040.39 & 10.25 & 0.25 & 63.68 \\
 & Incremental-\textit{Topic} & 41656.87 & 5062.44 & 12.25 & 0.57 & 182.19 \\ 
 \midrule
\multirow{3}{*}{3} & \modelTopic & 24801.22 & 4136.12 & 30.40 & 0.37 & 126.83 \\
 & Hierarchical & 19182.58 & 2141.91 & 10.25 & 0.26 & 53.32 \\
 & Incremental-\textit{Topic} & 43119.51 & 6532.92 & 13.25 & 0.63 & 152.44 \\ \midrule
\multirow{3}{*}{4} & \modelTopic & 30677.67 & 5037.31 & 38.00 & 0.46 & 120.64 \\
 & Hierarchical & 19203.30 & 2253.17 & 10.25 & 0.26 & 73.35 \\
 & Incremental-\textit{Topic} & 43922.02 & 7327.88 & 14.25 & 0.66 & 241.54 \\ \midrule
\multirow{3}{*}{5} & \modelTopic & 36988.41 & 6049.93 & 46.09 & 0.55 & 139.71 \\
 & Hierarchical & 19211.74 & 2356.01 & 10.25 & 0.26 & 49.41 \\
 & Incremental-\textit{Topic} & 45113.12 & 8504.59 & 15.25 & 0.71 & 186.40 \\ \bottomrule
\end{tabular}
\caption{\label{appendix:table:cost_debate} Number of LLM input/output tokens, LLM calls, GPT-4 Cost (USD), and Time (seconds) needed to run inference on a single DFQS example on DebateQFS with the top-3 models. We report 5 runs and 20 examples.}
\end{table*}

\begin{table*}[]
\small
\centering
\setlength{\tabcolsep}{3.5pt}
\renewcommand{\arraystretch}{0.8}
\begin{tabular}{@{}cl|ccccc@{}}
\toprule
\multicolumn{1}{l}{\textbf{\# Topics}} & \textbf{Model} & \multicolumn{1}{l}{\textbf{\# Input Tokens}} & \multicolumn{1}{l}{\textbf{\# Output Tokens}} & \multicolumn{1}{l}{\textbf{\# LLM Calls}} & \multicolumn{1}{l}{\textbf{Cost (GPT-4)}} & \multicolumn{1}{l}{\textbf{Time (seconds)}} \\ 
\midrule
\multirow{3}{*}{ConflictingQA} & \modelTopic & 47008.59 & 7918.92 & 58.94 & 0.71 & 186.32 \\
 & \modelTopic Pick All & 53733.70 & 9596.75 & 71.75 & 0.83 & 303.13 \\
 & Hierarchical-\emph{Topic} & 168160.85 & 7485.50 & 66.75 & 1.91 & 210.80 \\ \midrule
\multirow{3}{*}{DebateQFS} & \modelTopic & 36988.41 & 6049.93 & 46.09 & 0.55 & 139.71 \\
& \modelTopic Pick All & 43098.85 & 7612.45 & 57.25 & 0.66 & 242.35 \\
& Hierarchical-\emph{Topic} & 105237.25 & 5278.35 & 52.25 & 1.21 & 139.96 \\ \bottomrule
\end{tabular}
\caption{\label{appendix:table:cost_weird} Number of LLM input/output tokens, LLM calls, GPT-4 Cost (USD), and Time (seconds) needed to run inference on a single DFQS example on ConflictingQA and DebateQFS with \modelTopic, the version of \modelTopic with no Moderator, and the version of Hierarchical merging that runs on each topic paragraph ($m=5$). We report 5 runs and 20 examples.}
\end{table*}


\section{Research Trustworthiness, Limitations and Trade-offs}\label{sec:trust}

\subsection{Trustworthiness}

\noindent We implemented several techniques to address the requirements of research trustworthiness \citep{miles2014qualitative}. We reported \textit{Saturation} (Phase I), \textit{Member checking} (Phase I), and \emph{Feedback session} (Phase II) in Sect. \ref{sec:methods}.

\textit{Triangulation}: We triangulated data sources, including interviews, focus groups, and participant feedback sessions. This exercise allowed us to ensure that our findings are corroborated across different data sources and contexts.

\textit{Peer debriefing}: Although the analysis was primarily conducted by the first author, the second and third authors reviewed the proposed codes, and the results were continuously discussed and scrutinized by the other two authors in several meetings throughout the analysis process. The participation of two authors in the coding process helped minimize researcher biases \citep{miles2014qualitative}. This approach is grounded in our epistemological stance, constructivism, which posits that knowledge is socially constructed and that collective intellectual engagement can lead to more reliable understandings of the data \citep{fosnot2013constructivism}. 

\textit{Thick description}: We endeavored to provide a detailed explanation of our research process and the decisions we have made throughout (see Sect. \ref{sec:methods}). In addition, we assembled a comprehensive replication package (see Sect. \ref{sec:replication}).

\subsection{Limitations and Trade-offs}\label{sec:limit}

\textit{Homogeneous sample}: Our sample is composed only of software engineers. In line with roles theory \citep{katz1978social,frink2004advancing}, we limited our sample to the software engineer role to mitigate the potential for variations that may arise by the inclusion of multiple roles. Roles theory suggests that individuals' accountability is closely linked to roles \citep{katz1978social,frink2004advancing}. This narrow focus strengthens the internal validity of our study and allows for role-centric conclusions.

\textit{Focus on intrinsic drivers}: By focusing primarily on intrinsic drivers and their influence on accountability, we may have inadvertently undermined other factors. For example, in our previous work, we identified institutional factors, such as financial incentives or denial of promotions, that also promote accountability in SE environments \citep{alami2024understanding}. 

\textit{Limited variation in the focus group design}: Another tradeoff is the limited number of variations in the focus group configurations, and the code snippets we used were not of industrial caliber. The consistency across the four groups, shown in the collected data and findings, suggests that additional configurations might not have significantly altered the results. In addition, we prioritized in-depth discussions, which may have been diluted by overly complicated configurations and complex code.

Another tradeoff for this study design is with more realistic, complex, and context-aware code. However, we felt this would greatly limit the accessibility of the focus groups. A future study, examining the contextual intricacies of a proprietary codebase, would shed insight on the role of context in this setting.

We conducted focus groups synchronously and online. Often code reviews, in particular on GitHub and similar sites, are asynchronous and text-based. Open source projects have different dynamics than the ones we discuss here. Hence, our findings our findings may not fully transferable to asynchronous or open-source code reviews. Furthermore, the online setting may have influenced participants' behavior differently than an in-person setup. 

The implementation of a pre-focus group questionnaire to mitigate the risk of social desirability bias \citep{furnham1986response} and self-censorship \citep{yanos2008false} during the focus group discussions carries the risk of priming participants. To mitigate this risk, we avoided the explicit use of the word ``accountability'' in the questions. In addition, during the discussions, we asked participants to provide concrete examples to anchor their responses in their personal experiences, thereby avoiding generic or socially desired answers.


\section{Theoretical Analysis}
\label{derivation}
\subsection*{Derivation of \autoref{grad-eq}}
The gradient of the objective function \( J\left(\pi_\phi\right) \) (\autoref{objective}) with respect to the policy parameters \( \phi \) is given by:


\begin{equation}
\resizebox{0.8\hsize}{!}{%
$\displaystyle
\begin{aligned}
\nabla_{\phi} J\left(\pi_{\phi}\right) &= \int P^{\mathcal{S}}\left(x^{\mathcal{S}}, u^{\mathcal{S}}, x^{\prime \mathcal{S}}\right) \Bigg[ \sum_{w \in [0,1]^{N}} \nabla_{\phi} \pi_{\phi}(\mathcal{D_S}, w) \cdot r_\phi\left(x^{\mathcal{S}}, u^{\mathcal{S}}, x^{\prime \mathcal{S}}, \Delta_\theta\right) \\
&\quad + \sum_{w \in [0,1]^{N}} \pi_{\phi}(\mathcal{D_S}, w) \cdot \nabla_{\phi} r_\phi\left(x^{\mathcal{S}}, u^{\mathcal{S}}, x^{\prime \mathcal{S}}, \Delta_\theta\right) \Bigg] \, d\left(x^{\mathcal{S}}, u^{\mathcal{S}}, x^{\prime \mathcal{S}}\right),
\end{aligned}$
}
\end{equation}
% }

where \( P^{\mathcal{S}}\left(x^{\mathcal{S}}, u^{\mathcal{S}}, x^{\prime \mathcal{S}}\right) \) represents the probability distribution over state--action--next-state triples, \( \pi_{\phi}(\mathcal{D_S}, w) \) is the policy parameterized by \( \phi \), and \( r_{\phi} \) is the reward function.

We apply the log-derivative trick to the policy gradient:

\begin{equation}
\resizebox{0.6\hsize}{!}{%
$\displaystyle
\nabla_{\phi} \pi_{\phi}(\mathcal{D_S}, w) = \pi_{\phi}(\mathcal{D_S}, w) \nabla_{\phi} \log \pi_{\phi}(\mathcal{D_S}, w).
$
}
\end{equation}

Substituting this into the integral, we obtain:

\begin{equation}
\resizebox{0.98\hsize}{!}{%
$\displaystyle
\begin{aligned}
\nabla_{\phi} J(\pi_{\phi}) &= \int P^{\mathcal{S}}\left(x^{\mathcal{S}}, u^{\mathcal{S}}, x^{\prime \mathcal{S}}\right) \Bigg[ \sum_{w \in [0,1]^{N}} \pi_{\phi}(\mathcal{D_S}, w) \nabla_{\phi} \log \pi_{\phi}(\mathcal{D_S}, w) \cdot r_\phi\left(x^{\mathcal{S}}, u^{\mathcal{S}}, x^{\prime \mathcal{S}}, \Delta_\theta\right) \\
&\quad + \sum_{w \in [0,1]^{N}} \pi_{\phi}(\mathcal{D_S}, w) \cdot \nabla_{\phi} r_\phi\left(x^{\mathcal{S}}, u^{\mathcal{S}}, x^{\prime \mathcal{S}}, \Delta_\theta\right) \Bigg] \, d\left(x^{\mathcal{S}}, u^{\mathcal{S}}, x^{\prime \mathcal{S}}\right).
\end{aligned}$
}
\end{equation}

The sum over \( w \) can be interpreted as an expectation with respect to the policy distribution \( \pi_{\phi} \). Therefore, we simplify the expression to:


\begin{equation}
\resizebox{0.9\hsize}{!}{%
$\displaystyle
\begin{aligned}
\nabla_{\phi} J(\pi_{\phi}) &= \int P^{\mathcal{S}}\left(x^{\mathcal{S}}, u^{\mathcal{S}}, x^{\prime \mathcal{S}}\right) \, \mathbb{E}_{w \sim \pi_{\phi}(\mathcal{D_S}, \cdot)} \Bigg[ \nabla_{\phi} \log \pi_{\phi}(\mathcal{D_S}, w) \cdot r_\phi\left(x^{\mathcal{S}}, u^{\mathcal{S}}, x^{\prime \mathcal{S}}, \Delta_\theta\right) \\
&\quad + \nabla_{\phi} r_\phi\left(x^{\mathcal{S}}, u^{\mathcal{S}}, x^{\prime \mathcal{S}}, \Delta_\theta\right) \Bigg] \, d\left(x^{\mathcal{S}}, u^{\mathcal{S}}, x^{\prime \mathcal{S}}\right).
\end{aligned}$
}
\end{equation}


Finally, by interpreting the integral over \( P^{\mathcal{S}} \) and the expectation over \( \pi_{\phi} \) jointly as an expectation with respect to the distribution of trajectories under the current policy, we arrive at the final expression:


\begin{equation}
\resizebox{0.67\hsize}{!}{%
$\displaystyle
\begin{aligned}
    \nabla_{\phi} J\left(\pi_{\phi}\right) 
    &= \mathbb{E}_{\substack{(x^{\mathcal{S}}, u^{\mathcal{S}}, x^{\prime \mathcal{S}}) \sim P^{\mathcal{S}}\\ w \sim \pi_{\phi}(\mathcal{D_S}, \cdot)}} \Bigg[ r_\phi\left(x^{\mathcal{S}}, u^{\mathcal{S}}, x^{\prime \mathcal{S}}, \Delta_\theta\right) \cdot \nabla_{\phi} \log \pi_{\phi}(\mathcal{D_S}, w) \\
    & \quad + \nabla_{\phi} r_\phi\left(x^{\mathcal{S}}, u^{\mathcal{S}}, x^{\prime \mathcal{S}}, \Delta_\theta\right) \Bigg].
\end{aligned}$
}
\end{equation}
\section{Discussion of Assumptions}\label{sec:discussion}
In this paper, we have made several assumptions for the sake of clarity and simplicity. In this section, we discuss the rationale behind these assumptions, the extent to which these assumptions hold in practice, and the consequences for our protocol when these assumptions hold.

\subsection{Assumptions on the Demand}

There are two simplifying assumptions we make about the demand. First, we assume the demand at any time is relatively small compared to the channel capacities. Second, we take the demand to be constant over time. We elaborate upon both these points below.

\paragraph{Small demands} The assumption that demands are small relative to channel capacities is made precise in \eqref{eq:large_capacity_assumption}. This assumption simplifies two major aspects of our protocol. First, it largely removes congestion from consideration. In \eqref{eq:primal_problem}, there is no constraint ensuring that total flow in both directions stays below capacity--this is always met. Consequently, there is no Lagrange multiplier for congestion and no congestion pricing; only imbalance penalties apply. In contrast, protocols in \cite{sivaraman2020high, varma2021throughput, wang2024fence} include congestion fees due to explicit congestion constraints. Second, the bound \eqref{eq:large_capacity_assumption} ensures that as long as channels remain balanced, the network can always meet demand, no matter how the demand is routed. Since channels can rebalance when necessary, they never drop transactions. This allows prices and flows to adjust as per the equations in \eqref{eq:algorithm}, which makes it easier to prove the protocol's convergence guarantees. This also preserves the key property that a channel's price remains proportional to net money flow through it.

In practice, payment channel networks are used most often for micro-payments, for which on-chain transactions are prohibitively expensive; large transactions typically take place directly on the blockchain. For example, according to \cite{river2023lightning}, the average channel capacity is roughly $0.1$ BTC ($5,000$ BTC distributed over $50,000$ channels), while the average transaction amount is less than $0.0004$ BTC ($44.7k$ satoshis). Thus, the small demand assumption is not too unrealistic. Additionally, the occasional large transaction can be treated as a sequence of smaller transactions by breaking it into packets and executing each packet serially (as done by \cite{sivaraman2020high}).
Lastly, a good path discovery process that favors large capacity channels over small capacity ones can help ensure that the bound in \eqref{eq:large_capacity_assumption} holds.

\paragraph{Constant demands} 
In this work, we assume that any transacting pair of nodes have a steady transaction demand between them (see Section \ref{sec:transaction_requests}). Making this assumption is necessary to obtain the kind of guarantees that we have presented in this paper. Unless the demand is steady, it is unreasonable to expect that the flows converge to a steady value. Weaker assumptions on the demand lead to weaker guarantees. For example, with the more general setting of stochastic, but i.i.d. demand between any two nodes, \cite{varma2021throughput} shows that the channel queue lengths are bounded in expectation. If the demand can be arbitrary, then it is very hard to get any meaningful performance guarantees; \cite{wang2024fence} shows that even for a single bidirectional channel, the competitive ratio is infinite. Indeed, because a PCN is a decentralized system and decisions must be made based on local information alone, it is difficult for the network to find the optimal detailed balance flow at every time step with a time-varying demand.  With a steady demand, the network can discover the optimal flows in a reasonably short time, as our work shows.

We view the constant demand assumption as an approximation for a more general demand process that could be piece-wise constant, stochastic, or both (see simulations in Figure \ref{fig:five_nodes_variable_demand}).
We believe it should be possible to merge ideas from our work and \cite{varma2021throughput} to provide guarantees in a setting with random demands with arbitrary means. We leave this for future work. In addition, our work suggests that a reasonable method of handling stochastic demands is to queue the transaction requests \textit{at the source node} itself. This queuing action should be viewed in conjunction with flow-control. Indeed, a temporarily high unidirectional demand would raise prices for the sender, incentivizing the sender to stop sending the transactions. If the sender queues the transactions, they can send them later when prices drop. This form of queuing does not require any overhaul of the basic PCN infrastructure and is therefore simpler to implement than per-channel queues as suggested by \cite{sivaraman2020high} and \cite{varma2021throughput}.

\subsection{The Incentive of Channels}
The actions of the channels as prescribed by the DEBT control protocol can be summarized as follows. Channels adjust their prices in proportion to the net flow through them. They rebalance themselves whenever necessary and execute any transaction request that has been made of them. We discuss both these aspects below.

\paragraph{On Prices}
In this work, the exclusive role of channel prices is to ensure that the flows through each channel remains balanced. In practice, it would be important to include other components in a channel's price/fee as well: a congestion price  and an incentive price. The congestion price, as suggested by \cite{varma2021throughput}, would depend on the total flow of transactions through the channel, and would incentivize nodes to balance the load over different paths. The incentive price, which is commonly used in practice \cite{river2023lightning}, is necessary to provide channels with an incentive to serve as an intermediary for different channels. In practice, we expect both these components to be smaller than the imbalance price. Consequently, we expect the behavior of our protocol to be similar to our theoretical results even with these additional prices.

A key aspect of our protocol is that channel fees are allowed to be negative. Although the original Lightning network whitepaper \cite{poon2016bitcoin} suggests that negative channel prices may be a good solution to promote rebalancing, the idea of negative prices in not very popular in the literature. To our knowledge, the only prior work with this feature is \cite{varma2021throughput}. Indeed, in papers such as \cite{van2021merchant} and \cite{wang2024fence}, the price function is explicitly modified such that the channel price is never negative. The results of our paper show the benefits of negative prices. For one, in steady state, equal flows in both directions ensure that a channel doesn't loose any money (the other price components mentioned above ensure that the channel will only gain money). More importantly, negative prices are important to ensure that the protocol selectively stifles acyclic flows while allowing circulations to flow. Indeed, in the example of Section \ref{sec:flow_control_example}, the flows between nodes $A$ and $C$ are left on only because the large positive price over one channel is canceled by the corresponding negative price over the other channel, leading to a net zero price.

Lastly, observe that in the DEBT control protocol, the price charged by a channel does not depend on its capacity. This is a natural consequence of the price being the Lagrange multiplier for the net-zero flow constraint, which also does not depend on the channel capacity. In contrast, in many other works, the imbalance price is normalized by the channel capacity \cite{ren2018optimal, lin2020funds, wang2024fence}; this is shown to work well in practice. The rationale for such a price structure is explained well in \cite{wang2024fence}, where this fee is derived with the aim of always maintaining some balance (liquidity) at each end of every channel. This is a reasonable aim if a channel is to never rebalance itself; the experiments of the aforementioned papers are conducted in such a regime. In this work, however, we allow the channels to rebalance themselves a few times in order to settle on a detailed balance flow. This is because our focus is on the long-term steady state performance of the protocol. This difference in perspective also shows up in how the price depends on the channel imbalance. \cite{lin2020funds} and \cite{wang2024fence} advocate for strictly convex prices whereas this work and \cite{varma2021throughput} propose linear prices.

\paragraph{On Rebalancing} 
Recall that the DEBT control protocol ensures that the flows in the network converge to a detailed balance flow, which can be sustained perpetually without any rebalancing. However, during the transient phase (before convergence), channels may have to perform on-chain rebalancing a few times. Since rebalancing is an expensive operation, it is worthwhile discussing methods by which channels can reduce the extent of rebalancing. One option for the channels to reduce the extent of rebalancing is to increase their capacity; however, this comes at the cost of locking in more capital. Each channel can decide for itself the optimum amount of capital to lock in. Another option, which we discuss in Section \ref{sec:five_node}, is for channels to increase the rate $\gamma$ at which they adjust prices. 

Ultimately, whether or not it is beneficial for a channel to rebalance depends on the time-horizon under consideration. Our protocol is based on the assumption that the demand remains steady for a long period of time. If this is indeed the case, it would be worthwhile for a channel to rebalance itself as it can make up this cost through the incentive fees gained from the flow of transactions through it in steady state. If a channel chooses not to rebalance itself, however, there is a risk of being trapped in a deadlock, which is suboptimal for not only the nodes but also the channel.

\section{Conclusion}
This work presents DEBT control: a protocol for payment channel networks that uses source routing and flow control based on channel prices. The protocol is derived by posing a network utility maximization problem and analyzing its dual minimization. It is shown that under steady demands, the protocol guides the network to an optimal, sustainable point. Simulations show its robustness to demand variations. The work demonstrates that simple protocols with strong theoretical guarantees are possible for PCNs and we hope it inspires further theoretical research in this direction.
\section{Conclusion}
In this work, we propose a simple yet effective approach, called SMILE, for graph few-shot learning with fewer tasks. Specifically, we introduce a novel dual-level mixup strategy, including within-task and across-task mixup, for enriching the diversity of nodes within each task and the diversity of tasks. Also, we incorporate the degree-based prior information to learn expressive node embeddings. Theoretically, we prove that SMILE effectively enhances the model's generalization performance. Empirically, we conduct extensive experiments on multiple benchmarks and the results suggest that SMILE significantly outperforms other baselines, including both in-domain and cross-domain few-shot settings.

%%
%% The next two lines define the bibliography style to be used, and
%% the bibliography file.
\bibliographystyle{ACM-Reference-Format}
\bibliography{sample-base}


%%
%% If your work has an appendix, this is the place to put it.
\appendix
\section{Appendix: Background on quantum computing}

\subsection{Quantum circuit model}

In this part, we define the quantum circuit model \cite{Nielsen_Chuang_2010}. Classical computers operate with bits having discrete values of $0$ or $1$. Quantum computing is based on quantum bits, called qubits, that are formally represented as vectors in a complex-valued Hilbert space. For a single qubit system, the basis vectors of this space are $|0\rangle:= [1, 0]^{\top}$ and $|1\rangle:= [0, 1]^{\top}$. Every state in a single-qubit quantum computer can be represented as a complex-valued linear combination of these basis states as $|\varphi\rangle = \alpha |0\rangle + \beta |1\rangle$, where $\alpha, \beta \in \mathds{C}$ so that $|\alpha|^2 + |\beta|^2 = 1$. The values $\alpha$ and $\beta$ are also called amplitudes, and their squared lengths $|\alpha|^2 $ and $|\beta|^2$ can be interpreted as a probability distribution. Since a qubit is an element of a Hilbert space equipped with the standard tensor product, we can construct larger systems by applying tensor product operation between the smaller systems. For example, a two-qubit system is described by four basis states: $|0\rangle\otimes|0\rangle = |00\rangle$, $|01\rangle$, $|10 \rangle$ and $|11\rangle$.

The quantum algorithm is implemented by applying quantum logic gates to the qubits. These gates must respect the condition that the resulting quantum system has $\sum_{i}|\alpha_i|^2 = 1$ over the amplitudes $\alpha_i$. The operations which satisfy this property are the unitary matrices. A matrix $U$ over the complex numbers is unitary if its inverse is its conjugate transpose. The conjugate transpose is obtained by conjugating the matrice's complex-valued elements and then transposing the matrix. A quantum circuit represents a quantum computational system where the circuit's wires represent qubits, and operations are represented as gates acting on wires. 

The system is measured after the gates have been applied to the qubits. Measurement operation is especially characteristic of quantum computing and does not have a similar role in classical computing. Informally, measuring corresponds to the return statement at the end of a classical function. The most common measurement operation is measuring on a computational basis. This means that after a measurement, we obtain a classical bit string whose length is the number of qubits. For example, if we measure a 2-qubit system, the possible measurement results are $00$, $01$, $10$, and $11$. The connection to the amplitudes is the following: if the system has a state $\alpha_{11}|11\rangle$, then the value $|\alpha_{11}|^2$ is the probability that we obtain the result $11$ when we measure. 

In quantum computing, we can perform various measurements \cite{Nielsen_Chuang_2010}. In this work, we measure an expectation value of an observable, which is a typical measurement operation in practical applications. This measurement estimates the Hamiltonian's energy value corresponding to the cost we aim to minimize in the combinatorial optimization problem. The expectation value can be interpreted as the weighted average of all possible measurement outcomes, where each outcome is weighted by its probability. However, it does not necessarily correspond to the most probable measurement outcome. Formally, suppose $H$ is an operator such as a Hamiltonian. In that case, we compute $\langle \varphi | H | \varphi \rangle = \sum_{j}\lambda_j |\langle\varphi | \lambda_j\rangle|^2$, where $\lambda_j$ and $|\lambda_j \rangle$ are the eigenvalue and eigenvectors of the operator $H$ and the state $| \varphi \rangle$ is the state where the quantum mechanical system is before the measurement. Now the sum is the weighted average of the eigenvalues which are precisely the measurement results when measuring the observable: the eigenvalues $\lambda_j$ are the measurement outcomes, and $|\langle\varphi | \lambda_j\rangle|^2$ corresponds to the probability, which can be viewed as an overlap between the eigenstate $|\lambda_j \rangle$ and the system's current state $| \varphi \rangle$.

\subsection{QAOA and VQE}

On the gate-based universal quantum computers, we can apply the Quantum Approximate Optimization Algorithm (QAOA) \cite{farhi2014quantum} or Variational Quantum Eigensolver (VQE) \cite{Peruzzo_2014} to find the ground state of the Hamiltonian. Both QAOA and VQE are variational algorithms, meaning we execute them in two phases: first, we execute the circuit with a fixed parameter configuration on a quantum computer, then we estimate the circuit's gradient on a classical computer and tune the parameters. These two phases are repeated until a sufficiently good parameter configuration is found. 

First, we focus on QAOA. In this algorithm, we create two Hamiltonians: a mixer Hamiltonian $H_{\mathrm{mix}}$ and a problem Hamiltonian $H_{C}$. The most common mixer Hamiltonian is $H_{\mathrm{mix}} = \sum_{j}\sigma_{x}^{j}$. These Hamiltonians roughly correspond to the initial Hamiltonian and the problem Hamiltonian in Eq.~\eqref{eq:adiabatic_Hamiltonian}, and the cost Hamiltonian encodes the solution to the optimization problem. Then, we prepare two parametrized circuits for each Hamiltonian: $U_{C}:= e^{-i\gamma H_C}$ and $U_{\mathrm{mix}}:= e^{-i\alpha H_{\mathrm{mix}}}$. More concretely, implementing Hamiltonians as circuits, we follow the idea presented in \cite{Nielsen_Chuang_2010}. For example, if the Hamiltonian is $H = 2\sigma_{z}^{0} \otimes \sigma_{z}^{1} \otimes \sigma_{z}^{2}$, then the corresponding parametrized circuit is in Fig.~\ref{fig:circuit_example}. The important point is that we can natively encode and optimize HUBO problems using this method.
\begin{figure}
    \centering
    \scalebox{1.0}{
\Qcircuit @C=1.0em @R=0.2em @!R { \\
	 	\nghost{{q}_{0} :  } & \lstick{{q}_{0} :  } & \qw & \targ & \gate{\mathrm{R_Z}\,(\mathrm{2\gamma})} & \targ & \qw & \qw\\
	 	\nghost{{q}_{1} :  } & \lstick{{q}_{1} :  } & \targ & \ctrl{-1} & \qw & \ctrl{-1} & \targ & \qw \\
	 	\nghost{{q}_{2} :  } & \lstick{{q}_{2} :  } & \ctrl{-1} & \qw & \qw & \qw & \ctrl{-1} & \qw }}
    \caption{Circuit implementing the Hamiltonian $H = 2 \sigma_{z}^{0} \otimes \sigma_{z}^{1} \otimes \sigma_{z}^{2}$}
    \label{fig:circuit_example}
\end{figure}
Next, we define the complete QAOA circuit, which is a repeated application of circuits $U_C$ and $U_{\mathrm{mix}}$ for $n$ times. The precise value of $n$ depends on the problem, but already values such as $3$ have proved to be good \cite{farhi2014quantum}. Then, we prepare an equal superposition over all the basis states, choose random initial values for the parameters $\gamma$ and $\alpha$, and apply the complete QAOA circuit to the state. The equal superposition encodes that initially, every possible binary configuration has an equal probability of being selected as a solution. After preparing the circuit, we measure the expectation value of the cost Hamiltonian $H_C$. Then, we repeat the circuit execution with modified parameters $\gamma$ and $\alpha$. Since parametrized unitaries are differentiable, the circuits are also differentiable with respect to the parameters, and we can optimize the parameters $\gamma$ and $\alpha$ using classical stochastic gradient descent methods. The optimization goal is to find a parameter configuration that minimizes the expectation value of the Hamiltonian, which solves the combinatorial optimization problem.

The basic principles of VQE are similar to those of QAOA, except that the VQE circuit does not implement $U_C$ and $U_{\mathrm{mix}}$ circuits. The parameterized circuit in VQE is a sophisticated guess that is optimized to minimize the energy of the Hamiltonian using classical gradient descent methods. In VQE, we also measure the expectation value of the cost Hamiltonian $H_C$. QAOA and VQE have many modified versions that tackle challenges in the current noisy intermediate-scale quantum computing. 
\section{Appendix: HUBO to QUBO reduction}

Next, we describe how to reduce HUBO problems to QUBO problems. This rewrite process is necessary to utilize a wider variety of optimization platforms. In this work, we mainly rely on the D-wave's Ocean framework's utility of automatically translating HUBO problems into QUBO problems. The translation is based on a polynomial reduction by minimum selection or by substitution \cite{polynomial_reductions}.

The HUBO to QUBO reduction based on minimum selection \cite{polynomial_reductions} follows the scheme
\begin{equation*}
    xyz = \max_{w}w(x+y+z-2),
\end{equation*}
which iteratively replaces the higher order terms $xyz$ with lower order terms by introducing auxiliary binary variables $w$. Depending on the order in which variable terms are replaced, the QUBO formulation may vary in format, and the number of binary variables, but the minimum point remains unchanged.

Rewriting mechanism by substitution utilizes the following formula
\begin{equation*}
    xyz = \min_{w}\left\{wz + \mathrm{MP}(x, y ; w)\right\},
\end{equation*}
where $M > 1$  is a penalty weight and $P$ is a penalty function that has the lowest value when $w = xy$. The details of why these rewriting methods reach the same minimum are explained in \cite{polynomial_reductions}. These reduction methods provide a technically easy method to encode HUBOs as QUBOs, but they also introduce auxiliary variables depending on the number of higher-order terms.
\section{Dynamic programming and greedy algorithms}

The dynamic programming and greedy algorithms implemented here are based on \cite{Neumann_course}. Dynamic programming for join order selection comprises a general class of approaches to optimize the join order selection. In our work, we have fixed the cost function (Eq.~\eqref{eq:cost_function}) and employed the dynamic programming algorithm with and without cross-products. The algorithm without cross-products is presented in Alg.~\ref{alg:dp-join-order}. It relies on functions that create left-deep trees for trees $T_1$ and $T_2$ and return costs for join trees based on the cost function in Eq.~\eqref{eq:cost_function}.

\begin{algorithm}
\caption{Dynamic Programming for Join Order Optimization With Cross-Products}
\label{alg:dp-join-order}
\begin{algorithmic}[1]
\Require relations $R = \{r_1, r_2, \dots, r_n\}$, selectivities $S$
\Ensure optimal left-deep join tree in $\text{dp\_table}[R]$
\State initialize $\text{dp\_table}$
\For{$r \in R$}
    \State $\text{dp\_table}[\{r\}] \gets r$ \Comment{Base case: single relation}
\EndFor
\For{$s = 2 \ \mathbf{ to } \ |R|$} \Comment{Size of subsets from 2 to $n$}
    \For{$\text{subset} \subseteq R$ such that $|\text{subset}| = s-1$}
        \If{$\text{subset} \in \text{dp\_table}$}
            \For{$r \in R \setminus \text{subset}$} \Comment{Relations not in subset}
                \State $T_1 \gets \text{dp\_table}[\text{subset}]$
                \State $T_2 \gets \text{dp\_table}[\{r\}]$
                \State $\text{T}, \text{T\_cost} \gets \text{create\_join\_tree}(T_1, T_2, R, S)$
                \State $\text{join\_key} \gets \text{subset} \cup \{r\}$
                \If{$\text{join\_key} \notin \text{dp\_table}$}
                    \State $\text{dp\_table}[\text{join\_key}] \gets T$ \Comment{Update if key not in table}
                \EndIf
                \If{$\text{join\_key} \in \text{dp\_table}$}
                    \State $\text{prev\_cost} \gets \textbf{cost}(\text{dp\_table}[\text{join\_key}], R, S)$
                    \If{$\text{T\_cost} < \text{prev\_cost}$}
                        \State $\text{dp\_table}[\text{join\_key}] \gets T$ \Comment{Update if lower cost}
                    \EndIf
                \EndIf
            \EndFor
        \EndIf
    \EndFor
\EndFor
\end{algorithmic}
\end{algorithm}

The algorithm that computes the dynamic programming result without cross-products is similar except that for a query graph $G$, we change line 6: \textbf{for} connected subgraph $\subset G$ such that $|\text{subgraph}| = s - 1$ \textbf{do}. Then, the algorithm proceeds with the connected subgraphs of size $s - 1$ instead of all subsets of size $s - 1$.

The greedy algorithm is the other standard algorithm to optimize join order selection, and we represent it in Alg.~\ref{alg:greedy-join-order}. Similarly to the dynamic programming algorithm, we can consider only solutions without cross-products so that we iterate only over tables connected to one of the tables already joined. In other words, at each step, we compute a value called adjacent\_tables which contains those tables $R_i$ so that if edge $(R_i, R_j) \in G$ in the query graph $G$, then we require that $R_i \notin \text{joined\_tables}$ but $R_j \in \text{joined\_tables}$. 

\begin{algorithm}
\caption{Greedy Algorithm for Join Order Selection}
\label{alg:greedy-join-order}
\begin{algorithmic}[1]
\Require relations $R = \left\{ r_1, \ldots, r_n \right\}$, selectivities $S$
\Ensure Greedy join order tree

\State $\text{join\_result} \gets [r_i, r_j]$ so that $f_{i,j}|r_j||r_j|$ is the smallest

\For{$1 \ \text{to} \ |R| - 1$}
    \State $\text{min\_cost} \gets \infty$
    \State $\text{min\_table} \gets \text{None}$
    \For{$\text{table} \in \text{relations} \setminus \text{joined\_result}$} \Comment{Iterate over tables which are not joined}
        \State $\text{current\_join\_tree} \gets [\text{table}, \text{join\_result}]$
        \State $\text{current\_cost} \gets \textbf{cost}(\text{current\_join\_tree}, R, S)$
        \If{$\text{current\_cost} < \text{min\_cost}$} \Comment{Choose table with smallest cost}
            \State $\text{min\_cost} \gets \text{current\_cost}$
            \State $\text{min\_table} \gets \text{table}$
        \EndIf
    \EndFor
    \State $\text{join\_result} \gets [\text{min\_table}, \text{join\_result}]$
\EndFor
\end{algorithmic}
\end{algorithm}
\section{Appendix: Experimental results}
In this section, we present the results of the comprehensive experimental evaluation, which contains various combinations of query graphs, problem formulations, and classical and quantum optimizers. For each method proposed in this work, we evaluated the technique against five common query graph types: clique, star, chain, cycle, and tree. Each query graph is labeled using the format 'Graph name - number of nodes'. For each graph type and graph size, we randomly selected 20 query graph instances with cardinalities and selectivities. The cardinalities are randomly sampled from the range 10 to 50 and selectivity from interval $(0, 1]$. The costs are summed over 20 runs, describing a realistic cumulative error, and scaled with respect to the cost returned from the dynamic programming algorithm \textit{with} cross-products, which is the optimal left-deep plan. The anonymized code for this experimental evaluation can be found on GitHub \cite{anonymous2024qjoin}. Since we have used 20 query graph instances for five different graph types of sizes 3 to 60 and solved them with four different quantum and classical solvers, the total number of evaluated query graphs is in the thousands.

%After fixing one of the three proposed methods and a query graph instance, we have constructed the corresponding HUBO optimization problem. Then, we have submitted the problem for each selected solver. Depending on the solver's requirements, we might need to translate the HUBO problem into the equivalent QUBO problem. The available solvers include two quantum computing (D-wave quantum annealer, D-wave Hybrid quantum annealer) and two classical approaches (exact poly solver, Gurobi optimizer).

We have decided to focus on the quality of solutions instead of optimization time. Although time is crucial in real-life cases, integrating quantum computational systems with classical systems still brings an unavoidable overhead. Quantum computers work at the time scale of nano and milliseconds, but the classical pre-and post-processing makes the total computation time relatively long in practice. Concretely, these pre-and post-processing phases are demonstrated by such steps as encoding problems in HUBO/QUBO format, submitting them to a quantum computer over a possibly slow connection, and even waiting in line for the quantum computer to become available from the other users. After executing the workload, we need to translate the obtained results back into a format that allows us to interpret them in the light of the original problem.

\textbf{Summary of proposed methods.} We have proposed three algorithms to solve the join order selection problem with a higher-order unconstrained binary optimization model. Table \ref{table:methods} introduces names for these methods, which are used in this section.

\begin{table}[!ht]
\centering
\resizebox{0.5\columnwidth}{!}{
\begin{tabular}{|c|c|c|c|}
\hline
\textbf{Method name} & \textbf{Cost function} & \textbf{Validity constraint} \\ \hline
precise 1 & precise cost function & cost function dependent \\ \hline
precise 2 & precise cost function & cost function independent \\ \hline
heuristic & heuristic cost function & cost function dependent \\ \hline
\end{tabular}
}
\caption{Summary of proposed algorithms}
\label{table:methods}
\end{table}

\subsection{Evaluating \textrm{Precise 1} formulation}

First, we evaluate \textrm{Precise 1} formulation, which combines precise cost function and cost-dependent validity constraints. Fig.~\ref{fig:poly_solver_precise_1} shows the results of optimizing join order selection using the D-Wave's exact poly solver. Following the bounds given by Theorem \ref{thm:dp_bound}, the HUBO model consistently generates a plan that matches the quality of the plan produced by the dynamic programming algorithm without the cross products. We also see that the returned plans are only at most $0.7\%$ bigger than the optimal plan from dynamic programming with the cross products. We have excluded some results where the HUBO model produced the exact optimal plan: Clique-3, Cycle-3, Star-4, and Star-5.

\begin{figure*}[tbh]
    \centering
    \includegraphics[width=\textwidth]{results/precise_1_exact_poly_solver.png}
    \caption{Precise 1 results using the D-Wave's exact poly solver}
    \label{fig:poly_solver_precise_1}
    \Description[Other results]{}
\end{figure*}

% Graph Graph with 3 nodes and 3 edges is not included in the chart.
% costvalues: {'HUBO model': 1.0, 'Graph-DP': 1.0, 'Graph-Greedy': 1.0}
% graph type: clique
% Graph Graph with 3 nodes and 3 edges is not included in the chart.
% costvalues: {'HUBO model': 1.0, 'Graph-DP': 1.0, 'Graph-Greedy': 1.0}
% graph type: cycle
% Graph Graph with 4 nodes and 3 edges is not included in the chart.
% costvalues: {'HUBO model': 1.0000000000000002, 'Graph-DP': 1.0, 'Graph-Greedy': 1.0000000000000002}
% graph type: star
% Graph Graph with 5 nodes and 4 edges is not included in the chart.
% costvalues: {'HUBO model': 1.0000040588404762, 'Graph-DP': 1.000004058840476, 'Graph-Greedy': 1.0000040588404762}
% graph type: star
% Graph Graph with 5 nodes and 4 edges is not included in the chart.
% costvalues: {'HUBO model': 1.0, 'Graph-DP': 1.0, 'Graph-Greedy': 1.0}
% graph type: random

Second, we solved the same HUBO formulations using a classical Gurobi solver after the HUBO problem was translated into the equivalent QUBO problem. The results are presented in Fig.~\ref{fig:gurobi_precise_1}. The HUBO to QUBO translation does not decrease the algorithm's quality, and Gurobi finds the correct plans. The results stay very close to the optimal join tree, always being as good as a dynamic programming algorithm without cross products.

\begin{figure*}[tbh]
    \centering
    \includegraphics[width=\textwidth]{results/precise_1_gurobi.png}
    \caption{Precise 1 results using Gurobi solver}
    \label{fig:gurobi_precise_1}
    \Description[Other results]{}
\end{figure*}

% Graph Graph with 3 nodes and 3 edges is not included in the chart.
% costvalues: {'HUBO model': 1.0, 'Graph-DP': 1.0, 'Graph-Greedy': 1.0}
% graph type: clique
% Graph Graph with 3 nodes and 3 edges is not included in the chart.
% costvalues: {'HUBO model': 1.0, 'Graph-DP': 1.0, 'Graph-Greedy': 1.0}
% graph type: cycle
% Graph Graph with 4 nodes and 3 edges is not included in the chart.
% costvalues: {'HUBO model': 1.0000000000000002, 'Graph-DP': 1.0, 'Graph-Greedy': 1.0000000000000002}
% graph type: star
% Graph Graph with 5 nodes and 4 edges is not included in the chart.
% costvalues: {'HUBO model': 1.0000040588404762, 'Graph-DP': 1.000004058840476, 'Graph-Greedy': 1.0000040588404762}
% graph type: star
% Graph Graph with 4 nodes and 5 edges is not included in the chart.
% costvalues: {'HUBO model': 1.0, 'Graph-DP': 1.0, 'Graph-Greedy': 1.0}
% graph type: random

Third, we solved the same problems using D-wave's Leap Hybrid solver, a quantum-classical optimization platform in the cloud. In this case, the results are consistently as good as those from the dynamic program algorithm without the cross products, with some exceptions due to the heuristic nature of the quantum computer: Cycle-6, Chain-7, and Tree-6.

\begin{figure*}[tbh]
    \centering
    \includegraphics[width=\textwidth]{results/precise_1_dwave_LeapHybridSampler.png}
    \caption{Precise 1 results using D-Wave's Leap Hybrid solver}
    \label{fig:leap_precise_1}
    \Description[Other results]{}
\end{figure*}

% Graph Graph with 3 nodes and 3 edges is not included in the chart.
% costvalues: {'HUBO model': 1.0, 'Graph-DP': 1.0, 'Graph-Greedy': 1.0}
% graph type: clique
% Graph Graph with 3 nodes and 3 edges is not included in the chart.
% costvalues: {'HUBO model': 1.0, 'Graph-DP': 1.0, 'Graph-Greedy': 1.0}
% graph type: cycle
% Graph Graph with 4 nodes and 3 edges is not included in the chart.
% costvalues: {'HUBO model': 1.0000000000000002, 'Graph-DP': 1.0, 'Graph-Greedy': 1.0000000000000002}
% graph type: star
% Graph Graph with 5 nodes and 4 edges is not included in the chart.
% costvalues: {'HUBO model': 1.0000177876393395, 'Graph-DP': 1.000004058840476, 'Graph-Greedy': 1.0000040588404762}
% graph type: star

Finally, Fig.~\ref{fig:dwave_precise_1} shows the results from D-Wave's quantum annealer, which does not utilize hybrid features to increase solution quality. This resulted in performance that did not match the performance of the previous solvers, and this performance decrease was already identified in \cite{Schonberger_Scherzinger_Mauerer}. While quality was not as good as the previous solutions, the valid plans were still usable, mainly only a few percent from the global optimal. %Noise on the machine decreased the quality, and we could not embed many previous problems in the hardware. 

\begin{figure*}[tbh]
    \centering
    \includegraphics[width=\textwidth]{results/precise_1_dwave_DWaveSampler.png}
    \caption{Precise 1 results using D-Wave's standard solver}
    \label{fig:dwave_precise_1}
    \Description[Other results]{Everything is included in these results}
\end{figure*}
\subsection{Evaluating \textrm{Precise 2} formulation}

The key idea behind the Precise 2 formulation is to tackle larger join order optimization cases because the validity constraints are more efficient regarding the number of higher-order terms. We include the exact poly solver results to demonstrate that this formulation encodes precisely the correct plans. For the other solvers, we only show results that optimized larger queries compared to the previous Presice 1 method.

First, the results from the exact poly solver in Fig.~\ref{fig:precise_2_exact_poly_solver} demonstrate that this algorithm follows the bounds of Theorem \ref{thm:dp_bound}. In practice, the returned plans are again very close to the optimal plans. We can also see that compared to the Precise 1 method, the different sets of validity constraints work equally well.

\begin{figure*}[tbh]
    \centering
    \includegraphics[width=\textwidth]{results/precise_2_exact_poly_solver.png}
    \caption{Precise 2 results using the D-Wave's exact poly solver}
    \label{fig:precise_2_exact_poly_solver}
    \Description[Precise 2 results using the D-Wave's exact poly solver]{}
\end{figure*}

% Graph Graph with 3 nodes and 3 edges is not included in the chart.
% costvalues: {'Q-Join: Presice 2': 1.0, 'Dynamic programming w/o cross-products': 1.0, 'Greedy w/o cross-products': 1.0}
% graph type: clique
% Graph Graph with 3 nodes and 3 edges is not included in the chart.
% costvalues: {'Q-Join: Presice 2': 1.0, 'Dynamic programming w/o cross-products': 1.0, 'Greedy w/o cross-products': 1.0}
% graph type: cycle
% Graph Graph with 4 nodes and 3 edges is not included in the chart.
% costvalues: {'Q-Join: Presice 2': 1.0000000000000002, 'Dynamic programming w/o cross-products': 1.0, 'Greedy w/o cross-products': 1.0000000000000002}
% graph type: star
% Graph Graph with 5 nodes and 4 edges is not included in the chart.
% costvalues: {'Q-Join: Presice 2': 1.0000040588404762, 'Dynamic programming w/o cross-products': 1.000004058840476, 'Greedy w/o cross-products': 1.0000040588404762}
% graph type: star

Second, to evaluate the Gurobi solver, we scaled up the problem sizes remarkably from the Precise 1 method, although the experiments were performed on a standard laptop. The results are presented in Fig.~\ref{fig:precise_2_gurobi_1} and Fig.~\ref{fig:precise_2_gurobi_2}. We can see that finding the point that minimizes both cost and validity constraints becomes harder when the problem sizes increase. %The results remain close to optimal, with cost summed over multiple runs. However, even one suboptimal round can significantly impact the total cost and accuracy, so the results demonstrate solid performance considering the demanding evaluation setup.

\begin{figure*}[tbh]
    \centering
    \includegraphics[width=\textwidth]{results/precise_2_gurobi_1.png}
    \caption{Precise 2 results using Gurobi solver}
    \label{fig:precise_2_gurobi_1}
    \Description[Precise 2 results using Gurobi solver]{}
\end{figure*}

% Graph Graph with 3 nodes and 3 edges is not included in the chart.
% costvalues: {'Q-Join: Presice 2': 1.0, 'Dynamic programming w/o cross-products': 1.0, 'Greedy w/o cross-products': 1.0}
% graph type: clique
% Graph Graph with 3 nodes and 3 edges is not included in the chart.
% costvalues: {'Q-Join: Presice 2': 1.0, 'Dynamic programming w/o cross-products': 1.0, 'Greedy w/o cross-products': 1.0}
% graph type: clique
% Graph Graph with 3 nodes and 3 edges is not included in the chart.
% costvalues: {'Q-Join: Presice 2': 1.0, 'Dynamic programming w/o cross-products': 1.0, 'Greedy w/o cross-products': 1.0}
% graph type: cycle
% Graph Graph with 3 nodes and 3 edges is not included in the chart.
% costvalues: {'Q-Join: Presice 2': 1.0, 'Dynamic programming w/o cross-products': 1.0, 'Greedy w/o cross-products': 1.0}
% graph type: cycle
% Graph Graph with 4 nodes and 3 edges is not included in the chart.
% costvalues: {'Q-Join: Presice 2': 1.0000000000000002, 'Dynamic programming w/o cross-products': 1.0, 'Greedy w/o cross-products': 1.0000000000000002}
% graph type: star
% Graph Graph with 5 nodes and 4 edges is not included in the chart.
% costvalues: {'Q-Join: Presice 2': 1.0000040588404762, 'Dynamic programming w/o cross-products': 1.000004058840476, 'Greedy w/o cross-products': 1.0000040588404762}
% graph type: star
% Graph Graph with 6 nodes and 5 edges is not included in the chart.
% costvalues: {'Q-Join: Presice 2': 1.0000021564508312, 'Dynamic programming w/o cross-products': 1.0000011116483463, 'Greedy w/o cross-products': 1.0000011116483465}
% graph type: star
% Graph Graph with 4 nodes and 3 edges is not included in the chart.
% costvalues: {'Q-Join: Presice 2': 1.0000000000000002, 'Dynamic programming w/o cross-products': 1.0, 'Greedy w/o cross-products': 1.0000000000000002}
% graph type: star
% Graph Graph with 5 nodes and 4 edges is not included in the chart.
% costvalues: {'Q-Join: Presice 2': 1.0000040588404762, 'Dynamic programming w/o cross-products': 1.000004058840476, 'Greedy w/o cross-products': 1.0000040588404762}
% graph type: star

\begin{figure*}[tbh]
    \centering
    \includegraphics[width=\textwidth]{results/precise_2_gurobi_2.png}
    \caption{Precise 2 results using Gurobi solver}
    \label{fig:precise_2_gurobi_2}
    \Description[Precise 2 results using Gurobi solver]{}
\end{figure*}

Slightly unexpectedly, the Leap Hybrid solver did not perform as well as we expected, as shown in Fig.~\ref{fig:precise_2_dwave_LeapHybridSampler}. The solver does not have tunable hyperparameters, which we would be able to adjust to obtain better results. On the other hand, we used the developer access to the solver, which is limited to only one minute of quantum computing access per month. Finally, we did not include the results from the D-wave quantum solver due to space limitations since the solver did not scale to these cases. %It provided new results only for the Star-6 query graph with $7.07\%$ larger plans over 20 query graphs than the optimal ones.

\begin{figure*}[tbh]
    \centering
    \includegraphics[width=\textwidth]{results/precise_2_dwave_LeapHybridSampler.png}
    \caption{Precise 2 results using D-Wave's Leap Hybrid solver}
    \label{fig:precise_2_dwave_LeapHybridSampler}
    \Description[Precise 2 results using D-Wave's Leap Hybrid solver]{}
\end{figure*}

% Graph Graph with 3 nodes and 3 edges is not included in the chart.
% costvalues: {'Q-Join: Presice 2': 1.0, 'Dynamic programming w/o cross-products': 1.0, 'Greedy w/o cross-products': 1.0}
% graph type: clique
% Graph Graph with 3 nodes and 3 edges is not included in the chart.
% costvalues: {'Q-Join: Presice 2': 1.0, 'Dynamic programming w/o cross-products': 1.0, 'Greedy w/o cross-products': 1.0}
% graph type: clique
% Graph Graph with 3 nodes and 3 edges is not included in the chart.
% costvalues: {'Q-Join: Presice 2': 1.0, 'Dynamic programming w/o cross-products': 1.0, 'Greedy w/o cross-products': 1.0}
% graph type: cycle
% Graph Graph with 3 nodes and 3 edges is not included in the chart.
% costvalues: {'Q-Join: Presice 2': 1.0, 'Dynamic programming w/o cross-products': 1.0, 'Greedy w/o cross-products': 1.0}
% graph type: cycle
% Graph Graph with 4 nodes and 3 edges is not included in the chart.
% costvalues: {'Q-Join: Presice 2': 1.0000000000000002, 'Dynamic programming w/o cross-products': 1.0, 'Greedy w/o cross-products': 1.0000000000000002}
% graph type: star
% Graph Graph with 5 nodes and 4 edges is not included in the chart.
% costvalues: {'Q-Join: Presice 2': 1.0000141364642996, 'Dynamic programming w/o cross-products': 1.000004058840476, 'Greedy w/o cross-products': 1.0000040588404762}
% graph type: star
% Graph Graph with 6 nodes and 5 edges is not included in the chart.
% costvalues: {'Q-Join: Presice 2': 1.0000247692741169, 'Dynamic programming w/o cross-products': 1.0000011116483463, 'Greedy w/o cross-products': 1.0000011116483465}
% graph type: star
% Graph Graph with 4 nodes and 3 edges is not included in the chart.
% costvalues: {'Q-Join: Presice 2': 1.0000000000000002, 'Dynamic programming w/o cross-products': 1.0, 'Greedy w/o cross-products': 1.0000000000000002}
% graph type: star
% Graph Graph with 5 nodes and 4 edges is not included in the chart.
% costvalues: {'Q-Join: Presice 2': 1.0000177876393395, 'Dynamic programming w/o cross-products': 1.000004058840476, 'Greedy w/o cross-products': 1.0000040588404762}
% graph type: star
\subsection{Evaluating heuristic formulation}

The key motivation behind the heuristic formulation is to tackle even larger query graphs. Our main goal is to demonstrate that this algorithm reaches acceptable results with superior scalability compared to the previous Precise 1 and 2 formulations. The results also indicate that Theorem \ref{thm:greedy_bound} is respected in practice. The optimal results are computed with dynamic programming without cross products. Due to space limitations, we only included the results from the Gurobi solver, which we consider the most demonstrative, and we had unlimited access to it since it runs locally. 

The results are presented so that we have computed and scaled the difference between each pair of methods. A value that differs from 0 indicates that the two methods gave different join trees with different costs. Since one of the methods is near-optimal (DP without cross products), it is clear which method produced the suboptimal result. This way, we can compare all three methods at the same time. In all cases, we can see that the heuristic algorithm respects Theorem \ref{thm:greedy_bound} very well in practice, so the difference between quantum and greedy is always $0$.

Fig.~\ref{fig:clique_accuracies} shows the results of applying the heuristic method to clique query graphs. Although these results are good, the scalability in this hard case is modest. On the other hand, we are unaware of any quantum computing research that would have outperformed this scalability in the case of clique graphs. For example, the most scalable method \cite{10.14778/3632093.3632112} excluded clique graphs from their results.

\begin{figure}
    \centering
    \includegraphics[width=\linewidth]{results/clique_accuracies.png}
    \caption{Heuristic results for clique query graphs using Gurobi solver}
    \label{fig:clique_accuracies}
\end{figure}

The results for the tree (Fig.~\ref{fig:tree_accuracies}), chain (Fig.~\ref{fig:chain_accuracies}), cycle (Fig.~\ref{fig:cycle_accuracies}), and star graphs demonstrate the best scalability. We computed the results up to 60 tables to demonstrate advantageous scalability over the most scalable method in the previous research \cite{10.14778/3632093.3632112} where they considered queries up to 50 relations.

\begin{figure}
    \centering
    \includegraphics[width=\linewidth]{results/tree_accuracies.png}
    \caption{Heuristic results for tree query graphs using Gurobi solver}
    \label{fig:tree_accuracies}
\end{figure}

\begin{figure}
    \centering
    \includegraphics[width=\linewidth]{results/chain_accuracies.png}
    \caption{Heuristic results for chain query graphs using Gurobi solver}
    \label{fig:chain_accuracies}
\end{figure}
    
\begin{figure}
    \centering
    \includegraphics[width=\linewidth]{results/cycle_accuracies.png}
    \caption{Heuristic results for cycle query graphs using Gurobi solver}
    \label{fig:cycle_accuracies}
\end{figure}

We exclude the results for the star query graphs because, in this case, all three methods performed identically across up to 60 graphs and over 20 iterations, with no difference observed (a relative scaled difference of 0). These results may be because star graphs typically do not benefit from cross products \cite{10.14778/3632093.3632112}, and our method, which excludes them, performs better with such types of queries.


\section{Appendix: Comparison with quantum-inspired digital annealing}

We evaluated the method proposed in \cite{10.14778/3632093.3632112} with the same workloads we used and present the results in Figures \ref{fig:vldb24_chain}, \ref{fig:vldb24_clique}, \ref{fig:vldb24_cycle}, and \ref{fig:vldb24_star}. The method is the improved algorithm from \cite{Schonberger_Scherzinger_Mauerer}. The authors propose a novel readout technique that improves the results. Since we could not access special quantum-inspired hardware, such as a digital annealer, we used the Gurobi solver, which returns only a single result by default. Thus, the readout technique was not applicable. 

Nevertheless, the results still demonstrate that the method reaches a comparable accuracy to ours, which is optimal or close to optimal. Their method seems to be able to identify beneficial cross products. We have computed the exact results with dynamic programming and compared relative cumulative costs between the methods. Due to higher-order terms in our method, which currently have to be rewritten into quadratic format, their method is still more scalable than ours. On the other hand, our theoretical bounds, the more straightforward variable definitions, and the novel usage of the higher-order model show specific improvements over their methods. We are also positive that our model admits features that make it easier to expand for outer joins and include more complex dependencies between the predicates.

\begin{figure}[!ht]
\centering
\begin{subfigure}{.5\textwidth}
  \centering
  \includegraphics[width=.95\linewidth]{results/vldb24/chain_gurobi.png}
  \caption{Chain graphs with varying number of nodes solved with QUBO formulation proposed in \cite{10.14778/3632093.3632112}}
  \label{fig:vldb24_chain}
\end{subfigure}%
\hfill
\begin{subfigure}{.5\textwidth}
  \centering
  \includegraphics[width=.95\linewidth]{results/vldb24/clique_gurobi.png}
  \caption{Clique query graphs with varying number of nodes solved with QUBO formulation proposed in \cite{10.14778/3632093.3632112}}
  \label{fig:vldb24_clique}
\end{subfigure}
\caption{A figure with two subfigures}
\label{fig:test}
\end{figure}

\begin{figure}[!ht]
\centering
\begin{subfigure}{.5\textwidth}
  \centering
  \includegraphics[width=.95\linewidth]{results/vldb24/cycle_gurobi.png}
  \caption{Cycle query graphs with varying number of nodes solved with QUBO formulation proposed in \cite{10.14778/3632093.3632112}}
  \label{fig:vldb24_cycle}
\end{subfigure}%
\hfill
\begin{subfigure}{.5\textwidth}
  \centering
  \includegraphics[width=.95\linewidth]{results/vldb24/star_gurobi.png}
  \caption{Star query graphs with varying number of nodes solved with QUBO formulation proposed in \cite{10.14778/3632093.3632112}}
  \label{fig:vldb24_star}
\end{subfigure}
\caption{A figure with two subfigures}
\label{fig:test}
\end{figure}

\end{document}
\endinput
%%
%% End of file `sample-acmsmall.tex'.
