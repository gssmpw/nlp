%%
%% This is file `sample-acmsmall.tex',
%% generated with the docstrip utility.
%%
%% The original source files were:
%%
%% samples.dtx  (with options: `all,journal,bibtex,acmsmall')
%% 
%% IMPORTANT NOTICE:
%% 
%% For the copyright see the source file.
%% 
%% Any modified versions of this file must be renamed
%% with new filenames distinct from sample-acmsmall.tex.
%% 
%% For distribution of the original source see the terms
%% for copying and modification in the file samples.dtx.
%% 
%% This generated file may be distributed as long as the
%% original source files, as listed above, are part of the
%% same distribution. (The sources need not necessarily be
%% in the same archive or directory.)
%%
%%
%% Commands for TeXCount
%TC:macro \cite [option:text,text]
%TC:macro \citep [option:text,text]
%TC:macro \citet [option:text,text]
%TC:envir table 0 1
%TC:envir table* 0 1
%TC:envir tabular [ignore] word
%TC:envir displaymath 0 word
%TC:envir math 0 word
%TC:envir comment 0 0
%%
%% The first command in your LaTeX source must be the \documentclass
%% command.
%%
%% For submission and review of your manuscript please change the
%% command to \documentclass[manuscript, screen, review]{acmart}.
%%
%% When submitting camera ready or to TAPS, please change the command
%% to \documentclass[sigconf]{acmart} or whichever template is required
%% for your publication.
%%
%%
\documentclass[acmsmall,authorversion,nonacm]{acmart}

\usepackage{listings}
\lstset{breaklines=true}
\usepackage{graphicx}
\usepackage{dsfont}
\usepackage{amsthm}
\usepackage{amsmath}
\usepackage{algorithm}
\usepackage{algpseudocode}
\usepackage{makecell}
\usepackage{caption}
\usepackage{subcaption}
\theoremstyle{definition}
\newtheorem{example}{Example}[section]
\usepackage[braket, qm]{qcircuit}

\DeclareMathOperator*{\argmax}{arg\,max}
\DeclareMathOperator*{\argmin}{arg\,min}

\algrenewcommand\algorithmicrequire{\textbf{Input:}}
\algrenewcommand\algorithmicensure{\textbf{Output:}}

%\settopmatter{printacmref=false}

%% Rights management information.  This information is sent to you
%% when you complete the rights form.  These commands have SAMPLE
%% values in them; it is your responsibility as an author to replace
%% the commands and values with those provided to you when you
%% complete the rights form.
\setcopyright{acmlicensed}
\copyrightyear{2018}
\acmYear{2018}
\acmDOI{XXXXXXX.XXXXXXX}

%%
%% These commands are for a JOURNAL article.

\acmJournal{JACM}
\acmVolume{37}
\acmNumber{4}
\acmArticle{111}
\acmMonth{8}

%%
%% Submission ID.
%% Use this when submitting an article to a sponsored event. You'll
%% receive a unique submission ID from the organizers
%% of the event, and this ID should be used as the parameter to this command.
%%\acmSubmissionID{123-A56-BU3}

%%
%% For managing citations, it is recommended to use bibliography
%% files in BibTeX format.
%%
%% You can then either use BibTeX with the ACM-Reference-Format style,
%% or BibLaTeX with the acmnumeric or acmauthoryear sytles, that include
%% support for advanced citation of software artefact from the
%% biblatex-software package, also separately available on CTAN.
%%
%% Look at the sample-*-biblatex.tex files for templates showcasing
%% the biblatex styles.
%%

%%
%% The majority of ACM publications use numbered citations and
%% references.  The command \citestyle{authoryear} switches to the
%% "author year" style.
%%
%% If you are preparing content for an event
%% sponsored by ACM SIGGRAPH, you must use the "author year" style of
%% citations and references.
%% Uncommenting
%% the next command will enable that style.
%%\citestyle{acmauthoryear}


%%
%% end of the preamble, start of the body of the document source.
\begin{document}

%%
%% The "title" command has an optional parameter,
%% allowing the author to define a "short title" to be used in page headers.
\title{Left-Deep Join Order Selection with Higher-Order Unconstrained Binary Optimization on Quantum Computers}

%%
%% The "author" command and its associated commands are used to define
%% the authors and their affiliations.
%% Of note is the shared affiliation of the first two authors, and the
%% "authornote" and "authornotemark" commands
%% used to denote shared contribution to the research.
\author{Valter Uotila}
\email{valter.uotila@helsinki.fi}
\orcid{1234-5678-9012}
\affiliation{%
  \institution{University of Helsinki}
  \country{Finland}
}

%%
%% By default, the full list of authors will be used in the page
%% headers. Often, this list is too long, and will overlap
%% other information printed in the page headers. This command allows
%% the author to define a more concise list
%% of authors' names for this purpose.
\renewcommand{\shortauthors}{Uotila}

%%
%% The abstract is a short summary of the work to be presented in the
%% article.
\begin{abstract}
Join order optimization is among the most crucial query optimization problems, and its central position is also evident in the new research field where quantum computing is applied to database optimization and data management. In the field, join order optimization is the most studied database problem, usually tackled with a quadratic unconstrained binary optimization model, which is solved with various meta-heuristics such as quantum annealing, quantum approximate optimization algorithm, or variational quantum eigensolver. In this work, we continue developing quantum computing techniques for join order optimization by presenting three novel quantum optimization algorithms. These algorithms are based on a higher-order unconstrained binary optimization model, which is a generalization of the quadratic model and has not previously been applied to database problems. Theoretically, these optimization problems naturally map to universal quantum computers and quantum annealers. Compared to previous research, two of our algorithms are the first quantum algorithms to precisely model the join order cost function. We prove theoretical bounds by showing that these two methods encode the same plans as the dynamic programming algorithm without cross-products, which provides the optimal result up to cross-products. The third algorithm reaches at least as good plans as the greedy algorithm without cross-products. These results set an important theoretical connection between the classical and quantum algorithms for join order selection, which has not been studied in the previous research. To demonstrate our algorithms' practical usability, we have conducted an experimental evaluation on thousands of clique, cycle, star, tree, and chain query graphs using quantum and classical solvers.
\end{abstract}

%%
%% The code below is generated by the tool at http://dl.acm.org/ccs.cfm.
%% Please copy and paste the code instead of the example below.
%%
% \begin{CCSXML}
% <ccs2012>
%  <concept>
%   <concept_id>00000000.0000000.0000000</concept_id>
%   <concept_desc>Do Not Use This Code, Generate the Correct Terms for Your Paper</concept_desc>
%   <concept_significance>500</concept_significance>
%  </concept>
%  <concept>
%   <concept_id>00000000.00000000.00000000</concept_id>
%   <concept_desc>Do Not Use This Code, Generate the Correct Terms for Your Paper</concept_desc>
%   <concept_significance>300</concept_significance>
%  </concept>
%  <concept>
%   <concept_id>00000000.00000000.00000000</concept_id>
%   <concept_desc>Do Not Use This Code, Generate the Correct Terms for Your Paper</concept_desc>
%   <concept_significance>100</concept_significance>
%  </concept>
%  <concept>
%   <concept_id>00000000.00000000.00000000</concept_id>
%   <concept_desc>Do Not Use This Code, Generate the Correct Terms for Your Paper</concept_desc>
%   <concept_significance>100</concept_significance>
%  </concept>
% </ccs2012>
% \end{CCSXML}

% \ccsdesc[500]{Do Not Use This Code~Generate the Correct Terms for Your Paper}
% \ccsdesc[300]{Do Not Use This Code~Generate the Correct Terms for Your Paper}
% \ccsdesc{Do Not Use This Code~Generate the Correct Terms for Your Paper}
% \ccsdesc[100]{Do Not Use This Code~Generate the Correct Terms for Your Paper}

%%
%% Keywords. The author(s) should pick words that accurately describe
%% the work being presented. Separate the keywords with commas.
\keywords{quantum computing, join order selection, higher-order binary optimization}

% \received{20 February 2007}
% \received[revised]{12 March 2009}
% \received[accepted]{5 June 2009}

%%
%% This command processes the author and affiliation and title
%% information and builds the first part of the formatted document.
\maketitle

\documentclass[../main.tex]{subfiles}
\graphicspath{{../images/}}
\makeatletter
\def\input@path{{../images/}}
\makeatother
\begin{document}
\section{Introduction}
\begin{figure}
\centering
\begin{tikzpicture}
\node[inner sep=0pt] (ws) at (0, 0) {
\includegraphics[height=.4\textwidth, trim={10cm 0 10cm 0},clip]{world_space.png}};
\node[inner sep=0pt] (cs) at (6,0) {\includegraphics[height=.4\textwidth, trim={10cm 1cm 10cm 4cm},clip]{conf_space.png}};
\end{tikzpicture}
\vspace{-5pt}
\label{fig:pbrm_intro}
\caption{\textbf{Left}: Shows world space obstacles as grey spheres. Robots start and goal configuration is colored red and green, respectively. Configurations along the computed path are colored transparent blue. \textbf{Right:} Mapped world space scenario to configuration space. Obstacle region is the grey mesh. Red spheres are collision-free regions computed by the neural SCDF. The optimized shortest path in the convex corridor is the blue curve.}
\vspace{-25pt}
\end{figure}
Motion planning is the problem of finding a collision-free trajectory that connects a given start and goal configuration. The planning takes place in the configuration space of the robot. For single body robots, like mobile robots or drones, the configuration space and the world space are usually the same. This simplifies the planning, since explicit obstacle representations are available which enables geometrical tools like separating hyperplanes, smallest distance to obstacles etc., to be used when designing motion planning algorithms. For multi-body robots like manipulators, the situation is completely different. The world space obstacles are usually mapped to non-convex regions, and to make the problem even harder, the mapping is usually not known. Forming explicit representations of the obstacle region in the configuration space is usually too expensive or intractable. Despite all of this, sampling based planners are used with great success, which mainly is due to their use of implicit representations of the obstacle region. The basic idea is to construct a graph in the configuration space that covers and connects the collision-free region. From this graph, a path can be extracted that connects a given start and goal configuration. The approach is computationally expensive, since the graph is constructed with the smallest geometrical building block available, points, which represents a collision-check. Furthermore, the extracted paths from the graph are non-smooth and jagged due to the stochastic nature of the approach. This adds an additional post-processing step to the process, where the paths are shortcutted and smoothened, before the path can be used for tracking. Clearly a lot of time is invested to form this graph and produce smooth paths. Thus, if the obstacles start to move, then all of this work is done in no use, since all points that make up this graph need to be re-verified, which is simply too time consuming to be done in real time.
\\\\
In this work, we want to address the existing drawbacks of the sampling based planners. Our main contribution is an improved motion planner where each vertex in the graph covers a collision-free region in the form of a sphere instead of a point and where the edges are formed with neighboring intersecting spheres. This representation has the advantage of instead of returning piecewise linear paths, returning a sequence of overlapping spheres, i.e. a convex corridor, that connects a given start and goal configuration, illustrated in Figure \ref{fig:pbrm_intro}. This convex corridor allows us to use convex optimization to produce smooth trajectories, instead of computationally expensive post-processing methods. The representation further allows us to estimate the coverage of the collision-free space, which gives us awareness and feedback in the offline roadmap construction phase. Finally, our representation is simple to adapt to moving obstacles, simply requery for the new radii and recheck for intersections. 
\\\\
The spherical collision-free regions are formed using a signed distance function (SDF), which is a function that returns the smallest distance from an arbitrary point to the boundary of an obstacle. As the name implies, the distance is signed, thus if the point is inside the obstacle it is negative otherwise positive. If the distance is positive, a sphere with radius equal to the distance is guaranteed to cover a collision-free region. Using an SDF in motion planning is not new, but what is novel about our approach is that we express the distance in the configuration space instead of the world space and by doing so allows us to form these convex collision-free regions. We refer to the resulting SDF as a signed configuration distance function (SCDF). Computing an SCDF analytically is non-trivial, our approach is therefore to parameterize the SCDF with a deep neural network and learn the mapping by supervised learning. Our resulting neural SCDF can compute distances for different parameter values of obstacle shapes and we also show how multiple distances can be combined, thus making our approach flexible.
\section{Related work}
Motion planning algorithms can roughly be divided into three families, grid-based, sampling based and optimization based methods. Grid-based methods (GBM) discretize the planning space from which a graph is then compiled. A standard search method is A$^\star$ \citep{a_star}, which is classified as an \textit{informed} search method, since it employs a heuristic function to speed up the search. A$^\star$ guarantees to return an optimal path at the level of discretization used. GBMs usually discretize the planning space by a regular lattice and this limits the GBMs to problems with low dimensionality due to the curse of dimensionality. Thus, GBMs are usually limited to single-body robots where the degrees of freedom (DOF) are low. To overcome the inherent scaling problem with the GBMs, stochastic methods are usually used for multi-body robots. These methods are termed as sampling-based methods (SBM) and core members within this family are the rapidly-exploring random trees (RRT) \citep{rrt} and the probabilistic roadmap (PRM) \citep{prm}. RRT grows a tree from the start configuration and explores the collision-free region in a rapid way until it is able to connect to the goal region. RRT is usually improved by bi-directional planning \citep{rrt_connect}, i.e. an additional tree is grown from the goal configuration and the trees are tested for connection after any tree has been expanded. RRT is a single-query method, thus it searches for a path from scratch each time it is queried. Contrary to this, PRM is a multi-query method, which solves for multiple queries without starting from scratch. PRM does this by creating a roadmap (graph) that covers the collision-free space as an offline step. The graph is then used to solve for multiple queries. PRMs are used in cases where the environment does not change since the extra offline step is too computationally costly and needs to be re-done if the environment is changed. In our work, we address this inherent issue by using a different roadmap representation. Our vertices in the graph cover a collision-free region in the form of spheres and we form the edges by checking for intersecting spheres. If something in the environment changes, we recompute the spheres radii and recheck the intersections, without relying on collision detection. We use a trained neural network to compute the sphere radius, therefore querying for the radius can be done fast, hence our representation enables the PRM for dynamic environments.
\\\\
In the recent decades, optimization based methods (OBM) \citep{chomp, schulman, itomp, stomp} have been introduced as an alternative to SBM for multi-body robots. Like the SBM, the OBMs scale well to higher dimensional problems and produce smoother motion. It is common to use a SDF in the optimization since it is a smooth function, thus enabling gradient-based methods. However, the standard way of expressing the SDF is in world space. The distance therefore needs to be mapped to the configuration space by the forward kinematics. This mapping makes the optimization problem a non-linear program (NLP), which is computationally expensive to solve. Recently, a different approach has been proposed. In \cite{mp_gcs} motion planning is formulated as a convex optimization problem by using the graph of convex sets framework \citep{gcs}. The underlying idea is to decompose the collision-free space into intersecting convex sets from which a convex optimization problem is formulated. In cases where an explicit representation of the obstacles in the configuration space exists, like for single-body robots, creating collision-free convex regions can be done fast \citep{iris}. For multi-body robots, this is non-trivial. Existing work does this successfully \citep{iris_nlp, iris_c} by an optimization based approach, but the methods are still too time consuming to be used in the presence of moving obstacles. Our approach is instead to use deep learning to learn an SDF expressed in the configuration space. With this, we can query for shortest distances to the collision boundary, which allows us to expand spherical regions which are collision-free. Our approach is fast and therefore enables our suggested roadmap planner to be used in dynamic environments.
\\\\
Recent research has focused on learning collision detection \citep{fk_kernel_distance, diffco, graphdistnet} by predicting the signed distance between the robot links and the surrounding obstacles in the world space. The learned SDF is used in trajectory optimization but since the distance is expressed in the world space, the problem becomes an NLP and therefore takes a long time to solve. We take a novel approach and suggest to instead express the signed distance in the configuration space. This allows us to improve the PRM at the same time as it enables convex optimization for trajectory optimization, which runs faster and is more reliable than NLP solvers. In \cite{cspf} a learned signed distance function in the configuration space is proposed similar to our approach. However, their approach is restricted to point cloud representations, while we propose to represent the obstacles as parameterized geometric shapes, e.g. spheres. Furthermore, we also show how to use our learned SCDF to improve an existing roadmap planner.
\section{Problem formulation}
A robot is located in the world space, $\W \subset \R^3 $. The unique location of the robot is given by its configuration $\q \in \C$, where $\C$ is the configuration space. The set of points covered by the robots bodies at a certain configuration is expressed as $\B(\q) \subset \W$. The robot is surrounded by $\NrObst$ obstacles $\O = \bigcup_{i=1}^{\NrObst} \O_i$, where  $\O_i \subset \W$. The representation of the obstacle in the configuration space is the set $\C\O_i = \{\q \in \C \: |\: \B(\q) \cap \O_i \neq \emptyset \}$. The obstacle space is formed as $\Co = \bigcup_{i=1}^{\NrObst} \C \O_i$. The complement is referred to as the free space, $\Cf = \C \setminus \Co$. The path planning problem is a tuple, ($\Cf$, $\qStart$, $\qGoal$), where we want to connect a query pair, consisting of a start, $\qStart$, and goal configuration, $\qGoal$, with a geometric path, $\q(s): [0, 1] \mapsto \Cf$, such that $\q(0)=\qStart$ and $\q(1)=\qGoal$, or report correctly when such a path does not exist.
\end{document}

\section{Basic Background: Supervised Learning and the PAC Model}
\label{sec:background}

At this point almost everyone has heard of machine learning (ML). Anyone likely to stumble upon this article will have also heard of its most influential special case, supervised learning, and those theoretically inclined will also be familiar with the PAC model. Nonetheless, I will set the stage by  recapping the basics.

\subsection{Basics of Supervised Learning}%Let's set the stage in any case

\emph{Supervised Learning} is the task of ``coming up'' with a function $f: \X \to \Y$ to ``explain'' or ``fit'' a sequence of input/output examples   $(x_1,y_1), \ldots, (x_n,y_n)$, with $x_i \in \X$ and $y_i \in \Y$.  Here $\X$ is a \emph{data domain} consisting of \emph{datapoints} $x \in \X$, $\Y$ is a \emph{label set} consisting of \emph{labels} $y \in \Y$, and the sequence $(x_1,y_1),\ldots,(x_n,y_n)$ is the \emph{training data} consisting of \emph{labeled examples (a.k.a. samples)}~$(x_i,y_i)$.  I~will refer to the chosen function $f$ as a \emph{predictor}, and to $n$ as the \emph{sample size}. A \emph{learning algorithm} takes as input training data, and outputs (some representation of) a predictor $f \in \Y^\X$.\footnote{Note that this describes the usual \emph{batch}, a.k.a.~\emph{offline}, setting of supervised learning. I do not discuss other paradigms such as online or active learning in this article.} 



Success in supervised learning is defined as \emph{generalization} to  future examples: For a typical \emph{test example}  $(x_{\tst},y_{\tst})$, the predicted label $y'_{\tst}=f(x_{\tst})$ should ``equal'' $y_{\tst}$, perhaps approximately. We usually assume the test example is drawn from the same  ``source'' as the training data  --- commonly, i.i.d.~from the same distribution. The quality of the prediction is quantified by $\ell(y'_{\tst},y_{\tst})$, where $\ell:~\Y~\times~\Y \to \RR_{\geq 0}$ is a \emph{loss function} chosen as part of the problem definition. Common loss functions include the 0-1 loss $\ell_{0-1}(y',y) = [y' \neq y]$ for \emph{classification} problems,\footnote{The notation $[P]$ denotes $1$ when predicate $P$ is true, and denotes $0$ when $P$ is false.} as well as the absolute loss $|y'-y|$ or squared loss $(y'-y)^2$ for \emph{regression problems} featuring $\Y  \sse \RR$.

Nontrivial generalization properties are typically only possible if one assumes something about the data.\footnote{The need for such an assumption is formalized by the  \emph{no free lunch theorems} of supervised learning \cite{wolpert_connection_1992,wolpert_lack_1996,schaffer_conservation_1994}.} The Bayesian approach to  machine learning, common in many applications, assumes some parametric form for the distribution generating the data, and postulates a prior on the parameters. This is not the approach I will take in this article. Instead, I will focus on the frequentist --- and some would say ``worst-case'' or ``adversarial'' ---  approach that is common in the computational learning theory community, embodied by the PAC model. Here we assume that the (training and test) data can be explained, perhaps approximately, by a function in some ``simple enough to learn'' class of functions $\H \sse \Y^\X$, often called the \emph{hypotheses}. Equivalently, we  seek a predictor which explains the unseen data roughly  as well as the best hypothesis $h^* \in \H$, whether or not we assume that $h^*$ itself provides a perfect explanation.



 \paragraph{Common Algorithmic Templates.} Perhaps the best known general-purpose supervised learning algorithm is \emph{empirical risk minimization (ERM)}, which chooses as its predictor a hypothesis $f \in \H$ minimizing $\frac{1}{n} \sum_{i=1}^n \ell(f(x_i),y_i)$ --- a quantity called the \emph{training error}, \emph{empirical error}, or \emph{empirical risk} of $f$. %\footnote{When multiple hypotheses minimize the empirical risk, we assume ERM breaks ties arbitrarily.}
A common template for generalizing ERM involves adding a \emph{regularization term} $\psi(f)$ to the  objective function, typically chosen to measure some notion of ``hypothesis complexity.'' An algorithm instantiating this template is known as a \emph{structural risk minimizer (SRM)}, and chooses as its predictor the hypothesis $f \in \H$ minimizing the \emph{structural risk} $\frac{1}{n} \sum_{i=1}^n \ell(f(x_i),y_i) + \psi(f)$. Other well-known algorithms, such as gradient descent and its variations,  can frequently be interpreted as approximate implementations of ERM or SRM.


\paragraph{Proper vs Improper Learning.} A learning algorithm is said to be \emph{proper} if its predictor $f$ is always chosen from the hypothesis class, i.e., $f \in \H$, otherwise it is said to be \emph{improper}. ERM  is an example of a proper learning algorithm, as are SRM algorithms of the form described above.  In the \emph{proper regime} of learning, algorithms are required to be proper. This article will be concerned with the more flexible \emph{improper regime} (a.k.a \emph{representation-independent learning}), where no such constraint is placed on the learner. In other words, all we care about is predictive power at test time, rather than any insights derived from the functional form or representation of the predictor~itself.


\subsection{The PAC Model}
A standard mathematical setup for evaluation of supervised learning algorithms, at least in the theoretical computer science community, is Valiant's \emph{Probably Approximately Correct (PAC) model} of learning (see e.g.~\cite{kearns_introduction_1994,mohri_foundations_2018}). Here, we assume there is an unknown distribution $\D$ on $\X \times \Y$ from which training and test data are  drawn.  Specifically, the labeled datapoints of the training set  $(x_1,y_1), \ldots, (x_n,y_n)$, as well as the test data  $(x_\tst,y_\tst)$, are i.i.d.~from $\D$. Often it is assumed that $\D$ lies in some class of distributions of interest. The \emph{true expected loss}, or simply \emph{loss}, of a predictor $f: \X \to \Y$ is the expected loss it incurs on draws from $\D$, written $L_\D(f) = \Ex_{(x,y) \sim \D} \ell(f(x),y)$.


There are two main ``settings'' in PAC learning. The  \emph{realizable setting} only requires that the data be perfectly explained by some hypothesis in $\H$. More generally, the \emph{agnostic setting} makes no assumption relating the data to the hypotheses, but shifts the goalposts as necessary to allow nontrivial guarantees: the expected loss at test time is evaluated only ``relative'' to that of the best hypothesis $h^* \in \H$. There are other settings which make more nuanced assumptions, such as $\D$ being of a particular parametric form or its support living in some (unknown) lower-dimensional space, etc. I will mostly discuss the realizable and agnostic settings in this article, those being the simplest and most studied from a theoretical perspective. %TODO:We will briefly discuss other settings in Section ??

The PAC model demands high probability guarantees of learners, in the worst case over distributions of interest. Consider first the realizable setting, where $\D$ is such that $\min_{h \in \H} L_{\D}(h) = 0$. A PAC learner has \emph{error} $\epsilon=\epsilon(n)$ and \emph{confidence} $\delta=\delta(n)$ if, when training data consists of $n$ i.i.d~samples from a realizable distribution $\D$, it produces a predictor $f$  satisfying $L_\D(f) \leq \epsilon$ with probability at least $1-\delta$. In the agnostic setting, where $\D$ can be arbitrary, we require $L_\D(f) - \min_{h \in \H} L_\D(h) \leq \epsilon$ with probability $1-\delta$.

In both the realizable and agnostic settings, we look for PAC learners with small $\epsilon$ and $\delta$ as a function of the sample size $n$. An equivalent perspective looks at the sample complexity $m(\epsilon,\delta)$, which is the minimum sample size which guarantees error  at most $\epsilon$ with probability at least $1-\delta$. We say a problem is \emph{PAC learnable} if its PAC sample complexity is finite whenever $\epsilon,\delta > 0$.

For most PAC learning problems, learnability and sample complexity are characterized in terms of a  ``dimension'' of the hypothesis class. Most prominently this is the \emph{VC dimension} for binary classification, the \emph{fat shattering dimension} for agnostic regression, and the \emph{DS dimension} for multiclass classification (see \cite{anthony_neural_1999,daniely_optimal_2014,brukhim_characterization_2022}). Treatment of these is beyond the scope of this article. The unfamiliar reader need not worry, however,  as dimensions will feature only tangentially in our~discussion.




%\paragraph{Learning settings: Realizable, Agnostic, etc.} In learning theory, evaluating a supervised learning algorithm requires specifying a data model and an objective. We will leave the details of the data model flexible for now, to allow for both the PAC model and the adversarial transductive model. Nonetheless we will describe two variations, which we call ``settings'', which cut across different models. The  \emph{realizable setting}  requires only that the data be perfectly explained by some hypothesis $h \in \H$ --- i.e., there exists a hypothesis which is guaranteed to suffer a loss of $0$ on training and test data. The performance of the learning algorithm is its expected loss at test time for some ``worst case'' realizable instance. More generally, the \emph{agnostic setting} makes no assumption relating the data to the hypotheses, but shifts the goalposts as necessary to allow nontrivial guarantees: the expected loss at test time is evaluated only ``relative'' to that of the best hypothesis $h^* \in \H$, again for some ``worst case'' instance. There are other settings which make more nuanced assumptions about the data, such as it is drawn from a distribution of a particular parametric form, or that it lives in some (unknown) lower-dimensional space, etc. We will mostly discuss the realizable and agnostic settings, those being the simplest and most studied from a theoretical perspective.




%%% Local Variables:
%%% mode: latex
%%% TeX-master: "learning_matching"
%%% End:

\begin{table*}[]
\small
\centering
\setlength{\tabcolsep}{3.5pt}
\renewcommand{\arraystretch}{0.8}
\begin{tabular}{@{}cl|ccccc@{}}
\toprule
\textbf{\# Topics} & \textbf{Model} & \multicolumn{1}{l}{\textbf{\# Input Tokens}} & \multicolumn{1}{l}{\textbf{\# Output Tokens}} & \multicolumn{1}{l}{\textbf{\# LLM Calls}} & \multicolumn{1}{l}{\textbf{Cost (GPT-4)}} & \multicolumn{1}{l}{\textbf{Time (seconds)}} \\ \midrule
\multirow{3}{*}{2} & \modelTopic & 21383.08 & 3412.02 & 25.45 & 0.32 & 117.60 \\
 & Hierarchical & 31130.02 & 2536.66 & 13.15 & 0.39 & 83.13 \\
 & Incremental-\textit{Topic} & 59010.66 & 6115.04 & 15.15 & 0.77 & 214.39 \\ \midrule
\multirow{3}{*}{3} & \modelTopic & 30208.20 & 5040.38 & 37.38 & 0.45 & 149.54 \\
 & Hierarchical & 31144.83 & 2649.78 & 13.15 & 0.39 & 68.60 \\
 & Incremental-\textit{Topic} & 61344.07 & 8442.54 & 16.15 & 0.87 & 197.33 \\ \midrule
\multirow{3}{*}{4} & \modelTopic & 38286.40 & 6440.23 & 47.91 & 0.58 & 163.91 \\
 & Hierarchical & 31144.31 & 2740.31 & 13.15 & 0.39 & 88.75 \\
 & Incremental-\textit{Topic} & 62877.46 & 9966.45 & 17.15 & 0.93 & 312.55 \\ \midrule
\multirow{3}{*}{5} & \modelTopic & 47008.59 & 7918.92 & 58.94 & 0.71 & 186.32 \\
 & Hierarchical & 31160.88 & 2850.24 & 13.15 & 0.40 & 61.70 \\
 & Incremental-\textit{Topic} & 64893.95 & 11965.84 & 18.15 & 1.01 & 262.07 \\ \bottomrule
\end{tabular}
\caption{\label{appendix:table:cost_cqa} Number of LLM input/output tokens, LLM calls, GPT-4 Cost (USD), and Time (seconds) needed to run inference on a single DFQS example on ConflictingQA with the top-3 models. We report 5 runs and 20 examples.}
\end{table*}

\begin{table*}[]
\small
\centering
\setlength{\tabcolsep}{3.5pt}
\renewcommand{\arraystretch}{0.8}
\begin{tabular}{@{}cl|ccccc@{}}
\toprule
\multicolumn{1}{l}{\textbf{Dataset}} & \textbf{Model} & \multicolumn{1}{l}{\textbf{\# Input Tokens}} & \multicolumn{1}{l}{\textbf{\# Output Tokens}} & \multicolumn{1}{l}{\textbf{\# LLM Calls}} & \multicolumn{1}{l}{\textbf{Cost (GPT-4)}} & \multicolumn{1}{l}{\textbf{Time (seconds)}} \\ \midrule
\multirow{3}{*}{2} & \modelTopic & 17183.75 & 2722.40 & 20.30 & 0.25 & 94.81 \\
 & Hierarchical & 19181.59 & 2040.39 & 10.25 & 0.25 & 63.68 \\
 & Incremental-\textit{Topic} & 41656.87 & 5062.44 & 12.25 & 0.57 & 182.19 \\ 
 \midrule
\multirow{3}{*}{3} & \modelTopic & 24801.22 & 4136.12 & 30.40 & 0.37 & 126.83 \\
 & Hierarchical & 19182.58 & 2141.91 & 10.25 & 0.26 & 53.32 \\
 & Incremental-\textit{Topic} & 43119.51 & 6532.92 & 13.25 & 0.63 & 152.44 \\ \midrule
\multirow{3}{*}{4} & \modelTopic & 30677.67 & 5037.31 & 38.00 & 0.46 & 120.64 \\
 & Hierarchical & 19203.30 & 2253.17 & 10.25 & 0.26 & 73.35 \\
 & Incremental-\textit{Topic} & 43922.02 & 7327.88 & 14.25 & 0.66 & 241.54 \\ \midrule
\multirow{3}{*}{5} & \modelTopic & 36988.41 & 6049.93 & 46.09 & 0.55 & 139.71 \\
 & Hierarchical & 19211.74 & 2356.01 & 10.25 & 0.26 & 49.41 \\
 & Incremental-\textit{Topic} & 45113.12 & 8504.59 & 15.25 & 0.71 & 186.40 \\ \bottomrule
\end{tabular}
\caption{\label{appendix:table:cost_debate} Number of LLM input/output tokens, LLM calls, GPT-4 Cost (USD), and Time (seconds) needed to run inference on a single DFQS example on DebateQFS with the top-3 models. We report 5 runs and 20 examples.}
\end{table*}

\begin{table*}[]
\small
\centering
\setlength{\tabcolsep}{3.5pt}
\renewcommand{\arraystretch}{0.8}
\begin{tabular}{@{}cl|ccccc@{}}
\toprule
\multicolumn{1}{l}{\textbf{\# Topics}} & \textbf{Model} & \multicolumn{1}{l}{\textbf{\# Input Tokens}} & \multicolumn{1}{l}{\textbf{\# Output Tokens}} & \multicolumn{1}{l}{\textbf{\# LLM Calls}} & \multicolumn{1}{l}{\textbf{Cost (GPT-4)}} & \multicolumn{1}{l}{\textbf{Time (seconds)}} \\ 
\midrule
\multirow{3}{*}{ConflictingQA} & \modelTopic & 47008.59 & 7918.92 & 58.94 & 0.71 & 186.32 \\
 & \modelTopic Pick All & 53733.70 & 9596.75 & 71.75 & 0.83 & 303.13 \\
 & Hierarchical-\emph{Topic} & 168160.85 & 7485.50 & 66.75 & 1.91 & 210.80 \\ \midrule
\multirow{3}{*}{DebateQFS} & \modelTopic & 36988.41 & 6049.93 & 46.09 & 0.55 & 139.71 \\
& \modelTopic Pick All & 43098.85 & 7612.45 & 57.25 & 0.66 & 242.35 \\
& Hierarchical-\emph{Topic} & 105237.25 & 5278.35 & 52.25 & 1.21 & 139.96 \\ \bottomrule
\end{tabular}
\caption{\label{appendix:table:cost_weird} Number of LLM input/output tokens, LLM calls, GPT-4 Cost (USD), and Time (seconds) needed to run inference on a single DFQS example on ConflictingQA and DebateQFS with \modelTopic, the version of \modelTopic with no Moderator, and the version of Hierarchical merging that runs on each topic paragraph ($m=5$). We report 5 runs and 20 examples.}
\end{table*}


\section{Join order validity as HUBO}

HUBO problems, like QUBO problems, are required to return valid solutions. In this case, validity means that the join tree adheres to the query graph. All valid solutions are usable, although they might not minimize the cost. In this section, we present two approaches to encoding the validity of solutions: cost function dependent and independent. A cost-function-dependent approach is easy to construct but produces a larger number of higher-order terms. The cost-function-independent approach is closer to the standard QUBO formulations and identifies a collection of constraints the formulation needs to satisfy. The advantage of the second formulation is that the terms are primarily quadratic, which is easier to optimize in practice.

\subsection{Cost-function dependent validity}\label{subsubsection:cost_function_dependent}

Considering the cost function generation in the previous subsections, we have generated higher-order terms that encode valid join trees at rank $r = |V| - 2$. Considering the Alg.~\ref{alg:hubo_term_construction}, we can access these terms with $\texttt{HUBO[set of all relations]}$. Let us denote this set of terms as $H$. For example, one of those terms is represented in Fig.~\ref{fig:subgraph_generation}. The first validity constraint forces the model to select exactly one of these terms as true, which requires the final join tree to contain all the relations.

\textbf{Select one valid plan.} We utilize a generalized one-hot (or $k$-hot) encoding from QUBO formulations \cite{lucas_2014, Schonberger_Trummer_Mauerer_2023}. The encoding constructs a constraint that reaches its minimum when precisely $k$ variables are selected to be true from a given set of binary variables. In the generalized formulation, we construct an objective minimized when exactly $k$ terms are selected true from a set of higher-order terms. Let $H$ be the set of higher-order terms at rank $r = |V| - 2$. This functionality generalizes to higher-order cases with the following formulation:
\begin{equation}\label{eq:hubo_combinations}
    H_0 = \left( 1 - \sum_{h \in H} h \right)^2,
\end{equation}
where $h = \prod_{r = 0}^{|V| - 2} x_{i_r, j_r}^{r}$ for some indices $i_r$ and $j_r$ for $r = 0, \ldots, |V| - 2$ that depend on the query graph. The objective $H_0$ is always non-negative since it is squared. It is positive except when exactly one of the terms in the sum $\sum_{h \in H} h$ is $1$ when $H_0$ evaluates to $0$. The sum $\sum_{h \in H} h$ evaluates to $1$ when there is exactly one term $h$ such that all the variables in the product $\prod_{r = 0}^{|V| - 2} x_{i_r, j_r}^{r}$ are true. This combination is the valid join tree that the objective function returns as a solution to the minimization problem. Since these terms $h \in H$ have already been generated for the cost HUBO, we create this constraint based on them. Thus, we call this validity constraint cost-dependent.

\textbf{Every rank must appear exactly once in the solution.}
It is still possible to obtain solutions that include unnecessary true variables that do not affect the minimum of the final HUBO function. For example, consider the plan in Fig.~\ref{fig:subgraph_generation}. The HUBO that encodes this plan would have the same minimum even if we set that $x_{0,2}^{1} = 1$ because this variable would always be multiplied with variables set to $0$, and thus, activating a single variable would not affect the total cost. While we can solve this problem with classical post-processing, we still decided to fix it in the model itself. We include a constraint that every rank should appear exactly once in the solution. This is also an instance of one-hot encoding \cite{dimod_generators_combinations} and encoded with the following quadratic objective:
\begin{equation}\label{eq:every_rank_has_one_join}
    H_1 = \sum_{r = 0}^{|V| - 2}\left( 1 - \sum_{(R_i, R_j) \in E} x_{i,j}^{r} \right)^2.
\end{equation}
The objective $H_1$ is minimized when exactly one variable of type $x_{i,j}^{r}$ is selected to be true for each $0 \leq r <|V|- 1$.
\subsection{Cost-function independent validity}\label{subsubsection:cost_function_independent}

Although validity constraint \eqref{eq:hubo_combinations} is theoretically correct, it produces higher-degree terms due to exponentiation to the power of two, which we might want to avoid. Hence, we develop join tree validity constraints independently from the cost function. Notably, these validity constraints are often quadratic, i.e., QUBOs, and automatically supported by many solvers. Because we develop the theory considering the query graph's structure, we have slightly different constraints depending on the query graphs.

\subsubsection*{Clique graphs}
\textbf{Every rank must appear exactly once in the solution.} The first constraint is what we presented in Eq.~\eqref{eq:every_rank_has_one_join}. This constraint encodes that we perform exactly one join at every rank.

\textbf{Select connected, left-deep join tree.} The second constraint encodes that we penalize cases that do not form a connected, left-deep join tree. While it is clear that join trees must be connected, we must also encode that they are left deep because the cost HUBO does not evaluate bushy trees correctly. To achieve this, we include a constraint of the form
\begin{equation}\label{eq:respect_clique_graph}
    \sum_{r = 0}^{|V| - 2}\sum_{(i,j) \in E}\sum_{(i',j') \in E}Cx_{i,j}^{r}x_{i',j'}^{r+1},
\end{equation}
where the terms are included if $i \neq i'$ and $j \neq j'$, or $i = i'$ and $j = j'$, which means that we penalize these cases by increasing the objective's value by $C$. This makes the model favor cases where consecutive joins share one ''old'' table and include exactly one ''new'' table in the result. The penalizing term $C$ should be set high enough, and we will define its value later.

\textbf{Result contains all the tables.} The third constraint forces the fact that we join all the tables. Our framework identifies joins as pairs of tables $(R_i, R_j)$. This leads to one table $R_i$ appearing multiple times in variables, referring to different joins in a valid solution. An example of this is presented in Fig.~\ref{fig:subgraph_to_terms}, where table $2$ appears multiple times in the solution consisting of joins such as $(0,2)$ and $(1,2)$. This third constraint encodes that we count the number of tables and require that the count is at least one for each table. Counting tables requires minimally a logarithmic number of slack variables (the method to do the logarithmic encoding is presented in \cite{lucas_2014}) in terms of tables. For technical simplicity, we present the less efficient but equivalent method here:
\begin{equation*}
    \sum_{R_i}\left( 1 + \sum_{k = 2}^{|V| - 2}ky^{i}_{k} - \sum_{i}\sum_{r = 1}^{|V| - 2} x_{i,j}^{r} \right)^2.
\end{equation*}
The constraint reaches 0 if at least one variable of type $x_{i,j}^{r}$ is true for each table $R_i$. The other accepted cases are that we activate any number of $k \in \left\{ 2, \ldots, |V| - 2 \right\}$ many variables of type $x_{i,j}^{r}$ using the slack variables $y^{i}_{k}$. This means we must select at least one table but possibly $k$ many tables.
\subsubsection*{Chain, star, cycle, and tree graphs}

\textbf{At every rank, we have performed rank + 1 many joins.}
This constraint is related to the first validity constraint presented in \cite{Schonberger_Scherzinger_Mauerer,10.14778/3632093.3632112}. They present the constraint in terms of tables, whereas we have constructed it in terms of joins. The constraint encodes how many joins we must perform cumulatively at each rank. In other words, when rank is $0$, we select one variable of type $x_{i,j}^{0}$ to be true. When rank is $1$, we choose two variables of type $x_{i,j}^{1}$ to be true. Formally, this constraint is
\begin{equation}\label{eq:at_every_rank_select_rank_many_joins}
    \sum_{r = 0}^{|V| - 2}\left( r + 1 - \sum_{(i,j) \in E} x_{i,j}^{r} \right)^2.
\end{equation}
The constraint is minimized at $0$ when for each $0 \leq r < |V| - 1$, we have selected exactly $r + 1$ variables of type $x_{i,j}^{r}$ to be true.

\textbf{Include the previous joins in the proceeding ranks.}
This constraint is again similar to the second constraint presented in \cite{Schonberger_Scherzinger_Mauerer,10.14778/3632093.3632112} except that we express the constraint using joins. If the join happened at rank $r$, it should be included in every proceeding rank $\geq r$. In other words, we keep the information of the performed joins to the following ranks. This can be achieved with the following constraint
\begin{equation}\label{eq:select_same_join_for_proceeding_ranks}
    \sum_{(i,j) \in E}\sum_{r = 1}^{|V| - 2} x_{i,j}^{r - 1}(1 - x_{i,j}^{r}).
\end{equation}
Now, if $x_{i,j}^{r - 1}$ is active, then the model favors the case that $x_{i,j}^{r}$ is active too since in that case $1 - x_{i,j}^{r} = 0$. If $x_{i,j}^{r} = 0$, the term evaluates to $1$, which is considered penalizing.

\textbf{Respect query graph: chain, star, cycle.}
Since the cost functions are designed to respect structures of query graphs, we use the following constraint to encode the graph structure in chain, star, and cycle graphs:
\begin{equation}\label{eq:respect_query_graph1}
    \sum_{r = 0}^{|V| - 1}\sum_{(i,j) \in E}\sum_{(i',j') \in E} -C x_{i,j}^{r}x_{i',j'}^{r},
\end{equation}
where $|\left\{ i, j \right\} \cap \left\{ i', j' \right\}| = 1$, which means that the joins have to share exactly one table. Setting the coefficient $-C$ as negative, we favor the cases when the joins share precisely one table. This constraint is complementary to constraint \eqref{eq:respect_clique_graph}.

\textbf{Respect query graph: tree.}
Unfortunately, the previous constraint \eqref{eq:respect_query_graph1} fails to encode the minimum for certain proper trees, which contain nodes with at least three different degrees. The simplest, problematic tree shape is represented in Figure \ref{fig:subgraph_to_terms} (node 2 has degree 3, node 3 has degree 2, and the others have degree 1). The problem is that with constraint \eqref{eq:respect_query_graph1}, not all the join trees have the same minimum energy due to nodes' different degrees. To address this problem, we develop an alternative constraint
\begin{equation*}
    \sum_{r = 1}^{|V| - 1} \left( r - \sum_{(i,j) \in E}\sum_{(i',j') \in E} x_{i,j}^{r}x_{i',j'}^{r}\right)^2,
\end{equation*}
where again we form the sum over the elements if the indices satisfy $|\left\{ i, j \right\} \cap \left\{ i', j' \right\}| = 1$. At every rank, this constraint selects $r$-many pairs of type $x_{i,j}^{r}x_{i',j'}^{r}$ to be true. This forces the returned join tree to respect the query graph and be connected. This constraint is slightly more complex than the previous constraints since it is not a quadratic but a higher-order constraint that includes terms of four variables. On the other hand, it does not introduce additional variables.


\textbf{Scaling cost and validity.} Finally, we have to scale cost HUBO $H_{\text{cost}}$ and validity objective $H_{\text{val}}$ properly so that we favor valid solutions over minimizing cost:
\begin{displaymath}
H_{\text{full}} = H_{\text{cost}} + CH_{\text{val}}.
\end{displaymath}
We noticed that a value that worked consistently in practice is $C = H_{\text{cost}}(x)$ where $x = (1, \ldots, 1)$ is a binary vector containing only ones and $H_{\text{cost}}$ is normalized so that the coefficients are in the interval $(0,1]$.
\section{Theoretical analysis}
We prove two theorems that give bounds for the quality of the solutions, which are expected to be reached with the cost functions. We have the same initial assumptions for both proofs, which we state next. We consider that dynamic programming (DP) and greedy algorithms perform as many steps as we perform joins. We assume the reader is familiar with these two algorithms but have included their definitions in the appendix. 

By Def.~\eqref{def:binary_variables} of the binary variables, the number of steps coincides with the rank index in the cost functions. Thus, we can prove the theorems by induction on the rank parameter $r$ in the proposed algorithms, as well as the steps in the DP and greedy algorithms, by showing that for each $r$, there exists a solution from HUBO that corresponds to the same plan computed by the classical algorithms.

\begin{theorem}\label{thm:dp_bound}
Let  $H_{\text{cost}}$ be the cost HUBO defined in Subsection \ref{subsection:cost_function} and let $H_{\text{val}}$ be the binary formulation for the join order validity constraints. Let $x$ be the point that minimizes the full problem $H_{\text{cost}} + CH_{\text{val}}$. Then, the join order cost $H_{\text{cost}}(x)$ is equal to the cost computed by the dynamic programming algorithm without cross-products.
\end{theorem}
\begin{proof}
First, assume that the rank is $0$, and we are at the first iteration in the dynamic programming (DP) algorithm. By Alg.~\ref{alg:hubo_term_construction} (line 5), we include all the joins between the leaf tables to the cost function $H_{\text{cost}}$. Considering the DP algorithm, we compute the same costs for joining the leaf tables and include the combinations and costs in the DP table. Thus, the HUBO encodes the same plans as the DP algorithm.

Let us assume that we are at rank $r > 0$, and we have applied Algorithm \ref{alg:hubo_term_construction} and the DP algorithm up to $r - 1$ steps. The DP algorithm considers all the intermediate joins from the previous step (i.e., at rank $r-1$). For each of these joins, it performs the possible join for each table that is not yet included in the intermediate result with respect to the query graph. These results are kept in the DP table for the next iteration. While the DP algorithm keeps the total cost of each intermediate result in the DP table, our cost objective $H_{\text{cost}}$ encodes only the ''local'' intermediate costs, which are computed with Eq.~\eqref{eq:term_coefficient}. Based on Algorithm \ref{alg:hubo_term_construction}, we take the terms of rank $r-1$ and compute the corresponding new terms (i.e., intermediate join plans) and coefficients (i.e., intermediate join costs) for each table that is not yet included in the intermediate result with respect to the query graph. Instead of computing the total cost, we add the terms and coefficients to the cost objective $H_{\text{cost}}$. This process encodes the same cost that is stored in the DP table because we can choose to activate those $H_{\text{cost}}$ terms, which produce the total cost for each value stored in the DP table. This leads to the claim that there is a point where $H_{\text{cost}}$ achieves the same cost as the DP algorithm.
\end{proof}

\begin{theorem}\label{thm:greedy_bound}
Let $H_{\text{cost}}$ be the heuristic method's cost HUBO defined in Subsection \ref{subsection:heuristic_cost_function}, and let $H_{\text{val}}$ be the binary formulation for the join order validity constraints. Let $x$ be the point that minimizes the full problem $H_{\text{cost}} + CH_{\text{val}}$. Let $C_{\text{greedy}}$ be the cost computed by the greedy algorithm without cross-products. Then, $H(x) \leq C_{\text{greedy}}$, i.e., the cost from the greedy algorithm gives an upper bound for the cost from the heuristic algorithm.
\end{theorem}
\begin{proof}
First, assume that the rank is $0$, and we are at the first iteration in the greedy algorithm. By the definition of the heuristic method in Subsection \ref{subsection:heuristic_cost_function}, we include all terms corresponding to the joins between leaf tables to the cost function $H_{\text{cost}}$. The greedy algorithm includes only one join with the minimum cost to the join tree at the same step. Thus, our heuristic method encodes the first step of the greedy algorithm.

Let us consider a general rank $r > 0$ and assume that we have applied the heuristic HUBO construction and the greedy algorithm up to $r - 1$ steps. By the definition of our heuristic method, we select a subset of $n$-many table combinations from the previous rank $r - 1$ with the smallest coefficients. Simultaneously, in the greedy algorithm, we compute the costs of joining the previous step's intermediate result with the possible joins that are left with respect to the query graph and keep the join tree with the minimum total cost. This minimum total cost is always achieved by including the join tree corresponding to the terms with the smallest coefficient in the $H_{\text{cost}}$ at rank $r$. Since the terms with the smallest coefficients correspond to the cheapest plans, the heuristic algorithm will encode the same plan that the greedy algorithm finds. For larger numbers of $n$, $H_{\text{cost}}$ will also include other intermediate results. Since our heuristic method and the greedy algorithm keep the join plans with the minimum cost at each step, we can deduce that our algorithm encodes the same plan (and others depending on value $n$) as the greedy algorithm. This leads to the claim that $H(x) \leq C_{\text{greedy}}$.
\end{proof}

Besides the theorems, our method archives advantageous variable scalability compared to the most scalable methods \cite{Schonberger_Scherzinger_Mauerer, 10.14778/3632093.3632112}. The model in \cite{DBLP:conf/q-data/SaxenaSS24} uses the same variable definitions as \cite{10.14778/3632093.3632112}, so the scalability comparison also applies to this paper as well. The scalability is visualized in Fig.~\ref{fig:variable_scalability_cycle} for cycle query graphs. The chain, star, and tree graphs have identical relative scalability in all three cases. We want to point out that these are the only mandatory variables for the two compared methods. The previous techniques require more variables to estimate the cost thresholds, which depend on the problem. We excluded the variable scalability of \cite{Nayak_Winker_Groppe_Groppe_2024} since the growth of variable count is exponential in their work.

\begin{figure}
    \centering
    \includegraphics[width=0.4\linewidth]{figures/variable_scalability_cycle.png}
    \caption{We compare the number of mandatory variables in \cite{Schonberger_Scherzinger_Mauerer,10.14778/3632093.3632112} to all variables in our optimization model.}
    \label{fig:variable_scalability_cycle}
    \Description[Comparison of variable scalability between this and previous methods]{The plot shows that our method has the best variable scalability compared to the previous methods.}
\end{figure}
This work identifies signal collapse as a critical bottleneck in one-shot neural network pruning. Performance loss in pruned networks is due to \textbf{signal collapse} in addition to the removal of critical parameters. We propose \textbf{REFLOW} (\textbf{Re}storing \textbf{F}low of \textbf{Low}-variance signals), a simple yet effective method that mitigates signal collapse without computationally expensive weight updates. By focusing on signal preservation, REFLOW highlights the importance of mitigating signal collapse in sparse networks and enables magnitude pruning to match or surpass state-of-the-art one-shot pruning methods such as CHITA, CBS, and WF.

REFLOW consistently achieves state-of-the-art accuracy across diverse architectures, restoring ResNeXt-101 from under 4.1\% to 78.9\% top-1 accuracy at 80\% sparsity on ImageNet. Its lightweight design makes it a practical solution for both research and deployment, delivering high-quality sparse models without the overhead of traditional approaches. These findings challenge the traditional emphasis on weight selection strategies and underscore the critical role of signal propagation for achieving high-quality sparse networks in the context of one-shot pruning.



\section*{Conclusion}
This paper aims to enhance our understanding of the computational complexity of computing various Shapley value variants. We found that for various ML models --- including decision trees, regression tree ensembles, weighted automata, and linear regression --- both local and global interventional and baseline SHAP can be computed in polynomial time under HMM modeled distributions. This extends popular algorithms, such as TreeSHAP, beyond their empirical distributional scope. We also establish strict complexity gaps between the various SHAP variants (baseline, interventional, and conditional) and prove the intractability of computing SHAP for tree ensembles and neural networks in simplified scenarios. Overall, we present SHAP as a versatile framework whose complexity depends on four key factors: \begin{inparaenum}[(i)] \item model type, \item SHAP variant, \item distribution modeling approach, \item and local vs. global explanations\end{inparaenum}. We believe this perspective provides deeper insight into the computational complexity of SHAP, paving the way for future work.




%We believe that our framework provides a more intricate understanding of SHAP computation complexity across different models, distributions, and variants, paving the way for further research.

Our work opens promising directions for future research. First, expanding our computational analysis to other SHAP-related metrics, such as asymmetric SHAP~\citep{frye20} and SAGE~\citep{covert2020understanding}, would be valuable. Additionally, we aim to explore more expressive distribution classes and relaxed assumptions beyond those in Section \ref{sec:tractable} while maintaining tractable SHAP computation. Finally, when exact computation is intractable (Section \ref{sec:intractable}), investigating the approximability of SHAP metrics through approximation and parameterized complexity theory~\citep{downey2012parameterized} is an important direction.

%Our work opens several promising avenues for future research on the computational properties of explainable AI methods, with a particular focus on SHAP. First, it would be interesting to broaden the computational analysis conducted in this work to include other popular SHAP-related metrics in the literature, such as asymmetric SHAP \cite{frye20} and SAGE \cite{covert2020understanding}. Also, in the future, we aim to explore more expressive distribution classes and relaxed distributional assumptions—extending beyond those examined in Section \ref{sec:tractable} —that still yield tractable SHAP computation. Finally, when exact computation proves intractable (Section \ref{sec:intractable}), it is worthwhile to theoretically investigate the question of the approximability of computing the SHAP metrics across various configurations, through the lens of approximation and parametrized complexity theory \cite{arora2009computational}.

%This paper aims to deepen our understanding of the computational complexity involved in obtaining different Shapley value variants. We found that for a variety of ML models, including decision trees, tree ensembles for regression, weighted automata, and linear regression models — computing both local and global interventional and baseline SHAP can be done in polynomial time when distributions are modeled by HMMs. This extends the distributional scope of popular algorithms like TreeSHAP, which is limited to empirical distributions. Additionally, we demonstrate a strict complexity gap between SHAP variants, showing that interventional and baseline SHAP can be strictly easier to compute than conditional SHAP. Despite these positive results, we uncovered intractability for various SHAP variants in neural networks and tree ensembles. Finally, we provided generalized complexity relations across SHAP variants. We believe that our framework offers a deeper understanding of the complexity involved in computing SHAP across various variants, models, distributions, as well as in both local and global computations, laying the groundwork for future research.

%%
%% The next two lines define the bibliography style to be used, and
%% the bibliography file.
\bibliographystyle{ACM-Reference-Format}
\bibliography{sample-base}


%%
%% If your work has an appendix, this is the place to put it.
\appendix
\section{Appendix: Background on quantum computing}

\subsection{Quantum circuit model}

In this part, we define the quantum circuit model \cite{Nielsen_Chuang_2010}. Classical computers operate with bits having discrete values of $0$ or $1$. Quantum computing is based on quantum bits, called qubits, that are formally represented as vectors in a complex-valued Hilbert space. For a single qubit system, the basis vectors of this space are $|0\rangle:= [1, 0]^{\top}$ and $|1\rangle:= [0, 1]^{\top}$. Every state in a single-qubit quantum computer can be represented as a complex-valued linear combination of these basis states as $|\varphi\rangle = \alpha |0\rangle + \beta |1\rangle$, where $\alpha, \beta \in \mathds{C}$ so that $|\alpha|^2 + |\beta|^2 = 1$. The values $\alpha$ and $\beta$ are also called amplitudes, and their squared lengths $|\alpha|^2 $ and $|\beta|^2$ can be interpreted as a probability distribution. Since a qubit is an element of a Hilbert space equipped with the standard tensor product, we can construct larger systems by applying tensor product operation between the smaller systems. For example, a two-qubit system is described by four basis states: $|0\rangle\otimes|0\rangle = |00\rangle$, $|01\rangle$, $|10 \rangle$ and $|11\rangle$.

The quantum algorithm is implemented by applying quantum logic gates to the qubits. These gates must respect the condition that the resulting quantum system has $\sum_{i}|\alpha_i|^2 = 1$ over the amplitudes $\alpha_i$. The operations which satisfy this property are the unitary matrices. A matrix $U$ over the complex numbers is unitary if its inverse is its conjugate transpose. The conjugate transpose is obtained by conjugating the matrice's complex-valued elements and then transposing the matrix. A quantum circuit represents a quantum computational system where the circuit's wires represent qubits, and operations are represented as gates acting on wires. 

The system is measured after the gates have been applied to the qubits. Measurement operation is especially characteristic of quantum computing and does not have a similar role in classical computing. Informally, measuring corresponds to the return statement at the end of a classical function. The most common measurement operation is measuring on a computational basis. This means that after a measurement, we obtain a classical bit string whose length is the number of qubits. For example, if we measure a 2-qubit system, the possible measurement results are $00$, $01$, $10$, and $11$. The connection to the amplitudes is the following: if the system has a state $\alpha_{11}|11\rangle$, then the value $|\alpha_{11}|^2$ is the probability that we obtain the result $11$ when we measure. 

In quantum computing, we can perform various measurements \cite{Nielsen_Chuang_2010}. In this work, we measure an expectation value of an observable, which is a typical measurement operation in practical applications. This measurement estimates the Hamiltonian's energy value corresponding to the cost we aim to minimize in the combinatorial optimization problem. The expectation value can be interpreted as the weighted average of all possible measurement outcomes, where each outcome is weighted by its probability. However, it does not necessarily correspond to the most probable measurement outcome. Formally, suppose $H$ is an operator such as a Hamiltonian. In that case, we compute $\langle \varphi | H | \varphi \rangle = \sum_{j}\lambda_j |\langle\varphi | \lambda_j\rangle|^2$, where $\lambda_j$ and $|\lambda_j \rangle$ are the eigenvalue and eigenvectors of the operator $H$ and the state $| \varphi \rangle$ is the state where the quantum mechanical system is before the measurement. Now the sum is the weighted average of the eigenvalues which are precisely the measurement results when measuring the observable: the eigenvalues $\lambda_j$ are the measurement outcomes, and $|\langle\varphi | \lambda_j\rangle|^2$ corresponds to the probability, which can be viewed as an overlap between the eigenstate $|\lambda_j \rangle$ and the system's current state $| \varphi \rangle$.

\subsection{QAOA and VQE}

On the gate-based universal quantum computers, we can apply the Quantum Approximate Optimization Algorithm (QAOA) \cite{farhi2014quantum} or Variational Quantum Eigensolver (VQE) \cite{Peruzzo_2014} to find the ground state of the Hamiltonian. Both QAOA and VQE are variational algorithms, meaning we execute them in two phases: first, we execute the circuit with a fixed parameter configuration on a quantum computer, then we estimate the circuit's gradient on a classical computer and tune the parameters. These two phases are repeated until a sufficiently good parameter configuration is found. 

First, we focus on QAOA. In this algorithm, we create two Hamiltonians: a mixer Hamiltonian $H_{\mathrm{mix}}$ and a problem Hamiltonian $H_{C}$. The most common mixer Hamiltonian is $H_{\mathrm{mix}} = \sum_{j}\sigma_{x}^{j}$. These Hamiltonians roughly correspond to the initial Hamiltonian and the problem Hamiltonian in Eq.~\eqref{eq:adiabatic_Hamiltonian}, and the cost Hamiltonian encodes the solution to the optimization problem. Then, we prepare two parametrized circuits for each Hamiltonian: $U_{C}:= e^{-i\gamma H_C}$ and $U_{\mathrm{mix}}:= e^{-i\alpha H_{\mathrm{mix}}}$. More concretely, implementing Hamiltonians as circuits, we follow the idea presented in \cite{Nielsen_Chuang_2010}. For example, if the Hamiltonian is $H = 2\sigma_{z}^{0} \otimes \sigma_{z}^{1} \otimes \sigma_{z}^{2}$, then the corresponding parametrized circuit is in Fig.~\ref{fig:circuit_example}. The important point is that we can natively encode and optimize HUBO problems using this method.
\begin{figure}
    \centering
    \scalebox{1.0}{
\Qcircuit @C=1.0em @R=0.2em @!R { \\
	 	\nghost{{q}_{0} :  } & \lstick{{q}_{0} :  } & \qw & \targ & \gate{\mathrm{R_Z}\,(\mathrm{2\gamma})} & \targ & \qw & \qw\\
	 	\nghost{{q}_{1} :  } & \lstick{{q}_{1} :  } & \targ & \ctrl{-1} & \qw & \ctrl{-1} & \targ & \qw \\
	 	\nghost{{q}_{2} :  } & \lstick{{q}_{2} :  } & \ctrl{-1} & \qw & \qw & \qw & \ctrl{-1} & \qw }}
    \caption{Circuit implementing the Hamiltonian $H = 2 \sigma_{z}^{0} \otimes \sigma_{z}^{1} \otimes \sigma_{z}^{2}$}
    \label{fig:circuit_example}
\end{figure}
Next, we define the complete QAOA circuit, which is a repeated application of circuits $U_C$ and $U_{\mathrm{mix}}$ for $n$ times. The precise value of $n$ depends on the problem, but already values such as $3$ have proved to be good \cite{farhi2014quantum}. Then, we prepare an equal superposition over all the basis states, choose random initial values for the parameters $\gamma$ and $\alpha$, and apply the complete QAOA circuit to the state. The equal superposition encodes that initially, every possible binary configuration has an equal probability of being selected as a solution. After preparing the circuit, we measure the expectation value of the cost Hamiltonian $H_C$. Then, we repeat the circuit execution with modified parameters $\gamma$ and $\alpha$. Since parametrized unitaries are differentiable, the circuits are also differentiable with respect to the parameters, and we can optimize the parameters $\gamma$ and $\alpha$ using classical stochastic gradient descent methods. The optimization goal is to find a parameter configuration that minimizes the expectation value of the Hamiltonian, which solves the combinatorial optimization problem.

The basic principles of VQE are similar to those of QAOA, except that the VQE circuit does not implement $U_C$ and $U_{\mathrm{mix}}$ circuits. The parameterized circuit in VQE is a sophisticated guess that is optimized to minimize the energy of the Hamiltonian using classical gradient descent methods. In VQE, we also measure the expectation value of the cost Hamiltonian $H_C$. QAOA and VQE have many modified versions that tackle challenges in the current noisy intermediate-scale quantum computing. 
\section{Appendix: HUBO to QUBO reduction}

Next, we describe how to reduce HUBO problems to QUBO problems. This rewrite process is necessary to utilize a wider variety of optimization platforms. In this work, we mainly rely on the D-wave's Ocean framework's utility of automatically translating HUBO problems into QUBO problems. The translation is based on a polynomial reduction by minimum selection or by substitution \cite{polynomial_reductions}.

The HUBO to QUBO reduction based on minimum selection \cite{polynomial_reductions} follows the scheme
\begin{equation*}
    xyz = \max_{w}w(x+y+z-2),
\end{equation*}
which iteratively replaces the higher order terms $xyz$ with lower order terms by introducing auxiliary binary variables $w$. Depending on the order in which variable terms are replaced, the QUBO formulation may vary in format, and the number of binary variables, but the minimum point remains unchanged.

Rewriting mechanism by substitution utilizes the following formula
\begin{equation*}
    xyz = \min_{w}\left\{wz + \mathrm{MP}(x, y ; w)\right\},
\end{equation*}
where $M > 1$  is a penalty weight and $P$ is a penalty function that has the lowest value when $w = xy$. The details of why these rewriting methods reach the same minimum are explained in \cite{polynomial_reductions}. These reduction methods provide a technically easy method to encode HUBOs as QUBOs, but they also introduce auxiliary variables depending on the number of higher-order terms.
\section{Dynamic programming and greedy algorithms}

The dynamic programming and greedy algorithms implemented here are based on \cite{Neumann_course}. Dynamic programming for join order selection comprises a general class of approaches to optimize the join order selection. In our work, we have fixed the cost function (Eq.~\eqref{eq:cost_function}) and employed the dynamic programming algorithm with and without cross-products. The algorithm without cross-products is presented in Alg.~\ref{alg:dp-join-order}. It relies on functions that create left-deep trees for trees $T_1$ and $T_2$ and return costs for join trees based on the cost function in Eq.~\eqref{eq:cost_function}.

\begin{algorithm}
\caption{Dynamic Programming for Join Order Optimization With Cross-Products}
\label{alg:dp-join-order}
\begin{algorithmic}[1]
\Require relations $R = \{r_1, r_2, \dots, r_n\}$, selectivities $S$
\Ensure optimal left-deep join tree in $\text{dp\_table}[R]$
\State initialize $\text{dp\_table}$
\For{$r \in R$}
    \State $\text{dp\_table}[\{r\}] \gets r$ \Comment{Base case: single relation}
\EndFor
\For{$s = 2 \ \mathbf{ to } \ |R|$} \Comment{Size of subsets from 2 to $n$}
    \For{$\text{subset} \subseteq R$ such that $|\text{subset}| = s-1$}
        \If{$\text{subset} \in \text{dp\_table}$}
            \For{$r \in R \setminus \text{subset}$} \Comment{Relations not in subset}
                \State $T_1 \gets \text{dp\_table}[\text{subset}]$
                \State $T_2 \gets \text{dp\_table}[\{r\}]$
                \State $\text{T}, \text{T\_cost} \gets \text{create\_join\_tree}(T_1, T_2, R, S)$
                \State $\text{join\_key} \gets \text{subset} \cup \{r\}$
                \If{$\text{join\_key} \notin \text{dp\_table}$}
                    \State $\text{dp\_table}[\text{join\_key}] \gets T$ \Comment{Update if key not in table}
                \EndIf
                \If{$\text{join\_key} \in \text{dp\_table}$}
                    \State $\text{prev\_cost} \gets \textbf{cost}(\text{dp\_table}[\text{join\_key}], R, S)$
                    \If{$\text{T\_cost} < \text{prev\_cost}$}
                        \State $\text{dp\_table}[\text{join\_key}] \gets T$ \Comment{Update if lower cost}
                    \EndIf
                \EndIf
            \EndFor
        \EndIf
    \EndFor
\EndFor
\end{algorithmic}
\end{algorithm}

The algorithm that computes the dynamic programming result without cross-products is similar except that for a query graph $G$, we change line 6: \textbf{for} connected subgraph $\subset G$ such that $|\text{subgraph}| = s - 1$ \textbf{do}. Then, the algorithm proceeds with the connected subgraphs of size $s - 1$ instead of all subsets of size $s - 1$.

The greedy algorithm is the other standard algorithm to optimize join order selection, and we represent it in Alg.~\ref{alg:greedy-join-order}. Similarly to the dynamic programming algorithm, we can consider only solutions without cross-products so that we iterate only over tables connected to one of the tables already joined. In other words, at each step, we compute a value called adjacent\_tables which contains those tables $R_i$ so that if edge $(R_i, R_j) \in G$ in the query graph $G$, then we require that $R_i \notin \text{joined\_tables}$ but $R_j \in \text{joined\_tables}$. 

\begin{algorithm}
\caption{Greedy Algorithm for Join Order Selection}
\label{alg:greedy-join-order}
\begin{algorithmic}[1]
\Require relations $R = \left\{ r_1, \ldots, r_n \right\}$, selectivities $S$
\Ensure Greedy join order tree

\State $\text{join\_result} \gets [r_i, r_j]$ so that $f_{i,j}|r_j||r_j|$ is the smallest

\For{$1 \ \text{to} \ |R| - 1$}
    \State $\text{min\_cost} \gets \infty$
    \State $\text{min\_table} \gets \text{None}$
    \For{$\text{table} \in \text{relations} \setminus \text{joined\_result}$} \Comment{Iterate over tables which are not joined}
        \State $\text{current\_join\_tree} \gets [\text{table}, \text{join\_result}]$
        \State $\text{current\_cost} \gets \textbf{cost}(\text{current\_join\_tree}, R, S)$
        \If{$\text{current\_cost} < \text{min\_cost}$} \Comment{Choose table with smallest cost}
            \State $\text{min\_cost} \gets \text{current\_cost}$
            \State $\text{min\_table} \gets \text{table}$
        \EndIf
    \EndFor
    \State $\text{join\_result} \gets [\text{min\_table}, \text{join\_result}]$
\EndFor
\end{algorithmic}
\end{algorithm}
\begin{table*}[h!]
\small
\centering
\caption{The performance comparison of PRMs. The numbers represent the percentage of problems solved using verifiers. For all BoN experiments, we use the final state value to re-rank solution candidates following the definition of state value in LLM reasoning scenario.}
\label{tab:experiment-results}
\begin{tabular}{l|ccccc|cc}
\toprule
\multirow{2}{*}{Method}                           & \multicolumn{5}{c|}{Best-of-N}    & \multicolumn{2}{c}{Beam Search}   \\
% \multirow{2}{*}{Method}                           & \multicolumn{5}{c|}{Best-of-N}    & Beam Search   \\
                                & 8     & 16    & 32    & 64       & 128         & 4 & 8             \\ \midrule
Oracle                          & 29.8  & 35.8  & 40.4     & 45.7     & 49.6   & - & -                   \\
\rowcolor{gray!12}
\multicolumn{8}{c}{\textbf{Qwen2.5-Math-7B-Instruct}} \\
Scalar Regression (outcome)     & 24.4  & 28.2  & 29.4  & 30.0  & 31.2    & 44.2 & 45.8              \\
Scalar Regression (process)     & 24.4  & 28.2  & 29.0  & 29.8  & 30.6     & 51.8 & 56.0              \\
Expectation Regression (MSE)    & 24.4  & 27.8  & 29.0  & \textbf{30.6}  & \textbf{31.8}    & \textbf{52.0} & \textbf{56.4}               \\
Expectation Classification (HL) & \textbf{24.8}  & \textbf{28.4}  & \textbf{29.6}  & 30.4  & \textbf{31.8}   & 51.8 & 56.0                 \\ \midrule
\rowcolor{gray!12}
\multicolumn{8}{c}{\textbf{Deepseek-math-7b-instruct}} \\
Scalar Regression (outcome)     & 11.8  & 12.6  & 12.0  & 11.6   & 13.2    & 17.0 & 14.0              \\
Scalar Regression (process)     & 21.0  & 23.2  & 23.0  & 23.6  & 24.4     & 43.6 & 46.2              \\
Expectation Regression (MSE)    & 22.0  & \textbf{23.4}  & 25.0  & \textbf{25.8}  & 25.8     & \textbf{44.0} & 46.0                \\
Expectation Classification (HL) & \textbf{22.4}  & \textbf{23.4} & \textbf{25.4}  & 25.4  & \textbf{26.4}     & 42.6 & \textbf{47.0}              \\ \midrule
\end{tabular}
\end{table*}

\section{Appendix: Comparison with quantum-inspired digital annealing}

We evaluated the method proposed in \cite{10.14778/3632093.3632112} with the same workloads we used and present the results in Figures \ref{fig:vldb24_chain}, \ref{fig:vldb24_clique}, \ref{fig:vldb24_cycle}, and \ref{fig:vldb24_star}. The method is the improved algorithm from \cite{Schonberger_Scherzinger_Mauerer}. The authors propose a novel readout technique that improves the results. Since we could not access special quantum-inspired hardware, such as a digital annealer, we used the Gurobi solver, which returns only a single result by default. Thus, the readout technique was not applicable. 

Nevertheless, the results still demonstrate that the method reaches a comparable accuracy to ours, which is optimal or close to optimal. Their method seems to be able to identify beneficial cross products. We have computed the exact results with dynamic programming and compared relative cumulative costs between the methods. Due to higher-order terms in our method, which currently have to be rewritten into quadratic format, their method is still more scalable than ours. On the other hand, our theoretical bounds, the more straightforward variable definitions, and the novel usage of the higher-order model show specific improvements over their methods. We are also positive that our model admits features that make it easier to expand for outer joins and include more complex dependencies between the predicates.

\begin{figure}[!ht]
\centering
\begin{subfigure}{.5\textwidth}
  \centering
  \includegraphics[width=.95\linewidth]{results/vldb24/chain_gurobi.png}
  \caption{Chain graphs with varying number of nodes solved with QUBO formulation proposed in \cite{10.14778/3632093.3632112}}
  \label{fig:vldb24_chain}
\end{subfigure}%
\hfill
\begin{subfigure}{.5\textwidth}
  \centering
  \includegraphics[width=.95\linewidth]{results/vldb24/clique_gurobi.png}
  \caption{Clique query graphs with varying number of nodes solved with QUBO formulation proposed in \cite{10.14778/3632093.3632112}}
  \label{fig:vldb24_clique}
\end{subfigure}
\caption{A figure with two subfigures}
\label{fig:test}
\end{figure}

\begin{figure}[!ht]
\centering
\begin{subfigure}{.5\textwidth}
  \centering
  \includegraphics[width=.95\linewidth]{results/vldb24/cycle_gurobi.png}
  \caption{Cycle query graphs with varying number of nodes solved with QUBO formulation proposed in \cite{10.14778/3632093.3632112}}
  \label{fig:vldb24_cycle}
\end{subfigure}%
\hfill
\begin{subfigure}{.5\textwidth}
  \centering
  \includegraphics[width=.95\linewidth]{results/vldb24/star_gurobi.png}
  \caption{Star query graphs with varying number of nodes solved with QUBO formulation proposed in \cite{10.14778/3632093.3632112}}
  \label{fig:vldb24_star}
\end{subfigure}
\caption{A figure with two subfigures}
\label{fig:test}
\end{figure}

\end{document}
\endinput
%%
%% End of file `sample-acmsmall.tex'.
