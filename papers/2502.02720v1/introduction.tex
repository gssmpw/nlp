\IEEEPARstart{C}{loud} computing provides access to a pool of shared configurable resources such as datacenters, software, and infrastructure. Organizations utilize cloud computing to save the cost of acquiring and maintaining computing resources on-premise. The main feature of cloud computing is the ability to dynamically assign and release physical and virtual resources to customers based on their demand. However, security challenges present a major obstacle towards a complete shift to cloud computing paradigm \cite{DBLP:journals/cit/HuQLGTMBH11}. Cloud Security Alliance (CSA)\footnote{https://downloads.cloudsecurityalliance.org/assets/research/top-threats/Treacherous-12\_Cloud-Computing\_Top-Threats.pdf} identifies the top security threat in cloud computing as the breach of confidential data stored on the cloud. Such breach can be resulted due to human error, poor security practice, or software vulnerabilities. A recent example of such data breach is the leakage of US voter data, which was stored on a data warehouse service hosted by Amazon S3. The breach was caused by a misconfiguration of the database leaving the data exposed to the public \cite{upguard2017}. 

Data sharing among authorized users remains the main security concern among cloud tenants \cite{takabi2010security}. To mitigate the risk of data leakage, encryption-based techniques are used to preserve the confidentiality of the data stored on the cloud \cite{tang2016ensuring}. Recently, encryption primitive and homomorphic encryption techniques have been proposed to enable secure data search and computation on encrypted data. However, such techniques are computationally intensive and support limited operations. On the other hand, non-encryption based approaches to reduce the risk of data leakage by implementing resource isolation mechanisms, e.g., trusted virtual domain \cite{catuogno2010trusted}, secure hypervisor \cite{steinberg2010nova}, and Chinese wall policies \cite{berger2009security}. However, achieving resource isolation reduces resource utilization. 

In datacenter applications, the confidentiality of the shared data is often preserved using authorization mechanisms to manage data access \cite{alcaraz10}. In essence, cloud providers controls data operations using Role-Based Access Control (RBAC) policy, in which a user can perform an action on the shared data if the user is assigned a role and the role has the authorization to perform the action. For a particular datacenter application, cloud providers allocate a number of virtual resources, i.e., Virtual Machines (VMs), that are either hosted in a single or multiple Physical Machines (PMs) such that cost of resource provisioning is minimized. A user attempting to access the datacenter using an access control role is assigned to a particular VM subject to the Service Level Agreement (SLA). From security perspective, it is imperative that the assignment of virtual resources takes into account the risk associated with the data leakage resulted from assigning roles to VMs. In \cite{almutairi2014risk,almutairi2015risk}, risk-aware assignment techniques have been proposed with the objective to minimize the data leakage. 

In general, data contains embedded properties that can be extracted using data mining and information retrieval techniques. For example, social network dataset generally contains patterns that represent relationships among users \cite{cho2011friendship}, enterprise dataset contains association and business rules \cite{verykios2004association}, and healthcare data may contain infectious disease outbreak behavior \cite{lu2010prospective}. Some embedded properties can be sensitive and need to be preserved from authorized users. Furthermore, colocating roles in a single PM can increase the risk of disclosing the sensitive properties due to the risk of resource sharing among VMs. In this paper, we propose an efficient resource allocation technique for cloud datacenter applications with the aim to mitigate the risk associated with disclosing the sensitive properties. In particular, we assume that global sensitive properties are defined over the entire dataset, and each individual role perceives local sensitive properties associated with the role's permissible partial dataset. The local sensitive properties may disclose partial information about the global sensitive properties. To evaluate the information disclosure of the sensitive properties, we use information-theoretic measures to quantify the difference between global and local sensitive properties. 

\textbf{Motivating example:} Consider a cloud datacenter hosting a location-based social network application that allows to share information about the location and time of users' check-in activities. Furthermore, the access to check-in data is controlled using a fine-grained access control policy, e.g., Context-aware RBAC (CRBAC). In CRBAC, the time and location are used to define the context at which the access to check-in entries is controlled. Figure \ref{ch6:fig:map} depicts an example of a CRBAC policy with 5 roles with each role  is represented as a rectangle in the map. The rectangle boundary of the roles defines the spatial extent of the check-in entries during the last 24 hours. Several sensitive properties associated with the check-in data can be envisioned based on the demographics \cite{dong2014inferring} or commute distance \cite{cho2011friendship}. For example, the rate at which users visit a certain type of locations, e.g., restaurants, theaters, etc, during certain time of the day, e.g., morning, afternoon, or evening, is a sensitive property. Such a property can be modeled as a statistical function, e.g., probability mass function (pmf). For each role in Figure \ref{ch6:fig:map}, a local pmf of the sensitive property is computed and used to estimate the global pmf defined over the entire dataset. Subsequently, a role can exploit the vulnerabilities in the cloud architecture and attack the dataset assigned to other roles to gain more information about the global pmf. Any gain about the global pmf is a risk that needs to be controlled. 
\begin{figure}[t!]
\centering
\includegraphics[width = 0.5\textwidth]{fig/mapExample.pdf}
\caption{Example of check-in data controlled by a CRBAC policy}
\label{ch6:fig:map}
\end{figure}


The contributions of this paper are as follows. First, we introduce a virtual resource management architecture that facilitates the assignment of virtual resources with objective to reduce the risk of data leakage. Accordingly, we formulate the risk-aware assignment problem that protects the global sensitive properties of the dataset. The problem is shown to be NP-complete and subsequently two heuristics are proposed. Finally, experimental evaluations of the proposed heuristics is conducted using real check-in dataset.

The remainder of this paper proceeds as follows. In Section 2, the vulnerability and authentication policy models are presented. Section 3 presents the risk-aware sensitive property-driven assignment problem. Section 4 presents the assignment heuristics. In section 5, we provide performance evaluation. Section 6 discusses related work. Finally, Section 10 outlines the conclusion.
