In this section, we introduce three metrics to compare the performance of the proposed BFH and TDH. The first two metrics are concerned with the overall disclosure risk exhibited by all roles. The third metric focuses on the disclosure risk incurred per role. The formal definitions of these metrics are presented in this subsection and the experiment results are given in the next section. 

The  disclosure risk associated with sensitive property is the main metric that needs to be minimized and is computed over all roles. This metric depends on the  sensitive property function used in the cost function. The risk  metric associated  with $Div$ property, represented  as $RISK_{Div}$, measures the risk incurred due to the reduction in the distance between the attacker p.m.f. and the global p.m.f.. In essence, $Div$ indicates how close is the attacker's p.m.f. to the global p.m.f.. The risk metric corresponding to $MI$, denoted as $RISK_{MI}$, measures the risk incurred due to the reduction in the difference between the estimated $MI$ by a role and the global $MI$. Formally, these risk metrics are given as:\\
\begin{definition}
For the cost function associated with  $RISK_{Div}$ as per definition of Equation \ref{eq:cost}., we define the measure of disclosure  $g_i^A$ as following: \\
\begin{eqnarray*}
g_i^A &= &|D(P_A(X,Y) \parallel P_G(X,Y))\\
& -&D(P_{r_i}(X,Y) \parallel P_G(X,Y)) |
\end{eqnarray*}
Where:\\\
$P_A(X,Y)$: is the joint probability mass function  calculated based on data accessed by  the set of roles in $A$.\\
$P_G(X,Y)$: is the global  joint probability mass function.\\
$P_{r_i}(X,Y)$: is the joint probability mass function  for role $r_i$.
\end{definition}
Note, $f(A)$ is $D(P_A(X,Y) \parallel P_G(X,Y)) $ as per Definition \ref{def:SPRAP}. 

\begin{definition}
For the cost function associated with $RISK_{MI}$ as per definition of Equation \ref{eq:cost}, we define the measure of disclosure  $g_i^A$ as following: \\
\begin{eqnarray*}
g_i^A &= &| MI_A(X,Y) -MI_G(X,Y)|\\
&   - & | MI_{r_i}(X,Y) -MI_G(X,Y)|  |
\end{eqnarray*}
Where:\\
$MI_A(X,Y)$: is the mutual information between random variables. $X$ and  $Y$ calculated based on data accessed by the  set of roles in $A$.\\
$MI_G(X,Y)$: is the global mutual information between  $X$ and  $Y$.\\
$MI_{r_I}$: is the mutual information between  $X$ and $Y$ for role $r_i$.
\end{definition}
Note, $f(A)$ is $MI_A(X,Y) $ as per Definition \ref{def:SPRAP}. 

For an RBAC policy we define the \textit{Property Attackability} ($PA$) parameter as a measure corresponding to the maximum  risk associated with disclosure of the global property/knowledge for a given policy. It is given as $PA = \sum_{r_i \in R} f(r_i)  $. Accordingly, we can define the \emph{quality of risk reduction} $(\Delta)$ as  the second metric which describes the  relative difference between  $PA$  and the risk $RISK_{Div}$ and $RISK_{MI}$ associated  with the properties $Div$ and $MI$, respectively. 

Formally,
\begin{equation}
\label{eq:QoR}
\Delta_{Div} = \frac{(PA - RISK_{Div})}{PA}
\end{equation}
\begin{equation}
\label{eq:QoR}
\Delta_{MI} = \frac{(PA - RISK_{MI})}{PA} 
\end{equation}

Note, $ 1 \le \Delta \le 0$.

We define the  third metric know as  \textit{Discriminator Index} ($DI$)  which is concerned with the performance of heuristics at the role level.  In particular, $DI$  is an indirect indicator of performance of heuristic in term of risk reduction per role. Semantically, $DI$ provides  information about the  quality of reduction provided by a heuristic at  the role level, and indicates disproportionality in terms of risk management over all roles. Formally, it is defined as following: 
\begin{definition}
Given a scheduling heuristic, DI for a heuristic can be formally defined by extending the definition of generic discriminator index in \cite{jain1984quantitative}. For this purpose, given a role  $r_i$, we define the property attackability ($PA_i $), the associated risk ($RISK(r_i) $), and the role quality of risk reduction $\Delta_i$ as follows:\\
\begin{eqnarray*}
PA_i &=& f(r_i)\\
\Delta_i &= &(PA_i - RISK(r_i))  
\end{eqnarray*}
Accordingly, DI is defined as 
\begin{equation}
\label{eq:DI}
 DI = 1- \frac{(\sum_{i=1}^n \Delta_i)^2}{n\times \sum_{i=1}^{n}(\Delta_i)^2} 
\end{equation}
\end{definition}


