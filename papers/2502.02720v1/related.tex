Work related to this paper can be categorized into three categories: 1) access control in cloud datacenter, 2) vulnerability models in cloud datacenter, and 3) security-aware resource assignment in cloud datacenter.


First, several researchers have addressed access control issues for cloud computing. Nurmi et al. \cite{nurmi2009eucalyptus} provided an authorization system to control the execution of VMs to ensure that only administrators and owners could access them. Berger et al. \cite{berger2009security} promoted an authorization model based on both RBAC and security labels to control access to shared data, VMs, and network resources. Calero et al. \cite{alcaraz10} presented a centralized authorization system that provides a federated path-based access control mechanism. What distinguishes our work is that we address the problems of virtual resource vulnerability in the presence of multitenancy and virtualization. 

Second, vulnerability discovery models (VDMs) are used to estimate the probability of software vulnerability. The authors in \cite{manadhata2011attack} proposed an attack surface metric that measures the security of software by analyzing its source code and find the potential flows. In \cite{alhazmi2007measuring} the vulnerability discovery is modeled based on the empirical analysis of history of actual vulnerability dataset and followed by predicting the vulnerability discovery. A Static-analysis of vulnerability indicator tool to assess the risk of software built by external deleopers is proposed in \cite{walden2012savi}, while \cite{shar2015web} proposes a hybrid (static and dynamic) analysis and machine learning techniques to extract vulnerabilities in web applications.  In \cite{massacci2014empirical}, an empirical methodology to systematically evaluate the performance of VDMs is proposed. The source code quality and complexity are incorporated in the VDM to generate better estimation \cite{rahimi2013vulnerability}. Common Vulnerability Scoring System (CVSS) \cite{mell2007complete} metrics have been extensively used by both industry and academia to score the vulnerabilities of the systems and estimate risks. The  US National Vulnerability Database (NVD) provides a catalog of known vulnerabilities with their CVSS scores. 

Finally, various security-aware scheduling techniques for cloud computing have been reported. In \cite{wen2016cost}, an algorithm to deploy workflow application over federated cloud is proposed. The proposed algorithm guarantees reliability and security constraints while optimizing the monetary cost. In particular, a workflow application is distributed on two or more cloud platforms in order to benefit from the advantage of each cloud. In order to prevent from unauthorized information disclosure of the deployed workflow over the federated cloud, a multi-level security model is used. In \cite{chen2017scheduling}, a task-scheduling technique for security sensitive workflows is proposed. The proposed technique addresses the problem of scheduling tasks with data dependencies, i.e., sensitive intermediate data. Authors presented theoretical guidelines for duplicating workload tasks in order to minimize the makespans and monetary costs of executing the workflows in the cloud. In \cite{chase2017scalable}, authors presented an approach for optimal provisioning of resources in the cloud using stochastic optimization. The provisioning technique takes into account future pricing, incoming traffic, and cyber attacks. The above work do not consider the data confidentiality of application or the access control policy, which restricts permissions for a given task. A key feature that differentiates our scheduling algorithms from other approaches is that the privilege set for each task is the main parameter in the scheduling decision. The objective of our scheduling algorithm is to minimize the risk of data leakage due to vulnerability of virtual resources in cloud computing environments.


%To ensure the security of tasks, the intermediate data is encrypted before transmitting among VMs or storing to the cloud. However, data encryption inevitably increases time overhead. To reduce such time overhead, the proposed technique duplicates the task's predecessor tasks.

%
%
%
%
%In \cite{qin2008availability} the authors proposed an availability-aware secure scheduling algorithm for multi-classes applications with heterogeneous availability and performance requirements. The proposed algorithm provides high availability with the cost of increasing the response time. In \cite{xiaoyong2011novel}, the authors proposed a secure scheduling algorithm based on the trustworthiness measurements of the computing node in computer grid system. The application security overhead and risk probability are computed and then incorporated in the scheduling  decision. In addition, a risk-resilient scheduling algorithm was introduced in \cite{song2006risk } which ensures secure grid job execution by requiring grid node security index to be higher than user job’s trust requirement. In \cite{xie2008security}, the authors presented security-aware resource allocation for real-time applications in the case }
