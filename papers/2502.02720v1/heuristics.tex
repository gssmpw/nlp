In this section, we propose two heuristics for solving RKAP which can be used by the resource assignment component in VRM in Figure \ref{fig:arch}. The first heuristic follows a top-down clustering based approach. In each step, the cluster of roles is divided based on the risk associated with the disclosure of the sensitive property. The second heuristic is a neighbor-based heuristic, which uses a pairwise property disclosure measure for role scheduling. This measure is computed based on the inference function $f$. Each role is assigned to the best available virtual resource with respect to the probability of leakage.

\subsection{Top-Down Heuristic (TDH)}
In  this heuristic, a top down clustering approach is used. Initially, all the roles are assumed to be in one cluster. We begin with division of this root cluster and split it into two clusters such that total measure of property disclosure $g_i^A $ of both clusters is minimum. The resulting clusters are sorted based on their total $g_i^A$. TDH splits a cluster with the largest total $g_i^A$ further by moving roles with high value of $g_i^A$ from current cluster to the new cluster such that the total $g_i^A$ of both clusters is minimum. TDH repeats the cluster  splitting  step until it generates up to $m$ clusters that equals  the number of VMs. Subsequently, the cluster with the largest total $g_i^A$ is assigned to the VM with the minimum probability of leakage. After this initial assignment, TDH iterates over all the roles and changes the assignment of a role if it results in reduction of the total risk. The algorithm stops when any futher change in the assignment does not reduce the total  risk. 

 Algorithm \ref{ch6:alg:tdh}, formally represents the TDH. In Lines 1-2,  the initial cluster list has one cluster which is the set of all roles. In Lines 3-13,  TDH iterates the outer loops $m$ times and in each iteration the first cluster in  $\mathcal{C}$ is divided  into two new clusters $C1$ and $C2$. In the inner loop Lines 7-11, TBH moves the role $r_i$ from the current cluster $C1$ to the new cluster $C2$. The condition in Line 8 guarantees that this move does not increase the total  risk.  In Lines 12-13, TDH adds the new clusters to $\mathcal{C}$ and sorts the clusters based on  their  total $g_i^A$. Then,  the VMs  are sorted based on the intra probability of leakage in Line 14. The initial assignment is generated in Lines 15-17 by iterating over all the sorted VMs and the resulting sorted clusters.  TDH in Lines 18-22 iterates over all roles to improve the assignment. The final the assignment is stored in assignment matrix $I$.

\begin{lemma}
The  complexity of TDH is $O(n^3 \times m^2 )$
\end{lemma}
\begin{proof}
In TDH we need to generate $m$ clusters. For each of the cluster, the algorithm iterates over at most $n$ roles. For each of the roles the algorithm evaluate function  $f$ to compute the total disclosure risk of a cluster as illustrated in Line 8 in Algorithm \ref{ch6:alg:tdh}. The computation of $f$ has the complexity of $n^2$. Therefore, the complexity of TDH for computing the initial assignment is $O(n^3 \times m )$. To improve the assignment the algorithm iterates $n \times m$ times until no further improvement can be achieved. Each iteration requires $n^2\times m$ steps to calculate the total risk. Accordingly the complexity of the improvement step  is $O(n^3 \times m^2 )$ which results in the overall  complexity of TDH which is $O(n^3 \times m^2 )$.
  \end{proof}

\begin{algorithm}[t!]

\small
\SetNlSty{normal}{}{.}
\KwIn{sensitive property function $f$, vulnerability matrix $\mathcal{D}$.}
\KwOut{An assignment $I$ of roles to VMs.}
    Let $C1=\{R\}$ be the initial cluster\;
    Let $\mathcal{C} = \{C1\}$ be the set of clusters\;
   \While{$|\mathcal{C}| \neq m$}{
   	Let $C1 = \mathcal{C}[0]$ be the cluster with maximum property disclosure\;
	Let $C2 = \{\}$\;
	Let $dis = f(C1)$\;
		\ForEach{$r_i \in C1$}{
			\If{$f(C1-r_i) + f(C2 \cup r_i) < dis$}{
				$C1 = C1 - r_i$ \;
				$C2 = C2 \cap r_i$ \;
				$dis = f(C1) + f(C2)$\;
			}	
		}
	$\mathcal{C} = \mathcal{C} \cup C2$\;
	Sort $\mathcal{C}$ based on disclosure risk\;	
  }
 Let $B$ be the sorted list of VMs based on $d_{q,q}$\;	
 \ForEach{$i =1, \dot,m $}{
  	\ForEach{$r \in \mathcal{C}[i]$}{
		$I(r) = B[i]$\;  	
	}
} 
\ForEach{$r_i \in R$}{
	Let $t$ be the current risk for $r_i$\;
	\ForEach{$v_q \in V$}{
		\If{risk of $r_i$ when $I(r_i) = v_q$ less than $t$}{
			 $I(r_i) = v_q$\;
		} 
	}
} 
 return $I$\;

\caption{TDH}
\label{ch6:alg:tdh}
\end{algorithm}		

\subsection{Neighbor-Based Heuristic (NBH)}
The NBH algorithm is shown as Algorithm \ref{ch6:alg:nbh}. Prior to the assignment, NBH computes disclosure weights for each pair of roles as shown on Line  2-4 of the Algorithm \ref{ch6:alg:nbh}. These weights are based on the sensitive property function $f$ discussed in Section \ref{sec:SP}.  Subsequently, NBH follows a best fit strategy  whereby it initially selects the pair of roles  $(r_i,r_j)$ with the maximum weight and assigns them to the pair of VMs $(v_l,v_q)$  that has the least probability of leakage. It then selects the role $r_k$ that has the highest disclosure weight with any of perviously assigned roles.  $r_k$ is then assigned to  the VM that has the minimum probability of leakage to an already selected VM.  This step is repeated  until each VM is assigned exactly one role. The remaining $(n-m)$  roles are assigned to VMs based on the value of intra probability of leakage. Th reason for only considering intra probability is that we assume  that the probability of leakage  across VMs is always less than probability of leakage within a VM. 

The assigned roles are stored in list $A$. The unassigned  roles are kept in the list $F$. The list $G$ maintains IDs of the unassigned VMs. In Lines 12-18 , each iteration of  the loop assigns a role from list $F$ with the maximum value in matrix $C$ to a VM in $G$. Then the algorithm updates the assignment matrix $I$ and the lists $A$, $F$, and $G$. The outer loop in Lines 19-23 iterates over all the remaining unassigned roles to find the current best assignment. To find the best assignment for a given role in $F$, the inner loop (Lines 20-21) iterates over all VMs  and identifies the  VM that results in the minimum risk for that role. At the end NBH returns the assignment matrix.

\begin{lemma}
The  complexity of NBH is $O(n^2 \times m)$
\end{lemma}
\begin{proof}
The complexity of computing  $B_{i,q} $ in Line 22 of the NBH is $n \times m$. This  computational cost needs to be repeated $n$ times for each role in $F$. Therefore, the total complexity of NBH is  $O(n^2 \times m)$.

\end{proof}

\begin{algorithm}[t!]
\small
\SetNlSty{normal}{}{.}
\KwIn{sensitive property function $f$, vulnerability matrix $\mathcal{D}$.}
\KwOut{An assignment  $I$ of roles to VMs.}
   Let $C_{i,j}$ be the increase  in the measure of property disclosure due to the pair $(r_i,r_j)$ \; 
    \ForEach {$r_i \in R$}{
	\ForEach {$r_j \in R$}{
		$C_{i,j} =|f(r_i,r_j) - f(r_i)| +|f(r_i,r_j) - f(r_j)|$\;
		}
	}
    Find initial pair of vms   $(v_q,v_l)$ with minimum $d_{q,l}$\;
    Find initial pair of roles  $(r_i,r_j)$  with largest  $C_{i,j}$\;
    $I(r_i)=v_q$ and $I(r_j)=v_l$\;
     Let $A =\{ r_i,r_j\}$ be the set of assigned roles\;
     Let $F= R-\{r_i,r_j\}$ be the set of free roles\;
    %  Let $B = \{ v_q,v_l\}$ be the set of assigned VMs\;		
     Let $G= V-\{v_q,v_l\}$ be the set of free VMs\; 
     \While {G $\neq \phi$}{
     	Let $(r_i,r_j)$ be maximum $C_{i,j}$ where $r_i \in A$ and $r_j \in F$\;
	Find $(v_q,v_l)$ with minimum $d_{i,j}$ where $v_l\in F$  and  $I(r_i)=v_q$\;
	$I(r_j)=v_l$\;
	$A =A\cup \{ r_j\}$\;
	 $F =F -\{ r_j\}$\;
	 $G =G -\{ v_l\}$\;
	}
	 \ForEach {$r_i \in F$}{
	 	 \ForEach {$v_q \in V$}{
		 	\ForEach{$r_j \in A$}{
		 		Let $B_{i,q} =  C_{i,j} \times d_{q,q} $ such that $I(r_j)=v_q$\;
			}
		}
		 Let $r_i,v_q$ be the minimum  $B_{i,q}$\;
		$I(r_i)=v_q$\;
		$A =A\cup \{ r_i\}$\;
		 $F =F -\{ r_i\}$\;
	 }
  return $I$\;

\caption{NBH}
\label{ch6:alg:nbh}
\end{algorithm}



%\subsection{Performance Evaluation  Metrics for Resource Assignment}
%We introduce three metrics to compare the performance of the proposed NBH and TDH. The first two metrics are concerned with the overall disclosure risk exhibited by all roles. The third metric focuses on the disclosure risk incurred per role. The formal definitions of these metrics are presented in this subsection and the experiment results are given in next section. 
%
%The  sensitive property disclosure risk is the main metric that needs to be minimized and is computed overall roles. This metric depends on the  sensitive property function used in the cost function. In our evaluation, we used two sensitive property functions named \textit{divergence} $(Div)$ and \textit{mutual information} $(MI)$ as discussed in Section \ref{}. The risk  metric according to $Div$, noted as $RISK_{Div}$, measures the reduction in the distance between the attacker pmf and the sensitive pmf. The risk metric corresponding to $MI$, noted as $RISK_{MI}$, measure the reduction in the difference between the role view and the global $MI$. Formally, the risk metric is given as:\\
%\begin{definition}
%The \textit{divergence} risk $RISK_{Div}$ is defined as in Equation \ref{} where $g_i^A$ is as following: \\
%$g_i^A = |D(P_A(X,Y) \parallel P_R(X,Y)) -D(p_{r_i}(x,y) \parallel p_R(x,y)) |$\\
%Where:\\\
%$P_A(X,Y)$: is the joint probability mass function (pmf) of the set of roles in $A$
%\end{definition}
%
%\begin{definition}
%The \it{mutual information}  risk $RISK_{MI}$ is defined as in Equation \ref{} where $g_i^A$ is as following: \\
%$g_i^A = | I_A(X,Y) -I_R(X,Y)|  | - | I_{r_i}(X,Y) -I_R(X,Y)|  |$\\
%Where:\\
%$I_i^A(X,Y)$: is the mutual information between r.v. X and r.v. Y given the set of roles on $A$.
%\end{definition}
%
%The Attackability $AT$ measures the maximum mount of  property disclosure risk for a given policy. The second metric is the  relative difference between the Attackability of a policy and the resulting  risk represents the \emph{quality of risk reduction} $(\delta)$
%
%\begin{equation}
%\label{ch6:eq:QoR}
%\delta = \frac{(AT - RISK)}{AT}, \: \text{where}\: AT = \sum_{r_i \in R} f_{r_i}  
%\end{equation}
%
%The third metric is concerned with performance of heuristics at the role level.  One such metric is the Discriminator Index (DI)  which is an indirect indicator of performance of heuristic in term of risk reduction per role. Semantically, DI not only implies the role quality of reduction but also disproportionality in terms of risk management i.e. a heterogeneous bias in managing the risk. Formally, it is defined as following: 
%\begin{definition}
%Given a scheduling heuristic, DI for a heuristic can be formally defined by extending the definition of generic discriminator index in \cite{jain1984quantitative}. For this purpose, given a role  $r_i$ we define the role attackability ($AT_i $), the role  risk ($RISK(r_i) $), and the role quality of risk reduction $\delta_i$.  Let:\\
% $AT_i = f_{r_i}$ \\  
%$\delta_i = \frac{(AT_i - RISK(r_i))}{AT_i}$\\ 
%Accordingly, DI is defined as 
%\begin{equation}
%\label{ch6:eq:DI}
% DI = 1- \frac{(\sum_{i=1}^n \delta_i)^2}{n\times \sum_{i=1}^{n}(\delta_i)^2} 
%\end{equation}
%\end{definition}


