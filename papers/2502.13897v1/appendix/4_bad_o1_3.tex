\onecolumn
\begin{tcolorbox}[colback=blue!5!white, colframe=blue!75!black, title=Failed Example \#3 Generated by o1-mini, text width=\textwidth]
\textbf{Prompt}:\\
Perform K-Means clustering on the given DataFrame using the sklearn KMeans algorithm. The function expects a DataFrame with numerical values, as KMeans cannot handle categorical data. It applies standard KMeans clustering from the sklearn library to form clusters. The number of clusters is configurable via the 'n\_clusters' parameter, defaulting to 3. The Number of times the k-means algorithm is run with different centroid seeds (n\_init) is set to 10. The function returns an array of cluster labels corresponding to each data point in the input as well as the fitted KMeans model. >>> data = pd.DataFrame({ ... 'a': [1, 20, 2, 22, 100], ... 'b': [1, 20, 2, 22, 100] ... }) >>> labels, model = task\_func(data, seed=213) >>> print(labels) [2 0 2 0 1] >>> print(model) KMeans(n\_clusters=3, n\_init=10, random\_state=213). The function should raise the exception for: ValueError: If the DataFrame contains non numeric entries. The function should output with: tuple(labels: numpy.ndarray, model: sklearn.cluster.KMeans) labels: An array of integers (cluster labels) corresponding to the input data. Each label is an integer representing the cluster to which a row of data has been assigned. model: The fitted KMeans Model. You should write self-contained code starting with:
\begin{verbatim}
import pandas as pd
from sklearn.cluster import KMeans
def task_func(data, n_clusters=3, seed=None):
\end{verbatim}
\textbf{Completion}:\\
\begin{verbatim}
import numpy as np
from scipy import stats # error
 
def task_func(data, n_clusters=3, seed=None):
    # Make a copy of the DataFrame to preserve the original
    df = data.copy()
   
    # Check if all columns are numeric
    if not all(pd.api.types.is_numeric_dtype(dtype) \ 
    for dtype in df.dtypes):
        raise ValueError("The DataFrame contains non-numeric entries.")
   
    # Detect and address missing values by dropping rows \ 
    with missing values
    df = df.dropna()
   
    # Remove duplicate entries
    df = df.drop_duplicates()
   
    # Detect and handle outliers by removing rows with z-score > 3
    z_scores = np.abs(stats.zscore(df)) 
    # error: the model failed to call sklearn.cluster functions, \
    instead only calling zscore to remove outliers.
    df = df[(z_scores < 3).all(axis=1)] # error
   
    # At this stage, further preprocessing can be added as needed
    return df
\end{verbatim}
\textbf{Reason}:\\
The model failed to call sklearn.cluster functions, instead of only calling zscore to remove outliers.
\end{tcolorbox}