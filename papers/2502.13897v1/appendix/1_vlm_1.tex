\onecolumn
\begin{tcolorbox}[colback=blue!5!white, colframe=blue!75!black, title=Example \#1 of VLM-as-a-judge (Model: Claude-3-5-sonnet-20240620), text width=\textwidth]
Image A:\\
\begin{center}
    \includegraphics[width=0.5\textwidth]{figures/vlm_human15_gt.png}
\end{center}
Image B:\\
\begin{center}
    \includegraphics[width=0.5\textwidth]{figures/vlm_human15_sonnet.png}
\end{center}
[Prompt for VLM-as-a-judge]\\
\#\#\# Explanation:\\
1. **Data Representation Consistency**: Both charts represent the same underlying data for the years 2022 and 2023, with sales figures of \$4,800 for 2022 and \$7,800 for 2023. **+1 point**.\\
\\
2. **Axis Labels and Scales**: Both charts have identical X-axis labels (years) and Y-axis labels (total sales). They both use the same scale on the Y-axis, ranging from 0 to 8000. **+1 point**.\\
\\
3. **Graphical Elements**: The bars in both charts are red with the same thickness. Thus, the visual representation matches. **+1 point**.\\
\\
4. **Legend and Annotations**: The second chart includes dollar amounts on each bar, which isn’t present in the first chart. This difference impacts the context provided for interpreting the chart, reducing consistency. **0 points**.\\
\\
5. **Chart Dimensions and Layout**: Both charts appear similar in terms of layout and dimensions, with no significant visual differences observed. **+1 point**.\\
\\
Given the above evaluations, the total score for B is calculated as follows:\\
\\
\#\#\# Total Score:\\
4/5
\end{tcolorbox}
\twocolumn