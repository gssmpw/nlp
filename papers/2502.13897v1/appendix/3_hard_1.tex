\begin{tcolorbox}[colback=blue!5!white, colframe=blue!75!black, title=Hard-level Prompt:, text width=\textwidth]
\{ \texttt{"prompt":} \texttt{"}Write a Graph Recurrent Neural Network (GRNN) model based on attention mechanisms using Python for processing and analyzing time series data. Ensure to meet the following requirements:\\ \\1. \texttt{"}Graph network design\texttt{"}: Create a graph network where each graph represents an aerial formation, and the number of nodes corresponds to the number of vehicles in the formation.\\   - \texttt{"}Output\texttt{"}: Graph structure representation file.\\   - \texttt{"}Output filename\texttt{"}: \texttt{"}graph\_structure.json\texttt{"}\\ \\2. \texttt{"}Data format\texttt{"}: The time series data of aerial target formations is stored in Excel files located in the \texttt{"}.data/\texttt{"} directory, where each Excel file contains multiple sheets, with each sheet representing the time series data of a vehicle.\\   - \texttt{"}Input\texttt{"}: Excel files from \texttt{"}..data/\texttt{"} directory.\\   - \texttt{"}Input filename\texttt{"}: from \texttt{"}data\_1.xlsx\texttt{"} to \texttt{"}data\_5.xlsx\texttt{"}\\ \\3. \texttt{"}Data reading\texttt{"}: Read all Excel files from the \texttt{"}..data/\texttt{"} directory and extract the sheets for processing.\\   - \texttt{"}Output\texttt{"}: Combined time series data in a structured format.\\   - \texttt{"}Output filename\texttt{"}: \texttt{"}combined\_data.csv\texttt{"}\\ \\ 4. \texttt{"}Data segmentation\texttt{"}: Segment the time series data of each vehicle using a sliding window with length \texttt{"}l\texttt{"} and step size \texttt{"}s\texttt{"}.\\   - \texttt{"}Output\texttt{"}: Segmented time series data.\\   - \texttt{"}Output filename\texttt{"}: \texttt{"}segmented\_data.csv\texttt{"}\\\\5. \texttt{"}Training and testing set division\texttt{"}: Divide the segmented data into training and testing sets with a 7:3 ratio.\\   - \texttt{"}Output\texttt{"}: Training and testing datasets.\\   - \texttt{"}Output filenames\texttt{"}: \texttt{"}train\_data.csv\texttt{"}, \texttt{"}test\_data.csv\texttt{"} \\\\6. \texttt{"}Labels\texttt{"}: The last column of each Excel file contains label data, which should be preserved during the reading and segmentation process. \\\\7. \texttt{"}Model training\texttt{"}: Train the GRNN model using the training set to identify the action intentions of aerial target formations.\\   - \texttt{"}Output\texttt{"}: Trained model file.\\   - \texttt{"}Output filename\texttt{"}: \texttt{"}trained\_model.h5\texttt{"}\\\\8. \texttt{"}Training parameters\texttt{"}: Set the number of training epochs to 60, and record the loss value, accuracy, recall, and F1-Score for each epoch.\\   - \texttt{"}Output\texttt{"}: Training metrics log file.\\   - \texttt{"}Output filename\texttt{"}: \texttt{"}training\_metrics.csv\texttt{"}\\\\9. \texttt{"}Model evaluation\texttt{"}: Evaluate the trained model with the test set and generate a classification report and confusion matrix.\\   - \texttt{"}Output\texttt{"}: Classification report and confusion matrix visualization.\\   - \texttt{"}Output filenames\texttt{"}: \texttt{"}classification\_report.txt\texttt{"}, \texttt{"}confusion\_matrix.png\texttt{"} \\ \\ Please provide a complete code implementation and ensure that the code structure is clear and well-commented for understanding and evaluation.\texttt{"}, \\ \texttt{"data\_source\_type"}: \texttt{"}3=human written data\texttt{"}\}
\end{tcolorbox}
\twocolumn