\onecolumn
\begin{tcolorbox}[colback=blue!5!white, colframe=blue!75!black, title=Failed Example \#4 Generated by o1-mini, text width=\textwidth]
\textbf{Prompt}:\\
Reads data from a CSV file and generates a bar plot based on grouped mean values. The DataFrame is grouped by the column named 'col1\_name', and the mean for each group is calculated for the column 'col2\_name'. A bar plot is created using matplotlib. Each bar in the plot represents a group, and its height corresponds to the mean value of 'col2\_name' for that group. The plot is then configured with a title and axis labels: - The title is set as 'Mean of [col2\_name] Grouped by [col1\_name]'. This format dynamically inserts the names of the columns being analyzed into the title. - The xlabel (label for the x-axis) is set to the name of the column used for grouping (col1\_name). - The ylabel (label for the y-axis) is set as 'Mean of [col2\_name]', indicating that the y-axis represents the mean values of the specified column. Note that: Ensure that the CSV file exists at the specified path and has the required columns. The function does not handle missing data. Ensure that the CSV file has clean and complete data for accurate results. The bar plot is customizable using matplotlib's functionality after the function returns the Axes object. The function should output with: tuple(ax: matplotlib.axes.Axes, df\_mean: pandas.DataFrame) ax: The Axes object of the generated bar plot. This object can be used to further customize the plot, like adding labels or changing styles. df\_mean: The DataFrame containing the mean values calculated for each group. You should write self-contained code starting with:
\begin{verbatim}
import pandas as pd 
import matplotlib.pyplot as plt 
def task_func(csv_file_path, col1_name="column1", col2_name="column2"):
\end{verbatim}
\small
\textbf{Completion}:\\
\begin{verbatim}
# error: the model forgot to import relevant libraries.
def task_func(csv_file_path, col1_name="column1", col2_name="column2"):
    # Read the CSV file
    df = pd.read_csv(csv_file_path)
   
    # Create a copy for cleaning to preserve the original DataFrame
    df_clean = df.copy()
   
    # 1. Handle Missing Values
    # Check for missing values
    missing_counts = df_clean.isnull().sum()
    print("Missing Values:\n", missing_counts)
   
    # Option 1: Drop rows with missing values
    df_clean.dropna(inplace=True)
   
    # Option 2: Alternatively, you can fill missing values
    # df_clean[col2_name].fillna(df_clean[col2_name].mean(), inplace=True)
   
    # 2. Remove Duplicate Entries
    duplicates = df_clean.duplicated().sum()
    print(f"Number of duplicate rows: {duplicates}")
    df_clean.drop_duplicates(inplace=True)
   
    # 3. Handle Outliers
    # Example using Z-score for numerical columns
    numeric_cols = df_clean.select_dtypes(include=['float64', 'int64']).columns
    from scipy import stats
    z_scores = np.abs(stats.zscore(df_clean[numeric_cols]))
    df_clean = df_clean[(z_scores < 3).all(axis=1)]
   
    # 4. Standardize Data Formats
    # Example: Standardize date columns
    for col in df_clean.columns:
        if 'date' in col.lower():
            df_clean[col] = pd.to_datetime(df_clean[col], errors='coerce')
\end{verbatim}
\end{tcolorbox}

\onecolumn
\begin{tcolorbox}[colback=blue!5!white, colframe=blue!75!black, title=Failed Example \#4 Generated by o1-mini, text width=\textwidth]
\small
\begin{verbatim}
    # Drop rows with invalid date formats after conversion
    df_clean.dropna(subset=[col for col in df_clean.columns 
    if 'date' in col.lower()], inplace=True)
    
    # 5. Additional Preprocessing (if needed)
    # For example, convert categorical columns to consistent case
    categorical_cols = df_clean.select_dtypes(include=['object']).columns
    for col in categorical_cols:
        df_clean[col] = df_clean[col].str.strip().str.lower()
   
    # Continue with the original processing using the cleaned DataFrame
    df_mean = df_clean.groupby(col1_name)[col2_name].mean().reset_index()
    fig, ax = plt.subplots()
    ax.bar(df_mean[col1_name], df_mean[col2_name], color='skyblue')
    ax.set_title(f"Mean of {col2_name} Grouped by {col1_name}")
    ax.set_xlabel(col1_name)
    ax.set_ylabel(f"Mean of {col2_name}")
    plt.xticks(rotation=45)
    plt.tight_layout()
    plt.show()
   
    print("Basic Statistical Indicators:")
    print(df_clean.describe())
    print("\nMedian Values:")
    print(df_clean.median())
    print("\nStandard Deviation:")
    print(df_clean.std())
   
    df_clean[numeric_cols].hist(bins=10, figsize=(10, 8), edgecolor='black')
    plt.suptitle("Histograms of Numerical Columns")
    plt.tight_layout(rect=[0, 0.03, 1, 0.95])
    plt.show()
   
    df_clean.boxplot(column=numeric_cols, figsize=(10, 6))
    plt.title("Box Plots of Numerical Columns")
    plt.tight_layout()
    plt.show()
   
    corr_matrix = df_clean.corr()
    fig, ax = plt.subplots(figsize=(8, 6))
    cax = ax.matshow(corr_matrix, cmap='coolwarm')
    plt.title('Correlation Matrix Heatmap', pad=20)
    fig.colorbar(cax)
    ticks = range(len(corr_matrix.columns))
    ax.set_xticks(ticks)
    ax.set_yticks(ticks)
    ax.set_xticklabels(corr_matrix.columns, rotation=90)
    ax.set_yticklabels(corr_matrix.columns)
    plt.tight_layout()
    plt.show()
   
    return ax, df_mean
\end{verbatim}
\textbf{Reason}:\\
The model forgot to import relevant libraries.
\end{tcolorbox}
