\onecolumn
\begin{tcolorbox}[colback=blue!5!white, colframe=blue!75!black, title=Failed Example \#2 Generated by o1-mini, text width=\textwidth]
\textbf{Prompt}:\\
Reads data from a CSV file and generates a bar plot based on grouped mean values. The DataFrame is grouped by the column named 'col1\_name', and the mean for each group is calculated for the column 'col2\_name'. A bar plot is created using matplotlib. Each bar in the plot represents a group, and its height corresponds to the mean value of 'col2\_name' for that group. The plot is then configured with a title and axis labels: - The title is set as 'Mean of [col2\_name] Grouped by [col1\_name]'. This format dynamically inserts the names of the columns being analyzed into the title. - The xlabel (label for the x-axis) is set to the name of the column used for grouping (col1\_name). - The ylabel (label for the y-axis) is set as 'Mean of [col2\_name]', indicating that the y-axis represents the mean values of the specified column. Note that: Ensure that the CSV file exists at the specified path and has the required columns. The function does not handle missing data. Ensure that the CSV file has clean and complete data for accurate results. The bar plot is customizable using matplotlib's functionality after the function returns the Axes object. \textbf{The function should output with: tuple(ax: matplotlib.axes.Axes, df\_mean: pandas.DataFrame) ax: The Axes object of the generated bar plot. This object can be used to further customize the plot, like adding labels or changing styles. df\_mean: The DataFrame containing the mean values calculated for each group.} You should write self-contained code starting with:
\begin{verbatim}
import pandas as pd
import matplotlib.pyplot as plt
def task_func(csv_file_path, col1_name="column1", col2_name="column2"):
\end{verbatim}

\small
\textbf{Completion}:\\
\begin{verbatim}
import pandas as pd
import matplotlib.pyplot as plt
 
def task_func(csv_file_path, col1_name="column1", col2_name="column2"):
    # Read the CSV file into a pandas DataFrame
    df = pd.read_csv(csv_file_path)
    # Create a copy to preserve the original dataset
    df_clean = df.copy()
    # Handle missing values
    # Drop rows with any missing values
df_clean.dropna(inplace=True)
    # Remove duplicate entries  
    df_clean.drop_duplicates(inplace=True)
    # Handle outliers in 'col2_name' using the IQR method
    Q1 = df_clean[col2_name].quantile(0.25)
    Q3 = df_clean[col2_name].quantile(0.75)
    IQR = Q3 - Q1
    lower_bound = Q1 - 1.5 * IQR
    upper_bound = Q3 + 1.5 * IQR
    df_clean = df_clean[(df_clean[col2_name] >= lower_bound) \
    & (df_clean[col2_name] <= upper_bound)]  
    # Standardize data formats if necessary
    # Example: If 'col1_name' contains date information, \ 
    uncomment the following line
    # df_clean[col1_name] = pd.to_datetime(df_clean[col1_name])
   
return df_clean # error: the answer failed to output the required \ 
elements mentioned in the instruction.
\end{verbatim}
\textbf{Reason}:\\
The answer failed to output the required elements mentioned in the instruction.
\end{tcolorbox}