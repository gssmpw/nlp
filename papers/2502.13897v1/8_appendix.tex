\section{Appendix}
\label{sec: appendix}

\subsection{Task Definition}
\label{ssec: task_des}
We define six typical data science tasks as follows:

1) \textbf{Data cleaning and preprocessing.} This task detects and processes missing values, outliers, and duplicate data; and standardizes data formats, such as a uniform format for dates and times.
    
2) \textbf{Data exploration and statistics understanding.} This task calculates basic statistical indicators of data (mean, median, standard deviation, etc.), generates data distribution charts (histograms, box plots, etc.), calculates correlations between variables, and draws correlation matrices or maps. 
    
3) \textbf{Data visualization.} The goal of this task is to visualize and analyze data and create interactive charts so users can freely explore the data.
    
4) \textbf{Predictive modeling.} The task aims to select the appropriate machine learning algorithm, such as linear regression, decision tree, random forest, etc.; carry out feature engineering, such as feature selection, feature transformation, feature combination, etc.; the data set is divided into the training set and test set, and the model is trained and evaluated; and select the corresponding evaluation indicators for different prediction problems, such as classification, regression or clustering. 
    
5) \textbf{Data mining and Pattern recognition.} This task uses association rule mining, frequent item set mining, and other methods to find interesting patterns in the data; Text mining technology is used to extract keywords, topics, and other information from text data; and apply cluster analysis, classification algorithms, etc. to identify underlying patterns and structures. 
Pattern recognition tasks can conduct these functions: image recognition, text clustering, and time series detection.
    
6) \textbf{Interpretability and Report generation}. This task aims to provide explanations of model results, such as feature importance, model parameters, etc., and automatically generate reports and summaries that present the results of the analysis in a way that is easy to understand and share. 


\subsection{Comparison with Existing Benchmarks}
\label{app_ssec: comparison}
We perform correlation analysis to evaluate the alignment between \benchmark\space and coding evaluations like BCB and LCB.
To achieve correlation analysis, we calculate both Pearson’s $r$ and Spearman’s $p$ correlation coefficients, which provide insights into the strength and direction of relationships between our benchmark and these established metrics. 
This analysis not only validates our results but also ensures robustness across different evaluation dimensions. Our findings indicate strong positive correlations, suggesting that our benchmark aligns well with these established coding evaluation metrics.

% \vspace{-0.2cm}
\begin{table}[t!]
    \centering
    \caption{Correlation with LCB and BCB.}
    \vspace{-0.3cm}
    \label{tab: correlation}
    \resizebox{0.8\linewidth}{!}{
    \begin{tabular}{lcc} 
    \specialrule{.16em}{0pt}{.65ex}
    & \multicolumn{2}{c}{\benchmark} \\
    \cmidrule(lr){2-3} 
    \cmidrule(lr){2-3} & $r$ & $p$ \\ 
    \specialrule{.10em}{.4ex}{.65ex}
    LiveCodeBench (LCB) & 0.853 & 0.673  \\ 
    BigCodeBench (BCB) & 0.823 & 0.808 \\
    \specialrule{.16em}{.4ex}{0pt}
    \end{tabular}
    }
\end{table}

While \benchmark\space does show a correlation with LCB or BCB in Table~\ref{tab: correlation}, our benchmark offers several unique and important contributions:

\begin{itemize}[leftmargin=*,itemsep=0pt,parsep=0.5em,topsep=0.3em,partopsep=0.3em]
    \item \textbf{Domain-Specific Focus:} \benchmark\space specifically targets data science and analytics tasks. However, existing benchmarks primarily focus on general programming problems. This specialization helps evaluate models' capabilities in handling real-world data analysis scenarios.

    \item \textbf{Task Diversity:} Our \benchmark\space includes unique task types like data preprocessing, visualization, and statistical analysis. These tasks are underrepresented in current benchmarks. This provides deeper insights into models' data science-specific capabilities.

    \item \textbf{Complementary Insights:} While overall correlations exist, we observe meaningful differences in model rankings. For example, models like Meta-Llama-3-8B-Instruct and CodeLlama-34B-Instruct show distinct performance patterns. These differences highlight capabilities specific to data science tasks that other benchmarks may not capture.
\end{itemize}
\textcolor{black}{The correlation with existing benchmarks validates our evaluation methodology, while our domain-specific focus provides valuable new insights for assessing AI models in data science applications.}


\begin{figure}[t!]
    \centering
    \includegraphics[width=0.8\linewidth]{figures/TFC.pdf}
    \caption{An example of TFC tuples.}
    \label{fig: tfc}
\end{figure}


\subsection{Motivation and Example of Task-Function-Code (TFC)}
\textcolor{black}{
The TFC framework was developed to address several critical challenges in automated evaluation of data science tasks:}
\begin{itemize}[leftmargin=*,itemsep=0pt,parsep=0.5em,topsep=0.3em,partopsep=0.3em]
    \item \textbf{Systematic Task Selection:} TFC provides a structured approach to identify and categorize key tasks across six established types. This systematic organization ensures comprehensive coverage of essential data science operations and helps maintain evaluation consistency and completeness.
    \item \textbf{Standardized Evaluation Metrics:} Data science tasks often lack standardized evaluation criteria. TFC addresses this by explicitly defining appropriate evaluation functions for each task. For example, data preprocessing tasks require specific metrics that differ from visualization tasks. This standardization ensures fair and consistent assessment.
    \item \textbf{Automated Execution Framework:} TFC includes executable code components for both tasks and evaluation metrics. This automation significantly improves evaluation efficiency, result reproducibility, and testing scalability.
    \item \textbf{Ground Truth Generation:} TFC serves as a crucial foundation for establishing ground truth, particularly valuable for complex tasks where ground truth is not readily available, and enables systematic verification and validation of model outputs.
\end{itemize}
Overall, the TFC structure in Figure~\ref{fig: tfc} represents a novel contribution by providing a comprehensive framework that bridges the gap between task definition, evaluation criteria, and automated assessment in data science contexts.


\subsection{Programmatic Rules}
We list all programmatic rules in Table~\ref{tab: program_rules}.

\begin{table*}[t!]
\centering
\caption{Details of programmatic rules.}
\vspace{-0.2cm}
\resizebox{0.97\textwidth}{!}{
\begin{tabular}{c|c|c|c|c|c|c}
\specialrule{.16em}{0pt}{.65ex}
Aggregate Function                  & Task                                    &  Type  & Rule & 
Comparison            & GT & Threshold                   \\
\specialrule{.10em}{.4ex}{.65ex}
Mean Squared Error          & Calculate MSE                                   & Decimal & Bool & It $\leq$ to GT is 1, it $>$ GT is 0 & Yes     & -                     \\
Data Cleaning Completeness  & row/column number                               & Integer & Bool & If it ==  GT, it is 1; if it != GT, it is 0  & Yes     & -                     \\
Silhouette Score            & Calculate                                & Decimal & Bool & If it $\geq$ to GT is 1, if it $<$ GT is 0 & Yes     & 0.5                  \\
Model Accuracy              & Calculate F1                                    & Decimal & Bool & If it $\geq$ to GT is 1, if it $<$ GT is 0  & Yes     & 0.9                  \\
Model Accuracy              & Calculate Accuracy                              & Decimal & Bool & If it $\geq$ to GT is 1, if it $<$ GT is 0  & Yes     & 0.9                  \\
R-squared Value             & Calculate                              & Decimal & Bool & If it $\geq$ to GT is 1, if it $<$ GT is 0  & Yes     & 0.9                  \\
Data Cleaning Completeness  & number of nulls                            & Integer & Bool & If it ==  0, returns 1; if it $>$ 0, returns 0  & Yes     & -                     \\
Data Quality Score         & number of outlier                         & Integer & Bool & If it ==  GT, it is 1; if it != GT, it is 0  & Yes     & -                     \\
Data Accuracy               & MSE                                     & Decimal & Bool & Threshold judgment          &  Yes     & 0.05                 \\
Data Completeness           & Null number ratio                                & Decimal & Bool & If it ==  GT, it is 1; if it != GT, it is 0  &  Yes     & -                      \\
Data Quality Score          & Calculate loss                                  & Decimal & Bool & It $\leq$ to GT is 1, it $>$ GT is 0 &   Yes    & -                     \\
Association Rule Confidence & Association rule accuracy                                 & Decimal & Bool & Threshold judgment          & Yes      & 0.9                  \\                \specialrule{.16em}{.4ex}{0pt}
\end{tabular}
}
\label{tab: program_rules}
\end{table*}


\subsection{Caveats when Using LLMs for Data Science}
Here we list the issues that occurred during testcase generation, most of which have been addressed by modifying the prompts. We notice that some of the issues may be disruptive to the system and some may be subtle but important.
\begin{enumerate}
    \item Be careful when using LLMs on well-known open source datasets, especially with customized tasks and data split. LLMs may memorize some open-source datasets. For example, if we want to use part of the penguin dataset that does not contain certain columns, the model (GPT-4o) will still explicitly process those columns in the code.
    \item Hallucination during data pre-processing. For example, when the model is required to merge two CSV, it may hallucinate on a common column and not go through all the columns in the files to find the actual ones.
    \item On multilingual tasks. LLMs may not be able to select the correct encoding. For instance, when they are required to open a CSV file that has content in Chinese, they will struggle to choose the correct encoding to open the file. Even if they are hinted that the file is in Chinese, they may choose encodings other than ``gbk'', e.g., ``latin''.
\end{enumerate}


\subsection{Related Works}
\label{sec:related_works}
% In this section we will focus on differentiating from other related works (Table~\ref{tab:related-works}) and group work related to asynchronous planning. The field of LLM agent benchmarks for task planning performance consists of a large literature. \robotouille{} aims to extend the existing work with a LLM assessment environment that uses multi-agent asynchronous planning on long-horizon tasks.

In this section we will focus on our desiderata for LLM assistants and how \robotouille{} is different from other related works (Table~\ref{tab:related-works}).

% \begin{table*}[!h]
\centering
\setlength\tabcolsep{2.5pt}
\scalebox{0.80}{
\begin{tabular}{ccccccccccc}
    \toprule 
    Benchmark & \makecell{High-Level\\Actions} & Multi-agent & \makecell{Procedural\\Level Generation} & \makecell{Time\\Delays} & \makecell{Number of Tasks} & \makecell{Longest Plan \\Horizon} \\
    \midrule
    ALFWorld \citep{shridhar2021alfworldaligningtextembodied} & \cmark & \xmark & \xmark & \xmark & 3827 & 50 \\
    CuisineWorld \citep{gong2023mindagent} & \cmark & \cmark & \cmark & \xmark & 33 & 11 \\
    MiniWoB++ \citep{liu2018reinforcementlearningwebinterfaces} & \cmark & \xmark & \xmark & \xmark & 40 & 13 \\ 
    Overcooked-AI \citep{carroll2020utilitylearninghumanshumanai} & \xmark & \cmark & \xmark & \cmark & 1 & 100 \\
    PlanBench \citep{valmeekam2023planbenchextensiblebenchmarkevaluating} & \cmark & \xmark & \cmark & \xmark & 885 & 48\\
    $\tau$-bench \citep{yao2024taubenchbenchmarktoolagentuserinteraction} & \cmark & \xmark & \cmark & \xmark & 165 & 30 \\
    WebArena \citep{zhou2024webarenarealisticwebenvironment} & \cmark & \xmark & \cmark & \xmark & 812 & 30 \\
    WebShop \citep{yao2023webshopscalablerealworldweb} & \cmark & \xmark & \xmark & \xmark & 12087 & 90 \\
    AgentBench \citep{liu2023agentbenchevaluatingllmsagents} & \cmark & \cmark & \xmark & \xmark & 8 & 35 \\
    ARA \citep{kinniment2024evaluatinglanguagemodelagentsrealistic} & \cmark & \xmark & \xmark & \xmark & 12 & 4 \\
    AsyncHow \citep{lin2024graphenhancedlargelanguagemodels} & \cmark & \xmark & \xmark & \cmark & 1600 & 9 \\
    MAgIC \citep{xu2023magicinvestigationlargelanguage} & \cmark & \cmark & \xmark & \xmark & 5 & 20 \\
    T-Eval \citep{chen2024tevalevaluatingtoolutilization} & \cmark & \cmark & \xmark & \xmark & 23305 & 19 \\
    MLAgentBench \citep{huang2024mlagentbenchevaluatinglanguageagents} & \cmark & \xmark & \xmark & \xmark & 13 & 50 \\
    GAIA \citep{mialon2023gaiabenchmarkgeneralai} & \cmark & \xmark & \xmark & \xmark & 466 & 45 \\
    VirtualHome \citep{puig2018virtualhomesimulatinghouseholdactivities} & \cmark & \cmark & \cmark & \xmark & 2821 & 96 \\\midrule
    \robotouille (Ours) & \cmark & \cmark & \cmark & \cmark & 30 & 82 \\
    \bottomrule
\end{tabular}
}
\caption{Comparison between \robotouille and other benchmarks. See Appendix~\ref{sec:related_works} for more details.}
\label{tab:related-works}
\end{table*}

\textbf{Asynchronous Planning}
% While there exist many benchmarks that test the task planning abilities of LLMs agents (\cite{shridhar2021alfworldaligningtextembodied}; \cite{gong2023mindagent}; \cite{liu2018reinforcementlearningwebinterfaces}; \cite{valmeekam2023planbenchextensiblebenchmarkevaluating}; \cite{yao2024taubenchbenchmarktoolagentuserinteraction}; \cite{zhou2024webarenarealisticwebenvironment};\cite{yao2022webshop}), very few test the ability to plan asynchronously. The field of asynchronous planning benchmarks begins with benchmarks testing the temporal logic abilities of LLMs like TRAM (\cite{wang2024trambenchmarkingtemporalreasoning}). Of those benchmarks which test asynchronous planning, many use graph-based algorithms like GoT or GNNs (\cite{wu2024graphlearningimprovetask}; \cite{Besta_2024}). In fact, AsyncHow (\cite{lin2024graphenhancedlargelanguagemodels}), which extends both asynchronous and graph-based planning, proposes Plan Like a Graph (PLaG).
% When planning to bake potatoes or cut onions, it is important to realize that it is more efficient to cut onions while the potatoes are baking. Thus, a temporal component in tasks can be critical in task planning. While Overcooked AI (\cite{carroll2020utilitylearninghumanshumanai}) includes time delays for cooking in their environment, it has only 1 objective-based task so it is not able to test situations where multiple temporal tasks occur asynchronously. AsyncHow (\cite{lin2024graphenhancedlargelanguagemodels}) creates a benchmark for asynchronous planning with LLMs but uses GPT-4 to calculate estimates for each task. In contrast, \robotouille{} tests asynchronous task planning in a kitchen environment using temporal tasks like cutting and frying. In \robotouille{}, multiple tasks can occur sequentially or simultaneously which mimics real world interactions and planning.
Many benchmarks evaluate the task planning abilities of LLM agents \citep{shridhar2021alfworldaligningtextembodied, gong2023mindagent,liu2018reinforcementlearningwebinterfaces,valmeekam2023planbenchextensiblebenchmarkevaluating,yao2024taubenchbenchmarktoolagentuserinteraction,zhou2024webarenarealisticwebenvironment,yao2023webshopscalablerealworldweb} but few test the ability to plan asynchronously. Existing work relevant to asynchronous planning evaluate LLM capabilities on temporal logic \citep{wang2024trambenchmarkingtemporalreasoning} or use graph-based techniques \citep{wu2024graphlearningimprovetask}; \citep{Besta_2024}) but do not focus on it. \citep{lin2024graphenhancedlargelanguagemodels} proposes the Plan Like a Graph technique and a benchmark AsyncHow that focuses on asynchronous planning but makes a strong assumption that infinite agents exist. \citep{carroll2020utilitylearninghumanshumanai} proposes a benchmark, Overcooked-AI, that involves cooking onion soup which has time delays but has limited tasks and focuses on lower-level planning without LLM agents. \robotouille{} has a dataset focused on asynchronous planning that involves actions including cooking, frying, filling a pot with water, and boiling water.

% \textbf{Diverse Long-Horizon Task Planning}
% LLMs have a great store of semantic knowledge about the world but it is difficult for them to make decisions as agents because they have not experienced the real world. Works like SayCan (\cite{ahn2022icanisay}) have tried to solve this problem by grounding the LLM in real-world, long horizon tasks. Inner Monologue (\cite{huang2022innermonologueembodiedreasoning}) highlights the application of LLMs to embodied environments like robots where closed-loop language feedback allows better planning for robotic situations. Because of their wealth of knowledge about high-level tasks, reasoning capabilities, and ability to turn natural language into plans, LLMs can make proficient planners (\cite{singh2022progpromptgeneratingsituatedrobot}; \cite{liu2023llmpempoweringlargelanguage}; \cite{Lin_2023}; \cite{Wang_2024}; \cite{huang2024understandingplanningllmagents}). In many papers, LLMs shine as zero-shot generalizers, planners, and reasoners (\cite{huang2022languagemodelszeroshotplanners}; \cite{zeng2022socraticmodelscomposingzeroshot}). There have also been improvements in few-shot prompting with LLMs for the purpose of planning (\cite{yang2023couplinglargelanguagemodels}; \cite{liang2023codepolicieslanguagemodel}; \cite{song2023llmplannerfewshotgroundedplanning}). Now that LLMs can be used as agents in planners, how can we evaluate the effectiveness of the agents?

% Existing LLM agent benchmarks evaluate agents on only short-horizon tasks (\cite{shridhar2020alfworld}; \cite{puig2018virtualhome}; \cite{gong2023mindagent}). Firstly, many benchmarks contain very few goal-based tasks at all. The benchmarks AgentBench (\cite{liu2023agentbenchevaluatingllmsagents}), ARA (\cite{kinniment2024evaluatinglanguagemodelagentsrealistic}), and MLAgentBench (\cite{huang2024mlagentbenchevaluatinglanguageagents}) have 8, 12, 11 tasks respectively. Highlighting long-horizon tasks, the majority of the agent benchmarks in (Table~\ref{tab:related-works}) have less than 50 steps for their longest horizon plans. However, \robotouille{} creates a benchmark with 30 diverse long-horizon tasks with 80 steps as its longest plan, focused on cooking and better evaluates agents’ ability to solve plans with long sequences of steps.

% In addition, \robotouille{} leverages procedural generation. While other agent benchmarks like MiniWoB++ (\cite{liu2018reinforcementlearningwebinterfaces}), PlanBench (\cite{valmeekam2023planbenchextensiblebenchmarkevaluating}), $\tau$-bench (\cite{yao2024taubenchbenchmarktoolagentuserinteraction}), WebArena (\cite{zhou2024webarenarealisticwebenvironment}), and MAgIC (\cite{xu2023magicinvestigationlargelanguage}) support procedural generation, they contain only short-horizon tasks. \robotouille{}’s combination of long-horizon tasks and procedural generation gives it the ability to provide thorough testing of LLM agents on the randomization of tasks and environments.

\textbf{Diverse Long-Horizon Task Planning}
There is vast amount of work that use LLMs to plan \citep{ahn2022icanisay,huang2022innermonologueembodiedreasoning,zeng2022socraticmodelscomposingzeroshot,liang2023codepolicieslanguagemodel,singh2022progpromptgeneratingsituatedrobot,song2023llmplannerfewshotgroundedplanning,yang2023couplinglargelanguagemodels,song2023llmplannerfewshotgroundedplanning} but they tend to evaluate on short-horizon tasks with limited diversity in tasks. We present the number of tasks, longest plan horizon, and procedural generation capability of various benchmarks in Table~\ref{tab:related-works} to capture these axes. Notable LLM agent benchmarks that capture these axes include PlanBench \citep{valmeekam2023planbenchextensiblebenchmarkevaluating}, WebShop \citep{yao2023webshopscalablerealworldweb}, and VirtualHome \citep{puig2018virtualhomesimulatinghouseholdactivities}. \robotouille{} provides a focused set of diverse long-horizon tasks that can be procedurally generated.

% \textbf{Multi-agent Planning}
% Recently, after the development of using a singular LLM as an autonomous agent, multi-agent planners have made advancements in complex decision making and task planning (\cite{guo2024largelanguagemodelbased}; \cite{xi2023risepotentiallargelanguage}). Because of a better division of labor, having multiple agents can break down complex tasks and efficiently finish tasks. Agents can specialize in their roles, allowing for more collaboration. AutoGen (\cite{wu2023autogenenablingnextgenllm}) and ProAgent (\cite{zhang2024proagentbuildingproactivecooperative}) are frameworks used to build LLM-based multi-agent applications in conversation-based and cooperative task settings respectively.

% Many benchmarks have used LLMs as agents for collaboration in a shared environment (\cite{gong2023mindagent}; \cite{carroll2020utilitylearninghumanshumanai}; \cite{yao2024taubenchbenchmarktoolagentuserinteraction}; \cite{liu2023agentbenchevaluatingllmsagents}; \cite{xu2023magicinvestigationlargelanguage}; \cite{ma2024agentboardanalyticalevaluationboard}). However, many of these benchmarks focus on two-agent interactions with uneven task distribution like Overcooked AI (\cite{carroll2020utilitylearninghumanshumanai}). In response, CUISINEWORLD (\cite{gong2023mindagent}) tries to address this problem by assuming equal responsibilities across agents for tasks with competing resources. \robotouille{} instead extends this approach with turn-based environments where multiple LLM agents can control multiple agents to reach an objective within an environment. T-Eval (\cite{chen2024tevalevaluatingtoolutilization}) uses a three pronged approach to multi-agent frameworks with a planner, executor, and reviewers in which each agent can finish its job without switching the roles as dictated by the current plan. But, T-Eval uses multiple agents to annotate solutions which is different from the multi-agent planning of \robotouille{}. While MAgIC (\cite{xu2023magicinvestigationlargelanguage}) also investigates LLM powered multi-agents, it focuses on classic decision-making games like the Prisoner’s Dilemma and Chameleon. \robotouille{}’s cooking-based temporal tasks allow for better evaluation of a complex multi-agent environment.

\textbf{Multi-agent Planning} LLM agent benchmarks like \citep{liu2023agentbenchevaluatingllmsagents,xu2023magicinvestigationlargelanguage,ma2024agentboardanalyticalevaluationboard, gong2023mindagent} evaluate multi-agent interactions but do not involve time delays. OvercookedAI \citep{carroll2020utilitylearninghumanshumanai}, while not an LLM agent benchmark, incorporates time delays which brings the complexity of asynchronous planning to multi-agent settings. \robotouille{} provides a multi-agent dataset for 2-4 agents, a choice between turn-based or realtime planning, and incorporates asynchronous tasks for added complexity.


\subsection{Related Work in Data Science}
Recently, some evaluation benchmarks for large language models in data science have been proposed. 
Text2Analysis~\citep{he2024text2analysis} constructs the evaluation benchmark to evaluate the model's ability to handle data analysis functions and fuzzy questions on tabular data. Their prompts are obtained through manual annotation and large model generation.
Furthermore, DAEval~\citep{huinfiagent} is developed as another evaluation benchmark and it contains 257 data analysis questions on CSV data and questions, which are generated by LLMs.
However, the prompts in these two works often only involve one task, and these prompts involve relatively simple data analysis operations.
In practical data science analysis tasks, user questions often involve multiple tasks and involve performing complex data analysis operations.
Therefore, we aim to provide a data science evaluation benchmark that is more in line with practical scenarios, especially for problems involving multiple subtasks and complex data analysis operations.



\subsection{Results}
We present the detailed results for our collected prompt and BigCodeBench source in Table~\ref{tab:our_results} and Table~\ref{tab:bcb_results}.

\begin{table*}[t!]
    \centering
    \caption{Evaluation results for \benchmark$\space$ (our collected prompts).}
    \vspace{-0.2cm}
    \label{tab:our_results}
    \resizebox{0.97\textwidth}{!}{
    \begin{tabular}{c|c|c|c|c|c|c|c|c|c|c}
    \specialrule{.16em}{0pt}{.65ex}
        \multirow{2}{*}{Models} & \multirow{2}{*}{Size} & \multicolumn{2}{c|}{Coarse-grained Metrics} & \multicolumn{6}{c|}{Fine-grained Metrics} & \multirow{2}{*}{Score}
        \\
        \cmidrule{3-10}
        & & Success (\%) & CR (\%) & VLM & F1 & F2 & F3 & F4 & F5 & \\
        \specialrule{.10em}{.4ex}{.65ex}
        o1-mini                      & N/A                      & 13.45  & 15.43      & 2.87 & 53.75 & \text{0.00}  & 45.50 & 46.25 & 25.79 & 19.41 \\
        GPT-4o-2024-05-13            & N/A                      & \textbf{19.82}  & \textbf{17.89}      & \textbf{3.91} & \textbf{71.79} & \text{0.00}  & \textbf{54.50} & \underline{70.00} & 33.95 & \textbf{24.32} \\
        GPT-4o-mini                  & N/A                      & 12.73  & 17.35      & 3.05 & 62.86 & \textbf{15.00} & \underline{50.50} & 69.38 & \textbf{43.68} & 24.13 \\
        GPT-4-Turbo                  & N/A                      & \underline{17.27}  & \underline{17.36}      & \underline{3.09} & \underline{65.18} & 10.00 & 44.50 & \textbf{78.12} & \underline{41.05} & \underline{24.24} \\
        Claude-3-5-Sonnet-20240620   & N/A                      & 8.00      & 11.12      & 2.14 & 32.32 & \text{0.00}  & 36.00 & 63.12 & 17.63 & 15.19 \\
        GLM-4-Flash                  & N/A                      & 9.82   & 7.43       & 1.33 & 47.14 & \text{0.00}  & 22.00 & 24.38 & 3.95  & 10.26 \\
        \specialrule{.10em}{.4ex}{.65ex}
        Meta-Llama-3.1-8B-Instruct   & 8B                       & 10.00     & 7.72       & 1.29 & 45.18 & \text{0.00}  & 16.00 & 23.12 & 8.95  & 10.24 \\
        Meta-Llama-3-8B-Instruct     & 8B                       & 1.64   & 1.43       & 0.52 & 7.86  & \text{0.00}  & 4.50  & 6.88  & \text{0.00}  & 2.00  \\
        Gemma-2-9B-it                & 9B                       & 5.64   & 5.51       & 1.06 & 26.79 & \text{0.00}  & 13.00 & 22.50 & 2.89  & 7.18  \\
        GLM-4-9B-Chat                & 9B                       & 10.55  & 9.96       & 1.69 & 55.36 & \text{0.00}  & 31.00 & 28.75 & 21.32 & 13.91 \\
        Qwen2.5-7B-Instruct          & 7B                       & 11.64  & 10.11      & 1.43 & 55.36 & \text{0.00}  & 36.50 & 33.12 & 18.42 & 14.40 \\
        Qwen2-7B-Instruct            & 7B                       & 6.91   & 5.90       & 1.16 & 32.50 & \text{0.00}  & 18.00 & 21.88 & 2.37  & 7.97  \\
        Qwen2-1.5B-Instruct          & 1.5B                     & 1.82   & 1.60       & 0.40 & 3.57  & \text{0.00}  & 2.00  & 13.12 & 0.79  & 2.13  \\
        Yi-1.5-9B-Chat-16K           & 9B                       & 6.18   & 4.25       & 0.73 & 30.36 & \text{0.00}  & 16.00 & 8.75  & 3.95  & 6.06  \\
        \specialrule{.10em}{.4ex}{.65ex}
        CodeLlama-34B-Instruct    & 34B                      & \text{0.00}      & 0.03       & \text{0.00} & \text{0.00}  & \text{0.00}  & \text{0.00}  & \text{0.00}  & \text{0.00}  & 0.02  \\
        CodeLlama-13B-Instruct    & 13B                      & 0.73   & 0.50       & 0.04 & 4.46  & \text{0.00}  & \text{0.00}  & 3.75  & \text{0.00}  & 0.77  \\
        CodeLlama-7B-Instruct     & 7B                       & 0.55   & 0.27       & \text{0.00} & 1.96  & \text{0.00}  & \text{0.00}  & \text{0.00}  & \text{0.00}  & 0.30  \\
        StarCoder2-15B               & 15B                      & 0.18   & 0.20       & 0.07 & 0.54  & \text{0.00}  & \text{0.00}  & 0.62  & \text{0.00}  & 0.20 \\
        Deepseek-Coder-33B-instruct  & 33B                      & 12.55  & 13.53      & 2.29 & 62.86 & \text{0.00}  & 43.00 & 51.88 & 21.32 & 18.49 \\
        Deepseek-Coder-6.7B-instruct & 6.7B                     & 12.55  & 13.56      & 1.93 & 63.21 & \text{0.00}  & 39.00 & 53.75 & 21.05 & 18.39 \\
        Deepseek-Coder-1.3B-instruct & 1.3B                     & 0.73   & 0.61       & 0.10 & 3.39  & \text{0.00}  & \text{0.00}  & 1.25  & \text{0.00}  & 0.67  \\
        Qwen2.5-Coder-7B-Instruct    & 7B                       & 6.18   & 7.87       & 1.48 & 40.18 & \text{0.00}  & 27.50 & 33.75 & 4.47  & 10.79 \\
        Qwen2.5-Coder-1.5B-Instruct  & 1.5B                     & 6.18   & 7.52       & 0.81 & 38.57 & \text{0.00}  & 15.50 & 40.00 & 10.53 & 10.46 \\
        \specialrule{.16em}{.4ex}{0pt}
    \end{tabular}
    }
\end{table*}

\begin{table*}[t!]
    \centering
    \caption{Evaluation results for \benchmark$\space$ (BigCodeBench source).}
    \vspace{-0.2cm}
    \label{tab:bcb_results}
    \resizebox{0.97\textwidth}{!}{
    \begin{tabular}{c|c|c|c|c|c|c|c|c|c}
    \specialrule{.16em}{0pt}{.65ex}
    \multirow{2}{*}{Models} & \multirow{2}{*}{Size} & \multicolumn{2}{c|}{Coarse-grained Metrics} & \multicolumn{5}{c|}{Fine-grained Metrics} & \multirow{2}{*}{Score} \\
    \cmidrule{3-9}
    & & Success (\%) & CR (\%) & F1 & F2 & F3 & F4 & F5 & \\
    \specialrule{.10em}{.4ex}{.65ex}
    o1-mini                      & N/A                      & 35.15  & 55.08      & 41.62 & 25.62 & 32.89 & 25.90 & 23.16 & 47.77 \\
    GPT-4o-2024-05-13            & N/A                      & \textbf{81.62}  & \textbf{85.09}      & \textbf{77.30} & \textbf{74.63} & \textbf{74.21} & \textbf{71.79} & \textbf{65.48} & \textbf{81.81} \\
    GPT-4o-mini                  & N/A                      & 63.11  & 71.10       & 59.46 & 58.89 & 60.26 & 55.90 & 56.79 & 67.49 \\
    GPT-4-Turbo                  & N/A                      & 63.35  & 72.54      & 61.35 & 52.04 & 62.11 & 54.36 & 53.95 & 68.14 \\
    Claude-3-5-Sonnet-20240620   & N/A                      & 60.48  & 73.59      & 54.59 & 49.11 & 62.37 & 49.49 & 55.39 & 68.08 \\
    GLM-4-Flash                  & N/A                      & 37.07  & 42.8       & 33.04 & 39.11 & 36.05 & 28.72 & 17.89 & 39.55 \\
    \specialrule{.10em}{.4ex}{.65ex}
    Meta-Llama-3.1-8B-Instruct   & 8B                       & 29.58  & 42.51      & 35.95 & 24.26 & 23.95 & 22.82 & 31.41 & 38.16 \\
    Meta-Llama-3-8B-Instruct     & 8B                       & 3.29   & 4.74       & 2.97  & 1.67  & 2.11  & 1.28  & 1.96  & 3.98  \\
    Gemma-2-9B-it                & 9B                       & 7.54   & 12.81      & 25.95 & 22.46 & 27.37 & 16.67 & 21.84 & 15.06 \\
    GLM-4-9B-Chat                & 9B                       & 30.72  & 37.11      & 23.65 & 30.78 & 27.11 & 26.67 & 18.42 & 33.84 \\
    Qwen2.5-7B-Instruct          & 7B                       & 54.43  & 64.12      & 49.80 & 48.40 & 50.79 & 49.23 & 40.15 & 59.52 \\
    Qwen2-7B-Instruct            & 7B                       & 28.08  & 32.06      & 30.41 & 27.63 & 32.26 & 27.18 & 9.21  & 30.18 \\
    Qwen2-1.5B-Instruct          & 1.5B                     & 4.67   & 6.73       & 4.86  & 2.63  & 3.68  & 3.33  & 6.00  & 5.97  \\
    Yi-1.5-9B-Chat-16K           & 9B                       & 48.74  & 54.9       & 40.70 & 48.34 & 42.11 & 46.41 & 35.64 & 51.53 \\
    \specialrule{.10em}{.4ex}{.65ex}
    CodeLlama-34B-Instruct    & 34B                      & 1.20    & 1.94       & 1.35  & 1.11  & 2.63  & 2.05  & 1.58  & 1.85  \\
    CodeLlama-13B-Instruct    & 13B                      & 13.71  & 19.3       & 14.05 & 15.07 & 12.53 & 17.95 & 6.84  & 17.52 \\
    CodeLlama-7B-Instruct     & 7B                       & 3.65   & 5.19       & 4.05  & 3.15  & 3.42  & 2.31  & 2.11  & 4.57  \\
    StarCoder2-15B               & 15B                      & 2.69   & 3.41       & 3.24  & 2.41  & 2.11  & 4.36  & 1.58  & 3.21  \\
    Deepseek-Coder-33B-instruct  & 33B                      & \underline{70.12}  & \underline{76.94}      & \underline{66.58} & \underline{62.63} & \underline{63.16} & \underline{64.87} & \underline{57.59} & \underline{73.11} \\
    Deepseek-Coder-6.7B-instruct & 6.7B                     & 45.09  & 50.86      & 37.00 & 45.96 & 48.79 & 44.10 & 17.11 & 47.50 \\
    Deepseek-Coder-1.3B-instruct & 1.3B                     & 20.36  & 25.05      & 16.22 & 19.44 & 17.63 & 21.28 & 10.53 & 22.81 \\
    Qwen2.5-Coder-7B-Instruct    & 7B                       & 58.02  & 68.01      & 55.34 & 57.44 & 49.26 & 45.38 & 45.36 & 63.15 \\
    Qwen2.5-Coder-1.5B-Instruct  & 1.5B                     & 28.20   & 35.60       & 26.94 & 28.96 & 26.74 & 26.15 & 18.32 & 32.69 \\
    \specialrule{.16em}{.4ex}{0pt}
    \end{tabular}
    }
\end{table*}


\subsection{AI Assistants In Writing}
While writing this paper, we employed AI to rectify grammatical errors and revise unreasonable descriptions.



\subsection{VLM-as-a-judge Prompt and Examples}
\label{sec:vlm_prompt}
In this section, we define the criteria to measure the quality of LLM-generated figures/charts.


\onecolumn
\begin{tcolorbox}[colback=blue!5!white, colframe=blue!75!black, title=Prompt for VLM-as-a-judge, text width=\textwidth]
Above are two figures, which are A and B. The first figure is the ground truth image and the second figure is the predicted image. The total score is 5. Please score B following the criteria below:\\
- add 1 point for Data Representation Consistency: Ensure that the underlying data represented by the two charts is identical. This includes the values for all data points and the range of the data. Any variation in the dataset used would make the charts different.\\
- add 1 point for Axis Labels and Scales: Verify that both charts have identical axis labels, units, and scales. Any difference in how the axes are labeled or scaled, such as using logarithmic vs. linear scales, can affect the interpretation of the data.\\
- add 1 point for Graphical Elements: Check if the visual elements (such as lines, bars, markers, etc.) are represented the same way in both charts. Line thickness, marker styles, and colors should match across charts for them to be considered visually equal.\\
- add 1 point for Legend and Annotations: Confirm that any legends, titles, or annotations (e.g., text labels, arrows, or highlights) are the same in both charts. These elements often provide crucial context for interpreting the chart.\\
- add 1 point for Chart Dimensions and Layout: Ensure that the dimensions (height and width), aspect ratios, and layout of the charts are identical. Even if the content and representation are similar, a different aspect ratio or spacing between elements can change the chart’s overall appearance and interpretation.\\
\\
Please write down the total score for B based on the criteria above, and provide a brief explanation of your reasoning. If you believe that the two figures are not identical, please explain the differences you observed.\\
\\
\#\#\# Explanation:\\
your explanation here\\
\\
\#\#\# Total Score:\\
x/5
\end{tcolorbox}
\clearpage
\onecolumn
\begin{tcolorbox}[colback=blue!5!white, colframe=blue!75!black, title=Example \#1 of VLM-as-a-judge (Model: Claude-3-5-sonnet-20240620), text width=\textwidth]
Image A:\\
\begin{center}
    \includegraphics[width=0.5\textwidth]{figures/vlm_human15_gt.png}
\end{center}
Image B:\\
\begin{center}
    \includegraphics[width=0.5\textwidth]{figures/vlm_human15_sonnet.png}
\end{center}
[Prompt for VLM-as-a-judge]\\
\#\#\# Explanation:\\
1. **Data Representation Consistency**: Both charts represent the same underlying data for the years 2022 and 2023, with sales figures of \$4,800 for 2022 and \$7,800 for 2023. **+1 point**.\\
\\
2. **Axis Labels and Scales**: Both charts have identical X-axis labels (years) and Y-axis labels (total sales). They both use the same scale on the Y-axis, ranging from 0 to 8000. **+1 point**.\\
\\
3. **Graphical Elements**: The bars in both charts are red with the same thickness. Thus, the visual representation matches. **+1 point**.\\
\\
4. **Legend and Annotations**: The second chart includes dollar amounts on each bar, which isn’t present in the first chart. This difference impacts the context provided for interpreting the chart, reducing consistency. **0 points**.\\
\\
5. **Chart Dimensions and Layout**: Both charts appear similar in terms of layout and dimensions, with no significant visual differences observed. **+1 point**.\\
\\
Given the above evaluations, the total score for B is calculated as follows:\\
\\
\#\#\# Total Score:\\
4/5
\end{tcolorbox}
\twocolumn

\onecolumn
\begin{tcolorbox}[colback=blue!5!white, colframe=blue!75!black, title=Example \#2 of VLM-as-a-judge (Model: CodeLlama-13B-Instruct), text width=\textwidth]
Image A:\\
\begin{center}
    \includegraphics[width=0.5\textwidth]{figures/vlm_human19_gt.png}
\end{center}
Image B:\\
\begin{center}
    \includegraphics[width=0.5\textwidth]{figures/vlm_human19_codellama13b.png}
\end{center}
[Prompt for VLM-as-a-judge]\\
\#\#\# Explanation:\\
1. **Data Representation Consistency**: Both figures (A and B) represent the same underlying data with identical values for categories A, B, C, and D. Therefore, 1 point is awarded.\\
2. **Axis Labels and Scales**: The axis labels ("Category" for the x-axis and "Values" for the y-axis) appear to be identical in both images, and the scale for the y-axis seems consistent. Thus, 1 point is awarded.\\
3. **Graphical Elements**: Both charts use blue bars to represent the data, and the thickness and style of the bars are similar. Therefore, 1 point is awarded.\\
4. **Legend and Annotations**: There are no legends or additional annotations in either image, and the titles are the same ("Bar Chart"). Thus, 1 point is awarded.\\
5. **Chart Dimensions and Layout**: The dimensions and overall layout of the two charts differ. Image A has a different aspect ratio compared to Image B, which affects the total layout. Hence, 0 points are awarded.\\
\\
\#\#\# Total Score:\\
4/5
\end{tcolorbox}
\twocolumn

\onecolumn
\begin{tcolorbox}[colback=blue!5!white, colframe=blue!75!black, title=Example \#3 of VLM-as-a-judge (Model: o1-mini), width=\textwidth]
% Column 1
\begin{minipage}[t]{\columnwidth}
\textbf{Image A:}
\begin{center}
    \includegraphics[width=0.6\columnwidth]{figures/vlm_csv29_gt.png}
\end{center}
\textbf{Explanation:}
\begin{enumerate}[leftmargin=*]
    \item \textbf{Data Representation Consistency:}  
    Image A represents true values compared to predictions over a broader date range, while Image B only shows predicted values for a specific date range (2023-06-01 to 2023-07-19). The data sets are not identical. \textbf{(Score: 0)}
    
    \item \textbf{Axis Labels and Scales:}  
    Both images have similar axis labels ("Date" for x-axis, "Value1" for y-axis), but Image B lacks clarity on the date range. Image A provides a wider range, with different intervals. Scales are inconsistent. \textbf{(Score: 0)}
\end{enumerate}
\end{minipage}

% Column 2
\begin{minipage}[t]{\columnwidth}
\textbf{Image B:}
\begin{center}
    \includegraphics[width=0.6\columnwidth]{figures/vlm_csv29_o1mini.png}
\end{center}
\textbf{Explanation (cont.):}
\begin{enumerate}[leftmargin=*]
    \setcounter{enumi}{2} % Continue numbering
    \item \textbf{Graphical Elements:}  
    Image A uses a dashed red line for predictions, while Image B uses a solid blue line. This difference affects visual representation. \textbf{(Score: 0)}
    
    \item \textbf{Legend and Annotations:}  
    Image A includes a detailed legend, while Image B only identifies the predicted values. Titles are also different, affecting context and interpretability. \textbf{(Score: 0)}
    
    \item \textbf{Chart Dimensions and Layout:}  
    Image A has a wider aspect ratio for a broader date range, while Image B is more condensed. \textbf{(Score: 0)}
\end{enumerate}

\textbf{Total Score:} 0/5
\end{minipage}
\end{tcolorbox}
\twocolumn



\subsection{Qualified Prompts}
\begin{itemize}
    \item \textbf{Original Prompt 1:}    

There is a dataset with missing values in a CSV file, which records the region, height, weight, age, and salary of 36 individuals. Please address the following issues: \\(a) Calculate the proportion of missing values in each column and select the rows with at least two non-missing values in the last three columns. \\(b) Please fill in the weight column reasonably by combining the data from the height and region columns.
    
    \item \textbf{Qualified Prompt 1:}

\begin{tcolorbox}[colback=blue!5!white, colframe=blue!75!black, title=Qualified Prompt 1:]
Given a dataset with missing values in a file named \texttt{'data.csv'}which records the region, height, weight, age, and salary of 36 individuals, please address the following issues:\\
(a) Calculate the proportion of missing values in each column and select the rows with at least two non-missing values in the last three columns. Save your output in a CSV file named \texttt{'missing\_values\_proportion\\.csv'}.\\
(b) Fill in the weight column reasonably by combining the data from the height and region columns. Save this updated dataset in a CSV file named \texttt{'updated\_data.csv'}.
\end{tcolorbox}

\item \textbf{Original Prompt 2:}
You are required to analyze and visualize the "Global Terrorism Database" from Kaggle. Please load the dataset and perform data cleaning by handling missing values, removing duplicates, and correcting any anomalies. Conduct an exploratory data analysis (EDA) to understand the distribution and relationships within the dataset. Calculate basic statistical indicators such as mean, median, standard deviation, and provide summary statistics for key features like attack type, target type, and region. Generate visualizations to uncover patterns and insights. Create histograms and box plots to display the distribution of numerical features, and bar plots to show the frequency of categorical variables. Use scatter plots and heatmaps to visualize relationships and correlations between features. Identify patterns in the data related to terrorist activities. For instance, determine trends over time, geographical hotspots, and common attack methods. Use clustering techniques ( K-means clustering) to identify patterns and group similar incidents together.
  

    \item \textbf{Qualified Prompt 2:}


\begin{tcolorbox}[colback=blue!5!white, colframe=blue!75!black, title=Qualified Prompt 2:, text width=\textwidth]
You are required to analyze and visualize the \textit{Global Terrorism Database} from Kaggle. Please follow the steps below:
\begin{enumerate}
    \item Load the dataset \\
    Input: \texttt{gtd.csv} \\
    Output: \texttt{loaded\_data.csv} (This should contain the original data loaded without any modifications.)
    
    \item Data Cleaning 
    \begin{itemize}
        \item Handle missing values
        \item Remove duplicates
        \item Correct anomalies
    \end{itemize}
    Input: \texttt{loaded\_data.csv} \\
    Output: \texttt{cleaned\_data.csv} (This should reflect the cleaned dataset, ready for analysis.)
    
    \item Exploratory Data Analysis (EDA) 
    \begin{itemize}
        \item Calculate basic statistical indicators such as mean, median, and standard deviation
        \item Provide summary statistics for key features (attack type, target type, region)
    \end{itemize}
    Input: \texttt{cleaned\_data.csv} \\
    Output: \texttt{eda\_summary\_statistics.csv} (This should include all calculated statistics for key features.)
    
    \item Generate Visualizations 
    \begin{itemize}
        \item Create histograms and box plots for numerical features
        \item Generate bar plots for categorical variables
        \item Use scatter plots and heatmaps to visualize relationships and correlations
    \end{itemize}
    Input: \texttt{cleaned\_data.csv} \\
    Output: \texttt{visualizations.pdf} (This should include all visualizations generated in a single PDF file.)
    
    \item Identify Patterns in Data Related to Terrorist Activities 
    \begin{itemize}
        \item Determine trends over time
        \item Identify geographical hotspots
        \item Analyze common attack methods
    \end{itemize}
    Input: \texttt{cleaned\_data.csv} \\
    Output: \texttt{patterns\_analysis.csv} (This should summarize the identified patterns, trends, and hotspots.)
    
    \item Clustering Techniques 
    \begin{itemize}
        \item Use K-means clustering to identify patterns and group similar incidents
    \end{itemize}
    Input: \texttt{cleaned\_data.csv} \\
    Output: \texttt{clustering\_results.csv} (This should include the results of the clustering analysis, showing which group each incident belongs to.)
\end{enumerate}

Ensure that each output file reflects the quality of the completion of the respective subtask for further evaluation.
\end{tcolorbox}
    
\end{itemize}


\clearpage\newpage


\subsection{Modified Prompts}

\begin{itemize}
    \item \textbf{Original Prompt 1:}

\begin{tcolorbox}[colback=blue!5!white, colframe=blue!75!black, title=Modified Prompt 1:, text width=\textwidth]
Searches a directory for CSV files matching a given regular expression pattern, reads sales data from these files, and plots the sales data with month on the x-axis and sales on the y-axis.\\
Note that: Each CSV file contains two columns: \texttt{Month} and \texttt{Sales}.\\
The function should output with: \\
A list of \texttt{matplotlib.axes.\_axes.Axes} objects, each representing a plot of sales data from a matched CSV file.\\
You should write self-contained code starting with:
\begin{verbatim}
import os
import pandas as pd
import re
import matplotlib.pyplot as plt
def task_func(directory: str, pattern: str) -> list:
\end{verbatim}
\end{tcolorbox}



    \item \textbf{Modified Prompt 1:}

\onecolumn
\begin{tcolorbox}[colback=blue!5!white, colframe=blue!75!black, title=Modified Prompt 1:, text width=\textwidth]
Search a directory for CSV files matching a given regular expression pattern, read sales data from these files, and plot the sales data with month on the x-axis and sales on the y-axis.\\

Input Requirements:
\begin{itemize}
    \item Input Directory: \texttt{data}.
    \item Input Pattern: \texttt{"csv\_\textbackslash d+\textbackslash.csv"}.
\end{itemize}

Output Requirements:
\begin{enumerate}
    \item A list of \texttt{matplotlib.axes.\_axes.Axes} objects representing the plot of sales data from each matched CSV file.
    \item Save each plot as a separate image file:
    \begin{itemize}
        \item File format: PNG
        \item Output filenames: \texttt{"sales\_plot\_<filename>.png"} where \texttt{<filename>} is the name of the CSV file without the extension.
    \end{itemize}
\end{enumerate}
Input File Specification:
\begin{itemize}
    \item Each CSV file should contain two columns: \texttt{'Month'} and \texttt{'Sales'}. The input files will be located in the specified directory.
\end{itemize}

You should write self-contained code starting with:
\begin{verbatim}
import os
import pandas as pd
import re
import matplotlib.pyplot as plt

def task_func(directory: str, pattern: str) -> list:
\end{verbatim}

\end{tcolorbox}

\clearpage
\newpage

    
    \item \textbf{Original Prompt 2:}

\begin{tcolorbox}[colback=blue!5!white, colframe=blue!75!black, title=Modified Prompt 1:, text width=\textwidth]
Plot a scatter graph of tuples and highlight the tuple with the maximum value at index 1.\\
The function should output with: \\
\texttt{matplotlib.axes.Axes}: The Axes object of the plot for further manipulation and testing, with the title 'Max Tuple Highlighted', x-axis labeled 'x', y-axis labeled 'y', and a legend.\\
You should write self-contained code starting with:
\begin{verbatim}
import numpy as np
from operator import itemgetter
import matplotlib.pyplot as plt
def task_func(data):
\end{verbatim}
\end{tcolorbox}



    \item \textbf{Modified Prompt 2:}
    
\begin{tcolorbox}[colback=blue!5!white, colframe=blue!75!black, title=Modified Prompt 2:, text width=\textwidth]
Plot a scatter graph of tuples and highlight the tuple with the maximum value at index 1 using the input data from \texttt{"data.csv"}. The function should output the following: \\

A scatter plot saved as \texttt{"scatter\_plot.png"} with the title \texttt{'Max Tuple Highlighted'}, x-axis labeled \texttt{'x'}, y-axis labeled \texttt{'y'}, and a legend. The highlighted point should signify the tuple with the maximum value at index 1. \\

Please write self-contained code starting with:
\begin{verbatim}
import numpy as np
from operator import itemgetter
import matplotlib.pyplot as plt
def task_func(data):
\end{verbatim}
\end{tcolorbox}

\end{itemize}



\subsection{Prompt examples of different difficulty levels}
\begin{tcolorbox}[colback=blue!5!white, colframe=blue!75!black, title=Easy-level Prompt \#1:, text width=\textwidth]
\{ \\
    \texttt{"prompt":} \texttt{"}Read the dataset (input file: \texttt{"}Fish.csv\texttt{"}). Encode the dataset to divide it into training and test sets. From the dataset's four categories of Bream, Roach, Parkki, and Perch, randomly select 2 samples from each category for the test set. The remaining samples will be used as the training set. Output the number of samples in the training and test sets in a CSV file named \texttt{"}sample\_counts.csv\texttt{"}.\\ \\ Next, implement the KNN algorithm with K values of 1, 5, 15, and 100 to classify all samples in the test set. Output the classification results of the test samples to a CSV file named \texttt{"}classification\_results.csv\texttt{"}.\texttt{"}, 
    \\
    \\
    \texttt{"data\_source\_type"}: \texttt{"}2=open source data\texttt{"}
\\
\}
\end{tcolorbox}


\begin{tcolorbox}[colback=blue!5!white, colframe=blue!75!black, title=Easy-level Prompt \#2:, text width=\textwidth]
\{ 
\\
    \texttt{"prompt":} \texttt{"}Using the dataset of house prices and economic indicators provided below, please complete the following subtasks. Each subtask should generate an output file for evaluation, with specific input files where applicable.\\ \\ Dataset: \texttt{"}data.csv\texttt{"}\\ \\Subtasks:\\ \\ 1. Gaussian Naive Bayes Training: Train a Gaussian Naive Bayes model on the cleaned dataset and evaluate its accuracy.  \\   Input: \texttt{"}data.csv\texttt{"}  \\   Output: \texttt{"}gaussian\_model\_accuracy.csv\texttt{"} (contains accuracy metrics)\\ \\2. Multinomial Naive Bayes Training: Train a Multinomial Naive Bayes model on the cleaned dataset and evaluate its accuracy.  \\   Input: \texttt{"}data.csv\texttt{"}  \\   Output: `multinomial\_model\_accuracy.csv` (contains accuracy metrics)\\ \\ 3. Comparison of Models: Generate a comparative analysis of the accuracy of both models in a visual format.  \\   Input: \texttt{"}gaussian\_model\_accuracy.csv\texttt{"}, \texttt{"}multinomial\_model\_accuracy.csv\texttt{"}  \\   Output: \texttt{"}model\_comparison.png\texttt{"} \\ \\ Please ensure that the outputs contain necessary details for further evaluation.\texttt{"}, \\ \\
    \texttt{"data\_source\_type"}: \texttt{"}3=human written data\texttt{"}
\\
\}
\end{tcolorbox}


\begin{tcolorbox}[colback=blue!5!white, colframe=blue!75!black, title=Medium-level Prompt:, text width=\textwidth]
\{ \\
    \texttt{"prompt":} \texttt{"}Given a DataFrame \texttt{'df'} with sales data containing columns: TransactionID, ProductID, Quantity, SaleDate, StoreID, Revenue:\\\\1. Handle missing Revenue values: Impute any missing Revenue values with the median of the Revenue column.  \\   Input file: \texttt{'data.csv'}  \\   Output: \texttt{'step1\_imputed\_revenue.csv'} (contains the DataFrame after handling missing values)\\\\2. Detect and replace outliers: Detect and replace outliers in Quantity and Revenue. Outliers are defined as values below the 1st percentile or above the 99th percentile. Replace them with the corresponding 1st or 99th percentile value instead of removing them.  \\   Input file: \texttt{'step1\_imputed\_revenue.csv'}  \\   Output: \texttt{'step2\_replaced\_outliers.csv'} (contains the DataFrame after outlier replacement)\\\\3. Normalize Quantity and Revenue: Normalize Quantity and Revenue using Z-score normalization.  \\   Input file: \texttt{'step2\_replaced\_outliers.csv'}  \\   Output: \texttt{'step3\_normalized\_data.csv'} (contains the DataFrame after normalization)\\\\4. Ensure SaleDate format: Ensure SaleDate is in datetime format.  \\   Input file: \texttt{'step3\_normalized\_data.csv'}  \\   Output: \texttt{'step4\_formatted\_dates.csv'} (contains the DataFrame after ensuring datetime format)\\\\5. Encode ProductID and StoreID: Encode the ProductID and StoreID columns using separate label encoders to avoid any potential overlap in numerical values between categories from different columns.  \\   Input file: \texttt{'step4\_formatted\_dates.csv'}  \\   Output: \texttt{'final\_cleaned\_data.csv'} (contains the final cleaned DataFrame)\\\\Perform the specified data cleaning and preprocessing tasks and output the cleaned DataFrame as the final result.\texttt{"}, \\\\
    \texttt{"data\_source\_type"}: \texttt{"}3=human written data\texttt{"}
\\
\}
\end{tcolorbox}


\begin{tcolorbox}[colback=blue!5!white, colframe=blue!75!black, title=Hard-level Prompt:, text width=\textwidth]
\{ \texttt{"prompt":} \texttt{"}Write a Graph Recurrent Neural Network (GRNN) model based on attention mechanisms using Python for processing and analyzing time series data. Ensure to meet the following requirements:\\ \\1. \texttt{"}Graph network design\texttt{"}: Create a graph network where each graph represents an aerial formation, and the number of nodes corresponds to the number of vehicles in the formation.\\   - \texttt{"}Output\texttt{"}: Graph structure representation file.\\   - \texttt{"}Output filename\texttt{"}: \texttt{"}graph\_structure.json\texttt{"}\\ \\2. \texttt{"}Data format\texttt{"}: The time series data of aerial target formations is stored in Excel files located in the \texttt{"}.data/\texttt{"} directory, where each Excel file contains multiple sheets, with each sheet representing the time series data of a vehicle.\\   - \texttt{"}Input\texttt{"}: Excel files from \texttt{"}..data/\texttt{"} directory.\\   - \texttt{"}Input filename\texttt{"}: from \texttt{"}data\_1.xlsx\texttt{"} to \texttt{"}data\_5.xlsx\texttt{"}\\ \\3. \texttt{"}Data reading\texttt{"}: Read all Excel files from the \texttt{"}..data/\texttt{"} directory and extract the sheets for processing.\\   - \texttt{"}Output\texttt{"}: Combined time series data in a structured format.\\   - \texttt{"}Output filename\texttt{"}: \texttt{"}combined\_data.csv\texttt{"}\\ \\ 4. \texttt{"}Data segmentation\texttt{"}: Segment the time series data of each vehicle using a sliding window with length \texttt{"}l\texttt{"} and step size \texttt{"}s\texttt{"}.\\   - \texttt{"}Output\texttt{"}: Segmented time series data.\\   - \texttt{"}Output filename\texttt{"}: \texttt{"}segmented\_data.csv\texttt{"}\\\\5. \texttt{"}Training and testing set division\texttt{"}: Divide the segmented data into training and testing sets with a 7:3 ratio.\\   - \texttt{"}Output\texttt{"}: Training and testing datasets.\\   - \texttt{"}Output filenames\texttt{"}: \texttt{"}train\_data.csv\texttt{"}, \texttt{"}test\_data.csv\texttt{"} \\\\6. \texttt{"}Labels\texttt{"}: The last column of each Excel file contains label data, which should be preserved during the reading and segmentation process. \\\\7. \texttt{"}Model training\texttt{"}: Train the GRNN model using the training set to identify the action intentions of aerial target formations.\\   - \texttt{"}Output\texttt{"}: Trained model file.\\   - \texttt{"}Output filename\texttt{"}: \texttt{"}trained\_model.h5\texttt{"}\\\\8. \texttt{"}Training parameters\texttt{"}: Set the number of training epochs to 60, and record the loss value, accuracy, recall, and F1-Score for each epoch.\\   - \texttt{"}Output\texttt{"}: Training metrics log file.\\   - \texttt{"}Output filename\texttt{"}: \texttt{"}training\_metrics.csv\texttt{"}\\\\9. \texttt{"}Model evaluation\texttt{"}: Evaluate the trained model with the test set and generate a classification report and confusion matrix.\\   - \texttt{"}Output\texttt{"}: Classification report and confusion matrix visualization.\\   - \texttt{"}Output filenames\texttt{"}: \texttt{"}classification\_report.txt\texttt{"}, \texttt{"}confusion\_matrix.png\texttt{"} \\ \\ Please provide a complete code implementation and ensure that the code structure is clear and well-commented for understanding and evaluation.\texttt{"}, \\ \texttt{"data\_source\_type"}: \texttt{"}3=human written data\texttt{"}\}
\end{tcolorbox}
\twocolumn


\clearpage\newpage
\section{Error Analysis and Case Study}
We have analyzed the common errors during problem-solving, summarized them here, and then studied several cases, including a successful case and a failure case.
The common errors include:
\begin{enumerate}
    \item Coding errors when solving data science problems using codes. And based on our observation, the main kind of these is execution error. It may be due to different reasons. For example, hallucination on the column name of a csv file.
    \item Json format errors. These errors come from the agent framework side, where they use json format to wrap up actions, e.g. \texttt{WriteAnalysis}.
\end{enumerate}



\subsection{Successful cases}

\input{appendix/4_success_1}

% \onecolumn
\begin{tcolorbox}[colback=blue!5!white, colframe=blue!75!black, title=Successful case \#2 from Claude-3-5-sonnet-20240620, text width=\textwidth]
\textbf{Output}: The following example successfully fulfills the requirement of the given prompt by completing the function provided in the prompt and write the function to \texttt{most\_corr\_output.csv}.
\begin{verbatim}
[
    {
        'task_id': '1',
        'dependent_task_ids': [],
        'instruction': "Load and preprocess the input data from 
        'data.csv'",
        'task_type': 'preprocessing and cleaning',
        'code': '''
        import pandas as pd
        import numpy as np

        # Load the data from 'data.csv'
        prices = pd.read_csv('../data.csv')

        # Set the first column (assuming it's a date column) as
        the index
        prices.set_index(prices.columns[0], inplace=True)

        # Calculate daily percentage changes
        pct_changes = prices.pct_change()

        # Drop the first row (NaN values due to pct_change 
        calculation)
        pct_changes = pct_changes.dropna()

        print("Data loaded and preprocessed. Shape of \
        percentage changes:", pct_changes.shape)
        print("Columns (tickers):", \
        pct_changes.columns.tolist())
        print("First few rows of percentage changes:")
        print(pct_changes.head())
        ''',
        'result': "Data loaded and preprocessed. Shape of \
        percentage changes: (29, 4)
        Columns (tickers): ['Ticker2', 'Ticker3', 'Ticker4',\
        'Ticker5']
        First few rows of percentage changes:
        Ticker2   Ticker3   Ticker4   Ticker5
        Ticker1                                        
        65.60   -0.615289  0.108929  0.002184  0.841616
        52.06    1.633757 -0.024740 -0.353101 -0.435643
        68.34   -0.452888 -0.230862  0.308297  0.160458
        111.19  -0.204800 -0.227069 -0.070287  0.208418
        128.52   0.094136  0.280394  0.260849 -0.428407
        ",
        'is_success': True,
        'is_finished': True
    },
\end{verbatim}
\end{tcolorbox}
\twocolumn

\onecolumn
\vspace{-1cm}
\begin{tcolorbox}[colback=blue!5!white, colframe=blue!75!black, title=Successful case \#3 from Claude-3-5-sonnet-20240620, text width=\textwidth]
\begin{verbatim}
    {
        'task_id': '2',
        'dependent_task_ids': ['1'],
        'instruction': 'Implement the most_corr function to 
        find the most correlated pair of tickers',
        'task_type': 'data mining',
        'code': '''
        def most_corr(prices):
            """
            :param prices: (pandas.DataFrame) A dataframe 
            containing each ticker's 
                        daily closing prices.
            :returns: (container of strings) A container, 
            containing the two tickers that 
                    are the most highly (linearly) correlated 
                    by daily percentage change.
            """
            # Calculate percentage changes
            pct_changes = prices.pct_change().dropna()
            
            # Calculate correlation matrix
            corr_matrix = pct_changes.corr()
            
            # Set diagonal and lower triangle to NaN to avoid 
            self-correlation and duplicates
            corr_matrix = corr_matrix.where(np.triu( \
            np.ones(corr_matrix.shape), k=1).astype(bool))
            
            # Find the location of the maximum correlation
            max_corr = corr_matrix.max().max()
            max_corr_idx = corr_matrix.stack().idxmax()
            
            # Get the tickers with the highest correlation
            ticker1, ticker2 = max_corr_idx
            ...
            [ommited]

        result = most_corr(prices)
        print(f"The most correlated pair of tickers: {result}")

        output = pd.read_csv('./most_corr_output.csv')
        print("\nContents of most_corr_output.csv:")
        print(output)''',
        'result': "The most correlated pair of tickers: ...
        [omitted]
        'is_success': True,
        'is_finished': True
    },
\end{verbatim}
\end{tcolorbox}
\twocolumn

\begin{tcolorbox}[colback=blue!5!white, colframe=blue!75!black, title=Successful case \#4 from Claude-3-5-sonnet-20240620, text width=\textwidth]
\begin{verbatim}
    {
        'task_id': '3',
        'dependent_task_ids': ['2'],
        'instruction': "Generate and save the output CSV file 
        'most_corr_output.csv'",
        'task_type': 'other',
        'code': '''
        import pandas as pd

        # Read the output CSV file
        output_df = pd.read_csv('./most_corr_output.csv')
        
        # Display the contents of the file
        print("Contents of most_corr_output.csv:")
        print(output_df)
        
        # Verify that the file contains the expected columns
        expected_columns = ['Ticker 1', 'Ticker 2']
        assert all(col in output_df.columns for col in \
        expected_columns),"Output file is missing expected \
        columns"
        
        # Verify that the file contains exactly one row
        assert len(output_df) == 1,"Output file should \
        contain exactly one row"
        
        print("Output file verification complete. The file \
        meets the expected format.")''',
        'result': 'Contents of most_corr_output.csv:
        Ticker 1 Ticker 2
        0  Ticker4  Ticker5
        Output file verification complete. The file meets the 
        expected format.
        ',
        'is_success': True,
        'is_finished': True
    }
]
\end{verbatim}
\end{tcolorbox}




\subsection{Failure Case}
\label{appendix: worse_example}
\onecolumn
\begin{tcolorbox}[colback=blue!5!white, colframe=blue!75!black, title=Failed Example \#1 Generated by o1-mini, text width=\textwidth]
\textbf{Prompt}:\\
Draw normal distributions for multiple 'x' and 'y' arrays with labels. Each pair (x, y) represents a different chemical compound in the 'labels' list. \textbf{The function should output with: tuple(fig: Matplotlib.figure.Figure) fig: Matplotlib figure object containing the drawn normal distributions.} You should write self-contained code starting with:

\begin{verbatim}
import matplotlib.pyplot as plt
import numpy as np
import scipy.stats as stats
def task_func(x, y, labels):
\end{verbatim}

\textbf{Completion}:\\
\begin{verbatim}
import matplotlib.pyplot as plt
import numpy as np
import scipy.stats as stats
import pandas as pd
 
def task_func(x, y, labels):
    # Create a DataFrame from the input arrays
    data = pd.DataFrame({
        'x': x,
        'y': y,
        'label': labels
    }).copy()
   
    # Detect and handle missing values
    if data.isnull().values.any():
        data = data.dropna()
   
    # Remove duplicate entries
    data = data.drop_duplicates()
   
    # Detect outliers using Z-score and remove them
    numeric_cols = ['x', 'y']
    z_scores = np.abs(stats.zscore(data[numeric_cols]))
    data = data[(z_scores < 3).all(axis=1)]
   
    # Ensure that the lengths of x, y, and labels are aligned \ 
    after preprocessing
    x_processed = data['x'].values
    y_processed = data['y'].values
    labels_processed = data['label'].values
   
return x_processed, y_processed, labels_processed \ 
# error: the answer does not comply with required output instructions.
\end{verbatim}

\textbf{Reason}:\\
The answer does not comply with the required output instruction.
\end{tcolorbox}
\onecolumn
\begin{tcolorbox}[colback=blue!5!white, colframe=blue!75!black, title=Failed Example \#2 Generated by o1-mini, text width=\textwidth]
\textbf{Prompt}:\\
Reads data from a CSV file and generates a bar plot based on grouped mean values. The DataFrame is grouped by the column named 'col1\_name', and the mean for each group is calculated for the column 'col2\_name'. A bar plot is created using matplotlib. Each bar in the plot represents a group, and its height corresponds to the mean value of 'col2\_name' for that group. The plot is then configured with a title and axis labels: - The title is set as 'Mean of [col2\_name] Grouped by [col1\_name]'. This format dynamically inserts the names of the columns being analyzed into the title. - The xlabel (label for the x-axis) is set to the name of the column used for grouping (col1\_name). - The ylabel (label for the y-axis) is set as 'Mean of [col2\_name]', indicating that the y-axis represents the mean values of the specified column. Note that: Ensure that the CSV file exists at the specified path and has the required columns. The function does not handle missing data. Ensure that the CSV file has clean and complete data for accurate results. The bar plot is customizable using matplotlib's functionality after the function returns the Axes object. \textbf{The function should output with: tuple(ax: matplotlib.axes.Axes, df\_mean: pandas.DataFrame) ax: The Axes object of the generated bar plot. This object can be used to further customize the plot, like adding labels or changing styles. df\_mean: The DataFrame containing the mean values calculated for each group.} You should write self-contained code starting with:
\begin{verbatim}
import pandas as pd
import matplotlib.pyplot as plt
def task_func(csv_file_path, col1_name="column1", col2_name="column2"):
\end{verbatim}

\small
\textbf{Completion}:\\
\begin{verbatim}
import pandas as pd
import matplotlib.pyplot as plt
 
def task_func(csv_file_path, col1_name="column1", col2_name="column2"):
    # Read the CSV file into a pandas DataFrame
    df = pd.read_csv(csv_file_path)
    # Create a copy to preserve the original dataset
    df_clean = df.copy()
    # Handle missing values
    # Drop rows with any missing values
df_clean.dropna(inplace=True)
    # Remove duplicate entries  
    df_clean.drop_duplicates(inplace=True)
    # Handle outliers in 'col2_name' using the IQR method
    Q1 = df_clean[col2_name].quantile(0.25)
    Q3 = df_clean[col2_name].quantile(0.75)
    IQR = Q3 - Q1
    lower_bound = Q1 - 1.5 * IQR
    upper_bound = Q3 + 1.5 * IQR
    df_clean = df_clean[(df_clean[col2_name] >= lower_bound) \
    & (df_clean[col2_name] <= upper_bound)]  
    # Standardize data formats if necessary
    # Example: If 'col1_name' contains date information, \ 
    uncomment the following line
    # df_clean[col1_name] = pd.to_datetime(df_clean[col1_name])
   
return df_clean # error: the answer failed to output the required \ 
elements mentioned in the instruction.
\end{verbatim}
\textbf{Reason}:\\
The answer failed to output the required elements mentioned in the instruction.
\end{tcolorbox}
\onecolumn
\begin{tcolorbox}[colback=blue!5!white, colframe=blue!75!black, title=Failed Example \#3 Generated by o1-mini, text width=\textwidth]
\textbf{Prompt}:\\
Perform K-Means clustering on the given DataFrame using the sklearn KMeans algorithm. The function expects a DataFrame with numerical values, as KMeans cannot handle categorical data. It applies standard KMeans clustering from the sklearn library to form clusters. The number of clusters is configurable via the 'n\_clusters' parameter, defaulting to 3. The Number of times the k-means algorithm is run with different centroid seeds (n\_init) is set to 10. The function returns an array of cluster labels corresponding to each data point in the input as well as the fitted KMeans model. >>> data = pd.DataFrame({ ... 'a': [1, 20, 2, 22, 100], ... 'b': [1, 20, 2, 22, 100] ... }) >>> labels, model = task\_func(data, seed=213) >>> print(labels) [2 0 2 0 1] >>> print(model) KMeans(n\_clusters=3, n\_init=10, random\_state=213). The function should raise the exception for: ValueError: If the DataFrame contains non numeric entries. The function should output with: tuple(labels: numpy.ndarray, model: sklearn.cluster.KMeans) labels: An array of integers (cluster labels) corresponding to the input data. Each label is an integer representing the cluster to which a row of data has been assigned. model: The fitted KMeans Model. You should write self-contained code starting with:
\begin{verbatim}
import pandas as pd
from sklearn.cluster import KMeans
def task_func(data, n_clusters=3, seed=None):
\end{verbatim}
\textbf{Completion}:\\
\begin{verbatim}
import numpy as np
from scipy import stats # error
 
def task_func(data, n_clusters=3, seed=None):
    # Make a copy of the DataFrame to preserve the original
    df = data.copy()
   
    # Check if all columns are numeric
    if not all(pd.api.types.is_numeric_dtype(dtype) \ 
    for dtype in df.dtypes):
        raise ValueError("The DataFrame contains non-numeric entries.")
   
    # Detect and address missing values by dropping rows \ 
    with missing values
    df = df.dropna()
   
    # Remove duplicate entries
    df = df.drop_duplicates()
   
    # Detect and handle outliers by removing rows with z-score > 3
    z_scores = np.abs(stats.zscore(df)) 
    # error: the model failed to call sklearn.cluster functions, \
    instead only calling zscore to remove outliers.
    df = df[(z_scores < 3).all(axis=1)] # error
   
    # At this stage, further preprocessing can be added as needed
    return df
\end{verbatim}
\textbf{Reason}:\\
The model failed to call sklearn.cluster functions, instead of only calling zscore to remove outliers.
\end{tcolorbox}
\onecolumn
\begin{tcolorbox}[colback=blue!5!white, colframe=blue!75!black, title=Failed Example \#4 Generated by o1-mini, text width=\textwidth]
\textbf{Prompt}:\\
Reads data from a CSV file and generates a bar plot based on grouped mean values. The DataFrame is grouped by the column named 'col1\_name', and the mean for each group is calculated for the column 'col2\_name'. A bar plot is created using matplotlib. Each bar in the plot represents a group, and its height corresponds to the mean value of 'col2\_name' for that group. The plot is then configured with a title and axis labels: - The title is set as 'Mean of [col2\_name] Grouped by [col1\_name]'. This format dynamically inserts the names of the columns being analyzed into the title. - The xlabel (label for the x-axis) is set to the name of the column used for grouping (col1\_name). - The ylabel (label for the y-axis) is set as 'Mean of [col2\_name]', indicating that the y-axis represents the mean values of the specified column. Note that: Ensure that the CSV file exists at the specified path and has the required columns. The function does not handle missing data. Ensure that the CSV file has clean and complete data for accurate results. The bar plot is customizable using matplotlib's functionality after the function returns the Axes object. The function should output with: tuple(ax: matplotlib.axes.Axes, df\_mean: pandas.DataFrame) ax: The Axes object of the generated bar plot. This object can be used to further customize the plot, like adding labels or changing styles. df\_mean: The DataFrame containing the mean values calculated for each group. You should write self-contained code starting with:
\begin{verbatim}
import pandas as pd 
import matplotlib.pyplot as plt 
def task_func(csv_file_path, col1_name="column1", col2_name="column2"):
\end{verbatim}
\small
\textbf{Completion}:\\
\begin{verbatim}
# error: the model forgot to import relevant libraries.
def task_func(csv_file_path, col1_name="column1", col2_name="column2"):
    # Read the CSV file
    df = pd.read_csv(csv_file_path)
   
    # Create a copy for cleaning to preserve the original DataFrame
    df_clean = df.copy()
   
    # 1. Handle Missing Values
    # Check for missing values
    missing_counts = df_clean.isnull().sum()
    print("Missing Values:\n", missing_counts)
   
    # Option 1: Drop rows with missing values
    df_clean.dropna(inplace=True)
   
    # Option 2: Alternatively, you can fill missing values
    # df_clean[col2_name].fillna(df_clean[col2_name].mean(), inplace=True)
   
    # 2. Remove Duplicate Entries
    duplicates = df_clean.duplicated().sum()
    print(f"Number of duplicate rows: {duplicates}")
    df_clean.drop_duplicates(inplace=True)
   
    # 3. Handle Outliers
    # Example using Z-score for numerical columns
    numeric_cols = df_clean.select_dtypes(include=['float64', 'int64']).columns
    from scipy import stats
    z_scores = np.abs(stats.zscore(df_clean[numeric_cols]))
    df_clean = df_clean[(z_scores < 3).all(axis=1)]
   
    # 4. Standardize Data Formats
    # Example: Standardize date columns
    for col in df_clean.columns:
        if 'date' in col.lower():
            df_clean[col] = pd.to_datetime(df_clean[col], errors='coerce')
\end{verbatim}
\end{tcolorbox}

\onecolumn
\begin{tcolorbox}[colback=blue!5!white, colframe=blue!75!black, title=Failed Example \#4 Generated by o1-mini, text width=\textwidth]
\small
\begin{verbatim}
    # Drop rows with invalid date formats after conversion
    df_clean.dropna(subset=[col for col in df_clean.columns 
    if 'date' in col.lower()], inplace=True)
    
    # 5. Additional Preprocessing (if needed)
    # For example, convert categorical columns to consistent case
    categorical_cols = df_clean.select_dtypes(include=['object']).columns
    for col in categorical_cols:
        df_clean[col] = df_clean[col].str.strip().str.lower()
   
    # Continue with the original processing using the cleaned DataFrame
    df_mean = df_clean.groupby(col1_name)[col2_name].mean().reset_index()
    fig, ax = plt.subplots()
    ax.bar(df_mean[col1_name], df_mean[col2_name], color='skyblue')
    ax.set_title(f"Mean of {col2_name} Grouped by {col1_name}")
    ax.set_xlabel(col1_name)
    ax.set_ylabel(f"Mean of {col2_name}")
    plt.xticks(rotation=45)
    plt.tight_layout()
    plt.show()
   
    print("Basic Statistical Indicators:")
    print(df_clean.describe())
    print("\nMedian Values:")
    print(df_clean.median())
    print("\nStandard Deviation:")
    print(df_clean.std())
   
    df_clean[numeric_cols].hist(bins=10, figsize=(10, 8), edgecolor='black')
    plt.suptitle("Histograms of Numerical Columns")
    plt.tight_layout(rect=[0, 0.03, 1, 0.95])
    plt.show()
   
    df_clean.boxplot(column=numeric_cols, figsize=(10, 6))
    plt.title("Box Plots of Numerical Columns")
    plt.tight_layout()
    plt.show()
   
    corr_matrix = df_clean.corr()
    fig, ax = plt.subplots(figsize=(8, 6))
    cax = ax.matshow(corr_matrix, cmap='coolwarm')
    plt.title('Correlation Matrix Heatmap', pad=20)
    fig.colorbar(cax)
    ticks = range(len(corr_matrix.columns))
    ax.set_xticks(ticks)
    ax.set_yticks(ticks)
    ax.set_xticklabels(corr_matrix.columns, rotation=90)
    ax.set_yticklabels(corr_matrix.columns)
    plt.tight_layout()
    plt.show()
   
    return ax, df_mean
\end{verbatim}
\textbf{Reason}:\\
The model forgot to import relevant libraries.
\end{tcolorbox}



\onecolumn
\begin{tcolorbox}[colback=blue!5!white, colframe=blue!75!black, title=A failure case of execution error, text width=\textwidth]
\textbf{Prompt}:\\
I currently have a structured dataset named date.csv with 6 columns and 5960 rows of data. The first column is named X, the second column is named Y, both of which are structural parameters set for simulation. The following four columns, ff, ft, tt, and tf, represent the simulation results. Now, I need to train a model that can accurately predict the four simulation results for other structural parameters. The model and related content will be saved in the results folder. The model should be saved as "model.pkl" and the final accuracy should be saved in "acc.txt".\\
\textbf{One of the Generated Functions}:
\begin{verbatim}
# Import necessary libraries
import pandas as pd
import numpy as np
from sklearn.model_selection import train_test_split
from sklearn.preprocessing import StandardScaler
import matplotlib.pyplot as plt
from mpl_toolkits.mplot3d import Axes3D
from scipy.optimize import minimize

# Read in the input data from the 'data.csv' file
data = pd.read_csv('../data.csv')

# Preprocess the data by removing any unnecessary columns and 
handling missing values
data.dropna(inplace=True)
X = data.drop('target', axis=1)
y = data['target']
... (omitted)
\end{verbatim}
\textbf{Execution Error}:
\begin{verbatim}
-> 6992         raise KeyError(f"{labels[mask].tolist()} not 
found in axis")
   6993     indexer = indexer[~mask]
   6994 return self.delete(indexer)

KeyError: "['target'] not found in axis"
\end{verbatim}
\end{tcolorbox}

\onecolumn
\begin{tcolorbox}[colback=blue!5!white, colframe=blue!75!black, title=Generated by CodeLlama-34B-Instruct in one task:, text width=\textwidth]
\begin{verbatim}
[PYTHON]
# Define the context and task
context = {
    "data": "https://s3.amazonaws.com/benbensaid-public-data/data.csv",
    "target": "y"
}

task = {
    "task_type": "data exploration",
    "instruction": "Explore the data and identify any patterns or trends."
}

# Generate the plan
plan = generate_plan(context, task)

# Print the plan
print(plan)
[/PYTHON]
\end{verbatim}
\end{tcolorbox}
% \onecolumn
\begin{tcolorbox}[colback=blue!5!white, colframe=blue!75!black, title=Generated by StarCoder2-15B in one task:, text width=\textwidth]
\begin{verbatim}
Sure, here's a plan for your context:
1. Find the videos with titles containing "how" or "what"
2. Calculate the like ratios
3. Visualize the like ratios
\end{verbatim}
\end{tcolorbox}
\twocolumn


\begin{itemize}
\item \textbf{The expected format should at least begin and end with three backticks and must comply with JSON format. Here's an example:}


\begin{tcolorbox}[colback=blue!5!white, colframe=blue!75!black, title=Expected JSON format:, text width=\textwidth]
\begin{verbatim}
"
[
    {
        "task_id": "1",
        "dependent_task_ids": [],
        "instruction": "Preprocess and clean the data.",
        "task_type": "preprocessing and cleaning"
    },
    {
        "task_id": "2",
        "dependent_task_ids": ["1"],
        "instruction": "Explore the data to find 
        patterns and insights.",
        "task_type": "data exploration"
    },
    {
        "task_id": "3",
        "dependent_task_ids": ["2"],
        "instruction": "Visualize the data to communicate insights.",
        "task_type": "data visualization"
    },
    {
        "task_id": "4",
        "dependent_task_ids": ["3"],
        "instruction": "Generate an interpretability report.",
        "task_type": "interpretability report"
    }
]
"
\end{verbatim}
\end{tcolorbox}

\end{itemize}