\section{Improving the Blob Market}
The blob market is still in its early days and we have not consistently observed blob demand at the target. Still, type-3 transaction senders collectively pay a significant fee -- especially during congested periods such as the Blobscriptions craze and the LayerZero airdrop. However, most fees paid by type-3 transactions are either the EIP-4844 base fee or the EIP-1559 base fee, and only a small proportion of these fees is collected by the block's fee recipient. As a result, the financial incentives to include blobs are minimal. Even more so, the incentives for infrastructure investments are small and we presume that as a result, we have observed slow infrastructure development regarding blobs. One example would be the inefficiencies in blob packing we observe. Similarly, Geth (i.e., the biggest execution layer client) released a version with a bug in the blob mempool and it took nearly ten days to notice and fix~\parencite{szilagyi2024,szilagyi2024tweet2,szilagyi2024tweet3,szilagyi2024tweet4}. As a result of this bug, clients only had access to a small fraction of blob transactions. A similar bug for non-blob transactions persisting in the biggest execution layer client for more than ten days appears almost unimaginable. 

These incidents highlight the need for infrastructure investment in the blob market, for which the financial incentives currently do not appear to suffice. 

\subsection{Blob (Priority) Fees}
Some of the following steps could be taken to ensure that the ecosystem has the incentives to improve the blob market. 

\paragraph{Multidimensional (priority) fee market} Since the Decun hard fork, Ethereum has implemented a multidimensional gas market, separating the blob fee market from the general fee market. There are ongoing discussions to further divide non-blob gas usage into categories like computation, storage, and bandwidth~\parencite{buterin2024multidimensional}. Adding priority fees for each dimension would help scale fees appropriately during congestion. This approach addresses issues like those seen during the LayerZero airdrop, where EIP-1559 priority fees did not adjust adequately (see Appendix~\ref{app:prio}).

\paragraph{Increasing the blob target gradually} There are ongoing discussions within the Ethereum community about increasing the blob target soon. This increase is crucial for scaling the Ethereum ecosystem. However, a rapid target increase risks a mismatch, where demand takes months to catch up, resulting in extremely lower fees for type-3 transactions. A gradual, steady increase may help balance supply and demand. Nonetheless, creating an appropriate schedule for such a gradual increase presents its own challenges.

\paragraph{Speedup price discovery} During the LayerZero airdrop the base fee took more than six hours for price discovery to be achieved. This slow price discovery, in part, is a result of the minimum fee per blob gas being low and the maximum increase per block being 12.5\%. Thus, it is no surprise that price discovery took several hours when the base fee had to increase by 15 orders of magnitude. To address this issue, EIP-7762 seeks to increase the minimum blob fee per gas~\parencite{resnick2024eip7762,resnick2024tweet}. As a result of setting this minimum higher, the base fee would start higher and not have to move as much in times of congestion. Beyond that, this increase should not greatly impact blob transaction senders as they are still expected to pay less in the EIP-4844 gas market than they already do in the EIP-1559 gas market~\parencite{dataalways2024minimumblobfees}.

\subsection{Blob and Block Packing}
In addition to providing financial incentives to invest in the blob market, we must also outline the specific actions we can take.

\paragraph{Better packing algorithms} Our work shows that non-PBS and PBS blocks are often packed inefficiently. While some of this may result from builders avoiding blocks with blobs due to delayed propagation in the trusted-relay PBS system, many observed inefficiencies cannot be explained by this alone. If financial incentives in the blob market increase, better block-packing algorithms should emerge, especially for blob transactions. Importantly, block packing in a multidimensional market is complex, still, our proposed strategy shows potential improvements of up to 70\%. Since highly specialized builders currently construct most blocks through PBS, this added complexity is unlikely to pose a barrier.

\paragraph{Subset bidding} \Cref{subsec:market} makes the case for changing the market design for blob transactions. To fully resolve this structural market design issue that we point out, blobs would either need to go through a modified (more flexible) implementation of a mempool, or the transaction standard would need to be adjusted.
The fix that we delineate below is relatively straightforward in its description, even though it would require pervasive changes within most Ethereum clients: in its most general form, the standard should allow bidding for any subset of blobs in a transaction, and only a subset of those to be included on-chain (with the rest skipped in the final transaction).
Another way of implementing this change is to include transaction signatures for multiple subsets of blobs, in such a way that they are mutually exclusive so that only one can be chosen to be included on-chain at a time.

A special solution to this market design challenge has been introduced by Titan Builder, enabling senders to submit multiple permutations of blob transactions~\parencite{titanbuilderxyz2024api,titanbuilderxyz2024}. These blobs are then individually sorted to find the optimal combination before being included in a block.
This feature would attempt to alleviate the issue in \Cref{subsec:market} without changing the way type-3 transactions work (only for blobs submitted \emph{privately} through Titan Builder, when they win the block). While this improvement appears promising, we do not find signs of it being utilized to the extent it could and we conjecture that this might be related to the absence of congested periods since the feature's launch; we refer the interested reader to Appendix~\ref{app:private} for more details.



















\section{Conclusion}

In conclusion,
while the innovation of blob transactions through Ethereum's EIP-4844 offers promising advantages, including cost reductions for L2 data settlement, our study reveals several inefficiencies.
Such inefficiencies in block packing and economic incentives seem to have emerged as issues expected to be critical in a world where the temporary storage of blobs is dominant for Ethereum. We documented periods of up to 70\% fee loss due to suboptimal blob packing, highlighting the need for better utilization of blob space which would result in both higher builder profits and less blob inclusion delay. The lack of significant financial incentives for builders to prioritize blob transactions likely contributes to these inefficiencies, as most fees associated with blob transactions are burnt, leaving builders with minimal rewards.

This work provides valuable insights into the early days of blobs on Ethereum, shedding light on both the potential and challenges of this scaling solution. As Ethereum continues to evolve, we hope for these findings to inform future network upgrades, contributing to more efficient and scalable decentralized systems.

