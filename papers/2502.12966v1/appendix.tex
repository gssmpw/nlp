\section{Private Blob Transactions}
\label{app:private}

We use mempool data from the Mempool Dumpster project~\parencite{flashbots2024mempool} to assess whether a transaction was submitted privately or not. In Figure~\ref{fig:private}, we plot the daily share of privately submitted blob transactions. Initially, around April 2024, we observe up to 20\% of blob transactions being submitted privately. We could not identify the reason for this but presume that it is either blob transaction senders testing out various ways of submitting their transactions or the mempool loggers not being set up properly for the new transaction type yet. From May onwards we observe almost no privately submitted transactions until the start of September 2024. From then on the share or private blob transactions rose to 10\% to 15\%. The reason is that Taiko started to submit its blob transactions privately. As we highlight with the yellow dashed light essentially all private transactions from September onwards were submitted by Taiko.  

\begin{figure}[h]
\centering

\begin{minipage}[t]{.48\linewidth}
    \includegraphics[scale=1.2]{fc/figures/private.pdf}
    \caption{Proportion of privately submitted blob transactions over time. We further highlight those blob transactions privately submitted by Taiko. Notice that Taiko is responsible for nearly all privately submitted blob transactions from September onwards.}
    \label{fig:private}
\end{minipage}
\hfill
\begin{minipage}[t]{.48\linewidth}
     \includegraphics[scale=1.2]{fc/figures/titan.pdf}
    \caption{The daily average number of blobs per block for Titan Builder. Titan Builder implemented a new RPC call allowing the submission of multiple permutations for blob transactions on 5 July 2024 -- marked by the vertical grey line.  }
    \label{fig:titan}
\end{minipage}
\end{figure}

A further factor that could impact the share of privately submitted blob transactions is Titan Builder's special RPC call. The call allows subset bidding, offering more flexibility and promising higher efficiency during congested periods. Titan Builder implemented the call on 5 July 2024. Transactions submitted in this fashion would likely not hit the mempool as the additional flexibility gains would then be lost. 

In Figure~\ref{fig:titan}, we plot the daily average number of blobs per block built by Titan Builder and further indicate with the vertical grey line the data on which the RPC call was deployed. However, we did not notice an increase in the number of blobs included by Titan Builder in response to the call being implemented. Furthermore, there is also no increase in the share of private blob transactions around that time (see Figure~\ref{fig:private}), and the only rise in private transactions comes from Taiko who would not benefit from the implemented call. Taiko only submits a single blob per transaction (see Figure~\ref{fig:numBlobs}). Thus, the call does not seem to be utilized yet but this could be due to there not being any congested periods since it was implemented. 

\section{Priority Fees of Blob Transactions}\label{app:prio}



\begin{figure}[t!]
    \centering
    \begin{subfigure}[t]{0.48\columnwidth}
        \includegraphics[scale=1.2]{fc/figures/Blast_prio.pdf}
    \caption{Blast}
    \label{fig:Blast_prio}
    \end{subfigure}\hfill
    \begin{subfigure}[t]{0.48\columnwidth}
        \includegraphics[scale=1.2]{fc/figures/starknet_prio.pdf}
    \caption{Starknet}
    \label{fig:starknet_prio}
    \end{subfigure}    
    \begin{subfigure}[t]{0.48\columnwidth}
        \includegraphics[scale=1.2]{fc/figures/Linea_prio.pdf}
    \caption{Linea}
    \label{fig:Linea_prio}
    \end{subfigure}\hfill
    \begin{subfigure}[t]{0.48\columnwidth}
        \includegraphics[scale=1.2]{fc/figures/zksync_prio.pdf}
    \caption{zkSync}
    \label{fig:zksync_prio}
    \end{subfigure}    
    \caption{Distribution of priority fee per gas for blob transactions depending on the number of blobs per transaction for Blast (see Figure~\ref{fig:Blast_prio}), Starknet (see Figure~\ref{fig:starknet_prio}), Linea (see Figure~\ref{fig:Linea_prio}), and zkSync (see Figure~\ref{fig:zksync_prio}). }\label{fig:prio_dist}
\end{figure}

In the following, we dive further into the EIP-1559 priority fees of blob transactions. Recall, that priority fees do not scale with the number of blobs but with the EIP-1559 gas usage. We place a particular focus on the senders that submit blob transactions with different numbers of blobs simultaneously for extended periods: Blast, Starknet, Linea, and zkSync (see Figure~\ref{fig:numBlobs}). Given that these rollups submit different numbers of blobs per transaction at the same time, we can compare the priority fees they indicate when focusing on those periods in particular. 


In Figure~\ref{fig:prio_dist}, we plot the distribution of the priority fee per gas for blob transactions from these senders and restrict the analysis to periods where the respective sender was submitting differing numbers of blobs per transaction. We find that the priority fee per gas is not strongly positively correlated with the number of blobs for any of the rollups we look at. Instead, we the correlation to be either weakly positive for Blast, no correlation for Starknet, and weakly negative for Linea and zkSync. 












Nonetheless, it is important to highlight that gas usage---especially for ZK rollups---can correlate with the number of blobs. As a result, the priority fee might also scale with the number of blobs and the priority fee per gas would not necessarily be the correct metric to consider. Thus, we calculate total the priority fee for each blob transaction. 

We first investigate the distribution of this measure for each of the biggest blob users in Figure~\ref{fig:total_prio_per_rollup}. Here we notice that Taiko pays the highest priority fee per transaction on average, with zkSync, Linea, and Arbitrum also paying more than average in this measure. The remaining rollups have a quite small total priority fee. For some (i.e., Base, Optimism, and Blast), this is due to them being optimistic rollups and using very little gas (see Figure~\ref{fig:gas_used}), while for Scroll and Starknet this is due to them paying little priority fees per gas in general (see Figure~\ref{fig:prio}).

\begin{figure}[t]
    \centering
    \includegraphics[scale=1.2]{fc/figures/prio_per_blob.pdf}
    \caption{Distribution of total priority fee for blob transactions depending on the transaction sender.}
    \label{fig:total_prio_per_rollup}
\end{figure}

We also investigate the total priority fee for the four rollups submitting blob transactions with various numbers of blobs simultaneously for an extended period (see Figure~\ref{fig:prio_per_num_blobs}).
If rollups set the priority fee per gas in such a way that correlates positively with the number of blobs they would like to include, then we should expect to see this positive correlation between the total priority fee and the number of blobs in Figure~\ref{fig:prio_per_num_blobs}. However, we observe that this is not generally the case. Instead, for three of the rollups (Blast, Linea, and zkSync), we observe a generally insignificant (mildly negative or positive) correlation between the total priority fee and the number of blobs. We do observe a strong positive correlation of 0.92 for Starknet, but importantly correlation between the gas usage and the number of blobs is also exactly 0.92. Thus, the EIP-1559 gas usage is correlated with the blob gas usage for the rollup, and even without explicitly setting the priority fee in a way that scales with the number of blobs this would occur implicitly. Together this analysis indicates that the rollups do not appear to set the priority fee in a way that scales with the number of blobs they include in a transaction.


\begin{figure}[t]
    \centering
    \begin{subfigure}[t]{0.48\columnwidth}
        \includegraphics[scale=1.2]{fc/figures/Blast_prio_per_blob.pdf}
    \caption{Blast}
    \label{fig:Blast_prio_per_blob}
    \end{subfigure}\hfill
    \begin{subfigure}[t]{0.48\columnwidth}
        \includegraphics[scale=1.2]{fc/figures/starknet_prio_per_blob.pdf}
    \caption{Starknet}
    \label{fig:starknet_prio_per_blob}
    \end{subfigure}    
    \begin{subfigure}[t]{0.48\columnwidth}
        \includegraphics[scale=1.2]{fc/figures/Linea_prio_per_blob.pdf}
    \caption{Linea}
    \label{fig:Linea_prio_per_blob}
    \end{subfigure}\hfill
    \begin{subfigure}[t]{0.48\columnwidth}
        \includegraphics[scale=1.2]{fc/figures/zksync_prio_per_blob.pdf}
    \caption{zkSync}
    \label{fig:zksync_prio_per_blob}
    \end{subfigure}    
    \caption{Distribution of total priority fee for blob transactions depending on the number of blobs per transaction for Blast (see Figure~\ref{fig:Blast_prio_per_blob}), Starknet (see Figure~\ref{fig:starknet_prio_per_blob}), Linea (see Figure~\ref{fig:Linea_prio_per_blob}), and zkSync (see Figure~\ref{fig:zksync_prio_per_blob}). }\label{fig:prio_per_num_blobs}
\end{figure}

Such behavior is predicted especially in times when the blob market is congested~\parencite{buterin2024multidimensional,dataalways2024minimumblobfees}. The intuition is that when the blob market is congested and the blob base fee is still in price discovery, the blob transaction senders compete in a first-price auction. Given that blobs are the congested resource in this example, they would adjust the priority fee in a way that it would scale with the blob gas usage (i.e., the number of blobs) and then compete on the priority fee per blob measure. 

\begin{figure}[H]%
\centering
    \includegraphics[scale=1.2]{fc/figures/zksync_prio_per_blob_airdrop.pdf}
  \caption{Distribution of total priority fee for blob transactions depending on the number of blobs per transaction for zkSync during the LayerZero airdrop.}\label{fig:zksync_prio_per_blob_airdrop}
\end{figure}

The previous analysis, however, did not focus on congested periods. Thus, to analyze whether we observe this predicted behavior from the rollups we focus on zkSync during the congested period surrounding the LayerZero airdrop. zkSync was the only rollup already submitting blob transactions with different numbers of blobs during that period. We plot the distribution of total priority fee for blob transactions depending on the number of blobs per transaction in Figure~\ref{fig:zksync_prio_per_blob_airdrop}.
We find that the priority fee per blob seems more or less constant independent of the number of blobs, there is only a mild positive correlation of 0.14 which is not much higher than zkSync's correlation between gas usage and the number of blobs at 0.10. Thus, zkSync does not seem to be adjusting the priority fee for the number of blobs. As a result, our analysis is (partially) inconclusive in regard to how the rollups set their priority fees during congested periods but find little to no signs of them doing so. 
