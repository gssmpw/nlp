\section{Introduction}

\newcommand{\nume}{num\'eraire\xspace}

\subsection{The Blob Fee Market}

Ethereum's journey towards enhanced scalability has been a central focus in its development roadmap~\parencite{buterin2024tweet}, driven by the need to address increasing network demand while preserving decentralization and security. As part of this effort, \textit{Ethereum Improvement Proposal (EIP)} 4844 introduced a novel transaction type that enables the use of \textit{blobs}---large, temporary data objects that are particularly critical for \textit{layer-2 (L2)} solutions. These blobs, by virtue of their enshrined short lifespans, are designed to offer a significant reduction in the cost of storing L2 data on-chain, thus aligning with Ethereum's broader goal of scaling through L2s.

The introduction of blobs under EIP-4844 also brought about a transformative shift in Ethereum's fee market by establishing a separate yet intertwined \emph{blob fee market}, a critical step in the paradigm of multi-dimensional resource pricing. Previously, there was a unified transaction fee priced after fixing the relative prices of all operations (including storage ones) in multiples of a single unit of account (the so-called gas unit), which was then multiplied by the fee given by the transaction originator per gas unit. Such a technique, however, inhibits a scenario where the relative prices of resources shift regimes through time, and would be particularly inflexible with very distinct resources constantly shifting demand like execution and (temporary) storage. With the advent of blob transactions, Ethereum also introduced a new, parallel, dynamically varying blob gas fee (the inner workings of which we elucidate in \Cref{subsec:blob_fees}) that varies separately from the (traditional) EIP-1559 \textit{execution gas fee}.

Our key focus is this \emph{blob fee market} and economic outcomes within this type of temporary storage, especially as they relate to rollups. While this fee market significantly reduced transaction fees for rollups, it also introduced complexities in terms of transaction structuring, optimal block packing, and economic incentives.

\subsection{Main Contributions}

This work delivers the first rigorous and comprehensive analysis of the Ethereum blob fee market during its initial months, empirically studying it and the behavior of its participants. Our extensive data collection and empirical analysis allows obtaining novel insights on the market internals of this temporary storage and contributes to the ongoing, active debate about the blob space on Ethereum.

Analyzing transaction- and mempool-level data we look into behavioral patterns of the market participants and their effects on-chain.
Two major events that spiked demand for blobs may provide a clearer picture of what a congested blob market would look like (especially in a world like the roadmap envisions with more active utilization of blobspace) and we hone in on these periods.

We show that a surprisingly large percentage of blocks are suboptimally packed with blob transactions, leading to up to 70\% fee losses for the builders who build these suboptimal blocks compared to the optimally built block.
We document that the market has been becoming slightly more efficient over time, but is still stunningly inefficient. We believe the reason for this to be the underinvestment in needed infrastructure due to the low amount of direct economic incentives.
Further than that, in \Cref{subsec:market}, we identify a structural market design flaw having to do with the rigid structure of transactions preventing the most efficient use of available blob space, that we term subset bidding.
We offer solutions and improvement suggestions for these inefficiencies of the market, as well as point out fruitful future avenues of research.
Importantly, we note that the market structure problem and solution we identify does \emph{not only} apply to blobs included in transactions, but \emph{any} potentially discrete ``object'' allocated in the same all-or-nothing way through the standard Ethereum transaction format.













\section{Background}

In the following, we introduce the relevant background regarding blobs on Ethereum (see \Cref{subsec:intro_blobs}), the fees paid by transactions including them (see \Cref{subsec:blob_fees}), and Ethereum block building (see \Cref{subsec:block_building}).

\subsection{Blobs}
\label{subsec:intro_blobs}
To tackle Ethereum scalability issues, EIP-4844 introduced a new type of transactions---referred to as \textit{type-3 transactions} or \textit{blob transactions}---which allow the sender to submit blobs of data. L2s primarily use these blobs for transaction settlement on the L1. Blobs only persist for a short period (around 18 days) on the network, long enough for them to be retrieved but short enough to prevent excessive long-term usage of storage. This temporary nature allows blobs to be priced lower than calldata,\footnote{Before blobs were introduced, L2s used Ethereum calldata for settlement of L2 data on the L1.} which must be permanently stored on the blockchain and contribute to state growth. Up to six blobs may be included on each block, and a type-3 transaction can have between one and six blobs. There is space for 128kb of data in each blob and a transaction always pays for the entire blob even if it only uses a partial portion of the allowable storage space it provides.

Similarly to all other types of transactions, type-3 transactions that are publicly broadcast are stored in Ethereum's execution layer network \textit{mempool}, i.e., the public waiting area for transactions. Importantly, the blob data itself is not propagated in the execution layer network. Instead, blobs are propagated in the consensus layer network. Thus, type-3 transactions propagated through the execution layer network and included on-chain will only contain a reference to the blob.


\subsection{Blob Fees}\label{subsec:blob_fees}
Type-3 transactions simultaneously pay gas fees in: (1) the ``normal'' Ethereum gas market, and (2) the blob gas market.

Similar to other types of transactions, type-3 transactions can include calldata, transfer Ether, or interact with a smart contract. The fees paid by a type-3 transaction $tx$ included in block $n$ in the ``normal'' gas market, which we will refer to as the EIP-1159 gas market throughout, are as follows.

$$ \textsc{fee}_{1559} (tx,n) = \textsc{gas}(tx) \cdot (\textsc{base\_fee}(n) + \textsc{priority\_fee}(tx)),$$
where $\textsc{gas}(tx)$ is the transaction gas usage. Further, $\textsc{base\_fee}(n)$ is the block's base fee charged per unit of gas: it represents the minimum fee transactions must pay in the block and automatically updates based on past block gas usage~\parencite{ethereum2024eip1559}.  Finally, $ \textsc{priority\_fee}(tx)$ is the transaction's priority fee charged per unit of gas. Importantly, the part of the fee paid by transaction $tx$ associated with the base fee (i.e., $\textsc{gas}(tx) \cdot \textsc{base\_fee}(n)$) is burned. Only the remaining fees ($\textsc{gas}(tx) \cdot \textsc{priority\_fee}(tx)$ are received by the block's fee recipient. 

Additionally, type-3 transactions also pay fees per blob included in the blob gas market, which we will refer to as the EIP-4844 gas market throughout. To be precise, the fees a transaction $tx$ included in block $n$ pays in the EIP-4844 gas market are
$$ \textsc{fee}_{4844} (tx,n) = \textsc{num\_blobs}(tx) \cdot \textsc{blob\_base\_fee}(n),$$
where $\textsc{num\_blobs}(tx) $ is the number of blobs included by the transaction and $ \textsc{blob\_base\_fee}(n)$ is the blob base fee in block $n$ which is charged per blob gas. The blob base fee for block $n$ is derived as follows
$$\textsc{blob\_base\_fee}(n)=\textsc{min\_fee}\cdot \exp{\left(\frac{\textsc{total\_excess\_gas}(n-1)}{\textsc{update\_fraction}}\right)},$$
where $\textsc{total\_excess\_gas}(n-1)$ is the total blob gas used in excess of the target before the current block. Note that while there is space for six blobs per block, the target is three. Thus, whenever more than three blobs are included in a block the excess increases and decreases when less than three are included. Additionally, $\textsc{min\_fee}$ is a constant currently set to 1 wei, and $\textsc{update\_fraction}$ is a constant set such that the maximum increase in the blob base fee per block is 12.5\%~\parencite{ethereum2024eip4844}. Importantly, all fees associated with the EIP-4844 gas market are burned, i.e., the block's fee recipient receives no fees for blobs included.

\subsection{Block Building}\label{subsec:block_building}
The vast majority of blocks ($\approx$90\%) in Ethereum are built through a scheme called \textit{Proposer-Builder Separation (PBS)}~\parencite{wahrstatter2024}. With PBS the validator chosen as the block proposer is only responsible for proposing the block, while specialized \textit{builders} are responsible for building the blocks. The idea is that these specialized builders are better at building high-value blocks, i.e., blocks with significant fee revenue. Additionally, these specialized builders likely have access to value private order flow, i.e., transactions that are not broadcast to the public mempool. Note that in the current implementation of the scheme, validators and builders communicate with each other through a relay: a party trusted by the two~\parencite{flashbots2024}. 



\section{Related Work}
\label{subsec:litrev}


\paragraph{EIP-1559 fee market} The first major shift in the Ethereum transaction fee mechanism was the deployment of EIP-1559 in 2021. Before deployment, \textcite{roughgarden2020transaction} presented a game-theoretic analysis of the mechanism, while \textcite{leonardos2021dynamical,ferreira2021dynamic} conducted studies focusing on the dynamic update rule of the base fee. On the empirical side, multiple studies \parencite{reijsbergen2021transaction,liu2022empirical,leonardos2023optimality} demonstrate that the introduction of EIP-1559 made gas fees more predictable despite the short-term oscillation in block size. In addition, for farsighted validators, it can be rational to attack the mechanism leading to greater unpredictability of fees~\parencite{hougaard2023farsighted,azouvi2023base}. On the other hand, our work focuses on the blob fee market, which was introduced through EIP-4844. 




\paragraph{Multidimensional fee markets} A recent line of literature explores multidimensional blob markets. \textcite{diamandis2023designing} propose an efficient pricing mechanism for multiple resources, while \textcite{angeris2024multidimensional} show that these multidimensional blockchain fee markets are essentially optimal. In contrast, we empirically explore the blob fee market, which added a second dimension to Ethereum's fee market.

\paragraph{Blob fee market} \textcite{crapis2023eip} conducted an economic analysis of the blob market at its launch, examining whether rollups would prefer posting data in blobs or calldata (the original market). Their study concludes that large rollups with high transaction rates would opt for blobs, while those with lower rates would favor the original market. Our empirical findings, however, show that even when the blob market is congested and more expensive than calldata, rollups do not revert to the original market.

\textcite{soltani2024delay} analyze the delays of blob transactions in a \emph{time-average} fashion to conclude that average delays are lower if all transactions only carry a small number of blobs; this is a consequence of mainly standard limited-resource congestion. Contrary to that, one of our multifaceted contributions deals with examining the delays as a consequence of inefficiencies in block packing, where we study how this has changed \emph{throughout time} and its sensitivity, especially in demand-congested periods.

In the period before the introduction of blobs, \textcite{messias2024writing} performed a comprehensive transaction analysis focusing on inscriptions, which triggered the first significant demand for blob space on Ethereum. Their work, however, does not delve into the internal workings of the blob market, which is the central focus of our study.

A recent research post performs an empirical evaluation of the blob market, focusing on the blob base fees paid and the consequences of increasing the parameter of the minimum base fee~\parencite{dataalways2024minimumblobfees}. Concurrent work by \textcite{lee2024180} explores the possibility of rollups sharing blob space among each other (a practice called \textit{blob sharing}) and simulates the impact using the first 180 days of blobs on Ethereum. Their goal is to minimize the costs for rollups. In contrast, our work focuses on the \emph{blob fee market} as a whole and its participants (with an emphasis on \emph{block builders}) while also identifying and documenting market design issues, offering potential solutions to those.
Finally, likewise, auction considerations have been previously connected to market mechanisms, including in decentralized finance infrastructure \parencite{myersonian_mm,complexity,nftauctions}.

