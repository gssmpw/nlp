\section{Related Work}
\label{subsec:litrev}


\paragraph{EIP-1559 fee market} The first major shift in the Ethereum transaction fee mechanism was the deployment of EIP-1559 in 2021. Before deployment, \textcite{roughgarden2020transaction} presented a game-theoretic analysis of the mechanism, while \textcite{leonardos2021dynamical,ferreira2021dynamic} conducted studies focusing on the dynamic update rule of the base fee. On the empirical side, multiple studies \parencite{reijsbergen2021transaction,liu2022empirical,leonardos2023optimality} demonstrate that the introduction of EIP-1559 made gas fees more predictable despite the short-term oscillation in block size. In addition, for farsighted validators, it can be rational to attack the mechanism leading to greater unpredictability of fees~\parencite{hougaard2023farsighted,azouvi2023base}. On the other hand, our work focuses on the blob fee market, which was introduced through EIP-4844. 




\paragraph{Multidimensional fee markets} A recent line of literature explores multidimensional blob markets. \textcite{diamandis2023designing} propose an efficient pricing mechanism for multiple resources, while \textcite{angeris2024multidimensional} show that these multidimensional blockchain fee markets are essentially optimal. In contrast, we empirically explore the blob fee market, which added a second dimension to Ethereum's fee market.

\paragraph{Blob fee market} \textcite{crapis2023eip} conducted an economic analysis of the blob market at its launch, examining whether rollups would prefer posting data in blobs or calldata (the original market). Their study concludes that large rollups with high transaction rates would opt for blobs, while those with lower rates would favor the original market. Our empirical findings, however, show that even when the blob market is congested and more expensive than calldata, rollups do not revert to the original market.

\textcite{soltani2024delay} analyze the delays of blob transactions in a \emph{time-average} fashion to conclude that average delays are lower if all transactions only carry a small number of blobs; this is a consequence of mainly standard limited-resource congestion. Contrary to that, one of our multifaceted contributions deals with examining the delays as a consequence of inefficiencies in block packing, where we study how this has changed \emph{throughout time} and its sensitivity, especially in demand-congested periods.

In the period before the introduction of blobs, \textcite{messias2024writing} performed a comprehensive transaction analysis focusing on inscriptions, which triggered the first significant demand for blob space on Ethereum. Their work, however, does not delve into the internal workings of the blob market, which is the central focus of our study.

A recent research post performs an empirical evaluation of the blob market, focusing on the blob base fees paid and the consequences of increasing the parameter of the minimum base fee~\parencite{dataalways2024minimumblobfees}. Concurrent work by \textcite{lee2024180} explores the possibility of rollups sharing blob space among each other (a practice called \textit{blob sharing}) and simulates the impact using the first 180 days of blobs on Ethereum. Their goal is to minimize the costs for rollups. In contrast, our work focuses on the \emph{blob fee market} as a whole and its participants (with an emphasis on \emph{block builders}) while also identifying and documenting market design issues, offering potential solutions to those.
Finally, likewise, auction considerations have been previously connected to market mechanisms, including in decentralized finance infrastructure \parencite{myersonian_mm,complexity,nftauctions}.