\begin{algorithm}[t]
    \caption{Tree Reconstruction Attack (TRA)}
    \label{alg:TRA}
    \begin{algorithmic}[1]
       \STATE \textbf{Input:} Oracle \( \mathcal{O}_d \), target model \( f : \mathcal{X} \mapsto \mathcal{Y}\).
       \STATE \( \mathcal{Q} \gets \{\mathcal{X}\} \) \COMMENT{Initialize query list with the entire input space}
       \REPEAT
           \STATE \( \mathcal{E} \gets \mathcal{Q}.pop(0) \) \COMMENT{Retrieve the next input subset to investigate}
           \STATE \( x \gets center(\mathcal{E}) \) \COMMENT{Compute the center point of \( \mathcal{E} \)}
           \STATE \( y \gets f(x) \) \COMMENT{Obtain the label of the center point}
           \IF{ \( \mathcal{O}_d(f, x, \mathcal{E}) \) exists }
                \STATE \( x' \gets \mathcal{O}_d(f, x, \mathcal{E}) \) \COMMENT{Obtain counterfactual explanation}
                \STATE \( \mathcal{Q} \gets \mathcal{Q} \cup split(\mathcal{E}, x, x') \)
                \COMMENT{Split \( \mathcal{E} \) based on \( x' \) and add the resulting new subspaces to \( \mathcal{Q} \)}\label{line:split}
           \ELSE
                \STATE Assign label \( y \) to the subset \( \mathcal{E} \) \COMMENT{No counterfactual found; \( \mathcal{E} \) is a leaf node}
            \ENDIF
       \UNTIL{ \( \mathcal{Q} \) is empty }
    \end{algorithmic}
\end{algorithm}