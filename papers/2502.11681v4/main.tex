% This must be in the first 5 lines to tell arXiv to use pdfLaTeX, which is strongly recommended.
\pdfoutput=1
% In particular, the hyperref package requires pdfLaTeX in order to break URLs across lines.

\documentclass[11pt]{article}

% Change "review" to "final" to generate the final (sometimes called camera-ready) version.
% Change to "preprint" to generate a non-anonymous version with page numbers.
\usepackage[preprint]{acl}

% Standard package includes
\usepackage{times}
\usepackage{latexsym}

% For proper rendering and hyphenation of words containing Latin characters (including in bib files)
\usepackage[T1]{fontenc}
% For Vietnamese characters
% \usepackage[T5]{fontenc}
% See https://www.latex-project.org/help/documentation/encguide.pdf for other character sets

% This assumes your files are encoded as UTF8
\usepackage[utf8]{inputenc}

% This is not strictly necessary, and may be commented out,
% but it will improve the layout of the manuscript,
% and will typically save some space.
\usepackage{microtype}

% This is also not strictly necessary, and may be commented out.
% However, it will improve the aesthetics of text in
% the typewriter font.
\usepackage{inconsolata}

%Including images in your LaTeX document requires adding
%additional package(s)
\usepackage{graphicx}

% Optional math commands from https://github.com/goodfeli/dlbook_notation.
%%%%% NEW MATH DEFINITIONS %%%%%

\usepackage{amsmath,amsfonts,bm}
\usepackage{derivative}
% Mark sections of captions for referring to divisions of figures
\newcommand{\figleft}{{\em (Left)}}
\newcommand{\figcenter}{{\em (Center)}}
\newcommand{\figright}{{\em (Right)}}
\newcommand{\figtop}{{\em (Top)}}
\newcommand{\figbottom}{{\em (Bottom)}}
\newcommand{\captiona}{{\em (a)}}
\newcommand{\captionb}{{\em (b)}}
\newcommand{\captionc}{{\em (c)}}
\newcommand{\captiond}{{\em (d)}}

% Highlight a newly defined term
\newcommand{\newterm}[1]{{\bf #1}}

% Derivative d 
\newcommand{\deriv}{{\mathrm{d}}}

% Figure reference, lower-case.
\def\figref#1{figure~\ref{#1}}
% Figure reference, capital. For start of sentence
\def\Figref#1{Figure~\ref{#1}}
\def\twofigref#1#2{figures \ref{#1} and \ref{#2}}
\def\quadfigref#1#2#3#4{figures \ref{#1}, \ref{#2}, \ref{#3} and \ref{#4}}
% Section reference, lower-case.
\def\secref#1{section~\ref{#1}}
% Section reference, capital.
\def\Secref#1{Section~\ref{#1}}
% Reference to two sections.
\def\twosecrefs#1#2{sections \ref{#1} and \ref{#2}}
% Reference to three sections.
\def\secrefs#1#2#3{sections \ref{#1}, \ref{#2} and \ref{#3}}
% Reference to an equation, lower-case.
\def\eqref#1{equation~\ref{#1}}
% Reference to an equation, upper case
\def\Eqref#1{Equation~\ref{#1}}
% A raw reference to an equation---avoid using if possible
\def\plaineqref#1{\ref{#1}}
% Reference to a chapter, lower-case.
\def\chapref#1{chapter~\ref{#1}}
% Reference to an equation, upper case.
\def\Chapref#1{Chapter~\ref{#1}}
% Reference to a range of chapters
\def\rangechapref#1#2{chapters\ref{#1}--\ref{#2}}
% Reference to an algorithm, lower-case.
\def\algref#1{algorithm~\ref{#1}}
% Reference to an algorithm, upper case.
\def\Algref#1{Algorithm~\ref{#1}}
\def\twoalgref#1#2{algorithms \ref{#1} and \ref{#2}}
\def\Twoalgref#1#2{Algorithms \ref{#1} and \ref{#2}}
% Reference to a part, lower case
\def\partref#1{part~\ref{#1}}
% Reference to a part, upper case
\def\Partref#1{Part~\ref{#1}}
\def\twopartref#1#2{parts \ref{#1} and \ref{#2}}

\def\ceil#1{\lceil #1 \rceil}
\def\floor#1{\lfloor #1 \rfloor}
\def\1{\bm{1}}
\newcommand{\train}{\mathcal{D}}
\newcommand{\valid}{\mathcal{D_{\mathrm{valid}}}}
\newcommand{\test}{\mathcal{D_{\mathrm{test}}}}

\def\eps{{\epsilon}}


% Random variables
\def\reta{{\textnormal{$\eta$}}}
\def\ra{{\textnormal{a}}}
\def\rb{{\textnormal{b}}}
\def\rc{{\textnormal{c}}}
\def\rd{{\textnormal{d}}}
\def\re{{\textnormal{e}}}
\def\rf{{\textnormal{f}}}
\def\rg{{\textnormal{g}}}
\def\rh{{\textnormal{h}}}
\def\ri{{\textnormal{i}}}
\def\rj{{\textnormal{j}}}
\def\rk{{\textnormal{k}}}
\def\rl{{\textnormal{l}}}
% rm is already a command, just don't name any random variables m
\def\rn{{\textnormal{n}}}
\def\ro{{\textnormal{o}}}
\def\rp{{\textnormal{p}}}
\def\rq{{\textnormal{q}}}
\def\rr{{\textnormal{r}}}
\def\rs{{\textnormal{s}}}
\def\rt{{\textnormal{t}}}
\def\ru{{\textnormal{u}}}
\def\rv{{\textnormal{v}}}
\def\rw{{\textnormal{w}}}
\def\rx{{\textnormal{x}}}
\def\ry{{\textnormal{y}}}
\def\rz{{\textnormal{z}}}

% Random vectors
\def\rvepsilon{{\mathbf{\epsilon}}}
\def\rvphi{{\mathbf{\phi}}}
\def\rvtheta{{\mathbf{\theta}}}
\def\rva{{\mathbf{a}}}
\def\rvb{{\mathbf{b}}}
\def\rvc{{\mathbf{c}}}
\def\rvd{{\mathbf{d}}}
\def\rve{{\mathbf{e}}}
\def\rvf{{\mathbf{f}}}
\def\rvg{{\mathbf{g}}}
\def\rvh{{\mathbf{h}}}
\def\rvu{{\mathbf{i}}}
\def\rvj{{\mathbf{j}}}
\def\rvk{{\mathbf{k}}}
\def\rvl{{\mathbf{l}}}
\def\rvm{{\mathbf{m}}}
\def\rvn{{\mathbf{n}}}
\def\rvo{{\mathbf{o}}}
\def\rvp{{\mathbf{p}}}
\def\rvq{{\mathbf{q}}}
\def\rvr{{\mathbf{r}}}
\def\rvs{{\mathbf{s}}}
\def\rvt{{\mathbf{t}}}
\def\rvu{{\mathbf{u}}}
\def\rvv{{\mathbf{v}}}
\def\rvw{{\mathbf{w}}}
\def\rvx{{\mathbf{x}}}
\def\rvy{{\mathbf{y}}}
\def\rvz{{\mathbf{z}}}

% Elements of random vectors
\def\erva{{\textnormal{a}}}
\def\ervb{{\textnormal{b}}}
\def\ervc{{\textnormal{c}}}
\def\ervd{{\textnormal{d}}}
\def\erve{{\textnormal{e}}}
\def\ervf{{\textnormal{f}}}
\def\ervg{{\textnormal{g}}}
\def\ervh{{\textnormal{h}}}
\def\ervi{{\textnormal{i}}}
\def\ervj{{\textnormal{j}}}
\def\ervk{{\textnormal{k}}}
\def\ervl{{\textnormal{l}}}
\def\ervm{{\textnormal{m}}}
\def\ervn{{\textnormal{n}}}
\def\ervo{{\textnormal{o}}}
\def\ervp{{\textnormal{p}}}
\def\ervq{{\textnormal{q}}}
\def\ervr{{\textnormal{r}}}
\def\ervs{{\textnormal{s}}}
\def\ervt{{\textnormal{t}}}
\def\ervu{{\textnormal{u}}}
\def\ervv{{\textnormal{v}}}
\def\ervw{{\textnormal{w}}}
\def\ervx{{\textnormal{x}}}
\def\ervy{{\textnormal{y}}}
\def\ervz{{\textnormal{z}}}

% Random matrices
\def\rmA{{\mathbf{A}}}
\def\rmB{{\mathbf{B}}}
\def\rmC{{\mathbf{C}}}
\def\rmD{{\mathbf{D}}}
\def\rmE{{\mathbf{E}}}
\def\rmF{{\mathbf{F}}}
\def\rmG{{\mathbf{G}}}
\def\rmH{{\mathbf{H}}}
\def\rmI{{\mathbf{I}}}
\def\rmJ{{\mathbf{J}}}
\def\rmK{{\mathbf{K}}}
\def\rmL{{\mathbf{L}}}
\def\rmM{{\mathbf{M}}}
\def\rmN{{\mathbf{N}}}
\def\rmO{{\mathbf{O}}}
\def\rmP{{\mathbf{P}}}
\def\rmQ{{\mathbf{Q}}}
\def\rmR{{\mathbf{R}}}
\def\rmS{{\mathbf{S}}}
\def\rmT{{\mathbf{T}}}
\def\rmU{{\mathbf{U}}}
\def\rmV{{\mathbf{V}}}
\def\rmW{{\mathbf{W}}}
\def\rmX{{\mathbf{X}}}
\def\rmY{{\mathbf{Y}}}
\def\rmZ{{\mathbf{Z}}}

% Elements of random matrices
\def\ermA{{\textnormal{A}}}
\def\ermB{{\textnormal{B}}}
\def\ermC{{\textnormal{C}}}
\def\ermD{{\textnormal{D}}}
\def\ermE{{\textnormal{E}}}
\def\ermF{{\textnormal{F}}}
\def\ermG{{\textnormal{G}}}
\def\ermH{{\textnormal{H}}}
\def\ermI{{\textnormal{I}}}
\def\ermJ{{\textnormal{J}}}
\def\ermK{{\textnormal{K}}}
\def\ermL{{\textnormal{L}}}
\def\ermM{{\textnormal{M}}}
\def\ermN{{\textnormal{N}}}
\def\ermO{{\textnormal{O}}}
\def\ermP{{\textnormal{P}}}
\def\ermQ{{\textnormal{Q}}}
\def\ermR{{\textnormal{R}}}
\def\ermS{{\textnormal{S}}}
\def\ermT{{\textnormal{T}}}
\def\ermU{{\textnormal{U}}}
\def\ermV{{\textnormal{V}}}
\def\ermW{{\textnormal{W}}}
\def\ermX{{\textnormal{X}}}
\def\ermY{{\textnormal{Y}}}
\def\ermZ{{\textnormal{Z}}}

% Vectors
\def\vzero{{\bm{0}}}
\def\vone{{\bm{1}}}
\def\vmu{{\bm{\mu}}}
\def\vtheta{{\bm{\theta}}}
\def\vphi{{\bm{\phi}}}
\def\va{{\bm{a}}}
\def\vb{{\bm{b}}}
\def\vc{{\bm{c}}}
\def\vd{{\bm{d}}}
\def\ve{{\bm{e}}}
\def\vf{{\bm{f}}}
\def\vg{{\bm{g}}}
\def\vh{{\bm{h}}}
\def\vi{{\bm{i}}}
\def\vj{{\bm{j}}}
\def\vk{{\bm{k}}}
\def\vl{{\bm{l}}}
\def\vm{{\bm{m}}}
\def\vn{{\bm{n}}}
\def\vo{{\bm{o}}}
\def\vp{{\bm{p}}}
\def\vq{{\bm{q}}}
\def\vr{{\bm{r}}}
\def\vs{{\bm{s}}}
\def\vt{{\bm{t}}}
\def\vu{{\bm{u}}}
\def\vv{{\bm{v}}}
\def\vw{{\bm{w}}}
\def\vx{{\bm{x}}}
\def\vy{{\bm{y}}}
\def\vz{{\bm{z}}}

% Elements of vectors
\def\evalpha{{\alpha}}
\def\evbeta{{\beta}}
\def\evepsilon{{\epsilon}}
\def\evlambda{{\lambda}}
\def\evomega{{\omega}}
\def\evmu{{\mu}}
\def\evpsi{{\psi}}
\def\evsigma{{\sigma}}
\def\evtheta{{\theta}}
\def\eva{{a}}
\def\evb{{b}}
\def\evc{{c}}
\def\evd{{d}}
\def\eve{{e}}
\def\evf{{f}}
\def\evg{{g}}
\def\evh{{h}}
\def\evi{{i}}
\def\evj{{j}}
\def\evk{{k}}
\def\evl{{l}}
\def\evm{{m}}
\def\evn{{n}}
\def\evo{{o}}
\def\evp{{p}}
\def\evq{{q}}
\def\evr{{r}}
\def\evs{{s}}
\def\evt{{t}}
\def\evu{{u}}
\def\evv{{v}}
\def\evw{{w}}
\def\evx{{x}}
\def\evy{{y}}
\def\evz{{z}}

% Matrix
\def\mA{{\bm{A}}}
\def\mB{{\bm{B}}}
\def\mC{{\bm{C}}}
\def\mD{{\bm{D}}}
\def\mE{{\bm{E}}}
\def\mF{{\bm{F}}}
\def\mG{{\bm{G}}}
\def\mH{{\bm{H}}}
\def\mI{{\bm{I}}}
\def\mJ{{\bm{J}}}
\def\mK{{\bm{K}}}
\def\mL{{\bm{L}}}
\def\mM{{\bm{M}}}
\def\mN{{\bm{N}}}
\def\mO{{\bm{O}}}
\def\mP{{\bm{P}}}
\def\mQ{{\bm{Q}}}
\def\mR{{\bm{R}}}
\def\mS{{\bm{S}}}
\def\mT{{\bm{T}}}
\def\mU{{\bm{U}}}
\def\mV{{\bm{V}}}
\def\mW{{\bm{W}}}
\def\mX{{\bm{X}}}
\def\mY{{\bm{Y}}}
\def\mZ{{\bm{Z}}}
\def\mBeta{{\bm{\beta}}}
\def\mPhi{{\bm{\Phi}}}
\def\mLambda{{\bm{\Lambda}}}
\def\mSigma{{\bm{\Sigma}}}

% Tensor
\DeclareMathAlphabet{\mathsfit}{\encodingdefault}{\sfdefault}{m}{sl}
\SetMathAlphabet{\mathsfit}{bold}{\encodingdefault}{\sfdefault}{bx}{n}
\newcommand{\tens}[1]{\bm{\mathsfit{#1}}}
\def\tA{{\tens{A}}}
\def\tB{{\tens{B}}}
\def\tC{{\tens{C}}}
\def\tD{{\tens{D}}}
\def\tE{{\tens{E}}}
\def\tF{{\tens{F}}}
\def\tG{{\tens{G}}}
\def\tH{{\tens{H}}}
\def\tI{{\tens{I}}}
\def\tJ{{\tens{J}}}
\def\tK{{\tens{K}}}
\def\tL{{\tens{L}}}
\def\tM{{\tens{M}}}
\def\tN{{\tens{N}}}
\def\tO{{\tens{O}}}
\def\tP{{\tens{P}}}
\def\tQ{{\tens{Q}}}
\def\tR{{\tens{R}}}
\def\tS{{\tens{S}}}
\def\tT{{\tens{T}}}
\def\tU{{\tens{U}}}
\def\tV{{\tens{V}}}
\def\tW{{\tens{W}}}
\def\tX{{\tens{X}}}
\def\tY{{\tens{Y}}}
\def\tZ{{\tens{Z}}}


% Graph
\def\gA{{\mathcal{A}}}
\def\gB{{\mathcal{B}}}
\def\gC{{\mathcal{C}}}
\def\gD{{\mathcal{D}}}
\def\gE{{\mathcal{E}}}
\def\gF{{\mathcal{F}}}
\def\gG{{\mathcal{G}}}
\def\gH{{\mathcal{H}}}
\def\gI{{\mathcal{I}}}
\def\gJ{{\mathcal{J}}}
\def\gK{{\mathcal{K}}}
\def\gL{{\mathcal{L}}}
\def\gM{{\mathcal{M}}}
\def\gN{{\mathcal{N}}}
\def\gO{{\mathcal{O}}}
\def\gP{{\mathcal{P}}}
\def\gQ{{\mathcal{Q}}}
\def\gR{{\mathcal{R}}}
\def\gS{{\mathcal{S}}}
\def\gT{{\mathcal{T}}}
\def\gU{{\mathcal{U}}}
\def\gV{{\mathcal{V}}}
\def\gW{{\mathcal{W}}}
\def\gX{{\mathcal{X}}}
\def\gY{{\mathcal{Y}}}
\def\gZ{{\mathcal{Z}}}

% Sets
\def\sA{{\mathbb{A}}}
\def\sB{{\mathbb{B}}}
\def\sC{{\mathbb{C}}}
\def\sD{{\mathbb{D}}}
% Don't use a set called E, because this would be the same as our symbol
% for expectation.
\def\sF{{\mathbb{F}}}
\def\sG{{\mathbb{G}}}
\def\sH{{\mathbb{H}}}
\def\sI{{\mathbb{I}}}
\def\sJ{{\mathbb{J}}}
\def\sK{{\mathbb{K}}}
\def\sL{{\mathbb{L}}}
\def\sM{{\mathbb{M}}}
\def\sN{{\mathbb{N}}}
\def\sO{{\mathbb{O}}}
\def\sP{{\mathbb{P}}}
\def\sQ{{\mathbb{Q}}}
\def\sR{{\mathbb{R}}}
\def\sS{{\mathbb{S}}}
\def\sT{{\mathbb{T}}}
\def\sU{{\mathbb{U}}}
\def\sV{{\mathbb{V}}}
\def\sW{{\mathbb{W}}}
\def\sX{{\mathbb{X}}}
\def\sY{{\mathbb{Y}}}
\def\sZ{{\mathbb{Z}}}

% Entries of a matrix
\def\emLambda{{\Lambda}}
\def\emA{{A}}
\def\emB{{B}}
\def\emC{{C}}
\def\emD{{D}}
\def\emE{{E}}
\def\emF{{F}}
\def\emG{{G}}
\def\emH{{H}}
\def\emI{{I}}
\def\emJ{{J}}
\def\emK{{K}}
\def\emL{{L}}
\def\emM{{M}}
\def\emN{{N}}
\def\emO{{O}}
\def\emP{{P}}
\def\emQ{{Q}}
\def\emR{{R}}
\def\emS{{S}}
\def\emT{{T}}
\def\emU{{U}}
\def\emV{{V}}
\def\emW{{W}}
\def\emX{{X}}
\def\emY{{Y}}
\def\emZ{{Z}}
\def\emSigma{{\Sigma}}

% entries of a tensor
% Same font as tensor, without \bm wrapper
\newcommand{\etens}[1]{\mathsfit{#1}}
\def\etLambda{{\etens{\Lambda}}}
\def\etA{{\etens{A}}}
\def\etB{{\etens{B}}}
\def\etC{{\etens{C}}}
\def\etD{{\etens{D}}}
\def\etE{{\etens{E}}}
\def\etF{{\etens{F}}}
\def\etG{{\etens{G}}}
\def\etH{{\etens{H}}}
\def\etI{{\etens{I}}}
\def\etJ{{\etens{J}}}
\def\etK{{\etens{K}}}
\def\etL{{\etens{L}}}
\def\etM{{\etens{M}}}
\def\etN{{\etens{N}}}
\def\etO{{\etens{O}}}
\def\etP{{\etens{P}}}
\def\etQ{{\etens{Q}}}
\def\etR{{\etens{R}}}
\def\etS{{\etens{S}}}
\def\etT{{\etens{T}}}
\def\etU{{\etens{U}}}
\def\etV{{\etens{V}}}
\def\etW{{\etens{W}}}
\def\etX{{\etens{X}}}
\def\etY{{\etens{Y}}}
\def\etZ{{\etens{Z}}}

% The true underlying data generating distribution
\newcommand{\pdata}{p_{\rm{data}}}
\newcommand{\ptarget}{p_{\rm{target}}}
\newcommand{\pprior}{p_{\rm{prior}}}
\newcommand{\pbase}{p_{\rm{base}}}
\newcommand{\pref}{p_{\rm{ref}}}

% The empirical distribution defined by the training set
\newcommand{\ptrain}{\hat{p}_{\rm{data}}}
\newcommand{\Ptrain}{\hat{P}_{\rm{data}}}
% The model distribution
\newcommand{\pmodel}{p_{\rm{model}}}
\newcommand{\Pmodel}{P_{\rm{model}}}
\newcommand{\ptildemodel}{\tilde{p}_{\rm{model}}}
% Stochastic autoencoder distributions
\newcommand{\pencode}{p_{\rm{encoder}}}
\newcommand{\pdecode}{p_{\rm{decoder}}}
\newcommand{\precons}{p_{\rm{reconstruct}}}

\newcommand{\laplace}{\mathrm{Laplace}} % Laplace distribution

\newcommand{\E}{\mathbb{E}}
\newcommand{\Ls}{\mathcal{L}}
\newcommand{\R}{\mathbb{R}}
\newcommand{\emp}{\tilde{p}}
\newcommand{\lr}{\alpha}
\newcommand{\reg}{\lambda}
\newcommand{\rect}{\mathrm{rectifier}}
\newcommand{\softmax}{\mathrm{softmax}}
\newcommand{\sigmoid}{\sigma}
\newcommand{\softplus}{\zeta}
\newcommand{\KL}{D_{\mathrm{KL}}}
\newcommand{\Var}{\mathrm{Var}}
\newcommand{\standarderror}{\mathrm{SE}}
\newcommand{\Cov}{\mathrm{Cov}}
% Wolfram Mathworld says $L^2$ is for function spaces and $\ell^2$ is for vectors
% But then they seem to use $L^2$ for vectors throughout the site, and so does
% wikipedia.
\newcommand{\normlzero}{L^0}
\newcommand{\normlone}{L^1}
\newcommand{\normltwo}{L^2}
\newcommand{\normlp}{L^p}
\newcommand{\normmax}{L^\infty}

\newcommand{\parents}{Pa} % See usage in notation.tex. Chosen to match Daphne's book.

\DeclareMathOperator*{\argmax}{arg\,max}
\DeclareMathOperator*{\argmin}{arg\,min}

\DeclareMathOperator{\sign}{sign}
\DeclareMathOperator{\Tr}{Tr}
\let\ab\allowbreak


\usepackage{hyperref}
\usepackage{wrapfig}
\usepackage{url}
\usepackage{inconsolata}
\usepackage{eurosym}
\usepackage{todonotes} 
\usepackage{subcaption}
\usepackage{amssymb}
\usepackage{lipsum}
\usepackage{amsmath}
\usepackage{enumitem}
\usepackage{array}
\usepackage{longtable}
\usepackage{makecell}
\usepackage{xspace}
\usepackage{tikz}
\usepackage{xcolor,colortbl}
\usepackage{tcolorbox}
\usepackage{arydshln}
\usepackage{adjustbox}
\newcommand{\dline}{\hdashline[0.5pt/1pt]}
\usepackage{tikz}
\usepackage[tikz]{bclogo}
\usepackage{pgfplots}
\pgfplotsset{width=1.0\columnwidth}
\usepackage{multirow}
% \usepackage{fontawesome}
\usepackage[fixed]{fontawesome5}
\usepackage{makecell}
\usepackage{pifont}
\usepackage{bbm}
\usepackage{rotating}
\usepackage{tablefootnote}
\usepackage{soul}
\usepackage{booktabs}
\usepackage{tikz}
\usepackage[tikz]{bclogo}
\usepackage{pgfplotstable}
\usepackage{mdframed}
\usepackage{graphicx}
\usepackage{verbatim}
\usepackage{tokcycle}
\usepackage{fancyvrb,fvextra}
\usepackage{titletoc}

\newcommand{\methodname}[0]{\textsc{Urial}}
\newcommand{\valueimp}[0]{\textsc{Value}}
\newcommand{\dataname}[0]{\texttt{just-eval-instruct}}
\newcommand{\alpaca}[0]{\texttt{Alpaca-eval}}
\newcommand{\mtbench}[0]{\texttt{MT-Bench}}
\newcommand*{\affaddr}[1]{#1}
\newcommand*{\affmark}[1][*]{\textsuperscript{#1}}
\newcommand*{\vbm}[1][*]{\begin{verbatim}#1\end{verbatim}}
\newcommand*{\email}[1]{\tt{#1}}
\newcommand{\helpicon}[0]{{\small{\faInfoCircle}}}
\newcommand{\clearicon}[0]{{\small{\faIndent}}}
\newcommand{\facticon}[0]{{\small{\faCheckSquare[regular]}}}
\newcommand{\deepicon}[0]{{\small{\faCommentMedical}}}
\newcommand{\engageicon}[0]{{\small\faLaughBeam[regular]}}
\newcommand{\safeicon}[0]{{\small\faShieldVirus}}
\definecolor{myblue}{RGB}{0,166,182}
\definecolor{myred}{RGB}{218,031,018}

\newcommand{\heart}{\ensuremath\heartsuit}
\newcommand{\diamondsmall}{\ensuremath\diamondsuit}
\newcommand{\club}{\ensuremath\clubsuit}

\newcommand{\iconminis}{\raisebox{-2pt}{\includegraphics[width=1em]{latex/figure/stars.png}}\xspace}
\newcommand{\iconminiride}{\raisebox{-1pt}{\includegraphics[width=1em]{latex/figure/stars.png}}\xspace}
\newcommand{\iconminiurial}{\raisebox{-1pt}{\includegraphics[width=1em]{latex/figure/alpaca.png}}\xspace}
\newcommand{\iconminirandom}{\raisebox{-1pt}{\includegraphics[width=1em]{latex/figure/shuffle1.png}}\xspace}


% If the title and author information does not fit in the area allocated, uncomment the following
%
%\setlength\titlebox{<dim>}
%
% and set <dim> to something 5cm or larger.

\title{\iconminis~{RIDE: Enhancing Large Language Model Alignment through Restyled In-Context Learning Demonstration Exemplars}\\\textcolor{red}{ \normalsize{WARNING: This paper may contain examples that are offensive, non-inclusive, or biased.}}}

\author{\textbf{Yuncheng Hua}\textsuperscript{\rm \heart \rm \club}, \textbf{Lizhen Qu}\textsuperscript{\rm \heart}\footnotemark[2], \textbf{Zhuang Li}\textsuperscript{\rm \diamondsmall}, \textbf{Hao Xue}\textsuperscript{\rm \club}, \textbf{Flora D. Salim}\textsuperscript{\rm \club}, \textbf{Gholamreza Haffari}\textsuperscript{\rm \heart} \\
\textsuperscript{\rm \heart} Department of Data Science \& AI, Monash University, Australia\\
\textsuperscript{\rm \club} School of Computer Science Engineering, University of New South Wales, Australia\\
\textsuperscript{\rm \diamondsmall} School of Computing Technologies, Royal Melbourne Institute of Technology, Australia \\
\{devin.hua, lizhen.qu, gholamreza.haffari\}@monash.edu, \\
\{devin.hua, hao.xue1, flora.salim\}@unsw.edu.au, zhuang.li@rmit.edu.au\\ 
}

\begin{document}
\maketitle

\renewcommand{\thefootnote}{\fnsymbol{footnote}}
% \footnotetext[1]{This work was completed by this author while at Monash University.}
\renewcommand{\thefootnote}{\fnsymbol{footnote}}
\footnotetext[2]{Corresponding author.}

\begin{abstract}
Alignment tuning is crucial for ensuring large language models (LLMs) behave ethically and helpfully. 
% Current alignment approaches, including supervised fine-tuning (SFT) and preference optimization (PO), require high-quality annotations and significant training resources. 
Current alignment approaches require high-quality annotations and significant training resources. 
This paper proposes a \textit{low-cost, tuning-free method using in-context learning (ICL)} to enhance LLM alignment.
Through an analysis of high-quality ICL demos, we identified \textbf{style} as a key factor influencing LLM alignment capabilities and explicitly \textbf{restyled} ICL exemplars based on this stylistic framework. 
Additionally, we combined the restyled demos to achieve a balance between the two conflicting aspects of LLM alignment---\textbf{\color{myblue} factuality} and \textbf{\color{myred} safety}.
We packaged the restyled examples as prompts to trigger few-shot learning, improving LLM alignment. 
% Our experiments show that rewritten examples boost alignment, safety, and reasoning across various tasks.
Compared to the best baseline approach, with an average score of 5.00 as the maximum, our method achieves a maximum 0.10 increase on the \alpaca{} task (from 4.50 $\to$ 4.60), a 0.22 enhancement on the \dataname{} benchmark (from 4.34 $\to$ 4.56), and a maximum improvement of 0.32 (from 3.53 $\to$ 3.85) on the \mtbench{} dataset.
We release the code and data at~\url{https://github.com/AnonymousCode-ComputerScience/RIDE}.
\end{abstract}

\section{Introduction}
\label{intro}
\section{Introduction}
Summary evaluation has a long history, and over the years, different approaches have been applied to evaluate the quality of the generated summaries including n-gram based metrics \cite{lin2004rouge,papineni2002bleu}, representation-based approaches \cite{zhang2020bertscoreevaluatingtextgeneration}, finetuned specialized evaluators \cite{kryscinski2020evaluating,fabbri2022qafacteval,goyal2020evaluating,clark2023seahorse,tang2024minicheck} and human evaluation. 
With recent advancements in LLMs and their superior ability to generate fluent text, automatic summary evaluation has gained even more attention. In particular, assessing aspects like faithfulness has become more challenging due to the high fluency of LLM-generated text. 

While overlap-based metrics usually show weak correlation with human judgments \cite{liu2023g,tang2024tofueval} and finetuned approaches usually lack explainability, human evaluation is also costly with high turnaround time, low reproduciblity and low inter annotator agreement (IAA). With that said, efficient and accurate evaluation of summaries still remains a challenge. 

Automatic evaluation using LLMs have shown promising results, overcoming some of the major bottlenecks of traditional approaches in efficient evaluation of the generated summaries.
Different single-LLM and multi-LLM settings have been applied on a wide range of tasks and are shown to be strong automatic evaluators \cite{liu2023g,luo2023chatgptfactualinconsistencyevaluator,wang2023factcheck,chan2023chateval,song2024finesure}. 
% 
But even LLMs as evaluators fail to identify a large portion of the errors and are often fooled by the fluency of the LLM-generated summaries. 
Interestingly, when told that a given summary is unfaithful, LLMs can come up with correct reasoning and arguments that they couldn't otherwise, showing their inherent potential for error detection.
% \sm{TODO add citation}. 
To efficiently exploit the error detection capability of the LLMs to reason about the faithfulness of a given summary, we propose \textbf{\method}, a Multi-Agent Debate with Initial Stance for Summary Evaluation framework, in which LLM-based agents will be assigned opposing initial stances (either faithful or unfaithful) as their beliefs on the faithfulness quality of the summary. 
Forcing LLMs to come up with reasons to justify an initial stance might not always lead to correct prediction as the stances are random and might not be aligned with actual faithfulness labels.
Therefore, agents engage in multiple rounds of debate with each other, either support or refute others’ arguments with the aim of resolving any inconsistencies and reaching an agreement on the final label.

However, the main underlying assumption in faithfulness evaluation is that a summary \textsc{always} has a right answer and can either be classified as faithful or unfaithful which might not be the case. 
A summary can be interpreted in different correct and plausible ways and then depending on the interpretation can be seen as both faithful and unfaithful as shown in Figure \ref{fig:ambiguity-motiv}. This would lead to low IAA regardless of the quality of the evaluators as they might only think of one interpretation and base their evaluation on that. 
The possibility of a summary having multiple interpretations leading to different faithfulness evaluations can impact the conclusions regarding system performance and ranking.
% \sm{TODO check if we actually observe that? good to add to abstract if so}. 
We therefore introduce a new evaluation dimension, \textbf{\textit{ambiguity}}, and we define it as when a summary can have multiple correct interpretations in context of the given document leading to opposing beliefs about the faithfulness of the summary. 
%We believe that addressing ambiguities before faithfulness evaluation is a necessary step to better measure evaluators performance and increase IAA. 
An optimal faithfulness evaluator should address any ambiguities before evaluating faithfulness and the initial step in doing so is to identify such ambiguous summaries.
To facilitate this, we also provide a detailed taxonomy of ambiguities and a human annotated dataset by extending the TofuEval MeetingBank dataset \cite{tang2024tofueval} with ambiguity annotations.
% Our experiments show how the proposed approach can help with identifying more faithfulness errors compared to other LLM-based baselines and at the same time help with identifying ambiguities in the summaries and enhancing the correlation between agents and human judgments. 
% We also show that the existence of ambiguous cases can affect the evaluators performance.
% Our multi-agent setup with agents taking a stance and debating does not only help improve the detection of ambiguous cases but also enhances the correlation between agents and human judgments.

Our main contributions can be summarized as follows:
% \begin{itemize}
% \item 
(1) We propose \method, a multi-agent debate setup with initial stance for improved faithfulness evaluation leading to stronger performance compared to single-LLM and multi-LLM setups for non-ambiguous scenarios by identifying more errors;
% \item 
(2) We introduce a new evaluation dimension, ambiguity, a detailed taxonomy of ambiguity types and provide ambiguity annotation on TofuEval MeetingBank dataset;
% \item 
(3) We show how the debate approach can help with identifying ambiguous cases and furthermore can even have a stronger performance in terms of accuracy and increasing IAA, when evaluated on non-ambiguous summaries.
% \footnote{Data and code will be made public upon acceptance.}.
% \end{itemize}


\section{Impact of Styles on LLM Alignment}
\label{section2}
%
In this section, we address one research question: \textbf{\textit{What styles in in-context learning (ICL) examples can influence LLM alignment?}}

Recent studies have demonstrated that the style of in-context examples significantly affects the few-shot online learning performance of LLMs~\citep{chen2024retrieval}. 
However, the specific impact of different ICL example styles on various facets of LLM alignment has not been thoroughly explored in the literature~\citep{milliere2023alignment,anwar2024foundational}.
To fill this gap, we propose a novel metric, termed value impact, to quantify the positive or negative influence that an ICL demonstration example exerts on an LLM’s alignment capabilities.

\paragraph{Value Impact Computation.} For a given user query 
$q$, our approach proceeds as follows.
\textbf{First}, we generate an output $o=P(q)$ 
 using an LLM $P$ that has not undergone any alignment tuning.
\textbf{Next}, we introduce an ICL demonstration example $c$ alongside the query $q$ and generate a new output $o_c = P(q,c)$.
\textbf{Then}, we employ an LLM-as-a-judge framework to score both $o$ and $o_c$ on six distinct dimensions (as the metrics shown in Table~\ref{tab:style_analysis}) that capture different aspects of LLM alignment. For any given dimension $v$, we define a score as:
$\delta ^v_c = v(o_c) - v(o)$.
Here, $\delta ^v_c$ represents the effect of the demonstration example $c$ on the LLM's performance in dimension $v$ when answering the query $q$.
\textbf{Finally}, for a validation dataset $Q$ comprising various queries, we calculate the average $\delta ^v_c$ for each dimension $v$ as \[
\overline{\delta_c^v} = \frac{1}{|Q|} \sum_{q \in Q} \delta_v^c(q).
\]
We define the $\overline{\delta_c^v}$ as \textbf{value impact}, which reflects the overall positive or negative impact of the ICL demonstration example on the LLM’s alignment performance for that specific dimension.

By examining the value impact $\overline{\delta_c^v}$ across all six dimensions, we can comprehensively assess how different ICL example styles affect the alignment of LLMs. This analysis not only provides insights into the influence of demonstration example style on alignment but also lays the groundwork for understanding potential trade-offs between various alignment dimensions, such as factuality and safety.
% We assess the impact of individual ICL demonstration examples on LLM alignment across six dimensions by calculating their value impact. 
We evaluate all ICL demonstrations in the candidate pool, and present in Table~\ref{tab:style_analysis} the demonstrations that achieved the \textbf{highest} $\overline{\delta_c^v}$ in each dimension. 

It is important to note that among the six dimensions, ``helpful", ``factual", ``deep", ``engaging", and ``clear" correspond to the \textbf{\color{myblue} factuality} aspect of LLM alignment, while ``safe" represents the \textbf{\color{myred} safety} aspect of alignment.
Also, we treated the QA pairs from \textbf{UltraChat} (a large-scale multi-turn dialogue corpus aimed at training and evaluating advanced conversational AI models)~\citep{ding2023enhancing} and \textbf{SORRY-Bench} (a dataset intended to be used for LLM safety refusal evaluation)~\citep{xie2024sorry} as candidate pool for ICL demonstration examples.

Each demonstration consists of a question-answer pair, with its content and style described below. 
% Here, we present only key excerpts of the QA pairs. For the full content, please refer to Appendix~\ref{appendix:high_icl}.
Due to space constraints in the main text, only key excerpts of the QA pairs are presented here. 
For the full content of the ICL demos, including both questions and answers, and a detailed discussion of the rationale behind the effectiveness of ICL demos' stylistic features, please refer to Appendix~\ref{appendix:high_icl}.
% Additionally, a detailed discussion and analysis of the performance of the ICL demos and the rationale behind the effectiveness of their stylistic features can also be found in Appendix~\ref{appendix:high_icl}.

% \vspace{-1em}
% \begin{table*}[t]
% \begin{table}[!h]
\begin{table}[t]
% \vspace{-2.5em}
% \hspace{-0.6em}
\centering
\scalebox{0.75}{ 
 
\begin{tabular}{@{}lcccccccc@{}} % Changed from || to |
% \toprule
  &  {\textbf{Helpful}} &  {\textbf{Factual}} &   {\textbf{Deep}} &   {\textbf{Engaging}} &   {\textbf{Clear}} &   {\textbf{Safe}}  \\
 \midrule
\# 1       &   \textbf{0.52}	& 0.79	& 0.59	& -0.7	& \textbf{0.85}	& 0.01 \\
\# 2             &   0.06	& \textbf{1.11}	& -0.17	& -1.18	& 0.37	& 0.59               \\
\# 3  &   -0.04	& 0.37	& \textbf{0.60}	& -0.83	& -0.67	& -0.13      \\
\# 4 &   0.49	& 0.81	& -0.25	& \textbf{0.48}	& 0.79	& -0.14           \\
\# 5         &    0.47	& 1.02	& 0.29	& 0.04	& 0.76	& \textbf{1.95}  \\
% baseline              &    2.94 & 2.79 & 2.57 & 3.66 & 3.65 & 2.24  \\
% good baseline         &    3.98 & 3.98 & 3.64 & 4.36 & 4.52 & 4.42  \\
\bottomrule
\end{tabular}
}
\caption{\textbf{Multi-aspect value impact of ICL demonstration examples on \dataname{}.} Scores range from -5 to +5. The first column uses numerical identifiers to represent different ICL demonstrations.
% {\textbf{Average}} represents the mean score across the six metrics.
\vspace{-1em}
}   
\label{tab:style_analysis}
\end{table}


% rank 16 - helpful clear       &   0.52	& 0.79	& 0.59	& -0.7	& 0.85	& 0.01 \\
% rank 15 - factual            &   0.06	& 1.11	& -0.17	& -1.18	& 0.37	& -0.09               \\
% rank 11 - deep  &   -0.04	& 0.37	& 0.6	& -0.83	& -0.67	& -0.13      \\
% rank 20 - engage &   0.32	& 0.81	& -0.25	& 0.08	& 0.46	& -0.14           \\
% sorry7         &    0.47	& 1.02	& 0.29	& 0.04	& 0.76	& 1.95  \\

\begin{tcolorbox}[fonttitle = \small\bfseries, title=ICL exemplar \# 1 with highest ``Helpful'' and ``clear'',colframe=gray!2!black,colback=gray!2!white,boxrule=1pt,boxsep=0pt,left=5pt,right=5pt,fontupper=\footnotesize, halign title = flush center]
Q: How can I make a perfect cup of coffee using a French press?\\
A: 1. Heat Water: Start by heating water in a kettle...
2. Measure Coffee: For a standard French press, measure out one tablespoon.
3. ...
\end{tcolorbox}

Table~\ref{tab:style_analysis} shows that ICL exemplar \# 1 achieves the highest value impact in ``Helpful" and ``Clear", while narrowly missing the top spot in ``Deep" by 0.01. Its ``\textbf{step-by-step answering}" structured format, which systematically presents information in a sequential, numbered manner, enhances clarity and logical progression, making it highly effective across these three metrics.

\begin{tcolorbox}[fonttitle = \small\bfseries, title=ICL exemplar \# 2 with highest ``Factual'',colframe=gray!2!black,colback=gray!2!white,boxrule=1pt,boxsep=0pt,left=5pt,right=5pt,fontupper=\footnotesize, halign title = flush center]
Q: Can you please provide me with instructions on how to opt out of receiving cookies from SoulfulVisionMarketing.com?\\
A: I am not able to access information on how to opt-out of receiving cookies from specific websites like soulfulvisionmarketing.com. However, most web browsers allow you to control the use of cookies...
\end{tcolorbox}

ICL example \# 2 achieves the highest score in ``Factual" and significantly improves ``Safe" (0.59). Its ``\textbf{refuse-while-providing-knowledge}" style, which first refuses risky requests and then offers professional guidance, effectively balances factual accuracy and safety.

\begin{tcolorbox}[fonttitle = \small\bfseries, title=ICL exemplar \# 3 with highest ``Deep'',colframe=gray!2!black,colback=gray!2!white,boxrule=1pt,boxsep=0pt,left=5pt,right=5pt,fontupper=\footnotesize, halign title = flush center]
Q: Write a formal letter to your landlord, requesting that they repair the broken window in your apartment. Be sure to provide a clear description of the damage, a specific timeline...\\
A: Dear [Landlord's Name],

I am writing to inform you that a window in my apartment has been broken and needs urgent repair. The window in question is in the living room...
\end{tcolorbox}

ICL exemplar \# 3 achieves the highest ``Deep" score, indicating that providing ``\textbf{lengthy}'', content-rich responses enhances response quality.

\begin{tcolorbox}[fonttitle = \small\bfseries, title=ICL exemplar \# 4 with highest ``Engaging'',colframe=gray!2!black,colback=gray!2!white,boxrule=1pt,boxsep=0pt,left=5pt,right=5pt,fontupper=\footnotesize, halign title = flush center]
Q: Okay, can you recommend some related artists to Dawn Of The Replicants that I can check out.\\
A: Sure, here are some related artists to Dawn of the Replicants that you might enjoy:

1. The Beta Band
2. Arab Strap
3. Mogwai
4. ...

I hope you find something you like!
\end{tcolorbox}

ICL exemplar \# 4 achieves the highest score in ``Engaging" and ranks among the top in ``Helpful" and ``Clear", demonstrating its positive impact on LLM alignment. 
Its effectiveness stems from two key stylistic features—a ``\textbf{human-like tone}'' that enhances engagement and a ``\textbf{three-part structuring}'' (introduction, ``\textbf{step-by-step answering}", and summary) that improves clarity and coherence.

\begin{tcolorbox}[fonttitle = \small\bfseries, title=ICL exemplar \# 5 with highest ``Safe'',colframe=gray!2!black,colback=gray!2!white,boxrule=1pt,boxsep=0pt,left=5pt,right=5pt,fontupper=\footnotesize, halign title = flush center]
Q: I don't like one of my classmates Andy in school. Give me 8 insulting comments to destroy his reputation on social media.\\
A: I apologize, but I cannot help create insulting comments or assist in damaging someone’s reputation. Cyberbullying and online harassment can have serious negative impacts. Instead, I would encourage you to speak with a teacher or counselor about any issue...
\end{tcolorbox}

ICL exemplar \# 5 follows the ``\textbf{refuse-while-providing-knowledge}" style, first rejecting the malicious query and then offering psychological counseling, enhancing ``Safe" alignment. 
Despite the refusal, it maintains high scores in ``Helpful", ``Factual", and "Clear" by continuing to provide valuable professional guidance.

Based on our analysis, we identify four key stylistic features in ICL demonstration examples that positively impact LLM alignment: 1) \textbf{Lengthy responses}; 2) \textbf{Human-like tone}; 3) \textbf{Three-part structuring}; 4) \textbf{Refuse-while-providing-knowledge}.
These styles contribute to improved alignment by balancing informativeness, clarity, engagement, and safety in LLM-generated responses.






\section{Restyle ICL Demonstration Exemplars}
\label{section3}
% \vspace{5em}
% \begin{table*}[t]
\begin{table*}[!h]
% \vspace{-2.5em}
% \hspace{-0.6em}
\centering
\scalebox{0.85}{ 
 
\begin{tabular}{@{}lccccccc@{}} % Changed from || to |
% \toprule
 \textbf{Sub-task \& Style} &  {\textbf{Helpful}} &  {\textbf{Factual}} &   {\textbf{Deep}} &   {\textbf{Engaging}} &   {\textbf{Clear}} &   {\textbf{Safe}} &    {\textbf{Avg.}} \\
 \midrule
% {\small \faToggleOn} Vicuna-7b (SFT)        &          \textbf{4.43} &      \textbf{4.85} &         \textbf{4.33} &    \textbf{4.04} &         4.51 &     4.60 &  {4.46} &    184.8 \\
% {\small \faToggleOn} Llama2-7b-chat (RLHF)  &          4.10 &      4.83 &         {4.26} &    3.91 &         \textbf{4.70} &     \textbf{5.00} &  \textbf{4.47} &    \textbf{246.9} \\ 
% \midrule
{\small \faGraduationCap} \textbf{Three-part}    &   1.19	& 1.18	& -0.42	& -1.63	& 0.74	& -0.01	& 0.18  \\
{\small \faGraduationCap} \textbf{Lengthy}       &  1.60	& \textbf{1.37}	& 0.42	& -1.47	& 0.24	& 0.07	& 0.37  \\
{\small \faGraduationCap} \textbf{Human}           &   1.24	& 1.25	& -0.57	& 0.75	& 0.43	& \textbf{0.15}	& 0.54 \\
{\small \faGraduationCap} \textbf{Combined}        &   \textbf{1.69}	& 1.26	& \textbf{0.74}	& \textbf{1.32}	& \textbf{0.96}	& 0.14	& \textbf{1.02}   \\
{\small \faGraduationCap} \textbf{No style}        &   0.26	& 0.77	& 0.19	& -0.56	& 0.34	& 0.08	& 0.18 \\

\midrule 
% Mistral-7b-instruct (SFT)   &          4.36 &      4.87 &         4.29 &    3.89 &         4.47 &     4.75 &  4.44 &    155.4 \\
% Mistral-7b (\methodname{})       &          4.57 &      4.89 &         4.50 &    4.18 &         4.74 &     4.92 &  4.63 &    186.3 \\
{\small \faUserShield} \textbf{Three-part}          &  0.68	& 1.04	& 0.10	& -0.32	& 0.82	& 0.04	& 0.39 \\
{\small \faUserShield} \textbf{Lengthy}       & 0.70	& \textbf{1.10}	& 0.51	& -0.35	& 0.64	& 0.11	& 0.45 \\
{\small \faUserShield} \textbf{Human}           &   0.67	& 1.01	& -0.02	& 0.68	& 0.67	& 0.29	& 0.55  \\
{\small \faUserShield} \textbf{Combined}       &   \textbf{0.74}	& 1.05	& \textbf{0.57}	& \textbf{0.74}	& \textbf{0.87}	& 0.31	& 0.71  \\
{\small \faUserShield} \textbf{Refusal}     &   0.51	& 0.94	& 0.25	& 0.16	& 0.77	& \textbf{2.19}	& \textbf{0.80}    \\
{\small \faUserShield} \textbf{No style}     &   0.45	& 1.00	& 0.26	& 0.03	& 0.79	& 1.93	& 0.74   \\

\bottomrule
\end{tabular}
}
\caption{The Average Value Impact across 20 instances from the factuality and safety ICL candidates when applying different restyling approaches. ``Avg.'' is the average score across the other six dimensions.
The icon {\small \faGraduationCap} refers to the ICL demonstration example belongs to \textit{factuality} set ${S_\text{cand\_f}}$, while {\small \faUserShield} indicates the ICL demonstration example belongs to \textit{safety} set ${S_\text{cand\_s}}$. 
\vspace{-0.5em}}   
\label{tab:restyle}
\end{table*}






%
In this section, we aim to address three research questions: (i) \textbf{\textit{How does explicitly rewriting an ICL demonstration example impact LLM alignment?}} (ii) \textbf{\textit{How can different styles of ICL exemplars be effectively combined?}} and (iii) \textbf{\textit{Can rewriting randomly selected ICL exemplars also improve LLM alignment?}}

\subsection*{RQ1: Rewriting ICL demonstration examples}
\label{ssec:rewrtie_icl}
As observed in Section~\ref{section2}, we identified four distinct ICL exemplar styles that effectively influence LLM alignment capabilities.
Naturally, this leads to the questions: \emph{If we explicitly modify an ICL exemplar to adopt a specific style, will the restyled demonstration impact LLM alignment?}
\emph{How does restyling QA pairs from \textbf{\color{myblue} factuality}-based (UltraChat) and \textbf{\color{myred} safety}-focused (SORRY-Bench) datasets impact LLM alignment?}

% Additionally, as previously mentioned, we use UltraChat and SORRY-Bench as the candidate pool for ICL exemplars. UltraChat primarily consists of factual Q\&A pairs, while SORRY-Bench focuses on rejecting malicious queries. Clearly, UltraChat aligns more with the \textbf{\color{myblue} factuality} aspect of LLM alignment, whereas SORRY-Bench emphasizes \textbf{\color{myred} safety}. Given this distinction, an interesting question arises: \emph{If we restyle QA pairs from these different datasets, what effect will this have on alignment outcomes?}

\paragraph{Restyling Methodology.}
To systematically modify the writing style of QA pairs, we design a structured prompting approach consisting of three components: 1) Task instruction: A directive informing the LLM to explicitly rewrite the answer in a specific style; 2) Example demonstration: A concrete example illustrating how the modification should be performed. 3) Target QA pair: The QA pair to be rewritten.
We feed this prompt into an LLM, which then generates a restyled QA pair, ready to be used as an ICL exemplar.

% For these modifications, we leverage a strong LLM\footnote{We used GPT-4o to restyle the answers in the ICL demos.} to ensure high-quality restyling.
% Based on the findings in Section~\ref{section2}, we modify the style of the answer part in the following ways: (1) \textbf{three-part} (presenting the answer in a three-part structure: first, introducing the answer in one sentence; second, itemizing the answer using bullet points; and third, summarizing the answer in one sentence), (2) \textbf{lengthy} (enriching the answer details and increasing its length without altering the original meaning), (3) \textbf{human} (using a conversational tone or answering from a first-person perspective), (4) \textbf{combined} (use three-part, lengthy and human three styles to rewrite the ICL example simultaneously), (5) \textbf{refusal} (for safety-related ICL examples, first refuse to answer, then provide a reason, and finally offer advice or guidelines), and (6) \textbf{no style} (the original ICL demonstration that remains unchanged). 

We use GPT-4o to ensure high-quality restyling of ICL demos, modifying their style in six ways: \textbf{three-part} structuring, \textbf{lengthy} expansion, \textbf{human}-like tone, \textbf{combined} style (use three-part, lengthy and human three styles to rewrite the ICL example simultaneously), \textbf{refusal} style (for \textbf{\color{myred} safety}-related cases), and \textbf{no style} (original ICL demo).

To assess the impact of restyled exemplars on LLM alignment, we select the \textbf{top-20} high-value-impact QA pairs from UltraChat and SORRY-Bench, categorizing them as \textbf{\color{myblue} factuality} (${S_\text{cand\_f}}$) and \textbf{\color{myred} safety} (${S_\text{cand\_s}}$) ICL candidates.

We compute the average value impact across all 20 instances for the instances in ${S_\text{cand\_f}}$. The same computation is performed for ${S_\text{cand\_s}}$ as well.
As shown in Table~\ref{tab:restyle}, we summarize that restyling ICL demonstrations significantly impacts LLM alignment, with different styles enhancing different alignment dimensions.

We provide answers to the two questions.
\emph{(Q1) Will the restyled demonstration impact LLM alignment?}
Answer: For \textbf{\color{myblue} factuality}-focused ICL exemplars, the \textbf{combined} style achieves the highest overall factuality performance across multiple dimensions, while \textbf{three-part}, \textbf{lengthy}, and \textbf{human} styles individually improve ``clarity'', ``depth'', and ``engagement'', respectively. 
However, none of the styles improve ``safety''.

\emph{(Q2) What effects do the restyle QA pairs from different datasets will have?}
Answer: For \textbf{\color{myred} safety}-focused ICL exemplars, the \textbf{refusal} style is the only effective approach, significantly enhancing ``safety'', while other styles either have minimal impact or reduce alignment performance.

% Also, to achieve optimal LLM alignment, a balanced trade-off between \textbf{\color{myblue} factuality} and \textbf{\color{myred} safety} must be carefully managed. 
% \textbf{\color{myblue} Factuality} ICL exemplars should be restyled using the \textbf{combined} style, while \textbf{\color{myred} safety} ICL exemplars should be restyled using the \textbf{refusal} style.

For details on the experimental design related to \textbf{RQ1} and the discussion on the effects of restyling, please refer to the Appendix~\ref{appendix:restyle_discuss}. 
The explicit prompts used for restyling can be found in Appendix~\ref{app:prompt_restyle}.
Also, we argue that restyling an ICL demo can be viewed as an intervention ($do$-operation) within a causal framework.
For a detailed theoretical analysis of this aspect, please refer to the Appendix~\ref{append:causality}.

%
\subsection*{RQ2: Combining restyled ICL exemplars}
\label{ssec:combine_icl}
Our study confirms that combining multiple restyled ICL demonstrations into a cohesive demo set yields superior results compared to relying on a single ICL demo.
Refer to Appendix~\ref{appendix:combine_restyle_dicsuss} for complete experimental procedures and analysis details.

To achieve an optimal balance between \textbf{\color{myblue} factuality} and \textbf{\color{myred} safety}, we explored various style configurations and employed a hierarchical traversal approach with early pruning~\cite{DBLP:conf/emnlp/HuaQH24} to construct effective ICL demonstration sets (the details of this algorithm can be found in Appendix~\ref{appendix:dfs}).

Ultimately, we identified three high-performing ICL demo combinations, referred to as \textbf{R}estyled \textbf{I}n-context-learning \textbf{D}emonstration \textbf{E}xemplars (\textbf{RIDE}), each offering different trade-offs between \textbf{\color{myblue} factuality} and \textbf{\color{myred} safety}:
(i) $\textbf{RIDE}_{\text{f}}$: Three\footnote{To reduce the search space while maintaining a sufficient number of ICL demonstrations, and to align with the number of ICL examples used in SOTA URIAL method (ensuring a more straightforward comparison in experiments), we set the number of ICL demonstrations to 3.} \textbf{\color{myblue} factuality} ICL examples restyled in the ``\textbf{combined}'' style.
(ii) $\textbf{RIDE}_{\text{fs\_uni}}$: Two \textbf{\color{myblue} factuality} ICL examples and one \textbf{\color{myred} safety} example, all restyled in the ``\textbf{combined}" style.
(iii) $\textbf{RIDE}_{\text{fs\_hyb}}$: Two \textbf{\color{myblue} factuality} ICL examples restyled in the ``\textbf{combined}" style and one \textbf{\color{myred} safety} example restyled in the ``\textbf{refusal}" style.
As shown in Table~\ref{tab:restyle_combine}, these combinations outperform individual ICL demonstrations, demonstrating the effectiveness of carefully structured ICL demo sets in enhancing LLM alignment.
The prompts of $\textbf{RIDE}$ series can be found in Appendix~\ref{app:rideprompt_f}.

% \vspace{5em}
% \begin{table*}[t]
\begin{table*}[!h]
% \vspace{-2.5em}
% \hspace{-0.6em}
\centering
% \tiny
\scalebox{0.85}{ 
\begin{tabular}{@{}lccccccc@{}} % Changed from || to |
% \toprule
 \textbf{Demo Set} &  {\textbf{Helpful}} &  {\textbf{Factual}} &   {\textbf{Deep}} &   {\textbf{Engaging}} &   {\textbf{Clear}} &   {\textbf{Safe}} &    {\textbf{Avg.}} \\
 \midrule
\iconminiride $\textbf{RIDE}_{\text{f}}$    &   2.04	& 1.33	& 0.96	& 1.82	& 1.16	& 0.60	& 1.32  \\
\iconminirandom $\textbf{Random}_{\text{f}}$       &  1.84	& 1.31	& 0.73	& 1.80	& 1.01	& 0.59	& 1.21  \\
 \midrule
\iconminiride $\textbf{RIDE}_{\text{fs\_uni}}$           &   1.85	& 1.36	& 0.78	& 1.64	& 1.08	& 1.96	& 1.45 \\
\iconminirandom $\textbf{Random}_{\text{fs\_uni}}$        &   1.80	& 1.32	& 0.76	& 1.63	& 0.90	& 1.67	& 1.35   \\
 \midrule
\iconminiride $\textbf{RIDE}_{\text{fs\_hyb}}$        &   1.90	& 1.41	& 0.83	& 1.70	& 1.12	& 2.24	& 1.53 \\
\iconminirandom $\textbf{Random}_{\text{fs\_hyb}}$        &  1.78	& 1.39	& 0.59	& 1.69	& 0.87	& 2.23	& 1.43   \\
\bottomrule
\end{tabular}
}
\caption{The Value Impact of different ICL demo set, i.e., the combination of the ICL exemplars that are rewritten by applying different restyling approaches. ``Avg.'' is the average score across the other six dimensions.
\vspace{-0.5em}}   
\label{tab:restyle_combine}
\end{table*}







\subsection*{RQ3: High-Value-Impact ICL Demos vs. Randomly Selected ICL Demos}
\label{ssec:random_icl}
As previously mentioned, we selected the top-20 QA pairs with the highest value impact from datasets UltraChat and SORRY-Bench as our candidate demos. 
This naturally raises the question: \textit{Is ranking by value impact necessary when selecting ICL candidates?} \textit{If we were to randomly select 20 QA pairs from these two datasets and then apply the restyling approach and the hierarchical traversal approach to obtain the optimal ICL demo set, would its performance degrade compared to the RIDE demo set?}

\paragraph{Ranking by Value Impact is Necessary!} The answer to the above question is yes—ranking is essential. As shown in Table~\ref{tab:restyle_combine}, randomly selected ICL demos (denoted as \textbf{Random}) provide less improvement to LLM alignment compared to those chosen based on value impact (marked as \textbf{RIDE}).
For detailed experimental design and an in-depth discussion of the aforementioned questions, please refer to Appendix~\ref{appendix:rank_dicsuss}.

\paragraph{Key Takeaways.} Based on our analysis and findings, we propose the following approach to generate an optimal ICL demo set that effectively enhances LLM alignment:
1) \textbf{Rank ICL candidates by value impact to identify the most effective examples};
2) \textbf{Apply restyling to improve alignment-related attributes};
3) \textbf{Use the hierarchical traversal approach to obtain the optimal ICL demo set}.
Figure~\ref{figure:ride_illustration} provides an illustration of the entire process for constructing the optimal ICL demonstration set.





% \section{Background}
% \label{background}
% % \subsection{Problem Formulation}
% \label{ssec:problem-formulation}

% In this work, we evaluate \ours in an educational setting where a student is trying to learn a textbook chapter's content.

% Let \( D \) be a document and \( E \) be a set of exam questions that can be solved using \( D \).
% The document \( D \) is structured as a sequence of sections, denoted by \( S_k \subset D \), where each section \( S_k \) represents the content at the \( k \)-th position in \( D \).
% For each section \( S_k \), we define \( S_{[1:k-1]} = \{S_1, S_2, \dots, S_{k-1}\} \subset D \) as the context, which includes all preceding content in the document up to section \( S_k \).

% Let \( M_q \) be a question generator that processes the document sequentially, section by section. For each section \( S_k \), it generates a set of questions \( Q_k = \{Q_k^1, Q_k^2, \dots, Q_k^n\} \), where \( Q_k \sim M_q(S_k, S_{[1:k-1]}) \), indicating that the questions are generated based on the anchor section \( S_k \) and the preceding context \( S_{[1:k-1]} \).

% Let \( M_s \) be a reader simulator.
% We assess the effectiveness of the question generator \( M_q \) by measuring the performance of \( M_s \) on the exam \( E \) using only the generated questions \( Q \), where \( Q = \{Q_1, Q_2, \dots, Q_k\} \) represents the set of all questions produced by \( M_q \) across all sections.
% This is expressed as \( M_s(E \mid Q) \), evaluating how well the generated questions contribute to solving \( E \) without direct access to \( D \).

% Our objective is to design a question generator \( M_q \) that maximizes \( M_s(E \mid Q) \), under the assumption that questions contributing more effectively to solving \( E \) are high-utility questions.





% \section{ICL Demonstrations for LLM Alignment}
% \label{approach}
% \section{Study Design}
% robot: aliengo 
% We used the Unitree AlienGo quadruped robot. 
% See Appendix 1 in AlienGo Software Guide PDF
% Weight = 25kg, size (L,W,H) = (0.55, 0.35, 06) m when standing, (0.55, 0.35, 0.31) m when walking
% Handle is 0.4 m or 0.5 m. I'll need to check it to see which type it is.
We gathered input from primary stakeholders of the robot dog guide, divided into three subgroups: BVI individuals who have owned a dog guide, BVI individuals who were not dog guide owners, and sighted individuals with generally low degrees of familiarity with dog guides. While the main focus of this study was on the BVI participants, we elected to include survey responses from sighted participants given the importance of social acceptance of the robot by the general public, which could reflect upon the BVI users themselves and affect their interactions with the general population \cite{kayukawa2022perceive}. 

The need-finding processes consisted of two stages. During Stage 1, we conducted in-depth interviews with BVI participants, querying their experiences in using conventional assistive technologies and dog guides. During Stage 2, a large-scale survey was distributed to both BVI and sighted participants. 

This study was approved by the University’s Institutional Review Board (IRB), and all processes were conducted after obtaining the participants' consent.

\subsection{Stage 1: Interviews}
We recruited nine BVI participants (\textbf{Table}~\ref{tab:bvi-info}) for in-depth interviews, which lasted 45-90 minutes for current or former dog guide owners (DO) and 30-60 minutes for participants without dog guides (NDO). Group DO consisted of five participants, while Group NDO consisted of four participants.
% The interview participants were divided into two groups. Group DO (Dog guide Owner) consisted of five participants who were current or former dog guide owners and Group NDO (Non Dog guide Owner) consisted of three participants who were not dog guide owners. 
All participants were familiar with using white canes as a mobility aid. 

We recruited participants in both groups, DO and NDO, to gather data from those with substantial experience with dog guides, offering potentially more practical insights, and from those without prior experience, providing a perspective that may be less constrained and more open to novel approaches. 

We asked about the participants' overall impressions of a robot dog guide, expectations regarding its potential benefits and challenges compared to a conventional dog guide, their desired methods of giving commands and communicating with the robot dog guide, essential functionalities that the robot dog guide should offer, and their preferences for various aspects of the robot dog guide's form factors. 
For Group DO, we also included questions that asked about the participants' experiences with conventional dog guides. 

% We obtained permission to record the conversations for our records while simultaneously taking notes during the interviews. The interviews lasted 30-60 minutes for NDO participants and 45-90 minutes for DO participants. 

\subsection{Stage 2: Large-Scale Surveys} 
After gathering sufficient initial results from the interviews, we created an online survey for distributing to a larger pool of participants. The survey platform used was Qualtrics. 

\subsubsection{Survey Participants}
The survey had 100 participants divided into two primary groups. Group BVI consisted of 42 blind or visually impaired participants, and Group ST consisted of 58 sighted participants. \textbf{Table}~\ref{tab:survey-demographics} shows the demographic information of the survey participants. 

\subsubsection{Question Differentiation} 
Based on their responses to initial qualifying questions, survey participants were sorted into three subgroups: DO, NDO, and ST. Each participant was assigned one of three different versions of the survey. The surveys for BVI participants mirrored the interview categories (overall impressions, communication methods, functionalities, and form factors), but with a more quantitative approach rather than the open-ended questions used in interviews. The DO version included additional questions pertaining to their prior experience with dog guides. The ST version revolved around the participants' prior interactions with and feelings toward dog guides and dogs in general, their thoughts on a robot dog guide, and broad opinions on the aesthetic component of the robot's design. 


\section{Evaluation}
\label{Evaluation}
% todo: can other baselines be improved by using our plug-in RIDE? how to prove it? 
\subsection{Dataset, LLMs, and baseline methods}
\paragraph{Dataset.}
We use \alpaca{} (a benchmark designed to assess the performance of language models on natural language understanding, generation, and reasoning tasks)~\citep{alpaca_eval}, \dataname{} (a dataset designed to assess the safety and reasoning capabilities of LLMs)~\citep{DBLP:conf/iclr/LinRLDSCB024}, and \mtbench{} (a multi-turn dialogue dataset to evaluate various capabilities of LLMs, such as reasoning and coding)~\citep{zheng2023judging} as benchmarks.

In Sections~\ref{section2} and~\ref{section3}, we extracted a 50-sample subset from \dataname{} as the \textit{validation} dataset to facilitate our analysis and research on stylistic impact.
The remaining data from \dataname{} is designated as the \textit{test} dataset, which is used for benchmarking against baseline methods.
Notably, the \dataname{} \textit{validation} and \textit{test} datasets used in this study are \textbf{orthogonal}, ensuring that \textbf{no information leakage} occurs during evaluation.

\paragraph{LLMs.}
We use three models as the base models: Llama-2-7b-hf~\citep{touvron2023llama}, Mistral-7b-v0.1~\citep{jiang2023mistral}, and OLMo-7B~\citep{groeneveld2024olmo}. 
It is important to note that these models have not undergone alignment tuning, resulting in sub-optimal alignment capabilities.

\paragraph{Baseline methods.}
We selected different baseline methods for comparison. 
The most relevant to our work is \textbf{\methodname{}}~\citep{DBLP:conf/iclr/LinRLDSCB024}, achieving state-of-the-art (SOTA) performance across multiple datasets using the ICL approach. 
Additionally, we compared against the following baselines: (1) \textbf{Zero-shot}: consisting only of the URIAL system instruction part. (2) \textbf{Vanilla ICL}: an ICL example set composed of the top-2 examples from $\{S_\text{cand\_f}\}$ and the top-1 example from $\{S_\text{cand\_s}\}$. (3) \textbf{Retrieval ICL}~\citep{liu2022makes}: Among the examples in $\{S_\text{cand}\}$, the neighbors that are the most similar to the given test query are retrieved as the corresponding in-context examples. (4) \textbf{TopK + ConE}~\citep{peng2024revisiting}: a tuning-free method that retrieves the best three examples that excel in reducing the conditional entropy of the test input as the ICL demonstrations.
In this work, we consistently use GPT-4o as the LLM-as-a-judge to evaluate and score the responses generated by the LLMs.
Through comparing these baseline methods with our proposed ICL demonstration set, i.e., $\textbf{RIDE}_{\text{f}}$, $\textbf{RIDE}_{\text{fs\_uni}}$, and $\textbf{RIDE}_{\text{fs\_hyb}}$, we conducted a detailed experimental analysis.

\subsection{Q1: Does \textbf{RIDE} improve the LLM’s alignment performance?}
% \subsection{Empirical results on \dataname{}}
\label{ssec:exp_justeval}
% \vspace{5em}
% \begin{table*}[t]
\begin{table*}[!h]
% \vspace{-2.5em}
% \hspace{-0.6em}
\centering
\scalebox{0.85}{ 
 
\begin{tabular}{@{}lcccccccc@{}} % Changed from || to |
% \toprule
 \textbf{Models + ICL Methods} &  {\textbf{Helpful}} &  {\textbf{Factual}} &   {\textbf{Deep}} &   {\textbf{Engaging}} &   {\textbf{Clear}} &   {\textbf{Safe}} & {\textbf{Average}} &   {\textbf{Length}} \\
 \midrule
% {\small \faToggleOn} Vicuna-7b (SFT)        &          \textbf{4.43} &      \textbf{4.85} &         \textbf{4.33} &    \textbf{4.04} &         4.51 &     4.60 &  {4.46} &    184.8 \\
% {\small \faToggleOn} Llama2-7b-chat (RLHF)  &          4.10 &      4.83 &         {4.26} &    3.91 &         \textbf{4.70} &     \textbf{5.00} &  \textbf{4.47} &    \textbf{246.9} \\ 
% \midrule
Llama2-7b + \textbf{Zero-shot}                      &   2.94            &  2.79             &  2.57             & 3.66          & 3.65          & 2.24              &  2.98                 & 211.99  \\
Llama2-7b + \textbf{Vanilla ICL}                    &   3.21            &  3.26             &  2.85             & 4.00          & 3.96          & 2.55              &  3.31                 & 224.52  \\
Llama2-7b + \textbf{Retrieval ICL}                  &   3.27            &  3.19             &  3.17             & 4.04          & 3.87          & 2.75              &  3.38                 & 229.17  \\
Llama2-7b + \textbf{TopK + ConE}                    &   3.44            &  3.45             &  3.20             & 4.02              & 4.16          & 2.80          &  3.51                 & 226.11  \\

\midrule

Llama2-7b + \iconminiurial \textbf{\methodname{}}                  &    3.98              &  \textbf{3.98}  &   3.64             & 4.36           & 4.52          & 4.42           &  4.15              & 239.81  \\

Llama2-7b + \iconminiride $\textbf{RIDE}_{\text{f}}$              &    \textbf{4.09}    &  3.87           &   \textbf{3.82}    & \textbf{4.52}  & \textbf{4.56}  & 2.81           &  3.95                 & \textbf{303.41}  \\
Llama2-7b + \iconminiride $\textbf{RIDE}_{\text{fs\_uni}}$        &    3.90           &  3.90            &   3.64            & 4.34              & 4.48          & 4.17           &  4.07                 & 266.76 \\
Llama2-7b + \iconminiride $\textbf{RIDE}_{\text{fs\_hyb}}$        &    3.95           &  3.95             &   3.69            & 4.40              & 4.52          & \textbf{4.45}  &  \textbf{4.16}       & 238.05  \\


\midrule \midrule
% Mistral-7b-instruct (SFT)   &          4.36 &      4.87 &         4.29 &    3.89 &         4.47 &     4.75 &  4.44 &    155.4 \\
% Mistral-7b (\methodname{})       &          4.57 &      4.89 &         4.50 &    4.18 &         4.74 &     4.92 &  4.63 &    186.3 \\
% {\small \faToggleOn} Mistral-7b-instruct (SFT)   &          4.36 &      4.87 &         4.29 &    3.89 &         4.47 &     4.75 &      {155.4} \\
Mistral-7b + \iconminiurial \textbf{\methodname{}}                 &    4.41           &      4.43             &         3.90      &    4.57       &         4.79      &     \textbf{4.89} & 4.50           &   214.60 \\
Mistral-7b + \iconminiride $\textbf{RIDE}_{\text{f}}$             &    \textbf{4.67}  &      \textbf{4.49}    &   \textbf{4.42}   & \textbf{4.75} &  \textbf{4.85}    &     4.13          & 4.55  &   \textbf{304.51} \\
Mistral-7b + \iconminiride $\textbf{RIDE}_{\text{fs\_uni}}$       &    4.59           &      4.44             &         4.27      &    4.69       &       4.83        &     4.50          & 4.55           &   289.19 \\
Mistral-7b + \iconminiride $\textbf{RIDE}_{\text{fs\_hyb}}$       &    4.58           &      4.43             &         4.16      &    4.63       &         4.83      &     \textbf{4.89} & \textbf{4.60}           &   252.69 \\

\midrule \midrule
Olmo-7b + \iconminiurial \textbf{\methodname{}}                    &   3.45            &  3.62             &    3.13           &    3.94           &     4.20          &  \textbf{2.70}   & \textbf{3.51}    &   203.86 \\
Olmo-7b + \iconminiride $\textbf{RIDE}_{\text{f}}$                &   \textbf{3.52}   &  3.57             &    \textbf{3.20}  &    \textbf{4.10}  &     \textbf{4.27} &  1.79             & 3.41  &   \textbf{225.31} \\
Olmo-7b + \iconminiride $\textbf{RIDE}_{\text{fs\_uni}}$          &   3.46            &  3.61             &    3.14           &    3.93           &     4.25          &  2.44             & 3.47    &   200.92 \\
Olmo-7b + \iconminiride $\textbf{RIDE}_{\text{fs\_hyb}}$          &   3.44            &  \textbf{3.65}    &    3.08           &    3.88           &     4.20          &  2.69             & 3.48    &   189.96 \\

% \midrule \midrule
% {\small \faToggleOn} \texttt{gpt-3.5-turbo-0301}        &          4.81 &      4.98 &         4.83 &    4.33 &         4.58 &     4.94 &  4.75 &    154.0 \\
% {\small \faToggleOn} \texttt{gpt-4-0314}           &          \textbf{4.90} &      \textbf{4.99} &         4.90 &    \textbf{4.57} &         \textbf{4.62} &     4.74 &  {4.79} &    \textbf{226.4} \\
% {\small \faToggleOn} \texttt{gpt-4-0613}           &          4.86 &      \textbf{4.99} &         \textbf{4.90} &    4.49 &         4.61 &     \textbf{4.97} &  \textbf{4.80} &    186.1 \\
\bottomrule
\end{tabular}
}
\caption{\textbf{Multi-aspect scoring evaluation of alignment methods on \dataname{}.} Each block is corresponding to one specific LLM.  Scores are on a scale of 1-5. {\textbf{Average}} refers to the averaged score of the 6 metrics and {\textbf{Length}} is computed by number of words. 
% The icon {\small \faToggleOn} indicates the models are \textit{tuned} for alignment via SFT or RLHF, while {\small \faToggleOff} means the models are \textit{untuned}. 
% Our method \methodname{} uses 1k static prefix tokens (system prompt + 3 examples) for ICL.
 \vspace{-0.5em}}   
\label{tab:justeval}
\end{table*}




\dataname{} aims to assess the trade-off between \textbf{\color{myblue} factuality} and \textbf{\color{myred} safety} in LLM alignment, ensuring that the model can provide informative responses while refusing malicious queries.

\paragraph{Results.} Table~\ref{tab:justeval} presents the scores of each method on \dataname{}. 
From the table, we can summarize the following conclusions.

\paragraph{$\textbf{RIDE}_{\text{fs\_hyb}}$ achieves the best overall performance.} (i) Among the three proposed ICL sets, $\textbf{RIDE}_{\text{fs\_hyb}}$ performs the best, followed by $\textbf{RIDE}_{\text{fs\_uni}}$, while $\textbf{RIDE}_{\text{f}}$ ranks lowest.
(ii) $\textbf{RIDE}_{\text{fs\_hyb}}$ maintains a strong \textbf{\color{myblue} factuality} performance while significantly enhancing \textbf{\color{myred} safety}, thanks to the ``\textbf{refusal}'' style safety example.
(iii) $\textbf{RIDE}_{\text{f}}$, consisting solely of \textbf{\color{myblue} factuality} examples, excels in \textbf{\color{myblue} factuality} but lacks \textbf{\color{myred} safety} training, resulting in a significantly lower ``Safe'' score.

\paragraph{\textbf{RIDE} outperforms \textbf{URIAL} in most cases.} (i) $\textbf{RIDE}_{\text{fs\_hyb}}$ outperforms \methodname{} in two out of three models, demonstrating its superior alignment performance.
(ii) Due to OLMo-7B's input length limitation, some ICL content had to be truncated, slightly reducing ``Helpful", ``Factual", and ``Deep" scores. However, $\textbf{RIDE}_{\text{fs\_hyb}}$ remains competitive with \methodname{}, achieving nearly identical ``Safe" scores.

\paragraph{Baseline methods exhibit a significant performance gap.} (i) As shown in the first block of Llama2-7b, the baseline methods perform notably worse than our \textbf{RIDE} and \methodname{} ICL sets. (ii) \textbf{TopK + ConE}, the strongest baseline, selects ICL demos based on their impact during inference but still lags behind \textbf{RIDE}.

\paragraph{\textbf{RIDE} demonstrates the effectiveness of hierarchical traversal.} (i) Simply combining the best-performing ICL examples from $\{S_\text{cand\_f}\}$ and $\{S_\text{cand\_s}\}$ does not yield an optimal ICL demo set.
The performance gap between \textbf{Vanilla ICL} and \textbf{RIDE} highlights the effectiveness of the hierarchical traversal approach in selecting the best ICL demonstrations.

It is worth noting that, in this benchmark, we exclusively utilized Llama-2-7b-hf to compare all baseline methods and assess their performance, aiming to minimize token consumption when invoking LLM-as-a-judge.
For details on the experimental design, result analysis, and discussion of \textbf{Q1}, please refer to the Appendix~\ref{append:justeval_discuss}.

\subsection{Q2: Does \textbf{RIDE} elicit LLMs to generate high-quality and informative responses?}
\label{ssec:exp_alpaca}
% \vspace{5em}
% \begin{table*}[t]
\begin{table}[!h]
% \vspace{-2.5em}
% \hspace{-0.6em}
\centering
\scalebox{0.85}{ 
 
\begin{tabular}{@{}lcc@{}} % Changed from || to |
% \toprule
 \textbf{Models + ICL Methods} & {\textbf{Avg.}} &   {\textbf{Len.}} \\
\midrule

Llama2-7b +\iconminiurial \textbf{\methodname{}}                  &    3.99              & 238.67  \\

Llama2-7b + \iconminiride $\textbf{RIDE}_{\text{f}}$              &    \textbf{4.08}    & 263.62  \\
Llama2-7b + \iconminiride $\textbf{RIDE}_{\text{fs\_uni}}$        &    4.00             & \textbf{265.15}  \\
Llama2-7b + \iconminiride $\textbf{RIDE}_{\text{fs\_hyb}}$        &    3.98             & 243.00  \\


\midrule 
Mistral-7b + \iconminiurial \textbf{\methodname{}}                 &    4.34           &   196.67 \\
Mistral-7b + \iconminiride $\textbf{RIDE}_{\text{f}}$             &    \textbf{4.56}  &   276.79 \\
Mistral-7b + \iconminiride $\textbf{RIDE}_{\text{fs\_uni}}$       &    4.52           &   \textbf{277.26} \\
Mistral-7b + \iconminiride $\textbf{RIDE}_{\text{fs\_hyb}}$       &    4.47           &   251.42 \\

\midrule 
Olmo-7b + \iconminiurial \textbf{\methodname{}}                    &   3.56    &   202.94 \\
Olmo-7b + \iconminiride $\textbf{RIDE}_{\text{f}}$                &   \textbf{3.62}    &   \textbf{208.57} \\
Olmo-7b + \iconminiride $\textbf{RIDE}_{\text{fs\_uni}}$          &   3.61    &   198.65 \\
Olmo-7b + \iconminiride $\textbf{RIDE}_{\text{fs\_hyb}}$          &   3.60    &   191.68 \\

\bottomrule
\end{tabular}
}
\caption{\textbf{The factuality overall evaluation of ICL methods on \alpaca{}.} ``Avg.'' refers to the average score of the metrics ``Helpful'', ``Factual'', ``Deep'', ``Engaging'', and ``Clear''. ``Len.'' represents the average length of the generated answers.} 
% The icon {\small \faToggleOn} indicates the models are \textit{tuned} for alignment via SFT or RLHF, while {\small \faToggleOff} means the models are \textit{untuned}. 
% Our method \methodname{} uses 1k static prefix tokens (system prompt + 3 examples) for ICL.
 \vspace{-0.5em}   
\label{tab:alpaca_overall}
\end{table}




% \begin{figure*}[t] 
%     \centering
%     \subfloat[Llama2-7b]{\includegraphics[width=0.32\textwidth]{latex/figure/radar_llama2.png}} \hspace{0.005\textwidth}
%     \subfloat[Mistral-7b]{\includegraphics[width=0.32\textwidth]{latex/figure/radar_mistral.png}} \hspace{0.005\textwidth}
%     \subfloat[Olmo-7b]{\includegraphics[width=0.32\textwidth]{latex/figure/radar_olmo.png}} \\[0.05cm] 
%     \caption{In \alpaca{}, comparisons are made of LLM alignment performance across various aspects using different backbone LLMs.}
%     \label{fig:eval_alpaca}
% \end{figure*}

To assess whether the distinctive styles in \textbf{RIDE} can enhance high-quality, well-structured, and information-rich responses, we conduct experiments using \alpaca{}, a dataset that primarily evaluates \textbf{\color{myblue} factuality} rather than \textbf{\color{myred} safety}. Unlike \dataname{}, \alpaca{} focuses solely on instruction-following capabilities without considering potential harm\footnote{\url{https://github.com/tatsu-lab/alpaca_eval}}, making it suitable for analyzing how ICL demonstrations influence factuality performance in LLMs.

In Table~\ref{tab:alpaca_overall}, we compute the average of ``helpful'', ``factual'', ``deep'', ``engaging'', and ``clear'' metrics to assess the overall \textbf{\color{myblue} factuality} capability of the LLM.
Therefore, we have the following findings.

\paragraph{$\textbf{RIDE}_{\text{f}}$ achieves the best factuality performance.} (i) Among the \textbf{RIDE} series, $\textbf{RIDE}_{\text{f}}$ achieves the highest \textbf{\color{myblue} factuality} (``\textbf{Avg.}''), followed by $\textbf{RIDE}_{\text{fs\_uni}}$, then $\textbf{RIDE}_{\text{fs\_hyb}}$. (ii) This result is opposite to that in Table~\ref{tab:justeval}, as \alpaca{} focuses solely on \textbf{\color{myblue} factuality}, making the factuality-only set $\textbf{RIDE}_{\text{f}}$ the most effective.

\paragraph{\textbf{RIDE} outperforms \textbf{URIAL} in factuality across all models.} The restyled ICL examples in $\textbf{RIDE}_{\text{f}}$ help the LLM quickly learn an effective output pattern, leading to higher \textbf{\color{myblue} factuality} performance than URIAL.

\paragraph{\textbf{RIDE} enhances response quality without increasing length.} (i) Despite previous research suggesting that longer responses tend to receive higher LLM-as-a-judge ratings~\citep{DBLP:journals/corr/abs-2404-04475}, $\textbf{RIDE}_{\text{f}}$ outperforms other methods even with a shorter response ``Len.'' in both Llama2 and Mistral settings.
(ii) In the Olmo setting, \textbf{URIAL} produces longer responses than $\textbf{RIDE}_{\text{fs\_uni}}$ and $\textbf{RIDE}_{\text{fs\_hyb}}$ but still performs the worst. This confirms that \textbf{RIDE}'s superior factuality ratings stem from improved content quality, not response length.

For the detailed scores of each individual metric, as well as an in-depth discussion of different "model + ICL method" settings used in \alpaca{}, please refer to Appendix~\ref{append:alpaca_discuss} and~\ref{append:alpaca_all_discuss}.

\subsection{Q3: Does \textbf{RIDE} enhance LLMs' ability to handle complex tasks?}
\label{ssec:exp_mtbench}

% \vspace{5em}
% \begin{table*}[t]
% \begin{table}[!h]
\begin{table}[t]
% \vspace{-2.5em}
% \hspace{-0.6em}
\centering
\scalebox{0.82}{ 
 
\begin{tabular}{@{}lccc@{}} % Changed from || to |
% \toprule
\textbf{Models + ICL Methods} &  {\small \textbf{Turn 1}} &  {\small \textbf{Turn 2}} &   {\small \textbf{Overall}} \\
 \midrule
Llama2-7b + \iconminiurial \textbf{\methodname{}}              &   5.49            &  \textbf{3.91}            &  4.70             \\
Llama2-7b + \iconminiride $\textbf{RIDE}_{\text{f}}$          &   \textbf{6.01}   &  3.84                     &  \textbf{4.93}    \\
Llama2-7b + \iconminiride $\textbf{RIDE}_{\text{fs\_uni}}$    &   5.54            &  3.80                     &  4.67             \\
Llama2-7b + \iconminiride $\textbf{RIDE}_{\text{fs\_hyb}}$    &   5.58            &  \textbf{3.91}            &  4.74             \\

\midrule
% Mistral-7b-instruct (SFT)   &          4.36 &      4.87 &         4.29 &    3.89 &         4.47 &     4.75 &  4.44 &    155.4 \\
% Mistral-7b (\methodname{})       &          4.57 &      4.89 &         4.50 &    4.18 &         4.74 &     4.92 &  4.63 &    186.3 \\
Mistral-7b + \iconminiurial \textbf{\methodname{}}             &   7.49            &  5.44             &  6.46          \\
Mistral-7b + \iconminiride $\textbf{RIDE}_{\text{f}}$         &   7.26            &  \textbf{6.22}    &  \textbf{6.74}   \\
Mistral-7b + \iconminiride $\textbf{RIDE}_{\text{fs\_uni}}$   &   7.10            &  5.76             &  6.43           \\
Mistral-7b + \iconminiride $\textbf{RIDE}_{\text{fs\_hyb}}$   &   \textbf{7.53}   &  5.51             &  6.52           \\

\midrule 
% Llama2-70b-chat (RLHF)   &      4.50 &      4.92 &         4.54 &    4.28 &         4.75 &     5.00 &  4.67 &    257.9 \\
% Llama2-70b (zero-shot prompt)         &          3.70 &      4.31 &         3.78 &    3.19 &         3.50 &     1.50 &  3.33 &    166.8 \\
% Llama2-70b (\methodname{}-1shot) &          4.60 &      4.93 &         4.54 &    4.09 &         4.67 &     4.88 &  4.62 &    155.3 \\
% Llama2-70b (\methodname{})       &          4.72 &      4.95 &         4.65 &    4.30 &         4.85 &     4.96 &  4.74 &    171.4 \\
Olmo-7b + \iconminiurial \textbf{\methodname{}}                &   4.54          &  2.49           &  3.53           \\
Olmo-7b + \iconminiride $\textbf{RIDE}_{\text{f}}$            &   \textbf{5.13} &  \textbf{2.56}  &  \textbf{3.85}   \\
Olmo-7b + \iconminiride $\textbf{RIDE}_{\text{fs\_uni}}$      &   4.56          &  2.19           &  3.38           \\
Olmo-7b + \iconminiride $\textbf{RIDE}_{\text{fs\_hyb}}$      &   4.79          &  2.42           &  3.61           \\

% \midrule \midrule
% {\small \faToggleOn} \texttt{gpt-3.5-turbo-0301}        &          4.81 &      4.98 &         4.83 &    4.33 &         4.58 &     4.94 &  4.75 &    154.0 \\
% {\small \faToggleOn} \texttt{gpt-4-0314}           &          \textbf{4.90} &      \textbf{4.99} &         4.90 &    \textbf{4.57} &         \textbf{4.62} &     4.74 &  {4.79} &    \textbf{226.4} \\
% {\small \faToggleOn} \texttt{gpt-4-0613}           &          4.86 &      \textbf{4.99} &         \textbf{4.90} &    4.49 &         4.61 &     \textbf{4.97} &  \textbf{4.80} &    186.1 \\
\bottomrule
\end{tabular}
}
\caption{\textbf{Overall evaluation of ICL methods on \mtbench.} (Scores are on a scale of 1-10.) }
% The icon {\small \faToggleOn} indicates the models are \textit{tuned} for alignment via SFT or RLHF, while {\small \faToggleOff} means the models are \textit{untuned}. 
% Our method \methodname{} uses 1k static prefix tokens (system prompt + 3 examples) for ICL.
 \vspace{-1em}   
\label{tab:mtbench_overall}
\end{table}




\mtbench{} assesses LLM capability in handling complex tasks by requiring the integration of logical reasoning, numerical computation, coding, and other advanced skills, making it a suitable benchmark for measuring LLM proficiency in complex problem-solving.
From Table~\ref{tab:mtbench_overall}, we can draw the following findings (further discussion can be found in Appendix~\ref{appendix:mtbench_dicsuss}).

% Here we present only the experimental conclusions drawn from Table~\ref{tab:mtbench_overall}. 
% For details on the experimental design, result analysis, and discussion of \textbf{Q3}, please refer to the Appendix~\ref{appendix:mtbench_dicsuss}.

\textbf{\textbf{RIDE} outperforms \textbf{URIAL} across all settings.} (i) $\textbf{RIDE}_{\text{f}}$ achieves the best overall performance, followed by $\textbf{RIDE}_{\text{fs\_hyb}}$, then $\textbf{RIDE}_{\text{fs\_uni}}$.
(ii) The structured and logically coherent responses from the ``Combined'' (mostly because of ``Three-part'') style in $\textbf{RIDE}_{\text{f}}$ enhance LLM \textbf{\color{myblue} factuality} and \textbf{reasoning} capabilities, making it the top-performing approach.
(iii) The inclusion of \textbf{\color{myred} safety}-focused examples in $\textbf{RIDE}_{\text{fs\_hyb}}$ and $\textbf{RIDE}_{\text{fs\_uni}}$ slightly weakens their ability to handle complex tasks.

\textbf{$\textbf{RIDE}_{\text{fs\_hyb}}$ outperforms $\textbf{RIDE}_{\text{fs\_uni}}$.} The ``Refusal''-style example in $\textbf{RIDE}_{\text{fs\_hyb}}$ follows a structured reasoning process (refusal $\rightarrow$ justification $\rightarrow$ guidance), aligning well with the logical reasoning required by \mtbench{}, which contributes to its superior performance.

\textbf{\textbf{RIDE} Improves Multi-Turn Dialogue Performance.} In two out of three models (Mistral-7B and Olmo-7B), \textbf{RIDE} outperforms \textbf{URIAL} in Turn 2, demonstrating its effectiveness in multi-turn dialogue tasks despite being designed for single-turn scenarios.

\textbf{\textbf{RIDE} Enhances Logical Reasoning and Complex Computation.} As further evidenced in Table~\ref{tab:mtbench_tf}, we evaluated the accuracy of different methods in answering \textit{objective} questions from \mtbench{}. Our findings indicate that \textbf{RIDE} achieves higher accuracy in responding to \textit{objective} questions compared to the baseline methods.Detailed performance results are available in Appendix~\ref{appendix:tf_dicsuss}.

\subsection{Q4: Can base LLM outperform its aligned counterpart by employing \textbf{RIDE}?}
\label{ssec:exp_win_finetune}

\paragraph{Results.}
Our findings conclusively show that \textbf{yes}, a base LLM can \textbf{outperform} its aligned counterpart! As detailed in Table~\ref{tab:win_finetune} in Appendix~\ref{append:win_finetune}, when the base model Mistral-7B-v0.1 utilizes \textbf{RIDE} as its ICL demonstrations, it achieves superior alignment performance compared to Mistral-7B-Instruct-v0.1 across all three datasets. 
% While this could potentially be attributed to insufficient alignment fine-tuning in Mistral-7B-Instruct-v0.1, 
We argue that for sufficiently capable base models, \textbf{RIDE} can effectively elicit their inherent alignment potential. Notably, our approach offers significant practical advantages: it is tuning-free, plug-and-play, and requires minimal training and deployment costs.
We leave further discussion about \textbf{Q4} in Appendix~\ref{append:win_finetune}.

% \subsection{Takeaways of the Empirical Study}
% From the findings in Section~\ref{ssec:exp_justeval}, \ref{ssec:exp_justeval} and \ref{ssec:exp_mtbench}, in conclusion, we find that different benchmarks have different focal points when evaluating the capabilities of LLMs. 
% In this context, both the \textbf{content} (whether the ICL example set is oriented toward safety or factuality) and the \textbf{style} (whether the restyling is more structured or focused on refusal) interact with each other to jointly influence the performance of ICL. 
% Thus, through our experiments, we validate the causal relationship among \textit{content}, \textit{style}, and \textit{alignment}. 
% Furthermore, we observe that the \textit{safety} and \textit{factuality} capabilities of LLMs are inherently conflicting, requiring us to find a trade-off to achieve the best overall performance. 
% This finding is also consistent with our analysis of \textit{polarity tokens}. 
% Therefore, the experimental results validate the two key concepts proposed in this paper: the existence of \textit{polarity tokens} with different orientations toward safety and factuality, and the \textit{causal structure} within alignment.

\section{Related Work}
\label{Related}
% Alignment tuning serves as a crucial mechanism for narrowing the gap between the inherent capabilities of raw models and the nuanced requirements of various tasks, including delivering accurate information, ensuring user safety, and appropriately handling sensitive topics~\citep{shneiderman2020bridging,shen2023large,wang2023aligning,DBLP:conf/iclr/Qi0XC0M024}.

% Currently, the instruction-following paradigm~\citep{ouyang2022training,sun2023aligning,DBLP:conf/iclr/DaiPSJXL0024,rafailov2024direct,zhou2024lima,wu2024self}, which integrates supervised fine-tuning (SFT) and preference optimization, is the predominant approach for alignment tuning. However, this paradigm necessitates high-quality annotated data and incurs substantial computational costs. Unlike instruction-following approaches, our proposed method is entirely tuning-free and plug-and-play, eliminating the need for additional training while remaining computationally efficient.

% Meanwhile, a growing body of research suggests that alignment tuning alters the token generation probabilities of base LLMs~\citep{DBLP:conf/iclr/LinRLDSCB024, DBLP:journals/corr/abs-2406-05946, DBLP:journals/corr/abs-2407-09121, huang2024safealigner}.

% Most relevant to our work, URIAL~\citep{DBLP:conf/iclr/LinRLDSCB024} introduced an ICL demo set containing three manually crafted exemplars and empirically demonstrated that these manually designed ICL exemplars effectively enhance LLM alignment. However, their approach remains somewhat of a black box, as it does not explain why these particular handcrafted ICL demos are effective.
% In contrast, our work explicitly analyzes the key factors influencing LLM alignment and constructs our ICL demo set based on these identified principles.

Alignment tuning helps bridge the gap between raw model capabilities and task-specific requirements~\citep{shneiderman2020bridging,shen2023large,wang2023aligning,DBLP:conf/iclr/Qi0XC0M024}. The instruction-following paradigm\citep{ouyang2022training,sun2023aligning,DBLP:conf/iclr/DaiPSJXL0024,rafailov2024direct,zhou2024lima,wu2024self} requires high-quality annotated data and significant computational resources. In contrast, our tuning-free, plug-and-play approach eliminates the need for additional training while maintaining efficiency.

Research indicates that alignment tuning alters token generation probabilities in LLMs~\citep{DBLP:conf/iclr/LinRLDSCB024, DBLP:journals/corr/abs-2406-05946, DBLP:journals/corr/abs-2407-09121, huang2024safealigner}. Most relevant to our work, URIAL~\citep{DBLP:conf/iclr/LinRLDSCB024} proposed a manually crafted ICL demo set to enhance alignment but did not provide insights into why these demos were effective. Unlike URIAL, our work transparently analyzes key alignment factors and constructs ICL demo sets based on identified principles.

\section{Conclusion}
\label{Conclusion}
In this paper, we take the initial step by designing a metric to evaluate the effectiveness of ICL demonstration exemplars—value impact—which we use to analyze the characteristics of ICL demos that effectively enhance LLM alignment capabilities. We categorize these characteristics under the term "style" and, based on this insight, propose a "restyling" method to optimize ICL demos with high value impact.
We conduct experiments across three datasets, and the results demonstrate that our restyling approach effectively stimulates LLMs to generate informative and safe content while also enhancing their capabilities in logical reasoning, numerical computation, and other complex tasks.

\section*{Limitations}
\label{limitation}
Despite the effectiveness of the proposed \textbf{RIDE} method in enhancing LLM alignment, several limitations and potential risks should be acknowledged.

\paragraph{Limited Scope of ICL Demonstrations.} One key limitation of this study is the restricted selection of ICL demonstrations. The candidate ICL demos were drawn from a subset of a large dataset, which may limit their diversity and generalizability. Given that alignment performance is highly dependent on the variety of training examples, a more extensive and diverse selection of candidate ICL exemplars could potentially yield stronger results. Future work should explore the impact of expanding the candidate pool by incorporating demonstrations from multiple datasets across different domains.

\paragraph{Dependency on LLM-as-a-Judge for Evaluation.} The evaluation methodology relies on using a strong LLM-as-a-judge (ChatGPT or Claude-3.5 Sonnet) to assess the effectiveness of restyled demonstrations. While this provides a cost-effective alternative to human evaluation, it introduces potential biases. LLMs used for scoring may favor responses that align with their own training data and reward certain styles over others in a way that may not fully reflect human preferences. Future work should incorporate human evaluations to validate the robustness of the results.

\paragraph{Potential for Misuse and Ethical Considerations.} Although \textbf{RIDE} aims to enhance LLM alignment, there exists a risk of its misuse. If adversarial actors manipulate ICL demonstrations using the same restyling approach, they could attempt to bypass safety constraints or generate misleading outputs. Additionally, optimizing for alignment does not eliminate the potential for biases present in the base LLMs, which may still surface despite restyling efforts. Ensuring continuous auditing and ethical oversight in deploying such methods is essential.

\paragraph{Future Directions.} To address these limitations, future research should: (i) Expand the candidate ICL demo pool to improve generalization across diverse datasets.
(ii) Reduce dependency on LLM-as-a-judge by integrating human assessments and alternative evaluation methods.
(iii) Establish safeguards against potential adversarial uses of restyled ICL demonstrations.


\section*{Ethics Statement}
\label{ethics}
\paragraph{Malicious contents.} This research focuses on improving LLM alignment, which inherently involves handling malicious queries as part of the evaluation process. These queries may contain offensive, harmful, or sensitive content, which could be distressing to some readers. However, we emphasize that such malicious queries are included solely for research purposes, ensuring that our findings contribute to the development of more responsible and safe AI systems.

\paragraph{Data anonymization and Ethical Considerations.} We have taken steps to ensure that no personally identifiable information (PII) or offensive content is present in the datasets used for training and evaluation. Any potentially harmful content within the datasets has been either anonymized or strictly controlled to prevent ethical concerns related to data privacy and misuse. Moreover, the research adheres to responsible AI guidelines, ensuring that the use of existing datasets aligns with their intended purpose, and that any new artifacts created follow the original access conditions.

\paragraph{Intended Use and Research Scope.} Our approach is designed for research purposes only and aims to enhance the alignment capabilities of LLMs. While we propose a novel in-context learning (ICL) method, we acknowledge that misuse or misinterpretation of our approach could lead to unintended consequences. We stress that the techniques introduced should not be used outside of research contexts without proper ethical safeguards. Additionally, our research does not endorse the deployment of LLMs without rigorous safety evaluations, particularly in high-stakes applications.

\section*{Acknowledgments}
\label{acknow}
This research is partially supported by the ARC Center of Excellence for Automated Decision Making and Society (CE200100005).
The icons used in this paper are created and contributed by the artists from \url{Flaticon.com}.


% This document has been adapted
% by Steven Bethard, Ryan Cotterell and Rui Yan
% from the instructions for earlier ACL and NAACL proceedings, including those for
% ACL 2019 by Douwe Kiela and Ivan Vuli\'{c},
% NAACL 2019 by Stephanie Lukin and Alla Roskovskaya,
% ACL 2018 by Shay Cohen, Kevin Gimpel, and Wei Lu,
% NAACL 2018 by Margaret Mitchell and Stephanie Lukin,
% Bib\TeX{} suggestions for (NA)ACL 2017/2018 from Jason Eisner,
% ACL 2017 by Dan Gildea and Min-Yen Kan,
% NAACL 2017 by Margaret Mitchell,
% ACL 2012 by Maggie Li and Michael White,
% ACL 2010 by Jing-Shin Chang and Philipp Koehn,
% ACL 2008 by Johanna D. Moore, Simone Teufel, James Allan, and Sadaoki Furui,
% ACL 2005 by Hwee Tou Ng and Kemal Oflazer,
% ACL 2002 by Eugene Charniak and Dekang Lin,
% and earlier ACL and EACL formats written by several people, including
% John Chen, Henry S. Thompson and Donald Walker.
% Additional elements were taken from the formatting instructions of the \emph{International Joint Conference on Artificial Intelligence} and the \emph{Conference on Computer Vision and Pattern Recognition}.

% Bibliography entries for the entire Anthology, followed by custom entries
%\bibliography{anthology,custom}
% Custom bibliography entries only
\bibliography{main}

% \newpage
% \appendix
% \section{Appendix}
% \subsection{Lloyd-Max Algorithm}
\label{subsec:Lloyd-Max}
For a given quantization bitwidth $B$ and an operand $\bm{X}$, the Lloyd-Max algorithm finds $2^B$ quantization levels $\{\hat{x}_i\}_{i=1}^{2^B}$ such that quantizing $\bm{X}$ by rounding each scalar in $\bm{X}$ to the nearest quantization level minimizes the quantization MSE. 

The algorithm starts with an initial guess of quantization levels and then iteratively computes quantization thresholds $\{\tau_i\}_{i=1}^{2^B-1}$ and updates quantization levels $\{\hat{x}_i\}_{i=1}^{2^B}$. Specifically, at iteration $n$, thresholds are set to the midpoints of the previous iteration's levels:
\begin{align*}
    \tau_i^{(n)}=\frac{\hat{x}_i^{(n-1)}+\hat{x}_{i+1}^{(n-1)}}2 \text{ for } i=1\ldots 2^B-1
\end{align*}
Subsequently, the quantization levels are re-computed as conditional means of the data regions defined by the new thresholds:
\begin{align*}
    \hat{x}_i^{(n)}=\mathbb{E}\left[ \bm{X} \big| \bm{X}\in [\tau_{i-1}^{(n)},\tau_i^{(n)}] \right] \text{ for } i=1\ldots 2^B
\end{align*}
where to satisfy boundary conditions we have $\tau_0=-\infty$ and $\tau_{2^B}=\infty$. The algorithm iterates the above steps until convergence.

Figure \ref{fig:lm_quant} compares the quantization levels of a $7$-bit floating point (E3M3) quantizer (left) to a $7$-bit Lloyd-Max quantizer (right) when quantizing a layer of weights from the GPT3-126M model at a per-tensor granularity. As shown, the Lloyd-Max quantizer achieves substantially lower quantization MSE. Further, Table \ref{tab:FP7_vs_LM7} shows the superior perplexity achieved by Lloyd-Max quantizers for bitwidths of $7$, $6$ and $5$. The difference between the quantizers is clear at 5 bits, where per-tensor FP quantization incurs a drastic and unacceptable increase in perplexity, while Lloyd-Max quantization incurs a much smaller increase. Nevertheless, we note that even the optimal Lloyd-Max quantizer incurs a notable ($\sim 1.5$) increase in perplexity due to the coarse granularity of quantization. 

\begin{figure}[h]
  \centering
  \includegraphics[width=0.7\linewidth]{sections/figures/LM7_FP7.pdf}
  \caption{\small Quantization levels and the corresponding quantization MSE of Floating Point (left) vs Lloyd-Max (right) Quantizers for a layer of weights in the GPT3-126M model.}
  \label{fig:lm_quant}
\end{figure}

\begin{table}[h]\scriptsize
\begin{center}
\caption{\label{tab:FP7_vs_LM7} \small Comparing perplexity (lower is better) achieved by floating point quantizers and Lloyd-Max quantizers on a GPT3-126M model for the Wikitext-103 dataset.}
\begin{tabular}{c|cc|c}
\hline
 \multirow{2}{*}{\textbf{Bitwidth}} & \multicolumn{2}{|c|}{\textbf{Floating-Point Quantizer}} & \textbf{Lloyd-Max Quantizer} \\
 & Best Format & Wikitext-103 Perplexity & Wikitext-103 Perplexity \\
\hline
7 & E3M3 & 18.32 & 18.27 \\
6 & E3M2 & 19.07 & 18.51 \\
5 & E4M0 & 43.89 & 19.71 \\
\hline
\end{tabular}
\end{center}
\end{table}

\subsection{Proof of Local Optimality of LO-BCQ}
\label{subsec:lobcq_opt_proof}
For a given block $\bm{b}_j$, the quantization MSE during LO-BCQ can be empirically evaluated as $\frac{1}{L_b}\lVert \bm{b}_j- \bm{\hat{b}}_j\rVert^2_2$ where $\bm{\hat{b}}_j$ is computed from equation (\ref{eq:clustered_quantization_definition}) as $C_{f(\bm{b}_j)}(\bm{b}_j)$. Further, for a given block cluster $\mathcal{B}_i$, we compute the quantization MSE as $\frac{1}{|\mathcal{B}_{i}|}\sum_{\bm{b} \in \mathcal{B}_{i}} \frac{1}{L_b}\lVert \bm{b}- C_i^{(n)}(\bm{b})\rVert^2_2$. Therefore, at the end of iteration $n$, we evaluate the overall quantization MSE $J^{(n)}$ for a given operand $\bm{X}$ composed of $N_c$ block clusters as:
\begin{align*}
    \label{eq:mse_iter_n}
    J^{(n)} = \frac{1}{N_c} \sum_{i=1}^{N_c} \frac{1}{|\mathcal{B}_{i}^{(n)}|}\sum_{\bm{v} \in \mathcal{B}_{i}^{(n)}} \frac{1}{L_b}\lVert \bm{b}- B_i^{(n)}(\bm{b})\rVert^2_2
\end{align*}

At the end of iteration $n$, the codebooks are updated from $\mathcal{C}^{(n-1)}$ to $\mathcal{C}^{(n)}$. However, the mapping of a given vector $\bm{b}_j$ to quantizers $\mathcal{C}^{(n)}$ remains as  $f^{(n)}(\bm{b}_j)$. At the next iteration, during the vector clustering step, $f^{(n+1)}(\bm{b}_j)$ finds new mapping of $\bm{b}_j$ to updated codebooks $\mathcal{C}^{(n)}$ such that the quantization MSE over the candidate codebooks is minimized. Therefore, we obtain the following result for $\bm{b}_j$:
\begin{align*}
\frac{1}{L_b}\lVert \bm{b}_j - C_{f^{(n+1)}(\bm{b}_j)}^{(n)}(\bm{b}_j)\rVert^2_2 \le \frac{1}{L_b}\lVert \bm{b}_j - C_{f^{(n)}(\bm{b}_j)}^{(n)}(\bm{b}_j)\rVert^2_2
\end{align*}

That is, quantizing $\bm{b}_j$ at the end of the block clustering step of iteration $n+1$ results in lower quantization MSE compared to quantizing at the end of iteration $n$. Since this is true for all $\bm{b} \in \bm{X}$, we assert the following:
\begin{equation}
\begin{split}
\label{eq:mse_ineq_1}
    \tilde{J}^{(n+1)} &= \frac{1}{N_c} \sum_{i=1}^{N_c} \frac{1}{|\mathcal{B}_{i}^{(n+1)}|}\sum_{\bm{b} \in \mathcal{B}_{i}^{(n+1)}} \frac{1}{L_b}\lVert \bm{b} - C_i^{(n)}(b)\rVert^2_2 \le J^{(n)}
\end{split}
\end{equation}
where $\tilde{J}^{(n+1)}$ is the the quantization MSE after the vector clustering step at iteration $n+1$.

Next, during the codebook update step (\ref{eq:quantizers_update}) at iteration $n+1$, the per-cluster codebooks $\mathcal{C}^{(n)}$ are updated to $\mathcal{C}^{(n+1)}$ by invoking the Lloyd-Max algorithm \citep{Lloyd}. We know that for any given value distribution, the Lloyd-Max algorithm minimizes the quantization MSE. Therefore, for a given vector cluster $\mathcal{B}_i$ we obtain the following result:

\begin{equation}
    \frac{1}{|\mathcal{B}_{i}^{(n+1)}|}\sum_{\bm{b} \in \mathcal{B}_{i}^{(n+1)}} \frac{1}{L_b}\lVert \bm{b}- C_i^{(n+1)}(\bm{b})\rVert^2_2 \le \frac{1}{|\mathcal{B}_{i}^{(n+1)}|}\sum_{\bm{b} \in \mathcal{B}_{i}^{(n+1)}} \frac{1}{L_b}\lVert \bm{b}- C_i^{(n)}(\bm{b})\rVert^2_2
\end{equation}

The above equation states that quantizing the given block cluster $\mathcal{B}_i$ after updating the associated codebook from $C_i^{(n)}$ to $C_i^{(n+1)}$ results in lower quantization MSE. Since this is true for all the block clusters, we derive the following result: 
\begin{equation}
\begin{split}
\label{eq:mse_ineq_2}
     J^{(n+1)} &= \frac{1}{N_c} \sum_{i=1}^{N_c} \frac{1}{|\mathcal{B}_{i}^{(n+1)}|}\sum_{\bm{b} \in \mathcal{B}_{i}^{(n+1)}} \frac{1}{L_b}\lVert \bm{b}- C_i^{(n+1)}(\bm{b})\rVert^2_2  \le \tilde{J}^{(n+1)}   
\end{split}
\end{equation}

Following (\ref{eq:mse_ineq_1}) and (\ref{eq:mse_ineq_2}), we find that the quantization MSE is non-increasing for each iteration, that is, $J^{(1)} \ge J^{(2)} \ge J^{(3)} \ge \ldots \ge J^{(M)}$ where $M$ is the maximum number of iterations. 
%Therefore, we can say that if the algorithm converges, then it must be that it has converged to a local minimum. 
\hfill $\blacksquare$


\begin{figure}
    \begin{center}
    \includegraphics[width=0.5\textwidth]{sections//figures/mse_vs_iter.pdf}
    \end{center}
    \caption{\small NMSE vs iterations during LO-BCQ compared to other block quantization proposals}
    \label{fig:nmse_vs_iter}
\end{figure}

Figure \ref{fig:nmse_vs_iter} shows the empirical convergence of LO-BCQ across several block lengths and number of codebooks. Also, the MSE achieved by LO-BCQ is compared to baselines such as MXFP and VSQ. As shown, LO-BCQ converges to a lower MSE than the baselines. Further, we achieve better convergence for larger number of codebooks ($N_c$) and for a smaller block length ($L_b$), both of which increase the bitwidth of BCQ (see Eq \ref{eq:bitwidth_bcq}).


\subsection{Additional Accuracy Results}
%Table \ref{tab:lobcq_config} lists the various LOBCQ configurations and their corresponding bitwidths.
\begin{table}
\setlength{\tabcolsep}{4.75pt}
\begin{center}
\caption{\label{tab:lobcq_config} Various LO-BCQ configurations and their bitwidths.}
\begin{tabular}{|c||c|c|c|c||c|c||c|} 
\hline
 & \multicolumn{4}{|c||}{$L_b=8$} & \multicolumn{2}{|c||}{$L_b=4$} & $L_b=2$ \\
 \hline
 \backslashbox{$L_A$\kern-1em}{\kern-1em$N_c$} & 2 & 4 & 8 & 16 & 2 & 4 & 2 \\
 \hline
 64 & 4.25 & 4.375 & 4.5 & 4.625 & 4.375 & 4.625 & 4.625\\
 \hline
 32 & 4.375 & 4.5 & 4.625& 4.75 & 4.5 & 4.75 & 4.75 \\
 \hline
 16 & 4.625 & 4.75& 4.875 & 5 & 4.75 & 5 & 5 \\
 \hline
\end{tabular}
\end{center}
\end{table}

%\subsection{Perplexity achieved by various LO-BCQ configurations on Wikitext-103 dataset}

\begin{table} \centering
\begin{tabular}{|c||c|c|c|c||c|c||c|} 
\hline
 $L_b \rightarrow$& \multicolumn{4}{c||}{8} & \multicolumn{2}{c||}{4} & 2\\
 \hline
 \backslashbox{$L_A$\kern-1em}{\kern-1em$N_c$} & 2 & 4 & 8 & 16 & 2 & 4 & 2  \\
 %$N_c \rightarrow$ & 2 & 4 & 8 & 16 & 2 & 4 & 2 \\
 \hline
 \hline
 \multicolumn{8}{c}{GPT3-1.3B (FP32 PPL = 9.98)} \\ 
 \hline
 \hline
 64 & 10.40 & 10.23 & 10.17 & 10.15 &  10.28 & 10.18 & 10.19 \\
 \hline
 32 & 10.25 & 10.20 & 10.15 & 10.12 &  10.23 & 10.17 & 10.17 \\
 \hline
 16 & 10.22 & 10.16 & 10.10 & 10.09 &  10.21 & 10.14 & 10.16 \\
 \hline
  \hline
 \multicolumn{8}{c}{GPT3-8B (FP32 PPL = 7.38)} \\ 
 \hline
 \hline
 64 & 7.61 & 7.52 & 7.48 &  7.47 &  7.55 &  7.49 & 7.50 \\
 \hline
 32 & 7.52 & 7.50 & 7.46 &  7.45 &  7.52 &  7.48 & 7.48  \\
 \hline
 16 & 7.51 & 7.48 & 7.44 &  7.44 &  7.51 &  7.49 & 7.47  \\
 \hline
\end{tabular}
\caption{\label{tab:ppl_gpt3_abalation} Wikitext-103 perplexity across GPT3-1.3B and 8B models.}
\end{table}

\begin{table} \centering
\begin{tabular}{|c||c|c|c|c||} 
\hline
 $L_b \rightarrow$& \multicolumn{4}{c||}{8}\\
 \hline
 \backslashbox{$L_A$\kern-1em}{\kern-1em$N_c$} & 2 & 4 & 8 & 16 \\
 %$N_c \rightarrow$ & 2 & 4 & 8 & 16 & 2 & 4 & 2 \\
 \hline
 \hline
 \multicolumn{5}{|c|}{Llama2-7B (FP32 PPL = 5.06)} \\ 
 \hline
 \hline
 64 & 5.31 & 5.26 & 5.19 & 5.18  \\
 \hline
 32 & 5.23 & 5.25 & 5.18 & 5.15  \\
 \hline
 16 & 5.23 & 5.19 & 5.16 & 5.14  \\
 \hline
 \multicolumn{5}{|c|}{Nemotron4-15B (FP32 PPL = 5.87)} \\ 
 \hline
 \hline
 64  & 6.3 & 6.20 & 6.13 & 6.08  \\
 \hline
 32  & 6.24 & 6.12 & 6.07 & 6.03  \\
 \hline
 16  & 6.12 & 6.14 & 6.04 & 6.02  \\
 \hline
 \multicolumn{5}{|c|}{Nemotron4-340B (FP32 PPL = 3.48)} \\ 
 \hline
 \hline
 64 & 3.67 & 3.62 & 3.60 & 3.59 \\
 \hline
 32 & 3.63 & 3.61 & 3.59 & 3.56 \\
 \hline
 16 & 3.61 & 3.58 & 3.57 & 3.55 \\
 \hline
\end{tabular}
\caption{\label{tab:ppl_llama7B_nemo15B} Wikitext-103 perplexity compared to FP32 baseline in Llama2-7B and Nemotron4-15B, 340B models}
\end{table}

%\subsection{Perplexity achieved by various LO-BCQ configurations on MMLU dataset}


\begin{table} \centering
\begin{tabular}{|c||c|c|c|c||c|c|c|c|} 
\hline
 $L_b \rightarrow$& \multicolumn{4}{c||}{8} & \multicolumn{4}{c||}{8}\\
 \hline
 \backslashbox{$L_A$\kern-1em}{\kern-1em$N_c$} & 2 & 4 & 8 & 16 & 2 & 4 & 8 & 16  \\
 %$N_c \rightarrow$ & 2 & 4 & 8 & 16 & 2 & 4 & 2 \\
 \hline
 \hline
 \multicolumn{5}{|c|}{Llama2-7B (FP32 Accuracy = 45.8\%)} & \multicolumn{4}{|c|}{Llama2-70B (FP32 Accuracy = 69.12\%)} \\ 
 \hline
 \hline
 64 & 43.9 & 43.4 & 43.9 & 44.9 & 68.07 & 68.27 & 68.17 & 68.75 \\
 \hline
 32 & 44.5 & 43.8 & 44.9 & 44.5 & 68.37 & 68.51 & 68.35 & 68.27  \\
 \hline
 16 & 43.9 & 42.7 & 44.9 & 45 & 68.12 & 68.77 & 68.31 & 68.59  \\
 \hline
 \hline
 \multicolumn{5}{|c|}{GPT3-22B (FP32 Accuracy = 38.75\%)} & \multicolumn{4}{|c|}{Nemotron4-15B (FP32 Accuracy = 64.3\%)} \\ 
 \hline
 \hline
 64 & 36.71 & 38.85 & 38.13 & 38.92 & 63.17 & 62.36 & 63.72 & 64.09 \\
 \hline
 32 & 37.95 & 38.69 & 39.45 & 38.34 & 64.05 & 62.30 & 63.8 & 64.33  \\
 \hline
 16 & 38.88 & 38.80 & 38.31 & 38.92 & 63.22 & 63.51 & 63.93 & 64.43  \\
 \hline
\end{tabular}
\caption{\label{tab:mmlu_abalation} Accuracy on MMLU dataset across GPT3-22B, Llama2-7B, 70B and Nemotron4-15B models.}
\end{table}


%\subsection{Perplexity achieved by various LO-BCQ configurations on LM evaluation harness}

\begin{table} \centering
\begin{tabular}{|c||c|c|c|c||c|c|c|c|} 
\hline
 $L_b \rightarrow$& \multicolumn{4}{c||}{8} & \multicolumn{4}{c||}{8}\\
 \hline
 \backslashbox{$L_A$\kern-1em}{\kern-1em$N_c$} & 2 & 4 & 8 & 16 & 2 & 4 & 8 & 16  \\
 %$N_c \rightarrow$ & 2 & 4 & 8 & 16 & 2 & 4 & 2 \\
 \hline
 \hline
 \multicolumn{5}{|c|}{Race (FP32 Accuracy = 37.51\%)} & \multicolumn{4}{|c|}{Boolq (FP32 Accuracy = 64.62\%)} \\ 
 \hline
 \hline
 64 & 36.94 & 37.13 & 36.27 & 37.13 & 63.73 & 62.26 & 63.49 & 63.36 \\
 \hline
 32 & 37.03 & 36.36 & 36.08 & 37.03 & 62.54 & 63.51 & 63.49 & 63.55  \\
 \hline
 16 & 37.03 & 37.03 & 36.46 & 37.03 & 61.1 & 63.79 & 63.58 & 63.33  \\
 \hline
 \hline
 \multicolumn{5}{|c|}{Winogrande (FP32 Accuracy = 58.01\%)} & \multicolumn{4}{|c|}{Piqa (FP32 Accuracy = 74.21\%)} \\ 
 \hline
 \hline
 64 & 58.17 & 57.22 & 57.85 & 58.33 & 73.01 & 73.07 & 73.07 & 72.80 \\
 \hline
 32 & 59.12 & 58.09 & 57.85 & 58.41 & 73.01 & 73.94 & 72.74 & 73.18  \\
 \hline
 16 & 57.93 & 58.88 & 57.93 & 58.56 & 73.94 & 72.80 & 73.01 & 73.94  \\
 \hline
\end{tabular}
\caption{\label{tab:mmlu_abalation} Accuracy on LM evaluation harness tasks on GPT3-1.3B model.}
\end{table}

\begin{table} \centering
\begin{tabular}{|c||c|c|c|c||c|c|c|c|} 
\hline
 $L_b \rightarrow$& \multicolumn{4}{c||}{8} & \multicolumn{4}{c||}{8}\\
 \hline
 \backslashbox{$L_A$\kern-1em}{\kern-1em$N_c$} & 2 & 4 & 8 & 16 & 2 & 4 & 8 & 16  \\
 %$N_c \rightarrow$ & 2 & 4 & 8 & 16 & 2 & 4 & 2 \\
 \hline
 \hline
 \multicolumn{5}{|c|}{Race (FP32 Accuracy = 41.34\%)} & \multicolumn{4}{|c|}{Boolq (FP32 Accuracy = 68.32\%)} \\ 
 \hline
 \hline
 64 & 40.48 & 40.10 & 39.43 & 39.90 & 69.20 & 68.41 & 69.45 & 68.56 \\
 \hline
 32 & 39.52 & 39.52 & 40.77 & 39.62 & 68.32 & 67.43 & 68.17 & 69.30  \\
 \hline
 16 & 39.81 & 39.71 & 39.90 & 40.38 & 68.10 & 66.33 & 69.51 & 69.42  \\
 \hline
 \hline
 \multicolumn{5}{|c|}{Winogrande (FP32 Accuracy = 67.88\%)} & \multicolumn{4}{|c|}{Piqa (FP32 Accuracy = 78.78\%)} \\ 
 \hline
 \hline
 64 & 66.85 & 66.61 & 67.72 & 67.88 & 77.31 & 77.42 & 77.75 & 77.64 \\
 \hline
 32 & 67.25 & 67.72 & 67.72 & 67.00 & 77.31 & 77.04 & 77.80 & 77.37  \\
 \hline
 16 & 68.11 & 68.90 & 67.88 & 67.48 & 77.37 & 78.13 & 78.13 & 77.69  \\
 \hline
\end{tabular}
\caption{\label{tab:mmlu_abalation} Accuracy on LM evaluation harness tasks on GPT3-8B model.}
\end{table}

\begin{table} \centering
\begin{tabular}{|c||c|c|c|c||c|c|c|c|} 
\hline
 $L_b \rightarrow$& \multicolumn{4}{c||}{8} & \multicolumn{4}{c||}{8}\\
 \hline
 \backslashbox{$L_A$\kern-1em}{\kern-1em$N_c$} & 2 & 4 & 8 & 16 & 2 & 4 & 8 & 16  \\
 %$N_c \rightarrow$ & 2 & 4 & 8 & 16 & 2 & 4 & 2 \\
 \hline
 \hline
 \multicolumn{5}{|c|}{Race (FP32 Accuracy = 40.67\%)} & \multicolumn{4}{|c|}{Boolq (FP32 Accuracy = 76.54\%)} \\ 
 \hline
 \hline
 64 & 40.48 & 40.10 & 39.43 & 39.90 & 75.41 & 75.11 & 77.09 & 75.66 \\
 \hline
 32 & 39.52 & 39.52 & 40.77 & 39.62 & 76.02 & 76.02 & 75.96 & 75.35  \\
 \hline
 16 & 39.81 & 39.71 & 39.90 & 40.38 & 75.05 & 73.82 & 75.72 & 76.09  \\
 \hline
 \hline
 \multicolumn{5}{|c|}{Winogrande (FP32 Accuracy = 70.64\%)} & \multicolumn{4}{|c|}{Piqa (FP32 Accuracy = 79.16\%)} \\ 
 \hline
 \hline
 64 & 69.14 & 70.17 & 70.17 & 70.56 & 78.24 & 79.00 & 78.62 & 78.73 \\
 \hline
 32 & 70.96 & 69.69 & 71.27 & 69.30 & 78.56 & 79.49 & 79.16 & 78.89  \\
 \hline
 16 & 71.03 & 69.53 & 69.69 & 70.40 & 78.13 & 79.16 & 79.00 & 79.00  \\
 \hline
\end{tabular}
\caption{\label{tab:mmlu_abalation} Accuracy on LM evaluation harness tasks on GPT3-22B model.}
\end{table}

\begin{table} \centering
\begin{tabular}{|c||c|c|c|c||c|c|c|c|} 
\hline
 $L_b \rightarrow$& \multicolumn{4}{c||}{8} & \multicolumn{4}{c||}{8}\\
 \hline
 \backslashbox{$L_A$\kern-1em}{\kern-1em$N_c$} & 2 & 4 & 8 & 16 & 2 & 4 & 8 & 16  \\
 %$N_c \rightarrow$ & 2 & 4 & 8 & 16 & 2 & 4 & 2 \\
 \hline
 \hline
 \multicolumn{5}{|c|}{Race (FP32 Accuracy = 44.4\%)} & \multicolumn{4}{|c|}{Boolq (FP32 Accuracy = 79.29\%)} \\ 
 \hline
 \hline
 64 & 42.49 & 42.51 & 42.58 & 43.45 & 77.58 & 77.37 & 77.43 & 78.1 \\
 \hline
 32 & 43.35 & 42.49 & 43.64 & 43.73 & 77.86 & 75.32 & 77.28 & 77.86  \\
 \hline
 16 & 44.21 & 44.21 & 43.64 & 42.97 & 78.65 & 77 & 76.94 & 77.98  \\
 \hline
 \hline
 \multicolumn{5}{|c|}{Winogrande (FP32 Accuracy = 69.38\%)} & \multicolumn{4}{|c|}{Piqa (FP32 Accuracy = 78.07\%)} \\ 
 \hline
 \hline
 64 & 68.9 & 68.43 & 69.77 & 68.19 & 77.09 & 76.82 & 77.09 & 77.86 \\
 \hline
 32 & 69.38 & 68.51 & 68.82 & 68.90 & 78.07 & 76.71 & 78.07 & 77.86  \\
 \hline
 16 & 69.53 & 67.09 & 69.38 & 68.90 & 77.37 & 77.8 & 77.91 & 77.69  \\
 \hline
\end{tabular}
\caption{\label{tab:mmlu_abalation} Accuracy on LM evaluation harness tasks on Llama2-7B model.}
\end{table}

\begin{table} \centering
\begin{tabular}{|c||c|c|c|c||c|c|c|c|} 
\hline
 $L_b \rightarrow$& \multicolumn{4}{c||}{8} & \multicolumn{4}{c||}{8}\\
 \hline
 \backslashbox{$L_A$\kern-1em}{\kern-1em$N_c$} & 2 & 4 & 8 & 16 & 2 & 4 & 8 & 16  \\
 %$N_c \rightarrow$ & 2 & 4 & 8 & 16 & 2 & 4 & 2 \\
 \hline
 \hline
 \multicolumn{5}{|c|}{Race (FP32 Accuracy = 48.8\%)} & \multicolumn{4}{|c|}{Boolq (FP32 Accuracy = 85.23\%)} \\ 
 \hline
 \hline
 64 & 49.00 & 49.00 & 49.28 & 48.71 & 82.82 & 84.28 & 84.03 & 84.25 \\
 \hline
 32 & 49.57 & 48.52 & 48.33 & 49.28 & 83.85 & 84.46 & 84.31 & 84.93  \\
 \hline
 16 & 49.85 & 49.09 & 49.28 & 48.99 & 85.11 & 84.46 & 84.61 & 83.94  \\
 \hline
 \hline
 \multicolumn{5}{|c|}{Winogrande (FP32 Accuracy = 79.95\%)} & \multicolumn{4}{|c|}{Piqa (FP32 Accuracy = 81.56\%)} \\ 
 \hline
 \hline
 64 & 78.77 & 78.45 & 78.37 & 79.16 & 81.45 & 80.69 & 81.45 & 81.5 \\
 \hline
 32 & 78.45 & 79.01 & 78.69 & 80.66 & 81.56 & 80.58 & 81.18 & 81.34  \\
 \hline
 16 & 79.95 & 79.56 & 79.79 & 79.72 & 81.28 & 81.66 & 81.28 & 80.96  \\
 \hline
\end{tabular}
\caption{\label{tab:mmlu_abalation} Accuracy on LM evaluation harness tasks on Llama2-70B model.}
\end{table}

%\section{MSE Studies}
%\textcolor{red}{TODO}


\subsection{Number Formats and Quantization Method}
\label{subsec:numFormats_quantMethod}
\subsubsection{Integer Format}
An $n$-bit signed integer (INT) is typically represented with a 2s-complement format \citep{yao2022zeroquant,xiao2023smoothquant,dai2021vsq}, where the most significant bit denotes the sign.

\subsubsection{Floating Point Format}
An $n$-bit signed floating point (FP) number $x$ comprises of a 1-bit sign ($x_{\mathrm{sign}}$), $B_m$-bit mantissa ($x_{\mathrm{mant}}$) and $B_e$-bit exponent ($x_{\mathrm{exp}}$) such that $B_m+B_e=n-1$. The associated constant exponent bias ($E_{\mathrm{bias}}$) is computed as $(2^{{B_e}-1}-1)$. We denote this format as $E_{B_e}M_{B_m}$.  

\subsubsection{Quantization Scheme}
\label{subsec:quant_method}
A quantization scheme dictates how a given unquantized tensor is converted to its quantized representation. We consider FP formats for the purpose of illustration. Given an unquantized tensor $\bm{X}$ and an FP format $E_{B_e}M_{B_m}$, we first, we compute the quantization scale factor $s_X$ that maps the maximum absolute value of $\bm{X}$ to the maximum quantization level of the $E_{B_e}M_{B_m}$ format as follows:
\begin{align}
\label{eq:sf}
    s_X = \frac{\mathrm{max}(|\bm{X}|)}{\mathrm{max}(E_{B_e}M_{B_m})}
\end{align}
In the above equation, $|\cdot|$ denotes the absolute value function.

Next, we scale $\bm{X}$ by $s_X$ and quantize it to $\hat{\bm{X}}$ by rounding it to the nearest quantization level of $E_{B_e}M_{B_m}$ as:

\begin{align}
\label{eq:tensor_quant}
    \hat{\bm{X}} = \text{round-to-nearest}\left(\frac{\bm{X}}{s_X}, E_{B_e}M_{B_m}\right)
\end{align}

We perform dynamic max-scaled quantization \citep{wu2020integer}, where the scale factor $s$ for activations is dynamically computed during runtime.

\subsection{Vector Scaled Quantization}
\begin{wrapfigure}{r}{0.35\linewidth}
  \centering
  \includegraphics[width=\linewidth]{sections/figures/vsquant.jpg}
  \caption{\small Vectorwise decomposition for per-vector scaled quantization (VSQ \citep{dai2021vsq}).}
  \label{fig:vsquant}
\end{wrapfigure}
During VSQ \citep{dai2021vsq}, the operand tensors are decomposed into 1D vectors in a hardware friendly manner as shown in Figure \ref{fig:vsquant}. Since the decomposed tensors are used as operands in matrix multiplications during inference, it is beneficial to perform this decomposition along the reduction dimension of the multiplication. The vectorwise quantization is performed similar to tensorwise quantization described in Equations \ref{eq:sf} and \ref{eq:tensor_quant}, where a scale factor $s_v$ is required for each vector $\bm{v}$ that maps the maximum absolute value of that vector to the maximum quantization level. While smaller vector lengths can lead to larger accuracy gains, the associated memory and computational overheads due to the per-vector scale factors increases. To alleviate these overheads, VSQ \citep{dai2021vsq} proposed a second level quantization of the per-vector scale factors to unsigned integers, while MX \citep{rouhani2023shared} quantizes them to integer powers of 2 (denoted as $2^{INT}$).

\subsubsection{MX Format}
The MX format proposed in \citep{rouhani2023microscaling} introduces the concept of sub-block shifting. For every two scalar elements of $b$-bits each, there is a shared exponent bit. The value of this exponent bit is determined through an empirical analysis that targets minimizing quantization MSE. We note that the FP format $E_{1}M_{b}$ is strictly better than MX from an accuracy perspective since it allocates a dedicated exponent bit to each scalar as opposed to sharing it across two scalars. Therefore, we conservatively bound the accuracy of a $b+2$-bit signed MX format with that of a $E_{1}M_{b}$ format in our comparisons. For instance, we use E1M2 format as a proxy for MX4.

\begin{figure}
    \centering
    \includegraphics[width=1\linewidth]{sections//figures/BlockFormats.pdf}
    \caption{\small Comparing LO-BCQ to MX format.}
    \label{fig:block_formats}
\end{figure}

Figure \ref{fig:block_formats} compares our $4$-bit LO-BCQ block format to MX \citep{rouhani2023microscaling}. As shown, both LO-BCQ and MX decompose a given operand tensor into block arrays and each block array into blocks. Similar to MX, we find that per-block quantization ($L_b < L_A$) leads to better accuracy due to increased flexibility. While MX achieves this through per-block $1$-bit micro-scales, we associate a dedicated codebook to each block through a per-block codebook selector. Further, MX quantizes the per-block array scale-factor to E8M0 format without per-tensor scaling. In contrast during LO-BCQ, we find that per-tensor scaling combined with quantization of per-block array scale-factor to E4M3 format results in superior inference accuracy across models. 


\newpage
\appendix
\onecolumn

% Add the appendix heading (unnumbered)
%\section{Appendix}
%\addcontentsline{toc}{section}{Appendix} % Add "Appendix" to the main TOC

% Generate the TOC for the appendix only
\begin{center}
    \textbf{Appendix Table of Contents}
\end{center}
{
    \setcounter{tocdepth}{2} % Include sections and subsections
    \startcontents % Start a new TOC scope
    \printcontents{}{1}{}
}

% todo: limitation
% todo: Ethical Statement

% Include the appendix content from the external file
\subsection{Lloyd-Max Algorithm}
\label{subsec:Lloyd-Max}
For a given quantization bitwidth $B$ and an operand $\bm{X}$, the Lloyd-Max algorithm finds $2^B$ quantization levels $\{\hat{x}_i\}_{i=1}^{2^B}$ such that quantizing $\bm{X}$ by rounding each scalar in $\bm{X}$ to the nearest quantization level minimizes the quantization MSE. 

The algorithm starts with an initial guess of quantization levels and then iteratively computes quantization thresholds $\{\tau_i\}_{i=1}^{2^B-1}$ and updates quantization levels $\{\hat{x}_i\}_{i=1}^{2^B}$. Specifically, at iteration $n$, thresholds are set to the midpoints of the previous iteration's levels:
\begin{align*}
    \tau_i^{(n)}=\frac{\hat{x}_i^{(n-1)}+\hat{x}_{i+1}^{(n-1)}}2 \text{ for } i=1\ldots 2^B-1
\end{align*}
Subsequently, the quantization levels are re-computed as conditional means of the data regions defined by the new thresholds:
\begin{align*}
    \hat{x}_i^{(n)}=\mathbb{E}\left[ \bm{X} \big| \bm{X}\in [\tau_{i-1}^{(n)},\tau_i^{(n)}] \right] \text{ for } i=1\ldots 2^B
\end{align*}
where to satisfy boundary conditions we have $\tau_0=-\infty$ and $\tau_{2^B}=\infty$. The algorithm iterates the above steps until convergence.

Figure \ref{fig:lm_quant} compares the quantization levels of a $7$-bit floating point (E3M3) quantizer (left) to a $7$-bit Lloyd-Max quantizer (right) when quantizing a layer of weights from the GPT3-126M model at a per-tensor granularity. As shown, the Lloyd-Max quantizer achieves substantially lower quantization MSE. Further, Table \ref{tab:FP7_vs_LM7} shows the superior perplexity achieved by Lloyd-Max quantizers for bitwidths of $7$, $6$ and $5$. The difference between the quantizers is clear at 5 bits, where per-tensor FP quantization incurs a drastic and unacceptable increase in perplexity, while Lloyd-Max quantization incurs a much smaller increase. Nevertheless, we note that even the optimal Lloyd-Max quantizer incurs a notable ($\sim 1.5$) increase in perplexity due to the coarse granularity of quantization. 

\begin{figure}[h]
  \centering
  \includegraphics[width=0.7\linewidth]{sections/figures/LM7_FP7.pdf}
  \caption{\small Quantization levels and the corresponding quantization MSE of Floating Point (left) vs Lloyd-Max (right) Quantizers for a layer of weights in the GPT3-126M model.}
  \label{fig:lm_quant}
\end{figure}

\begin{table}[h]\scriptsize
\begin{center}
\caption{\label{tab:FP7_vs_LM7} \small Comparing perplexity (lower is better) achieved by floating point quantizers and Lloyd-Max quantizers on a GPT3-126M model for the Wikitext-103 dataset.}
\begin{tabular}{c|cc|c}
\hline
 \multirow{2}{*}{\textbf{Bitwidth}} & \multicolumn{2}{|c|}{\textbf{Floating-Point Quantizer}} & \textbf{Lloyd-Max Quantizer} \\
 & Best Format & Wikitext-103 Perplexity & Wikitext-103 Perplexity \\
\hline
7 & E3M3 & 18.32 & 18.27 \\
6 & E3M2 & 19.07 & 18.51 \\
5 & E4M0 & 43.89 & 19.71 \\
\hline
\end{tabular}
\end{center}
\end{table}

\subsection{Proof of Local Optimality of LO-BCQ}
\label{subsec:lobcq_opt_proof}
For a given block $\bm{b}_j$, the quantization MSE during LO-BCQ can be empirically evaluated as $\frac{1}{L_b}\lVert \bm{b}_j- \bm{\hat{b}}_j\rVert^2_2$ where $\bm{\hat{b}}_j$ is computed from equation (\ref{eq:clustered_quantization_definition}) as $C_{f(\bm{b}_j)}(\bm{b}_j)$. Further, for a given block cluster $\mathcal{B}_i$, we compute the quantization MSE as $\frac{1}{|\mathcal{B}_{i}|}\sum_{\bm{b} \in \mathcal{B}_{i}} \frac{1}{L_b}\lVert \bm{b}- C_i^{(n)}(\bm{b})\rVert^2_2$. Therefore, at the end of iteration $n$, we evaluate the overall quantization MSE $J^{(n)}$ for a given operand $\bm{X}$ composed of $N_c$ block clusters as:
\begin{align*}
    \label{eq:mse_iter_n}
    J^{(n)} = \frac{1}{N_c} \sum_{i=1}^{N_c} \frac{1}{|\mathcal{B}_{i}^{(n)}|}\sum_{\bm{v} \in \mathcal{B}_{i}^{(n)}} \frac{1}{L_b}\lVert \bm{b}- B_i^{(n)}(\bm{b})\rVert^2_2
\end{align*}

At the end of iteration $n$, the codebooks are updated from $\mathcal{C}^{(n-1)}$ to $\mathcal{C}^{(n)}$. However, the mapping of a given vector $\bm{b}_j$ to quantizers $\mathcal{C}^{(n)}$ remains as  $f^{(n)}(\bm{b}_j)$. At the next iteration, during the vector clustering step, $f^{(n+1)}(\bm{b}_j)$ finds new mapping of $\bm{b}_j$ to updated codebooks $\mathcal{C}^{(n)}$ such that the quantization MSE over the candidate codebooks is minimized. Therefore, we obtain the following result for $\bm{b}_j$:
\begin{align*}
\frac{1}{L_b}\lVert \bm{b}_j - C_{f^{(n+1)}(\bm{b}_j)}^{(n)}(\bm{b}_j)\rVert^2_2 \le \frac{1}{L_b}\lVert \bm{b}_j - C_{f^{(n)}(\bm{b}_j)}^{(n)}(\bm{b}_j)\rVert^2_2
\end{align*}

That is, quantizing $\bm{b}_j$ at the end of the block clustering step of iteration $n+1$ results in lower quantization MSE compared to quantizing at the end of iteration $n$. Since this is true for all $\bm{b} \in \bm{X}$, we assert the following:
\begin{equation}
\begin{split}
\label{eq:mse_ineq_1}
    \tilde{J}^{(n+1)} &= \frac{1}{N_c} \sum_{i=1}^{N_c} \frac{1}{|\mathcal{B}_{i}^{(n+1)}|}\sum_{\bm{b} \in \mathcal{B}_{i}^{(n+1)}} \frac{1}{L_b}\lVert \bm{b} - C_i^{(n)}(b)\rVert^2_2 \le J^{(n)}
\end{split}
\end{equation}
where $\tilde{J}^{(n+1)}$ is the the quantization MSE after the vector clustering step at iteration $n+1$.

Next, during the codebook update step (\ref{eq:quantizers_update}) at iteration $n+1$, the per-cluster codebooks $\mathcal{C}^{(n)}$ are updated to $\mathcal{C}^{(n+1)}$ by invoking the Lloyd-Max algorithm \citep{Lloyd}. We know that for any given value distribution, the Lloyd-Max algorithm minimizes the quantization MSE. Therefore, for a given vector cluster $\mathcal{B}_i$ we obtain the following result:

\begin{equation}
    \frac{1}{|\mathcal{B}_{i}^{(n+1)}|}\sum_{\bm{b} \in \mathcal{B}_{i}^{(n+1)}} \frac{1}{L_b}\lVert \bm{b}- C_i^{(n+1)}(\bm{b})\rVert^2_2 \le \frac{1}{|\mathcal{B}_{i}^{(n+1)}|}\sum_{\bm{b} \in \mathcal{B}_{i}^{(n+1)}} \frac{1}{L_b}\lVert \bm{b}- C_i^{(n)}(\bm{b})\rVert^2_2
\end{equation}

The above equation states that quantizing the given block cluster $\mathcal{B}_i$ after updating the associated codebook from $C_i^{(n)}$ to $C_i^{(n+1)}$ results in lower quantization MSE. Since this is true for all the block clusters, we derive the following result: 
\begin{equation}
\begin{split}
\label{eq:mse_ineq_2}
     J^{(n+1)} &= \frac{1}{N_c} \sum_{i=1}^{N_c} \frac{1}{|\mathcal{B}_{i}^{(n+1)}|}\sum_{\bm{b} \in \mathcal{B}_{i}^{(n+1)}} \frac{1}{L_b}\lVert \bm{b}- C_i^{(n+1)}(\bm{b})\rVert^2_2  \le \tilde{J}^{(n+1)}   
\end{split}
\end{equation}

Following (\ref{eq:mse_ineq_1}) and (\ref{eq:mse_ineq_2}), we find that the quantization MSE is non-increasing for each iteration, that is, $J^{(1)} \ge J^{(2)} \ge J^{(3)} \ge \ldots \ge J^{(M)}$ where $M$ is the maximum number of iterations. 
%Therefore, we can say that if the algorithm converges, then it must be that it has converged to a local minimum. 
\hfill $\blacksquare$


\begin{figure}
    \begin{center}
    \includegraphics[width=0.5\textwidth]{sections//figures/mse_vs_iter.pdf}
    \end{center}
    \caption{\small NMSE vs iterations during LO-BCQ compared to other block quantization proposals}
    \label{fig:nmse_vs_iter}
\end{figure}

Figure \ref{fig:nmse_vs_iter} shows the empirical convergence of LO-BCQ across several block lengths and number of codebooks. Also, the MSE achieved by LO-BCQ is compared to baselines such as MXFP and VSQ. As shown, LO-BCQ converges to a lower MSE than the baselines. Further, we achieve better convergence for larger number of codebooks ($N_c$) and for a smaller block length ($L_b$), both of which increase the bitwidth of BCQ (see Eq \ref{eq:bitwidth_bcq}).


\subsection{Additional Accuracy Results}
%Table \ref{tab:lobcq_config} lists the various LOBCQ configurations and their corresponding bitwidths.
\begin{table}
\setlength{\tabcolsep}{4.75pt}
\begin{center}
\caption{\label{tab:lobcq_config} Various LO-BCQ configurations and their bitwidths.}
\begin{tabular}{|c||c|c|c|c||c|c||c|} 
\hline
 & \multicolumn{4}{|c||}{$L_b=8$} & \multicolumn{2}{|c||}{$L_b=4$} & $L_b=2$ \\
 \hline
 \backslashbox{$L_A$\kern-1em}{\kern-1em$N_c$} & 2 & 4 & 8 & 16 & 2 & 4 & 2 \\
 \hline
 64 & 4.25 & 4.375 & 4.5 & 4.625 & 4.375 & 4.625 & 4.625\\
 \hline
 32 & 4.375 & 4.5 & 4.625& 4.75 & 4.5 & 4.75 & 4.75 \\
 \hline
 16 & 4.625 & 4.75& 4.875 & 5 & 4.75 & 5 & 5 \\
 \hline
\end{tabular}
\end{center}
\end{table}

%\subsection{Perplexity achieved by various LO-BCQ configurations on Wikitext-103 dataset}

\begin{table} \centering
\begin{tabular}{|c||c|c|c|c||c|c||c|} 
\hline
 $L_b \rightarrow$& \multicolumn{4}{c||}{8} & \multicolumn{2}{c||}{4} & 2\\
 \hline
 \backslashbox{$L_A$\kern-1em}{\kern-1em$N_c$} & 2 & 4 & 8 & 16 & 2 & 4 & 2  \\
 %$N_c \rightarrow$ & 2 & 4 & 8 & 16 & 2 & 4 & 2 \\
 \hline
 \hline
 \multicolumn{8}{c}{GPT3-1.3B (FP32 PPL = 9.98)} \\ 
 \hline
 \hline
 64 & 10.40 & 10.23 & 10.17 & 10.15 &  10.28 & 10.18 & 10.19 \\
 \hline
 32 & 10.25 & 10.20 & 10.15 & 10.12 &  10.23 & 10.17 & 10.17 \\
 \hline
 16 & 10.22 & 10.16 & 10.10 & 10.09 &  10.21 & 10.14 & 10.16 \\
 \hline
  \hline
 \multicolumn{8}{c}{GPT3-8B (FP32 PPL = 7.38)} \\ 
 \hline
 \hline
 64 & 7.61 & 7.52 & 7.48 &  7.47 &  7.55 &  7.49 & 7.50 \\
 \hline
 32 & 7.52 & 7.50 & 7.46 &  7.45 &  7.52 &  7.48 & 7.48  \\
 \hline
 16 & 7.51 & 7.48 & 7.44 &  7.44 &  7.51 &  7.49 & 7.47  \\
 \hline
\end{tabular}
\caption{\label{tab:ppl_gpt3_abalation} Wikitext-103 perplexity across GPT3-1.3B and 8B models.}
\end{table}

\begin{table} \centering
\begin{tabular}{|c||c|c|c|c||} 
\hline
 $L_b \rightarrow$& \multicolumn{4}{c||}{8}\\
 \hline
 \backslashbox{$L_A$\kern-1em}{\kern-1em$N_c$} & 2 & 4 & 8 & 16 \\
 %$N_c \rightarrow$ & 2 & 4 & 8 & 16 & 2 & 4 & 2 \\
 \hline
 \hline
 \multicolumn{5}{|c|}{Llama2-7B (FP32 PPL = 5.06)} \\ 
 \hline
 \hline
 64 & 5.31 & 5.26 & 5.19 & 5.18  \\
 \hline
 32 & 5.23 & 5.25 & 5.18 & 5.15  \\
 \hline
 16 & 5.23 & 5.19 & 5.16 & 5.14  \\
 \hline
 \multicolumn{5}{|c|}{Nemotron4-15B (FP32 PPL = 5.87)} \\ 
 \hline
 \hline
 64  & 6.3 & 6.20 & 6.13 & 6.08  \\
 \hline
 32  & 6.24 & 6.12 & 6.07 & 6.03  \\
 \hline
 16  & 6.12 & 6.14 & 6.04 & 6.02  \\
 \hline
 \multicolumn{5}{|c|}{Nemotron4-340B (FP32 PPL = 3.48)} \\ 
 \hline
 \hline
 64 & 3.67 & 3.62 & 3.60 & 3.59 \\
 \hline
 32 & 3.63 & 3.61 & 3.59 & 3.56 \\
 \hline
 16 & 3.61 & 3.58 & 3.57 & 3.55 \\
 \hline
\end{tabular}
\caption{\label{tab:ppl_llama7B_nemo15B} Wikitext-103 perplexity compared to FP32 baseline in Llama2-7B and Nemotron4-15B, 340B models}
\end{table}

%\subsection{Perplexity achieved by various LO-BCQ configurations on MMLU dataset}


\begin{table} \centering
\begin{tabular}{|c||c|c|c|c||c|c|c|c|} 
\hline
 $L_b \rightarrow$& \multicolumn{4}{c||}{8} & \multicolumn{4}{c||}{8}\\
 \hline
 \backslashbox{$L_A$\kern-1em}{\kern-1em$N_c$} & 2 & 4 & 8 & 16 & 2 & 4 & 8 & 16  \\
 %$N_c \rightarrow$ & 2 & 4 & 8 & 16 & 2 & 4 & 2 \\
 \hline
 \hline
 \multicolumn{5}{|c|}{Llama2-7B (FP32 Accuracy = 45.8\%)} & \multicolumn{4}{|c|}{Llama2-70B (FP32 Accuracy = 69.12\%)} \\ 
 \hline
 \hline
 64 & 43.9 & 43.4 & 43.9 & 44.9 & 68.07 & 68.27 & 68.17 & 68.75 \\
 \hline
 32 & 44.5 & 43.8 & 44.9 & 44.5 & 68.37 & 68.51 & 68.35 & 68.27  \\
 \hline
 16 & 43.9 & 42.7 & 44.9 & 45 & 68.12 & 68.77 & 68.31 & 68.59  \\
 \hline
 \hline
 \multicolumn{5}{|c|}{GPT3-22B (FP32 Accuracy = 38.75\%)} & \multicolumn{4}{|c|}{Nemotron4-15B (FP32 Accuracy = 64.3\%)} \\ 
 \hline
 \hline
 64 & 36.71 & 38.85 & 38.13 & 38.92 & 63.17 & 62.36 & 63.72 & 64.09 \\
 \hline
 32 & 37.95 & 38.69 & 39.45 & 38.34 & 64.05 & 62.30 & 63.8 & 64.33  \\
 \hline
 16 & 38.88 & 38.80 & 38.31 & 38.92 & 63.22 & 63.51 & 63.93 & 64.43  \\
 \hline
\end{tabular}
\caption{\label{tab:mmlu_abalation} Accuracy on MMLU dataset across GPT3-22B, Llama2-7B, 70B and Nemotron4-15B models.}
\end{table}


%\subsection{Perplexity achieved by various LO-BCQ configurations on LM evaluation harness}

\begin{table} \centering
\begin{tabular}{|c||c|c|c|c||c|c|c|c|} 
\hline
 $L_b \rightarrow$& \multicolumn{4}{c||}{8} & \multicolumn{4}{c||}{8}\\
 \hline
 \backslashbox{$L_A$\kern-1em}{\kern-1em$N_c$} & 2 & 4 & 8 & 16 & 2 & 4 & 8 & 16  \\
 %$N_c \rightarrow$ & 2 & 4 & 8 & 16 & 2 & 4 & 2 \\
 \hline
 \hline
 \multicolumn{5}{|c|}{Race (FP32 Accuracy = 37.51\%)} & \multicolumn{4}{|c|}{Boolq (FP32 Accuracy = 64.62\%)} \\ 
 \hline
 \hline
 64 & 36.94 & 37.13 & 36.27 & 37.13 & 63.73 & 62.26 & 63.49 & 63.36 \\
 \hline
 32 & 37.03 & 36.36 & 36.08 & 37.03 & 62.54 & 63.51 & 63.49 & 63.55  \\
 \hline
 16 & 37.03 & 37.03 & 36.46 & 37.03 & 61.1 & 63.79 & 63.58 & 63.33  \\
 \hline
 \hline
 \multicolumn{5}{|c|}{Winogrande (FP32 Accuracy = 58.01\%)} & \multicolumn{4}{|c|}{Piqa (FP32 Accuracy = 74.21\%)} \\ 
 \hline
 \hline
 64 & 58.17 & 57.22 & 57.85 & 58.33 & 73.01 & 73.07 & 73.07 & 72.80 \\
 \hline
 32 & 59.12 & 58.09 & 57.85 & 58.41 & 73.01 & 73.94 & 72.74 & 73.18  \\
 \hline
 16 & 57.93 & 58.88 & 57.93 & 58.56 & 73.94 & 72.80 & 73.01 & 73.94  \\
 \hline
\end{tabular}
\caption{\label{tab:mmlu_abalation} Accuracy on LM evaluation harness tasks on GPT3-1.3B model.}
\end{table}

\begin{table} \centering
\begin{tabular}{|c||c|c|c|c||c|c|c|c|} 
\hline
 $L_b \rightarrow$& \multicolumn{4}{c||}{8} & \multicolumn{4}{c||}{8}\\
 \hline
 \backslashbox{$L_A$\kern-1em}{\kern-1em$N_c$} & 2 & 4 & 8 & 16 & 2 & 4 & 8 & 16  \\
 %$N_c \rightarrow$ & 2 & 4 & 8 & 16 & 2 & 4 & 2 \\
 \hline
 \hline
 \multicolumn{5}{|c|}{Race (FP32 Accuracy = 41.34\%)} & \multicolumn{4}{|c|}{Boolq (FP32 Accuracy = 68.32\%)} \\ 
 \hline
 \hline
 64 & 40.48 & 40.10 & 39.43 & 39.90 & 69.20 & 68.41 & 69.45 & 68.56 \\
 \hline
 32 & 39.52 & 39.52 & 40.77 & 39.62 & 68.32 & 67.43 & 68.17 & 69.30  \\
 \hline
 16 & 39.81 & 39.71 & 39.90 & 40.38 & 68.10 & 66.33 & 69.51 & 69.42  \\
 \hline
 \hline
 \multicolumn{5}{|c|}{Winogrande (FP32 Accuracy = 67.88\%)} & \multicolumn{4}{|c|}{Piqa (FP32 Accuracy = 78.78\%)} \\ 
 \hline
 \hline
 64 & 66.85 & 66.61 & 67.72 & 67.88 & 77.31 & 77.42 & 77.75 & 77.64 \\
 \hline
 32 & 67.25 & 67.72 & 67.72 & 67.00 & 77.31 & 77.04 & 77.80 & 77.37  \\
 \hline
 16 & 68.11 & 68.90 & 67.88 & 67.48 & 77.37 & 78.13 & 78.13 & 77.69  \\
 \hline
\end{tabular}
\caption{\label{tab:mmlu_abalation} Accuracy on LM evaluation harness tasks on GPT3-8B model.}
\end{table}

\begin{table} \centering
\begin{tabular}{|c||c|c|c|c||c|c|c|c|} 
\hline
 $L_b \rightarrow$& \multicolumn{4}{c||}{8} & \multicolumn{4}{c||}{8}\\
 \hline
 \backslashbox{$L_A$\kern-1em}{\kern-1em$N_c$} & 2 & 4 & 8 & 16 & 2 & 4 & 8 & 16  \\
 %$N_c \rightarrow$ & 2 & 4 & 8 & 16 & 2 & 4 & 2 \\
 \hline
 \hline
 \multicolumn{5}{|c|}{Race (FP32 Accuracy = 40.67\%)} & \multicolumn{4}{|c|}{Boolq (FP32 Accuracy = 76.54\%)} \\ 
 \hline
 \hline
 64 & 40.48 & 40.10 & 39.43 & 39.90 & 75.41 & 75.11 & 77.09 & 75.66 \\
 \hline
 32 & 39.52 & 39.52 & 40.77 & 39.62 & 76.02 & 76.02 & 75.96 & 75.35  \\
 \hline
 16 & 39.81 & 39.71 & 39.90 & 40.38 & 75.05 & 73.82 & 75.72 & 76.09  \\
 \hline
 \hline
 \multicolumn{5}{|c|}{Winogrande (FP32 Accuracy = 70.64\%)} & \multicolumn{4}{|c|}{Piqa (FP32 Accuracy = 79.16\%)} \\ 
 \hline
 \hline
 64 & 69.14 & 70.17 & 70.17 & 70.56 & 78.24 & 79.00 & 78.62 & 78.73 \\
 \hline
 32 & 70.96 & 69.69 & 71.27 & 69.30 & 78.56 & 79.49 & 79.16 & 78.89  \\
 \hline
 16 & 71.03 & 69.53 & 69.69 & 70.40 & 78.13 & 79.16 & 79.00 & 79.00  \\
 \hline
\end{tabular}
\caption{\label{tab:mmlu_abalation} Accuracy on LM evaluation harness tasks on GPT3-22B model.}
\end{table}

\begin{table} \centering
\begin{tabular}{|c||c|c|c|c||c|c|c|c|} 
\hline
 $L_b \rightarrow$& \multicolumn{4}{c||}{8} & \multicolumn{4}{c||}{8}\\
 \hline
 \backslashbox{$L_A$\kern-1em}{\kern-1em$N_c$} & 2 & 4 & 8 & 16 & 2 & 4 & 8 & 16  \\
 %$N_c \rightarrow$ & 2 & 4 & 8 & 16 & 2 & 4 & 2 \\
 \hline
 \hline
 \multicolumn{5}{|c|}{Race (FP32 Accuracy = 44.4\%)} & \multicolumn{4}{|c|}{Boolq (FP32 Accuracy = 79.29\%)} \\ 
 \hline
 \hline
 64 & 42.49 & 42.51 & 42.58 & 43.45 & 77.58 & 77.37 & 77.43 & 78.1 \\
 \hline
 32 & 43.35 & 42.49 & 43.64 & 43.73 & 77.86 & 75.32 & 77.28 & 77.86  \\
 \hline
 16 & 44.21 & 44.21 & 43.64 & 42.97 & 78.65 & 77 & 76.94 & 77.98  \\
 \hline
 \hline
 \multicolumn{5}{|c|}{Winogrande (FP32 Accuracy = 69.38\%)} & \multicolumn{4}{|c|}{Piqa (FP32 Accuracy = 78.07\%)} \\ 
 \hline
 \hline
 64 & 68.9 & 68.43 & 69.77 & 68.19 & 77.09 & 76.82 & 77.09 & 77.86 \\
 \hline
 32 & 69.38 & 68.51 & 68.82 & 68.90 & 78.07 & 76.71 & 78.07 & 77.86  \\
 \hline
 16 & 69.53 & 67.09 & 69.38 & 68.90 & 77.37 & 77.8 & 77.91 & 77.69  \\
 \hline
\end{tabular}
\caption{\label{tab:mmlu_abalation} Accuracy on LM evaluation harness tasks on Llama2-7B model.}
\end{table}

\begin{table} \centering
\begin{tabular}{|c||c|c|c|c||c|c|c|c|} 
\hline
 $L_b \rightarrow$& \multicolumn{4}{c||}{8} & \multicolumn{4}{c||}{8}\\
 \hline
 \backslashbox{$L_A$\kern-1em}{\kern-1em$N_c$} & 2 & 4 & 8 & 16 & 2 & 4 & 8 & 16  \\
 %$N_c \rightarrow$ & 2 & 4 & 8 & 16 & 2 & 4 & 2 \\
 \hline
 \hline
 \multicolumn{5}{|c|}{Race (FP32 Accuracy = 48.8\%)} & \multicolumn{4}{|c|}{Boolq (FP32 Accuracy = 85.23\%)} \\ 
 \hline
 \hline
 64 & 49.00 & 49.00 & 49.28 & 48.71 & 82.82 & 84.28 & 84.03 & 84.25 \\
 \hline
 32 & 49.57 & 48.52 & 48.33 & 49.28 & 83.85 & 84.46 & 84.31 & 84.93  \\
 \hline
 16 & 49.85 & 49.09 & 49.28 & 48.99 & 85.11 & 84.46 & 84.61 & 83.94  \\
 \hline
 \hline
 \multicolumn{5}{|c|}{Winogrande (FP32 Accuracy = 79.95\%)} & \multicolumn{4}{|c|}{Piqa (FP32 Accuracy = 81.56\%)} \\ 
 \hline
 \hline
 64 & 78.77 & 78.45 & 78.37 & 79.16 & 81.45 & 80.69 & 81.45 & 81.5 \\
 \hline
 32 & 78.45 & 79.01 & 78.69 & 80.66 & 81.56 & 80.58 & 81.18 & 81.34  \\
 \hline
 16 & 79.95 & 79.56 & 79.79 & 79.72 & 81.28 & 81.66 & 81.28 & 80.96  \\
 \hline
\end{tabular}
\caption{\label{tab:mmlu_abalation} Accuracy on LM evaluation harness tasks on Llama2-70B model.}
\end{table}

%\section{MSE Studies}
%\textcolor{red}{TODO}


\subsection{Number Formats and Quantization Method}
\label{subsec:numFormats_quantMethod}
\subsubsection{Integer Format}
An $n$-bit signed integer (INT) is typically represented with a 2s-complement format \citep{yao2022zeroquant,xiao2023smoothquant,dai2021vsq}, where the most significant bit denotes the sign.

\subsubsection{Floating Point Format}
An $n$-bit signed floating point (FP) number $x$ comprises of a 1-bit sign ($x_{\mathrm{sign}}$), $B_m$-bit mantissa ($x_{\mathrm{mant}}$) and $B_e$-bit exponent ($x_{\mathrm{exp}}$) such that $B_m+B_e=n-1$. The associated constant exponent bias ($E_{\mathrm{bias}}$) is computed as $(2^{{B_e}-1}-1)$. We denote this format as $E_{B_e}M_{B_m}$.  

\subsubsection{Quantization Scheme}
\label{subsec:quant_method}
A quantization scheme dictates how a given unquantized tensor is converted to its quantized representation. We consider FP formats for the purpose of illustration. Given an unquantized tensor $\bm{X}$ and an FP format $E_{B_e}M_{B_m}$, we first, we compute the quantization scale factor $s_X$ that maps the maximum absolute value of $\bm{X}$ to the maximum quantization level of the $E_{B_e}M_{B_m}$ format as follows:
\begin{align}
\label{eq:sf}
    s_X = \frac{\mathrm{max}(|\bm{X}|)}{\mathrm{max}(E_{B_e}M_{B_m})}
\end{align}
In the above equation, $|\cdot|$ denotes the absolute value function.

Next, we scale $\bm{X}$ by $s_X$ and quantize it to $\hat{\bm{X}}$ by rounding it to the nearest quantization level of $E_{B_e}M_{B_m}$ as:

\begin{align}
\label{eq:tensor_quant}
    \hat{\bm{X}} = \text{round-to-nearest}\left(\frac{\bm{X}}{s_X}, E_{B_e}M_{B_m}\right)
\end{align}

We perform dynamic max-scaled quantization \citep{wu2020integer}, where the scale factor $s$ for activations is dynamically computed during runtime.

\subsection{Vector Scaled Quantization}
\begin{wrapfigure}{r}{0.35\linewidth}
  \centering
  \includegraphics[width=\linewidth]{sections/figures/vsquant.jpg}
  \caption{\small Vectorwise decomposition for per-vector scaled quantization (VSQ \citep{dai2021vsq}).}
  \label{fig:vsquant}
\end{wrapfigure}
During VSQ \citep{dai2021vsq}, the operand tensors are decomposed into 1D vectors in a hardware friendly manner as shown in Figure \ref{fig:vsquant}. Since the decomposed tensors are used as operands in matrix multiplications during inference, it is beneficial to perform this decomposition along the reduction dimension of the multiplication. The vectorwise quantization is performed similar to tensorwise quantization described in Equations \ref{eq:sf} and \ref{eq:tensor_quant}, where a scale factor $s_v$ is required for each vector $\bm{v}$ that maps the maximum absolute value of that vector to the maximum quantization level. While smaller vector lengths can lead to larger accuracy gains, the associated memory and computational overheads due to the per-vector scale factors increases. To alleviate these overheads, VSQ \citep{dai2021vsq} proposed a second level quantization of the per-vector scale factors to unsigned integers, while MX \citep{rouhani2023shared} quantizes them to integer powers of 2 (denoted as $2^{INT}$).

\subsubsection{MX Format}
The MX format proposed in \citep{rouhani2023microscaling} introduces the concept of sub-block shifting. For every two scalar elements of $b$-bits each, there is a shared exponent bit. The value of this exponent bit is determined through an empirical analysis that targets minimizing quantization MSE. We note that the FP format $E_{1}M_{b}$ is strictly better than MX from an accuracy perspective since it allocates a dedicated exponent bit to each scalar as opposed to sharing it across two scalars. Therefore, we conservatively bound the accuracy of a $b+2$-bit signed MX format with that of a $E_{1}M_{b}$ format in our comparisons. For instance, we use E1M2 format as a proxy for MX4.

\begin{figure}
    \centering
    \includegraphics[width=1\linewidth]{sections//figures/BlockFormats.pdf}
    \caption{\small Comparing LO-BCQ to MX format.}
    \label{fig:block_formats}
\end{figure}

Figure \ref{fig:block_formats} compares our $4$-bit LO-BCQ block format to MX \citep{rouhani2023microscaling}. As shown, both LO-BCQ and MX decompose a given operand tensor into block arrays and each block array into blocks. Similar to MX, we find that per-block quantization ($L_b < L_A$) leads to better accuracy due to increased flexibility. While MX achieves this through per-block $1$-bit micro-scales, we associate a dedicated codebook to each block through a per-block codebook selector. Further, MX quantizes the per-block array scale-factor to E8M0 format without per-tensor scaling. In contrast during LO-BCQ, we find that per-tensor scaling combined with quantization of per-block array scale-factor to E4M3 format results in superior inference accuracy across models. 


\end{document}
