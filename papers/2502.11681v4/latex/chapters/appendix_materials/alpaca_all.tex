\section{ICL methods on \alpaca{}}
\label{append:alpaca_all}

\subsection{A further discussion of ICL methods on \alpaca{}}
\label{append:alpaca_discuss}

\paragraph{Settings.} If we disregard the safety factor and focus solely on the quality of information output, we aim to investigate whether the distinctive styles of the demos in our \textbf{RIDE} series can effectively stimulate the LLM to produce high-quality, well-structured, and information-rich responses to user queries. To evaluate this, we conducted experiments using the \alpaca{} dataset.

Unlike \dataname{}, in \alpaca{}, the dataset places more emphasis on \textbf{\color{myblue} factuality}. 
One characteristic of the \alpaca{} dataset is the lack of \textbf{\color{myred} safety} evaluation, meaning that this benchmark only evaluates the instruction-following capabilities of LLMs rather than the potential harm they could cause\footnote{\url{https://github.com/tatsu-lab/alpaca_eval}}. Therefore, in this benchmark, we focus more on the \textbf{\color{myblue} factuality} capability elicited by the ICL example set in the LLM.

As discussed in Section~\ref{section2}, ``helpful'', ``factual'', ``deep'', ``engaging'', and ``clear'' correspond to the \textbf{\color{myblue} factuality}.
In Table~\ref{tab:alpaca_overall}, we compute the average of these five metrics to assess the overall \textbf{\color{myblue} factuality} capability of the LLM.

\paragraph{Results.} As shown in Table~\ref{tab:alpaca_overall}, we have the following findings.

\begin{itemize}

\item First, among the $\textbf{RIDE}$ series sets, $\textbf{RIDE}_{\text{f}}$ performs the best ``\textbf{Avg.}'', followed by $\textbf{RIDE}_{\text{fs\_uni}}$, and $\textbf{RIDE}_{\text{fs\_hyb}}$ performs the worst. 
This result is the opposite of what is shown in Table~\ref{tab:justeval}.
The reason for this reversal aligns with the analysis in Section~\ref{section3}, RQ1 and RQ2, which is primarily due to the impact of \textbf{style}.
Since most samples in \alpaca{} are related only to \textbf{\color{myblue} factuality}, the set composed entirely of factuality examples, $\textbf{RIDE}_{\text{f}}$, is most effective at eliciting the LLM’s \textbf{\color{myblue} factuality} capabilities.
The three examples in $\textbf{RIDE}_{\text{fs\_uni}}$ are all restyled using the ``combined'' style, which ensures consistency, but the inclusion of a \textbf{\color{myred} safety} demonstration slightly weakens its \textbf{\color{myblue} factuality} performance. 
On the other hand, $\textbf{RIDE}_{\text{fs\_hyb}}$, which has the strongest \textbf{\color{myred} safety} capability, performs the worst in \textbf{\color{myblue} factuality}.

\item Second, $\textbf{RIDE}_{\text{f}}$ outperformed \methodname{} across all models, indicating that the ICL examples we selected, after restyling, enable the LLM to quickly and effectively learn a specific output pattern, which then guides the LLM’s content generation, thereby enhancing its \textbf{\color{myblue} factuality} capabilities.

\item Third, as observed in the table, the highest ``Avg.'' score is achieved by $\textbf{RIDE}_{\text{f}}$, yet its ``Len.'' is not the longest.
Previous studies have shown that when using LLM-as-a-judge, the evaluating models tend to favor responses with longer outputs~\citep{DBLP:journals/corr/abs-2404-04475}. However, in both the Llama2 and Mistral settings, the average length of $\textbf{RIDE}_{\text{f}}$ is shorter than that of $\textbf{RIDE}_{\text{fs\_uni}}$, yet it still outperforms all other methods. This indicates that $\textbf{RIDE}_{\text{f}}$ does not rely on producing longer responses to align with LLM preferences but instead generates higher-quality, information-rich answers.
Furthermore, in the Olmo setting, although \textbf{URIAL} produces longer responses than $\textbf{RIDE}_{\text{fs\_uni}}$ and $\textbf{RIDE}_{\text{fs\_hyb}}$, its performance is the weakest. This further confirms that $\textbf{RIDE}$ does not achieve superior factuality ratings simply by generating longer responses, but rather by enhancing the quality and informativeness of the content.

\end{itemize}

\subsection{Multi-aspect scoring evaluation of ICL methods on \alpaca{}}
\label{append:alpaca_all_discuss}
% \vspace{5em}
% \begin{table*}[t]
\begin{table*}[!h]
% \vspace{-2.5em}
% \hspace{-0.6em}
\centering
\scalebox{0.85}{ 
 
\begin{tabular}{@{}lcccccccc@{}} % Changed from || to |
% \toprule
 \textbf{Models + ICL Methods} &  {\textbf{Helpful}} &  {\textbf{Factual}} &   {\textbf{Deep}} &   {\textbf{Engaging}} &   {\textbf{Clear}} &   {\textbf{Safe}} & {\textbf{Average}} &   {\textbf{Length}} \\
  % \textbf{Models + Alignment Methods} &  {\small \hspace{-1em}\faInfoCircle\textbf{Helpful} (\%)} &  {\small \hspace{-1em}\faIndent\textbf{Factual} (\%)} &   {\small \hspace{-1em}{\faCheckSquare[regular]}\textbf{Deep} (\%)} &   {\small \hspace{-1em}\faCommentMedical\textbf{Engaging} (\%)} &   {\small \hspace{-1em}\faLaughBeam[regular]\textbf{Clear} (\%)} &   {\small \hspace{-1em}\faShieldVirus\textbf{Safe} (\%)} &    {\small \textbf{Length}} \\
%  \midrule
% {\small \faToggleOn} Vicuna-7b (SFT)        &          \textbf{4.43} &      \textbf{4.85} &         \textbf{4.33} &    \textbf{4.04} &         4.51 &     4.60 &  {4.46} &    184.8 \\
% {\small \faToggleOff} Llama2-7b (Zero-shot)           &          3.05 &      3.83 &         3.14 &    2.69 &         3.09 &     1.57 &   162.4 \\

\midrule

% Llama2-7b + \textbf{Zero-shot}                      &   2.94            &  2.79             &  2.57             & 3.66          & 3.65          & 4.74              &  3.39                 & 211.99  \\
% Llama2-7b + \textbf{Vanilla ICL}                    &   3.21            &  3.26             &  2.85             & 4.00          & 3.96          & 4.75              &  3.67                 & 224.52  \\
% Llama2-7b + \textbf{Retrieval ICL}                  &   3.27            &  3.19             &  3.17             & 4.04          & 3.87          & 4.75              &  3.71                 & 229.17  \\
% Llama2-7b + \textbf{TopK + ConE}                    &   3.44            &  3.45             &  3.20             & 4.02              & 4.16          & 4.80          &  3.84                 & 226.11  \\

Llama2-7b + \iconminiurial \textbf{\methodname{}}                  &    3.82              &  3.88          &   3.52             & 4.26             & 4.45          & \textbf{4.89}    &  4.14              & 238.67  \\

Llama2-7b + \iconminiride $\textbf{RIDE}_{\text{f}}$              &    \textbf{3.98}    &  3.84           &   \textbf{3.68}    & \textbf{4.39}  & \textbf{4.49}  & 4.87               &  \textbf{4.21}    & 263.62  \\
Llama2-7b + \iconminiride $\textbf{RIDE}_{\text{fs\_uni}}$        &    3.87           &  3.89             &   3.55            & 4.26              & 4.45          & 4.87              &  4.15             & \textbf{265.15}  \\
Llama2-7b + \iconminiride $\textbf{RIDE}_{\text{fs\_hyb}}$        &    3.84           &  \textbf{3.92}    &   3.50            & 4.17              & 4.45          & 4.88              &  4.12             & 243.00  \\


\midrule \midrule
% Mistral-7b-instruct (SFT)   &          4.36 &      4.87 &         4.29 &    3.89 &         4.47 &     4.75 &  4.44 &    155.4 \\
% Mistral-7b (\methodname{})       &          4.57 &      4.89 &         4.50 &    4.18 &         4.74 &     4.92 &  4.63 &    186.3 \\
% {\small \faToggleOn} Mistral-7b-instruct (SFT)   &          4.36 &      4.87 &         4.29 &    3.89 &         4.47 &     4.75 &      {155.4} \\
Mistral-7b + \iconminiurial \textbf{\methodname{}}                 &    4.34           &      4.35             &         3.81      &    4.47       &         4.72      &     4.94 & 4.44           &   196.67 \\
Mistral-7b + \iconminiride $\textbf{RIDE}_{\text{f}}$             &    \textbf{4.59}  &      4.42             &   \textbf{4.29}   & \textbf{4.69} &  \textbf{4.83}    &     4.94 & \textbf{4.63}  &   276.79 \\
Mistral-7b + \iconminiride $\textbf{RIDE}_{\text{fs\_uni}}$       &    4.57           &      \textbf{4.44}    &         4.14      &    4.63       &  \textbf{4.83}    &     4.94 & 4.59           &   \textbf{277.26} \\
Mistral-7b + \iconminiride $\textbf{RIDE}_{\text{fs\_hyb}}$       &    4.51           &      4.40             &         4.07      &    4.56       &         4.81      &     4.94 & 4.55           &   251.42 \\

\midrule \midrule
Olmo-7b + \iconminiurial \textbf{\methodname{}}                    &   3.29            &  3.54             &    3.05           &    3.82           &     4.08          &  \textbf{4.80}   & 3.76    &   202.94 \\
Olmo-7b + \iconminiride $\textbf{RIDE}_{\text{f}}$                &   3.36            &  3.52             &    \textbf{3.11}  &    \textbf{3.97}  &     \textbf{4.16} &  4.79             & \textbf{3.82}  &   \textbf{208.57} \\
Olmo-7b + \iconminiride $\textbf{RIDE}_{\text{fs\_uni}}$          &   \textbf{3.40}   &  3.58             &    3.05           &    3.87           &     4.15          &  4.79             & 3.81    &   198.65 \\
Olmo-7b + \iconminiride $\textbf{RIDE}_{\text{fs\_hyb}}$          &   3.35            &  \textbf{3.63}    &    3.05           &    3.83           &     4.15          &  4.79             & 3.80    &   191.68 \\

% \midrule \midrule
% {\small \faToggleOn} \texttt{gpt-3.5-turbo-0301}        &          4.81 &      4.98 &         4.83 &    4.33 &         4.58 &     4.94 &    154.0 \\
% {\small \faToggleOn} \texttt{gpt-4-0314}           &          \textbf{4.90} &      \textbf{4.99} &         4.90 &    \textbf{4.57} &         \textbf{4.62} &     4.74 &  \textbf{226.4} \\
% {\small \faToggleOn} \texttt{gpt-4-0613}           &          4.86 &      \textbf{4.99} &         \textbf{4.90} &    4.49 &         4.61 &     \textbf{4.97} &  186.1 \\
\bottomrule
\end{tabular}
}
\caption{\textbf{Multi-aspect scoring evaluation of ICL methods on \alpaca{}.}} 
% The icon {\small \faToggleOn} indicates the models are \textit{tuned} for alignment via SFT or RLHF, while {\small \faToggleOff} means the models are \textit{untuned}. 
% Our method \methodname{} uses 1k static prefix tokens (system prompt + 3 examples) for ICL.
 \vspace{-0.5em}   
\label{tab:alpaca}
\end{table*}




\paragraph{Settings.} Table~\ref{tab:alpaca} presents the multi-aspect performance evaluation of different ICL methods applied to three LLMs (Llama2-7B, Mistral-7B, and Olmo-7B) on the \alpaca{} dataset. The evaluation metrics include ``Helpful'', ``Factual'', ``Deep'', ``Engaging'', ``Clear'', ``Safe''.
The ``Average" metric refers to the mean value of the six preceding metrics, while ``Length" represents the average response length generated by the LLM under a specific model + ICL method setting.

% \vspace{1em}

\paragraph{Results.} From Table~\ref{tab:alpaca}, we can observe that:

\begin{itemize}
    \item Overall Performance Trends: Across all three models (Llama2-7B, Mistral-7B, and Olmo-7B), \textbf{RIDE}-based ICL methods consistently outperform \textbf{URIAL} in terms of average scores.

    \item ``Helpful" and ``Deep" scores show notable improvements with \textbf{RIDE}, particularly in Mistral-7B and Llama2-7B settings.

    \item Longer responses do not always correlate with better performance. For Llama2-7B, $\textbf{RIDE}_{f}$ has a slightly shorter response length (263.62) than $\textbf{RIDE}_{\text{fs\_uri}}$ (265.15), yet achieves a higher average score (4.18 vs. 4.16). For Mistral-7B, $\textbf{RIDE}_{f}$ has a longer response (276.79) and achieves the best performance. Olmo-7B shows a decrease in performance despite longer responses. For example, $\textbf{RIDE}_{f}$ has a longer response length (208.57) but does not perform as well as $\textbf{RIDE}_{\text{fs\_uri}}$ (average 3.84 vs. 3.82). This suggests that \textbf{RIDE improves alignment through structured responses rather than artificially increasing output length}.

    \item Among the three \textbf{RIDE} variations, Mistral-7B + \textbf{RIDE} variants consistently achieve the best scores, with $\textbf{RIDE}_{f}$ obtaining the highest average (4.42). Also, Llama2-7B benefits significantly from \textbf{RIDE}, with $\textbf{RIDE}_{f}$ achieving the highest factuality score (3.98). Furthremore, Olmo-7B + \textbf{RIDE} still lags behind the other models but sees notable improvement in ``Deep" scores with $\textbf{RIDE}_{\text{fs\_uri}}$ (3.63).

    \item A comprehensive analysis of the ``Safe'' scores across all methods shows that they are largely consistent, further proving that \alpaca{} has little discriminative power for evaluating the safety capabilities of LLMs. Thus, $\textbf{RIDE}_{\text{fs\_hyb}}$, which exhibited excellent safety performance in \dataname{}, performs worse in this benchmark.

    \item Mistral-7B consistently achieves the highest scores across most aspects, followed by Llama2-7B, while Olmo-7B exhibits the lowest performance.

\end{itemize}
