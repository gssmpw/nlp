\section{Selection of a Set of ICL Demonstrations - The Description of the Algorithm}
\label{appendix:dfs}

\paragraph{Search for Optimal exemplars.} The combination of multiple ICL exemplars often provides more assistance to the model in tackling tasks, compared to a single ICL exemplar.
% %
% % Moreover, the  order of ICL exemplars  influences the  inference results.
% After completing the first step, we now need to find an approximate optimal ICL example \emph{set} to build an effective policy. 
% %and their order that can most effectively help the model solve the current task.

The steps of the whole algorithm are as follows.

\begin{itemize}
    
\item We have candidate sets ${S_\text{cand\_f}}$ and ${S_\text{cand\_s}}$, the previous one focus on factuality QA answering while the latter one is biased to refusal answering.  
For the \textit{factuality} candidates $\{S_\text{cand\_f}\}$, we restyled them using the ``\textbf{combined}” style, while for the \textit{safety} candidates $\{S_\text{cand\_s}\}$, we restyled them using both the ``\textbf{combined}” and ``\textbf{refusal}” styles. 
To achieve the optimal trade-off between factuality and safety, we merged the restyled \textit{factuality} and \textit{safety} candidates into a set $\{S_{\text{cand}}\}$.

\item Following the hierarchical traversal approach outlined in~\citep{DBLP:conf/emnlp/HuaQH24}, we first rank all exemplars in $\{S_{\text{cand}}\}$ in descending order based on their average value impact across the six evaluation dimensions. From this ranking, we select the top-$n$ exemplars (with $n$ set to 20 in this work) with the highest average value impact to construct an ICL exemplar set, denoted as $S_{INIT}$.
The remaining exemplars in $\{S_{\text{cand}}\}$, also sorted in descending order of average value impact, constitute the candidate ICL exemplar pool, referred to as $S_{CAND}$. We designate $S_{INIT}$ as the initial ICL example \emph{set}, denoted as $S_{ICL}$.

\item Our objective is to combine ICL examples from $S_{INIT}$ and $S_{CAND}$ using a hierarchical traversal algorithm. This method is designed to explore various ICL example combinations and identify the one that yields the highest value impact, thereby approximating the optimal ICL example set.

\item 
Through empirical analysis, we observed that the value impact of an individual ICL exemplar carries predictive significance. Specifically, exemplars with higher value impact tend to contribute more significantly to the overall value impact when included in the ICL example \emph{set}. 
Consequently, such exemplars are more likely to be retained in the final ICL example \emph{set} compared to those with lower value impact. Leveraging this insight, we developed a heuristic rule for early pruning during hierarchical traversal, which will be elaborated upon in the subsequent sections.

\item We begin the hierarchical traversal by initializing an empty queue $q$ and enqueueing $S_{INIT}$. During each iteration, we dequeue the elements at the current level from $q$, where each element represents a combination of ICL exemplars, denoted as $S'_{ICL}$.

\item For every ICL exemplar $a$ originally present in $S_{INIT}$ within $S'_{ICL}$, we sequentially select an exemplar $b$ from $S_{CAND}$ in its sorted order and substitute $a$ with $b$ in $S'{ICL}$ to generate a new set, $S''_{ICL}$. This newly formed set becomes a child node of $S'_{ICL}$.
We then compute the value impact change of $S''_{ICL}$ and enqueue it into $q$ for further exploration in the next level of traversal. The value impact change is given by:
$\Delta := V^{\textrm{impact}}_{S''_{ICL}} - V^{\textrm{impact}}_{S'_{ICL}}$.

\item Importantly, if the value impact change $\Delta$ for $S''_{ICL}$ remains negative for $M$ consecutive replacements, we determine that further substitutions of $a$ with lower-ranked exemplars $b$ from $S_{CAND}$ are unnecessary. As a result, we terminate the exploration of the current branch and refrain from enqueuing additional child nodes of $S'_{ICL}$ (generated by replacing $a$) into $q$, thereby implementing early pruning.

\item Once all elements in the queue $q$ have been dequeued and explored, the hierarchical traversal concludes. At this point, we select the ICL example \emph{set} with the highest value impact as our final solution. Consequently, we obtain $\pi^* := \pi_{S^*_{ICL}}$, which serves as an approximately locally optimal policy for remediation.
 
\end{itemize}

% \begin{figure*}[!t]
%     \centering
%     \includegraphics[width=0.6\textwidth]{latex/figure/DFS_1.png}
%     \caption{An illustration of using Hierarchical traversal with early pruning to search for the optimal exemplars.}
%     \label{fig:DFS}
%     % \vspace{-15pt}
% \end{figure*}