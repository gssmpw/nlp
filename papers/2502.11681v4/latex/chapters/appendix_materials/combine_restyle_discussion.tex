\section{Combining restyled ICL exemplars - A Further Discussion}
\label{appendix:combine_restyle_dicsuss}

Research has shown that LLMs generalize better when provided with multiple diverse demonstrations, enabling them to infer task-specific patterns more effectively~\citep{brown2020language, DBLP:conf/iclr/LinRLDSCB024}. Moreover, as raised in \textbf{RQ1}, for certain tasks where LLMs must simultaneously provide useful information while resisting malicious attacks, they require a balance between \textbf{\color{myblue} factuality} and \textbf{\color{myred} safety} as part of their alignment capabilities. 
Theoretically, combining multiple restyled ICL demonstrations into an ICL demo set should yield better results than relying on a single ICL demo.

However, the process of finding the optimal ICL demo set is NP-hard~\citep{DBLP:conf/icml/Ye0F0K23}, and so heuristic approaches should be used in general to get an (approximate) optimal approximation solution~\citep{liu2024se2}.

Previous research has shown that subtle interactions between the demonstrations in an ICL example set can significantly influence the performance of LLMs in few-shot online learning~\cite{DBLP:conf/emnlp/HuaQH24}. 
On the one hand, maintaining a consistent response style across ICL demonstration examples can effectively enhance LLM performance on downstream tasks~\citep{DBLP:conf/iclr/LinRLDSCB024,li2024scar}. 
On the other hand, the multiple ICL demonstrations needs to be sufficiently diverse and complementary to fully elicit LLMs' task-oriented capabilities~\citep{min2022rethinking}. 
Notably, when dealing with \textbf{\color{myred} safety} tasks, having refusal demonstration in the set becomes particularly crucial. 

As mentioned above, we already formed candidate sets ${S_\text{cand\_f}}$ and ${S_\text{cand\_s}}$.  
Therefore, for the \textbf{\color{myblue} factuality} candidates $\{S_\text{cand\_f}\}$, we restyled them using the ``\textbf{combined}” style, while for the \textbf{\color{myred} safety} candidates $\{S_\text{cand\_s}\}$, we restyled them using both the ``\textbf{combined}” and ``\textbf{refusal}” styles. 
To achieve the optimal trade-off between \textbf{\color{myblue} factuality} and \textbf{\color{myred} safety}, we merged the restyled \textbf{\color{myblue} factuality} and \textbf{\color{myred} safety} candidates into a set $\{S_{\text{cand}}\}$ and employed a hierarchical traversal approach with early pruning~\cite{DBLP:conf/emnlp/HuaQH24} to select three  ICL examples\footnote{To reduce the search space while maintaining a sufficient number of ICL demonstrations, and to align with the number of ICL examples used in SOTA URIAL method (ensuring a more straightforward comparison in experiments), we set the number of ICL demonstrations to 3.} from $\{S_{\text{cand}}\}$ to construct different demonstration sets. 
The details of the hierarchical traversal algorithm are provided in Appendix~\ref{appendix:dfs}.
We computed the value impact of different combinations on the \dataname{} validation dataset.

Ultimately, as shown in Table~\ref{tab:restyle_combine}, we identified the three best combinations of the ICL examples.
The first combination consists of three \textbf{\color{myblue} factuality} ICL examples restyled with the ``combined” style. 
The second combination includes two \textbf{\color{myblue} factuality} ICL examples and one \textbf{\color{myred} safety} example, all restyled using the ``\textbf{combined}” style.
The third combination consists of two \textbf{\color{myblue} factuality} ICL examples restyled with the ``\textbf{combined}” style and one \textbf{\color{myred} safety} example restyled with the ``\textbf{refusal}” style. 
We refer to these combinations as \textbf{R}estyled \textbf{I}n-context-learning \textbf{D}emonstration \textbf{E}xemplars (\textbf{RIDE}), with the first combination denoted as $\textbf{RIDE}_{\text{f}}$, the second as $\textbf{RIDE}_{\text{fs\_uni}}$, and the third as $\textbf{RIDE}_{\text{fs\_hyb}}$.
We use these notations in the following sections. The prompts of $\textbf{RIDE}$ series can be found in Appendix~\ref{app:rideprompt_f}.
Furthermore, a comparison between Table~\ref{tab:restyle} and Table~\ref{tab:restyle_combine} reveals that the ICL demo set, after being combined, outperforms individual ICL demonstrations in overall performance.