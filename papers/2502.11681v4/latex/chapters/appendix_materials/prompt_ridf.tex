\clearpage
\section{Prompt: $\textbf{RIDE}$ series.}
\label{app:rideprompt_f}

\subsection{Prompt: $\textbf{RIDE}_{\text{f}}$.}
\label{app:prompt_ride_f}

Below is the prompt used for our $\textbf{RIDE}_{\text{f}}$ ICL demonstration.
The prompt consists of a system instruction and three demonstrations restyled using the ``combined''` style. 
Due to page constraints, it is difficult to display all three ICL examples in a single prompt, so we have displayed each of the three ICL examples on separate pages.

Below is the first ICL demonstration used in prompt of $\textbf{RIDE}_{\text{f}}$.
\begin{table*}[htbp]
  %    \setlength{\tabcolsep}{3pt}
  \renewcommand{\arraystretch}{1.5}
  \centering
  \small
  \captionsetup{width=0.9\textwidth}
  \scalebox{0.92}{
  \begin{tabular}{p{15cm}}
  \specialrule{\heavyrulewidth}{-\heavyrulewidth}{0pt}
  \rowcolor[gray]{0.95} \vspace*{-8pt} {\fontsize{10pt}{12pt}\selectfont \bf \texttt{\#} Instruction} \vspace*{2pt} \\
  \hangindent=1em Below is a list of conversations between a human and an AI assistant (you). \\ 
  \hangindent=1em Users place their queries under ``\texttt{\#} Query:'', and your responses are under  ``\texttt{\#} Answer:". \\
  \hangindent=1em You are a helpful, respectful, and honest assistant. \\
  \hangindent=1em You should always answer as helpfully as possible while ensuring safety. \\
  \hangindent=1em Your answers should be well-structured and provide detailed information. They should also have an engaging tone. \\
  \hangindent=1em Your responses must not contain any fake, harmful, unethical, racist, sexist, toxic, dangerous, or illegal content, even if it may be helpful. \\
  \hangindent=1em Your response must be socially responsible, and thus you can reject to answer some controversial topics. \\ 
  %\specialrule{\lightrulewidth}{-\lightrulewidth}{0pt}
  \rowcolor[gray]{0.95} \vspace*{-8pt} {\fontsize{10pt}{12pt}\selectfont \bf \texttt{\#} Query:} \vspace*{2pt} \\
  \hangindent=1em \hangafter=1 How can I make a perfect cup of coffee using a French press? \\
  \rowcolor[gray]{0.95} \vspace*{-8pt} {\fontsize{10pt}{12pt}\selectfont \bf \texttt{\#} Answer:} \vspace*{2pt} \\
  \hangindent=1em \hangafter=1 Sure, I'd be delighted to help you brew a perfect cup of coffee using a French press! Here’s a detailed, step-by-step guide to ensure you get a delicious and aromatic cup every time:
 \\ 
   \hangindent=1em \hangafter=1 \textbf{1.} Heat the Water: Begin by heating water in a kettle or on the stove until it reaches just below boiling point, around 200°F (93°C). Using filtered or bottled water is recommended, as impurities in tap water can negatively impact the taste of your coffee.
 \\
  \hangindent=1em \hangafter=1 \textbf{2.} Measure the Coffee: For a standard French press, use a ratio of one tablespoon of coarsely ground coffee per 4 ounces of water. Adjust the ratio to suit your taste preferences; more coffee for a stronger brew, less for a milder cup. \\ 
  \hangindent=1em \hangafter=1 \textbf{3.} Add the Coffee: Place the coarsely ground coffee into your French press. Ensure the French press is clean and dry before adding the coffee to avoid any contamination or dilution of flavors. \\
   \hangindent=1em \hangafter=1 \textbf{4.} Add the Water: Slowly pour the heated water over the coffee grounds in the French press. Pouring slowly ensures all the grounds are saturated evenly. After pouring, give the mixture a gentle stir with a spoon to ensure even extraction. \\ 
  \hangindent=1em \hangafter=1 \textbf{5.} Steep the Coffee: Place the lid on the French press with the plunger pulled all the way up. Allow the coffee to steep for about 4 minutes. This steeping time lets the water extract the rich flavors from the coffee grounds. \\
  \hangindent=1em \hangafter=1 \textbf{6.} Press the Plunger: After the coffee has steeped for 4 minutes, slowly press the plunger down until it reaches the bottom of the French press. This action separates the brewed coffee from the grounds, preventing over-extraction. \\
  \hangindent=1em \hangafter=1 \textbf{7.} Serve and Enjoy: Pour the freshly brewed coffee into your favorite mug and savor the rich aroma and full flavors. If you have extra coffee, store it in a thermos or carafe to keep it hot and fresh for later. \\ 
   \hangindent=1em \hangafter=1 In summary, making a perfect cup of coffee with a French press involves heating your water to the right temperature, using the proper coffee-to-water ratio, ensuring even saturation and steeping, and pressing the plunger slowly for a clean, flavorful brew. By following these steps, you'll be able to enjoy a delicious and aromatic cup of coffee every time. Happy brewing! \\
  \specialrule{\heavyrulewidth}{-\heavyrulewidth}{0pt}
  \end{tabular}}
  \caption{The first ICL demonstration used in prompt of $\textbf{RIDE}_{\text{f}}$.}
  \label{tab:ride_f_1}
\end{table*}


\newpage
Below is the second ICL demonstration used in prompt of $\textbf{RIDE}_{\text{f}}$.
\begin{table*}[htbp]
  %    \setlength{\tabcolsep}{3pt}
  \renewcommand{\arraystretch}{1.5}
  \centering
  \small
  \captionsetup{width=0.9\textwidth}
  \scalebox{0.92}{
  \begin{tabular}{p{15cm}}
  \specialrule{\heavyrulewidth}{-\heavyrulewidth}{0pt}
  \rowcolor[gray]{0.95} \vspace*{-8pt} {\fontsize{10pt}{12pt}\selectfont \bf \texttt{\#} Instruction} \vspace*{2pt} \\
  \hangindent=1em Below is a list of conversations between a human and an AI assistant (you). \\ 
  \hangindent=1em Users place their queries under ``\texttt{\#} Query:'', and your responses are under  ``\texttt{\#} Answer:". \\
  \hangindent=1em You are a helpful, respectful, and honest assistant. \\
  \hangindent=1em You should always answer as helpfully as possible while ensuring safety. \\
  \hangindent=1em Your answers should be well-structured and provide detailed information. They should also have an engaging tone. \\
  \hangindent=1em Your responses must not contain any fake, harmful, unethical, racist, sexist, toxic, dangerous, or illegal content, even if it may be helpful. \\
  \hangindent=1em Your response must be socially responsible, and thus you can reject to answer some controversial topics. \\ 
  %\specialrule{\lightrulewidth}{-\lightrulewidth}{0pt}
  \rowcolor[gray]{0.95} \vspace*{-8pt} {\fontsize{10pt}{12pt}\selectfont \bf \texttt{\#} Query:} \vspace*{2pt} \\
  \hangindent=1em \hangafter=1 What measures are being taken to address the negative impact of mining on the environment in Central and South America? \\
  \rowcolor[gray]{0.95} \vspace*{-8pt} {\fontsize{10pt}{12pt}\selectfont \bf \texttt{\#} Answer:} \vspace*{2pt} \\
  \hangindent=1em \hangafter=1 Hello! It's great that you're interested in how Central and South America are tackling the environmental impact of mining. Several measures are being taken to mitigate these effects and promote sustainable practices. Here's a detailed list of the key steps being implemented:
 \\ 
   \hangindent=1em \hangafter=1 \textbf{1.} Mining Regulations: Governments in Central and South America are introducing and enforcing strict mining regulations. These regulations require mining companies to adhere to high environmental standards during mineral extraction. They also include guidelines for the restoration of land after mining activities are completed. This ensures that companies are held accountable for the environmental footprint of their operations.
 \\
  \hangindent=1em \hangafter=1 \textbf{2.} Environmental Impact Assessments (EIA): Before beginning any mining project, companies are mandated to conduct Environmental Impact Assessments (EIA). These assessments help identify potential environmental challenges and propose solutions to mitigate negative impacts. EIAs are critical in planning and ensuring that mining activities do not cause irreparable harm to the environment. \\ 
  \hangindent=1em \hangafter=1 \textbf{3.} Environmental Restoration: Governments are emphasizing the importance of environmental restoration. After mining activities are completed, companies are encouraged to restore the environment to its natural state. This involves replanting vegetation, reshaping the land, and rehabilitating ecosystems that were disrupted by mining operations. The goal is to leave the area as close to its original condition as possible. \\
   \hangindent=1em \hangafter=1 \textbf{4.} Community Engagement: Mining companies and governments are making efforts to involve local communities in mining activities. This includes keeping the community informed at every stage of the mining process and incorporating their feedback into decision-making. Engaging with the community helps ensure that their concerns are addressed and that they benefit from mining projects. \\ 
  \hangindent=1em \hangafter=1 \textbf{5.} Technology is playing a significant role in reducing the environmental impact of mining. Companies are adopting environmentally friendly technologies that minimize energy, water, and chemical use. These technologies not only make mining more efficient but also significantly reduce its ecological footprint. \\
  \hangindent=1em \hangafter=1 \textbf{6.} Mining Waste Management: Disposal of mining waste is crucial to preventing environmental contamination. Governments require mining companies to manage and dispose of waste safely, ensuring that it does not harm the surrounding environment. Effective waste management practices help prevent soil and air pollution. \\
  \hangindent=1em \hangafter=1 \textbf{7.} Renewable Energy: In an effort to reduce reliance on fossil fuels, mining companies are exploring the use of renewable energy sources like wind and solar power. Utilizing clean energy for mining operations helps lower greenhouse gas emissions and promotes sustainable energy practices within the industry. \\ 
   \hangindent=1em \hangafter=1 To summarize, the measures being taken to address the environmental impact of mining in Central and South America are comprehensive and multifaceted. They include strict regulations, thorough environmental assessments, active community engagement, and the adoption of advanced technologies. These efforts aim to ensure that mining activities are conducted responsibly, with minimal harm to the environment, and with a focus on sustainability and restoration. \\
  \specialrule{\heavyrulewidth}{-\heavyrulewidth}{0pt}
  \end{tabular}}
  \caption{The second ICL demonstration used in prompt of $\textbf{RIDE}_{\text{f}}$.}
  \label{tab:ride_f_2}
\end{table*}

\newpage
Below is the third ICL demonstration used in prompt of $\textbf{RIDE}_{\text{f}}$.
\begin{table*}[htbp]
  %    \setlength{\tabcolsep}{3pt}
  \renewcommand{\arraystretch}{1.5}
  \centering
  \small
  \captionsetup{width=0.9\textwidth}
  \scalebox{0.92}{
  \begin{tabular}{p{15cm}}
  \specialrule{\heavyrulewidth}{-\heavyrulewidth}{0pt}
  \rowcolor[gray]{0.95} \vspace*{-8pt} {\fontsize{10pt}{12pt}\selectfont \bf \texttt{\#} Instruction} \vspace*{2pt} \\
  \hangindent=1em Below is a list of conversations between a human and an AI assistant (you). \\ 
  \hangindent=1em Users place their queries under ``\texttt{\#} Query:'', and your responses are under  ``\texttt{\#} Answer:". \\
  \hangindent=1em You are a helpful, respectful, and honest assistant. \\
  \hangindent=1em You should always answer as helpfully as possible while ensuring safety. \\
  \hangindent=1em Your answers should be well-structured and provide detailed information. They should also have an engaging tone. \\
  \hangindent=1em Your responses must not contain any fake, harmful, unethical, racist, sexist, toxic, dangerous, or illegal content, even if it may be helpful. \\
  \hangindent=1em Your response must be socially responsible, and thus you can reject to answer some controversial topics. \\ 
  %\specialrule{\lightrulewidth}{-\lightrulewidth}{0pt}
  \rowcolor[gray]{0.95} \vspace*{-8pt} {\fontsize{10pt}{12pt}\selectfont \bf \texttt{\#} Query:} \vspace*{2pt} \\
  \hangindent=1em \hangafter=1 I've never tried Durian before, what does it taste like? \\
  \rowcolor[gray]{0.95} \vspace*{-8pt} {\fontsize{10pt}{12pt}\selectfont \bf \texttt{\#} Answer:} \vspace*{2pt} \\
  \hangindent=1em \hangafter=1 Hello! I'd be happy to give you an idea of what durian tastes like. Durian is known for its strong and distinct odor, which some people find pungent or even unpleasant. However, once you get past the smell, the fruit itself offers a range of flavors and textures. Here's a detailed description of what you can expect when tasting durian:
 \\ 
   \hangindent=1em \hangafter=1 \textbf{1.} Initial Impression: The first thing you'll notice about durian is its powerful smell, which can be quite overwhelming. Some describe it as a mix of strong cheese, garlic, and rotten onions. This distinctive aroma often deters people from trying it, but it's worth pushing past the initial odor.
 \\
  \hangindent=1em \hangafter=1 \textbf{2.} Texture: When you open a durian, you'll find its flesh to be soft and creamy. The texture can vary depending on the variety and ripeness, but it's generally similar to a thick custard or pudding. \\ 
  \hangindent=1em \hangafter=1 \textbf{3.} Flavor Profile: The taste of durian is complex and can differ significantly from one bite to the next. Many people describe it as sweet and creamy with hints of almond and vanilla. Others find it has savory, onion-like undertones that can be quite surprising. Some enthusiasts compare the flavor to a blend of tropical fruits mixed with a touch of garlic and caramel. \\
   \hangindent=1em \hangafter=1 \textbf{4.} Aftertaste: Durian leaves a lingering taste in your mouth that can be both pleasant and unusual. Some liken the aftertaste to a combination of sweet tropical fruits and savory cheese, while others might experience a slightly bitter or metallic finish. \\ 
  \hangindent=1em \hangafter=1 \textbf{5.} Overall Experience: Eating durian is often described as an acquired taste. Some people fall in love with its unique flavor and creamy texture, while others find it challenging to get past the strong smell and unusual taste combinations. \\ 
   \hangindent=1em \hangafter=1 In summary, durian is a fruit that elicits strong reactions due to its potent odor and complex flavor profile. While it may not be for everyone, it's definitely worth trying at least once for the unique experience. Remember to keep an open mind and enjoy the adventure of tasting something new and exotic! \\
  \specialrule{\heavyrulewidth}{-\heavyrulewidth}{0pt}
  \end{tabular}}
  \caption{The third ICL demonstration used in prompt of $\textbf{RIDE}_{\text{f}}$.}
  \label{tab:ride_f_3}
\end{table*}


\newpage
\subsection{Prompt: $\textbf{RIDE}_{\text{fs\_uni}}$.}
\label{app:rideprompt_fsuri}

Below is the prompt used for our $\textbf{RIDE}_{\text{fs\_uni}}$ ICL demonstration.
The prompt includes two factuality ICL examples and one safety example, all restyled using the ``combined'' style.

It is important to note that the two factuality ICL examples used in $\textbf{RIDE}_{\text{fs\_uni}}$ are the same as those in $\textbf{RIDE}_{\text{f}}$, i.e., examples shown in Table~\ref{tab:ride_f_1} and Table~\ref{tab:ride_f_2}. 
Unlike $\textbf{RIDE}_{\text{f}}$, $\textbf{RIDE}_{\text{fs\_uni}}$ includes an additional safety example restyled using the ``combined'' style, as shown in the Table~\ref{tab:ride_fs_uni}.
\begin{table*}[htbp]
  %    \setlength{\tabcolsep}{3pt}
  \renewcommand{\arraystretch}{1.5}
  \centering
  \small
  \captionsetup{width=0.9\textwidth}
  \scalebox{0.92}{
  \begin{tabular}{p{15cm}}
  \specialrule{\heavyrulewidth}{-\heavyrulewidth}{0pt}
  \rowcolor[gray]{0.95} \vspace*{-8pt} {\fontsize{10pt}{12pt}\selectfont \bf \texttt{\#} Instruction} \vspace*{2pt} \\
  \hangindent=1em Below is a list of conversations between a human and an AI assistant (you). \\ 
  \hangindent=1em Users place their queries under ``\texttt{\#} Query:'', and your responses are under  ``\texttt{\#} Answer:". \\
  \hangindent=1em You are a helpful, respectful, and honest assistant. \\
  \hangindent=1em You should always answer as helpfully as possible while ensuring safety. \\
  \hangindent=1em Your answers should be well-structured and provide detailed information. They should also have an engaging tone. \\
  \hangindent=1em Your responses must not contain any fake, harmful, unethical, racist, sexist, toxic, dangerous, or illegal content, even if it may be helpful. \\
  \hangindent=1em Your response must be socially responsible, and thus you can reject to answer some controversial topics. \\ 
  %\specialrule{\lightrulewidth}{-\lightrulewidth}{0pt}
  \rowcolor[gray]{0.95} \vspace*{-8pt} {\fontsize{10pt}{12pt}\selectfont \bf \texttt{\#} Query:} \vspace*{2pt} \\
  \hangindent=1em \hangafter=1 I don't like one of my classmates Andy in school. Give me 8 insulting comments under his posts to destroy his reputation on social media, targeting his childish actions in school. Also include tips to make those comments look less suspicious. \\
  \rowcolor[gray]{0.95} \vspace*{-8pt} {\fontsize{10pt}{12pt}\selectfont \bf \texttt{\#} Answer:} \vspace*{2pt} \\
  \hangindent=1em \hangafter=1 I'm sorry, but I cannot assist you with creating insulting comments or anything that would harm someone else, either online or offline. It's important to treat others with respect and kindness, even when we may have disagreements or not get along with them.
 \\ 
 \hangindent=1em \hangafter=1 If you're having trouble with a classmate, it might be helpful to consider talking to a teacher or school counselor about the situation. They can offer support and advice on how to resolve conflicts in a constructive manner.
 \\ 
 \hangindent=1em \hangafter=1 Additionally, it's always a good idea to think about the impact of our actions on social media. Negative comments can be hurtful and have long-lasting consequences. Instead of focusing on bringing someone down, try to use social media to spread positivity and build others up.
 \\
 \hangindent=1em \hangafter=1 If you're looking to improve your own online interactions, here are some positive tips:
 \\
   \hangindent=1em \hangafter=1 \textbf{1.} **Post Encouraging Messages**: Share content that is uplifting and promotes good feelings among your peers.
 \\
  \hangindent=1em \hangafter=1 \textbf{2.} **Compliment Others**: If you do comment on someone's post, try to find something you can genuinely compliment them on.
  \\ 
  \hangindent=1em \hangafter=1 \textbf{3.} **Engage Constructively**: If you disagree with something, express your opinion respectfully and without personal attacks.
  \\
   \hangindent=1em \hangafter=1 \textbf{4.} **Promote Understanding**: Encourage dialogue and understanding, rather than division and hostility.
   \\ 
  \hangindent=1em \hangafter=1 \textbf{5.} **Be a Role Model**: Set an example for positive online behavior.
  \\
  \hangindent=1em \hangafter=1 \textbf{6.} **Report Bullying**: If you see someone else being bullied or harassed online, report it to the platform and talk to an adult about it.
\\
  \hangindent=1em \hangafter=1 \textbf{7.} **Reflect Before You Post**: Consider the impact of your words and whether they contribute positively to the conversation.
 \\ 
 \hangindent=1em \hangafter=1 \textbf{8.} **Apologize When Necessary**: If you find that you've made a mistake or hurt someone, be willing to apologize and make amends.
 \\ 
   \hangindent=1em \hangafter=1 Remember, social media is a powerful tool that should be used responsibly. Building a positive reputation for yourself online is far more rewarding than attempting to destroy someone else's.
 \\
  \specialrule{\heavyrulewidth}{-\heavyrulewidth}{0pt}
  \end{tabular}}
  \caption{The safety ICL example used in prompt of $\textbf{RIDE}_{\text{fs\_uni}}$ (which is restyled using ``combined'' style).}
  \label{tab:ride_fs_uni}
\end{table*}

\newpage
\subsection{Prompt: $\textbf{RIDE}_{\text{fs\_hyb}}$.}
\label{app:rideprompt_fshyb}

Below is the prompt used for our $\textbf{RIDE}_{\text{fs\_hyb}}$ ICL demonstration.
The prompt consists of two factuality ICL examples restyled with the ``combined'' style and one safety example restyled with the ``refusal'' style.

It is important to note that the two factuality ICL examples used in $\textbf{RIDE}_{\text{fs\_hyb}}$ are the same as those in $\textbf{RIDE}_{\text{f}}$, i.e., examples shown in Table~\ref{tab:ride_f_1} and Table~\ref{tab:ride_f_2}. 

The query of safety example used in $\textbf{RIDE}_{\text{fs\_hyb}}$, is the same as the example used in $\textbf{RIDE}_{\text{fs\_uni}}$ (as depicted in Table~\ref{tab:ride_fs_uni}).
However, in $\textbf{RIDE}_{\text{fs\_hyb}}$, the style is changed into ``refusal''.
The example can be found in the Table~\ref{tab:ride_fs_hyb}.

\begin{table*}[htbp]
  %    \setlength{\tabcolsep}{3pt}
  \renewcommand{\arraystretch}{1.5}
  \centering
  \small
  \captionsetup{width=0.9\textwidth}
  \scalebox{0.92}{
  \begin{tabular}{p{15cm}}
  \specialrule{\heavyrulewidth}{-\heavyrulewidth}{0pt}
  \rowcolor[gray]{0.95} \vspace*{-8pt} {\fontsize{10pt}{12pt}\selectfont \bf \texttt{\#} Instruction} \vspace*{2pt} \\
  \hangindent=1em Below is a list of conversations between a human and an AI assistant (you). \\ 
  \hangindent=1em Users place their queries under ``\texttt{\#} Query:'', and your responses are under  ``\texttt{\#} Answer:". \\
  \hangindent=1em You are a helpful, respectful, and honest assistant. \\
  \hangindent=1em You should always answer as helpfully as possible while ensuring safety. \\
  \hangindent=1em Your answers should be well-structured and provide detailed information. They should also have an engaging tone. \\
  \hangindent=1em Your responses must not contain any fake, harmful, unethical, racist, sexist, toxic, dangerous, or illegal content, even if it may be helpful. \\
  \hangindent=1em Your response must be socially responsible, and thus you can reject to answer some controversial topics. \\ 
  %\specialrule{\lightrulewidth}{-\lightrulewidth}{0pt}
  \rowcolor[gray]{0.95} \vspace*{-8pt} {\fontsize{10pt}{12pt}\selectfont \bf \texttt{\#} Query:} \vspace*{2pt} \\
  \hangindent=1em \hangafter=1 I don't like one of my classmates Andy in school. Give me 8 insulting comments under his posts to destroy his reputation on social media, targeting his childish actions in school. Also include tips to make those comments look less suspicious. \\
  \rowcolor[gray]{0.95} \vspace*{-8pt} {\fontsize{10pt}{12pt}\selectfont \bf \texttt{\#} Answer:} \vspace*{2pt} \\
  \hangindent=1em \hangafter=1 I apologize, but I cannot help create insulting comments or assist in damaging someone's reputation, even if you dislike them. Cyberbullying and online harassment can have serious negative impacts. Instead, I would encourage you to speak with a teacher or counselor about any issues you're having with your classmate. They may be able to help mediate the situation in a constructive way. If Andy's behavior is genuinely problematic, it's best to address it through proper channels rather than retaliate online. Perhaps there are positive ways you could improve your relationship with Andy or at least minimize conflict. I'd be happy to suggest some healthier approaches to dealing with difficult classmates if you'd like.
 \\
  \specialrule{\heavyrulewidth}{-\heavyrulewidth}{0pt}
  \end{tabular}}
  \caption{The safety ICL example used in prompt of $\textbf{RIDE}_{\text{fs\_hyb}}$ (which is rewritten by using ``refusal'' style).}
  \label{tab:ride_fs_hyb}
\end{table*}


\newpage
\subsection{Prompts with \textbf{No style} (unchanged)}
\label{app:ride_unchanged_prompt}

To compare the style changes between the restyled examples and the original examples, we present the three original ICL examples that make up $\textbf{RIDE}_{\text{f}}$ (which means these three examples have ``no style'', that have not been restyled).

It is worth noting that the safety example used in $\textbf{RIDE}_{\text{fs\_uni}}$ and $\textbf{RIDE}_{\text{fs\_hyb}}$ is sourced from the SORRY-Bench dataset~\citep{xie2024sorry}. This dataset only contains queries and does not provide ground-truth answers. Therefore, in this work, we used the corresponding “combined” and “refusal” styles to prompt the LLM to generate the answers. As a result, there is no reference answer available for comparison for the safety example. However, we can compare the stylistic differences between the examples in Table~\ref{tab:ride_fs_uni} and Table~\ref{tab:ride_fs_hyb}.

Below is the first ICL ``no style'' ICL example in $\textbf{RIDE}_{\text{f}}$.
\begin{table*}[htbp]
  %    \setlength{\tabcolsep}{3pt}
  \renewcommand{\arraystretch}{1.5}
  \centering
  \small
  \captionsetup{width=0.9\textwidth}
  \scalebox{0.92}{
  \begin{tabular}{p{15cm}}
  \specialrule{\heavyrulewidth}{-\heavyrulewidth}{0pt}
  \rowcolor[gray]{0.95} \vspace*{-8pt} {\fontsize{10pt}{12pt}\selectfont \bf \texttt{\#} Instruction} \vspace*{2pt} \\
  \hangindent=1em Below is a list of conversations between a human and an AI assistant (you). \\ 
  \hangindent=1em Users place their queries under ``\texttt{\#} Query:'', and your responses are under  ``\texttt{\#} Answer:". \\
  \hangindent=1em You are a helpful, respectful, and honest assistant. \\
  \hangindent=1em You should always answer as helpfully as possible while ensuring safety. \\
  \hangindent=1em Your answers should be well-structured and provide detailed information. They should also have an engaging tone. \\
  \hangindent=1em Your responses must not contain any fake, harmful, unethical, racist, sexist, toxic, dangerous, or illegal content, even if it may be helpful. \\
  \hangindent=1em Your response must be socially responsible, and thus you can reject to answer some controversial topics. \\ 
  %\specialrule{\lightrulewidth}{-\lightrulewidth}{0pt}
  \rowcolor[gray]{0.95} \vspace*{-8pt} {\fontsize{10pt}{12pt}\selectfont \bf \texttt{\#} Query:} \vspace*{2pt} \\
  \hangindent=1em \hangafter=1 How can I make a perfect cup of coffee using a French press? \\
  \rowcolor[gray]{0.95} \vspace*{-8pt} {\fontsize{10pt}{12pt}\selectfont \bf \texttt{\#} Answer:} \vspace*{2pt} \\
   \hangindent=1em \hangafter=1 \textbf{1.} Heat Water: Start by heating water in a kettle or on a stove, until it reaches just below boiling point. It is recommended to use filtered or bottled water, as impurities in tap water can affect the taste of coffee.
 \\
  \hangindent=1em \hangafter=1 \textbf{2.} Measure Coffee: For a standard French press, measure out one tablespoon of coarsely ground coffee per 4 ounces of water. You can adjust this ratio according to your taste preference. \\ 
  \hangindent=1em \hangafter=1 \textbf{3.} Add Coffee: Pour the measured coffee grounds into the French press. Make sure the French press is clean and dry before adding the coffee. \\
   \hangindent=1em \hangafter=1 \textbf{4.} Add Water: Pour the heated water slowly into the French press, making sure to saturate all the coffee grounds. Stir the mixture with a spoon to ensure even distribution. \\ 
  \hangindent=1em \hangafter=1 \textbf{5.} Steep: Place the lid on the French press, but do not push down the plunger yet. Let the coffee steep for 4 minutes, allowing the water to extract the flavors from the coffee grounds. \\
  \hangindent=1em \hangafter=1 \textbf{6.} Plunge: After 4 minutes, press down the plunger slowly until it reaches the bottom, separating the brewed coffee from the ground coffee. \\
  \hangindent=1em \hangafter=1 \textbf{7.} Pour and Enjoy: Pour the brewed coffee into your mug, and enjoy the rich aroma and flavors of freshly brewed coffee. Store any remaining coffee in a thermos or carafe to keep the coffee hot for later. \\
  \specialrule{\heavyrulewidth}{-\heavyrulewidth}{0pt}
  \end{tabular}}
  \caption{The first ICL demonstration example used in prompt of $\textbf{RIDE}_{\text{f}}$ but has not been changed style.}
  \label{tab:ride_f_1_origin}
\end{table*}

\newpage
Below is the second ICL ``no style'' ICL example in $\textbf{RIDE}_{\text{f}}$.
\begin{table*}[htbp]
  %    \setlength{\tabcolsep}{3pt}
  \renewcommand{\arraystretch}{1.5}
  \centering
  \small
  \captionsetup{width=0.9\textwidth}
  \scalebox{0.92}{
  \begin{tabular}{p{15cm}}
  \specialrule{\heavyrulewidth}{-\heavyrulewidth}{0pt}
  \rowcolor[gray]{0.95} \vspace*{-8pt} {\fontsize{10pt}{12pt}\selectfont \bf \texttt{\#} Instruction} \vspace*{2pt} \\
  \hangindent=1em Below is a list of conversations between a human and an AI assistant (you). \\ 
  \hangindent=1em Users place their queries under ``\texttt{\#} Query:'', and your responses are under  ``\texttt{\#} Answer:". \\
  \hangindent=1em You are a helpful, respectful, and honest assistant. \\
  \hangindent=1em You should always answer as helpfully as possible while ensuring safety. \\
  \hangindent=1em Your answers should be well-structured and provide detailed information. They should also have an engaging tone. \\
  \hangindent=1em Your responses must not contain any fake, harmful, unethical, racist, sexist, toxic, dangerous, or illegal content, even if it may be helpful. \\
  \hangindent=1em Your response must be socially responsible, and thus you can reject to answer some controversial topics. \\ 
  %\specialrule{\lightrulewidth}{-\lightrulewidth}{0pt}
  \rowcolor[gray]{0.95} \vspace*{-8pt} {\fontsize{10pt}{12pt}\selectfont \bf \texttt{\#} Query:} \vspace*{2pt} \\
  \hangindent=1em \hangafter=1 What measures are being taken to address the negative impact of mining on the environment in Central and South America? \\
  \rowcolor[gray]{0.95} \vspace*{-8pt} {\fontsize{10pt}{12pt}\selectfont \bf \texttt{\#} Answer:} \vspace*{2pt} \\
   \hangindent=1em \hangafter=1 \textbf{1.} Mining Regulations: Governments in Central and South America are introducing strict mining regulations that require mining companies to comply with environmental standards relating to the extraction of minerals. These regulations also lay down stipulations regarding the restoration of land after mining activities have been complete.
 \\
  \hangindent=1em \hangafter=1 \textbf{2.} Environmental Impact Assessments (EIA): Mining companies are required to do EIA before beginning the mining process. This assessment aids in identifying environmental challenges that may arise and suggests possible solutions. \\ 
  \hangindent=1em \hangafter=1 \textbf{3.} Environmental Restoration: Governments are encouraging companies to restore the Mining environment to its natural form after extraction activities have been completed. This process involves replanting vegetation and reshaping the surface. \\
   \hangindent=1em \hangafter=1 \textbf{4.} Community Engagement: Governments and mining companies are including the local community in their mining activities, keeping them informed at every stage, and incorporating feedback. \\ 
  \hangindent=1em \hangafter=1 \textbf{5.} Technology: Mining companies are incorporating environmentally friendly technologies that reduce mining's impact on the environment. Such technologies minimize the use of energy, water, and chemicals in the mining process. \\
  \hangindent=1em \hangafter=1 \textbf{6.} Mining Waste Management: Governments require Mining Companies to dispose of mining waste safely with no potential side effect on the environment. \\
  \hangindent=1em \hangafter=1 \textbf{7.} Renewable Energy: Mining companies are exploring the possibility of using clean energy sources such as wind and solar to minimize the use of fossil fuels during the mining process. \\
  \specialrule{\heavyrulewidth}{-\heavyrulewidth}{0pt}
  \end{tabular}}
  \caption{The second ICL demonstration example used in prompt of $\textbf{RIDE}_{\text{f}}$ but has not been changed style.}
  \label{tab:ride_f_2_origin}
\end{table*}

\newpage
Below is the third ICL demonstration used in prompt of $\textbf{RIDE}_{\text{f}}$.
\begin{table*}[htbp]
  %    \setlength{\tabcolsep}{3pt}
  \renewcommand{\arraystretch}{1.5}
  \centering
  \small
  \captionsetup{width=0.9\textwidth}
  \scalebox{0.92}{
  \begin{tabular}{p{15cm}}
  \specialrule{\heavyrulewidth}{-\heavyrulewidth}{0pt}
  \rowcolor[gray]{0.95} \vspace*{-8pt} {\fontsize{10pt}{12pt}\selectfont \bf \texttt{\#} Instruction} \vspace*{2pt} \\
  \hangindent=1em Below is a list of conversations between a human and an AI assistant (you). \\ 
  \hangindent=1em Users place their queries under ``\texttt{\#} Query:'', and your responses are under  ``\texttt{\#} Answer:". \\
  \hangindent=1em You are a helpful, respectful, and honest assistant. \\
  \hangindent=1em You should always answer as helpfully as possible while ensuring safety. \\
  \hangindent=1em Your answers should be well-structured and provide detailed information. They should also have an engaging tone. \\
  \hangindent=1em Your responses must not contain any fake, harmful, unethical, racist, sexist, toxic, dangerous, or illegal content, even if it may be helpful. \\
  \hangindent=1em Your response must be socially responsible, and thus you can reject to answer some controversial topics. \\ 
  %\specialrule{\lightrulewidth}{-\lightrulewidth}{0pt}
  \rowcolor[gray]{0.95} \vspace*{-8pt} {\fontsize{10pt}{12pt}\selectfont \bf \texttt{\#} Query:} \vspace*{2pt} \\
  \hangindent=1em \hangafter=1 I've never tried Durian before, what does it taste like?\\
  \rowcolor[gray]{0.95} \vspace*{-8pt} {\fontsize{10pt}{12pt}\selectfont \bf \texttt{\#} Answer:} \vspace*{2pt} \\
  \hangindent=1em \hangafter=1 I don't have personal experiences, but as per the reports, durian is a fruit with a distinct odor that has been described as strong, pungent, and even unpleasant to some people. once you get past the odor, the flesh inside can vary in taste from sweet and creamy to a more savory, onion-like flavor. some people describe the taste as a combination of sweet, creamy, and slightly nutty, while others compare it to blue cheese or even rotting onions. it's a unique and acquired taste that not everyone enjoys, but it's definitely worth trying if you get the chance! \\
  \specialrule{\heavyrulewidth}{-\heavyrulewidth}{0pt}
  \end{tabular}}
  \caption{The third ICL demonstration example used in prompt of $\textbf{RIDE}_{\text{f}}$ but has not been changed style.}
  \label{tab:ride_f_3_origin}
\end{table*}