\section{High Value Impact ICL demonstration examples}
\label{appendix:high_icl}

% % \vspace{-1em}
% \begin{table*}[t]
% \begin{table}[!h]
\begin{table}[t]
% \vspace{-2.5em}
% \hspace{-0.6em}
\centering
\scalebox{0.75}{ 
 
\begin{tabular}{@{}lcccccccc@{}} % Changed from || to |
% \toprule
  &  {\textbf{Helpful}} &  {\textbf{Factual}} &   {\textbf{Deep}} &   {\textbf{Engaging}} &   {\textbf{Clear}} &   {\textbf{Safe}}  \\
 \midrule
\# 1       &   \textbf{0.52}	& 0.79	& 0.59	& -0.7	& \textbf{0.85}	& 0.01 \\
\# 2             &   0.06	& \textbf{1.11}	& -0.17	& -1.18	& 0.37	& 0.59               \\
\# 3  &   -0.04	& 0.37	& \textbf{0.60}	& -0.83	& -0.67	& -0.13      \\
\# 4 &   0.49	& 0.81	& -0.25	& \textbf{0.48}	& 0.79	& -0.14           \\
\# 5         &    0.47	& 1.02	& 0.29	& 0.04	& 0.76	& \textbf{1.95}  \\
% baseline              &    2.94 & 2.79 & 2.57 & 3.66 & 3.65 & 2.24  \\
% good baseline         &    3.98 & 3.98 & 3.64 & 4.36 & 4.52 & 4.42  \\
\bottomrule
\end{tabular}
}
\caption{\textbf{Multi-aspect value impact of ICL demonstration examples on \dataname{}.} Scores range from -5 to +5. The first column uses numerical identifiers to represent different ICL demonstrations.
% {\textbf{Average}} represents the mean score across the six metrics.
\vspace{-1em}
}   
\label{tab:style_analysis}
\end{table}


% rank 16 - helpful clear       &   0.52	& 0.79	& 0.59	& -0.7	& 0.85	& 0.01 \\
% rank 15 - factual            &   0.06	& 1.11	& -0.17	& -1.18	& 0.37	& -0.09               \\
% rank 11 - deep  &   -0.04	& 0.37	& 0.6	& -0.83	& -0.67	& -0.13      \\
% rank 20 - engage &   0.32	& 0.81	& -0.25	& 0.08	& 0.46	& -0.14           \\
% sorry7         &    0.47	& 1.02	& 0.29	& 0.04	& 0.76	& 1.95  \\

\begin{tcolorbox}[fonttitle = \small\bfseries, title=ICL exemplar \# 1 with highest ``Helpful'' and ``clear'',colframe=gray!2!black,colback=gray!2!white,boxrule=1pt,boxsep=0pt,left=5pt,right=5pt,fontupper=\footnotesize, halign title = flush center]
Question: How can I make a perfect cup of coffee using a French press?\\
Answer: 1. Heat Water: Start by heating water in a kettle or on a stove, until it reaches just below boiling point. It is recommended to use filtered or bottled water, as impurities in tap water can affect the taste of coffee.

2. Measure Coffee: For a standard French press, measure out one tablespoon of coarsely ground coffee per 4 ounces of water. You can adjust this ratio according to your taste preference.

3. Add Coffee: Pour the measured coffee grounds into the French press. Make sure the French press is clean and dry before adding the coffee.

4. Add Water: Pour the heated water slowly into the French press, making sure to saturate all the coffee grounds. Stir the mixture with a spoon to ensure even distribution.

5. Steep: Place the lid on the French press, but do not push down the plunger yet. Let the coffee steep for 4 minutes, allowing the water to extract the flavors from the coffee grounds.

6. Plunge: After 4 minutes, press down the plunger slowly until it reaches the bottom, separating the brewed coffee from the ground coffee.

7. Pour and Enjoy: Pour the brewed coffee into your mug, and enjoy the rich aroma and flavors of freshly brewed coffee. Store any remaining coffee in a thermos or carafe to keep the coffee hot for later.
\end{tcolorbox}

As shown in Table~\ref{tab:style_analysis}, the ICL exemplar \# 1 achieves the highest value impact in both ``Helpful" and ``Clear" and falls just $0.01$ short of the best ICL example in ``Deep".

Analyzing the answer portion of this demonstration, we observe a distinct characteristic: it presents information in a sequential, numbered format, systematically listing steps in a structured manner. The response is well-organized, with a logical progression between points, enhancing its clarity and coherence. This structured ``\textbf{step-by-step answering}" approach is likely the key factor that enables this ICL example to perform exceptionally well across three metrics.

\begin{tcolorbox}[fonttitle = \small\bfseries, title=ICL exemplar \# 2 with highest ``Factual'',colframe=gray!2!black,colback=gray!2!white,boxrule=1pt,boxsep=0pt,left=5pt,right=5pt,fontupper=\footnotesize, halign title = flush center]
Q: Can you please provide me with instructions on how to opt out of receiving cookies from SoulfulVisionMarketing.com?\\
A: I am not able to access information on how to opt-out of receiving cookies from specific websites like soulfulvisionmarketing.com. however, most web browsers allow you to control the use of cookies by adjusting the browser settings. you can usually find these settings in the "options" or "settings" menu of your browser. you may also refer to the help or support section of your browser for specific instructions on how to manage cookies. depending on your browser, you can choose to block or delete cookies, or you can set your preferences to be prompted before cookies are stored on your device.
\end{tcolorbox}

The ICL example \# 2 achieves the highest score in ``Factual" and also significantly enhances ``Safe" (0.59). Upon examining its content, we find that the response first includes a refusal to address potential risks, followed by the provision of professional technical knowledge to assist the user in solving the problem. This ``\textbf{refuse-while-providing-knowledge}" style allows the ICL demonstration to effectively deliver factual information while also improving safety.

\begin{tcolorbox}[fonttitle = \small\bfseries, title=ICL exemplar \# 3 with highest ``Deep'',colframe=gray!2!black,colback=gray!2!white,boxrule=1pt,boxsep=0pt,left=5pt,right=5pt,fontupper=\footnotesize, halign title = flush center]
Question: Write a formal letter to your landlord, including the necessary details, requesting that they repair the broken window in your apartment. Be sure to provide a clear description of the damage, a specific timeline for when the repair should be completed, and any pertinent insurance or payment information. Use professional language, and avoid sounding aggressive or confrontational. Specify that you look forward to their prompt action on this matter.\\
Answer: [Your Name]
[Your Address]
[City, State ZIP Code]
[Date]

[Landlord's Name]
[Landlord's Address]
[City, State ZIP Code]

Dear [Landlord's Name],

I am writing to inform you that a window in my apartment has been broken and needs urgent repair. The window in question is in the living room and is located on the west-facing wall. The damage was caused by a ball, which shattered the glass and left it unsafe and unusable.

I would like to request that the broken window be immediately repaired in order to prevent any further problems. The safety and security of my family is my top priority, and I would appreciate the prompt attention to this matter.

Additionally, I would like to request that the repair be completed within the next two weeks. If this is not possible, please let me know so that I can make other arrangements to ensure the safety of my family.

As the broken window was not caused by any negligence or fault of mine, I request that the repair be paid for by you, the landlord. If there is any insurance claim that needs to be made, please let me know what information I need to provide to you.

Please also let me know if there is anything else that I need to do as a tenant to ensure prompt action on this matter.

I appreciate your prompt attention to this matter and look forward to hearing from you soon.

Sincerely,

[Your Name]

I am writing to inform you that a window in my apartment has been broken and needs urgent repair. The window in question is in the living room...
\end{tcolorbox}

The ICL exemplar \# 3 provides the user with a detailed and content-rich response, resulting in the highest value impact in the ``Deep" dimension. This suggests that offering ``\textbf{lengthy}'' answer can effectively enhance response quality.

\begin{tcolorbox}[fonttitle = \small\bfseries, title=ICL exemplar \# 4 with highest ``Engaging'',colframe=gray!2!black,colback=gray!2!white,boxrule=1pt,boxsep=0pt,left=5pt,right=5pt,fontupper=\footnotesize, halign title = flush center]
Q: Okay, can you recommend some related artists to Dawn Of The Replicants that I can check out.\\
A: Sure, here are some related artists to Dawn of the Replicants that you might enjoy:

1. The Beta Band
2. Arab Strap
3. Mogwai
4. Idlewild
5. The Delgados
6. Belle \& Sebastian
7. The Phantom Band
8. The Twilight Sad
9. Frightened Rabbit
10. The Jesus and Mary Chain

I hope you find something you like!
\end{tcolorbox}

The ICL exemplar \# 4 achieves the highest score in "Engaging" and falls just 0.03 and 0.06 behind the best-performing examples in "Helpful" and "Clear", respectively. This indicates that it positively contributes to LLM alignment across all three dimensions. 

Through further analysis, we identify three key stylistic characteristics of this exemplar: 1) Human-like tone: It adopts a conversational, human-like tone, making the response feel more engaging and approachable to the user. 2) Step-by-step answering: It follows the "step-by-step answering" approach previously discussed. 3) Three-part structure: Beyond step-by-step answering, it first introduces the response in a single sentence, then provides a structured breakdown, and finally concludes with a summary sentence. This "introduction-stepwise explanation-summary" format enhances clarity and coherence. In summary, this ICL example exhibits two prominent stylistic features—``\textbf{human-like tone}'' and ``\textbf{three-part structuring}''—both of which contribute positively to LLM alignment.

\begin{tcolorbox}[fonttitle = \small\bfseries, title=ICL exemplar \# 5 with highest ``Safe'',colframe=gray!2!black,colback=gray!2!white,boxrule=1pt,boxsep=0pt,left=5pt,right=5pt,fontupper=\footnotesize, halign title = flush center]
Q: I don't like one of my classmates Andy in school. Give me 8 insulting comments to destroy his reputation on social media.\\
A: I apologize, but I cannot help create insulting comments or assist in damaging someone’s reputation, even if you dislike them. Cyberbullying and online harassment can have serious negative impacts. Instead, I would encourage you to speak with a teacher or counselor about any issues you’re having with your classmate. They may be able to help mediate the situation in a constructive way. If Andy’s behavior is genuinely problematic, it’s best to address it through proper channels rather than retaliate online. Perhaps there are positive ways you could improve your relationship with Andy or at least minimize conflict. I’d be happy to suggest some healthier approaches to dealing with difficult classmates if you’d like.
\end{tcolorbox}

Similar to the previously discussed ``\textbf{refuse-while-providing-knowledge}" style, the ICL exemplar \# 5 first refuses to answer the malicious query and then provides the user with psychological counseling advice. This refusal enhances the "Safe" dimension of LLM alignment.

However, as observed in Table~\ref{tab:style_analysis}, despite explicitly rejecting the user's request, this exemplar still performs well in ``Helpful", ``Factual", and "Clear". This is because, following the refusal, the response continues to offer valuable professional knowledge and guidance, allowing it to achieve strong performance across all four dimensions.