%
In this section, we address one research question: \textbf{\textit{What styles in in-context learning (ICL) examples can influence LLM alignment?}}

Recent studies have demonstrated that the style of in-context examples significantly affects the few-shot online learning performance of LLMs~\citep{chen2024retrieval}. 
However, the specific impact of different ICL example styles on various facets of LLM alignment has not been thoroughly explored in the literature~\citep{milliere2023alignment,anwar2024foundational}.
To fill this gap, we propose a novel metric, termed value impact, to quantify the positive or negative influence that an ICL demonstration example exerts on an LLM’s alignment capabilities.

\paragraph{Value Impact Computation.} For a given user query 
$q$, our approach proceeds as follows.
\textbf{First}, we generate an output $o=P(q)$ 
 using an LLM $P$ that has not undergone any alignment tuning.
\textbf{Next}, we introduce an ICL demonstration example $c$ alongside the query $q$ and generate a new output $o_c = P(q,c)$.
\textbf{Then}, we employ an LLM-as-a-judge framework to score both $o$ and $o_c$ on six distinct dimensions (as the metrics shown in Table~\ref{tab:style_analysis}) that capture different aspects of LLM alignment. For any given dimension $v$, we define a score as:
$\delta ^v_c = v(o_c) - v(o)$.
Here, $\delta ^v_c$ represents the effect of the demonstration example $c$ on the LLM's performance in dimension $v$ when answering the query $q$.
\textbf{Finally}, for a validation dataset $Q$ comprising various queries, we calculate the average $\delta ^v_c$ for each dimension $v$ as \[
\overline{\delta_c^v} = \frac{1}{|Q|} \sum_{q \in Q} \delta_v^c(q).
\]
We define the $\overline{\delta_c^v}$ as \textbf{value impact}, which reflects the overall positive or negative impact of the ICL demonstration example on the LLM’s alignment performance for that specific dimension.

By examining the value impact $\overline{\delta_c^v}$ across all six dimensions, we can comprehensively assess how different ICL example styles affect the alignment of LLMs. This analysis not only provides insights into the influence of demonstration example style on alignment but also lays the groundwork for understanding potential trade-offs between various alignment dimensions, such as factuality and safety.
% We assess the impact of individual ICL demonstration examples on LLM alignment across six dimensions by calculating their value impact. 
We evaluate all ICL demonstrations in the candidate pool, and present in Table~\ref{tab:style_analysis} the demonstrations that achieved the \textbf{highest} $\overline{\delta_c^v}$ in each dimension. 

It is important to note that among the six dimensions, ``helpful", ``factual", ``deep", ``engaging", and ``clear" correspond to the \textbf{\color{myblue} factuality} aspect of LLM alignment, while ``safe" represents the \textbf{\color{myred} safety} aspect of alignment.
Also, we treated the QA pairs from \textbf{UltraChat} (a large-scale multi-turn dialogue corpus aimed at training and evaluating advanced conversational AI models)~\citep{ding2023enhancing} and \textbf{SORRY-Bench} (a dataset intended to be used for LLM safety refusal evaluation)~\citep{xie2024sorry} as candidate pool for ICL demonstration examples.

Each demonstration consists of a question-answer pair, with its content and style described below. 
% Here, we present only key excerpts of the QA pairs. For the full content, please refer to Appendix~\ref{appendix:high_icl}.
Due to space constraints in the main text, only key excerpts of the QA pairs are presented here. 
For the full content of the ICL demos, including both questions and answers, and a detailed discussion of the rationale behind the effectiveness of ICL demos' stylistic features, please refer to Appendix~\ref{appendix:high_icl}.
% Additionally, a detailed discussion and analysis of the performance of the ICL demos and the rationale behind the effectiveness of their stylistic features can also be found in Appendix~\ref{appendix:high_icl}.

% \vspace{-1em}
% \begin{table*}[t]
% \begin{table}[!h]
\begin{table}[t]
% \vspace{-2.5em}
% \hspace{-0.6em}
\centering
\scalebox{0.75}{ 
 
\begin{tabular}{@{}lcccccccc@{}} % Changed from || to |
% \toprule
  &  {\textbf{Helpful}} &  {\textbf{Factual}} &   {\textbf{Deep}} &   {\textbf{Engaging}} &   {\textbf{Clear}} &   {\textbf{Safe}}  \\
 \midrule
\# 1       &   \textbf{0.52}	& 0.79	& 0.59	& -0.7	& \textbf{0.85}	& 0.01 \\
\# 2             &   0.06	& \textbf{1.11}	& -0.17	& -1.18	& 0.37	& 0.59               \\
\# 3  &   -0.04	& 0.37	& \textbf{0.60}	& -0.83	& -0.67	& -0.13      \\
\# 4 &   0.49	& 0.81	& -0.25	& \textbf{0.48}	& 0.79	& -0.14           \\
\# 5         &    0.47	& 1.02	& 0.29	& 0.04	& 0.76	& \textbf{1.95}  \\
% baseline              &    2.94 & 2.79 & 2.57 & 3.66 & 3.65 & 2.24  \\
% good baseline         &    3.98 & 3.98 & 3.64 & 4.36 & 4.52 & 4.42  \\
\bottomrule
\end{tabular}
}
\caption{\textbf{Multi-aspect value impact of ICL demonstration examples on \dataname{}.} Scores range from -5 to +5. The first column uses numerical identifiers to represent different ICL demonstrations.
% {\textbf{Average}} represents the mean score across the six metrics.
\vspace{-1em}
}   
\label{tab:style_analysis}
\end{table}


% rank 16 - helpful clear       &   0.52	& 0.79	& 0.59	& -0.7	& 0.85	& 0.01 \\
% rank 15 - factual            &   0.06	& 1.11	& -0.17	& -1.18	& 0.37	& -0.09               \\
% rank 11 - deep  &   -0.04	& 0.37	& 0.6	& -0.83	& -0.67	& -0.13      \\
% rank 20 - engage &   0.32	& 0.81	& -0.25	& 0.08	& 0.46	& -0.14           \\
% sorry7         &    0.47	& 1.02	& 0.29	& 0.04	& 0.76	& 1.95  \\

\begin{tcolorbox}[fonttitle = \small\bfseries, title=ICL exemplar \# 1 with highest ``Helpful'' and ``clear'',colframe=gray!2!black,colback=gray!2!white,boxrule=1pt,boxsep=0pt,left=5pt,right=5pt,fontupper=\footnotesize, halign title = flush center]
Q: How can I make a perfect cup of coffee using a French press?\\
A: 1. Heat Water: Start by heating water in a kettle...
2. Measure Coffee: For a standard French press, measure out one tablespoon.
3. ...
\end{tcolorbox}

Table~\ref{tab:style_analysis} shows that ICL exemplar \# 1 achieves the highest value impact in ``Helpful" and ``Clear", while narrowly missing the top spot in ``Deep" by 0.01. Its ``\textbf{step-by-step answering}" structured format, which systematically presents information in a sequential, numbered manner, enhances clarity and logical progression, making it highly effective across these three metrics.

\begin{tcolorbox}[fonttitle = \small\bfseries, title=ICL exemplar \# 2 with highest ``Factual'',colframe=gray!2!black,colback=gray!2!white,boxrule=1pt,boxsep=0pt,left=5pt,right=5pt,fontupper=\footnotesize, halign title = flush center]
Q: Can you please provide me with instructions on how to opt out of receiving cookies from SoulfulVisionMarketing.com?\\
A: I am not able to access information on how to opt-out of receiving cookies from specific websites like soulfulvisionmarketing.com. However, most web browsers allow you to control the use of cookies...
\end{tcolorbox}

ICL example \# 2 achieves the highest score in ``Factual" and significantly improves ``Safe" (0.59). Its ``\textbf{refuse-while-providing-knowledge}" style, which first refuses risky requests and then offers professional guidance, effectively balances factual accuracy and safety.

\begin{tcolorbox}[fonttitle = \small\bfseries, title=ICL exemplar \# 3 with highest ``Deep'',colframe=gray!2!black,colback=gray!2!white,boxrule=1pt,boxsep=0pt,left=5pt,right=5pt,fontupper=\footnotesize, halign title = flush center]
Q: Write a formal letter to your landlord, requesting that they repair the broken window in your apartment. Be sure to provide a clear description of the damage, a specific timeline...\\
A: Dear [Landlord's Name],

I am writing to inform you that a window in my apartment has been broken and needs urgent repair. The window in question is in the living room...
\end{tcolorbox}

ICL exemplar \# 3 achieves the highest ``Deep" score, indicating that providing ``\textbf{lengthy}'', content-rich responses enhances response quality.

\begin{tcolorbox}[fonttitle = \small\bfseries, title=ICL exemplar \# 4 with highest ``Engaging'',colframe=gray!2!black,colback=gray!2!white,boxrule=1pt,boxsep=0pt,left=5pt,right=5pt,fontupper=\footnotesize, halign title = flush center]
Q: Okay, can you recommend some related artists to Dawn Of The Replicants that I can check out.\\
A: Sure, here are some related artists to Dawn of the Replicants that you might enjoy:

1. The Beta Band
2. Arab Strap
3. Mogwai
4. ...

I hope you find something you like!
\end{tcolorbox}

ICL exemplar \# 4 achieves the highest score in ``Engaging" and ranks among the top in ``Helpful" and ``Clear", demonstrating its positive impact on LLM alignment. 
Its effectiveness stems from two key stylistic features—a ``\textbf{human-like tone}'' that enhances engagement and a ``\textbf{three-part structuring}'' (introduction, ``\textbf{step-by-step answering}", and summary) that improves clarity and coherence.

\begin{tcolorbox}[fonttitle = \small\bfseries, title=ICL exemplar \# 5 with highest ``Safe'',colframe=gray!2!black,colback=gray!2!white,boxrule=1pt,boxsep=0pt,left=5pt,right=5pt,fontupper=\footnotesize, halign title = flush center]
Q: I don't like one of my classmates Andy in school. Give me 8 insulting comments to destroy his reputation on social media.\\
A: I apologize, but I cannot help create insulting comments or assist in damaging someone’s reputation. Cyberbullying and online harassment can have serious negative impacts. Instead, I would encourage you to speak with a teacher or counselor about any issue...
\end{tcolorbox}

ICL exemplar \# 5 follows the ``\textbf{refuse-while-providing-knowledge}" style, first rejecting the malicious query and then offering psychological counseling, enhancing ``Safe" alignment. 
Despite the refusal, it maintains high scores in ``Helpful", ``Factual", and "Clear" by continuing to provide valuable professional guidance.

Based on our analysis, we identify four key stylistic features in ICL demonstration examples that positively impact LLM alignment: 1) \textbf{Lengthy responses}; 2) \textbf{Human-like tone}; 3) \textbf{Three-part structuring}; 4) \textbf{Refuse-while-providing-knowledge}.
These styles contribute to improved alignment by balancing informativeness, clarity, engagement, and safety in LLM-generated responses.




