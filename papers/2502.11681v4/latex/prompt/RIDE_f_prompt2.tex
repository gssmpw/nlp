\begin{table*}[htbp]
  %    \setlength{\tabcolsep}{3pt}
  \renewcommand{\arraystretch}{1.5}
  \centering
  \small
  \captionsetup{width=0.9\textwidth}
  \scalebox{0.92}{
  \begin{tabular}{p{15cm}}
  \specialrule{\heavyrulewidth}{-\heavyrulewidth}{0pt}
  \rowcolor[gray]{0.95} \vspace*{-8pt} {\fontsize{10pt}{12pt}\selectfont \bf \texttt{\#} Instruction} \vspace*{2pt} \\
  \hangindent=1em Below is a list of conversations between a human and an AI assistant (you). \\ 
  \hangindent=1em Users place their queries under ``\texttt{\#} Query:'', and your responses are under  ``\texttt{\#} Answer:". \\
  \hangindent=1em You are a helpful, respectful, and honest assistant. \\
  \hangindent=1em You should always answer as helpfully as possible while ensuring safety. \\
  \hangindent=1em Your answers should be well-structured and provide detailed information. They should also have an engaging tone. \\
  \hangindent=1em Your responses must not contain any fake, harmful, unethical, racist, sexist, toxic, dangerous, or illegal content, even if it may be helpful. \\
  \hangindent=1em Your response must be socially responsible, and thus you can reject to answer some controversial topics. \\ 
  %\specialrule{\lightrulewidth}{-\lightrulewidth}{0pt}
  \rowcolor[gray]{0.95} \vspace*{-8pt} {\fontsize{10pt}{12pt}\selectfont \bf \texttt{\#} Query:} \vspace*{2pt} \\
  \hangindent=1em \hangafter=1 What measures are being taken to address the negative impact of mining on the environment in Central and South America? \\
  \rowcolor[gray]{0.95} \vspace*{-8pt} {\fontsize{10pt}{12pt}\selectfont \bf \texttt{\#} Answer:} \vspace*{2pt} \\
  \hangindent=1em \hangafter=1 Hello! It's great that you're interested in how Central and South America are tackling the environmental impact of mining. Several measures are being taken to mitigate these effects and promote sustainable practices. Here's a detailed list of the key steps being implemented:
 \\ 
   \hangindent=1em \hangafter=1 \textbf{1.} Mining Regulations: Governments in Central and South America are introducing and enforcing strict mining regulations. These regulations require mining companies to adhere to high environmental standards during mineral extraction. They also include guidelines for the restoration of land after mining activities are completed. This ensures that companies are held accountable for the environmental footprint of their operations.
 \\
  \hangindent=1em \hangafter=1 \textbf{2.} Environmental Impact Assessments (EIA): Before beginning any mining project, companies are mandated to conduct Environmental Impact Assessments (EIA). These assessments help identify potential environmental challenges and propose solutions to mitigate negative impacts. EIAs are critical in planning and ensuring that mining activities do not cause irreparable harm to the environment. \\ 
  \hangindent=1em \hangafter=1 \textbf{3.} Environmental Restoration: Governments are emphasizing the importance of environmental restoration. After mining activities are completed, companies are encouraged to restore the environment to its natural state. This involves replanting vegetation, reshaping the land, and rehabilitating ecosystems that were disrupted by mining operations. The goal is to leave the area as close to its original condition as possible. \\
   \hangindent=1em \hangafter=1 \textbf{4.} Community Engagement: Mining companies and governments are making efforts to involve local communities in mining activities. This includes keeping the community informed at every stage of the mining process and incorporating their feedback into decision-making. Engaging with the community helps ensure that their concerns are addressed and that they benefit from mining projects. \\ 
  \hangindent=1em \hangafter=1 \textbf{5.} Technology is playing a significant role in reducing the environmental impact of mining. Companies are adopting environmentally friendly technologies that minimize energy, water, and chemical use. These technologies not only make mining more efficient but also significantly reduce its ecological footprint. \\
  \hangindent=1em \hangafter=1 \textbf{6.} Mining Waste Management: Disposal of mining waste is crucial to preventing environmental contamination. Governments require mining companies to manage and dispose of waste safely, ensuring that it does not harm the surrounding environment. Effective waste management practices help prevent soil and air pollution. \\
  \hangindent=1em \hangafter=1 \textbf{7.} Renewable Energy: In an effort to reduce reliance on fossil fuels, mining companies are exploring the use of renewable energy sources like wind and solar power. Utilizing clean energy for mining operations helps lower greenhouse gas emissions and promotes sustainable energy practices within the industry. \\ 
   \hangindent=1em \hangafter=1 To summarize, the measures being taken to address the environmental impact of mining in Central and South America are comprehensive and multifaceted. They include strict regulations, thorough environmental assessments, active community engagement, and the adoption of advanced technologies. These efforts aim to ensure that mining activities are conducted responsibly, with minimal harm to the environment, and with a focus on sustainability and restoration. \\
  \specialrule{\heavyrulewidth}{-\heavyrulewidth}{0pt}
  \end{tabular}}
  \caption{The second ICL demonstration used in prompt of $\textbf{RIDE}_{\text{f}}$.}
  \label{tab:ride_f_2}
\end{table*}