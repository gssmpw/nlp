\begin{table*}[htbp]
  %    \setlength{\tabcolsep}{3pt}
  \renewcommand{\arraystretch}{1.5}
  \centering
  \small
  \captionsetup{width=0.9\textwidth}
  \scalebox{0.92}{
  \begin{tabular}{p{15cm}}
  \specialrule{\heavyrulewidth}{-\heavyrulewidth}{0pt}
  \rowcolor[gray]{0.95} \vspace*{-8pt} {\fontsize{10pt}{12pt}\selectfont \bf \texttt{\#} Instruction} \vspace*{2pt} \\
  \hangindent=1em Below is a list of conversations between a human and an AI assistant (you). \\ 
  \hangindent=1em Users place their queries under ``\texttt{\#} Query:'', and your responses are under  ``\texttt{\#} Answer:". \\
  \hangindent=1em You are a helpful, respectful, and honest assistant. \\
  \hangindent=1em You should always answer as helpfully as possible while ensuring safety. \\
  \hangindent=1em Your answers should be well-structured and provide detailed information. They should also have an engaging tone. \\
  \hangindent=1em Your responses must not contain any fake, harmful, unethical, racist, sexist, toxic, dangerous, or illegal content, even if it may be helpful. \\
  \hangindent=1em Your response must be socially responsible, and thus you can reject to answer some controversial topics. \\ 
  %\specialrule{\lightrulewidth}{-\lightrulewidth}{0pt}
  \rowcolor[gray]{0.95} \vspace*{-8pt} {\fontsize{10pt}{12pt}\selectfont \bf \texttt{\#} Query:} \vspace*{2pt} \\
  \hangindent=1em \hangafter=1 What measures are being taken to address the negative impact of mining on the environment in Central and South America? \\
  \rowcolor[gray]{0.95} \vspace*{-8pt} {\fontsize{10pt}{12pt}\selectfont \bf \texttt{\#} Answer:} \vspace*{2pt} \\
   \hangindent=1em \hangafter=1 \textbf{1.} Mining Regulations: Governments in Central and South America are introducing strict mining regulations that require mining companies to comply with environmental standards relating to the extraction of minerals. These regulations also lay down stipulations regarding the restoration of land after mining activities have been complete.
 \\
  \hangindent=1em \hangafter=1 \textbf{2.} Environmental Impact Assessments (EIA): Mining companies are required to do EIA before beginning the mining process. This assessment aids in identifying environmental challenges that may arise and suggests possible solutions. \\ 
  \hangindent=1em \hangafter=1 \textbf{3.} Environmental Restoration: Governments are encouraging companies to restore the Mining environment to its natural form after extraction activities have been completed. This process involves replanting vegetation and reshaping the surface. \\
   \hangindent=1em \hangafter=1 \textbf{4.} Community Engagement: Governments and mining companies are including the local community in their mining activities, keeping them informed at every stage, and incorporating feedback. \\ 
  \hangindent=1em \hangafter=1 \textbf{5.} Technology: Mining companies are incorporating environmentally friendly technologies that reduce mining's impact on the environment. Such technologies minimize the use of energy, water, and chemicals in the mining process. \\
  \hangindent=1em \hangafter=1 \textbf{6.} Mining Waste Management: Governments require Mining Companies to dispose of mining waste safely with no potential side effect on the environment. \\
  \hangindent=1em \hangafter=1 \textbf{7.} Renewable Energy: Mining companies are exploring the possibility of using clean energy sources such as wind and solar to minimize the use of fossil fuels during the mining process. \\
  \specialrule{\heavyrulewidth}{-\heavyrulewidth}{0pt}
  \end{tabular}}
  \caption{The second ICL demonstration example used in prompt of $\textbf{RIDE}_{\text{f}}$ but has not been changed style.}
  \label{tab:ride_f_2_origin}
\end{table*}