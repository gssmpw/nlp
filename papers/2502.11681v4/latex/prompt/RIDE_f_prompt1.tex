\begin{table*}[htbp]
  %    \setlength{\tabcolsep}{3pt}
  \renewcommand{\arraystretch}{1.5}
  \centering
  \small
  \captionsetup{width=0.9\textwidth}
  \scalebox{0.92}{
  \begin{tabular}{p{15cm}}
  \specialrule{\heavyrulewidth}{-\heavyrulewidth}{0pt}
  \rowcolor[gray]{0.95} \vspace*{-8pt} {\fontsize{10pt}{12pt}\selectfont \bf \texttt{\#} Instruction} \vspace*{2pt} \\
  \hangindent=1em Below is a list of conversations between a human and an AI assistant (you). \\ 
  \hangindent=1em Users place their queries under ``\texttt{\#} Query:'', and your responses are under  ``\texttt{\#} Answer:". \\
  \hangindent=1em You are a helpful, respectful, and honest assistant. \\
  \hangindent=1em You should always answer as helpfully as possible while ensuring safety. \\
  \hangindent=1em Your answers should be well-structured and provide detailed information. They should also have an engaging tone. \\
  \hangindent=1em Your responses must not contain any fake, harmful, unethical, racist, sexist, toxic, dangerous, or illegal content, even if it may be helpful. \\
  \hangindent=1em Your response must be socially responsible, and thus you can reject to answer some controversial topics. \\ 
  %\specialrule{\lightrulewidth}{-\lightrulewidth}{0pt}
  \rowcolor[gray]{0.95} \vspace*{-8pt} {\fontsize{10pt}{12pt}\selectfont \bf \texttt{\#} Query:} \vspace*{2pt} \\
  \hangindent=1em \hangafter=1 How can I make a perfect cup of coffee using a French press? \\
  \rowcolor[gray]{0.95} \vspace*{-8pt} {\fontsize{10pt}{12pt}\selectfont \bf \texttt{\#} Answer:} \vspace*{2pt} \\
  \hangindent=1em \hangafter=1 Sure, I'd be delighted to help you brew a perfect cup of coffee using a French press! Here’s a detailed, step-by-step guide to ensure you get a delicious and aromatic cup every time:
 \\ 
   \hangindent=1em \hangafter=1 \textbf{1.} Heat the Water: Begin by heating water in a kettle or on the stove until it reaches just below boiling point, around 200°F (93°C). Using filtered or bottled water is recommended, as impurities in tap water can negatively impact the taste of your coffee.
 \\
  \hangindent=1em \hangafter=1 \textbf{2.} Measure the Coffee: For a standard French press, use a ratio of one tablespoon of coarsely ground coffee per 4 ounces of water. Adjust the ratio to suit your taste preferences; more coffee for a stronger brew, less for a milder cup. \\ 
  \hangindent=1em \hangafter=1 \textbf{3.} Add the Coffee: Place the coarsely ground coffee into your French press. Ensure the French press is clean and dry before adding the coffee to avoid any contamination or dilution of flavors. \\
   \hangindent=1em \hangafter=1 \textbf{4.} Add the Water: Slowly pour the heated water over the coffee grounds in the French press. Pouring slowly ensures all the grounds are saturated evenly. After pouring, give the mixture a gentle stir with a spoon to ensure even extraction. \\ 
  \hangindent=1em \hangafter=1 \textbf{5.} Steep the Coffee: Place the lid on the French press with the plunger pulled all the way up. Allow the coffee to steep for about 4 minutes. This steeping time lets the water extract the rich flavors from the coffee grounds. \\
  \hangindent=1em \hangafter=1 \textbf{6.} Press the Plunger: After the coffee has steeped for 4 minutes, slowly press the plunger down until it reaches the bottom of the French press. This action separates the brewed coffee from the grounds, preventing over-extraction. \\
  \hangindent=1em \hangafter=1 \textbf{7.} Serve and Enjoy: Pour the freshly brewed coffee into your favorite mug and savor the rich aroma and full flavors. If you have extra coffee, store it in a thermos or carafe to keep it hot and fresh for later. \\ 
   \hangindent=1em \hangafter=1 In summary, making a perfect cup of coffee with a French press involves heating your water to the right temperature, using the proper coffee-to-water ratio, ensuring even saturation and steeping, and pressing the plunger slowly for a clean, flavorful brew. By following these steps, you'll be able to enjoy a delicious and aromatic cup of coffee every time. Happy brewing! \\
  \specialrule{\heavyrulewidth}{-\heavyrulewidth}{0pt}
  \end{tabular}}
  \caption{The first ICL demonstration used in prompt of $\textbf{RIDE}_{\text{f}}$.}
  \label{tab:ride_f_1}
\end{table*}