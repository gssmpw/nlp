\section{RELATED WORK}
\subsection{Reinforcement Learning Locomotion}
RL has been successfully used to generate robust locomotion policies for quadrupedal robots~\cite{anymal_terrain, anymal_perceptive, Choi2023-cf, Wu2023-nz}. 
All of the previously mentioned works focus on robustness but ignore the user-interaction aspect of the locomotion policy, namely the amount of noise the robot generates when moving around. 
% The resulting policies are often loud and stompy, mainly because they do not prioritize quietness. 
Even locomotion policies designed to be energy efficient do not inadvertently reduce the amount of noise they produce, as seen in the work of Yang \textit{et al.}~\cite{energyefficient_learning_locomotion}. 
Other works on energy efficient gaits~\cite{energyefficient_amp_locomotion} or imitation learning-based gaits do not address or evaluate the impact noise of the learned policy~\cite{fuchioka2023opt, reske2021imitation}. 
While many of these works show promising results for robustness and efficiency, they are not designed to suppress loudness, which we aim to solve.

\subsection{Sound Engineering in Robotics} Sound plays a crucial role in how humans perceive and interact with various products.
Extensive engineering efforts have been made in the automotive industry~\cite{parizet2008analysis} to optimize the acoustic experience for drivers. Recently, the robotics industry has also recognized the importance of sound engineering, with research focusing on robotic arms~\cite{tennent2017good, trovato2018sound}, bipedal robots~\cite{stelthwalking}, rolling robots~\cite{quietwalkingrobot}, and servomotors~\cite{servo_sound}. These studies have investigated how the generated sounds influence human comfort and perception during human-robot interactions, leading to the development of more pleasant and even cute motion patterns~\cite{arm_motion_sound_study}. 
However, previous works have not thoroughly explored the acoustic impact of quadruped robot footsteps. 
This paper aims to address this gap by focusing on evaluating and reducing the sound generated by the footsteps of quadruped robots. 
