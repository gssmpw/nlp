\section{CONCLUSION}
In this work, we propose quiet walking based on a sim-to-real \ac{rl} approach to minimize foot contact velocity, which is highly correlated with footstep sound for a small home robot. Our framework consists of three key elements: adaptive PD gains, utilization of foot contact sensors, and implementation of curriculum learning. Through ablation studies, we demonstrate that each element contributes to minimizing the footstep sound for achieving a quieter walk. From the perspective of sound magnitude in the audible range from 20 Hz to 20 kHz, our approach achieves the quietest locomotion controller compared to Sony's commercial locomotion controllers and our own \ac{rl} baseline. We also identify a trade-off between quietness and robustness. 

Our work is the first to demonstrate and highlight key factors needed for the sim-to-real based \ac{rl} to achieve quiet walking. These findings have important implications for future research in human-robot interaction and the commercialization of quadruped robots for home use. 

% Our work is the first to demonstrate and highlight key factors necessary for an \ac{rl} based policy to achieve a quieter gait in quadruped robots. These findings have important implications for future research in human-robot interaction and the commercialization of quadruped robots for home use. By addressing the issue of noise generation during locomotion, our approach paves the way for more socially acceptable and less disruptive robotic companions in home environments.

