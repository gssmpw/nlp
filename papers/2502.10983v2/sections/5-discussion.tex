\section{DISCUSSION}
While our proposed \ac{rl} policy demonstrates improved quietness, its robustness is not as high as the baseline \ac{rl} policy, resulting in reduced adaptability to unknown terrains. We have shown that the selection of \ac{dr} parameters, such as ground friction and terrain height, can influence the trade-off between robustness and quietness of the policy. Future work could incorporate the approach of Chen et al.~\cite{chen2024identifying}, which utilizes perception information to estimate ground characteristics like friction, height, and damping. This integration could enable the selection of an appropriate policy based on these factors, potentially allowing the robot to traverse slippery surfaces and employ robust walking strategies on various terrains.

Rather than directly minimizing footstep sound, which is challenging to emulate in physics simulator, we proposed sim-to-real \ac{rl} framework to minimize foot contact velocity, which correlates with footstep sound. This approach could be extended to other applications where real-world parameters are difficult to emulate in the typical locomotion simulator. For instance, battery life optimization could be addressed by minimizing torque in the simulator, as torque correlates with energy consumption. 

In this study, we focused primarily on reducing footstep sound during locomotion. However, other sources of noise during walking, such as actuator movement and mechanical friction, warrant further investigation. Although these sounds are not as dominant as footstep noise, addressing them remains crucial for developing quieter home-legged robots.