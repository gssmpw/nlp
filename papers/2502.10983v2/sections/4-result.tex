\section{Experiments and Results}
We evaluate the quietness by using the microphone in the robot depending on the measured base velocity. By showing that the selective reward such as the foot contact velocity, base angular acceleration and joint angular acceleration at the proposed policy are smaller than the baseline policy, we prove those observable parameters in a physics simulator correlate to the footstep sound in the real. As part of our ablation study, we demonstrate the trade-off between robustness and quietness by validating three key factors of our framework and the effectiveness of \ac{dr}.  
\begin{figure}[!t]
    \centering
    \vspace{-20pt}
    \includegraphics[trim={0 0 0 0}, width=\linewidth]
    % \includegraphics[trim={0 6ex 0 0}, width=\linewidth]
    {figures/sound_magnitude_depending_velocity.pdf}
    %\includegraphics[width=\linewidth]{figures/sound_magnitude_depending_velocity.png}
    \caption{Comparison of the average sound magnitude between $20$ Hz and $20$ kHz for the measured base velocity of aibo. The baseline and propose are RL policies without and with our three key factors for each. Sony normal and Sony quiet are the commercial controllers provided by Sony.}
    \label{fig:sound_eval_vel}
    \vspace{-2ex}
\end{figure}

\subsection{Quietness Evaluation}
\label{subsec:quietness_eval}
We evaluate the performance of our proposed quiet \ac{rl} walking policy and compare it to three other methods: a baseline \ac{rl} policy and two commercial locomotion controllers provided by Sony. 
The baseline \ac{rl} policy is trained using the standard rewards from the previous work~\cite{legged_gym} which we tuned to adapt to the small-scale robot, aibo. Unlike our proposed policy, the baseline \ac{rl} policy does not output PD gain scale, does not include rewards for noisy walking penalties, and does not employ curriculum learning during training. 
Sony's two baseline controllers are designed for normal walking and quiet walking, respectively. 
As these are proprietary algorithms, we compare our method to them without knowledge of their internal workings.

To evaluate the sound level of each locomotion policy, we use one of the microphones on the rear side of aibo's head, which has a sampling frequency of $48$ kHz. We focus our analysis on the standard human audible range between $20$ Hz and $20$ kHz~\cite{neuroscience}, as this range is most relevant for assessing perceived noise levels. In this experiment, all locomotion controllers are set to track the same base velocity moving forward for the comparison. We preprocess the data using a Hamming window and then use Welch's method~\cite{welch1967use} to perform the frequency analysis with \acp{fft}. We use a window size of $4096$ samples corresponding to approximately $85$ milliseconds. For the power analysis, we use as a reference a $10^{-10}$ amplitude sinusoid for each frequency bin.

The sound level of walking increases with higher commanded base velocities, as the robot has less time to slow its feet before contact while maintaining the commanded velocity. Therefore, we experimented to measure the average sound level for each policy at different measured linear velocity commands. It is important to note that the microphone is approximately $10$ cm away from the feet, whereas the sound level would be much lower for a human standing further away. We only use the human audible frequency range and a $1$ kHz sinusoidal reference signal with an amplitude of $1.0$. \Figref{fig:sound_eval_vel} shows that the quietness of proposed method outperforms the other methods such as the baselines and Sony commercial controllers at different measured velocities. 
% We achieve an average of $10.9$ dB noise reduction with respect to the baseline \ac{rl} policy. We also achieve a $1.73$ dB reduction in noise compared to the best baseline, i.e. Sony's quiet walk, which represents a substantial improvement in quietness, equivalent to a $49$\% reduction.

\begin{table}[!t]
\caption{Analysis of Noisy Walking Penalties at Simulator}
%  for \\ Validating the Correlation of Footstep Sound at Real
\label{table:reward_analysis}
\begin{center}
\begin{tabular}{cccc}
\toprule
\textbf{Physical Term} & \textbf{Unit} & \textbf{Baseline} & \textbf{Proposed} \\
\midrule
Foot contact velocity & $\rm{m/s}$ & 0.417 & 0.123 \\
Joint acceleration & $\rm{rad/s^2}$ & 114.3 & 76.7 \\
Base angular acceleration & $\rm{rad/s^2}$ & 57.2 & 23.7\\
\bottomrule
\end{tabular}
\end{center}
\end{table}

\subsection{Analysis of Noisy Walking Penalties Rewards}
To validate our approach of minimizing noisy walking penalties in the simulator for reducing aibo's footstep sound in the real world, as illustrated in \figref{fig:quiet_walking_concept}, we compared the noisy walking penalties between the baseline and proposed policies. 
~\tabref{table:reward_analysis} presents the average values for 10 seconds calculated using the equations in Table~\ref{table:reward} in the Isaac Gym simulator.
The policies used for the baseline and proposed are those employed in \figref{fig:sound_eval_vel}. 
As shown in \tabref{table:reward_analysis}, our analysis reveals that the proposed policy, which produces quieter walking compared to the baseline, exhibits lower values across all observed parameters. As mentioned in subsection \ref{subsec:quietness_eval}, the results indicate that the footstep sounds generated by the proposed policy are quieter than those of the baseline.
This finding suggests a correlation between the real-world footstep sound and the noisy walking penalties in the simulator. Consequently, these penalties such as foot contact velocity, joint acceleration, and base angular acceleration effectively enable aibo to walk quietly in the real world through our sim-to-real transfer approach.

\begin{figure}
    \centering
    % \vspace{-5pt}
    \includegraphics[trim={0 0 0 0}, width=\linewidth]{figures/tradeoff.pdf}
    \caption{Trade-off between robustness and quietness are shown. The experiment evaluates robustness by having aibo climb a slope. Sound magnitude is calculated using the same method as in \figref{fig:sound_eval_vel}. The controllers such as \ac{rl} baseline, ablation test, SONY commercial controllers, and \ac{rl} proposed are shown.}
    \label{fig:tradeoff}
    \vspace{-2ex}
\end{figure}

\subsection{Trade-off Between Robustness and Quietness}
To assess robustness, we designed an experiment where aibo had to traverse a 0.5 m long slope within 20 seconds. We varied the slope angle and used this value as a metric to compare robustness, assuming that adaptability to unknown environments is one of the important aspect of robustness. We measured sound magnitude within the audible range using the method described in subsection \ref{subsec:quietness_eval}. We prepared policies including our proposed method, baseline, Sony commercial controllers, and ablation study conditions for the framework, which consists of three key factors.

\Figref{fig:tradeoff} shows that the loudest baseline policy can overcome the steepest slope of 7 degrees. Although our proposed method is the quietest among all settings, its robustness is the lowest. This observation suggests a trade-off between quietness and robustness, as quieter policies struggle to overcome steeper slopes.

We conducted an ablation study by training aibo in eight different settings. First, we evaluated policies without curriculum learning, using parameter scales from the noisy walking and quiet walking phases in \tabref{table:reward}. As shown in \figref{fig:tradeoff} (No Curriculum), only using the noisy walking phase parameters without curriculum achieves about half of the noise reduction compared to the proposed method. Using the quiet walking phase parameters in Table~\ref{table:reward} without curriculum, the training fails to converge. In other words, aibo doesn't learn to walk, due to too severe penalties for quiet walking. 

As a next ablation study, We tested the impact of removing switch contact sensors. Although the training converges, aibo doesn't walk at all. At the beginning of the quiet walking phase, aibo falls frequently, ending episodes early. We hypothesize that this is due to the lack of information about when to stiffen joints to support the whole body. Consequently, aibo learns to avoid walking to satisfy the strong noisy walking penalties.

As a third ablation study, a comparable result in terms of robustness is obtained using our proposed method without allowing the network to change the PD gains, as shown in \figref{fig:tradeoff} (Fixed PD gain). While this demonstrates that the noisy walking penalties and curriculum can reduce noise, the sound magnitude differs from our proposed method. PD gain scale change is beneficial for quiet walking, as mentioned in previous work \cite{bogdanovic2020learning}, because quiet walking is also a sensitive contact task.

Finally, we expanded our training of the proposed method by incorporating additional domain randomization parameters for ground characteristics, specifically the friction coefficient and terrain height. When modifying the friction coefficient range, increasing the maximum from 0.7 to 0.9 and decreasing the minimum from 0.4 to 0.2, we observed a relatively small sound magnitude but enhanced robustness compared to the originally proposed method as shown in \figref{fig:tradeoff} as More DR (terrain friction). Conversely, increasing the maximum terrain height from 0.01 to 0.03 resulted in the third-lowest sound magnitude while maintaining robustness comparable to the baseline as shown in \figref{fig:tradeoff} as More DR (terrain height). These findings suggest that the selection of domain randomization parameters enables tuning of the trade-off between robustness and quietness.

\begin{figure}
    \centering
    \vspace{2pt}
    \includegraphics[trim={0 0 0 0}, width=\linewidth]{figures/gain_scale_change.pdf}
    \caption{PD gain scale of $\sigma(x)$ in the \eqnref{eq:pd_gain_scale} at the fore right leg during locomotion. Right foreleg is (A): in the air to move forward, (B): approaching contact with the ground, (C): in contact to support the ensuing movement of the other legs.  The plot at the bottom shows the edge of the foot velocity. The blue area in the plot shows when the right foreleg is in contact with the ground, based on the foot switch contact sensor.}
    \label{fig:gain_scale_change}
    \vspace{-2ex}
\end{figure}

\subsection{PD Gain Scale Analysis}
\Figref{fig:gain_scale_change} illustrates the changes in PD gain scales and the edge of foot velocity exhibited by the quiet walking policy for the three actuators on the right foreleg. 
The plot of PD gain scale shows the scaled $\sigma(x_i)$ value outputted by the network, which is later scaled according to Equation~\ref{eq:pd_gain_scale}. 

As shown in the middle plot of \figref{fig:gain_scale_change}, just before the foot makes contact with the ground, the gains for all the actuators of the right foreleg decrease, effectively dampening the motion. This drop in gain scales at the shoulder pitch joint is followed by an increase, supporting the ensuing movement of the other legs. This behavior suggests that the policy learns to walk quietly by adjusting the PD gains to dampen the joint at the moment of ground contact.

The foot velocity, calculated through forward kinematics using measurement data of the three joint angles for the right foreleg, is shown in \figref{fig:gain_scale_change}. The foot velocity decreases just before contact with the ground. Conversely, when the foot is in the air, the velocity of the foot increases, along with the PD gains for shoulder roll and ankle pitch. This indicates that the policy actively controls the PD gains to maintain tracking of the base velocity.

% \begin{figure}
%     \centering
%     \vspace{-20pt}
%     \includegraphics[trim={0 0 0 0}, width=\linewidth]{figures/tradeoff.pdf}
%     \caption{UPDATING JUST NOW!!: Tradeoff between robustness and quietness are shown. By using the same policy as. }
%     \label{fig:gain_scale_change}
%     % \vspace{-2ex}
% \end{figure}

% \begin{table*}
% \caption{Ablation study for the different components of the quiet walking training, showing they are all necessary. \\ Additionally, \acf{dr} can be used to tune the tradeoff between robustness and quietness.}
% \label{table:ablation_study}
% \begin{center}
% %\begin{tabular}{lllcclccl}
% %\toprule
% %\textbf{} & \textbf{baseline} & \textbf{noisy walking} & \textbf{quiet walking} & \textbf{without switch} & \textbf{without pd} & \textbf{more \ac{dr}} & \textbf{more \ac{dr}} & \textbf{proposed} \\
% % & & \textbf{no curriculum} & \textbf{no curriculum} & \textbf{contact sensor} & \textbf{gain change} & \textbf{terrain friction} & \textbf{terrain height}\\
% %\midrule
% %sound magnitude & $35.9$ $\rm{dB}$ & $30.4$ $\rm{dB}$ & not convergent & not walk & $31.8$ $\rm{dB}$ & $28.9$ & $28.4$ $\rm{dB}$ & $22.7$ $\rm{dB}$ \\
% %slope angle & $7.0$ $\rm{deg}$ & $5.0$ $\rm{deg}$ & at training & at all & $7.0$ $\rm{deg}$ & $5.0$ $\rm{deg}$ & $7.0$ $\rm{deg}$ & $3.0$ $\rm{deg}$\\
% \begin{tabular}{ccccccccc}
% \toprule
% \textbf{} & \textbf{Baseline} & \textbf{Noisy walking} & \textbf{Quiet walking} & \textbf{No contact} & \textbf{Fixed PD} & \textbf{More \ac{dr}} & \textbf{More \ac{dr}} & \textbf{Proposed} \\
%  & & (\textbf{no curriculum}) & (\textbf{no curriculum}) & \textbf{sensor} & \textbf{gains} & (\textbf{terrain friction}) & (\textbf{terrain height})\\
% \midrule
% Sound [$\rm{dB}$] & $35.9$ & $30.4$ & Does not & Does not & $31.8$ & $28.9$ & $28.4$ & $22.7$ \\
% slope [$\rm{deg}$] & $7.0$ & $5.0$ & converge & walk & $7.0$ & $5.0$ & $7.0$ & $3.0$ \\
% \bottomrule
% \end{tabular}
% \end{center}
% \end{table*}

% To demonstrate the effectiveness of the Noisy Walking Penalties in reducing the footstep sound, which consists of joint acceleration, base angular acceleration, and foot contact velocity, we compare the baseline and proposed policies with the same command velocity on the real robot. Table~\ref{table:reward_analysis} presents these observations, calculated using the equations in Table~\ref{table:reward} and averaged over a sequence.
% The foot contact velocity is determined by utilizing the timing information provided by the foot switch contact sensors. The joint acceleration and base angular acceleration are computed by differentiating the joint encoder and gyro sensor measurements on the hardware. Forward kinematics is then applied to the joint encoder measurements to calculate the foot positions, which are subsequently differentiated to obtain their respective velocities. 
% The analysis reveals that the proposed policy, which generates quieter walking compared to the baseline, exhibits reduced values for all the aforementioned observations. This finding suggests a correlation between the footstep sound and the noisy walking penalties, indicating that these penalties effectively enable aibo to walk quietly in the real world.

% \begin{table*}
% \caption{Ablation Study}
% \label{table:ablation_study}
% \begin{center}
% \begin{tabular}{lllllcccl}
% \toprule
% \textbf{} & \textbf{baseline} & \textbf{without pd} & \textbf{noisy scales} & \textbf{more domain} & \textbf{quiet scales} & \textbf{without switch} & \textbf{separate pd} & \textbf{proposed} \\
%  & & \textbf{gain change} & \textbf{no curriculum} & \textbf{randomization} & \textbf{no curriculum} & \textbf{contact sensor} & \textbf{gain change}\\
% \midrule
% sound magnitude & $35.9$ $\rm{dB}$ & $0.0$ $\rm{dB}$ & $0.0$ $\rm{dB}$ & $0.0$ $\rm{dB}$ & not convergent & not walk & not convergent & $22.7$ $\rm{dB}$ \\
% slope angle & $7.0$ $\rm{deg}$ & $0.0$ $\rm{deg}$ & $0.0$ $\rm{deg}$ & $0.0$ $\rm{deg}$ & at training & at all & at training & $3.0$ $\rm{deg}$\\
% \bottomrule
% \end{tabular}
% \end{center}
% \end{table*}






