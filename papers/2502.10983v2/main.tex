%%%%%%%%%%%%%%%%%%%%%%%%%%%%%%%%%%%%%%%%%%%%%%%%%%%%%%%%%%%%%%%%%%%%%%%%%%%%%%%%
%2345678901234567890123456789012345678901234567890123456789012345678901234567890
%        1         2         3         4         5         6         7         8

\documentclass[letterpaper, 10 pt, conference]{ieeeconf}  % Comment this line out if you need a4paper


%\documentclass[a4paper, 10pt, conference]{ieeeconf}      % Use this line for a4 paper

\IEEEoverridecommandlockouts                              % This command is only needed if 
                                                          % you want to use the \thanks command

\overrideIEEEmargins                                      % Needed to meet printer requirements.

%In case you encounter the following error:
%Error 1010 The PDF file may be corrupt (unable to open PDF file) OR
%Error 1000 An error occurred while parsing a contents stream. Unable to analyze the PDF file.
%This is a known problem with pdfLaTeX conversion filter. The file cannot be opened with acrobat reader
%Please use one of the alternatives below to circumvent this error by uncommenting one or the other
%\pdfobjcompresslevel=0
%\pdfminorversion=4

% See the \addtolength command later in the file to balance the column lengths
% on the last page of the document

% The following packages can be found on http:\\www.ctan.org
%\usepackage{graphics} % for pdf, bitmapped graphics files
%\usepackage{epsfig} % for postscript graphics files
%\usepackage{mathptmx} % assumes new font selection scheme installed
%\usepackage{times} % assumes new font selection scheme installed
%\usepackage{amsmath} % assumes amsmath package installed
%\usepackage{amssymb}  % assumes amsmath package installed

\usepackage{graphics}
\usepackage{multirow}
% Needed to meet printer requirements.

% \overrideIEEEmargins                                      % Needed to meet printer requirements. Uncommented for RA-L

% The following packages can be found on http:\\www.ctan.org
\usepackage{graphics} % for pdf, bitmapped graphics files
\usepackage{epsfig} % for postscript graphics files
%\usepackage{mathptmx} % assumes new font selection scheme installed
\usepackage{times} % assumes new font selection scheme installed
\usepackage{amsmath} % assumes amsmath package installed
\usepackage{amssymb}  % assumes amsmath package installed

\usepackage{array}
\usepackage{booktabs}

\usepackage[hidelinks]{hyperref}
\usepackage{cite}

% Cross and Check mark
\usepackage{xcolor,pifont}
\usepackage{color}
\usepackage[nolist]{acronym}
\usepackage{amsmath}
\usepackage{multirow}
\usepackage{microtype}

\newcommand\Tstrut{\rule{0pt}{2.6ex}}         % = `top' strut
\newcommand\Bstrut{\rule[-0.9ex]{0pt}{0pt}}   % = `bottom' strut

%%% useful macros from chenhao's premables
\newcommand{\Figref}[1]{Figure~\ref{#1}}  % beginning of sentence
\newcommand{\figref}[1]{Fig.~\ref{#1}}    % somewhere
\newcommand{\Tabref}[1]{Table~\ref{#1}}
\newcommand{\tabref}[1]{Table~\ref{#1}}
\newcommand{\Eqnref}[1]{Equation~\ref{#1}}
\newcommand{\eqnref}[1]{Eq.~\ref{#1}} % Eq. (1)
\newcommand{\eqnpref}[1]{(Eq.~\ref{#1})} % (Eq. 1)
\newcommand{\Real}{\ensuremath{\mathbb R}}  

\DeclareMathOperator*{\MSE}{MSE}
\DeclareMathOperator*{\enc}{\textbf{enc}}
\DeclareMathOperator*{\dec}{\textbf{dec}}

\title{\LARGE \bf
DFM: Deep Fourier Mimic for Expressive Dance Motion Learning
% DFM: Deep Fourier Mimic for Expressive Dance Motion Learning \\ on Entertainment Robots
}

\pdfminorversion=5


\title{\LARGE \bf
Learning Quiet Walking for a Small Home Robot
%Quiet Walk Learning for Small Home Robot aibo
}

\author{Ryo Watanabe$^{1}$$^{,}$$^{2}$, Takahiro Miki$^{1}$, Fan Shi$^{1}$$^{,}$$^{3}$, Yuki Kadokawa$^{1}$$^{,}$$^{4}$, Filip Bjelonic$^{1}$, \\
Kento Kawaharazuka$^{1}$$^{,}$$^{5}$, Andrei Cramariuc$^{1}$ and Marco Hutter$^{1}$% <-this % stops a space
\thanks{$^{1}$Ryo Watanabe, Takahiro Miki, Fan Shi, Yuki Kadokawa, Filip Bjelonic, Kento Kawaharazuka, Andrei Cramariuc, and Marco Hutter are with the Robotic Systems Lab, Department of Mechanical Engineering, ETH Zurich, Switzerland.}
\thanks{$^{2}$Ryo Watanabe is at Sony Group Corporation, Japan}%
\thanks{$^{3}$Fan Shi is at National University of Singapore}%
\thanks{$^{4}$Yuki Kadokawa is at Nara Institute of Science and Technology, Japan}%
\thanks{$^{5}$Kento Kawaharazuka is at The University of Tokyo, Japan}%
}

\begin{document}

\maketitle
\thispagestyle{empty}
\pagestyle{empty}

%%%%%%%%%%%%%%%%%%%%%%%%%%%%%%%%%%%%%%%%%%%%%%%%%%%%%%%%%%%%%%%%%%%%%%%%%%%%%%%%
\begin{abstract}
As home robotics gains traction, robots are increasingly integrated into households, offering companionship and assistance. Quadruped robots, particularly those resembling dogs, have emerged as popular alternatives for traditional pets. However, user feedback highlights concerns about the noise these robots generate during walking at home, particularly the loud footstep sound.
To address this issue, we propose a sim-to-real
based \ac{rl} approach to minimize the foot contact velocity highly related to the footstep sound. Our framework incorporates three key elements: learning varying PD gains to actively dampen and stiffen each joint, utilizing foot contact sensors, and employing curriculum learning to gradually enforce penalties on foot contact velocity. Experiments demonstrate that our learned policy achieves superior quietness compared to a \ac{rl} baseline and the carefully handcrafted Sony commercial controllers. Furthermore, the trade-off between robustness and quietness is shown. This research contributes to developing quieter and more user-friendly robotic companions in home environments.
% This paper addresses this issue by introducing a reinforcement learning (RL) based locomotion policy learned in a simulator and successfully transferred to the real home robot, aibo. Our approach includes three key elements: learning varying PD gains, using foot contact sensors, and utilizing curriculum learning. Evaluation of the resulting locomotion controller shows that it achieves quieter locomotion than Sony's commercial locomotion and our RL baseline locomotion, as measured by sound magnitude within the audible range. This research opens avenues for quieter and thus more user-friendly robotic companions in home environments.
%Our contributions include the first implementation of a quiet walking policy in RL for quadrupedal robots, deployment on a real consumer-grade small platform, and superior performance compared to baseline RL and Sony commercial controllers. => repetitive. 
\end{abstract}

\section{Introduction}
\label{sec:intro}

Foundational models (FMs)~\cite{zhang2024data, zhou2023comprehensive} have shown remarkable progress in the healthcare domain, enabling professional-like assessment of disease diagnosis, treatment decision-making, and monitoring~\cite{zhang2023text, wang2022medclip, lu2023mi-zero}. 
Examples include LLaVA-Med~\cite{li2023llava}, Med-PaLM Multimodal~\cite{tu2024towards}, and Med-Flamingo~\cite{moor2023med}, have demonstrated their capacity on question answering, medical image analysis, and report generation.
These studies follow a predominant top-down model development strategy that requires upstream developers to collect data and train models for downstream tasks. 
Consequently, the developed model capabilities are heavily dependent on the training data, limiting their generalization performance in diverse clinical scenarios. 
For instance, Med-Gemini~\cite{yang2024advancing} reveals promising general capabilities in report generation while it lags behind state-of-the-art (SoTA) models on classification tasks, especially for out-of-domain applications. 
This indicates that while the generalizability of the foundation model is promising, more solutions are expected to meet the various specialized clinical needs.

To address these challenges, multi-center data centralization becomes essential to enhance model capacity and robustness across varied clinical scenarios~\cite{rajpurkar2022ai}. 
Centralizing distributed data can significantly improve model training and inference performance.
However, the process of medical data storage, transfer, and aggregation among centers requires extra efforts to ensure data security and system interoperability~\cite{bradford2020international}.
Moreover, a growing concern for patient privacy makes large-scale multi-center data sharing particularly challenging. 
While efforts like federated learning~\cite{wen2023survey, li2020review} can achieve good model performance on local data, the need for synchronized system coordination presents significant challenges, as clients are unable to update asynchronously. This limitation greatly restricts the practical capability of such approaches.
As a result, without a flexible collaboration, medical community still struggles to fully utilize the isolated data and local computation resources for comprehensive medical AI model development. 
To address this dilemma, open-source platforms encourage public data sharing and knowledge integration~\cite{markiewicz2021openneuro, zenodo}.
However, these platforms focus solely on raw data sharing while seldom providing collaborative model training or cooperation between different institutions.
Recently, collaborative learning has emerged as a viable approach for enhancing multi-model robustness~\cite{boulemtafes2020review}. 
For instance, software-like model development~\cite{raffel2023building} mimics software engineering practices by introducing structured workflows, enabling merging, version control, and continuous model integration.
Under this design, model ability can be strengthened with incremental knowledge updates similar to the version updating in software development. 

Although collaborative learning provides a multi-model collaboration, two key challenges remain in the leakage of raw data during collaboration~\cite{huang2023lorahub} and the synchronization of multiple collaborators~\cite{mcmahan2017communication} in the medical AI community. It is still challenging to integrate decentralized, privacy-sensitive data across institutions, leading to under-utilized insights and fragmented knowledge sharing~\cite{kaissis2020secure, rajpurkar2022ai, abdullah2021ethics}.
 To address these challenges, inspired by the collaborative software development, we propose \textbf{Med}ical \textbf{Fo}undation Models Me\textbf{rg}ing (\textbf{MedForge}), a cooperative workflow enabling continuously community-driven foundation model (FM) development.
MedForge enables a lightweight manner for individual centers to share their knowledge among multiple centers, minimizing the burden of data transmission and integration while enhancing model robustness.
Meanwhile, MedForge facilitates asynchronous and flexible collaboration, allowing individual centers to continuously update and improve medical FMs without the need for real-time synchronization.
Similar to open-source software development, MedForge incrementally updates medical knowledge and follows a sustainable model development scheme. 
This key design emphasizes a bottom-up construction of a multi-task medical FM, allowing downstream users to collaboratively build, refine, and update the upstream model according to their local resources. Our major contributions of MedForge are as below: 
\begin{enumerate}
    \item[$\bullet$] We introduce a collaborative workflow to promote the merging scheme of open-source software development. Our proposed MedForge allows distributed clinical centers to asynchronously contribute to comprehensive medical model construction while reducing transmitting costs among centers and avoiding the leakage of raw data, thus enhancing the utilization of private resources in the healthcare system. 
    \item[$\bullet$] We propose two effective knowledge-merging strategies for the asynchronous branch contribution. The MedForge-Fusion strategy updates the plugin module parameters of the main model during the merging phase, whereas the MedForge-Mixture strategy integrates the output of the plugin module by memorizing each contributor's coefficient. These strategies make MedForge more flexible and versatile. MedForge-Fusion is friendly to implement, while the MedForge-Mixture offers better performance and robustness.
    \item[$\bullet$]  We comprehensively evaluate model merging strategies to accumulate medical knowledge among multiple branch plugin modules. MedForge yields superior performance on medical classification tasks compared to other collaborative baselines across multiple datasets. We demonstrate the robustness of MedForge by shuffling the task order and evaluating various configurations of plugin modules and dataset distillation methods.
\end{enumerate}



\section{Related Work}
\label{sec:related}
\subsection{Collaborative Systems}
In the era of rapid growth in medical foundational models~\cite{huang2023visual,wang2022medclip, zhang2024data}, the top-down model development paradigm limits model capabilities by heavily relying on the resources available to the model builders. 
Such paradigm often restricts the potential of these models, as they cannot effectively utilize the diverse, private, and decentralized resources that exist within the broader medical community.
In contrast, collaborative systems present a promising alternative, offering a more flexible approach to model development.

Collaborative systems enable institutions to share knowledge by allowing distributed collaborators to contribute to a common goal~\cite{boulemtafes2020review}. 
To further protect patient privacy, federated learning (FL)~\cite{mcmahan2017communication} was proposed to alleviate such privacy concerns as server aggregating parameter updates from multiple clients without sharing their local data. 
While subsequent optimizations, such as aggregation algorithms~\cite{mcmahan2017communication, zhao2018federated, li2020federated}, secure learning~\cite{hardy2017private, xie2021crfl}, fairness improvements~\cite{sharma2022federated, zhao2022dynamic} and its application in medicine~\cite{kumar2024privacy}, have enhanced the capacity and applicability of FL, its real-world flexibility remains limited. This is primarily due to the need for synchronous updates, which require the server and clients to stay in sync, or model updates will be blocked.
This synchrony issue can be mitigated by open-source software platforms (e.g., GitHub~\cite{github}), allowing independent contributions from individual developers asynchronously. Such an asynchronous scheme enables faster iteration and the integration of specialized expertise, thus offering a more flexible and scalable approach.

Unlike synchronous collaboration, asynchronous collaboration does not require collaborators to work simultaneously and collaborators can individually complete their updates.
While the concept of asynchronous collaboration has been widely used in software development, its machine-learning applications remain limited~\cite{kandpal2023git, raffel2023building}. 
With the rise of global collaboration, large models~\cite{sahajBERT, le2023bloom} are usually co-developed by collaborators given various levels of data availability. However, this collaborative scheme requires the aggregation of local data and online synchronous cooperation of developers.
Software-like model update system~\cite{raffel2023building} alleviates the synchronous problem, where models are updated incrementally, similar to software development, by introducing merging and version control to model development.
However, the existing collaborative version control system~\cite{kandpal2023git} fails to address the complexities of medical scenarios because of the heavy dependency on plain parameter averaging across the full model without accounting for the varying requirements of different tasks.
To respond, we propose MedForge, which enables an asynchronous collaborative system and ensures strong robustness toward a continuous, community-driven enhancement of medical models while overcoming potential data leakage.

\begin{figure*}[t]
\begin{center}
\includegraphics[width=.85\linewidth]{fig_overview_v3.pdf}
\end{center}
\caption{
FastAtlas Overview: In each frame, we compute charts spanning fully or partially visible triangles (a), determine texture space bounding boxes for the visible portions of the view-space projections of each chart, and tightly pack these boxes into atlases (b, here $2K \times 2K$). We simultaneously bijectively parameterize and shade the charts into their atlas boxes, obtaining high quality texture space shading (c), and use this shading to render the shaded frames (d).}
\label{fig:overview}
\label{fig:alg_overview}
\end{figure*}

\section{Overview}
\label{sec:overview}
Our work has two core contributions: a real-time, GPU-based algorithm for tight packing of general parameterized charts into compact atlases; and a real-time TSS method that
utilizes this packing.  

\paragraph*{FastAtlas Packing.}
FastAtlas runs entirely on the GPU as a series of compute shaders. It takes the bounding boxes of parameterized charts as input, and packs them into an atlas (Fig~\ref{fig:overview}b, Sec.~\ref{sec:pack}). As such, the only input it requires are the dimensions of the bounding boxes.
Its outputs are deterministic; identical input charts are packed into identical atlases. This is critical for TSS and similar applications, as it ensures that consecutive frames taken from the same camera view have the same shading. Even minute shading differences across such frames can cause sampling jitter, leading to undesirable flicker \cite{baker2012rock}. 
While prior methods such as \cite{mueller2018shading,hladky2019tessellated,hladky2021snakebinning,Neff2022MSA} cap the dimensions of the charts that can be packed as-is for a given atlas size, and scale down all charts that exceed these dimensions, we scale all charts by the same factor, and do so only when strictly necessary to achieve packing success (Figs~\ref{fig:atlas},~\ref{fig:sas_issues}). 

\paragraph*{TSS using FastAtlas.}
Our end-to-end TSS atlas generation method combines the packing method above with a novel approach for computing seamless per-frame charts. 
We define our charts as the connected components of the visible surfaces in each frame (Fig.~\ref{fig:overview}a), and efficiently compute them using a parallel union-find algorithm (Sec.~\ref{sec:visible}). Since the boundaries of these charts coincide with the contours of the rendered surface, they are {\em invisible} to the viewer. This approach 
eliminates the artifacts caused by shading discontinuities along visible seams (Fig.~\ref{fig:seams}). 

\begin{parWithWrapFigure}
\begin{wrapfigure}{l}{.27\columnwidth}%
\includegraphics[width=\linewidth]{fig_inset_view_plane.pdf}%
\end{wrapfigure}
We bijectively parametrize the {\em visible portions} of our charts by projecting them to view space (inset). This maps a constant number of texels to each pixel in the final rendered output, evenly distributing residual undersampling error across all image pixels. While conceptually straightforward, efficiently parameterizing charts containing partially visible triangles using viewspace projection is non-trivial, as the visible portions may no longer be triangular (e.g. green triangle in the inset); applying naive projection to triangles with vertices behind the camera may produce ill-posed results. Clipping triangles before projection is both computationally expensive and significantly complicates downstream operations. We avoid explicit clipping by observing that all that is required for atlas packing is the dimensions of, potentially conservative, bounding boxes of these projected visible portions. We compute such bounding boxes without explicit chart clipping by adapting a conservative screen coverage estimator \shortcite{Blinn:CalculatingScreenCoverage} (Sec.~\ref{sec:box}). We then pack the computed boxes using FastAtlas. 
\end{parWithWrapFigure}

Finally, we shade the visible portion of each chart into its corresponding atlas bounding box (Fig~\ref{fig:overview}c). 
The resulting texture is then used during rasterization as a standard texture map (Fig. ~\ref{fig:overview}d). 
Our framework is compatible with all existing approaches for texture space shading, including forward shading, raytraced illumination, or deferred shading in texture space \cite{baker:2016}. In the examples shown, we use the standard forward shading based rendering pipeline included in the G3D Innovation Engine \cite{G3D17}, a commercial grade renderer.


\subsection{Model Merging}
In collaborative systems, proper model merging becomes increasingly vital for improving model knowledge integration from multiple sources in a resource-limited environment~\cite{li2023deep, yang2024model, goddard2024arcee}. Conceptually, model merging strategies can be categorized into entire model merging and partial model merging.

Entire model merging involves combining multiple model parameters to participate in the merging process by several means. Entire model merging can be viewed as an optimization problem~\cite{Matena_Raffel_2021, jin2022dataless, mavromatis2024packllm} or an alignment problem~\cite{ainsworth2022git, jordan2022repair, xu2024training, ainsworth2022git}, each offering unique advantages depending on the task at hand.
In the optimization-based approach, the goal is to find the best combination of multiple models to enhance performance and efficiency. For instance, using Fisher information approximation~\cite{Matena_Raffel_2021}, the optimization-based model merging can be interpreted as selecting parameters that maximize the joint likelihood of the models' posterior distributions. The optimization of model merging can also be guided by minimizing the prediction differences between the merged model and individual models~\cite{jin2022dataless}. 
With the development of large language models (LLM), optimization-based method is used to fuse multiple LLMs at test-time by minimizing perplexity over the input prompt~\cite{mavromatis2024packllm}.
To highlight, optimization-based methods are beneficial for scenarios requiring enhanced model performance and efficiency to integrate model parameters, while alignment-based methods~\cite{ainsworth2022git, jordan2022repair} are better suited for maintaining consistency and interpretability, facilitating critical information sharing across models.
For example, a training-free model merging strategy aligns relevant models by using a similarity matrix of their representations in both activation and weight spaces~\cite{xu2024training}.
Further, the alignment between the independently trained model and a reference model not only works for models with the same architecture but also for arbitrary model architectures~\cite{ainsworth2022git}.
In summary, the entire model merging methods can effectively integrate existing models into a merged model with enhanced functionality. However, they could lead to increased computational complexity and reduced flexibility, making them less scalable and harder to implement across diverse tasks.

Partial model merging refers to combining only specific components or layers of models to improve model merging efficiency and decrease the computational cost. 
Such specific components can come from the same network~\cite{kingetsu2021neural}, where the original network is divided into subnetworks for different purposes, and these subnetworks can then be recombined for new tasks.
Additionally, modules can originate from different functional networks and be merged using various strategies. For instance, arithmetic operations are applied in \cite{zhang2023composing} to fuse parameter-efficient modules.
While merging modules from different networks provides flexibility, the process also requires a selection strategy to ensure the resulting model aligns with the specific needs of the inference stage. 
The selection strategies are commonly designed based on the similarity of task~\cite{lv2023parameter} and domain clustering performance~\cite{chronopoulou2023adaptersoup}. Alternatively, the mixture-of-experts methods use a routing strategy to select appropriate component modules~\cite{ponti2023combining}. However, these strategies often require significant time and computational resources to filter through a large model pool. 
In contrast, LoRAHub~\cite{huang2023lorahub} offers a more lightweight approach, combining various LoRA modules for different tasks with minimal model training. Nevertheless, LoRAHub lacks flexibility for incremental updates, especially when handling unseen tasks.

Although the existing model merging approaches effectively combine the capabilities of individual models, these approaches often rely on raw data, leading to potential privacy risks. Our proposed MedForge emphasizes the prevention of raw data usage, which is particularly crucial in medical scenarios. Additionally, MedForge offers an extensible capability for incremental learning, enabling continuous model improvement.



\vspace{-5pt}
\section{Method}
\label{sec:method}
\begin{figure*}[t]
\begin{center}
\includegraphics[width=.85\linewidth]{fig_overview_v3.pdf}
\end{center}
\caption{
FastAtlas Overview: In each frame, we compute charts spanning fully or partially visible triangles (a), determine texture space bounding boxes for the visible portions of the view-space projections of each chart, and tightly pack these boxes into atlases (b, here $2K \times 2K$). We simultaneously bijectively parameterize and shade the charts into their atlas boxes, obtaining high quality texture space shading (c), and use this shading to render the shaded frames (d).}
\label{fig:overview}
\label{fig:alg_overview}
\end{figure*}

\section{Overview}
\label{sec:overview}
Our work has two core contributions: a real-time, GPU-based algorithm for tight packing of general parameterized charts into compact atlases; and a real-time TSS method that
utilizes this packing.  

\paragraph*{FastAtlas Packing.}
FastAtlas runs entirely on the GPU as a series of compute shaders. It takes the bounding boxes of parameterized charts as input, and packs them into an atlas (Fig~\ref{fig:overview}b, Sec.~\ref{sec:pack}). As such, the only input it requires are the dimensions of the bounding boxes.
Its outputs are deterministic; identical input charts are packed into identical atlases. This is critical for TSS and similar applications, as it ensures that consecutive frames taken from the same camera view have the same shading. Even minute shading differences across such frames can cause sampling jitter, leading to undesirable flicker \cite{baker2012rock}. 
While prior methods such as \cite{mueller2018shading,hladky2019tessellated,hladky2021snakebinning,Neff2022MSA} cap the dimensions of the charts that can be packed as-is for a given atlas size, and scale down all charts that exceed these dimensions, we scale all charts by the same factor, and do so only when strictly necessary to achieve packing success (Figs~\ref{fig:atlas},~\ref{fig:sas_issues}). 

\paragraph*{TSS using FastAtlas.}
Our end-to-end TSS atlas generation method combines the packing method above with a novel approach for computing seamless per-frame charts. 
We define our charts as the connected components of the visible surfaces in each frame (Fig.~\ref{fig:overview}a), and efficiently compute them using a parallel union-find algorithm (Sec.~\ref{sec:visible}). Since the boundaries of these charts coincide with the contours of the rendered surface, they are {\em invisible} to the viewer. This approach 
eliminates the artifacts caused by shading discontinuities along visible seams (Fig.~\ref{fig:seams}). 

\begin{parWithWrapFigure}
\begin{wrapfigure}{l}{.27\columnwidth}%
\includegraphics[width=\linewidth]{fig_inset_view_plane.pdf}%
\end{wrapfigure}
We bijectively parametrize the {\em visible portions} of our charts by projecting them to view space (inset). This maps a constant number of texels to each pixel in the final rendered output, evenly distributing residual undersampling error across all image pixels. While conceptually straightforward, efficiently parameterizing charts containing partially visible triangles using viewspace projection is non-trivial, as the visible portions may no longer be triangular (e.g. green triangle in the inset); applying naive projection to triangles with vertices behind the camera may produce ill-posed results. Clipping triangles before projection is both computationally expensive and significantly complicates downstream operations. We avoid explicit clipping by observing that all that is required for atlas packing is the dimensions of, potentially conservative, bounding boxes of these projected visible portions. We compute such bounding boxes without explicit chart clipping by adapting a conservative screen coverage estimator \shortcite{Blinn:CalculatingScreenCoverage} (Sec.~\ref{sec:box}). We then pack the computed boxes using FastAtlas. 
\end{parWithWrapFigure}

Finally, we shade the visible portion of each chart into its corresponding atlas bounding box (Fig~\ref{fig:overview}c). 
The resulting texture is then used during rasterization as a standard texture map (Fig. ~\ref{fig:overview}d). 
Our framework is compatible with all existing approaches for texture space shading, including forward shading, raytraced illumination, or deferred shading in texture space \cite{baker:2016}. In the examples shown, we use the standard forward shading based rendering pipeline included in the G3D Innovation Engine \cite{G3D17}, a commercial grade renderer.


Our goal is to increase the robustness of T2I models, particularly with rare or unseen concepts, which they struggle to generate. To do so, we investigate a retrieval-augmented generation approach, through which we dynamically select images that can provide the model with missing visual cues. Importantly, we focus on models that were not trained for RAG, and show that existing image conditioning tools can be leveraged to support RAG post-hoc.
As depicted in \cref{fig:overview}, given a text prompt and a T2I generative model, we start by generating an image with the given prompt. Then, we query a VLM with the image, and ask it to decide if the image matches the prompt. If it does not, we aim to retrieve images representing the concepts that are missing from the image, and provide them as additional context to the model to guide it toward better alignment with the prompt.
In the following sections, we describe our method by answering key questions:
(1) How do we know which images to retrieve? 
(2) How can we retrieve the required images? 
and (3) How can we use the retrieved images for unknown concept generation?
By answering these questions, we achieve our goal of generating new concepts that the model struggles to generate on its own.

\vspace{-3pt}
\subsection{Which images to retrieve?}
The amount of images we can pass to a model is limited, hence we need to decide which images to pass as references to guide the generation of a base model. As T2I models are already capable of generating many concepts successfully, an efficient strategy would be passing only concepts they struggle to generate as references, and not all the concepts in a prompt.
To find the challenging concepts,
we utilize a VLM and apply a step-by-step method, as depicted in the bottom part of \cref{fig:overview}. First, we generate an initial image with a T2I model. Then, we provide the VLM with the initial prompt and image, and ask it if they match. If not, we ask the VLM to identify missing concepts and
focus on content and style, since these are easy to convey through visual cues.
As demonstrated in \cref{tab:ablations}, empirical experiments show that image retrieval from detailed image captions yields better results than retrieval from brief, generic concept descriptions.
Therefore, after identifying the missing concepts, we ask the VLM to suggest detailed image captions for images that describe each of the concepts. 

\vspace{-4pt}
\subsubsection{Error Handling}
\label{subsec:err_hand}

The VLM may sometimes fail to identify the missing concepts in an image, and will respond that it is ``unable to respond''. In these rare cases, we allow up to 3 query repetitions, while increasing the query temperature in each repetition. Increasing the temperature allows for more diverse responses by encouraging the model to sample less probable words.
In most cases, using our suggested step-by-step method yields better results than retrieving images directly from the given prompt (see 
\cref{subsec:ablations}).
However, if the VLM still fails to identify the missing concepts after multiple attempts, we fall back to retrieving images directly from the prompt, as it usually means the VLM does not know what is the meaning of the prompt.

The used prompts can be found in \cref{app:prompts}.
Next, we turn to retrieve images based on the acquired image captions.

\vspace{-3pt}
\subsection{How to retrieve the required images?}

Given $n$ image captions, our goal is to retrieve the images that are most similar to these captions from a dataset. 
To retrieve images matching a given image caption, we compare the caption to all the images in the dataset using a text-image similarity metric and retrieve the top $k$ most similar images.
Text-to-image retrieval is an active research field~\cite{radford2021learning, zhai2023sigmoid, ray2024cola, vendrowinquire}, where no single method is perfect.
Retrieval is especially hard when the dataset does not contain an exact match to the query \cite{biswas2024efficient} or when the task is fine-grained retrieval, that depends on subtle details~\cite{wei2022fine}.
Hence, a common retrieval workflow is to first retrieve image candidates using pre-computed embeddings, and then re-rank the retrieved candidates using a different, often more expensive but accurate, method \cite{vendrowinquire}.
Following this workflow, we experimented with cosine similarity over different embeddings, and with multiple re-ranking methods of reference candidates.
Although re-ranking sometimes yields better results compared to simply using cosine similarity between CLIP~\cite{radford2021learning} embeddings, the difference was not significant in most of our experiments. Therefore, for simplicity, we use cosine similarity between CLIP embeddings as our similarity metric (see \cref{tab:sim_metrics}, \cref{subsec:ablations} for more details about our experiments with different similarity metrics).

\vspace{-3pt}
\subsection{How to use the retrieved images?}
Putting it all together, after retrieving relevant images, all that is left to do is to use them as context so they are beneficial for the model.
We experimented with two types of models; models that are trained to receive images as input in addition to text and have ICL capabilities (e.g., OmniGen~\cite{xiao2024omnigen}), and T2I models augmented with an image encoder in post-training (e.g., SDXL~\cite{podellsdxl} with IP-adapter~\cite{ye2023ip}).
As the first model type has ICL capabilities, we can supply the retrieved images as examples that it can learn from, by adjusting the original prompt.
Although the second model type lacks true ICL capabilities, it offers image-based control functionalities, which we can leverage for applying RAG over it with our method.
Hence, for both model types, we augment the input prompt to contain a reference of the retrieved images as examples.
Formally, given a prompt $p$, $n$ concepts, and $k$ compatible images for each concept, we use the following template to create a new prompt:
``According to these examples of 
$\mathord{<}c_1\mathord{>:<}img_{1,1}\mathord{>}, ... , \mathord{<}img_{1,k}\mathord{>}, ... , \mathord{<}c_n\mathord{>:<}img_{n,1}\mathord{>}, ... , $
$\mathord{<}img_{n,k}\mathord{>}$,
generate $\mathord{<}p\mathord{>}$'', 
where $c_i$ for $i\in{[1,n]}$ is a compatible image caption of the image $\mathord{<}img_{i,j}\mathord{>},  j\in{[1,k]}$. 

This prompt allows models to learn missing concepts from the images, guiding them to generate the required result. 

\textbf{Personalized Generation}: 
For models that support multiple input images, we can apply our method for personalized generation as well, to generate rare concept combinations with personal concepts. In this case, we use one image for personal content, and 1+ other reference images for missing concepts. For example, given an image of a specific cat, we can generate diverse images of it, ranging from a mug featuring the cat to a lego of it or atypical situations like the cat writing code or teaching a classroom of dogs (\cref{fig:personalization}).
\vspace{-2pt}
\begin{figure}[htp]
  \centering
   \includegraphics[width=\linewidth]{Assets/personalization.pdf}
   \caption{\textbf{Personalized generation example.}
   \emph{ImageRAG} can work in parallel with personalization methods and enhance their capabilities. For example, although OmniGen can generate images of a subject based on an image, it struggles to generate some concepts. Using references retrieved by our method, it can generate the required result.
}
   \label{fig:personalization}\vspace{-10pt}
\end{figure}
\section{Experiments and Results}
We evaluate the performance of \ac{dfm} in terms of tracking accuracy relative to reference motions and the naturalness of transitions between motions. 
Additionally, we demonstrate the multitasking capabilities of our approach through locomotion and gaze control during stylized dancing.

\subsection{Tracking Accuracy}
To assess tracking accuracy, we calculate the difference between the reference motion and the observed joint positions on the robot hardware.
For this evaluation, we select a dancing motion that involved lifting the rear legs.
\figref{fig:moveup_rear_leg} illustrates the height of the rear right leg during this stylized dance.
Although \ac{dfm} does not perfectly replicate the height of the reference motion, it significantly outperforms the \ac{fld} baseline.
To provide a more quantitative comparison between the baseline and our method, we analyze three types of joint angles as shown in \figref{fig:joint_angle_tracking_acc}.
The results indicate that the motion reconstructed by \ac{fld} is overly smooth due to its strong enforcement of quasi-constant parameterization and periodicity assumption with $N = 100$.
In contrast, \ac{dfm} achieves a more accurate reconstruction with $N = 0$, preserving intricate details that may not follow periodic patterns.
When examining the joint encoder data measured from the robot, \ac{fld} again shows excessive smoothing, which we attribute to the overly strong periodic assumptions applied to the local time during \ac{rl} training, as described in \eqnref{eqn:periodic_assumption}.
\Figref{fig:latent_parameters} presents the $\sin{\phi}$ and frequency values derived from the latent parameters across eight channels during the same dancing motion.
On the left side, \ac{fld} shows that all channels of $\sin{\phi}$ are periodic, with little change in frequency.
In contrast, \ac{dfm} demonstrates variability in some channels of $\sin{\phi}$ and frequency during the upward movement of the rear leg, retaining non-periodic features that characterize the dance motions.
Finally, \tabref{table:tracking_accuracy} reports the mean absolute tracking error (MAE) across all joints for all 170 evaluated motions using the real aibo hardware.
Additionally, we test \ac{dfm} using the MIT Humanoid environment~\cite{chignoli2021humanoid} in Isaac Gym.
Our method consistently demonstrates superior tracking accuracy in both robot environments compared to \ac{fld}.

\begin{figure}[t]
    \vspace{1ex}
    \centering
    \includegraphics[trim={0 0 0 0}, width=\linewidth]{figures/fig3.pdf}
    \caption{Height reached by rear right leg. Left, middle and right depict reference motion, \ac{fld} and \ac{dfm} motions respectively. The red dash line illustrates the height of the right rear leg at reference motion.}
    \label{fig:moveup_rear_leg}
    \vspace{-1ex}
\end{figure}

\begin{figure}[t]
    % \vspace{-3ex}
    \centering
    %\includegraphics[trim={0 6ex 0 0}, width=\linewidth]
    \includegraphics[trim={0 2ex 0 0}, width=\linewidth]{figures/tracking_acc.pdf}
    \caption{Comparison of tracking accuracy for the \ac{fld} and \ac{dfm}. Blue: reference motion created by the motion designer. Orange: reconstructed motions from motion representation parts by conditioning the reference motion. Green: joint encoder reading activated by the \ac{rl} policy.}
    % \caption{Comparison of tracking accuracy for \ac{fld} and \ac{dfm}. Motion reconstruction by FLD is overly smooth. In contrast, DFM achieves a more accurate reconstruction, preserving intricate details that may not follow periodic patterns. }
    \label{fig:joint_angle_tracking_acc}
    \vspace{-3ex}
\end{figure}


\begin{figure}[!t]
    \vspace{1ex}
    \centering
    \includegraphics[trim={0 0 0 0}, width=\linewidth]{figures/periodic.pdf}
    \caption{Comparison of 8 channel latent parameters for \ac{fld} at the left and \ac{dfm}  at the right side by conditioning the same dancing motion as \figref{fig:moveup_rear_leg}. The upper and bottom of plots are $\sin{\phi}$ and frequency for each.}
    \label{fig:latent_parameters}
    % \vspace{-3ex}
\end{figure}

\begin{table}[!t]
\caption{Mean Absolute Tracking Accuracy}
\label{table:tracking_accuracy}
\begin{center}
\begin{tabular}{llcc}
\toprule
\textbf{Robot} & \textbf{reference motion} & \textbf{FLD} & \textbf{DFM (ours)} \\
\midrule
aibo & dance & $0.132$ $\rm{rad}$ & $0.094$ 
 $\rm{rad}$ \\
% aibo & locomotion & 0.141 $\rm{rad}$ & 0.123 $\rm{rad}$ \\
MIT humanoid  & locomotion &  $0.125$ $\rm{rad}$  & $0.103$ $\rm{rad}$   \\
\bottomrule
\end{tabular}
\end{center}
\vspace{-5ex}
\end{table}

\subsection{Natural Transition}
The motion representation employed by \ac{dfm} enables continuous frequency interpolation and smooth transitions between different dancing motions.

\Figref{fig:frequency_interpolation} shows the estimated latent frequency parameters conditioned on the reference motion, which primarily involves head movements transitioning from higher to lower dancing frequencies.
% , as demonstrated in the supplementary video.
While most frequency channels remain relatively constant, channels 3 and 4 exhibit gradual changes as shown in \figref{fig:frequency_interpolation}.
The linear interpolation of frequencies in these channels adjusts in response to the changing frequency of the reference dancing motion.
Even though the training dataset consists of discrete frequency types, the motion representation allows for continuous frequency interpolation.
This capability results in smooth, periodic changes in joint positions without abrupt movements, even for previously unseen datasets as shown at the bottom plot in \figref{fig:frequency_interpolation}.

\Figref{fig:natural_transient} illustrates the joint angular velocity at the head pitch and yaw during the transition from motion A to motion B, which primarily involves the head pitch and yaw actuators, as shown in the supplementary video.
During the transition times at 1 and 2.5 seconds, joint positions experience abrupt changes with switches between reference motions.
We compare the transition performance of \ac{dfm} with DeepMimic~\cite{deepmimic}, a well-known learning from demonstration approach that yields high tracking performance on single trajectories but lacks capabilities to deal with multiple motions.
Jerky transitions are observed in this case if the reference dataset and its representation are not carefully crafted.
In contrast, \ac{dfm} achieves smooth transitions without abrupt movements by interpolating in the latent space using \eqnref{eqn:natural_transient}.

\begin{figure}[!t]
    % \vspace{1ex}
    \centering
    \includegraphics[trim={0 0 0 0}, width=\linewidth]{figures/frequency_interpolation.pdf}
    \caption{Frequency modulation during head-moving dance. The upper plot displays the frequency of two representative latent channels out of eight. Solid and dashed curves represent raw and linearly interpolated data, respectively. The bottom plot shows head pitch (HP) and the head yaw (HY) joint angles.}
    \label{fig:frequency_interpolation}
    \vspace{-1ex}
\end{figure}


        % \small {
        % {\color{ourred}\rule[.5ex]{1em}{1pt}$\bullet$\rule[.5ex]{1em}{1pt}} step in place \qquad
        % {\color{ourorange}\rule[.5ex]{1em}{1pt}$\bullet$\rule[.5ex]{1em}{1pt}} forward run \qquad
        % {\color{ourgreen}\rule[.5ex]{1em}{1pt}$\bullet$\rule[.5ex]{1em}{1pt}} forward stride
        % }

% Continuous frequency transient. All plot are shown during dancing frequency which moves mostly head angle positions. Upper plot is all 8 channel of frequency from latent parameter. Solid and dash are raw and linear interpolation for each. middle plots is two plot (3 and 4 channel) are picked up and zoomed up from upper plot. Bottom plot are the joint angle of head yaw (HY) and head pitch (HP) actuator.

\begin{figure}[!t]
    %\vspace{-3ex}
    \centering
    \includegraphics[trim={0 0 0 0}, width=\linewidth]{figures/natural_transient.pdf}
    \caption{Transition between different dance types. The background color indicates the dance motion type. The left and right plots demonstrate hard switches between Dance A and Dance B, with DeepMimic and \ac{dfm}, respectively. Angular velocities at the head pitch (HP) and the head yaw (HY) are shown.}
    \label{fig:natural_transient}
    \vspace{-3ex}
\end{figure}

\subsection{Multi-task Demonstration}
We evaluate the multitasking capability of \ac{dfm} with auxiliary tasks, including locomotion and gaze, respectively.

\Figref{fig:dancing_locomotion} illustrates the locomotion policy during dancing, where an angular velocity command is used to facilitate in-place rotation.
In the reference motion, only the rear legs move alternately, while the forelegs remain stationary.
After training the policy with the reward structure defined in the locomotion curriculum (\tabref{table:reward}), aibo learns to rotate in response to the angular velocity command in the base frame.
To allow this rotation without hindering the movement of the rear legs, the right foreleg is lifted, enabling the execution of the stylized dancing, as shown in the supplementary video.

Similarly, a policy for auxiliary gaze control is trained using the reward scale from the gaze curriculum in \tabref{table:reward}.
This policy enables aibo to adjust its head orientation in response to pitch and yaw commands during dancing, as demonstrated in \figref{fig:dancing_gaze}.
The supplementary video shows that the dance sequence continues smoothly while the robot adjusts its pitch and yaw angles.
aibo utilizes its head and legs to track the commanded pitch and yaw angles, as illustrated in \figref{fig:dancing_gaze_plot}.
For instance, when a pitch of $0.3$ $\rm{rad}$ and a yaw of $0.0$ $\rm{rad}$ are commanded, both the directions of head are moved up with legs.
In contrast, a $-0.5$ $\rm{rad}$ pitch command prompts the head and legs to move in opposite directions.

\begin{figure}[!t]
    %\vspace{-3ex}
    \centering
    \includegraphics[trim={0 0 0 0}, width=\linewidth]{figures/fig8.pdf}
    \caption{Locomotion during dance. The reference dance motion alternates lifting the rear legs while keeping the forelegs stationary. Applying an angular velocity command results in locomotion by lifting the left foreleg.}
    \label{fig:dancing_locomotion}
    %\vspace{-3ex}
\end{figure}

\begin{figure}[!t]
    \vspace{-2ex}
    \centering
    \includegraphics[trim={0 0 0 0}, width=\linewidth]{figures/fig9.pdf}
    \caption{Gaze during dance. The reference motion is the same as in \figref{fig:dancing_locomotion}. The images depict commanded pitch angles of $0.0$, $0.3$ $\rm{rad}$, and $-0.5$ $\rm{rad}$, respectively. The command of the yaw angle is held at zero.}
    \label{fig:dancing_gaze}
    % \vspace{-3ex}
\end{figure}

\begin{figure}[!t]
    % \vspace{-1ex}
    \centering
    \includegraphics[trim={0 0 0 0}, width=\linewidth]{figures/dancing_gaze_plot.pdf}
    \caption{Joint readings for fore left shoulder pitch (FLSP), fore left shoulder roll (FLSR), fore left foot pitch (FLFP), and head pitch (HP) during dancing gaze are shown. The background color of the plot indicates command for pitch angle at the head frame (yellow: $0.0$, red: $0.3$ $\rm{rad}$, blue: $-0.5$ $\rm{rad}$).}
    \label{fig:dancing_gaze_plot}
    \vspace{-3ex}
\end{figure}


\section{Discussions}

\subsection{Transparency in Ride-Sharing Platform Algorithms}
The publicly available Chicago Transportation Network Provider dataset helped us answer many research questions, but ride-sharing platforms still make many of their mechanisms opaque. The lack of transparency in key platform mechanisms---such as pricing models, driver--rider matching algorithms, and driver ranking systems---makes it difficult to pinpoint the exact causes of these disparities. Without greater visibility into these proprietary algorithms, drivers also remain at an information disadvantage, unable to anticipate fare fluctuations or optimize their work schedules effectively.

Pricing models remain opaque, with our analysis revealing that fare adjustments over time have failed to keep pace with inflation, effectively reducing real driver earnings (\cref{sec:results-pricing-stablization}). While platforms advertise dynamic pricing mechanisms that respond to demand surges, drivers have limited insight into how much of the fare they actually receive after platform fees~\cite{santos2020dynamic}. Previous research has shown that drivers tend to work more during peaks for higher compensation~\cite{chen2016dynamic}. A real-time, large-scale understanding of the surge pricing model can help drivers become more informed in planning and organizing their workday, beyond anecdotal observations. Furthermore, researchers can provide prediction models of price surges, helping both drivers and riders adjust plans accordingly. Another key limitation of using the Chicago dataset is the lack of driver earning information. As a result, our analysis can only use the trip fare as a proxy for driver earning. Making such information available can significantly increase transparency into platform operations.

Similarly, the driver-rider matching algorithm remains a black box. Our inferred driver profiles suggest that trip assignments may systematically disadvantage certain groups, particularly those operating in lower-income areas. If the matching algorithm disproportionately favors drivers in high-demand or high-fare regions, it could reinforce existing geographic disparities in earnings. However, such analysis is hard to conduct without access to driver-level information. As discussed in \cref{sec:methods-driver-simulation}, releasing such data may lead to privacy concerns. Our approach is an effort to approximate driver working conditions without needing detailed driver data. However, researchers should still work with ride-sharing platforms to come up with privacy-preserving ways to analyze such data for insights. Also, driver ranking algorithms---which determine access to high-value trips---are equally opaque. While platforms often cite factors such as acceptance rate, customer ratings, and trip history, the lack of public accountability raises concerns regarding potential biases. Accessing such information can support researchers in identifying potential biases, also help drivers provide more desired services to riders.

In all, we call for increased regulatory oversight and platform-level efforts to improve algorithmic transparency. Without clear disclosures on how these systems operate, ride-sharing drivers remain vulnerable to unfair decision-making and fluctuating incomes that they cannot predict or control.

\subsection{Data Analysis Methodology Improvements}
Our study demonstrates the feasibility of simulating reasonable driver profiles from trip-level data, even in the absence of driver-related information. By leveraging a simulation-based approach, we were able to approximate driver earnings, work patterns, and geographic activity. However, there are still areas for improvement for our methodology.

First, a robust evaluation benchmark is needed to validate the accuracy of inferred driver profiles. While our approach provides valuable insights and matches previous empirical findings, the lack of direct ground truth data means we rely on approximations. We need alternative data sources to cross-verify our inferred driver activities. Tools for driver task management, such as Driver's Seat~\cite{calacci2023access}, asks drivers to upload their work tasks and can serve as a potential data source. More autonomous approaches that uses UI understanding techniques and directly collects data from drivers' phones can also scale up this effort~\cite{lu2024crepe}. 

Moreover, expanding the scope of inferred information would provide deeper insights into platform operations. Currently, we infer earnings and work patterns for drivers. Newer algorithms can be developed to analyze additional opaque platform mechanisms as discussed above. Future studies could aim to reconstruct other aspects of opaque platform algorithms, as discussed above, directly from publicly available, large-scale datasets.

Given a large-scale dataset that misses key information aspects, a potential future approach is to self-collect a smaller dataset that contains the necessary details and conduct a joint analysis of both datasets. For example, a smaller dataset that we collect directly from drivers, containing both driver and trip information, can serve both as a benchmark and a basis for use to train machine learning models that predict driver profiles from existing large-scale datasets. Future research can investigate effective measures to combine these different data sources~\cite{harris2018federal} for joint analysis. These methodological advancements can help us to use large-scale ridesharing datasets more effectively and accurately while maintaining driver and rider privacy.


\subsection{Societal Implications: Ride-Sharing as a Reflection of Broader Inequalities}

Our findings revealed regional ride-sharing disparities in the city of Chicago, which largely reflect the broader existing societal inequalities. Drivers working in lower-income neighborhoods---in our case, drivers that service the southern regions of Chicago---consistently earn less, even despite longer work hours. Structural disadvantages, such as lower infrastructure quality, longer wait times, and increased safety concerns---compound the challenges faced by gig workers. Chicago South Side, as a community suffering from violence and poverty, has been an example of social segregation and studied by numerous researchers~\cite{moore2016south, bachin2004building, bell1993community}. As an aspect of a deep-rooted societal issue, ride-sharing inequality in lower-income neighborhoods calls for holistic policymaking efforts from multiple stakeholders.

Our findings provide practical implications for labor activists and policy makers. By providing a more transparent view of drivers’ potential workday experiences, policymakers can better evaluate the labor conditions these platforms create, ensuring that emerging mobility systems align with equity goals. Urban planners and regulators can use these insights to inform policy interventions---such as driver support programs, driver caps, or incentive structures---that promote fairness and mitigate algorithmic biases. Similarly, platform operators themselves might harness these findings to improve their matching algorithms, advancing a more equitable ecosystem that benefits both drivers and passengers.

Research has shown that transportation access can have a positive impact on regional economic growth and productivity~\cite{targa2005economic, banerjee2020road, alstadt2012relationship}. Ride-sharing, as an increasingly critical way of transportation, especially where public transportation is scarce, can support individual and community access to growth opportunities. The persistence of regional earning gaps raises important questions about equity in urban transportation. If ride-sharing platforms are designed primarily to maximize efficiency and revenue, they may inadvertently exacerbate existing economic inequalities by steering high-value rides away from underserved areas~\cite{durand2022access, bocarejo2012transport}.

To address these issues, we call for policy interventions aimed at ensuring fair compensation and equitable access to earning opportunities. Regulators should consider implementing transparency mandates, income stability measures, and algorithmic accountability frameworks to prevent platforms from disproportionately disadvantaging certain driver groups. Moreover, these efforts should be in orchestration with existing efforts to promote infrastructural improvements and public safety in underserved regions. Collaborative initiatives between policymakers, ride-sharing companies, and community organizations can help create a more inclusive transportation ecosystem that benefits both drivers and passengers alike~\cite{baber2022new}.
\section{Conclusion}
We introduced \methodname, an effective training framework defending against MIAs for LLMs. The extensive experiments demonstrate its robustness in protecting privacy while maintaining strong language modeling performance across various datasets and architectures. Although our study focuses on fine-tuning due to computational constraints, \methodname can be seamlessly applied to large-scale pretraining, as done in prior selective pretraining work~\cite{lin2024not}. By categorizing tokens and treating them appropriately, \methodname opens a novel pathway for MIA defense. Future work can explore improved token selection strategies and multi-objective training approaches.
\begin{acronym}
\acro{gan}[GANs]{Generative Adversarial Networks}
\acro{rl}[RL]{Reinforcement Learning}
\acro{pae}[PAE]{Periodic Autoencoder}
\acro{fld}[FLD]{Fourier Latent Dynamics}
\acro{ppo}[PPO]{Proximal Policy Optimization}
\acro{fft}[FFT]{Fast Fourier Transform}
\acro{pca}[PCA]{Principal Component Analysis}
\acro{dfm}[DFM]{Deep Fourier Mimic}
\acro{dof}[DoF]{Degrees of Freedom}
\acro{mlp}[MLPs]{Multi-Layer Perceptrons}
\end{acronym}



%%%%%%%%%%%%%%%%%%%%%%%%%%%%%%%%%%%%%%%%%%%%%%%%%%%%%%%%%%%%%%%%%%%%%%%%%%%%%%%%
\iffalse
\section*{APPENDIX}

\begin{itemize}
    \item Furter resources on mono, stereo and event camera latency.
    \item Datasheet extracts to back our claims, if needed.
    \item Elaboration of the testing setup.
\end{itemize}
\fi

\section*{ACKNOWLEDGMENT}
The authors would like to thank Hiroyuki Izumi, Kensuke Kitamura, Ichitaro Kohara, Fumitaka Joo, Toshihisa Sambommatsu, Takuma Morita, and Yuichiro To at Sony Group Corporation for software integration help. 
Thanks to Yuntao Ma, Mayank Mittal and Jonas Frey at ETH Zurich for discussion about reinforcement learning.  

\clearpage

%%%%%%%%%%%%%%%%%%%%%%%%%%%%%%%%%%%%%%%%%%%%%%%%%%%%%%%%%%%%%%%%%%%%%%%%%%%%%%%%

\bibliographystyle{IEEEtran}
\bibliography{main} 

\end{document}
