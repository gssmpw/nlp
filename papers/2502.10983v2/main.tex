%%%%%%%%%%%%%%%%%%%%%%%%%%%%%%%%%%%%%%%%%%%%%%%%%%%%%%%%%%%%%%%%%%%%%%%%%%%%%%%%
%2345678901234567890123456789012345678901234567890123456789012345678901234567890
%        1         2         3         4         5         6         7         8

\documentclass[letterpaper, 10 pt, conference]{ieeeconf}  % Comment this line out if you need a4paper


%\documentclass[a4paper, 10pt, conference]{ieeeconf}      % Use this line for a4 paper

\IEEEoverridecommandlockouts                              % This command is only needed if 
                                                          % you want to use the \thanks command

\overrideIEEEmargins                                      % Needed to meet printer requirements.

%In case you encounter the following error:
%Error 1010 The PDF file may be corrupt (unable to open PDF file) OR
%Error 1000 An error occurred while parsing a contents stream. Unable to analyze the PDF file.
%This is a known problem with pdfLaTeX conversion filter. The file cannot be opened with acrobat reader
%Please use one of the alternatives below to circumvent this error by uncommenting one or the other
%\pdfobjcompresslevel=0
%\pdfminorversion=4

% See the \addtolength command later in the file to balance the column lengths
% on the last page of the document

% The following packages can be found on http:\\www.ctan.org
%\usepackage{graphics} % for pdf, bitmapped graphics files
%\usepackage{epsfig} % for postscript graphics files
%\usepackage{mathptmx} % assumes new font selection scheme installed
%\usepackage{times} % assumes new font selection scheme installed
%\usepackage{amsmath} % assumes amsmath package installed
%\usepackage{amssymb}  % assumes amsmath package installed

\usepackage{graphics}
\usepackage{multirow}
% Needed to meet printer requirements.

% \overrideIEEEmargins                                      % Needed to meet printer requirements. Uncommented for RA-L

% The following packages can be found on http:\\www.ctan.org
\usepackage{graphics} % for pdf, bitmapped graphics files
\usepackage{epsfig} % for postscript graphics files
%\usepackage{mathptmx} % assumes new font selection scheme installed
\usepackage{times} % assumes new font selection scheme installed
\usepackage{amsmath} % assumes amsmath package installed
\usepackage{amssymb}  % assumes amsmath package installed

\usepackage{array}
\usepackage{booktabs}

\usepackage[hidelinks]{hyperref}
\usepackage{cite}

% Cross and Check mark
\usepackage{xcolor,pifont}
\usepackage{color}
\usepackage[nolist]{acronym}
\usepackage{amsmath}
\usepackage{multirow}
\usepackage{microtype}

\newcommand\Tstrut{\rule{0pt}{2.6ex}}         % = `top' strut
\newcommand\Bstrut{\rule[-0.9ex]{0pt}{0pt}}   % = `bottom' strut

%%% useful macros from chenhao's premables
\newcommand{\Figref}[1]{Figure~\ref{#1}}  % beginning of sentence
\newcommand{\figref}[1]{Fig.~\ref{#1}}    % somewhere
\newcommand{\Tabref}[1]{Table~\ref{#1}}
\newcommand{\tabref}[1]{Table~\ref{#1}}
\newcommand{\Eqnref}[1]{Equation~\ref{#1}}
\newcommand{\eqnref}[1]{Eq.~\ref{#1}} % Eq. (1)
\newcommand{\eqnpref}[1]{(Eq.~\ref{#1})} % (Eq. 1)
\newcommand{\Real}{\ensuremath{\mathbb R}}  

\DeclareMathOperator*{\MSE}{MSE}
\DeclareMathOperator*{\enc}{\textbf{enc}}
\DeclareMathOperator*{\dec}{\textbf{dec}}

\title{\LARGE \bf
DFM: Deep Fourier Mimic for Expressive Dance Motion Learning
% DFM: Deep Fourier Mimic for Expressive Dance Motion Learning \\ on Entertainment Robots
}

\pdfminorversion=5


\title{\LARGE \bf
Learning Quiet Walking for a Small Home Robot
%Quiet Walk Learning for Small Home Robot aibo
}

\author{Ryo Watanabe$^{1}$$^{,}$$^{2}$, Takahiro Miki$^{1}$, Fan Shi$^{1}$$^{,}$$^{3}$, Yuki Kadokawa$^{1}$$^{,}$$^{4}$, Filip Bjelonic$^{1}$, \\
Kento Kawaharazuka$^{1}$$^{,}$$^{5}$, Andrei Cramariuc$^{1}$ and Marco Hutter$^{1}$% <-this % stops a space
\thanks{$^{1}$Ryo Watanabe, Takahiro Miki, Fan Shi, Yuki Kadokawa, Filip Bjelonic, Kento Kawaharazuka, Andrei Cramariuc, and Marco Hutter are with the Robotic Systems Lab, Department of Mechanical Engineering, ETH Zurich, Switzerland.}
\thanks{$^{2}$Ryo Watanabe is at Sony Group Corporation, Japan}%
\thanks{$^{3}$Fan Shi is at National University of Singapore}%
\thanks{$^{4}$Yuki Kadokawa is at Nara Institute of Science and Technology, Japan}%
\thanks{$^{5}$Kento Kawaharazuka is at The University of Tokyo, Japan}%
}

\begin{document}

\maketitle
\thispagestyle{empty}
\pagestyle{empty}

%%%%%%%%%%%%%%%%%%%%%%%%%%%%%%%%%%%%%%%%%%%%%%%%%%%%%%%%%%%%%%%%%%%%%%%%%%%%%%%%
\begin{abstract}
As home robotics gains traction, robots are increasingly integrated into households, offering companionship and assistance. Quadruped robots, particularly those resembling dogs, have emerged as popular alternatives for traditional pets. However, user feedback highlights concerns about the noise these robots generate during walking at home, particularly the loud footstep sound.
To address this issue, we propose a sim-to-real
based \ac{rl} approach to minimize the foot contact velocity highly related to the footstep sound. Our framework incorporates three key elements: learning varying PD gains to actively dampen and stiffen each joint, utilizing foot contact sensors, and employing curriculum learning to gradually enforce penalties on foot contact velocity. Experiments demonstrate that our learned policy achieves superior quietness compared to a \ac{rl} baseline and the carefully handcrafted Sony commercial controllers. Furthermore, the trade-off between robustness and quietness is shown. This research contributes to developing quieter and more user-friendly robotic companions in home environments.
% This paper addresses this issue by introducing a reinforcement learning (RL) based locomotion policy learned in a simulator and successfully transferred to the real home robot, aibo. Our approach includes three key elements: learning varying PD gains, using foot contact sensors, and utilizing curriculum learning. Evaluation of the resulting locomotion controller shows that it achieves quieter locomotion than Sony's commercial locomotion and our RL baseline locomotion, as measured by sound magnitude within the audible range. This research opens avenues for quieter and thus more user-friendly robotic companions in home environments.
%Our contributions include the first implementation of a quiet walking policy in RL for quadrupedal robots, deployment on a real consumer-grade small platform, and superior performance compared to baseline RL and Sony commercial controllers. => repetitive. 
\end{abstract}

%!TEX root = gcn.tex
\section{Introduction}
Graphs, representing structural data and topology, are widely used across various domains, such as social networks and merchandising transactions.
Graph convolutional networks (GCN)~\cite{iclr/KipfW17} have significantly enhanced model training on these interconnected nodes.
However, these graphs often contain sensitive information that should not be leaked to untrusted parties.
For example, companies may analyze sensitive demographic and behavioral data about users for applications ranging from targeted advertising to personalized medicine.
Given the data-centric nature and analytical power of GCN training, addressing these privacy concerns is imperative.

Secure multi-party computation (MPC)~\cite{crypto/ChaumDG87,crypto/ChenC06,eurocrypt/CiampiRSW22} is a critical tool for privacy-preserving machine learning, enabling mutually distrustful parties to collaboratively train models with privacy protection over inputs and (intermediate) computations.
While research advances (\eg,~\cite{ccs/RatheeRKCGRS20,uss/NgC21,sp21/TanKTW,uss/WatsonWP22,icml/Keller022,ccs/ABY318,folkerts2023redsec}) support secure training on convolutional neural networks (CNNs) efficiently, private GCN training with MPC over graphs remains challenging.

Graph convolutional layers in GCNs involve multiplications with a (normalized) adjacency matrix containing $\numedge$ non-zero values in a $\numnode \times \numnode$ matrix for a graph with $\numnode$ nodes and $\numedge$ edges.
The graphs are typically sparse but large.
One could use the standard Beaver-triple-based protocol to securely perform these sparse matrix multiplications by treating graph convolution as ordinary dense matrix multiplication.
However, this approach incurs $O(\numnode^2)$ communication and memory costs due to computations on irrelevant nodes.
%
Integrating existing cryptographic advances, the initial effort of SecGNN~\cite{tsc/WangZJ23,nips/RanXLWQW23} requires heavy communication or computational overhead.
Recently, CoGNN~\cite{ccs/ZouLSLXX24} optimizes the overhead in terms of  horizontal data partitioning, proposing a semi-honest secure framework.
Research for secure GCN over vertical data  remains nascent.

Current MPC studies, for GCN or not, have primarily targeted settings where participants own different data samples, \ie, horizontally partitioned data~\cite{ccs/ZouLSLXX24}.
MPC specialized for scenarios where parties hold different types of features~\cite{tkde/LiuKZPHYOZY24,icml/CastigliaZ0KBP23,nips/Wang0ZLWL23} is rare.
This paper studies $2$-party secure GCN training for these vertical partition cases, where one party holds private graph topology (\eg, edges) while the other owns private node features.
For instance, LinkedIn holds private social relationships between users, while banks own users' private bank statements.
Such real-world graph structures underpin the relevance of our focus.
To our knowledge, no prior work tackles secure GCN training in this context, which is crucial for cross-silo collaboration.


To realize secure GCN over vertically split data, we tailor MPC protocols for sparse graph convolution, which fundamentally involves sparse (adjacency) matrix multiplication.
Recent studies have begun exploring MPC protocols for sparse matrix multiplication (SMM).
ROOM~\cite{ccs/SchoppmannG0P19}, a seminal work on SMM, requires foreknowledge of sparsity types: whether the input matrices are row-sparse or column-sparse.
Unfortunately, GCN typically trains on graphs with arbitrary sparsity, where nodes have varying degrees and no specific sparsity constraints.
Moreover, the adjacency matrix in GCN often contains a self-loop operation represented by adding the identity matrix, which is neither row- nor column-sparse.
Araki~\etal~\cite{ccs/Araki0OPRT21} avoid this limitation in their scalable, secure graph analysis work, yet it does not cover vertical partition.

% and related primitives
To bridge this gap, we propose a secure sparse matrix multiplication protocol, \osmm, achieving \emph{accurate, efficient, and secure GCN training over vertical data} for the first time.

\subsection{New Techniques for Sparse Matrices}
The cost of evaluating a GCN layer is dominated by SMM in the form of $\adjmat\feamat$, where $\adjmat$ is a sparse adjacency matrix of a (directed) graph $\graph$ and $\feamat$ is a dense matrix of node features.
For unrelated nodes, which often constitute a substantial portion, the element-wise products $0\cdot x$ are always zero.
Our efficient MPC design 
avoids unnecessary secure computation over unrelated nodes by focusing on computing non-zero results while concealing the sparse topology.
We achieve this~by:
1) decomposing the sparse matrix $\adjmat$ into a product of matrices (\S\ref{sec::sgc}), including permutation and binary diagonal matrices, that can \emph{faithfully} represent the original graph topology;
2) devising specialized protocols (\S\ref{sec::smm_protocol}) for efficiently multiplying the structured matrices while hiding sparsity topology.


 
\subsubsection{Sparse Matrix Decomposition}
We decompose adjacency matrix $\adjmat$ of $\graph$ into two bipartite graphs: one represented by sparse matrix $\adjout$, linking the out-degree nodes to edges, the other 
by sparse matrix $\adjin$,
linking edges to in-degree nodes.

%\ie, we decompose $\adjmat$ into $\adjout \adjin$, where $\adjout$ and $\adjin$ are sparse matrices representing these connections.
%linking out-degree nodes to edges and edges to in-degree nodes of $\graph$, respectively.

We then permute the columns of $\adjout$ and the rows of $\adjin$ so that the permuted matrices $\adjout'$ and $\adjin'$ have non-zero positions with \emph{monotonically non-decreasing} row and column indices.
A permutation $\sigma$ is used to preserve the edge topology, leading to an initial decomposition of $\adjmat = \adjout'\sigma \adjin'$.
This is further refined into a sequence of \emph{linear transformations}, 
which can be efficiently computed by our MPC protocols for 
\emph{oblivious permutation}
%($\Pi_{\ssp}$) 
and \emph{oblivious selection-multiplication}.
% ($\Pi_\SM$)
\iffalse
Our approach leverages bipartite graph representation and the monotonicity of non-zero positions to decompose a general sparse matrix into linear transformations, enhancing the efficiency of our MPC protocols.
\fi
Our decomposition approach is not limited to GCNs but also general~SMM 
by 
%simply 
treating them 
as adjacency matrices.
%of a graph.
%Since any sparse matrix can be viewed 

%allowing the same technique to be applied.

 
\subsubsection{New Protocols for Linear Transformations}
\emph{Oblivious permutation} (OP) is a two-party protocol taking a private permutation $\sigma$ and a private vector $\xvec$ from the two parties, respectively, and generating a secret share $\l\sigma \xvec\r$ between them.
Our OP protocol employs correlated randomnesses generated in an input-independent offline phase to mask $\sigma$ and $\xvec$ for secure computations on intermediate results, requiring only $1$ round in the online phase (\cf, $\ge 2$ in previous works~\cite{ccs/AsharovHIKNPTT22, ccs/Araki0OPRT21}).

Another crucial two-party protocol in our work is \emph{oblivious selection-multiplication} (OSM).
It takes a private bit~$s$ from a party and secret share $\l x\r$ of an arithmetic number~$x$ owned by the two parties as input and generates secret share $\l sx\r$.
%between them.
%Like our OP protocol, o
Our $1$-round OSM protocol also uses pre-computed randomnesses to mask $s$ and $x$.
%for secure computations.
Compared to the Beaver-triple-based~\cite{crypto/Beaver91a} and oblivious-transfer (OT)-based approaches~\cite{pkc/Tzeng02}, our protocol saves ${\sim}50\%$ of online communication while having the same offline communication and round complexities.

By decomposing the sparse matrix into linear transformations and applying our specialized protocols, our \osmm protocol
%($\prosmm$) 
reduces the complexity of evaluating $\numnode \times \numnode$ sparse matrices with $\numedge$ non-zero values from $O(\numnode^2)$ to $O(\numedge)$.

%(\S\ref{sec::secgcn})
\subsection{\cgnn: Secure GCN made Efficient}
Supported by our new sparsity techniques, we build \cgnn, 
a two-party computation (2PC) framework for GCN inference and training over vertical
%ly split
data.
Our contributions include:

1) We are the first to explore sparsity over vertically split, secret-shared data in MPC, enabling decompositions of sparse matrices with arbitrary sparsity and isolating computations that can be performed in plaintext without sacrificing privacy.

2) We propose two efficient $2$PC primitives for OP and OSM, both optimally single-round.
Combined with our sparse matrix decomposition approach, our \osmm protocol ($\prosmm$) achieves constant-round communication costs of $O(\numedge)$, reducing memory requirements and avoiding out-of-memory errors for large matrices.
In practice, it saves $99\%+$ communication
%(Table~\ref{table:comm_smm}) 
and reduces ${\sim}72\%$ memory usage over large $(5000\times5000)$ matrices compared with using Beaver triples.
%(Table~\ref{table:mem_smm_sparse}) ${\sim}16\%$-

3) We build an end-to-end secure GCN framework for inference and training over vertically split data, maintaining accuracy on par with plaintext computations.
We will open-source our evaluation code for research and deployment.

To evaluate the performance of $\cgnn$, we conducted extensive experiments over three standard graph datasets (Cora~\cite{aim/SenNBGGE08}, Citeseer~\cite{dl/GilesBL98}, and Pubmed~\cite{ijcnlp/DernoncourtL17}),
reporting communication, memory usage, accuracy, and running time under varying network conditions, along with an ablation study with or without \osmm.
Below, we highlight our key achievements.

\textit{Communication (\S\ref{sec::comm_compare_gcn}).}
$\cgnn$ saves communication by $50$-$80\%$.
(\cf,~CoGNN~\cite{ccs/KotiKPG24}, OblivGNN~\cite{uss/XuL0AYY24}).

\textit{Memory usage (\S\ref{sec::smmmemory}).}
\cgnn alleviates out-of-memory problems of using %the standard 
Beaver-triples~\cite{crypto/Beaver91a} for large datasets.

\textit{Accuracy (\S\ref{sec::acc_compare_gcn}).}
$\cgnn$ achieves inference and training accuracy comparable to plaintext counterparts.
%training accuracy $\{76\%$, $65.1\%$, $75.2\%\}$ comparable to $\{75.7\%$, $65.4\%$, $74.5\%\}$ in plaintext.

{\textit{Computational efficiency (\S\ref{sec::time_net}).}} 
%If the network is worse in bandwidth and better in latency, $\cgnn$ shows more benefits.
$\cgnn$ is faster by $6$-$45\%$ in inference and $28$-$95\%$ in training across various networks and excels in narrow-bandwidth and low-latency~ones.

{\textit{Impact of \osmm (\S\ref{sec:ablation}).}}
Our \osmm protocol shows a $10$-$42\times$ speed-up for $5000\times 5000$ matrices and saves $10$-2$1\%$ memory for ``small'' datasets and up to $90\%$+ for larger ones.

\section{Related Works}

\textbf{Enhancing LLMs' Theory of Mind.} There has been systematic evaluation that revealed LLMs' limitations in achieving robust Theory of Mind inference \citep{ullman2023large, shapira2023clever}. To enhance LLMs' Theory of Mind capacity, recent works have proposed various prompting techniques. For instance, SimToM \citep{wilf2023think} encourages LLMs to adopt perspective-taking, PercepToM \citep{jung2024perceptions} improves perception-to-belief inference by extracting relevant contextual details, and \citet{huang2024notion} utilize an LLM as a world model to track environmental changes and refine prompts. Explicit symbolic modules also seem to improve LLM's accuracy through dynamic updates based on inputs. Specifically, TimeToM \citep{hou2024timetom} constructs a temporal reasoning framework to support inference, while SymbolicToM \citep{sclar2023minding} uses graphical representations to track characters' beliefs. Additionally, \citet{wagner2024mind} investigates ToM's necessity and the level of recursion required for specific tasks. However, these approaches continue to exhibit systematic errors in long contexts, complex behaviors, and recursive reasoning due to inherent limitations in inference and modeling \citep{jin2024mmtom,shi2024muma}. Most of them rely on domain-specific designs, lacking open-endedness.


\textbf{Model-based Theory of Mind inference.} Model-based Theory of Mind inference, in particular, Bayesian inverse planning (BIP) \citep{baker2009action,ullman2009help,baker2017rational,zhi2020online}, explicitly constructs representations of agents' mental states and how mental states guide agents' behavior via Bayesian Theory of Mind (BToM) models. These methods can reverse engineer human ToM inference in simple domains \citep[e.g.,][]{baker2017rational,netanyahu2021phase,shu2021agent}. Recent works have proposed to combine BIP with LLMs to achieve robust ToM inference in more realistic settings \citep{ying2023neuro, jin2024mmtom, shi2024muma}. However, these methods require manual specification of the BToM models as well as rigid, domain-specific implementations of Bayesian inference, limiting their adaptability to open-ended scenarios. To overcome this limitation, we propose \ours, a method capable of automatically modeling mental variables across diverse conditions and conducting automated BIP without domain-specific knowledge or implementations.


\begin{figure*}[ht]
  \centering
  \includegraphics[width=\linewidth]{figures/benchmarks_and_models.pdf}
    \vspace{-15pt}
  \caption{Examples questions (top panels) and the necessary Bayesian Theory of Mind (BToM) model for Bayesian inverse planning (bottom panels) in diverse Theory of Mind benchmarks. \ours aims to answer any Theory of Mind question in a variety of benchmarks, encompassing different mental variables, observable contexts, numbers of agents, the presence or absence of utterances, wording styles, and modalities. It proposes and iteratively adjusts an appropriate BToM and conducts automated Bayesian inverse planning based on the model.
  There can be more types of questions/models in each benchmark beyond the examples shown in this figure.}
  \label{fig:benchmarks_and_models}
  %\vspace{-0.75em}
  \vspace{-10pt}
\end{figure*}



\textbf{Automated Modeling with LLMs.} There has been an increasing interest in integrating LLMs with inductive reasoning and probabilistic inference for automated modeling. \citet{piriyakulkij2024doing} combine LLMs with Sequential Monte Carlo to perform probabilistic inference about underlying rules. Iterative hypothesis refinement techniques \citep{qiu2023phenomenal} further enhance LLM-based inductive reasoning by iteratively proposing, selecting, and refining textual hypotheses of rules. Beyond rule-based hypotheses, \citet{wang2023hypothesis} prompt LLMs to generate natural language hypotheses that are then implemented as verifiable programs, while \citet{li2024automated} propose a method in which LLMs construct, critique, and refine statistical models represented as probabilistic programs for data modeling. \citet{cross2024hypothetical} leverage LLMs to propose and evaluate agent strategies for multi-agent planning but do not specifically infer individual mental variables. Our method also aims to achieve automated modeling with LLMs. Unlike prior works, we propose a novel automated model discovery approach for Bayesian inverse planning, where the objective is to confidently infer any mental variable given any context via constructing a suitable Bayesian Theory of Mind model.
\vspace{-5pt}
\section{Method}
\label{sec:method}
\section{Overview}

\revision{In this section, we first explain the foundational concept of Hausdorff distance-based penetration depth algorithms, which are essential for understanding our method (Sec.~\ref{sec:preliminary}).
We then provide a brief overview of our proposed RT-based penetration depth algorithm (Sec.~\ref{subsec:algo_overview}).}



\section{Preliminaries }
\label{sec:Preliminaries}

% Before we introduce our method, we first overview the important basics of 3D dynamic human modeling with Gaussian splatting. Then, we discuss the diffusion-based 3d generation techniques, and how they can be applied to human modeling.
% \ZY{I stopp here. TBC.}
% \subsection{Dynamic human modeling with Gaussian splatting}
\subsection{3D Gaussian Splatting}
3D Gaussian splatting~\cite{kerbl3Dgaussians} is an explicit scene representation that allows high-quality real-time rendering. The given scene is represented by a set of static 3D Gaussians, which are parameterized as follows: Gaussian center $x\in {\mathbb{R}^3}$, color $c\in {\mathbb{R}^3}$, opacity $\alpha\in {\mathbb{R}}$, spatial rotation in the form of quaternion $q\in {\mathbb{R}^4}$, and scaling factor $s\in {\mathbb{R}^3}$. Given these properties, the rendering process is represented as:
\begin{equation}
  I = Splatting(x, c, s, \alpha, q, r),
  \label{eq:splattingGA}
\end{equation}
where $I$ is the rendered image, $r$ is a set of query rays crossing the scene, and $Splatting(\cdot)$ is a differentiable rendering process. We refer readers to Kerbl et al.'s paper~\cite{kerbl3Dgaussians} for the details of Gaussian splatting. 



% \ZY{I would suggest move this part to the method part.}
% GaissianAvatar is a dynamic human generation model based on Gaussian splitting. Given a sequence of RGB images, this method utilizes fitted SMPLs and sampled points on its surface to obtain a pose-dependent feature map by a pose encoder. The pose-dependent features and a geometry feature are fed in a Gaussian decoder, which is employed to establish a functional mapping from the underlying geometry of the human form to diverse attributes of 3D Gaussians on the canonical surfaces. The parameter prediction process is articulated as follows:
% \begin{equation}
%   (\Delta x,c,s)=G_{\theta}(S+P),
%   \label{eq:gaussiandecoder}
% \end{equation}
%  where $G_{\theta}$ represents the Gaussian decoder, and $(S+P)$ is the multiplication of geometry feature S and pose feature P. Instead of optimizing all attributes of Gaussian, this decoder predicts 3D positional offset $\Delta{x} \in {\mathbb{R}^3}$, color $c\in\mathbb{R}^3$, and 3D scaling factor $ s\in\mathbb{R}^3$. To enhance geometry reconstruction accuracy, the opacity $\alpha$ and 3D rotation $q$ are set to fixed values of $1$ and $(1,0,0,0)$ respectively.
 
%  To render the canonical avatar in observation space, we seamlessly combine the Linear Blend Skinning function with the Gaussian Splatting~\cite{kerbl3Dgaussians} rendering process: 
% \begin{equation}
%   I_{\theta}=Splatting(x_o,Q,d),
%   \label{eq:splatting}
% \end{equation}
% \begin{equation}
%   x_o = T_{lbs}(x_c,p,w),
%   \label{eq:LBS}
% \end{equation}
% where $I_{\theta}$ represents the final rendered image, and the canonical Gaussian position $x_c$ is the sum of the initial position $x$ and the predicted offset $\Delta x$. The LBS function $T_{lbs}$ applies the SMPL skeleton pose $p$ and blending weights $w$ to deform $x_c$ into observation space as $x_o$. $Q$ denotes the remaining attributes of the Gaussians. With the rendering process, they can now reposition these canonical 3D Gaussians into the observation space.



\subsection{Score Distillation Sampling}
Score Distillation Sampling (SDS)~\cite{poole2022dreamfusion} builds a bridge between diffusion models and 3D representations. In SDS, the noised input is denoised in one time-step, and the difference between added noise and predicted noise is considered SDS loss, expressed as:

% \begin{equation}
%   \mathcal{L}_{SDS}(I_{\Phi}) \triangleq E_{t,\epsilon}[w(t)(\epsilon_{\phi}(z_t,y,t)-\epsilon)\frac{\partial I_{\Phi}}{\partial\Phi}],
%   \label{eq:SDSObserv}
% \end{equation}
\begin{equation}
    \mathcal{L}_{\text{SDS}}(I_{\Phi}) \triangleq \mathbb{E}_{t,\epsilon} \left[ w(t) \left( \epsilon_{\phi}(z_t, y, t) - \epsilon \right) \frac{\partial I_{\Phi}}{\partial \Phi} \right],
  \label{eq:SDSObservGA}
\end{equation}
where the input $I_{\Phi}$ represents a rendered image from a 3D representation, such as 3D Gaussians, with optimizable parameters $\Phi$. $\epsilon_{\phi}$ corresponds to the predicted noise of diffusion networks, which is produced by incorporating the noise image $z_t$ as input and conditioning it with a text or image $y$ at timestep $t$. The noise image $z_t$ is derived by introducing noise $\epsilon$ into $I_{\Phi}$ at timestep $t$. The loss is weighted by the diffusion scheduler $w(t)$. 
% \vspace{-3mm}

\subsection{Overview of the RTPD Algorithm}\label{subsec:algo_overview}
Fig.~\ref{fig:Overview} presents an overview of our RTPD algorithm.
It is grounded in the Hausdorff distance-based penetration depth calculation method (Sec.~\ref{sec:preliminary}).
%, similar to that of Tang et al.~\shortcite{SIG09HIST}.
The process consists of two primary phases: penetration surface extraction and Hausdorff distance calculation.
We leverage the RTX platform's capabilities to accelerate both of these steps.

\begin{figure*}[t]
    \centering
    \includegraphics[width=0.8\textwidth]{Image/overview.pdf}
    \caption{The overview of RT-based penetration depth calculation algorithm overview}
    \label{fig:Overview}
\end{figure*}

The penetration surface extraction phase focuses on identifying the overlapped region between two objects.
\revision{The penetration surface is defined as a set of polygons from one object, where at least one of its vertices lies within the other object. 
Note that in our work, we focus on triangles rather than general polygons, as they are processed most efficiently on the RTX platform.}
To facilitate this extraction, we introduce a ray-tracing-based \revision{Point-in-Polyhedron} test (RT-PIP), significantly accelerated through the use of RT cores (Sec.~\ref{sec:RT-PIP}).
This test capitalizes on the ray-surface intersection capabilities of the RTX platform.
%
Initially, a Geometry Acceleration Structure (GAS) is generated for each object, as required by the RTX platform.
The RT-PIP module takes the GAS of one object (e.g., $GAS_{A}$) and the point set of the other object (e.g., $P_{B}$).
It outputs a set of points (e.g., $P_{\partial B}$) representing the penetration region, indicating their location inside the opposing object.
Subsequently, a penetration surface (e.g., $\partial B$) is constructed using this point set (e.g., $P_{\partial B}$) (Sec.~\ref{subsec:surfaceGen}).
%
The generated penetration surfaces (e.g., $\partial A$ and $\partial B$) are then forwarded to the next step. 

The Hausdorff distance calculation phase utilizes the ray-surface intersection test of the RTX platform (Sec.~\ref{sec:RT-Hausdorff}) to compute the Hausdorff distance between two objects.
We introduce a novel Ray-Tracing-based Hausdorff DISTance algorithm, RT-HDIST.
It begins by generating GAS for the two penetration surfaces, $P_{\partial A}$ and $P_{\partial B}$, derived from the preceding step.
RT-HDIST processes the GAS of a penetration surface (e.g., $GAS_{\partial A}$) alongside the point set of the other penetration surface (e.g., $P_{\partial B}$) to compute the penetration depth between them.
The algorithm operates bidirectionally, considering both directions ($\partial A \to \partial B$ and $\partial B \to \partial A$).
The final penetration depth between the two objects, A and B, is determined by selecting the larger value from these two directional computations.

%In the Hausdorff distance calculation step, we compute the Hausdorff distance between given two objects using a ray-surface-intersection test. (Sec.~\ref{sec:RT-Hausdorff}) Initially, we construct the GAS for both $\partial A$ and $\partial B$ to utilize the RT-core effectively. The RT-based Hausdorff distance algorithms then determine the Hausdorff distance by processing the GAS of one object (e.g. $GAS_{\partial A}$) and set of the vertices of the other (e.g. $P_{\partial B}$). Following the Hausdorff distance definition (Eq.~\ref{equation:hausdorff_definition}), we compute the Hausdorff distance to both directions ($\partial A \to \partial B$) and ($\partial B \to \partial A$). As a result, the bigger one is the final Hausdorff distance, and also it is the penetration depth between input object $A$ and $B$.


%the proposed RT-based penetration depth calculation pipeline.
%Our proposed methods adopt Tang's Hausdorff-based penetration depth methods~\cite{SIG09HIST}. The pipeline is divided into the penetration surface extraction step and the Hausdorff distance calculation between the penetration surface steps. However, since Tang's approach is not suitable for the RT platform in detail, we modified and applied it with appropriate methods.

%The penetration surface extraction step is extracting overlapped surfaces on other objects. To utilize the RT core, we use the ray-intersection-based PIP(Point-In-Polygon) algorithms instead of collision detection between two objects which Tang et al.~\cite{SIG09HIST} used. (Sec.~\ref{sec:RT-PIP})
%RT core-based PIP test uses a ray-surface intersection test. For purpose this, we generate the GAS(Geometry Acceleration Structure) for each object. RT core-based PIP test takes the GAS of one object (e.g. $GAS_{A}$) and a set of vertex of another one (e.g. $P_{B}$). Then this computes the penetrated vertex set of another one (e.g. $P_{\partial B}$). To calculate the Hausdorff distance, these vertex sets change to objects constructed by penetrated surface (e.g. $\partial B$). Finally, the two generated overlapped surface objects $\partial A$ and $\partial B$ are used in the Hausdorff distance calculation step.

Our goal is to increase the robustness of T2I models, particularly with rare or unseen concepts, which they struggle to generate. To do so, we investigate a retrieval-augmented generation approach, through which we dynamically select images that can provide the model with missing visual cues. Importantly, we focus on models that were not trained for RAG, and show that existing image conditioning tools can be leveraged to support RAG post-hoc.
As depicted in \cref{fig:overview}, given a text prompt and a T2I generative model, we start by generating an image with the given prompt. Then, we query a VLM with the image, and ask it to decide if the image matches the prompt. If it does not, we aim to retrieve images representing the concepts that are missing from the image, and provide them as additional context to the model to guide it toward better alignment with the prompt.
In the following sections, we describe our method by answering key questions:
(1) How do we know which images to retrieve? 
(2) How can we retrieve the required images? 
and (3) How can we use the retrieved images for unknown concept generation?
By answering these questions, we achieve our goal of generating new concepts that the model struggles to generate on its own.

\vspace{-3pt}
\subsection{Which images to retrieve?}
The amount of images we can pass to a model is limited, hence we need to decide which images to pass as references to guide the generation of a base model. As T2I models are already capable of generating many concepts successfully, an efficient strategy would be passing only concepts they struggle to generate as references, and not all the concepts in a prompt.
To find the challenging concepts,
we utilize a VLM and apply a step-by-step method, as depicted in the bottom part of \cref{fig:overview}. First, we generate an initial image with a T2I model. Then, we provide the VLM with the initial prompt and image, and ask it if they match. If not, we ask the VLM to identify missing concepts and
focus on content and style, since these are easy to convey through visual cues.
As demonstrated in \cref{tab:ablations}, empirical experiments show that image retrieval from detailed image captions yields better results than retrieval from brief, generic concept descriptions.
Therefore, after identifying the missing concepts, we ask the VLM to suggest detailed image captions for images that describe each of the concepts. 

\vspace{-4pt}
\subsubsection{Error Handling}
\label{subsec:err_hand}

The VLM may sometimes fail to identify the missing concepts in an image, and will respond that it is ``unable to respond''. In these rare cases, we allow up to 3 query repetitions, while increasing the query temperature in each repetition. Increasing the temperature allows for more diverse responses by encouraging the model to sample less probable words.
In most cases, using our suggested step-by-step method yields better results than retrieving images directly from the given prompt (see 
\cref{subsec:ablations}).
However, if the VLM still fails to identify the missing concepts after multiple attempts, we fall back to retrieving images directly from the prompt, as it usually means the VLM does not know what is the meaning of the prompt.

The used prompts can be found in \cref{app:prompts}.
Next, we turn to retrieve images based on the acquired image captions.

\vspace{-3pt}
\subsection{How to retrieve the required images?}

Given $n$ image captions, our goal is to retrieve the images that are most similar to these captions from a dataset. 
To retrieve images matching a given image caption, we compare the caption to all the images in the dataset using a text-image similarity metric and retrieve the top $k$ most similar images.
Text-to-image retrieval is an active research field~\cite{radford2021learning, zhai2023sigmoid, ray2024cola, vendrowinquire}, where no single method is perfect.
Retrieval is especially hard when the dataset does not contain an exact match to the query \cite{biswas2024efficient} or when the task is fine-grained retrieval, that depends on subtle details~\cite{wei2022fine}.
Hence, a common retrieval workflow is to first retrieve image candidates using pre-computed embeddings, and then re-rank the retrieved candidates using a different, often more expensive but accurate, method \cite{vendrowinquire}.
Following this workflow, we experimented with cosine similarity over different embeddings, and with multiple re-ranking methods of reference candidates.
Although re-ranking sometimes yields better results compared to simply using cosine similarity between CLIP~\cite{radford2021learning} embeddings, the difference was not significant in most of our experiments. Therefore, for simplicity, we use cosine similarity between CLIP embeddings as our similarity metric (see \cref{tab:sim_metrics}, \cref{subsec:ablations} for more details about our experiments with different similarity metrics).

\vspace{-3pt}
\subsection{How to use the retrieved images?}
Putting it all together, after retrieving relevant images, all that is left to do is to use them as context so they are beneficial for the model.
We experimented with two types of models; models that are trained to receive images as input in addition to text and have ICL capabilities (e.g., OmniGen~\cite{xiao2024omnigen}), and T2I models augmented with an image encoder in post-training (e.g., SDXL~\cite{podellsdxl} with IP-adapter~\cite{ye2023ip}).
As the first model type has ICL capabilities, we can supply the retrieved images as examples that it can learn from, by adjusting the original prompt.
Although the second model type lacks true ICL capabilities, it offers image-based control functionalities, which we can leverage for applying RAG over it with our method.
Hence, for both model types, we augment the input prompt to contain a reference of the retrieved images as examples.
Formally, given a prompt $p$, $n$ concepts, and $k$ compatible images for each concept, we use the following template to create a new prompt:
``According to these examples of 
$\mathord{<}c_1\mathord{>:<}img_{1,1}\mathord{>}, ... , \mathord{<}img_{1,k}\mathord{>}, ... , \mathord{<}c_n\mathord{>:<}img_{n,1}\mathord{>}, ... , $
$\mathord{<}img_{n,k}\mathord{>}$,
generate $\mathord{<}p\mathord{>}$'', 
where $c_i$ for $i\in{[1,n]}$ is a compatible image caption of the image $\mathord{<}img_{i,j}\mathord{>},  j\in{[1,k]}$. 

This prompt allows models to learn missing concepts from the images, guiding them to generate the required result. 

\textbf{Personalized Generation}: 
For models that support multiple input images, we can apply our method for personalized generation as well, to generate rare concept combinations with personal concepts. In this case, we use one image for personal content, and 1+ other reference images for missing concepts. For example, given an image of a specific cat, we can generate diverse images of it, ranging from a mug featuring the cat to a lego of it or atypical situations like the cat writing code or teaching a classroom of dogs (\cref{fig:personalization}).
\vspace{-2pt}
\begin{figure}[htp]
  \centering
   \includegraphics[width=\linewidth]{Assets/personalization.pdf}
   \caption{\textbf{Personalized generation example.}
   \emph{ImageRAG} can work in parallel with personalization methods and enhance their capabilities. For example, although OmniGen can generate images of a subject based on an image, it struggles to generate some concepts. Using references retrieved by our method, it can generate the required result.
}
   \label{fig:personalization}\vspace{-10pt}
\end{figure}
\section{Result}

\subsection{Quantative Metrices}
In this paper, we use the distribution of drawn object categories and approaches to material utilization to illustrate the narrow creativity of human and GenAI. The object categories were determined based on a coding framework developed from an initial analysis of common design features and functional elements observed in the dataset. Each drawing was manually coded into a category according to its primary function and physical characteristics, following a structured coding process. This process involved two coders who categorized designs independently and resolved disagreements through discussion to ensure consistency and reliability in the coding scheme. The material utilization approaches were categorized by identifying and coding how materials were incorporated into the designs (e.g., direct use of materials and complex composition).

According to research on creativity evaluation \cite{tromp2024creativity, knight2015managing,li2008exploration}, creativity on generating new ideas are best understood by evaluating the exploration and exploitation of design space. 
To investigate the narrow creativity of human and GenAI by analyzing their exploration and exploitation of the explored design space, we developed the following quantitative metrics. 

\begin{itemize}
  \item \textbf{Number of used categories (\# of used cat.)} This metric represents the average number of distinct categories used by each participant during the task.
  \item \textbf{Number of frequent categories (\# of the equation cat.)} To identify the frequently used categories, we calculate the average and standard deviation of the number of circles within each category for an individual. A category is classified as 'frequent' if the number of circles in that category exceeds the average of the individual.
  \item \textbf{Number of highly frequent categories (\# of highly freq. cat.)} This metric identifies categories that are "highly frequent." A category is classified as such if the number of circles it contains exceeds the average by more than one standard deviation.

  \item \textbf{Proportations of drawings in the frequent categories (\% of the frequency category)} This metric counts the proportions of drawings made by an individual within their frequent categories.
  \item \textbf{Proportations of drawings in the highly frequent categories (\% of freq cat.)} Similar to the previous metric, this counts the total number of drawings within an individual’s highly frequent categories.
\end{itemize}

Exploration activity can be measured by the number of categories that an individual has used, while exploitation activity can be assessed by the number of frequent and highly frequent categories that an individual has. An individual who possesses a strong exploration mindset will have a relatively high number of categories used. In addition, the gap between the number of categories used and the number of frequent categories reveals the balance between exploration and exploitation.
When individuals explore multiple categories, a small gap indicates a good balance.

In order to further investigate the inclination between exploration and exploitation, we analyze the distribution of the circles among categories, which reveals how concentratedly an individual focuses on the frequent categories when doing the assignment. If a person has a large portion of circles that fall into the frequent or highly frequent categories, we can conclude that they lean toward exploitation and have narrow creativity issues.
Furthermore, conveying newly generated ideas is a part of the creativity process. In this work, we analyze the categories of artistic expression techniques that humans and GenAI frequently adopt. 

\subsection{Insights into Human Narrow Creativity from the Circles Excercise}

Human creative fixation often reflects a tendency to operate within familiar and constrained boundaries when engaging in creative tasks.
This tendency is evident in the results of the circle test, where individuals demonstrated preferences in object categories, artistic expression, and approaches to material utilization.
By clustering their creative output into key thematic aspects, we gain insight into the strategies humans used to interpret the task. Structurally encoding these strategies further allows us to pinpoint how fixation manifests in human creativity.
Furthermore, analysis of human creativity serves as a valuable baseline for interpreting similar patterns in the results generated by GenAI, enabling a deeper understanding of its capability and limitations.

\subsubsection{Distribution of Drawn Object Categories}

Human creativity tends to cluster around familiar categories of objects. 
Through analysis of students' drawings in the 28 Circles test, we observe that \textbf{daily objects} is the category most frequently used, while the \textbf{ mechanical} is the least adopted category. 
% We intentionally created this category because the students are mostly in the Department of Mechanical Engineering. 
The distribution suggests a tendency to draw inspiration from familiar, easily recognizable elements (daily objects), and reveals a clear preference for relatable, concrete objects over abstract forms or highly imaginative constructs. We provide the frequency of the average number of used categories in Figure \ref{humanobject}.
It shows that in the creativity task, humans exhibit a narrow and skewed bandwidth of creativity, with a significant inclination toward certain categories.



\begin{figure*}
    \includegraphics[width=0.9\textwidth]{figures/Object.pdf} % Adjust width as needed
    \captionof{figure}{The frequency distribution of categories of objects drawn by humans and GenAI, with a more even distribution across different categories indicating better performance.}
    \label{humanobject}
\end{figure*}

\begin{table*}
  \caption{Quantative analysis of human and GenAI narrow creativity based on the distribution of drawn object categories. For the number of used categories, frequent categories, and highly frequent categories, a higher value indicates better performance. Conversely, for the percentage of objects within frequent and highly frequent categories, a lower value reflects better performance.}
  \label{expression_table_1}
  \begin{tabular}{ccccccccc}
    \toprule
    Object Categories                    &  \multicolumn{2}{c}{Human}          & \multicolumn{2}{c}{Zero-shot} & \multicolumn{2}{c}{Few-shot} &\multicolumn{2}{c}{CoT}\\
    \midrule
                                               & Mean & Std.          & Mean & Std.     & Mean & Std.  & Mean & Std. \\
    \texttt{\# of used cat.}             & 5.6 & 2.1            & 6.6 & 1.6       & 5.6 & 2.0     & 7.5 & 0.97 \\
    \texttt{\# of freq. cat.}            & 2.5 & 1.2            & 2.5 & 0.66       & 2.8 & 1.1     & 3.2 & 0.78 \\
    \texttt{\# of highly freq. cat.}     & 1.4 & 0.96            & 1.4 & 0.65      & 1.2 & 0.28     & 1.6 & 0.69\\ \hline
    \texttt{\% of freq cat.}             & \multicolumn{2}{c}{70}             & \multicolumn{2}{c}{81}        & \multicolumn{2}{c}{77}      & \multicolumn{2}{c}{70}   \\
    \texttt{\% of highly freq. cat. }    & \multicolumn{2}{c}{47}             & \multicolumn{2}{c}{48}        & \multicolumn{2}{c}{43}      &\multicolumn{2}{c}{45}  \\
    \bottomrule
  \end{tabular}
\end{table*}

In this creativity exercise, we observe that human creativity only explores a narrow range when participating in the circle exercise. 
Specifically, an individual likely produces circles concentrated within only a limited number of categories. 
The average number of categories used, frequent categories, and highly frequent categories in all individuals is reported in Table \ref{expression_table_1}. 
Compared with the total number of object categories (10), these results suggest that humans tend to explore a limited subset of categories and frequently narrow their focus even further. 
Humans not only explore narrow perspectives, but also exploit even narrower ones. 
The percentage of objects belonging to frequent categories reveals that the majority of the drawn objects (70\%) fall into frequent categories. 
Thus, the result indicates that the creativity of individuals is strongly inclined towards a limited set of frequent categories.

\begin{figure*}
    \includegraphics[width=0.9\textwidth]{figures/Drawn_Object_Categories.pdf} % Adjust width as needed
    \captionof{figure}{Example of Drawn Object Categories: A) Human Sketched; B) GenAI-Generated, categorized into: 1) Animals, 2) Sport Equipment, 3) Foods, 4) Icons, 5) Daily Objects, and 6) Natural Elements.}
\end{figure*}

\subsubsection{Approaches to Material Utilization}
Students demonstrate diverse strategies for utilizing the circles for their ideas. These include \textbf{direct use}, where students transform the circles into recognizable objects such as clocks, wheels, or buttons by drawing directly within them; \textbf{personification}, where features of human faces are added to anthropomorphize the circles; \textbf{circle-based abstraction}, where the circles are used as references for similar shapes existing in other objects, such as a tennis racket, lollipop, and gear; \textbf{complex compositions}, where multiple circles were combined to form intricate objects, such as bicycles, ice-cream, and glasses; and \textbf{use as background}, where objects are draw within the circles. The frequency distribution of approaches to material utilization is shown in Figure \ref{humanutilization}. It indicates that humans also present narrow creativity and a skewed preference in terms of material utilization approaches.

\begin{figure*}
    \includegraphics[width=0.9\textwidth]{figures/Material_Utilization.pdf} % Adjust width as needed
    \captionof{figure}{Example of Approaches to Material Utilization: A) Human Sketched; B) GenAI Generated, categorized into: 1) Complex Compositions 2) Circle-based Abstraction 3) Use as Background.}
\end{figure*}


% \begin{center}
%     \includegraphics[width=0.6\textwidth]{figures/Distribution of Drawn Object Categories.png} % Adjust width as needed
%     \captionof{figure}{Place holder for example of Distribution of Drawn Object Categories}
% \end{center}

\begin{figure*}
    \includegraphics[width=0.7\textwidth]{figures/Utilization.pdf} % Adjust width as needed
    \captionof{figure}{The frequency distribution of approaches to material utilization is analyzed, with a more even distribution across different approaches indicating better performance.}
    \label{humanutilization}
\end{figure*}

Similarl to the analysis in the above subsection, Table \ref{expression_table_2} provides the average number of approaches used, frequent approaches, and highly frequent approaches. 
Based on the percentage of the frequent approaches, 80\% of the object are proposed based on one frequent approach of using the circles.

\begin{table*}
  \caption{Quantative analysis of human and GenAI narrow creativity based on material utilization approaches. For the number of used approaches, frequent approaches, and highly frequent approaches, a higher value indicates better performance. Conversely, for the percentage of objects within frequent and highly frequent approaches, a lower value reflects better performance.}
  \label{expression_table_2}
  \begin{tabular}{ccccccccc}
    \toprule
    Utilization Approaches                     &  \multicolumn{2}{c}{Human}          & \multicolumn{2}{c}{Zero-shot} & \multicolumn{2}{c}{Few-shot} &\multicolumn{2}{c}{CoT}\\
    \midrule
                                               & Mean & Std.          & Mean & Std.     & Mean & Std.  & Mean & Std. \\
    \texttt{\# of used apch.}                  & 3.0 & 1.2            & 3.5 & 0.96       & 3.4 & 1.2     & 3.6 & 0.69 \\
    \texttt{\# of freq. apch.}                 & 1.5 & 0.61           & 1.9 & 0.86       & 1.4 & 0.50    & 1.7 & 0.67 \\
    \texttt{\# of highly freq. apch.}          & 1.1 & 0.40           & 1.1 & 0.27       & - & -     & - & - \\ \hline
    \texttt{\% of freq apch.}                  & \multicolumn{2}{c}{80}             & \multicolumn{2}{c}{81}        & \multicolumn{2}{c}{63}      & \multicolumn{2}{c}{68}  \\
    \texttt{\% of highly freq. apch. }         & \multicolumn{2}{c}{68}             & \multicolumn{2}{c}{48}        & \multicolumn{2}{c}{51}      & \multicolumn{2}{c}{45}  \\
    \bottomrule
  \end{tabular}
\end{table*}

\subsubsection{Variation in Artistic Expression}

Although students' object categories and artistic expressions exhibit considerable diversity, the artistic styles and techniques they employ show limited variation, reflecting a monotonic and uniform approach to conveying their ideas. The approaches include \textbf{simple sketches}, where many students opt for minimalistic, black-and-white drawings focusing on the core concept; \textbf{detailed illustrations}, with some students enhancing their designs through intricate details that add depth and character. The portion of the two is presented in Table \ref{expression_table_2}, which suggests that humans almost adopt simple sketches. A possible reason might be that humans emphasize task efficiency and believe simple sketches are effective enough to convey their ideas.

Besides sketching, some students incorporate additional elements to facilitate their expression. The elements are summarized as follows: \textbf{use of color}, where some students incorporate vibrant colors to enrich their visual representations; and \textbf{annotations and labels}, where some drawings include textual annotation to explain or narrate their drawings, adding an interpretive layer to their visual output. 
The portion of drawings that use the additional elements is reported in Table \ref{expression_table_3}. 
The relatively low portion of drawings with additional elements (5\%) suggests that humans have limited capability to convey their ideas with detailed expressions. It might be because adding additional elements is time-consuming and requires extra resources. 
Figure \ref{fig: example of art} shows that while the GenAI outputs demonstrate enhanced refinement, they require explicit instructions to incorporate annotations or labels.


\begin{figure*}
    \includegraphics[width=0.9\textwidth]{figures/Artistic_Expression.pdf} % Adjust width as needed
    \captionof{figure}{Example of Artistic Expression: A) Human Sketched; B) GenAI-Generated, categorized into: 1) Simple Sketches 2) Detailed Illustrations 3) Use of Color 4) Annotations and Labels. }
    \label{fig: example of art}
\end{figure*}

\begin{table*}
  \caption{The portion of the frequent artistic expression of humans and GenAI.}
  \label{expression_table_3}
  \begin{tabular}{ccccc}
    \toprule
    Artistic Expressions                     & Human          & Zero-shot & Few-shot & CoT\\
    \midrule
    \texttt{\% of simple sketches}           & 95             & 2.9        & 10      & 3.2\\
    \texttt{\% of detailed illustrations}    & 5              & 97       & 89      & 97\\ \hline
    \texttt{\% of use of colors}             & 16             & 52        & 20      & 70\\
    \texttt{\% of use of annotations}        & 13             & 0.0        & 0.0     & 0.0\\
    \bottomrule
  \end{tabular}
\end{table*}

\subsection{Understanding GenAI Narrow Creativity with Different Prompting Strategies}

We present the results of different prompting strategies that are frequently adopted in GenAI-augmented creativity support tools.
We clustered the GenAI result based on the same aspects of narrow creativity as we used for human result.
This analysis offers insights into how AI-augmented creativity can complement or challenge human tendencies, revealing both the limitations and opportunities of prompting strategies in addressing narrow creative issues.

\subsubsection{Naive Prompting}

We conducted a pilot study to evaluate the results of GenAI under the condition of naive prompting.
We adopted the same quantitative metrics that were used to evaluate human creativity.
The zero-shot prompts provided to GenAI were based on a pre-articulated template, as described in the appendix.

For few-shot prompting, we provide each prompt with an examples from the results.

The statistics for zero-shot and few-shot prompting align closely with observations from human data (object categories: human=5.6, zero-shot=6.6, few-shot=5.6; utilization approaches: human=3.0, zero-shot=3.5, few-shot=3.4). 
The distribution of object categories suggests that GenAI, under naive prompting strategies, exhibits a similar pattern of narrow creativity as humans in this task. Interestingly, compared to zero-shot prompting, few-shot prompting produces GenAI results that are more quantitatively aligned with human performance. This observation implies that providing human examples may lead to a similar representation of narrow creativity in GenAI, particularly if the prompts are not further refined or articulated to encourage more diverse outputs.


In terms of artistic expression, humans and GenAI demonstrate differing preferences. Most humans (95\%) prefer to express their ideas through simple sketches. By contrast, GenAI models tend to favor detailed illustrations (zero-shot=97\%, few-shot=89\%), likely due to their stronger image-generation capabilities.
A similar pattern is observed in the use of color. Few-shot prompted GenAI exhibits preferences more aligned with human behavior (human=16\%, few-shot=20\%) due to exposure to human examples during training. This suggests that the inclusion of human examples in few-shot prompting can guide GenAI to mimic certain human tendencies, albeit within the constraints of its learned patterns.


\subsubsection{Chain-of-Thought Prompting}

To better understand the capability of GenAI, we adopted the Chain-of-Thought (CoT) prompting strategy to perform a circle test with GenAI. 
While CoT is considered an advanced technique to enhance GenAI's reasoning capabilities, the experiment results reveal GenAI under CoT prompting demonstrate similiar pattern of narrow creativity as human does.

The result demonstrate that a significant proportion of the objects generated by GenAI under CoT prompting belong to frequently used categories (70\%). This mirrors the behavior seen in other strategies (e.g., zero-shot=81\% and few-shot=70\%), showing that CoT does not substantially expand the variety of generated object categories.
Moreover, 45\% of the objects belong to highly frequent categories, reinforcing the observation that GenAI under CoT also tend to explore on narrowed regions of design space, rather than exploring more diverse ideas.
In terms of material utilization approaches, CoT-generated ideas also exhibit constrained diversity. Approximately 68\% of the approaches employed by GenAI under CoT prompting narrow to the most frequently used methods.
This suggests that CoT does not effectively overcome the bias toward relying on dominant patterns of material utilization.

These results highlight a persistent reliance on frequent object categories and approaches. While CoT improves reasoning capabilities, it does not necessarily enhance the creative breadth of GenAI, as it struggles to generate ideas that break away from the narrow design space.

\section{Discussion}

% Shift from findings to discussion
This study on robotic art explores human-machine relationships in creative processes.
It first contributes as an empirical description of artistic creativity in robotic art practice---an unconventional use of robots---examined through the artists' perspectives on their creative experiences. Our analysis reveals three facets of creativity in robotic art practices: the \textit{social}, \textit{material}, and \textit{temporal}. Creativity emerges from the co-constitution between artists, robots, audience, and environment in spatial-temporal dimensions, as illustrated in \autoref{PracticeDiagram}. Acknowledging the audience as an important actor reflects the social dimension in that creativity does not stem from the artists but from their interactions with the audience. Robots are the major material and technological actants characterizing creative practices, shaping the conditions for creativity to emerge. The axis of the temporal process signifies that the practice is situated within a time continuum, and the actors/actants and their relations shift over time. In this way, temporality constitutes another dimension of creativity in robotic art.

Accordingly, as the second contribution, this study outlines implications for \textit{socially informed}, \textit{material-attentive}, and \textit{process-oriented} creation with computing systems\footnote{For the sake of clarity, we mean ``creation with computing systems'' by three types of scenarios: human creator(s) create computing system(s) as the final artifact(s) (e.g., robots are artworks themselves); human creator(s) use computing system(s) to create the artifact(s) (e.g., robots create artworks as human planned); and human creator(s) and system(s) work in tandem to produce the artifact(s) (e.g., human-robot co-creation).} to facilitate creation practices. These insights can inform related HCI research on media/art creation, crafting, digital fabrication, and tangible computing.
In each following subsection, we present each implication with a grounding in corresponding findings from our study and relevant literature in HCI and adjacent fields on art, creativity, and creation.

\begin{figure*}[htbp]
    \centering
    \includegraphics[width=0.88\textwidth]{Writings/figure/PracticeDiagram.pdf}
    \caption{Actors/actants in robotic art practice and their interactive relations. Robotic art practice unfolds primarily in two spaces: the creation space where interactions happen mainly between artists and robots, and the exhibition space where interactions mostly involve audiences and robots. The two spaces constitute the ENVIRONMENT plane. Within the plane, directed arrows between the actors indicate the types of interaction. For example, the \textit{Design} arrow indicates that the artist designs the robot(s), and the \textit{Revise} arrow indicates that the robot(s) make the artist revise artistic ideas. All the actors/actants may also intra-act with the ENVIRONMENT. The actors/actants and their interactive relations may differ at different times along the axis of TEMPORAL PROCESS that is orthogonal to the plane.}
    \Description{This figure shows the actors/actants in robotic art practice and their interactive relations.}
    \label{PracticeDiagram}
\end{figure*}

\subsection{Socially Informed Creation}

% Introduce social aspect of distributed creativity
The sociality of creativity means that creativity is distributed among different human actors, commonly within the creators or between the creators and the audience. Glăveanu’s ethnographic study on Easter egg decoration in northern Romania~\cite{glaveanu_distributed_2014} showed that artisans anticipate how others might appreciate their work and adjust their creative decisions accordingly. Even in the absence of direct interaction, the audience’s potential responses become part of the creative process, as artisans imagine feedback and predict reactions. In this sense, the sociologist Katherine Giuffre argues that ``\textit{creative individuals are embedded within specific network contexts so that creativity itself, rather than being an individual personality characteristic is, instead, a collective phenomenon}''~\cite[p. 1]{giuffre2012collective}.

% Recall findings about audience feedback
We found that the practice of robotic art manifests this sociality as it involves, particularly artists and audiences. 
Our artists take audiences' reactions to their artwork as feedback and then revise the robots' functions and aesthetics accordingly. 
For example, as shown earlier, Robert added a protective fuse onto his robot because he expected that children would squeeze the springs together and cause a short circuit; Alex's enthusiasm and attention to the audience's imagination about his robots led him to new aesthetic designs of both the robots and the scene layouts. The artists may directly ask about the audience's judgment of quality but they often receive feedback just by observing the audience's reactions or sometimes by learning from the audience's imagination about the robots.
% Recall findings about audience's sociocultural expectations and codes
Meanwhile, our findings reveal that audience reception is not an individual matter but is often associated with their sociocultural codes, including shared cultural norms, beliefs, expectations, and aesthetic values. The audience can be seen as representatives of these broader cultural codes. For example, Mark and Robert observed that the animist tendency in some East Asian societies is associated with higher acceptance of and interest among the audience in intelligence and agency of robots and non-human entities. Such sociocultural contexts influence not only how audiences interpret the work but also how artists anticipate and respond to these perspectives in their creative process.

% Situate in HCI literature
A creative process, including creation and reception, is essentially a social activity. The second wave of creativity research in psychology has argued for creativity's dependency on sociocultural settings and group dynamics~\cite{sawyer2024explaining}. Recent discussions from creativity-support and social computing researchers also called for more attention to the social aspect of creativity~\cite{kato2023special, fischer2005beyond, fischer2009creativity}. There is a clear need to consider the audience when producing creative content. For instance, researchers studying video-creation support have examined audience preferences to inform system designs that align with these preferences~\cite{wang2024podreels}. Such work highlights how creative activities extend beyond individual creators to co-creators and heterogeneous audiences. Some HCI researchers conceptualize creativity as by large a socially constructed concept, perceived and determined by social groups~\cite{fischer2009creativity}. 
Prior HCI work examined the social aspects between art creators. For example, creators and performers in music and dance form social relationships through artifacts, making the final work a collaborative outcome~\cite{hsueh2019deconstructing}. There is also a system designed to support collaborative creation between artists~\cite{striner2022co}. However, the social creative process between creators and audience is less articulated in HCI. Jeon et al.'s work~\cite{jeon2019rituals} stands as an exception, suggesting that professional interactive art can involve evaluation with the audience in the creation stage. 
Another relevant approach in HCI involves enabling the general public to participate in co-creation alongside professional creators. ~\citet{matarasso2019restless}, for instance, promoted ``participatory art'' as ``\textit{the creation of an artwork by professional artists and non-professional artists working together}'' with non-professional artists referring to the general public engaged in the art-making process. Similarly, socially inclusive community-based art also considers target communities' perception of the artwork during creation~\cite{clark2016situated, clarke2014socially}. But like participatory design~\cite{schuler1993participatory}, these art projects aim for social justice more than creativity in the work~\cite{murray2024designing}, let alone that direct participation in art creation is not always feasible. Our findings suggest that feedback from the audience can lead to creative ideas, as well as that the feedback can be generative and remain low-effort for the audience.

Unlike conventional design feedback---which is typically expected to be specific, justified, and actionable~\cite{yen2024give, krishna2021ready}---the feedback that resonates with our artists is often implicit, creative, and generative. Such feedback may include audiences' imaginations stimulated by the work, personal and societal reflections, and even emotions, facial expressions, micro-actions, and observable behaviors following the art experience. Our artists gathered this implicit feedback not by posing evaluative questions, as commonly done in typical design processes (e.g., usability testing, think-aloud protocols), which seek to elicit clear, relatively structured responses. Instead, they closely observe the audience's reactions and interpret their subjective perceptions. This form of implicit feedback, while indirect, can lead to more creative ideas by embracing open, multifaceted interpretations of the work~\cite{sengers2006staying}. Computing systems for creation should better incorporate implicit feedback in addition to explicit ones from the audience into the creation process. Implicit feedback can be indirect, creative, inspirational, and heuristic about functions and aesthetics. A hypothetical instance of such design can be a system that helps creators perceive audiences' implicit reactions and perceptions and variously interpret them, for further iteration.

% Recall findings about audience interacting with robots as a performative art
Moreover, as seen in Robert and Daniel's experiences, the audience may participate in robotic live performances by interacting with the robots, who may change actions accordingly, triggering a loop of simultaneous mutual influence that makes the work performative and improvisational.
% Situate in HCI
HCI researchers explored performative and improvisational creation with machines, focusing on developing and evaluating systems with performative capabilities, including music improvisation with robots~\cite{hoffman2010shimon}, dance with virtual agents~\cite{jacob2015viewpoints, triebus2023precious}, and narrative theatre~\cite{magerko2011employing, piplica2012full}. \citet{kang2018intermodulation} discussed the improvisational nature of interactions between humans and computers and argued that an HCI researcher-designers' improvisation with the environment facilitates the emergence of creativity and knowledge. Designs of computing systems for creation can leverage performativity in service of creative experience. One possible direction could be to allow the audience to embed themselves in and interact with elements of static artwork in a virtual space, turning the exhibition into an improvisational on-site creation~\cite{zhou2023painterly}.
% Our new implication different from current discussion on perf and impr
While interactions with machines during performance are mostly physical or embodied, we posit that they can also be a \textit{symbolic engagement}. Alex's audience projected themselves and their personalities onto his robots, which established a symbolic relevance, generating creative imaginations. During exhibitions, East Asian audiences carried the animist views shaped by their sociocultural backgrounds, and robots, through the performance, were successful in symbolically matching the views, stimulating aesthetic satisfaction. Symbolic engagement resonates with what ~\citet{nam2014interactive} called the ``reference'' of the interactive installation performance to participants' sociocultural conditions.
As such, we propose that designers of computing systems for creation may consider establishing symbolic engagement between the produced artifacts and the audience as a way to enhance perceived creativity or enrich the creative experience. One example is an interactive installation, \textit{Boundary Functions}~\cite{snibbe1998}, which encourages viewers to reflect on their personal spaces while interacting with the installation and others. Another example is \textit{Blendie}, a voice-controlled blender that requires a user to ``speak'' the machine's language to use it. This interaction builds a symbolic connection between the user and the device, transforming the act of blending into a novel experience~\cite{dobson2004blendie}.


\subsection{Material-Attentive Creation}

% Intro paragraph to the importance of materiality for creative activities with machines and the end goal of this discussion--- design suggestions
The theory of distributed creativity by Glaveanu claims that creativity distributes across humans and materials, so the creation practice itself is inevitably shaped by objects~\cite{glaveanu_distributed_2014}. In his case of Easter egg decoration, materials are not passive objects but active participants in artistic creation; e.g., the egg decorators face challenges from color pigments not matching the shell, wax not melted at the desired temperature, to eggs that break at the last step of decoration; hence, materials often go against the decorators' intentions and influence future creative pathways~\cite{glaveanu_distributed_2014}.
Materials manifest specific properties, which afford certain uses of the materials while constraining others~\cite{leonardi2012materiality}. Our findings highlight the critical role of materiality in artistic practice, showing that artists intentionally arrange materials to enhance the creative values of their work.

% Materiality aspect One: physicality and embodiment
% Embodiment or physicality fascilitates creative interaction with machines
Robotic art relies on the material properties of robots and other objects. An apparent property of most materials is their physicality~\cite{leonardi2012materiality}, meaning they possess a tangible presence that enables interaction with other physical entities. Here, we consider physicality and embodiment interchangeable as computational creativity researchers have conceptualized~\cite{guckelsberger2021embodiment}.
% Recall findings on embodiment's value in making art
Our findings support both the conceptual and operational contributions of embodiment for creative activities. For the conceptual aspect, the embodied presence of robotic systems supports creative thinking for our artists, exemplary in Linda's case where she found new art ideas around the difference between human and robot bodies through bodily engagement with robots. 
For the operational aspect, the embodied nature of robotic artworks and their creation processes exhibit original aesthetics that are based on physics much different from disembodied works, e.g., embodied drawings by David's non-industrial robotic arms are dynamic due to physical movements and thus artistically pleasant, which is hard to replicate in simulated programs.

% References: embodied interaction, embodied cognition theories, tangible computing
These findings on embodiment of robotic art (Section \ref{f:emb}) closely relate to HCI's attention on embodied interaction as a way to leverage human bodies and environmental objects to expand disembodied user experiences. 
For example, as~\citet{hollan2000distributed} explained, a blind person's cane and a cell biologist's microscope as embodied materials are part of the distributed system of cognitive control, showing that cognition is distributed and embodied. 
Similarly, theories of embodied interaction in HCI explicate how bodily interactions shape perception, experience, and cognition~\cite{marshall2013introduction, antle2011workshop, antle2009body}, backed up by the framework of 4E cognition (embodied, embedded, enactive, and extended)~\cite{wheeler2005reconstructing, newen20184E}. 
Prior works suggest that creative activities with interactive machines rely on similar embodied cognitive mechanisms ~\cite{guckelsberger2021embodiment, malinin2019radical}, which are operationalized by tangible computing~\cite{hornecker2011role}. 
% References: embodiment's consequence in creation
As related to robots in creation, HCI researchers show that physicality or embodiment of robots in creation may lead to some beneficial outcomes, such as curiosity from the audience, feelings of co-presence, body engagement, and mutuality, which are hard to simulate through computer programs~\cite{dell2022ah, hoggenmueller2020woodie}. Embodied robotic motions convey emotional expressions and social cues that potentially enrich and facilitate creation activities like drawings~\cite{ariccia2022make, grinberg2023implicit, dietz2017human, santos2021motions}. Guckelsberger et al.~\cite{guckelsberger2021embodiment} showed in their review that embodiment-related constraints (e.g., the physical limitations of a moving robotic arm) can also stimulate creativity. These constraints push creators to develop new and useful movements, echoing the broader principle that encountering obstacles in forms or materials can lead to generative processes. This phenomenon is similarly observed in activities such as art and digital fabrication~\cite{devendorf2015being, hirsch2023nothing}. In co-drawing with robots, physical touch and textures of drawing materials made the artists prefer tangible mediums (e.g., pencils) than digital tools (e.g., tablets) that fall short in these respects~\cite{jansen2021exploring}.

% Transit to materiality aspect two
% Materiality aspect Two: malfunction as manifestation of unique materiality of robots
% Intro to materials of robots
Materiality plays a crucial role in the embodiment of robots, as the choice of materials fundamentally shapes the physical forms and properties. This focus on materials extends to art practices, where robots made with soft materials introduce new aesthetics and sensory experiences~\cite{jorgensen2019constructing, belling2021rhythm}, and the use of plants and soil in robotic printing creates unique visual effects~\cite{harmon2022living}. Following Leonardi's ~\cite{leonardi2012materiality} conceptualization of materiality, we refer to the materials of robots as encompassing physical and digital components---including the shell, hardware, mechanical parts, software, programs, data, and controllers---each significant to the artist's intent. ~\citet{nam2023dreams} found that the material constraints of robots can limit creative expression but simultaneously stimulate creativity when artists push the boundaries.

%-----maybe here the real "malfuction" start ------------------
% Move to introduce malfunctions as unique materiality

Even carefully designed, digital and mechanical components in robots are prone to errors or bugs in everyday runs, causing malfunctions or unexpected consequences. This reflects the unique materiality of robots as complex computing systems. From an engineering perspective, errors signal unreliability and must be eliminated, driving advancements in robotics---where error detection and recovery are central~\cite{gini1987monitoring}---as well as in digital fabrication, which prioritizes precision over creative exploration~\cite{yildirim2020digital}. % Recall findings on embracing malfunctions
However, material failures and accidents are inevitable, exemplifying what has been called the `craftsmanship of risk'~\cite{glaveanu_distributed_2014} in material art. For our artists, these risks are often creatively utilized and incorporated into their work: these moments of breakdown---whether physical or digital---become resources for new creative expression. Errors are anticipated and intentionally designed into the process and work of our artists. In some cases, such as for Alex, the entire concept of one of his works is machine errors.

% Situate in literature
Reports on how artists view errors within engineering and creation processes are dispersed throughout HCI literature. ~\citet{nam2023dreams} showed that the accumulation of ``contingency'' and ``accidents''---unexpected, serendipitous, and emergent events during art creation like errors---meaningfully constituted the final presentation of the artwork. Song and Paulos's concept of ``unmaking'' highlighted the values of material failures in enabling new aesthetics and creativity~\cite{song2021unmaking}. Kang et al.~\cite{kang2022electronicists, kang2023lady} introduced the notion of an ``error-engaged studio'' for design research in which errors in creative processes are identified, accommodated, and leveraged for their creative potential. Collectively, these works advocate for reframing errors from something to avoid to something to embrace and recognize. We want to push this further by arguing that errors can be intended and be part or sometimes entire of the design. Several artists, including participants from our study, have been deliberately seeking errors to formulate their designs. Roboticist Damith Herath recounted when he mistakenly programmed a motion sequence of a robotic arm, his collaborator, robotic artist Stelac responded with ``[W]e need to make more mistakes;'' as many mistakes were made, the initial pointless movements became beautiful, rendering the robot ``alive'' and ``seductive'' \cite{herath2016robots}. Similarly, AI artists sometimes look for program glitches to generate unusual styles and content~\cite{chang2023prompt}. Therefore, creators may not only passively accept errors but can actively seek and utilize them. Errors can be integral to the design itself---errors can \textit{be designed into} an artifact, and the design/idea of the artifact can be all about errors.

Thus, to focus on material-attentive creation---considering the creative arrangement of materials---we suggest exploring the embodiment and materiality of creation materials, objects, and environments to recognize their creative potential. %This perspective aligns with insights from professional digital fabrication practitioners, who advocate for systems that integrate support for machine settings and material properties~\cite{hirsch2023nothing}.
Specifically, we propose using a design method/probe that enables creators to realize both the conceptual and operational contributions of materiality. This approach may build on the material probe developed by~\citet{jung2010material}, which calls for exploring the materiality of digital artifacts. A material-attentive probe would enable creators to engage with diverse materials, objects, and environments through embodied interaction, encouraging them to speculate on material preferences and limitations, and to compare and contrast material qualities---insights that can inform creative decisions.
To accommodate, seek, and actively harness the creative potential of errors, we propose embracing failures, glitches, randomness, and malfunctions in computing systems as critical design materials---elements that creators can intentionally control and manipulate. By doing so, we can begin to systematically approach errors. For instance, as part of the design process, we may document how to replicate these errors and changes, allowing creators to explore them further at their discretion. This could include intentionally inducing errors or random changes to influence the creative process or outcomes.

\subsection{Process-Oriented Creation}

% Introduce the key idea: process itself embeds creative value and can be pursued as the goal of creation
As shown in our findings, the creation process itself embeds creative values and meanings, and experiencing the process can be pursued as the goal of creation with computing systems.
% Recall findings
For the robotic artists in our study, artistic values were often placed on the creation process rather than the outcome.  For example, in Alex's robotic live drawing performance, the drawing process is more important than the drawn pattern on canvas. Techniques used, decisions made, or stimuli received by robots during creation or exhibition reflect artistic ideas and nuanced thinking, as seen in Sophie's exploration of interactive decision-making in robotic drawing.

% Situate in HCI lit
Previous HCI work has touched on the value of the process of creation. ~\citet{bremers2024designing} shared a vignette where a robotic pen plotter simultaneously imitates the creator's drawing, serving as a material presence rather than a pragmatic co-creator; here the focus of the work is no longer the outcome but the process of drawing itself. ~\citet{devendorf2015reimagining} concluded that performative actions of digital fabrication systems, rather than the fabricated products themselves, convey artistic meanings tied to histories, public spaces, time, environments, audiences, and gestures. This emphasis on process is particularly significant for media such as improvisational theatre, where the creation itself is an integral part of the final work~\cite{o2011knowledge}. ~\citet{davis2016empirically} named their improvisational co-drawing robotic agents as ``casual creators,'' who are meant to creatively engage users and provide enjoyable creative experiences rather than necessarily helping users make a higher quality product. Shifting the focus from product to process and experiences \textit{in} creation may generate alternative creative meanings.

% Findings about process extends beyond creation
Our artists pointed out that even a ``finished'' artwork in an exhibition is not truly finished. A crack in Daniel's robotic artwork introduced a new artistic meaning, ultimately subverting the entire work. As the properties of the work change over time---whether due to the artist's intent, material characteristics, or environmental factors---the artwork evolves, revealing new aesthetics and meanings. % Situate in HCI lit
Based on these observations, we argue that creation processes should not be regarded as one-shot transactions, as creative artifacts, particularly physical ones, continue to change and generate artistic values. For instance, material wear and destruction bring unique aesthetics, often contrasting with the original form ~\cite{zoran2013hybrid}, and are seen as signs of mature use~\cite{giaccardi2014growing}.
Changes such as material failure, destruction, decay, and deformation---what~\citet{song2021unmaking} referred to as ``unmaking,'' a process that occurs after making---meaningfully transforms the original objects. Similarly, through Broken Probes, a process of assembling fractured objects, ~\citet{ikemiya2014broken} demonstrated that personal connections, reminiscence, and reflections related to material wear and breakage add new values to the objects. Drawing from Japanese philosophy Wabi-Sabi, ~\citet{tsaknaki2016expanding} reflected on the creeds of `Nothing lasts,' `Nothing is finished,' and `Nothing is perfect' and pointed to the impermanence, incompleteness, and imperfection of artifacts as a resource that designers, producers, and users can utilize to achieve long-term, improving, and richer interactive experience~\cite{tsaknaki2016things}. Insights from this study contribute to this line of thought by showing how robotic artists appreciate the aesthetics and meanings of temporal changes after the creation phase.

The findings underscore the need to reconceptualize creation as encompassing more than just the process aimed at producing a final product; it also includes what we term \textit{post-creation}. Distinct from repair, maintenance, or recycle, \textit{post-creation} entails anticipating and managing how an artifact evolves after its ``completion'' in the conventional sense. Specifically, we encourage creators to anticipate and strategically engage with the post-creation phase, considering potential changes to the artifact and their consequences for interactions with human users. For instance, during the creation process, creators may focus on possible material changes the artifact might undergo post-creation, allowing them to either mitigate or creatively exploit these potential changes. This expanded view of creation invites us to trace post-creation developments and to plan how our creative intentions can be embedded in its potential degradation, transformation, or evolution over time.

% A conclusion paragraph
We categorize the design implications into three aspects, but we do not suggest that a computing system must implement all simultaneously, nor that each aspect should be considered in isolation. Social interactions, such as those between artists and audiences, already presume the presence of material actants like robots, and these interactions inform future arrangements of materials. Thus the social and material aspects can be entangled and mutually constitutive as seen in sociomaterial practices~\cite{orlikowski2007sociomaterial, cheatle2015digital, rosner2012material}. The temporal aspect is orthogonal to the other aspects because both social interactions and material manifestations unfold and shift in a temporal continuum.

\section{Conclusion }
This paper introduces the Latent Radiance Field (LRF), which to our knowledge, is the first work to construct radiance field representations directly in the 2D latent space for 3D reconstruction. We present a novel framework for incorporating 3D awareness into 2D representation learning, featuring a correspondence-aware autoencoding method and a VAE-Radiance Field (VAE-RF) alignment strategy to bridge the domain gap between the 2D latent space and the natural 3D space, thereby significantly enhancing the visual quality of our LRF.
Future work will focus on incorporating our method with more compact 3D representations, efficient NVS, few-shot NVS in latent space, as well as exploring its application with potential 3D latent diffusion models.

\newacronym{rl}{RL}{Reinforcement Learning}
\newacronym{drl}{DRL}{Deep Reinforcement Learning}
\newacronym{mdp}{MDP}{Markov Decision Process}
\newacronym{ppo}{PPO}{Proximal Policy Optimization}
\newacronym{sac}{SAC}{Soft Actor-Critic}
\newacronym{epvf}{EPVF}{Explicit Policy-conditioned Value Function}
\newacronym{unf}{UNF}{Universal Neural Functional}

%%%%%%%%%%%%%%%%%%%%%%%%%%%%%%%%%%%%%%%%%%%%%%%%%%%%%%%%%%%%%%%%%%%%%%%%%%%%%%%%
\iffalse
\section*{APPENDIX}

\begin{itemize}
    \item Furter resources on mono, stereo and event camera latency.
    \item Datasheet extracts to back our claims, if needed.
    \item Elaboration of the testing setup.
\end{itemize}
\fi

\section*{ACKNOWLEDGMENT}
The authors would like to thank Hiroyuki Izumi, Kensuke Kitamura, Ichitaro Kohara, Fumitaka Joo, Toshihisa Sambommatsu, Takuma Morita, and Yuichiro To at Sony Group Corporation for software integration help. 
Thanks to Yuntao Ma, Mayank Mittal and Jonas Frey at ETH Zurich for discussion about reinforcement learning.  

\clearpage

%%%%%%%%%%%%%%%%%%%%%%%%%%%%%%%%%%%%%%%%%%%%%%%%%%%%%%%%%%%%%%%%%%%%%%%%%%%%%%%%

\bibliographystyle{IEEEtran}
\bibliography{main} 

\end{document}
