\section{Conclusion}
This work investigates effective data synthesis for long-context instruction-tuning.
Through a pilot study on controllable needle-in-a-haystack tasks, we identify that instruction difficulty, context composition, and context length all play crucial roles.
Based on these insights, we propose a novel synthesis approach called ``context synthesis''.
Experiment results on document-level question-answering and document-level summarization tasks demonstrate that our method not only outperforms the previous instruction synthesis approach but also achieves comparable performance to oracle human-annotated data.
Furthermore, our approach shows robust generalization to unseen tasks not covered during data synthesis. 
Additionally, we quantitatively assess instruction-context coherence, revealing new insights for designing effective long-context instruction data.

\section*{Limitations}
Our work has certain limitations.
While we evaluate on document-level question-answering and summarization tasks, these may not fully cover all real-world scenarios.
We plan to extend our scope as more practical long-context benchmarks emerge.
Additionally, as our context synthesis approaches require context-aware instructions as a starting point, an automated verification framework would be beneficial to filter such instructions from large instruction pools - a direction we leave for future work.
Moreover, while experiments in this study focuses on LLMs with quadratic attention mechanisms, we notice the recent trend of developing large language models with linear attention mechanisms. 
This shift may raise new research questions about long-context modeling, which we plan to explore in future work.

\section*{Acknowledgement}
We would like to thank Qipeng Guo, Yijun Yang for their suggestions and feedback.
This work is supported by National Science Foundation of China (No. 62376116, 62176120), the Fundamental Research Funds for the Central Universities (No. 2024300507) and research project of Nanjing University-China Mobile Joint Institute.
This project has also received funding from UK Research and Innovation (UKRI) under the UK government's Horizon Europe funding guarantee (grant numbers 10039436 and 10052546).
Wenhao Zhu is also supported by China Scholarship Council (No.202306190172).
Shujian Huang is the corresponding author. 