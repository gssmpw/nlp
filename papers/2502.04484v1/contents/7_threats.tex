
\textbf{Construct Validity.} 
We rely on information that is available through the Hugging Face API in a machine readable format. We addressed cases where information was missing from this source by applying Python's \texttt{request} and BeautifulSoup libraries to retrieve missing information. However, this means that we likely missed licensing and dependency information that was presented in natural language.  We measure and mitigate this risk with a review of the top-100 ``unlicensed'' models sorted by number of downloads, which showed that the majority did not, in fact, declare a license or declared a license in a non-standard or transitive way.  %
 
\textbf{Internal Validity}.
We followed best practices in our mining and data cleaning approaches. Conclusions were drawn based on empirical evidence found in the data. We relied on tags applied to components to detect their licenses, but such information might be present in other, human-readable forms or through links to other resources, such as a research paper or another repository. 
The labeling of license categories is based on our understanding of the licenses. To mitigate bias in these labels, two researchers examined each license to confirm the label applied to it. %

\textbf{External Validity.}
The generalizability of our findings is restricted by our choice of Hugging Face, which might not be fully generalizable to all model-sharing platforms. It is possible that we missed other challenges and obstacles faced by other model hubs or that different model-sharing sites have different levels of documentation and dependency management techniques. Additionally, given that the field of AI is rapidly evolving and new models are created every day, the conclusions we glean may be different if we were to take an updated snapshot of the supply chain at some future point. The extent to which this factor might change our results is unknown, but we maintain that it is important to capture an understanding of this nascent supply chain so that future investigations can understand the ways in which the supply chain has changed over time.
