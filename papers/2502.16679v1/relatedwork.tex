\section{Related Work}
\label{sec:related_work}

Algorithms for the problem of finding Tarski fixed points on the $k$-dimensional grid of side length $n$ have only recently been considered. \cite{dang2020tarskialgorithm} gave an $O(\log^k (n))$ divide-and-conquer algorithm. \cite{fearnley2022faster} gave an $O(\log^2 (n))$ algorithm for the 3D grid and used it to construct an $O(\log^{2 \lceil k/3 \rceil} (n))$ algorithm for the $k$-dimensional grid of side length $n$. \cite{chen2022improved} extended their ideas to get an $O(\log^{\lceil (k+1)/2 \rceil}(n))$ algorithm.

 \cite{etessami2019tarski} showed a lower bound of $\Omega(\log^2(n))$ for the 2D grid, implying the same lower bound for the $k$-dimensional grid of side length $n$. This bound is tight for $k = 2$ and $k=3$, but there is an exponential gap for larger $k$. They also showed that the problem is in both PLS and PPAD, which by the results of \cite{fearnley2022cls} implies it is in CLS.

\cite{CLY23} give a black-box reduction from the Tarski problem to the same problem with an additional promise that the input function has a unique fixed point. This result  implies that the Tarski problem and the unique Tarski problem have the same query complexity.

Two problems related conceptually to that of finding a Tarski fixed point are finding a Brouwer fixed point~\cite{hirsch1989exponential,chen2005algorithms,chen2007paths} and finding a local minimum (i.e. local search)~\cite{aldous1983minimization,Aaronson06,zhang2009tight,llewellyn1989local,sun2009quantum,santha2004quantum,dinh2010quantum}. The query complexity lower bounds for Brouwer and local search also rely on hidden path (``spine'') constructions. However, the monotonicity condition of the function in the Tarski setting poses an extra challenge.