\section{Related work}
\subsection{3D Object Representation Learning}
Traditional methods render a three-dimensional object into two dimensions as an image or a set of images with different views \citep{qi2016volumetric, su2015multiview}. These methods involve significant information loss and cannot truly represent 3D objects. Some recent works \citep{qi2017pointnet, 8579057} utilize spatial point cloud to depict objects. PointNet \citep{qi2017pointnet} introduced a deep learning framework for directly processing point clouds, significantly advancing object classification and segmentation tasks. This was further expanded by \citet{8579057} through PointGrid, which combines point clouds with voxel grids to enhance geometric understanding. Voxel grid representation offers a volumetric approach to 3D shape analysis. \citet{chen2023polygnn} develop PolyGNN to reconstruct 3D building models using polyhedral decomposition from point cloud. \citet{wu20153dshapenets} developed 3D ShapeNets, a method that leverages convolutional neural networks on voxel grids to perform 3D shape recognition, providing a robust framework for capturing complex shapes. \citet{wang2017ocnn} introduced the Octree-based CNN, which improves efficiency by using octree structures for adaptive resolution in 3D space. These discrete methods fail to leverage the structural information like edges inherently by points or grids, making them less compatible with structured data. 
Mesh representation focuses on using triangles or quads to model 3D objects. \citet{bruna2013spectral} proposed spectral networks to operate on meshes. \citet{henaff2015deep} extended this concept by introducing convolutional networks for structured data, enhancing the analysis of mesh topology. Further advancements by \citet{defferrard2016convolutional} and \citet{monti2017geometric} applied localized filtering and mixture model CNNs to learn geometric features on meshes. \citet{Pang_2023} proposes a GNN-based approach to learn geodesic embeddings for polyhedral faces.  While these methods have significantly advanced the processing of 3D object, they face limitations due to their computational intensity with high-resolution models and their struggles with irregular geometries, inherent to the mesh format. Directly modeling objects with polyhedra is a promising method to address these issues.

\subsection{Polyhedral Representation Learning}
Recent advancements in the field of polyhedral geometry representation learning have been significant. Traditional feature engineering approaches \citep{pham2010fast, yan2019graph, he2018recognition}  transform polygonal shapes into predefined shape descriptors. GNNs are utilized to improve the handling of spatial relationships and structural complexities \citep{qi2017graph, shi2020pointgnn, wang2019dynamic}. However, these descriptors oversimplify the data, failing to capture the complete spectrum of shape information.  Polygon shape encoding methods \citep{van2019deep, mai2023towards, yan2021graph}, have demonstrated their effectiveness in shape classification and retrieval tasks. While beneficial for certain types of analysis, these methods do not fully meet the needs of polyhedral representation learning that requires capturing complex topological relationships between polygonal geometries. PolygonGNN \cite{yu2024polygongnn} is the first GNN-based method that captures both multi-polygon relationships and individual polygon shape information. However, it is designed specifically for 2D shapes. In relation to polyline representation learning \citep{jiang2021weakly, jiang2022weakly}, current works focus on processing continuous lines and curves that delineate the boundaries and configurations of shapes in spatial data.  Another category of research focuses on polyhedron generation. \citet{gillsjo2023polygon} extracts polygons from images by using heterogeneous graphs and wireframes to learn feature space. \citet{zorzi2023re} utilizes edge-aware GNNs to enhance polygon detection accuracy and applicability in scene parsing by considering both node and edge features. 
% Recently, in \citet{jones2022selfsupervised}, unsupervised techniques within encoder-decoder architectures have been utilized to extract geometric and topological data from unlabeled CAD sources, enabling automated interpretation of complex B-Rep faces without predefined labels. Diffusion model-based approaches have also been employed to manipulate and learn polyhedral structures in latent spaces \citep{koo2024salad, cheng2023sdfusion}. Additionally, \citet{zhang20233dshape2vecset} using radial basis functions and transformers to encode 3D shapes into neural field vectors provide a versatile and efficient approach to processing polyhedral data.

% \subsection{Graph Neural Network for Polyhedron}
% Recent studies have shown that Graph Neural Networks (GNNs) are highly effective in handling polyhedron, as they process various forms of polyhedral structures including point clouds, polygons, and directly managing polyhedra. For GNNs to process point clouds \cite{qi2017graph, shi2020pointgnn, wang2019dynamic}, they leverage the graph-based representation of point clouds to improve handling of spatial relationships and structural complexities inherent in 3D environments. 
% When it comes to polygons, \citet{antonietti2024agglomeration} enhance the analysis of geometric structures by maintaining mesh quality and improving computational processes, as demonstrated in multigrid solvers and scene parsing tasks. \citet{zorzi2023re} utilizes edge-aware GNNs to enhance polygon detection accuracy and applicability in scene parsing by considering both vertex and edge features.  \citet{gillsjo2023polygon} optimize room layout estimation by using heterogeneous graphs and wireframes, benefiting urban planning and interior design applications. For polyhedron, recent research highlights several advanced methods leveraging GNNs. \citet{Pang_2023} proposes a GNN-based approach to learn geodesic embeddings for polyhedral faces, significantly enhancing shape matching and segmentation tasks. \citet{allen2022learning} introduce a method to model face-to-face interactions within polyhedra, leading to more accurate rigid body dynamics simulations and superior performance on real-world data.  \citet{chen2023polygnn} develop PolyGNN to reconstruct 3D building models using polyhedral decomposition, achieving compact and semantically rich building reconstructions.