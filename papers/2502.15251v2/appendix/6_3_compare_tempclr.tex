\subsection{Comparison with TempCLR Method}

\renewcommand{\arraystretch}{1.2}
\begin{table}[!t]
\centering
\resizebox{0.6\textwidth}{!}{
    \begin{tabular}{cc|cc}
     \Xhline{1.0pt}

    \rowcolor{COLOR_MEAN} &  & \multicolumn{2}{c}{\textbf{FreiHand*}} \\
     
    \rowcolor{COLOR_MEAN}  \multirow{-2}{*}{\textbf{Method}} &  \multirow{-2}{*}{\textbf{Pre-training size}}  & \textit{MPJPE}  $\downarrow$ & \textit{PCK-AUC}  $\uparrow$ \\ 
    
    \Xhline{0.6pt}

        PeCLR & \multirow{3}{*}{\begin{tabular}{c} Ego-50K \end{tabular}} & 47.42 & 49.85 \\
        TempCLR & & 45.17 & 52.40 \\
        \Ours & & \textbf{35.32} & \textbf{63.35} \\

       \Xhline{0.6pt}

        PeCLR & \multirow{3}{*}{\begin{tabular}{c} Ego-100K \end{tabular}} & 46.00 & 51.50 \\
        TempCLR & & 44.54 & 53.28 \\
        \Ours & &  \cellcolor{blue!10}\textbf{31.06} &  \cellcolor{blue!10}\textbf{68.66} \\

     \Xhline{1.0pt}
   \end{tabular}
}
\captionof{table}{\textbf{Comparison with the TempCLR method.}'*' indicates that we use a small amount of training data for fine-tuning to validate the effectiveness of the pre-trained model. TempCLR outperforms PeCLR by a modest margin, whereas \Ours achieves a significant performance improvement over TempCLR.
}
\label{tab:exp_tempclr}
\end{table}

We conduct an experimental comparison with the TempCLR~\citep{ziani:3dv22} method. TempCLR proposes a pre-training framework for 3D hand reconstruction using time-coherent contrastive learning and demonstrates better performance compared to PeCLR~\citep{spurr:iccv21}. Although TempCLR primarily focuses on reconstruction tasks, the parametric model it uses can also output 3D pose results, making it valuable to further compare our method with TempCLR.

However, TempCLR has certain limitations in data collection and the effectiveness of contrastive learning. First, TempCLR treats hands from adjacent frames as positive samples during training. In dynamic egocentric videos, hand occlusions or detection failures often lead to missed hand crops in neighboring frames. In addition, images from adjacent frames typically lack background diversity, limiting the contribution of positive sample pairs formed from neighboring frames in contrastive learning.

In contrast to TempCLR, our method, \Ours, significantly improves performance. \Ours leverages similar hand images, which provide richer diversity in features, including various types of hand-object interactions, diverse backgrounds, and varying appearances. These features allow \Ours to effectively increase the diversity of positive samples in contrastive learning, resulting in superior pre-training performance.

We further validate our approach on two different size of pre-training data, consisting of 50K and 100K hand images from the Ego4D dataset~\citep{grauman:cvpr22}. Tab.~\ref{tab:exp_tempclr} shows the significant progress made by \Ours compared to TempCLR and PeCLR.

From the experimental results, TempCLR demonstrates better performance than PeCLR, which matches the conclusion of the original paper. However, \Ours provides more valuable positive samples for contrastive learning, leading to better results during the fine-tuning phase of 3D hand pose estimation tasks.