\renewcommand{\arraystretch}{1.2}
\begin{table}[!t]
\centering
\resizebox{0.6\textwidth}{!}{
    \begin{tabular}{cc|cc}
     \Xhline{1.0pt}

    \rowcolor{COLOR_MEAN} &  & \multicolumn{2}{c}{\textbf{FreiHand*}} \\
     
    \rowcolor{COLOR_MEAN}  \multirow{-2}{*}{\textbf{Method}} &  \multirow{-2}{*}{\textbf{Pre-training size}}  & \textit{MPJPE}  $\downarrow$ & \textit{PCK-AUC}  $\uparrow$ \\ 
    
    \Xhline{0.6pt}

        PeCLR & \multirow{3}{*}{\begin{tabular}{c} Ego-50K \end{tabular}} & 47.42 & 49.85 \\
        TempCLR & & 45.17 & 52.40 \\
        \Ours & & \textbf{35.32} & \textbf{63.35} \\

       \Xhline{0.6pt}

        PeCLR & \multirow{3}{*}{\begin{tabular}{c} Ego-100K \end{tabular}} & 46.00 & 51.50 \\
        TempCLR & & 44.54 & 53.28 \\
        \Ours & &  \cellcolor{blue!10}\textbf{31.06} &  \cellcolor{blue!10}\textbf{68.66} \\

     \Xhline{1.0pt}
   \end{tabular}
}
\captionof{table}{\textbf{Comparison with the TempCLR method.}'*' indicates that we use a small amount of training data for fine-tuning to validate the effectiveness of the pre-trained model. TempCLR outperforms PeCLR by a modest margin, whereas \Ours achieves a significant performance improvement over TempCLR.
}
\label{tab:exp_tempclr}
\end{table}