\subsection{Visualization of Similar hands}\label{sec:appendix_vis_simhands}

We present the visualization of Top-K similar hand images used to create positive pairs. As shown in Fig. ~\ref{fig:similar_hands}, we visualize a set of Top-K similar hand images. The figure displays the query image alongside its corresponding similar hand sequence (Top-K). At the top of Fig. ~\ref{fig:similar_hands}, a timeline indicates that the images are deliberately sampled from consecutive frames of the same video. 

From these visualizations, we derive three key insights: 1) Using adjacent frames from the same video as positive samples in pre-training lacks diversity, as substantial variations may still exist between samples. 2) As the ranking increases, the similarity between hand images decreases significantly, leading to greater differences that may result in inaccurate feature representations during pre-training. 3) Therefore, selecting the Top-1 image is a proper design to assign diverse yet similar positive samples for the query images.

\begin{figure}[t!]
\vspace{-2mm}
    \begin{center}
    \includegraphics[width=1.0\textwidth]{appendix/figures/figure7_top_k.pdf}
    \end{center}
    \vspace{-3mm}
    \caption{
    \textbf{Visualization of similar hand samples in Top-K.} As the ranking increases, the differences between hand samples become more pronounced.
   }
    \label{fig:similar_hands}
    \vspace{-5mm}
\end{figure}