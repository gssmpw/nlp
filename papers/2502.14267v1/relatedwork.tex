\section{Literature review}
\subsection{Related work}
The most common way VIP use banknotes is by placing those in various parts of their wallets. A relative does this task on behalf of the individual. Thus, it is convenient for the blind person to handle banknotes. Nonetheless, the visually impaired individual should always depend on others to put the money in the appropriate places.

Furthermore, the Central Bank of Sri Lanka (CBSL) has taken initiatives to aid VIP in identifying denominations. So, as illustrated in Fig.~\ref{fig:my_image_1}, heavily printed dots appear on the left side of the currency note.
\begin{figure}[htbp]
    \centerline{\includegraphics[width=0.45\textwidth]{fig_1.png}}
    \caption{Heavily printed dots on Sri Lankan banknotes \cite{b5}.}
    \label{fig:my_image_1}
\end{figure}

Although it is an important step towards helping the VIP, the heavily printed dots wear out with use. As a result, the VIP finds it increasingly difficult to use banknotes.

Due to recent advancements in Artificial Intelligence (AI), some excellent work has been done to address this issue. Nevertheless, more research needs to be done on Sri Lankan banknotes. Hence, the work that has been done in Sri Lanka and other countries is taken into consideration. Furthermore, only the work done in the last five years was considered for the literature survey of this research.

A study on the Sri Lankan notes and coins is presented in \cite{b6}. Convolutional Neural network (CNN) was used to classify the denominations. The note detection accuracy was 0.6667 and the coin detection accuracy was 0.6970. The low accuracy might have occurred due to CNN training from scratch. However, in their approach, an image should be captured and then the inference should be done. Thus, it can lead to a less user-friendly application.

Another research on the Sri Lankan banknotes is presented in \cite{b7}. A CNN-based Transfer Learning (TL) model was fine-tuned to classify the Sri Lankan banknotes. Popular pre-trained models such as MobileNetV2, NASNetMobile, EfficientNetB0, DenseNet121, DenseNet201, ResNet50V2, VGG16, and VGG19 were trained and tested to identify the best-performing model. As a result, DenseNet121 performed well by achieving 98\% of the test dataset. A focus mechanism is used to recognize the denominations. Hence, the model will detect the currency note when it is approximately 10 cm from the camera. The solution is less user-friendly from the perspective of a blind person, despite the model's exceptional accuracy.

There has not been much research done on this area in recent times in Sri Lanka. Thus, research done in other countries is also considered to derive the research gap.

A study was done on the Bangladeshi banknotes utlizing the Oriented FAST and Rotated BRIEF (ORB) algorithm \cite{b8}. OpenCV was used to generate the key points, which were then stored in a database. In production, when an image is received, it is converted to grayscale to eliminate the lighting conditions. The ORB keypoint detection algorithm was then used to identify the key points, and the ORB descriptor extractor was used to extract them. The brute force Hamming algorithm was then utilized to match the key points. The highest number of matching points is finally measured. The average matching time was 0.17s, and the accuracy was 100\%. This approach is, however, not a user-friendly solution.

Another research was done for the Pakistani currency notes using DL \cite{b9}. The dataset used to train the AlexNet was enhanced by incorporating data augmentation techniques. The system receives a video with the currency note in it. The trained model then identifies the banknotes in the frames after they have been separated. The approach achieved a testing accuracy of 96.85\%. Yet, this method is time-consuming.

A research was conducted to identify Yemeni currency using MobileNet \cite{b10}. A collected dataset was used to train the model. Additionally, the collection included notes in a variety of formats, including damaged and worn notes. The testing accuracy was 100\%. The model identifies the correct denomination after capturing an image of the currency note. The deployed model performs efficiently since MobileNet was utilized as the pre-trained model. However, this method is not real-time, and VIP may find it more difficult.

The literature review of the above research concluded that neither of the works is based on real-time approaches. Thus, research based on OD algorithms was reviewed. However, no research has been done on Sri Lankan currency notes incorporating OD algorithms. Conversely, OD approaches have been used in research in other nations to identify banknotes in real-time.

A study on Iraqi banknotes has leveraged the popular YOLOv3 (You Only Look Once) algorithm to train its model \cite{b11}. The dataset was created manually, and geometric transformations were applied to augment it. The images were then annotated using the LabelImg tool. The mean average precision (mAP) was 97.405\%.

A study was done on the Indian currency notes utilizing the YOLOv3 algorithm \cite{b12}. Data augmentation was done on a self-built dataset, and it was manually annotated to train the model. The standalone system has a detection and recognition accuracy of 95.71\% and 100\%, respectively. Furthermore, the system is robust enough to recognize even in cases of partial occlusions as well as wrinkled or torn currency notes.

A research done on the Ethiopian currency stands out from all the research done to date \cite{b13}. The YOLOv5m was trained on a dataset created after data augmentation and manual annotation. The model achieved an mAP of 97.9\%. Moreover, the total loss of the model was 0.044. The model performed well for the testing data by achieving a 92\% accuracy. In this work, not only variants of the YOLOv5 model were trained and tested, but also Faster\_RCNN\_Inception\_v2 and SSD\_Mobilenet\_v2 models were trained and tested. Yet, YOLOv5m outperformed the other models.

\subsection{Research gap}
The proposed system is based on OD techniques. Thus, it will allow real-time currency note recognition. Moreover, the proposed solution focuses on training and testing a model that could be easily deployed to a smartphone. To the best of knowledge, this is the first Sri Lankan currency recognition system to identify currency notes in one stage.

\vspace{0.15cm}