\section{Related Work}
% Coordination Mechanisms
Scheduling games were initially studied in the setting in which each machine processes its jobs in parallel so that the completion time of each job depends on the total load on the machine. The corresponding papers analyze the inefficiency of selfish behavior with respect to the makespan. Czumaj and Vöcking~\cite{CzumajV07} gave tight bounds on the price of anarchy for related machines, whereas Awerbuch et al.~\cite{AART06} and Gairing et al.~\cite{GLM10} provided tight bounds for restricted machine settings. % We refer to Vöcking [27] for an overview.
These tight bounds grow with the number of machines and motivated the study of coordination mechanisms, i.e., local scheduling policies, to reduce the price of anarchy.

Christodoulou et al.~\cite{CKN04} introduced coordination mechanisms and studied the price of anarchy with priority lists based on longest processing time (LPT) first. Immorlica et al.~\cite{ILMS09} generalized their results and studied several different scheduling policies. Other results were given by Cohen et al.~\cite{CDN11}, who proved that a Nash equilibrium is expected to exist for two unrelated machines with a random order. Also, Azar et al.~\cite{AJM08} showed that for unrelated machines with priorities based on the ratio of a job's processing time to its shortest processing time a Nash equilibrium need not exist; Lu and Yu \cite{LY12} suggested a mechanism that guarantees the existence of a Nash equilibrium; Kollias \cite{K13} showed that non-preemptive coordination mechanisms need not induce a pure Nash equilibrium. Vijayalakshmi et al.~\cite{RST21} considered a more general setting in which machines have arbitrary individual priority lists. The paper characterizes four classes of instances in which a pure Nash equilibrium is guaranteed to exist, and analyzes the equilibrium inefficiency for these classes. Caragiannis et al.~\cite{CF19} introduced {\em DCOORD} coordination mechanism with price of anarchy $O(\log m)$ and price of stability $O(1)$.

% BRD and Sink eq.
Best Response Dynamics (BRD) has been a significant area of study in the analysis of congestion games.
Research in this domain often examines the existence of NE, the convergence of BRD to these equilibria, and the inefficiencies introduced by selfish behavior. The papers \cite{EKM03,FT15,FST17,KBL13} analyze various deviator rules, such as Max-Weight-Job, Min-Weight-Job, and Max-Cost, and compared their effects on the convergence time and the solution quality in several classes of congestion games.

Research has also delved into scenarios where BRD does not converge to a pure Nash Equilibrium.
% Sink eq.
Goemans et al.~\cite{GMV05} introduced the concept of sink equilibria, to address situations where pure NE do not exist. They defined the {\em Price of Sinking} that quantifies the inefficiency of sink equilibria. %that quantifies the inefficiency of sink equilibria by comparing their worst-case value to that of the social optimum. Specifically, the cost of a sink equilibrium is determined by the expected social cost derived from the steady-state distribution of a random walk on the sink.
%
Berger et al.~\cite{BFNR11} introduced the notion of dynamic inefficiency, examining the average social cost across an infinite sequence of best responses. They explored how different deviator rules, such as Random Walk, Round Robin, and Best Improvement, impact social costs in games lacking the finite improvement property.


% Competition
Rank-based scheduling games where studied so far only on parallel machines (without priority lists). Ashlagi et al.~\cite{AKT08} presented a social context game with rank competition. The competition structure in their model is arbitrary and defined by a network. For the case of disjoint competition sets, that is, when the network is a collection of disjoint cliques, the paper shows that a NE may not exist and is guaranteed to exist in a game with identical resources. Immorlica et al.~\cite{ILMS09} consider a model with arbitrary competition structure, in which players' utility combine their payoff and ranking. General ranking games are studied in Brandt et al.~\cite{BFHS09}, where it is shown that computing a NE is NP-complete in most cases. In Goldberg et al.~\cite{GGKV13} a player's utilization combines its rank with the effort expended to achieve it. Rosner and Tamir introduced in~\cite{RT23} a scheduling game with rank-based utilities (SRBG), where the players are partitioned into competition sets, and the goal of every player is to perform well relative to its competitors. These works show that the analysis of games with rank-based utilities tends to be very different from the analysis of classical games.

Our model combines the study of coordination mechanism with rank-based utilities. This combination was not considered in the past.

%%%%%%%%%%%%%%%%%%%%%%%%%%%%%%%%%%%%%%%%%%%%%%%%%%%%%%%%%%%%%%%%%