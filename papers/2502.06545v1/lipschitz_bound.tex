\section{The Spectral Filtering Property Lemma~\ref{lemma:spectral_filtering_property}}
\label{appendix:spectral_filtering_property}


In order to prove Lemma~\ref{lemma:spectral_filtering_property} we require two helper lemmas. The first is Lemma~\ref{lemma:lipschitz}, which roughly argues that the Lipschitz constant of a function $f: \complex_{\beta} \to \R$ can be bounded by a polynomial of the expectation of the function on $\complex_{\beta}$. The second is Lemma~\ref{lemma:hankel_decay} , which argues that the Hankel matrix which we construct exhibits exponentially fast decay in its spectrum. 




\begin{lemma}
\label{lemma:lipschitz}
Recall 
\[
\complex_{\beta} = \{ \alpha\in\mathbb{C} : |\alpha|\le 1 \text{ and } \operatorname{Im}(\alpha)\le \beta \},
\]
and let $\mathcal{F}$ denote the set of functions $f:\mathbb{C}_{\beta} \to\mathbb{R}$ that are $L$-Lipschitz and attain the maximum value $g_{\max}$ on $\complex_\beta$. Then 
\[
\min_{f\in\mathcal{F}} \int_{\complex_\beta} f(z)\,dz \le \frac{\arcsin(\beta)\,g_{\max}^3}{L^2}.
\]
\end{lemma}

\begin{proof}

Let $ z_0 = \sqrt{1-\beta^2} + i\beta$.
Then $|z_0|=1$ and $\operatorname{Im}(z_0)=\beta$, so $z_0\in S_\beta$. Define
\[
f(z) = \max\{ g_{\max} - L\,|z-z_0|,\,0\}.
\]
Since $f(z_0)=g_{\max}$ and the function $z\mapsto g_{\max}-L\,|z-z_0|$ is $L$-Lipschitz (and the maximum with $0$ does not increase the Lipschitz constant), we have $f\in\mathcal{F}$.

\medskip

Notice that $f(z)>0$ only when
\[
|z-z_0| < R,\quad\text{with } R = \frac{g_{\max}}{L}.
\]
Thus the support of $f$ is contained in the disk
\[
D(z_0,R) = \{ z\in\mathbb{C} : |z-z_0| < R \}.
\]
Since $z_0$ lies on the horizontal line $\operatorname{Im}(z)=\beta$, one may verify that the intersection
\[
D(z_0,R)\cap S_\beta
\]
is contained in a circular sector of radius $R$ with central angle at most $2\arcsin(\beta)$. An upper bound for the area of this sector is
\[
\operatorname{Area}\bigl(D(z_0,R)\cap S_\beta\bigr) \le \frac{2\arcsin(\beta)}{2\pi} \cdot \pi R^2 
= \arcsin(\beta) \, R^2 
= \arcsin(\beta) \left(\frac{g_{\max}}{L}\right)^2.
\]

Since $f(z)\le g_{\max}$ for all $z$, we deduce
\[
\int_{S_\beta} f(z)\,dz \le g_{\max}\cdot \operatorname{Area}\bigl(D(z_0,R)\cap S_\beta\bigr)
\le g_{\max}\,\arcsin(\beta) \left(\frac{g_{\max}}{L}\right)^2
= \frac{\arcsin(\beta)\, g_{\max}^3}{L^2}.
\]

\medskip
Since we have exhibited an $f\in\mathcal{F}$ satisfying
\[
\int_{S_\beta} f(z)\,dz \le \frac{\arcsin(\beta)\, g_{\max}^3}{L^2},
\]
the inequality follows.

\end{proof}



\ignore{
\begin{proof}[Proof of Lemma~\ref{lemma:lipschitz}]
\annie{We actually need to argue that $f^*$ is in fact the function which minimizes its area.}
   Consider the following function $g^*$
    \begin{align*}
        g^*(r) & = \begin{cases}
            Lr , & r \leq g_{\textrm{max}}/L, \\
            \max \left \{g_{\textrm{max}} - L(r - g_{\textrm{max}}/L), 0 \right \} & g_{\textrm{max}}/L < r \leq 1 
            \end{cases} \\
            & = \begin{cases}
            L r, & r \leq g_{\textrm{max}}/L, \\
           2 g_{\textrm{max}} - Lr ,& g_{\textrm{max}}/L < r\leq 2g_{\textrm{max}}/L.
        \end{cases}
    \end{align*}
    Let $f^*(re^{i\theta}) = g(r) \left( 1 - \frac{\theta}{\beta} \right)$.
    Then
    \begin{align*}
        \int_{\alpha \in S} f^*(\abs{\alpha}) \left( 1 - \frac{\theta}{\beta} \right) d \alpha & = \int_{\alpha: \abs{\alpha} \leq g_{\textrm{max}}/L} f^*(\alpha) d \alpha + \int_{\alpha:g_{\textrm{max}}/L < \abs{\alpha} \leq 1} f^*(\alpha) d \alpha\\
        & = \int_{r=0}^{g_{\textrm{max}}/L} \int_{\theta = -\textrm{arcsin}(\beta)}^{\textrm{arcsin}(\beta)} L r \left( 1 - \frac{\theta}{\beta} \right) (r dr d \theta) \\
        & \qquad + \int_{r=g_{\textrm{max}}/L}^{2g_{\textrm{max}}/L} \int_{\theta = -\textrm{arcsin}(\beta)}^{\textrm{arcsin}(\beta)} \left( 2 g_{\textrm{max}} - Lr \right)\left( 1 - \frac{\theta}{\beta} \right)  (r dr d \theta) \\
        & =  \textrm{arcsin}(\beta) g_{\textrm{max}}^3/L^2.
    \end{align*}
\end{proof}
}


\begin{fact}
    \label{fact:Z_entry_bound}
    Let $Z_{S, p_n, T} \defeq \int_{\alpha \in S} \mu_{p_n}(\alpha) \overline{\mu_{p_n}(\alpha)}^{\top} d \alpha$. Then
    \begin{equation*}
        \abs{  Z_{S, p_n, T} }_{jk} \leq 2 \arcsin(\beta) \cdot \max_{\alpha \in S} \abs{p_n(\alpha)}^2  \cdot  \frac{1}{j + k}.
    \end{equation*}
\end{fact}
\begin{proof}[Proof of Fact~\ref{fact:Z_entry_bound}]
    Observe
\begin{align*}
        \abs{ Z_{S, p_n, T} }_{jk} &  = \abs{ \int_{\alpha \in S} p_n(\alpha) p_n(\overline{\alpha}) \alpha^{j-1} \overline{\alpha}^{k-1} d \alpha } \\
        & \leq \max_{\alpha \in S} \abs{p_n(\alpha)}^2 \cdot  \int_{\alpha \in S}  \abs{\alpha}^{j + k -2} d \alpha  \\
        & = \max_{\alpha \in S} \abs{p_n(\alpha)}^2 \cdot \abs{ \int_{\theta = - \arcsin(\beta)}^{\arcsin(\beta)} \int_{r=0}^1 r^{j +  k - 2}  (r dr d \theta) } \\
         & = 2 \arcsin(\beta) \cdot \max_{\alpha \in S} \abs{p_n(\alpha)}^2  \cdot  \frac{1}{j + k}.
        \end{align*}
\end{proof}
\begin{lemma}[Adapted from Lemma E.2 from \cite{hazan2017learning}]
\label{lemma:hankel_decay}
Given a degree-$n$ polynomial $p_n$, horizon $T$, and complex tolerance $\beta$, let $S= \left \{ \alpha \in \C \textrm{ s.t. } \abs{\alpha} \leq 1 \textrm{ and } \mathrm{Im}(\alpha) \leq \beta \right \}$ and let
\begin{equation*}
    Z_{S, p_n, T} \defeq \int_{\alpha \in S} \mu_{p_n}(\alpha) \overline{\mu_{p_n}(\alpha)} d \alpha.
\end{equation*}
    Let $\sigma_j$ be the $j$-th singular value of $Z_{S, p_n, T}$. Then for absolute constants $c = e^{\pi^2/4}$ and $C = 2225$,
    \begin{equation}
        \sigma_j \leq  C \cdot   \arcsin(\beta)  \cdot \max_{\alpha \in S} \abs{p_n(\alpha)}^2 \cdot (1 +  \ln(T-n)) \cdot c^{-j/\log(T-n)}.
    \end{equation}
\end{lemma}
We prove Lemma~\ref{lemma:hankel_decay} using the same machinery as was introduced in \cite{hazan2017learning}. We use the following result from \cite{beckermann2017singular} which bounds the singular values of any positive semidefinite Hankel matrix. 
\begin{lemma}[Cor. 5.4 in \cite{beckermann2017singular}]
\label{lemma:hankel_result}
    Let $H_L$ be a real psd Hankel matrix of dimension $L$. then,
    \begin{equation}
    \label{eqn:eig_decay1}
        \sigma_{j + 2k}(H_L) \leq 16 \left( \exp \left( \frac{\pi^2}{4 \log(8 \lfloor L/2 \rfloor / \pi ) }\right) \right)^{-2k + 2} \sigma_j(H_L).
    \end{equation}
\end{lemma}
\begin{proof}[Proof of Lemma~\ref{lemma:hankel_decay}]
First we must show that $Z_{S, p_n, T}$ has real entries. This is true simply because we integrate over $z \in \C$ such that $\abs{\mathrm{Im}(z)} \leq \beta$ and so the imaginary components cancel each other out.  Therefore Lemma~\ref{lemma:hankel_result} applies to $Z_{S, p_n, T}$. 
        Note that Lemma~\ref{lemma:hankel_result} implies,
        \begin{equation*}
        \sigma_{j + 2k}(H_L) \leq 16 \left( \exp \left( \frac{\pi^2}{4 \log(8 \lfloor L/2 \rfloor / \pi ) }\right) \right)^{-2k + 2} \mathrm{Tr}(H_L).
        \end{equation*}
        Using Fact~\ref{fact:Z_entry_bound} we have,
        \begin{align*}
            \mathrm{Tr}\left( Z_{S, p_n, T} \right)   & = \sum_{j = 1}^{T-n} \left( Z_{S, p_n, T} \right)_{jj} \\
            & \leq   \arcsin(\beta)  \cdot \max_{\alpha \in S} \abs{p_n(\alpha)}^2  \cdot \sum_{j = 1}^{T-n} \frac{1}{j} \\
            & \leq   \arcsin(\beta)  \cdot \max_{\alpha \in S} \abs{p_n(\alpha)}^2 \cdot (1 +  \ln(T-n)).
        \end{align*}
        Therefore,
        \begin{align*}
        \sigma_{j} & = \begin{cases} \sigma_{2(j/2)}, & \textrm{$j$ even}, \\
        \sigma_{1 + 2(j-1/2)}, & \textrm{$j$ odd.}
        \end{cases} \\
        & \leq 16 \left( \exp \left( \frac{\pi^2}{4 \log(8 \lfloor (T-n) /2 \rfloor / \pi ) }\right) \right)^{-2(j/2) + 2} \left(  \arcsin(\beta)  \cdot \max_{\alpha \in S} \abs{p_n(\alpha)}^2 \cdot (1 +  \ln(T-n)) \right) \\
        &  \leq 16 \exp(\pi^2/4)^2 \arcsin(\beta) \max_{\alpha \in S} \abs{p_n(\alpha)}^2 (1 + \ln(T-n)) c^{-j/\log(T-n)} \tag{For $c = e^{\pi^2/4}$.}\\
        &  \leq 2225 \cdot   \arcsin(\beta)  \cdot \max_{\alpha \in S} \abs{p_n(\alpha)}^2 \cdot (1 +  \ln(T-n)) \cdot c^{-j/\log(T-n)}.
        \end{align*}
\end{proof}
With Lemma~\ref{lemma:lipschitz} and Lemma~\ref{lemma:hankel_decay} in hand, we are ready to prove Lemma~\ref{lemma:spectral_filtering_property}. 
\begin{proof}[Proof of Lemma~\ref{lemma:spectral_filtering_property}]
Let
\begin{equation*}
    f_{j, p_n}(\alpha) \defeq \abs{\phi_j^{\top} \mu_{p_n}(\alpha)}^2.
\end{equation*}

If $f_{j, p_n}(\alpha)$ is $L$-Lipschitz and $S_{\beta}= \left \{ \alpha \in \C \textrm{ s.t. } \abs{\alpha} \leq 1 \textrm{ and } \mathrm{Im}(\alpha) \leq \beta \right \}$ then by Lemma~\ref{lemma:lipschitz},
\begin{equation*}
    \int_{\alpha \in S} f_{j, p_n} (\alpha) d\alpha \geq \frac{2 \arcsin(\beta) \left( \max_{\alpha \in S} f_{j, p_n}(\alpha) \right)^3  }{L^2},
\end{equation*}
or equivalently,
\begin{equation*}
   \max_{\alpha \in S} f_{j, p_n}(\alpha)  \leq \left( \frac{L^2}{2 \arcsin(\beta)}  \int_{\alpha \in S} f_{j, p_n}(\alpha) d\alpha \right)^{1/6}.
\end{equation*}
Observe that
\begin{align*}
\int_{\alpha \in S} f_{j, p_n}(\alpha) d\alpha & = \int_{\alpha \in S}  \abs{\phi_j^{\top} \mu_{p_n}(\alpha)}^2 d \alpha \\
& = \phi_j^{\top} \left( \int_{\alpha \in S}   \mu_{p_n}(\alpha) \mu_{p_n}(\overline{\alpha})^{\top} d \alpha   \right) \overline{\phi_j}  \\
& = \sigma_j.
\end{align*}
Therefore have the following bound
\begin{equation}
\label{eqn:bound1}
    \max_{\alpha \in S} \abs{\mu_{p_n}(\alpha)^{\top} \phi_j} = \max_{\alpha \in S} \sqrt{f_{j, p_n}(\alpha)}  \leq \left( \frac{L^2}{2 \arcsin(\beta)} \sigma_j \right)^{1/6}.
\end{equation}
The remainder of the proof consists of bounding the Lipschitz constant $L$ and bounding the eigenvalue $\sigma_j$.
To bound the Lipschitz constant of $f_{j,p_n}$,
\begin{align*}
  L & \leq \max_{\alpha \in S} \abs{ f'_{j, p_n}(\alpha) } \\
  & =  \max_{\alpha \in S} 2 \abs{ \mathrm{Re} \left( \phi_j^{\top} \mu_{p_n}'(\alpha) \cdot \phi_j^{\top} \mu_{p_n}(\alpha) \right)} \\
    & \leq \max_{\alpha \in S} \norm{\phi_j}_2^2 \cdot \norm{\mu_{p_n}(\alpha)}_2 \cdot \norm{\mu'_{p_n}(\alpha)}_2 \\
    & = \max_{\alpha \in S} \norm{\mu_{p_n}(\alpha)}_2 \cdot \norm{\mu'_{p_n}(\alpha)}_2.
\end{align*}
We have
\begin{align*}
    \norm{\mu_{p_n}(\alpha)}_2^2 & = \abs{p_n(\alpha)}^2 \left( \sum_{s = 0}^{T-n-1} \abs{\alpha}^{2s} \right) \\
    & \leq \abs{p_n(\alpha)}^2  T \\
    & \leq \pbnd^2 T.\tag{Assumption that $\max_{\alpha \in S} \abs{p_n(\alpha)}  \leq \pbnd$. }
\end{align*}
Next, to bound $\norm{\mu_{p_n}'(\alpha)}_2$ we observe,
\begin{align*}
    \left( \mu_{p_n}'(\alpha) \right)_j & = \frac{d}{d \alpha} p_n(\alpha) \alpha^j  =  p_n'(\alpha) \alpha^j + j \alpha^{j-1} p_n(\alpha) \mathbf{1}_{j > 0}.
\end{align*}
Therefore, $ \abs{\left( \mu_{p_n}'(\alpha) \right)_j} \leq \pbnd(1 + T)$ and $\norm{\mu_{p_n}'(\alpha)}_2^2 \leq 2 \pbnd^2 T^3$. Thus,
\begin{equation}
    L^2 \leq 2 \pbnd^4 T^4.
\end{equation}
Next we bound $\sigma_j$. To bound $\sigma_j$, we observe that $Z_{S,p_n}$ is a highly structured matrix called a Hankel matrix and, as such, its spectrum decays exponentially. Indeed, by Lemma~\ref{lemma:hankel_decay}, there are universal constants $C,c >0$ such that
\begin{equation}
\label{eqn:bound2}
   \sigma_j \leq   C \arcsin(\beta)  \cdot \max_{\alpha \in S} \abs{p_n(\alpha)}^2 \cdot \log T \cdot c^{-j/\log T}.
\end{equation}
Recalling Eq~\eqref{eqn:bound1} and the assumption that $\max_{\alpha \in S} \abs{p_n(\alpha)} \leq \pbnd$, we observe
\begin{equation*}
    \max_{\alpha \in S} \abs{\mu_{p_n}(\alpha)^{\top} \phi_j} \leq \left( C \pbnd^6 T^4 \log T c^{-j/\log T} \right)^{1/6}.
\end{equation*}
Therefore, there is a universal constant $C>0$ such that 
\begin{equation*}
    \max_{\alpha \in S} \abs{\mu_{p_n}(\alpha)^{\top} \phi_j} \leq C \log(T)^{1/6}  \pbnd T^{2/3} c^{-j/6 \log T} .
\end{equation*}
\end{proof}



