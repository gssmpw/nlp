%\documentclass[anon,12pt]{colt2025} % Anonymized submission
%\documentclass[final,12pt]{colt2025} % Include author names
% \documentclass{article}
% \usepackage{arxiv} 
% \usepackage{graphicx} % Required for inserting images
% \usepackage{algorithm,algorithmic}
% %\usepackage{fullpage}
% \usepackage{url}
% \usepackage{amssymb}

%\usepackage[affil-it]{authblk}
\documentclass{article}
\usepackage{graphicx} % Required for inserting images
\usepackage{algorithm,algorithmic}
\usepackage{fullpage}
\usepackage{url}
\usepackage{amssymb}
\usepackage{natbib}
\usepackage{hyperref}
\usepackage{arxiv}
\usepackage{fancyhdr}

\usepackage{times}
\usepackage{latexsym}

\usepackage[T1]{fontenc}

\usepackage[utf8]{inputenc}

\usepackage{microtype}

\usepackage{inconsolata}

\usepackage{graphicx}


\usepackage{amsmath}
\usepackage{amssymb}
\usepackage{multirow}
\usepackage{booktabs}
\usepackage{catchfile}

\usepackage[boxed]{algorithm}
\usepackage{varwidth}
\usepackage[noEnd=true,indLines=false]{algpseudocodex}
\usepackage{cleveref}
\makeatletter
\@addtoreset{ALG@line}{algorithm}
\renewcommand{\ALG@beginalgorithmic}{\small}
\algrenewcommand\alglinenumber[1]{\small #1:}
\makeatother

\usepackage[normalem]{ulem}
\usepackage{todonotes}

\usepackage{lipsum}    %
\usepackage{comment}   %
\usepackage{graphicx}  %
\usepackage{pifont}    %

\usepackage[font=small,labelfont=bf]{caption}
\usepackage{float}     %
\usepackage{booktabs}  %
\usepackage{subcaption}  %

\usepackage{listings}

\usepackage{amsthm}  %

\pagestyle{fancy}
\fancyhf{}  % Clear default header/footer
\fancyhead[L]{Dimension-free Regret for Learning Asymmetric Linear Dynamical Systems}  % Left-side header
\fancyhead[R]{\thepage}  % Right-side header with page number


% The following packages will be automatically loaded:
% amsmath, amssymb, natbib, graphicx, url, algorithm2e

%\title[DRAFT: NOT FOR DISTRIBUTION]{DRAFT: NOT FOR DISTRIBUTION \\ Dimension-free Regret for Learning \\ Asymmetric Linear Dynamical Systems}
\title{Dimension-free Regret for Learning \\ Asymmetric Linear Dynamical Systems}
\usepackage{times}
%
\author{  Annie Marsden \And Elad Hazan\thanks{Google DeepMind, \texttt{\{anniemarsden,ehazan\}@google.com} }}

\begin{document}

\maketitle

\begin{abstract}%
Previously, methods for learning marginally stable linear dynamical systems either (1) required the transition matrix to be symmetric or (2) incurred regret bounds that scale polynomially with the system’s hidden dimension. In this work, we introduce a novel method that overcomes this trade-off, achieving dimension-free regret despite asymmetric matrices and marginal stability. Our method combines spectral filtering with linear predictors and employs Chebyshev polynomials in the complex plane to construct a novel spectral filtering basis. This construction guarantees sublinear regret in an online learning framework, without relying on any statistical or generative assumptions. Specifically, we prove that as long as the transition matrix has eigenvalues with complex component bounded by $1/\poly \log(T)$, then our method achieves regret $\tilde{O}(T^{9/10})$ when compared to the best linear dynamical predictor in hindsight. 
\end{abstract}


 
\section{Introduction}


\begin{figure}[t]
\centering
\includegraphics[width=0.6\columnwidth]{figures/evaluation_desiderata_V5.pdf}
\vspace{-0.5cm}
\caption{\systemName is a platform for conducting realistic evaluations of code LLMs, collecting human preferences of coding models with real users, real tasks, and in realistic environments, aimed at addressing the limitations of existing evaluations.
}
\label{fig:motivation}
\end{figure}

\begin{figure*}[t]
\centering
\includegraphics[width=\textwidth]{figures/system_design_v2.png}
\caption{We introduce \systemName, a VSCode extension to collect human preferences of code directly in a developer's IDE. \systemName enables developers to use code completions from various models. The system comprises a) the interface in the user's IDE which presents paired completions to users (left), b) a sampling strategy that picks model pairs to reduce latency (right, top), and c) a prompting scheme that allows diverse LLMs to perform code completions with high fidelity.
Users can select between the top completion (green box) using \texttt{tab} or the bottom completion (blue box) using \texttt{shift+tab}.}
\label{fig:overview}
\end{figure*}

As model capabilities improve, large language models (LLMs) are increasingly integrated into user environments and workflows.
For example, software developers code with AI in integrated developer environments (IDEs)~\citep{peng2023impact}, doctors rely on notes generated through ambient listening~\citep{oberst2024science}, and lawyers consider case evidence identified by electronic discovery systems~\citep{yang2024beyond}.
Increasing deployment of models in productivity tools demands evaluation that more closely reflects real-world circumstances~\citep{hutchinson2022evaluation, saxon2024benchmarks, kapoor2024ai}.
While newer benchmarks and live platforms incorporate human feedback to capture real-world usage, they almost exclusively focus on evaluating LLMs in chat conversations~\citep{zheng2023judging,dubois2023alpacafarm,chiang2024chatbot, kirk2024the}.
Model evaluation must move beyond chat-based interactions and into specialized user environments.



 

In this work, we focus on evaluating LLM-based coding assistants. 
Despite the popularity of these tools---millions of developers use Github Copilot~\citep{Copilot}---existing
evaluations of the coding capabilities of new models exhibit multiple limitations (Figure~\ref{fig:motivation}, bottom).
Traditional ML benchmarks evaluate LLM capabilities by measuring how well a model can complete static, interview-style coding tasks~\citep{chen2021evaluating,austin2021program,jain2024livecodebench, white2024livebench} and lack \emph{real users}. 
User studies recruit real users to evaluate the effectiveness of LLMs as coding assistants, but are often limited to simple programming tasks as opposed to \emph{real tasks}~\citep{vaithilingam2022expectation,ross2023programmer, mozannar2024realhumaneval}.
Recent efforts to collect human feedback such as Chatbot Arena~\citep{chiang2024chatbot} are still removed from a \emph{realistic environment}, resulting in users and data that deviate from typical software development processes.
We introduce \systemName to address these limitations (Figure~\ref{fig:motivation}, top), and we describe our three main contributions below.


\textbf{We deploy \systemName in-the-wild to collect human preferences on code.} 
\systemName is a Visual Studio Code extension, collecting preferences directly in a developer's IDE within their actual workflow (Figure~\ref{fig:overview}).
\systemName provides developers with code completions, akin to the type of support provided by Github Copilot~\citep{Copilot}. 
Over the past 3 months, \systemName has served over~\completions suggestions from 10 state-of-the-art LLMs, 
gathering \sampleCount~votes from \userCount~users.
To collect user preferences,
\systemName presents a novel interface that shows users paired code completions from two different LLMs, which are determined based on a sampling strategy that aims to 
mitigate latency while preserving coverage across model comparisons.
Additionally, we devise a prompting scheme that allows a diverse set of models to perform code completions with high fidelity.
See Section~\ref{sec:system} and Section~\ref{sec:deployment} for details about system design and deployment respectively.



\textbf{We construct a leaderboard of user preferences and find notable differences from existing static benchmarks and human preference leaderboards.}
In general, we observe that smaller models seem to overperform in static benchmarks compared to our leaderboard, while performance among larger models is mixed (Section~\ref{sec:leaderboard_calculation}).
We attribute these differences to the fact that \systemName is exposed to users and tasks that differ drastically from code evaluations in the past. 
Our data spans 103 programming languages and 24 natural languages as well as a variety of real-world applications and code structures, while static benchmarks tend to focus on a specific programming and natural language and task (e.g. coding competition problems).
Additionally, while all of \systemName interactions contain code contexts and the majority involve infilling tasks, a much smaller fraction of Chatbot Arena's coding tasks contain code context, with infilling tasks appearing even more rarely. 
We analyze our data in depth in Section~\ref{subsec:comparison}.



\textbf{We derive new insights into user preferences of code by analyzing \systemName's diverse and distinct data distribution.}
We compare user preferences across different stratifications of input data (e.g., common versus rare languages) and observe which affect observed preferences most (Section~\ref{sec:analysis}).
For example, while user preferences stay relatively consistent across various programming languages, they differ drastically between different task categories (e.g. frontend/backend versus algorithm design).
We also observe variations in user preference due to different features related to code structure 
(e.g., context length and completion patterns).
We open-source \systemName and release a curated subset of code contexts.
Altogether, our results highlight the necessity of model evaluation in realistic and domain-specific settings.







\section{Discussion of Result}

\subsection{Decomposing the Components of Learning a Linear Dynamical Systems}
\label{section:lds_breakdown}
Consider a noiseless linear dynamical system (LDS) with input vectors $\uv_1, \dots, \uv_T \in \R^{d_{\textrm{in}}}$ and output vectors $\y_1, \dots \y_T \in \R^{d_{\textrm{out}}}$ which follow the law 
\begin{align*}
    \mat{x}_{t+1} & = \A \mat{x}_t + \B \uv_t \\
    \y_t & = \C \mat{x}_t + \mat{D} \uv_t,
\end{align*}
where $\mat{x}_0, \dots, \mat{x}_T \in \R^{d_{\textrm{hidden}}}$ is a sequence of hidden states and $(\A, \B, \C, \mat{D})$ are matrices which parameterize the LDS. We assume w.l.o.g. that $\mat{D} = 0$. Observe that $\y_t$ can be equivalently expressed as
\begin{equation}
\label{eqn:lds}
    \y_t = \sum_{s = 1}^{t} \C \A^{t-s} \B \uv_{s}.
\end{equation}
By algebraically manipulating Eq~\eqref{eqn:lds} one can see that any linear combination of $n$ autoregressive terms $\y_t, \dots, \y_{t-n}$ with coefficients $c_0, \dots, c_n$ is equivalent to 
\begin{equation}
\label{eqn:polynomial_translation}
    \sum_{i = 0}^n c_i \y_{t-i} = \sum_{s = 0}^{n-1} \sum_{i = 1}^n c_i \C \A^{\max (0, i-n+s)}\B \uv_{t-s} + \sum_{s = 1}^{t-n-1} \C p_n(\A) \A^s \B \uv_{t-n-s},
\end{equation}
where we define the degree-$n$ polynomial $p_n(x)$ based on the coefficients $c_0, \dots, c_n$,
\begin{equation*}
    p_n(x) = c_0 x^n + \dots + c_n x^0.
\end{equation*}
Letting $\Q_j =  \C \A^j \B$ and assuming for simplicity that $c_0=1$, this immediately shows that there are matrices $\Q_1, \dots, \Q_n$ such that
\begin{center}
%\fbox{
\begin{equation}
\label{eqn:lds_breakdown}
    \y_t = \underbrace{\sum_{i = 1}^n -c_i \y_{t-i} + \sum_{s = 0}^{n-1} \sum_{i = 1}^n c_i \Q_{\max (0, i-n+s)} \uv_{t-s} }_{\text{Linear Autoregressive Term}: \prod_{LIN}^{n,n}}+ \underbrace{\sum_{s = 0}^{t-n-1} \C p_n(\A) \A^s \B \uv_{t-n-s}}_{\text{Spectral Filtering Term:} \prod_{SF}^k}.
\end{equation}
%}
\end{center}

\subsection{Auto-regressive vs. Spectral Learning}

Eq~\eqref{eqn:lds_breakdown} gives the key intuition for our result. The two left most terms are captured by the class of autoregressive predictors using a history $n$ inputs and using $n$ autoregressive terms. The right-most term has a special structure that is captures by the class of spectral filtering predictors. 

The main insight we derive from this expression is the inherent tension between the two approaches. The coefficients $c_i$ control two terms that are competing: The polynomial $p_n(x)$ and its corresponding coefficients $c_0, \dots, c_n$ control two terms that are competing:
\begin{enumerate}
    \item The auto-regressive coefficients grow larger with the degree $n$ of the polynomial and the magnitude of the coefficients $c_i$. A higher degree polynomial and larger coefficients increase the diameter of the search space over the auto-regressive coefficients, and therefore increase the regret bound. 

    \item On the other hand, a larger search space can allow a broader class of polynomials $p_n(\cdot)$ which can better control of the magnitude of $p_n(\A)$, and therefore reduce the search space of the spectral component. 
\end{enumerate}
This intuition is formalized by Theorem~\ref{thm:main_regret} which guarantees a regret bound of 
$$ \tilde{O} \left(  (1 + \max_{i \in [n]} \abs{c_i}) (1 + T^{7/6} \max_{\alpha \in \complex_\beta} \max \{ \abs{p_n(\alpha)}, \abs{p_n'(\alpha)} \}) \sqrt{T}\right), $$
when compared to the best linear dynamical predictor which assumes the spectrum of $\A$ is contained in $\complex_\beta$. 

A prime example of this tension is exhibited by the characteristic polynomial of $\A$. Let $p_n(x)$ be the characteristic polynomial for $\A$ and note that it has degree hidden dimension, i.e. $n =  \dhidden$. Then by the Cayley-Hamilton theorem, $ p_n(\A) = 0$.
This means that the spectral filtering term in Eq~\eqref{eqn:lds_breakdown} is canceled out and therefore it is sufficient to use only the class of autoregressive predictors $\prod_{\textrm{LIN}}^{\dhidden,\dhidden}$ to accurately model the LDS. However, there is a major obstacle to using this insight in order to design an efficient algorithm-- the resulting algorithm would require learning hidden dimension many matrices $\Q_1, \dots, \Q_{\dhidden}$ which can be prohibitive when the hidden dimension is large.

\subsection{Universal vs. Learned Polynomials}

The characteristic polynomial coefficients depend on the system matrix $\mat{A}$. This is a serious downside: imprecision or noise will render the spectral component non-zero, and rule out exact learning. Another downside of learned polynomials is that implementing the auto-regressive component is inherently sequential, and prohibits parallel implementations. We thus ask: {\bf are there universal polynomials that guarantee efficient learning in linear dynamical systems?} 

The answer is positive, as we show, and in a very strong sense. The coefficients of the Chebychev polynomial of the first type have favorable properties in several regards. First and foremost - they are {\it universal}, meaning that they do not depend on the system at hand and can be applied to any LDS learning setting without prior knowledge. Second, we rigorously prove that only $O(\log T)$ coefficients are needed to strongly bound the diameter of the spectral component, ie we can use the degree $O(\log(T))$ Chebyshev polynomial. This is entirely dependent of hidden dimension, unlike the characteristic polynomial and implies that we only need to learn $\log(T)$ many autoregressive components.


\section{Proof Overview}
The proof proceeds in two parts. The first is to show that any linear dynamical predictor in the class $\prod_{L}^{\beta, \gamma }$ is well approximated by a predictor in the hybrid class $\prod_{H}^{\K}$. 
\begin{lemma}[Approximation Lemma]
\label{lemma:approximation}
    Let $\pi(\A, \B, \C) \in \prod_{L}^{\beta, \gamma }$ be any LDS predictor. Suppose $$\max_{ \substack{ \alpha \in \complex \\  \mathrm{Im}(\alpha) \leq \beta  \\ \abs{\alpha} \leq 1 } } \{ \abs{p_n(\alpha)}, \abs{p_n'(\alpha)} \} \leq \pbnd . $$ 
    Then for $k \geq \log(n^2 BC T /\epsilon)^2$ and domain 
    \begin{align*}
    \K =  \{ (Q_1, \dots, Q_n, M_1, \dots, M_k) \textrm{ s.t. } & \norm{Q_i} \leq \gamma, \\
    & \norm{M_i} \leq C' \log(T)^{1/6} T^{2/3} \pbnd \}, 
\end{align*}
 there exists $\pi(\Q, \M) \in  \prod_H^{\K}$ such that for any $t \in [T]$,
    \begin{equation}
        \norm{ \y^{\pi(\A, \B, \C)}_t - \y^{\pi(\Q, \M)}_t} \leq \epsilon. 
    \end{equation}
\end{lemma}
The next result provides the regret of Online Gradient Descent when compared to the best $\pi(\Q^*, \M^*) \in \prod_{H}^{\K}$. 
\begin{lemma}[Online Gradient Descent]
\label{lemma:ogd}
Let the domain $\K$ in Algorithm~\ref{alg:new_sf} be
\begin{align*}
    \K =  \{ (Q_1, \dots, Q_n, M_1, \dots, M_k) \textrm{ s.t. } & \norm{Q_i} \leq R_Q \textrm{ and } \norm{M_i} \leq R_M \}. 
\end{align*}
Given coefficients $c_{1:n}$, let $p_n(x) = x^n + c_1 x^{n-1} + \dots + c_n$ and let $r$ denote the maximum absolute value of the coefficients. 
The iterates $\hat{y}_t^{\mA}$ output by Algorithm~\ref{alg:new_sf} satisfy 
\begin{equation*}
        \sum_{t = 1}^T f_t\left( \y_t^{\mA}\right) - \min_{\pi^* \in \prod_{H}^{\K}} f_t \left( \y_t^{\pi^*}\right) \leq  6 n^4 k^2 \sqrt{d_{\textrm{out}}} (1 + r) (R_Q + R_M).
    \end{equation*}
\end{lemma}
Combining Lemma~\ref{lemma:approximation} and Lemma~\ref{lemma:ogd} proves Theorem~\ref{thm:main_regret}. 
\begin{proof}[Proof of Theorem~\ref{thm:main_regret}]
    By Lemma~\ref{lemma:ogd} the iterates from Algorithm~\ref{alg:new_sf} $(Q^1, M^1)$, \dots, $(Q^T, M^T)$ satisfy
    \begin{equation*}
    \sum_{t = 1}^T f_t\left( \y_t^{\mA}\right) - \min_{\pi^* \in \prod_{H}^{\K}} f_t \left( \y_t^{\pi}\right)  \leq \tilde{O} \left(\gamma \sqrt{d_{\textrm{out}}} (1 + r) (1 + T^{ \frac{7}{6}} \pbnd) \sqrt{T}\right).
\end{equation*}
Next, by Lemma~\ref{lemma:approximation}, given an LDS predictor $\pi_L \in \prod_L^{\beta, \gamma}$, there exists $\pi_H \in \prod_{H}^{\K}$ such that 
\begin{equation}
        \norm{ \y^{\pi_L}_t - \y^{\pi_H}_t } \leq \epsilon. 
\end{equation}
Given any output sequence $\y_1, \dots, \y_t$, 
\begin{align*}
    f_t(\y^{\pi_L}_t) & = \norm{ \y_t - \y^{\pi_L}_t }_1 \\
    & \leq \norm{ \y_t - \y_t^{\pi_H} }_1 + \norm{ \y^{\pi_H}_t- \y^{\pi_L}_t}_1 \tag{Triangle inequality}\\
    & = f_t(\y_t^{\pi_H} ) + \epsilon \tag{Lemma~\ref{lemma:approximation} }\\
    & \leq f_t(\y_t^{\pi^*}) + \epsilon. \tag{$\y_t^{\pi^*}$ minimizes $f_t$}
\end{align*}
Therefore, if $\epsilon\leq T^{3/2}$, we get the stated result. 
\end{proof}

\subsection{Proving Approximation Lemma~\ref{lemma:approximation}}
The difficult result to prove is Lemma~\ref{lemma:approximation}. As discussed in Section~\ref{section:lds_breakdown}, any linear dynamical predictor can be written in two parts: a linear autoregressive term and a spectral filtering term (see Eq~\eqref{eqn:lds_breakdown}). The ``spectral filtering term'' is $\sum_{s = 0}^{t-n-1} \C p_n(\A) \A^s \B \uv_{t-n-s}$. We call it as such because its structure is amenable to spectral filtering. Indeed, we construct a set of spectral filters that satisfy the following.
\begin{lemma}
\label{lemma:M_matrices}
For any $(\A, \B, \C)$ there exist matrices $\M_1^*, \dots, \M^*_{T-n-1}$ such that, 
    \begin{equation}
        \sum_{s = 0}^{t-n-1}  \C p_n(\A) \A^s \B \uv_{t-n-s} = \sum_{j = 1}^{T} \M^*_{j} \langle \uv_{t-n-1:0}, \phi_{\ell} \rangle,
    \end{equation}
    for any $t \in [T]$.
    Moreover, if $\A$ has eigenvalues with imaginary component bounded by $\beta$, then for any $j \in [T-n-1]$,
    \begin{equation*}
        \norm{\M^*_j} \leq \tilde{O}(T^{2/3}) \max_{\complex_{\beta}} \max \left \{ \abs{p_n(\alpha)}, \abs{p_n'(\alpha)} \right \} e^{-j/6 \log T}.
    \end{equation*}
\end{lemma}

We prove Lemma~\ref{lemma:M_matrices} formally in Appendix~\ref{appendix:M_matrices}. This proof is inspired by previous spectral filtering literature but requires several key and nontrivial adaptations to incorporate the polynomial $p_n(\cdot)$, which impacts the choice of spectral filters, and to extend the analysis to the case where the spectrum of $\A$ may lie in the complex plane. 


\subsection{Using the Chebyshev Polynomial over the Complex Plane}

Theorem~\ref{thm:main_regret} states that Algorithm~\ref{alg:new_sf} instantiated with some choice of polynomial $p_n(\cdot)$ achieves regret
\begin{equation*}
    \tilde{O} \left(\gamma \sqrt{d_{\textrm{out}}} (1 + r) (1 + T^{ \frac{7}{6}} \cdot \max_{\alpha \in \complex_{\beta}} \max \{\abs{p_n(\alpha)}, \abs{p_n'(\alpha} \}) \sqrt{T}\right),
\end{equation*}
where we recall that $r$ bounds the maximum coefficient of $p_n(\cdot)$.  This leads us to the following question: \textbf{Is there a universal choice of polynomial $p_n(x)$, where $n$ is independent of hidden dimension, which guarantees sublinear regret? } \\ % for $\beta = \Omega(1/\log(T))$, the Spectral Filtering algorithm provides sub-linear regret?}\\

For the real line, the answer to this question is known to be positive using the Chebyshev polynomials of the first kind. In general, the $n^{\textrm{th}}$ (monic) Chebyshev polynomial $M_n(x)$ satisfies
\begin{align*}
    \max_{x \in [-1,1]} \abs{M_n(x)} \leq 2^{-(n-1)} \qquad \textrm{ and } \qquad  \max_{x \in [-1,1]} \abs{M_n'(x)} \leq n 2^{-(n-1)}.
\end{align*}

However, we are interested in a more general question over the complex plane. Since we care about linear dynamical systems that evolve according to a general asymetric matrix,  we need to extending our analysis to $\complex_{\beta}$. This is a nontrivial extension since, in general, functions that are bounded on the real line can grow exponentially on the complex plane. Indeed, $2^{n-1} M_n(x) = \cos ( n \arccos(x) )$ and while $\cos(x)$ is bounded within $[-1,1]$ for any $x \in \R$, over the complex numbers we have 
$$\cos(z) = \frac{1}{2}(e^{iz} + e^{-iz}) , $$ 
which is unbounded. Thus, we analyze the Chebyshev polynomial on the complex plane and provide the following bound.
\begin{lemma}
\label{lemma:cheby_bound}
    Let $z \in \complex$ be some complex number with magnitude $\abs{\alpha} \leq 1$. Let $M_n(\cdot)$ denote the $n$-th monic Chebyshev polynomial. If $\abs{\mathrm{Im}(z)} \leq 1/64n^2$, then
    \begin{equation*}
        \abs{M_n(z)} \leq \frac{1}{2^{n-2}} \qquad \textrm{ and } \qquad 
        \abs{M_n'(z)} \leq \frac{n^2}{2^{n-1}}.
    \end{equation*}
\end{lemma}
We provide the proof in Appendix~\ref{appendix:chebyshev}. We also must analyze the magnitude of the coefficients of the Chebyshev polynomial, which can grow exponentially with $n$. We provide the following result. 
\begin{lemma}
\label{lemma:cheby_coeffs_bound}
    Let $M_n(\cdot)$ have coefficients $c_0, \dots, c_n$. Then,
    \begin{equation*}
        \max_{k = 0, \dots, n} \abs{c_k} \leq 2^{0.3n}.
    \end{equation*}
\end{lemma}
The proof of Lemma~\ref{lemma:cheby_coeffs_bound} is in Appendix~\ref{appendix:chebyshev}. 
\section{Discussion of Assumptions}\label{sec:discussion}
In this paper, we have made several assumptions for the sake of clarity and simplicity. In this section, we discuss the rationale behind these assumptions, the extent to which these assumptions hold in practice, and the consequences for our protocol when these assumptions hold.

\subsection{Assumptions on the Demand}

There are two simplifying assumptions we make about the demand. First, we assume the demand at any time is relatively small compared to the channel capacities. Second, we take the demand to be constant over time. We elaborate upon both these points below.

\paragraph{Small demands} The assumption that demands are small relative to channel capacities is made precise in \eqref{eq:large_capacity_assumption}. This assumption simplifies two major aspects of our protocol. First, it largely removes congestion from consideration. In \eqref{eq:primal_problem}, there is no constraint ensuring that total flow in both directions stays below capacity--this is always met. Consequently, there is no Lagrange multiplier for congestion and no congestion pricing; only imbalance penalties apply. In contrast, protocols in \cite{sivaraman2020high, varma2021throughput, wang2024fence} include congestion fees due to explicit congestion constraints. Second, the bound \eqref{eq:large_capacity_assumption} ensures that as long as channels remain balanced, the network can always meet demand, no matter how the demand is routed. Since channels can rebalance when necessary, they never drop transactions. This allows prices and flows to adjust as per the equations in \eqref{eq:algorithm}, which makes it easier to prove the protocol's convergence guarantees. This also preserves the key property that a channel's price remains proportional to net money flow through it.

In practice, payment channel networks are used most often for micro-payments, for which on-chain transactions are prohibitively expensive; large transactions typically take place directly on the blockchain. For example, according to \cite{river2023lightning}, the average channel capacity is roughly $0.1$ BTC ($5,000$ BTC distributed over $50,000$ channels), while the average transaction amount is less than $0.0004$ BTC ($44.7k$ satoshis). Thus, the small demand assumption is not too unrealistic. Additionally, the occasional large transaction can be treated as a sequence of smaller transactions by breaking it into packets and executing each packet serially (as done by \cite{sivaraman2020high}).
Lastly, a good path discovery process that favors large capacity channels over small capacity ones can help ensure that the bound in \eqref{eq:large_capacity_assumption} holds.

\paragraph{Constant demands} 
In this work, we assume that any transacting pair of nodes have a steady transaction demand between them (see Section \ref{sec:transaction_requests}). Making this assumption is necessary to obtain the kind of guarantees that we have presented in this paper. Unless the demand is steady, it is unreasonable to expect that the flows converge to a steady value. Weaker assumptions on the demand lead to weaker guarantees. For example, with the more general setting of stochastic, but i.i.d. demand between any two nodes, \cite{varma2021throughput} shows that the channel queue lengths are bounded in expectation. If the demand can be arbitrary, then it is very hard to get any meaningful performance guarantees; \cite{wang2024fence} shows that even for a single bidirectional channel, the competitive ratio is infinite. Indeed, because a PCN is a decentralized system and decisions must be made based on local information alone, it is difficult for the network to find the optimal detailed balance flow at every time step with a time-varying demand.  With a steady demand, the network can discover the optimal flows in a reasonably short time, as our work shows.

We view the constant demand assumption as an approximation for a more general demand process that could be piece-wise constant, stochastic, or both (see simulations in Figure \ref{fig:five_nodes_variable_demand}).
We believe it should be possible to merge ideas from our work and \cite{varma2021throughput} to provide guarantees in a setting with random demands with arbitrary means. We leave this for future work. In addition, our work suggests that a reasonable method of handling stochastic demands is to queue the transaction requests \textit{at the source node} itself. This queuing action should be viewed in conjunction with flow-control. Indeed, a temporarily high unidirectional demand would raise prices for the sender, incentivizing the sender to stop sending the transactions. If the sender queues the transactions, they can send them later when prices drop. This form of queuing does not require any overhaul of the basic PCN infrastructure and is therefore simpler to implement than per-channel queues as suggested by \cite{sivaraman2020high} and \cite{varma2021throughput}.

\subsection{The Incentive of Channels}
The actions of the channels as prescribed by the DEBT control protocol can be summarized as follows. Channels adjust their prices in proportion to the net flow through them. They rebalance themselves whenever necessary and execute any transaction request that has been made of them. We discuss both these aspects below.

\paragraph{On Prices}
In this work, the exclusive role of channel prices is to ensure that the flows through each channel remains balanced. In practice, it would be important to include other components in a channel's price/fee as well: a congestion price  and an incentive price. The congestion price, as suggested by \cite{varma2021throughput}, would depend on the total flow of transactions through the channel, and would incentivize nodes to balance the load over different paths. The incentive price, which is commonly used in practice \cite{river2023lightning}, is necessary to provide channels with an incentive to serve as an intermediary for different channels. In practice, we expect both these components to be smaller than the imbalance price. Consequently, we expect the behavior of our protocol to be similar to our theoretical results even with these additional prices.

A key aspect of our protocol is that channel fees are allowed to be negative. Although the original Lightning network whitepaper \cite{poon2016bitcoin} suggests that negative channel prices may be a good solution to promote rebalancing, the idea of negative prices in not very popular in the literature. To our knowledge, the only prior work with this feature is \cite{varma2021throughput}. Indeed, in papers such as \cite{van2021merchant} and \cite{wang2024fence}, the price function is explicitly modified such that the channel price is never negative. The results of our paper show the benefits of negative prices. For one, in steady state, equal flows in both directions ensure that a channel doesn't loose any money (the other price components mentioned above ensure that the channel will only gain money). More importantly, negative prices are important to ensure that the protocol selectively stifles acyclic flows while allowing circulations to flow. Indeed, in the example of Section \ref{sec:flow_control_example}, the flows between nodes $A$ and $C$ are left on only because the large positive price over one channel is canceled by the corresponding negative price over the other channel, leading to a net zero price.

Lastly, observe that in the DEBT control protocol, the price charged by a channel does not depend on its capacity. This is a natural consequence of the price being the Lagrange multiplier for the net-zero flow constraint, which also does not depend on the channel capacity. In contrast, in many other works, the imbalance price is normalized by the channel capacity \cite{ren2018optimal, lin2020funds, wang2024fence}; this is shown to work well in practice. The rationale for such a price structure is explained well in \cite{wang2024fence}, where this fee is derived with the aim of always maintaining some balance (liquidity) at each end of every channel. This is a reasonable aim if a channel is to never rebalance itself; the experiments of the aforementioned papers are conducted in such a regime. In this work, however, we allow the channels to rebalance themselves a few times in order to settle on a detailed balance flow. This is because our focus is on the long-term steady state performance of the protocol. This difference in perspective also shows up in how the price depends on the channel imbalance. \cite{lin2020funds} and \cite{wang2024fence} advocate for strictly convex prices whereas this work and \cite{varma2021throughput} propose linear prices.

\paragraph{On Rebalancing} 
Recall that the DEBT control protocol ensures that the flows in the network converge to a detailed balance flow, which can be sustained perpetually without any rebalancing. However, during the transient phase (before convergence), channels may have to perform on-chain rebalancing a few times. Since rebalancing is an expensive operation, it is worthwhile discussing methods by which channels can reduce the extent of rebalancing. One option for the channels to reduce the extent of rebalancing is to increase their capacity; however, this comes at the cost of locking in more capital. Each channel can decide for itself the optimum amount of capital to lock in. Another option, which we discuss in Section \ref{sec:five_node}, is for channels to increase the rate $\gamma$ at which they adjust prices. 

Ultimately, whether or not it is beneficial for a channel to rebalance depends on the time-horizon under consideration. Our protocol is based on the assumption that the demand remains steady for a long period of time. If this is indeed the case, it would be worthwhile for a channel to rebalance itself as it can make up this cost through the incentive fees gained from the flow of transactions through it in steady state. If a channel chooses not to rebalance itself, however, there is a risk of being trapped in a deadlock, which is suboptimal for not only the nodes but also the channel.

\section{Conclusion}
This work presents DEBT control: a protocol for payment channel networks that uses source routing and flow control based on channel prices. The protocol is derived by posing a network utility maximization problem and analyzing its dual minimization. It is shown that under steady demands, the protocol guides the network to an optimal, sustainable point. Simulations show its robustness to demand variations. The work demonstrates that simple protocols with strong theoretical guarantees are possible for PCNs and we hope it inspires further theoretical research in this direction.


 
\newpage

\bibliographystyle{apalike}
\bibliography{main}

\newpage
\appendix
\section{Chebyshev Polynomials Evaluated in the Complex Plane}
\label{appendix:chebyshev}
In this section we let $T_n$ denote the $\th{n}$ Chebyshev polynomial and let $M_n$ denote the monic form. 

\begin{proof}[Proof of Lemma~\ref{lemma:cheby_bound}]
We use that $M_n(z) = T_n(z)/2^{n-1}$ and
\begin{equation}
T_n(z) =  \cos \left( n \arccos(z)\right).
\end{equation}
If $\abs{\mathrm{Im}(z)} \leq 1/64n^2$ then by Lemma~\ref{lemma:arccos}, $\arccos(z) \leq 1/n$. Therefore $n \arccos(z) \leq 1$ and so by Fact~\ref{fact:cos_bound}, 
\begin{equation}
\label{eqn:tbound}
    T_n(z) = \cos( n \arccos(z)) \leq 2
\end{equation}
Now we turn to the derivative $M_n'(z)$. It's a fact that
\begin{equation}
    M_n'(z) = \frac{n}{2^{n-1}} U_{n-1}(z),
\end{equation}
where $U_{n-1}$ is the Chebyshev polynomial of the second kind. We next use the fact that
\begin{equation*}
    U_{n-1}(z) = \begin{cases} 2 \sum_{\substack{j \geq 0 \\ j \textrm{ even } }}^n T_j(z), & n \textrm{ even}, \\
    2 \sum_{\substack{j \geq 0 \\ j \textrm{ odd } }}^n T_j(z), & n \textrm{ odd} .
    \end{cases}
\end{equation*}
By Eq~\eqref{eqn:tbound}, $\abs{T_j(z)} \leq 2$ for any $j$ and therefore
\begin{equation}
    \abs{U_{n-1}(z)} \leq n.
\end{equation}
Therefore, 
\begin{equation*}
    \abs{M_n'(z)} \leq \frac{n^2}{2^{n-1}}.
\end{equation*}
\end{proof}


\begin{fact}
\label{fact:cos_bound}
    Let $z \in \complex$. Then $\abs{\cos(z)} \leq 2$ whenever $\abs{\mathrm{Im}} \leq 1$.
\end{fact}
\begin{proof}[Proof of Fact~\ref{fact:cos_bound}]
   \begin{align*}
       \abs{ \cos(x + i y) } & = \left( \cos^2 x \cosh^2 y + \sin^2 x \sinh^2 y \right)^{1/2} \tag{Uses standard complex cosine identity.} \\
       & = \left( \cos^2 x  +  \cos^2 x \left(\cosh^2 y - 1 \right) + \sin^2 x \sinh^2 y \right)^{1/2}  \\
        & = \left( \cos^2 x  +  \cos^2 x \sinh^2 y + \sin^2 x \sinh^2 y \right)^{1/2} \tag{$\cosh^2 y - \sinh^2 y = 1$} \\
         & = \left( \cos^2 x  +   \sinh^2 y \right)^{1/2} \tag{$\cos^2 x + \sin^2 x = 1$} \\
         & \leq \left( 1  +   \sinh^2 y \right)^{1/2} \tag{$\sinh^2 y \leq 2$ when $\abs{y} \leq 1$.} \\
          & \leq 2.
   \end{align*}
\end{proof}

\begin{lemma}
\label{lemma:arccos}
    Let $z \in \complex$ with $\abs{z} \leq 1$. Then $\abs{ \mathrm{Im} \left( \arccos(z) \right) } \leq 1/n$ whenever $\abs{\mathrm{Im}(z)} \leq 1/64n^2$.
\end{lemma}
\begin{proof}[Proof of Lemma~\ref{lemma:arccos}]
    Let $re^{i \theta} = z$. We use the Taylor series for $\arccos(\cdot)$,
    \begin{align*}
        \arccos(re^{i \theta}) & = \frac{\pi}{2} - \sum_{k = 0}^{\infty} a_k (re^{i \theta})^{2k + 1} \tag{For $a_k = \frac{(2k)!}{4^k (k!)^2 (2k + 1)}$} \\
        & = \frac{\pi}{2} - \sum_{k = 0}^{\infty} a_k r^{2k + 1} e^{i (2k + 1) \theta} \tag{De Moivre's Theorem}\\
        & = \frac{\pi}{2} - \sum_{k = 0}^{\infty} a_k r^{2k + 1}  \cos ( (2k + 1) \theta)  - i \sum_{k = 0}^{\infty} a_k r^{2k + 1} \sin( (2k + 1) \theta). \tag{$e^{i \theta} = \cos \theta + i \sin \theta$} 
    \end{align*}
    Therefore,
    \begin{equation*}
         \mathrm{Im} \left( \arccos(re^{i \theta}) \right)  = \sum_{k = 0}^{\infty} a_k r^{2k + 1} \sin( (2k + 1) \theta).
    \end{equation*}
    Then
    \begin{align*}
         \abs{ \mathrm{Im} \left( \arccos(re^{i \theta}) \right)  } & \leq \sum_{k = 0}^{\infty} a_k  \abs{r}^{2k+1} \abs{ \sin( (2k + 1) \theta)}  \\
         & \leq \sum_{k = 0}^{\infty} a_k \abs{r}^{2k+1} \min (1, (2k + 1) \abs{ \theta} ) \tag{$\abs{\sin(x)} \leq \min (\abs{x}, 1)$} \\
          & \leq 2 \sum_{k = 0}^{\infty} a_k  \min (1, (2k + 1) \abs{r \sin (\theta) }  ) \tag{$\abs{r} \leq 1$ and $\abs{x} \leq 2 \abs{\sin{x}}$} \\
           & \leq 2 \sum_{k = 0}^{\infty} a_k  \min (1, (2k + 1) \abs{\mathrm{Im}(z) }  ) \tag{$ \mathrm{Im}(z) = r \sin \theta$} \\
         & \leq \sum_{k = 0}^{K} a_k (2k+1) \abs{\mathrm{Im}(z)} + \sum_{k = K+1}^{\infty} a_k  . \tag{For any arbitrary $K \geq 0$}
    \end{align*}
    Now we bound $a_k$.
\begin{align*}
    a_k (2k + 1)& = \frac{(2k)!}{4^k (k!)^2} \\
    & \leq \frac{\sqrt{2 \pi (2k)} (2k/e)^{2k} \left( 1 + \frac{1}{2k} \right)}{4^k \left( \sqrt{2 \pi k} (k/e)^k \right)^2 } \tag{Stirling's Formula }%, $\sqrt{2 \pi x} (x/e)^x \leq x! \leq \sqrt{2 \pi x} (x/e)^x (1 + (1/x))$} 
    \\
    & = \left( 1 + \frac{1}{2k} \right)/\sqrt{ \pi k}   \\
    & \leq 1/\sqrt{k}.
\end{align*}
Therefore we also have that $a_k \leq 1/k^{3/2}$. Using this (and noting that $a_0 = 1$) we see,
 \begin{align*}
         \abs{ \mathrm{Im} \left( \arccos(z) \right)  } & \leq \abs{\mathrm{Im}(z)} \left( 1 + \sum_{k =1}^K \frac{1}{\sqrt{k}} \right) + \sum_{k = K}^{\infty} \frac{1}{k^{3/2}} \\
         & \leq 4 \left( \abs{\mathrm{Im}(z)} \sqrt{K} + \frac{1}{\sqrt{K}} \right) \\
         & \leq \frac{8}{\sqrt{K}} \tag{For $\abs{\mathrm{Im}(z)}\leq 1/K$} \\
         & \leq \frac{1}{n} \tag{For $K \geq 64 n^2$.}
         \end{align*}
         Therefore, for $\abs{\mathrm{Im}(z)} \leq 1/64n^2$, we have that
         \begin{equation*}
             \abs{\mathrm{Im} ( \arccos (z) ) } \leq 1/n.
         \end{equation*}
\end{proof}

\begin{proof}[Proof of Lemma~\ref{lemma:cheby_coeffs_bound}]
We bound the coefficients of the Chebyshev polynomial. From Chapter 22 of \cite{abramowitz1948handbook},
\begin{equation}
\label{eqn:coefficients_cheby}
    T_n(x) = \frac{n}{2} \sum_{m=0}^{\lfloor n/2 \rfloor} (-1)^m \frac{(n-m-1)!}{m!(n-2m)!} (2x)^{n-2m}.
\end{equation}
Therefore
\begin{equation*}
    M_n(x) = \frac{1}{2^{n-1}} T_n(x) = n \sum_{m=0}^{\lfloor n/2 \rfloor} (-1)^m \frac{(n-m-1)!}{m!(n-2m)!} 2^{-2m} x^{n-2m}.
\end{equation*}
Let $c_m = \frac{(n-m-1)!}{m!(n-2m)!} 2^{-2m}$. Then
\begin{align*}
    \max_{m = 0, \dots, n} c_m & \leq \max_{m = 0, \dots, n}  {n-m \choose m} 4^{-m} \\
    & \leq \max_{m = 0, \dots, n} \left( \frac{(n-m)e}{4m}\right)^m \tag{${n \choose k }\leq (ne/k)^k$} \\
    & \leq \max_{c \in [0,1]} \left( \frac{(1-c)e}{4c}\right)^{cn} \tag{Letting $m = cn$} \\
    & \leq 2^{0.3n}. \tag{$\max_{c \in [0,1]} ((1-c)e/4c)^c \leq 2^{0.3}$}
\end{align*}
\end{proof}
\section{Proof of Theorem~\ref{thm:main_regret}}
\label{appendix:main_regret}
We prove the regret bound on the following (equivalent) algorithm, where we rescale the parameter $\M_j$ by $\sqrt{T}$ and the input $\langle \phi_j, \uv_{(t-n-i):1} \rangle$ by $1/\sqrt{T}$. We account for this rescaling by increasing the size of the domain for $\M$ by $\sqrt{T}$. 
The proof of Theorem~\ref{thm:main_regret} is composed of two parts: first, we show in Lemma~\ref{lemma:ogd} that Algorithm~\ref{alg:new_sf} achieves the stated regret with respect to the best predictor in the class of hybrid spectral-auto-regressive predictor. Then we proceed with an approximation lemma, Lemma~\ref{lemma:approximation}, showing that the best predictor in the class of hybrid spectral-auto-regressive predictors approximates the best LDS predictor. 


\subsection{Online Gradient Descent Convergence}
The following lemma provides the regret of Algorithm~\ref{alg:new_sf} when compared to the best predictor in the class of hybrid spectral-auto-regressive predictors, denoted by $\K$.




\begin{proof}[Proof of Lemma~\ref{lemma:ogd}]
Let $G = \max_{t \in [T]} \norm{\nabla_{\Q,\M} \ell_t(\Q^t, \M^t, L)}$ and let $$D = \max_{(\Q^1, \M^1), (\Q^2, \M^2) \in \K} \norm{(\Q^1, \M^1) - (\Q^2, \M^2)}.$$ By Theorem 3.1 from \cite{hazan2016introduction}, 
    \begin{equation*}
        \sum_{t = 1}^T \ell_t(\Q^t, \M^t) -  \min_{\M^* \in \K}  \sum_{t = 1}^T \ell_t(\Q^*, \M^*) \leq \frac{3}{2} GD \sqrt{T}.
    \end{equation*}
    Therefore it remains to bound $G$ and $D$. By definition of $\K$ we have
    \begin{equation*}
        D \leq n^2 R_\Q + k R_\M \leq C' k n^2 \gamma \log(T)^{1/6} \left( 1 + T^{2/3}2^{-n}\right).
    \end{equation*}
    Therefore, up to poly logarithmic factors in $T$ (assuming $n$ is polynomially bounded in $T$),
     \begin{equation*}
        D \leq c' \gamma \left( 1 + T^{2/3}2^{-n}\right).
    \end{equation*}
    For $G$ we compute the subgradient at any $\Q_i$ and $\M_i$. 
    For notational simplicity, let $v_j = \uv_{t-n-1:1} \phi_j T^{-1/2}$ and note that $\norm{v_j}_2 \leq 1$ since $\norm{u_{t-n-1:1} \phi_j T^{-1/2}}_2 \leq \norm{u_{t-n-1:1}}_{\infty} \norm{\phi_j}_1 T^{-1/2} \leq 1$, where we used that $\norm{\phi_j}_2 = 1$ implies $\norm{\phi_j}_1 \leq \sqrt{T}$. Also for convenience, let $\mathrm{ind}(i,s) = \max(0, i-n+s) + 1$. Recall $r = \max_{i \in [n]} \abs{c_i}$ denotes the maximum coefficient of the polynomial. Using this, we bound the subgradient norm as follows,
  \begin{align*}
        \norm{ \nabla_{\M_j} \ell_t(\Q,\M) }  
        & = \left\|  \mathrm{sign} \left( y_t - \left(\sum_{i = 1}^n -c_i y_{t-i} + \sum_{s = 0}^{n-1} \sum_{i = 1}^n c_i \Q_{\mathrm{ind}(i,s)} \uv_{t-s} + \sum_{j = 1}^k \M_j v_j \right)  \right)  v_j^{\top} \right\| \\
            & \leq   \sqrt{d_{\textrm{out}}}. \tag{$\norm{v_j}_2 \leq 1$ and the $\sign(\cdot)$ has entries in $\{ \pm 1 \}$ and is dimension $d_{out}$.}
    \end{align*}
    Next we bound the subgradient with respect to $\Q$. 
    \begin{align*}
        \norm{ \nabla_{\Q_{\mathrm{ind}(i,s)}} \ell_t(\Q,\M) }  
        & = \left\|  \mathrm{sign}  \left( y_t - \left(\sum_{i = 1}^n -c_i y_{t-i} + \sum_{s = 0}^{n-1} \sum_{i = 1}^n c_i \Q_{\mathrm{ind}(i,s)} \uv_{t-s} + \sum_{j = 1}^k \M_j v_j \right)  \right)  c_i \uv_{t-s}^{\top} \right\| \\
        & \leq \sqrt{d_{\textrm{out}}} r. \tag{$\norm{u_{t-s}}_2 \leq 1$, $\max_{i \in [n]} \abs{c_i} \leq r$, and the $\sign(\cdot)$ has entries in $\{ \pm 1 \}$ and is dimension $d_{out}$.}
    \end{align*}
    Then $G \leq n^2 r \sqrt{d_{out}} + k \sqrt{d_{out}}$. Therefore, 
    \begin{align*}
        D G \leq (n^2 R_\Q + k R_\M) ( n^2 r +k) \sqrt{d_{out}} \leq 4 n^4 k^2 \sqrt{d_{\textrm{out}}} (1 + r) (R_\Q + R_\M).
    \end{align*}
    Therefore, we have 
\begin{equation*}
        \sum_{t = 1}^T f_t(\hat{y}_t)  -  \min_{(\Q^*,\M^*) \in \K}  \sum_{t = 1}^T f_t(\hat{y}_t(\Q^*, \M^*)) \leq  6 n^4 k^2 \sqrt{d_{\textrm{out}}} (1 + r) (R_\Q + R_\M).
    \end{equation*}

\end{proof}


\subsection{The hybrid autoregressive-spectral-filtering class contains the LDS predictor class}
\label{appendix:M_matrices}
In this section we prove Lemma~\ref{lemma:approximation} which includes the proof of Lemma~\ref{lemma:M_matrices} from the main body of the paper. We require a critical lemma in order to do so, which we present now. 

\begin{lemma}
\label{lemma:spectral_filtering_property}
Given polynomial $p_n(\cdot)$, let $\mu_{p_n}(\alpha)$ be as defined in Eq~\eqref{eqn:mu_alpha} and $ \mat{Z}_{p_n} $ be as defined in Eq~\eqref{eqn:Z_def}.
There is a universal constant $C>0$ such that for $\phi_1, \dots, \phi_{T-n}$ as the eigenvectors of $\mat{Z}_{p_n}$
\begin{equation}
    \max_{\alpha \in S} \abs{\mu_{p_n}(\alpha)^{\top} \phi_j} \leq C \log(T)^{1/6} T^{2/3} c^{-j/6 \log T} \cdot \max_{\complex_{\beta}} \max \left \{ \abs{p_n(\alpha)}, \abs{p_n'(\alpha)} \right \}.
\end{equation}
\end{lemma}

The proof of Lemma~\ref{lemma:spectral_filtering_property} is nontrivial and therefore we dedicate a separate section to it in Appendix~\ref{appendix:spectral_filtering_property}. 
\begin{proof}[Proof of Lemma~\ref{lemma:approximation}]
If $\pi(\A, \B, \C)$ is an LDS predictor paramterized by matrices $(\A, \B, \C)$ then
\begin{equation*}
 \y_t^{\pi(\A, \B, \C)} = \sum_{s = 1}^{t} \C \A^{t-s} \B \uv_{s}.
\end{equation*}
Using the derivation of Eq~\eqref{eqn:lds_breakdown} we have
\begin{equation}
\label{eqn:helper}
     \y_t^{\pi(\A, \B, \C)} = \sum_{i = 1}^n -c_i \y_{t-i} + \sum_{s = 0}^{n-1} \sum_{i = 1}^n c_i \C\A^{\max (0, i-n+s)} \B \uv_{t-s} + \sum_{s = 0}^{t-n-1} \C p_n(\A) \A^s \B \uv_{t-n-s}.
\end{equation}
Since $\max_{i \in [n]} \norm{\C\A^i \B} \leq \gamma$ then for $\Q_s \defeq \C\A^{\max (0, i-n+s)} \B $ it is the case that $\norm{\Q_s} \leq R_\Q$. Next we turn our attention to the spectral filtering parameters. Recall our definition of $\mu_{p_n}(\alpha)$,
\begin{equation*}
    \mu_{p_n}(\alpha) \defeq p_n(\alpha) \begin{bmatrix}
        1 & \alpha & \dots & \alpha^{t-n-1}
    \end{bmatrix}^{\top}
\end{equation*}
and
\begin{equation}
    \uv_{t:1} \defeq \begin{bmatrix}
        & \uv_t \\
        & \uv_{t-1} \\
        & \vdots \\
        & \uv_1
    \end{bmatrix}.
\end{equation}
We henceforth assume that $\A$ is diagonalizable over the complex numbers. This is w.l.o.g., since we can perturb $\A$ with an arbitrary small perturbation, and we know that the set of diagonalizable matrices over the complex numbers is dense.  
Observe that if $P$ diagonalizes $\A$ and if $D_A$ is the diagonalization of $\A$, 
\begin{align*}
    \sum_{s = 1}^{t-n-1} \C p_n(\A) \A^s \B \uv_{t-n-s} & =  \sum_{s = 1}^{t-n-1} \C P  p_n(D_A) D_A^s P^{\top} \B \uv_{t-n-s}  \\
     & =  \sum_{s = 1}^{t-n-1} \C P  \sum_{\ell = 1}^{d_{\textrm{hidden}}}  p_n(\alpha_{\ell}) \alpha_{\ell}^s e_{\ell} e_{\ell}^{\top}  P^{\top} \B \uv_{t-n-s}  \\
  & = \sum_{\ell=1}^{d_{\textrm{hidden}}}  \left( \C P  e_{\ell} \right)  
   \left( P^{\top} \B^{\top}  e_{\ell} \right)^{\top} \sum_{s = 1}^{t-n-1} p_n(\alpha_{\ell}) \alpha_{\ell}^s \uv_{t-n-s} \\
   & = \sum_{\ell=1}^{d_{\textrm{hidden}}}  \left( \C P  e_{\ell} \right)  
   \left( P^{\top} \B^{\top}  e_{\ell} \right)^{\top} \mu_{p_n}(\alpha)^{\top} \uv_{(t-n-1):1} \\
   & = \sum_{\ell=1}^{d_{\textrm{hidden}}}  \left( \C P  e_{\ell} \right)  
   \left( P^{\top} \B^{\top}  e_{\ell} \right)^{\top} \mu_{p_n}(\alpha) \left( \sum_{j = 1}^{T-n} \phi_j \phi_j^{\top} \right)u_{(t-n-1):1} \tag{Orthonormality of the filters} \\
   & = \sum_{j = 1}^{T-n} \left( \sum_{\ell=1}^{d_{\textrm{hidden}}}  \left( \C P  e_{\ell} \right)  
   \left( P^{\top} \B^{\top}  e_{\ell} \right)^{\top} \mu_{p_n}(\alpha)^{\top} \phi_j \right) \phi_j^{\top} \uv_{(t-n-1):1} \\
   & = \sum_{j = 1}^{T-n} \left(  C\B^{\top} \mu_{p_n}(\alpha)^{\top} \phi_j \right) \phi_j^{\top} \uv_{(t-n-1):1} \\
   & =  \sum_{j = 1}^{T-n} \left(  T^{1/2} \C \B^{\top} \mu_{p_n}(\alpha)^{\top} \phi_j \right) \left( \phi_j^{\top} \uv_{(t-n-1):1} T^{-1/2} \right) .
\end{align*}
Then for $\M_j =  T^{1/2} C\B^{\top} \mu_{p_n}(\alpha)^{\top} \phi_j$,
\begin{align*}
    \sum_{s = 1}^{t-n-1} \C p_n(\A) \A^s \B \uv_{t-n-s}= \sum_{j = 1}^{T-n} \M_j \left( \phi_j^{\top} \uv_{(t-n-1):1} T^{-1/2} \right) .
\end{align*}
By Lemma~\ref{lemma:spectral_filtering_property} and the assumption that $\max_{\alpha \in S} \{ \abs{p_n(\alpha)}, \abs{p_n'(\alpha)} \} \leq \pbnd$,
\begin{align*}
    \norm{\M_j}& \leq T^{1/2} \gamma\cdot \max_{\alpha \in S} \abs{\mu_{p_n}(\alpha)^{\top} \phi_j} \\
    & \leq  T^{1/2} n^2 \norm{\C}  \norm{\B} \cdot \left( \C\log(T)^{1/6}  T^{2/3} c^{-j/6 \log T} \right) \cdot \max_{\alpha \in S} \{ \abs{p_n(\alpha)}, \abs{p_n'(\alpha)} \} \tag{Lemma~\ref{lemma:spectral_filtering_property}}\\
    & \leq C \gamma \log(T)^{1/6} T^{7/6}  c^{-j/6 \log T} \pbnd \tag{Assumption that $\max_{\alpha \in S} \{ \abs{p_n(\alpha)}, \abs{p_n'(\alpha)} \} \leq \pbnd$} \\
    & \leq R_\M  c^{-j/6 \log T} \\
    & \leq R_\M.
\end{align*}
Define
\begin{equation*}
    \y^{\pi(\Q, \M)} \defeq \sum_{i = 1}^n -c_i y_{t-i} + \sum_{s = 0}^{n-1} \Q_{\max(0, n-i+s)} \uv_{t-s} + \sum_{j = 0}^{k} \M_j \phi_j^{\top} \uv_{(t-n-1):1}.
\end{equation*}
Observe that $\pi(\Q, \M) \in \prod_{H}^{\K}$. Next we show that $\pi(\Q, \M)$ is an $\epsilon$-approximation for $\y_t^{\pi(\A, \B, \C)}$. We have
\begin{align*}
    \norm{ \y_t^{\pi(\A, \B, \C)} -  \y^{\pi(\Q, \M)}}_2^2 & = \norm{ \sum_{j = k+1}^{T-n} \M_j \phi_j^{\top} \uv_{(t-n-1):1}T^{-1/2}}_2^2 \\
    & \leq \sum_{j = k+1}^{T-n} \norm{ \M_j } \norm{\phi_j^{\top} \uv_{(t-n-1):1}T^{-1/2}}_2 \\
     & \leq \sum_{j = k+1}^{T-n} \norm{ \M_j } \norm{\phi_j^{\top}}_1 \norm{ \uv_{(t-n-1):1}}_{\infty} T^{-1/2} \\
     & \leq \sum_{j = k+1}^{T-n} \norm{ \M_j } \tag{$\norm{\phi_j}_2 = 1$ and therefore $\norm{\phi_j}_1 \leq \sqrt{T} $ } \\
      & \leq R_\M  c^{-j/6 \log T}  \\
      & \leq \epsilon. \tag{Definition of $k$.} 
\end{align*}
\end{proof}


\section{The Spectral Filtering Property Lemma~\ref{lemma:spectral_filtering_property}}
\label{appendix:spectral_filtering_property}


In order to prove Lemma~\ref{lemma:spectral_filtering_property} we require two helper lemmas. The first is Lemma~\ref{lemma:lipschitz}, which roughly argues that the Lipschitz constant of a function $f: \complex_{\beta} \to \R$ can be bounded by a polynomial of the expectation of the function on $\complex_{\beta}$. The second is Lemma~\ref{lemma:hankel_decay} , which argues that the Hankel matrix which we construct exhibits exponentially fast decay in its spectrum. 




\begin{lemma}
\label{lemma:lipschitz}
Recall 
\[
\complex_{\beta} = \{ \alpha\in\mathbb{C} : |\alpha|\le 1 \text{ and } \operatorname{Im}(\alpha)\le \beta \},
\]
and let $\mathcal{F}$ denote the set of functions $f:\mathbb{C}_{\beta} \to\mathbb{R}$ that are $L$-Lipschitz and attain the maximum value $g_{\max}$ on $\complex_\beta$. Then 
\[
\min_{f\in\mathcal{F}} \int_{\complex_\beta} f(z)\,dz \le \frac{\arcsin(\beta)\,g_{\max}^3}{L^2}.
\]
\end{lemma}

\begin{proof}

Let $ z_0 = \sqrt{1-\beta^2} + i\beta$.
Then $|z_0|=1$ and $\operatorname{Im}(z_0)=\beta$, so $z_0\in S_\beta$. Define
\[
f(z) = \max\{ g_{\max} - L\,|z-z_0|,\,0\}.
\]
Since $f(z_0)=g_{\max}$ and the function $z\mapsto g_{\max}-L\,|z-z_0|$ is $L$-Lipschitz (and the maximum with $0$ does not increase the Lipschitz constant), we have $f\in\mathcal{F}$.

\medskip

Notice that $f(z)>0$ only when
\[
|z-z_0| < R,\quad\text{with } R = \frac{g_{\max}}{L}.
\]
Thus the support of $f$ is contained in the disk
\[
D(z_0,R) = \{ z\in\mathbb{C} : |z-z_0| < R \}.
\]
Since $z_0$ lies on the horizontal line $\operatorname{Im}(z)=\beta$, one may verify that the intersection
\[
D(z_0,R)\cap S_\beta
\]
is contained in a circular sector of radius $R$ with central angle at most $2\arcsin(\beta)$. An upper bound for the area of this sector is
\[
\operatorname{Area}\bigl(D(z_0,R)\cap S_\beta\bigr) \le \frac{2\arcsin(\beta)}{2\pi} \cdot \pi R^2 
= \arcsin(\beta) \, R^2 
= \arcsin(\beta) \left(\frac{g_{\max}}{L}\right)^2.
\]

Since $f(z)\le g_{\max}$ for all $z$, we deduce
\[
\int_{S_\beta} f(z)\,dz \le g_{\max}\cdot \operatorname{Area}\bigl(D(z_0,R)\cap S_\beta\bigr)
\le g_{\max}\,\arcsin(\beta) \left(\frac{g_{\max}}{L}\right)^2
= \frac{\arcsin(\beta)\, g_{\max}^3}{L^2}.
\]

\medskip
Since we have exhibited an $f\in\mathcal{F}$ satisfying
\[
\int_{S_\beta} f(z)\,dz \le \frac{\arcsin(\beta)\, g_{\max}^3}{L^2},
\]
the inequality follows.

\end{proof}



\ignore{
\begin{proof}[Proof of Lemma~\ref{lemma:lipschitz}]
\annie{We actually need to argue that $f^*$ is in fact the function which minimizes its area.}
   Consider the following function $g^*$
    \begin{align*}
        g^*(r) & = \begin{cases}
            Lr , & r \leq g_{\textrm{max}}/L, \\
            \max \left \{g_{\textrm{max}} - L(r - g_{\textrm{max}}/L), 0 \right \} & g_{\textrm{max}}/L < r \leq 1 
            \end{cases} \\
            & = \begin{cases}
            L r, & r \leq g_{\textrm{max}}/L, \\
           2 g_{\textrm{max}} - Lr ,& g_{\textrm{max}}/L < r\leq 2g_{\textrm{max}}/L.
        \end{cases}
    \end{align*}
    Let $f^*(re^{i\theta}) = g(r) \left( 1 - \frac{\theta}{\beta} \right)$.
    Then
    \begin{align*}
        \int_{\alpha \in S} f^*(\abs{\alpha}) \left( 1 - \frac{\theta}{\beta} \right) d \alpha & = \int_{\alpha: \abs{\alpha} \leq g_{\textrm{max}}/L} f^*(\alpha) d \alpha + \int_{\alpha:g_{\textrm{max}}/L < \abs{\alpha} \leq 1} f^*(\alpha) d \alpha\\
        & = \int_{r=0}^{g_{\textrm{max}}/L} \int_{\theta = -\textrm{arcsin}(\beta)}^{\textrm{arcsin}(\beta)} L r \left( 1 - \frac{\theta}{\beta} \right) (r dr d \theta) \\
        & \qquad + \int_{r=g_{\textrm{max}}/L}^{2g_{\textrm{max}}/L} \int_{\theta = -\textrm{arcsin}(\beta)}^{\textrm{arcsin}(\beta)} \left( 2 g_{\textrm{max}} - Lr \right)\left( 1 - \frac{\theta}{\beta} \right)  (r dr d \theta) \\
        & =  \textrm{arcsin}(\beta) g_{\textrm{max}}^3/L^2.
    \end{align*}
\end{proof}
}


\begin{fact}
    \label{fact:Z_entry_bound}
    Let $Z_{S, p_n, T} \defeq \int_{\alpha \in S} \mu_{p_n}(\alpha) \overline{\mu_{p_n}(\alpha)}^{\top} d \alpha$. Then
    \begin{equation*}
        \abs{  Z_{S, p_n, T} }_{jk} \leq 2 \arcsin(\beta) \cdot \max_{\alpha \in S} \abs{p_n(\alpha)}^2  \cdot  \frac{1}{j + k}.
    \end{equation*}
\end{fact}
\begin{proof}[Proof of Fact~\ref{fact:Z_entry_bound}]
    Observe
\begin{align*}
        \abs{ Z_{S, p_n, T} }_{jk} &  = \abs{ \int_{\alpha \in S} p_n(\alpha) p_n(\overline{\alpha}) \alpha^{j-1} \overline{\alpha}^{k-1} d \alpha } \\
        & \leq \max_{\alpha \in S} \abs{p_n(\alpha)}^2 \cdot  \int_{\alpha \in S}  \abs{\alpha}^{j + k -2} d \alpha  \\
        & = \max_{\alpha \in S} \abs{p_n(\alpha)}^2 \cdot \abs{ \int_{\theta = - \arcsin(\beta)}^{\arcsin(\beta)} \int_{r=0}^1 r^{j +  k - 2}  (r dr d \theta) } \\
         & = 2 \arcsin(\beta) \cdot \max_{\alpha \in S} \abs{p_n(\alpha)}^2  \cdot  \frac{1}{j + k}.
        \end{align*}
\end{proof}
\begin{lemma}[Adapted from Lemma E.2 from \cite{hazan2017learning}]
\label{lemma:hankel_decay}
Given a degree-$n$ polynomial $p_n$, horizon $T$, and complex tolerance $\beta$, let $S= \left \{ \alpha \in \C \textrm{ s.t. } \abs{\alpha} \leq 1 \textrm{ and } \mathrm{Im}(\alpha) \leq \beta \right \}$ and let
\begin{equation*}
    Z_{S, p_n, T} \defeq \int_{\alpha \in S} \mu_{p_n}(\alpha) \overline{\mu_{p_n}(\alpha)} d \alpha.
\end{equation*}
    Let $\sigma_j$ be the $j$-th singular value of $Z_{S, p_n, T}$. Then for absolute constants $c = e^{\pi^2/4}$ and $C = 2225$,
    \begin{equation}
        \sigma_j \leq  C \cdot   \arcsin(\beta)  \cdot \max_{\alpha \in S} \abs{p_n(\alpha)}^2 \cdot (1 +  \ln(T-n)) \cdot c^{-j/\log(T-n)}.
    \end{equation}
\end{lemma}
We prove Lemma~\ref{lemma:hankel_decay} using the same machinery as was introduced in \cite{hazan2017learning}. We use the following result from \cite{beckermann2017singular} which bounds the singular values of any positive semidefinite Hankel matrix. 
\begin{lemma}[Cor. 5.4 in \cite{beckermann2017singular}]
\label{lemma:hankel_result}
    Let $H_L$ be a real psd Hankel matrix of dimension $L$. then,
    \begin{equation}
    \label{eqn:eig_decay1}
        \sigma_{j + 2k}(H_L) \leq 16 \left( \exp \left( \frac{\pi^2}{4 \log(8 \lfloor L/2 \rfloor / \pi ) }\right) \right)^{-2k + 2} \sigma_j(H_L).
    \end{equation}
\end{lemma}
\begin{proof}[Proof of Lemma~\ref{lemma:hankel_decay}]
First we must show that $Z_{S, p_n, T}$ has real entries. This is true simply because we integrate over $z \in \C$ such that $\abs{\mathrm{Im}(z)} \leq \beta$ and so the imaginary components cancel each other out.  Therefore Lemma~\ref{lemma:hankel_result} applies to $Z_{S, p_n, T}$. 
        Note that Lemma~\ref{lemma:hankel_result} implies,
        \begin{equation*}
        \sigma_{j + 2k}(H_L) \leq 16 \left( \exp \left( \frac{\pi^2}{4 \log(8 \lfloor L/2 \rfloor / \pi ) }\right) \right)^{-2k + 2} \mathrm{Tr}(H_L).
        \end{equation*}
        Using Fact~\ref{fact:Z_entry_bound} we have,
        \begin{align*}
            \mathrm{Tr}\left( Z_{S, p_n, T} \right)   & = \sum_{j = 1}^{T-n} \left( Z_{S, p_n, T} \right)_{jj} \\
            & \leq   \arcsin(\beta)  \cdot \max_{\alpha \in S} \abs{p_n(\alpha)}^2  \cdot \sum_{j = 1}^{T-n} \frac{1}{j} \\
            & \leq   \arcsin(\beta)  \cdot \max_{\alpha \in S} \abs{p_n(\alpha)}^2 \cdot (1 +  \ln(T-n)).
        \end{align*}
        Therefore,
        \begin{align*}
        \sigma_{j} & = \begin{cases} \sigma_{2(j/2)}, & \textrm{$j$ even}, \\
        \sigma_{1 + 2(j-1/2)}, & \textrm{$j$ odd.}
        \end{cases} \\
        & \leq 16 \left( \exp \left( \frac{\pi^2}{4 \log(8 \lfloor (T-n) /2 \rfloor / \pi ) }\right) \right)^{-2(j/2) + 2} \left(  \arcsin(\beta)  \cdot \max_{\alpha \in S} \abs{p_n(\alpha)}^2 \cdot (1 +  \ln(T-n)) \right) \\
        &  \leq 16 \exp(\pi^2/4)^2 \arcsin(\beta) \max_{\alpha \in S} \abs{p_n(\alpha)}^2 (1 + \ln(T-n)) c^{-j/\log(T-n)} \tag{For $c = e^{\pi^2/4}$.}\\
        &  \leq 2225 \cdot   \arcsin(\beta)  \cdot \max_{\alpha \in S} \abs{p_n(\alpha)}^2 \cdot (1 +  \ln(T-n)) \cdot c^{-j/\log(T-n)}.
        \end{align*}
\end{proof}
With Lemma~\ref{lemma:lipschitz} and Lemma~\ref{lemma:hankel_decay} in hand, we are ready to prove Lemma~\ref{lemma:spectral_filtering_property}. 
\begin{proof}[Proof of Lemma~\ref{lemma:spectral_filtering_property}]
Let
\begin{equation*}
    f_{j, p_n}(\alpha) \defeq \abs{\phi_j^{\top} \mu_{p_n}(\alpha)}^2.
\end{equation*}

If $f_{j, p_n}(\alpha)$ is $L$-Lipschitz and $S_{\beta}= \left \{ \alpha \in \C \textrm{ s.t. } \abs{\alpha} \leq 1 \textrm{ and } \mathrm{Im}(\alpha) \leq \beta \right \}$ then by Lemma~\ref{lemma:lipschitz},
\begin{equation*}
    \int_{\alpha \in S} f_{j, p_n} (\alpha) d\alpha \geq \frac{2 \arcsin(\beta) \left( \max_{\alpha \in S} f_{j, p_n}(\alpha) \right)^3  }{L^2},
\end{equation*}
or equivalently,
\begin{equation*}
   \max_{\alpha \in S} f_{j, p_n}(\alpha)  \leq \left( \frac{L^2}{2 \arcsin(\beta)}  \int_{\alpha \in S} f_{j, p_n}(\alpha) d\alpha \right)^{1/6}.
\end{equation*}
Observe that
\begin{align*}
\int_{\alpha \in S} f_{j, p_n}(\alpha) d\alpha & = \int_{\alpha \in S}  \abs{\phi_j^{\top} \mu_{p_n}(\alpha)}^2 d \alpha \\
& = \phi_j^{\top} \left( \int_{\alpha \in S}   \mu_{p_n}(\alpha) \mu_{p_n}(\overline{\alpha})^{\top} d \alpha   \right) \overline{\phi_j}  \\
& = \sigma_j.
\end{align*}
Therefore have the following bound
\begin{equation}
\label{eqn:bound1}
    \max_{\alpha \in S} \abs{\mu_{p_n}(\alpha)^{\top} \phi_j} = \max_{\alpha \in S} \sqrt{f_{j, p_n}(\alpha)}  \leq \left( \frac{L^2}{2 \arcsin(\beta)} \sigma_j \right)^{1/6}.
\end{equation}
The remainder of the proof consists of bounding the Lipschitz constant $L$ and bounding the eigenvalue $\sigma_j$.
To bound the Lipschitz constant of $f_{j,p_n}$,
\begin{align*}
  L & \leq \max_{\alpha \in S} \abs{ f'_{j, p_n}(\alpha) } \\
  & =  \max_{\alpha \in S} 2 \abs{ \mathrm{Re} \left( \phi_j^{\top} \mu_{p_n}'(\alpha) \cdot \phi_j^{\top} \mu_{p_n}(\alpha) \right)} \\
    & \leq \max_{\alpha \in S} \norm{\phi_j}_2^2 \cdot \norm{\mu_{p_n}(\alpha)}_2 \cdot \norm{\mu'_{p_n}(\alpha)}_2 \\
    & = \max_{\alpha \in S} \norm{\mu_{p_n}(\alpha)}_2 \cdot \norm{\mu'_{p_n}(\alpha)}_2.
\end{align*}
We have
\begin{align*}
    \norm{\mu_{p_n}(\alpha)}_2^2 & = \abs{p_n(\alpha)}^2 \left( \sum_{s = 0}^{T-n-1} \abs{\alpha}^{2s} \right) \\
    & \leq \abs{p_n(\alpha)}^2  T \\
    & \leq \pbnd^2 T.\tag{Assumption that $\max_{\alpha \in S} \abs{p_n(\alpha)}  \leq \pbnd$. }
\end{align*}
Next, to bound $\norm{\mu_{p_n}'(\alpha)}_2$ we observe,
\begin{align*}
    \left( \mu_{p_n}'(\alpha) \right)_j & = \frac{d}{d \alpha} p_n(\alpha) \alpha^j  =  p_n'(\alpha) \alpha^j + j \alpha^{j-1} p_n(\alpha) \mathbf{1}_{j > 0}.
\end{align*}
Therefore, $ \abs{\left( \mu_{p_n}'(\alpha) \right)_j} \leq \pbnd(1 + T)$ and $\norm{\mu_{p_n}'(\alpha)}_2^2 \leq 2 \pbnd^2 T^3$. Thus,
\begin{equation}
    L^2 \leq 2 \pbnd^4 T^4.
\end{equation}
Next we bound $\sigma_j$. To bound $\sigma_j$, we observe that $Z_{S,p_n}$ is a highly structured matrix called a Hankel matrix and, as such, its spectrum decays exponentially. Indeed, by Lemma~\ref{lemma:hankel_decay}, there are universal constants $C,c >0$ such that
\begin{equation}
\label{eqn:bound2}
   \sigma_j \leq   C \arcsin(\beta)  \cdot \max_{\alpha \in S} \abs{p_n(\alpha)}^2 \cdot \log T \cdot c^{-j/\log T}.
\end{equation}
Recalling Eq~\eqref{eqn:bound1} and the assumption that $\max_{\alpha \in S} \abs{p_n(\alpha)} \leq \pbnd$, we observe
\begin{equation*}
    \max_{\alpha \in S} \abs{\mu_{p_n}(\alpha)^{\top} \phi_j} \leq \left( C \pbnd^6 T^4 \log T c^{-j/\log T} \right)^{1/6}.
\end{equation*}
Therefore, there is a universal constant $C>0$ such that 
\begin{equation*}
    \max_{\alpha \in S} \abs{\mu_{p_n}(\alpha)^{\top} \phi_j} \leq C \log(T)^{1/6}  \pbnd T^{2/3} c^{-j/6 \log T} .
\end{equation*}
\end{proof}





\end{document}

