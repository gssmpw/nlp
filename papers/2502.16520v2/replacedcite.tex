\section{RELATED WORKS}
\label{sec:relatedwork}
The application of predictive analytics to quality control and defect detection in supply chains has garnered increasing attention in recent years. Traditional statistical methods, such as Autoregressive Integrated Moving Average (ARIMA) models, have been extensively utilized for time series forecasting in industrial contexts ____. While effective for linear and stationary data, ARIMA models face challenges with non-linear, high-dimensional, or noisy datasets common in modern supply chains, prompting exploration of alternative approaches ____. For instance, machine learning techniques like Support Vector Machines (SVM) and Random Forests have been applied for anomaly detection and defect prediction in manufacturing processes, but these methods often struggle to model long-term temporal dependencies critical for bad goods prediction.

Advanced time series methods, particularly deep learning approaches, have revolutionized forecasting in supply chains. Recurrent Neural Networks (RNNs), especially Long Short-Term Memory (LSTM) networks, have proven effective for modeling sequential data and predicting equipment failures or product defects over extended periods ____. Convolutional Neural Networks (CNNs) have also been adapted for time series tasks, extracting local features from multivariate inputs to improve prediction accuracy ____. Hybrid models combining CNNs and LSTMs have shown promise in supply chain demand forecasting, as demonstrated by ____, yet their computational complexity and data requirements can limit scalability for real-time applications. Despite these advancements, few studies have specifically targeted bad goods prediction, leaving a gap in quality-specific predictive frameworks for supply chains.

Risk assessment, a cornerstone of supply chain management, has been integrated with predictive models to prioritize critical events. Techniques such as Failure Mode and Effects Analysis (FMEA) and probabilistic risk scoring have been employed to evaluate defect impacts and guide decision-making ____. However, their integration with time series models like ARIMA remains underexplored. Existing research often treats forecasting and risk analysis as separate tasks, reducing the ability to provide unified, actionable insights for bad goods management. This study addresses this gap by proposing a novel framework that combines Time Series ARIMA with risk assessment, leveraging historical data to predict and score bad goods risks for Organic Beer-G 1 Liter. By building on state-of-the-art time series and risk management techniques, this research advances predictive analytics for quality assurance in supply chains.
%%%%%%%%%%%%%%%%%%%%%%%%%%%%%%%%%%%%%%%%%