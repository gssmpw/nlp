
\subsection{Bounds on the Hessian, Gradients and the Predictor}


The convergence analysis makes use of the gradients and Hessians of the empirical loss with respect to the parameters $\vtheta$, namely, 
\begin{align}
    \nabla_{\vtheta} \gL(\vtheta) = 
    \begin{bmatrix}
        \nabla_{\vtheta_f}\gL^\top \; \nabla_{\vtheta_g}\gL^\top
    \end{bmatrix}^\top,\quad \text{and} \qquad 
    \nabla_{\vtheta}^2\gL(\vtheta) = 
    \mH
    \left({\vtheta}\right) 
    & = \left[
        \begin{array}{c c}
        H_{ff} & H_{fg} \\
        H_{gf} & H_{gg}
        \end{array}
    \right],
    \label{eq:gradHessDeepONetLoss}
\end{align}
where $\nabla_{\vtheta_f}\gL(\vtheta)=\partial \gL(\vtheta)/\partial\vtheta_f\in \R^{p_f}$ and  $\nabla_{\vtheta_g}\gL(\vtheta)=\partial \gL(\vtheta)/\partial\vtheta_g\in \R^{p_g}$. Note that we make use of the notation $\nabla_{\vtheta_f}(\cdot)$ to denote the derivative with respect to the parameters $\vtheta_f$ and this \emph{is not} a functional gradient. Similarly, the individual blocks in the $2\times 2$ block Hessian $\mH(\vtheta)$ are given by 
\begin{equation}
    H_{ff} = \nabla^2_{\vtheta_f}\gL = \ddel{\gL}{\vtheta_f},\quad H_{fg} = \frac{{\partial^2\gL}}{\partial\vtheta_f\partial\vtheta_g},\quad H_{gf} = H_{fg}^\top= \frac{{\partial^2\gL}}{\partial\vtheta_g\partial\vtheta_f}, \quad H_{gg} = \nabla^2_{\vtheta_g}\gL = \ddel{\gL}{\vtheta_g},\label{eq:hessian_blocks_expanded}
\end{equation}
where $H_{ff}\in\R^{p_f\times p_f}$,\ $H_{gg}\in\R^{p_g\times p_g}$,  $H_{fg}\in\R^{p_f\times p_g}$, $H_{gf}\in \R^{p_g\times p_f}$ and the argument $\vtheta$ is ignored for clarity of exposition. Using \eqref{eq:loss-don} and rewriting the derivatives in \eqref{eq:gradHessDeepONetLoss} and \eqref{eq:hessian_blocks_expanded}, 
recalling that 
$\ell_{i,j}=(G_{\vtheta}(u^{(i)})(\vy_{j}^{(i)})-G^\dagger(u^{(i)})(\vy^{(i)}_j))^2$, 
we get
\begin{align}
    \del{\gL}{\vtheta_f}
    = 
    \frac{1}{n}\sum_{i=1}^n \frac{1}{q_i} \sum_{j=1}^{q_i} 
    \ell^{\prime}_{i,j} 
    \sum_{k=1}^K g_{k,j}^{(i)} \nabla_{\vtheta_f} f_k^{(i)}
    \quad \text{and}\quad
    \del{\gL}{\vtheta_g} 
    =
    \frac{1}{n}\sum_{i=1}^n \frac{1}{q_i} \sum_{j=1}^{q_i} 
    \ell^{\prime}_{ij} 
    \sum_{k=1}^K f_k^{(i)}\nabla_{\vtheta_g} g_{k,j}^{(i)},
    \label{eq:Gradients_fully_expanded}
\end{align}
% for the gradients, and %\todo{BS: hessians (start here)}
\begin{align}
    \begin{aligned}
        \ddel{\gL}{\vtheta_f} &= 
        \frac{1}{n} \sum_{i=1}^n \frac{1}{q_i}\sum_{j=1}^{q_i} \ell_{i,j}^{\prime} \sum_{k=1}^{K} g_{k,j}^{(i)} \nabla_{\vtheta_f}^{2} f_{k}^{(i)} 
        +
        \frac{1}{n}\sum_{i=1}^n \frac{1}{q_i} \sum_{j=1}^{q_i} \ell_{i,j}^{\prime \prime}
        \left(
            \sum_{k,k^\prime=1}^{K} g_{k,j}^{(i)}g_{k^\prime,j}^{(i)} \nabla_{\vtheta_f} f_{k}^{(i)} \nabla_{\vtheta_f} f_{k^\prime}^{{(i)}\top} \right),\\
        \ddel{\gL}{\vtheta_g} 
        &=
        \frac{1}{n} \sum_{i=1}^{n} \frac{1}{q_i} \sum_{j=1}^{q_i} \ell_{i,j}^{\prime} \sum_{k=1}^{k} f_{k}^{(i)} \nabla_{\vtheta_g}^{2} g_{k,j}^{(i)}
        +
        \frac{1}{n}\sum_{i=1}^n \frac{1}{q_i} \sum_{j=1}^{q_i}
        \ell_{i,j}^{\prime \prime}
        \left(\sum_{k,k^\prime=1}^{K} f_{k}^{(i)}f_{k^\prime}^{(i)} \nabla_{\vtheta_g} g_{k,j}^{(i)
        } \nabla_{
\vtheta_g} {g_{k^\prime,j}}^{{(i)}\top}
        \right),\\
        \frac{{\partial^2\gL}}{\partial\vtheta_f\partial\vtheta_g}
        &=
            \frac{1}{n}\sum_{i=1}^n \frac{1}{q_i} \sum_{j=1}^{q_i} 
        \ell_{i,j}^{\prime} 
        \sum_{k=1}^{K} \nabla_{\vtheta_f} f_{k}^{(i)} \nabla_{\vtheta_g} g_{k,j}^{(i){\top}}
        +
            \frac{1}{n}\sum_{i=1}^n \frac{1}{q_i} \sum_{j=1}^{q_i} \ell_{i,j}^{\prime \prime}
        \left(
            \sum_{k,k'=1}^{K} g_{k,j}^{(i)}f_{k'}^{(i)} \nabla_{\vtheta_f} f_{k}^{(i)}
             \nabla_{\vtheta_g} {g_{k',j}}^{{(i)}\top}
        \right),
    \end{aligned}
    \label{eq:Hessian_blocks_fully_Expanded}
\end{align}
for the individual blocks of the Hessian \eqref{eq:gradHessDeepONetLoss} where we make use of the notation $f^{(i)}_k = f_k (\vtheta_f; u^{(i)})$ and $g^{(i)}_{k,j} = g_k(\vtheta_g;\vy^{(i)}_j)$. In the rest of the paper, with some abuse of notation, we also make use of the implicit notation $f^{(i)}_k(\vtheta_f) = f_k (\vtheta_f; u^{(i)})$ and $g^{(i)}_{k,j}(\vtheta_g) = g_k(\vtheta_g;\vy^{(i)}_j)$.

In order to prove the RSC and smoothness properties of the empirical loss $\cL$, we need to upper bound the spectral norm of its Hessian. As can be seen above, the gradient and Hessian of the predictors (i.e., the branch $f_k^{(i)}$ and trunk $g^{(i)}_{k,j}$ networks, $k\in[K]$, $j\in[q_i]$, $i\in[n]$) appear in the Hessian of $\cL$, and thus, we will eventually need the upper bound of their norms. For this, we will make use of the next lemma.  
\begin{lemm}[{\bf Bounds on the predictors}]
\label{lemm:hessgradbounds}
Under \TwoAsmpsref{asmp:Activation_Function}{asmp:smoothinit}, and for $\vtheta \in B^{\mathrm{Euc}}_{\rho,\rho_1}(\vtheta_0)$, with probability at least {$1-2KL\left(\frac{1}{m_f}+\frac{1}{m_g}\right)$}, we have for every $k\in [K]$, $i\in[n]$, $j\in[q_i]$,
\begin{align}
    \begin{aligned}
        % \max _{i \in[n]}\max_{j\in[q_i]}
        \left\|\nabla^2_{\vtheta_f} f^{(i)}_k \right\| \leq \frac{c^{(f)}}{\sqrt{m_f}} \quad \text{and}\quad 
        %\max _{i \in[n]}\max_{j\in[q_i]}
        \left\|\nabla^2_{\vtheta_g} g^{(i)}_{k,j} \right\| \leq \frac{c^{(g)}}{\sqrt{m_g}}~,
        \\
    \left\| \nabla_{\vtheta_f} f_k^{(i)}\right\|_2 \leq \varrho^{(f)} \quad \text{and}\quad \left\| \nabla_{\vtheta_g} g^{(i)}_{k,j}\right\|_2 \leq \varrho^{(g)}~,\\
    %
     |f_k^{(i)}| \leq \lambda_1, \quad \text{and}\quad |  g^{(i)}_{k,j}| \leq \lambda_2~,
    \end{aligned}\label{eq:gradientBoundG_fg}
\end{align}
where $c^{(f)}$, $c^{(g)}$, $\varrho^{(f)}$, $\varrho^{(g)}$, $\lambda_1$, and $\lambda_2$ are suitable constants that depend on $\sigma_0$, the depth $L$ and the radii $\rho$, $\rho_1$. 
%\pcedit{The dependence of the constants is polynomial and only on the depth and the radii whenever $\sigma_0\leq 1-\rho\max\{\frac{1}{\sqrt{m_f}},\frac{1}{\sqrt{m_g}}\}$.}
%
\pcedit{The dependence of the constants reduces to the depth and the radii and becomes polynomial whenever $\sigma_0\leq 1-\rho\max\{\frac{1}{\sqrt{m_f}},\frac{1}{\sqrt{m_g}}\}$.}
%
\end{lemm}
\proof The proof follows from a direct adaptation of Theorem~4.1 \pcedit{and of both the statement and proof of} Lemma~4.1 in~\citep{banerjee2022restricted} to our setting. \pcedit{Indeed, these results show that $c^{(f)}$, $\varrho^{(f)}$, and $\lambda_1$ depend on $\sigma_0$, the depth $L$ and the radii $\rho$, $\rho_1$; and that such dependence reduces to the depth and the radii and becomes polynomial whenever $\sigma_0\leq 1-\frac{\rho}{\sqrt{m_f}}$. A similar dependence is obtained for the constants $c^{(g)}$, $\varrho^{(g)}$, and $\lambda_2$ whenever $\sigma_0\leq 1-\frac{\rho}{\sqrt{m_g}}$. The last statement in Lemma~\ref{eq:gradientBoundG_fg} follows immediately.} 
Finally, since the bound for a single branch network output holds with probability at least $1-\frac{2L}{m_f}$ and for a single trunk network output holds with probability at least $1-\frac{2L}{m_g}$, then in order for these bounds to hold for the $K$ outputs of all predictors, we obtain the overall probability using De Morgan's law and a union bound.\qed

% % 

\subsection{RSC and Smoothness Results}
%\label{subsec:SmoothActivationProofs}
Using the results from the previous section, we derive the RSC and smoothness results. 

\RSCLoss*


\proof 
%
We start by proving the first part of the theorem's statement. 
%We immediately see that $\vtheta_{t}\notin Q_{\kappa}^t$ by definition of $Q_{\kappa}^t$, and thus $\vtheta_{t}\notin B^t_{\kappa}$. Then, 
%
Since $B^t_{\kappa}\subset B^{\mathrm{Euc}}_{\rho,\rho_1}(\vtheta_0)$, we satisfy Condition~\ref{cond:rsc}(a). We now need to satisfy Condition~\ref{cond:rsc}(b). For this, we first show the existence of an element $\vtheta' \in B^t_{\kappa}$. For such $\vtheta'$, it must be true that $\vtheta' \in Q_{\kappa}^t$. From Definition~\ref{defn:qset_DON}, 
% the definition of $Q^t_{\kappa}$, 
$\vtheta'$ needs to satisfy three conditions: 
%which we refer to as the cosine similarity condition, average condition, and output condition, for convenience:
\begin{align*}
|\cos(\vtheta' - \vtheta_t, \nabla_{\vtheta} \bar{G}_{\vtheta_t})| & \geq \kappa \quad \text{(cosine similarity condition)}~,\\
%
(\vtheta'_f-\vtheta_{f,t})^\top\left(\frac{1}{n} \sum_{i=1}^n \frac{1}{q_i} \sum_{j=1}^{q_i} \ell'_{i,j} \sum_{k=1}^K \nabla_{\vtheta_{f}} f_k^{(i)} \nabla_{\vtheta_{g}} g_{k,j}^{(i)~\top}\right)(\vtheta'_g-\vtheta_{g,t})  &\geq 0\quad \text{(average condition)}~,\\
%
(\vtheta'_f-\vtheta_{f,t})^\top\left( \sum_{k=1}^K \nabla_{\vtheta_{f}} f_k^{(i)} \nabla_{\vtheta_{g}} g_{k,j}^{(i)~\top}\right)(\vtheta'_g-\vtheta_{g,t})&\leq 0,\forall i\in[n],\forall j\in[q_i]
\quad \text{(output condition)}.
\end{align*}
%We also notice that the output condition is in reality a system of $q=\sum^n_{i=1}q_i$ quadratic inequalities, each one with $p_g+p_f$ variables, since $\vtheta'\in\R^{p_f+p_g}$ is the variable of interest.
%
%Now, we prove that the $Q^t_{\kappa}$ set is non-empty. 
Let us consider 
%$\vtheta'=[{\vtheta'_f}^{\top}\;{\vtheta'_g}^{\top}]^{\top}=[{\bar{\vtheta}_f}^{\top}\;\vzero_{p_g}^\top]^{\top} + \vtheta_t$ 
%
$\vtheta'=[{\vtheta'_f}^{\top}\;{\vtheta'_g}^{\top}]^{\top}$,
%
%(where $\vzero_{q}$ is the zero vector of dimension $q$), 
where 
$\vtheta'_f\in\R^{p_f}$ will be specified later and $\vtheta'_g = \vtheta_{g,t}$. 
%$\bar{\vtheta}_f\in\R^{p_f}$. 
%Let $\bar{\g} = [\bar{\g}_f^{\top}\;\bar{\g}_g^{\top}]^{\top} :=  \frac{\nabla_{\vtheta} \bar{G}_{\vtheta_t}}{\| \nabla_{\vtheta} \bar{G}_{\vtheta_t}\|_2}\in\R^{p_f+p_g}$.
Then, belonging to the $Q_\kappa^t$ set conveniently reduces to the feasibility of the cosine similarity condition as follows: 
\begin{align}
\label{eq:cs-cond}
  %(\langle \vtheta'_f , \bar{\g}_f \rangle)^2  & \geq \kappa^2 \quad \text{(cosine similarity condition)}.
  |\cos(\vtheta'_f-\vtheta_{f,t}\,, \bar{\g}_f \rangle)| \geq \kappa~,
\end{align}
where $\bar{\g}_f$ is the first $p_f$ components of the gradient $\nabla_{\vtheta}\bar{G}_{\vtheta_t}$ (recall that the cosine computation is invariant to the vector norms).
%
%It is without loss of generalization that we have the condition above in terms of $\bar{\g}_f$ instead of the first $p_f$ components of the gradient $\nabla_{\vtheta}\bar{G}_{\vtheta_t}$, since the cosine computation is invariant to the vector norms.
%
%
%$\bar{\g}=[\bar{\g}_f^{\top}\;\bar{\g}_g^{\top}]^{\top} \in \R^{p_f+p_g}$ and $\vtheta' =[\vtheta'_f;\vtheta'_g] \in \R^{p_f+p_g}$ are unit vectors, and
%
%We will choose a suitable $\kappa \in (0,\frac{1}{2}]$. 
%Now, to construct $\vtheta'_f$ satisfying the above condition, without loss of generality, let us assume $\| \bar{\g}_f\|_2 \geq \| \bar{\g}_g \|_2$ (the case $\| \bar{\g}_f\|_2 \leq \| \bar{\g}_g \|_2$ is treated similarly) so that $\| \bar{\g}_f\|_2 \geq \frac{1}{\sqrt{2}}$. Then, we simply set $\vtheta'_f = \frac{\bar{\g}_f}{\| \bar{\g}_f \|_2}$ so that 
%\begin{align*}
%\langle \vtheta'_f, \bar{\g}_f \rangle^2 = \left( \frac{\bar{\g}_f}{\| \bar{\g}_f \|_2} \bar{\g}_f  \right)^2 = \| \bar{\g}_f \|_2^2 \geq \frac{1}{2}~,
%\end{align*}
%and so the cosine similarity condition is satisfied given that 
% so that the feasibility condition is satisfied for 
%$\kappa \in (0,\frac{1}{\sqrt{2}}]$ (see Definition~\eqref{defn:qset}). This %completes the proof. 
%
%======**************************
%

With all of this in mind, we proceed to show the existence of an element $\vtheta' \in B^t_{\kappa}$ of the form $\vtheta'=[{\vtheta'_f}^{\top}\;{\vtheta_{g,t}}^{\top}]^{\top}$ 
%$\vtheta'=[\bar{\vtheta}_f^{\top}\;\vzero_{p_g}^\top]^{\top}+\vtheta_t$ 
%
%
satisfying condition~\eqref{eq:cs-cond} and the following two conditions:
%, using the notation $\vtheta_{t}={[{\vtheta'_f}^{\top}\;\vzero_{p_g}^\top]}^{\top}$
%:
%: %such that 
\begin{enumerate}[{Condition} (A):]%[(a)]
\item  $\|\vtheta'_f - \vtheta_{f,t} \|_2 = \epsilon$ for some $\epsilon< \frac{2 \norm{\nabla_{\vtheta_f} \cL(\vtheta_t)}_2 \sqrt{1-\kappa^2}}{\beta}$; and \label{cond-1-don}
\item the angle $\nu'$ between $(\vtheta'_f - \vtheta_{f,t})$ and $-\nabla_{\vtheta_f} \cL(\vtheta_t)$ is acute, so that $\cos(\nu') > 0$. \label{cond-2-don}\end{enumerate}
%

%Now, let us take the following three points: $\vtheta_{t+1}$, $\bar{\ggamma}_t + \vtheta_t$, and $\vtheta_t$.
%\begin{itemize}
%    \item If the points are not collinear, then they define a plane and we take $\vtheta'$ as a point in this plane with the conditions above.
%    \item If the points are collinear, then we take some $\vtheta'$ according to the conditions above and
%    %defining the acute angle with respect to $\vtheta_{t+1}$ (which is $\nu'$). Now 
%    define a new plane with this third point $\vtheta'$. %, and then the proof follows.
%\end{itemize}
%Notice that $\vtheta_t$ is taken as the origin of the plane being formed (by the definition of gradient descent, $\theta_{t+1}=-\nabla_{\vtheta}\cL(\vtheta_t)$.
%===
%If $\cos(\nu')\geq \sqrt{1-\kappa^2}$, we then have that Conditions~\eqref{cond-1-don} and~\eqref{cond-2-don} altogether satisfy the sought Condition~\ref{cond:rsc}(b) and finishes the proof of the first part of the theorem's statement. 

To show the existence of such element $\vtheta' \in B_t$, we propose two possible constructions:
%whose conditions of existence we will be shown afterwards:
%make \pcedit{either of} the \pcedit{specific choices}:% choice specific: 
\begin{enumerate}[{Choice} (A):]
\item  
If the points $\vtheta_{f,t+1}$, $\bar{\g}_f + \vtheta_{f,t}$, and $\vtheta_{f,t}$ are not collinear, then they define a hyperplane $\mathcal{P}$ that contains the vectors $\bar{\g}_f$ and $-\nabla_{\vtheta_f} \cL(\vtheta_t)$ (recall that $\vtheta_{f,t+1} - \vtheta_{f,t}=-\nabla_{\vtheta_f}\cL(\vtheta_t)$ by gradient descent). We choose $\vtheta'_f$ such that the vector 
$\vtheta'_f-\vtheta_{f,t}$ 
%(or, equivalently, the vector $\bar{\vtheta}_f$)
lies in 
$\mathcal{P}$ 
%in the hyperplane $\mathcal{P}$ containing $\bar{g}_t$ and $-\nabla \cL(\vtheta_t)$ in 
%
%(between the vectors 
%$\bar{\ggamma}_t$ and $-\nabla \cL(\vtheta_t)$) 
%
%such that
with 
$\cos(\vtheta'_f-\vtheta_{f,t},\bar{\g}_f)=\kappa$ (i.e., it satisfies condition~\eqref{eq:cs-cond} with equality) while simultaneously satisfying Condition~\eqref{cond-2-don}.
%
%
If the points $\vtheta_{f,t+1}$, $\bar{\g}_f + \vtheta_{f,t}$, and $\vtheta_{f,t}$ are collinear, we choose $\vtheta'_f$ such that it is not collinear with these points, thus defining a hyperplane $\mathcal{P}$ %(which contains the vectors $(\vtheta'-\vtheta_t)$, $\bar{\ggamma}_t$ and $-\nabla_{\vtheta}\cL(\vtheta_t)$ sharing the same origin $\vtheta_t$). Such %
with these other three points, 
and such that 
$\vtheta'_f$ is also taken so that $\cos(\vtheta'_f-\vtheta_{f,t},\bar{\g}_f)=\kappa$ while simultaneously satisfying Condition~\eqref{cond-2-don}.
%

Thus far we have only defined \emph{angle} (or \emph{direction}) conditions on the vector $\vtheta'_f-\vtheta_{f,t}$, and so there could be an infinite number of values for $\vtheta'_f$ satisfying such angle conditions without $\vtheta'$ belonging to the set $B^{\mathrm{Euc}}_{\rho,\rho_1}(\vtheta_0)$ nor $\vtheta'_f$ satisfying Condition~\eqref{cond-1-don}. To determine the feasible values for $\vtheta'_f$, we observe that $\vtheta_t$ is \emph{strictly inside} the set $B^{\mathrm{Euc}}_{\rho,\rho_1}(\vtheta_0)$ by Assumption~\ref{asmp:iter-1}, and so 
%
%and so now we set the magnitude of $\vtheta'_f$ such that $\vtheta'\in  B^{\mathrm{Euc}}_{\rho,\rho_1}(\vtheta_0)$, which is possible since  and 
%
$\vtheta'_f$ can be taken arbitrarily close to $\vtheta_{f,t}$ so that $\vtheta'\in B^{\mathrm{Euc}}_{\rho,\rho_1}(\vtheta_0)$ and Condition~\eqref{cond-1-don} is satisfied. 
%

We remark that, regardless of the collinearity of the points 
$\vtheta_{f,t+1}$, $\bar{\g}_f + \vtheta_{f,t}$, and $\vtheta_{f,t}$, hyperplane $\mathcal{P}$ contains the vectors $\vtheta'_f-\vtheta_{f,t}$, $\bar{\g}_f$, and $-\nabla_{\vtheta_f}\cL(\vtheta_t)$, all sharing its origin at $\vtheta_{f,t}\in\mathcal{P}$. \label{ch-A-don}
%
%; or
\item  
%
%$(\vtheta'-\vtheta_t)$
%lies in the hyperplane containing $-\bar{g}_t$ and $-\nabla_{\vtheta} \cL(\vtheta_t)$ in between these vectors 
%such that $\cos(\vtheta'-\vtheta_t,-\bar{\ggamma}_t)=\kappa$, i.e., the boundary of the $Q^t_{\kappa}$ set.
%
We choose $\vtheta'$ as in Choice~\eqref{ch-A-don} but with $\bar{\g}_f$ replaced by $-\bar{\g}_f$.
%
%$(\vtheta'-\vtheta_t)$
%lies in the hyperplane containing $-\bar{g}_t$ and $-\nabla_{\vtheta} \cL(\vtheta_t)$ in between these vectors 
%such that $\cos(\vtheta'-\vtheta_t,-\bar{\ggamma}_t)=\kappa$, i.e., the boundary of the $Q^t_{\kappa}$ set.
%We choose 
\label{ch-B-don}
%
%
\end{enumerate}
We immediately notice that $\vtheta'$ defined by either Choice~\eqref{ch-A-don} or Choice~\eqref{ch-B-don} satisfies 
$\vtheta'\in Q^t_\kappa \cap B^{\mathrm{Euc}}_{\rho,\rho_1}(\vtheta_0)$. To make $\vtheta'$ belong to the set $B^t_\kappa$, we need to find a radius $\rho_2$ such that $\vtheta'\in B^{\mathrm{Euc}}_{\rho_2}(\vtheta_t)$, or, equivalently, such that $\vtheta'_f\in B^{\mathrm{Euc}}_{\rho_2}(\vtheta_{f,t})$ due to our construction of $\vtheta'$. Such $\rho_2$ is found by taking $\rho_2>\epsilon$ with $\epsilon$ as in Condition~\eqref{cond-1-don}. 
%
Finally, it is straightforward to verify that such $\vtheta'\in B^t_\kappa$ defined by either Choice~\eqref{ch-A-don} or Choice~\eqref{ch-B-don} will always exist, by considering the following cases for the angle $\nu$ between $\bar{\g}_f$ and $-\nabla_{\vtheta_f} \cL(\vtheta_t)$:
\begin{enumerate}[(i)]
\item If $\nu \in [0, \pi/2]$ or $\nu \in [3\pi/2, 2\pi]$, then Choice~\eqref{ch-A-don} will be true, since $-\nabla_{\vtheta_f} \cL(\vtheta_t)$ is in the positive half space\footnote{We say $\a$ is in the positive half-space of $\b$ if $\langle \a, \b \rangle \geq 0$.} of $\bar{\g}_f$; and
\label{it-i-don}
\item if $\nu \in [\pi/2,\pi]$ or $\nu \in [\pi, 3\pi/2]$, then Choice~\eqref{ch-B-don} will be true, since $-\nabla_{\vtheta_f} \cL(\vtheta_t)$ is in the positive half space of $-\bar{\g}_f$.\label{it-ii-don}
\end{enumerate}

%To finish the proof, we only need to prove $\cos(\nu')\geq \sqrt{1-\kappa^2}$. 
%
Now, let us assume we are in the case of item~\eqref{it-i-don} above, so that $\vtheta'$ is constructed according to Choice~\eqref{ch-A-don} (the rest of the proof can be adapted to the case of item~\eqref{it-ii-don} by using a symmetrical argument and so it is omitted). %Assume the angle conditions in item~\eqref{it-i-don}. 
% Now, we show that
% \begin{enumerate}[(a)]
% \item $|\cos(\vtheta'-\vtheta_t,\nabla \cL(\vtheta_t)| \geq \sqrt{1-\kappa^2}$, and 
% \item .
% \end{enumerate}
% To get (a)
Let $\nu_1$ be the angle between $\vtheta'_f-\vtheta_{f,t}$ and $\bar{\g}_f$, so that $\cos(\nu_1)=\kappa$ according to Choice~\eqref{ch-A-don}.
%
%unknown, $\nu_2 = \nu'$, and $\nu_3 = \nu$ respectively denoting the angles between , $((\vtheta'-\vtheta_t), - \nabla \cL(\vtheta_t))$, and $(\bar{\ggamma}_t, - \nabla \cL(\vtheta_t))$, we have $\cos(\nu_1)=\kappa$.
Then, we have that 
\begin{align*}
|\cos(\nu')| = |\cos(\nu - \nu_1)| \geq |\cos(\pi/2 - \nu_1)| = |\sin(\nu_1)| = \sqrt{1-\cos^2(\nu_1)} = \sqrt{1-\kappa^2}~.
\end{align*}
Further, by the construction in Condition~\eqref{cond-2-don}, $\cos(\nu') > 0$, which implies 
%\begin{equation}
$\cos(\nu') \geq  \sqrt{1-\kappa^2}>0$.
%\label{eq:cosnu}
%\end{equation}
%$$. 
%
%
%Now, we take $\rho_2$ sufficiently large so that $\rho_2>\epsilon$, which immediately implies $\vtheta'\in B^{\mathrm{Euc}}_{\rho_2}(\vtheta_0)$ and which, along with the construction in Choice~\ref{ch-A-don}, implies that $\vtheta'\in B^t_\kappa$. Now, equation~\eqref{eq:cosnu}
%All the above derivations imply that and so $\vtheta'\in B^{\mathrm{Euc}}_{\rho_2}(\vtheta_0)$
%
Now, 
%Let $\nu$ be the angle between $\vtheta'-\vtheta_t$ and $-\nabla_\vtheta\cL(\vtheta_t)$, so that $\cos(\nu)>0$ by assumption. Now, for such $\vtheta' \in \mathcal{N}_t$ and with  $\epsilon:=\norm{\vtheta'-\vtheta_t}_2$, 
by the smoothness property of the empirical loss $\cL$ we have
\begin{align*}
\cL(\vtheta') & \leq \cL(\vtheta_t) - \langle \vtheta' - \vtheta_t, -\nabla_\vtheta \cL(\vtheta_t) \rangle + \frac{\beta}{2}\| \vtheta' - \vtheta_t \|_2^2 \\ 
& = \cL(\vtheta_t) - \|\vtheta'_f - \vtheta_{f,t}\|_2 \|\nabla_{\vtheta_f} \cL(\vtheta_t) \|_2 \cos(\nu) + \frac{\beta}{2}\| \vtheta'_f - \vtheta_{f,t} \|_2^2 \\
& =  \cL(\vtheta_t) -  \epsilon \|\nabla_{\vtheta_f} \cL(\vtheta_t) \|_2 \cos(\nu) + \frac{\beta}{2} \epsilon^2 \\
& = \cL(\vtheta_t) -  \frac{\beta \epsilon}{2} \left( \frac{2 \|\nabla_{\vtheta_f} \cL(\vtheta_t) \|_2 \cos(\nu)}{\beta} - \epsilon \right)\\
&<\cL(\vtheta_t)~.
\end{align*}
where the last inequality follows by the construction of $\epsilon$ in Condition~\eqref{ch-A-don}. Note that this implies that the constructed $\vtheta'$ is as described in Condition~\ref{cond:rsc}(b.2). This finishes the proof for Condition~\ref{cond:rsc}(b).

%We choose 
%%To get (b), \abedit{we know that $\theta_2$ is an acute angle and $\cos(-\theta_2) = -\cos(\theta_2)$.
%%
%$\epsilon$ such that $\epsilon < \frac{2 \norm{\nabla_\vtheta \cL(\vtheta_t)}_2 \sqrt{1-\kappa^2}}{\beta}$ and $\epsilon > 0$. With this choice of $\epsilon$, we take $\rho_2>\epsilon$ so that $\vtheta'\in B^{\mathrm{Euc}}_{\rho_2}(\vtheta_0)$.
%%we take  $\rho_2>\epsilon$. 
%%This implies that $\vtheta'\in B^t_\kappa$ and so, keeping in mind the choice of $\epsilon$ we made, it immediately follows that $\vtheta'\in B^t_{\kappa}$ and that $B^t_{\kappa}$ 
%%is a non-empty set that 
%satisfies Condition~\ref{cond:rsc}(a) and (b). This finishes the first part of the proof.
%===


%=========================================================================
%=========================================================================
%=========================================================================
%=========================================================================



We now proceed to prove the second part of the proof. 
For any $\vtheta' \in B^t_{\kappa}$, by the second order Taylor expansion of the DON loss with respect to iterate $\vtheta_t$, we have
\[
\cL(\vtheta') = \cL(\vtheta_t) + \langle \vtheta' - \vtheta_t, \nabla_\vtheta\cL(\vtheta_t) \rangle + \frac{1}{2} (\vtheta'-\vtheta_t)^\top \frac{\partial^2 \cL(\tilde{\vtheta})}{\partial \vtheta^2} (\vtheta'-\vtheta_t)~,
\]
where $\tilde{\vtheta} = \xi \vtheta' + (1-\xi) \vtheta_t$ for some $\xi \in [0,1]$. To establish $\alpha_t$-RSC of the loss with $\alpha_t$ as in \eqref{eq:RSCLoss}, it suffices to focus on the quadratic form of the Hessian and show 
\begin{equation}
(\vtheta' - \vtheta_t)^{\top} \frac{\partial^2 \cL(\tilde{\vtheta})}{\partial \vtheta^2}  (\vtheta - \vtheta_t) \geq \alpha_t \| \vtheta' - \vtheta_t \|_2^2~.
\end{equation}
Note that the Hessian, by chain rule, is given by
\begin{align*}
\mH (\tilde{\vtheta}) & := \frac{\partial^2 \cL(\tilde{\vtheta})}{\partial \vtheta^2}  
=  \frac{1}{n} \sum_{i=1}^n \frac{1}{q_i} \sum_{j=1}^{q_i}  \left( \ell^{\prime\prime}_{i,j} \nabla_{\vtheta} G_{\tilde{\vtheta}}(u^{(i)})(\vy^{(i)}_j) \nabla_{\vtheta} G_{\tilde{\vtheta}}(u^{(i)})(\vy^{(i)}_j)^\top  
+ \ell^{\prime}_{i,j}   \nabla^2 G_{\tilde{\vtheta}}(u^{(i)})(\vy^{(i)}_j) \right)~. 
\end{align*}
%from the Taylor expansion in \eqref{eq:local_taylor_expansion_loss}, 
where 
$\ell_{i,j}=(G_{\tilde{\vtheta}}(u^{(i)})(\vy_{j}^{(i)})-G^\dagger(u^{(i)})(\vy^{(i)}_j))^2$. 
Given the $2 \times 2$ block structure of the Hessian as in \eqref{eq:gradHessDeepONetLoss}, denoting $\delta \vtheta := \vtheta' - \vtheta_t$ for compactness, the quadratic form on the Hessian is given by
\begin{equation}
\delta \vtheta^{\top} \mH (\tilde{\vtheta}) \delta \vtheta 
= \underbrace{\delta \vtheta_{f}^{\top} H_{ff}(\tilde{\vtheta}) \delta \vtheta_{f}}_{T_1}
        + \underbrace{2 \delta \vtheta_{f}^{\top} H_{fg}(\tilde{\vtheta}) \delta \vtheta_{g}}_{T_2}
        + \underbrace{\delta \vtheta_{g}^{\top} H_{gg}(\tilde{\vtheta}) \delta \vtheta_{g}}_{T_3}~.
\end{equation}
Focusing on $T_1$ and using the exact form of $H_{ff}(\tilde{\vtheta})$ as in~\eqref{eq:Hessian_blocks_fully_Expanded}, we have
\begin{align*}
    T_1 & = \frac{1}{n} \sum_{i=1}^n \frac{1}{q_i} \sum_{j=1}^{q_i} \ell^{\prime\prime}_{i,j} \left\langle \delta \vtheta_f , \sum_{k=1}^K g_{k,j}^{(i)}(\tilde{\vtheta}_g) \nabla_{\vtheta_f} f_k^{(i)}(\tilde{\vtheta}_f) \right\rangle^2 
    + \frac{1}{n} \sum_{i=1}^n \frac{1}{q_i} \sum_{j=1}^{q_i} \ell'_{ij} \sum_{k=1}^K g^{(i)}_{k,j}(\tilde{\vtheta}_g) \delta \vtheta_f^{\top} \nabla_{\vtheta_f}^2 f_k^{(i)}(\tilde{\vtheta}_f) \delta \vtheta_f \nonumber \\
    & \overset{(a)}{\geq} \frac{2}{n} \sum_{i=1}^n \frac{1}{q_i} \sum_{j=1}^{q_i}  \left\langle \delta \vtheta_f , \nabla_{\vtheta_f} G_{\tilde{\vtheta}}(u^{(i)})(\vy^{(i)}_j)  \right\rangle^2 - \frac{(2K\lambda_1\lambda_2+\tilde{c})\lambda_2 c^{(f)}}{\sqrt{m_f}} \| \delta \vtheta_f \|_2^2~,
%    \label{eq:hesst1}
\end{align*}
where (a) follows from $\ell''_{ij}=2$ and the different bounds in Lemma~\ref{lemm:hessgradbounds} since $\tilde{\vtheta}\in B^{\mathrm{Euc}}_{\rho,\rho_1}(\vtheta_0)$, so that $|\ell'_{ij}|\leq 2K\lambda_1\lambda_2+\tilde{c}$  with $\tilde{c}=\max_{i\in[n],j\in[q_i]}|G^\dagger(u^{(i)})(\vy^{(i)}_j)|$.
Similarly, for $T_3$ we get
\begin{align*}
    T_3 \geq \frac{2}{n} \sum_{i=1}^n \frac{1}{q_i} \sum_{j=1}^{q_i}  \left\langle \delta \vtheta_g , \nabla_{\vtheta_g} G_{\tilde{\vtheta}}(u^{(i)})(\vy^{(i)}_j)  \right\rangle^2 - \frac{(2K\lambda_1\lambda_2+\tilde{c})\lambda_1 c^{(g)}}{\sqrt{m_g}} \| \delta \vtheta_g \|_2^2~.
%    \label{eq:hesst2}
\end{align*}
Then, 
\begin{align*}
T_1+T_3 &\overset{(a)}{\geq} \frac{2}{n} \sum_{i=1}^n \frac{1}{q_i} \sum_{j=1}^{q_i}  \left(\left\langle \delta \vtheta_g ,  \nabla_{\vtheta_g} G_{\tilde{\vtheta}}(u^{(i)})(\vy^{(i)}_j) \right\rangle^2
+
\left\langle \delta \vtheta_f ,  \nabla_{\vtheta_f} G_{\tilde{\vtheta}}(u^{(i)})(\vy^{(i)}_j) \right\rangle^2
\right)\\
&\quad- 
    (2K\lambda_1\lambda_2+\tilde{c})
    \left(\frac{\lambda_1 c^{(g)}}{\sqrt{m_g}} + \frac{\lambda_2 c^{(f)}}{\sqrt{m_f}}\right) \| \delta \vtheta \|_2^2\\
%
&= \frac{2}{n} \sum_{i=1}^n \frac{1}{q_i} \sum_{j=1}^{q_i}  \left(\left\langle \delta \vtheta_g ,  \nabla_{\vtheta_g} G_{\tilde{\vtheta}}(u^{(i)})(\vy^{(i)}_j) \right\rangle
+
\left\langle \delta \vtheta_f ,  \nabla_{\vtheta_f} G_{\tilde{\vtheta}}(u^{(i)})(\vy^{(i)}_j) \right\rangle
\right)^2\\
&\quad -\frac{4}{n} \sum_{i=1}^n \frac{1}{q_i} \sum_{j=1}^{q_i}  \left\langle \delta \vtheta_g ,  \nabla_{\vtheta_g} G_{\tilde{\vtheta}}(u^{(i)})(\vy^{(i)}_j) \right\rangle
\left\langle \delta \vtheta_f ,  \nabla_{\vtheta_f} G_{\tilde{\vtheta}}(u^{(i)})(\vy^{(i)}_j) \right\rangle\\
&\quad- 
    (2K\lambda_1\lambda_2+\tilde{c})
    \left(\frac{\lambda_1 c^{(g)}}{\sqrt{m_g}} + \frac{\lambda_2 c^{(f)}}{\sqrt{m_f}}\right) \| \delta \vtheta \|_2^2~,
\end{align*}
where (a) follows from $\norm{\vtheta_f}_2,\norm{\vtheta_g}_2\leq\norm{\vtheta}_2$.

Focusing on $T_2$ and using the exact form as in~\eqref{eq:Hessian_blocks_fully_Expanded}, we have
\begin{align*}
    T_2 & = 2\delta \vtheta_f^\top \left(\frac{1}{n} \sum_{i=1}^n \frac{1}{q_i} \sum_{j=1}^{q_i} \ell'_{ij} \sum_{k=1}^K \nabla_{\vtheta_f} f_k^{(i)}(\tilde{\vtheta}_f) \nabla_{\vtheta_g} g_{k,j}^{(i)}(\tilde{\vtheta}_g)^\top \right) \delta \vtheta_g \nonumber \\
    & \qquad \qquad + 2\delta \vtheta_f^\top \left( \frac{1}{n} \sum_{i=1}^n \frac{1}{q_i} \sum_{j=1}^{q_i}\ell^{\prime\prime}_{i,j} \left(\sum_{k=1}^K g_{k,j}^{(i)} \nabla_{\vtheta_f} f_k^{(i)}(\tilde{\vtheta}_f) \right) \left( \sum_{k'=1}^K f_{k'}^{(i)} \nabla_{\vtheta_g} g_{k',j}^{(i)}(\tilde{\vtheta}_g)^\top \right) \right) \delta \vtheta_g\\
    & \overset{(a)}{=} \underbrace{2\delta \vtheta_f^\top \left(\frac{1}{n} \sum_{i=1}^n \frac{1}{q_i} \sum_{j=1}^{q_i} \ell'_{ij} \sum_{k=1}^K \nabla_{\vtheta_f} f_k^{(i)}(\tilde{\vtheta}_f) \nabla_{\vtheta_g} g_{k,j}^{(i)}(\tilde{\vtheta}_g)^\top \right) \delta \vtheta_g}_{I_1} \nonumber \\
    & \qquad \qquad + \left( \frac{4}{n} \sum_{i=1}^n \frac{1}{q_i} 
    \sum_{j=1}^{q_i} 
    \left\langle \delta \vtheta_g ,  \nabla_{\vtheta_g} G_{\vtheta_t}(u^{(i)})(\vy^{(i)}_j) \right\rangle
\left\langle \delta \vtheta_f ,  \nabla_{\vtheta_f} G_{\vtheta_t}(u^{(i)})(\vy^{(i)}_j)\right\rangle\right)~,
\end{align*}
where (a) follows from $\ell^{\prime\prime}_{i,j}=2$.

For $I_1$ our goal is to first transfer the dependence of the gradient terms on $\tilde{\vtheta}$ to $\vtheta_t$, so that we can use properties of the restricted set $Q^t_{\kappa}$ which is based on $\vtheta_t$ to simplify the analysis. Towards that end, note that 
\begin{align*}
    \frac{1}{2}I_1 & = \delta \vtheta_f^\top \left(\frac{1}{n} \sum_{i=1}^n \frac{1}{q_i} \sum_{j=1}^{q_i} \ell'_{ij} \sum_{k=1}^K \nabla_{\vtheta_f} f_k^{(i)}(\vtheta_{t,f}) \nabla_{\vtheta_g} g_{k,j}^{(i)}(\vtheta_{t,g})^\top \right) \delta \vtheta_g \\
    & \quad + \delta \vtheta_f^\top \left(\frac{1}{n} \sum_{i=1}^n \frac{1}{q_i} \sum_{j=1}^{q_i} \ell'_{ij} \sum_{k=1}^K \left( \nabla_{\vtheta_f} f_k^{(i)}(\tilde{\vtheta}_f) - \nabla_{\vtheta_f} f_k^{(i)}(\vtheta_{t,f}) \right) \nabla_{\vtheta_g} g_{k,j}^{(i)}(\tilde{\vtheta}_g)^\top \right) \delta \vtheta_g \\
    & \quad + \delta \vtheta_f^\top \left(\frac{1}{n} \sum_{i=1}^n \frac{1}{q_i} \sum_{j=1}^{q_i} \ell'_{ij} \sum_{k=1}^K \nabla_{\vtheta_f} f_k^{(i)}(\vtheta_{t,f}) \left( \nabla_{\vtheta_g} g_{k,j}^{(i)}(\tilde{\vtheta}_g) - \nabla_{\vtheta_g} g_{k,j}^{(i)}(\vtheta_{t,g}) \right)^\top \right) \delta \vtheta_g \\
    %& \quad + \delta \vtheta_f^\top \left(\frac{1}{n} \sum_{i=1}^n \frac{1}{q_i} \sum_{j=1}^{q_i} \ell'_{ij} \sum_{k=1}^K \nabla_{\vtheta_f} f_k^{(i)}(\tilde{\vtheta}_f) \nabla_{\vtheta_g} g_{k,j}^{(i)}(\tilde{\vtheta}_g)^\top \right) \delta \vtheta_g \\
    %& \overset{(a)}{\geq} \delta \vtheta_f^\top \left( \sum_{h=1}^{\tilde{q}} \sigma_{h,t} \va_{h,t} \vb_{h,t}^\top \right) \delta \vtheta_g \\
    %& \quad -  \frac{(2K\lambda_1\lambda_2+\tilde{c})}{n} \sum_{i=1}^n \frac{1}{q_i} \sum_{j=1}^{q_i}  \left\| \nabla_{\vtheta_f} f_k^{(i)}(\tilde{\vtheta}_f) - \nabla_{\vtheta_f} f_k^{(i)}(\vtheta_{t,f}) \right\|_2 \left\| \nabla_{\vtheta_g} g_{k,j}^{(i)}(\tilde{\vtheta}_g) \right\|_2 \| \delta \vtheta_f \|_2 \| \delta \vtheta_g \|_2  \\
    %& \quad - \frac{(2K\lambda_1\lambda_2+\tilde{c})}{n} \sum_{i=1}^n \frac{1}{q_i} \sum_{j=1}^{q_i} \sum_{k=1}^K \left\| \nabla_{\vtheta_f} f_k^{(i)}(\vtheta_{t,f}) \right\|_2 \left\| \nabla_{\vtheta_g} g_{k,j}^{(i)}(\tilde{\vtheta}_g) - \nabla_{\vtheta_g} g_{k,j}^{(i)}(\vtheta_{t,g})  \right\|_2 \delta \vtheta_f^\top \|_2 \| \delta \vtheta_g \|_2 \\
    %
    & \overset{(a)}{\geq} -  \frac{(2K\lambda_1\lambda_2+\tilde{c})}{n} \sum_{i=1}^n \frac{1}{q_i} \sum_{j=1}^{q_i}  \left\| \nabla_{\vtheta_f} f_k^{(i)}(\tilde{\vtheta}_f) - \nabla_{\vtheta_f} f_k^{(i)}(\vtheta_{t,f}) \right\|_2 \left\| \nabla_{\vtheta_g} g_{k,j}^{(i)}(\tilde{\vtheta}_g) \right\|_2 \| \delta \vtheta_f \|_2 \| \delta \vtheta_g \|_2  \\
    & \quad - \frac{(2K\lambda_1\lambda_2+\tilde{c})}{n} \sum_{i=1}^n \frac{1}{q_i} \sum_{j=1}^{q_i} \sum_{k=1}^K \left\| \nabla_{\vtheta_f} f_k^{(i)}(\vtheta_{t,f}) \right\|_2 \left\| \nabla_{\vtheta_g} g_{k,j}^{(i)}(\tilde{\vtheta}_g) - \nabla_{\vtheta_g} g_{k,j}^{(i)}(\vtheta_{t,g})  \right\|_2 \| \delta \vtheta_f \|_2 \| \delta \vtheta_g \|_2 \\
    %
    & \overset{(b)}{=} -  \frac{(2K\lambda_1\lambda_2+\tilde{c})}{n} \sum_{i=1}^n \frac{1}{q_i} \sum_{j=1}^{q_i}  \left\| \nabla^2_{\vtheta_f} f_k^{(i)}(\bar{\vtheta}_f)\right\|_2 \norm{\tilde{\vtheta}_f-\vtheta_{t,f}}_2 \left\| \nabla_{\vtheta_g} g_{k,j}^{(i)}(\tilde{\vtheta}_g) \right\|_2 \| \delta \vtheta_f \|_2 \| \delta \vtheta_g \|_2 \\
    & \quad - \frac{(2K\lambda_1\lambda_2+\tilde{c})}{n} \sum_{i=1}^n \frac{1}{q_i} \sum_{j=1}^{q_i} \sum_{k=1}^K \left\| \nabla_{\vtheta_f} f_k^{(i)}(\vtheta_{t,f}) \right\|_2 \left\| \nabla^2_{\vtheta_g} g_{k,j}^{(i)}(\bar{\vtheta}_g) \right\|_2\norm{\tilde{\vtheta}_g-\vtheta_{t,g}}_2 \| \delta \vtheta_f \|_2 \| \delta \vtheta_g \|_2 \\
    %
    & \overset{(c)}{\geq} - (2K\lambda_1\lambda_2+\tilde{c})\left( \frac{c^{(f)} \varrho^{(g)}}{\sqrt{m_f}} \right) \| \delta \vtheta_f \|_2^2 \| \delta \vtheta_g \|_2 
    - (2K\lambda_1\lambda_2+\tilde{c})\left( \frac{c^{(g)} \varrho^{(f)}}{\sqrt{m_g}} \right) \| \delta \vtheta_f \|_2 \| \delta \vtheta_g \|_2^2
    \\
    & \overset{(d)}{\geq} - (2K\lambda_1\lambda_2+\tilde{c})\left( \frac{c^{(g)} \varrho^{(f)}}{\sqrt{m_f}} + \frac{c^{(g)} \varrho^{(f)}}{\sqrt{m_f}} \right) \| \delta \vtheta \|_2^3\\
    & \geq - (2K\lambda_1\lambda_2+\tilde{c})\rho_2 \left( \frac{c^{(g)} \varrho^{(f)}}{\sqrt{m_f}} + \frac{c^{(g)} \varrho^{(f)}}{\sqrt{m_f}} \right) \| \delta \vtheta \|_2^2 ~,
\end{align*}
where (a) follows from the definition of $Q_{\kappa}^t$ set %from the SVD in Definition~\ref{defn:qset}, (b) follows 
(Definition~\ref{defn:qset}) 
since $\vtheta' \in B^t_\kappa\subset Q^t_\kappa$; %and so $\vtheta' \in Q^t_\kappa$ 
%and $\delta \vtheta = \vtheta' - \vtheta_t$, we have $\sum_h \sigma_{h,t} \langle \delta \vtheta_f, \va_{h,t} \rangle \langle \delta \vtheta_g , \vb_{h,t} \rangle \geq 0$ by the definition of $Q^t_{\kappa}$, 
(b) follows from the generalized mean value theorem with $\bar{\vtheta_f}=\xi_1\tilde{\vtheta}_f+(1-\xi_1)\vtheta_{t,f}$ for some $\xi_1\in[0,1]$ and $\bar{\vtheta_g}=\xi_2\tilde{\vtheta}_g+(1-\xi_2)\vtheta_{t,g}$ for some $\xi_2\in[0,1]$; (c) follows from the results in Lemma~\ref{lemm:hessgradbounds} since $[\bar{\vtheta}_f^\top\; \bar{\vtheta}_g^\top]^\top \in B^{\mathrm{Euc}}_{\rho,\rho_1}(\vtheta_0)$, and the fact that $\| \tilde{\vtheta}_f - \vtheta_{f,t} \|_2 \leq \| \delta \vtheta_f \|_2$ and $\| \tilde{\vtheta}_g - \vtheta_{g,t} \|_2 \leq \| \delta \vtheta_g \|_2$; and (d) follows from $\norm{\delta\vtheta_f}_2,\norm{\delta\vtheta_g}_2\leq \norm{\delta\vtheta}_2$.

Replacing $I_1$ back in $T_2$ and then combining the bounds on $T_1+T_3$ and $T_2$, we have
\begin{equation}
\label{eq:Hess_tildetheta}
\begin{aligned}
    \delta \vtheta^{\top} \mH (\tilde{\vtheta}) \delta \vtheta  
&\geq \frac{2}{n} \sum_{i=1}^n \frac{1}{q_i} \sum_{j=1}^{q_i}  \left(\left\langle \delta \vtheta_g ,  \nabla_{\vtheta_g} G_{\tilde{\vtheta}}(u^{(i)})(\vy^{(i)}_j) \right\rangle
+
\left\langle \delta \vtheta_f ,  \nabla_{\vtheta_f} G_{\tilde{\vtheta}}(u^{(i)})(\vy^{(i)}_j) \right\rangle
\right)^2\\
&\quad- 
    (2K\lambda_1\lambda_2+\tilde{c})
    \left(\frac{\lambda_1 c^{(g)}}{\sqrt{m_g}} + \frac{\lambda_2 c^{(f)}}{\sqrt{m_f}}\right) \| \delta \vtheta \|_2^2\\
& \quad- 2(2K\lambda_1\lambda_2+\tilde{c})\rho_2 \left( \frac{c^{(g)} \varrho^{(f)}}{\sqrt{m_f}} + \frac{c^{(g)} \varrho^{(f)}}{\sqrt{m_f}} \right) \| \delta \vtheta \|_2^2\\
%
&=\frac{2}{n} \sum_{i=1}^n \frac{1}{q_i} \sum_{j=1}^{q_i}  \left\langle \delta \vtheta ,  \nabla_{\vtheta} G_{\tilde{\vtheta}}(u^{(i)})(\vy^{(i)}_j) \right\rangle
^2\\
&\quad- 
    (2K\lambda_1\lambda_2+\tilde{c})
    \left(\frac{\lambda_1 c^{(g)}}{\sqrt{m_g}} + \frac{\lambda_2 c^{(f)}}{\sqrt{m_f}}\right) \| \delta \vtheta \|_2^2\\
& \quad- 2(2K\lambda_1\lambda_2+\tilde{c})\rho_2 \left( \frac{c^{(g)} \varrho^{(f)}}{\sqrt{m_f}} + \frac{c^{(g)} \varrho^{(f)}}{\sqrt{m_f}} \right) \| \delta \vtheta \|_2^2\\
%
&=\underbrace{\frac{2}{n} \sum_{i=1}^n \frac{1}{q_i} \sum_{j=1}^{q_i}  \left(\left\langle \delta \vtheta ,  \nabla_{\vtheta} G_{\vtheta_t}(u^{(i)})(\vy^{(i)}_j) \right\rangle+\left(\left\langle \delta \vtheta ,  \nabla_{\vtheta} G_{\tilde{\vtheta}}(u^{(i)})(\vy^{(i)}_j) \right\rangle
-
\left\langle \delta \vtheta ,  \nabla_{\vtheta} G_{\vtheta_t}(u^{(i)})(\vy^{(i)}_j) \right\rangle
\right)\right)
^2}_{I_2}\\
&\quad- 
    (2K\lambda_1\lambda_2+\tilde{c})
    \left(\frac{\lambda_1 c^{(g)}}{\sqrt{m_g}} + \frac{\lambda_2 c^{(f)}}{\sqrt{m_f}}\right) \| \delta \vtheta \|_2^2\\
& \quad- 2(2K\lambda_1\lambda_2+\tilde{c})\rho_2 \left( \frac{c^{(g)} \varrho^{(f)}}{\sqrt{m_f}} + \frac{c^{(g)} \varrho^{(f)}}{\sqrt{m_f}} \right) \| \delta \vtheta \|_2^2
%
%
~.
\end{aligned}
\end{equation}
%
%
Then, 
\begin{equation}
\label{eq:I3-1}
\begin{aligned}
    I_2 & = \frac{2}{n} \sum_{i=1}^n \frac{1}{q_i} \sum_{j=1}^{q_i}  \left\langle \delta \vtheta , \nabla_{\vtheta} G_{\vtheta_t}(u^{(i)})(\vy^{(i)}_j) \right\rangle^2 + \sum_{i=1}^n \frac{2}{q_i} \sum_{j=1}^{q_i}  \left\langle \delta \vtheta , \nabla_{\vtheta} G_{\tilde{\vtheta}}(u^{(i)})(\vy^{(i)}_j) - \nabla_{\vtheta} G_{\vtheta_t}(u^{(i)})(\vy^{(i)}_j) \right\rangle^2 \\
    & \quad + \frac{4}{n} \sum_{i=1}^n \frac{1}{q_i} \sum_{j=1}^{q_i} \left\langle \delta \vtheta_f , \nabla_{\vtheta} G_{\vtheta_t}(u^{(i)})(\vy^{(i)}_j) \right\rangle \left\langle \delta \vtheta , \nabla_{\vtheta} G_{\tilde{\vtheta}}(u^{(i)})(\vy^{(i)}_j) - \nabla_{\vtheta} G_{\vtheta_t}(u^{(i)})(\vy^{(i)}_j) \right\rangle \\
    & \overset{(a)}{=}\frac{2}{n} \sum_{i=1}^n \frac{1}{q_i} \sum_{j=1}^{q_i}  \left\langle \delta \vtheta , \nabla_{\vtheta} G_{\vtheta_t}(u^{(i)})(\vy^{(i)}_j) \right\rangle^2 + \sum_{i=1}^n \frac{2}{q_i} \sum_{j=1}^{q_i}  \left\langle \delta \vtheta , \nabla_{\vtheta} G_{\tilde{\vtheta}}(u^{(i)})(\vy^{(i)}_j) - \nabla_{\vtheta} G_{\vtheta_t}(u^{(i)})(\vy^{(i)}_j) \right\rangle^2 \\
    & \quad + \frac{4}{n} \sum_{i=1}^n \frac{1}{q_i} \sum_{j=1}^{q_i} \left\langle \delta \vtheta , \nabla_{\vtheta} G_{\vtheta_t}(u^{(i)})(\vy^{(i)}_j) \right\rangle \left((\delta \vtheta)^\top \nabla_{\vtheta}^2 G_{\tilde{\tilde{\vtheta}}}(u^{(i)})(\vy^{(i)}_j)(\tilde{\vtheta}-\vtheta_t)\right)
    \\
    & \geq\frac{2}{n} \sum_{i=1}^n \frac{1}{q_i} \sum_{j=1}^{q_i}  \left\langle \delta \vtheta , \nabla_{\vtheta} G_{\vtheta_t}(u^{(i)})(\vy^{(i)}_j) \right\rangle^2  \\
    & \quad + \frac{4}{n} \sum_{i=1}^n \frac{1}{q_i} \sum_{j=1}^{q_i} \left\langle \delta \vtheta , \nabla_{\vtheta} G_{\vtheta_t}(u^{(i)})(\vy^{(i)}_j) \right\rangle \left((\delta \vtheta)^\top \nabla_{\vtheta}^2 G_{\tilde{\tilde{\vtheta}}}(u^{(i)})(\vy^{(i)}_j)(\tilde{\vtheta}-\vtheta_t)\right)
    \\
%     
    & \overset{(b)}{\geq}\frac{2}{n} \sum_{i=1}^n \frac{1}{q_i} \sum_{j=1}^{q_i}  \left\langle \delta \vtheta , \nabla_{\vtheta} G_{\vtheta_t}(u^{(i)})(\vy^{(i)}_j) \right\rangle^2  \\
    & \quad - 
4\xi_3K(\lambda_2\varrho^{(f)}+\lambda_1\varrho^{(g)})\left\|\delta\vtheta\right\|_2
    \frac{1}{n} \sum_{i=1}^n \frac{1}{q_i} \sum_{j=1}^{q_i}
    \left|(\delta \vtheta)^\top \nabla_{\vtheta}^2 G_{\tilde{\tilde{\vtheta}}}(u^{(i)})(\vy^{(i)}_j)(\delta\vtheta)\right|
%%    
\end{aligned}
\end{equation}
where (a) follows from the generalized mean value theorem and has $\tilde{\tilde{\vtheta}}\in\xi_3 \tilde{\vtheta}+(1-\xi_3)\vtheta_t$ for some $\xi_3\in[0,1]$; and (b) follows from the fact that $\tilde{\vtheta}-\vtheta_t=\xi_3(\theta'-\vtheta')$ and 
\begin{align*}
\left\| \nabla_{\vtheta} G_{\vtheta_t}(u^{(i)})(\vy^{(i)}_j) \right\|_2 &\leq 
\left\| \nabla_{\vtheta_f} G_{\vtheta_t}(u^{(i)})(\vy^{(i)}_j) \right\|_2 +
\left\| \nabla_{\vtheta_g} G_{\vtheta_t}(u^{(i)})(\vy^{(i)}_j) \right\|_2\\
&\leq 
\left\| \sum_{k=1}^K g^{(i)}_{k,j}(\vtheta_t)\nabla_{\vtheta_f}f^{(i)}_k(\vtheta_t)\right\|_2
+
\left\| \sum_{k=1}^K \nabla_{\vtheta_g}g^{(i)}_{k,j}(\vtheta_t)(f^{(i)}_k(\vtheta_t))\right\|_2\\
&\leq 
\sum_{k=1}^K|g^{(i)}_{k,j}(\vtheta_t)|\norm{\nabla_{\vtheta_f}f^{(i)}_k(\vtheta_t)}_2 + 
\sum_{k=1}^K|f^{(i)}_{k}(\vtheta_t)|\norm{\nabla_{\vtheta_g}g^{(i)}_{k,j}(\vtheta_t)}_2\\
&\leq K\lambda_2\varrho^{(f)}+K\lambda_1\varrho^{(g)}~, 
\end{align*}
where the last inequality follows from Lemma~\ref{lemm:hessgradbounds}.

Now, we have that
\begin{equation}
\label{eq:dd_G}
\begin{aligned}
    (\delta \vtheta)^\top \nabla_{\vtheta}^2 G_{\tilde{\tilde{\vtheta}}}(u^{(i)})(\vy^{(i)}_j)(\delta\vtheta) &= 
\sum^K_{k=1}(\delta\vtheta_f)^\top(g^{(i)}_{k,j}(\tilde{\tilde{\vtheta}}_g)\nabla^2_{\vtheta_f}f^{(i)}_{k}(\tilde{\tilde{\vtheta}}_f)(\delta\vtheta_f)
+ 
\sum^K_{k=1}(\delta\vtheta_g)^\top(f^{(i)}_{k}(\tilde{\tilde{\vtheta}}_f)\nabla^2_{\vtheta_g}g^{(i)}_{k,j}(\tilde{\tilde{\vtheta}}_g)(\delta\vtheta_g)\\
&\quad 
+2\sum^K_{k=1}(\delta\vtheta_g)^\top(\nabla_{\vtheta_g}g^{(i)}_{k,j}(\tilde{\tilde{\vtheta}}_g)(\nabla_{\vtheta_f}f^{(i)}_{k}(\tilde{\tilde{\vtheta}}_f))^\top(\delta\vtheta_f)\\
&\overset{(a)}{\leq} 
K\left(\frac{\lambda_2c^{(f)}}{\sqrt{m_f}}+ \frac{\lambda_1c^{(g)}}{\sqrt{m_g}}\right)\norm{\delta\vtheta}_2^2+\underbrace{2\sum^K_{k=1}(\delta\vtheta_g)^\top(\nabla_{\vtheta_g}g^{(i)}_{k,j}(\tilde{\tilde{\vtheta}}_g)(\nabla_{\vtheta_f}f^{(i)}_{k}(\tilde{\tilde{\vtheta}}_f))^\top(\delta\vtheta_f)}_{I_3}
\end{aligned}
\end{equation}
where (a) follows from Lemma~\ref{lemm:hessgradbounds} since $\tilde{\tilde{\vtheta}}\in B^{\mathrm{Euc}}_{\rho,\rho_1}(\vtheta_0)$.

Now, for $I_3$,
\begin{align*}
    \frac{1}{2}I_3 & = \delta \vtheta_f^\top \left( \sum_{k=1}^K \nabla_{\vtheta_f} f_k^{(i)}(\vtheta_{t,f}) \nabla_{\vtheta_g} g_{k,j}^{(i)}(\vtheta_{t,g})^\top \right) \delta \vtheta_g \\
    & \quad + \delta \vtheta_f^\top \left(\sum_{k=1}^K \left( \nabla_{\vtheta_f} f_k^{(i)}(\tilde{\tilde{\vtheta}}_f) - \nabla_{\vtheta_f} f_k^{(i)}(\vtheta_{t,f}) \right) \nabla_{\vtheta_g} g_{k,j}^{(i)}(\tilde{\tilde{\vtheta}}_g)^\top \right) \delta \vtheta_g \\
    & \quad + \delta \vtheta_f^\top \left(\sum_{k=1}^K \nabla_{\vtheta_f} f_k^{(i)}(\vtheta_{t,f}) \left( \nabla_{\vtheta_g} g_{k,j}^{(i)}(\tilde{\tilde{\vtheta}}_g) - \nabla_{\vtheta_g} g_{k,j}^{(i)}(\vtheta_{t,g}) \right)^\top \right) \delta \vtheta_g \\
    %& \quad + \delta \vtheta_f^\top \left(\frac{1}{n} \sum_{i=1}^n \frac{1}{q_i} \sum_{j=1}^{q_i} \ell'_{ij} \sum_{k=1}^K \nabla_{\vtheta_f} f_k^{(i)}(\tilde{\vtheta}_f) \nabla_{\vtheta_g} g_{k,j}^{(i)}(\tilde{\vtheta}_g)^\top \right) \delta \vtheta_g \\
    %& \overset{(a)}{\geq} \delta \vtheta_f^\top \left( \sum_{h=1}^{\tilde{q}} \sigma_{h,t} \va_{h,t} \vb_{h,t}^\top \right) \delta \vtheta_g \\
    %& \quad -  \frac{(2K\lambda_1\lambda_2+\tilde{c})}{n} \sum_{i=1}^n \frac{1}{q_i} \sum_{j=1}^{q_i}  \left\| \nabla_{\vtheta_f} f_k^{(i)}(\tilde{\vtheta}_f) - \nabla_{\vtheta_f} f_k^{(i)}(\vtheta_{t,f}) \right\|_2 \left\| \nabla_{\vtheta_g} g_{k,j}^{(i)}(\tilde{\vtheta}_g) \right\|_2 \| \delta \vtheta_f \|_2 \| \delta \vtheta_g \|_2  \\
    %& \quad - \frac{(2K\lambda_1\lambda_2+\tilde{c})}{n} \sum_{i=1}^n \frac{1}{q_i} \sum_{j=1}^{q_i} \sum_{k=1}^K \left\| \nabla_{\vtheta_f} f_k^{(i)}(\vtheta_{t,f}) \right\|_2 \left\| \nabla_{\vtheta_g} g_{k,j}^{(i)}(\tilde{\vtheta}_g) - \nabla_{\vtheta_g} g_{k,j}^{(i)}(\vtheta_{t,g})  \right\|_2 \delta \vtheta_f^\top \|_2 \| \delta \vtheta_g \|_2 \\
    & \leq
    \delta \vtheta_f^\top \left( \sum_{k=1}^K \nabla_{\vtheta_f} f_k^{(i)}(\vtheta_{t,f}) \nabla_{\vtheta_g} g_{k,j}^{(i)}(\vtheta_{t,g})^\top \right) \delta \vtheta_g 
    \\
    &\quad+\sum^K_{k=1}  \left\| \nabla_{\vtheta_f} f_k^{(i)}(\tilde{\tilde{\vtheta}}_f) - \nabla_{\vtheta_f} f_k^{(i)}(\vtheta_{t,f}) \right\|_2 \left\| \nabla_{\vtheta_g} g_{k,j}^{(i)}(\tilde{\tilde{\vtheta}}_g) \right\|_2 \| \delta \vtheta_f \|_2 \| \delta \vtheta_g \|_2  \\
    & \quad +  \sum_{k=1}^K \left\| \nabla_{\vtheta_f} f_k^{(i)}(\vtheta_{t,f}) \right\|_2 \left\| \nabla_{\vtheta_g} g_{k,j}^{(i)}(\tilde{\tilde{\vtheta}}_g) - \nabla_{\vtheta_g} g_{k,j}^{(i)}(\vtheta_{t,g})  \right\|_2 \| \delta \vtheta_f \|_2 \| \delta \vtheta_g \|_2 \\
    %
    & \overset{(a)}{\leq} 
    \delta \vtheta_f^\top \left( \sum_{k=1}^K \nabla_{\vtheta_f} f_k^{(i)}(\vtheta_{t,f}) \nabla_{\vtheta_g} g_{k,j}^{(i)}(\vtheta_{t,g})^\top \right) \delta \vtheta_g 
    \\
    &\quad+\xi_3\xi K\left( \frac{c^{(f)} \varrho^{(g)}}{\sqrt{m_f}} \right) \| \delta \vtheta_f \|_2^2 \| \delta \vtheta_g \|_2 
    +\xi_3\xi K\left( \frac{c^{(g)} \varrho^{(f)}}{\sqrt{m_g}} \right) \| \delta \vtheta_f \|_2 \| \delta \vtheta_g \|_2^2
    \\
    & \overset{(b)}{\leq} \delta \vtheta_f^\top \left( \sum_{k=1}^K \nabla_{\vtheta_f} f_k^{(i)}(\vtheta_{t,f}) \nabla_{\vtheta_g} g_{k,j}^{(i)}(\vtheta_{t,g})^\top \right) \delta \vtheta_g\\
    &\quad+ K\left( \frac{c^{(g)} \varrho^{(f)}}{\sqrt{m_f}} + \frac{c^{(f)} \varrho^{(g)}}{\sqrt{m_g}} \right) \| \delta \vtheta \|_2^3\\
    & \leq \delta \vtheta_f^\top \left( \sum_{k=1}^K \nabla_{\vtheta_f} f_k^{(i)}(\vtheta_{t,f}) \nabla_{\vtheta_g} g_{k,j}^{(i)}(\vtheta_{t,g})^\top \right) \delta \vtheta_g\\
    &\quad+ K\rho_2\left( \frac{c^{(g)} \varrho^{(f)}}{\sqrt{m_g}} + \frac{c^{(f)} \varrho^{(g)}}{\sqrt{m_f}} \right) \| \delta \vtheta \|_2^2~,
\end{align*}
where (a) follows from the generalized mean value theorem, from $\norm{\tilde{\tilde{\vtheta}}_f-\vtheta_{t,f}}_2= \norm{\xi_3 \tilde{\vtheta}_f+(1-\xi_3)\vtheta_{t,f}-\vtheta_{t,f}}_2=\xi_3\xi\norm{\vtheta'-\vtheta_{t,f}}_2=\xi_3\xi\norm{\delta\vtheta_f}_2$, and from the results in Lemma~\ref{lemm:hessgradbounds} since $\tilde{\tilde{\vtheta}}\in B^{\mathrm{Euc}}_{\rho,\rho_1}(\vtheta_0)$; and (b) follows from $\norm{\vtheta_f}_2,\norm{\vtheta_g}_2\leq \norm{\vtheta}_2$ and $\xi_3\xi\leq 1$.

Replacing the bound on $I_3$ back to~\eqref{eq:dd_G}, we obtain 
\begin{equation}
\label{eq:dd_G1}
\begin{aligned}
    (\delta \vtheta)^\top \nabla_{\vtheta}^2 G_{\tilde{\tilde{\vtheta}}}(u^{(i)})(\vy^{(i)}_j)(\delta\vtheta) 
& \leq \delta \vtheta_f^\top \left( \sum_{k=1}^K \nabla_{\vtheta_f} f_k^{(i)}(\vtheta_{t,f}) \nabla_{\vtheta_g} g_{k,j}^{(i)}(\vtheta_{t,g})^\top \right) \delta \vtheta_g\\
&\quad+ K(1+\rho_2)\left( \frac{c^{(g)} (\lambda_1+2\varrho^{(f)})}{\sqrt{m_g}} + \frac{c^{(f)} (\lambda_2+2\varrho^{(g)})}{\sqrt{m_f}} \right) \| \delta \vtheta \|_2^2\\
&\leq K(1+\rho_2)\left( \frac{c^{(g)} (\lambda_1+2\varrho^{(f)})}{\sqrt{m_g}} + \frac{c^{(f)} (\lambda_2+2\varrho^{(g)})}{\sqrt{m_f}} \right) \| \delta \vtheta \|_2^2~,
%
%
\end{aligned}
\end{equation}
where the last inequality follows from the fact that  $\vtheta^{\prime}\in Q^t_{\kappa}$, using the properties of the restricted set $Q^t_{\kappa}$ in Definition~\ref{defn:qset}.

Replacing~\eqref{eq:dd_G1} back to $I_2$ in~\eqref{eq:I3-1}, we obtain
\begin{equation}
\label{eq:I2-21}
\begin{aligned}
    I_2
%     
    & \geq\frac{2}{n} \sum_{i=1}^n \frac{1}{q_i} \sum_{j=1}^{q_i}  \left\langle \delta \vtheta , \nabla_{\vtheta} G_{\vtheta_t}(u^{(i)})(\vy^{(i)}_j) \right\rangle^2  \\
    & \quad - 
4\xi_3K(\lambda_2\varrho^{(f)}+\lambda_1\varrho^{(g)})\left\|\delta\vtheta\right\|_2
    \times K(1+\rho_2)\left( \frac{c^{(g)} (\lambda_1+2\varrho^{(f)})}{\sqrt{m_g}} + \frac{c^{(f)} (\lambda_2+2\varrho^{(g)})}{\sqrt{m_f}} \right) \| \delta \vtheta \|_2^2\\
%
& =\frac{2}{n} \sum_{i=1}^n \frac{1}{q_i} \sum_{j=1}^{q_i}  \left\langle \delta \vtheta , \nabla_{\vtheta} G_{\vtheta_t}(u^{(i)})(\vy^{(i)}_j) \right\rangle^2  \\
    & \quad - 
4\xi_3K^2(1+\rho_2)\rho_2(\lambda_2\varrho^{(f)}+\lambda_1\varrho^{(g)})
    \left( \frac{c^{(g)} (\lambda_1+2\varrho^{(f)})}{\sqrt{m_g}} + \frac{c^{(f)} (\lambda_2+2\varrho^{(g)})}{\sqrt{m_f}} \right) \| \delta \vtheta \|_2^2~.    
%%    
\end{aligned}
\end{equation}
Replacing this lower bound~\eqref{eq:I2-21} back to the Hessian expression in~\eqref{eq:Hess_tildetheta_1},
\begin{equation}
\label{eq:Hess_tildetheta_1}
\begin{aligned}
    \delta \vtheta^{\top} \mH (\tilde{\vtheta}) \delta \vtheta
&\geq\frac{2}{n} \sum_{i=1}^n \frac{1}{q_i} \sum_{j=1}^{q_i}  \left\langle \delta \vtheta , \nabla_{\vtheta} G_{\vtheta_t}(u^{(i)})(\vy^{(i)}_j) \right\rangle^2  \\
    & \quad - 
4\xi_3K^2(1+\rho_2)\rho_2(\lambda_2\varrho^{(f)}+\lambda_1\varrho^{(g)})
    \left( \frac{c^{(g)} (\lambda_1+2\varrho^{(f)})}{\sqrt{m_g}} + \frac{c^{(f)} (\lambda_2+2\varrho^{(g)})}{\sqrt{m_f}} \right) \| \delta \vtheta \|_2^2\\
&\quad- 
    (2K\lambda_1\lambda_2+\tilde{c})
    \left(\frac{\lambda_1 c^{(g)}}{\sqrt{m_g}} + \frac{\lambda_2 c^{(f)}}{\sqrt{m_f}}\right) \| \delta \vtheta \|_2^2\\
& \quad- 2(2K\lambda_1\lambda_2+\tilde{c})\rho_2 \left( \frac{c^{(g)} \varrho^{(f)}}{\sqrt{m_f}} + \frac{c^{(g)} \varrho^{(f)}}{\sqrt{m_f}} \right) \| \delta \vtheta \|_2^2\\
%
&\overset{(a)}{\geq} 2 \left\langle \delta \vtheta , \nabla_{\vtheta} \bar{G}_{\vtheta_t} \right\rangle^2  - c_1 K^2
\left( \frac{1}{\sqrt{m_f}} + \frac{1}{\sqrt{m_g}} \right) \| \delta \vtheta \|_2^2\\
&\overset{(b)}{\geq} 2\kappa^2 \norm{\nabla_{\vtheta} \bar{G}_{\vtheta_t}}^2_2\norm{\delta\vtheta}_2^2  - c_1 K^2
\left( \frac{1}{\sqrt{m_f}} + \frac{1}{\sqrt{m_g}} \right) \| \delta \vtheta \|_2^2\\
%
&=\alpha_t\norm{\delta\vtheta}_2^2~,
%
\end{aligned}
\end{equation}
where (a) follows from Jensen's inequality with $\bar{G}_{\vtheta} = \frac{1}{n} \sum_{i=1}^n \frac{1}{q_i} \sum_{j=1}^{q_i} G_{\vtheta}(u^{(i)})(\vy^{(i)}_j)$; where (b) follows from the fact that  $\vtheta^{\prime}\in Q^t_{\kappa}$ and using the properties of the restricted set $Q^t_{\kappa}$ in Definition~\ref{defn:qset}; and where $\alpha_t = 2\kappa^2 \| \nabla_{\vtheta} \bar{G}_{\vtheta} \|_2^2 - c_1K^2\left(\frac{1}{m_f}+\frac{1}{m_g}\right)$. 
%\pcedit{Note that the defined constant $c_1$ inherits its dependency from Lemma~\ref{lemm:hessgradbounds} as stated in Theorem~\ref{theo:rsc_main_DON}'s statement.}
%
\pcedit{Note that adding all the constants from the second to the fourth line in~\eqref{eq:Hess_tildetheta_1} define the constant $c_1$, and so $c_1$ depends on $\sigma_1$, the depth $L$, and the radii $\rho$, $\rho_1$, and $\rho_2$ due to Lemma~\ref{lemm:hessgradbounds}. As in the statement of Lemma~\ref{lemm:hessgradbounds}, this dependence reduces to the depth and the radii and becomes polynomial whenever $\sigma_0\leq 1-\rho\max\{\frac{1}{\sqrt{m_f}},\frac{1}{\sqrt{m_g}}\}$.}
This completes the proof. \qed
% \end{proof}




% \begin{theo}[Smoothness of Loss]
% Under the assumptions \ThreeAsmpsref{asmp:Activation_Function}{asmp:Loss}{asmp:smoothinit}, with high probability, 
% for $\vtheta \in B_{\rho}^{\mathrm{Euc}}(\vtheta_0)$, $\gL(\vtheta)$ is $\beta$-smooth with  $\beta = b \varrho^2 + \frac{c\sqrt{\lambda}}{\sqrt{m}}$ with $c= \max(c^{(f)}, c^{(g)}), \varrho = \max(\varrho^{(f)}, \varrho^{(g)})$ with $c^{(f)}, c^{(g)}, \varrho^{(f)}, \varrho^{(g)}$ as in Lemma~\ref{lemm:hessgradbounds}.
% \label{theo:smooth}
% \end{theo}
\RSS*
\begin{proof}
By the second order Taylor expansion of $\gL(\vtheta^\prime)$ about the point $\bar{\vtheta}$ with $\vtheta^\prime,\bar{\vtheta}\in B^{\mathrm{Euc}}_{\rho,\rho_1}(\vtheta_0)$, we have
$\cL(\vtheta') = \cL(\bar{\vtheta}) + \langle \vtheta' - \bar{\vtheta}, \nabla_\vtheta\cL(\bar{\vtheta}) \rangle + \frac{1}{2} (\vtheta'-\bar{\vtheta})^\top \frac{\partial^2 \cL(\tilde{\vtheta})}{\partial \vtheta^2} (\vtheta'-\bar{\vtheta})$, 
where $\tilde{\vtheta} = \xi \vtheta' + (1-\xi) \bar{\vtheta}$ for some $\xi \in [0,1]$. Then, 
\begin{align*}
    (\vtheta'-\bar{\vtheta})^\top \frac{\partial^2 \cL(\tilde{\vtheta})}{\partial \vtheta^2} (\vtheta'-\bar{\vtheta}) 
    & = (\vtheta'-\bar{\vtheta})^\top \bigg( \frac{1}{n} \sum_{i=1}^n \frac{1}{q_i} \sum_{j=1}^{q_i}  \ell^{\prime\prime}_{i,j} \nabla_{\vtheta} G_{\tilde{\vtheta}}(u^{(i)})(\vy^{(i)}_j) \nabla_{\vtheta} G_{\tilde{\vtheta}}(u^{(i)})(\vy^{(i)}_j)^\top  \\
    & \qquad \qquad \qquad \qquad + \ell^{\prime}_{i,j}   \nabla^2_{\vtheta} G_{\tilde{\vtheta}}(u^{(i)})(\vy^{(i)}_j)  \bigg)  (\vtheta'-\bar{\vtheta}) \\
    & = \underbrace{\frac{1}{n} \sum_{i=1}^n  \frac{1}{q_i} \sum_{j=1}^{q_i} \ell^{\prime\prime}_{i,j} \left\langle \vtheta'-\bar{\vtheta}, \nabla_{\vtheta} G_{\tilde{\vtheta}}(u^{(i)})(\vy^{(i)}_j) \right\rangle^2}_{I_1} \\
    & \qquad \qquad + \underbrace{\frac{1}{n} \sum_{i=1}^n \frac{1}{q_i} \sum_{j=1}^{q_i} \ell^{\prime}_{i,j}  (\vtheta'-\bar{\vtheta})^\top \nabla^2_{\vtheta} G_{\tilde{\vtheta}}(u^{(i)})(\vy^{(i)}_j)  (\vtheta'-\bar{\vtheta}) }_{I_2}~,
\end{align*}
where $\ell_{i,j}=(G_{\tilde{\vtheta}}(u^{(i)})(\vy_{j}^{(i)})-G^\dagger(u^{(i)})(\vy^{(i)}_j))^2$. 

Now, note that
\begin{align*}
I_1 & = \frac{1}{n} \sum_{i=1}^n  \frac{1}{q_i} \sum_{j=1}^{q_i} \ell^{\prime\prime}_{i,j} \left\langle \vtheta'-\bar{\vtheta}, \nabla_{\vtheta} G_{\tilde{\vtheta}}(u^{(i)})(\vy^{(i)}_j) \right\rangle^2 \\
& \overset{(a)}{\leq} \frac{2}{n} \sum_{i=1}^n \frac{1}{q_i}\sum_{j=1}^{q_i} \left\| \nabla_{\vtheta} G_{\tilde{\vtheta}}(u^{(i)})(\vy^{(i)}_j) \right\|_2^2 \|\vtheta' - \bar{\vtheta} \|_2^2 \\
& \overset{(b)}{\leq} 4K^2(\lambda_2\varrho^{(f)}+\lambda_1\varrho^{(g)})^2 \| \vtheta' - \bar{\vtheta} \|_2^2~,
\end{align*}
where (a) follows by the Cauchy-Schwartz inequality and (b) from Lemma~\ref{lemm:hessgradbounds} as follows
\begin{equation*}
\norm{\nabla_{\vtheta}G_{\tilde{\vtheta}}(u^{(i)})(\vy^{(i)}_j)}_2\leq\sum^K_{k=1}(
\norm{g_{k,j}^{(i)}(\tilde{\vtheta}_g)\nabla_{\vtheta_f}f_k^{(i)}(\tilde{\vtheta}_f)}_2
+
\norm{
f_{k}^{(i)}(\tilde{\vtheta}_f)\nabla_{\vtheta_g}g_{k,j}^{(i)}(\tilde{\vtheta}_g)
}_2
)
\leq
%\leq \sqrt{2}\sum^K_{k=1}\norm{\nabla_{\tilde{\vtheta}_f}f_k^{(i)}}_2|g_{k,j}^{(i)}|+\norm{\nabla_{\tilde{\vtheta}_g}g_{k,j}^{(i)}}_2|f_{k}^{(i)}|\leq 
K(\lambda_2\varrho^{(f)}+\lambda_1\varrho^{(g)}),
\end{equation*}
since $\tilde{\vtheta}\in B^{\mathrm{Euc}}_{\rho,\rho_1}(\vtheta_0)$.

Now, for $I_2$, 
\begin{align*}
I_2 & \leq \frac{1}{n} \sum_{i=1}^n \frac{1}{q_i} \sum_{j=1}^{q_i} |\ell^{\prime}_{i,j}|  |(\vtheta'-\bar{\vtheta})^\top \nabla^2 G_{\tilde{\vtheta}}(u^{(i)})(\vy^{(i)}_j) (\vtheta'-\bar{\vtheta})| \\
%& \leq \left| \sum_{i=1}^n \left( \frac{1}{\sqrt{n}} \ell'_{i} \right) \left( \frac{1}{\sqrt{n}}Q_{i} \right) \right| \\
%& \overset{(a)}{\leq} \lambda \left( \frac{1}{n} \sum_{i=1}^n \frac{1}{q_i} Q^2_{t,(i,j)} \right)^{1/2}  \\
& \overset{(a)}{\leq}
(2K\lambda_1\lambda_2+\tilde{c})\left(K\varrho^{(f)}\varrho^{(g)}+K(1+\rho_2) 
\left( \frac{c^{(g)} (\lambda_1+\varrho^{(f)})}{\sqrt{m_g}} + \frac{c^{(f)} (\lambda_2+\varrho^{(g)})}{\sqrt{m_f}} \right)\right)
\| \vtheta' - \bar{\vtheta} \|_2^2~,
%& \leq  \frac{\lambda}{n} \sum_{i=1}^n   \| \vtheta'-\bar{\vtheta}\|_2^2 \left\| \frac{\partial^2 f(\tilde{\vtheta};\x_i)}{\partial \vtheta^2} \right\|_2 \\
%& \leq \frac{\lambda c_H}{\sqrt{m}} \| \vtheta' - \bar{\vtheta} \|_2^2~. 
\end{align*}
%where (a) follows by Cauchy-Schwartz.
with  $\tilde{c}=\max_{i\in[n],j\in[q_i]}|G^\dagger(u^{(i)})(\vy^{(i)}_j)|$, and where (a) follows from modifying the result in equation~\eqref{eq:dd_G1} from Theorem~\ref{theo:rsc_main_DON} according to our setting.
%

Putting the upper bounds on $I_1$ and $I_2$ back, we have
\begin{align*}
(\vtheta'-\bar{\vtheta})^\top \frac{\partial^2 \cL(\tilde{\vtheta})}{\partial \vtheta^2} (\vtheta'-\bar{\vtheta})
& \leq \left[ 
4K^2(\lambda_2\varrho^{(f)}+\lambda_1\varrho^{(g)})^2 \right.\\
&\left.\quad 
+
(2K\lambda_1\lambda_2+\tilde{c})\left(K\varrho^{(f)}\varrho^{(g)}+K(1+\rho_2) 
\left( \frac{c^{(g)} (\lambda_1+\varrho^{(f)})}{\sqrt{m_g}} + \frac{c^{(f)} (\lambda_2+\varrho^{(g)})}{\sqrt{m_f}} \right)\right)
\right]\\
&\quad \times \| \vtheta' - \bar{\vtheta} \|_2^2~.
\end{align*}
\pcedit{Note that all the constants on the right-hand side of the inequality above form an expression that depends on $K$ and on $\sigma_1$, the depth $L$, and the radii $\rho$, $\rho_1$, and $\rho_2$ due to Lemma~\ref{lemm:hessgradbounds}. As in the statement of Lemma~\ref{lemm:hessgradbounds}, the dependence of such expression reduces to the depth and the radii and becomes polynomial whenever $\sigma_0\leq 1-\rho\max\{\frac{1}{\sqrt{m_f}},\frac{1}{\sqrt{m_g}}\}$.} 
This completes the proof.
\label{theo:smooth}
\end{proof}

\begin{prop}[{\bf RSC to smoothness ratio}]
\label{prop:RSC-smooth-DON}
Under the same conditions as in Theorems~\ref{theo:rsc_main_DON} and~\ref{theo:smooth_main}, we have that $\alpha_t/\beta<1$ with probability at least $1-2LK(\frac{1}{m_f}+\frac{1}{m_g})$.
\end{prop}
\begin{proof}
From the proofs of both Theorems~\ref{theo:rsc_main_DON} and~\ref{theo:smooth_main},
$\alpha_t<2\kappa^2\norm{\nabla_{\vtheta_t}\bar{G}_t}_2^2\leq 2\kappa^2 K^2(\lambda_2\varrho^{(f)}+\lambda_1\varrho^{(g)})^2\leq
4 K^2(\lambda_2\varrho^{(f)}+\lambda_1\varrho^{(g)})^2<
\beta$, and so $\frac{\alpha_t}{\beta}<1$.
\end{proof}
