We recall the FNO model
\begin{align}
    \begin{aligned}
\aalpha^{(0)} &= P(u)(\vx)\\
\aalpha^{(1)} &= \phi\left(
        \frac{1}{\sqrt{m}} W^{(1)} \aalpha^{(0)}
    \right)\\
\aalpha^{(l)} & = \phi\left(
        \frac{1}{\sqrt{m}} W^{(l)} \aalpha^{(l-1)} +
        \frac{1}{\sqrt{m}} F^{*} R^{(l)} F \aalpha^{(l-1)}
    \right),\quad l\in \{2,\dots,L+1\}\\
    f(\vtheta;\x)  = \aalpha^{(L+2)} &:= \frac{1}{\sqrt{m}} \v^\top \aalpha^{(L+1)}~,
    \end{aligned}
\label{eq:FNO_predictor_app}
\end{align}
where $W^{(l)}, R^{(l)} \in \R^{m \times m}$ for $l \in \{2,\ldots,L+1\}$, $W^{(1)} \in \R^{m \times d}$. 
%

\subsection{Bounds on the Hessian, Gradients and the Predictor}

\begin{restatable}[{\bf Bounds on the Predictor}]{lemm}{HessDiag2}
\label{lemm:hessgradbounds-FNO}
Under Assumptions~\ref{asmp:Activation_Function_FNO} and \ref{asmp:smoothinit_FNO} and for $\vtheta \in B^{\mathrm{Euc}}_{\rho_w,\rho_r\rho_1}(\vtheta_0)$ we have with probability at least {$1-\frac{2(L+2)}{m}$}, that for any input function $u$ and evaluation point $\vx$ as in Section~\ref{sec:optFNO},
\begin{align}
    %\max _{i \in[n]}%\max_{j\in[q_i]}
        \left\|\nabla^2_{\vtheta} f\right\| \leq \frac{c}{\sqrt{m}}, \label{eq:hessianBoundG_fg_FNO} \\
    \left\| \nabla_{\vtheta} f\right\|_2 \leq \varrho~,\label{eq:gradientBoundG_fg_FNO}\\
    %
     |f| \leq \lambda~,\label{eq:predictorBoundG_fg_FNO}
\end{align}
where $c,\;\varrho,\;\lambda$ are suitable constants that depend on $\sigma_{1,w}$, $\sigma_{1,r}$, the depth $L$, and the radii $\rho_w$, $\rho_r$, and $\rho_1$. 
\pcedit{The dependence of the constants reduces to depth and the radii and becomes polynomial whenever $\sigma_{1,w}+\sigma_{1,r}\leq 1-\frac{\rho_w+\rho_r}{\sqrt{m}}$.}
\end{restatable}

In this section we will prove all the bounds in Lemma~\ref{lemm:hessgradbounds-FNO}.
%

\begin{lemm}[{\bf Initialization of the Parameters}]
    \label{lemm:InitParamFNO}
    Under Assumption~\ref{asmp:smoothinit_FNO},with probability at least $1-\frac{2}{m}$ we have
\begin{equation}
    \|W^{(l)}_0\|_2 \leq \sigma_{1,w}\sqrt{m}, \quad \text{and} \quad \|R^{(l)}_0\|_2 \leq \sigma_{1,r}\sqrt{m}.
    \label{eq:Wl_zero_Rl_zero}
\end{equation}
\end{lemm}
\proof The proof follows directly from Lemma A.1 in~\citep{banerjee2022restricted}. 
%We reproduce it here for the sake of completeness.  For ($m_l \times m_{l-1}$) random matrices $W^{(l)}_0$ and $R^{(l)}_0$ with i.i.d entries $w^{(l)}_{0,ij} \in \gN(0, \sigma_{0,w}^2)$ and $r^{(l)}_{0,ij} \in \gN(0, \sigma_{0,r}^2)$, the largest singular values are bounded from above with probabilities $(1-2\exp (-t^2/2\sigma_{0,w}^2))$ and $(1-2\exp (-t^2/2\sigma_{0,r}^2))$ respectively, namely,
%\begin{equation}
%    \sigma_{\max}(W^{(l)}_0) \leq \sigma_{0,w}(\sqrt{m_l} + \sqrt{m_{l-1}}) + t,\quad \text{and}\quad
%    \sigma_{\max}(R^{(l)}_0) \leq \sigma_{0,r}(\sqrt{m_l} + \sqrt{m_{l-1}}) + t.
%    \label{eq:signularValueW_l_R_l_zero}
%\end{equation}
%In order to derive the above concentration result note that $\Wlzero = \sigma_{0,w}\bar{W}^{(l)}_0$ and $\Rlzero = \sigma_{0,r}\bar{R}^{(l)}_0$, where the entries $\bar{w}^{(l)}_{0,ij} \in \gN (0, 1)$ and $\bar{r}^{(l)}_{0,ij} \in \gN (0,1)$. We can then write
%\begin{align*}
%    &\E[\|\Wlzero \|_2] = \sigma_{0,w}\E[\| \bar{W}^{(l)}_0\|_2] = \sigma_{0,w}(\sqrt{m_l} + \sqrt{m_{l-1}}),\\
%    &\E[\|\Rlzero \|_2] = \sigma_{0,r}\E[\| \bar{R}^{(l)}_0\|_2] = \sigma_{0,r}(\sqrt{m_l} + \sqrt{m_{l-1}})
%\end{align*}
%from Gordon's Theorem for Gaussian random matrices (see Theorem 5.32, Proposition 3.4 in \citep{vershynin2010introduction}) where  the function $f: B\rightarrow \| \sigma_0 B\|_2$ is a $\sigma_0$-Lipchitz function (where the matrix $B$ can be treated as a vector). Finally, choosing $t_w = \sigma_{0,w}\sqrt{2\log m}$ so that \eqref{eq:signularValueW_l_R_l_zero} holds with probability at least $\left( 1 - \frac{2}{m}\right)$. In order to obtain the result in \eqref{eq:Wl_zero_Rl_zero} we consider the following cases: 
%\begin{itemize}
%    \item {\bf Case 1:} $l=1$. With $m_0 = d_u$ and $m_1 = m$.
%    \begin{align*}
%        \left\|W_0^{(1)}\right\|_2 &\leq \sigma_0(\sqrt{d}+\sqrt{m}+\sqrt{2 \log m}) \leq \sigma_0(2 \sqrt{m}+\sqrt{2 \log m}),\\
%        R^{(1)}_0 &= 0.
%    \end{align*}
%    \item {\bf Case 2:} $2\leq l \leq L$. With $m_l = m_{l-1} = m$
%    \begin{align*}
%        &\| \Wlzero\|_2 \leq \sigma_{0,w}\left( 2\sqrt{m} + \sqrt{2\log m}\right),
%        &\| \Rlzero\|_2 \leq \sigma_{0,r}\left( 2\sqrt{m} + \sqrt{2\log m}\right).
%    \end{align*}
%    Now, using $\sigma_{0,w} = \dfrac{\sigma_{1,w}}{2(1 + \sqrt{\frac{\log m}{2m}})}$ and $\sigma_{0,r} = \dfrac{\sigma_{1,r}}{2(1 + \sqrt{\frac{\log m}{2m}})}$ completes the proof.\qed
%\end{itemize}
%\end{lemm}
\begin{prop}[{\bf Layer-wise matrices}]
\label{prop:W_l_R_l_bound_FNO}
Under Assumption~\ref{asmp:smoothinit_FNO}, for $\vtheta\in B^{\mathrm{Euc}}_{\rho_w,\rho_r\rho_1}(\vtheta_0)$, with probability at least $1-\frac{2}{m}$ we have
\begin{equation}
    \left\|W^{(l)}\right\|_2 \leq\left(\sigma_{1,w}+\frac{\rho_w}{\sqrt{m}}\right) \sqrt{m},\;l\in[L+1] \quad \text{and}\quad 
    \left\|R^{(l)}\right\|_2 \leq\left(\sigma_{1,r}+\frac{\rho_r}{\sqrt{m}}\right) \sqrt{m},\;l\in\{2,\dots,L+1\}
\end{equation}
\proof By the triangle inequality and Lemma~\ref{lemm:InitParamFNO},
\begin{align*}
    &\| W^{(l)}\|_2 \leq \| \Wlzero\|_2 + \| W^{(l)} - \Wlzero\|_2 \leq\sigma_{1,w}\sqrt{m} + \rho_w, \\ 
    &\| R^{(l)}\|_2 \leq \| \Rlzero\|_2 + \| R^{(l)} - \Rlzero\|_2 \leq \sigma_{1,r}\sqrt{m} + \rho_r~.
\end{align*}
\qed
\end{prop}
We now bound the norm of the output $\aalpha^{(l)}$ at the layer $l\in[L+1]$.
%
\begin{lemm}[{\bf Norm of the $l$-th layer output}]
\label{lemm:TwoNormOutputFNOBlock}
For $l\in [L+1]$, under Assumptions~\ref{asmp:Activation_Function_FNO} and \ref{asmp:smoothinit_FNO} for $\theta\in B^{\mathrm{Euc}}_{\rho_w,\rho_r\rho_1}(\vtheta_0)$, with probability at least $1 - \frac{2l}{m}$, we have
\begin{equation}
    \left\|\aalpha^{(l)}\right\|_2 \leq \sqrt{m}\left(\sigma_1+\frac{\rho}{\sqrt{m}}\right)^l+\sqrt{m} \sum_{i=1}^l\left(\sigma_1+\frac{\rho}{\sqrt{m}}\right)^{i-1}|\phi(0)|=\left(\gamma^l+|\phi(0)| \sum_{i=1}^l \gamma^{i-1}\right) \sqrt{m},
\end{equation}
where,
\begin{equation*}
    \sigma_1 = \sigma_{1,w} + \sigma_{1,r},\quad\rho = \rho_w + \rho_r,\quad \text{and}\quad \gamma=\sigma_1+\frac{\rho}{\sqrt{m}}.
\end{equation*}
\proof We prove the result using induction (e.g., see Lemma A.2 in~\citep{banerjee2022restricted}). First, note that for the first hidden layer, using the fact that $\phi$ is $1$-Lipschitz,
\begin{equation}
    \left\|\phi\left(\frac{1}{\sqrt{d}} W^{(1)} \aalpha^{(0)}\right)\right\|_2-\|\phi(\mathbf{0})\|_2 \leq\left\|\phi\left(\frac{1}{\sqrt{d}} W^{(1)} \aalpha^{(0)}\right)-\phi(\mathbf{0})\right\|_2 \leq\left\|\frac{1}{\sqrt{d}} W^{(1)} \aalpha^{(0)}\right\|_2,
\end{equation}
where $\vzero$ denotes the zero vector of appropriate size. This in turn gives, using $\|\aalpha^{(0)} \|_2 = \sqrt{d}$,
\begin{align*}
    \begin{aligned}
        \left\|\aalpha^{(1)}\right\|_2 & =\left\|\phi\left(\frac{1}{\sqrt{d}} W^{(1)} \aalpha^{(0)}\right)\right\|_2 \leq\left\|\frac{1}{\sqrt{d}} W^{(1)} \aalpha^{(0)}\right\|_2+\|\phi(\mathbf{0})\|_2 \\
        & \leq \frac{1}{\sqrt{d}}\left\|W^{(1)}\right\|_2\left\|\aalpha^{(0)}\right\|_2+|\phi(0)| \sqrt{m} \\
        & \leq\left(\sigma_{1,w}+\frac{\rho_w}{\sqrt{m}}\right) \sqrt{m}+|\phi(0)| \sqrt{m} \\
        & \leq\left(\sigma_{1,w}+\sigma_{1,r}+\frac{\rho_w + \rho_r}{\sqrt{m}}\right) \sqrt{m}+|\phi(0)| \sqrt{m}~.
    \end{aligned}
\end{align*}
Now, consider also the output at layer $2$, namely,
\begin{align*}
   \| \aalpha^{(2)}\|_2 
   = 
   \left\|\phi\left(\frac{1}{\sqrt{m}} W^{(2)} \aalpha^{(1)}
    +
    \frac{1}{\sqrt{m}} F^{*} R^{(2)} F \aalpha^{(1)}\right)\right\|_2,
\end{align*}
which gives,
\begin{align*}
    & \left\|\phi\left(\frac{1}{\sqrt{m}} W^{(2)} \aalpha^{(1)}
    +
    \frac{1}{\sqrt{m}} F^{*} R^{(2)} F \aalpha^{(1)}\right)\right\|_2 - \| \phi(\mathbf{0})\|_2 \\
    &\leq 
    \left\|\phi\left(\frac{1}{\sqrt{m}} W^{(2)} \aalpha^{(1)}
    +
    \frac{1}{\sqrt{m}} F^{*} R^{(2)} F \aalpha^{(1)}\right)
    - \phi(\mathbf{0})
    \right\|_2 \leq 
    \left\|\frac{1}{\sqrt{m}} W^{(2)} \aalpha^{(1)}
    +
    \frac{1}{\sqrt{m}} F^{*} R^{(2)} F \aalpha^{(1)}\right\|_2,
\end{align*}
and, in turn,
\begin{align*}
    \| \aalpha^{(2)}\|_2 
    &\leq \left\|\frac{1}{\sqrt{m}} W^{(2)} \aalpha^{(1)}
    +
    \frac{1}{\sqrt{m}} F^{*} R^{(2)} F \aalpha^{(1)}\right\|_2 + \| \phi(\mathbf{0})\|_2 \\
    &\leq \left\|\frac{1}{\sqrt{m}} W^{(2)} \aalpha^{(1)}\right\|_2+\left\|\frac{1}{\sqrt{m}} F^* R^{(2)} F \aalpha^{(1)}\right\|_2+|\phi(0)| \sqrt{m} \\
    &\overset{(a)}{\leq} \frac{1}{\sqrt{m}} \|W^{(2)}\|_2 \|\aalpha^{(1)}\|_2 + \frac{1}{\sqrt{m}} \|R^{(2)}\|_2 \| \aalpha^{(1)}\|_2 + \sqrt{m}|\phi(0)| \\
    &\leq \left( \sigma_{1,w} + \frac{\rho_w}{\sqrt{m}} + \sigma_{1,r} + \frac{\rho_r}{\sqrt{m}}\right) \| \aalpha^{(1)}\|_2 + \sqrt{m}|\phi(0)|\\
    &\leq \sqrt{m}\left( \sigma_1 + \frac{\rho}{\sqrt{m}}\right)^2 + \left(1 + \left( \sigma_1 + \frac{\rho}{\sqrt{m}}\right) \right)\sqrt{m}|\phi(0)|,
\end{align*}
where (a) follows from the fact that the operator $F$ is a unitary matrix. Now, for the inductive step, consider that the output at layer $l-1$ satisfies
\begin{align*}
    \left\|\aalpha^{(l-1)}\right\|_2 
    \leq \sqrt{m}\left(\sigma_1
    +
    \frac{\rho}{\sqrt{m}}\right)^{l-1}
    +\sqrt{m} \sum_{i=1}^{l-1}
    \left(\sigma_1
        +
        \frac{\rho}{\sqrt{m}}\right)^{i-1}|\phi(0)|.
\end{align*}
Finally, at layer $l$, we have
\begin{align}
    \left\|\aalpha^{(l)}\right\|_2 
    &\leq \frac{1}{\sqrt{m}}\left(\underbrace{\left\|W^{(l)}\right\|_2+\left\|F^* R^{(l)} F\right\|_2}\right) \| \aalpha^{(l-1)}\|_2
    +
    \sqrt{m}|\phi (0)|\\
    &\leq \left( \sigma_{1,w} + \sigma_{1,r} + \frac{\rho_w + \rho_r}{\sqrt{m}}\right) \| \aalpha^{(l-1)}\|_2
    +
    \sqrt{m}|\phi (0)|\\
    &\leq \sqrt{m}\left(\sigma_1
    +
    \frac{\rho}{\sqrt{m}}\right)^{l}
    +\sqrt{m} \sum_{i=1}^{l}
    \left(\sigma_{1}
        +
        \frac{\rho}{\sqrt{m}}\right)^{i-1}|\phi(0)|.
\end{align}
Introducing $\gamma = \sigma_1 + \dfrac{\rho}{\sqrt{m}}$, we can write
\begin{equation}
    \|\aalpha^{(l)}\|_2 \leq \sqrt{m}\left(\gamma^l+|\phi(0)| \sum_{i=1}^{l} \gamma^{i-1}\right).
    \label{eq:alpha_l_2_norm_bound}
\end{equation}
This completes the proof.\hfill \qed
\end{lemm}

From now on, we will use the notation $\rho$, $\sigma_1$, and $\gamma$ as defined in Lemma~\ref{lemm:TwoNormOutputFNOBlock}.

\begin{lemm}
    \label{lemm:firstDerivativeBoundFNO}
    For $l\in \{2,\dots,L+1\}$, under Assumptions~\ref{asmp:Activation_Function_FNO} and \ref{asmp:smoothinit_FNO} for $\theta\in B^{\mathrm{Euc}}_{\rho_w,\rho_r\rho_1}(\theta_0)$, with probability at least $1 - \frac{2}{m}$, we have
    \begin{equation}
        \left\|\frac{\partial \aalpha^{(l)}}{\partial \aalpha^{(l-1)}}\right\|_2 \leq
        %\left(\sigma_{1,w} + \frac{\rho_w}{\sqrt{m}}\right)
        %+
        %\left(\sigma_{1,r} + \frac{\rho_r}{\sqrt{m}}\right)
        %=
        \gamma.
    \end{equation}
    \proof
    We first note that 
    \begin{align*}
        \left[
            \del{\aalpha^{(l)}}{\aalpha^{(l-1)}}
        \right]_{ij} = 
        \frac{1}{\sqrt{m}} \phi'(\widetilde{\aalpha}^{(l-1)}) \left[ 
            W^{(l)}_{ij} + [F^*R^{(l)}F]_{ij}
        \right].
    \end{align*}
    Now, from the definition $\|A\|_2 = \sup_{\| \vv\|_2 = 1} \| A\vv\|_2$ we have,
    \begin{align}
        \begin{aligned}
            \left\|\frac{\partial \aalpha^{(l)}}{\partial \aalpha^{(l-1)}}\right\|_2 
            &= 
            \sup_{\|\vv\|_2 = 1} \frac{1}{\sqrt{m}}
            \left(
                {\phi'} \left\|\left( W^{(l)} + F^*R^{(l)}F \right)\vv\right\|_2
            \right)\\
            &\overset{(a)}{\leq} \sup_{\|\vv\|_2 = 1} \frac{1}{\sqrt{m}}
            \left(
                \|W^{(l)}\vv\|_2 +  \|F^*R^{(l)}F\vv\|_2
            \right)
            \\
            &\overset{(b)}{=} \sup_{\|\vv\|_2 = 1} \frac{1}{\sqrt{m}}
            \left(
                \|W^{(l)}\vv\|_2 +  \|R^{(l)}F\vv\|_2
            \right)\\
            &\overset{(c)}{\leq}\sup_{\|\vv\|_2 = 1} \frac{1}{\sqrt{m}}
            \left(
                \|W^{(l)}\|_2\norm{\vv}_2 +  \norm{R^{(l)}}_2\norm{\vv}_2
            \right)\\
            &=\frac{1}{\sqrt{m}}
            \left(
                \|W^{(l)}\|_2 +  \norm{R^{(l)}}_2
            \right)~,            
        \end{aligned}
    \end{align}
    where $(a)$ follows from the fact that $\phi$ is $1$-Lipchitz and by using the triangle inequality, and $(b)$ and $(c)$ follow from the fact that $F^*$ and $F$ are isometries with respect to the $L_2$-norm, i.e. $\|F \vv\|_2=\|\vv\|_2$ and $\|F^* \vv\|_2=\|\vv\|_2$ for $\vv\in\R^{m}$.
    This finally gives 
    \begin{align*}
        \left\|\frac{\partial \aalpha^{(l)}}{\partial \aalpha^{(l-1)}}\right\|_2  
        \leq 
        \frac{1}{\sqrt{m}}\left(\|W^{(l)}\|_2 + \|R^{(l)}\|_2 \right) &\leq
            \left(\sigma_{1,w} + \frac{\rho_w}{\sqrt{m}}\right)
            + \left(\sigma_{1,r} + \frac{\rho_r}{\sqrt{m}}\right) \\
            & = \gamma~,
    \end{align*}
where we used Proposition~\ref{prop:W_l_R_l_bound_FNO}. This completes the proof.\hfill\qed
\end{lemm}

We make use of the Einstein summation convention, i.e. repeated indices imply summation, unless explicitly stated. We also use the notation $\text{vec}(\cdot)$ to denote the vectorization of the matrix argument according to some fixed manner (e.g., row-wise vectorization).

\begin{lemm}
    \label{lemm:gradient_alpha_params}
    Under Assumptions~\ref{asmp:Activation_Function_FNO} and \ref{asmp:smoothinit_FNO} and for $\theta\in B^{\mathrm{Euc}}_{\rho_w,\rho_r\rho_1}(\vtheta_0)$, with probability at least $ 1 - \frac{2l}{m}$,
    \begin{equation*}
\left\|\dfrac{\partial\aalpha^{(l)}}{\partial\mathbf{w}^{(l)}}\right\|_2,\left\|\dfrac{\partial\aalpha^{(l)}}{\partial\mathbf{r}^{(l)}}\right\|_2
\leq         
\left(\gamma^{l-1} + |\phi(0)|\sum_{i=1}^{l-1}\gamma^{i-1}\right)
    \end{equation*}
where, $\mathbf{w}^{(l)} = \text{vec}(W^{(l)})$ for $l\in[L+1]$, and $\r^{(l)} = \text{vec}(R^{(l)})$ for $l\in\{2,\dots,L+1\}$. 
\end{lemm} 

    \proof
    We can index the vectors $\vw^{(l)}$ and $\r^{(l)}$ according to their matrix form $W^{(l)}_{jj'}$ and $R^{(l)}_{jj'}$, respectively, with the indices $j\in [m]$, and $j'\in [d]$ when $l=1$ or $j'\in [m]$ when $l\in \{2,\dots,L+1\}$. Therefore,
    \begin{equation*}
        \left[\frac{\partial \aalpha^{(l)}}{\partial \mathbf{w}^{(l)}}\right]_{i, j j^{\prime}} = \frac{1}{\sqrt{m}}\phi' (\widetilde{\aalpha}^{(l)}_i)\delta_{ij}\aalpha^{(l-1)}_{j'},
        \quad \delta_{ij} = \begin{cases}
            1 &{i=j}\\
            0 &\text{otherwise}
        \end{cases}.
    \end{equation*}
    Now, for $l\in \{2,\dots,L+1\}$, we can write the $L_2$-norm of the matrices as follows
    \begin{align*}
        \left\|\frac{\partial \aalpha^{(l)}}{\partial \mathbf{w}^{(l)}}\right\|_2^2
        &=
        \sup _{\|V\|_F=1} \frac{1}{m} \sum_{i=1}^m\left(\phi^{\prime}\left(\widetilde{\aalpha}_i^{(l)}\right) \sum_{j, j^{\prime}=1}^m \aalpha_{j^{\prime}}^{(l-1)} \delta_{ij} V_{j j^{\prime}}\right)^2\\
        &\leq \sup_{\|V\|_F = 1} \frac{1}{m} \| V\aalpha^{(l-1)}\|_2^2\\
        & \leq \sup_{\|V\|_F = 1}\frac{1}{m} \|V\|_2^2 \|\aalpha^{(l-1)}\|_2^2 \\ 
        & \overset{(a)}{\leq} \sup_{\|V\|_F = 1}\frac{1}{m} \|V\|_F^2 \|\aalpha^{(l-1)}\|_2^2 \\
        & = \frac{1}{m}\|\aalpha^{(l-1)}\|_2^2 \\
        & \overset{(b)}{\leq} \frac{1}{m} \left[ 
            \sqrt{m}\left(\gamma^{l-1}+|\phi(0)| \sum_{i=1}^{l-1} \gamma^{i-1}\right)
        \right]^2 = \left(\gamma^{l-1}+|\phi(0)| \sum_{i=1}^{l-1} \gamma^{i-1}\right)^2,
    \end{align*}
    where $(a)$ follows from the fact that $\|V\|_2 \leq \|V\|_F$ and $(b)$ from Lemma~\ref{lemm:TwoNormOutputFNOBlock}. The $l=1$ case follows in a similar fashion:
    \begin{equation*}
        \left\|\frac{\partial \aalpha^{(1)}}{\partial \mathbf{w}^{(1)}}\right\|_2^2 \leq 
        \frac{1}{d}\|\aalpha^{(0)} \|_2^2 = 1.
    \end{equation*}
    Similarly, for $l\in\{2,\dots,L+1\}$,
    \begin{align*}
        \left\|\frac{\partial \aalpha^{(l)}}{\partial \mathbf{r}^{(l)}}\right\|_2^2
        &=
        \sup _{\|V\|_F=1} \frac{1}{m} \sum_{i=1}^m\left(\phi^{\prime}\left(\widetilde{\aalpha}_i^{(l)}\right) 
        F^*_{ij}F_{j'p}\aalpha^{(l-1)}_p
        V_{j j^{\prime}}\right)^2\\
        &\leq \sup_{\|V\|_F = 1} \frac{1}{m} \| (F^*VF)\aalpha^{(l-1)}\|_2^2\\
        & \leq \sup_{\|V\|_F = 1}\frac{1}{m} \|F^*VF\|_2^2 \|\aalpha^{(l-1)}\|_2^2 \\ 
        & \leq \sup_{\|V\|_F = 1}\frac{1}{m} \|F^*\|^2_2 \|V\|_2^2 \|F\|_2^2  \|\aalpha^{(l-1)}\|_2^2 \\ 
        & \overset{(a)}{\leq} \sup_{\|V\|_F = 1}\frac{1}{m} \|V\|_F^2 \|\aalpha^{(l-1)}\|_2^2 \\
        & = \frac{1}{m}\|\aalpha^{(l-1)}\|_2^2 \\
        & \overset{(b)}{\leq} \frac{1}{m} \left[ 
            \sqrt{m}\left(\gamma^{l-1}+|\phi(0)| \sum_{i=1}^{l-1} \gamma^{i-1}\right)
        \right]^2 = \left(\gamma^{l-1}+|\phi(0)| \sum_{i=1}^{l-1} \gamma^{i-1}\right)^2,
    \end{align*}
    where $(a)$ follows again by $\|V\|_2 \leq \|V\|_F$ and the fact that $F^*$ and $F$ are unitary matrices, and $(b)$ from Lemma~\ref{lemm:TwoNormOutputFNOBlock}. This completes the proof. \qed


\textbf{Hessians.} We now focus on bounding the Hessian of the predictor $f$ in equation~\eqref{eq:FNO_predictor_app}. Note that the FNO model can be considered as having $L+1$ layers, with Layer 1 being a feedforward single layer encoder on top of the encoder $P$, the $L$ layers from Layer $2$ to Layer $L+1$ being FNO hidden layers, and Layer $L+2$ being the output of the linear decoder.
%
%
Likewise, we decompose the Hessian matrix $\mH$ of the FNO in three different blocks corresponding to the aforementioned encoder, FNO hidden layers, and decoder, respectively.

Firstly, the Hessian blocks associated to the hidden FNO layers are: 
%:  layers can be viewed as $2 \times 2$ block matrix, namely,
%\begin{align*}
%    H = \begin{bmatrix}
%        H_{w}^{(l_1, l_2)} & H_{w,r}^{(l_1, l_2)}\\
%        H_{r,w}^{(l_1, l_2)} & H_{r}^{(l_1, l_2)}
%    \end{bmatrix}
%\end{align*}
%where 
\begin{itemize}
\item the %$(1,1)$ block has 
$L \times L$ sub-blocks corresponding to $H_{w}^{(l_1,l_2)} := \frac{\partial^2 f}{\partial \w^{(l_1)} \partial \w^{(l_2)}}$ for $l_1, l_2 \in \{2,\ldots,L+1\}$, 
\item the %$(2,2)$ block has 
$L \times L$ sub-blocks corresponding to $H_{r}^{(l_1,l_2)} := \frac{\partial^2 f}{\partial \r^{(l_1)} \partial \r^{(l_2)}}$ for $l_1, l_2 \in \{2,\ldots,L+1\}$, and 
\item the 
%$(1,2)$ and $(2,1)$ 
cross blocks have terms of the form $H_{w,r}^{(l_1,l_2)} := \frac{\partial^2 f}{\partial \w^{(l_1)} \partial \r^{(l_2)}}$ for $l_1, l_2 \in \{2,\ldots,L+1\}$. 
\end{itemize}
%

Secondly, the Hessian blocks corresponding to the single layer encoder, i.e., with respect to weight $W^{(1)}$:
\begin{itemize}
\item diagonal block $H_{w}^{(1,1)} := \frac{\partial^2 f}{\partial \w^{{(1)}^2}}$,
\item off-diagonal blocks $H_{w}^{(1,l_1)} := \frac{\partial^2 f}{\partial \w^{(1)} \partial \w^{(l_1)}}$ and $H_{w}^{(l_1,1)}$ for $l_1 \in \{2,\ldots,L+1\}$, and 
\item off-diagonal blocks $H_{w,r}^{(1,l_2)} := \frac{\partial^2 f}{\partial \w^{(1)} \partial \r^{(l_2)}}$ ans $H_{r,w}^{(l_2,1)}$ for $l_2 \in \{2,\ldots,L+1\}$.
\end{itemize}

Finally, the Hessian blocks corresponding to the decoder, i.e., with respect to weight $\v$:
\begin{itemize}
\item diagonal block $H_{v} := \frac{\partial^2 f}{\partial \v^2}$, which is the zero matrix $\vzero_{m\times m}$,
\item off-diagonal block $H_{w,v}^{(l_1)} := \frac{\partial^2 f}{\partial \w^{(l_1)} \partial \v}$ and $H_{v,w}^{(l_1)}$ for $l_1 \in \{1,\ldots,L+1\}$, and
\item off-diagonal block $H_{r,v}^{(l_2)} := \frac{\partial^2 f}{\partial \r^{(l_2)} \partial \v}$ and $H_{v,r}^{(l_2)}$ for $l_2 \in \{2,\ldots,L+1\}$.
\end{itemize}


%Note that the definitions suitably generalize the equivalent quantities in \citep{liu_linearity_2021}[BCZB'23].

First, we note that due to the symmetry of the Hessian matrix of the FNO model $\mH$:
%that for the Hessian matrix of the FNO model $\mH$:
\begin{align}
\label{eq:Hessian_big}
    \| \mH \|_2 & \leq \sum_{l_1,l_2=1}^{L+1} \| H_{w}^{(l_1,l_2)} \|_2 + \sum_{l_1,l_2=2}^{L+1} \| H_{r}^{(l_1,l_2)} \|_2 + 2 \sum_{l_1=1}^{L+1} \sum_{l_2=2}^{L+1} \| H_{w,r}^{(l_1,l_2)} \|_2 
    + 2 \sum_{l_1=1}^{L+1} \| H_{w,v}^{(l_1)} \|_2 + 2 \sum_{l_2=2}^{L+1} \| H_{r,v}^{(l_2)} \|_2~.
\end{align}

%For convenience of notation, we assume that whenever we find the differential operator ``$\partial \r^{(1)}$'', the associated quantity to this operator will be equal to the zero matrix. 

We define
\begin{equation}
\label{eq:qwr2}
\begin{split}
\cQ_{\infty}(f) & := \max_{l \in [L+1]} ~\left\| \frac{\partial f}{\partial \aalpha^{(l)}} \right\|_{\infty}~,  \\
\cQ^{(w,r)}_2(f) & := \max_{l \in [L+1]} ~\left\{ \left\| \frac{\partial \aalpha^{(l)}}{\partial \w^{(l)}} \right\|_2, ~\left\| \frac{\partial \aalpha^{(l)}}{\partial \r^{(l)}} \right\|_2\right\} ~,  \\
\cQ^{(w,r)}_{2,2,1}(f) & := \max_{\substack{1\leq l_1 \leq L+1\\2\leq l_2 \leq L+1\\3\leq l_3 \leq L+1}}~ \left\{ 
\left\| \frac{\partial^2 \aalpha^{(l_2)}}{\partial \w^{(l_2)}  \partial \r^{(l_2)}} \right\|_{2,2,1} , 
\left\| \frac{\partial \aalpha^{(l_1)}}{\partial \w^{(l_1)}} \right\|_2  \left\| \frac{\partial^2 \aalpha^{(l_2)}}{\partial \aalpha^{(l_2-1)} \partial \r^{(l_2)}} \right\|_{2,2,1} ,\right. \\ 
& \left.\phantom{:= \max_{1\leq l_1 \leq l_2 \leq l_3 \leq L+1}~~~~~~ }\left\| \frac{\partial \aalpha^{(l_1)}}{\partial \r^{(l_1)}} \right\|_2  \left\| \frac{\partial^2 \aalpha^{(l_2)}}{\partial \aalpha^{(l_2-1)} \partial \w^{(l_2)}} \right\|_{2,2,1} ,\left\| \frac{\partial \aalpha^{(l_1)}}{\partial \w^{(l_1)}} \right\|_2  \left\| \frac{\partial \aalpha^{(l_2)}}{\partial \r^{(l_2)}} \right\|_2  \left\| \frac{\partial^2 \aalpha^{(l_3)}}{(\partial \aalpha^{(l_3-1)})^2} \right\|_{2,2,1}\right\} ~,\\
\cQ^{(w)}_{2,2,1}(f) & := \max_{\substack{1\leq l_1 \leq L+1\\2\leq l_2 \leq L+1\\3\leq l_3 \leq L+1}}~ \left\{ 
\left\| \frac{\partial^2 \aalpha^{(l_1)}}{(\partial \w^{(l_1)})^2} \right\|_{2,2,1} , 
\left\| \frac{\partial \aalpha^{(l_1)}}{\partial \w^{(l_1)}} \right\|_2  \left\| \frac{\partial^2 \aalpha^{(l_2)}}{\partial \aalpha^{(l_2-1)} \partial \w^{(l_2)}} \right\|_{2,2,1} ,\right.\\
&\left.\phantom{:= \max_{1\leq l_1 \leq l_2 \leq l_3 \leq L+1}~~~~~~ }\left\| \frac{\partial \aalpha^{(l_1)}}{\partial \w^{(l_1)}} \right\|_2  \left\| \frac{\partial \aalpha^{(l_2)}}{\partial \w^{(l_2)}} \right\|_2  \left\| \frac{\partial^2 \aalpha^{(l_3)}}{(\partial \aalpha^{(l_3-1)})^2} \right\|_{2,2,1}
\right\} ~,\\
\cQ^{(r)}_{2,2,1}(f) & := \max_{\substack{2\leq l_1 \leq L+1\\3\leq l_2 \leq L+1\\4\leq l_3 \leq L+1}}~ \left\{ 
\left\| \frac{\partial^2 \aalpha^{(l_1)}}{(\partial \r^{(l_1)})^2} \right\|_{2,2,1} , 
\left\| \frac{\partial \aalpha^{(l_1)}}{\partial \r^{(l_1)}} \right\|_2  \left\| \frac{\partial^2 \aalpha^{(l_2)}}{\partial \aalpha^{(l_2-1)} \partial \r^{(l_2)}} \right\|_{2,2,1} ,\right.\\
&\left.\phantom{:= \max_{1\leq l_1 \leq l_2 \leq l_3 \leq L+1}~~~~~~ }\left\| \frac{\partial \aalpha^{(l_1)}}{\partial \r^{(l_1)}} \right\|_2  \left\| \frac{\partial \aalpha^{(l_2)}}{\partial \r^{(l_2)}} \right\|_2  \left\| \frac{\partial^2 \aalpha^{(l_3)}}{(\partial \aalpha^{(l_3-1)})^2} \right\|_{2,2,1}
\right\}~,
\end{split}
\end{equation}
where, for an order-3 tensor $T \in \R^{d_1 \times d_2 \times d_3}$ we define the operator $\norm{\cdot}_{2,2,1}$ as follows, 
\begin{align}
\| T \|_{2,2,1} := \sup_{\|\a\|_2 = \|\b\|_2 = 1} \sum_{k=1}^{d_3} \left| \sum_{i=1}^{d_1} \sum_{j=1}^{d_2} T_{ijk} a_i b_j \right|~,~~\a \in \R^{d_1}, \b \in \R^{d_2}~.
\label{eq:norm-221}
\end{align}
Note that it seems from~\eqref{eq:qwr2} that we need the depth $L$ of the FNO to be $L\geq 3$. However, the bounds presented in Lemma~\ref{lemm:hessgradbounds-FNO} also hold for FNOS with depth $L<3$: indeed, the upper bounds we derive in this section for an FNO with depth $L$ will trivially hold for FNOS with depths $L-1,\dots,1$.

\begin{lemm}
\label{lemm:cross-w-r-bound}
    Under Assumptions~\ref{asmp:Activation_Function_FNO} and \ref{asmp:smoothinit_FNO} for $\vtheta \in B^{\mathrm{Euc}}_{\rho_w,\rho_r\rho_1} (\vtheta_0)$, the following inequalities hold with probability at least $1 - \frac{2(L+2)}{m}$, for $l_1\in[L+1]$,
    % \leq \beta_{\phi} \left(                 \gamma^{l_1 - 1} + |\phi(0)| \sum_{i=1}^{l_1 - 1} \gamma^{i-1}       \right)^2
    %
    \begin{equation}
    \label{eq:norm_d_alpha_l_w_l_w_l}\left\|\frac{\partial^2\aalpha^{{(l_1)}}}{(\partial{\mathbf{w}^{(l_1)}})^2} \right\|_{2,2,1} \leq \beta_{\phi}(1+\gamma^L)^2(1+L|\phi(0)|)^2~,
    \end{equation}
% \leq 
%        \beta_{\phi} \left( 
%                \gamma^2 + \left(
%                    \gamma^{l_1 - 1} + |\phi(0)| \sum_{i=1}^{l_1 - 1} \gamma^{i-1}
%                \right)^2
%            \right) + 1
%    
and for $l_2\in\{2,\dots,L+1\}$,
    \begin{equation}
        \label{eq:norm_d_alpha_l_w_l_r_l}
        \left\|\frac{\partial^2\aalpha^{{(l_2)}}}{\partial{\mathbf{w}^{(l_2)}}\partial \r^{(l_2)}} \right\|_{2,2,1} \leq
        \beta_{\phi}(1+\gamma^L)^2(1+L|\phi(0)|)^2        ~,
    \end{equation}
    %
    \begin{equation}
        \label{eq:norm_d_alpha_l_alpha_l_minus_1}
        \left\| \frac{\partial^2 \aalpha^{(l_2)}}{(\partial \aalpha^{(l_2-1)})^2} \right\|_{2,2,1} \leq 2\beta_\phi\gamma^2~,
    \end{equation}
    \begin{equation}
    \label{eq:norm_d_alpha_l_alpha_l_w_l}\left\|\frac{\partial^2\aalpha^{{(l_2)}}}{\partial \aalpha^{(l_2-1)}\partial{\mathbf{w}^{(l_2)}}} \right\|_{2,2,1} \leq \beta_{\phi}(1+\gamma^L)^2(1+(1+L|\phi(0)|)^2)+1~,
    \end{equation}
    \begin{equation}
        \label{eq:norm_d_alpha_l2_alpha_l2_minus_1_dr}
        \left\|\frac{\partial^2\aalpha^{{(l_2)}}}{\partial{\aalpha^{(l_2-1)}}\partial \r^{(l_2)}} \right\|_{2,2,1}
        \leq \beta_{\phi}(1+\gamma^L)^2(1+(1+L|\phi(0)|)^2)+1
        ~, \text{ and}
    \end{equation}
        \begin{equation}
    \label{eq:norm_d_alpha_l_r_l_r_l}\left\|\frac{\partial^2\aalpha^{{(l_2)}}}{(\partial{\mathbf{r}^{(l_2)}})^2} \right\|_{2,2,1} \leq \beta_{\phi}(1+\gamma^L)^2(1+L|\phi(0)|)^2~.
    \end{equation}
% \leq \beta_{\phi} \left( 
%            \gamma^2 + \left(
%                \gamma^{l_2 - 1} + |\phi(0)| \sum_{i=1}^{l_2 - 1} \gamma^{i-1}
%            \right)^2
%        \right) + 1    
    \end{lemm}
    
    \proof
    We first begin by proving \eqref{eq:norm_d_alpha_l_w_l_r_l}. Note that from \eqref{eq:FNO_predictor_app} we have
    \begin{align*}
        \frac{\partial^2 \aalpha_i^{\left(l_2\right)}}{\partial \w_{j j^{\prime}}^{\left(l_2\right)} \partial \r_{k k^{\prime}}^{\left(l_2\right)}}
        =
        \frac{1}{m} \phi^{\prime \prime}
        \left(\tilde{\aalpha}^{\left(l_2\right)}\right) \cdot \aalpha_{j^{\prime}}^{\left(l_2 - 1\right)} \delta_{i j} F_{i k}^* F_{k^{\prime} q} \aalpha_q^{\left(l_2-1\right)},
    \end{align*}
    where we make use of the Einstein notation. Now,
    \begin{align}
        \begin{aligned}
            &\left\| 
                \frac{\partial^2 \aalpha_i^{\left(l_2\right)}}{\partial \w^{\left(l_2\right)} \partial \r^{\left(l_2\right)}}
            \right\|_{2,2,1} \\
            &= 
            \sup_{\|V_1\|_F = 1, \|V_2\|_F = 1} \sum_{i=1}^m
            \left| 
                \frac{1}{m} \phi^{\prime \prime}(\tilde{\aalpha}^{(l_2)}_i) \aalpha_{j^{\prime}}^{\left(l_2 -1\right)} \delta_{i j} F_{i k}^* F_{k^{\prime} q} \aalpha_q^{\left(l_2-1\right)} V_{1_{j j^{\prime}}} V_{2_{k k^{\prime}}} 
            \right| \\ 
            &= 
            \sup_{\|V_1\|_F = 1, \|V_2\|_F = 1}\sum_{i=1}^m
            \left| 
            \frac{\phi^{\prime \prime}(\tilde{\aalpha}^{(l_2)}_i)}{m}
            \left(
                V_{1_{ij^{\prime}}} \aalpha^{(l_2 - 1)}_{j^{'}}
            \right) 
            \left(
                F_{i k}^* V_{2_{k k^{\prime}}} F_{k^{\prime} q} \aalpha_q^{\left(l_2{-1}\right)}\right)
            \right| \\
            &\leq
            \sup_{\|V_1\|_F = 1, \|V_2\|_F = 1}
            \frac{\beta_\phi}{m}
            \sum^m_{i=1}
            \left| 
            (V_1 \aalpha^{(l_2-1)})_i((F^*V_2F)\aalpha^{(l_2-1)})_i      \right| 
            \\
            &\overset{(a)}{\leq}
            \sup_{\|V_1\|_F = 1, \|V_2\|_F = 1}
            \frac{\beta_{\phi}}{2m} \left(\left\|V_1 \aalpha^{\left(l_2 - 1\right)}\right\|_2^2+\left\|F^* V_2 F \aalpha^{(l_2-1)}\right\|_2^2\right) \\
            &\overset{(b)}{\leq}
            \frac{\beta_{\phi}}{2m}\left(\left\|\aalpha^{\left(l_2 - 1\right)}\right\|_2^2+\left\|\aalpha^{\left(l_2-1\right)}\right\|_2^2\right) \leq \beta_{\phi}
                \left(\gamma^{l_2-1}+|\phi(0)| \sum_{i=1}^{l_2-1} \gamma^{i-1}\right)^2,
        \end{aligned}
    \end{align}
    where $(a)$ follows from the quadratic expression; where $(b)$ follows from $\|V_1\aalpha^{(l_2 - 1)}\|_{2} \leq \|V_1\|_2 \| \aalpha^{(l_2 - 1)}\|_2$, $\|V_1\|_2 \leq \|V_1\|_F$, $\|V_2\|_2 \leq \|V_2\|_F$,  
    $\|F^*V_2F\aalpha^{(l_2 - 1)}\|_{2} =\| V_2F \aalpha^{(l_2 - 1)} \|_2 \leq \|V_2\|_2 \| F\aalpha^{(l_2 - 1)}\|_2= \|V_2\|_2 \| \aalpha^{(l_2 - 1)}\|_2$ due to $F$ being a unitary operator; and where the last inequality follows from~\eqref{eq:alpha_l_2_norm_bound}. Finally, we can upper bound the last quantity above as in~\eqref{eq:norm_d_alpha_l_w_l_r_l} and complete the proof.

    For proving \eqref{eq:norm_d_alpha_l_alpha_l_minus_1}, again note from \eqref{eq:FNO_predictor_app} that
    \begin{align}
            \begin{aligned}
            \left[ \ddel{\aalpha^{(l_2)}}{\aalpha^{(l_2-1)}} \right]_{i,j,k} 
            &= \frac{1}{m}
            \phi''(\tilde{\aalpha}^{(l_2)}) \left( 
                W^{(l_2)}_{ij} + F^*_{ip}R^{(l_2)}_{pq}F_{qj}
            \right) \cdot \left(W^{(l_2)}_{ik} + F^*_{iu}R^{(l_2)}_{uv}F_{vk} \right) \\
            &= \frac{\phi''}{m} \left[ \underbrace{W^{(l_2)}_{ij} W^{(l_2)}_{ik}}_{T_1} + \underbrace{W^{(l_2)}_{ij}F^{*}_{iu}R^{(l_2)}_{uv}F_{vk}}_{T_2} + \underbrace{F^*_{ip}R^{(l_2)}_{pq}F_{qj}W^{(l_2)}_{ik}}_{T_3} +  \underbrace{F^*_{ip}R^{(l_2)}_{pq}F_{qj} F^{*}_{iu}R^{(l_2)}_{uv}F_{vk}}_{T_4} \right].
            \end{aligned}
    \end{align}
    Then, we can write
    \begin{align*}
        \left\| \ddel{\aalpha^{(l_2)}}{\aalpha^{(l_2-1)}}  \right\|_{2,2,1}=
        \sup_{\|\vv_1\|_2 = 1,\|\vv_2\|_2 = 1} 
        \sum_{i=1}^m\left| \left[ 
            \ddel{\aalpha^{(l_1 )}}{\aalpha^{(l_1 - 1)}}
        \right]_{i,j,k}v_{1_j}v_{2_k}\right|.
    \end{align*}
    Let us consider the notation $\gamma_w = \sigma_{1,w}+\frac{\rho_w}{\sqrt{m}}$ and $\gamma_r = \sigma_{1,r}+\frac{\rho_r}{\sqrt{m}}$.
    Now, we handle each of the terms separately:
    \begin{align}
    \label{eq:T1_norm_d2_alpha2_dalpha2}
        \begin{aligned}
            \sup_{\|\vv_1\|_2 = 1,\|\vv_2\|_2 = 1} \sum_{i=1}^m \left|\frac{\phi''}{m}T_{1_{i,j,k}}v_{1_j}v_{2_k}\right| 
            &= \frac{|\phi''|}{m} \sup_{\|\vv_1\|_2 = 1,\|\vv_2\|_2 = 1} \sum_{i=1}^m
            \left|\left( W^{(l_2)}_{ij} v_{1_j}\right)\cdot \left( W^{(l_2)}_{ik} v_{2_k}\right)\right|\\
            &\leq \frac{\beta_{\phi}}{2m}\sup_{\|\vv_1\|_2 = 1,\|\vv_2\|_2 = 1}\left(\|W^{(l_2)}\|_2^2 \|\vv_1\|_2^2 + \|W^{(l_2)}\|_2^2 \|\vv_2\|_2^2 \right) \\
            &=\beta_{\phi}\left( \sigma_{1,w} + \frac{\rho_w}{\sqrt{m}}\right)^2 = \beta_{\phi} \gamma_w^2.
        \end{aligned}
    \end{align}
    \begin{align}
    \label{eq:T4_norm_d2_alpha2_dalpha2}
        \begin{aligned}
            \sup_{\|\vv_1\|_2 = 1,\|\vv_2\|_2 = 1} \sum_{i=1}^m\left| \frac{\phi''}{m}T_{4_{i,j,k}}v_{1_j}v_{2_k} \right|
            &= \frac{|\phi''|}{m} \sup_{\|\vv_1\|_2 = 1,\|\vv_2\|_2 = 1} \sum_{i=1}^m
            \left| 
            \left((F^*R^{(l_2)}F)_{ij}v_{1_j} \right)\cdot \left((F^*R^{(l_2)}F)_{ik}v_{2_k} \right)
            \right|\\
            &\leq \frac{\beta_{\phi}}{2m} \sup_{\|\vv_1\|_2 = 1,\|\vv_2\|_2 = 1} \left( 
                \|F^*R^{(l_2)}F\|_2^2 \|\vv_1\|_2^2 + \|F^*R^{(l_2)}F\|_2^2 \|\vv_2\|_2^2
            \right)\\
            &=\frac{\beta_{\phi}}{m} \| F^*R^{(l_2)}F\|_2^2\\
            &\leq \frac{\beta_{\phi}}{m}\|R^{(l_2)}\|_2^2 \leq \beta_{\phi}\gamma^2_r.
        \end{aligned}
    \end{align}
    \begin{align}
    \label{eq:T2_norm_d2_alpha2_dalpha2}
        \begin{aligned}
            \sup_{\|\vv_1\|_2 = 1,\|\vv_2\|_2 = 1} \sum_{i=1}^m \left|\frac{\phi''}{m}T_{2_{i,j,k}}v_{1_j}v_{2_k}\right|
            &= \frac{|\phi''|}{m} \sup_{\|\vv_1\|_2 = 1,\|\vv_2\|_2 = 1} \sum_{i=1}^m
            \left|
                (W^{(l_2)}_{ij}v_{1_j})\cdot (F^*R^{(l_2)}F)_{ik}v_{2_k}
            \right|\\
            &\leq \frac{\beta_{\phi}}{2m} \sup_{\|\vv_1\|_2 = 1,\|\vv_2\|_2 = 1}
            \left( \| W^{(l_2)}\|_2^2 \| \vv_1\|_2^2 + \| F^*R^{(l_2)}F\|_2^2 \| \vv_2\|_2^2 \right)\\
            &\leq\frac{\beta_{\phi}}{2m}\left( \|W^{(l_2)}\|_2^2 + \|R^{(l_2)}\|_2^2\right) \leq \frac{\beta_{\phi}}{2} \left( \gamma_w^2 + \gamma_r^2\right).
        \end{aligned}
    \end{align}
    Similarly, for the term corresponding to $T_3$ we obtain
    \begin{align}
    \label{eq:T3_norm_d2_alpha2_dalpha2}
        \begin{aligned}
            \sup_{\|\vv_1\|_2 = 1,\|\vv_2\|_2 = 1} \sum_{i=1}^m \left|\frac{\phi''}{m}T_{3_{i,j,k}}v_{1_j}v_{2_k}\right| \leq \frac{\beta_{\phi}}{2} \left( \gamma_w^2 + \gamma_r^2\right)~.
        \end{aligned}
    \end{align}
    Putting together \eqref{eq:T1_norm_d2_alpha2_dalpha2}, \eqref{eq:T4_norm_d2_alpha2_dalpha2}, \eqref{eq:T2_norm_d2_alpha2_dalpha2} and \eqref{eq:T3_norm_d2_alpha2_dalpha2}, we get
    \begin{align}
        \left\| \ddel{\aalpha^{(l_2)}}{\aalpha^{(l_2-1)}}\right\|_{2,2,1}^2 \leq 2 \beta_{\phi} (\gamma_w^2 + \gamma_r^2) \leq 2\beta_\phi (\gamma_w^2 + \gamma_r^2 + 2\gamma_w\gamma_r) = 2\beta_\phi\gamma^2.
    \end{align}
    This completes the proof for \eqref{eq:norm_d_alpha_l_alpha_l_minus_1}. 
    
    We now look at the proof for \eqref{eq:norm_d_alpha_l2_alpha_l2_minus_1_dr}. First note that
    \begin{align*}
        \begin{aligned}            
            \frac{\partial^2 \aalpha^{\left(l_2\right)}_i}{\partial \aalpha^{\left(l_2-1\right)}_{k} \partial \r_{j j^{\prime}}^{\left(l_2\right)}}
        &=
        \frac{1}{m}\phi''(\tilde{\aalpha_i}) \left(W_{i k}^{\left(l_2\right)}
        +
        F_{i p}^* R_{p q}^{\left(l_2\right)} F_{q k}\right) F_{i j}^{*} F_{j' q} \aalpha_q^{\left(l_2-1\right)} + \frac{1}{\sqrt{m}}\phi' (\tilde{\aalpha}_i^{(l_2)}) F^{*}_{ij}F_{j'k} \\
        &= \underbrace{\frac{\phi''}{m} \left(W_{i k}^{\left(l_2\right)}F_{i j}^{*} F_{j' q} \aalpha_q^{\left(l_2-1\right)} \right)}_{T_1}
        + \underbrace{\frac{\phi''}{m}
        \left( 
            F_{i p}^* R_{p q}^{\left(l_2\right)} F_{q k}F_{i j}^{*} F_{j'q}\aalpha_q^{\left(l_2-1\right)}
        \right)}_{T_2}
        + \underbrace{\frac{1}{\sqrt{m}}\phi' (\tilde{\aalpha}_i^{(l_2)}) F^{*}_{ij}F_{j'k}}_{T_3}.
        \end{aligned}
    \end{align*}
    Again, we analyze each of the terms separately
    \begin{align}
    \label{eq:T1_norm_d2_alpha_d_alpha_d_r}
        \begin{aligned}
            \left\|T_{1_{i,jj',k}} \right\|_{2,2,1} 
            &= \sup_{\|\vv_1\|_2 = 1, \|V_2\|_F =1} \sum_{i=1}^m \left| 
                \frac{\phi''}{m} \left(W_{i k}v_{1_k}^{\left(l_2\right)}F_{i j}^{*}V_{2_{jj'}} F_{j' q} \aalpha_q^{\left(l_2-1\right)} \right)
            \right|\\
            &\leq \frac{\beta_{\phi}}{2m} \left( 
                \|W^{(l_2)}\vv_1 \|_{2}^2 + \|F^*V_2F\aalpha^{(l_2 - 1)}\|_{2}^2 
            \right)\\
            &\leq \frac{\beta_{\phi}}{2}\left( 
                \gamma_w^2 + \left( \gamma^{l_2 - 1} + |\phi(0)|\sum_{i=1}^{l_2 - 1} \gamma^{i-1}\right)^2
            \right)
        \end{aligned}
    \end{align}
    % \begin{align}
    %     \begin{aligned}
    %         \left\|\frac{\partial^2 \aalpha_i^{\left(l_2\right)}}{\partial \aalpha_k^{\left(l_2-1\right)} \partial r_{\left(j^{\prime}\right)}^{\left(l_2\right)}}\right\|_{2,2,1} = 
    %         \sup_{\|\vv_1\|_2 = 1, \|V_2\|_F =1} \sum_{i=1}^m \left(\left| \frac{\phi''}{m}
    %         \left( 
    %             W^{(l_2)}_{ik}v_{1_k} \cdot F^*_{ij}F_{j'q}\aalpha^{(l_2 - 1)}_q V_{2_{jj'}}
    %         \right) \right|
    %         + \left| \frac{\phi'}{\sqrt{m}} F^*_{ij}F_{j'k}v_{1_k}V_{2_{jj'}}\right|\right).
    %     \end{aligned}
    % \end{align}
    \begin{align}
    \label{eq:T2_norm_d2_alpha_d_alpha_d_r}
        \begin{aligned}
            \left\|T_{2_{i,jj',k}} \right\|_{2,2,1} 
            &= \sup_{\|\vv_1\|_2 = 1, \|V_2\|_F =1} \sum_{i=1}^m \left| 
                \frac{\phi''}{m} \left(
                    F^*_{ip}R^{(l_2)}_{pq}F_{qk}v_{1_k} F^{*}_{ij}V_{2_{jj'}}F_{j'q}\aalpha^{(l_2 - 1)}_q
                \right)
            \right|\\
            &\leq \sup_{\|\vv_1\|_2 = 1, \|V_2\|_F =1} 
            \frac{\beta_{\phi}}{2m}\left( 
                \| F^*R^{(l_2)}F\vv_1\|_2^2 + \| F^*V_2F\aalpha^{(l_2 - 1)}\|_2^2
            \right)\\
            &\leq \sup_{\|\vv_1\|_2 = 1, \|V_2\|_F =1} 
            \frac{\beta_{\phi}}{2m} \left( 
                \| F^*R^{(l_2)}F\|_2^2 \|\vv_1\|_2^2 +  \| F^*V_2F\|_2^2 \| \aalpha^{(l_2 - 1)}\|^2_{2}
            \right)\\
            &\overset{(a)}{\leq}
            \frac{\beta_{\phi}}{2m} \left( 
                \| R^{(l_2)}\|_2^2 +  \| \aalpha^{(l_2 - 1)}\|^2_{2}
            \right) \leq \frac{\beta_{\phi}}{2}\left( 
                \gamma_r^2 + \left( \gamma^{l_2 - 1} + |\phi(0)|\sum_{i=1}^{l_2 - 1} \gamma^{i-1}\right)^2
            \right)
        \end{aligned}
    \end{align}
    where $(a)$ follows, again, by exploiting the isometry of $F^*$ and $F$ with respect to the $L_2$ norm, and using $\|V_2\|_2 \leq \|V_2\|_F$. Finally,
    \begin{align}
    \label{eq:T3_norm_d2_alpha_d_alpha_d_r}
        \begin{aligned}
            \left\|T_{3_{i,jj',k}} \right\|_{2,2,1} 
            &= \sup_{\|\vv_1\|_2 = 1, \|V_2\|_F =1} \sum_{i=1}^m \left| 
                \frac{\phi'}{\sqrt{m}} F^{*}_{ij}V_{2_{jj'}}F_{j'k}v_{1_k}
            \right|\\
            &\leq
            \frac{1}{\sqrt{m}} \sup_{\|\vv_1\|_2 = 1, \|V_2\|_F =1}\sum^m_{i=1}\left|  
            (F^*V_2F\vv_1)_i
            \right|\\
            &\leq \sup_{\|\vv_1\|_2 = 1, \|V_2\|_F =1}
                \|F^*V_2F\vv_1 \|_2\\ 
            &\leq \sup_{\|\vv_1\|_2 = 1, \|V_2\|_F =1} 
            \|V_{2}\|_2\norm{\vv_1}_2 
            = 1~.
        \end{aligned}
    \end{align}
Combining \eqref{eq:T1_norm_d2_alpha_d_alpha_d_r}, \eqref{eq:T2_norm_d2_alpha_d_alpha_d_r} and \eqref{eq:T3_norm_d2_alpha_d_alpha_d_r}, we get
    \begin{align}
        \begin{aligned}
                \left\|
                \frac{\partial^2 \aalpha^{\left(l_2\right)}}{\partial \aalpha^{\left(l_2-1\right)} \partial \r^{\left(l_2\right)}}
            \right\|_{2,2,1} 
            &\leq \frac{\beta_{\phi}}{2} \left( 
                \gamma_w^2 + \gamma_r^2
            \right) + \beta_{\phi} \left( \gamma^{l_2 - 1} + |\phi(0)|\sum_{i=1}^{l_2 - 1} \gamma^{i-1}\right)^2 + 1\\
            &\leq \beta_{\phi} \left( 
                \gamma^2 + \left(
                    \gamma^{l_2 - 1} + |\phi(0)| \sum_{i=1}^{l_2 - 1} \gamma^{i-1}
                \right)^2
            \right) + 1~,
        \end{aligned}
    \end{align}
and finally we can upper bound the last quantity above as in~\eqref{eq:norm_d_alpha_l2_alpha_l2_minus_1_dr} and complete the proof.
%
%

For proving~\eqref{eq:norm_d_alpha_l_alpha_l_w_l} consider the following
\begin{align*}
        \begin{aligned}
            \left[ 
                \frac{\partial^2 \aalpha^{(l_2)}}{\partial \aalpha^{(l_2 - 1)}\partial \vw^{(l_2)}}
            \right]_{i,jj',k} =\left(
            \underbrace{\frac{\phi''(\widetilde{\aalpha}^{(l_2)}_i)  }{m}
                W^{(l_2)}_{ik}\aalpha^{(l_2-1)}_{j'}\delta_{ij}}_{T_1}
                + 
                \underbrace{\frac{\phi''(\widetilde{\aalpha}^{(l_2)}_i)}{m} F^*_{ip}R^{(l_2)}_{pq}F_{qk}\aalpha^{(l_2-1)}_{j'}\delta_{ij}}_{T_2}
            \right)
            + \underbrace{\frac{1}{\sqrt{m}} \phi'(\widetilde{\aalpha}^{(l_2)})\delta_{ij}\delta_{kj'}}_{T_3}
        \end{aligned}
    \end{align*}
    Then analyzing each term separately, we get
    \begin{align}
        \begin{aligned}
            \left\|T_{1_{i,jj',k}} \right\|_{2,2,1} 
            &= \sup_{\|\vv_1\|_2 = 1, \|V_2\|_F =1} \sum_{i=1}^m
            \left| 
                \frac{\phi''}{m} W^{(l_2)}_{ik}v_{1_k} V_{2_{ij'}}\aalpha^{(l_2-1)}_{j'}
            \right|\\
            &\leq \sup_{\|\vv_1\|_2 = 1, \|V_2\|_F =1}
            \frac{\beta_{\phi}}{2m} 
            \left( 
                \|W^{(l_2)}\|_2^2 \|\vv_1\|_2^2 + \|V_2\|_2^2 \|\aalpha^{(l_2-1)}\|_2^2
            \right)\\
            &=
            \frac{\beta_{\phi}}{2m} 
            \left( 
                \|W^{(l_2)}\|_2^2 + \|\aalpha^{(l_2-1)}\|_2^2
            \right) \leq\frac{\beta_{\phi}}{2}  
                \left(\gamma_w^2 + \left(\gamma^{l_1-1}+|\phi(0)| \sum_{i=1}^{l_1-1} \gamma^{i-1}\right)^2\right),
        \end{aligned}
    \end{align}
    \begin{align}
        \begin{aligned}
            \left\|T_{2_{i,jj',k}} \right\|_{2,2,1} 
            &= \sup_{\|\vv_1\|_2 = 1, \|V_2\|_F =1} \sum_{i=1}^m
            \left| 
                \frac{\phi''}{m} F^*_{ip}R^{(l_2)}_{pq}F_{qk}v_{1_k} V_{2_{ij'}}\aalpha^{(l_2-1)}_{j'}
            \right|\\
             &\leq \sup_{\|\vv_1\|_2 = 1, \|V_2\|_F =1}
             \frac{\beta_{\phi}}{2m}\left( 
                \| F^*R^{(l_2)}F\vv_1\|_2^2 + \| V_2\aalpha^{(l_2-1)}\|_2^2 
             \right)\\
             &\leq \sup_{\|\vv_1\|_2 = 1, \|V_2\|_F =1}
             \frac{\beta_{\phi}}{2m}\left( 
                \| F^*R^{(l_2)}F\|_2^2\norm{\vv_1}_2^2 + \norm{V_2}_2^2\| \aalpha^{(l_2-1)}\|_2^2 
             \right)\\
             &=            \frac{\beta_{\phi}}{2m}\left( 
                \| F^*R^{(l_2)}F\|_2^2 + \| \aalpha^{(l_2-1)}\|_2^2 
             \right)\\
             &\leq \frac{\beta_{\phi}}{2}  
                \left(\gamma_r^2 + \left(\gamma^{l_1-1}+|\phi(0)| \sum_{i=1}^{l_1-1} \gamma^{i-1}\right)^2\right),
        \end{aligned}
    \end{align}
    and, finally,
    \begin{align}
        \begin{aligned}
            \left\|T_{3_{i,jj'k}} \right\|_{2,2,1} 
            &= \sup_{\|\vv_1\|_2 = 1, \|V_2\|_F =1} \sum_{i=1}^m
            \left| 
                \frac{\phi'}{\sqrt{m}}V_{2_{ik}}v_{1_k}
            \right|\\
            &\leq \sup_{\|\vv_1\|_2 = 1, \|V_2\|_F =1} \sum_{i=1}^m
            \frac{1}{\sqrt{m}}\|\vv_1\|_2 \|V_{2,_{i,:}}\|_2\\
            &\leq \sup_{\|V_2\|_F =1}\sqrt{\sum_{i=1}^m
            \|V_{2,_{i,:}}\|_2^2}\\
            &= 1~.
        \end{aligned}
    \end{align}
    Hence, we have
    \begin{align}
        \begin{aligned}
            \left\|\frac{\partial^2\aalpha^{{(l_2)}}}{\partial \aalpha^{(l_2-1)}\partial{\mathbf{w}^{(l_2)}}} \right\|_{2,2,1}^2
            &\leq 
            \frac{\beta_{\phi}}{2}\left( \gamma_w^2 + \gamma_r^2\right) + \beta_{\phi} \left( 
                \gamma^{l_1 - 1} + |\phi(0)| \sum_{i=1}^{l_1 - 1} \gamma^{i-1}
        \right)^2+1\\
        &\leq 
            \beta_{\phi}\left( \gamma^2 +
                \left(\gamma^{l_1 - 1} + |\phi(0)| \sum_{i=1}^{l_1 - 1} \gamma^{i-1}\right)^2
        \right)+1~,
        \end{aligned}
    \end{align}
and finally we can upper bound the last quantity above as in~\eqref{eq:norm_d_alpha_l_alpha_l_w_l}.

We now focus on proving~\eqref{eq:norm_d_alpha_l_w_l_w_l}. Note that from \eqref{eq:FNO_predictor_app} we have
    \begin{align*}
        \left[\frac{\partial^2 \aalpha^{\left(l_1\right)}}{(\partial \w^{\left(l_1\right)})^2}\right]_{i,jj',kk'}
        =
        \frac{1}{m} \phi^{\prime \prime}
        \left(\tilde{\aalpha}^{\left(l_1\right)}\right) \cdot \aalpha_{j^{\prime}}^{\left(l_1 - 1\right)}
        \aalpha_{k^{\prime}}^{\left(l_1 - 1\right)}
        \delta_{i j}\delta_{ik}.
    \end{align*}
Now,
    \begin{align}
        \begin{aligned}
            &\left\| 
                \frac{\partial^2 \aalpha^{\left(l_1\right)}}{(\partial \w^{\left(l_1\right)})^2}
            \right\|_{2,2,1} \\
            &= 
            \sup_{\|V_1\|_F = 1, \|V_2\|_F = 1} \sum_{i=1}^m
            \left| 
                \frac{1}{m} \phi^{\prime \prime}(\tilde{\aalpha}^{(l_1)}_i) 
                V_{1_{i j^{\prime}}}\aalpha_{j^{\prime}}^{\left(l_1 -1\right)}
                V_{2_{i k^{\prime}}}\aalpha_{k^{\prime}}^{\left(l_1 -1\right)} 
            \right| \\ 
            &\leq
            \sup_{\|V_1\|_F = 1, \|V_2\|_F = 1}
            \frac{\beta_\phi}{m}
            \sum^m_{i=1}
            \left| 
            (V_1 \aalpha^{(l_1-1)})_i(V_2\aalpha^{(l_1-1)})_i \right| 
            \\
            &\leq
            \sup_{\|V_1\|_F = 1, \|V_2\|_F = 1}
            \frac{\beta_{\phi}}{2m} \left(\left\|V_1 \aalpha^{\left(l_1 - 1\right)}\right\|_2^2+\left\|V_2\aalpha^{(l_1-1)}\right\|_2^2\right) \\
            &\leq
            \frac{\beta_{\phi}}{m}\left\|\aalpha^{\left(l_1 - 1\right)}\right\|_2^2 \leq \beta_{\phi}
                \left(\gamma^{l_1-1}+|\phi(0)| \sum_{i=1}^{l_1-1} \gamma^{i-1}\right)^2.
        \end{aligned}
    \end{align}
Finally, we can upper bound the last quantity above as in~\eqref{eq:norm_d_alpha_l_w_l_w_l} and complete the proof.

Now, the last result to prove is~\eqref{eq:norm_d_alpha_l_r_l_r_l}. Note that from \eqref{eq:FNO_predictor_app} we have
    \begin{align*}
        \left[\frac{\partial^2 \aalpha^{\left(l_2\right)}}{(\partial \r^{\left(l_2\right)})^2}\right]_{i,jj',kk'}
        =
        \frac{1}{m} \phi^{\prime \prime}\left(\tilde{\aalpha}^{\left(l_2\right)}\right) \cdot 
        F_{i j}^* F_{j^{\prime} p} \aalpha_p^{\left(l_2-1\right)} 
        F_{i k}^* F_{k^{\prime} q} \aalpha_q^{\left(l_2-1\right)},
    \end{align*}
where we make use of the Einstein notation. Now,
    \begin{align}
        \begin{aligned}
            &\left\| 
                \frac{\partial^2 \aalpha^{\left(l_2\right)}}{(\partial \r^{\left(l_2\right)})^2}
            \right\|_{2,2,1} \\
            &= 
            \sup_{\|V_1\|_F = 1, \|V_2\|_F = 1} \sum_{i=1}^m
            \left| 
                \frac{1}{m} \phi^{\prime \prime}(\tilde{\aalpha}^{(l_2)}_i) F_{i j}^* V_{1_{j j^{\prime}}} F_{j^{\prime} p} \aalpha_p^{\left(l_2-1\right)}  
                F_{i k}^* V_{2_{k k^{\prime}}} F_{k^{\prime} q} \aalpha_q^{\left(l_2-1\right)}
            \right| \\ 
            &\leq
            \sup_{\|V_1\|_F = 1, \|V_2\|_F = 1}
            \frac{\beta_\phi}{m}
            \sum^m_{i=1}
            \left| 
            ((F^*V_1F)\aalpha^{(l_2-1)})_i ((F^*V_2F)\aalpha^{(l_2-1)})_i      \right| 
            \\
            &\leq
            \sup_{\|V_1\|_F = 1, \|V_2\|_F = 1}
            \frac{\beta_{\phi}}{2m} \left(\left\|F^* V_1 F \aalpha^{\left(l_2 - 1\right)}\right\|_2^2+\left\|F^* V_2 F \aalpha^{(l_2-1)}\right\|_2^2\right) \\
            &\leq
            \frac{\beta_{\phi}}{m}\left\|\aalpha^{\left(l_2 - 1\right)}\right\|_2^2 \leq \beta_{\phi}
                \left(\gamma^{l_2-1}+|\phi(0)| \sum_{i=1}^{l_2-1} \gamma^{i-1}\right)^2.
        \end{aligned}
    \end{align}
Finally, we can upper bound the last quantity above as in~\eqref{eq:norm_d_alpha_l_r_l_r_l} and complete the proof.
%    
%    This completes the proof of \eqref{eq:norm_d2_alpha_dw_dalpha}.
%    
%This completes the proof.
\qed

Now we upper bound the terms in equation~\eqref{eq:qwr2}. Thus, we obtain that, with probability at least $1-\frac{2(L+2)}{m}$,
\begin{equation}
\label{eq:Q_1}
 \cQ_{\infty}(f)=\max_{l\in[L+1]}\left\| \frac{\partial f}{\partial \aalpha^{(l)}} \right\|_{\infty}\overset{(a)}{\leq}
 \max_{l\in[L+1]}\frac{1}{\sqrt{m}}\gamma^{L+1-l}(1+\rho_1)\leq
 \frac{1}{\sqrt{m}}(1+\gamma^{L})(1+\rho_1)~,
\end{equation}
%
%
\begin{equation}
\label{eq:Q_2}
\cQ^{(w,r)}_2(f)\overset{(b)}{\leq}\max_{l\in[L+1]}(\gamma^{l-1}+|\phi(0)|\sum^{l-1}_{i=1}\gamma^{i-1})
\leq (1+\gamma^L)(1+L|\phi(0)|)~,
\end{equation}
and 
%
\begin{equation}
\label{eq:Q_3}
\cQ^{(w,r)}_{2,2,1}(f),\cQ^{(w)}_{2,2,1}(f),\cQ^{(r)}_{2,2,1}(f)  \overset{(c)}{\leq} 
2\beta_{\phi}(1+\gamma^L)^2(1+\gamma)^2(1+(1+L|\phi(0)|)^2)+1~,
\end{equation}
%
where (a) follows from a direct adaptation of the results from Section~A.6 in~\cite{banerjee2022restricted}, (b) follows from Lemma~\ref{lemm:gradient_alpha_params}, and (c) follows from Lemma~\ref{lemm:cross-w-r-bound}.

We now proceed to analyze the Hessian. We also recall that $\max_{l \in \{2,\dots,L+1\}} \left\| \frac{\partial \aalpha^{(l)}}{\partial \aalpha^{(l-1)}} \right\|_2 \leq \gamma$ from Lemma~\ref{lemm:firstDerivativeBoundFNO}. 

%====
We introduce some notation. Given an order-3 tensor $T \in \R^{d_1 \times d_2 \times d_3}$, we have that its first dimension has $d_1$ entries, the second has $d_2$ entries, and the third has $d_3$ entries. Consider the matrices $X\in\R^{k_1\times d_1}$, $Y\in\R^{k_2\times d_2}$, and $Z\in\R^{k_3\times d_3}$. We use the notation $(X)(Y)T(Z)\in\R^{k_1\times k_2\times k_3}$ to denote $X$ multiplying $A$ long its first dimension, $Y$ along its second dimension, and $Z$ along its third dimension. We use the notation $(X)T(Z)\in\R^{k_1\times d_2\times k_3}$ to denote $X$ multiplying $A$ long its first dimension and $Z$ along its third dimension.
%====



\paragraph{Off-Diagonal Blocks.} For the off-diagonal blocks, we focus on bounding $\| H_{w,r}^{(l_1,l_2)} \|_2$ for (Case 1.A) $l_1 \leq l_2$, (Case 1.B) $l_2 \leq l_1$. Further, we bound (Case 2.A) $\| H_{v,w}^{(l_1)} \|_2$ and (Case 2.B) $\| H_{v,r}^{(l_2)} \|_2$.


{\bf Case 1.A:} $2 \leq l_1 \leq l_2 \leq L+1$. By building on the form of the gradient, we have 
\begin{align*}
H_{w,r}^{(l_1,l_2)} & = \frac{\partial^2 \aalpha^{(l_1)}}{\partial \w^{(l_1)} \partial \r^{(l_1)}} \frac{\partial f}{\partial \aalpha^{(l_1)}} \1_{[l_1 =l_2]} +  \1_{[l_1 <l_2]}\left( \frac{\partial \aalpha^{(l_1)}}{\partial \w^{(l_1)}}  \prod_{l'=l_1+1}^{l_2-1} \frac{\partial \aalpha^{(l')}}{\partial \aalpha^{(l'-1)}} \right) \frac{\partial^2 \aalpha^{(l_2)}}{\partial \aalpha^{(l_2-1)} \partial \r^{(l_2)}} \left(\frac{\partial f}{\partial \aalpha^{(l_2)}} \right) \\
& ~~~~ + \1_{[l_1 <l_2]}\sum_{l=l_2+1}^{L+1} \left( \frac{\partial \aalpha^{(l_1)}}{\partial \w^{(l_1)}} \prod_{l'=l_1+1}^{l-1} \frac{\partial \aalpha^{(l')}}{\partial \aalpha^{(l'-1)}} \right)  \left( \frac{\partial \aalpha^{(l_2)}}{\partial \r^{(l_2)}} \prod_{l'=l_2+1}^{l-1} \frac{\partial \aalpha^{(l')}}{\partial \aalpha^{(l'-1)}} \right) \frac{\partial^2 \aalpha^{(l)}}{(\partial \aalpha^{(l-1)})^2}  \left(\frac{\partial f}{\partial \aalpha^{(l)}} \right) ~. 
\end{align*}
Then,
\begin{align*}
\| H_{w,r}^{(l_1,l_2)} \|_2 
& \leq  \left\| \frac{\partial^2 \aalpha^{(l_1)}}{\partial \w^{(l_1)}  \partial \r^{(l_1)}} \right\|_{2,2,1} \left\| \frac{\partial f}{\partial \aalpha^{(l_1)}} \right\|_{\infty}\1_{[l_1=l_2]} \\
& ~~~ +\1_{[l_1 <l_2]}  \left\| \frac{\partial \aalpha^{(l_1)}}{\partial \w^{(l_1)}} \right\|_2 \prod_{l'=l_1+1}^{l_2-1} \left\| \frac{\partial \aalpha^{(l')}}{\partial \aalpha^{(l'-1)}} \right\|_2 \left\| \frac{\partial^2 \aalpha^{(l_2)}}{\partial \aalpha^{(l_2-1)} \partial \r^{(l_2)}} \right\|_{2,2,1} \left\| \frac{\partial f}{\partial \aalpha^{(l_2)}} \right\|_{\infty} \\
& ~~~~ + \1_{[l_1 <l_2]}\sum_{l=l_2+1}^{L+1} \left( \left\| \frac{\partial \aalpha^{(l_1)}}{\partial \w^{(l_1)}} \right\|_2  \prod_{l'=l_1+1}^{l-1} \left\| \frac{\partial \aalpha^{(l')}}{\partial \aalpha^{(l'-1)}} \right\|_2 \right)  \left( \left\| \frac{\partial \aalpha^{(l_2)}}{\partial \r^{(l_2)}} \right\|_2 \prod_{l'=l_2+1}^{l-1} \left\| \frac{\partial \aalpha^{(l')}}{\partial \aalpha^{(l'-1)}} \right\|_{2} \right)\\
& ~~~~ \times\left\| \frac{\partial^2 \aalpha^{(l)}}{(\partial \aalpha^{(l-1)})^2} \right\|_{2,2,1} \left\| \frac{\partial f}{\partial \aalpha^{(l)}} \right\|_{\infty} \\
& \leq  \left\| \frac{\partial^2 \aalpha^{(l_1)}}{\partial \w^{(l_1)}  \partial \r^{(l_1)}} \right\|_{2,2,1} \left\| \frac{\partial f}{\partial \aalpha^{(l_1)}} \right\|_{\infty} \1_{[l_1 =l_2]}\\
& ~~~~ + \1_{[l_1 <l_2]} \gamma^{l_2-l_1-1} \left\| \frac{\partial \aalpha^{(l_1)}}{\partial \w^{(l_1)}} \right\|_2 \left\| \frac{\partial^2 \aalpha^{(l_2)}}{\partial \aalpha^{(l_2-1)} \partial \r^{(l_2)}} \right\|_{2,2,1} \left\| \frac{\partial f}{\partial \aalpha^{(l_2)}} \right\|_{\infty} \\
& ~~~~ + \1_{[l_1 <l_2]}\sum_{l=l_2+1}^{L+1} \gamma^{2l - l_2 - l_1-2} \left\| \frac{\partial \aalpha^{(l_1)}}{\partial \w^{(l_1)}} \right\|_2  \left\| \frac{\partial \aalpha^{(l_2)}}{\partial \r^{(l_2)}} \right\|_2  \left\| \frac{\partial^2 \aalpha^{(l)}}{(\partial \aalpha^{(l-1)})^2} \right\|_{2,2,1} \left\| \frac{\partial f}{\partial \aalpha^{(l)}} \right\|_{\infty}~.
\end{align*}
Then, based on the definitions in \eqref{eq:qwr2}, we have
\begin{align*}
 \| H_{w,r}^{(l_1,l_2)} \|_2 &\leq (L+1)(1+\gamma^{2L}) \cQ^{(w,r)}_{2,2,1}(f) \cQ_{\infty}(f)\\
 &\overset{(a)}{\leq}
 \frac{(L+1)(1+\rho_1)}{\sqrt{m}}(1+\gamma^{2L})^2(2\beta_{\phi}(1+\gamma^L)^2(1+\gamma)^2(1+(1+L|\phi(0)|)^2)+1)
 ~,
\end{align*}
where for (a) we used equations~\eqref{eq:Q_1} and~\eqref{eq:Q_3}. 

{\bf Case 1.B:} $2 \leq l_2 \leq l_1 \leq L+1$. By building on the form of the gradient, we have 
\begin{align*}
H_{w,r}^{(l_1,l_2)} & = \frac{\partial^2 \aalpha^{(l_2)}}{\partial \w^{(l_2)} \partial \r^{(l_2)}} \frac{\partial f}{\partial \aalpha^{(l_2)}} \1_{[l_1 =l_2]} + \1_{[l_2 <l_1]} \left( \frac{\partial \aalpha^{(l_2)}}{\partial \r^{(l_2)}}  \prod_{l'=l_2+1}^{l_1-1} \frac{\partial \aalpha^{(l')}}{\partial \aalpha^{(l'-1)}} \right) \frac{\partial^2 \aalpha^{(l_1)}}{\partial \aalpha^{(l_1-1)} \partial \w^{(l_1)}} \left(\frac{\partial f}{\partial \aalpha^{(l_1)}} \right) \\
& ~~~~ + \1_{[l_2 <l_1]}\sum_{l=l_1+1}^{L+1} \left( \frac{\partial \aalpha^{(l_2)}}{\partial \r^{(l_2)}} \prod_{l'=l_2+1}^{l-1} \frac{\partial \aalpha^{(l')}}{\partial \aalpha^{(l'-1)}} \right)  \left( \frac{\partial \aalpha^{(l_1)}}{\partial \w^{(l_1)}} \prod_{l'=l_1+1}^{l-1} \frac{\partial \aalpha^{(l')}}{\partial \aalpha^{(l'-1)}} \right) \frac{\partial^2 \aalpha^{(l)}}{(\partial \aalpha^{(l-1)})^2}  \left(\frac{\partial f}{\partial \aalpha^{(l)}} \right) ~. 
\end{align*}
Then,
\begin{align*}
\| H_{w,r}^{(l_1,l_2)} \|_2 
& \leq  \left\| \frac{\partial^2 \aalpha^{(l_2)}}{\partial \w^{(l_2)}  \partial \r^{(l_2)}} \right\|_{2,2,1} \left\| \frac{\partial f}{\partial \aalpha^{(l_2)}} \right\|_{\infty} \1_{[l_1=l_2]}\\
& ~~~~ + \1_{[l_2 <l_1]} \left\| \frac{\partial \aalpha^{(l_2)}}{\partial \r^{(l_2)}} \right\|_2 \prod_{l'=l_2+1}^{l_1-1} \left\| \frac{\partial \aalpha^{(l')}}{\partial \aalpha^{(l'-1)}} \right\|_2 \left\| \frac{\partial^2 \aalpha^{(l_1)}}{\partial \aalpha^{(l_1-1)} \partial \w^{(l_1)}} \right\|_{2,2,1} \left\| \frac{\partial f}{\partial \aalpha^{(l_1)}} \right\|_{\infty} \\
& ~~~~ + \1_{[l_2 <l_1]}\sum_{l=l_1+1}^{L+1} \left( \left\| \frac{\partial \aalpha^{(l_2)}}{\partial \r^{(l_2)}} \right\|_2  \prod_{l'=l_2+1}^{l-1} \left\| \frac{\partial \aalpha^{(l')}}{\partial \aalpha^{(l'-1)}} \right\|_2 \right)  \left( \left\| \frac{\partial \aalpha^{(l_1)}}{\partial \w^{(l_1)}} \right\|_2 \prod_{l'=l_1+1}^{l-1} \left\| \frac{\partial \aalpha^{(l')}}{\partial \aalpha^{(l'-1)}} \right\|_{2} \right)\\
& ~~~~ \times \left\| \frac{\partial^2 \aalpha^{(l)}}{(\partial \aalpha^{(l-1)})^2} \right\|_{2,2,1} \left\| \frac{\partial f}{\partial \aalpha^{(l)}} \right\|_{\infty} \\
& \leq  \left\| \frac{\partial^2 \aalpha^{(l_2)}}{\partial \w^{(l_2)}  \partial \r^{(l_2)}} \right\|_{2,2,1} \left\| \frac{\partial f}{\partial \aalpha^{(l_2)}} \right\|_{\infty} \1_{[l_1 =l_2]}\\
& ~~~~ + \1_{[l_2 <l_1]} \gamma^{l_1-l_2-1} \left\| \frac{\partial \aalpha^{(l_2)}}{\partial \r^{(l_2)}} \right\|_2 \left\| \frac{\partial^2 \aalpha^{(l_1)}}{\partial \aalpha^{(l_1-1)} \partial \w^{(l_1)}} \right\|_{2,2,1} \left\| \frac{\partial f}{\partial \aalpha^{(l_1)}} \right\|_{\infty} \\
& ~~~~ + \1_{[l_2 <l_1]}\sum_{l=l_1+1}^{L+1} \gamma^{2l - l_1 - l_2-2} \left\| \frac{\partial \aalpha^{(l_2)}}{\partial \r^{(l_2)}} \right\|_2  \left\| \frac{\partial \aalpha^{(l_1)}}{\partial \w^{(l_1)}} \right\|_2  \left\| \frac{\partial^2 \aalpha^{(l)}}{(\partial \aalpha^{(l-1)})^2} \right\|_{2,2,1} \left\| \frac{\partial f}{\partial \aalpha^{(l)}} \right\|_{\infty} ~.
\end{align*}

Then, the upper bound is similar to the case {\bf Case 1.A},
\begin{align*}
 \| H_{w,r}^{(l_1,l_2)} \|_2 &\leq (L+1)(1+\gamma^{2L}) \cQ^{(w,r)}_{2,2,1}(f) \cQ_{\infty}(f)\\
 &\overset{(a)}{\leq}
 \frac{(L+1)(1+\rho_1)}{\sqrt{m}}(1+\gamma^{2L})^2(2\beta_{\phi}(1+\gamma^L)^2(1+\gamma)^2(1+(1+L|\phi(0)|)^2)+1)
 ~.
\end{align*}
%where for (a) we used equations~\eqref{eq:Q_1} and~\eqref{eq:Q_3}. 

{\bf Case 2.A:} $1 \leq l_1 \leq L+1$. For Hessian terms involving $(w,v)$, since $\frac{\partial f}{\partial \v} = \frac{1}{\sqrt{m}} \aalpha^{(L+1)}$, we have 
\begin{align*}
H_{w,v}^{(l_1)} = \frac{1}{\sqrt{m}}  \frac{\partial \aalpha^{(L+1)}}{\partial \w^{(l_1)}} = \frac{1}{\sqrt{m}} \left( \frac{\partial \aalpha^{(l_1)}}{\partial \w^{(l_1)}} \prod_{l'=l_1+1}^{L+1} \frac{\partial \aalpha^{(l')}}{\partial \aalpha^{(l'-1)}}  \right)~.
\end{align*}
Then,
\begin{align*}
 \| H_{w,v}^{(l_1,L+1)} \|_2 \leq \frac{1}{\sqrt{m}} \left\| \frac{\partial \aalpha^{(l_1)}}{\partial \w^{(l_1)}} \right\|_2 \prod_{l'=l_1+1}^{L+1} \left\| \frac{\partial \aalpha^{(l')}}{\partial \aalpha^{(l'-1)}} \right\|_2 \leq \frac{1}{\sqrt{m}} \gamma^L \cQ_2^{(w,r)}(f)\overset{(a)}{\leq} \frac{1}{\sqrt{m}}\gamma^L(1+\gamma^L)(1+L|\phi(0)|)~,   
\end{align*}
where (a) follows from equation~\eqref{eq:Q_2}.

{\bf Case 2.B:} $2 \leq l_2 \leq L+1$. For Hessian terms involving $(r,v)$, since $\frac{\partial f}{\partial \v} = \frac{1}{\sqrt{m}} \aalpha^{(L+1)}$,  we have 
\begin{align*}
H_{r,v}^{(l_2)} = \frac{1}{\sqrt{m}}  \frac{\partial \aalpha^{(L+1)}}{\partial \r^{(l_2)}}
= \frac{1}{\sqrt{m}} \left( \frac{\partial \aalpha^{(l_2)}}{\partial \r^{(l_2)}} \prod_{l'=l_2+1}^{L+1} \frac{\partial \aalpha^{(l')}}{\partial \aalpha^{(l'-1)}}  \right)~.
\end{align*}
Then,
\begin{align*}
 \| H_{r,v}^{(l_2)} \|_2 \leq \frac{1}{\sqrt{m}} \left\| \frac{\partial \aalpha^{(l_2)}}{\partial \r^{(l_2)}} \right\|_2 \prod_{l'=l_2+1}^{L+1} \left\| \frac{\partial \aalpha^{(l')}}{\partial \aalpha^{(l'-1)}} \right\|_2 \overset{(a)}{\leq} \frac{1}{\sqrt{m}}\gamma^L(1+\gamma^L)(1+L|\phi(0)|)~,   
\end{align*}
where (a) follows from equation~\eqref{eq:Q_2}.



\paragraph{Diagonal Blocks.} 
For the diagonal blocks, we focus only on bounding (Case 3.A) $\| H_{w}^{(l_1,l_2)} \|_2$ and (Case 3.B) $\| H_{r}^{(l_1,l_2)} \|_2$ for $l_1 \leq l_2$, since the case $l_2 \leq l_1$ is just symmetrical and will have the same bounds. 

{\bf Case 3.A:} $1 \leq l_1 \leq l_2 \leq L+1$. By building on the form of the gradient, we have 
\begin{align*}
H_{w}^{(l_1,l_2)} & = \frac{\partial^2 \aalpha^{(l_1)}}{(\partial \w^{(l_1)})^2} \frac{\partial f}{\partial \aalpha^{(l_1)}} \1_{[l_1 =l_2]} +  \1_{[l_1 <l_2]}\left( \frac{\partial \aalpha^{(l_1)}}{\partial \w^{(l_1)}}  \prod_{l'=l_1+1}^{l_2-1} \frac{\partial \aalpha^{(l')}}{\partial \aalpha^{(l'-1)}} \right) \frac{\partial^2 \aalpha^{(l_2)}}{\partial \aalpha^{(l_2-1)} \partial \w^{(l_2)}} \left(\frac{\partial f}{\partial \aalpha^{(l_2)}} \right) \\
& ~~~~ + \1_{[l_1 <l_2]}\sum_{l=l_2+1}^{L+1} \left( \frac{\partial \aalpha^{(l_1)}}{\partial \w^{(l_1)}} \prod_{l'=l_1+1}^{l-1} \frac{\partial \aalpha^{(l')}}{\partial \aalpha^{(l'-1)}} \right)  \left( \frac{\partial \aalpha^{(l_2)}}{\partial \w^{(l_2)}} \prod_{l'=l_2+1}^{l-1} \frac{\partial \aalpha^{(l')}}{\partial \aalpha^{(l'-1)}} \right) \frac{\partial^2 \aalpha^{(l)}}{(\partial \aalpha^{(l-1)})^2}  \left(\frac{\partial f}{\partial \aalpha^{(l)}} \right) ~. 
\end{align*}
Then,
\begin{align*}
\| H_{w}^{(l_1,l_2)} \|_2 
& \leq  \left\| \frac{\partial^2 \aalpha^{(l_1)}}{(\partial \w^{(l_1)})^2} \right\|_{2,2,1} \left\| \frac{\partial f}{\partial \aalpha^{(l_1)}} \right\|_{\infty} \1_{[l_1=l_2]}\\
& ~~~~ + \1_{[l_1 <l_2]} \left\| \frac{\partial \aalpha^{(l_1)}}{\partial \w^{(l_1)}} \right\|_2 \prod_{l'=l_1+1}^{l_2-1} \left\| \frac{\partial \aalpha^{(l')}}{\partial \aalpha^{(l'-1)}} \right\|_2 \left\| \frac{\partial^2 \aalpha^{(l_2)}}{\partial \aalpha^{(l_2-1)} \partial \w^{(l_2)}} \right\|_{2,2,1} \left\| \frac{\partial f}{\partial \aalpha^{(l_2)}} \right\|_{\infty} \\
& ~~~~ + \1_{[l_1 <l_2]}\sum_{l=l_2+1}^{L+1} \left( \left\| \frac{\partial \aalpha^{(l_1)}}{\partial \w^{(l_1)}} \right\|_2  \prod_{l'=l_1+1}^{l-1} \left\| \frac{\partial \aalpha^{(l')}}{\partial \aalpha^{(l'-1)}} \right\|_2 \right)  \left( \left\| \frac{\partial \aalpha^{(l_2)}}{\partial \w^{(l_2)}} \right\|_2 \prod_{l'=l_2+1}^{l-1} \left\| \frac{\partial \aalpha^{(l')}}{\partial \aalpha^{(l'-1)}} \right\|_{2} \right)\\
& ~~~~ \times\left\| \frac{\partial^2 \aalpha^{(l)}}{(\partial \aalpha^{(l-1)})^2} \right\|_{2,2,1} \left\| \frac{\partial f}{\partial \aalpha^{(l)}} \right\|_{\infty} \\
& \leq  \left\| \frac{\partial^2 \aalpha^{(l_1)}}{(\partial \w^{(l_1)})^2} \right\|_{2,2,1} \left\| \frac{\partial f}{\partial \aalpha^{(l_1)}} \right\|_{\infty}\1_{[l_1=l_2]}\\
& ~~~~ +  \1_{[l_1 <l_2]}\gamma^{l_2-l_1-1} \left\| \frac{\partial \aalpha^{(l_1)}}{\partial \w^{(l_1)}} \right\|_2 \left\| \frac{\partial^2 \aalpha^{(l_2)}}{\partial \aalpha^{(l_2-1)} \partial \w^{(l_2)}} \right\|_{2,2,1} \left\| \frac{\partial f}{\partial \aalpha^{(l_2)}} \right\|_{\infty} \\
& ~~~~ + \1_{[l_1 <l_2]}\sum_{l=l_2+1}^{L+1} \gamma^{2l - l_2 - l_1-2} \left\| \frac{\partial \aalpha^{(l_1)}}{\partial \w^{(l_1)}} \right\|_2  \left\| \frac{\partial \aalpha^{(l_2)}}{\partial \w^{(l_2)}} \right\|_2  \left\| \frac{\partial^2 \aalpha^{(l)}}{(\partial \aalpha^{(l-1)})^2} \right\|_{2,2,1} \left\| \frac{\partial f}{\partial \aalpha^{(l)}} \right\|_{\infty}~.
\end{align*}
Then, based on the definitions in \eqref{eq:qwr2}, we have
\begin{align*}
 \| H_{(w)}^{(l_1,l_2)} \|_2 &\leq (L+1)(1+\gamma^{2L}) \cQ^{(w)}_{2,2,1}(f) \cQ_{\infty}(f)\\
 &\overset{(a)}{\leq}
 \frac{(L+1)(1+\rho_1)}{\sqrt{m}}(1+\gamma^{2L})^2(2\beta_{\phi}(1+\gamma^L)^2(1+\gamma)^2(1+(1+L|\phi(0)|)^2)+1)
 ~,
\end{align*}
where for (a) we used equations~\eqref{eq:Q_1} and~\eqref{eq:Q_3}.


{\bf Case 3.B:} $2 \leq l_1 \leq l_2 \leq L+1$. By building on the form of the gradient, we have 
\begin{align*}
H_{r}^{(l_1,l_2)} & = \frac{\partial^2 \aalpha^{(l_1)}}{(\partial \r^{(l_1)})^2} \frac{\partial f}{\partial \aalpha^{(l_1)}} \1_{[l_1 =l_2]} +  \1_{[l_1 <l_2]}\left( \frac{\partial \aalpha^{(l_1)}}{\partial \r^{(l_1)}}  \prod_{l'=l_1+1}^{l_2-1} \frac{\partial \aalpha^{(l')}}{\partial \aalpha^{(l'-1)}} \right) \frac{\partial^2 \aalpha^{(l_2)}}{\partial \aalpha^{(l_2-1)} \partial \r^{(l_2)}} \left(\frac{\partial f}{\partial \aalpha^{(l_2)}} \right) \\
& ~~~~ + \1_{[l_1 <l_2]}\sum_{l=l_2+1}^{L+1} \left( \frac{\partial \aalpha^{(l_1)}}{\partial \r^{(l_1)}} \prod_{l'=l_1+1}^{l-1} \frac{\partial \aalpha^{(l')}}{\partial \aalpha^{(l'-1)}} \right)  \left( \frac{\partial \aalpha^{(l_2)}}{\partial \r^{(l_2)}} \prod_{l'=l_2+1}^{l-1} \frac{\partial \aalpha^{(l')}}{\partial \aalpha^{(l'-1)}} \right) \frac{\partial^2 \aalpha^{(l)}}{(\partial \aalpha^{(l-1)})^2}  \left(\frac{\partial f}{\partial \aalpha^{(l)}} \right) ~. 
\end{align*}
Then, we can obtain  prove the following upper bound in a similar way to Case 3.A based on the definitions in \eqref{eq:qwr2},
\begin{align*}
 \| H_{(r)}^{(l_1,l_2)} \|_2 &\leq (L+1)(1+\gamma^{2L}) \cQ^{(r)}_{2,2,1}(f) \cQ_{\infty}(f)\\
 &\overset{(a)}{\leq}
 \frac{(L+1)(1+\rho_1)}{\sqrt{m}}(1+\gamma^{2L})^2(2\beta_{\phi}(1+\gamma^L)^2(1+\gamma)^2(1+(1+L|\phi(0)|)^2)+1)
 ~,
\end{align*}
where for (a) we used equations~\eqref{eq:Q_1} and~\eqref{eq:Q_3}.


Putting all the shown results back in~\eqref{eq:Hessian_big}, we prove equation~\eqref{eq:hessianBoundG_fg_FNO}. \pcedit{We also note that all the constants in the Hessian bound depend on $\sigma_{1,w}$, $\sigma_{1,r}$, the depth $L$, and the radii $\rho_w$, $\rho_r$, $\rho_1$, and $\rho_2$. This dependence of this bound reduces to the depth and the radii and becomes polynomial whenever $\gamma\leq 1$, which is equivalent to $\sigma_{1,w}+\sigma_{1,r}\leq 1-\frac{\rho_w+\rho_r}{\sqrt{m}}$.} 



Now, we focus on proving the rest of equations in Lemma~\ref{lemm:hessgradbounds-FNO}, namely, equations~\eqref{eq:gradientBoundG_fg_FNO} and~\eqref{eq:predictorBoundG_fg_FNO}. 

{\bf Gradient and predictor bounds.} 
%
We observe that for $l\in[L]$,
$
\frac{\partial f}{\partial \w^{(l)}} = \frac{\partial \aalpha^{(l)}}{\partial \w^{(l)}}  \left( \prod_{l'=l}^L \frac{\partial \aalpha^{(l'+1)}}{\partial \aalpha^{(l')}} \right) \frac{\partial f}{\partial \aalpha^{(L+1)}}
$, and so


\begin{align*}
\left\|\frac{\partial f}{\partial \w^{(l)}}\right\|_2 & \leq \left\| \frac{\partial \aalpha^{(l)}}{\partial \w^{(l)}}\right\|_2 \gamma^{L-l+1} \left\|\frac{\partial f}{\partial \aalpha^{(L+1)}}\right\|_2\\
&\leq 
\left\| \frac{\partial \aalpha^{(l)}}{\partial \w^{(l)}}\right\|_2 \gamma^{L-l+1} \frac{1}{\sqrt{m}}(1+\rho_1)\\
&\leq 
(1+\gamma^L)(1+L|\phi(0)|)\gamma^{L} \frac{1}{\sqrt{m}}(1+\rho_1)~,
\end{align*}
where the last inequality follows from Lemma~\ref{lemm:gradient_alpha_params}.

We also have that
\begin{align*}
\left\|\frac{\partial f}{\partial \w^{(L+1)}}\right\|_2 & =
\left\|\frac{\partial \aalpha^{(L+1)}}{\partial \w^{(L+1)}} \frac{\partial f}{\partial \aalpha^{(L+1)}}\right\|_2\\
& =
\left\|\frac{\partial \aalpha^{(L+1)}}{\partial \w^{(L+1)}}\right\|_2\left\| \frac{\partial f}{\partial \aalpha^{(L+1)}}\right\|_2\\
&\leq (1+\gamma^L)(1+L|\phi(0)|)\frac{1}{\sqrt{m}}(1+\rho_1)~.
\end{align*}

Similarly, we can obtain for $l_2\in\{2,\dots,L\}$,
\begin{align*}
\left\|\frac{\partial f}{\partial \r^{(l_2)}}\right\|_2 &\leq 
(1+\gamma^L)(1+L|\phi(0)|)\gamma^{L} \frac{1}{\sqrt{m}}(1+\rho_1)~,
\end{align*}
and
\begin{align*}
\left\|\frac{\partial f}{\partial \r^{(L+1)}}\right\|_2 
&\leq (1+\gamma^L)(1+L|\phi(0)|)\frac{1}{\sqrt{m}}(1+\rho_1)~.
\end{align*}
Using all these derivations, 
\begin{align*}
\norm{\nabla_{\vtheta}f}_2^2 &=\sum^{L+1}_{l=1}\left\|\frac{\partial f}{\partial \w^{(l)}}\right\|_2^2 + \sum^{L+1}_{l=2}\left\|\frac{\partial f}{\partial \r^{(l)}}\right\|_2^2\\
&\leq \frac{2}{m}(L+1)(1+\gamma^L)^2(1+L|\phi(0)|)^2(1+\rho_1)^2~,
\end{align*}
which finishes the proof for equation~\eqref{eq:gradientBoundG_fg_FNO}.

Now, 
\begin{align*}
|f| &=\left|\frac{1}{\sqrt{m}}\v^\top\aalpha^{(L+1)}\right|\\
&\leq \frac{1}{\sqrt{m}}\norm{\v}_2\norm{\aalpha^{(L+1)}}_2\\
&\leq (1+\rho_1)(1+\gamma^L)(1+L|\phi(0)|)~,
\end{align*}
which finishes the proof for equation~\eqref{eq:predictorBoundG_fg_FNO}. 
\pcedit{Again, we notice that all these bounds have a polynomial dependence on the depth $L$, and the radii $\rho_w$, $\rho_r$, $\rho_1$, and $\rho_2$ whenever $\gamma\leq 1$, i.e., whenever $\sigma_{1,w}+\sigma_{1,r}\leq 1-\frac{\rho_w+\rho_r}{\sqrt{m}}$.}

Thus, we finish the proof for Lemma~\ref{lemm:hessgradbounds-FNO}.

\subsection{RSC and Smoothness Results}

Using the results from the previous section, we immediately obtain the RSC and smoothness results. 

\RSCLossFNO*
%
\begin{proof}
We start by proving the first part of the theorem's statement. 
We immediately see that, 
since $B^t_{\kappa}\subset B^{\mathrm{Euc}}_{\rho_w,\rho_r\rho_1}(\vtheta_0)$, we satisfy Condition~\ref{cond:rsc}(a). We now need to satisfy Condition~\ref{cond:rsc}(b). For this, we proceed to show the existence of an element $\vtheta' \in B^t_{\kappa}$ that is an element of the set $Q^{t}_{\kappa}$ as in Definition~\ref{defn:qset_FNO}, i.e., satisfies
\begin{equation}
|\cos(\vtheta'-\vtheta_t, \nabla_\vtheta\bar{G}_t)| \geq \kappa~,
\label{eq:kap-fno}
\end{equation}
and that also satisfies the following two conditions:
\begin{enumerate}[{Condition} (A):]%[(a)]
\item  $\|\vtheta' - \vtheta_{t} \|_2 = \epsilon$ for some $\epsilon< \frac{2 \norm{\nabla_{\vtheta} \cL(\vtheta_t)}_2 \sqrt{1-\kappa^2}}{\beta}$; and \label{cond-1-fno}
\item the angle $\nu'$ between $(\vtheta' - \vtheta_{t})$ and $-\nabla_{\vtheta} \cL(\vtheta_t)$ is acute, so that $\cos(\nu') > 0$. \label{cond-2-fno}\end{enumerate}
%

To show the existence of such element $\vtheta' \in B_t$, we propose two possible constructions:
\begin{enumerate}[{Choice} (A):]
\item  
If the points $\vtheta_{t+1}$, $\nabla_\vtheta\bar{G}_t + \vtheta_{t}$, and $\vtheta_{t}$ are not collinear, then they define a hyperplane $\mathcal{P}$ that contains the vectors $\nabla_\vtheta\bar{G}_t$ and $-\nabla_{\vtheta} \cL(\vtheta_t)$ (recall that $\vtheta_{t+1} - \vtheta_{t}=-\nabla_{\vtheta}\cL(\vtheta_t)$ by gradient descent). We choose $\vtheta'$ such that the vector 
$\vtheta'-\vtheta_{t}$ 
%(or, equivalently, the vector $\bar{\vtheta}_f$)
lies in 
$\mathcal{P}$ 
with 
$\cos(\vtheta'-\vtheta_{t},\nabla_\vtheta\bar{G}_t)=\kappa$ (i.e., it satisfies condition~\eqref{eq:kap-fno} with equality) while simultaneously satisfying Condition~\eqref{cond-2-fno}.
%
%
If the points $\vtheta_{t+1}$, $\nabla_\vtheta\bar{G}_t + \vtheta_{t}$, and $\vtheta_{t}$ are collinear, we choose $\vtheta'$ such that it is not collinear with these points, thus defining a hyperplane $\mathcal{P}$ 
with these other three points, 
and such that 
$\vtheta'$ is also taken so that $\cos(\vtheta'-\vtheta_{t},\nabla_\vtheta\bar{G}_t)=\kappa$ while simultaneously satisfying Condition~\eqref{cond-2-fno}.
%

Thus far we have only defined \emph{angle} (or \emph{direction}) conditions on the vector $\vtheta'-\vtheta_{t}$, and so there could be an infinite number of values for $\vtheta'_f$ satisfying such angle conditions without $\vtheta'$ belonging to the set $B^{\mathrm{Euc}}_{\rho_w,\rho_r,\rho_1}(\vtheta_0)$ nor $\vtheta'$ satisfying Condition~\eqref{cond-1-fno}. To determine the feasible values for $\vtheta'$, we observe that $\vtheta_t$ is \emph{strictly inside} the set $B^{\mathrm{Euc}}_{\rho_w,\rho_r,\rho_1}(\vtheta_0)$ by Assumption~\ref{asmp:iter-2}, and so 
%
$\vtheta'$ can be taken arbitrarily close to $\vtheta_{t}$ so that $\vtheta'\in B^{\mathrm{Euc}}_{\rho_w,\rho_r,\rho_1}(\vtheta_0)$ and Condition~\eqref{cond-1-fno} is satisfied. 
%

We remark that, regardless of the collinearity of the points 
$\vtheta_{t+1}$, $\nabla_\vtheta\bar{G}_t + \vtheta_{t}$, and $\vtheta_{t}$, hyperplane $\mathcal{P}$ contains the vectors $\vtheta'-\vtheta_{t}$, $\nabla_\vtheta\bar{G}_t$, and $-\nabla_{\vtheta}\cL(\vtheta_t)$, all sharing its origin at $\vtheta_{f}\in\mathcal{P}$. \label{ch-A-fno}
%
%; or
\item  
%
%
We choose $\vtheta'$ as in Choice~\eqref{ch-A-don} but with $\nabla_\vtheta\bar{G}_t$ replaced by $-\nabla_\vtheta\bar{G}_t$.
%
\label{ch-B-fno}
%
%
\end{enumerate}
We immediately notice that $\vtheta'$ defined by either Choice~\eqref{ch-A-fno} or Choice~\eqref{ch-B-fno} satisfies 
$\vtheta'\in Q^t_\kappa \cap B^{\mathrm{Euc}}_{\rho_w,\rho_r,\rho_1}(\vtheta_0)
$. To make $\vtheta'$ belong to the set $B^t_\kappa$, we need to find a radius $\rho_2$ such that $\vtheta'\in B^{\mathrm{Euc}}_{\rho_w,\rho_r,\rho_1}(\vtheta_0)$, which is done by taking $\rho_2>\epsilon$ with $\epsilon$ as in Condition~\eqref{cond-1-fno}. 
%
Finally, it is straightforward to verify that such $\vtheta'\in B^t_\kappa$ defined by either Choice~\eqref{ch-A-fno} or Choice~\eqref{ch-B-fno} will always exist, by considering the following cases for the angle $\nu$ between $\nabla_\vtheta\bar{G}_t$ and $-\nabla_{\vtheta} \cL(\vtheta_t)$:
\begin{enumerate}[(i)]
\item If $\nu \in [0, \pi/2]$ or $\nu \in [3\pi/2, 2\pi]$, then Choice~\eqref{ch-A-fno} will be true, since $-\nabla_{\vtheta_f} \cL(\vtheta_t)$ is in the positive half space\footnote{We say $\a$ is in the positive half-space of $\b$ if $\langle \a, \b \rangle \geq 0$.} of $\nabla_\vtheta\bar{G}_t$; and
\label{it-i-fno}
\item if $\nu \in [\pi/2,\pi]$ or $\nu \in [\pi, 3\pi/2]$, then Choice~\eqref{ch-B-fno} will be true, since $-\nabla_{\vtheta_f} \cL(\vtheta_t)$ is in the positive half space of $-\nabla_\vtheta\bar{G}_t$.\label{it-ii-fno}
\end{enumerate}

%
Now, let us assume we are in the case of item~\eqref{it-i-fno} above, so that $\vtheta'$ is constructed according to Choice~\eqref{ch-A-fno} (the rest of the proof can be adapted to the case of item~\eqref{it-ii-fno} by using a symmetrical argument and so it is omitted).  
% \end{enumerate}
% To get (a)
Let $\nu_1$ be the angle between $\vtheta'-\vtheta_{t}$ and $\nabla_\vtheta\bar{G}_t$, so that $\cos(\nu_1)=\kappa$ according to Choice~\eqref{ch-A-fno}.
%
Then, we have that 
\begin{align*}
|\cos(\nu')| = |\cos(\nu - \nu_1)| \geq |\cos(\pi/2 - \nu_1)| = |\sin(\nu_1)| = \sqrt{1-\cos^2(\nu_1)} = \sqrt{1-\kappa^2}~.
\end{align*}
Further, by the construction in Condition~\eqref{cond-2-fno}, $\cos(\nu') > 0$, which implies 
%\begin{equation}
$\cos(\nu') \geq  \sqrt{1-\kappa^2}>0$.
%
Now, 
by the smoothness property of the empirical loss $\cL$ we have
\begin{align*}
\cL(\vtheta') & \leq \cL(\vtheta_t) - \langle \vtheta' - \vtheta_t, -\nabla_\vtheta \cL(\vtheta_t) \rangle + \frac{\beta}{2}\| \vtheta' - \vtheta_t \|_2^2 \\ 
& = \cL(\vtheta_t) - \|\vtheta' - \vtheta_{t}\|_2 \|\nabla_{\vtheta} \cL(\vtheta_t) \|_2 \cos(\nu) + \frac{\beta}{2}\| \vtheta' - \vtheta_{t} \|_2^2 \\
& =  \cL(\vtheta_t) -  \epsilon \|\nabla_{\vtheta} \cL(\vtheta_t) \|_2 \cos(\nu) + \frac{\beta}{2} \epsilon^2 \\
& = \cL(\vtheta_t) -  \frac{\beta \epsilon}{2} \left( \frac{2 \|\nabla_{\vtheta} \cL(\vtheta_t) \|_2 \cos(\nu)}{\beta} - \epsilon \right)\\
&<\cL(\vtheta_t)~.
\end{align*}
where the last inequality follows by the construction of $\epsilon$ in Condition~\eqref{ch-A-fno}. Note that this implies that the constructed $\vtheta'$ is as described in Condition~\ref{cond:rsc}(b.2). This finishes the proof for Condition~\ref{cond:rsc}(b).

The second part of the proof, i.e., the RSC condition over the non-empty set $B^t_{\kappa}$, %. This proof 
follows from a direct adaptation of Theorem~5.1 in~\citep{banerjee2022restricted} using Lemma~\ref{lemm:hessgradbounds-FNO}. \pcedit{Since we are using Lemma~\ref{lemm:hessgradbounds-FNO}, the condition for polynomial dependence on the bounds carries on.} 
%
%
\end{proof}


\RSSFNO*
%
\begin{proof}
The proof follows from a direct adaptation of the proof of Theorem~5.2 in~\citep{banerjee2022restricted} using Lemma~\ref{lemm:hessgradbounds-FNO}, where it can be shown that $\beta = 2\varrho^2 + \frac{\bar{c}}{\sqrt{m}}$ for some positive constant \pcedit{$\bar{c}$ which inherits the dependence on the constants $\sigma_{1,w}$, $\sigma_{1,r}$, the depth $L$ and the radii $\rho_w$, $\rho_r$, and $\rho_1$ from Lemma~\ref{lemm:hessgradbounds-FNO}}.
\end{proof}

\begin{prop}[{\bf RSC to smoothness ratio}] 
\label{prop:RSC-smooth-FNO}
Under the same conditions as in Theorems~\ref{theo:rsc_main_fno} and~\ref{theo:smooth_main_fno}, we have that $\alpha_t/\beta<1$ with probability at least $1-\frac{2(L+2)}{m}$.
\end{prop}
\begin{proof}
From the direct adaptation of the proof of Theorem~5.2 in~\citep{banerjee2022restricted} using Lemma~\ref{lemm:hessgradbounds-FNO}, we can obtain $\norm{\nabla_{\vtheta}\bar{G}_t}_2^2\leq \varrho^2$. Then, $\alpha_t\overset{(a)}{<}2\kappa^2\norm{\nabla_{\vtheta}\bar{G}_t}_2^2\leq 2\kappa^2 \varrho^2\leq 2\varrho^2\overset{(b)}{<}
\beta$, where (a) follows from~\eqref{eq:RSCLoss_FNO} and (b) from Theorem~\ref{theo:smooth_main_fno}. This result shows that $\frac{\alpha_t}{\beta}<1$.
\end{proof}



%====================
%====================
