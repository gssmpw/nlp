\pdfoutput=1

\documentclass[letterpaper]{article}

\usepackage[margin=1in]{geometry}

\usepackage{microtype}
\newcommand{\theHalgorithm}{\arabic{algorithm}}

\usepackage{amsfonts,amsmath,amssymb,amsthm}       % blackboard math symbols
\usepackage{natbib}
\usepackage{array}
\usepackage{todonotes}
%\usepackage[dvipsnames]{xcolor}
\usepackage{xcolor}
\usepackage{subcaption,float,graphicx}
\usepackage{thmtools,thm-restate}
\usepackage{url}            % simple URL typesetting
\usepackage{booktabs}       % professional-quality tables
\usepackage{nicefrac}       % compact symbols for 1/2, etc.
\usepackage{microtype}      % microtypography
\usepackage{mathrsfs}
% \usepackage{mathpazo}
\usepackage[
    colorlinks=true,
    linkcolor=blue,
    filecolor=magenta,      
    urlcolor=cyan,
    citecolor=blue,
]{hyperref}
% Assuming math_commands.tex and notation.tex are present in the same directory
%%%%% NEW MATH DEFINITIONS %%%%%

\usepackage{amsmath,amsfonts,bm}
\usepackage{derivative}
% Mark sections of captions for referring to divisions of figures
\newcommand{\figleft}{{\em (Left)}}
\newcommand{\figcenter}{{\em (Center)}}
\newcommand{\figright}{{\em (Right)}}
\newcommand{\figtop}{{\em (Top)}}
\newcommand{\figbottom}{{\em (Bottom)}}
\newcommand{\captiona}{{\em (a)}}
\newcommand{\captionb}{{\em (b)}}
\newcommand{\captionc}{{\em (c)}}
\newcommand{\captiond}{{\em (d)}}

% Highlight a newly defined term
\newcommand{\newterm}[1]{{\bf #1}}

% Derivative d 
\newcommand{\deriv}{{\mathrm{d}}}

% Figure reference, lower-case.
\def\figref#1{figure~\ref{#1}}
% Figure reference, capital. For start of sentence
\def\Figref#1{Figure~\ref{#1}}
\def\twofigref#1#2{figures \ref{#1} and \ref{#2}}
\def\quadfigref#1#2#3#4{figures \ref{#1}, \ref{#2}, \ref{#3} and \ref{#4}}
% Section reference, lower-case.
\def\secref#1{section~\ref{#1}}
% Section reference, capital.
\def\Secref#1{Section~\ref{#1}}
% Reference to two sections.
\def\twosecrefs#1#2{sections \ref{#1} and \ref{#2}}
% Reference to three sections.
\def\secrefs#1#2#3{sections \ref{#1}, \ref{#2} and \ref{#3}}
% Reference to an equation, lower-case.
\def\eqref#1{equation~\ref{#1}}
% Reference to an equation, upper case
\def\Eqref#1{Equation~\ref{#1}}
% A raw reference to an equation---avoid using if possible
\def\plaineqref#1{\ref{#1}}
% Reference to a chapter, lower-case.
\def\chapref#1{chapter~\ref{#1}}
% Reference to an equation, upper case.
\def\Chapref#1{Chapter~\ref{#1}}
% Reference to a range of chapters
\def\rangechapref#1#2{chapters\ref{#1}--\ref{#2}}
% Reference to an algorithm, lower-case.
\def\algref#1{algorithm~\ref{#1}}
% Reference to an algorithm, upper case.
\def\Algref#1{Algorithm~\ref{#1}}
\def\twoalgref#1#2{algorithms \ref{#1} and \ref{#2}}
\def\Twoalgref#1#2{Algorithms \ref{#1} and \ref{#2}}
% Reference to a part, lower case
\def\partref#1{part~\ref{#1}}
% Reference to a part, upper case
\def\Partref#1{Part~\ref{#1}}
\def\twopartref#1#2{parts \ref{#1} and \ref{#2}}

\def\ceil#1{\lceil #1 \rceil}
\def\floor#1{\lfloor #1 \rfloor}
\def\1{\bm{1}}
\newcommand{\train}{\mathcal{D}}
\newcommand{\valid}{\mathcal{D_{\mathrm{valid}}}}
\newcommand{\test}{\mathcal{D_{\mathrm{test}}}}

\def\eps{{\epsilon}}


% Random variables
\def\reta{{\textnormal{$\eta$}}}
\def\ra{{\textnormal{a}}}
\def\rb{{\textnormal{b}}}
\def\rc{{\textnormal{c}}}
\def\rd{{\textnormal{d}}}
\def\re{{\textnormal{e}}}
\def\rf{{\textnormal{f}}}
\def\rg{{\textnormal{g}}}
\def\rh{{\textnormal{h}}}
\def\ri{{\textnormal{i}}}
\def\rj{{\textnormal{j}}}
\def\rk{{\textnormal{k}}}
\def\rl{{\textnormal{l}}}
% rm is already a command, just don't name any random variables m
\def\rn{{\textnormal{n}}}
\def\ro{{\textnormal{o}}}
\def\rp{{\textnormal{p}}}
\def\rq{{\textnormal{q}}}
\def\rr{{\textnormal{r}}}
\def\rs{{\textnormal{s}}}
\def\rt{{\textnormal{t}}}
\def\ru{{\textnormal{u}}}
\def\rv{{\textnormal{v}}}
\def\rw{{\textnormal{w}}}
\def\rx{{\textnormal{x}}}
\def\ry{{\textnormal{y}}}
\def\rz{{\textnormal{z}}}

% Random vectors
\def\rvepsilon{{\mathbf{\epsilon}}}
\def\rvphi{{\mathbf{\phi}}}
\def\rvtheta{{\mathbf{\theta}}}
\def\rva{{\mathbf{a}}}
\def\rvb{{\mathbf{b}}}
\def\rvc{{\mathbf{c}}}
\def\rvd{{\mathbf{d}}}
\def\rve{{\mathbf{e}}}
\def\rvf{{\mathbf{f}}}
\def\rvg{{\mathbf{g}}}
\def\rvh{{\mathbf{h}}}
\def\rvu{{\mathbf{i}}}
\def\rvj{{\mathbf{j}}}
\def\rvk{{\mathbf{k}}}
\def\rvl{{\mathbf{l}}}
\def\rvm{{\mathbf{m}}}
\def\rvn{{\mathbf{n}}}
\def\rvo{{\mathbf{o}}}
\def\rvp{{\mathbf{p}}}
\def\rvq{{\mathbf{q}}}
\def\rvr{{\mathbf{r}}}
\def\rvs{{\mathbf{s}}}
\def\rvt{{\mathbf{t}}}
\def\rvu{{\mathbf{u}}}
\def\rvv{{\mathbf{v}}}
\def\rvw{{\mathbf{w}}}
\def\rvx{{\mathbf{x}}}
\def\rvy{{\mathbf{y}}}
\def\rvz{{\mathbf{z}}}

% Elements of random vectors
\def\erva{{\textnormal{a}}}
\def\ervb{{\textnormal{b}}}
\def\ervc{{\textnormal{c}}}
\def\ervd{{\textnormal{d}}}
\def\erve{{\textnormal{e}}}
\def\ervf{{\textnormal{f}}}
\def\ervg{{\textnormal{g}}}
\def\ervh{{\textnormal{h}}}
\def\ervi{{\textnormal{i}}}
\def\ervj{{\textnormal{j}}}
\def\ervk{{\textnormal{k}}}
\def\ervl{{\textnormal{l}}}
\def\ervm{{\textnormal{m}}}
\def\ervn{{\textnormal{n}}}
\def\ervo{{\textnormal{o}}}
\def\ervp{{\textnormal{p}}}
\def\ervq{{\textnormal{q}}}
\def\ervr{{\textnormal{r}}}
\def\ervs{{\textnormal{s}}}
\def\ervt{{\textnormal{t}}}
\def\ervu{{\textnormal{u}}}
\def\ervv{{\textnormal{v}}}
\def\ervw{{\textnormal{w}}}
\def\ervx{{\textnormal{x}}}
\def\ervy{{\textnormal{y}}}
\def\ervz{{\textnormal{z}}}

% Random matrices
\def\rmA{{\mathbf{A}}}
\def\rmB{{\mathbf{B}}}
\def\rmC{{\mathbf{C}}}
\def\rmD{{\mathbf{D}}}
\def\rmE{{\mathbf{E}}}
\def\rmF{{\mathbf{F}}}
\def\rmG{{\mathbf{G}}}
\def\rmH{{\mathbf{H}}}
\def\rmI{{\mathbf{I}}}
\def\rmJ{{\mathbf{J}}}
\def\rmK{{\mathbf{K}}}
\def\rmL{{\mathbf{L}}}
\def\rmM{{\mathbf{M}}}
\def\rmN{{\mathbf{N}}}
\def\rmO{{\mathbf{O}}}
\def\rmP{{\mathbf{P}}}
\def\rmQ{{\mathbf{Q}}}
\def\rmR{{\mathbf{R}}}
\def\rmS{{\mathbf{S}}}
\def\rmT{{\mathbf{T}}}
\def\rmU{{\mathbf{U}}}
\def\rmV{{\mathbf{V}}}
\def\rmW{{\mathbf{W}}}
\def\rmX{{\mathbf{X}}}
\def\rmY{{\mathbf{Y}}}
\def\rmZ{{\mathbf{Z}}}

% Elements of random matrices
\def\ermA{{\textnormal{A}}}
\def\ermB{{\textnormal{B}}}
\def\ermC{{\textnormal{C}}}
\def\ermD{{\textnormal{D}}}
\def\ermE{{\textnormal{E}}}
\def\ermF{{\textnormal{F}}}
\def\ermG{{\textnormal{G}}}
\def\ermH{{\textnormal{H}}}
\def\ermI{{\textnormal{I}}}
\def\ermJ{{\textnormal{J}}}
\def\ermK{{\textnormal{K}}}
\def\ermL{{\textnormal{L}}}
\def\ermM{{\textnormal{M}}}
\def\ermN{{\textnormal{N}}}
\def\ermO{{\textnormal{O}}}
\def\ermP{{\textnormal{P}}}
\def\ermQ{{\textnormal{Q}}}
\def\ermR{{\textnormal{R}}}
\def\ermS{{\textnormal{S}}}
\def\ermT{{\textnormal{T}}}
\def\ermU{{\textnormal{U}}}
\def\ermV{{\textnormal{V}}}
\def\ermW{{\textnormal{W}}}
\def\ermX{{\textnormal{X}}}
\def\ermY{{\textnormal{Y}}}
\def\ermZ{{\textnormal{Z}}}

% Vectors
\def\vzero{{\bm{0}}}
\def\vone{{\bm{1}}}
\def\vmu{{\bm{\mu}}}
\def\vtheta{{\bm{\theta}}}
\def\vphi{{\bm{\phi}}}
\def\va{{\bm{a}}}
\def\vb{{\bm{b}}}
\def\vc{{\bm{c}}}
\def\vd{{\bm{d}}}
\def\ve{{\bm{e}}}
\def\vf{{\bm{f}}}
\def\vg{{\bm{g}}}
\def\vh{{\bm{h}}}
\def\vi{{\bm{i}}}
\def\vj{{\bm{j}}}
\def\vk{{\bm{k}}}
\def\vl{{\bm{l}}}
\def\vm{{\bm{m}}}
\def\vn{{\bm{n}}}
\def\vo{{\bm{o}}}
\def\vp{{\bm{p}}}
\def\vq{{\bm{q}}}
\def\vr{{\bm{r}}}
\def\vs{{\bm{s}}}
\def\vt{{\bm{t}}}
\def\vu{{\bm{u}}}
\def\vv{{\bm{v}}}
\def\vw{{\bm{w}}}
\def\vx{{\bm{x}}}
\def\vy{{\bm{y}}}
\def\vz{{\bm{z}}}

% Elements of vectors
\def\evalpha{{\alpha}}
\def\evbeta{{\beta}}
\def\evepsilon{{\epsilon}}
\def\evlambda{{\lambda}}
\def\evomega{{\omega}}
\def\evmu{{\mu}}
\def\evpsi{{\psi}}
\def\evsigma{{\sigma}}
\def\evtheta{{\theta}}
\def\eva{{a}}
\def\evb{{b}}
\def\evc{{c}}
\def\evd{{d}}
\def\eve{{e}}
\def\evf{{f}}
\def\evg{{g}}
\def\evh{{h}}
\def\evi{{i}}
\def\evj{{j}}
\def\evk{{k}}
\def\evl{{l}}
\def\evm{{m}}
\def\evn{{n}}
\def\evo{{o}}
\def\evp{{p}}
\def\evq{{q}}
\def\evr{{r}}
\def\evs{{s}}
\def\evt{{t}}
\def\evu{{u}}
\def\evv{{v}}
\def\evw{{w}}
\def\evx{{x}}
\def\evy{{y}}
\def\evz{{z}}

% Matrix
\def\mA{{\bm{A}}}
\def\mB{{\bm{B}}}
\def\mC{{\bm{C}}}
\def\mD{{\bm{D}}}
\def\mE{{\bm{E}}}
\def\mF{{\bm{F}}}
\def\mG{{\bm{G}}}
\def\mH{{\bm{H}}}
\def\mI{{\bm{I}}}
\def\mJ{{\bm{J}}}
\def\mK{{\bm{K}}}
\def\mL{{\bm{L}}}
\def\mM{{\bm{M}}}
\def\mN{{\bm{N}}}
\def\mO{{\bm{O}}}
\def\mP{{\bm{P}}}
\def\mQ{{\bm{Q}}}
\def\mR{{\bm{R}}}
\def\mS{{\bm{S}}}
\def\mT{{\bm{T}}}
\def\mU{{\bm{U}}}
\def\mV{{\bm{V}}}
\def\mW{{\bm{W}}}
\def\mX{{\bm{X}}}
\def\mY{{\bm{Y}}}
\def\mZ{{\bm{Z}}}
\def\mBeta{{\bm{\beta}}}
\def\mPhi{{\bm{\Phi}}}
\def\mLambda{{\bm{\Lambda}}}
\def\mSigma{{\bm{\Sigma}}}

% Tensor
\DeclareMathAlphabet{\mathsfit}{\encodingdefault}{\sfdefault}{m}{sl}
\SetMathAlphabet{\mathsfit}{bold}{\encodingdefault}{\sfdefault}{bx}{n}
\newcommand{\tens}[1]{\bm{\mathsfit{#1}}}
\def\tA{{\tens{A}}}
\def\tB{{\tens{B}}}
\def\tC{{\tens{C}}}
\def\tD{{\tens{D}}}
\def\tE{{\tens{E}}}
\def\tF{{\tens{F}}}
\def\tG{{\tens{G}}}
\def\tH{{\tens{H}}}
\def\tI{{\tens{I}}}
\def\tJ{{\tens{J}}}
\def\tK{{\tens{K}}}
\def\tL{{\tens{L}}}
\def\tM{{\tens{M}}}
\def\tN{{\tens{N}}}
\def\tO{{\tens{O}}}
\def\tP{{\tens{P}}}
\def\tQ{{\tens{Q}}}
\def\tR{{\tens{R}}}
\def\tS{{\tens{S}}}
\def\tT{{\tens{T}}}
\def\tU{{\tens{U}}}
\def\tV{{\tens{V}}}
\def\tW{{\tens{W}}}
\def\tX{{\tens{X}}}
\def\tY{{\tens{Y}}}
\def\tZ{{\tens{Z}}}


% Graph
\def\gA{{\mathcal{A}}}
\def\gB{{\mathcal{B}}}
\def\gC{{\mathcal{C}}}
\def\gD{{\mathcal{D}}}
\def\gE{{\mathcal{E}}}
\def\gF{{\mathcal{F}}}
\def\gG{{\mathcal{G}}}
\def\gH{{\mathcal{H}}}
\def\gI{{\mathcal{I}}}
\def\gJ{{\mathcal{J}}}
\def\gK{{\mathcal{K}}}
\def\gL{{\mathcal{L}}}
\def\gM{{\mathcal{M}}}
\def\gN{{\mathcal{N}}}
\def\gO{{\mathcal{O}}}
\def\gP{{\mathcal{P}}}
\def\gQ{{\mathcal{Q}}}
\def\gR{{\mathcal{R}}}
\def\gS{{\mathcal{S}}}
\def\gT{{\mathcal{T}}}
\def\gU{{\mathcal{U}}}
\def\gV{{\mathcal{V}}}
\def\gW{{\mathcal{W}}}
\def\gX{{\mathcal{X}}}
\def\gY{{\mathcal{Y}}}
\def\gZ{{\mathcal{Z}}}

% Sets
\def\sA{{\mathbb{A}}}
\def\sB{{\mathbb{B}}}
\def\sC{{\mathbb{C}}}
\def\sD{{\mathbb{D}}}
% Don't use a set called E, because this would be the same as our symbol
% for expectation.
\def\sF{{\mathbb{F}}}
\def\sG{{\mathbb{G}}}
\def\sH{{\mathbb{H}}}
\def\sI{{\mathbb{I}}}
\def\sJ{{\mathbb{J}}}
\def\sK{{\mathbb{K}}}
\def\sL{{\mathbb{L}}}
\def\sM{{\mathbb{M}}}
\def\sN{{\mathbb{N}}}
\def\sO{{\mathbb{O}}}
\def\sP{{\mathbb{P}}}
\def\sQ{{\mathbb{Q}}}
\def\sR{{\mathbb{R}}}
\def\sS{{\mathbb{S}}}
\def\sT{{\mathbb{T}}}
\def\sU{{\mathbb{U}}}
\def\sV{{\mathbb{V}}}
\def\sW{{\mathbb{W}}}
\def\sX{{\mathbb{X}}}
\def\sY{{\mathbb{Y}}}
\def\sZ{{\mathbb{Z}}}

% Entries of a matrix
\def\emLambda{{\Lambda}}
\def\emA{{A}}
\def\emB{{B}}
\def\emC{{C}}
\def\emD{{D}}
\def\emE{{E}}
\def\emF{{F}}
\def\emG{{G}}
\def\emH{{H}}
\def\emI{{I}}
\def\emJ{{J}}
\def\emK{{K}}
\def\emL{{L}}
\def\emM{{M}}
\def\emN{{N}}
\def\emO{{O}}
\def\emP{{P}}
\def\emQ{{Q}}
\def\emR{{R}}
\def\emS{{S}}
\def\emT{{T}}
\def\emU{{U}}
\def\emV{{V}}
\def\emW{{W}}
\def\emX{{X}}
\def\emY{{Y}}
\def\emZ{{Z}}
\def\emSigma{{\Sigma}}

% entries of a tensor
% Same font as tensor, without \bm wrapper
\newcommand{\etens}[1]{\mathsfit{#1}}
\def\etLambda{{\etens{\Lambda}}}
\def\etA{{\etens{A}}}
\def\etB{{\etens{B}}}
\def\etC{{\etens{C}}}
\def\etD{{\etens{D}}}
\def\etE{{\etens{E}}}
\def\etF{{\etens{F}}}
\def\etG{{\etens{G}}}
\def\etH{{\etens{H}}}
\def\etI{{\etens{I}}}
\def\etJ{{\etens{J}}}
\def\etK{{\etens{K}}}
\def\etL{{\etens{L}}}
\def\etM{{\etens{M}}}
\def\etN{{\etens{N}}}
\def\etO{{\etens{O}}}
\def\etP{{\etens{P}}}
\def\etQ{{\etens{Q}}}
\def\etR{{\etens{R}}}
\def\etS{{\etens{S}}}
\def\etT{{\etens{T}}}
\def\etU{{\etens{U}}}
\def\etV{{\etens{V}}}
\def\etW{{\etens{W}}}
\def\etX{{\etens{X}}}
\def\etY{{\etens{Y}}}
\def\etZ{{\etens{Z}}}

% The true underlying data generating distribution
\newcommand{\pdata}{p_{\rm{data}}}
\newcommand{\ptarget}{p_{\rm{target}}}
\newcommand{\pprior}{p_{\rm{prior}}}
\newcommand{\pbase}{p_{\rm{base}}}
\newcommand{\pref}{p_{\rm{ref}}}

% The empirical distribution defined by the training set
\newcommand{\ptrain}{\hat{p}_{\rm{data}}}
\newcommand{\Ptrain}{\hat{P}_{\rm{data}}}
% The model distribution
\newcommand{\pmodel}{p_{\rm{model}}}
\newcommand{\Pmodel}{P_{\rm{model}}}
\newcommand{\ptildemodel}{\tilde{p}_{\rm{model}}}
% Stochastic autoencoder distributions
\newcommand{\pencode}{p_{\rm{encoder}}}
\newcommand{\pdecode}{p_{\rm{decoder}}}
\newcommand{\precons}{p_{\rm{reconstruct}}}

\newcommand{\laplace}{\mathrm{Laplace}} % Laplace distribution

\newcommand{\E}{\mathbb{E}}
\newcommand{\Ls}{\mathcal{L}}
\newcommand{\R}{\mathbb{R}}
\newcommand{\emp}{\tilde{p}}
\newcommand{\lr}{\alpha}
\newcommand{\reg}{\lambda}
\newcommand{\rect}{\mathrm{rectifier}}
\newcommand{\softmax}{\mathrm{softmax}}
\newcommand{\sigmoid}{\sigma}
\newcommand{\softplus}{\zeta}
\newcommand{\KL}{D_{\mathrm{KL}}}
\newcommand{\Var}{\mathrm{Var}}
\newcommand{\standarderror}{\mathrm{SE}}
\newcommand{\Cov}{\mathrm{Cov}}
% Wolfram Mathworld says $L^2$ is for function spaces and $\ell^2$ is for vectors
% But then they seem to use $L^2$ for vectors throughout the site, and so does
% wikipedia.
\newcommand{\normlzero}{L^0}
\newcommand{\normlone}{L^1}
\newcommand{\normltwo}{L^2}
\newcommand{\normlp}{L^p}
\newcommand{\normmax}{L^\infty}

\newcommand{\parents}{Pa} % See usage in notation.tex. Chosen to match Daphne's book.

\DeclareMathOperator*{\argmax}{arg\,max}
\DeclareMathOperator*{\argmin}{arg\,min}

\DeclareMathOperator{\sign}{sign}
\DeclareMathOperator{\Tr}{Tr}
\let\ab\allowbreak
 
% !TeX root = main.tex 


\newcommand{\lnote}{\textcolor[rgb]{1,0,0}{Lydia: }\textcolor[rgb]{0,0,1}}
\newcommand{\todo}{\textcolor[rgb]{1,0,0.5}{To do: }\textcolor[rgb]{0.5,0,1}}


\newcommand{\state}{S}
\newcommand{\meas}{M}
\newcommand{\out}{\mathrm{out}}
\newcommand{\piv}{\mathrm{piv}}
\newcommand{\pivotal}{\mathrm{pivotal}}
\newcommand{\isnot}{\mathrm{not}}
\newcommand{\pred}{^\mathrm{predict}}
\newcommand{\act}{^\mathrm{act}}
\newcommand{\pre}{^\mathrm{pre}}
\newcommand{\post}{^\mathrm{post}}
\newcommand{\calM}{\mathcal{M}}

\newcommand{\game}{\mathbf{V}}
\newcommand{\strategyspace}{S}
\newcommand{\payoff}[1]{V^{#1}}
\newcommand{\eff}[1]{E^{#1}}
\newcommand{\p}{\vect{p}}
\newcommand{\simplex}[1]{\Delta^{#1}}

\newcommand{\recdec}[1]{\bar{D}(\hat{Y}_{#1})}





\newcommand{\sphereone}{\calS^1}
\newcommand{\samplen}{S^n}
\newcommand{\wA}{w}%{w_{\mathfrak{a}}}
\newcommand{\Awa}{A_{\wA}}
\newcommand{\Ytil}{\widetilde{Y}}
\newcommand{\Xtil}{\widetilde{X}}
\newcommand{\wst}{w_*}
\newcommand{\wls}{\widehat{w}_{\mathrm{LS}}}
\newcommand{\dec}{^\mathrm{dec}}
\newcommand{\sub}{^\mathrm{sub}}

\newcommand{\calP}{\mathcal{P}}
\newcommand{\totspace}{\calZ}
\newcommand{\clspace}{\calX}
\newcommand{\attspace}{\calA}

\newcommand{\Ftil}{\widetilde{\calF}}

\newcommand{\totx}{Z}
\newcommand{\classx}{X}
\newcommand{\attx}{A}
\newcommand{\calL}{\mathcal{L}}



\newcommand{\defeq}{\mathrel{\mathop:}=}
\newcommand{\vect}[1]{\ensuremath{\mathbf{#1}}}
\newcommand{\mat}[1]{\ensuremath{\mathbf{#1}}}
\newcommand{\dd}{\mathrm{d}}
\newcommand{\grad}{\nabla}
\newcommand{\hess}{\nabla^2}
\newcommand{\argmin}{\mathop{\rm argmin}}
\newcommand{\argmax}{\mathop{\rm argmax}}
\newcommand{\Ind}[1]{\mathbf{1}\{#1\}}

\newcommand{\norm}[1]{\left\|{#1}\right\|}
\newcommand{\fnorm}[1]{\|{#1}\|_{\text{F}}}
\newcommand{\spnorm}[2]{\left\| {#1} \right\|_{\text{S}({#2})}}
\newcommand{\sigmin}{\sigma_{\min}}
\newcommand{\tr}{\text{tr}}
\renewcommand{\det}{\text{det}}
\newcommand{\rank}{\text{rank}}
\newcommand{\logdet}{\text{logdet}}
\newcommand{\trans}{^{\top}}
\newcommand{\poly}{\text{poly}}
\newcommand{\polylog}{\text{polylog}}
\newcommand{\st}{\text{s.t.~}}
\newcommand{\proj}{\mathcal{P}}
\newcommand{\projII}{\mathcal{P}_{\parallel}}
\newcommand{\projT}{\mathcal{P}_{\perp}}
\newcommand{\projX}{\mathcal{P}_{\mathcal{X}^\star}}
\newcommand{\inner}[1]{\langle #1 \rangle}

\renewcommand{\Pr}{\mathbb{P}}
\newcommand{\Z}{\mathbb{Z}}
\newcommand{\N}{\mathbb{N}}
\newcommand{\R}{\mathbb{R}}
\newcommand{\E}{\mathbb{E}}
\newcommand{\F}{\mathcal{F}}
\newcommand{\var}{\mathrm{var}}
\newcommand{\cov}{\mathrm{cov}}


\newcommand{\calN}{\mathcal{N}}

\newcommand{\jccomment}{\textcolor[rgb]{1,0,0}{C: }\textcolor[rgb]{1,0,1}}
\newcommand{\fracpar}[2]{\frac{\partial #1}{\partial  #2}}

\newcommand{\A}{\mathcal{A}}
\newcommand{\B}{\mat{B}}
%\newcommand{\C}{\mat{C}}

\newcommand{\I}{\mat{I}}
\newcommand{\M}{\mat{M}}
\newcommand{\D}{\mat{D}}
%\newcommand{\U}{\mat{U}}
\newcommand{\V}{\mat{V}}
\newcommand{\W}{\mat{W}}
\newcommand{\X}{\mat{X}}
\newcommand{\Y}{\mat{Y}}
\newcommand{\mSigma}{\mat{\Sigma}}
\newcommand{\mLambda}{\mat{\Lambda}}
\newcommand{\e}{\vect{e}}
\newcommand{\g}{\vect{g}}
\renewcommand{\u}{\vect{u}}
\newcommand{\w}{\vect{w}}
\newcommand{\x}{\vect{x}}
\newcommand{\y}{\vect{y}}
\newcommand{\z}{\vect{z}}
\newcommand{\fI}{\mathfrak{I}}
\newcommand{\fS}{\mathfrak{S}}
\newcommand{\fE}{\mathfrak{E}}
\newcommand{\fF}{\mathfrak{F}}

\newcommand{\Risk}{\mathcal{R}}

\renewcommand{\L}{\mathcal{L}}
\renewcommand{\H}{\mathcal{H}}

\newcommand{\cn}{\kappa}
\newcommand{\nn}{\nonumber}


\newcommand{\Hess}{\nabla^2}
\newcommand{\tlO}{\tilde{O}}
\newcommand{\tlOmega}{\tilde{\Omega}}

\newcommand{\calF}{\mathcal{F}}
\newcommand{\fhat}{\widehat{f}}
\newcommand{\calS}{\mathcal{S}}

\newcommand{\calX}{\mathcal{X}}
\newcommand{\calY}{\mathcal{Y}}
\newcommand{\calD}{\mathcal{D}}
\newcommand{\calZ}{\mathcal{Z}}
\newcommand{\calA}{\mathcal{A}}
\newcommand{\fbayes}{f^B}
\newcommand{\func}{f^U}


\newcommand{\bayscore}{\text{calibrated Bayes score}}
\newcommand{\bayrisk}{\text{calibrated Bayes risk}}

\newtheorem{example}{Example}[section]
\newtheorem{exc}{Exercise}[section]
%\newtheorem{rem}{Remark}[section]

\newtheorem{theorem}{Theorem}[section]
\newtheorem{definition}{Definition}
\newtheorem{proposition}[theorem]{Proposition}
\newtheorem{corollary}[theorem]{Corollary}

\newtheorem{remark}{Remark}[section]
\newtheorem{lemma}[theorem]{Lemma}
\newtheorem{claim}[theorem]{Claim}
\newtheorem{fact}[theorem]{Fact}
\newtheorem{assumption}{Assumption}

\newcommand{\iidsim}{\overset{\mathrm{i.i.d.}}{\sim}}
\newcommand{\unifsim}{\overset{\mathrm{unif}}{\sim}}
\newcommand{\sign}{\mathrm{sign}}
\newcommand{\wbar}{\overline{w}}
\newcommand{\what}{\widehat{w}}
\newcommand{\KL}{\mathrm{KL}}
\newcommand{\Bern}{\mathrm{Bernoulli}}
\newcommand{\ihat}{\widehat{i}}
\newcommand{\Dwst}{\calD^{w_*}}
\newcommand{\fls}{\widehat{f}_{n}}


\newcommand{\brpi}{\pi^{br}}
\newcommand{\brtheta}{\theta^{br}}

% \newcommand{\M}{\mat{M}}
% \newcommand\Mmh{\mat{M}^{-1/2}}
% \newcommand{\A}{\mat{A}}
% \newcommand{\B}{\mat{B}}
% \newcommand{\C}{\mat{C}}
% \newcommand{\Et}[1][t]{\mat{E_{#1}}}
% \newcommand{\Etp}{\Et[t+1]}
% \newcommand{\Errt}[1][t]{\mat{\bigtriangleup_{#1}}}
% \newcommand\cnM{\kappa}
% \newcommand{\cn}[1]{\kappa\left(#1\right)}
% \newcommand\X{\mat{X}}
% \newcommand\fstar{f_*}
% \newcommand\Xt[1][t]{\mat{X_{#1}}}
% \newcommand\ut[1][t]{{u_{#1}}}
% \newcommand\Xtinv{\inv{\Xt}}
% \newcommand\Xtp{\mat{X_{t+1}}}
% \newcommand\Xtpinv{\inv{\left(\mat{X_{t+1}}\right)}}
% \newcommand\U{\mat{U}}
% \newcommand\UTr{\trans{\mat{U}}}
% \newcommand{\Ut}[1][t]{\mat{U_{#1}}}
% \newcommand{\Utinv}{\inv{\Ut}}
% \newcommand{\UtTr}[1][t]{\trans{\mat{U_{#1}}}}
% \newcommand\Utp{\mat{U_{t+1}}}
% \newcommand\UtpTr{\trans{\mat{U}_{t+1}}}
% \newcommand\Utptild{\mat{\widetilde{U}_{t+1}}}
% \newcommand\Us{\mat{U^*}}
% \newcommand\UsTr{\trans{\mat{U^*}}}
% \newcommand{\Sigs}{\mat{\Sigma}}
% \newcommand{\Sigsmh}{\Sigs^{-1/2}}
% \newcommand{\eye}{\mat{I}}
% \newcommand{\twonormbound}{\left(4+\DPhi{\M}{\Xt[0]}\right)\twonorm{\M}}
% \newcommand{\lamj}{\lambda_j}

% \renewcommand\u{\vect{u}}
% \newcommand\uTr{\trans{\vect{u}}}
% \renewcommand\v{\vect{v}}
% \newcommand\vTr{\trans{\vect{v}}}
% \newcommand\w{\vect{w}}
% \newcommand\wTr{\trans{\vect{w}}}
% \newcommand\wperp{\vect{w}_{\perp}}
% \newcommand\wperpTr{\trans{\vect{w}_{\perp}}}
% \newcommand\wj{\vect{w_j}}
% \newcommand\vj{\vect{v_j}}
% \newcommand\wjTr{\trans{\vect{w_j}}}
% \newcommand\vjTr{\trans{\vect{v_j}}}

% \newcommand{\DPhi}[2]{\ensuremath{D_{\Phi}\left(#1,#2\right)}}
% \newcommand\matmult{{\omega}}


\usepackage{enumerate,color,multirow}
\usepackage{xpatch}
\usepackage[suppress]{color-edits}
\addauthor{ab}{violet}
%\addauthor{abb}{magenta}
\addauthor{bs}{blue}
\addauthor{pc}{olive}


\title{Optimization for Neural Operators can Benefit from Width}

\author{%
%\hspace*{-10mm}
  Pedro Cisneros-Velarde$^\dagger$\\
    \normalsize{VMware Research}\\
    \texttt{pacisne@gmail.com}
    \and
    Bhavesh Shrimali$^\dagger$\\
    \normalsize{Corporate Research, Kimberly-Clark}\\
    \texttt{bhavesh.shrimali@gmail.com}
    \and
  Arindam Banerjee \\ 
  \normalsize{University of Illinois Urbana-Champaign}\\
  \texttt{arindamb@illinois.edu}
} 


%\vspace{50pt}


\date{}



\begin{document}
\maketitle
\def\thefootnote{$\dagger$}\footnotetext{These authors contributed equally to this work.}\def\thefootnote{\arabic{footnote}}



\begin{abstract}
    Neural Operators that directly learn mappings between function spaces, such as Deep Operator Networks (DONs) and Fourier Neural Operators (FNOs), have received considerable attention. Despite the universal approximation guarantees for DONs and FNOs, there is currently no optimization convergence guarantee for learning such networks using gradient descent (GD). In this paper, we address this open problem by presenting a unified framework for optimization based on GD and applying it to establish convergence guarantees for both DONs and FNOs. 
    In particular, we show that the losses associated with both of these neural operators satisfy two conditions---restricted strong convexity (RSC) and smoothness---that guarantee a decrease on their loss values due to GD. Remarkably, these two conditions are satisfied for each neural operator due to different reasons associated with the architectural differences of the respective models. One takeaway that emerges from the theory is that wider networks should lead to better optimization convergence for both DONs and FNOs. 
    We present empirical results on canonical operator learning problems to support our theoretical results.
\end{abstract}

\section{Introduction}
\label{sec:intro}
\section{Introduction}
\label{sec:introduction}
The business processes of organizations are experiencing ever-increasing complexity due to the large amount of data, high number of users, and high-tech devices involved \cite{martin2021pmopportunitieschallenges, beerepoot2023biggestbpmproblems}. This complexity may cause business processes to deviate from normal control flow due to unforeseen and disruptive anomalies \cite{adams2023proceddsriftdetection}. These control-flow anomalies manifest as unknown, skipped, and wrongly-ordered activities in the traces of event logs monitored from the execution of business processes \cite{ko2023adsystematicreview}. For the sake of clarity, let us consider an illustrative example of such anomalies. Figure \ref{FP_ANOMALIES} shows a so-called event log footprint, which captures the control flow relations of four activities of a hypothetical event log. In particular, this footprint captures the control-flow relations between activities \texttt{a}, \texttt{b}, \texttt{c} and \texttt{d}. These are the causal ($\rightarrow$) relation, concurrent ($\parallel$) relation, and other ($\#$) relations such as exclusivity or non-local dependency \cite{aalst2022pmhandbook}. In addition, on the right are six traces, of which five exhibit skipped, wrongly-ordered and unknown control-flow anomalies. For example, $\langle$\texttt{a b d}$\rangle$ has a skipped activity, which is \texttt{c}. Because of this skipped activity, the control-flow relation \texttt{b}$\,\#\,$\texttt{d} is violated, since \texttt{d} directly follows \texttt{b} in the anomalous trace.
\begin{figure}[!t]
\centering
\includegraphics[width=0.9\columnwidth]{images/FP_ANOMALIES.png}
\caption{An example event log footprint with six traces, of which five exhibit control-flow anomalies.}
\label{FP_ANOMALIES}
\end{figure}

\subsection{Control-flow anomaly detection}
Control-flow anomaly detection techniques aim to characterize the normal control flow from event logs and verify whether these deviations occur in new event logs \cite{ko2023adsystematicreview}. To develop control-flow anomaly detection techniques, \revision{process mining} has seen widespread adoption owing to process discovery and \revision{conformance checking}. On the one hand, process discovery is a set of algorithms that encode control-flow relations as a set of model elements and constraints according to a given modeling formalism \cite{aalst2022pmhandbook}; hereafter, we refer to the Petri net, a widespread modeling formalism. On the other hand, \revision{conformance checking} is an explainable set of algorithms that allows linking any deviations with the reference Petri net and providing the fitness measure, namely a measure of how much the Petri net fits the new event log \cite{aalst2022pmhandbook}. Many control-flow anomaly detection techniques based on \revision{conformance checking} (hereafter, \revision{conformance checking}-based techniques) use the fitness measure to determine whether an event log is anomalous \cite{bezerra2009pmad, bezerra2013adlogspais, myers2018icsadpm, pecchia2020applicationfailuresanalysispm}. 

The scientific literature also includes many \revision{conformance checking}-independent techniques for control-flow anomaly detection that combine specific types of trace encodings with machine/deep learning \cite{ko2023adsystematicreview, tavares2023pmtraceencoding}. Whereas these techniques are very effective, their explainability is challenging due to both the type of trace encoding employed and the machine/deep learning model used \cite{rawal2022trustworthyaiadvances,li2023explainablead}. Hence, in the following, we focus on the shortcomings of \revision{conformance checking}-based techniques to investigate whether it is possible to support the development of competitive control-flow anomaly detection techniques while maintaining the explainable nature of \revision{conformance checking}.
\begin{figure}[!t]
\centering
\includegraphics[width=\columnwidth]{images/HIGH_LEVEL_VIEW.png}
\caption{A high-level view of the proposed framework for combining \revision{process mining}-based feature extraction with dimensionality reduction for control-flow anomaly detection.}
\label{HIGH_LEVEL_VIEW}
\end{figure}

\subsection{Shortcomings of \revision{conformance checking}-based techniques}
Unfortunately, the detection effectiveness of \revision{conformance checking}-based techniques is affected by noisy data and low-quality Petri nets, which may be due to human errors in the modeling process or representational bias of process discovery algorithms \cite{bezerra2013adlogspais, pecchia2020applicationfailuresanalysispm, aalst2016pm}. Specifically, on the one hand, noisy data may introduce infrequent and deceptive control-flow relations that may result in inconsistent fitness measures, whereas, on the other hand, checking event logs against a low-quality Petri net could lead to an unreliable distribution of fitness measures. Nonetheless, such Petri nets can still be used as references to obtain insightful information for \revision{process mining}-based feature extraction, supporting the development of competitive and explainable \revision{conformance checking}-based techniques for control-flow anomaly detection despite the problems above. For example, a few works outline that token-based \revision{conformance checking} can be used for \revision{process mining}-based feature extraction to build tabular data and develop effective \revision{conformance checking}-based techniques for control-flow anomaly detection \cite{singh2022lapmsh, debenedictis2023dtadiiot}. However, to the best of our knowledge, the scientific literature lacks a structured proposal for \revision{process mining}-based feature extraction using the state-of-the-art \revision{conformance checking} variant, namely alignment-based \revision{conformance checking}.

\subsection{Contributions}
We propose a novel \revision{process mining}-based feature extraction approach with alignment-based \revision{conformance checking}. This variant aligns the deviating control flow with a reference Petri net; the resulting alignment can be inspected to extract additional statistics such as the number of times a given activity caused mismatches \cite{aalst2022pmhandbook}. We integrate this approach into a flexible and explainable framework for developing techniques for control-flow anomaly detection. The framework combines \revision{process mining}-based feature extraction and dimensionality reduction to handle high-dimensional feature sets, achieve detection effectiveness, and support explainability. Notably, in addition to our proposed \revision{process mining}-based feature extraction approach, the framework allows employing other approaches, enabling a fair comparison of multiple \revision{conformance checking}-based and \revision{conformance checking}-independent techniques for control-flow anomaly detection. Figure \ref{HIGH_LEVEL_VIEW} shows a high-level view of the framework. Business processes are monitored, and event logs obtained from the database of information systems. Subsequently, \revision{process mining}-based feature extraction is applied to these event logs and tabular data input to dimensionality reduction to identify control-flow anomalies. We apply several \revision{conformance checking}-based and \revision{conformance checking}-independent framework techniques to publicly available datasets, simulated data of a case study from railways, and real-world data of a case study from healthcare. We show that the framework techniques implementing our approach outperform the baseline \revision{conformance checking}-based techniques while maintaining the explainable nature of \revision{conformance checking}.

In summary, the contributions of this paper are as follows.
\begin{itemize}
    \item{
        A novel \revision{process mining}-based feature extraction approach to support the development of competitive and explainable \revision{conformance checking}-based techniques for control-flow anomaly detection.
    }
    \item{
        A flexible and explainable framework for developing techniques for control-flow anomaly detection using \revision{process mining}-based feature extraction and dimensionality reduction.
    }
    \item{
        Application to synthetic and real-world datasets of several \revision{conformance checking}-based and \revision{conformance checking}-independent framework techniques, evaluating their detection effectiveness and explainability.
    }
\end{itemize}

The rest of the paper is organized as follows.
\begin{itemize}
    \item Section \ref{sec:related_work} reviews the existing techniques for control-flow anomaly detection, categorizing them into \revision{conformance checking}-based and \revision{conformance checking}-independent techniques.
    \item Section \ref{sec:abccfe} provides the preliminaries of \revision{process mining} to establish the notation used throughout the paper, and delves into the details of the proposed \revision{process mining}-based feature extraction approach with alignment-based \revision{conformance checking}.
    \item Section \ref{sec:framework} describes the framework for developing \revision{conformance checking}-based and \revision{conformance checking}-independent techniques for control-flow anomaly detection that combine \revision{process mining}-based feature extraction and dimensionality reduction.
    \item Section \ref{sec:evaluation} presents the experiments conducted with multiple framework and baseline techniques using data from publicly available datasets and case studies.
    \item Section \ref{sec:conclusions} draws the conclusions and presents future work.
\end{itemize}
%
\section{Related Work}
\label{sec:related}

We only provide a brief overview of the literature related to our work and provide a more extensive treatment in Appendix~\ref{app:related}. In the case of DONs, approximation~\citep{lu20201DeepONet} and generalization~\citep{kontolati2022_Over_parameterization} properties have been formally studied, as well as several applications of DONs~\citep{goswami_physics-informed_2022,wang_long-time_2021,diab2024u,centofanti2024learning,sun2023deepgraphonet}. Nevertheless, optimization guarantees for DONs is an open problem.
Approximation properties for FNOs have been formally studied~\citep{kovachki2021universal}, and diverse applications of FNOs and various Fourier-based operators have been formulated~\citep{li_multipole_2020,liu_learning-based_2022,wen_u-fnoenhanced_2022,pathak_fourcastnet_2022,centofanti2024learning,li2023fourier,yang2023fourier,harder2023hard}. Nevertheless, optimization guarantees for DONs is also an open problem. 
%
Though formal optimization guarantees for neural operators are largely absent, there is a more established literature on such guarantees for neural networks. We highlight two particular approaches for optimization analysis: based on the NTK approach~\citep{jacot2018neural,liu_linearity_2021,banerjee23a,du2019gradient,allen-zhu_convergence_2019} and on the RSC approach~\citep{banerjee2022restricted,cisnerosvelarde2024optgenWeightNorm}---our work is related to the latter.
%
\section{Learning Neural Operators}
\label{sec:modelSetup}
%In general, 
A neural operator~\citep{li_fourier_2021,li_neural_2020,lu20201DeepONet} is a parametric model based on neural networks that aims to best approximate a 
%(possibly) nonlinear 
%\emph{ground truth} 
mapping between two function spaces, which %---these mappings 
can be linear, such as the antiderivative or integral operator, or nonlinear such as the solution operator of a nonlinear PDE. % (e.g., Hyperelasticity). 
\pcedit{Thus, letting $G^\dagger$ denote the ground-truth operator we are trying to approximate and $G_{\vtheta}$ denote the neural operator parameterized by the parameter vector $\vtheta$, the objective is to \emph{learn} $\vtheta$ such that, given an input function $\vu$, we have $G_{\vtheta}(\vu)\approx G^{\dagger}(\vu)$.} 
%Learning such approximation 
Such learning 
is done by solving an optimization problem  
%where learning such best approximation is done 
using data samples consisting of tuples of input and output function values of $G^\dagger$. %the ground truth operator.
%
This optimization problem is analogous to the notion of learning in finite dimensions, which is precisely the setup for which classical deep learning is used.

We now introduce %two widely known neural operators: 
DONs and FNOs. %\pcedit{We denote by $G^\dagger$ the true operator we are trying to approximate, and denote the neural operator (DON or FNO) by $G_{\vtheta}$ and parameterized by the parameter vector $\vtheta$. The idea would be to \emph{learn} $\vtheta$ such that, given an input function $\vu$, we have $G_{\vtheta}(\vu)\approx G^{\dagger}(\vu)$.} 
More information about neural operators and the schematics of both DONs and FNOs %these two 
are found in Appendix~\ref{app:learning_fno_don}.


%
%
\subsection{Learning Deep Operator Networks (DONs)}
\label{subsec:DON_Setup} 
%
The DON model~\citep{lu20201DeepONet} is defined as the inner product of two deep feedforward neural networks, each one with $K$ output neurons. Given the 
% of real output: 
the branch net $\vf = \{f_k\}_{k=1}^{K}$ and {the trunk net}  $\vg = \{g_k\}_{k=1}^{K}$, the DON is 
\begin{equation}
    G_{\vtheta}(\vu)(\vy) := \sum_{k=1}^K f_k(\vtheta_f;\vu) g_k(\vtheta_g;\vy),
    \label{eq:DONoutput}
\end{equation}
where the input function $\vu$ has $\operatorname{ran}(\vu)\subseteq\R^{d_u}$ 
and $\vy \in \dom (G_{\vtheta}(\vu))\subseteq \R^{d_y}$ is the output location on which the operator is evaluated. 
The training data is composed of $n$ input functions $\{\vu^{(i)}\}_{i=1}^n$ and $q_i$ output locations for each $G^\dagger(\vu^{(i)})$, i.e., $ \{\{\vy^{(i)}_j\}_{j=1}^{q_i}\}_{i=1}^n$ with $\vy^{(i)}_j\in\R^{d_y}$ denoting the $j$-th output location for $G^\dagger_{\vtheta}(\vu^{(i)})$. 
Each $\vu^{(i)}$ is represented in $R$ locations $\{\bfx_r\}_{r=1}^{R}$
%\in \dom (\vu)\subseteq\R^{d_x}$ 
so that $\vu^{(i)}(\bfx_r)\in\R^{d_u}$, $r\in [R]$. 
% 
%
The entire set of parameters is $\vtheta = [\vtheta_f^{\top}\;\vtheta_g^{\top}]^\top \in \R^{p_f+p_g}$, where $\vtheta_f\in \R^{p_f}$ and $\vtheta_g\in\R^{p_g}$ are the parameter vectors of $\vf$ and $\vg$ respectively. 

We only consider scalar input functions, i.e., 
$d_u = 1$.  
%$\operatorname{ran}(u)\subseteq\R$. 
For each $i\in[n]$, we stack $\{u^{(i)}(\vx_r)\}_{r=1}^R$ as an input vector to $\vf$, thus, $\vf:\R^R\to \R^K$. Note that $\vg:\R^{d_y}\to\R^K$. Then, the DON learning problem is %formulated as 
the minimization: 
\begin{align}
    \begin{aligned}
        \vtheta^\dagger_{\rm (don)} &\in 
    \underset{\vtheta\in\R^{p_f+p_g}}{\argmin}~
    \gL\left(G_{\vtheta}, G^{\dagger}\right)
    \label{eq:empirical_risk}
    \end{aligned}
\end{align}
where 
\begin{equation}
\label{eq:loss-don}
\begin{aligned}
&\gL\left(G_{\vtheta}, G^{\dagger}\right)= \frac{1}{n}\sum_{i=1}^n \frac{1}{q_i} \sum_{j=1}^{q_i} 
    \left(
        G_{\vtheta}(u^{(i)})(\vy^{(i)}_j) - G^{\dagger}(u^{(i)})(\vy^{(i)}_j) 
    \right)^2
\end{aligned}
\end{equation}
is the empirical loss function that measures the approximation between $G_{\vtheta}$ and $G^\dagger$, and where
$G_{\vtheta}(u^{(i)})(\vy^{(i)}_j)=\sum_{k=1}^K f_k\left(\vtheta_f;\{u^{(i)}(\vx_r)\}_{r=1}^R\right) g_k\left(\vtheta_g;\vy^{(i)}_j\right)$.

Note that the ground truth operator $G^{\dagger}$  
can either be explicit, e.g. integral of a function, or implicit, e.g. the solution to a nonlinear partial differential equation (PDE). 

\subsection{Learning Fourier Neural Operators (FNOs)}
\label{subsec:FNO_setup} 
The FNO model~\citep{li_fourier_2021} is defined as follows: $G_{\vtheta}(\vu)(\vx):=f(\vtheta;\vx)$ with 
\begin{equation}
\label{eq:continuous_fno}
\begin{aligned}
    \aalpha^{(0)}(\vx) &= P(\vu;\vtheta_p)(\vx)\\ 
    \aalpha^{(l)}(\vx) &= \gF^{(l)}(\aalpha^{(l-1)}(\vx); \vtheta_{F^{{(l)}}}),\; l\in[L+1]\\ 
    %G_{\vtheta}(\vu)(\vx)=f(\vtheta;\vx) &= Q(\aalpha^{({L+1})};\vtheta_q)(\vx),
    f(\vtheta;\vx) &= Q(\aalpha^{({L+1})};\vtheta_q)(\vx),
\end{aligned}
\end{equation} 
where the input function $\vu$ has $\operatorname{ran}(\vu)\subseteq\R^{d_u}$, 
$G_{\vtheta}(\vu)(\vx)\in\R$ is the output of the FNO evaluated at output location $\vx\in\R^{d_x}$,
%where 
$\{\gF^{(l)}\}_{l=1}^{L+1}$ are nonlinear transformations with learnable parameters $\vtheta_{F} = [\vtheta_{F^{(1)}}^{\top}, \dots, \vtheta_{F^{(L+1)}}^\top]^\top\in\R^{F}$ and which may contain operations in the Fourier domain, $P$ is an encoder that maps $\vu$ and $\vx$ to an ambient space of dimension $d$ and has parameter vector $\vtheta_p\in\R^{p}$, and $Q$ is a decoder that maps the output from the block $\aalpha^{(L+1)}(\vx)$ to a scalar output 
%the desired output space of dimension $d_f$ 
with parameter vector $\vtheta_q\in\R^q$. 
The entire set of parameters for the FNO can be written as $\vtheta = \left[ \vtheta_p^{\top} \ \vtheta_{F}^{\top}\ \vtheta_q^{\top}\right]^{\top}$. 
With a slight abuse of notation, the FNO is simply written as $G_{\vtheta}(\vu)(\vx)=f(\vtheta;\vx)$ in~\eqref{eq:continuous_fno} when the input function $\vu$ is known by the context.
%
%
%
%\end{equation}

The training data is composed of $n$ input-output pairs $\{(\vu^{(i)},G^\dagger(\vu^{(i)})\}_{i=1}^n$ and a computational grid of evaluations $\{\vx_{r}\}_{r=1}^R$. We let $f^{(i)}(\vtheta;\vx_r)$ denote the FNO model~\eqref{eq:continuous_fno} with input function $\vu^{(i)}$ and evaluated at $\vx_r$. Then, the FNO learning problem is %formulated as 
the minimization:
%of the following empirical risk:
\begin{equation}
    \vtheta^{\dagger}_{\rm (fno)} \in \underset{{\vtheta \in \R^{p+F+q}}}{\argmin}~ \gL (G_{\vtheta}, G^{\dagger}) 
    \label{eq:fno_loss}
\end{equation}
with empirical loss function
\begin{equation}
\label{eq:loss-fno}
\begin{aligned}
    &\gL (G_{\vtheta}, G^{\dagger})=\frac{1}{n}\sum_{i=1}^n \frac{1}{R} \sum_{r=1}^R \left( 
        G_{\vtheta}({\vu^{(i)}})(\vx_r) - G^{\dagger}(\vu^{(i)})(\vx_r)
    \right)^2
\end{aligned}
\end{equation}
and where $G_{\vtheta}(\vu^{(i)})(\vx_{r})=\vf^{(i)}(\vtheta;\vx_{r})$.

%where $\ell(z) = \frac{1}{2}z^2$.  
%


% \section{Restricted Strong Convexity of Neural Operators Models}
% \label{sec:rscNeuralModels}
% \input{Sections/rsc}

\section{Optimization Convergence Framework}
\label{sec:optmain}
We now establish {\em two conditions}---Conditions \ref{cond:rsc} and~\ref{cond:smooth} below---for the convergence of gradient descent (GD) when minimizing a 
%of any model whose training can be expressed as the minimization of a 
loss function $\cL$.
%---including neural operators. 
We show that as long as these two conditions are satisfied, the loss will decrease in value. 
In the following sections we show how the empirical losses used for training DONs (Section~\ref{sec:optDON}) and FNOs (Section~\ref{sec:optFNO}), as in \eqref{eq:loss-don} and~\eqref{eq:loss-fno} respectively, satisfy these two conditions. % properties specific to these models.
%
    
%

\pcedit{We consider $\vtheta\mapsto\gL(\vtheta)$ to be continuously differentiable.} Let $\vtheta_0\in\R^p$ be a suitable initialization point and $\{\vtheta_t\}_{t\geq 1}$ be the sequence of iterates obtained from GD on loss $\cL$ for some step-size $\eta_t>0$, i.e.,
\begin{equation}
\vtheta_{t+1} = \vtheta_{t} - \eta_{t} \nabla_{\vtheta} \gL(\vtheta_t )~.
    \label{eq:gd_at_t}
\end{equation}
We consider a non-empty set $\cB(\vtheta_0)\subseteq\R^p$ around and including $\vtheta_0$. 

\begin{asmp}[{\bf Iterates inside $\cB(\vtheta_0)$}]
\label{asmp:iter-0}
All iterates $\{\vtheta_t\}_{t\geq 1}$ follow GD as in~\eqref{eq:gd_at_t} and are inside the set $\cB(\vtheta_0)$.
\end{asmp}

The first condition is based on the concept of Restricted Strong Convexity (RSC) being satisfied for $\cL$. % at time step $t$. 
\begin{defn}[{\bf Restricted strong convexity (RSC)}] A function $\mathcal{L}$ is said to satisfy $\alpha$-restricted strong convexity ($\alpha$-RSC) w.r.t.~the tuple $(\mathcal{S}, \vtheta)$ if for any $\vtheta^{\prime} \in \mathcal{S} \subseteq \mathbb{R}^p$ and some fixed $\vtheta \in \mathbb{R}^p$, we have 
\begin{align}
\label{eq:RSC-prim}
\mathcal{L}\left(\vtheta^{\prime}\right) \geq \mathcal{L}(\vtheta) + \left\langle\vtheta^{\prime}-\vtheta, \nabla_\vtheta \mathcal{L}(\vtheta)\right\rangle+\frac{\alpha}{2}\left\|\vtheta^{\prime}-\vtheta\right\|_2^2~, 
\end{align}
with $\alpha>0$.
\end{defn}
%
%
%

%Thus, the first condition of interest stipulates that the loss $\cL$ satisfies the RSC condition at step $t$.
%
\begin{cond}[{\bf RSC}]
Consider Assumption~\ref{asmp:iter-0}.
At step $t$, 
there exists a non-empty set $\mathcal{N}_t$ 
% B^{\mathrm{Euc}}_{\rho}(\vtheta_0)
such that:
\begin{itemize}
\item[(a)] %$\vtheta_0$
%$\vtheta_t,
$\mathcal{N}_t\subseteq \cB(\vtheta_0)$;
\item[(b)] one of these two conditions hold:
\begin{itemize}
    \item[(b.1)] $\vtheta_{t+1}\in \mathcal{N}_t$ with either $\vtheta_{t}\notin \mathcal{N}_t$ or $\gL(\vtheta_{t})\neq\inf_{\vtheta\in\mathcal{N}_t}\cL(\vtheta)$,\label{condb1} 
    \item[(b.2)] there exists some $\vtheta'\in\mathcal{N}_t$ such that $\cL(\vtheta')<\cL(\vtheta_{t})$;\label{condb2}
\end{itemize}
\item[(c)] $\gL$ satisfies $\alpha_t$-RSC w.r.t.~$(\mathcal{N}_t,\vtheta_t)$ for some $\alpha_t > 0$.
\end{itemize}
\label{cond:rsc}
\end{cond}

%
Note that $\gL$ need not be convex for it to satisfy $\alpha_t$-RSC.
%==
%==

The second condition is based on the smoothness of $\gL$. 
\begin{cond}[{\bf Smoothness}]
The function $\gL$ is $\beta$-smooth, i.e., for $\vtheta',\vtheta \in \cB(\vtheta_0)$
%B^{\mathrm{Euc}}_{\rho}(\vtheta_0)$ 
and some $\beta>0$, 
%\begin{align}
$\gL(\vtheta') \leq \gL(\vtheta) + \langle \vtheta'-\vtheta, \nabla_{\vtheta} \gL(\vtheta) \rangle + \frac{\beta}{2} \| \vtheta'-\vtheta \|^2_2$.
%\end{align}
\label{cond:smooth}
\end{cond}
As long as Conditions~\ref{cond:rsc} and~\ref{cond:smooth} are satisfied at step $t$ of the GD update in \eqref{eq:gd_at_t}, the loss is guaranteed to decrease with a suitable step-size choice. 
%
%
\begin{restatable}[{\bf Global loss reduction}]{theo}{GlobalLossSmooth}
Consider Assumption~\ref{asmp:iter-0} and Conditions~\ref{cond:rsc} and \ref{cond:smooth} with $\alpha_t \leq \beta$ at step $t$ of the GD update~\eqref{eq:gd_at_t} with step-size $\eta_t=\frac{\omega_t}{\beta}$ for some $\omega_t \in(0,2)$. 
\pcedit{If $\gL(\vtheta_t)\neq \underset{\vtheta \in \cB(\vtheta_0)}{\inf} \gL(\vtheta)$, then}
%Then,
%
we have $0 \leq \gamma_t := \frac{\underset{\vtheta \in \mathcal{N}_t}{\inf} \gL(\vtheta) - \underset{\vtheta \in \cB(\vtheta_0)}{\inf} \gL(\vtheta)}{\gL(\vtheta_t) - \underset{\vtheta \in \cB(\vtheta_0)}{\inf} \gL(\vtheta)} < 1$ 
%
and 
\begin{equation}
\begin{aligned}
    &\gL(\vtheta_{t+1}) - \underset{\vtheta \in \cB(\vtheta_0)}{\inf} \gL(\vtheta)\leq \left(1-\frac{\alpha_t \omega_t (1-\gamma_t)}{\beta}(2-\omega_t) \right) (\gL(\vtheta_t) - \underset{\vtheta \in \cB(\vtheta_0)}{\inf} \gL(\vtheta)).
\end{aligned}
\label{eq:conv-0}
\end{equation}
%
%==================================
%%%
%===============================
\label{theo:global_main}
\end{restatable}
Theorem~\ref{theo:global_main}'s proof is found in Appendix~\ref{app:rscopt}.
\pcedit{We note that if the infimum loss inside $\cB(\vtheta_0)$ is attained at time $t$, i.e., $\cL(\vtheta_t)=\underset{\vtheta \in \cB(\vtheta_0)}{\inf} \gL(\vtheta)$,
%\min_{\vtheta\in\cB(\vtheta_0)}\cL(\vtheta_0)$, 
then there is nothing to prove---hence the conditional in the second sentence of Theorem~\ref{theo:global_main}.}

\pcedit{
\begin{remark}[The RSC to smoothness ratio] 
\label{rem:ratio}
Theorem~\ref{theo:global_main} requires $\alpha_t/\beta\leq 1$, which needs to be proved for the particular function $\gL$ being considered. 
%
If \eqref{cond:rsc} were to hold for any $\vtheta,\vtheta'\in\cB(\vtheta_0)$, then $\gL$ would be a \emph{locally strongly convex} function in the set $\cB(\vtheta_0)$~\citep{Boyd_Vandenberghe_2004}. This is a stronger condition on $\gL$ which makes $\alpha$ in \eqref{cond:rsc} \emph{independent} from the choice of $\vtheta$ (in the context of Theorem~\ref{theo:global_main}, $\alpha_t$ would be independent from $t$), which immediately implies $\alpha/\beta<1$.
\qed
\end{remark}}

Our analysis is inspired by the recent works \citep{banerjee2022restricted} and~\citep{cisnerosvelarde2024optgenWeightNorm}, where optimization guarantees were done for feedforward networks and normalization. We abstract out from those special cases, and demonstrate that our analysis works for any losses satisfying Conditions~\ref{cond:rsc} and \ref{cond:smooth}---indeed, \citep{cisnerosvelarde2024optgenWeightNorm} particularly satisfies Condition~\ref{cond:rsc}(b.1) and   \citep{banerjee2022restricted} satisfies Condition~\ref{cond:rsc}(b.2). Thus, in the context of our paper, \textbf{the largest effort in %heavy lifting for 
establishing optimization guarantees for DONs and FNOs is to show these two models satisfy Conditions~\ref{cond:rsc} and \ref{cond:smooth} \pcedit{with $\alpha_t/\beta\leq 1$}}.
%
\section{Optimization Analysis for DON}
\label{sec:optDON}
We consider, analogous to \citep{liu_loss_2021}, the branch net as a fully connected feedforward neural network:
\begin{align}
    \begin{aligned}
        \aalpha^{(0)}_{f} &= \bfu(\vx) \\
        \aalpha^{(l)}_{f} &= \phi\left( \frac{1}{\sqrt{m_{\rf}}} W^{(l)}_f\aalpha^{(l-1)}_{f}  \right),\; l \in [L-1]\\ 
        \vf =  \aalpha^{(L)}_{f} &= 
        % \frac{1}{\sqrt{m_{f}}}
        \frac{1}{\sqrt{m_{f}}} W_f^{(L)}\aalpha^{(L-1)}_f
%        f_k &= \aalpha^{(L)}_{f,k},\quad \forall k\in [K]~,
    \end{aligned}
    \label{eq:BranchNetRegularized}
\end{align}
where with some abuse of notation $\bfu(\vx):=[u(\vx_1),\dots,u(\vx_R)]^\top$ is the vector of all scalar evaluations of $u$ at each of the $R$ locations, $\phi$ is a pointwise smooth activation function, $\aalpha^{(l)}_f$ is the output at layer $l\in[L]$, and the weight matrices are $W^{(1)}_f\in\R^{m_f\times R}$ and $W^{(l)}_f \in \R^{m_f\times m_f}$ at layer $l\in\{2,\dots,L-1\}$. The branch net has width $m_f$ (all hidden layers have the same width). % and $L$ respectively. 
%
Similarly, the trunk net is a fully connected feedforward network:
\begin{align}
    \begin{aligned}
        \aalpha^{(0)}_{g} &= \vy\\
        \aalpha^{(l)}_{g} &= \phi\left( \frac{1}{\sqrt{m_{g}}} W^{(l)}_g \aalpha^{(l-1)}_{g}  \right),\; l \in [L-1]\\ 
        \vg = \aalpha^{(L)}_{g} &= 
        % \frac{1}{\sqrt{m_{g}}}
        \frac{1}{\sqrt{m_{g}}} W_g^{(L)}\aalpha^{(L - 1)}_{g}
%        g_k &= \aalpha^{(L)}_{g,k},\quad \forall k\in [K]~,
    \end{aligned}
    \label{eq:TrunkNetRegularized}
\end{align}
where $\vy\in\R^{d_y}$ is the output location, and the weight matrices are $W^{(1)}_g\in\R^{m_g\times d_y}$ and $W^{(l)}_g\in \R^{m_g\times m_g}$ at layer $l\in\{2,\dots,L-1\}$. The trunk net 
has width $m_g$
(all hidden layers have the same width).
%and $L$ respectively. 
Finally, we recall that we have $K$ outputs on each network, i.e., $W^{(L)}_f\in\mathbb{R}^{K\times m_f}$ and $W^{(L)}_g\in\mathbb{R}^{K\times m_g}$.
% 
Given $l\in[L]$, we denote by $(w^{(l)}_{f,{k}})^\top$ and $(w^{(l)}_{g,{k}})^\top$ the $k$-th row of the matrices $W^{(l)}_f$ and $W^{(l)}_g$ respectively, and by $w_{f,{ij}}^{(l)}$ and $w_{g,{ij}}^{(l)}$ their respective $ij$-entry. 
%
%
Using the notation in Section~\ref{subsec:DON_Setup}, the set of trainable parameters is $\vtheta = [\vtheta_f^{\top}\
\vtheta_g^{\top}]^{\top}\in\R^{p_f+p_g}$, with $\vtheta_f = [\text{vec}(W^{(1)}_f)^{\top},\dots,\text{vec}(W^{(L)}_f)^{\top}]^{\top}$ and $\vtheta_g = [\text{vec}(W^{(2)}_g)^\top,\dots,\text{vec}(W^{(L)}_g)^{\top}]^{\top}$. 
Let $\vtheta_0$ be the parameter vector at initialization and $\vtheta_t$ be it at time step $t$. 

We make the following assumptions for our analysis: 
%on the activations, the loss, and the weights:
\begin{asmp}[{\bf Activation functions}]
\label{asmp:Activation_Function}
The activation function $\phi$ of the DON is $1$-Lipschitz and $\beta_{\phi}$-smooth (i.e. $\phi^{\prime\prime}\leq \beta_{\phi}$) for some $\beta_{\phi} > 0$.
\end{asmp}
\begin{asmp}[{\bf Initialization of weights}]
\label{asmp:smoothinit}
All weights of the branch and trunk nets are initialized independently as follows: (i) $w^{(l)}_{f_{0,\,ij}}\sim \gN (0, \sigma^2_{f,0})$ and $w^{(l)}_{g_{0,\,ij}}\sim \gN (0, \sigma^2_{g,0})$ for $l\in [L-1]$ where {$\sigma_{f,0} = \frac{\sigma_0}{2(1+\frac{\sqrt{\log m_f}}{\sqrt{2m_f}})}$ and $\sigma_{g,0} = \frac{\sigma_0}{2(1+\frac{\sqrt{\log m_g}}{\sqrt{2m_g}})}$}, $\sigma_0>0$; (ii) $w^{(L)}_{f_{0},k}$ and $w^{(L)}_{g_{0},k}~$, $k\in[K]$, are random vectors with unit norms, i.e., $\norm{w^{(L)}_{f_{0},k}}_2=1$ and $\norm{w^{(L)}_{g_{0},k}}_2=1$. Further, we assume the input to the branches are normalized as $\norm{\vu(\vx)}_2=\sqrt{R}$ and $\norm{\vy}_2=\sqrt{d_y}$.
\end{asmp}

For a given parameter vector $\bar{\vtheta}=[\bar{\vtheta}_f^\top,\bar{\vtheta}_g^\top]\in\R^{p_f+p_g}$, we introduce the neighborhood set  
$B^{\mathrm{Euc}}_{\rho,\rho_1}(\bar{\vtheta})=\{\vtheta\in\mathbb{R}^{p_f+p_g}\,:\,\norm{W^{(l)}_f-\bar{W}^{(l)}_f}_2\leq \rho,\,\norm{W^{(l)}_g-\bar{W}^{(l)}_g }_2\leq \rho,\,l\in[L-1],\,\norm{w^{(L)}_{f,k}-\bar{w}^{(L)}_{f,k}}_2\leq\rho_1,\,\norm{w^{(L)}_{g,k}-\bar{w}^{(L)}_{g,k}}_2\leq\rho_1,\,k\in[K]\}$ for $\rho,\rho_1>0$. 
%
We say that an element of $B^{\mathrm{Euc}}_{\rho,\rho_1}(\bar{\vtheta})$ is \emph{strictly inside} $B^{\mathrm{Euc}}_{\rho,\rho_1}(\bar{\vtheta})$ when it satisfies every inequality in the set's definition without equality.
%
We also define $B^{\mathrm{Euc}}_{\rho}(\bar{\vtheta})$ as an Euclidean ball around $\bar{\vtheta}$ with radius $\rho>0$.

The following is an assumption analogous to the general Assumption~\ref{asmp:iter-0}.

\begin{asmp}[{\bf Iterates inside $B^{\mathrm{Euc}}_{\rho,\rho_1}(\vtheta_0)$}]
\label{asmp:iter-1}
All iterates $\{\vtheta_t\}_{t\geq 1}$ follow GD as in~\eqref{eq:gd_at_t} and are strictly inside the set $B^{\mathrm{Euc}}_{\rho,\rho_1}(\vtheta_0)$ for fixed $\rho,\rho_1>0$.
\end{asmp}

We now focus on showing that the two conditions needed for optimization using GD as discussed in Section~\ref{sec:optmain} are indeed satisfied by DONs. We start with the definition of a set $Q^t_{\kappa}$ parameterized by $\kappa \in (0, \frac{1}{2}]$, which will help construct the set $\cN_t$ in Condition~\ref{cond:rsc} for RSC. Due to the interaction of two neural networks (branch and trunk), the definition of $Q^t_{\kappa}$ looks seemingly involved. However, note that $Q^t_{\kappa}$ is only needed for establishing the RSC condition for the analysis and does not change the computation of the optimization algorithm, which is simply GD run over all the branch and trunk network parameters. 
%
%
\begin{restatable}[{\bf $Q^{t}_{\kappa}$ sets for DONs}]{defn}{qset} 
%
For an iterate $\vtheta_t = [\vtheta_{f,t}^{\top}\; \vtheta_{g,t}^{\top}]^{\top}$ and $\kappa \in (0,\frac{1}{\sqrt{2}}]$, we define the set:
{\small 
\begin{equation}
    \begin{aligned} 
Q^t_{\kappa} &:= \bigg\{ \vtheta' = {[{\vtheta'_{f}}^{\top}\; {\vtheta'_{g}}^{\top}]}^{\top}\in \R^{p_f+p_g}:\\
&\;\;|\cos(\vtheta' - \vtheta_t, \nabla_{\vtheta} \bar{G}_{\vtheta_t})| \geq \kappa~, \\
%
&\;\;(\vtheta'_f-\vtheta_{f,t})^\top\left(\frac{1}{n} \sum_{i=1}^n \frac{1}{q_i} \sum_{j=1}^{q_i} \ell'_{i,j} \sum_{k=1}^K \nabla_{\vtheta_{f}} f_k^{(i)} \nabla_{\vtheta_{g}} g_{k,j}^{(i)~\top}\right)(\vtheta'_g-\vtheta_{g,t})  \geq 0~,\\
&\;\;(\vtheta'_f-\vtheta_{f,t})^\top\left( \sum_{k=1}^K \nabla_{\vtheta_{f}} f_k^{(i)} \nabla_{\vtheta_{g}} g_{k,j}^{(i)~\top}\right)(\vtheta'_g-\vtheta_{g,t})\leq 0,\forall i\in[n],\forall j\in[q_i] ~ \bigg\}~,
    \end{aligned}
\label{defn:qset}
\end{equation}}where $\nabla_{\vtheta}\bar{G}_{\vtheta_t} = \frac{1}{n} \sum_{i=1}^n \frac{1}{q_i} \sum_{j=1}^{q_i} \nabla_{\vtheta}G_{\vtheta_t}(u^{(i)})(\vy^{(i)}_j)$, $\ell_{i,j}=(G_{\vtheta_t}(u^{(i)})(\vy_{j}^{(i)})-G^\dagger(u^{(i)})(\vy^{(i)}_j))^2$, and both $\nabla_{\vtheta_{f}}f^{(i)}_k$ and $\nabla_{\vtheta_{g}}g^{(i)}_{k,j}$ are evaluated on $\vtheta_t$.
%\begin{align}
\label{defn:qset_DON}
\end{restatable}

We now prove the RSC and smoothness conditions (corresponding to Conditions~\ref{cond:rsc} and~\ref{cond:smooth}, respectively).
Using the nomenclature of Section~\ref{sec:optmain}, the set $B^{\mathrm{Euc}}_{\rho,\rho_1}(\vtheta_0)$ corresponds to $\mathcal{B}(\vtheta_0)$, and 
$B^t_{\kappa} := Q_{\kappa}^t \cap B^{\mathrm{Euc}}_{\rho,\rho_1}(\vtheta_0) \cap B^{\mathrm{Euc}}_{\rho_2}(\vtheta_t)$ corresponds to $\mathcal{N}_t$.
%
%

\begin{restatable}[{\bf RSC for DONs}]{theo}{RSCLoss}
Consider Assumptions~\ref{asmp:Activation_Function}, \ref{asmp:smoothinit}, and~\ref{asmp:iter-1}, and $Q^t_{\kappa}$ as in Definition~\ref{defn:qset_DON}.
%with $\kappa\in(0,\frac{1}{2}]$. 
Then, the set $B^t_{\kappa} := Q_{\kappa}^t \cap B^{\mathrm{Euc}}_{\rho,\rho_1}(\vtheta_0) \cap B^{\mathrm{Euc}}_{\rho_2}(\vtheta_t)$ is a non-empty set that satisfies Condition~\ref{cond:rsc}(a) and (b) for suitable $\rho_2$. 
%, and 
%(ii) 
Moreover, 
with probability at least {$1-2KL(\frac{1}{m_f}+\frac{1}{m_g})$}, at step $t$ of GD, % 
%$\forall \vtheta^{\prime} \in B^t_{\kappa}$,
the DON loss $\gL$~\eqref{eq:loss-don} satisfies
equation~\eqref{eq:RSC-prim} with
\begin{equation}
\alpha_t = 2\kappa^2 \| \nabla_{\vtheta} \bar{G}_t \|_2^2 - 
    c_1K^2\left(\frac{1}{\sqrt{m_f}}+\frac{1}{\sqrt{m_g}}\right)
        \label{eq:RSCLoss}
\end{equation}
where $\nabla_{\vtheta}\bar{G}_t = \frac{1}{n} \sum_{i=1}^n \frac{1}{q_i}  \sum_{j=1}^{q_i} \nabla_{\vtheta}G_{\vtheta_t}(u^{(i)})(\vy^{(i)}_j)$, 
and for some constant $c_1 >0$ 
%
which depends polynomially on the depth $L$, and the radii $\rho$, $\rho_1$, and $\rho_2$ whenever $\sigma_0\leq 1-\rho\max\{\frac{1}{\sqrt{m_f}},\frac{1}{\sqrt{m_g}}\}$.
%=======
%=========
Thus, the loss $\gL$ satisfies RSC w.r.t $(B_{\kappa}^t, \vtheta_t)$, i.e., Condition~\ref{cond:rsc}(c), whenever $\| \nabla_{\vtheta} \bar{G}_t \|_2^2 = \Omega(\frac{1}{\sqrt{m_f}}+\frac{1}{\sqrt{m_g}})$.
\label{theo:rsc_main_DON}
\end{restatable}
%

\begin{restatable}[{\bf Smoothness for DONs}]{theo}{RSS}
Under \TwoAsmpsref{asmp:Activation_Function}{asmp:smoothinit}, with probability at least {$1-2KL(\frac{1}{m_f}+\frac{1}{m_g})$}, the DON loss $\cL$~\eqref{eq:loss-don} is $\beta$-smooth in $B^{\mathrm{Euc}}_{\rho,\rho_1}(\vtheta_0)$ with $\beta =c_2K^2$, where 
%
$c_2>0$ is a constant which depends polynomially on the depth $L$, and the radii $\rho$, $\rho_1$, and $\rho_2$ whenever $\sigma_0\leq 1-\rho\max\{\frac{1}{\sqrt{m_f}},\frac{1}{\sqrt{m_g}}\}$.
%===
%%
%%
%===
%
\label{theo:smooth_main}
\end{restatable}
%

\begin{remark}[Ensuring that $\alpha_t/\beta<1$] 
\label{rem:abeta-DON}
\pcedit{As mentioned in Remark~\ref{rem:ratio}}, in order to use the optimization framework from Section~\ref{sec:optmain}, the statement of Theorem~\ref{theo:global_main} requires 
$\alpha_t/\beta\leq 1$. We prove that this condition is satisfied with a strict inequality for DONs in Proposition~\ref{prop:RSC-smooth-DON} in Appendix~\ref{app:donopt}. 
\qed
\end{remark}

%
\paragraph{\textbf{Optimization Under Gradient Descent for DONs.}} We have that Theorem~\ref{theo:rsc_main_DON} satisfies Condition~\ref{cond:rsc} and Theorem~\ref{theo:smooth_main} satisfies Condition~\ref{cond:smooth}. We also proved that $\alpha_t/\beta<1$. Thus, when $\| \nabla_{\vtheta} \bar{G}_t \|_2^2 = \Omega(\frac{1}{\sqrt{m_f}}+\frac{1}{\sqrt{m_g}})$, i.e., $\alpha_t>0$, a decrease on the loss function by GD is ensured with probability at least $1-2KL(\frac{1}{m_f}+\frac{1}{m_g})$ towards its minimum value taken within the set $B^{\mathrm{Euc}}_{\rho,\rho_1}(\vtheta_0)$ due to Theorem~\ref{theo:global_main}.

\begin{remark}[The benefit of over-parameterization for the RSC property]
According to~\eqref{eq:RSCLoss}, $\| \nabla_{\vtheta} \bar{G}_t \|_2^2 = \Omega(\frac{1}{\sqrt{m_f}}+\frac{1}{\sqrt{m_g}})$ is needed to ensure that $\alpha_t>0$, i.e., to ensure that the empirical loss $\gL$ satisfies the RSC property at time $t$. Thus, as both widths $m_f$ and $m_g$ increase, $\gL$ attains the RSC property at a lower value of $\| \nabla_{\vtheta} \bar{G}_t \|_2^2$.
%
%
%
%
%
%
  \qed
  \label{rem:RSC-m}
\end{remark}
%
\begin{remark}[Over-parameterization allows for a larger neighborhood around initialization]
%
%
%
%
%
\pcedit{The condition $\sigma_0\leq 1-\rho\max\{\frac{1}{\sqrt{m_f}},\frac{1}{\sqrt{m_g}}\}$ (required for obtaining a polynomial dependence on $L$ for both RSC and smoothness parameters) implies %and the radii to be at most polynomial on $L$.}
%
%requires 
$\rho\leq\min\{m_f,m_g\}$ since $\sigma_0$ must be positive.}
%
%This means that}
%
%Moreover, since 
\pcedit{Thus,} it is possible to \pcedit{increase the} %make the 
radius $\rho$ %larger 
as we increase 
both $m_f$ and $m_g$. 
%
Thus, we can \pcedit{enlarge}
%it is possible to enlarge 
the neighborhood around the initialization point where our guarantees hold \pcedit{as the widths increase}.
%
\label{rem:largerNeighb-m}\qed
\end{remark}
%
\section{Optimization Analysis for FNO}
\label{sec:optFNO}
As in the case of DONs, we also focus on scalar input functions $u$. 
%, i.e., $\operatorname{ran}(u)\subseteq\R$. 
To pass the 
input function $u$,
we discretize it by sampling it 
on $\bar{R}$ locations, 
forming a vector of dimension $\bar{R}$.
Thus,
the encoder $P(u;\vtheta_p)(\bfx)$ in equation~\eqref{eq:continuous_fno} takes a vector of dimension $\bar{R}+d_x$ ($\bar{R}$ from the sampled $u$ and $d_x$ from the output location where we evaluate the operator on). For our purposes, we consider a fixed (not trainable) encoder with output dimension $d$: $P(u;\vtheta_p)(\bfx)\equiv P(u)(\bfx)\in\R^d$; and a linear decoder $Q(\aalpha^{({L+1})};\vtheta_q)(\vx)=\frac{1}{\sqrt{m}}\v^\top \aalpha^{({L+1})}(\vx)\in\R$ with $\vtheta_q\equiv \v\in\R^m$ assuming $\aalpha^{(L+1)}(\bfx)\in\R^m$. 
Thus, following~\citep{li_fourier_2021}, the FNO model is:
\begin{align*}
\aalpha^{(0)} &= P(u)(\bfx)\\
\aalpha^{(1)} &= \phi\left(
        \frac{1}{\sqrt{m}} W^{(1)} \aalpha^{(0)}
    \right)\\
\aalpha^{(l)} & = \phi\left(
        \frac{1}{\sqrt{m}} W^{(l)} \aalpha^{(l-1)} +
        \frac{1}{\sqrt{m}} F^{*} R^{(l)} F \aalpha^{(l-1)}
    \right),\; l\in \{2,\dots,L+1\}\\
    f(\vtheta;\vx)  &= \frac{1}{\sqrt{m}} \v^\top \aalpha^{(L+1)}~,
\end{align*}
where $\phi$ is a pointwise smooth activation function, $F$ is the discrete Fourier transform kernel (as a matrix) with $F^*$ being its conjugate transpose, 
%where 
the weight matrices are $W^{(1)}\in\R^{m\times d}$, $W^{(l)}\in\R^{m\times m}$ and $R^{(l)}\in\R^{m\times m}$ for layer $l\in\{2,\dots, L+1\}$ (all hidden layers have the same width $m$). The ij-entries of $W^{(l)}$ and $R^{(l)}$ are $w^{(l)}_{ij}$ and $r^{(l)}_{ij}$, respectively, for an appropriate $l$. With some abuse of notation, we denote the entire set of trainable parameters by $\vtheta = [\vtheta_w^{\top}\ \vtheta_r^{\top}]^{\top}$, with $\vtheta_w = [\text{vec}(W^{(1)})^{\top},\dots,\text{vec}(W^{(L+1)})^{\top}\ \mathbf{v}^{\top}]^{\top}$ and $\vtheta_r = [\text{vec}(R^{(2)})^\top,\dots,\text{vec}(R^{(L+1)})^{\top}]^{\top}$. We denote the number of parameters by $p_w + p_r$, where $\vtheta_w\in\R^{p_w}$ and $\vtheta_r\in\R^{p_r}$.  
%
%
Let $\vtheta_0$ be the parameter vector at initialization and $\vtheta_t$ be it at time step $t$.

We remark that our model uses an $m\times m$ Discrete Fourier Transform %(DFT) 
kernel $F$, whose $kj$-entry is $F_{kj}=e^{-\frac{2\pi \iota}{m}(k-1)(j-1)}$, with $\iota$ representing the imaginary unit.

\begin{asmp}[{\bf Activation functions}]
\label{asmp:Activation_Function_FNO}
The activation function $\phi$ is $1$-Lipschitz and $\beta_{\phi}$-smooth (i.e. $\phi_l^{\prime\prime}\leq \beta_{\phi}$) for some $\beta_{\phi} > 0$.
\end{asmp}
\begin{asmp}[{\bf Initialization of weights}]
\label{asmp:smoothinit_FNO}
All weights of the FNO are initialized independently as follows: (i) $w^{(l)}_{{0,\,ij}}\sim \gN (0, \sigma^2_{0_w})$ and $r^{(l)}_{{0,\,ij}}\sim \gN (0, \sigma^2_{0_r})$ for $l\in [L+1]$ where $\sigma_{0,w} = \frac{\sigma_{1,w}}{2(1+\frac{\sqrt{\log m}}{\sqrt{2m}})}$ and $\sigma_{0,r} = \frac{\sigma_{1,r}}{2(1+\frac{\sqrt{\log m}}{\sqrt{2m}})}$, where $\sigma_{1,w},\ \sigma_{1,r} > 0$; (ii) the decoder parameter $\v$ is a random vector with unit norm $\norm{\v}_2=1$. Further, we assume the encoder output satisfies $\norm{\aalpha^{(0)}}_2=\sqrt{d}$.
\end{asmp}

For a given parameter vector $\bar{\vtheta}\in\R^{p_w+p_r}$, we introduce the neighborhood set  
$B^{\mathrm{Euc}}_{\rho_w,\rho_r,\rho_1}(\bar{\vtheta})=\{\vtheta\in\mathbb{R}^{p_w+p_r}\,:\,\norm{W^{(l)}-\bar{W}^{(l)}}_2\leq \rho_w,\,l\in[L+1],\,\norm{R^{(l)}-\bar{R}^{(l)}}_2\leq \rho_r,\,l\in\{2,\dots,L+1\},\,\norm{\v-\bar{\v}}_2\leq \rho_1\}$ for $\rho_w,\rho_r,\rho_1>0$. 
%
We say that an element of $B^{\mathrm{Euc}}_{\rho_w,\rho_r,\rho_1}(\bar{\vtheta})$ is \emph{strictly inside} $B^{\mathrm{Euc}}_{\rho_w,\rho_r,\rho_1}(\bar{\vtheta})$ when it satisfies every inequality in the set's definition without equality.

The following assumption is analogous to Assumption~\ref{asmp:iter-0}.

\begin{asmp}[{\bf Iterates inside $B^{\mathrm{Euc}}_{\rho_w,\rho_r,\rho_1}(\vtheta_0)$}]
\label{asmp:iter-2}
All iterates $\{\vtheta_t\}_{t\geq 1}$ follow GD as in~\eqref{eq:gd_at_t} and are strictly inside the set $B^{\mathrm{Euc}}_{\rho_w,\rho_r,\rho_1}(\vtheta_0)$ for fixed $\rho_w,\rho_r,\rho_1>0$.
\end{asmp}

We also introduce the following auxiliary set. %
\begin{restatable}[{\bf $Q^{t}_{\kappa}$ sets for FNOs}]{defn}{qset_FNO} 
For an iterate $\vtheta_t$, let $\nabla_{\vtheta}\bar{G}_t =  \frac{1}{n} \sum_{i=1}^n \frac{1}{R}  \sum_{j=1}^{R} \nabla_{\vtheta}G_{\vtheta_t}(u^{(i)})(x_j)$. For $\kappa \in (0,1)$, define $Q^t_{\kappa} := \{ \vtheta \in \R^{p_w + p_r} \mid |\cos(\vtheta-\vtheta_t, \nabla_{\vtheta}\bar{G}_t)| \geq \kappa \}$.
\label{defn:qset_FNO}
\end{restatable}
Note that unlike DONs, the $Q^t_{\kappa}$ sets for FNOs are relatively simpler due to a single network architecture. 

Next, we prove the RSC and smoothness conditions (corresponding to Conditions~\ref{cond:rsc} and~\ref{cond:smooth}, respectively). 
%
Using the nomenclature of Section~\ref{sec:optmain}, the set $B^{\mathrm{Euc}}_{\rho_w,\rho_r,\rho_1}(\vtheta_0)$ corresponds to $\mathcal{B}(\vtheta_0)$, and 
$B^t_{\kappa} := Q_{\kappa}^t \cap B^{\mathrm{Euc}}_{\rho_w,\rho_r\rho_1}(\vtheta_0) \cap B^{\mathrm{Euc}}_{\rho_2}(\vtheta_t)$ corresponds to $\mathcal{N}_t$.

\begin{restatable}[{\bf RSC for FNOs}]{theo}{RSCLossFNO}
Consider Assumptions~\ref{asmp:Activation_Function_FNO}, \ref{asmp:smoothinit_FNO}, and~\ref{asmp:iter-2}, and $Q^t_{\kappa}$ as in Definition~\ref{defn:qset_FNO}. Then, the set $B^t_{\kappa} := Q_{\kappa}^t \cap B^{\mathrm{Euc}}_{\rho_w,\rho_r\rho_1}(\vtheta_0) \cap B^{\mathrm{Euc}}_{\rho_2}(\vtheta_t)$ is a non-empty set that satisfies Condition~\ref{cond:rsc}(a) and (b) for suitable $\rho_2$. 
Moreover, 
with probability at least $1-\frac{2(L + 2)}{m}$, at step $t$ of GD,  
%$\forall \vtheta^{\prime} \in B^t_{\kappa}$ 
the FNO loss $\gL$~\eqref{eq:loss-fno} satisfies 
equation~\eqref{eq:RSC-prim} with
\begin{equation}
\alpha_t = 2\kappa^2 \| \nabla_{\vtheta} \bar{G}_t \|_2^2 - \frac{c_1}{\sqrt{m}}~,
        \label{eq:RSCLoss_FNO}
\end{equation}
where $\nabla_{\vtheta}\bar{G}_t = \frac{1}{n} \sum_{i=1}^n \frac{1}{R}  \sum_{j=1}^{R} \nabla_{\vtheta}G_{\vtheta_t}(u^{(i)})(x_j)$, and for some constant $c_1 >0$ 
which depends polynomially on the depth $L$, and the radii $\rho_w$, $\rho_r$, $\rho_1$, and $\rho_2$ whenever $\sigma_{1,w}+\sigma_{1,r}\leq 1-\frac{\rho_w+\rho_r}{\sqrt{m}}$.
%====
%%
%====
%
Thus, the loss $\gL(\vtheta)$ satisfies RSC w.r.t $(B_{\kappa}^t, \vtheta_t)$, i.e., Condition~\ref{cond:rsc}(c), whenever $\| \nabla_{\vtheta} \bar{G}_t \|_2^2 = \Omega(\frac{1}{\sqrt{m}})$.
\label{theo:rsc_main_fno}
\end{restatable}
%

\begin{restatable}[{\bf Smoothness for FNOs}]{theo}{RSSFNO}
Under Assumptions~\ref{asmp:Activation_Function_FNO} and \ref{asmp:smoothinit_FNO}, with probability at least $ 1 - \frac{2(L+2)}{m}$, the FNO loss $\cL$~\eqref{eq:loss-fno} is $\beta$-smooth in $B^{\mathrm{Euc}}_{\rho_w,\rho_r\rho_1}(\vtheta_0)$ with $\beta$ being a positive constant which 
%
depends polynomially on the depth $L$, and the radii $\rho_w$, $\rho_r$, and $\rho_1$ whenever $\sigma_{1,w}+\sigma_{1,r}\leq 1-\frac{\rho_w+\rho_r}{\sqrt{m}}$.
%===
%%
%===
%
\label{theo:smooth_main_fno}
\end{restatable}

\begin{remark}[Ensuring that $\alpha_t/\beta<1$] 
Similar to our discussion in Remark~\ref{rem:abeta-DON}, we prove that 
%the needed ratio between the RSC and smoothness parameters is less than one, i.e., 
$\alpha_t/\beta<1$ in Proposition~\ref{prop:RSC-smooth-FNO} from Appendix~\ref{app:fnoopt}, satisfying the condition required in the statement of Theorem~\ref{theo:global_main}.
%
%In the appendix we provide a proof for $\alpha_t/\beta<1$.
%
\qed
\end{remark}

\paragraph{\textbf{Optimization Under Gradient Descent for FNOs.}} 
We have that 
Theorem~\ref{theo:rsc_main_fno} satisfies Condition~\ref{cond:rsc} and Theorem~\ref{theo:smooth_main_fno} satisfies Condition~\ref{cond:smooth}. We also proved that $\alpha_t/\beta<1$. Thus, when $\| \nabla_{\vtheta} \bar{G}_t \|_2^2 = \Omega(\frac{1}{\sqrt{m}})$, i.e., $\alpha_t>0$, a decrease on the loss function by GD is ensured with probability at least $1- \frac{2(L+2)}{m}$ towards its minimum value taken within the set $B^{\mathrm{Euc}}_{\rho_w,\rho_r\rho_1}(\vtheta_0)$ due to Theorem~\ref{theo:global_main}.
%
%

\begin{remark}[The effects of over-parameterization for FNOs]
Similar observations to Remarks~\ref{rem:RSC-m} and~\ref{rem:largerNeighb-m} hold for FNOs, i.e., that over-parameterization ensures (i) a better condition for ensuring the RSC property, and (ii) a larger neighborhood %\pcedit{(in terms of larger $\rho_w$ and $\rho_r$)} 
around the initialization point over which our guarantees hold. Item (ii) follows from the relationship $\rho_w+\rho_r\leq \sqrt{m}$ obtained when choosing $\sigma_{1,w}$ and $\sigma_{1,r}$ to ensure a polynomial dependence as in Theorems~\ref{theo:rsc_main_fno} and~\ref{theo:smooth_main_fno}.
%
\end{remark}
%
\section{Comparison between Neural Operators and Feedforward Neural Networks}
\label{sec:Comparison}

% \begin{wraptable}{r}{0.65\textwidth}
% \centering
% \caption{The comparison of resource requirements between Eurus-2-7B-PRIME and Qwen2.5-Math-7B-Instruct.}
% \label{tab:comparision}
% \resizebox{0.65\textwidth}{!}{
% \begin{tabular}{l >{\columncolor[HTML]{D7E8E8}}l l}
% \toprule
% \textbf{Model} & \textbf{Eurus-2-7B-PRIME} & \textbf{Qwen2.5-Math-7B-Instruct} \\ \midrule
% Base Model     & Qwen2.5-Math-7B           & Qwen2.5-Math-7B                  \\
% SFT Data       & 230K (open-source)        & 2.5M (open-source and in-house)  \\
% RM Data        & 0                         & 618K (in-house)                 \\
% RM             & Eurus-2-7B-SFT            & Qwen2.5-Math-RM (72B)           \\
% RL Data        & 150K queries $\times$ 4 samples & 66K queries $\times$ 32 samples \\ \bottomrule
% \end{tabular}
% }
% \end{wraptable}

\begin{wraptable}{r}{0.65\textwidth}
\centering
\caption{The comparison of resource requirements between Eurus-2-7B-PRIME and Qwen2.5-Math-7B-Instruct.}
\label{tab:comparision}
% \resizebox{0.65\textwidth}{!}{
\resizebox{\linewidth}{!}{
\begin{tabular}{l >{\columncolor[HTML]{D7E8E8}}l l}
\toprule
\textbf{Model} & \textbf{Eurus-2-7B-PRIME} & \textbf{Qwen2.5-Math-7B-Instruct} \\ \midrule
Base Model     & Qwen2.5-Math-7B           & Qwen2.5-Math-7B                  \\
SFT Data       & 230K (open-source)        & 2.5M (open-source \& in-house)  \\
RM Data        & 0                         & 618K (in-house)                 \\
RM             & Eurus-2-7B-SFT            & Qwen2.5-Math-RM (72B)           \\
RL Data        & 150K queries $\times$ 4 samples & 66K queries $\times$ 32 samples \\ \bottomrule
\end{tabular}
}
\end{wraptable}





% 字体标橙色
% \begin{wraptable}{r}{0.65\textwidth}  % r表示表格在右侧,0.5\textwidth表示表格宽度为文本宽度的50%
% \centering
% \caption{The comparison of resource requirements between Eurus-2-7B-PRIME and Qwen2.5-Math-7B-Instruct.}
% \label{tab:comparision}
% \resizebox{0.65\textwidth}{!}{
% \begin{tabular}{lll}
% \toprule
% \textbf{Model} & {\color[HTML]{F8A102}\textbf{Eurus-2-7B-PRIME}}                             & \textbf{Qwen2.5-Math-7B-Instruct}            \\ \midrule
% Base Model     & {\color[HTML]{F8A102}Qwen2.5-Math-7B}                                       & Qwen2.5-Math-7B                              \\
% SFT Data       & {\color[HTML]{F8A102}\textbf{230K (open-source)}}                           & 2.5M (open-source and in-house)              \\
% RM Data        & {\color[HTML]{F8A102}\textbf{0}}                                            & 618K (in-house)                              \\
% RM             & {\color[HTML]{F8A102}\textbf{Eurus-2-7B-SFT}}                               & Qwen2.5-Math-RM (72B)                        \\
% RL Data        & {\color[HTML]{F8A102}\textbf{150K queries $\times$ 4 samples}} & 66K queries $\times$ 32 samples \\ \bottomrule
% \end{tabular}

% }
% \end{wraptable}




% \begin{table}[]
% \centering
% \caption{The comparison of resource requirements between Eurus-2-7B-PRIME and Qwen2.5-Math-7B-Instruct.\hanbin{Embed into text}}
% \label{tab:comparision}
% \resizebox{0.8\textwidth}{!}{
% \begin{tabular}{lll}
% \midrule
% \textbf{Model} & \textbf{Eurus-2-7B-PRIME}                             & \textbf{Qwen2.5-Math-7B-Instruct}            \\ \midrule
% Base Model     & Qwen2.5-Math-7B                                       & Qwen2.5-Math-7B                              \\
% SFT Data       & \textbf{230K (open-source)}                           & 2.5M (open-source and in-house)              \\
% RM Data        & \textbf{0}                                            & 618K (in-house)                              \\
% RM             & \textbf{Eurus-2-7B-SFT}                               & Qwen2.5-Math-RM (72B)                        \\
% RL Data        & \textbf{150K queries $\times$ 4 samples} & 66K queries $\times$ 32 samples \\ \midrule
% \end{tabular}
% }
% \end{table}


%
\section{Experiments}
\label{sec:Experiments}
\section{Experiments}
\label{sec:experiments}
The experiments are designed to address two key research questions.
First, \textbf{RQ1} evaluates whether the average $L_2$-norm of the counterfactual perturbation vectors ($\overline{||\perturb||}$) decreases as the model overfits the data, thereby providing further empirical validation for our hypothesis.
Second, \textbf{RQ2} evaluates the ability of the proposed counterfactual regularized loss, as defined in (\ref{eq:regularized_loss2}), to mitigate overfitting when compared to existing regularization techniques.

% The experiments are designed to address three key research questions. First, \textbf{RQ1} investigates whether the mean perturbation vector norm decreases as the model overfits the data, aiming to further validate our intuition. Second, \textbf{RQ2} explores whether the mean perturbation vector norm can be effectively leveraged as a regularization term during training, offering insights into its potential role in mitigating overfitting. Finally, \textbf{RQ3} examines whether our counterfactual regularizer enables the model to achieve superior performance compared to existing regularization methods, thus highlighting its practical advantage.

\subsection{Experimental Setup}
\textbf{\textit{Datasets, Models, and Tasks.}}
The experiments are conducted on three datasets: \textit{Water Potability}~\cite{kadiwal2020waterpotability}, \textit{Phomene}~\cite{phomene}, and \textit{CIFAR-10}~\cite{krizhevsky2009learning}. For \textit{Water Potability} and \textit{Phomene}, we randomly select $80\%$ of the samples for the training set, and the remaining $20\%$ for the test set, \textit{CIFAR-10} comes already split. Furthermore, we consider the following models: Logistic Regression, Multi-Layer Perceptron (MLP) with 100 and 30 neurons on each hidden layer, and PreactResNet-18~\cite{he2016cvecvv} as a Convolutional Neural Network (CNN) architecture.
We focus on binary classification tasks and leave the extension to multiclass scenarios for future work. However, for datasets that are inherently multiclass, we transform the problem into a binary classification task by selecting two classes, aligning with our assumption.

\smallskip
\noindent\textbf{\textit{Evaluation Measures.}} To characterize the degree of overfitting, we use the test loss, as it serves as a reliable indicator of the model's generalization capability to unseen data. Additionally, we evaluate the predictive performance of each model using the test accuracy.

\smallskip
\noindent\textbf{\textit{Baselines.}} We compare CF-Reg with the following regularization techniques: L1 (``Lasso''), L2 (``Ridge''), and Dropout.

\smallskip
\noindent\textbf{\textit{Configurations.}}
For each model, we adopt specific configurations as follows.
\begin{itemize}
\item \textit{Logistic Regression:} To induce overfitting in the model, we artificially increase the dimensionality of the data beyond the number of training samples by applying a polynomial feature expansion. This approach ensures that the model has enough capacity to overfit the training data, allowing us to analyze the impact of our counterfactual regularizer. The degree of the polynomial is chosen as the smallest degree that makes the number of features greater than the number of data.
\item \textit{Neural Networks (MLP and CNN):} To take advantage of the closed-form solution for computing the optimal perturbation vector as defined in (\ref{eq:opt-delta}), we use a local linear approximation of the neural network models. Hence, given an instance $\inst_i$, we consider the (optimal) counterfactual not with respect to $\model$ but with respect to:
\begin{equation}
\label{eq:taylor}
    \model^{lin}(\inst) = \model(\inst_i) + \nabla_{\inst}\model(\inst_i)(\inst - \inst_i),
\end{equation}
where $\model^{lin}$ represents the first-order Taylor approximation of $\model$ at $\inst_i$.
Note that this step is unnecessary for Logistic Regression, as it is inherently a linear model.
\end{itemize}

\smallskip
\noindent \textbf{\textit{Implementation Details.}} We run all experiments on a machine equipped with an AMD Ryzen 9 7900 12-Core Processor and an NVIDIA GeForce RTX 4090 GPU. Our implementation is based on the PyTorch Lightning framework. We use stochastic gradient descent as the optimizer with a learning rate of $\eta = 0.001$ and no weight decay. We use a batch size of $128$. The training and test steps are conducted for $6000$ epochs on the \textit{Water Potability} and \textit{Phoneme} datasets, while for the \textit{CIFAR-10} dataset, they are performed for $200$ epochs.
Finally, the contribution $w_i^{\varepsilon}$ of each training point $\inst_i$ is uniformly set as $w_i^{\varepsilon} = 1~\forall i\in \{1,\ldots,m\}$.

The source code implementation for our experiments is available at the following GitHub repository: \url{https://anonymous.4open.science/r/COCE-80B4/README.md} 

\subsection{RQ1: Counterfactual Perturbation vs. Overfitting}
To address \textbf{RQ1}, we analyze the relationship between the test loss and the average $L_2$-norm of the counterfactual perturbation vectors ($\overline{||\perturb||}$) over training epochs.

In particular, Figure~\ref{fig:delta_loss_epochs} depicts the evolution of $\overline{||\perturb||}$ alongside the test loss for an MLP trained \textit{without} regularization on the \textit{Water Potability} dataset. 
\begin{figure}[ht]
    \centering
    \includegraphics[width=0.85\linewidth]{img/delta_loss_epochs.png}
    \caption{The average counterfactual perturbation vector $\overline{||\perturb||}$ (left $y$-axis) and the cross-entropy test loss (right $y$-axis) over training epochs ($x$-axis) for an MLP trained on the \textit{Water Potability} dataset \textit{without} regularization.}
    \label{fig:delta_loss_epochs}
\end{figure}

The plot shows a clear trend as the model starts to overfit the data (evidenced by an increase in test loss). 
Notably, $\overline{||\perturb||}$ begins to decrease, which aligns with the hypothesis that the average distance to the optimal counterfactual example gets smaller as the model's decision boundary becomes increasingly adherent to the training data.

It is worth noting that this trend is heavily influenced by the choice of the counterfactual generator model. In particular, the relationship between $\overline{||\perturb||}$ and the degree of overfitting may become even more pronounced when leveraging more accurate counterfactual generators. However, these models often come at the cost of higher computational complexity, and their exploration is left to future work.

Nonetheless, we expect that $\overline{||\perturb||}$ will eventually stabilize at a plateau, as the average $L_2$-norm of the optimal counterfactual perturbations cannot vanish to zero.

% Additionally, the choice of employing the score-based counterfactual explanation framework to generate counterfactuals was driven to promote computational efficiency.

% Future enhancements to the framework may involve adopting models capable of generating more precise counterfactuals. While such approaches may yield to performance improvements, they are likely to come at the cost of increased computational complexity.


\subsection{RQ2: Counterfactual Regularization Performance}
To answer \textbf{RQ2}, we evaluate the effectiveness of the proposed counterfactual regularization (CF-Reg) by comparing its performance against existing baselines: unregularized training loss (No-Reg), L1 regularization (L1-Reg), L2 regularization (L2-Reg), and Dropout.
Specifically, for each model and dataset combination, Table~\ref{tab:regularization_comparison} presents the mean value and standard deviation of test accuracy achieved by each method across 5 random initialization. 

The table illustrates that our regularization technique consistently delivers better results than existing methods across all evaluated scenarios, except for one case -- i.e., Logistic Regression on the \textit{Phomene} dataset. 
However, this setting exhibits an unusual pattern, as the highest model accuracy is achieved without any regularization. Even in this case, CF-Reg still surpasses other regularization baselines.

From the results above, we derive the following key insights. First, CF-Reg proves to be effective across various model types, ranging from simple linear models (Logistic Regression) to deep architectures like MLPs and CNNs, and across diverse datasets, including both tabular and image data. 
Second, CF-Reg's strong performance on the \textit{Water} dataset with Logistic Regression suggests that its benefits may be more pronounced when applied to simpler models. However, the unexpected outcome on the \textit{Phoneme} dataset calls for further investigation into this phenomenon.


\begin{table*}[h!]
    \centering
    \caption{Mean value and standard deviation of test accuracy across 5 random initializations for different model, dataset, and regularization method. The best results are highlighted in \textbf{bold}.}
    \label{tab:regularization_comparison}
    \begin{tabular}{|c|c|c|c|c|c|c|}
        \hline
        \textbf{Model} & \textbf{Dataset} & \textbf{No-Reg} & \textbf{L1-Reg} & \textbf{L2-Reg} & \textbf{Dropout} & \textbf{CF-Reg (ours)} \\ \hline
        Logistic Regression   & \textit{Water}   & $0.6595 \pm 0.0038$   & $0.6729 \pm 0.0056$   & $0.6756 \pm 0.0046$  & N/A    & $\mathbf{0.6918 \pm 0.0036}$                     \\ \hline
        MLP   & \textit{Water}   & $0.6756 \pm 0.0042$   & $0.6790 \pm 0.0058$   & $0.6790 \pm 0.0023$  & $0.6750 \pm 0.0036$    & $\mathbf{0.6802 \pm 0.0046}$                    \\ \hline
%        MLP   & \textit{Adult}   & $0.8404 \pm 0.0010$   & $\mathbf{0.8495 \pm 0.0007}$   & $0.8489 \pm 0.0014$  & $\mathbf{0.8495 \pm 0.0016}$     & $0.8449 \pm 0.0019$                    \\ \hline
        Logistic Regression   & \textit{Phomene}   & $\mathbf{0.8148 \pm 0.0020}$   & $0.8041 \pm 0.0028$   & $0.7835 \pm 0.0176$  & N/A    & $0.8098 \pm 0.0055$                     \\ \hline
        MLP   & \textit{Phomene}   & $0.8677 \pm 0.0033$   & $0.8374 \pm 0.0080$   & $0.8673 \pm 0.0045$  & $0.8672 \pm 0.0042$     & $\mathbf{0.8718 \pm 0.0040}$                    \\ \hline
        CNN   & \textit{CIFAR-10} & $0.6670 \pm 0.0233$   & $0.6229 \pm 0.0850$   & $0.7348 \pm 0.0365$   & N/A    & $\mathbf{0.7427 \pm 0.0571}$                     \\ \hline
    \end{tabular}
\end{table*}

\begin{table*}[htb!]
    \centering
    \caption{Hyperparameter configurations utilized for the generation of Table \ref{tab:regularization_comparison}. For our regularization the hyperparameters are reported as $\mathbf{\alpha/\beta}$.}
    \label{tab:performance_parameters}
    \begin{tabular}{|c|c|c|c|c|c|c|}
        \hline
        \textbf{Model} & \textbf{Dataset} & \textbf{No-Reg} & \textbf{L1-Reg} & \textbf{L2-Reg} & \textbf{Dropout} & \textbf{CF-Reg (ours)} \\ \hline
        Logistic Regression   & \textit{Water}   & N/A   & $0.0093$   & $0.6927$  & N/A    & $0.3791/1.0355$                     \\ \hline
        MLP   & \textit{Water}   & N/A   & $0.0007$   & $0.0022$  & $0.0002$    & $0.2567/1.9775$                    \\ \hline
        Logistic Regression   &
        \textit{Phomene}   & N/A   & $0.0097$   & $0.7979$  & N/A    & $0.0571/1.8516$                     \\ \hline
        MLP   & \textit{Phomene}   & N/A   & $0.0007$   & $4.24\cdot10^{-5}$  & $0.0015$    & $0.0516/2.2700$                    \\ \hline
       % MLP   & \textit{Adult}   & N/A   & $0.0018$   & $0.0018$  & $0.0601$     & $0.0764/2.2068$                    \\ \hline
        CNN   & \textit{CIFAR-10} & N/A   & $0.0050$   & $0.0864$ & N/A    & $0.3018/
        2.1502$                     \\ \hline
    \end{tabular}
\end{table*}

\begin{table*}[htb!]
    \centering
    \caption{Mean value and standard deviation of training time across 5 different runs. The reported time (in seconds) corresponds to the generation of each entry in Table \ref{tab:regularization_comparison}. Times are }
    \label{tab:times}
    \begin{tabular}{|c|c|c|c|c|c|c|}
        \hline
        \textbf{Model} & \textbf{Dataset} & \textbf{No-Reg} & \textbf{L1-Reg} & \textbf{L2-Reg} & \textbf{Dropout} & \textbf{CF-Reg (ours)} \\ \hline
        Logistic Regression   & \textit{Water}   & $222.98 \pm 1.07$   & $239.94 \pm 2.59$   & $241.60 \pm 1.88$  & N/A    & $251.50 \pm 1.93$                     \\ \hline
        MLP   & \textit{Water}   & $225.71 \pm 3.85$   & $250.13 \pm 4.44$   & $255.78 \pm 2.38$  & $237.83 \pm 3.45$    & $266.48 \pm 3.46$                    \\ \hline
        Logistic Regression   & \textit{Phomene}   & $266.39 \pm 0.82$ & $367.52 \pm 6.85$   & $361.69 \pm 4.04$  & N/A   & $310.48 \pm 0.76$                    \\ \hline
        MLP   &
        \textit{Phomene} & $335.62 \pm 1.77$   & $390.86 \pm 2.11$   & $393.96 \pm 1.95$ & $363.51 \pm 5.07$    & $403.14 \pm 1.92$                     \\ \hline
       % MLP   & \textit{Adult}   & N/A   & $0.0018$   & $0.0018$  & $0.0601$     & $0.0764/2.2068$                    \\ \hline
        CNN   & \textit{CIFAR-10} & $370.09 \pm 0.18$   & $395.71 \pm 0.55$   & $401.38 \pm 0.16$ & N/A    & $1287.8 \pm 0.26$                     \\ \hline
    \end{tabular}
\end{table*}

\subsection{Feasibility of our Method}
A crucial requirement for any regularization technique is that it should impose minimal impact on the overall training process.
In this respect, CF-Reg introduces an overhead that depends on the time required to find the optimal counterfactual example for each training instance. 
As such, the more sophisticated the counterfactual generator model probed during training the higher would be the time required. However, a more advanced counterfactual generator might provide a more effective regularization. We discuss this trade-off in more details in Section~\ref{sec:discussion}.

Table~\ref{tab:times} presents the average training time ($\pm$ standard deviation) for each model and dataset combination listed in Table~\ref{tab:regularization_comparison}.
We can observe that the higher accuracy achieved by CF-Reg using the score-based counterfactual generator comes with only minimal overhead. However, when applied to deep neural networks with many hidden layers, such as \textit{PreactResNet-18}, the forward derivative computation required for the linearization of the network introduces a more noticeable computational cost, explaining the longer training times in the table.

\subsection{Hyperparameter Sensitivity Analysis}
The proposed counterfactual regularization technique relies on two key hyperparameters: $\alpha$ and $\beta$. The former is intrinsic to the loss formulation defined in (\ref{eq:cf-train}), while the latter is closely tied to the choice of the score-based counterfactual explanation method used.

Figure~\ref{fig:test_alpha_beta} illustrates how the test accuracy of an MLP trained on the \textit{Water Potability} dataset changes for different combinations of $\alpha$ and $\beta$.

\begin{figure}[ht]
    \centering
    \includegraphics[width=0.85\linewidth]{img/test_acc_alpha_beta.png}
    \caption{The test accuracy of an MLP trained on the \textit{Water Potability} dataset, evaluated while varying the weight of our counterfactual regularizer ($\alpha$) for different values of $\beta$.}
    \label{fig:test_alpha_beta}
\end{figure}

We observe that, for a fixed $\beta$, increasing the weight of our counterfactual regularizer ($\alpha$) can slightly improve test accuracy until a sudden drop is noticed for $\alpha > 0.1$.
This behavior was expected, as the impact of our penalty, like any regularization term, can be disruptive if not properly controlled.

Moreover, this finding further demonstrates that our regularization method, CF-Reg, is inherently data-driven. Therefore, it requires specific fine-tuning based on the combination of the model and dataset at hand.
%\vspace*{-2ex}

\section{Conclusion}
\label{sec:Discussion}
\section{Discussion of Assumptions}\label{sec:discussion}
In this paper, we have made several assumptions for the sake of clarity and simplicity. In this section, we discuss the rationale behind these assumptions, the extent to which these assumptions hold in practice, and the consequences for our protocol when these assumptions hold.

\subsection{Assumptions on the Demand}

There are two simplifying assumptions we make about the demand. First, we assume the demand at any time is relatively small compared to the channel capacities. Second, we take the demand to be constant over time. We elaborate upon both these points below.

\paragraph{Small demands} The assumption that demands are small relative to channel capacities is made precise in \eqref{eq:large_capacity_assumption}. This assumption simplifies two major aspects of our protocol. First, it largely removes congestion from consideration. In \eqref{eq:primal_problem}, there is no constraint ensuring that total flow in both directions stays below capacity--this is always met. Consequently, there is no Lagrange multiplier for congestion and no congestion pricing; only imbalance penalties apply. In contrast, protocols in \cite{sivaraman2020high, varma2021throughput, wang2024fence} include congestion fees due to explicit congestion constraints. Second, the bound \eqref{eq:large_capacity_assumption} ensures that as long as channels remain balanced, the network can always meet demand, no matter how the demand is routed. Since channels can rebalance when necessary, they never drop transactions. This allows prices and flows to adjust as per the equations in \eqref{eq:algorithm}, which makes it easier to prove the protocol's convergence guarantees. This also preserves the key property that a channel's price remains proportional to net money flow through it.

In practice, payment channel networks are used most often for micro-payments, for which on-chain transactions are prohibitively expensive; large transactions typically take place directly on the blockchain. For example, according to \cite{river2023lightning}, the average channel capacity is roughly $0.1$ BTC ($5,000$ BTC distributed over $50,000$ channels), while the average transaction amount is less than $0.0004$ BTC ($44.7k$ satoshis). Thus, the small demand assumption is not too unrealistic. Additionally, the occasional large transaction can be treated as a sequence of smaller transactions by breaking it into packets and executing each packet serially (as done by \cite{sivaraman2020high}).
Lastly, a good path discovery process that favors large capacity channels over small capacity ones can help ensure that the bound in \eqref{eq:large_capacity_assumption} holds.

\paragraph{Constant demands} 
In this work, we assume that any transacting pair of nodes have a steady transaction demand between them (see Section \ref{sec:transaction_requests}). Making this assumption is necessary to obtain the kind of guarantees that we have presented in this paper. Unless the demand is steady, it is unreasonable to expect that the flows converge to a steady value. Weaker assumptions on the demand lead to weaker guarantees. For example, with the more general setting of stochastic, but i.i.d. demand between any two nodes, \cite{varma2021throughput} shows that the channel queue lengths are bounded in expectation. If the demand can be arbitrary, then it is very hard to get any meaningful performance guarantees; \cite{wang2024fence} shows that even for a single bidirectional channel, the competitive ratio is infinite. Indeed, because a PCN is a decentralized system and decisions must be made based on local information alone, it is difficult for the network to find the optimal detailed balance flow at every time step with a time-varying demand.  With a steady demand, the network can discover the optimal flows in a reasonably short time, as our work shows.

We view the constant demand assumption as an approximation for a more general demand process that could be piece-wise constant, stochastic, or both (see simulations in Figure \ref{fig:five_nodes_variable_demand}).
We believe it should be possible to merge ideas from our work and \cite{varma2021throughput} to provide guarantees in a setting with random demands with arbitrary means. We leave this for future work. In addition, our work suggests that a reasonable method of handling stochastic demands is to queue the transaction requests \textit{at the source node} itself. This queuing action should be viewed in conjunction with flow-control. Indeed, a temporarily high unidirectional demand would raise prices for the sender, incentivizing the sender to stop sending the transactions. If the sender queues the transactions, they can send them later when prices drop. This form of queuing does not require any overhaul of the basic PCN infrastructure and is therefore simpler to implement than per-channel queues as suggested by \cite{sivaraman2020high} and \cite{varma2021throughput}.

\subsection{The Incentive of Channels}
The actions of the channels as prescribed by the DEBT control protocol can be summarized as follows. Channels adjust their prices in proportion to the net flow through them. They rebalance themselves whenever necessary and execute any transaction request that has been made of them. We discuss both these aspects below.

\paragraph{On Prices}
In this work, the exclusive role of channel prices is to ensure that the flows through each channel remains balanced. In practice, it would be important to include other components in a channel's price/fee as well: a congestion price  and an incentive price. The congestion price, as suggested by \cite{varma2021throughput}, would depend on the total flow of transactions through the channel, and would incentivize nodes to balance the load over different paths. The incentive price, which is commonly used in practice \cite{river2023lightning}, is necessary to provide channels with an incentive to serve as an intermediary for different channels. In practice, we expect both these components to be smaller than the imbalance price. Consequently, we expect the behavior of our protocol to be similar to our theoretical results even with these additional prices.

A key aspect of our protocol is that channel fees are allowed to be negative. Although the original Lightning network whitepaper \cite{poon2016bitcoin} suggests that negative channel prices may be a good solution to promote rebalancing, the idea of negative prices in not very popular in the literature. To our knowledge, the only prior work with this feature is \cite{varma2021throughput}. Indeed, in papers such as \cite{van2021merchant} and \cite{wang2024fence}, the price function is explicitly modified such that the channel price is never negative. The results of our paper show the benefits of negative prices. For one, in steady state, equal flows in both directions ensure that a channel doesn't loose any money (the other price components mentioned above ensure that the channel will only gain money). More importantly, negative prices are important to ensure that the protocol selectively stifles acyclic flows while allowing circulations to flow. Indeed, in the example of Section \ref{sec:flow_control_example}, the flows between nodes $A$ and $C$ are left on only because the large positive price over one channel is canceled by the corresponding negative price over the other channel, leading to a net zero price.

Lastly, observe that in the DEBT control protocol, the price charged by a channel does not depend on its capacity. This is a natural consequence of the price being the Lagrange multiplier for the net-zero flow constraint, which also does not depend on the channel capacity. In contrast, in many other works, the imbalance price is normalized by the channel capacity \cite{ren2018optimal, lin2020funds, wang2024fence}; this is shown to work well in practice. The rationale for such a price structure is explained well in \cite{wang2024fence}, where this fee is derived with the aim of always maintaining some balance (liquidity) at each end of every channel. This is a reasonable aim if a channel is to never rebalance itself; the experiments of the aforementioned papers are conducted in such a regime. In this work, however, we allow the channels to rebalance themselves a few times in order to settle on a detailed balance flow. This is because our focus is on the long-term steady state performance of the protocol. This difference in perspective also shows up in how the price depends on the channel imbalance. \cite{lin2020funds} and \cite{wang2024fence} advocate for strictly convex prices whereas this work and \cite{varma2021throughput} propose linear prices.

\paragraph{On Rebalancing} 
Recall that the DEBT control protocol ensures that the flows in the network converge to a detailed balance flow, which can be sustained perpetually without any rebalancing. However, during the transient phase (before convergence), channels may have to perform on-chain rebalancing a few times. Since rebalancing is an expensive operation, it is worthwhile discussing methods by which channels can reduce the extent of rebalancing. One option for the channels to reduce the extent of rebalancing is to increase their capacity; however, this comes at the cost of locking in more capital. Each channel can decide for itself the optimum amount of capital to lock in. Another option, which we discuss in Section \ref{sec:five_node}, is for channels to increase the rate $\gamma$ at which they adjust prices. 

Ultimately, whether or not it is beneficial for a channel to rebalance depends on the time-horizon under consideration. Our protocol is based on the assumption that the demand remains steady for a long period of time. If this is indeed the case, it would be worthwhile for a channel to rebalance itself as it can make up this cost through the incentive fees gained from the flow of transactions through it in steady state. If a channel chooses not to rebalance itself, however, there is a risk of being trapped in a deadlock, which is suboptimal for not only the nodes but also the channel.

\section{Conclusion}
This work presents DEBT control: a protocol for payment channel networks that uses source routing and flow control based on channel prices. The protocol is derived by posing a network utility maximization problem and analyzing its dual minimization. It is shown that under steady demands, the protocol guides the network to an optimal, sustainable point. Simulations show its robustness to demand variations. The work demonstrates that simple protocols with strong theoretical guarantees are possible for PCNs and we hope it inspires further theoretical research in this direction.
%

%\newpage
%\bibliographystyle{unsrt}
\bibliographystyle{my-plainnat}
%\bibliographystyle{icml2025}
\bibliography{references}


% APPENDIX
%%%%%%%%%%%%%%%%%%%%%%%%%%%%%%%%%%%%%%%%%%%%%%%%%%%%%%%%%%%%%%%%%%%%%%%%%%%%%%%
%%%%%%%%%%%%%%%%%%%%%%%%%%%%%%%%%%%%%%%%%%%%%%%%%%%%%%%%%%%%%%%%%%%%%%%%%%%%%%%

\newpage
\appendix
\onecolumn

\section{Related Work}
\label{app:related}
\putsec{related}{Related Work}

\noindent \textbf{Efficient Radiance Field Rendering.}
%
The introduction of Neural Radiance Fields (NeRF)~\cite{mil:sri20} has
generated significant interest in efficient 3D scene representation and
rendering for radiance fields.
%
Over the past years, there has been a large amount of research aimed at
accelerating NeRFs through algorithmic or software
optimizations~\cite{mul:eva22,fri:yu22,che:fun23,sun:sun22}, and the
development of hardware
accelerators~\cite{lee:cho23,li:li23,son:wen23,mub:kan23,fen:liu24}.
%
The state-of-the-art method, 3D Gaussian splatting~\cite{ker:kop23}, has
further fueled interest in accelerating radiance field
rendering~\cite{rad:ste24,lee:lee24,nie:stu24,lee:rho24,ham:mel24} as it
employs rasterization primitives that can be rendered much faster than NeRFs.
%
However, previous research focused on software graphics rendering on
programmable cores or building dedicated hardware accelerators. In contrast,
\name{} investigates the potential of efficient radiance field rendering while
utilizing fixed-function units in graphics hardware.
%
To our knowledge, this is the first work that assesses the performance
implications of rendering Gaussian-based radiance fields on the hardware
graphics pipeline with software and hardware optimizations.

%%%%%%%%%%%%%%%%%%%%%%%%%%%%%%%%%%%%%%%%%%%%%%%%%%%%%%%%%%%%%%%%%%%%%%%%%%
\myparagraph{Enhancing Graphics Rendering Hardware.}
%
The performance advantage of executing graphics rendering on either
programmable shader cores or fixed-function units varies depending on the
rendering methods and hardware designs.
%
Previous studies have explored the performance implication of graphics hardware
design by developing simulation infrastructures for graphics
workloads~\cite{bar:gon06,gub:aam19,tin:sax23,arn:par13}.
%
Additionally, several studies have aimed to improve the performance of
special-purpose hardware such as ray tracing units in graphics
hardware~\cite{cho:now23,liu:cha21} and proposed hardware accelerators for
graphics applications~\cite{lu:hua17,ram:gri09}.
%
In contrast to these works, which primarily evaluate traditional graphics
workloads, our work focuses on improving the performance of volume rendering
workloads, such as Gaussian splatting, which require blending a huge number of
fragments per pixel.

%%%%%%%%%%%%%%%%%%%%%%%%%%%%%%%%%%%%%%%%%%%%%%%%%%%%%%%%%%%%%%%%%%%%%%%%%%
%
In the context of multi-sample anti-aliasing, prior work proposed reducing the
amount of redundant shading by merging fragments from adjacent triangles in a
mesh at the quad granularity~\cite{fat:bou10}.
%
While both our work and quad-fragment merging (QFM)~\cite{fat:bou10} aim to
reduce operations by merging quads, our proposed technique differs from QFM in
many aspects.
%
Our method aims to blend \emph{overlapping primitives} along the depth
direction and applies to quads from any primitive. In contrast, QFM merges quad
fragments from small (e.g., pixel-sized) triangles that \emph{share} an edge
(i.e., \emph{connected}, \emph{non-overlapping} triangles).
%
As such, QFM is not applicable to the scenes consisting of a number of
unconnected transparent triangles, such as those in 3D Gaussian splatting.
%
In addition, our method computes the \emph{exact} color for each pixel by
offloading blending operations from ROPs to shader units, whereas QFM
\emph{approximates} pixel colors by using the color from one triangle when
multiple triangles are merged into a single quad.




\section{Additional Information on Neural Operators}
\label{app:learning_fno_don}
\subsection{Learning Operators}\label{subsec:LearningInfiniteDimensions}
\label{subsec:Learning_Operators}We briefly outline the notion of learning for neural operators \citep{li_fourier_2021,li_neural_2020,lu20201DeepONet}. 
Consider two separable Banach spaces, the input space $\gU$ and the output space $\gV$, and a possibly nonlinear operator $G^\dagger: \gU\to \gV$. 

The standard operator learning problem seeks to approximate $G^\dagger$ by a parametric operator $G_{\vtheta}: \gU\to\gV$ that depends on the parameter vector $\vtheta\in \Theta$ defined over some parameter space $\Theta$. 

This is done by proposing an optimization framework where we learn a vector $\vtheta^\dagger\in\Theta$ that ``best'' approximates $G^\dagger$ in some sense. 
Given observations $\{\vu^{(j)}\}_{j=1}^n\in\gU$ and $\{G^{\dagger}(\vu^{(j)})\}_{j=1}^n\in \gV$ where $\vu^{(j)}\sim\mu$, $j=1,\dots,n$, is an i.i.d sequence from the probability measure $\mu$ supported on $\gU$, we take $\vtheta^{\dagger}$ as the solution of the minimization problem
\begin{equation}
    \vtheta^{\dagger}
    =
    \argmin_{\vtheta \in \Theta} \mathbb{E}_{\vu \sim \mu}\left[
        \gC\left(G_{\vtheta}(\vu), G^{\dagger}(\vu)\right)
    \right],
    \label{eq:learningProblemInfiniteDimensions}
\end{equation}
where $\gC$ is a suitable cost functional that measures the discrepancy on the approximation between the operators $G_{\vtheta}(\vu)$ and $G^{\dagger}(\vu)$ for a given $\vu\in\gU$. This optimization problem is analogous to the notion of learning in finite dimensions, which is precisely the setup for which classical deep learning is used.

\subsection{DON Architecture}
The schematic for the Deep Operator Network's architecture is presented in Figure~\ref{fig:deeponetArchitecture}.

\begin{figure}[H]
    \centering
    \includegraphics[width=0.9\textwidth]{Figures/DeepONet_Architecture_ModNew.pdf}
    \caption{
    A schematic of the %\emph{unstacked} 
    DON architecture by \citet{lu20201DeepONet} used in our study. We refer to the notation used in our paper. Note that the input functions need not be sampled on a structured grid of points.
    }
    \label{fig:deeponetArchitecture}
\end{figure}

\subsection{FNO Architecture}
A schematic for the Fourier Neural Operator's architecture is presented in Figure~\ref{fig:fnoArchitecture}.
\begin{figure}[H]
    \centering
    \includegraphics[width=\textwidth]{Figures/FNO_arch.pdf}
    \caption{
    A schematic of the FNO architecture by \citet{li_fourier_2021} used in our study. We refer to the notation used in our paper. ``Spectral convolution'' and ``bypass convolution'' are terms used in the FNO literature to denote the effect of the linear mappings in the spectral and spatial domain, respectively.
    %and point out that  its connection to FNO notation/terminology used in literature.
    }
    \label{fig:fnoArchitecture}
\end{figure}


%\section{Learning Neural Operators}
%\label{app:learning_fno_don}
%%\bsdelete{\subsection{DON Architecture}\label{subsec:ArchitectureDON}
%We now briefly outline the architecture used {in the rest of the paper.} \pcdelete{throughout the analysis in Sections~\ref{sec:optimization} and \ref{sec:ntk} and in the numerical experiments in \Secref{sec:Experiments}.}We adopt fully connected feedforward neural networks (FNNs) for both the branch and trunk nets which is also the baseline DON model in \citet{lu20201DeepONet}.  \Figref{fig:deeponetArchitecture} in the Appendix shows a schematic of the architecture and the notation used throughout this paper. For the architecture we adopt the unstacked configuration (see, Fig 1d in \citet{lu20201DeepONet}). \pccomment{Not sure if this last sentence should be included, especially since citing the figure makes the paper look less self-contained.}}

%
%\bsdelete{{The DON training tuple is $\gD^{(i)}:=\left(\{u^{(i)}(x_r)\}_{r=1}^R,\, \{y^{(i)}_j\}_{j=1}^{q_i}, \{G(u^{(i)})(y^{(i)}_j)\}_{j=1}^{q_i}\right)$, $i\in[n]$ (recall that $n$ is the sample size).}
%\pcdelete{\begin{remark}[DON %Training Tuple]

%\end{remark}}}
%
%\bsdelete{{We denote the width of the branch net by $m_f$ and the trunk net by $m_g$. We assume that both nets have the same width in their hidden layers, and further assume $m_f = m_g = m$ in Section~\ref{sec:optimization}. Similarly, for the experiments we use $m_f = m_g$ unless otherwise stated, in which case (when $m_f\neq m_g$), the analysis in Section~\ref{sec:optimization} still holds with $m=\min (m_f, m_g)$.} \pccomment{I added that both branch and trunk nets have all hidden layers of the same size! This was missing!}}
\subsection{DON training tuple}
Each DON training data comprises of the tuple $\gD^{(i)}:=\left(\{u^{(i)}(x_r)\}_{r=1}^R,\, \{y^{(i)}_j\}_{j=1}^{q_i}, \{G(u^{(i)})(y^{(i)}_j)\}_{j=1}^{q_i}\right)$. The total training dataset comprises of all such training tuples $\gD = \{ \gD^{(i)}\}_{i=1}^n$.

\subsection{Motivation for FNOs}
FNOs are closely related to the notion of fundamental solutions. This allows us to write ${k_{l}:=}k\left(x, y, a(x), a(y) ; \vtheta_{\gF^{(l)}}\right) := k\left(x-y; \vtheta_{\gF^{(l)}}\right)$ \citep{li_fourier_2021}. Taking the Fourier Transform ($\mathscr{F}$) and applying the convolution theorem gives
$\left(\mathcal{K}^{(l)}(a ;  \vtheta_{\gF^{(l)}}) \alpha^{(l-1)}\right)(x)=
\mathscr{F}^{-1}\left(\mathscr{F}\left(k_l\right) \cdot \mathscr{F}\left(\alpha^{(l-1)}\right)\right)(x)$, $\forall x \in \gT$, $l\in [L]$.

This helps in parameterizing the \emph{kernel operator} $R^{(l)} = \mathscr{F}(k_l)$ directly in the Fourier space {and in simplified notation obtain}, 
$\left(\mathcal{K}^{(l)} \alpha^{({l-1})}\right)(x)
=\mathscr{F}^{-1}\left(R^{(l)} \cdot\left(\mathscr{F} \alpha^{({l-1})}\right)\right)(x)$, $\forall x \in \gT$, $l\in [L]$.

{Replacing this quantity back in~\eqref{eq:nonFnoLay},} 
following \citep{li_fourier_2021}, we define each Fourier block as follows
\begin{align}
    \alpha^{(l)}(x)
    =
    \phi\left(W^{(l)} \alpha^{(l-1)} 
    + 
    \mathscr{F}^{-1}\left(R^{(l)} \cdot \mathscr{F}\left(\alpha^{(l-1)}\right)\right)\right)(x), \quad x\in\gT, \quad l \in [L],
    \label{eq:ContinuousFNOBlocks}
\end{align}
where 
the Fourier transform of the input function $\alpha^{({l-1})}$ is $\mathscr{F}\alpha^{({l-1})}(\xi) := \int_{\gT} e^{-2\pi i <\xi ,y>} \alpha^{({l-1})}(y) dy$, $y\in \gT$.
%
{Notice that $R^{(l)}$ is defined by the set of unknown parameters $\vtheta_{\gF^{(l)}}$, whereas $\vtheta_{F^{(l)}}$ is defined by both the affine operator $W^{(l)}$ and parameters $\vtheta_{\gF^{(l)}}$.}
%
% \begin{remark}
% The input $a$, output $u$ and the intermediate functions $\alpha^{(l)}$, $l\in[L]$, in~\eqref{eq:ContinuousFNOBlocks} are all functions, {unlike the case of standard neural networks where they would be vectors.} \pcdelete{(like in standard neural networks).} \bscomment{maybe be a bit more precise on the what spaces these belong to}\pccomment{I agree -- at least the intermediate functions should be such that a Fourier transform exists (i.e., is well-defined)}\qed  
% \end{remark}
We now turn to the discrete version of \eqref{eq:ContinuousFNOBlocks} and the associated architecture.


{Since we have a discrete domain,} we employ the Discrete Fourier Transform (DFT). The entries of the ${m}\times {m}$ DFT kernel ($F$) can be written (up to a suitable scaling) as $F_{kj}:= e^{\frac{-2\pi i}{{m}}(k-1)(j-1)}$, where $k, j \in [{m}]$ which allows us to write its action on an input vector $\aalpha^{(l-1)}{\in\mathbb{R}^m}$ as
    \begin{equation}
        v_k := \sum_{j=1}^{{m}} F_{kj}\alpha^{(l-1)}_j = \alpha^{(l-1)}_j e^{\frac{-2\pi i}{{m}}(k-1)(j-1)},
    \end{equation}
where $i^2 = -1$.

% \pccomment{I think we should establish here our assumption that the decoder is linear and that the encoder is basically an identity function over the input data (at least, it seems to be the identify function to me, unless I am missing something!). The reason is because it is here where we define the architecture. For example, when we introduced the architecture of the DeepOnet, we established all of our architecture assumptions in the section without declaring it as a separate assumption environment.}

% {
% Taking the FNO~\eqref{eq:FNOTotal}, we establish the following correspondence of parameters: $\vtheta_p$ is defined by (????) \pccomment{Complete here!},   $\vtheta_F=[\operatorname{vec}(W^{(1)})^\top,\operatorname{vec}(R^{(1)})^\top, \dots,\operatorname{vec}(W^{(L)})^\top,\operatorname{vec}(R^{(L)})^\top]^\top$, and $\vtheta_q=\v$.}
% %
% Now, given observations $\{a(x_i),u(x_i)\}_{i=1}^n$, the FNO learning problem amounts to determining the optimal values of the parameters $\vtheta = [\vtheta_p^\top\;\vtheta_F^\top\;\vtheta_q^\top]^\top$ by minimizing the empirical risk: 
% % \bscomment{the FNO implementation also define the $L^2$ norm of the error, namely, $e^2 :=\int_D | (u(x) - G_{\vtheta}(a)(x)|^2 dx$ (and using quadrature to evaluate the integral). The FNO paper never explicitly said which error norm they used in the results!}
% \begin{align}
% \label{eq:fno-loss}
%     \vtheta^\dagger = \argmin_{\vtheta\in\Theta}\gL_{\text{fno}}(
%         G_{\vtheta}(\vu), G^{\dagger}(\vu)
%     )
%     =
%     \frac{1}{n}\sum_{i=1}^n \left( G^{\dagger}(\vu)(x_i) - G_{\vtheta}(\vu)(x_i)\right)^2
% \end{align}
% \bscomment{change the FNO notation to have $\vu$ as the input}
% {for some parameter domain $\Theta$ of interest.}
% }

% We now outline the FNO model used in the optimization analysis in Section~\ref{sec:optFNO}. To begin with, we use a feedforward architecture for the FNO blocks as in $1$D examples in \citep{li_fourier_2021}, i.e. when $\gT\subset \R$. We consider the simplest possible setting where both the input $u$ and FNO output $f$ are scalar-valued functions. \bsdelete{ and the domain $\gT$ \pccomment{Question: $\gT{\subseteq \mathbb{R}}$ or $\gT{\subseteq \mathbb{R}^d}$?} is approximated using a $n$-point discretization $\gT^h$, where $\gT^h=\left\{x_1, \cdots x_n\right\}$. \pccomment{Why do you have $\gT^h$ instead of $\gT^n$?}}
% The FNO training dataset $\gD$ corresponds to the point values of the functions $u$ and $f$, i.e. $D:=\left\{u_i, f_i\right\}_{i=1}^n$, where $a_i:=a\left(x_i\right)$ and $f_i = f(x_i)$.
% %
% The widths for each of the FNO blocks are denoted by $m_l$, $l\in [L]$. We will assume in this paper that all Fourier blocks have equal widths throughout the FNO. Furthermore, as is often done in practice, we use discrete fourier transform (DFT). {The DFT let us obtain the discrete counterpart of~\eqref{eq:ContinuousFNOBlocks} which, along with the incorporation of the encoder $P$ and decoder $Q$  as extra layers of the network, let us obtain the entire FNO model. To this end, we will use a slightly modified model with a single-layer feedforward encoder and a single linear layer decoder, namely
% \begin{equation}
% \label{eq:FNOTotal}
% \begin{aligned}
% \alpha^{(0)} &= u \\
% \alpha^{(l)} & = \phi\left(
%         \frac{1}{\sqrt{m}} W^{(l)} \alpha^{(l-1)} +
%         \frac{1}{\sqrt{m}} F^H R^{(l)} F \alpha^{(l-1)}
%     \right),\quad l\in [{L}]\\
%     {f := }  \alpha^{(L+{1})} &:= \frac{1}{\sqrt{m}} \v^T \alpha^{({L})}~,
% \end{aligned}
% \end{equation}
% where $W^{(l)}, R^{(l)} \in \R^{m \times m}$ for $l \in[L]$, $W^{(1)} \in \R^{m \times d}$, $R^{(1)} = 0$. {Notice that we have explicitly scaled the parameters $W^{l}$ and $R^{(l)}$ with $1/\sqrt{m}$, which will be helpful for our analysis.} Furthermore, we denote the parameters of the FNO by $\vtheta = \left[\vtheta_w^\top\ \vtheta_r^\top\ \mathbf{v}^\top \right]^{\top}$, where $\vtheta_w = \left[\text{vec}(W^{(1)})^{\top},\dots ,\text{vec}(W^{(L)})^{\top}\right]^{\top}$ and $\vtheta_r = \left[\text{vec}(R^{(1)})^{\top},\dots ,\text{vec}(R^{(L)})^{\top}\right]^{\top}$.

\section{Optimization Convergence Analysis for Section~\ref{sec:optmain}}
\label{app:rscopt}

We establish relevant results for Section~\ref{sec:optmain}. Our analysis follows very closely the recent work by~\citet{banerjee2022restricted} and generalizes it. We now provide all the relevant proofs.
%, and we provide all proofs here for the sake of completeness.

We start with the following lemma which shows that Condition~\ref{cond:rsc} implies a form of restricted PL condition. 
%
\begin{lemm}[{\bf Restricted PL}]\label{lemm:RPL}
Assume Condition~\ref{cond:rsc} is satisfied. Then, $\gL$ satisfies a restricted form of the Polyak-Łojasiewicz (PL) condition w.r.t.~$(\mathcal{N}_t,\vtheta_t)$: 
\begin{equation}
\label{eq:RPL}
\gL(\vtheta_t) - \inf_{\vtheta \in \mathcal{N}_t} \gL(\vtheta) \leq \frac{1}{2\alpha_t} \| \nabla_\vtheta\gL(\vtheta_t) \|_2^2~.
\end{equation}
\label{lemm:rsc2}
\end{lemm}
\begin{proof}
Define 
\begin{equation*}
\hat{\gL}_{\vtheta_t}(\vtheta) := \gL(\vtheta_t) + \langle \vtheta - \vtheta_t, \nabla_\vtheta\gL(\vtheta_t) \rangle + \frac{\alpha_t}{2} \| \vtheta - \vtheta_t \|_2^2 ~.
\end{equation*}
%Since $\gL$ satisfies $\alpha_t$-RSC w.r.t.~$(Q^t_{\kappa}\cap B_{\rho}(\vtheta_0),\vtheta_t)$ 
By the $\alpha_t$-RSC property of Condition~\ref{cond:rsc}(\pcedit{c}), $\forall \vtheta' \in \mathcal{N}_t$, we have
\begin{align}
    \gL(\vtheta') \geq \hat{\gL}_{\vtheta_t}(\vtheta')~.
\label{eq:dom1}
\end{align}
Further, note that $\hat{\gL}_{\vtheta_t}(\cdot)$ is minimized at $\hat{\vtheta}_{t+1} := \vtheta_t - \nabla_\vtheta\gL(\vtheta_t)/\alpha_t$ and the minimum value is:
\begin{equation*}
    \inf_{\vtheta\in\R^p} \hat{\gL}_{\vtheta_t}(\vtheta) = \hat{\gL}_{\vtheta_t}(\hat{\vtheta}_{t+1}) = \gL(\vtheta_t) - \frac{1}{2\alpha_t} \| \nabla_\vtheta\gL(\vtheta_t) \|_2^2 ~.
\end{equation*}
% In addition, $\hat{\vtheta}_{t+1} \in Q^t_{\kappa}$ since
% \begin{align*}
%     |\cos(\hat{\vtheta}_{t+1} - \vtheta_t, \nabla_\vtheta\gL(\vtheta_t)| = |\cos(- \nabla_\vtheta\gL(\vtheta_t)/\alpha_t, \nabla_\vtheta\gL(\vtheta_t))| = 1 \geq \kappa~,
% \end{align*}
% for any $\kappa \in (0,1/2]$.
%
Then, we have that 
\begin{equation}
\inf_{\vtheta \in \mathcal{N}_t} \hat{\gL}_{\vtheta_t} (\vtheta) \geq \inf_{\vtheta\in\R^p} \hat{\gL}_{\vtheta_t}(\vtheta) = \gL(\vtheta_t) - \frac{1}{2\alpha_t} \| \nabla_\vtheta\gL(\vtheta_t) \|_2^2~.
\label{eq:oxii}
\end{equation}
This means that that $\hat{\gL}_{\vtheta_t}(\cdot)$ is lower bounded by the expression on the right-hand side of~\eqref{eq:oxii} and so we can take the infimum over $\mathcal{N}_t$ on both sides of~\eqref{eq:dom1} and obtain   
\begin{equation}
\inf_{\vtheta \in \mathcal{N}_t} \gL(\vtheta) \geq \inf_{\vtheta \in \mathcal{N}_t} \hat{\gL}_{\vtheta_t} (\vtheta)~.
\label{eq:oxii2}
\end{equation}
Finally, we obtain the expression in~\eqref{eq:RPL} by using both inequalities in~\eqref{eq:oxii} and~\eqref{eq:oxii2} and rearranging terms. 
%
\label{lemm:QTk} 
\end{proof}

Next, we show that the restricted PL condition on $\mathcal{N}_t$ in Lemma~\ref{lemm:RPL} along with smoothness (Condition~\ref{cond:smooth}) can be used to show a loss reduction on $\mathcal{N}_t$. 
%
\begin{lemm}[{\bf Local loss reduction}]
Assume Conditions~\ref{cond:rsc} and \ref{cond:smooth} with $\alpha_t \leq \beta$ at step $t$ of the GD update as in \eqref{eq:gd_at_t} with step-size $\eta_t=\frac{\omega_t}{\beta}$ for some $\omega_t \in(0,2)$.
Then, we have 
%\pcedit{either $\gL(\vtheta_{t+1})\leq \inf_{\vtheta \in \mathcal{N}_t} \gL(\vtheta)$ or }
\begin{equation}
    \gL(\vtheta_{t+1}) - \inf_{\vtheta \in \mathcal{N}_t} \gL(\vtheta)  \leq \left(1-\frac{\alpha_t \omega_t}{\beta}(2-\omega_t) \right) (\gL(\vtheta_t) - \inf_{\vtheta \in \mathcal{N}_t} \gL(\vtheta))~.
    \label{eq:conv-1_app}
\end{equation}
\label{lemm:local_loss}
\end{lemm}
%
\proof 
%\pcedit{If $\gL(\vtheta_{t+1})\leq \inf_{\vtheta \in \mathcal{N}_t} \gL(\vtheta)$, then there is nothing to prove, and so we focus on the case $\gL(\vtheta_{t+1})>\inf_{\vtheta \in \mathcal{N}_t} \gL(\vtheta)$.} 
%
Since $\gL$ is $\beta$-smooth by Condition~\ref{cond:smooth}, we have
\begin{equation}
    \label{eq:lemma2-aux}
\begin{aligned}
\gL(\vtheta_{t+1}) & \leq \gL(\vtheta_t) + \langle \vtheta_{t+1} - \vtheta_t, \nabla_\vtheta\gL(\vtheta_t) \rangle + \frac{\beta}{2}\| \vtheta_{t+1} - \vtheta_t \|_2^2 \\
& = \gL(\vtheta_t) - \eta_t \| \nabla_\vtheta\gL(\vtheta_t) \|_2^2  + \frac{\beta \eta_t^2}{2} \| \nabla_\vtheta\gL(\vtheta_t) \|_2^2 \\
& = \gL(\vtheta_t) - \eta_t \left(1 - \frac{\beta \eta_t}{2} \right) \| \nabla_\vtheta\gL(\vtheta_t) \|_2^2~.
\end{aligned}
\end{equation}
Since $\alpha_t>0$ by assumption, from Lemma~\ref{lemm:RPL} we obtain
\begin{align*}
    - \| \nabla_\vtheta\gL(\vtheta_t) \|_2^2 \leq - 2\alpha_t (\gL(\vtheta_t) - \inf_{\vtheta \in \mathcal{N}_t} \gL(\vtheta) )~.
\end{align*}
Hence
\begin{align*}
\gL(\vtheta_{t+1}) - \inf_{\vtheta \in \mathcal{N}_t} \gL(\vtheta) & \leq \gL(\vtheta_t) - \inf_{\vtheta \in \mathcal{N}_t} \gL(\vtheta)  - \eta_t \left(1 - \frac{\beta \eta_t}{2} \right) \| \nabla_\vtheta\gL(\vtheta_t) \|_2^2  \\
& \overset{(a)}{\leq} \gL(\vtheta_t) - \inf_{\vtheta \in \mathcal{N}_t} \gL(\vtheta) - 
\eta_t \left(1 - \frac{\beta \eta_t}{2} \right)2\alpha_t(\gL(\vtheta_t) - \inf_{\vtheta \in \mathcal{N}_t} \gL(\vtheta)) \\
& = \left(1 - 
2\alpha_t\eta_t \left(1 - \frac{\beta \eta_t}{2} \right)
\right) (\gL(\vtheta_t) - \inf_{\vtheta \in \mathcal{N}_t} \gL(\vtheta))
\end{align*}
where (a) follows for any $\eta_t\leq \frac{2}{\beta}$ because this implies $1-\frac{\beta\eta_t}{2}\geq 0$. Choosing $\eta_t = \frac{\omega_t}{\beta}, \omega_t \in (0,2)$,
$$
    \gL(\vtheta_{t+1}) - \inf_{\vtheta \in \mathcal{N}_t} \gL(\vtheta)  \leq \left(1-\frac{\alpha_t \omega_t}{\beta}(2-\omega_t)\right) (\gL(\vtheta_t) - \inf_{\vtheta \in \mathcal{N}_t} \gL(\vtheta))~.
$$
This completes the proof. \qed

Finally, we show that the local loss reduction result in $\mathcal{N}_t$ from Lemma~\ref{lemm:local_loss} can be extended to show loss reduction in $\mathcal{B}(\vtheta_0)$, which is the main optimization result. 
%
\GlobalLossSmooth*
\begin{proof} We start by showing $\gamma_t = \frac{\inf_{\vtheta \in \mathcal{N}_t} \gL(\vtheta) - \underset{\vtheta \in \cB(\vtheta_0)}{\inf} \gL(\vtheta)}{\gL(\vtheta_t) - \underset{\vtheta \in \cB(\vtheta_0)}{\inf} \gL(\vtheta)}$ satisfies $0 \leq \gamma_t < 1$. \pcedit{First of all, we note that this quantity is well-defined because we are assuming that $\gL(\vtheta_t)\neq \underset{\vtheta \in \cB(\vtheta_0)}{\inf} \gL(\vtheta)$, i.e., that the current iterate does not attain the minimum loss.}
The fact that $\gamma_t \geq 0$ follows immediately from $\underset{\vtheta \in \cB(\vtheta_0)}{\inf} \gL(\vtheta)< \gL(\vtheta_{t})$ and $\underset{\vtheta \in \cB(\vtheta_0)}{\inf} \gL(\vtheta)\leq \underset{\vtheta \in \mathcal{N}_t}{\inf} \gL(\vtheta)$ since $\mathcal{N}_t\subseteq\cB(\vtheta_0)$ by Condition~\ref{cond:rsc}(a).
%
%First, we observe that, since $\bar{\vtheta} \in \underset{\vtheta \in \mathcal{B}(\vtheta_0)}{\arginf} \gL(\vtheta)$, $\bar{\vtheta}_{t+1} \in \underset{\vtheta \in \mathcal{N}_t}{\arginf}~\gL(\vtheta)$ and $\vtheta_{t+1} \in \mathcal{N}_t$ by assumption, we have 
%\begin{align}
%\label{eq:oinkaaa1}
%\gL(\bar{\vtheta}) \leq \inf_{\vtheta \in \mathcal{N}_t} \gL(\vtheta) \leq \gL(\vtheta_{t+1}) \overset{(a)}{\leq} \gL(\vtheta_t) - \eta_t \left( 1 - \frac{\beta \eta_t}{2} \right) \| \nabla_\vtheta\gL(\vtheta_t) \|_2^2 \overset{(b)}{\leq} \gL(\vtheta_t)~, 
%\end{align}
%where (a) follows from~\eqref{eq:lemma2-aux} and (b) follows since $\eta_t = \frac{\omega_t}{\beta}$ with $\omega_t \in (0,2)$. Thus, equation~\eqref{eq:oinkaaa1} immediately shows that $\gamma_t\geq 0$ and that $\gamma_t$ cannot be greater than $1$; i.e., we have shown that $0\leq \gamma_t\leq 1$.
%
%
Now, there are two ways to prove that $\gamma_t<1$ depending on whether we consider 
%element that belongs to the set $\mathcal{N}_t$ in 
Condition~\ref{cond:rsc}(b.1) or Condition~\ref{cond:rsc}(b.2).
%we consider. 

We start by considering Condition~\ref{cond:rsc}(b.1) and 
%, which has $\vtheta_{t+1}\in \mathcal{N}_t$.
%Now, we 
prove by contradiction that $\gamma_t<1$. Assume that $\gamma_t\geq 1$, i.e., 
%Then, 
$\underset{\vtheta \in \mathcal{N}_t}{\inf}\gL(\vtheta)\geq \cL(\vtheta_{t})$. Then, we note that 
\begin{align}
\label{eq:oinkaaa1}
%\gL(\bar{\vtheta}) \leq 
\inf_{\vtheta \in \mathcal{N}_t} \gL(\vtheta) \overset{(a)}{\leq} \gL(\vtheta_{t+1}) \overset{(b)}{\leq} \gL(\vtheta_t) - \eta_t \left( 1 - \frac{\beta \eta_t}{2} \right) \| \nabla_\vtheta\gL(\vtheta_t) \|_2^2 \overset{(c)}{\leq} \inf_{\vtheta \in \mathcal{N}_t} \gL(\vtheta)- \eta_t \left( 1 - \frac{\beta \eta_t}{2} \right) \| \nabla_\vtheta\gL(\vtheta_t) \|_2^2~, 
\end{align}
where (a) follows from $\vtheta_{t+1}\in\mathcal{N}_t$, (b) from~\eqref{eq:lemma2-aux}, and (c) from $\gamma_t\geq 1$. Then, comparing the leftmost and rightmost inequalities in~\eqref{eq:oinkaaa1}, we must have $\| \nabla_\vtheta\gL(\vtheta_t) \|_2 = 0$ since $\eta_t \left( 1 - \frac{\beta \eta_t}{2} \right)>0$ because of $\eta_t = \frac{\omega_t}{\beta}$ with $\omega_t \in (0,2)$. 
%Then, $\| \nabla_\vtheta\gL(\vtheta_t) \|_2 = 0$ 
%
%This then implies that all the inequalities in~\eqref{eq:oinkaaa1} are also equalities, which then implies that $\vtheta_t=\inf_{\vtheta \in \mathcal{N}_t} \gL(\vtheta)$, 
%
Now, when considering Condition~\ref{cond:rsc}(b.1), we either assumed that $\vtheta\notin\mathcal{N}_t$ or that $\gL(\vtheta_t)\neq\underset{\vtheta \in \mathcal{N}_t}{\inf} \gL(\vtheta)$; thus, we analyze both cases.
\begin{enumerate}[(i)]
    \item \label{condi}\textbf{Assuming $\vtheta\notin\mathcal{N}_t$:} Note that $\| \nabla_\vtheta\gL(\vtheta_t) \|_2 = 0$ implies $ \nabla_\vtheta\gL(\vtheta_t) = \vzero_p$ (i.e., the gradient evaluated at $\vtheta_t$ is the zero vector), which then, due to GD in~\eqref{eq:gd_at_t}, implies $\vtheta_t=\vtheta_{t+1}$. Since we had $\vtheta_{t+1}\in\mathcal{N}_t$, this then means that $\vtheta_t\in\mathcal{N}_t$---a contradiction to our assumption.
    \item \label{condii}\textbf{Assuming $\gL(\vtheta_t)\neq\underset{\vtheta \in \mathcal{N}_t}{\inf}\gL(\vtheta)$:} Note that $\| \nabla_\vtheta\gL(\vtheta_t) \|_2 = 0$ implies that all the inequalities in~\eqref{eq:oinkaaa1} are also equalities. This then implies that $\gL(\vtheta_t)=\underset{\vtheta \in \mathcal{N}_t}{\inf}\gL(\vtheta)$---a contradiction to our assumption.
\end{enumerate}
%=============
%This implies $ \nabla_\vtheta\gL(\vtheta_t) = \vzero_p$ (i.e., the gradient is the zero vector), which then, due to GD, implies $\vtheta_t=\vtheta_{t+1}$. This then means that $\vtheta_t\in\mathcal{N}_t$, 
%which contradicts our assumption in Condition~\ref{cond:rsc}(b.1) if we consider the condition $\vtheta_{t}\notin \mathcal{N}_t$. 
%%i.e., that $\vtheta_{t}\notin \arg\inf_{\vtheta\in\mathcal{N}_t}\cL(\vtheta)$. 
%%
%Otherwise, we also observe that all the inequalities in~\eqref{eq:oinkaaa1} are also equalities, which then implies that $\gL(\vtheta_t)=\inf_{\vtheta \in \mathcal{N}_t} \gL(\vtheta)$, which contradicts our assumption in Condition~\ref{cond:rsc}(b.1) if we consider the condition $\gL(\vtheta_t)\neq\inf_{\vtheta \in \mathcal{N}_t} \gL(\vtheta)$.
%% 
%==============
In either case~\eqref{condi} or~\eqref{condii}, our proof by contradiction shows that 
%Therefore, we have 
$\gamma_t < 1$. 
%
%
%===
%Then, using the RSC condition satisfied in $\mathcal{N}_t$ and %$\vtheta_t, 
%$\bar{\vtheta}_{t+1} \in \mathcal{N}_t$, we have 
%\begin{align*}
%\inf_{\vtheta \in \mathcal{N}_t} \gL(\vtheta) & \geq \gL(\vtheta_t) + \langle \bar{\vtheta}_{t+1} - \vtheta_t, \nabla _\vtheta\gL(\vtheta_t)  \rangle + \frac{\alpha_t}{2} \| \bar{\vtheta}_{t+1} - \vtheta_t \|^2_2 \\
%& \overset{(a)}{=}  \gL(\vtheta_t) + \frac{\alpha_t}{2} \| \bar{\vtheta}_{t+1} - \vtheta_t \|^2_2~,
%\end{align*}
%where (a) follows from $\| \nabla_\vtheta\gL(\vtheta_t) \|_2 = 0$. Now, using this expression, we observe that 
%$\inf_{\vtheta \in \mathcal{N}_t} \gL(\vtheta)=\cL(\vtheta_{t})$ can only hold if and only if $\bar{\vtheta}_{t+1} = \vtheta_t$, but this contradicts our assumption in Condition~\ref{cond:rsc}(a), i.e., that $\vtheta_{t}\notin \arg\inf_{\vtheta\in\mathcal{N}_t}\cL(\vtheta)$. Therefore,  
%%since $\inf_{\vtheta \in \mathcal{N}_t} \gL(\vtheta) = \gL(\vtheta_t)$.
%%Further, since $\inf_{\vtheta \in \mathcal{N}_t} \gL(\vtheta) < \gL(\vtheta_t)$, 
%we have $\gamma_t < 1$.
%===

We now consider Condition~\ref{cond:rsc}(b.2) with the 
%Now, we consider the second way of proving $\gamma_t < 1$ by considering 
element $\vtheta'\in\mathcal{N}_t$ as described in the condition's statement.  %Condition~\ref{cond:rsc}(b.2). 
We immediately obtain that $\vtheta'$ satisfies $\underset{\vtheta \in \mathcal{N}_t}{\inf}\gL(\vtheta) \leq \cL(\vtheta') < \cL(\vtheta_t)$, which then implies $\gamma_t<1$.

%Note that we know $a_t>0$ because, otherwise, the iterate $\vtheta_t$ would be in a stationary point, i.e., $\vtheta_t$ would already be a local minimum we are trying to find.

%Having proved that $\gamma_t<1$, 

Having shown that $\gamma_t<1$ according to Condition~\ref{cond:rsc}(b), we now proceed to prove equation~\eqref{eq:conv-0}.
%~\eqref{eq:conv-1}.
\pcedit{We consider two cases: (A) $\gamma_t>0$ and (B) $\gamma_t = 0$.} 

\pcedit{We start by considering Case (A), which holds if and only if $\underset{\vtheta \in \mathcal{N}_t}{\inf}\gL(\vtheta)>\underset{\vtheta \in \cB(\vtheta_0)}{\inf} \gL(\vtheta)$.}
%
%\pcedit{Then, $\gL(\vtheta_{t+1})\neq\inf_{\vtheta \in \mathcal{N}_t} \gL(\vtheta)$ since, otherwise, we would have that $\gL(\vtheta_{t+1})=\inf_{\vtheta \in \mathcal{N}_t} \gL(\vtheta)>\gL(\bar{\vtheta}$ which contradicts the assumption of Case (A).
%
%\pcedit{
%We then need to consider three Sub-cases: (A.1) $\gL(\vtheta_{t+1})=\inf_{\vtheta \in \mathcal{N}_t} \gL(\vtheta)$, (A.2) $\gL(\vtheta_{t+1})< \inf_{\vtheta \in \mathcal{N}_t} \gL(\vtheta)$, and (A.3) $\gL(\vtheta_{t+1})> \inf_{\vtheta \in \mathcal{N}_t} \gL(\vtheta)$.}
%
%\pcedit{For Sub-case (A.1)
%%, if $\gL(\vtheta_{t+1})=\inf_{\vtheta \in \mathcal{N}_t} \gL(\vtheta)$, then 
%it immediately follows that $\gL(\vtheta_{t+1})>\gL(\bar{\vtheta})$ and so 
%\pcedit{\begin{align*}
%\gL(\vtheta_{t+1}) - \gL(\bar{\vtheta}) 
%&\overset{(a)}{=} \inf_{\vtheta \in \mathcal{N}_t} \gL(\vtheta) - \gL(\bar{\vtheta}))\\
%&= \gamma_t (\gL(\vtheta_{t}) - \gL(\bar{\vtheta}))
%~,
%\end{align*}where (a) follows by assumption of Sub-case (A.1).}
%%
%%
%%it immediately follows that $\gL(\vtheta_{t+1})=\gL(\bar{\vtheta})$, i.e., the next iteration attains the minimum loss and there nothing else to prove. 
%%
%For Sub-case (A.2),
%%if $\gL(\vtheta_{t+1})< \inf_{\vtheta \in \mathcal{N}_t} \gL(\vtheta)$, then, 
%since we already have $\inf_{\vtheta \in \mathcal{N}_t} \gL(\vtheta)>\gL(\bar{\vtheta})$, there is nothing much to affirm than the obvious fact that $\gL(\bar{\vtheta})\leq \gL(\vtheta_{t+1})$ must hold. We focus on $\gL(\bar{\vtheta})< \gL(\vtheta_{t+1})$ since there is nothing to prove for  $\gL(\vtheta_{t+1})=\gL(\bar{\vtheta})$. Thus, } 
%\pcedit{\begin{align*}
%\gL(\vtheta_{t+1}) - \gL(\bar{\vtheta}) 
%&\overset{(a)}{<} \inf_{\vtheta \in \mathcal{N}_t} \gL(\vtheta) - \gL(\bar{\vtheta}))\\
%&= \gamma_t (\gL(\vtheta_{t}) - \gL(\bar{\vtheta}))
%~,
%\end{align*}where (a) follows by assumption of Sub-case (A.2).} 
%\pcedit{Now, moving to Sub-case (A.3), let us first
%} 
\pcedit{We now 
define $\delta_t:=\frac{\gL(\vtheta_t)-\inf_{\vtheta \in \mathcal{N}_t} \gL(\vtheta)}{\gL(\vtheta_t)-\underset{\vtheta \in \cB(\vtheta_0)}{\inf} \gL(\vtheta)}$ and note that $\delta_t\in(0,1)$ since $\delta_t=1-\gamma_t$.} %Notice that $\delta_t=1$ only happens whenever $\gL(\bar{\vtheta})=\inf_{\vtheta \in \mathcal{N}_t} \gL(\vtheta)$. 
%
%
Now, with $\omega_t \in (0,2)$, we have
\pcedit{\begin{align*}
\gL(\vtheta_{t+1}) - \underset{\vtheta \in \cB(\vtheta_0)}{\inf} \gL(\vtheta) 
& = \gL(\vtheta_{t+1}) - \inf_{\vtheta \in \mathcal{N}_t} \gL(\vtheta) + \inf_{\vtheta \in \mathcal{N}_t} \gL(\vtheta) - \underset{\vtheta \in \cB(\vtheta_0)}{\inf} \gL(\vtheta) \\
& \overset{(a)}{\leq} \left(1-\frac{\alpha_t \omega_t}{\beta}(2-\omega_t) \right) (\gL(\vtheta_{t}) - \inf_{\vtheta \in \mathcal{N}_t} \gL(\vtheta)) +  (\inf_{\vtheta \in \mathcal{N}_t} \gL(\vtheta) - \underset{\vtheta \in \cB(\vtheta_0)}{\inf} \gL(\vtheta)) \\
& = \left(1-\frac{\alpha_t \omega_t}{\beta}(2-\omega_t)\right)\delta_t (\gL(\vtheta_{t}) - \underset{\vtheta \in \cB(\vtheta_0)}{\inf} \gL(\vtheta)) + (1-\delta_t)(\gL(\vtheta_t) - \underset{\vtheta \in \cB(\vtheta_0)}{\inf} \gL(\vtheta)) \\
& \overset{(b)}{=} \left(1-\frac{\alpha_t \omega_t}{\beta}(2-\omega_t)(1-\gamma_t)\right) (\gL(\vtheta_{t}) - \underset{\vtheta \in \cB(\vtheta_0)}{\inf} \gL(\vtheta))~,
\end{align*}}\pcedit{which is~\eqref{eq:conv-0}, and} where (a) follows from Lemma~\ref{lemm:local_loss} %\pcedit{considering that $\cL(\vtheta_{t+1})>\inf_{\vtheta \in \mathcal{N}_t} \gL(\vtheta)$ by assumption of Sub-case (A.3),} 
\pcedit{and (b) 
follows from
%
$$
\left(1-\frac{\alpha_t\omega_t}{\beta}(2-\omega_t)\right)\delta_t+(1-\delta_t)=1-\frac{\alpha_t\omega_t}{\beta}(2-\omega_t)\delta_t=1-\frac{\alpha_t\omega_t}{\beta}(2-\omega_t)(1-\gamma_t)~.
$$
}

\pcedit{We now consider Case (B), i.e., $\gamma_t=0$, which holds if and only if $\underset{\vtheta \in \mathcal{N}_t}{\inf}\gL(\vtheta)=\underset{\vtheta \in \cB(\vtheta_0)}{\inf} \gL(\vtheta)$.} 
\pcedit{Then, we have 
\pcedit{\begin{align*}
\gL(\vtheta_{t+1}) - \underset{\vtheta \in \cB(\vtheta_0)}{\inf} \gL(\vtheta) 
& = \gL(\vtheta_{t+1}) - \inf_{\vtheta \in \mathcal{N}_t} \gL(\vtheta) \\
& \overset{(a)}{\leq} \left(1-\frac{\alpha_t \omega_t}{\beta}(2-\omega_t) \right) (\gL(\vtheta_{t}) - \inf_{\vtheta \in \mathcal{N}_t} \gL(\vtheta))\\
& = \left(1-\frac{\alpha_t \omega_t}{\beta}(2-\omega_t)\right) (\gL(\vtheta_{t}) - \underset{\vtheta \in \cB(\vtheta_0)}{\inf} \gL(\vtheta))~,
\end{align*}}which is~\eqref{eq:conv-0} when $\gamma_t=0$, and where (a) follows from Lemma~\ref{lemm:local_loss}.}
%considering that $\cL(\vtheta_{t+1})>\inf_{\vtheta \in \mathcal{N}_t} \gL(\vtheta)$ by assumption of Sub-case (B.2).} 
This completes the proof.
\label{theo:global}
\end{proof}



\section{Analysis for Deep Operator Networks}
\label{app:donopt}

\subsection{Bounds on the Hessian, Gradients and the Predictor}


The convergence analysis makes use of the gradients and Hessians of the empirical loss with respect to the parameters $\vtheta$, namely, 
\begin{align}
    \nabla_{\vtheta} \gL(\vtheta) = 
    \begin{bmatrix}
        \nabla_{\vtheta_f}\gL^\top \; \nabla_{\vtheta_g}\gL^\top
    \end{bmatrix}^\top,\quad \text{and} \qquad 
    \nabla_{\vtheta}^2\gL(\vtheta) = 
    \mH
    \left({\vtheta}\right) 
    & = \left[
        \begin{array}{c c}
        H_{ff} & H_{fg} \\
        H_{gf} & H_{gg}
        \end{array}
    \right],
    \label{eq:gradHessDeepONetLoss}
\end{align}
where $\nabla_{\vtheta_f}\gL(\vtheta)=\partial \gL(\vtheta)/\partial\vtheta_f\in \R^{p_f}$ and  $\nabla_{\vtheta_g}\gL(\vtheta)=\partial \gL(\vtheta)/\partial\vtheta_g\in \R^{p_g}$. Note that we make use of the notation $\nabla_{\vtheta_f}(\cdot)$ to denote the derivative with respect to the parameters $\vtheta_f$ and this \emph{is not} a functional gradient. Similarly, the individual blocks in the $2\times 2$ block Hessian $\mH(\vtheta)$ are given by 
\begin{equation}
    H_{ff} = \nabla^2_{\vtheta_f}\gL = \ddel{\gL}{\vtheta_f},\quad H_{fg} = \frac{{\partial^2\gL}}{\partial\vtheta_f\partial\vtheta_g},\quad H_{gf} = H_{fg}^\top= \frac{{\partial^2\gL}}{\partial\vtheta_g\partial\vtheta_f}, \quad H_{gg} = \nabla^2_{\vtheta_g}\gL = \ddel{\gL}{\vtheta_g},\label{eq:hessian_blocks_expanded}
\end{equation}
where $H_{ff}\in\R^{p_f\times p_f}$,\ $H_{gg}\in\R^{p_g\times p_g}$,  $H_{fg}\in\R^{p_f\times p_g}$, $H_{gf}\in \R^{p_g\times p_f}$ and the argument $\vtheta$ is ignored for clarity of exposition. Using \eqref{eq:loss-don} and rewriting the derivatives in \eqref{eq:gradHessDeepONetLoss} and \eqref{eq:hessian_blocks_expanded}, 
recalling that 
$\ell_{i,j}=(G_{\vtheta}(u^{(i)})(\vy_{j}^{(i)})-G^\dagger(u^{(i)})(\vy^{(i)}_j))^2$, 
we get
\begin{align}
    \del{\gL}{\vtheta_f}
    = 
    \frac{1}{n}\sum_{i=1}^n \frac{1}{q_i} \sum_{j=1}^{q_i} 
    \ell^{\prime}_{i,j} 
    \sum_{k=1}^K g_{k,j}^{(i)} \nabla_{\vtheta_f} f_k^{(i)}
    \quad \text{and}\quad
    \del{\gL}{\vtheta_g} 
    =
    \frac{1}{n}\sum_{i=1}^n \frac{1}{q_i} \sum_{j=1}^{q_i} 
    \ell^{\prime}_{ij} 
    \sum_{k=1}^K f_k^{(i)}\nabla_{\vtheta_g} g_{k,j}^{(i)},
    \label{eq:Gradients_fully_expanded}
\end{align}
% for the gradients, and %\todo{BS: hessians (start here)}
\begin{align}
    \begin{aligned}
        \ddel{\gL}{\vtheta_f} &= 
        \frac{1}{n} \sum_{i=1}^n \frac{1}{q_i}\sum_{j=1}^{q_i} \ell_{i,j}^{\prime} \sum_{k=1}^{K} g_{k,j}^{(i)} \nabla_{\vtheta_f}^{2} f_{k}^{(i)} 
        +
        \frac{1}{n}\sum_{i=1}^n \frac{1}{q_i} \sum_{j=1}^{q_i} \ell_{i,j}^{\prime \prime}
        \left(
            \sum_{k,k^\prime=1}^{K} g_{k,j}^{(i)}g_{k^\prime,j}^{(i)} \nabla_{\vtheta_f} f_{k}^{(i)} \nabla_{\vtheta_f} f_{k^\prime}^{{(i)}\top} \right),\\
        \ddel{\gL}{\vtheta_g} 
        &=
        \frac{1}{n} \sum_{i=1}^{n} \frac{1}{q_i} \sum_{j=1}^{q_i} \ell_{i,j}^{\prime} \sum_{k=1}^{k} f_{k}^{(i)} \nabla_{\vtheta_g}^{2} g_{k,j}^{(i)}
        +
        \frac{1}{n}\sum_{i=1}^n \frac{1}{q_i} \sum_{j=1}^{q_i}
        \ell_{i,j}^{\prime \prime}
        \left(\sum_{k,k^\prime=1}^{K} f_{k}^{(i)}f_{k^\prime}^{(i)} \nabla_{\vtheta_g} g_{k,j}^{(i)
        } \nabla_{
\vtheta_g} {g_{k^\prime,j}}^{{(i)}\top}
        \right),\\
        \frac{{\partial^2\gL}}{\partial\vtheta_f\partial\vtheta_g}
        &=
            \frac{1}{n}\sum_{i=1}^n \frac{1}{q_i} \sum_{j=1}^{q_i} 
        \ell_{i,j}^{\prime} 
        \sum_{k=1}^{K} \nabla_{\vtheta_f} f_{k}^{(i)} \nabla_{\vtheta_g} g_{k,j}^{(i){\top}}
        +
            \frac{1}{n}\sum_{i=1}^n \frac{1}{q_i} \sum_{j=1}^{q_i} \ell_{i,j}^{\prime \prime}
        \left(
            \sum_{k,k'=1}^{K} g_{k,j}^{(i)}f_{k'}^{(i)} \nabla_{\vtheta_f} f_{k}^{(i)}
             \nabla_{\vtheta_g} {g_{k',j}}^{{(i)}\top}
        \right),
    \end{aligned}
    \label{eq:Hessian_blocks_fully_Expanded}
\end{align}
for the individual blocks of the Hessian \eqref{eq:gradHessDeepONetLoss} where we make use of the notation $f^{(i)}_k = f_k (\vtheta_f; u^{(i)})$ and $g^{(i)}_{k,j} = g_k(\vtheta_g;\vy^{(i)}_j)$. In the rest of the paper, with some abuse of notation, we also make use of the implicit notation $f^{(i)}_k(\vtheta_f) = f_k (\vtheta_f; u^{(i)})$ and $g^{(i)}_{k,j}(\vtheta_g) = g_k(\vtheta_g;\vy^{(i)}_j)$.

In order to prove the RSC and smoothness properties of the empirical loss $\cL$, we need to upper bound the spectral norm of its Hessian. As can be seen above, the gradient and Hessian of the predictors (i.e., the branch $f_k^{(i)}$ and trunk $g^{(i)}_{k,j}$ networks, $k\in[K]$, $j\in[q_i]$, $i\in[n]$) appear in the Hessian of $\cL$, and thus, we will eventually need the upper bound of their norms. For this, we will make use of the next lemma.  
\begin{lemm}[{\bf Bounds on the predictors}]
\label{lemm:hessgradbounds}
Under \TwoAsmpsref{asmp:Activation_Function}{asmp:smoothinit}, and for $\vtheta \in B^{\mathrm{Euc}}_{\rho,\rho_1}(\vtheta_0)$, with probability at least {$1-2KL\left(\frac{1}{m_f}+\frac{1}{m_g}\right)$}, we have for every $k\in [K]$, $i\in[n]$, $j\in[q_i]$,
\begin{align}
    \begin{aligned}
        % \max _{i \in[n]}\max_{j\in[q_i]}
        \left\|\nabla^2_{\vtheta_f} f^{(i)}_k \right\| \leq \frac{c^{(f)}}{\sqrt{m_f}} \quad \text{and}\quad 
        %\max _{i \in[n]}\max_{j\in[q_i]}
        \left\|\nabla^2_{\vtheta_g} g^{(i)}_{k,j} \right\| \leq \frac{c^{(g)}}{\sqrt{m_g}}~,
        \\
    \left\| \nabla_{\vtheta_f} f_k^{(i)}\right\|_2 \leq \varrho^{(f)} \quad \text{and}\quad \left\| \nabla_{\vtheta_g} g^{(i)}_{k,j}\right\|_2 \leq \varrho^{(g)}~,\\
    %
     |f_k^{(i)}| \leq \lambda_1, \quad \text{and}\quad |  g^{(i)}_{k,j}| \leq \lambda_2~,
    \end{aligned}\label{eq:gradientBoundG_fg}
\end{align}
where $c^{(f)}$, $c^{(g)}$, $\varrho^{(f)}$, $\varrho^{(g)}$, $\lambda_1$, and $\lambda_2$ are suitable constants that depend on $\sigma_0$, the depth $L$ and the radii $\rho$, $\rho_1$. 
%\pcedit{The dependence of the constants is polynomial and only on the depth and the radii whenever $\sigma_0\leq 1-\rho\max\{\frac{1}{\sqrt{m_f}},\frac{1}{\sqrt{m_g}}\}$.}
%
\pcedit{The dependence of the constants reduces to the depth and the radii and becomes polynomial whenever $\sigma_0\leq 1-\rho\max\{\frac{1}{\sqrt{m_f}},\frac{1}{\sqrt{m_g}}\}$.}
%
\end{lemm}
\proof The proof follows from a direct adaptation of Theorem~4.1 \pcedit{and of both the statement and proof of} Lemma~4.1 in~\citep{banerjee2022restricted} to our setting. \pcedit{Indeed, these results show that $c^{(f)}$, $\varrho^{(f)}$, and $\lambda_1$ depend on $\sigma_0$, the depth $L$ and the radii $\rho$, $\rho_1$; and that such dependence reduces to the depth and the radii and becomes polynomial whenever $\sigma_0\leq 1-\frac{\rho}{\sqrt{m_f}}$. A similar dependence is obtained for the constants $c^{(g)}$, $\varrho^{(g)}$, and $\lambda_2$ whenever $\sigma_0\leq 1-\frac{\rho}{\sqrt{m_g}}$. The last statement in Lemma~\ref{eq:gradientBoundG_fg} follows immediately.} 
Finally, since the bound for a single branch network output holds with probability at least $1-\frac{2L}{m_f}$ and for a single trunk network output holds with probability at least $1-\frac{2L}{m_g}$, then in order for these bounds to hold for the $K$ outputs of all predictors, we obtain the overall probability using De Morgan's law and a union bound.\qed

% % 

\subsection{RSC and Smoothness Results}
%\label{subsec:SmoothActivationProofs}
Using the results from the previous section, we derive the RSC and smoothness results. 

\RSCLoss*


\proof 
%
We start by proving the first part of the theorem's statement. 
%We immediately see that $\vtheta_{t}\notin Q_{\kappa}^t$ by definition of $Q_{\kappa}^t$, and thus $\vtheta_{t}\notin B^t_{\kappa}$. Then, 
%
Since $B^t_{\kappa}\subset B^{\mathrm{Euc}}_{\rho,\rho_1}(\vtheta_0)$, we satisfy Condition~\ref{cond:rsc}(a). We now need to satisfy Condition~\ref{cond:rsc}(b). For this, we first show the existence of an element $\vtheta' \in B^t_{\kappa}$. For such $\vtheta'$, it must be true that $\vtheta' \in Q_{\kappa}^t$. From Definition~\ref{defn:qset_DON}, 
% the definition of $Q^t_{\kappa}$, 
$\vtheta'$ needs to satisfy three conditions: 
%which we refer to as the cosine similarity condition, average condition, and output condition, for convenience:
\begin{align*}
|\cos(\vtheta' - \vtheta_t, \nabla_{\vtheta} \bar{G}_{\vtheta_t})| & \geq \kappa \quad \text{(cosine similarity condition)}~,\\
%
(\vtheta'_f-\vtheta_{f,t})^\top\left(\frac{1}{n} \sum_{i=1}^n \frac{1}{q_i} \sum_{j=1}^{q_i} \ell'_{i,j} \sum_{k=1}^K \nabla_{\vtheta_{f}} f_k^{(i)} \nabla_{\vtheta_{g}} g_{k,j}^{(i)~\top}\right)(\vtheta'_g-\vtheta_{g,t})  &\geq 0\quad \text{(average condition)}~,\\
%
(\vtheta'_f-\vtheta_{f,t})^\top\left( \sum_{k=1}^K \nabla_{\vtheta_{f}} f_k^{(i)} \nabla_{\vtheta_{g}} g_{k,j}^{(i)~\top}\right)(\vtheta'_g-\vtheta_{g,t})&\leq 0,\forall i\in[n],\forall j\in[q_i]
\quad \text{(output condition)}.
\end{align*}
%We also notice that the output condition is in reality a system of $q=\sum^n_{i=1}q_i$ quadratic inequalities, each one with $p_g+p_f$ variables, since $\vtheta'\in\R^{p_f+p_g}$ is the variable of interest.
%
%Now, we prove that the $Q^t_{\kappa}$ set is non-empty. 
Let us consider 
%$\vtheta'=[{\vtheta'_f}^{\top}\;{\vtheta'_g}^{\top}]^{\top}=[{\bar{\vtheta}_f}^{\top}\;\vzero_{p_g}^\top]^{\top} + \vtheta_t$ 
%
$\vtheta'=[{\vtheta'_f}^{\top}\;{\vtheta'_g}^{\top}]^{\top}$,
%
%(where $\vzero_{q}$ is the zero vector of dimension $q$), 
where 
$\vtheta'_f\in\R^{p_f}$ will be specified later and $\vtheta'_g = \vtheta_{g,t}$. 
%$\bar{\vtheta}_f\in\R^{p_f}$. 
%Let $\bar{\g} = [\bar{\g}_f^{\top}\;\bar{\g}_g^{\top}]^{\top} :=  \frac{\nabla_{\vtheta} \bar{G}_{\vtheta_t}}{\| \nabla_{\vtheta} \bar{G}_{\vtheta_t}\|_2}\in\R^{p_f+p_g}$.
Then, belonging to the $Q_\kappa^t$ set conveniently reduces to the feasibility of the cosine similarity condition as follows: 
\begin{align}
\label{eq:cs-cond}
  %(\langle \vtheta'_f , \bar{\g}_f \rangle)^2  & \geq \kappa^2 \quad \text{(cosine similarity condition)}.
  |\cos(\vtheta'_f-\vtheta_{f,t}\,, \bar{\g}_f \rangle)| \geq \kappa~,
\end{align}
where $\bar{\g}_f$ is the first $p_f$ components of the gradient $\nabla_{\vtheta}\bar{G}_{\vtheta_t}$ (recall that the cosine computation is invariant to the vector norms).
%
%It is without loss of generalization that we have the condition above in terms of $\bar{\g}_f$ instead of the first $p_f$ components of the gradient $\nabla_{\vtheta}\bar{G}_{\vtheta_t}$, since the cosine computation is invariant to the vector norms.
%
%
%$\bar{\g}=[\bar{\g}_f^{\top}\;\bar{\g}_g^{\top}]^{\top} \in \R^{p_f+p_g}$ and $\vtheta' =[\vtheta'_f;\vtheta'_g] \in \R^{p_f+p_g}$ are unit vectors, and
%
%We will choose a suitable $\kappa \in (0,\frac{1}{2}]$. 
%Now, to construct $\vtheta'_f$ satisfying the above condition, without loss of generality, let us assume $\| \bar{\g}_f\|_2 \geq \| \bar{\g}_g \|_2$ (the case $\| \bar{\g}_f\|_2 \leq \| \bar{\g}_g \|_2$ is treated similarly) so that $\| \bar{\g}_f\|_2 \geq \frac{1}{\sqrt{2}}$. Then, we simply set $\vtheta'_f = \frac{\bar{\g}_f}{\| \bar{\g}_f \|_2}$ so that 
%\begin{align*}
%\langle \vtheta'_f, \bar{\g}_f \rangle^2 = \left( \frac{\bar{\g}_f}{\| \bar{\g}_f \|_2} \bar{\g}_f  \right)^2 = \| \bar{\g}_f \|_2^2 \geq \frac{1}{2}~,
%\end{align*}
%and so the cosine similarity condition is satisfied given that 
% so that the feasibility condition is satisfied for 
%$\kappa \in (0,\frac{1}{\sqrt{2}}]$ (see Definition~\eqref{defn:qset}). This %completes the proof. 
%
%======**************************
%

With all of this in mind, we proceed to show the existence of an element $\vtheta' \in B^t_{\kappa}$ of the form $\vtheta'=[{\vtheta'_f}^{\top}\;{\vtheta_{g,t}}^{\top}]^{\top}$ 
%$\vtheta'=[\bar{\vtheta}_f^{\top}\;\vzero_{p_g}^\top]^{\top}+\vtheta_t$ 
%
%
satisfying condition~\eqref{eq:cs-cond} and the following two conditions:
%, using the notation $\vtheta_{t}={[{\vtheta'_f}^{\top}\;\vzero_{p_g}^\top]}^{\top}$
%:
%: %such that 
\begin{enumerate}[{Condition} (A):]%[(a)]
\item  $\|\vtheta'_f - \vtheta_{f,t} \|_2 = \epsilon$ for some $\epsilon< \frac{2 \norm{\nabla_{\vtheta_f} \cL(\vtheta_t)}_2 \sqrt{1-\kappa^2}}{\beta}$; and \label{cond-1-don}
\item the angle $\nu'$ between $(\vtheta'_f - \vtheta_{f,t})$ and $-\nabla_{\vtheta_f} \cL(\vtheta_t)$ is acute, so that $\cos(\nu') > 0$. \label{cond-2-don}\end{enumerate}
%

%Now, let us take the following three points: $\vtheta_{t+1}$, $\bar{\ggamma}_t + \vtheta_t$, and $\vtheta_t$.
%\begin{itemize}
%    \item If the points are not collinear, then they define a plane and we take $\vtheta'$ as a point in this plane with the conditions above.
%    \item If the points are collinear, then we take some $\vtheta'$ according to the conditions above and
%    %defining the acute angle with respect to $\vtheta_{t+1}$ (which is $\nu'$). Now 
%    define a new plane with this third point $\vtheta'$. %, and then the proof follows.
%\end{itemize}
%Notice that $\vtheta_t$ is taken as the origin of the plane being formed (by the definition of gradient descent, $\theta_{t+1}=-\nabla_{\vtheta}\cL(\vtheta_t)$.
%===
%If $\cos(\nu')\geq \sqrt{1-\kappa^2}$, we then have that Conditions~\eqref{cond-1-don} and~\eqref{cond-2-don} altogether satisfy the sought Condition~\ref{cond:rsc}(b) and finishes the proof of the first part of the theorem's statement. 

To show the existence of such element $\vtheta' \in B_t$, we propose two possible constructions:
%whose conditions of existence we will be shown afterwards:
%make \pcedit{either of} the \pcedit{specific choices}:% choice specific: 
\begin{enumerate}[{Choice} (A):]
\item  
If the points $\vtheta_{f,t+1}$, $\bar{\g}_f + \vtheta_{f,t}$, and $\vtheta_{f,t}$ are not collinear, then they define a hyperplane $\mathcal{P}$ that contains the vectors $\bar{\g}_f$ and $-\nabla_{\vtheta_f} \cL(\vtheta_t)$ (recall that $\vtheta_{f,t+1} - \vtheta_{f,t}=-\nabla_{\vtheta_f}\cL(\vtheta_t)$ by gradient descent). We choose $\vtheta'_f$ such that the vector 
$\vtheta'_f-\vtheta_{f,t}$ 
%(or, equivalently, the vector $\bar{\vtheta}_f$)
lies in 
$\mathcal{P}$ 
%in the hyperplane $\mathcal{P}$ containing $\bar{g}_t$ and $-\nabla \cL(\vtheta_t)$ in 
%
%(between the vectors 
%$\bar{\ggamma}_t$ and $-\nabla \cL(\vtheta_t)$) 
%
%such that
with 
$\cos(\vtheta'_f-\vtheta_{f,t},\bar{\g}_f)=\kappa$ (i.e., it satisfies condition~\eqref{eq:cs-cond} with equality) while simultaneously satisfying Condition~\eqref{cond-2-don}.
%
%
If the points $\vtheta_{f,t+1}$, $\bar{\g}_f + \vtheta_{f,t}$, and $\vtheta_{f,t}$ are collinear, we choose $\vtheta'_f$ such that it is not collinear with these points, thus defining a hyperplane $\mathcal{P}$ %(which contains the vectors $(\vtheta'-\vtheta_t)$, $\bar{\ggamma}_t$ and $-\nabla_{\vtheta}\cL(\vtheta_t)$ sharing the same origin $\vtheta_t$). Such %
with these other three points, 
and such that 
$\vtheta'_f$ is also taken so that $\cos(\vtheta'_f-\vtheta_{f,t},\bar{\g}_f)=\kappa$ while simultaneously satisfying Condition~\eqref{cond-2-don}.
%

Thus far we have only defined \emph{angle} (or \emph{direction}) conditions on the vector $\vtheta'_f-\vtheta_{f,t}$, and so there could be an infinite number of values for $\vtheta'_f$ satisfying such angle conditions without $\vtheta'$ belonging to the set $B^{\mathrm{Euc}}_{\rho,\rho_1}(\vtheta_0)$ nor $\vtheta'_f$ satisfying Condition~\eqref{cond-1-don}. To determine the feasible values for $\vtheta'_f$, we observe that $\vtheta_t$ is \emph{strictly inside} the set $B^{\mathrm{Euc}}_{\rho,\rho_1}(\vtheta_0)$ by Assumption~\ref{asmp:iter-1}, and so 
%
%and so now we set the magnitude of $\vtheta'_f$ such that $\vtheta'\in  B^{\mathrm{Euc}}_{\rho,\rho_1}(\vtheta_0)$, which is possible since  and 
%
$\vtheta'_f$ can be taken arbitrarily close to $\vtheta_{f,t}$ so that $\vtheta'\in B^{\mathrm{Euc}}_{\rho,\rho_1}(\vtheta_0)$ and Condition~\eqref{cond-1-don} is satisfied. 
%

We remark that, regardless of the collinearity of the points 
$\vtheta_{f,t+1}$, $\bar{\g}_f + \vtheta_{f,t}$, and $\vtheta_{f,t}$, hyperplane $\mathcal{P}$ contains the vectors $\vtheta'_f-\vtheta_{f,t}$, $\bar{\g}_f$, and $-\nabla_{\vtheta_f}\cL(\vtheta_t)$, all sharing its origin at $\vtheta_{f,t}\in\mathcal{P}$. \label{ch-A-don}
%
%; or
\item  
%
%$(\vtheta'-\vtheta_t)$
%lies in the hyperplane containing $-\bar{g}_t$ and $-\nabla_{\vtheta} \cL(\vtheta_t)$ in between these vectors 
%such that $\cos(\vtheta'-\vtheta_t,-\bar{\ggamma}_t)=\kappa$, i.e., the boundary of the $Q^t_{\kappa}$ set.
%
We choose $\vtheta'$ as in Choice~\eqref{ch-A-don} but with $\bar{\g}_f$ replaced by $-\bar{\g}_f$.
%
%$(\vtheta'-\vtheta_t)$
%lies in the hyperplane containing $-\bar{g}_t$ and $-\nabla_{\vtheta} \cL(\vtheta_t)$ in between these vectors 
%such that $\cos(\vtheta'-\vtheta_t,-\bar{\ggamma}_t)=\kappa$, i.e., the boundary of the $Q^t_{\kappa}$ set.
%We choose 
\label{ch-B-don}
%
%
\end{enumerate}
We immediately notice that $\vtheta'$ defined by either Choice~\eqref{ch-A-don} or Choice~\eqref{ch-B-don} satisfies 
$\vtheta'\in Q^t_\kappa \cap B^{\mathrm{Euc}}_{\rho,\rho_1}(\vtheta_0)$. To make $\vtheta'$ belong to the set $B^t_\kappa$, we need to find a radius $\rho_2$ such that $\vtheta'\in B^{\mathrm{Euc}}_{\rho_2}(\vtheta_t)$, or, equivalently, such that $\vtheta'_f\in B^{\mathrm{Euc}}_{\rho_2}(\vtheta_{f,t})$ due to our construction of $\vtheta'$. Such $\rho_2$ is found by taking $\rho_2>\epsilon$ with $\epsilon$ as in Condition~\eqref{cond-1-don}. 
%
Finally, it is straightforward to verify that such $\vtheta'\in B^t_\kappa$ defined by either Choice~\eqref{ch-A-don} or Choice~\eqref{ch-B-don} will always exist, by considering the following cases for the angle $\nu$ between $\bar{\g}_f$ and $-\nabla_{\vtheta_f} \cL(\vtheta_t)$:
\begin{enumerate}[(i)]
\item If $\nu \in [0, \pi/2]$ or $\nu \in [3\pi/2, 2\pi]$, then Choice~\eqref{ch-A-don} will be true, since $-\nabla_{\vtheta_f} \cL(\vtheta_t)$ is in the positive half space\footnote{We say $\a$ is in the positive half-space of $\b$ if $\langle \a, \b \rangle \geq 0$.} of $\bar{\g}_f$; and
\label{it-i-don}
\item if $\nu \in [\pi/2,\pi]$ or $\nu \in [\pi, 3\pi/2]$, then Choice~\eqref{ch-B-don} will be true, since $-\nabla_{\vtheta_f} \cL(\vtheta_t)$ is in the positive half space of $-\bar{\g}_f$.\label{it-ii-don}
\end{enumerate}

%To finish the proof, we only need to prove $\cos(\nu')\geq \sqrt{1-\kappa^2}$. 
%
Now, let us assume we are in the case of item~\eqref{it-i-don} above, so that $\vtheta'$ is constructed according to Choice~\eqref{ch-A-don} (the rest of the proof can be adapted to the case of item~\eqref{it-ii-don} by using a symmetrical argument and so it is omitted). %Assume the angle conditions in item~\eqref{it-i-don}. 
% Now, we show that
% \begin{enumerate}[(a)]
% \item $|\cos(\vtheta'-\vtheta_t,\nabla \cL(\vtheta_t)| \geq \sqrt{1-\kappa^2}$, and 
% \item .
% \end{enumerate}
% To get (a)
Let $\nu_1$ be the angle between $\vtheta'_f-\vtheta_{f,t}$ and $\bar{\g}_f$, so that $\cos(\nu_1)=\kappa$ according to Choice~\eqref{ch-A-don}.
%
%unknown, $\nu_2 = \nu'$, and $\nu_3 = \nu$ respectively denoting the angles between , $((\vtheta'-\vtheta_t), - \nabla \cL(\vtheta_t))$, and $(\bar{\ggamma}_t, - \nabla \cL(\vtheta_t))$, we have $\cos(\nu_1)=\kappa$.
Then, we have that 
\begin{align*}
|\cos(\nu')| = |\cos(\nu - \nu_1)| \geq |\cos(\pi/2 - \nu_1)| = |\sin(\nu_1)| = \sqrt{1-\cos^2(\nu_1)} = \sqrt{1-\kappa^2}~.
\end{align*}
Further, by the construction in Condition~\eqref{cond-2-don}, $\cos(\nu') > 0$, which implies 
%\begin{equation}
$\cos(\nu') \geq  \sqrt{1-\kappa^2}>0$.
%\label{eq:cosnu}
%\end{equation}
%$$. 
%
%
%Now, we take $\rho_2$ sufficiently large so that $\rho_2>\epsilon$, which immediately implies $\vtheta'\in B^{\mathrm{Euc}}_{\rho_2}(\vtheta_0)$ and which, along with the construction in Choice~\ref{ch-A-don}, implies that $\vtheta'\in B^t_\kappa$. Now, equation~\eqref{eq:cosnu}
%All the above derivations imply that and so $\vtheta'\in B^{\mathrm{Euc}}_{\rho_2}(\vtheta_0)$
%
Now, 
%Let $\nu$ be the angle between $\vtheta'-\vtheta_t$ and $-\nabla_\vtheta\cL(\vtheta_t)$, so that $\cos(\nu)>0$ by assumption. Now, for such $\vtheta' \in \mathcal{N}_t$ and with  $\epsilon:=\norm{\vtheta'-\vtheta_t}_2$, 
by the smoothness property of the empirical loss $\cL$ we have
\begin{align*}
\cL(\vtheta') & \leq \cL(\vtheta_t) - \langle \vtheta' - \vtheta_t, -\nabla_\vtheta \cL(\vtheta_t) \rangle + \frac{\beta}{2}\| \vtheta' - \vtheta_t \|_2^2 \\ 
& = \cL(\vtheta_t) - \|\vtheta'_f - \vtheta_{f,t}\|_2 \|\nabla_{\vtheta_f} \cL(\vtheta_t) \|_2 \cos(\nu) + \frac{\beta}{2}\| \vtheta'_f - \vtheta_{f,t} \|_2^2 \\
& =  \cL(\vtheta_t) -  \epsilon \|\nabla_{\vtheta_f} \cL(\vtheta_t) \|_2 \cos(\nu) + \frac{\beta}{2} \epsilon^2 \\
& = \cL(\vtheta_t) -  \frac{\beta \epsilon}{2} \left( \frac{2 \|\nabla_{\vtheta_f} \cL(\vtheta_t) \|_2 \cos(\nu)}{\beta} - \epsilon \right)\\
&<\cL(\vtheta_t)~.
\end{align*}
where the last inequality follows by the construction of $\epsilon$ in Condition~\eqref{ch-A-don}. Note that this implies that the constructed $\vtheta'$ is as described in Condition~\ref{cond:rsc}(b.2). This finishes the proof for Condition~\ref{cond:rsc}(b).

%We choose 
%%To get (b), \abedit{we know that $\theta_2$ is an acute angle and $\cos(-\theta_2) = -\cos(\theta_2)$.
%%
%$\epsilon$ such that $\epsilon < \frac{2 \norm{\nabla_\vtheta \cL(\vtheta_t)}_2 \sqrt{1-\kappa^2}}{\beta}$ and $\epsilon > 0$. With this choice of $\epsilon$, we take $\rho_2>\epsilon$ so that $\vtheta'\in B^{\mathrm{Euc}}_{\rho_2}(\vtheta_0)$.
%%we take  $\rho_2>\epsilon$. 
%%This implies that $\vtheta'\in B^t_\kappa$ and so, keeping in mind the choice of $\epsilon$ we made, it immediately follows that $\vtheta'\in B^t_{\kappa}$ and that $B^t_{\kappa}$ 
%%is a non-empty set that 
%satisfies Condition~\ref{cond:rsc}(a) and (b). This finishes the first part of the proof.
%===


%=========================================================================
%=========================================================================
%=========================================================================
%=========================================================================



We now proceed to prove the second part of the proof. 
For any $\vtheta' \in B^t_{\kappa}$, by the second order Taylor expansion of the DON loss with respect to iterate $\vtheta_t$, we have
\[
\cL(\vtheta') = \cL(\vtheta_t) + \langle \vtheta' - \vtheta_t, \nabla_\vtheta\cL(\vtheta_t) \rangle + \frac{1}{2} (\vtheta'-\vtheta_t)^\top \frac{\partial^2 \cL(\tilde{\vtheta})}{\partial \vtheta^2} (\vtheta'-\vtheta_t)~,
\]
where $\tilde{\vtheta} = \xi \vtheta' + (1-\xi) \vtheta_t$ for some $\xi \in [0,1]$. To establish $\alpha_t$-RSC of the loss with $\alpha_t$ as in \eqref{eq:RSCLoss}, it suffices to focus on the quadratic form of the Hessian and show 
\begin{equation}
(\vtheta' - \vtheta_t)^{\top} \frac{\partial^2 \cL(\tilde{\vtheta})}{\partial \vtheta^2}  (\vtheta - \vtheta_t) \geq \alpha_t \| \vtheta' - \vtheta_t \|_2^2~.
\end{equation}
Note that the Hessian, by chain rule, is given by
\begin{align*}
\mH (\tilde{\vtheta}) & := \frac{\partial^2 \cL(\tilde{\vtheta})}{\partial \vtheta^2}  
=  \frac{1}{n} \sum_{i=1}^n \frac{1}{q_i} \sum_{j=1}^{q_i}  \left( \ell^{\prime\prime}_{i,j} \nabla_{\vtheta} G_{\tilde{\vtheta}}(u^{(i)})(\vy^{(i)}_j) \nabla_{\vtheta} G_{\tilde{\vtheta}}(u^{(i)})(\vy^{(i)}_j)^\top  
+ \ell^{\prime}_{i,j}   \nabla^2 G_{\tilde{\vtheta}}(u^{(i)})(\vy^{(i)}_j) \right)~. 
\end{align*}
%from the Taylor expansion in \eqref{eq:local_taylor_expansion_loss}, 
where 
$\ell_{i,j}=(G_{\tilde{\vtheta}}(u^{(i)})(\vy_{j}^{(i)})-G^\dagger(u^{(i)})(\vy^{(i)}_j))^2$. 
Given the $2 \times 2$ block structure of the Hessian as in \eqref{eq:gradHessDeepONetLoss}, denoting $\delta \vtheta := \vtheta' - \vtheta_t$ for compactness, the quadratic form on the Hessian is given by
\begin{equation}
\delta \vtheta^{\top} \mH (\tilde{\vtheta}) \delta \vtheta 
= \underbrace{\delta \vtheta_{f}^{\top} H_{ff}(\tilde{\vtheta}) \delta \vtheta_{f}}_{T_1}
        + \underbrace{2 \delta \vtheta_{f}^{\top} H_{fg}(\tilde{\vtheta}) \delta \vtheta_{g}}_{T_2}
        + \underbrace{\delta \vtheta_{g}^{\top} H_{gg}(\tilde{\vtheta}) \delta \vtheta_{g}}_{T_3}~.
\end{equation}
Focusing on $T_1$ and using the exact form of $H_{ff}(\tilde{\vtheta})$ as in~\eqref{eq:Hessian_blocks_fully_Expanded}, we have
\begin{align*}
    T_1 & = \frac{1}{n} \sum_{i=1}^n \frac{1}{q_i} \sum_{j=1}^{q_i} \ell^{\prime\prime}_{i,j} \left\langle \delta \vtheta_f , \sum_{k=1}^K g_{k,j}^{(i)}(\tilde{\vtheta}_g) \nabla_{\vtheta_f} f_k^{(i)}(\tilde{\vtheta}_f) \right\rangle^2 
    + \frac{1}{n} \sum_{i=1}^n \frac{1}{q_i} \sum_{j=1}^{q_i} \ell'_{ij} \sum_{k=1}^K g^{(i)}_{k,j}(\tilde{\vtheta}_g) \delta \vtheta_f^{\top} \nabla_{\vtheta_f}^2 f_k^{(i)}(\tilde{\vtheta}_f) \delta \vtheta_f \nonumber \\
    & \overset{(a)}{\geq} \frac{2}{n} \sum_{i=1}^n \frac{1}{q_i} \sum_{j=1}^{q_i}  \left\langle \delta \vtheta_f , \nabla_{\vtheta_f} G_{\tilde{\vtheta}}(u^{(i)})(\vy^{(i)}_j)  \right\rangle^2 - \frac{(2K\lambda_1\lambda_2+\tilde{c})\lambda_2 c^{(f)}}{\sqrt{m_f}} \| \delta \vtheta_f \|_2^2~,
%    \label{eq:hesst1}
\end{align*}
where (a) follows from $\ell''_{ij}=2$ and the different bounds in Lemma~\ref{lemm:hessgradbounds} since $\tilde{\vtheta}\in B^{\mathrm{Euc}}_{\rho,\rho_1}(\vtheta_0)$, so that $|\ell'_{ij}|\leq 2K\lambda_1\lambda_2+\tilde{c}$  with $\tilde{c}=\max_{i\in[n],j\in[q_i]}|G^\dagger(u^{(i)})(\vy^{(i)}_j)|$.
Similarly, for $T_3$ we get
\begin{align*}
    T_3 \geq \frac{2}{n} \sum_{i=1}^n \frac{1}{q_i} \sum_{j=1}^{q_i}  \left\langle \delta \vtheta_g , \nabla_{\vtheta_g} G_{\tilde{\vtheta}}(u^{(i)})(\vy^{(i)}_j)  \right\rangle^2 - \frac{(2K\lambda_1\lambda_2+\tilde{c})\lambda_1 c^{(g)}}{\sqrt{m_g}} \| \delta \vtheta_g \|_2^2~.
%    \label{eq:hesst2}
\end{align*}
Then, 
\begin{align*}
T_1+T_3 &\overset{(a)}{\geq} \frac{2}{n} \sum_{i=1}^n \frac{1}{q_i} \sum_{j=1}^{q_i}  \left(\left\langle \delta \vtheta_g ,  \nabla_{\vtheta_g} G_{\tilde{\vtheta}}(u^{(i)})(\vy^{(i)}_j) \right\rangle^2
+
\left\langle \delta \vtheta_f ,  \nabla_{\vtheta_f} G_{\tilde{\vtheta}}(u^{(i)})(\vy^{(i)}_j) \right\rangle^2
\right)\\
&\quad- 
    (2K\lambda_1\lambda_2+\tilde{c})
    \left(\frac{\lambda_1 c^{(g)}}{\sqrt{m_g}} + \frac{\lambda_2 c^{(f)}}{\sqrt{m_f}}\right) \| \delta \vtheta \|_2^2\\
%
&= \frac{2}{n} \sum_{i=1}^n \frac{1}{q_i} \sum_{j=1}^{q_i}  \left(\left\langle \delta \vtheta_g ,  \nabla_{\vtheta_g} G_{\tilde{\vtheta}}(u^{(i)})(\vy^{(i)}_j) \right\rangle
+
\left\langle \delta \vtheta_f ,  \nabla_{\vtheta_f} G_{\tilde{\vtheta}}(u^{(i)})(\vy^{(i)}_j) \right\rangle
\right)^2\\
&\quad -\frac{4}{n} \sum_{i=1}^n \frac{1}{q_i} \sum_{j=1}^{q_i}  \left\langle \delta \vtheta_g ,  \nabla_{\vtheta_g} G_{\tilde{\vtheta}}(u^{(i)})(\vy^{(i)}_j) \right\rangle
\left\langle \delta \vtheta_f ,  \nabla_{\vtheta_f} G_{\tilde{\vtheta}}(u^{(i)})(\vy^{(i)}_j) \right\rangle\\
&\quad- 
    (2K\lambda_1\lambda_2+\tilde{c})
    \left(\frac{\lambda_1 c^{(g)}}{\sqrt{m_g}} + \frac{\lambda_2 c^{(f)}}{\sqrt{m_f}}\right) \| \delta \vtheta \|_2^2~,
\end{align*}
where (a) follows from $\norm{\vtheta_f}_2,\norm{\vtheta_g}_2\leq\norm{\vtheta}_2$.

Focusing on $T_2$ and using the exact form as in~\eqref{eq:Hessian_blocks_fully_Expanded}, we have
\begin{align*}
    T_2 & = 2\delta \vtheta_f^\top \left(\frac{1}{n} \sum_{i=1}^n \frac{1}{q_i} \sum_{j=1}^{q_i} \ell'_{ij} \sum_{k=1}^K \nabla_{\vtheta_f} f_k^{(i)}(\tilde{\vtheta}_f) \nabla_{\vtheta_g} g_{k,j}^{(i)}(\tilde{\vtheta}_g)^\top \right) \delta \vtheta_g \nonumber \\
    & \qquad \qquad + 2\delta \vtheta_f^\top \left( \frac{1}{n} \sum_{i=1}^n \frac{1}{q_i} \sum_{j=1}^{q_i}\ell^{\prime\prime}_{i,j} \left(\sum_{k=1}^K g_{k,j}^{(i)} \nabla_{\vtheta_f} f_k^{(i)}(\tilde{\vtheta}_f) \right) \left( \sum_{k'=1}^K f_{k'}^{(i)} \nabla_{\vtheta_g} g_{k',j}^{(i)}(\tilde{\vtheta}_g)^\top \right) \right) \delta \vtheta_g\\
    & \overset{(a)}{=} \underbrace{2\delta \vtheta_f^\top \left(\frac{1}{n} \sum_{i=1}^n \frac{1}{q_i} \sum_{j=1}^{q_i} \ell'_{ij} \sum_{k=1}^K \nabla_{\vtheta_f} f_k^{(i)}(\tilde{\vtheta}_f) \nabla_{\vtheta_g} g_{k,j}^{(i)}(\tilde{\vtheta}_g)^\top \right) \delta \vtheta_g}_{I_1} \nonumber \\
    & \qquad \qquad + \left( \frac{4}{n} \sum_{i=1}^n \frac{1}{q_i} 
    \sum_{j=1}^{q_i} 
    \left\langle \delta \vtheta_g ,  \nabla_{\vtheta_g} G_{\vtheta_t}(u^{(i)})(\vy^{(i)}_j) \right\rangle
\left\langle \delta \vtheta_f ,  \nabla_{\vtheta_f} G_{\vtheta_t}(u^{(i)})(\vy^{(i)}_j)\right\rangle\right)~,
\end{align*}
where (a) follows from $\ell^{\prime\prime}_{i,j}=2$.

For $I_1$ our goal is to first transfer the dependence of the gradient terms on $\tilde{\vtheta}$ to $\vtheta_t$, so that we can use properties of the restricted set $Q^t_{\kappa}$ which is based on $\vtheta_t$ to simplify the analysis. Towards that end, note that 
\begin{align*}
    \frac{1}{2}I_1 & = \delta \vtheta_f^\top \left(\frac{1}{n} \sum_{i=1}^n \frac{1}{q_i} \sum_{j=1}^{q_i} \ell'_{ij} \sum_{k=1}^K \nabla_{\vtheta_f} f_k^{(i)}(\vtheta_{t,f}) \nabla_{\vtheta_g} g_{k,j}^{(i)}(\vtheta_{t,g})^\top \right) \delta \vtheta_g \\
    & \quad + \delta \vtheta_f^\top \left(\frac{1}{n} \sum_{i=1}^n \frac{1}{q_i} \sum_{j=1}^{q_i} \ell'_{ij} \sum_{k=1}^K \left( \nabla_{\vtheta_f} f_k^{(i)}(\tilde{\vtheta}_f) - \nabla_{\vtheta_f} f_k^{(i)}(\vtheta_{t,f}) \right) \nabla_{\vtheta_g} g_{k,j}^{(i)}(\tilde{\vtheta}_g)^\top \right) \delta \vtheta_g \\
    & \quad + \delta \vtheta_f^\top \left(\frac{1}{n} \sum_{i=1}^n \frac{1}{q_i} \sum_{j=1}^{q_i} \ell'_{ij} \sum_{k=1}^K \nabla_{\vtheta_f} f_k^{(i)}(\vtheta_{t,f}) \left( \nabla_{\vtheta_g} g_{k,j}^{(i)}(\tilde{\vtheta}_g) - \nabla_{\vtheta_g} g_{k,j}^{(i)}(\vtheta_{t,g}) \right)^\top \right) \delta \vtheta_g \\
    %& \quad + \delta \vtheta_f^\top \left(\frac{1}{n} \sum_{i=1}^n \frac{1}{q_i} \sum_{j=1}^{q_i} \ell'_{ij} \sum_{k=1}^K \nabla_{\vtheta_f} f_k^{(i)}(\tilde{\vtheta}_f) \nabla_{\vtheta_g} g_{k,j}^{(i)}(\tilde{\vtheta}_g)^\top \right) \delta \vtheta_g \\
    %& \overset{(a)}{\geq} \delta \vtheta_f^\top \left( \sum_{h=1}^{\tilde{q}} \sigma_{h,t} \va_{h,t} \vb_{h,t}^\top \right) \delta \vtheta_g \\
    %& \quad -  \frac{(2K\lambda_1\lambda_2+\tilde{c})}{n} \sum_{i=1}^n \frac{1}{q_i} \sum_{j=1}^{q_i}  \left\| \nabla_{\vtheta_f} f_k^{(i)}(\tilde{\vtheta}_f) - \nabla_{\vtheta_f} f_k^{(i)}(\vtheta_{t,f}) \right\|_2 \left\| \nabla_{\vtheta_g} g_{k,j}^{(i)}(\tilde{\vtheta}_g) \right\|_2 \| \delta \vtheta_f \|_2 \| \delta \vtheta_g \|_2  \\
    %& \quad - \frac{(2K\lambda_1\lambda_2+\tilde{c})}{n} \sum_{i=1}^n \frac{1}{q_i} \sum_{j=1}^{q_i} \sum_{k=1}^K \left\| \nabla_{\vtheta_f} f_k^{(i)}(\vtheta_{t,f}) \right\|_2 \left\| \nabla_{\vtheta_g} g_{k,j}^{(i)}(\tilde{\vtheta}_g) - \nabla_{\vtheta_g} g_{k,j}^{(i)}(\vtheta_{t,g})  \right\|_2 \delta \vtheta_f^\top \|_2 \| \delta \vtheta_g \|_2 \\
    %
    & \overset{(a)}{\geq} -  \frac{(2K\lambda_1\lambda_2+\tilde{c})}{n} \sum_{i=1}^n \frac{1}{q_i} \sum_{j=1}^{q_i}  \left\| \nabla_{\vtheta_f} f_k^{(i)}(\tilde{\vtheta}_f) - \nabla_{\vtheta_f} f_k^{(i)}(\vtheta_{t,f}) \right\|_2 \left\| \nabla_{\vtheta_g} g_{k,j}^{(i)}(\tilde{\vtheta}_g) \right\|_2 \| \delta \vtheta_f \|_2 \| \delta \vtheta_g \|_2  \\
    & \quad - \frac{(2K\lambda_1\lambda_2+\tilde{c})}{n} \sum_{i=1}^n \frac{1}{q_i} \sum_{j=1}^{q_i} \sum_{k=1}^K \left\| \nabla_{\vtheta_f} f_k^{(i)}(\vtheta_{t,f}) \right\|_2 \left\| \nabla_{\vtheta_g} g_{k,j}^{(i)}(\tilde{\vtheta}_g) - \nabla_{\vtheta_g} g_{k,j}^{(i)}(\vtheta_{t,g})  \right\|_2 \| \delta \vtheta_f \|_2 \| \delta \vtheta_g \|_2 \\
    %
    & \overset{(b)}{=} -  \frac{(2K\lambda_1\lambda_2+\tilde{c})}{n} \sum_{i=1}^n \frac{1}{q_i} \sum_{j=1}^{q_i}  \left\| \nabla^2_{\vtheta_f} f_k^{(i)}(\bar{\vtheta}_f)\right\|_2 \norm{\tilde{\vtheta}_f-\vtheta_{t,f}}_2 \left\| \nabla_{\vtheta_g} g_{k,j}^{(i)}(\tilde{\vtheta}_g) \right\|_2 \| \delta \vtheta_f \|_2 \| \delta \vtheta_g \|_2 \\
    & \quad - \frac{(2K\lambda_1\lambda_2+\tilde{c})}{n} \sum_{i=1}^n \frac{1}{q_i} \sum_{j=1}^{q_i} \sum_{k=1}^K \left\| \nabla_{\vtheta_f} f_k^{(i)}(\vtheta_{t,f}) \right\|_2 \left\| \nabla^2_{\vtheta_g} g_{k,j}^{(i)}(\bar{\vtheta}_g) \right\|_2\norm{\tilde{\vtheta}_g-\vtheta_{t,g}}_2 \| \delta \vtheta_f \|_2 \| \delta \vtheta_g \|_2 \\
    %
    & \overset{(c)}{\geq} - (2K\lambda_1\lambda_2+\tilde{c})\left( \frac{c^{(f)} \varrho^{(g)}}{\sqrt{m_f}} \right) \| \delta \vtheta_f \|_2^2 \| \delta \vtheta_g \|_2 
    - (2K\lambda_1\lambda_2+\tilde{c})\left( \frac{c^{(g)} \varrho^{(f)}}{\sqrt{m_g}} \right) \| \delta \vtheta_f \|_2 \| \delta \vtheta_g \|_2^2
    \\
    & \overset{(d)}{\geq} - (2K\lambda_1\lambda_2+\tilde{c})\left( \frac{c^{(g)} \varrho^{(f)}}{\sqrt{m_f}} + \frac{c^{(g)} \varrho^{(f)}}{\sqrt{m_f}} \right) \| \delta \vtheta \|_2^3\\
    & \geq - (2K\lambda_1\lambda_2+\tilde{c})\rho_2 \left( \frac{c^{(g)} \varrho^{(f)}}{\sqrt{m_f}} + \frac{c^{(g)} \varrho^{(f)}}{\sqrt{m_f}} \right) \| \delta \vtheta \|_2^2 ~,
\end{align*}
where (a) follows from the definition of $Q_{\kappa}^t$ set %from the SVD in Definition~\ref{defn:qset}, (b) follows 
(Definition~\ref{defn:qset}) 
since $\vtheta' \in B^t_\kappa\subset Q^t_\kappa$; %and so $\vtheta' \in Q^t_\kappa$ 
%and $\delta \vtheta = \vtheta' - \vtheta_t$, we have $\sum_h \sigma_{h,t} \langle \delta \vtheta_f, \va_{h,t} \rangle \langle \delta \vtheta_g , \vb_{h,t} \rangle \geq 0$ by the definition of $Q^t_{\kappa}$, 
(b) follows from the generalized mean value theorem with $\bar{\vtheta_f}=\xi_1\tilde{\vtheta}_f+(1-\xi_1)\vtheta_{t,f}$ for some $\xi_1\in[0,1]$ and $\bar{\vtheta_g}=\xi_2\tilde{\vtheta}_g+(1-\xi_2)\vtheta_{t,g}$ for some $\xi_2\in[0,1]$; (c) follows from the results in Lemma~\ref{lemm:hessgradbounds} since $[\bar{\vtheta}_f^\top\; \bar{\vtheta}_g^\top]^\top \in B^{\mathrm{Euc}}_{\rho,\rho_1}(\vtheta_0)$, and the fact that $\| \tilde{\vtheta}_f - \vtheta_{f,t} \|_2 \leq \| \delta \vtheta_f \|_2$ and $\| \tilde{\vtheta}_g - \vtheta_{g,t} \|_2 \leq \| \delta \vtheta_g \|_2$; and (d) follows from $\norm{\delta\vtheta_f}_2,\norm{\delta\vtheta_g}_2\leq \norm{\delta\vtheta}_2$.

Replacing $I_1$ back in $T_2$ and then combining the bounds on $T_1+T_3$ and $T_2$, we have
\begin{equation}
\label{eq:Hess_tildetheta}
\begin{aligned}
    \delta \vtheta^{\top} \mH (\tilde{\vtheta}) \delta \vtheta  
&\geq \frac{2}{n} \sum_{i=1}^n \frac{1}{q_i} \sum_{j=1}^{q_i}  \left(\left\langle \delta \vtheta_g ,  \nabla_{\vtheta_g} G_{\tilde{\vtheta}}(u^{(i)})(\vy^{(i)}_j) \right\rangle
+
\left\langle \delta \vtheta_f ,  \nabla_{\vtheta_f} G_{\tilde{\vtheta}}(u^{(i)})(\vy^{(i)}_j) \right\rangle
\right)^2\\
&\quad- 
    (2K\lambda_1\lambda_2+\tilde{c})
    \left(\frac{\lambda_1 c^{(g)}}{\sqrt{m_g}} + \frac{\lambda_2 c^{(f)}}{\sqrt{m_f}}\right) \| \delta \vtheta \|_2^2\\
& \quad- 2(2K\lambda_1\lambda_2+\tilde{c})\rho_2 \left( \frac{c^{(g)} \varrho^{(f)}}{\sqrt{m_f}} + \frac{c^{(g)} \varrho^{(f)}}{\sqrt{m_f}} \right) \| \delta \vtheta \|_2^2\\
%
&=\frac{2}{n} \sum_{i=1}^n \frac{1}{q_i} \sum_{j=1}^{q_i}  \left\langle \delta \vtheta ,  \nabla_{\vtheta} G_{\tilde{\vtheta}}(u^{(i)})(\vy^{(i)}_j) \right\rangle
^2\\
&\quad- 
    (2K\lambda_1\lambda_2+\tilde{c})
    \left(\frac{\lambda_1 c^{(g)}}{\sqrt{m_g}} + \frac{\lambda_2 c^{(f)}}{\sqrt{m_f}}\right) \| \delta \vtheta \|_2^2\\
& \quad- 2(2K\lambda_1\lambda_2+\tilde{c})\rho_2 \left( \frac{c^{(g)} \varrho^{(f)}}{\sqrt{m_f}} + \frac{c^{(g)} \varrho^{(f)}}{\sqrt{m_f}} \right) \| \delta \vtheta \|_2^2\\
%
&=\underbrace{\frac{2}{n} \sum_{i=1}^n \frac{1}{q_i} \sum_{j=1}^{q_i}  \left(\left\langle \delta \vtheta ,  \nabla_{\vtheta} G_{\vtheta_t}(u^{(i)})(\vy^{(i)}_j) \right\rangle+\left(\left\langle \delta \vtheta ,  \nabla_{\vtheta} G_{\tilde{\vtheta}}(u^{(i)})(\vy^{(i)}_j) \right\rangle
-
\left\langle \delta \vtheta ,  \nabla_{\vtheta} G_{\vtheta_t}(u^{(i)})(\vy^{(i)}_j) \right\rangle
\right)\right)
^2}_{I_2}\\
&\quad- 
    (2K\lambda_1\lambda_2+\tilde{c})
    \left(\frac{\lambda_1 c^{(g)}}{\sqrt{m_g}} + \frac{\lambda_2 c^{(f)}}{\sqrt{m_f}}\right) \| \delta \vtheta \|_2^2\\
& \quad- 2(2K\lambda_1\lambda_2+\tilde{c})\rho_2 \left( \frac{c^{(g)} \varrho^{(f)}}{\sqrt{m_f}} + \frac{c^{(g)} \varrho^{(f)}}{\sqrt{m_f}} \right) \| \delta \vtheta \|_2^2
%
%
~.
\end{aligned}
\end{equation}
%
%
Then, 
\begin{equation}
\label{eq:I3-1}
\begin{aligned}
    I_2 & = \frac{2}{n} \sum_{i=1}^n \frac{1}{q_i} \sum_{j=1}^{q_i}  \left\langle \delta \vtheta , \nabla_{\vtheta} G_{\vtheta_t}(u^{(i)})(\vy^{(i)}_j) \right\rangle^2 + \sum_{i=1}^n \frac{2}{q_i} \sum_{j=1}^{q_i}  \left\langle \delta \vtheta , \nabla_{\vtheta} G_{\tilde{\vtheta}}(u^{(i)})(\vy^{(i)}_j) - \nabla_{\vtheta} G_{\vtheta_t}(u^{(i)})(\vy^{(i)}_j) \right\rangle^2 \\
    & \quad + \frac{4}{n} \sum_{i=1}^n \frac{1}{q_i} \sum_{j=1}^{q_i} \left\langle \delta \vtheta_f , \nabla_{\vtheta} G_{\vtheta_t}(u^{(i)})(\vy^{(i)}_j) \right\rangle \left\langle \delta \vtheta , \nabla_{\vtheta} G_{\tilde{\vtheta}}(u^{(i)})(\vy^{(i)}_j) - \nabla_{\vtheta} G_{\vtheta_t}(u^{(i)})(\vy^{(i)}_j) \right\rangle \\
    & \overset{(a)}{=}\frac{2}{n} \sum_{i=1}^n \frac{1}{q_i} \sum_{j=1}^{q_i}  \left\langle \delta \vtheta , \nabla_{\vtheta} G_{\vtheta_t}(u^{(i)})(\vy^{(i)}_j) \right\rangle^2 + \sum_{i=1}^n \frac{2}{q_i} \sum_{j=1}^{q_i}  \left\langle \delta \vtheta , \nabla_{\vtheta} G_{\tilde{\vtheta}}(u^{(i)})(\vy^{(i)}_j) - \nabla_{\vtheta} G_{\vtheta_t}(u^{(i)})(\vy^{(i)}_j) \right\rangle^2 \\
    & \quad + \frac{4}{n} \sum_{i=1}^n \frac{1}{q_i} \sum_{j=1}^{q_i} \left\langle \delta \vtheta , \nabla_{\vtheta} G_{\vtheta_t}(u^{(i)})(\vy^{(i)}_j) \right\rangle \left((\delta \vtheta)^\top \nabla_{\vtheta}^2 G_{\tilde{\tilde{\vtheta}}}(u^{(i)})(\vy^{(i)}_j)(\tilde{\vtheta}-\vtheta_t)\right)
    \\
    & \geq\frac{2}{n} \sum_{i=1}^n \frac{1}{q_i} \sum_{j=1}^{q_i}  \left\langle \delta \vtheta , \nabla_{\vtheta} G_{\vtheta_t}(u^{(i)})(\vy^{(i)}_j) \right\rangle^2  \\
    & \quad + \frac{4}{n} \sum_{i=1}^n \frac{1}{q_i} \sum_{j=1}^{q_i} \left\langle \delta \vtheta , \nabla_{\vtheta} G_{\vtheta_t}(u^{(i)})(\vy^{(i)}_j) \right\rangle \left((\delta \vtheta)^\top \nabla_{\vtheta}^2 G_{\tilde{\tilde{\vtheta}}}(u^{(i)})(\vy^{(i)}_j)(\tilde{\vtheta}-\vtheta_t)\right)
    \\
%     
    & \overset{(b)}{\geq}\frac{2}{n} \sum_{i=1}^n \frac{1}{q_i} \sum_{j=1}^{q_i}  \left\langle \delta \vtheta , \nabla_{\vtheta} G_{\vtheta_t}(u^{(i)})(\vy^{(i)}_j) \right\rangle^2  \\
    & \quad - 
4\xi_3K(\lambda_2\varrho^{(f)}+\lambda_1\varrho^{(g)})\left\|\delta\vtheta\right\|_2
    \frac{1}{n} \sum_{i=1}^n \frac{1}{q_i} \sum_{j=1}^{q_i}
    \left|(\delta \vtheta)^\top \nabla_{\vtheta}^2 G_{\tilde{\tilde{\vtheta}}}(u^{(i)})(\vy^{(i)}_j)(\delta\vtheta)\right|
%%    
\end{aligned}
\end{equation}
where (a) follows from the generalized mean value theorem and has $\tilde{\tilde{\vtheta}}\in\xi_3 \tilde{\vtheta}+(1-\xi_3)\vtheta_t$ for some $\xi_3\in[0,1]$; and (b) follows from the fact that $\tilde{\vtheta}-\vtheta_t=\xi_3(\theta'-\vtheta')$ and 
\begin{align*}
\left\| \nabla_{\vtheta} G_{\vtheta_t}(u^{(i)})(\vy^{(i)}_j) \right\|_2 &\leq 
\left\| \nabla_{\vtheta_f} G_{\vtheta_t}(u^{(i)})(\vy^{(i)}_j) \right\|_2 +
\left\| \nabla_{\vtheta_g} G_{\vtheta_t}(u^{(i)})(\vy^{(i)}_j) \right\|_2\\
&\leq 
\left\| \sum_{k=1}^K g^{(i)}_{k,j}(\vtheta_t)\nabla_{\vtheta_f}f^{(i)}_k(\vtheta_t)\right\|_2
+
\left\| \sum_{k=1}^K \nabla_{\vtheta_g}g^{(i)}_{k,j}(\vtheta_t)(f^{(i)}_k(\vtheta_t))\right\|_2\\
&\leq 
\sum_{k=1}^K|g^{(i)}_{k,j}(\vtheta_t)|\norm{\nabla_{\vtheta_f}f^{(i)}_k(\vtheta_t)}_2 + 
\sum_{k=1}^K|f^{(i)}_{k}(\vtheta_t)|\norm{\nabla_{\vtheta_g}g^{(i)}_{k,j}(\vtheta_t)}_2\\
&\leq K\lambda_2\varrho^{(f)}+K\lambda_1\varrho^{(g)}~, 
\end{align*}
where the last inequality follows from Lemma~\ref{lemm:hessgradbounds}.

Now, we have that
\begin{equation}
\label{eq:dd_G}
\begin{aligned}
    (\delta \vtheta)^\top \nabla_{\vtheta}^2 G_{\tilde{\tilde{\vtheta}}}(u^{(i)})(\vy^{(i)}_j)(\delta\vtheta) &= 
\sum^K_{k=1}(\delta\vtheta_f)^\top(g^{(i)}_{k,j}(\tilde{\tilde{\vtheta}}_g)\nabla^2_{\vtheta_f}f^{(i)}_{k}(\tilde{\tilde{\vtheta}}_f)(\delta\vtheta_f)
+ 
\sum^K_{k=1}(\delta\vtheta_g)^\top(f^{(i)}_{k}(\tilde{\tilde{\vtheta}}_f)\nabla^2_{\vtheta_g}g^{(i)}_{k,j}(\tilde{\tilde{\vtheta}}_g)(\delta\vtheta_g)\\
&\quad 
+2\sum^K_{k=1}(\delta\vtheta_g)^\top(\nabla_{\vtheta_g}g^{(i)}_{k,j}(\tilde{\tilde{\vtheta}}_g)(\nabla_{\vtheta_f}f^{(i)}_{k}(\tilde{\tilde{\vtheta}}_f))^\top(\delta\vtheta_f)\\
&\overset{(a)}{\leq} 
K\left(\frac{\lambda_2c^{(f)}}{\sqrt{m_f}}+ \frac{\lambda_1c^{(g)}}{\sqrt{m_g}}\right)\norm{\delta\vtheta}_2^2+\underbrace{2\sum^K_{k=1}(\delta\vtheta_g)^\top(\nabla_{\vtheta_g}g^{(i)}_{k,j}(\tilde{\tilde{\vtheta}}_g)(\nabla_{\vtheta_f}f^{(i)}_{k}(\tilde{\tilde{\vtheta}}_f))^\top(\delta\vtheta_f)}_{I_3}
\end{aligned}
\end{equation}
where (a) follows from Lemma~\ref{lemm:hessgradbounds} since $\tilde{\tilde{\vtheta}}\in B^{\mathrm{Euc}}_{\rho,\rho_1}(\vtheta_0)$.

Now, for $I_3$,
\begin{align*}
    \frac{1}{2}I_3 & = \delta \vtheta_f^\top \left( \sum_{k=1}^K \nabla_{\vtheta_f} f_k^{(i)}(\vtheta_{t,f}) \nabla_{\vtheta_g} g_{k,j}^{(i)}(\vtheta_{t,g})^\top \right) \delta \vtheta_g \\
    & \quad + \delta \vtheta_f^\top \left(\sum_{k=1}^K \left( \nabla_{\vtheta_f} f_k^{(i)}(\tilde{\tilde{\vtheta}}_f) - \nabla_{\vtheta_f} f_k^{(i)}(\vtheta_{t,f}) \right) \nabla_{\vtheta_g} g_{k,j}^{(i)}(\tilde{\tilde{\vtheta}}_g)^\top \right) \delta \vtheta_g \\
    & \quad + \delta \vtheta_f^\top \left(\sum_{k=1}^K \nabla_{\vtheta_f} f_k^{(i)}(\vtheta_{t,f}) \left( \nabla_{\vtheta_g} g_{k,j}^{(i)}(\tilde{\tilde{\vtheta}}_g) - \nabla_{\vtheta_g} g_{k,j}^{(i)}(\vtheta_{t,g}) \right)^\top \right) \delta \vtheta_g \\
    %& \quad + \delta \vtheta_f^\top \left(\frac{1}{n} \sum_{i=1}^n \frac{1}{q_i} \sum_{j=1}^{q_i} \ell'_{ij} \sum_{k=1}^K \nabla_{\vtheta_f} f_k^{(i)}(\tilde{\vtheta}_f) \nabla_{\vtheta_g} g_{k,j}^{(i)}(\tilde{\vtheta}_g)^\top \right) \delta \vtheta_g \\
    %& \overset{(a)}{\geq} \delta \vtheta_f^\top \left( \sum_{h=1}^{\tilde{q}} \sigma_{h,t} \va_{h,t} \vb_{h,t}^\top \right) \delta \vtheta_g \\
    %& \quad -  \frac{(2K\lambda_1\lambda_2+\tilde{c})}{n} \sum_{i=1}^n \frac{1}{q_i} \sum_{j=1}^{q_i}  \left\| \nabla_{\vtheta_f} f_k^{(i)}(\tilde{\vtheta}_f) - \nabla_{\vtheta_f} f_k^{(i)}(\vtheta_{t,f}) \right\|_2 \left\| \nabla_{\vtheta_g} g_{k,j}^{(i)}(\tilde{\vtheta}_g) \right\|_2 \| \delta \vtheta_f \|_2 \| \delta \vtheta_g \|_2  \\
    %& \quad - \frac{(2K\lambda_1\lambda_2+\tilde{c})}{n} \sum_{i=1}^n \frac{1}{q_i} \sum_{j=1}^{q_i} \sum_{k=1}^K \left\| \nabla_{\vtheta_f} f_k^{(i)}(\vtheta_{t,f}) \right\|_2 \left\| \nabla_{\vtheta_g} g_{k,j}^{(i)}(\tilde{\vtheta}_g) - \nabla_{\vtheta_g} g_{k,j}^{(i)}(\vtheta_{t,g})  \right\|_2 \delta \vtheta_f^\top \|_2 \| \delta \vtheta_g \|_2 \\
    & \leq
    \delta \vtheta_f^\top \left( \sum_{k=1}^K \nabla_{\vtheta_f} f_k^{(i)}(\vtheta_{t,f}) \nabla_{\vtheta_g} g_{k,j}^{(i)}(\vtheta_{t,g})^\top \right) \delta \vtheta_g 
    \\
    &\quad+\sum^K_{k=1}  \left\| \nabla_{\vtheta_f} f_k^{(i)}(\tilde{\tilde{\vtheta}}_f) - \nabla_{\vtheta_f} f_k^{(i)}(\vtheta_{t,f}) \right\|_2 \left\| \nabla_{\vtheta_g} g_{k,j}^{(i)}(\tilde{\tilde{\vtheta}}_g) \right\|_2 \| \delta \vtheta_f \|_2 \| \delta \vtheta_g \|_2  \\
    & \quad +  \sum_{k=1}^K \left\| \nabla_{\vtheta_f} f_k^{(i)}(\vtheta_{t,f}) \right\|_2 \left\| \nabla_{\vtheta_g} g_{k,j}^{(i)}(\tilde{\tilde{\vtheta}}_g) - \nabla_{\vtheta_g} g_{k,j}^{(i)}(\vtheta_{t,g})  \right\|_2 \| \delta \vtheta_f \|_2 \| \delta \vtheta_g \|_2 \\
    %
    & \overset{(a)}{\leq} 
    \delta \vtheta_f^\top \left( \sum_{k=1}^K \nabla_{\vtheta_f} f_k^{(i)}(\vtheta_{t,f}) \nabla_{\vtheta_g} g_{k,j}^{(i)}(\vtheta_{t,g})^\top \right) \delta \vtheta_g 
    \\
    &\quad+\xi_3\xi K\left( \frac{c^{(f)} \varrho^{(g)}}{\sqrt{m_f}} \right) \| \delta \vtheta_f \|_2^2 \| \delta \vtheta_g \|_2 
    +\xi_3\xi K\left( \frac{c^{(g)} \varrho^{(f)}}{\sqrt{m_g}} \right) \| \delta \vtheta_f \|_2 \| \delta \vtheta_g \|_2^2
    \\
    & \overset{(b)}{\leq} \delta \vtheta_f^\top \left( \sum_{k=1}^K \nabla_{\vtheta_f} f_k^{(i)}(\vtheta_{t,f}) \nabla_{\vtheta_g} g_{k,j}^{(i)}(\vtheta_{t,g})^\top \right) \delta \vtheta_g\\
    &\quad+ K\left( \frac{c^{(g)} \varrho^{(f)}}{\sqrt{m_f}} + \frac{c^{(f)} \varrho^{(g)}}{\sqrt{m_g}} \right) \| \delta \vtheta \|_2^3\\
    & \leq \delta \vtheta_f^\top \left( \sum_{k=1}^K \nabla_{\vtheta_f} f_k^{(i)}(\vtheta_{t,f}) \nabla_{\vtheta_g} g_{k,j}^{(i)}(\vtheta_{t,g})^\top \right) \delta \vtheta_g\\
    &\quad+ K\rho_2\left( \frac{c^{(g)} \varrho^{(f)}}{\sqrt{m_g}} + \frac{c^{(f)} \varrho^{(g)}}{\sqrt{m_f}} \right) \| \delta \vtheta \|_2^2~,
\end{align*}
where (a) follows from the generalized mean value theorem, from $\norm{\tilde{\tilde{\vtheta}}_f-\vtheta_{t,f}}_2= \norm{\xi_3 \tilde{\vtheta}_f+(1-\xi_3)\vtheta_{t,f}-\vtheta_{t,f}}_2=\xi_3\xi\norm{\vtheta'-\vtheta_{t,f}}_2=\xi_3\xi\norm{\delta\vtheta_f}_2$, and from the results in Lemma~\ref{lemm:hessgradbounds} since $\tilde{\tilde{\vtheta}}\in B^{\mathrm{Euc}}_{\rho,\rho_1}(\vtheta_0)$; and (b) follows from $\norm{\vtheta_f}_2,\norm{\vtheta_g}_2\leq \norm{\vtheta}_2$ and $\xi_3\xi\leq 1$.

Replacing the bound on $I_3$ back to~\eqref{eq:dd_G}, we obtain 
\begin{equation}
\label{eq:dd_G1}
\begin{aligned}
    (\delta \vtheta)^\top \nabla_{\vtheta}^2 G_{\tilde{\tilde{\vtheta}}}(u^{(i)})(\vy^{(i)}_j)(\delta\vtheta) 
& \leq \delta \vtheta_f^\top \left( \sum_{k=1}^K \nabla_{\vtheta_f} f_k^{(i)}(\vtheta_{t,f}) \nabla_{\vtheta_g} g_{k,j}^{(i)}(\vtheta_{t,g})^\top \right) \delta \vtheta_g\\
&\quad+ K(1+\rho_2)\left( \frac{c^{(g)} (\lambda_1+2\varrho^{(f)})}{\sqrt{m_g}} + \frac{c^{(f)} (\lambda_2+2\varrho^{(g)})}{\sqrt{m_f}} \right) \| \delta \vtheta \|_2^2\\
&\leq K(1+\rho_2)\left( \frac{c^{(g)} (\lambda_1+2\varrho^{(f)})}{\sqrt{m_g}} + \frac{c^{(f)} (\lambda_2+2\varrho^{(g)})}{\sqrt{m_f}} \right) \| \delta \vtheta \|_2^2~,
%
%
\end{aligned}
\end{equation}
where the last inequality follows from the fact that  $\vtheta^{\prime}\in Q^t_{\kappa}$, using the properties of the restricted set $Q^t_{\kappa}$ in Definition~\ref{defn:qset}.

Replacing~\eqref{eq:dd_G1} back to $I_2$ in~\eqref{eq:I3-1}, we obtain
\begin{equation}
\label{eq:I2-21}
\begin{aligned}
    I_2
%     
    & \geq\frac{2}{n} \sum_{i=1}^n \frac{1}{q_i} \sum_{j=1}^{q_i}  \left\langle \delta \vtheta , \nabla_{\vtheta} G_{\vtheta_t}(u^{(i)})(\vy^{(i)}_j) \right\rangle^2  \\
    & \quad - 
4\xi_3K(\lambda_2\varrho^{(f)}+\lambda_1\varrho^{(g)})\left\|\delta\vtheta\right\|_2
    \times K(1+\rho_2)\left( \frac{c^{(g)} (\lambda_1+2\varrho^{(f)})}{\sqrt{m_g}} + \frac{c^{(f)} (\lambda_2+2\varrho^{(g)})}{\sqrt{m_f}} \right) \| \delta \vtheta \|_2^2\\
%
& =\frac{2}{n} \sum_{i=1}^n \frac{1}{q_i} \sum_{j=1}^{q_i}  \left\langle \delta \vtheta , \nabla_{\vtheta} G_{\vtheta_t}(u^{(i)})(\vy^{(i)}_j) \right\rangle^2  \\
    & \quad - 
4\xi_3K^2(1+\rho_2)\rho_2(\lambda_2\varrho^{(f)}+\lambda_1\varrho^{(g)})
    \left( \frac{c^{(g)} (\lambda_1+2\varrho^{(f)})}{\sqrt{m_g}} + \frac{c^{(f)} (\lambda_2+2\varrho^{(g)})}{\sqrt{m_f}} \right) \| \delta \vtheta \|_2^2~.    
%%    
\end{aligned}
\end{equation}
Replacing this lower bound~\eqref{eq:I2-21} back to the Hessian expression in~\eqref{eq:Hess_tildetheta_1},
\begin{equation}
\label{eq:Hess_tildetheta_1}
\begin{aligned}
    \delta \vtheta^{\top} \mH (\tilde{\vtheta}) \delta \vtheta
&\geq\frac{2}{n} \sum_{i=1}^n \frac{1}{q_i} \sum_{j=1}^{q_i}  \left\langle \delta \vtheta , \nabla_{\vtheta} G_{\vtheta_t}(u^{(i)})(\vy^{(i)}_j) \right\rangle^2  \\
    & \quad - 
4\xi_3K^2(1+\rho_2)\rho_2(\lambda_2\varrho^{(f)}+\lambda_1\varrho^{(g)})
    \left( \frac{c^{(g)} (\lambda_1+2\varrho^{(f)})}{\sqrt{m_g}} + \frac{c^{(f)} (\lambda_2+2\varrho^{(g)})}{\sqrt{m_f}} \right) \| \delta \vtheta \|_2^2\\
&\quad- 
    (2K\lambda_1\lambda_2+\tilde{c})
    \left(\frac{\lambda_1 c^{(g)}}{\sqrt{m_g}} + \frac{\lambda_2 c^{(f)}}{\sqrt{m_f}}\right) \| \delta \vtheta \|_2^2\\
& \quad- 2(2K\lambda_1\lambda_2+\tilde{c})\rho_2 \left( \frac{c^{(g)} \varrho^{(f)}}{\sqrt{m_f}} + \frac{c^{(g)} \varrho^{(f)}}{\sqrt{m_f}} \right) \| \delta \vtheta \|_2^2\\
%
&\overset{(a)}{\geq} 2 \left\langle \delta \vtheta , \nabla_{\vtheta} \bar{G}_{\vtheta_t} \right\rangle^2  - c_1 K^2
\left( \frac{1}{\sqrt{m_f}} + \frac{1}{\sqrt{m_g}} \right) \| \delta \vtheta \|_2^2\\
&\overset{(b)}{\geq} 2\kappa^2 \norm{\nabla_{\vtheta} \bar{G}_{\vtheta_t}}^2_2\norm{\delta\vtheta}_2^2  - c_1 K^2
\left( \frac{1}{\sqrt{m_f}} + \frac{1}{\sqrt{m_g}} \right) \| \delta \vtheta \|_2^2\\
%
&=\alpha_t\norm{\delta\vtheta}_2^2~,
%
\end{aligned}
\end{equation}
where (a) follows from Jensen's inequality with $\bar{G}_{\vtheta} = \frac{1}{n} \sum_{i=1}^n \frac{1}{q_i} \sum_{j=1}^{q_i} G_{\vtheta}(u^{(i)})(\vy^{(i)}_j)$; where (b) follows from the fact that  $\vtheta^{\prime}\in Q^t_{\kappa}$ and using the properties of the restricted set $Q^t_{\kappa}$ in Definition~\ref{defn:qset}; and where $\alpha_t = 2\kappa^2 \| \nabla_{\vtheta} \bar{G}_{\vtheta} \|_2^2 - c_1K^2\left(\frac{1}{m_f}+\frac{1}{m_g}\right)$. 
%\pcedit{Note that the defined constant $c_1$ inherits its dependency from Lemma~\ref{lemm:hessgradbounds} as stated in Theorem~\ref{theo:rsc_main_DON}'s statement.}
%
\pcedit{Note that adding all the constants from the second to the fourth line in~\eqref{eq:Hess_tildetheta_1} define the constant $c_1$, and so $c_1$ depends on $\sigma_1$, the depth $L$, and the radii $\rho$, $\rho_1$, and $\rho_2$ due to Lemma~\ref{lemm:hessgradbounds}. As in the statement of Lemma~\ref{lemm:hessgradbounds}, this dependence reduces to the depth and the radii and becomes polynomial whenever $\sigma_0\leq 1-\rho\max\{\frac{1}{\sqrt{m_f}},\frac{1}{\sqrt{m_g}}\}$.}
This completes the proof. \qed
% \end{proof}




% \begin{theo}[Smoothness of Loss]
% Under the assumptions \ThreeAsmpsref{asmp:Activation_Function}{asmp:Loss}{asmp:smoothinit}, with high probability, 
% for $\vtheta \in B_{\rho}^{\mathrm{Euc}}(\vtheta_0)$, $\gL(\vtheta)$ is $\beta$-smooth with  $\beta = b \varrho^2 + \frac{c\sqrt{\lambda}}{\sqrt{m}}$ with $c= \max(c^{(f)}, c^{(g)}), \varrho = \max(\varrho^{(f)}, \varrho^{(g)})$ with $c^{(f)}, c^{(g)}, \varrho^{(f)}, \varrho^{(g)}$ as in Lemma~\ref{lemm:hessgradbounds}.
% \label{theo:smooth}
% \end{theo}
\RSS*
\begin{proof}
By the second order Taylor expansion of $\gL(\vtheta^\prime)$ about the point $\bar{\vtheta}$ with $\vtheta^\prime,\bar{\vtheta}\in B^{\mathrm{Euc}}_{\rho,\rho_1}(\vtheta_0)$, we have
$\cL(\vtheta') = \cL(\bar{\vtheta}) + \langle \vtheta' - \bar{\vtheta}, \nabla_\vtheta\cL(\bar{\vtheta}) \rangle + \frac{1}{2} (\vtheta'-\bar{\vtheta})^\top \frac{\partial^2 \cL(\tilde{\vtheta})}{\partial \vtheta^2} (\vtheta'-\bar{\vtheta})$, 
where $\tilde{\vtheta} = \xi \vtheta' + (1-\xi) \bar{\vtheta}$ for some $\xi \in [0,1]$. Then, 
\begin{align*}
    (\vtheta'-\bar{\vtheta})^\top \frac{\partial^2 \cL(\tilde{\vtheta})}{\partial \vtheta^2} (\vtheta'-\bar{\vtheta}) 
    & = (\vtheta'-\bar{\vtheta})^\top \bigg( \frac{1}{n} \sum_{i=1}^n \frac{1}{q_i} \sum_{j=1}^{q_i}  \ell^{\prime\prime}_{i,j} \nabla_{\vtheta} G_{\tilde{\vtheta}}(u^{(i)})(\vy^{(i)}_j) \nabla_{\vtheta} G_{\tilde{\vtheta}}(u^{(i)})(\vy^{(i)}_j)^\top  \\
    & \qquad \qquad \qquad \qquad + \ell^{\prime}_{i,j}   \nabla^2_{\vtheta} G_{\tilde{\vtheta}}(u^{(i)})(\vy^{(i)}_j)  \bigg)  (\vtheta'-\bar{\vtheta}) \\
    & = \underbrace{\frac{1}{n} \sum_{i=1}^n  \frac{1}{q_i} \sum_{j=1}^{q_i} \ell^{\prime\prime}_{i,j} \left\langle \vtheta'-\bar{\vtheta}, \nabla_{\vtheta} G_{\tilde{\vtheta}}(u^{(i)})(\vy^{(i)}_j) \right\rangle^2}_{I_1} \\
    & \qquad \qquad + \underbrace{\frac{1}{n} \sum_{i=1}^n \frac{1}{q_i} \sum_{j=1}^{q_i} \ell^{\prime}_{i,j}  (\vtheta'-\bar{\vtheta})^\top \nabla^2_{\vtheta} G_{\tilde{\vtheta}}(u^{(i)})(\vy^{(i)}_j)  (\vtheta'-\bar{\vtheta}) }_{I_2}~,
\end{align*}
where $\ell_{i,j}=(G_{\tilde{\vtheta}}(u^{(i)})(\vy_{j}^{(i)})-G^\dagger(u^{(i)})(\vy^{(i)}_j))^2$. 

Now, note that
\begin{align*}
I_1 & = \frac{1}{n} \sum_{i=1}^n  \frac{1}{q_i} \sum_{j=1}^{q_i} \ell^{\prime\prime}_{i,j} \left\langle \vtheta'-\bar{\vtheta}, \nabla_{\vtheta} G_{\tilde{\vtheta}}(u^{(i)})(\vy^{(i)}_j) \right\rangle^2 \\
& \overset{(a)}{\leq} \frac{2}{n} \sum_{i=1}^n \frac{1}{q_i}\sum_{j=1}^{q_i} \left\| \nabla_{\vtheta} G_{\tilde{\vtheta}}(u^{(i)})(\vy^{(i)}_j) \right\|_2^2 \|\vtheta' - \bar{\vtheta} \|_2^2 \\
& \overset{(b)}{\leq} 4K^2(\lambda_2\varrho^{(f)}+\lambda_1\varrho^{(g)})^2 \| \vtheta' - \bar{\vtheta} \|_2^2~,
\end{align*}
where (a) follows by the Cauchy-Schwartz inequality and (b) from Lemma~\ref{lemm:hessgradbounds} as follows
\begin{equation*}
\norm{\nabla_{\vtheta}G_{\tilde{\vtheta}}(u^{(i)})(\vy^{(i)}_j)}_2\leq\sum^K_{k=1}(
\norm{g_{k,j}^{(i)}(\tilde{\vtheta}_g)\nabla_{\vtheta_f}f_k^{(i)}(\tilde{\vtheta}_f)}_2
+
\norm{
f_{k}^{(i)}(\tilde{\vtheta}_f)\nabla_{\vtheta_g}g_{k,j}^{(i)}(\tilde{\vtheta}_g)
}_2
)
\leq
%\leq \sqrt{2}\sum^K_{k=1}\norm{\nabla_{\tilde{\vtheta}_f}f_k^{(i)}}_2|g_{k,j}^{(i)}|+\norm{\nabla_{\tilde{\vtheta}_g}g_{k,j}^{(i)}}_2|f_{k}^{(i)}|\leq 
K(\lambda_2\varrho^{(f)}+\lambda_1\varrho^{(g)}),
\end{equation*}
since $\tilde{\vtheta}\in B^{\mathrm{Euc}}_{\rho,\rho_1}(\vtheta_0)$.

Now, for $I_2$, 
\begin{align*}
I_2 & \leq \frac{1}{n} \sum_{i=1}^n \frac{1}{q_i} \sum_{j=1}^{q_i} |\ell^{\prime}_{i,j}|  |(\vtheta'-\bar{\vtheta})^\top \nabla^2 G_{\tilde{\vtheta}}(u^{(i)})(\vy^{(i)}_j) (\vtheta'-\bar{\vtheta})| \\
%& \leq \left| \sum_{i=1}^n \left( \frac{1}{\sqrt{n}} \ell'_{i} \right) \left( \frac{1}{\sqrt{n}}Q_{i} \right) \right| \\
%& \overset{(a)}{\leq} \lambda \left( \frac{1}{n} \sum_{i=1}^n \frac{1}{q_i} Q^2_{t,(i,j)} \right)^{1/2}  \\
& \overset{(a)}{\leq}
(2K\lambda_1\lambda_2+\tilde{c})\left(K\varrho^{(f)}\varrho^{(g)}+K(1+\rho_2) 
\left( \frac{c^{(g)} (\lambda_1+\varrho^{(f)})}{\sqrt{m_g}} + \frac{c^{(f)} (\lambda_2+\varrho^{(g)})}{\sqrt{m_f}} \right)\right)
\| \vtheta' - \bar{\vtheta} \|_2^2~,
%& \leq  \frac{\lambda}{n} \sum_{i=1}^n   \| \vtheta'-\bar{\vtheta}\|_2^2 \left\| \frac{\partial^2 f(\tilde{\vtheta};\x_i)}{\partial \vtheta^2} \right\|_2 \\
%& \leq \frac{\lambda c_H}{\sqrt{m}} \| \vtheta' - \bar{\vtheta} \|_2^2~. 
\end{align*}
%where (a) follows by Cauchy-Schwartz.
with  $\tilde{c}=\max_{i\in[n],j\in[q_i]}|G^\dagger(u^{(i)})(\vy^{(i)}_j)|$, and where (a) follows from modifying the result in equation~\eqref{eq:dd_G1} from Theorem~\ref{theo:rsc_main_DON} according to our setting.
%

Putting the upper bounds on $I_1$ and $I_2$ back, we have
\begin{align*}
(\vtheta'-\bar{\vtheta})^\top \frac{\partial^2 \cL(\tilde{\vtheta})}{\partial \vtheta^2} (\vtheta'-\bar{\vtheta})
& \leq \left[ 
4K^2(\lambda_2\varrho^{(f)}+\lambda_1\varrho^{(g)})^2 \right.\\
&\left.\quad 
+
(2K\lambda_1\lambda_2+\tilde{c})\left(K\varrho^{(f)}\varrho^{(g)}+K(1+\rho_2) 
\left( \frac{c^{(g)} (\lambda_1+\varrho^{(f)})}{\sqrt{m_g}} + \frac{c^{(f)} (\lambda_2+\varrho^{(g)})}{\sqrt{m_f}} \right)\right)
\right]\\
&\quad \times \| \vtheta' - \bar{\vtheta} \|_2^2~.
\end{align*}
\pcedit{Note that all the constants on the right-hand side of the inequality above form an expression that depends on $K$ and on $\sigma_1$, the depth $L$, and the radii $\rho$, $\rho_1$, and $\rho_2$ due to Lemma~\ref{lemm:hessgradbounds}. As in the statement of Lemma~\ref{lemm:hessgradbounds}, the dependence of such expression reduces to the depth and the radii and becomes polynomial whenever $\sigma_0\leq 1-\rho\max\{\frac{1}{\sqrt{m_f}},\frac{1}{\sqrt{m_g}}\}$.} 
This completes the proof.
\label{theo:smooth}
\end{proof}

\begin{prop}[{\bf RSC to smoothness ratio}]
\label{prop:RSC-smooth-DON}
Under the same conditions as in Theorems~\ref{theo:rsc_main_DON} and~\ref{theo:smooth_main}, we have that $\alpha_t/\beta<1$ with probability at least $1-2LK(\frac{1}{m_f}+\frac{1}{m_g})$.
\end{prop}
\begin{proof}
From the proofs of both Theorems~\ref{theo:rsc_main_DON} and~\ref{theo:smooth_main},
$\alpha_t<2\kappa^2\norm{\nabla_{\vtheta_t}\bar{G}_t}_2^2\leq 2\kappa^2 K^2(\lambda_2\varrho^{(f)}+\lambda_1\varrho^{(g)})^2\leq
4 K^2(\lambda_2\varrho^{(f)}+\lambda_1\varrho^{(g)})^2<
\beta$, and so $\frac{\alpha_t}{\beta}<1$.
\end{proof}



\section{Analysis for Fourier Neural Operators}
\label{app:fnoopt}
We recall the FNO model
\begin{align}
    \begin{aligned}
\aalpha^{(0)} &= P(u)(\vx)\\
\aalpha^{(1)} &= \phi\left(
        \frac{1}{\sqrt{m}} W^{(1)} \aalpha^{(0)}
    \right)\\
\aalpha^{(l)} & = \phi\left(
        \frac{1}{\sqrt{m}} W^{(l)} \aalpha^{(l-1)} +
        \frac{1}{\sqrt{m}} F^{*} R^{(l)} F \aalpha^{(l-1)}
    \right),\quad l\in \{2,\dots,L+1\}\\
    f(\vtheta;\x)  = \aalpha^{(L+2)} &:= \frac{1}{\sqrt{m}} \v^\top \aalpha^{(L+1)}~,
    \end{aligned}
\label{eq:FNO_predictor_app}
\end{align}
where $W^{(l)}, R^{(l)} \in \R^{m \times m}$ for $l \in \{2,\ldots,L+1\}$, $W^{(1)} \in \R^{m \times d}$. 
%

\subsection{Bounds on the Hessian, Gradients and the Predictor}

\begin{restatable}[{\bf Bounds on the Predictor}]{lemm}{HessDiag2}
\label{lemm:hessgradbounds-FNO}
Under Assumptions~\ref{asmp:Activation_Function_FNO} and \ref{asmp:smoothinit_FNO} and for $\vtheta \in B^{\mathrm{Euc}}_{\rho_w,\rho_r\rho_1}(\vtheta_0)$ we have with probability at least {$1-\frac{2(L+2)}{m}$}, that for any input function $u$ and evaluation point $\vx$ as in Section~\ref{sec:optFNO},
\begin{align}
    %\max _{i \in[n]}%\max_{j\in[q_i]}
        \left\|\nabla^2_{\vtheta} f\right\| \leq \frac{c}{\sqrt{m}}, \label{eq:hessianBoundG_fg_FNO} \\
    \left\| \nabla_{\vtheta} f\right\|_2 \leq \varrho~,\label{eq:gradientBoundG_fg_FNO}\\
    %
     |f| \leq \lambda~,\label{eq:predictorBoundG_fg_FNO}
\end{align}
where $c,\;\varrho,\;\lambda$ are suitable constants that depend on $\sigma_{1,w}$, $\sigma_{1,r}$, the depth $L$, and the radii $\rho_w$, $\rho_r$, and $\rho_1$. 
\pcedit{The dependence of the constants reduces to depth and the radii and becomes polynomial whenever $\sigma_{1,w}+\sigma_{1,r}\leq 1-\frac{\rho_w+\rho_r}{\sqrt{m}}$.}
\end{restatable}

In this section we will prove all the bounds in Lemma~\ref{lemm:hessgradbounds-FNO}.
%

\begin{lemm}[{\bf Initialization of the Parameters}]
    \label{lemm:InitParamFNO}
    Under Assumption~\ref{asmp:smoothinit_FNO},with probability at least $1-\frac{2}{m}$ we have
\begin{equation}
    \|W^{(l)}_0\|_2 \leq \sigma_{1,w}\sqrt{m}, \quad \text{and} \quad \|R^{(l)}_0\|_2 \leq \sigma_{1,r}\sqrt{m}.
    \label{eq:Wl_zero_Rl_zero}
\end{equation}
\end{lemm}
\proof The proof follows directly from Lemma A.1 in~\citep{banerjee2022restricted}. 
%We reproduce it here for the sake of completeness.  For ($m_l \times m_{l-1}$) random matrices $W^{(l)}_0$ and $R^{(l)}_0$ with i.i.d entries $w^{(l)}_{0,ij} \in \gN(0, \sigma_{0,w}^2)$ and $r^{(l)}_{0,ij} \in \gN(0, \sigma_{0,r}^2)$, the largest singular values are bounded from above with probabilities $(1-2\exp (-t^2/2\sigma_{0,w}^2))$ and $(1-2\exp (-t^2/2\sigma_{0,r}^2))$ respectively, namely,
%\begin{equation}
%    \sigma_{\max}(W^{(l)}_0) \leq \sigma_{0,w}(\sqrt{m_l} + \sqrt{m_{l-1}}) + t,\quad \text{and}\quad
%    \sigma_{\max}(R^{(l)}_0) \leq \sigma_{0,r}(\sqrt{m_l} + \sqrt{m_{l-1}}) + t.
%    \label{eq:signularValueW_l_R_l_zero}
%\end{equation}
%In order to derive the above concentration result note that $\Wlzero = \sigma_{0,w}\bar{W}^{(l)}_0$ and $\Rlzero = \sigma_{0,r}\bar{R}^{(l)}_0$, where the entries $\bar{w}^{(l)}_{0,ij} \in \gN (0, 1)$ and $\bar{r}^{(l)}_{0,ij} \in \gN (0,1)$. We can then write
%\begin{align*}
%    &\E[\|\Wlzero \|_2] = \sigma_{0,w}\E[\| \bar{W}^{(l)}_0\|_2] = \sigma_{0,w}(\sqrt{m_l} + \sqrt{m_{l-1}}),\\
%    &\E[\|\Rlzero \|_2] = \sigma_{0,r}\E[\| \bar{R}^{(l)}_0\|_2] = \sigma_{0,r}(\sqrt{m_l} + \sqrt{m_{l-1}})
%\end{align*}
%from Gordon's Theorem for Gaussian random matrices (see Theorem 5.32, Proposition 3.4 in \citep{vershynin2010introduction}) where  the function $f: B\rightarrow \| \sigma_0 B\|_2$ is a $\sigma_0$-Lipchitz function (where the matrix $B$ can be treated as a vector). Finally, choosing $t_w = \sigma_{0,w}\sqrt{2\log m}$ so that \eqref{eq:signularValueW_l_R_l_zero} holds with probability at least $\left( 1 - \frac{2}{m}\right)$. In order to obtain the result in \eqref{eq:Wl_zero_Rl_zero} we consider the following cases: 
%\begin{itemize}
%    \item {\bf Case 1:} $l=1$. With $m_0 = d_u$ and $m_1 = m$.
%    \begin{align*}
%        \left\|W_0^{(1)}\right\|_2 &\leq \sigma_0(\sqrt{d}+\sqrt{m}+\sqrt{2 \log m}) \leq \sigma_0(2 \sqrt{m}+\sqrt{2 \log m}),\\
%        R^{(1)}_0 &= 0.
%    \end{align*}
%    \item {\bf Case 2:} $2\leq l \leq L$. With $m_l = m_{l-1} = m$
%    \begin{align*}
%        &\| \Wlzero\|_2 \leq \sigma_{0,w}\left( 2\sqrt{m} + \sqrt{2\log m}\right),
%        &\| \Rlzero\|_2 \leq \sigma_{0,r}\left( 2\sqrt{m} + \sqrt{2\log m}\right).
%    \end{align*}
%    Now, using $\sigma_{0,w} = \dfrac{\sigma_{1,w}}{2(1 + \sqrt{\frac{\log m}{2m}})}$ and $\sigma_{0,r} = \dfrac{\sigma_{1,r}}{2(1 + \sqrt{\frac{\log m}{2m}})}$ completes the proof.\qed
%\end{itemize}
%\end{lemm}
\begin{prop}[{\bf Layer-wise matrices}]
\label{prop:W_l_R_l_bound_FNO}
Under Assumption~\ref{asmp:smoothinit_FNO}, for $\vtheta\in B^{\mathrm{Euc}}_{\rho_w,\rho_r\rho_1}(\vtheta_0)$, with probability at least $1-\frac{2}{m}$ we have
\begin{equation}
    \left\|W^{(l)}\right\|_2 \leq\left(\sigma_{1,w}+\frac{\rho_w}{\sqrt{m}}\right) \sqrt{m},\;l\in[L+1] \quad \text{and}\quad 
    \left\|R^{(l)}\right\|_2 \leq\left(\sigma_{1,r}+\frac{\rho_r}{\sqrt{m}}\right) \sqrt{m},\;l\in\{2,\dots,L+1\}
\end{equation}
\proof By the triangle inequality and Lemma~\ref{lemm:InitParamFNO},
\begin{align*}
    &\| W^{(l)}\|_2 \leq \| \Wlzero\|_2 + \| W^{(l)} - \Wlzero\|_2 \leq\sigma_{1,w}\sqrt{m} + \rho_w, \\ 
    &\| R^{(l)}\|_2 \leq \| \Rlzero\|_2 + \| R^{(l)} - \Rlzero\|_2 \leq \sigma_{1,r}\sqrt{m} + \rho_r~.
\end{align*}
\qed
\end{prop}
We now bound the norm of the output $\aalpha^{(l)}$ at the layer $l\in[L+1]$.
%
\begin{lemm}[{\bf Norm of the $l$-th layer output}]
\label{lemm:TwoNormOutputFNOBlock}
For $l\in [L+1]$, under Assumptions~\ref{asmp:Activation_Function_FNO} and \ref{asmp:smoothinit_FNO} for $\theta\in B^{\mathrm{Euc}}_{\rho_w,\rho_r\rho_1}(\vtheta_0)$, with probability at least $1 - \frac{2l}{m}$, we have
\begin{equation}
    \left\|\aalpha^{(l)}\right\|_2 \leq \sqrt{m}\left(\sigma_1+\frac{\rho}{\sqrt{m}}\right)^l+\sqrt{m} \sum_{i=1}^l\left(\sigma_1+\frac{\rho}{\sqrt{m}}\right)^{i-1}|\phi(0)|=\left(\gamma^l+|\phi(0)| \sum_{i=1}^l \gamma^{i-1}\right) \sqrt{m},
\end{equation}
where,
\begin{equation*}
    \sigma_1 = \sigma_{1,w} + \sigma_{1,r},\quad\rho = \rho_w + \rho_r,\quad \text{and}\quad \gamma=\sigma_1+\frac{\rho}{\sqrt{m}}.
\end{equation*}
\proof We prove the result using induction (e.g., see Lemma A.2 in~\citep{banerjee2022restricted}). First, note that for the first hidden layer, using the fact that $\phi$ is $1$-Lipschitz,
\begin{equation}
    \left\|\phi\left(\frac{1}{\sqrt{d}} W^{(1)} \aalpha^{(0)}\right)\right\|_2-\|\phi(\mathbf{0})\|_2 \leq\left\|\phi\left(\frac{1}{\sqrt{d}} W^{(1)} \aalpha^{(0)}\right)-\phi(\mathbf{0})\right\|_2 \leq\left\|\frac{1}{\sqrt{d}} W^{(1)} \aalpha^{(0)}\right\|_2,
\end{equation}
where $\vzero$ denotes the zero vector of appropriate size. This in turn gives, using $\|\aalpha^{(0)} \|_2 = \sqrt{d}$,
\begin{align*}
    \begin{aligned}
        \left\|\aalpha^{(1)}\right\|_2 & =\left\|\phi\left(\frac{1}{\sqrt{d}} W^{(1)} \aalpha^{(0)}\right)\right\|_2 \leq\left\|\frac{1}{\sqrt{d}} W^{(1)} \aalpha^{(0)}\right\|_2+\|\phi(\mathbf{0})\|_2 \\
        & \leq \frac{1}{\sqrt{d}}\left\|W^{(1)}\right\|_2\left\|\aalpha^{(0)}\right\|_2+|\phi(0)| \sqrt{m} \\
        & \leq\left(\sigma_{1,w}+\frac{\rho_w}{\sqrt{m}}\right) \sqrt{m}+|\phi(0)| \sqrt{m} \\
        & \leq\left(\sigma_{1,w}+\sigma_{1,r}+\frac{\rho_w + \rho_r}{\sqrt{m}}\right) \sqrt{m}+|\phi(0)| \sqrt{m}~.
    \end{aligned}
\end{align*}
Now, consider also the output at layer $2$, namely,
\begin{align*}
   \| \aalpha^{(2)}\|_2 
   = 
   \left\|\phi\left(\frac{1}{\sqrt{m}} W^{(2)} \aalpha^{(1)}
    +
    \frac{1}{\sqrt{m}} F^{*} R^{(2)} F \aalpha^{(1)}\right)\right\|_2,
\end{align*}
which gives,
\begin{align*}
    & \left\|\phi\left(\frac{1}{\sqrt{m}} W^{(2)} \aalpha^{(1)}
    +
    \frac{1}{\sqrt{m}} F^{*} R^{(2)} F \aalpha^{(1)}\right)\right\|_2 - \| \phi(\mathbf{0})\|_2 \\
    &\leq 
    \left\|\phi\left(\frac{1}{\sqrt{m}} W^{(2)} \aalpha^{(1)}
    +
    \frac{1}{\sqrt{m}} F^{*} R^{(2)} F \aalpha^{(1)}\right)
    - \phi(\mathbf{0})
    \right\|_2 \leq 
    \left\|\frac{1}{\sqrt{m}} W^{(2)} \aalpha^{(1)}
    +
    \frac{1}{\sqrt{m}} F^{*} R^{(2)} F \aalpha^{(1)}\right\|_2,
\end{align*}
and, in turn,
\begin{align*}
    \| \aalpha^{(2)}\|_2 
    &\leq \left\|\frac{1}{\sqrt{m}} W^{(2)} \aalpha^{(1)}
    +
    \frac{1}{\sqrt{m}} F^{*} R^{(2)} F \aalpha^{(1)}\right\|_2 + \| \phi(\mathbf{0})\|_2 \\
    &\leq \left\|\frac{1}{\sqrt{m}} W^{(2)} \aalpha^{(1)}\right\|_2+\left\|\frac{1}{\sqrt{m}} F^* R^{(2)} F \aalpha^{(1)}\right\|_2+|\phi(0)| \sqrt{m} \\
    &\overset{(a)}{\leq} \frac{1}{\sqrt{m}} \|W^{(2)}\|_2 \|\aalpha^{(1)}\|_2 + \frac{1}{\sqrt{m}} \|R^{(2)}\|_2 \| \aalpha^{(1)}\|_2 + \sqrt{m}|\phi(0)| \\
    &\leq \left( \sigma_{1,w} + \frac{\rho_w}{\sqrt{m}} + \sigma_{1,r} + \frac{\rho_r}{\sqrt{m}}\right) \| \aalpha^{(1)}\|_2 + \sqrt{m}|\phi(0)|\\
    &\leq \sqrt{m}\left( \sigma_1 + \frac{\rho}{\sqrt{m}}\right)^2 + \left(1 + \left( \sigma_1 + \frac{\rho}{\sqrt{m}}\right) \right)\sqrt{m}|\phi(0)|,
\end{align*}
where (a) follows from the fact that the operator $F$ is a unitary matrix. Now, for the inductive step, consider that the output at layer $l-1$ satisfies
\begin{align*}
    \left\|\aalpha^{(l-1)}\right\|_2 
    \leq \sqrt{m}\left(\sigma_1
    +
    \frac{\rho}{\sqrt{m}}\right)^{l-1}
    +\sqrt{m} \sum_{i=1}^{l-1}
    \left(\sigma_1
        +
        \frac{\rho}{\sqrt{m}}\right)^{i-1}|\phi(0)|.
\end{align*}
Finally, at layer $l$, we have
\begin{align}
    \left\|\aalpha^{(l)}\right\|_2 
    &\leq \frac{1}{\sqrt{m}}\left(\underbrace{\left\|W^{(l)}\right\|_2+\left\|F^* R^{(l)} F\right\|_2}\right) \| \aalpha^{(l-1)}\|_2
    +
    \sqrt{m}|\phi (0)|\\
    &\leq \left( \sigma_{1,w} + \sigma_{1,r} + \frac{\rho_w + \rho_r}{\sqrt{m}}\right) \| \aalpha^{(l-1)}\|_2
    +
    \sqrt{m}|\phi (0)|\\
    &\leq \sqrt{m}\left(\sigma_1
    +
    \frac{\rho}{\sqrt{m}}\right)^{l}
    +\sqrt{m} \sum_{i=1}^{l}
    \left(\sigma_{1}
        +
        \frac{\rho}{\sqrt{m}}\right)^{i-1}|\phi(0)|.
\end{align}
Introducing $\gamma = \sigma_1 + \dfrac{\rho}{\sqrt{m}}$, we can write
\begin{equation}
    \|\aalpha^{(l)}\|_2 \leq \sqrt{m}\left(\gamma^l+|\phi(0)| \sum_{i=1}^{l} \gamma^{i-1}\right).
    \label{eq:alpha_l_2_norm_bound}
\end{equation}
This completes the proof.\hfill \qed
\end{lemm}

From now on, we will use the notation $\rho$, $\sigma_1$, and $\gamma$ as defined in Lemma~\ref{lemm:TwoNormOutputFNOBlock}.

\begin{lemm}
    \label{lemm:firstDerivativeBoundFNO}
    For $l\in \{2,\dots,L+1\}$, under Assumptions~\ref{asmp:Activation_Function_FNO} and \ref{asmp:smoothinit_FNO} for $\theta\in B^{\mathrm{Euc}}_{\rho_w,\rho_r\rho_1}(\theta_0)$, with probability at least $1 - \frac{2}{m}$, we have
    \begin{equation}
        \left\|\frac{\partial \aalpha^{(l)}}{\partial \aalpha^{(l-1)}}\right\|_2 \leq
        %\left(\sigma_{1,w} + \frac{\rho_w}{\sqrt{m}}\right)
        %+
        %\left(\sigma_{1,r} + \frac{\rho_r}{\sqrt{m}}\right)
        %=
        \gamma.
    \end{equation}
    \proof
    We first note that 
    \begin{align*}
        \left[
            \del{\aalpha^{(l)}}{\aalpha^{(l-1)}}
        \right]_{ij} = 
        \frac{1}{\sqrt{m}} \phi'(\widetilde{\aalpha}^{(l-1)}) \left[ 
            W^{(l)}_{ij} + [F^*R^{(l)}F]_{ij}
        \right].
    \end{align*}
    Now, from the definition $\|A\|_2 = \sup_{\| \vv\|_2 = 1} \| A\vv\|_2$ we have,
    \begin{align}
        \begin{aligned}
            \left\|\frac{\partial \aalpha^{(l)}}{\partial \aalpha^{(l-1)}}\right\|_2 
            &= 
            \sup_{\|\vv\|_2 = 1} \frac{1}{\sqrt{m}}
            \left(
                {\phi'} \left\|\left( W^{(l)} + F^*R^{(l)}F \right)\vv\right\|_2
            \right)\\
            &\overset{(a)}{\leq} \sup_{\|\vv\|_2 = 1} \frac{1}{\sqrt{m}}
            \left(
                \|W^{(l)}\vv\|_2 +  \|F^*R^{(l)}F\vv\|_2
            \right)
            \\
            &\overset{(b)}{=} \sup_{\|\vv\|_2 = 1} \frac{1}{\sqrt{m}}
            \left(
                \|W^{(l)}\vv\|_2 +  \|R^{(l)}F\vv\|_2
            \right)\\
            &\overset{(c)}{\leq}\sup_{\|\vv\|_2 = 1} \frac{1}{\sqrt{m}}
            \left(
                \|W^{(l)}\|_2\norm{\vv}_2 +  \norm{R^{(l)}}_2\norm{\vv}_2
            \right)\\
            &=\frac{1}{\sqrt{m}}
            \left(
                \|W^{(l)}\|_2 +  \norm{R^{(l)}}_2
            \right)~,            
        \end{aligned}
    \end{align}
    where $(a)$ follows from the fact that $\phi$ is $1$-Lipchitz and by using the triangle inequality, and $(b)$ and $(c)$ follow from the fact that $F^*$ and $F$ are isometries with respect to the $L_2$-norm, i.e. $\|F \vv\|_2=\|\vv\|_2$ and $\|F^* \vv\|_2=\|\vv\|_2$ for $\vv\in\R^{m}$.
    This finally gives 
    \begin{align*}
        \left\|\frac{\partial \aalpha^{(l)}}{\partial \aalpha^{(l-1)}}\right\|_2  
        \leq 
        \frac{1}{\sqrt{m}}\left(\|W^{(l)}\|_2 + \|R^{(l)}\|_2 \right) &\leq
            \left(\sigma_{1,w} + \frac{\rho_w}{\sqrt{m}}\right)
            + \left(\sigma_{1,r} + \frac{\rho_r}{\sqrt{m}}\right) \\
            & = \gamma~,
    \end{align*}
where we used Proposition~\ref{prop:W_l_R_l_bound_FNO}. This completes the proof.\hfill\qed
\end{lemm}

We make use of the Einstein summation convention, i.e. repeated indices imply summation, unless explicitly stated. We also use the notation $\text{vec}(\cdot)$ to denote the vectorization of the matrix argument according to some fixed manner (e.g., row-wise vectorization).

\begin{lemm}
    \label{lemm:gradient_alpha_params}
    Under Assumptions~\ref{asmp:Activation_Function_FNO} and \ref{asmp:smoothinit_FNO} and for $\theta\in B^{\mathrm{Euc}}_{\rho_w,\rho_r\rho_1}(\vtheta_0)$, with probability at least $ 1 - \frac{2l}{m}$,
    \begin{equation*}
\left\|\dfrac{\partial\aalpha^{(l)}}{\partial\mathbf{w}^{(l)}}\right\|_2,\left\|\dfrac{\partial\aalpha^{(l)}}{\partial\mathbf{r}^{(l)}}\right\|_2
\leq         
\left(\gamma^{l-1} + |\phi(0)|\sum_{i=1}^{l-1}\gamma^{i-1}\right)
    \end{equation*}
where, $\mathbf{w}^{(l)} = \text{vec}(W^{(l)})$ for $l\in[L+1]$, and $\r^{(l)} = \text{vec}(R^{(l)})$ for $l\in\{2,\dots,L+1\}$. 
\end{lemm} 

    \proof
    We can index the vectors $\vw^{(l)}$ and $\r^{(l)}$ according to their matrix form $W^{(l)}_{jj'}$ and $R^{(l)}_{jj'}$, respectively, with the indices $j\in [m]$, and $j'\in [d]$ when $l=1$ or $j'\in [m]$ when $l\in \{2,\dots,L+1\}$. Therefore,
    \begin{equation*}
        \left[\frac{\partial \aalpha^{(l)}}{\partial \mathbf{w}^{(l)}}\right]_{i, j j^{\prime}} = \frac{1}{\sqrt{m}}\phi' (\widetilde{\aalpha}^{(l)}_i)\delta_{ij}\aalpha^{(l-1)}_{j'},
        \quad \delta_{ij} = \begin{cases}
            1 &{i=j}\\
            0 &\text{otherwise}
        \end{cases}.
    \end{equation*}
    Now, for $l\in \{2,\dots,L+1\}$, we can write the $L_2$-norm of the matrices as follows
    \begin{align*}
        \left\|\frac{\partial \aalpha^{(l)}}{\partial \mathbf{w}^{(l)}}\right\|_2^2
        &=
        \sup _{\|V\|_F=1} \frac{1}{m} \sum_{i=1}^m\left(\phi^{\prime}\left(\widetilde{\aalpha}_i^{(l)}\right) \sum_{j, j^{\prime}=1}^m \aalpha_{j^{\prime}}^{(l-1)} \delta_{ij} V_{j j^{\prime}}\right)^2\\
        &\leq \sup_{\|V\|_F = 1} \frac{1}{m} \| V\aalpha^{(l-1)}\|_2^2\\
        & \leq \sup_{\|V\|_F = 1}\frac{1}{m} \|V\|_2^2 \|\aalpha^{(l-1)}\|_2^2 \\ 
        & \overset{(a)}{\leq} \sup_{\|V\|_F = 1}\frac{1}{m} \|V\|_F^2 \|\aalpha^{(l-1)}\|_2^2 \\
        & = \frac{1}{m}\|\aalpha^{(l-1)}\|_2^2 \\
        & \overset{(b)}{\leq} \frac{1}{m} \left[ 
            \sqrt{m}\left(\gamma^{l-1}+|\phi(0)| \sum_{i=1}^{l-1} \gamma^{i-1}\right)
        \right]^2 = \left(\gamma^{l-1}+|\phi(0)| \sum_{i=1}^{l-1} \gamma^{i-1}\right)^2,
    \end{align*}
    where $(a)$ follows from the fact that $\|V\|_2 \leq \|V\|_F$ and $(b)$ from Lemma~\ref{lemm:TwoNormOutputFNOBlock}. The $l=1$ case follows in a similar fashion:
    \begin{equation*}
        \left\|\frac{\partial \aalpha^{(1)}}{\partial \mathbf{w}^{(1)}}\right\|_2^2 \leq 
        \frac{1}{d}\|\aalpha^{(0)} \|_2^2 = 1.
    \end{equation*}
    Similarly, for $l\in\{2,\dots,L+1\}$,
    \begin{align*}
        \left\|\frac{\partial \aalpha^{(l)}}{\partial \mathbf{r}^{(l)}}\right\|_2^2
        &=
        \sup _{\|V\|_F=1} \frac{1}{m} \sum_{i=1}^m\left(\phi^{\prime}\left(\widetilde{\aalpha}_i^{(l)}\right) 
        F^*_{ij}F_{j'p}\aalpha^{(l-1)}_p
        V_{j j^{\prime}}\right)^2\\
        &\leq \sup_{\|V\|_F = 1} \frac{1}{m} \| (F^*VF)\aalpha^{(l-1)}\|_2^2\\
        & \leq \sup_{\|V\|_F = 1}\frac{1}{m} \|F^*VF\|_2^2 \|\aalpha^{(l-1)}\|_2^2 \\ 
        & \leq \sup_{\|V\|_F = 1}\frac{1}{m} \|F^*\|^2_2 \|V\|_2^2 \|F\|_2^2  \|\aalpha^{(l-1)}\|_2^2 \\ 
        & \overset{(a)}{\leq} \sup_{\|V\|_F = 1}\frac{1}{m} \|V\|_F^2 \|\aalpha^{(l-1)}\|_2^2 \\
        & = \frac{1}{m}\|\aalpha^{(l-1)}\|_2^2 \\
        & \overset{(b)}{\leq} \frac{1}{m} \left[ 
            \sqrt{m}\left(\gamma^{l-1}+|\phi(0)| \sum_{i=1}^{l-1} \gamma^{i-1}\right)
        \right]^2 = \left(\gamma^{l-1}+|\phi(0)| \sum_{i=1}^{l-1} \gamma^{i-1}\right)^2,
    \end{align*}
    where $(a)$ follows again by $\|V\|_2 \leq \|V\|_F$ and the fact that $F^*$ and $F$ are unitary matrices, and $(b)$ from Lemma~\ref{lemm:TwoNormOutputFNOBlock}. This completes the proof. \qed


\textbf{Hessians.} We now focus on bounding the Hessian of the predictor $f$ in equation~\eqref{eq:FNO_predictor_app}. Note that the FNO model can be considered as having $L+1$ layers, with Layer 1 being a feedforward single layer encoder on top of the encoder $P$, the $L$ layers from Layer $2$ to Layer $L+1$ being FNO hidden layers, and Layer $L+2$ being the output of the linear decoder.
%
%
Likewise, we decompose the Hessian matrix $\mH$ of the FNO in three different blocks corresponding to the aforementioned encoder, FNO hidden layers, and decoder, respectively.

Firstly, the Hessian blocks associated to the hidden FNO layers are: 
%:  layers can be viewed as $2 \times 2$ block matrix, namely,
%\begin{align*}
%    H = \begin{bmatrix}
%        H_{w}^{(l_1, l_2)} & H_{w,r}^{(l_1, l_2)}\\
%        H_{r,w}^{(l_1, l_2)} & H_{r}^{(l_1, l_2)}
%    \end{bmatrix}
%\end{align*}
%where 
\begin{itemize}
\item the %$(1,1)$ block has 
$L \times L$ sub-blocks corresponding to $H_{w}^{(l_1,l_2)} := \frac{\partial^2 f}{\partial \w^{(l_1)} \partial \w^{(l_2)}}$ for $l_1, l_2 \in \{2,\ldots,L+1\}$, 
\item the %$(2,2)$ block has 
$L \times L$ sub-blocks corresponding to $H_{r}^{(l_1,l_2)} := \frac{\partial^2 f}{\partial \r^{(l_1)} \partial \r^{(l_2)}}$ for $l_1, l_2 \in \{2,\ldots,L+1\}$, and 
\item the 
%$(1,2)$ and $(2,1)$ 
cross blocks have terms of the form $H_{w,r}^{(l_1,l_2)} := \frac{\partial^2 f}{\partial \w^{(l_1)} \partial \r^{(l_2)}}$ for $l_1, l_2 \in \{2,\ldots,L+1\}$. 
\end{itemize}
%

Secondly, the Hessian blocks corresponding to the single layer encoder, i.e., with respect to weight $W^{(1)}$:
\begin{itemize}
\item diagonal block $H_{w}^{(1,1)} := \frac{\partial^2 f}{\partial \w^{{(1)}^2}}$,
\item off-diagonal blocks $H_{w}^{(1,l_1)} := \frac{\partial^2 f}{\partial \w^{(1)} \partial \w^{(l_1)}}$ and $H_{w}^{(l_1,1)}$ for $l_1 \in \{2,\ldots,L+1\}$, and 
\item off-diagonal blocks $H_{w,r}^{(1,l_2)} := \frac{\partial^2 f}{\partial \w^{(1)} \partial \r^{(l_2)}}$ ans $H_{r,w}^{(l_2,1)}$ for $l_2 \in \{2,\ldots,L+1\}$.
\end{itemize}

Finally, the Hessian blocks corresponding to the decoder, i.e., with respect to weight $\v$:
\begin{itemize}
\item diagonal block $H_{v} := \frac{\partial^2 f}{\partial \v^2}$, which is the zero matrix $\vzero_{m\times m}$,
\item off-diagonal block $H_{w,v}^{(l_1)} := \frac{\partial^2 f}{\partial \w^{(l_1)} \partial \v}$ and $H_{v,w}^{(l_1)}$ for $l_1 \in \{1,\ldots,L+1\}$, and
\item off-diagonal block $H_{r,v}^{(l_2)} := \frac{\partial^2 f}{\partial \r^{(l_2)} \partial \v}$ and $H_{v,r}^{(l_2)}$ for $l_2 \in \{2,\ldots,L+1\}$.
\end{itemize}


%Note that the definitions suitably generalize the equivalent quantities in \citep{liu_linearity_2021}[BCZB'23].

First, we note that due to the symmetry of the Hessian matrix of the FNO model $\mH$:
%that for the Hessian matrix of the FNO model $\mH$:
\begin{align}
\label{eq:Hessian_big}
    \| \mH \|_2 & \leq \sum_{l_1,l_2=1}^{L+1} \| H_{w}^{(l_1,l_2)} \|_2 + \sum_{l_1,l_2=2}^{L+1} \| H_{r}^{(l_1,l_2)} \|_2 + 2 \sum_{l_1=1}^{L+1} \sum_{l_2=2}^{L+1} \| H_{w,r}^{(l_1,l_2)} \|_2 
    + 2 \sum_{l_1=1}^{L+1} \| H_{w,v}^{(l_1)} \|_2 + 2 \sum_{l_2=2}^{L+1} \| H_{r,v}^{(l_2)} \|_2~.
\end{align}

%For convenience of notation, we assume that whenever we find the differential operator ``$\partial \r^{(1)}$'', the associated quantity to this operator will be equal to the zero matrix. 

We define
\begin{equation}
\label{eq:qwr2}
\begin{split}
\cQ_{\infty}(f) & := \max_{l \in [L+1]} ~\left\| \frac{\partial f}{\partial \aalpha^{(l)}} \right\|_{\infty}~,  \\
\cQ^{(w,r)}_2(f) & := \max_{l \in [L+1]} ~\left\{ \left\| \frac{\partial \aalpha^{(l)}}{\partial \w^{(l)}} \right\|_2, ~\left\| \frac{\partial \aalpha^{(l)}}{\partial \r^{(l)}} \right\|_2\right\} ~,  \\
\cQ^{(w,r)}_{2,2,1}(f) & := \max_{\substack{1\leq l_1 \leq L+1\\2\leq l_2 \leq L+1\\3\leq l_3 \leq L+1}}~ \left\{ 
\left\| \frac{\partial^2 \aalpha^{(l_2)}}{\partial \w^{(l_2)}  \partial \r^{(l_2)}} \right\|_{2,2,1} , 
\left\| \frac{\partial \aalpha^{(l_1)}}{\partial \w^{(l_1)}} \right\|_2  \left\| \frac{\partial^2 \aalpha^{(l_2)}}{\partial \aalpha^{(l_2-1)} \partial \r^{(l_2)}} \right\|_{2,2,1} ,\right. \\ 
& \left.\phantom{:= \max_{1\leq l_1 \leq l_2 \leq l_3 \leq L+1}~~~~~~ }\left\| \frac{\partial \aalpha^{(l_1)}}{\partial \r^{(l_1)}} \right\|_2  \left\| \frac{\partial^2 \aalpha^{(l_2)}}{\partial \aalpha^{(l_2-1)} \partial \w^{(l_2)}} \right\|_{2,2,1} ,\left\| \frac{\partial \aalpha^{(l_1)}}{\partial \w^{(l_1)}} \right\|_2  \left\| \frac{\partial \aalpha^{(l_2)}}{\partial \r^{(l_2)}} \right\|_2  \left\| \frac{\partial^2 \aalpha^{(l_3)}}{(\partial \aalpha^{(l_3-1)})^2} \right\|_{2,2,1}\right\} ~,\\
\cQ^{(w)}_{2,2,1}(f) & := \max_{\substack{1\leq l_1 \leq L+1\\2\leq l_2 \leq L+1\\3\leq l_3 \leq L+1}}~ \left\{ 
\left\| \frac{\partial^2 \aalpha^{(l_1)}}{(\partial \w^{(l_1)})^2} \right\|_{2,2,1} , 
\left\| \frac{\partial \aalpha^{(l_1)}}{\partial \w^{(l_1)}} \right\|_2  \left\| \frac{\partial^2 \aalpha^{(l_2)}}{\partial \aalpha^{(l_2-1)} \partial \w^{(l_2)}} \right\|_{2,2,1} ,\right.\\
&\left.\phantom{:= \max_{1\leq l_1 \leq l_2 \leq l_3 \leq L+1}~~~~~~ }\left\| \frac{\partial \aalpha^{(l_1)}}{\partial \w^{(l_1)}} \right\|_2  \left\| \frac{\partial \aalpha^{(l_2)}}{\partial \w^{(l_2)}} \right\|_2  \left\| \frac{\partial^2 \aalpha^{(l_3)}}{(\partial \aalpha^{(l_3-1)})^2} \right\|_{2,2,1}
\right\} ~,\\
\cQ^{(r)}_{2,2,1}(f) & := \max_{\substack{2\leq l_1 \leq L+1\\3\leq l_2 \leq L+1\\4\leq l_3 \leq L+1}}~ \left\{ 
\left\| \frac{\partial^2 \aalpha^{(l_1)}}{(\partial \r^{(l_1)})^2} \right\|_{2,2,1} , 
\left\| \frac{\partial \aalpha^{(l_1)}}{\partial \r^{(l_1)}} \right\|_2  \left\| \frac{\partial^2 \aalpha^{(l_2)}}{\partial \aalpha^{(l_2-1)} \partial \r^{(l_2)}} \right\|_{2,2,1} ,\right.\\
&\left.\phantom{:= \max_{1\leq l_1 \leq l_2 \leq l_3 \leq L+1}~~~~~~ }\left\| \frac{\partial \aalpha^{(l_1)}}{\partial \r^{(l_1)}} \right\|_2  \left\| \frac{\partial \aalpha^{(l_2)}}{\partial \r^{(l_2)}} \right\|_2  \left\| \frac{\partial^2 \aalpha^{(l_3)}}{(\partial \aalpha^{(l_3-1)})^2} \right\|_{2,2,1}
\right\}~,
\end{split}
\end{equation}
where, for an order-3 tensor $T \in \R^{d_1 \times d_2 \times d_3}$ we define the operator $\norm{\cdot}_{2,2,1}$ as follows, 
\begin{align}
\| T \|_{2,2,1} := \sup_{\|\a\|_2 = \|\b\|_2 = 1} \sum_{k=1}^{d_3} \left| \sum_{i=1}^{d_1} \sum_{j=1}^{d_2} T_{ijk} a_i b_j \right|~,~~\a \in \R^{d_1}, \b \in \R^{d_2}~.
\label{eq:norm-221}
\end{align}
Note that it seems from~\eqref{eq:qwr2} that we need the depth $L$ of the FNO to be $L\geq 3$. However, the bounds presented in Lemma~\ref{lemm:hessgradbounds-FNO} also hold for FNOS with depth $L<3$: indeed, the upper bounds we derive in this section for an FNO with depth $L$ will trivially hold for FNOS with depths $L-1,\dots,1$.

\begin{lemm}
\label{lemm:cross-w-r-bound}
    Under Assumptions~\ref{asmp:Activation_Function_FNO} and \ref{asmp:smoothinit_FNO} for $\vtheta \in B^{\mathrm{Euc}}_{\rho_w,\rho_r\rho_1} (\vtheta_0)$, the following inequalities hold with probability at least $1 - \frac{2(L+2)}{m}$, for $l_1\in[L+1]$,
    % \leq \beta_{\phi} \left(                 \gamma^{l_1 - 1} + |\phi(0)| \sum_{i=1}^{l_1 - 1} \gamma^{i-1}       \right)^2
    %
    \begin{equation}
    \label{eq:norm_d_alpha_l_w_l_w_l}\left\|\frac{\partial^2\aalpha^{{(l_1)}}}{(\partial{\mathbf{w}^{(l_1)}})^2} \right\|_{2,2,1} \leq \beta_{\phi}(1+\gamma^L)^2(1+L|\phi(0)|)^2~,
    \end{equation}
% \leq 
%        \beta_{\phi} \left( 
%                \gamma^2 + \left(
%                    \gamma^{l_1 - 1} + |\phi(0)| \sum_{i=1}^{l_1 - 1} \gamma^{i-1}
%                \right)^2
%            \right) + 1
%    
and for $l_2\in\{2,\dots,L+1\}$,
    \begin{equation}
        \label{eq:norm_d_alpha_l_w_l_r_l}
        \left\|\frac{\partial^2\aalpha^{{(l_2)}}}{\partial{\mathbf{w}^{(l_2)}}\partial \r^{(l_2)}} \right\|_{2,2,1} \leq
        \beta_{\phi}(1+\gamma^L)^2(1+L|\phi(0)|)^2        ~,
    \end{equation}
    %
    \begin{equation}
        \label{eq:norm_d_alpha_l_alpha_l_minus_1}
        \left\| \frac{\partial^2 \aalpha^{(l_2)}}{(\partial \aalpha^{(l_2-1)})^2} \right\|_{2,2,1} \leq 2\beta_\phi\gamma^2~,
    \end{equation}
    \begin{equation}
    \label{eq:norm_d_alpha_l_alpha_l_w_l}\left\|\frac{\partial^2\aalpha^{{(l_2)}}}{\partial \aalpha^{(l_2-1)}\partial{\mathbf{w}^{(l_2)}}} \right\|_{2,2,1} \leq \beta_{\phi}(1+\gamma^L)^2(1+(1+L|\phi(0)|)^2)+1~,
    \end{equation}
    \begin{equation}
        \label{eq:norm_d_alpha_l2_alpha_l2_minus_1_dr}
        \left\|\frac{\partial^2\aalpha^{{(l_2)}}}{\partial{\aalpha^{(l_2-1)}}\partial \r^{(l_2)}} \right\|_{2,2,1}
        \leq \beta_{\phi}(1+\gamma^L)^2(1+(1+L|\phi(0)|)^2)+1
        ~, \text{ and}
    \end{equation}
        \begin{equation}
    \label{eq:norm_d_alpha_l_r_l_r_l}\left\|\frac{\partial^2\aalpha^{{(l_2)}}}{(\partial{\mathbf{r}^{(l_2)}})^2} \right\|_{2,2,1} \leq \beta_{\phi}(1+\gamma^L)^2(1+L|\phi(0)|)^2~.
    \end{equation}
% \leq \beta_{\phi} \left( 
%            \gamma^2 + \left(
%                \gamma^{l_2 - 1} + |\phi(0)| \sum_{i=1}^{l_2 - 1} \gamma^{i-1}
%            \right)^2
%        \right) + 1    
    \end{lemm}
    
    \proof
    We first begin by proving \eqref{eq:norm_d_alpha_l_w_l_r_l}. Note that from \eqref{eq:FNO_predictor_app} we have
    \begin{align*}
        \frac{\partial^2 \aalpha_i^{\left(l_2\right)}}{\partial \w_{j j^{\prime}}^{\left(l_2\right)} \partial \r_{k k^{\prime}}^{\left(l_2\right)}}
        =
        \frac{1}{m} \phi^{\prime \prime}
        \left(\tilde{\aalpha}^{\left(l_2\right)}\right) \cdot \aalpha_{j^{\prime}}^{\left(l_2 - 1\right)} \delta_{i j} F_{i k}^* F_{k^{\prime} q} \aalpha_q^{\left(l_2-1\right)},
    \end{align*}
    where we make use of the Einstein notation. Now,
    \begin{align}
        \begin{aligned}
            &\left\| 
                \frac{\partial^2 \aalpha_i^{\left(l_2\right)}}{\partial \w^{\left(l_2\right)} \partial \r^{\left(l_2\right)}}
            \right\|_{2,2,1} \\
            &= 
            \sup_{\|V_1\|_F = 1, \|V_2\|_F = 1} \sum_{i=1}^m
            \left| 
                \frac{1}{m} \phi^{\prime \prime}(\tilde{\aalpha}^{(l_2)}_i) \aalpha_{j^{\prime}}^{\left(l_2 -1\right)} \delta_{i j} F_{i k}^* F_{k^{\prime} q} \aalpha_q^{\left(l_2-1\right)} V_{1_{j j^{\prime}}} V_{2_{k k^{\prime}}} 
            \right| \\ 
            &= 
            \sup_{\|V_1\|_F = 1, \|V_2\|_F = 1}\sum_{i=1}^m
            \left| 
            \frac{\phi^{\prime \prime}(\tilde{\aalpha}^{(l_2)}_i)}{m}
            \left(
                V_{1_{ij^{\prime}}} \aalpha^{(l_2 - 1)}_{j^{'}}
            \right) 
            \left(
                F_{i k}^* V_{2_{k k^{\prime}}} F_{k^{\prime} q} \aalpha_q^{\left(l_2{-1}\right)}\right)
            \right| \\
            &\leq
            \sup_{\|V_1\|_F = 1, \|V_2\|_F = 1}
            \frac{\beta_\phi}{m}
            \sum^m_{i=1}
            \left| 
            (V_1 \aalpha^{(l_2-1)})_i((F^*V_2F)\aalpha^{(l_2-1)})_i      \right| 
            \\
            &\overset{(a)}{\leq}
            \sup_{\|V_1\|_F = 1, \|V_2\|_F = 1}
            \frac{\beta_{\phi}}{2m} \left(\left\|V_1 \aalpha^{\left(l_2 - 1\right)}\right\|_2^2+\left\|F^* V_2 F \aalpha^{(l_2-1)}\right\|_2^2\right) \\
            &\overset{(b)}{\leq}
            \frac{\beta_{\phi}}{2m}\left(\left\|\aalpha^{\left(l_2 - 1\right)}\right\|_2^2+\left\|\aalpha^{\left(l_2-1\right)}\right\|_2^2\right) \leq \beta_{\phi}
                \left(\gamma^{l_2-1}+|\phi(0)| \sum_{i=1}^{l_2-1} \gamma^{i-1}\right)^2,
        \end{aligned}
    \end{align}
    where $(a)$ follows from the quadratic expression; where $(b)$ follows from $\|V_1\aalpha^{(l_2 - 1)}\|_{2} \leq \|V_1\|_2 \| \aalpha^{(l_2 - 1)}\|_2$, $\|V_1\|_2 \leq \|V_1\|_F$, $\|V_2\|_2 \leq \|V_2\|_F$,  
    $\|F^*V_2F\aalpha^{(l_2 - 1)}\|_{2} =\| V_2F \aalpha^{(l_2 - 1)} \|_2 \leq \|V_2\|_2 \| F\aalpha^{(l_2 - 1)}\|_2= \|V_2\|_2 \| \aalpha^{(l_2 - 1)}\|_2$ due to $F$ being a unitary operator; and where the last inequality follows from~\eqref{eq:alpha_l_2_norm_bound}. Finally, we can upper bound the last quantity above as in~\eqref{eq:norm_d_alpha_l_w_l_r_l} and complete the proof.

    For proving \eqref{eq:norm_d_alpha_l_alpha_l_minus_1}, again note from \eqref{eq:FNO_predictor_app} that
    \begin{align}
            \begin{aligned}
            \left[ \ddel{\aalpha^{(l_2)}}{\aalpha^{(l_2-1)}} \right]_{i,j,k} 
            &= \frac{1}{m}
            \phi''(\tilde{\aalpha}^{(l_2)}) \left( 
                W^{(l_2)}_{ij} + F^*_{ip}R^{(l_2)}_{pq}F_{qj}
            \right) \cdot \left(W^{(l_2)}_{ik} + F^*_{iu}R^{(l_2)}_{uv}F_{vk} \right) \\
            &= \frac{\phi''}{m} \left[ \underbrace{W^{(l_2)}_{ij} W^{(l_2)}_{ik}}_{T_1} + \underbrace{W^{(l_2)}_{ij}F^{*}_{iu}R^{(l_2)}_{uv}F_{vk}}_{T_2} + \underbrace{F^*_{ip}R^{(l_2)}_{pq}F_{qj}W^{(l_2)}_{ik}}_{T_3} +  \underbrace{F^*_{ip}R^{(l_2)}_{pq}F_{qj} F^{*}_{iu}R^{(l_2)}_{uv}F_{vk}}_{T_4} \right].
            \end{aligned}
    \end{align}
    Then, we can write
    \begin{align*}
        \left\| \ddel{\aalpha^{(l_2)}}{\aalpha^{(l_2-1)}}  \right\|_{2,2,1}=
        \sup_{\|\vv_1\|_2 = 1,\|\vv_2\|_2 = 1} 
        \sum_{i=1}^m\left| \left[ 
            \ddel{\aalpha^{(l_1 )}}{\aalpha^{(l_1 - 1)}}
        \right]_{i,j,k}v_{1_j}v_{2_k}\right|.
    \end{align*}
    Let us consider the notation $\gamma_w = \sigma_{1,w}+\frac{\rho_w}{\sqrt{m}}$ and $\gamma_r = \sigma_{1,r}+\frac{\rho_r}{\sqrt{m}}$.
    Now, we handle each of the terms separately:
    \begin{align}
    \label{eq:T1_norm_d2_alpha2_dalpha2}
        \begin{aligned}
            \sup_{\|\vv_1\|_2 = 1,\|\vv_2\|_2 = 1} \sum_{i=1}^m \left|\frac{\phi''}{m}T_{1_{i,j,k}}v_{1_j}v_{2_k}\right| 
            &= \frac{|\phi''|}{m} \sup_{\|\vv_1\|_2 = 1,\|\vv_2\|_2 = 1} \sum_{i=1}^m
            \left|\left( W^{(l_2)}_{ij} v_{1_j}\right)\cdot \left( W^{(l_2)}_{ik} v_{2_k}\right)\right|\\
            &\leq \frac{\beta_{\phi}}{2m}\sup_{\|\vv_1\|_2 = 1,\|\vv_2\|_2 = 1}\left(\|W^{(l_2)}\|_2^2 \|\vv_1\|_2^2 + \|W^{(l_2)}\|_2^2 \|\vv_2\|_2^2 \right) \\
            &=\beta_{\phi}\left( \sigma_{1,w} + \frac{\rho_w}{\sqrt{m}}\right)^2 = \beta_{\phi} \gamma_w^2.
        \end{aligned}
    \end{align}
    \begin{align}
    \label{eq:T4_norm_d2_alpha2_dalpha2}
        \begin{aligned}
            \sup_{\|\vv_1\|_2 = 1,\|\vv_2\|_2 = 1} \sum_{i=1}^m\left| \frac{\phi''}{m}T_{4_{i,j,k}}v_{1_j}v_{2_k} \right|
            &= \frac{|\phi''|}{m} \sup_{\|\vv_1\|_2 = 1,\|\vv_2\|_2 = 1} \sum_{i=1}^m
            \left| 
            \left((F^*R^{(l_2)}F)_{ij}v_{1_j} \right)\cdot \left((F^*R^{(l_2)}F)_{ik}v_{2_k} \right)
            \right|\\
            &\leq \frac{\beta_{\phi}}{2m} \sup_{\|\vv_1\|_2 = 1,\|\vv_2\|_2 = 1} \left( 
                \|F^*R^{(l_2)}F\|_2^2 \|\vv_1\|_2^2 + \|F^*R^{(l_2)}F\|_2^2 \|\vv_2\|_2^2
            \right)\\
            &=\frac{\beta_{\phi}}{m} \| F^*R^{(l_2)}F\|_2^2\\
            &\leq \frac{\beta_{\phi}}{m}\|R^{(l_2)}\|_2^2 \leq \beta_{\phi}\gamma^2_r.
        \end{aligned}
    \end{align}
    \begin{align}
    \label{eq:T2_norm_d2_alpha2_dalpha2}
        \begin{aligned}
            \sup_{\|\vv_1\|_2 = 1,\|\vv_2\|_2 = 1} \sum_{i=1}^m \left|\frac{\phi''}{m}T_{2_{i,j,k}}v_{1_j}v_{2_k}\right|
            &= \frac{|\phi''|}{m} \sup_{\|\vv_1\|_2 = 1,\|\vv_2\|_2 = 1} \sum_{i=1}^m
            \left|
                (W^{(l_2)}_{ij}v_{1_j})\cdot (F^*R^{(l_2)}F)_{ik}v_{2_k}
            \right|\\
            &\leq \frac{\beta_{\phi}}{2m} \sup_{\|\vv_1\|_2 = 1,\|\vv_2\|_2 = 1}
            \left( \| W^{(l_2)}\|_2^2 \| \vv_1\|_2^2 + \| F^*R^{(l_2)}F\|_2^2 \| \vv_2\|_2^2 \right)\\
            &\leq\frac{\beta_{\phi}}{2m}\left( \|W^{(l_2)}\|_2^2 + \|R^{(l_2)}\|_2^2\right) \leq \frac{\beta_{\phi}}{2} \left( \gamma_w^2 + \gamma_r^2\right).
        \end{aligned}
    \end{align}
    Similarly, for the term corresponding to $T_3$ we obtain
    \begin{align}
    \label{eq:T3_norm_d2_alpha2_dalpha2}
        \begin{aligned}
            \sup_{\|\vv_1\|_2 = 1,\|\vv_2\|_2 = 1} \sum_{i=1}^m \left|\frac{\phi''}{m}T_{3_{i,j,k}}v_{1_j}v_{2_k}\right| \leq \frac{\beta_{\phi}}{2} \left( \gamma_w^2 + \gamma_r^2\right)~.
        \end{aligned}
    \end{align}
    Putting together \eqref{eq:T1_norm_d2_alpha2_dalpha2}, \eqref{eq:T4_norm_d2_alpha2_dalpha2}, \eqref{eq:T2_norm_d2_alpha2_dalpha2} and \eqref{eq:T3_norm_d2_alpha2_dalpha2}, we get
    \begin{align}
        \left\| \ddel{\aalpha^{(l_2)}}{\aalpha^{(l_2-1)}}\right\|_{2,2,1}^2 \leq 2 \beta_{\phi} (\gamma_w^2 + \gamma_r^2) \leq 2\beta_\phi (\gamma_w^2 + \gamma_r^2 + 2\gamma_w\gamma_r) = 2\beta_\phi\gamma^2.
    \end{align}
    This completes the proof for \eqref{eq:norm_d_alpha_l_alpha_l_minus_1}. 
    
    We now look at the proof for \eqref{eq:norm_d_alpha_l2_alpha_l2_minus_1_dr}. First note that
    \begin{align*}
        \begin{aligned}            
            \frac{\partial^2 \aalpha^{\left(l_2\right)}_i}{\partial \aalpha^{\left(l_2-1\right)}_{k} \partial \r_{j j^{\prime}}^{\left(l_2\right)}}
        &=
        \frac{1}{m}\phi''(\tilde{\aalpha_i}) \left(W_{i k}^{\left(l_2\right)}
        +
        F_{i p}^* R_{p q}^{\left(l_2\right)} F_{q k}\right) F_{i j}^{*} F_{j' q} \aalpha_q^{\left(l_2-1\right)} + \frac{1}{\sqrt{m}}\phi' (\tilde{\aalpha}_i^{(l_2)}) F^{*}_{ij}F_{j'k} \\
        &= \underbrace{\frac{\phi''}{m} \left(W_{i k}^{\left(l_2\right)}F_{i j}^{*} F_{j' q} \aalpha_q^{\left(l_2-1\right)} \right)}_{T_1}
        + \underbrace{\frac{\phi''}{m}
        \left( 
            F_{i p}^* R_{p q}^{\left(l_2\right)} F_{q k}F_{i j}^{*} F_{j'q}\aalpha_q^{\left(l_2-1\right)}
        \right)}_{T_2}
        + \underbrace{\frac{1}{\sqrt{m}}\phi' (\tilde{\aalpha}_i^{(l_2)}) F^{*}_{ij}F_{j'k}}_{T_3}.
        \end{aligned}
    \end{align*}
    Again, we analyze each of the terms separately
    \begin{align}
    \label{eq:T1_norm_d2_alpha_d_alpha_d_r}
        \begin{aligned}
            \left\|T_{1_{i,jj',k}} \right\|_{2,2,1} 
            &= \sup_{\|\vv_1\|_2 = 1, \|V_2\|_F =1} \sum_{i=1}^m \left| 
                \frac{\phi''}{m} \left(W_{i k}v_{1_k}^{\left(l_2\right)}F_{i j}^{*}V_{2_{jj'}} F_{j' q} \aalpha_q^{\left(l_2-1\right)} \right)
            \right|\\
            &\leq \frac{\beta_{\phi}}{2m} \left( 
                \|W^{(l_2)}\vv_1 \|_{2}^2 + \|F^*V_2F\aalpha^{(l_2 - 1)}\|_{2}^2 
            \right)\\
            &\leq \frac{\beta_{\phi}}{2}\left( 
                \gamma_w^2 + \left( \gamma^{l_2 - 1} + |\phi(0)|\sum_{i=1}^{l_2 - 1} \gamma^{i-1}\right)^2
            \right)
        \end{aligned}
    \end{align}
    % \begin{align}
    %     \begin{aligned}
    %         \left\|\frac{\partial^2 \aalpha_i^{\left(l_2\right)}}{\partial \aalpha_k^{\left(l_2-1\right)} \partial r_{\left(j^{\prime}\right)}^{\left(l_2\right)}}\right\|_{2,2,1} = 
    %         \sup_{\|\vv_1\|_2 = 1, \|V_2\|_F =1} \sum_{i=1}^m \left(\left| \frac{\phi''}{m}
    %         \left( 
    %             W^{(l_2)}_{ik}v_{1_k} \cdot F^*_{ij}F_{j'q}\aalpha^{(l_2 - 1)}_q V_{2_{jj'}}
    %         \right) \right|
    %         + \left| \frac{\phi'}{\sqrt{m}} F^*_{ij}F_{j'k}v_{1_k}V_{2_{jj'}}\right|\right).
    %     \end{aligned}
    % \end{align}
    \begin{align}
    \label{eq:T2_norm_d2_alpha_d_alpha_d_r}
        \begin{aligned}
            \left\|T_{2_{i,jj',k}} \right\|_{2,2,1} 
            &= \sup_{\|\vv_1\|_2 = 1, \|V_2\|_F =1} \sum_{i=1}^m \left| 
                \frac{\phi''}{m} \left(
                    F^*_{ip}R^{(l_2)}_{pq}F_{qk}v_{1_k} F^{*}_{ij}V_{2_{jj'}}F_{j'q}\aalpha^{(l_2 - 1)}_q
                \right)
            \right|\\
            &\leq \sup_{\|\vv_1\|_2 = 1, \|V_2\|_F =1} 
            \frac{\beta_{\phi}}{2m}\left( 
                \| F^*R^{(l_2)}F\vv_1\|_2^2 + \| F^*V_2F\aalpha^{(l_2 - 1)}\|_2^2
            \right)\\
            &\leq \sup_{\|\vv_1\|_2 = 1, \|V_2\|_F =1} 
            \frac{\beta_{\phi}}{2m} \left( 
                \| F^*R^{(l_2)}F\|_2^2 \|\vv_1\|_2^2 +  \| F^*V_2F\|_2^2 \| \aalpha^{(l_2 - 1)}\|^2_{2}
            \right)\\
            &\overset{(a)}{\leq}
            \frac{\beta_{\phi}}{2m} \left( 
                \| R^{(l_2)}\|_2^2 +  \| \aalpha^{(l_2 - 1)}\|^2_{2}
            \right) \leq \frac{\beta_{\phi}}{2}\left( 
                \gamma_r^2 + \left( \gamma^{l_2 - 1} + |\phi(0)|\sum_{i=1}^{l_2 - 1} \gamma^{i-1}\right)^2
            \right)
        \end{aligned}
    \end{align}
    where $(a)$ follows, again, by exploiting the isometry of $F^*$ and $F$ with respect to the $L_2$ norm, and using $\|V_2\|_2 \leq \|V_2\|_F$. Finally,
    \begin{align}
    \label{eq:T3_norm_d2_alpha_d_alpha_d_r}
        \begin{aligned}
            \left\|T_{3_{i,jj',k}} \right\|_{2,2,1} 
            &= \sup_{\|\vv_1\|_2 = 1, \|V_2\|_F =1} \sum_{i=1}^m \left| 
                \frac{\phi'}{\sqrt{m}} F^{*}_{ij}V_{2_{jj'}}F_{j'k}v_{1_k}
            \right|\\
            &\leq
            \frac{1}{\sqrt{m}} \sup_{\|\vv_1\|_2 = 1, \|V_2\|_F =1}\sum^m_{i=1}\left|  
            (F^*V_2F\vv_1)_i
            \right|\\
            &\leq \sup_{\|\vv_1\|_2 = 1, \|V_2\|_F =1}
                \|F^*V_2F\vv_1 \|_2\\ 
            &\leq \sup_{\|\vv_1\|_2 = 1, \|V_2\|_F =1} 
            \|V_{2}\|_2\norm{\vv_1}_2 
            = 1~.
        \end{aligned}
    \end{align}
Combining \eqref{eq:T1_norm_d2_alpha_d_alpha_d_r}, \eqref{eq:T2_norm_d2_alpha_d_alpha_d_r} and \eqref{eq:T3_norm_d2_alpha_d_alpha_d_r}, we get
    \begin{align}
        \begin{aligned}
                \left\|
                \frac{\partial^2 \aalpha^{\left(l_2\right)}}{\partial \aalpha^{\left(l_2-1\right)} \partial \r^{\left(l_2\right)}}
            \right\|_{2,2,1} 
            &\leq \frac{\beta_{\phi}}{2} \left( 
                \gamma_w^2 + \gamma_r^2
            \right) + \beta_{\phi} \left( \gamma^{l_2 - 1} + |\phi(0)|\sum_{i=1}^{l_2 - 1} \gamma^{i-1}\right)^2 + 1\\
            &\leq \beta_{\phi} \left( 
                \gamma^2 + \left(
                    \gamma^{l_2 - 1} + |\phi(0)| \sum_{i=1}^{l_2 - 1} \gamma^{i-1}
                \right)^2
            \right) + 1~,
        \end{aligned}
    \end{align}
and finally we can upper bound the last quantity above as in~\eqref{eq:norm_d_alpha_l2_alpha_l2_minus_1_dr} and complete the proof.
%
%

For proving~\eqref{eq:norm_d_alpha_l_alpha_l_w_l} consider the following
\begin{align*}
        \begin{aligned}
            \left[ 
                \frac{\partial^2 \aalpha^{(l_2)}}{\partial \aalpha^{(l_2 - 1)}\partial \vw^{(l_2)}}
            \right]_{i,jj',k} =\left(
            \underbrace{\frac{\phi''(\widetilde{\aalpha}^{(l_2)}_i)  }{m}
                W^{(l_2)}_{ik}\aalpha^{(l_2-1)}_{j'}\delta_{ij}}_{T_1}
                + 
                \underbrace{\frac{\phi''(\widetilde{\aalpha}^{(l_2)}_i)}{m} F^*_{ip}R^{(l_2)}_{pq}F_{qk}\aalpha^{(l_2-1)}_{j'}\delta_{ij}}_{T_2}
            \right)
            + \underbrace{\frac{1}{\sqrt{m}} \phi'(\widetilde{\aalpha}^{(l_2)})\delta_{ij}\delta_{kj'}}_{T_3}
        \end{aligned}
    \end{align*}
    Then analyzing each term separately, we get
    \begin{align}
        \begin{aligned}
            \left\|T_{1_{i,jj',k}} \right\|_{2,2,1} 
            &= \sup_{\|\vv_1\|_2 = 1, \|V_2\|_F =1} \sum_{i=1}^m
            \left| 
                \frac{\phi''}{m} W^{(l_2)}_{ik}v_{1_k} V_{2_{ij'}}\aalpha^{(l_2-1)}_{j'}
            \right|\\
            &\leq \sup_{\|\vv_1\|_2 = 1, \|V_2\|_F =1}
            \frac{\beta_{\phi}}{2m} 
            \left( 
                \|W^{(l_2)}\|_2^2 \|\vv_1\|_2^2 + \|V_2\|_2^2 \|\aalpha^{(l_2-1)}\|_2^2
            \right)\\
            &=
            \frac{\beta_{\phi}}{2m} 
            \left( 
                \|W^{(l_2)}\|_2^2 + \|\aalpha^{(l_2-1)}\|_2^2
            \right) \leq\frac{\beta_{\phi}}{2}  
                \left(\gamma_w^2 + \left(\gamma^{l_1-1}+|\phi(0)| \sum_{i=1}^{l_1-1} \gamma^{i-1}\right)^2\right),
        \end{aligned}
    \end{align}
    \begin{align}
        \begin{aligned}
            \left\|T_{2_{i,jj',k}} \right\|_{2,2,1} 
            &= \sup_{\|\vv_1\|_2 = 1, \|V_2\|_F =1} \sum_{i=1}^m
            \left| 
                \frac{\phi''}{m} F^*_{ip}R^{(l_2)}_{pq}F_{qk}v_{1_k} V_{2_{ij'}}\aalpha^{(l_2-1)}_{j'}
            \right|\\
             &\leq \sup_{\|\vv_1\|_2 = 1, \|V_2\|_F =1}
             \frac{\beta_{\phi}}{2m}\left( 
                \| F^*R^{(l_2)}F\vv_1\|_2^2 + \| V_2\aalpha^{(l_2-1)}\|_2^2 
             \right)\\
             &\leq \sup_{\|\vv_1\|_2 = 1, \|V_2\|_F =1}
             \frac{\beta_{\phi}}{2m}\left( 
                \| F^*R^{(l_2)}F\|_2^2\norm{\vv_1}_2^2 + \norm{V_2}_2^2\| \aalpha^{(l_2-1)}\|_2^2 
             \right)\\
             &=            \frac{\beta_{\phi}}{2m}\left( 
                \| F^*R^{(l_2)}F\|_2^2 + \| \aalpha^{(l_2-1)}\|_2^2 
             \right)\\
             &\leq \frac{\beta_{\phi}}{2}  
                \left(\gamma_r^2 + \left(\gamma^{l_1-1}+|\phi(0)| \sum_{i=1}^{l_1-1} \gamma^{i-1}\right)^2\right),
        \end{aligned}
    \end{align}
    and, finally,
    \begin{align}
        \begin{aligned}
            \left\|T_{3_{i,jj'k}} \right\|_{2,2,1} 
            &= \sup_{\|\vv_1\|_2 = 1, \|V_2\|_F =1} \sum_{i=1}^m
            \left| 
                \frac{\phi'}{\sqrt{m}}V_{2_{ik}}v_{1_k}
            \right|\\
            &\leq \sup_{\|\vv_1\|_2 = 1, \|V_2\|_F =1} \sum_{i=1}^m
            \frac{1}{\sqrt{m}}\|\vv_1\|_2 \|V_{2,_{i,:}}\|_2\\
            &\leq \sup_{\|V_2\|_F =1}\sqrt{\sum_{i=1}^m
            \|V_{2,_{i,:}}\|_2^2}\\
            &= 1~.
        \end{aligned}
    \end{align}
    Hence, we have
    \begin{align}
        \begin{aligned}
            \left\|\frac{\partial^2\aalpha^{{(l_2)}}}{\partial \aalpha^{(l_2-1)}\partial{\mathbf{w}^{(l_2)}}} \right\|_{2,2,1}^2
            &\leq 
            \frac{\beta_{\phi}}{2}\left( \gamma_w^2 + \gamma_r^2\right) + \beta_{\phi} \left( 
                \gamma^{l_1 - 1} + |\phi(0)| \sum_{i=1}^{l_1 - 1} \gamma^{i-1}
        \right)^2+1\\
        &\leq 
            \beta_{\phi}\left( \gamma^2 +
                \left(\gamma^{l_1 - 1} + |\phi(0)| \sum_{i=1}^{l_1 - 1} \gamma^{i-1}\right)^2
        \right)+1~,
        \end{aligned}
    \end{align}
and finally we can upper bound the last quantity above as in~\eqref{eq:norm_d_alpha_l_alpha_l_w_l}.

We now focus on proving~\eqref{eq:norm_d_alpha_l_w_l_w_l}. Note that from \eqref{eq:FNO_predictor_app} we have
    \begin{align*}
        \left[\frac{\partial^2 \aalpha^{\left(l_1\right)}}{(\partial \w^{\left(l_1\right)})^2}\right]_{i,jj',kk'}
        =
        \frac{1}{m} \phi^{\prime \prime}
        \left(\tilde{\aalpha}^{\left(l_1\right)}\right) \cdot \aalpha_{j^{\prime}}^{\left(l_1 - 1\right)}
        \aalpha_{k^{\prime}}^{\left(l_1 - 1\right)}
        \delta_{i j}\delta_{ik}.
    \end{align*}
Now,
    \begin{align}
        \begin{aligned}
            &\left\| 
                \frac{\partial^2 \aalpha^{\left(l_1\right)}}{(\partial \w^{\left(l_1\right)})^2}
            \right\|_{2,2,1} \\
            &= 
            \sup_{\|V_1\|_F = 1, \|V_2\|_F = 1} \sum_{i=1}^m
            \left| 
                \frac{1}{m} \phi^{\prime \prime}(\tilde{\aalpha}^{(l_1)}_i) 
                V_{1_{i j^{\prime}}}\aalpha_{j^{\prime}}^{\left(l_1 -1\right)}
                V_{2_{i k^{\prime}}}\aalpha_{k^{\prime}}^{\left(l_1 -1\right)} 
            \right| \\ 
            &\leq
            \sup_{\|V_1\|_F = 1, \|V_2\|_F = 1}
            \frac{\beta_\phi}{m}
            \sum^m_{i=1}
            \left| 
            (V_1 \aalpha^{(l_1-1)})_i(V_2\aalpha^{(l_1-1)})_i \right| 
            \\
            &\leq
            \sup_{\|V_1\|_F = 1, \|V_2\|_F = 1}
            \frac{\beta_{\phi}}{2m} \left(\left\|V_1 \aalpha^{\left(l_1 - 1\right)}\right\|_2^2+\left\|V_2\aalpha^{(l_1-1)}\right\|_2^2\right) \\
            &\leq
            \frac{\beta_{\phi}}{m}\left\|\aalpha^{\left(l_1 - 1\right)}\right\|_2^2 \leq \beta_{\phi}
                \left(\gamma^{l_1-1}+|\phi(0)| \sum_{i=1}^{l_1-1} \gamma^{i-1}\right)^2.
        \end{aligned}
    \end{align}
Finally, we can upper bound the last quantity above as in~\eqref{eq:norm_d_alpha_l_w_l_w_l} and complete the proof.

Now, the last result to prove is~\eqref{eq:norm_d_alpha_l_r_l_r_l}. Note that from \eqref{eq:FNO_predictor_app} we have
    \begin{align*}
        \left[\frac{\partial^2 \aalpha^{\left(l_2\right)}}{(\partial \r^{\left(l_2\right)})^2}\right]_{i,jj',kk'}
        =
        \frac{1}{m} \phi^{\prime \prime}\left(\tilde{\aalpha}^{\left(l_2\right)}\right) \cdot 
        F_{i j}^* F_{j^{\prime} p} \aalpha_p^{\left(l_2-1\right)} 
        F_{i k}^* F_{k^{\prime} q} \aalpha_q^{\left(l_2-1\right)},
    \end{align*}
where we make use of the Einstein notation. Now,
    \begin{align}
        \begin{aligned}
            &\left\| 
                \frac{\partial^2 \aalpha^{\left(l_2\right)}}{(\partial \r^{\left(l_2\right)})^2}
            \right\|_{2,2,1} \\
            &= 
            \sup_{\|V_1\|_F = 1, \|V_2\|_F = 1} \sum_{i=1}^m
            \left| 
                \frac{1}{m} \phi^{\prime \prime}(\tilde{\aalpha}^{(l_2)}_i) F_{i j}^* V_{1_{j j^{\prime}}} F_{j^{\prime} p} \aalpha_p^{\left(l_2-1\right)}  
                F_{i k}^* V_{2_{k k^{\prime}}} F_{k^{\prime} q} \aalpha_q^{\left(l_2-1\right)}
            \right| \\ 
            &\leq
            \sup_{\|V_1\|_F = 1, \|V_2\|_F = 1}
            \frac{\beta_\phi}{m}
            \sum^m_{i=1}
            \left| 
            ((F^*V_1F)\aalpha^{(l_2-1)})_i ((F^*V_2F)\aalpha^{(l_2-1)})_i      \right| 
            \\
            &\leq
            \sup_{\|V_1\|_F = 1, \|V_2\|_F = 1}
            \frac{\beta_{\phi}}{2m} \left(\left\|F^* V_1 F \aalpha^{\left(l_2 - 1\right)}\right\|_2^2+\left\|F^* V_2 F \aalpha^{(l_2-1)}\right\|_2^2\right) \\
            &\leq
            \frac{\beta_{\phi}}{m}\left\|\aalpha^{\left(l_2 - 1\right)}\right\|_2^2 \leq \beta_{\phi}
                \left(\gamma^{l_2-1}+|\phi(0)| \sum_{i=1}^{l_2-1} \gamma^{i-1}\right)^2.
        \end{aligned}
    \end{align}
Finally, we can upper bound the last quantity above as in~\eqref{eq:norm_d_alpha_l_r_l_r_l} and complete the proof.
%    
%    This completes the proof of \eqref{eq:norm_d2_alpha_dw_dalpha}.
%    
%This completes the proof.
\qed

Now we upper bound the terms in equation~\eqref{eq:qwr2}. Thus, we obtain that, with probability at least $1-\frac{2(L+2)}{m}$,
\begin{equation}
\label{eq:Q_1}
 \cQ_{\infty}(f)=\max_{l\in[L+1]}\left\| \frac{\partial f}{\partial \aalpha^{(l)}} \right\|_{\infty}\overset{(a)}{\leq}
 \max_{l\in[L+1]}\frac{1}{\sqrt{m}}\gamma^{L+1-l}(1+\rho_1)\leq
 \frac{1}{\sqrt{m}}(1+\gamma^{L})(1+\rho_1)~,
\end{equation}
%
%
\begin{equation}
\label{eq:Q_2}
\cQ^{(w,r)}_2(f)\overset{(b)}{\leq}\max_{l\in[L+1]}(\gamma^{l-1}+|\phi(0)|\sum^{l-1}_{i=1}\gamma^{i-1})
\leq (1+\gamma^L)(1+L|\phi(0)|)~,
\end{equation}
and 
%
\begin{equation}
\label{eq:Q_3}
\cQ^{(w,r)}_{2,2,1}(f),\cQ^{(w)}_{2,2,1}(f),\cQ^{(r)}_{2,2,1}(f)  \overset{(c)}{\leq} 
2\beta_{\phi}(1+\gamma^L)^2(1+\gamma)^2(1+(1+L|\phi(0)|)^2)+1~,
\end{equation}
%
where (a) follows from a direct adaptation of the results from Section~A.6 in~\cite{banerjee2022restricted}, (b) follows from Lemma~\ref{lemm:gradient_alpha_params}, and (c) follows from Lemma~\ref{lemm:cross-w-r-bound}.

We now proceed to analyze the Hessian. We also recall that $\max_{l \in \{2,\dots,L+1\}} \left\| \frac{\partial \aalpha^{(l)}}{\partial \aalpha^{(l-1)}} \right\|_2 \leq \gamma$ from Lemma~\ref{lemm:firstDerivativeBoundFNO}. 

%====
We introduce some notation. Given an order-3 tensor $T \in \R^{d_1 \times d_2 \times d_3}$, we have that its first dimension has $d_1$ entries, the second has $d_2$ entries, and the third has $d_3$ entries. Consider the matrices $X\in\R^{k_1\times d_1}$, $Y\in\R^{k_2\times d_2}$, and $Z\in\R^{k_3\times d_3}$. We use the notation $(X)(Y)T(Z)\in\R^{k_1\times k_2\times k_3}$ to denote $X$ multiplying $A$ long its first dimension, $Y$ along its second dimension, and $Z$ along its third dimension. We use the notation $(X)T(Z)\in\R^{k_1\times d_2\times k_3}$ to denote $X$ multiplying $A$ long its first dimension and $Z$ along its third dimension.
%====



\paragraph{Off-Diagonal Blocks.} For the off-diagonal blocks, we focus on bounding $\| H_{w,r}^{(l_1,l_2)} \|_2$ for (Case 1.A) $l_1 \leq l_2$, (Case 1.B) $l_2 \leq l_1$. Further, we bound (Case 2.A) $\| H_{v,w}^{(l_1)} \|_2$ and (Case 2.B) $\| H_{v,r}^{(l_2)} \|_2$.


{\bf Case 1.A:} $2 \leq l_1 \leq l_2 \leq L+1$. By building on the form of the gradient, we have 
\begin{align*}
H_{w,r}^{(l_1,l_2)} & = \frac{\partial^2 \aalpha^{(l_1)}}{\partial \w^{(l_1)} \partial \r^{(l_1)}} \frac{\partial f}{\partial \aalpha^{(l_1)}} \1_{[l_1 =l_2]} +  \1_{[l_1 <l_2]}\left( \frac{\partial \aalpha^{(l_1)}}{\partial \w^{(l_1)}}  \prod_{l'=l_1+1}^{l_2-1} \frac{\partial \aalpha^{(l')}}{\partial \aalpha^{(l'-1)}} \right) \frac{\partial^2 \aalpha^{(l_2)}}{\partial \aalpha^{(l_2-1)} \partial \r^{(l_2)}} \left(\frac{\partial f}{\partial \aalpha^{(l_2)}} \right) \\
& ~~~~ + \1_{[l_1 <l_2]}\sum_{l=l_2+1}^{L+1} \left( \frac{\partial \aalpha^{(l_1)}}{\partial \w^{(l_1)}} \prod_{l'=l_1+1}^{l-1} \frac{\partial \aalpha^{(l')}}{\partial \aalpha^{(l'-1)}} \right)  \left( \frac{\partial \aalpha^{(l_2)}}{\partial \r^{(l_2)}} \prod_{l'=l_2+1}^{l-1} \frac{\partial \aalpha^{(l')}}{\partial \aalpha^{(l'-1)}} \right) \frac{\partial^2 \aalpha^{(l)}}{(\partial \aalpha^{(l-1)})^2}  \left(\frac{\partial f}{\partial \aalpha^{(l)}} \right) ~. 
\end{align*}
Then,
\begin{align*}
\| H_{w,r}^{(l_1,l_2)} \|_2 
& \leq  \left\| \frac{\partial^2 \aalpha^{(l_1)}}{\partial \w^{(l_1)}  \partial \r^{(l_1)}} \right\|_{2,2,1} \left\| \frac{\partial f}{\partial \aalpha^{(l_1)}} \right\|_{\infty}\1_{[l_1=l_2]} \\
& ~~~ +\1_{[l_1 <l_2]}  \left\| \frac{\partial \aalpha^{(l_1)}}{\partial \w^{(l_1)}} \right\|_2 \prod_{l'=l_1+1}^{l_2-1} \left\| \frac{\partial \aalpha^{(l')}}{\partial \aalpha^{(l'-1)}} \right\|_2 \left\| \frac{\partial^2 \aalpha^{(l_2)}}{\partial \aalpha^{(l_2-1)} \partial \r^{(l_2)}} \right\|_{2,2,1} \left\| \frac{\partial f}{\partial \aalpha^{(l_2)}} \right\|_{\infty} \\
& ~~~~ + \1_{[l_1 <l_2]}\sum_{l=l_2+1}^{L+1} \left( \left\| \frac{\partial \aalpha^{(l_1)}}{\partial \w^{(l_1)}} \right\|_2  \prod_{l'=l_1+1}^{l-1} \left\| \frac{\partial \aalpha^{(l')}}{\partial \aalpha^{(l'-1)}} \right\|_2 \right)  \left( \left\| \frac{\partial \aalpha^{(l_2)}}{\partial \r^{(l_2)}} \right\|_2 \prod_{l'=l_2+1}^{l-1} \left\| \frac{\partial \aalpha^{(l')}}{\partial \aalpha^{(l'-1)}} \right\|_{2} \right)\\
& ~~~~ \times\left\| \frac{\partial^2 \aalpha^{(l)}}{(\partial \aalpha^{(l-1)})^2} \right\|_{2,2,1} \left\| \frac{\partial f}{\partial \aalpha^{(l)}} \right\|_{\infty} \\
& \leq  \left\| \frac{\partial^2 \aalpha^{(l_1)}}{\partial \w^{(l_1)}  \partial \r^{(l_1)}} \right\|_{2,2,1} \left\| \frac{\partial f}{\partial \aalpha^{(l_1)}} \right\|_{\infty} \1_{[l_1 =l_2]}\\
& ~~~~ + \1_{[l_1 <l_2]} \gamma^{l_2-l_1-1} \left\| \frac{\partial \aalpha^{(l_1)}}{\partial \w^{(l_1)}} \right\|_2 \left\| \frac{\partial^2 \aalpha^{(l_2)}}{\partial \aalpha^{(l_2-1)} \partial \r^{(l_2)}} \right\|_{2,2,1} \left\| \frac{\partial f}{\partial \aalpha^{(l_2)}} \right\|_{\infty} \\
& ~~~~ + \1_{[l_1 <l_2]}\sum_{l=l_2+1}^{L+1} \gamma^{2l - l_2 - l_1-2} \left\| \frac{\partial \aalpha^{(l_1)}}{\partial \w^{(l_1)}} \right\|_2  \left\| \frac{\partial \aalpha^{(l_2)}}{\partial \r^{(l_2)}} \right\|_2  \left\| \frac{\partial^2 \aalpha^{(l)}}{(\partial \aalpha^{(l-1)})^2} \right\|_{2,2,1} \left\| \frac{\partial f}{\partial \aalpha^{(l)}} \right\|_{\infty}~.
\end{align*}
Then, based on the definitions in \eqref{eq:qwr2}, we have
\begin{align*}
 \| H_{w,r}^{(l_1,l_2)} \|_2 &\leq (L+1)(1+\gamma^{2L}) \cQ^{(w,r)}_{2,2,1}(f) \cQ_{\infty}(f)\\
 &\overset{(a)}{\leq}
 \frac{(L+1)(1+\rho_1)}{\sqrt{m}}(1+\gamma^{2L})^2(2\beta_{\phi}(1+\gamma^L)^2(1+\gamma)^2(1+(1+L|\phi(0)|)^2)+1)
 ~,
\end{align*}
where for (a) we used equations~\eqref{eq:Q_1} and~\eqref{eq:Q_3}. 

{\bf Case 1.B:} $2 \leq l_2 \leq l_1 \leq L+1$. By building on the form of the gradient, we have 
\begin{align*}
H_{w,r}^{(l_1,l_2)} & = \frac{\partial^2 \aalpha^{(l_2)}}{\partial \w^{(l_2)} \partial \r^{(l_2)}} \frac{\partial f}{\partial \aalpha^{(l_2)}} \1_{[l_1 =l_2]} + \1_{[l_2 <l_1]} \left( \frac{\partial \aalpha^{(l_2)}}{\partial \r^{(l_2)}}  \prod_{l'=l_2+1}^{l_1-1} \frac{\partial \aalpha^{(l')}}{\partial \aalpha^{(l'-1)}} \right) \frac{\partial^2 \aalpha^{(l_1)}}{\partial \aalpha^{(l_1-1)} \partial \w^{(l_1)}} \left(\frac{\partial f}{\partial \aalpha^{(l_1)}} \right) \\
& ~~~~ + \1_{[l_2 <l_1]}\sum_{l=l_1+1}^{L+1} \left( \frac{\partial \aalpha^{(l_2)}}{\partial \r^{(l_2)}} \prod_{l'=l_2+1}^{l-1} \frac{\partial \aalpha^{(l')}}{\partial \aalpha^{(l'-1)}} \right)  \left( \frac{\partial \aalpha^{(l_1)}}{\partial \w^{(l_1)}} \prod_{l'=l_1+1}^{l-1} \frac{\partial \aalpha^{(l')}}{\partial \aalpha^{(l'-1)}} \right) \frac{\partial^2 \aalpha^{(l)}}{(\partial \aalpha^{(l-1)})^2}  \left(\frac{\partial f}{\partial \aalpha^{(l)}} \right) ~. 
\end{align*}
Then,
\begin{align*}
\| H_{w,r}^{(l_1,l_2)} \|_2 
& \leq  \left\| \frac{\partial^2 \aalpha^{(l_2)}}{\partial \w^{(l_2)}  \partial \r^{(l_2)}} \right\|_{2,2,1} \left\| \frac{\partial f}{\partial \aalpha^{(l_2)}} \right\|_{\infty} \1_{[l_1=l_2]}\\
& ~~~~ + \1_{[l_2 <l_1]} \left\| \frac{\partial \aalpha^{(l_2)}}{\partial \r^{(l_2)}} \right\|_2 \prod_{l'=l_2+1}^{l_1-1} \left\| \frac{\partial \aalpha^{(l')}}{\partial \aalpha^{(l'-1)}} \right\|_2 \left\| \frac{\partial^2 \aalpha^{(l_1)}}{\partial \aalpha^{(l_1-1)} \partial \w^{(l_1)}} \right\|_{2,2,1} \left\| \frac{\partial f}{\partial \aalpha^{(l_1)}} \right\|_{\infty} \\
& ~~~~ + \1_{[l_2 <l_1]}\sum_{l=l_1+1}^{L+1} \left( \left\| \frac{\partial \aalpha^{(l_2)}}{\partial \r^{(l_2)}} \right\|_2  \prod_{l'=l_2+1}^{l-1} \left\| \frac{\partial \aalpha^{(l')}}{\partial \aalpha^{(l'-1)}} \right\|_2 \right)  \left( \left\| \frac{\partial \aalpha^{(l_1)}}{\partial \w^{(l_1)}} \right\|_2 \prod_{l'=l_1+1}^{l-1} \left\| \frac{\partial \aalpha^{(l')}}{\partial \aalpha^{(l'-1)}} \right\|_{2} \right)\\
& ~~~~ \times \left\| \frac{\partial^2 \aalpha^{(l)}}{(\partial \aalpha^{(l-1)})^2} \right\|_{2,2,1} \left\| \frac{\partial f}{\partial \aalpha^{(l)}} \right\|_{\infty} \\
& \leq  \left\| \frac{\partial^2 \aalpha^{(l_2)}}{\partial \w^{(l_2)}  \partial \r^{(l_2)}} \right\|_{2,2,1} \left\| \frac{\partial f}{\partial \aalpha^{(l_2)}} \right\|_{\infty} \1_{[l_1 =l_2]}\\
& ~~~~ + \1_{[l_2 <l_1]} \gamma^{l_1-l_2-1} \left\| \frac{\partial \aalpha^{(l_2)}}{\partial \r^{(l_2)}} \right\|_2 \left\| \frac{\partial^2 \aalpha^{(l_1)}}{\partial \aalpha^{(l_1-1)} \partial \w^{(l_1)}} \right\|_{2,2,1} \left\| \frac{\partial f}{\partial \aalpha^{(l_1)}} \right\|_{\infty} \\
& ~~~~ + \1_{[l_2 <l_1]}\sum_{l=l_1+1}^{L+1} \gamma^{2l - l_1 - l_2-2} \left\| \frac{\partial \aalpha^{(l_2)}}{\partial \r^{(l_2)}} \right\|_2  \left\| \frac{\partial \aalpha^{(l_1)}}{\partial \w^{(l_1)}} \right\|_2  \left\| \frac{\partial^2 \aalpha^{(l)}}{(\partial \aalpha^{(l-1)})^2} \right\|_{2,2,1} \left\| \frac{\partial f}{\partial \aalpha^{(l)}} \right\|_{\infty} ~.
\end{align*}

Then, the upper bound is similar to the case {\bf Case 1.A},
\begin{align*}
 \| H_{w,r}^{(l_1,l_2)} \|_2 &\leq (L+1)(1+\gamma^{2L}) \cQ^{(w,r)}_{2,2,1}(f) \cQ_{\infty}(f)\\
 &\overset{(a)}{\leq}
 \frac{(L+1)(1+\rho_1)}{\sqrt{m}}(1+\gamma^{2L})^2(2\beta_{\phi}(1+\gamma^L)^2(1+\gamma)^2(1+(1+L|\phi(0)|)^2)+1)
 ~.
\end{align*}
%where for (a) we used equations~\eqref{eq:Q_1} and~\eqref{eq:Q_3}. 

{\bf Case 2.A:} $1 \leq l_1 \leq L+1$. For Hessian terms involving $(w,v)$, since $\frac{\partial f}{\partial \v} = \frac{1}{\sqrt{m}} \aalpha^{(L+1)}$, we have 
\begin{align*}
H_{w,v}^{(l_1)} = \frac{1}{\sqrt{m}}  \frac{\partial \aalpha^{(L+1)}}{\partial \w^{(l_1)}} = \frac{1}{\sqrt{m}} \left( \frac{\partial \aalpha^{(l_1)}}{\partial \w^{(l_1)}} \prod_{l'=l_1+1}^{L+1} \frac{\partial \aalpha^{(l')}}{\partial \aalpha^{(l'-1)}}  \right)~.
\end{align*}
Then,
\begin{align*}
 \| H_{w,v}^{(l_1,L+1)} \|_2 \leq \frac{1}{\sqrt{m}} \left\| \frac{\partial \aalpha^{(l_1)}}{\partial \w^{(l_1)}} \right\|_2 \prod_{l'=l_1+1}^{L+1} \left\| \frac{\partial \aalpha^{(l')}}{\partial \aalpha^{(l'-1)}} \right\|_2 \leq \frac{1}{\sqrt{m}} \gamma^L \cQ_2^{(w,r)}(f)\overset{(a)}{\leq} \frac{1}{\sqrt{m}}\gamma^L(1+\gamma^L)(1+L|\phi(0)|)~,   
\end{align*}
where (a) follows from equation~\eqref{eq:Q_2}.

{\bf Case 2.B:} $2 \leq l_2 \leq L+1$. For Hessian terms involving $(r,v)$, since $\frac{\partial f}{\partial \v} = \frac{1}{\sqrt{m}} \aalpha^{(L+1)}$,  we have 
\begin{align*}
H_{r,v}^{(l_2)} = \frac{1}{\sqrt{m}}  \frac{\partial \aalpha^{(L+1)}}{\partial \r^{(l_2)}}
= \frac{1}{\sqrt{m}} \left( \frac{\partial \aalpha^{(l_2)}}{\partial \r^{(l_2)}} \prod_{l'=l_2+1}^{L+1} \frac{\partial \aalpha^{(l')}}{\partial \aalpha^{(l'-1)}}  \right)~.
\end{align*}
Then,
\begin{align*}
 \| H_{r,v}^{(l_2)} \|_2 \leq \frac{1}{\sqrt{m}} \left\| \frac{\partial \aalpha^{(l_2)}}{\partial \r^{(l_2)}} \right\|_2 \prod_{l'=l_2+1}^{L+1} \left\| \frac{\partial \aalpha^{(l')}}{\partial \aalpha^{(l'-1)}} \right\|_2 \overset{(a)}{\leq} \frac{1}{\sqrt{m}}\gamma^L(1+\gamma^L)(1+L|\phi(0)|)~,   
\end{align*}
where (a) follows from equation~\eqref{eq:Q_2}.



\paragraph{Diagonal Blocks.} 
For the diagonal blocks, we focus only on bounding (Case 3.A) $\| H_{w}^{(l_1,l_2)} \|_2$ and (Case 3.B) $\| H_{r}^{(l_1,l_2)} \|_2$ for $l_1 \leq l_2$, since the case $l_2 \leq l_1$ is just symmetrical and will have the same bounds. 

{\bf Case 3.A:} $1 \leq l_1 \leq l_2 \leq L+1$. By building on the form of the gradient, we have 
\begin{align*}
H_{w}^{(l_1,l_2)} & = \frac{\partial^2 \aalpha^{(l_1)}}{(\partial \w^{(l_1)})^2} \frac{\partial f}{\partial \aalpha^{(l_1)}} \1_{[l_1 =l_2]} +  \1_{[l_1 <l_2]}\left( \frac{\partial \aalpha^{(l_1)}}{\partial \w^{(l_1)}}  \prod_{l'=l_1+1}^{l_2-1} \frac{\partial \aalpha^{(l')}}{\partial \aalpha^{(l'-1)}} \right) \frac{\partial^2 \aalpha^{(l_2)}}{\partial \aalpha^{(l_2-1)} \partial \w^{(l_2)}} \left(\frac{\partial f}{\partial \aalpha^{(l_2)}} \right) \\
& ~~~~ + \1_{[l_1 <l_2]}\sum_{l=l_2+1}^{L+1} \left( \frac{\partial \aalpha^{(l_1)}}{\partial \w^{(l_1)}} \prod_{l'=l_1+1}^{l-1} \frac{\partial \aalpha^{(l')}}{\partial \aalpha^{(l'-1)}} \right)  \left( \frac{\partial \aalpha^{(l_2)}}{\partial \w^{(l_2)}} \prod_{l'=l_2+1}^{l-1} \frac{\partial \aalpha^{(l')}}{\partial \aalpha^{(l'-1)}} \right) \frac{\partial^2 \aalpha^{(l)}}{(\partial \aalpha^{(l-1)})^2}  \left(\frac{\partial f}{\partial \aalpha^{(l)}} \right) ~. 
\end{align*}
Then,
\begin{align*}
\| H_{w}^{(l_1,l_2)} \|_2 
& \leq  \left\| \frac{\partial^2 \aalpha^{(l_1)}}{(\partial \w^{(l_1)})^2} \right\|_{2,2,1} \left\| \frac{\partial f}{\partial \aalpha^{(l_1)}} \right\|_{\infty} \1_{[l_1=l_2]}\\
& ~~~~ + \1_{[l_1 <l_2]} \left\| \frac{\partial \aalpha^{(l_1)}}{\partial \w^{(l_1)}} \right\|_2 \prod_{l'=l_1+1}^{l_2-1} \left\| \frac{\partial \aalpha^{(l')}}{\partial \aalpha^{(l'-1)}} \right\|_2 \left\| \frac{\partial^2 \aalpha^{(l_2)}}{\partial \aalpha^{(l_2-1)} \partial \w^{(l_2)}} \right\|_{2,2,1} \left\| \frac{\partial f}{\partial \aalpha^{(l_2)}} \right\|_{\infty} \\
& ~~~~ + \1_{[l_1 <l_2]}\sum_{l=l_2+1}^{L+1} \left( \left\| \frac{\partial \aalpha^{(l_1)}}{\partial \w^{(l_1)}} \right\|_2  \prod_{l'=l_1+1}^{l-1} \left\| \frac{\partial \aalpha^{(l')}}{\partial \aalpha^{(l'-1)}} \right\|_2 \right)  \left( \left\| \frac{\partial \aalpha^{(l_2)}}{\partial \w^{(l_2)}} \right\|_2 \prod_{l'=l_2+1}^{l-1} \left\| \frac{\partial \aalpha^{(l')}}{\partial \aalpha^{(l'-1)}} \right\|_{2} \right)\\
& ~~~~ \times\left\| \frac{\partial^2 \aalpha^{(l)}}{(\partial \aalpha^{(l-1)})^2} \right\|_{2,2,1} \left\| \frac{\partial f}{\partial \aalpha^{(l)}} \right\|_{\infty} \\
& \leq  \left\| \frac{\partial^2 \aalpha^{(l_1)}}{(\partial \w^{(l_1)})^2} \right\|_{2,2,1} \left\| \frac{\partial f}{\partial \aalpha^{(l_1)}} \right\|_{\infty}\1_{[l_1=l_2]}\\
& ~~~~ +  \1_{[l_1 <l_2]}\gamma^{l_2-l_1-1} \left\| \frac{\partial \aalpha^{(l_1)}}{\partial \w^{(l_1)}} \right\|_2 \left\| \frac{\partial^2 \aalpha^{(l_2)}}{\partial \aalpha^{(l_2-1)} \partial \w^{(l_2)}} \right\|_{2,2,1} \left\| \frac{\partial f}{\partial \aalpha^{(l_2)}} \right\|_{\infty} \\
& ~~~~ + \1_{[l_1 <l_2]}\sum_{l=l_2+1}^{L+1} \gamma^{2l - l_2 - l_1-2} \left\| \frac{\partial \aalpha^{(l_1)}}{\partial \w^{(l_1)}} \right\|_2  \left\| \frac{\partial \aalpha^{(l_2)}}{\partial \w^{(l_2)}} \right\|_2  \left\| \frac{\partial^2 \aalpha^{(l)}}{(\partial \aalpha^{(l-1)})^2} \right\|_{2,2,1} \left\| \frac{\partial f}{\partial \aalpha^{(l)}} \right\|_{\infty}~.
\end{align*}
Then, based on the definitions in \eqref{eq:qwr2}, we have
\begin{align*}
 \| H_{(w)}^{(l_1,l_2)} \|_2 &\leq (L+1)(1+\gamma^{2L}) \cQ^{(w)}_{2,2,1}(f) \cQ_{\infty}(f)\\
 &\overset{(a)}{\leq}
 \frac{(L+1)(1+\rho_1)}{\sqrt{m}}(1+\gamma^{2L})^2(2\beta_{\phi}(1+\gamma^L)^2(1+\gamma)^2(1+(1+L|\phi(0)|)^2)+1)
 ~,
\end{align*}
where for (a) we used equations~\eqref{eq:Q_1} and~\eqref{eq:Q_3}.


{\bf Case 3.B:} $2 \leq l_1 \leq l_2 \leq L+1$. By building on the form of the gradient, we have 
\begin{align*}
H_{r}^{(l_1,l_2)} & = \frac{\partial^2 \aalpha^{(l_1)}}{(\partial \r^{(l_1)})^2} \frac{\partial f}{\partial \aalpha^{(l_1)}} \1_{[l_1 =l_2]} +  \1_{[l_1 <l_2]}\left( \frac{\partial \aalpha^{(l_1)}}{\partial \r^{(l_1)}}  \prod_{l'=l_1+1}^{l_2-1} \frac{\partial \aalpha^{(l')}}{\partial \aalpha^{(l'-1)}} \right) \frac{\partial^2 \aalpha^{(l_2)}}{\partial \aalpha^{(l_2-1)} \partial \r^{(l_2)}} \left(\frac{\partial f}{\partial \aalpha^{(l_2)}} \right) \\
& ~~~~ + \1_{[l_1 <l_2]}\sum_{l=l_2+1}^{L+1} \left( \frac{\partial \aalpha^{(l_1)}}{\partial \r^{(l_1)}} \prod_{l'=l_1+1}^{l-1} \frac{\partial \aalpha^{(l')}}{\partial \aalpha^{(l'-1)}} \right)  \left( \frac{\partial \aalpha^{(l_2)}}{\partial \r^{(l_2)}} \prod_{l'=l_2+1}^{l-1} \frac{\partial \aalpha^{(l')}}{\partial \aalpha^{(l'-1)}} \right) \frac{\partial^2 \aalpha^{(l)}}{(\partial \aalpha^{(l-1)})^2}  \left(\frac{\partial f}{\partial \aalpha^{(l)}} \right) ~. 
\end{align*}
Then, we can obtain  prove the following upper bound in a similar way to Case 3.A based on the definitions in \eqref{eq:qwr2},
\begin{align*}
 \| H_{(r)}^{(l_1,l_2)} \|_2 &\leq (L+1)(1+\gamma^{2L}) \cQ^{(r)}_{2,2,1}(f) \cQ_{\infty}(f)\\
 &\overset{(a)}{\leq}
 \frac{(L+1)(1+\rho_1)}{\sqrt{m}}(1+\gamma^{2L})^2(2\beta_{\phi}(1+\gamma^L)^2(1+\gamma)^2(1+(1+L|\phi(0)|)^2)+1)
 ~,
\end{align*}
where for (a) we used equations~\eqref{eq:Q_1} and~\eqref{eq:Q_3}.


Putting all the shown results back in~\eqref{eq:Hessian_big}, we prove equation~\eqref{eq:hessianBoundG_fg_FNO}. \pcedit{We also note that all the constants in the Hessian bound depend on $\sigma_{1,w}$, $\sigma_{1,r}$, the depth $L$, and the radii $\rho_w$, $\rho_r$, $\rho_1$, and $\rho_2$. This dependence of this bound reduces to the depth and the radii and becomes polynomial whenever $\gamma\leq 1$, which is equivalent to $\sigma_{1,w}+\sigma_{1,r}\leq 1-\frac{\rho_w+\rho_r}{\sqrt{m}}$.} 



Now, we focus on proving the rest of equations in Lemma~\ref{lemm:hessgradbounds-FNO}, namely, equations~\eqref{eq:gradientBoundG_fg_FNO} and~\eqref{eq:predictorBoundG_fg_FNO}. 

{\bf Gradient and predictor bounds.} 
%
We observe that for $l\in[L]$,
$
\frac{\partial f}{\partial \w^{(l)}} = \frac{\partial \aalpha^{(l)}}{\partial \w^{(l)}}  \left( \prod_{l'=l}^L \frac{\partial \aalpha^{(l'+1)}}{\partial \aalpha^{(l')}} \right) \frac{\partial f}{\partial \aalpha^{(L+1)}}
$, and so


\begin{align*}
\left\|\frac{\partial f}{\partial \w^{(l)}}\right\|_2 & \leq \left\| \frac{\partial \aalpha^{(l)}}{\partial \w^{(l)}}\right\|_2 \gamma^{L-l+1} \left\|\frac{\partial f}{\partial \aalpha^{(L+1)}}\right\|_2\\
&\leq 
\left\| \frac{\partial \aalpha^{(l)}}{\partial \w^{(l)}}\right\|_2 \gamma^{L-l+1} \frac{1}{\sqrt{m}}(1+\rho_1)\\
&\leq 
(1+\gamma^L)(1+L|\phi(0)|)\gamma^{L} \frac{1}{\sqrt{m}}(1+\rho_1)~,
\end{align*}
where the last inequality follows from Lemma~\ref{lemm:gradient_alpha_params}.

We also have that
\begin{align*}
\left\|\frac{\partial f}{\partial \w^{(L+1)}}\right\|_2 & =
\left\|\frac{\partial \aalpha^{(L+1)}}{\partial \w^{(L+1)}} \frac{\partial f}{\partial \aalpha^{(L+1)}}\right\|_2\\
& =
\left\|\frac{\partial \aalpha^{(L+1)}}{\partial \w^{(L+1)}}\right\|_2\left\| \frac{\partial f}{\partial \aalpha^{(L+1)}}\right\|_2\\
&\leq (1+\gamma^L)(1+L|\phi(0)|)\frac{1}{\sqrt{m}}(1+\rho_1)~.
\end{align*}

Similarly, we can obtain for $l_2\in\{2,\dots,L\}$,
\begin{align*}
\left\|\frac{\partial f}{\partial \r^{(l_2)}}\right\|_2 &\leq 
(1+\gamma^L)(1+L|\phi(0)|)\gamma^{L} \frac{1}{\sqrt{m}}(1+\rho_1)~,
\end{align*}
and
\begin{align*}
\left\|\frac{\partial f}{\partial \r^{(L+1)}}\right\|_2 
&\leq (1+\gamma^L)(1+L|\phi(0)|)\frac{1}{\sqrt{m}}(1+\rho_1)~.
\end{align*}
Using all these derivations, 
\begin{align*}
\norm{\nabla_{\vtheta}f}_2^2 &=\sum^{L+1}_{l=1}\left\|\frac{\partial f}{\partial \w^{(l)}}\right\|_2^2 + \sum^{L+1}_{l=2}\left\|\frac{\partial f}{\partial \r^{(l)}}\right\|_2^2\\
&\leq \frac{2}{m}(L+1)(1+\gamma^L)^2(1+L|\phi(0)|)^2(1+\rho_1)^2~,
\end{align*}
which finishes the proof for equation~\eqref{eq:gradientBoundG_fg_FNO}.

Now, 
\begin{align*}
|f| &=\left|\frac{1}{\sqrt{m}}\v^\top\aalpha^{(L+1)}\right|\\
&\leq \frac{1}{\sqrt{m}}\norm{\v}_2\norm{\aalpha^{(L+1)}}_2\\
&\leq (1+\rho_1)(1+\gamma^L)(1+L|\phi(0)|)~,
\end{align*}
which finishes the proof for equation~\eqref{eq:predictorBoundG_fg_FNO}. 
\pcedit{Again, we notice that all these bounds have a polynomial dependence on the depth $L$, and the radii $\rho_w$, $\rho_r$, $\rho_1$, and $\rho_2$ whenever $\gamma\leq 1$, i.e., whenever $\sigma_{1,w}+\sigma_{1,r}\leq 1-\frac{\rho_w+\rho_r}{\sqrt{m}}$.}

Thus, we finish the proof for Lemma~\ref{lemm:hessgradbounds-FNO}.

\subsection{RSC and Smoothness Results}

Using the results from the previous section, we immediately obtain the RSC and smoothness results. 

\RSCLossFNO*
%
\begin{proof}
We start by proving the first part of the theorem's statement. 
We immediately see that, 
since $B^t_{\kappa}\subset B^{\mathrm{Euc}}_{\rho_w,\rho_r\rho_1}(\vtheta_0)$, we satisfy Condition~\ref{cond:rsc}(a). We now need to satisfy Condition~\ref{cond:rsc}(b). For this, we proceed to show the existence of an element $\vtheta' \in B^t_{\kappa}$ that is an element of the set $Q^{t}_{\kappa}$ as in Definition~\ref{defn:qset_FNO}, i.e., satisfies
\begin{equation}
|\cos(\vtheta'-\vtheta_t, \nabla_\vtheta\bar{G}_t)| \geq \kappa~,
\label{eq:kap-fno}
\end{equation}
and that also satisfies the following two conditions:
\begin{enumerate}[{Condition} (A):]%[(a)]
\item  $\|\vtheta' - \vtheta_{t} \|_2 = \epsilon$ for some $\epsilon< \frac{2 \norm{\nabla_{\vtheta} \cL(\vtheta_t)}_2 \sqrt{1-\kappa^2}}{\beta}$; and \label{cond-1-fno}
\item the angle $\nu'$ between $(\vtheta' - \vtheta_{t})$ and $-\nabla_{\vtheta} \cL(\vtheta_t)$ is acute, so that $\cos(\nu') > 0$. \label{cond-2-fno}\end{enumerate}
%

To show the existence of such element $\vtheta' \in B_t$, we propose two possible constructions:
\begin{enumerate}[{Choice} (A):]
\item  
If the points $\vtheta_{t+1}$, $\nabla_\vtheta\bar{G}_t + \vtheta_{t}$, and $\vtheta_{t}$ are not collinear, then they define a hyperplane $\mathcal{P}$ that contains the vectors $\nabla_\vtheta\bar{G}_t$ and $-\nabla_{\vtheta} \cL(\vtheta_t)$ (recall that $\vtheta_{t+1} - \vtheta_{t}=-\nabla_{\vtheta}\cL(\vtheta_t)$ by gradient descent). We choose $\vtheta'$ such that the vector 
$\vtheta'-\vtheta_{t}$ 
%(or, equivalently, the vector $\bar{\vtheta}_f$)
lies in 
$\mathcal{P}$ 
with 
$\cos(\vtheta'-\vtheta_{t},\nabla_\vtheta\bar{G}_t)=\kappa$ (i.e., it satisfies condition~\eqref{eq:kap-fno} with equality) while simultaneously satisfying Condition~\eqref{cond-2-fno}.
%
%
If the points $\vtheta_{t+1}$, $\nabla_\vtheta\bar{G}_t + \vtheta_{t}$, and $\vtheta_{t}$ are collinear, we choose $\vtheta'$ such that it is not collinear with these points, thus defining a hyperplane $\mathcal{P}$ 
with these other three points, 
and such that 
$\vtheta'$ is also taken so that $\cos(\vtheta'-\vtheta_{t},\nabla_\vtheta\bar{G}_t)=\kappa$ while simultaneously satisfying Condition~\eqref{cond-2-fno}.
%

Thus far we have only defined \emph{angle} (or \emph{direction}) conditions on the vector $\vtheta'-\vtheta_{t}$, and so there could be an infinite number of values for $\vtheta'_f$ satisfying such angle conditions without $\vtheta'$ belonging to the set $B^{\mathrm{Euc}}_{\rho_w,\rho_r,\rho_1}(\vtheta_0)$ nor $\vtheta'$ satisfying Condition~\eqref{cond-1-fno}. To determine the feasible values for $\vtheta'$, we observe that $\vtheta_t$ is \emph{strictly inside} the set $B^{\mathrm{Euc}}_{\rho_w,\rho_r,\rho_1}(\vtheta_0)$ by Assumption~\ref{asmp:iter-2}, and so 
%
$\vtheta'$ can be taken arbitrarily close to $\vtheta_{t}$ so that $\vtheta'\in B^{\mathrm{Euc}}_{\rho_w,\rho_r,\rho_1}(\vtheta_0)$ and Condition~\eqref{cond-1-fno} is satisfied. 
%

We remark that, regardless of the collinearity of the points 
$\vtheta_{t+1}$, $\nabla_\vtheta\bar{G}_t + \vtheta_{t}$, and $\vtheta_{t}$, hyperplane $\mathcal{P}$ contains the vectors $\vtheta'-\vtheta_{t}$, $\nabla_\vtheta\bar{G}_t$, and $-\nabla_{\vtheta}\cL(\vtheta_t)$, all sharing its origin at $\vtheta_{f}\in\mathcal{P}$. \label{ch-A-fno}
%
%; or
\item  
%
%
We choose $\vtheta'$ as in Choice~\eqref{ch-A-don} but with $\nabla_\vtheta\bar{G}_t$ replaced by $-\nabla_\vtheta\bar{G}_t$.
%
\label{ch-B-fno}
%
%
\end{enumerate}
We immediately notice that $\vtheta'$ defined by either Choice~\eqref{ch-A-fno} or Choice~\eqref{ch-B-fno} satisfies 
$\vtheta'\in Q^t_\kappa \cap B^{\mathrm{Euc}}_{\rho_w,\rho_r,\rho_1}(\vtheta_0)
$. To make $\vtheta'$ belong to the set $B^t_\kappa$, we need to find a radius $\rho_2$ such that $\vtheta'\in B^{\mathrm{Euc}}_{\rho_w,\rho_r,\rho_1}(\vtheta_0)$, which is done by taking $\rho_2>\epsilon$ with $\epsilon$ as in Condition~\eqref{cond-1-fno}. 
%
Finally, it is straightforward to verify that such $\vtheta'\in B^t_\kappa$ defined by either Choice~\eqref{ch-A-fno} or Choice~\eqref{ch-B-fno} will always exist, by considering the following cases for the angle $\nu$ between $\nabla_\vtheta\bar{G}_t$ and $-\nabla_{\vtheta} \cL(\vtheta_t)$:
\begin{enumerate}[(i)]
\item If $\nu \in [0, \pi/2]$ or $\nu \in [3\pi/2, 2\pi]$, then Choice~\eqref{ch-A-fno} will be true, since $-\nabla_{\vtheta_f} \cL(\vtheta_t)$ is in the positive half space\footnote{We say $\a$ is in the positive half-space of $\b$ if $\langle \a, \b \rangle \geq 0$.} of $\nabla_\vtheta\bar{G}_t$; and
\label{it-i-fno}
\item if $\nu \in [\pi/2,\pi]$ or $\nu \in [\pi, 3\pi/2]$, then Choice~\eqref{ch-B-fno} will be true, since $-\nabla_{\vtheta_f} \cL(\vtheta_t)$ is in the positive half space of $-\nabla_\vtheta\bar{G}_t$.\label{it-ii-fno}
\end{enumerate}

%
Now, let us assume we are in the case of item~\eqref{it-i-fno} above, so that $\vtheta'$ is constructed according to Choice~\eqref{ch-A-fno} (the rest of the proof can be adapted to the case of item~\eqref{it-ii-fno} by using a symmetrical argument and so it is omitted).  
% \end{enumerate}
% To get (a)
Let $\nu_1$ be the angle between $\vtheta'-\vtheta_{t}$ and $\nabla_\vtheta\bar{G}_t$, so that $\cos(\nu_1)=\kappa$ according to Choice~\eqref{ch-A-fno}.
%
Then, we have that 
\begin{align*}
|\cos(\nu')| = |\cos(\nu - \nu_1)| \geq |\cos(\pi/2 - \nu_1)| = |\sin(\nu_1)| = \sqrt{1-\cos^2(\nu_1)} = \sqrt{1-\kappa^2}~.
\end{align*}
Further, by the construction in Condition~\eqref{cond-2-fno}, $\cos(\nu') > 0$, which implies 
%\begin{equation}
$\cos(\nu') \geq  \sqrt{1-\kappa^2}>0$.
%
Now, 
by the smoothness property of the empirical loss $\cL$ we have
\begin{align*}
\cL(\vtheta') & \leq \cL(\vtheta_t) - \langle \vtheta' - \vtheta_t, -\nabla_\vtheta \cL(\vtheta_t) \rangle + \frac{\beta}{2}\| \vtheta' - \vtheta_t \|_2^2 \\ 
& = \cL(\vtheta_t) - \|\vtheta' - \vtheta_{t}\|_2 \|\nabla_{\vtheta} \cL(\vtheta_t) \|_2 \cos(\nu) + \frac{\beta}{2}\| \vtheta' - \vtheta_{t} \|_2^2 \\
& =  \cL(\vtheta_t) -  \epsilon \|\nabla_{\vtheta} \cL(\vtheta_t) \|_2 \cos(\nu) + \frac{\beta}{2} \epsilon^2 \\
& = \cL(\vtheta_t) -  \frac{\beta \epsilon}{2} \left( \frac{2 \|\nabla_{\vtheta} \cL(\vtheta_t) \|_2 \cos(\nu)}{\beta} - \epsilon \right)\\
&<\cL(\vtheta_t)~.
\end{align*}
where the last inequality follows by the construction of $\epsilon$ in Condition~\eqref{ch-A-fno}. Note that this implies that the constructed $\vtheta'$ is as described in Condition~\ref{cond:rsc}(b.2). This finishes the proof for Condition~\ref{cond:rsc}(b).

The second part of the proof, i.e., the RSC condition over the non-empty set $B^t_{\kappa}$, %. This proof 
follows from a direct adaptation of Theorem~5.1 in~\citep{banerjee2022restricted} using Lemma~\ref{lemm:hessgradbounds-FNO}. \pcedit{Since we are using Lemma~\ref{lemm:hessgradbounds-FNO}, the condition for polynomial dependence on the bounds carries on.} 
%
%
\end{proof}


\RSSFNO*
%
\begin{proof}
The proof follows from a direct adaptation of the proof of Theorem~5.2 in~\citep{banerjee2022restricted} using Lemma~\ref{lemm:hessgradbounds-FNO}, where it can be shown that $\beta = 2\varrho^2 + \frac{\bar{c}}{\sqrt{m}}$ for some positive constant \pcedit{$\bar{c}$ which inherits the dependence on the constants $\sigma_{1,w}$, $\sigma_{1,r}$, the depth $L$ and the radii $\rho_w$, $\rho_r$, and $\rho_1$ from Lemma~\ref{lemm:hessgradbounds-FNO}}.
\end{proof}

\begin{prop}[{\bf RSC to smoothness ratio}] 
\label{prop:RSC-smooth-FNO}
Under the same conditions as in Theorems~\ref{theo:rsc_main_fno} and~\ref{theo:smooth_main_fno}, we have that $\alpha_t/\beta<1$ with probability at least $1-\frac{2(L+2)}{m}$.
\end{prop}
\begin{proof}
From the direct adaptation of the proof of Theorem~5.2 in~\citep{banerjee2022restricted} using Lemma~\ref{lemm:hessgradbounds-FNO}, we can obtain $\norm{\nabla_{\vtheta}\bar{G}_t}_2^2\leq \varrho^2$. Then, $\alpha_t\overset{(a)}{<}2\kappa^2\norm{\nabla_{\vtheta}\bar{G}_t}_2^2\leq 2\kappa^2 \varrho^2\leq 2\varrho^2\overset{(b)}{<}
\beta$, where (a) follows from~\eqref{eq:RSCLoss_FNO} and (b) from Theorem~\ref{theo:smooth_main_fno}. This result shows that $\frac{\alpha_t}{\beta}<1$.
\end{proof}



%====================
%====================


\section{Supplementary Information for the Experiments}
\label{app:exp_don_fno}
In this section we expand on the mathematical description of each operator learning problem studied in Section~\ref{sec:Experiments}. We also present further results on how the accuracy of each neural operator model improves as the width $m$ increases. Finally, we provide details about the hyperparameters and datasets used in the training of the corresponding models. 

We remark that all experiments with widths $m\in\{10,50\}$ were run on a personal computer with one NVIDIA Quadro GPU, while the rest of widths were on Google Colab with single NVIDIA L4 and A100 GPUs. 

\subsection{Antiderivative Operator}
We consider a simple one-dimensional Antiderivative or Integral operator given by
\begin{equation}
    s(x) := G(u)(x) = \int_0^x u(\xi)\,\mathrm{d}\xi, \qquad x\in [0, 1]~.
\end{equation}
Note that $G(u)$ is a linear operator and therefore learnable up to high accuracy. This is evident from the training loss in Figure~\ref{fig:seLU_Loss_DON} as well as from the sample solutions presented in Figure~\ref{fig:DON_solns_vs_width} for DONs and in Figure~\ref{fig:FNO_solns_vs_width} for FNOs. We observe that overall an increase in the width $m$ leads to higher training accuracy and lower training loss.

The sample size of the training data is $n=2000$, with every input function $u^{(i)}$, $i\in[n]$, being a one-dimensional Gaussian Random Fields (GRF).
For DON training, we choose $R=100$ input locations and we choose $100$ output locations for each input function, i.e., $q_i=100$, $i\in[n]$ (according to the notation in Section~\ref{subsec:DON_Setup}).\footnote{During training; however, for each $i\in[n]$, instead of averaging the loss over all the $q_i$ points, we simply randomly choose one of the $q_i$ points and evaluate the loss on it. This is strictly done in the interest of computational efficiency, since it is known to not reduce the accuracy of the results for the Antiderivative operator; e.g.,  see~\citep{lu20201DeepONet}.}  
%
\begin{figure}[t!]
    \centering
    \includegraphics[width=\linewidth]{Figures/soln_DON_AD_FNO.pdf}
    \caption{Sample solutions obtained for the Antiderivative operator for DONs for $m\in \{10, 50, 500\}$ at the end of the training process (80,000 epochs) for a randomly chosen input function. The ``data'' refers to the ground truth (obtained by a standard numerical solver) and ``pred'' corresponds to the learned operator.}
    \label{fig:DON_solns_vs_width}
\end{figure}
%
For the FNO, the input function is also sampled across $100$ locations (i.e., $\bar{R}=100$ using the notation in Section~\ref{sec:optFNO}); however, since we are interested in the model to provide an output of $100$ output locations, we modify the FNO architecture to provide this vector-valued output.
\footnote{This means that we have $R=1$ (according to the notation in Section~\ref{subsec:FNO_setup}) with the understanding that for each input function, we output a vector of size $100$. This is done as an alternative to an FNO with a scalar output which is averaged across the $100$ locations (for which we would have $R=100$), which is what we described in Section~\ref{sec:modelSetup}. We remark that we considered this modification on the output of the FNO just for the sake of computational efficiency, and this only empirically works for the case of the Antiderivative operator. Again, as in the case of DONs, we randomly sample one of the output locations to compute the loss during training.} 
%
For all the experiments we fix the learning rate for the \texttt{Adam} optimizer at $10^{-3}$ and with full-batch training, i.e., the batch size of $2000$ for both DONs and FNOs. For testing the trained neural operators, we generate another one-dimensional GRF. 
%
\begin{figure}[t!]
    \centering
    \includegraphics[width=\linewidth]{Figures/soln_FNO_AD_FNO.pdf}
    \caption{Sample solutions obtained for the Antiderivative operator for FNOs for $m\in \{10, 50, 500\}$. The setting is the same as in Figure~\ref{fig:DON_solns_vs_width}.}
    \label{fig:FNO_solns_vs_width}
\end{figure}

\subsection{Diffusion-Reaction Operator}
We are interested in learning an operator $G: u(x) \to s(x,t)$ for the solution operator of the one-dimensional Diffusion-Reaction equation implicitly given by
\begin{equation}
\label{eq:dr-eq}
\begin{aligned}
    \frac{\partial s}{\partial t}&=D \frac{\partial^2 s}{\partial x^2}+k s^2+u(x), \quad(x, t) \in(0,1] \times(0,1]~,
\end{aligned}
\end{equation}
with $D>0$ and zero initial and boundary conditions, namely, 
\begin{equation*}
    s(0, t) = s(1, t) = 0 \quad \text{and}\quad s(x, 0) = 0~,
\end{equation*}
along with a forcing function $u(x)$ defined by a GRF. This is the same setup as in~\citep{physicsInformed202WangPerdikaris,lu20201DeepONet}. The corresponding solutions for DONs and FNOs are presented in Figures~\ref{fig:DON_DR_solns_vs_width} and~\ref{fig:FNO_DR_solns_vs_width} respectively. Again, a larger width $m$ leads to a more accurate solution.

The neural operator aims to learn a mapping from the forcing function to the solution at different times in the interval $(0,1]$, in other words, the forcing function would be the \emph{input function} as defined in Section~\ref{sec:modelSetup}. We make use of a slightly modified solver provided at \url{https://github.com/PredictiveIntelligenceLab/Physics-informed-DeepONets} to generate the training data for the equation. We generate solutions for $n=5000$ input functions which are sampled on $100$ points in the space dimension (i.e., the interval $(0,1]$ for $x$ in~\eqref{eq:dr-eq} is divided in $100$ points) so that $R=100$ for DON and $\bar{R}=100$ for FNO (according to the notations in Section~\ref{subsec:DON_Setup} and Section~\ref{sec:optFNO} respectively). For computing the solutions, we are interested in computing them at $100$ different times $t$ within the time interval $(0,1]$ in~\eqref{eq:dr-eq} (in order to be able to plot the two-dimensional map on $x$ and $t$ in Figures~\ref{fig:DON_DR_solns_vs_width} and~\ref{fig:FNO_DR_solns_vs_width}). This division of both spatial and time dimensions results in a grid of $10,000$ points that can be chosen as output locations.
%
For the training of DONs, for each $i\in[n]$, we only select $100$ scattered points from the grid of output locations (out of their $10,000$ points), so that $q^{(i)}=100$, which will become the input to the trunk net. 
However, for the training of FNOs, we do choose the full grid as output locations and thus we modify the FNO to provide $10,000$ outputs instead of the scalar output provided in our theoretical analysis.%: $100$ values of time $t$ per location $x$.
\footnote{A scalar output is needed if we were interested in evaluating the operator at only one specific spatial location $x$ and one specific value of time $t$; however, as can be seen in Figures~\ref{fig:DON_DR_solns_vs_width} and~\ref{fig:FNO_DR_solns_vs_width}, we are interested in plotting solutions at multiple locations and times.} We fix the diffusivity as $D = 0.01$.
%
%
For all the experiments we use a constant learning rate of $3\times 10^{-4}$ and \texttt{Adam} optimizer with a batch size of $4000$. 
For testing the trained neural operators, we generate another one-dimensional GRF. 

\begin{figure}[t!]
    \centering
    \includegraphics[width=\linewidth]{Figures/soln_DON_DR_FNO.pdf}
    \caption{Sample solutions $s(x,t)$ obtained for the Diffusion-Reaction operator for DONs for $m\in \{10,500\}$ given an input $u(x)$. The top row corresponds to $m=10$ and the bottom row to $m=500$. The third column represents the pointwise difference of the ground truth or ``Data'' (first column) minus the obtained results from the learned DON or ``Pred'' (second column).}
    \label{fig:DON_DR_solns_vs_width}
\end{figure}
\begin{figure}[t!]
    \centering
    \includegraphics[width=\linewidth]{Figures/soln_FNO_DR_FNO.pdf}
    \caption{Sample solutions obtained for the Diffusion-Reaction operator $s(x, t)$ for FNOs for $m\in \{10,500\}$. The setting of the plots is the same as in Figure~\ref{fig:DON_DR_solns_vs_width} where the top row corresponds to $m=10$ and the bottom row to $m=500$.}
    \label{fig:FNO_DR_solns_vs_width}
\end{figure}

\subsection{Burger's Equation}
The Burger's equation operator learns an operator $G: u(x) \to s(x, 1)$, where 
\begin{align}
    \begin{aligned}
        & \frac{\partial s}{\partial t}
        +
        s \frac{\partial s}{\partial x}
        -
        \nu \frac{\partial^2 s}{\partial x^2}
        =
        0, \quad(x, t) \in(0, 2\pi] \times(0,1] \\
        & s(x, 0)=u(x), \quad x \in(0,2\pi]
    \end{aligned}\label{eq:Burgers}
\end{align}
with $\nu>0$ and periodic boundary conditions
\begin{align*}
    \begin{aligned}
        & s(0, t)=s(2\pi, t) \\
        & \frac{\partial s}{\partial x}(0, t)=\frac{\partial s}{\partial x}(2\pi, t)~.
    \end{aligned}
\end{align*}
%
The corresponding solutions for DONs and FNOs are presented in Figures~\ref{fig:DON_solns_Burgers_vs_width} and~\ref{fig:FNO_solns_Burgers_vs_width} respectively. Again, a larger width $m$ leads to a more accurate solution.

The neural operator aims to learn a mapping from the initial condition to the solution at time $t=1$, i.e. the mapping from $u(x)$ to the final solution $s(x, 1)$.  This is the operator learning problem originally studied in~\citep{li_fourier_2021}. 
We note that the initial condition would then be the \emph{input function} as defined in Section~\ref{sec:modelSetup}. We make use of the datasets publicly available at \url{https://github.com/neuraloperator/neuraloperator}, specifically the \texttt{Burgers\_R10.mat} dataset available at \url{https://drive.google.com/drive/folders/1UnbQh2WWc6knEHbLn-ZaXrKUZhp7pjt-}, which comprises of 2048 input functions and corresponding final solution (i.e., $u^{(i)}$ with associated solution $s^{(i)}(\cdot,1)$, $i\in[2048]$). All solutions are calculated for a single viscosity $\nu=0.01$. 
%
For all the experiments we use a constant learning rate of $10^{-3}$ and \texttt{Adam} optimizer with a batch size of $800$. We test the trained neural operators on a simple GRF sampled from the training dataset.
%
%
%
\begin{figure}[t!]
    \centering
    \includegraphics[width=\linewidth]{Figures/soln_DON_Burgers_FNO.pdf}
    \caption{Sample solutions obtained for the Burger's equation for DONs for $m\in \{10, 50, 500\}$. The setting is similar to the one in Figure~\ref{fig:DON_solns_vs_width} where we plot the obtained solution from the learned operator (denoted by ``pred'') along with the ground truth (denoted by ``data'') for different widths. 
    }
    \label{fig:DON_solns_Burgers_vs_width}
\end{figure}
%
\begin{figure}[t!]
    \centering
    \includegraphics[width=\linewidth]{Figures/soln_FNO_Burgers_FNO.pdf}
    \caption{Sample solutions obtained for the Burgers equation for FNOs for $m\in \{10, 50, 500\}$. The setting is the same as in Figure~\ref{fig:DON_solns_Burgers_vs_width}.}
    \label{fig:FNO_solns_Burgers_vs_width}
\end{figure}


\end{document}