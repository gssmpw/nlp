\section{Related work}
In the field of pHRI, the control of robots during interaction tasks is an active area of research. There are several key areas of pHRI control that are relevant to the development of a controller that enables robots to resist impact forces while maintaining compliance with traction forces. These areas are admittance control parameters tuning, nonlinear damping control, and the technology of distinguishing between impact forces and intentional contacts.
In this section, we will discuss these areas of pHRI control and summarize the related research progress.

\subsection{9.1 Admittance control parameters tuning}
Admittance control, as favored by \cite{chen2022human}, is a preferred approach in physical human-robot interaction (pHRI) due to its ability to allow robots to move based on contact forces and comply with the operator's intentions. 
Compared with impedance control, admittance control is more beneficial for non-backdrivable or challenging-to-identify dynamic parameter robots to achieve flexible interaction, as noted by both~\cite{chen2020compliance,arduengo2021human}.  Tuning admittance control parameters is crucial for optimizing the robot's response during pHRI, and this compliant interaction response mainly reflects three aspects: stability, passivity, and transparency.

1) In the context of stability, researchers have conducted valuable studies from the perspectives of frequency domain detection, feedforward control, velocity bandwidth, virtual damping augmentation, and energy allocation. They have provided beneficial design principles and adaptive adjustment strategies (\cite{dimeas2016online,keemink2018admittance,ferraguti2019variable}). However, these analyses primarily focus on tightly coupled human-machine interactions, with limited research on contact transitions and the impact of impulsive forces. 2) Passivity is an effective approach for analyzing the stability of complex systems. Preserving the power envelope, unaffected by the passivity condition, prevents mismatches in admittance parameters caused by unpredictable human interaction (\cite{ferraguti2015energy}). Motivated by the concept of passivity, our paper investigates the dissipation of the proposed controller under impact disturbances to ensure the safety of human-robot interaction. 3) To achieve transparency in pHRI, the robot should minimize its resistance to human motion (\cite{laghi2020unifying}). Researchers have explored methods such as fractional-order control and integrating admittance control with arm redundancy resolution to enhance transparency (\cite{sirintuna2020variable,aydin2021towards,kim2012admittance} ). However, transparency is not always the optimal solution in pHRI. In situations where impact disturbances occur, high transparency can lead to sudden changes in robot motion speed, posing a safety threat to collaborators. In such cases, lower transparency may be preferable to avoid unnecessary secondary injuries. 

When designing pHRI systems, researchers have mainly focused on the trade-off between passivity, stability, and transparency in admittance control's dynamic properties. However, distinguishing between impact forces and intentional contacts has not been addressed from a dynamical perspective. In fact, an admittance controller with linear inertial-damping-stiffness formulation does not differentiate forces of different amplitudes and frequencies. Therefore, this paper considers nonlinear control to achieve the desired dynamic properties while meeting the requirements for passivity and stability.

\subsection{9.2 Nonlinear damping control}
Nonlinear damping control has been proposed as an effective method for regulating the force interaction between a robot and a human. 
Incorporating nonlinear damping in control system design can improve error dynamics convergence, counteract estimation errors, achieve robust control, reduce overshoot, and ensure stability, passivity, convergence time, and accuracy for second-order systems (\cite{6202693,7954629,lai2014improving,ruderman2021optimal}). In recent years, variable damping controllers have gained significant attention in research. The focus of these studies includes addressing changes in system dynamics caused by unknown disturbances, balancing stability and agility, minimizing energy consumption (\cite{6809166,zahedi2021variable,zahedi2022user}).

Compared to linear cases, nonlinear damping control offers better robustness in uncertain systems. However, designing such controllers is complex and typically involves system modeling, simulation, and parameter tuning tailored to specific applications, which poses technical difficulties and high workload. While some biologically inspired parameter regulation methods, such as imitating human muscle stiffness (~\cite{al2018active}) or implementing an angle-related damping curve for impact-bearing joints (~\cite{hamid2021state}), have been proposed, they often aim to follow contact forces instead of specific dynamic properties. Moreover, since nonlinear damping control methods may contain multiple adjustable parameters, the tuning and experimental verification process can be time-consuming. Thus, automatic parameter tuning methods for nonlinear controllers are necessary to alleviate workload associated with manual tuning while ensuring optimal performance.

% Additionally, researching nonlinear damping models and their stability analysis is crucial. One proposed model by \cite{61008} presented an energy-based approach, allowing for the explicit determination of white noise input's corresponding stationary probability density. Another study by \cite{798060} examined a compliant contact model with nonlinear damping to analyze impact response while avoiding discontinuous impact forces seen in linear models. Meanwhile, \cite{marx2019stability} proposed a general strategy utilizing Lyapunov functionals to analyze global asymptotic stability of linear infinite-dimensional systems subjected to nonlinear dampings. Practically speaking, \cite{colonnese2016stability} explored the stability and quantization error of haptic rendering of virtual damping, while \cite{elliott2015nonlinear} reviewed various sources of nonlinear damping within quasi-linear modeling. The studies excel in investigating nonlinear damping models' performance, stability, and practical applications; however, they fail to meet our dynamic requirements of compliance with traction and resistance to impacts. This paper analyzes the proposed nonlinear controller's performance based on these studies to comprehend the system's behavior better while avoiding problems seen in linear models.


Researchers have extensively studied nonlinear damping and its stability analysis using various approaches, such as compliant contact models, quasi-linear modeling methods, energy-based approaches, and Lyapunov functionals (\cite{61008,798060,marx2019stability,colonnese2016stability,elliott2015nonlinear}). These studies have provided valuable insights into the performance, stability, and practical applications of nonlinear damping models. However, these models do not meet our specific requirements for dynamic compliance with traction and resistance to impacts. To address these limitations observed in linear models, this paper analyzes the performance of a proposed nonlinear controller, drawing from the aforementioned studies to gain a better understanding of the system's behavior.


\subsection{9.3 Distinguish between impact and compliance}

Distinguishing between impact and compliance is critical in pHRI tasks as it enables robots to differentiate between unintended collisions and intentional contact. One approach is to use force sensing technologies to detect the level of force during interaction tasks. \cite{li2018stable} proposed a method that combines variable admittance control and adaptive control, utilizing two independent force sensors. The objective is to maintain human-like compliance in direct control of the robot while ensuring smooth transitions and stable motion when the robot interacts with the environment. Other approaches address impact and compliance issues through mechanical design, including adjusting the stiffness of parallel compliant elements, utilizing passive compliant links and joints, or designing variable stiffness mechanisms (\cite{niehues2015compliance,she2020comparative,ayoubi2020safe}). However, it is crucial to note that compliant mechanisms may suffer from hysteresis, fatigue, and creep effects over time, which can lead to performance degradation.



% Other approaches address impact and compliance through mechanical design. \cite{niehues2015compliance} improved stability in response to impacts by adjusting the stiffness of parallel compliant elements. \cite{she2020comparative} compared collision results of manipulators with compliant links and compliant joints, showing that compliant links have a potential advantage in dealing with collision safety due to their smaller maximum collision force. \cite{ayoubi2020safe} proposed integrating variable stiffness safety-oriented mechanisms at each rotational joint of a series collaborative robot, ensuring safe pHRI by adjusting between high and low stiffness. However, it's essential to note that compliant mechanisms may suffer from hysteresis, fatigue, and creep effects over time, which may lead to performance degradation.

Energy observation-based methods are also common in addressing impact disturbance, where real-time model-based collision detection, isolation, and recognition can be achieved through energy observers and momentum observers \cite{haddadin2017robot}. \cite{lachner2021energy} allocated a safe energy budget for robots and modifies parameters to restrict the exchange of kinetic and potential energy during collisions, allowing automatic tuning of controller parameters and reducing impact (\cite{raiola2018development}, \cite{munoz2019time}, \cite{navarro2016iso10218}). In unstructured environments, some studies focused on detecting collisions and ensuring robots react quickly to prevent potential hazards (\cite{de2012integrated}, \cite{khan2014compliance}). Nonetheless, techniques like math model matching or signal threshold filtering may introduce time delays and recovery errors, ultimately affecting the real-time responsiveness and accuracy of pHRI (\cite{haddadin2017robot}, \cite{lin2021adaptive}). Our SFC controller does not require online identification of external forces. Instead, it achieves different responses to different forces through the inherent nonlinearity of the controller, which contributes to better real-time performance.

\subsection{9.4 Inverse kinematics}
Inverse kinematics is a crucial aspect in pHRI. However, the computation of the inverse Jacobian matrix can present safety issues in terms of singularities and redundancy management. Impedance control is an effective approach for addressing singularities as it maps end-effector forces to joint forces using the transpose of the Jacobian matrix, thereby avoiding the issues associated with inverting the Jacobian (\cite{albu2002cartesian}). Nonetheless, impedance control also has some limitations. Uncertainties in the robot's dynamic parameters and noise in acceleration measurements can have a detrimental impact on the accuracy of impedance control (\cite{lynch2017modern}). Moreover, impedance control encounters difficulties in generating stiff virtual dynamic models (\cite{keemink2018admittance}) and is not feasible for non-backdrivable and high-inertia robots (\cite{villani2016force}). 

In addition to impedance control, there exist multiple alternative methods for addressing singularities and managing redundancy in inverse kinematics. For example, the weighted pseudo-inverse Jacobian method (\cite{mussa1991integrable}) or the damped least squares method (\cite{zhan2021adaptive}) can be utilized to effectively deal with singularity problems in inverse kinematics. Another strategy involves decoupling the mobile platform from the robot manipulation system to address redundancy challenges (\cite{chung1998interaction}). Moreover, employing null-space projection techniques enables dynamic control of redundant manipulators, enabling them to effectively execute multiple priority tasks concurrently (\cite{ott2015prioritized}). In our case, we employed a damped least squares approach to calculate the pseudo-inverse of the Jacobian. When approaching singularity, we only apply small damping along the singular directions to reduce the impact of the singularity. It is important to note that inverse kinematics is not the main focus of this paper, hence the limited discussion on this topic.

To sum up, pHRI is an area of ongoing research, and the studies reviewed in this section have demonstrated promising outcomes in enhancing the efficiency, safety, and performance of human-robot interaction through creative control methods. However, to achieve more natural and secure interactions, it is necessary for robots to withstand external impact disturbances while complying with human traction. Currently, this dynamic ability has not been attained through the aforementioned research. With inspiration from shear-thickening fluids, there is potential to design a controller that can achieve the desired dynamic characteristics of pHRI.



% In addition to impedance control, some studies have been consistently devoted to programming expected robot behaviors by mimicking the dynamic properties of various non-robotic systems. 
% Yang et al.~\cite{mrelated1} proposed an interaction model which incorporates a bio-inspired neurodynamics component with a back-stepping technique so that the sudden velocity jumps in the back-stepping controllers are eliminated due to the smooth dynamics in the shunting neural model. Inspired by the human cognition and human control-thinking mechanism, Wang et al.~\cite{mrelated2} proposed a method to deal with the irreconcilable conflict between stability, speediness, and accuracy. However, their performance does not meet our dynamic requirements of compliance with traction and resistance to impacts.