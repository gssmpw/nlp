\section{Related Work}
\subsection{Nuclear instance segmentation}
Before the emergence of deep learning, instance segmentation methods predominantly relied on classical machine learning algorithms and image processing techniques. 
These included statistical features, thresholding-based methods~\citep{latorre2013segmentation, abbas2014occluded, sheeba2014splitting}, and morphological features for contour and shape identification. 
Energy-based solutions gained prominence during the 1990s. 
Marker-based approaches such as the Watershed algorithm proved beneficial for nuclear instance segmentation, as demonstrated by \citet{cheng2008segmentation}. 
However, these techniques faced limitations in accurately segmenting nuclear due to their diverse structures, textures, intensities, leading to unreliable outcomes. 
Moreover, the effectiveness of these systems heavily depended on manual parameter tuning, including thresholds and weights, rendering them unsuitable for widespread application due to their inherent unreliability.

With the rapid development of deep learning, \citet{ronneberger2015u} proposed the UNet model, which has since become one of the most fundamental models in medical image segmentation. 
\citet{raza2019micro} introduced Micro-Net, achieving robustness to large internal and external variances in nuclear size by utilizing multi-resolution and weighted loss functions. \citet{qu2019improving} developed a full-resolution convolutional neural network (FullNet), which enhances localization accuracy by eliminating downsampling operations in the network structure. 
\citet{he2021hybrid} presented a hybrid attention nested U-shaped network (Han-Net) to extract effective feature information from multiple layers.

To leverage contour information for distinguishing contact/overlapping nuclear, \citet{chen2017dcan} initially proposed incorporating contour information into a multi-level fully convolutional network (FCN) to create a deep contour-aware network for nuclear instance segmentation. Subsequently,
\citet{zhou2019cia} introduced the contour-aware information aggregation network, which combines spatial and textural features between nuclear and contours. 
Additionally, some models have approached nuclear instance segmentation as an object detection task, such as contour proposal networks (CPN)~\citep{upschulte2022contour}, which use a sparse list of contour representations to define a nuclear instance.

Several works~\citep{chen2023cpp, graham2019hover, liu2021mdc, naylor2018segmentation} have introduced distance maps to separate contact/overlapping nuclear. 
\citet{naylor2018segmentation} addressed the issue of segmenting touching nuclear by formulating the segmentation task as a regression task of intra-nuclear distance maps.
\citet{graham2019hover} proposed Hover-Net, a network for simultaneous nuclear segmentation and classification, which uses the vertical and horizontal distances between a nuclear pixel and its center of mass to separate clusters of nuclear. 
Moreover, \citet{he2021cdnet} introduced a centripetal directional network (CDNet) for nuclear instance segmentation, incorporating directional information into the network. \citet{he2023toposeg} take topological information into consideration to further split overlapping nuclear instance.
These works mainly are proposed for full-supervised nuclear instance segmentation, and don't consider the characteristic of nuclear itself.
% However, nuclear annotations is merely in real world.
In this work, we make full use of the characteristic of nuclear instance (we name it as \textbf{nuclear guidance}), introduce a new framework with guidance, not only can solve full-supervised but also weak-supervised nuclear instance segmentation efficiently.

\subsection{Weakly-supervised segmentation}
Weakly supervised approaches offer the advantage of reducing manual annotation effort compared to fully supervised methods. 
In natural image segmentation, ~\citet{papandreou2015weakly} proposed an Expectation-Maximization (EM) method for training with image-level or bounding-box annotations. ~\citet{pathak2015constrained} added a set of linear constraints on the output space in the loss function to leverage information from image-level labels.

In contrast to image-level annotations, point annotations provide more precise location information for each object. ~\citet{bearman2016s} incorporated an objectness prior in the loss function to guide the training of a CNN, aiding in the separation of objects from the background. Scribble annotations, which require at least one scribble per object, are a more informative type of weak label. ~\citet{lin2016scribblesup} used scribble annotations to train a graphical model that propagates information from the scribbles to the unmarked pixels.

Bounding boxes are the most widely used form of weak annotation, applied in both natural images~\citep{dai2015boxsup, rajchl2016deepcut} and medical images~\citep{yang2018boxnet, zhao2018deep}. 
~\citet{kervadec2019constrained} utilized a small fraction of full labels and imposed a size constraint in their loss function, achieving good performance, though this method is not applicable to scenarios involving multiple objects of the same class. Another method~\citep{qu2020weakly} proposed a two-stage approach that uses only a small fraction of nuclear locations.

In this work, we propose a novel end-to-end framework to solve both fully-supervised and weakly-supervised nuclear instance segmentation tasks.
We start from the characteristics of the nuclear image itself, and use the information of this feature as the a priori information to build the corresponding module to guide the model for training.
With only a small amount of labeled data, our model is able to approach fully-supervised results.

\subsection{Contrastive learning}
Contrastive learning is a highly regarded technique for learning representations from unlabeled features these days~\citep{chen2020simple, chen2020improved, grill2020bootstrap}. 
It aims to enhance representation learning by contrasting similar features (positive pairs) against dissimilar features (negative pairs). A key innovation in contrastive learning lies in the selection of positive and negative pairs. Additionally, the use of a memory bank to store more negative samples has been adopted, as this can lead to improved performance~\citep{chen2020simple}.

In the field of segmentation, numerous works leverage contrastive learning for the pre-training of models~\citep{chaitanya2020contrastive, wang2021dense, xie2021propagate}. 
Recently, ~\citet{wang2021exploring} demonstrated the advantages of applying contrastive learning in a cross-image pixel-wise manner for supervised segmentation. 
The CAC approach~\citep{lai2021semi} shows improvement in semi-supervised segmentation by performing directional contrastive learning pixel-to-pixel, aligning lower-quality features towards their high-quality counterparts.

Unlike these works, we construct the guide-based insatnce level contrastive (GILC) module from the image characteristics of the nuclear, which relies on the automatically generated guide mask to further enhance the feature representation of the nuclear, and is able to be applied to both fully-supervised and weakly-supervised nuclear instance segmentation tasks.

% \section{Study Design}
% robot: aliengo 
% We used the Unitree AlienGo quadruped robot. 
% See Appendix 1 in AlienGo Software Guide PDF
% Weight = 25kg, size (L,W,H) = (0.55, 0.35, 06) m when standing, (0.55, 0.35, 0.31) m when walking
% Handle is 0.4 m or 0.5 m. I'll need to check it to see which type it is.
We gathered input from primary stakeholders of the robot dog guide, divided into three subgroups: BVI individuals who have owned a dog guide, BVI individuals who were not dog guide owners, and sighted individuals with generally low degrees of familiarity with dog guides. While the main focus of this study was on the BVI participants, we elected to include survey responses from sighted participants given the importance of social acceptance of the robot by the general public, which could reflect upon the BVI users themselves and affect their interactions with the general population \cite{kayukawa2022perceive}. 

The need-finding processes consisted of two stages. During Stage 1, we conducted in-depth interviews with BVI participants, querying their experiences in using conventional assistive technologies and dog guides. During Stage 2, a large-scale survey was distributed to both BVI and sighted participants. 

This study was approved by the University’s Institutional Review Board (IRB), and all processes were conducted after obtaining the participants' consent.

\subsection{Stage 1: Interviews}
We recruited nine BVI participants (\textbf{Table}~\ref{tab:bvi-info}) for in-depth interviews, which lasted 45-90 minutes for current or former dog guide owners (DO) and 30-60 minutes for participants without dog guides (NDO). Group DO consisted of five participants, while Group NDO consisted of four participants.
% The interview participants were divided into two groups. Group DO (Dog guide Owner) consisted of five participants who were current or former dog guide owners and Group NDO (Non Dog guide Owner) consisted of three participants who were not dog guide owners. 
All participants were familiar with using white canes as a mobility aid. 

We recruited participants in both groups, DO and NDO, to gather data from those with substantial experience with dog guides, offering potentially more practical insights, and from those without prior experience, providing a perspective that may be less constrained and more open to novel approaches. 

We asked about the participants' overall impressions of a robot dog guide, expectations regarding its potential benefits and challenges compared to a conventional dog guide, their desired methods of giving commands and communicating with the robot dog guide, essential functionalities that the robot dog guide should offer, and their preferences for various aspects of the robot dog guide's form factors. 
For Group DO, we also included questions that asked about the participants' experiences with conventional dog guides. 

% We obtained permission to record the conversations for our records while simultaneously taking notes during the interviews. The interviews lasted 30-60 minutes for NDO participants and 45-90 minutes for DO participants. 

\subsection{Stage 2: Large-Scale Surveys} 
After gathering sufficient initial results from the interviews, we created an online survey for distributing to a larger pool of participants. The survey platform used was Qualtrics. 

\subsubsection{Survey Participants}
The survey had 100 participants divided into two primary groups. Group BVI consisted of 42 blind or visually impaired participants, and Group ST consisted of 58 sighted participants. \textbf{Table}~\ref{tab:survey-demographics} shows the demographic information of the survey participants. 

\subsubsection{Question Differentiation} 
Based on their responses to initial qualifying questions, survey participants were sorted into three subgroups: DO, NDO, and ST. Each participant was assigned one of three different versions of the survey. The surveys for BVI participants mirrored the interview categories (overall impressions, communication methods, functionalities, and form factors), but with a more quantitative approach rather than the open-ended questions used in interviews. The DO version included additional questions pertaining to their prior experience with dog guides. The ST version revolved around the participants' prior interactions with and feelings toward dog guides and dogs in general, their thoughts on a robot dog guide, and broad opinions on the aesthetic component of the robot's design. 
