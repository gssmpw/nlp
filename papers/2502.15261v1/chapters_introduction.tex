\section{Introduction}\label{sec:intro}
Writing proficiently poses significant challenges for language learners who often struggle to produce grammatically correct and coherent texts~\citep{li2024rethinking,DBLP:conf/acl/LiZLLLSWLCZ22,DBLP:conf/emnlp/YeLZLM0023}. Therefore, GEC systems~\citep{bryant2023grammatical} are developed to detect and rectify all grammatical errors in texts~\citep{ye-etal-2023-mixedit,huang-etal-2023-frustratingly}. Research advancements in GEC encompass multi-language~\citep{rothe-etal-2021-simple,DBLP:journals/corr/abs-2307-09007,DBLP:conf/emnlp/MaLSZHZLLLCZS22}, multi-modality~\citep{fang-etal-2023-improving,DBLP:conf/emnlp/LiMZLLHLLC022,DBLP:conf/acl/LiXC0LMJLZZS24}, and document-level~\citep{yuan-bryant-2021-document,DBLP:conf/emnlp/DuW0D0LZVZSZGL024}.


However, the explainability of GEC~\citep{hanawa-etal-2021-exploring,DBLP:journals/corr/abs-2407-00924,DBLP:journals/corr/abs-2407-00934} remains under-explored due to its intrinsic difficulties. Given that neural GEC systems generally function as intricate black-box models, their internal processes are not transparent~\citep{zhao2023explainability}. The absence of explainability can result in inadequacies in educational scenarios, where L2 learners might find it difficult to completely understand outputs from GEC systems without knowing the rationale behind corrections. Providing corrections with explanations fosters appropriate trust by clarifying the linguistic principles and logical mechanisms underpinning model decisions in a comprehensible way, thereby aiding educationally K-12 students and L2 speakers~\citep{bitchener2005effect,sheen2007effect,DBLP:journals/eswa/LiMCHHLZS25,ye2025position}. Moreover, explainability offers insights that help identify unintentional biases and risks for researchers, functioning as a debugging tool to enhance model performance~\citep{ludan-etal-2023-explanation,DBLP:conf/icassp/ZhangLZMLCZ23}.

\section{Introduction}
\label{sec:intro}

\begin{figure*}[tb]
    \centering
    \includegraphics[width=0.848\linewidth]{figs/circuitnn.pdf} 
    \caption{Illustration of differentiable CircuitNN. CircuitNN is designed based on differentiable NAND gates. After DAS is guided by PI and PO pairs of the truth table, CircuitNN can get the precise circuit architecture logic equivalent to the truth table.}
    \label{fig:circuitnn}
\end{figure*}

% 1. Describe the importance of logic synthesis
% 2. Existing Problems
% (a) Neural Architecture Search: Unstable, Predefined Setting, etc.
% (b) Circuit Generation: Probabilistic Model, Logic Equivalence

With the rapid advancement of technology, the scale of integrated circuits (ICs) has expanded exponentially. 
This expansion has introduced significant challenges in chip manufacturing, particularly concerning power and area metrics.
A primary objective in IC design is achieving the same circuit function with fewer transistors, thereby reducing power usage and area occupancy.

Logic synthesis~\cite{hachtel2005logicsynth}, a critical step in electronic design automation (EDA), transforms behavioral-level circuit designs into optimized gate-level circuits, ultimately yielding the final IC layout. 
The primary goal of logic synthesis is to identify the physical implementation with the fewest gates for a given circuit function. 
This task constitutes a challenging NP-hard combinatorial optimization problem. 
Current logic synthesis tools~\cite{brayton2010abc, wolf2013yosys} rely on human-designed heuristics, often leading to sub-optimal outcomes.

Differentiable architecture search (DAS) techniques~\cite{liu2018darts, chu2020darts} offer novel perspectives on addressing challenges in this problem.
Circuit functions can be represented through truth tables, which map binary inputs to their corresponding outputs. 
Truth tables provide a precise representation of input-output relationships, ensuring the design of functionally equivalent circuits.
Inspired by this, researchers~\cite{deepmind2024ai4sys, wang2024tnet} have begun exploring the application of DAS to synthesize circuits directly from truth tables.
Specifically, \citet{deepmind2024ai4sys} proposed CircuitNN, a framework that learns differentiable connection structures with logic gates, enabling the automatic generation of logic circuits from truth tables.
This approach significantly reduces the complexity of traditional circuit generation. 
Building on this, \citet{wang2024tnet} introduced T-Net, a triangle-shaped variant of CircuitNN, incorporating regularization techniques to enhance the efficiency of DAS.

Despite these advancements, several challenges remain. 
The computational complexity of DAS grows quadratically with the number of gates, posing scalability issues.
Although triangle-shaped architecture~\cite{wang2024tnet} partially mitigates this problem, redundancy persists. 
%Additionally, DAS is susceptible to converging to local optima, limiting the ability to search architectures that satisfy the given truth tables~\cite{liu2018darts}. 
%Furthermore, hyperparameters (network depth and layer width) require extensive searches, introducing complexity and prolonging the synthesis process. 
Additionally, DAS is susceptible to converging to local optima~\cite{liu2018darts} and hyperparameters (network depth and layer width) require extensive searches. 
The challenges arise from the vast search space in DAS. 
% Even with predefined settings for CircuitNN, finding a configuration that meets the truth table requires extensive trial and error during the DAS process. 
Intuitively, limiting the search space through predefined parameters (network depth, gates per layer, and connection probabilities) can significantly reduce the complexity.

Recent advances~\cite{openai2023gpt4, abramson2024alphafold3, esser2024sd3, li2024mar} in conditional generative models have demonstrated remarkable performance across language, vision, and graph generation tasks. 
Motivated by these developments, we propose a novel approach to circuit generation that generates preliminary circuit structures to guide DAS in generating refined circuits matching specified truth tables. 
Firstly, we introduce CircuitVQ, a tokenizer with a discrete codebook for circuit tokenization. 
Built upon our Circuit AutoEncoder framework~\cite{hou2022graphmae,li2023maskgae,wu2025mgvga}, CircuitVQ is trained through a circuit reconstruction task. 
Specifically, the CircuitVQ encoder encodes input circuits into discrete tokens using a learnable codebook, while the decoder reconstructs the circuit adjacency matrix based on these tokens.
Subsequently, the CircuitVQ encoder serves as a circuit tokenizer for CircuitAR pretraining, which employs a masked autoregressive modeling paradigm~\cite{chang2022maskgit, li2023mage}. 
In this process, the discrete codes function as supervision signals. 
After training, CircuitAR can generate discrete tokens progressively, which can be decoded into initial circuit structures by the decoder of the CircuitVQ. 
These prior insights can guide DAS in producing refined circuits that match the target truth tables precisely.

Our key contributions can be summarized as follows:
\begin{itemize}
\item We introduce CircuitVQ, a circuit tokenizer that facilitates graph autoregressive modeling for circuit generation, based on our Circuit AutoEncoder framework;
\item Develop CircuitAR, a model trained using masked autoregressive modeling, which generates initial circuit structures conditioned on given truth tables;
\item Propose a refinement framework that integrates differentiable architecture search to produce functionally equivalent circuits guided by target truth tables;
\item Comprehensive experiments demonstrating the scalability and capability emergence of our CircuitAR and the superior performance of the proposed circuit generation approach.
\end{itemize}

% Motivation
% (a) Diffusion (Vision, Graph), Autoregressive (Language, Vision)
% (b) Circuit Generation for Predefined Setting
% (c) Neural Architecture Search for Strict Logic Equivalence

% Contribution
% (a) Circuit Tokenizer (new transformer arch, training strategy)
% (b) CircuitAR (train and gen strategies, post-ar strategy)
% (c) Extensive Evaluation including BitD (Bit Distance) for Scalability


The paper focuses on EXPECT~\citep{fei-etal-2023-enhancing}, an explainable GEC dataset characterized by human-labeled \textit{evidence words} and \textit{grammatical error types} annotations, designed to assist language learners in understanding the corrections from GEC systems. These evidence words, referred to as extractive rationales\footnote{Following EXPECT~\citep{fei-etal-2023-enhancing}, we use the term ``evidence words'' throughout the paper except in Section~\ref{sec:related_works}.}, provide precise cues for corrections, thereby enabling learners to comprehend the rationale underlying the corrections. The error types within EXPECT encompass 15 categories grounded in pragmatism~\citep{skehan1998cognitive,shichungui}, facilitating learners in inferring abstract grammatical rules from particular errors through inductive reasoning. However, existing studies~\citep{song-etal-2024-gee} primarily concentrate on post hoc explanation, neglecting the interaction between the explanation and correction tasks as represented in Figure~\ref{fig:intro}.

To explore the interaction between explanation and correction tasks, we introduce \textbf{EXGEC} (\textbf{EX}plainable \textbf{G}rammatical \textbf{E}rror \textbf{C}orrection), a unified multi-task explainable GEC framework that formulates the multi-task problem as a generative task. The framework can jointly correct ungrammatical sentences, extract evidence words, and classify grammatical errors~\cite{zou2025revisiting} in different prediction orders within an architecture. Our research indicates that learning correction and explanation tasks together can be mutually beneficial and the prediction orders affect the task performance. More specifically, pre-explaining models achieve better correction performance but lower explanation performance compared to post-explaining models. Nevertheless, both models show improved or comparable correction and explanation performance compared to their respective baselines.

Moreover, we find that EXPECT is not an ideal dataset for explainable GEC. This is due to the presence of numerous unidentified grammatical errors in EXPECT, which would disturb the extraction of evidence words and the prediction of grammatical errors. Therefore, it will lead to a bias in the training and evaluation process. Consequently, we reconstruct EXPECT to correct the unidentified errors while ensuring each sentence contains only one distinct error~\citep{fei-etal-2023-enhancing}. The resulting dataset is called \Dataset{}. By training and evaluating EXGEC models on our proposed \Dataset{}, we can obtain unbiased results reflecting their real abilities in both the correction and the explanation tasks. In summary, our contributions are three folds:

\begin{itemize}
\item [(1)] We present EXGEC, a comprehensive framework that integrates correction and explanation components. This adaptable design facilitates the investigation of the interplay between correction and explanation tasks when utilizing various prediction sequences.

\item [(2)] We recognize a potential critical limitation in EXPECT and reconstruct it into \Dataset{}, thereby enhancing the training and evaluation framework for EXGEC models.

\item [(3)] We perform extensive experiments employing three language models (BART, T5, and Llama3) to demonstrate the beneficial interaction between the two tasks and substantiate the efficacy of our approach.

\end{itemize}

% The organization of this paper is as follows. Section~\ref{sec:related_works} provides a literature review of existing studies on explainable GEC and relevant techniques. Section~\ref{sec:dataset} delineates the drawback of EXPECT and the construction of the \Dataset{} dataset. Section~\ref{sec:method} presents the details of the presented framework (EXGEC). The experimental design and results are discussed in Section~\ref{sec:experiments}. Section~\ref{sec:analyses} provides a discussion and analysis of the results and limitations of the presented framework. Section~\ref{sec:discussion} discusses the implications and limitations of our work. Finally, Section~\ref{sec:conclusion} presents the conclusion and future work directions.

