\section{Introduction}\label{sec:intro}
Writing proficiently poses significant challenges for language learners who often struggle to produce grammatically correct and coherent texts~\citep{li2024rethinking,DBLP:conf/acl/LiZLLLSWLCZ22,DBLP:conf/emnlp/YeLZLM0023}. Therefore, GEC systems~\citep{bryant2023grammatical} are developed to detect and rectify all grammatical errors in texts~\citep{ye-etal-2023-mixedit,huang-etal-2023-frustratingly}. Research advancements in GEC encompass multi-language~\citep{rothe-etal-2021-simple,DBLP:journals/corr/abs-2307-09007,DBLP:conf/emnlp/MaLSZHZLLLCZS22}, multi-modality~\citep{fang-etal-2023-improving,DBLP:conf/emnlp/LiMZLLHLLC022,DBLP:conf/acl/LiXC0LMJLZZS24}, and document-level~\citep{yuan-bryant-2021-document,DBLP:conf/emnlp/DuW0D0LZVZSZGL024}.


However, the explainability of GEC~\citep{hanawa-etal-2021-exploring,DBLP:journals/corr/abs-2407-00924,DBLP:journals/corr/abs-2407-00934} remains under-explored due to its intrinsic difficulties. Given that neural GEC systems generally function as intricate black-box models, their internal processes are not transparent~\citep{zhao2023explainability}. The absence of explainability can result in inadequacies in educational scenarios, where L2 learners might find it difficult to completely understand outputs from GEC systems without knowing the rationale behind corrections. Providing corrections with explanations fosters appropriate trust by clarifying the linguistic principles and logical mechanisms underpinning model decisions in a comprehensible way, thereby aiding educationally K-12 students and L2 speakers~\citep{bitchener2005effect,sheen2007effect,DBLP:journals/eswa/LiMCHHLZS25,ye2025position}. Moreover, explainability offers insights that help identify unintentional biases and risks for researchers, functioning as a debugging tool to enhance model performance~\citep{ludan-etal-2023-explanation,DBLP:conf/icassp/ZhangLZMLCZ23}.

\section{Introduction}


\begin{figure}[t]
\centering
\includegraphics[width=0.6\columnwidth]{figures/evaluation_desiderata_V5.pdf}
\vspace{-0.5cm}
\caption{\systemName is a platform for conducting realistic evaluations of code LLMs, collecting human preferences of coding models with real users, real tasks, and in realistic environments, aimed at addressing the limitations of existing evaluations.
}
\label{fig:motivation}
\end{figure}

\begin{figure*}[t]
\centering
\includegraphics[width=\textwidth]{figures/system_design_v2.png}
\caption{We introduce \systemName, a VSCode extension to collect human preferences of code directly in a developer's IDE. \systemName enables developers to use code completions from various models. The system comprises a) the interface in the user's IDE which presents paired completions to users (left), b) a sampling strategy that picks model pairs to reduce latency (right, top), and c) a prompting scheme that allows diverse LLMs to perform code completions with high fidelity.
Users can select between the top completion (green box) using \texttt{tab} or the bottom completion (blue box) using \texttt{shift+tab}.}
\label{fig:overview}
\end{figure*}

As model capabilities improve, large language models (LLMs) are increasingly integrated into user environments and workflows.
For example, software developers code with AI in integrated developer environments (IDEs)~\citep{peng2023impact}, doctors rely on notes generated through ambient listening~\citep{oberst2024science}, and lawyers consider case evidence identified by electronic discovery systems~\citep{yang2024beyond}.
Increasing deployment of models in productivity tools demands evaluation that more closely reflects real-world circumstances~\citep{hutchinson2022evaluation, saxon2024benchmarks, kapoor2024ai}.
While newer benchmarks and live platforms incorporate human feedback to capture real-world usage, they almost exclusively focus on evaluating LLMs in chat conversations~\citep{zheng2023judging,dubois2023alpacafarm,chiang2024chatbot, kirk2024the}.
Model evaluation must move beyond chat-based interactions and into specialized user environments.



 

In this work, we focus on evaluating LLM-based coding assistants. 
Despite the popularity of these tools---millions of developers use Github Copilot~\citep{Copilot}---existing
evaluations of the coding capabilities of new models exhibit multiple limitations (Figure~\ref{fig:motivation}, bottom).
Traditional ML benchmarks evaluate LLM capabilities by measuring how well a model can complete static, interview-style coding tasks~\citep{chen2021evaluating,austin2021program,jain2024livecodebench, white2024livebench} and lack \emph{real users}. 
User studies recruit real users to evaluate the effectiveness of LLMs as coding assistants, but are often limited to simple programming tasks as opposed to \emph{real tasks}~\citep{vaithilingam2022expectation,ross2023programmer, mozannar2024realhumaneval}.
Recent efforts to collect human feedback such as Chatbot Arena~\citep{chiang2024chatbot} are still removed from a \emph{realistic environment}, resulting in users and data that deviate from typical software development processes.
We introduce \systemName to address these limitations (Figure~\ref{fig:motivation}, top), and we describe our three main contributions below.


\textbf{We deploy \systemName in-the-wild to collect human preferences on code.} 
\systemName is a Visual Studio Code extension, collecting preferences directly in a developer's IDE within their actual workflow (Figure~\ref{fig:overview}).
\systemName provides developers with code completions, akin to the type of support provided by Github Copilot~\citep{Copilot}. 
Over the past 3 months, \systemName has served over~\completions suggestions from 10 state-of-the-art LLMs, 
gathering \sampleCount~votes from \userCount~users.
To collect user preferences,
\systemName presents a novel interface that shows users paired code completions from two different LLMs, which are determined based on a sampling strategy that aims to 
mitigate latency while preserving coverage across model comparisons.
Additionally, we devise a prompting scheme that allows a diverse set of models to perform code completions with high fidelity.
See Section~\ref{sec:system} and Section~\ref{sec:deployment} for details about system design and deployment respectively.



\textbf{We construct a leaderboard of user preferences and find notable differences from existing static benchmarks and human preference leaderboards.}
In general, we observe that smaller models seem to overperform in static benchmarks compared to our leaderboard, while performance among larger models is mixed (Section~\ref{sec:leaderboard_calculation}).
We attribute these differences to the fact that \systemName is exposed to users and tasks that differ drastically from code evaluations in the past. 
Our data spans 103 programming languages and 24 natural languages as well as a variety of real-world applications and code structures, while static benchmarks tend to focus on a specific programming and natural language and task (e.g. coding competition problems).
Additionally, while all of \systemName interactions contain code contexts and the majority involve infilling tasks, a much smaller fraction of Chatbot Arena's coding tasks contain code context, with infilling tasks appearing even more rarely. 
We analyze our data in depth in Section~\ref{subsec:comparison}.



\textbf{We derive new insights into user preferences of code by analyzing \systemName's diverse and distinct data distribution.}
We compare user preferences across different stratifications of input data (e.g., common versus rare languages) and observe which affect observed preferences most (Section~\ref{sec:analysis}).
For example, while user preferences stay relatively consistent across various programming languages, they differ drastically between different task categories (e.g. frontend/backend versus algorithm design).
We also observe variations in user preference due to different features related to code structure 
(e.g., context length and completion patterns).
We open-source \systemName and release a curated subset of code contexts.
Altogether, our results highlight the necessity of model evaluation in realistic and domain-specific settings.






The paper focuses on EXPECT~\citep{fei-etal-2023-enhancing}, an explainable GEC dataset characterized by human-labeled \textit{evidence words} and \textit{grammatical error types} annotations, designed to assist language learners in understanding the corrections from GEC systems. These evidence words, referred to as extractive rationales\footnote{Following EXPECT~\citep{fei-etal-2023-enhancing}, we use the term ``evidence words'' throughout the paper except in Section~\ref{sec:related_works}.}, provide precise cues for corrections, thereby enabling learners to comprehend the rationale underlying the corrections. The error types within EXPECT encompass 15 categories grounded in pragmatism~\citep{skehan1998cognitive,shichungui}, facilitating learners in inferring abstract grammatical rules from particular errors through inductive reasoning. However, existing studies~\citep{song-etal-2024-gee} primarily concentrate on post hoc explanation, neglecting the interaction between the explanation and correction tasks as represented in Figure~\ref{fig:intro}.

To explore the interaction between explanation and correction tasks, we introduce \textbf{EXGEC} (\textbf{EX}plainable \textbf{G}rammatical \textbf{E}rror \textbf{C}orrection), a unified multi-task explainable GEC framework that formulates the multi-task problem as a generative task. The framework can jointly correct ungrammatical sentences, extract evidence words, and classify grammatical errors~\cite{zou2025revisiting} in different prediction orders within an architecture. Our research indicates that learning correction and explanation tasks together can be mutually beneficial and the prediction orders affect the task performance. More specifically, pre-explaining models achieve better correction performance but lower explanation performance compared to post-explaining models. Nevertheless, both models show improved or comparable correction and explanation performance compared to their respective baselines.

Moreover, we find that EXPECT is not an ideal dataset for explainable GEC. This is due to the presence of numerous unidentified grammatical errors in EXPECT, which would disturb the extraction of evidence words and the prediction of grammatical errors. Therefore, it will lead to a bias in the training and evaluation process. Consequently, we reconstruct EXPECT to correct the unidentified errors while ensuring each sentence contains only one distinct error~\citep{fei-etal-2023-enhancing}. The resulting dataset is called \Dataset{}. By training and evaluating EXGEC models on our proposed \Dataset{}, we can obtain unbiased results reflecting their real abilities in both the correction and the explanation tasks. In summary, our contributions are three folds:

\begin{itemize}
\item [(1)] We present EXGEC, a comprehensive framework that integrates correction and explanation components. This adaptable design facilitates the investigation of the interplay between correction and explanation tasks when utilizing various prediction sequences.

\item [(2)] We recognize a potential critical limitation in EXPECT and reconstruct it into \Dataset{}, thereby enhancing the training and evaluation framework for EXGEC models.

\item [(3)] We perform extensive experiments employing three language models (BART, T5, and Llama3) to demonstrate the beneficial interaction between the two tasks and substantiate the efficacy of our approach.

\end{itemize}

% The organization of this paper is as follows. Section~\ref{sec:related_works} provides a literature review of existing studies on explainable GEC and relevant techniques. Section~\ref{sec:dataset} delineates the drawback of EXPECT and the construction of the \Dataset{} dataset. Section~\ref{sec:method} presents the details of the presented framework (EXGEC). The experimental design and results are discussed in Section~\ref{sec:experiments}. Section~\ref{sec:analyses} provides a discussion and analysis of the results and limitations of the presented framework. Section~\ref{sec:discussion} discusses the implications and limitations of our work. Finally, Section~\ref{sec:conclusion} presents the conclusion and future work directions.

