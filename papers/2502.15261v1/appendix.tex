\begin{table*}[!t]
    \centering
    % \begin{tabular}{ c l c m{25em} }
    \begin{tabular}{ m{10em} c c c}
    % \toprule[1pt]
    % \thickhline
    \toprule
    \textbf{Defense}  & \textbf{Adaptive Attack} &\textbf{ASR-defense} &\textbf{ASR-adaptive attack}\\
    \midrule
    -&-&\multicolumn{2}{c}{0.52}\\
    \midrule 
    Fine-tuned detector & \multirowcell{2.5}{Multi-objective GCG} & 0.19 & 0.77\\
    \cmidrule(lr){1-1}\cmidrule(lr){3-4}
    LLM-based detector & & 0.26 & 0.66\\
    \midrule
    Perplexity filtering &AutoDAN&0.41&0.78\\
    \midrule
    Instructional prevention &\multirowcell{7}{GCG}& 0.48 & 0.83\\
   \cmidrule(lr){1-1}\cmidrule(lr){3-4}
     Data prompt isolation & & 0.45 & 0.88\\
    \cmidrule(lr){1-1}\cmidrule(lr){3-4}
    Sandwich prevention & & 0.15 & 0.63\\
    \cmidrule(lr){1-1}\cmidrule(lr){3-4}
     Paraphrasing && 0.56&0.58\\
    \cmidrule(lr){1-1}\cmidrule(lr){3-4}
    Adversarial finetuning& &0.05&0.53\\
    \bottomrule
    \end{tabular}
    \caption{ASR results}
    \label{tab:results}
\end{table*}
\begin{table*}[!tbh]
\centering
\renewcommand{\arraystretch}{1.2}
% \renewcommand{\tabcolsep}{4pt}
%\caption{Hyper-parameter values used in our experiments.}
\label{tab:hp}
\scalebox{0.80}{
%\begin{tabular}{CLCCC}
\begin{tabular}{clcccc}
\toprule

& \textbf{Configuration}  & \textbf{BART}  & \textbf{T5}  & \textbf{LLaMA3} \\ 
\midrule

\multirow{9}{*}{\textbf{Training}}
& Backbone  & BART-Large & T5-Base & LLaMA3-8B  \\
& Epochs    & 60 & 80 & 3 (LoRA) \\
& Batch size per GPU & 4096 tokens & 4096 tokens & 8192 tokens \\
& Gradient Accumulation & 4 & 4 & 8 \\

\cdashline{2-5}

& Loss weight $\lambda$& \multicolumn{3}{c}{ 1.0 } \\
& Learning rate &  \multicolumn{3}{c}{$3 \times 10^{-5}$ } \\
& Devices   & \multicolumn{3}{c}{1 Tesla A100 GPU (80GB)} \\
& \multirow{2}{*}{Optimizer} & \multicolumn{3}{c}{Adam \citep{kingma2014adam}}  \\
& & \multicolumn{3}{c}{($\beta_1=0.9,\beta_2=0.999,\epsilon=1\times 10^{-8}$) } \\

\midrule

\multirow{2}{*}{\textbf{Inference}}
& Beam size          & 5 & 5 & 5 \\
& Max length         & 256 & 512 & 512 \\

\bottomrule
\end{tabular}}
\caption{Hyper-parameters used in our experiments.}
\label{tab:hyper-parameter}
\end{table*}

\begin{table}[tp!]
\renewcommand{\arraystretch}{1.3}
\renewcommand{\tabcolsep}{10pt}

\resizebox{\linewidth}{!}{
\begin{tabular}{lcc}
\toprule

$\lambda$  &
\textbf{Cor. (P / R / $\textbf{F}_{0.5}$)$\uparrow$}  &
\textbf{Exp. (P / R / $\textbf{F}_{1}$ / $\textbf{F}_{0.5}$ / Acc)$\uparrow$}  \\

\midrule

0.5  &  35.40 / 38.03 / 35.90  &
39.77 / 38.88 / 39.32 / 39.59 / 32.02  \\

1.0  &  36.34 / 40.15 / \textbf{37.05}  &
48.95 / 42.72 / \textbf{45.63} / \textbf{47.56} / \textbf{40.32}  \\

1.5  &  36.03 / 38.42 / 36.49  &
43.90 / 42.82 / 43.35 / 43.68 / 36.88  \\

2.0  &  35.41 / 38.61 / 36.00  &
47.98 / 42.86 / 45.28 / 46.86 / 40.07  \\

\bottomrule

\end{tabular}}
\caption{Results of \textit{post-explaining} models with varying loss weights $\lambda$ on \textbf{\Dataset{}-\textit{dev}}.}
\label{tab:exp_loss}
\end{table}


\section{Fine-tuning Details}
\label{app:training_details}
We update all parameters of BART-Large and T5-Base during fine-tuning. For Llama3-8B, we leverage LoRA~\citep{hu2022lora} to implement parameter-efficient fine-tuning. Detailed hyper-parameter configurations are provided in Table~\ref{tab:hyper-parameter}. 

\section{Detailed Results}
\label{app:detailed_results}
Table~\ref{tab:detailed_results} lists detailed results on the \Dataset{}-\textit{dev} set, providing further insight into model behaviors in different training settings. Infusion (with Evidence and with Evidence \& Type) models gain higher correction TP but lower FP and FN, demonstrating that evidence words significantly benefit the correction task. Additionally, pre-explaining models tend to extract more evidence words ($\approx$ 4200\footnote{This is equal to TP plus FP.}) but predict fewer correction edits ($\approx$ 2300).
On the other hand, post-explaining models are declined to extract fewer evidence words ($\approx$ 3700), but predict more correction edits ($\approx$ 2900). We speculate that models are more likely to make predictions when prior information is unavailable. However, models become more cautious with the available prior information. Therefore, pre-explaining models achieve higher correction precision, whereas post-explaining models exhibit higher correction recall.

\section{Impact of Loss Weighting}
\label{app:loss_weight}
This section examines the balance between learning for correction and explanation tasks by adjusting the loss weight $\lambda$ in Equation~\eqref{eq:loss}. An elevated value of $\lambda$ indicates an increased focus on learning the explanation task. We experiment with various models employing different loss weights $\lambda$ selected from $\{0.5, 1.0, 1.5, 2.0\}$, and present our results in \Dataset{}-$\textit{dev}$ in Table~\ref{tab:exp_loss}. The findings indicate that focusing predominantly on a single task may cause a deterioration in the performance of both tasks. We hypothesize that with a reduced $\lambda$, the weak supervisory signals in the explanations hinder the model's capacity to generate accurate explanations, thereby compromising the correction learning process. Conversely, a larger $\lambda$ might neglect correction learning, leading to subpar explanation performance since the quality of explanations depends on the predicted corrections in post-explaining models. An optimal $\lambda$ must be chosen to attain a balance between the learning of both tasks and achieve mutually beneficial results.
