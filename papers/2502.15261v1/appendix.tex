\begin{table*}[!tb]
\renewcommand{\arraystretch}{1.2}
\centering

\resizebox{1.0\linewidth}{!}{
\begin{tabular}{lcccc}
\toprule

\textbf{System}
&  \textbf{Cor. (TP / FP / FN)}
&  \textbf{Cor. (P / R / $\textbf{F}_{0.5}$)$\uparrow$}
&  \textbf{Exp. (TP / FP / FN)}
&  \textbf{Exp. (P / R / $\textbf{F}_{1}$ / $\textbf{F}_{0.5}$ / Acc)$\uparrow$} \\

\hline

\textbf{BART Baseline} &
910 / 1604 / 1695  &
36.14 / 34.87 / 35.88  &
-  &
-  \\

\hline

\textbf{Infusion}  \\

\hspace{0.3cm} \textbf{+ Evidence}   &
1149 / 1345 / 1459  &
45.78 / 44.55 / \textbf{45.53}  &
-  &
-  \\

\hspace{0.3cm} \textbf{+ Type}   &
879 / 1608 / 1716  &
35.31 / 47.87 / 35.22  &
-  &
-  \\

\hspace{0.3cm} \textbf{+ Evidence\&Type}   &
1244 / 1600 / 1351  &
44.28 / 47.55 / 44.90  &
-  &
-  \\

\hline 

\textbf{Self-Rationalization}  \\

\hspace{0.3cm} \textbf{Pre-explaining}  &
885 / 1437 / 1721  &
38.25 / 34.18 / \textbf{37.36}  &
1525 / 2701 / 2737  &
36.01 / 35.58 / 35.79 / 35.92 / 26.56  \\


\hspace{0.3cm} \textbf{Post-explaining}  &
1038 / 1821 / 1548  &
36.34 / 40.15 / 37.05  &
1829 / 1911 / 2456  &
48.95 / 42.72 / \textbf{45.63} / \textbf{47.56} / \textbf{40.32}  \\

\bottomrule

\end{tabular}}

\caption{Detailed results of BART-Large on \Dataset{}-\textit{dev}, including the number of True Positive (TP), False Positive (FP), and False Negative (FP) for both correction and explanation tasks. TP / FP / FN counts are taken from one checkpoint, while P / R / F / Acc scores are averaged over three runs.}
\label{tab:detailed_results}
\end{table*}

\begin{table*}[h]
    \centering
    %\small % Apply small font size to the entire table
    \begin{tabular}{c c c c c| c c c c }
    \toprule
        \multicolumn{5}{c}{Expriment Settings} & $\alpha$ & $\lambda_1$ &$\lambda_2$ & \# Training Steps \\
        Attack & Rt. & Rr. & Gen. & Task\\
        \midrule
        LPA-Rt & CLIP & - & - & MMQA&0.005&-&-&50 \\
        LPA-Rt & CLIP & - & - & WebQA&0.005&-&-&50 \\
        GPA-Rt & CLIP & - & - &MMQA&0.01&-&-& 500\\
        GPA-Rt & CLIP & - & - &WebQA&0.01&-&-&500 \\
        GPA-RtRrGen& CLIP& Llava &Llava &MMQA & 0.01&0.2&0.3&2000\\
                GPA-RtRrGen& CLIP& Qwen &Qwen &MMQA & 0.005&0.2&0.3&2500 \\
                GPA-RtRrGen& CLIP& Llava &Qwen &MMQA & 0.01&0.08&0.9&2500\\
        GPA-RtRrGen& CLIP& Llava &Llava &WebQA & 0.01&0.2&0.3&2000\\
                GPA-RtRrGen& CLIP& Qwen &Qwen &WebQA & 0.01&0.3&0.3&1000\\
                GPA-RtRrGen& CLIP& Llava &Qwen &WebQA & 0.01&0.1&0.8&3000\\
         \bottomrule
    \end{tabular}%
    \caption{Hyper-parameters for training adversarial images.}
    \vspace{-0.1in}
    \label{tab:hyper_parameters}
\end{table*}

\begin{table}[tp!]
\renewcommand{\arraystretch}{1.3}
\renewcommand{\tabcolsep}{10pt}

\resizebox{\linewidth}{!}{
\begin{tabular}{lcc}
\toprule

$\lambda$  &
\textbf{Cor. (P / R / $\textbf{F}_{0.5}$)$\uparrow$}  &
\textbf{Exp. (P / R / $\textbf{F}_{1}$ / $\textbf{F}_{0.5}$ / Acc)$\uparrow$}  \\

\midrule

0.5  &  35.40 / 38.03 / 35.90  &
39.77 / 38.88 / 39.32 / 39.59 / 32.02  \\

1.0  &  36.34 / 40.15 / \textbf{37.05}  &
48.95 / 42.72 / \textbf{45.63} / \textbf{47.56} / \textbf{40.32}  \\

1.5  &  36.03 / 38.42 / 36.49  &
43.90 / 42.82 / 43.35 / 43.68 / 36.88  \\

2.0  &  35.41 / 38.61 / 36.00  &
47.98 / 42.86 / 45.28 / 46.86 / 40.07  \\

\bottomrule

\end{tabular}}
\caption{Results of \textit{post-explaining} models with varying loss weights $\lambda$ on \textbf{\Dataset{}-\textit{dev}}.}
\label{tab:exp_loss}
\end{table}


\section{Fine-tuning Details}
\label{app:training_details}
We update all parameters of BART-Large and T5-Base during fine-tuning. For Llama3-8B, we leverage LoRA~\citep{hu2022lora} to implement parameter-efficient fine-tuning. Detailed hyper-parameter configurations are provided in Table~\ref{tab:hyper-parameter}. 

\section{Detailed Results}
\label{app:detailed_results}
Table~\ref{tab:detailed_results} lists detailed results on the \Dataset{}-\textit{dev} set, providing further insight into model behaviors in different training settings. Infusion (with Evidence and with Evidence \& Type) models gain higher correction TP but lower FP and FN, demonstrating that evidence words significantly benefit the correction task. Additionally, pre-explaining models tend to extract more evidence words ($\approx$ 4200\footnote{This is equal to TP plus FP.}) but predict fewer correction edits ($\approx$ 2300).
On the other hand, post-explaining models are declined to extract fewer evidence words ($\approx$ 3700), but predict more correction edits ($\approx$ 2900). We speculate that models are more likely to make predictions when prior information is unavailable. However, models become more cautious with the available prior information. Therefore, pre-explaining models achieve higher correction precision, whereas post-explaining models exhibit higher correction recall.

\section{Impact of Loss Weighting}
\label{app:loss_weight}
This section examines the balance between learning for correction and explanation tasks by adjusting the loss weight $\lambda$ in Equation~\eqref{eq:loss}. An elevated value of $\lambda$ indicates an increased focus on learning the explanation task. We experiment with various models employing different loss weights $\lambda$ selected from $\{0.5, 1.0, 1.5, 2.0\}$, and present our results in \Dataset{}-$\textit{dev}$ in Table~\ref{tab:exp_loss}. The findings indicate that focusing predominantly on a single task may cause a deterioration in the performance of both tasks. We hypothesize that with a reduced $\lambda$, the weak supervisory signals in the explanations hinder the model's capacity to generate accurate explanations, thereby compromising the correction learning process. Conversely, a larger $\lambda$ might neglect correction learning, leading to subpar explanation performance since the quality of explanations depends on the predicted corrections in post-explaining models. An optimal $\lambda$ must be chosen to attain a balance between the learning of both tasks and achieve mutually beneficial results.
