\begingroup
\begin{spacing}{1.0}
\begin{longtable}{p{3.5cm}p{3.9cm}p{5.5cm}}
    \caption{Key Features for Envisioned Social Media from Participant Interviews} \\
    \toprule
    \textbf{Desired Feature} & \textbf{Key Idea} & \textbf{Example Quote} \\
    \midrule
    \endfirsthead
    \toprule
    \textbf{Desired Feature} & \textbf{Key Idea} & \textbf{Example Quote} \\
    \midrule
    \endhead
    \midrule
    \multicolumn{3}{r}{\textit{Continued on the next page}} \\
    \midrule
    \endfoot
    \bottomrule
    \endlastfoot    
    \multicolumn{3}{c}{
        \parbox{13cm}{\centering 
        \textbf{Immersive Experiences Through Spatial Elements (Section~\ref{lab:4-1})} \\ 
        \tablequote{In 2D, you're just observing\ldots{} But in a 3D setting, even if you don't leave physical footprints, in a way, they're still there.}{01}
        }
    } \\
    \midrule
    {\textbf{[Section~\ref{lab:4-1-1}]}}\newline{}Bringing Realism and Presence Through 3D & Emphasized need for physical presence and embodied interactions beyond flat screens & \tablequote{You turn on your voice, chat\ldots{} walking around a physical world\ldots{} It feels like you're actually in the area of these people; versus traditional social media like Instagram, there's a separation of the screen.}{06} \\
    {\textbf{[Section~\ref{lab:4-1-2}]}}\newline{}Personal Spaces and Community-Centered Landscapes & Desired personal homes and neighborhoods that mirror real-world social structures & \tablequote{You each get your own house\ldots{} anything you want.}{03} \\
    {\textbf{[Section~\ref{lab:4-1-3}]}}\newline{}Holographic Display and Virtual Reality & Sought more natural and immersive interfaces beyond traditional devices & \tablequote{If you're already carrying around like a wand everywhere with you, why would you also want to carry a phone?}{01} \\
    \midrule
    \multicolumn{3}{c}{
        \parbox{13cm}{\centering 
        \textbf{Organic and Intentional Social Interactions (Section~\ref{lab:4-2})} \\ 
        \tablequote{The antonym to pressure, stress-free, community-oriented, wholesome, accessible, and accepting.}{09}
        }
    } \\
    \midrule
    {\textbf{[Section~\ref{lab:4-2-1}]}}\newline{}Flexible and Safe Spaces for Meaningful Conversations & Valued direct conversations in protected spaces & \tablequote{Actual conversations for real connections, individualized chatting\ldots{} to get to know people\ldots{} rather than just posting on a thread.}{22} \\
    {\textbf{[Section~\ref{lab:4-2-2}]}}\newline{}Community Gathering Places & Desired spaces for shared activities and casual encounters & \tablequote{a little card game area, shared experiences, something fun}{06} \\
    {\textbf{[Section~\ref{lab:4-2-2}] (1)}}\newline{}Activity Spaces for Shared Experiences & Valued collaborative activities and shared projects & \tablequote{Part of the fun is not only are we actually in the space, but doing something together\ldots{} we're experiencing something together, almost.}{06} \\
    {\textbf{[Section~\ref{lab:4-2-2}] (2)}}\newline{}(Public) Third Places for Casual Encounters & Desired spaces for serendipitous interactions & \tablequote{You strike a conversation because you're both picking up the same shirt\ldots{} little connections like that happen just because you're in a public space.}{03} \\
    {\textbf{[Section~\ref{lab:4-2-3}]}}\newline{}Ambient Co-Presence & Valued low-intensity connections through shared virtual spaces & \tablequote{At one point, literally, we sat for an hour in silence, listening to a whole playlist\ldots{} we were just sitting there listening, vibing the music.}{19} \\
    {\textbf{[Section~\ref{lab:4-2-4}]}}\newline{}Effortful and Intentional Navigation & Saw physical movement as a meaningful way to engage & \tablequote{Ideally, SMH would have some way that you specifically have to search for stuff\ldots{} where you have to intentionally go in looking for the account or the post versus it just being handed to you.}{05} \\
    \midrule
    \multicolumn{3}{c}{
        \parbox{13cm}{\centering 
        \textbf{Expressive and Lower-Stakes Sharing (Section~\ref{lab:4-3})} \\ 
        \tablequote{I'm into multiple things---you could have a room for Harry Potter, a room for neuroscience, and a room for another fandom or interest.}{01}
        }
    } \\
    \midrule
    {\textbf{[Section~\ref{lab:4-3-1}]}}\newline{}Expression Through Personal Area and Avatar Customization & Wanted customizable avatars and environments & \tablequote{decorat[ing] a house or customizing an avatar allows others to get to know more about my personality\ldots{} through my style.}{16} \\
    {\textbf{[Section~\ref{lab:4-3-2}]}}\newline{}Real-Time Sharing: Capturing States and Changes & Desired ways to share real-time states & \tablequote{what I'm doing at the moment. For example, if I'm at the shops, maybe I'd want to share that}{04} \\
    {\textbf{[Section~\ref{lab:4-3-3}]}}\newline{}Sharing Memories: Multisensory and Emotional Experiences & Wanted to share memories in an immersive and clearer way & \tablequote{jumping into a pool or taking the first bite of an ice cream sundae\ldots{} from your brain, pull it out, and just be able to share that}{04} \\
    \midrule
    \multicolumn{3}{c}{
        \parbox{13cm}{\centering 
        \textbf{Granular, Intuitive, and Fun Privacy Mechanisms (Section~\ref{lab:4-4})}\\ 
        \tablequote{If people are posting more like their genuine real life and their experiences, privacy would be a little bit more complicated.}{11}
        }
    } \\
    \midrule
    {\textbf{[Section~\ref{lab:4-4-1}]}}\newline{}Space-Based Privacy & Wanted privacy boundaries based on spatial metaphors & \tablequote{A bedroom is a respected personal space\ldots{} If you label it as a bedroom, it just feels more intimate.}{13} \\
    {\textbf{[Section~\ref{lab:4-4-2}]}}\newline{}Playful Privacy & Desired engaging privacy mechanisms with elements of fantasy & \tablequote{Secret knock or an object that reveals a secret passage to\ldots{} secret rooms.}{11} \\
    {\textbf{[Section~\ref{lab:4-4-3}]}}\newline{}Contextual Privacy & Wanted context-aware privacy settings & \tablequote{Each of them should have their own setting, because I feel like everybody has different levels of what they want to keep to themselves.}{14} \\
    {\textbf{[Section~\ref{lab:4-4-4}]}}\newline{}Invisibility & Sought robust blocking and visibility control & \tablequote{I don't wanna be able to tell that the person I blocked is even around me.}{22} \\
    {\textbf{[Section~\ref{lab:4-4-5}]}}\newline{}Age-Appropriate Spaces & Emphasized need for age separation & \tablequote{People who are 50 should not be engaging with people who are 13.}{21} \\
    \midrule
    \multicolumn{3}{c}{
        \parbox{13cm}{\centering 
        \textbf{Refocusing Interaction Priorities for Meaningful Engagement (Section~\ref{lab:4-5})} \\ 
        \tablequote{We need old Instagram back in 2010, when my mom was posting me and my brother just for the family and no one else. We didn't need the explore page.}{19}
        }
    } \\
    \midrule
    {\textbf{[Section~\ref{lab:4-5-1}]}}\newline{}Intentional Content Consumption & Desired more control over content exposure & \tablequote{You press which floor you want to go to, and then, when you ride that elevator, it just goes straight to that floor.}{09} \\
    {\textbf{[Section~\ref{lab:4-5-2}]}}\newline{}Reduced Celebrity and Commercial Influence & Wanted focus on real friends over celebrities & \tablequote{If I'm just gonna open the app for like 5 seconds\ldots{} I'm gonna wanna see my best friend's post, not some girl I talked to once 3 years ago.}{07} \\
    {\textbf{[Section~\ref{lab:4-5-3}]}}\newline{}Spatial Representation of Relationships & Sought physical distance to represent emotional closeness & \tablequote{The top 7 people you're closest to would be the 7 houses in your cul-de-sac area.}{15} \\
    {\textbf{[Section~\ref{lab:4-5-4}]}}\newline{}Intentional and Meaningful Use of Time & Valued active engagement over passive consumption & \tablequote{Anytime not scrolling and instead talking to people feels worthwhile, even if it's just light conversation.}{06} \\
\end{longtable}
\end{spacing}
\endgroup
\label{tab:results-overview}
\section{EVALUATION}
We evaluate PerfSeer by addressing the following research questions (RQs).

\textbf{RQ1:} How effective are the selected features and optimization components? 

\textbf{RQ2:} How does SeerNet compare with baseline models? 

\textbf{RQ3:} How effective is the multi-metric performance prediction model, SeerNet-Multi? 

\textbf{RQ4:} What is the application scope and overhead of PerfSeer? 

\subsection{Evaluation Setup}
\subsubsection{Training Settings.}\label{sec:train-setting}
The dataset is divided into 2:1:1 for training, validation, and testing. We use a batch size of 128 and an initial learning rate of 1e-3, halving it after five epochs without improvement, down to 1e-6. Training runs for up to 500 epochs, with Mean Squared Error (MSE) as the loss function and Adam as the optimizer.

\subsubsection{Evaluation Metrics.}\label{sec:metrics}
To ensure consistency across metrics with varying value ranges, we use percentage errors. The metrics include MAPE, Root Mean Square Percentage Error (RMSPE), and accuracy within a relative error of \(x\%\) (x\%Acc) \cite{Brp-nas}.

%% SeerNet
\subsection{Ablation Study (RQ1)}\label{sec:ablation}
The accuracy of the performance prediction depends on both representation and prediction. We conducted an ablation study to validate the features selection and construction (\emph{representation}) and the prediction model design (\emph{prediction}).
\subsubsection{Ablation Study of Features.} \label{sec:features ablation}
From the results shown in Table \ref{tab:ablation of features}, we observe the following:

\emph{Features Importance.} 
Node features are much more important than global features, which are in turn more important than edge features. 
The mean MAPE of SeerNet decreases from 58.23\% to 28.41\% as the node features are used increasingly, as they directly correspond to each operation node.
Adding the global features further improves accuracy, significantly reducing the mean MAPE from 25.40\% to 7.41\%, as they provide an overall representation of the model.
Edge features, though less important, improved MAPE slightly from 28.41\% to 25.40\% by providing additional memory access information, further enhancing prediction accuracy.

\emph{Feature Selection.} 
Each group of the node, edge, and global features we selected improved the performance prediction accuracy, demonstrating their relevance to model performance.
In contrast, node categories, which are commonly used in other predictors, reduced accuracy. We consider SeerNet extracts category-related information from other semantically rich features, so we excluded node categories from our feature set.

\emph{Feature Construction.} 
We constructed several unique and effective features. 
For node features, we included arithmetic intensity and computation/memory access proportions, reducing the mean MAPE from 35.41\% to 28.41\%. 
For global features, we introduced graph profiles, computation and memory access trend statistics, tensor size delivered per-edge, and arithmetic intensity, reducing the mean MAPE from 25.40\% to 7.53\%. 
These features, not used by any previous predictors, significantly enhance prediction accuracy.

\subsubsection{Ablation Study of Components.}\label{sec:ablation component}
From the results shown in Table \ref{tab:ablation of components}, we observe the following:

\emph{SynMM.}
Using SynMM to aggregate the node features reduces the mean MAPE from 7.41\% to 6.10\%. This demonstrates that SynMM combines max and mean aggregation to create a more comprehensive and robust representation, thereby enhancing prediction accuracy.

\emph{GNPB.}
With the introduction of GNPB, the mean MAPE further decreases from 6.10\% to 5.14\%. This result proves that GNPB enables complementary learning between the node and global features, enriching both perspectives and further improving prediction accuracy.
% The experiments verify that the optimization components can capture the performance information of a model effectively.

% \vspace{-0.3cm}
\setlength{\tabcolsep}{0pt}
\renewcommand{\arraystretch}{0.95}
\setcounter{table}{1}
\begin{table*}[b]
    \small
    % \vspace{-5mm}
    \caption{Parametric models included in the experiments. Cond. = conditioning method, R.F. = receptive field in samples.
    PEQ = Parametric EQ, G = Gain, O = Offset, MLP = Multilayer Perceptron, RNL = Rational Non Linearity. Controllers: 
    .s = static, .d = dynamic, .sc = static conditional, .dc = dynamic conditional}
    \label{tab:models}
    % \vspace{-2mm}
    \centerline{
        \begin{tabular}{L{2.8cm}C{1.3cm}R{1.1cm}C{1.1cm}C{1.1cm}C{1.3cm}C{1.5cm}R{1.4cm}R{1.3cm}R{1.3cm}}
            \hline
            \hline
            Model
                & Cond.
                    & R.F.
                        & Blocks
                            & Kernel
                                & Dilation
                                    & Channels
                                        & \# Params 
                                            & FLOP/s 
                                                & MAC/s\\ 
            \hline
            TCN-F-45-S-16 & FiLM & 2047 & 5 & 7 & 4 & 16 & 15.0k & 736.5M & 364.3M\\
            TCN-TF-45-S-16 & TFiLM & 2047 & 5 & 7 & 4 & 16 & 42.0k & 762.8M & 364.2M\\
            TCN-TTF-45-S-16 & TTFiLM & 2047 & 5 & 7 & 4 & 16 & 17.3k & 744.0M & 367.4M\\
            TCN-TVF-45-S-16 & TVFiLM & 2047 & 5 & 7 & 4 & 16 & 17.7k & 740.4M & 366.2M\\
            \hline
            \hline
        \end{tabular}
    }
    \centerline{
        \begin{tabular}{L{2.8cm}C{1.3cm}R{1.1cm}C{1.2cm}C{2.3cm}C{1.5cm}R{1.4cm}R{1.3cm}R{1.3cm}}
            Model
                & Cond.
                    & R.F.
                        & Blocks
                            & State Dimension
                                & Channels
                                    & \# Params
                                        & FLOP/s 
                                            & MAC/s\\ 
            \hline
            S4-F-S-16 & FiLM & - & 4 & 4 & 16 & 8.9k & 135.2M & 53.8M\\
            S4-TF-S-16 & TFiLM & - & 4 & 4 & 16 & 30.0k & 155.6M & 53.8M\\
            S4-TTF-S-16 & TTFiLM & - & 4 & 4 & 16 & 10.2k & 141.0M & 56.3M\\
            S4-TVF-S-16 & TVFiLM & - & 4 & 4 & 16 & 11.6k & 138.9M & 55.3M\\
            \hline
            \hline
        \end{tabular}
    }
    \centerline{
        \begin{tabular}{L{3cm}C{7.2cm}R{1.4cm}R{1.3cm}R{1.3cm}}
            Model
                & Signal Chain
                    & \# Params
                        & FLOP/s 
                            & MAC/s\\
            \hline
            GB-C-DIST-MLP & PEQ.sc $\rightarrow$ G.sc $\rightarrow$ O.sc $\rightarrow$ MLP $\rightarrow$ G.sc $\rightarrow$ PEQ.sc & 4.5k & 202.8M & 101.4M\\
            GB-C-DIST-RNL & PEQ.sc $\rightarrow$ G.sc $\rightarrow$ O.sc $\rightarrow$ RNL $\rightarrow$ G.sc $\rightarrow$ PEQ.sc & 2.3k & 920.5k & 4.3k\\
            \hline
            GB-C-FUZZ-MLP & PEQ.sc $\rightarrow$ G.sc $\rightarrow$ O.dc $\rightarrow$ MLP $\rightarrow$ G.sc $\rightarrow$ PEQ.sc & 4.2k & 202.8M & 101.4M\\
            GB-C-FUZZ-RNL & PEQ.sc $\rightarrow$ G.sc $\rightarrow$ O.dc $\rightarrow$ RNL $\rightarrow$ G.sc $\rightarrow$ PEQ.sc & 2.0k & 988.9k & 3.6k\\
            \hline
            \hline
        \end{tabular}
    }
    % \vspace{-4mm}
\end{table*}
\begin{table}[!htb]

\belowrulesep=0pt
\aboverulesep=0pt
\renewcommand{\arraystretch}{1}
    \resizebox{1.0\columnwidth}{!}{\begin{tabular}{lcccccccc}
    
    \specialrule{1.5pt}{0pt}{0pt}
    % summury title
    \multirow{3}{*}{Method} & \multicolumn{7}{c}{Accuracy (MAPE[\%]$\downarrow$)} & \multirow{3}{*}{\makecell{Params$\downarrow$ \\ {[}M{]}}} \\
    \cmidrule(lr){2-8}
    
    & \multicolumn{3}{c}{Training} & \multicolumn{3}{c}{Inference} & \multirow{2}{*}{\makecell{\textit{Mean} \\ (6 metrics)}}  \\
    \cmidrule(lr){2-4} \cmidrule(lr){5-7}
    
    % title
    & {Util} & {Mem} & {Time} & {Util} & {Mem} & {Time} & \multicolumn{1}{c}{} \\
    \specialrule{1.0pt}{0pt}{0pt}
    
    MLP-Node (MLP) & 21.27 & 5.41 & 12.37 & 35.21 & 6.38 & 23.77 & \textit{17.40} & 4.15 \\ % 4,149,249
    PMGNS (GraphSAGE) & 8.23 & 9.62 & 10.53 & 6.71 & 8.76 & 13.72 & \textit{9.60} & 3.45 \\ % 3,448,321
    Eagle-p (GCN) & 82.45 & 82.30 & 66.66 & 71.85 & 48.51 & 94.80 & \textit{74.43} & 1.11 \\ % 1,107,001
    Eagle-s (GCN) & 6.14 & 4.25 & 8.60 & 5.22 & 5.27 & 16.63 & \textit{7.69} & 1.10 \\ % 1,099,801
    \textbf{SeerNet (Our)} & \textbf{4.94} & \textbf{2.47} & \textbf{6.71} & \textbf{4.37} & \textbf{3.46} & \textbf{8.91} & \textbf{\textit{5.14}} & \textbf{1.02} \\ % 1,015,831

    \specialrule{1.5pt}{0pt}{0pt}
    
  \end{tabular}}
\caption{SeerNet comparison on our dataset}
\label{tab:comparison our}
\end{table}

\begin{table*}[!htb]
\belowrulesep=0pt
\aboverulesep=0pt
\newcolumntype{C}{>{\centering\arraybackslash}p{30pt}}
\renewcommand{\arraystretch}{1}
    \resizebox{2.0\columnwidth}{!}{
    \begin{tabular}{llCCCCCCCCCCCC}

    \specialrule{1.5pt}{0pt}{0pt}
    % summury title
    \multirow{3}{*}{Method} & \multirow{3}{*}{Model} & \multicolumn{3}{c}{Mobile CPU (CortexA76)} & \multicolumn{3}{c}{Mobile GPU (Adreno 640)} & \multicolumn{3}{c}{Intel VPU (MyriadX)} & \multicolumn{3}{c}{\textit{Mean (3 devices)}} \\
    
    % title
    & & RMSPE$\downarrow$ & Acc$\uparrow$ & Acc$\uparrow$ & RMSPE$\downarrow$ & Acc$\uparrow$ & Acc$\uparrow$ & RMSPE$\downarrow$ & Acc$\uparrow$ & Acc$\uparrow$ & RMSPE$\downarrow$ & Acc$\uparrow$ & Acc$\uparrow$ \\
    & & [\%] & [5\%] & [10\%] & [\%] & [5\%] & [10\%] & [\%] & [5\%] & [10\%] & [\%] & [5\%] &  [10\%] \\
    \specialrule{1.5pt}{0pt}{0pt}
    
    % content
    \multirow{13}{*}{\textit{nn-Meter}} & AlexNets & 3.90 & 81.0 & 98.6 & 5.32 & 72.0 & 94.0 & 10.74 & 23.4 & 60.9 & \textit{6.65} & \textit{58.8} & \textit{84.5} \\
    & DenseNets & 2.76 & 93.1 & 99.9 & 4.52 & 68.6 & 99.9 & 5.89 & 75.6 & 86.3 & \textit{4.39} & \textit{79.1} & \textit{95.4} \\
    & GoogleNets & 3.27 & 85.9 & 100.0 & 1.35 & 100.0 & 100.0 & 5.86 & 39.7 & 98.4 & \textit{3.49} & \textit{75.2} & \textit{99.5} \\
    & MnasNets & 5.54 & 50.9 & 99.2 & 1.86 & 100.0 & 100.0 & 4.34 & 77.3 & 97.7 & \textit{3.91} & \textit{76.1} & \textit{99.0} \\
    & MobileNetv1s & 4.98 & 63.8 & 97.8 & 2.56 & 96.9 & 100.0 & 5.90 & 54.2 & 93.3 & \textit{4.48} & \textit{71.6} & \textit{97.0} \\
    & MobileNetv2s & 4.84 & 67.6 & 97.7 & 3.93 & 80.0 & 99.0 & 4.26 & 78.3 & 97.6 & \textit{4.34} & \textit{75.3} & \textit{98.1} \\
    & MobileNetv3s & 4.34 & 73.8 & 99.0 & 4.02 & 84.4 & 100.0 & 5.72 & 47.6 & 98.5 & \textit{4.69} & \textit{68.6} & \textit{99.1} \\
    & NASBench201 & 3.51 & 82.4 & 99.9 & 3.80 & 75.9 & 100.0 & 18.20 & 19.3 & 40.6 & \textit{8.50} & \textit{59.2} & \textit{80.1} \\
    & ProxylessNas & 3.44 & 84.6 & 100.0 & 3.28 & 95.6 & 98.9 & 5.05 & 65.6 & 96.9 & \textit{3.92} & \textit{81.9} & \textit{98.6} \\
    & ResNets & 4.41 & 72.3 & 98.1 & 3.16 & 88.8 & 99.9 & 7.42 & 37.9 & 84.2 & \textit{5.00} & \textit{66.3} & \textit{94.1} \\
    & ShuffleNetv2s & 5.01 & 61.6 & 98.3 & - & - & - & 6.37 & 45.6 & 91.3 & \textit{5.69} & \textit{53.6} & \textit{94.8} \\
    & SqueezeNets & 3.59 & 84.5 & 99.9 & 3.85 & 81.9 & 97.9 & 7.08 & 66.1 & 88.5 & \textit{4.84} & \textit{77.5} & \textit{95.4} \\
    & VGGs & 4.84 & 66.1 & 98.2 & 2.97 & 91.8 & 99.8 & 22.25 & 27.1 & 50.6 & \textit{10.02} & \textit{61.7} & \textit{82.9} \\
    & \textit{Mean (13 models)} & \textit{4.19}& \textit{74.4} & \textit{99.0} & \textit{3.39} & \textit{86.3} & \textit{99.1} & \textit{8.39} & \textit{50.6} & \textit{83.4} & \underline{\textit{5.38}} & \underline{\textit{69.6}} & \underline{\textit{93.7}} \\
    \specialrule{1.0pt}{0pt}{0pt}
    
    % \multirow{13}{*}{\textit{\makecell{SeerNet \\ (Our)}}}
    \multirow{13}{*}{\textit{SeerNet (Our)}}
    & AlexNets & 3.91$\dagger$ & 83.3 & 97.9$\dagger$ & 3.34 & 89.6 & 97.7 & 3.48 & 84.6 & 99.5 & \textit{3.58} & \textit{85.9} & \textit{98.4} \\
    & DenseNets & 2.33 & 95.8 & 100.0 & 1.15 & 100.0 & 100.0 & 1.82 & 99.2 & 100.0 & \textit{1.77} & \textit{98.4} & \textit{100.0} \\
    & GoogleNets & 2.04 & 99.0 & 100.0 & 1.06 & 100.0 & 100.0 & 1.36 & 100.0 & 100.0 & \textit{1.49} & \textit{99.7} & \textit{100.0} \\
    & MnasNets & 2.49 & 97.9 & 100.0 & 1.70 & 100.0 & 100.0 & 3.23 & 88.8 & 99.0 & \textit{2.47} & \textit{95.6} & \textit{99.7} \\
    & MobileNetv1s & 2.53 & 96.4 & 100.0 & 1.47 & 100.0 & 100.0 & 3.71 & 84.9 & 98.2 & \textit{2.57} & \textit{93.8} & \textit{99.4} \\
    & MobileNetv2s & 3.18 & 87.8 & 100.0 & 2.66 & 93.2 & 100.0 & 4.13 & 78.4 & 98.7 & \textit{3.32} & \textit{86.5} & \textit{99.6} \\
    & MobileNetv3s & 2.80 & 93.0 & 100.0 & 2.23 & 96.9 & 100.0 & 2.06 & 98.2 & 100.0 & \textit{2.36} & \textit{96.0} & \textit{100.0} \\
    & NasBench201s & 2.77 & 94.8 & 100.0 & 2.32 & 97.7 & 100.0 & 3.74 & 82.3 & 98.2 & \textit{2.94} & \textit{91.6} & \textit{99.4} \\
    & ProxylessNas & 2.40 & 97.4 & 100.0 & 2.07 & 97.4 & 100.0 & 1.93 & 97.9 & 100.0 & \textit{2.13} & \textit{97.6} & \textit{100.0} \\
    & ResNets & 4.51$\dagger$ & 76.4 & 96.3$\dagger$ & 2.72 & 90.9 & 99.4$\dagger$ & 3.08 & 88.9 & 99.4 & \textit{3.44} & \textit{85.4} & \textit{98.4} \\
    & ShuffleNetv2s & 2.69 & 95.6 & 100.0 & - & - & - & 2.35 & 96.9 & 100.0 & \textit{2.52} & \textit{96.2} & \textit{100.0} \\
    & SqueezeNets & 3.33 & 87.0 & 100.0 & 2.36 & 96.9 & 100.0 & 4.44 & 75.8 & 96.6 & \textit{3.38} & \textit{86.6} & \textit{98.9} \\
    & VGGs & 6.03$\dagger$ & 61.9 & 93.2$\dagger$ & 2.44 & 94.9 & 99.4$\dagger$ & 13.52 & 38.4 & 62.2 & \textit{7.33} & \textit{65.1} & \textit{84.9} \\
    & \textit{Mean (13 models)} & \textit{3.15} & \textit{89.7} & \textit{99.0} & \textit{2.13} & \textit{96.5} & \textit{99.7} & \textit{3.76} & \textit{85.7} & \textit{96.3} & \underline{\textit{3.02}} & \underline{\textit{90.6}} & \underline{\textit{98.4}} \\
    \specialrule{1.5pt}{0pt}{0pt}
    
  \end{tabular}}
\caption{Comparison with nn-Meter. "$\dagger$" indicates SeerNet underperforms than nn-Meter, and the italicized and underlined entries represent the prediction results of execution time across 13 model types on 3 devices.}
\label{tab:comparison nnm}
\end{table*}
\subsection{Baseline Comparison (RQ2)}\label{sec:seernet compare}
\subsubsection{Baseline.} 
We compare SeerNet with the following methods:
% MLP-Node
(i) MLP-Node uses an MLP to predict model performance, concatenating features from all nodes in the graph and handling variable node numbers by padding or truncating them to a fixed size.
% PMGNS
(ii) PMGNS \cite{DIPPM} (described in Section \ref{sec:related work}) utilizes a single prediction head for predicting one performance metric.
% Eagel
(iii) Eagle \cite{Brp-nas} (described in Section \ref{sec:related work}) is implemented for cell-based models, but for non-cell-based models, it uses features from PMGNS (Eagle-p) and our features from SeerPerf (Eagle-s).
% nn-Meter
(iv) nn-Meter \cite{Nn-meter} (described in Section \ref{sec:related work}) is a kernel-based predictor.

\subsubsection{Performance Comparison.}
\emph{Baseline Comparison on our Dataset.}
Table \ref{tab:comparison our} shows that SeerNet has the smallest parameter size (1.02M) and the highest accuracy, with a mean MAPE of 5.14\%.
Compared to SeerNet,
MLP-Node contains more than four times the parameters and achieves a MAPE over three times higher, while PMGNS contains more than three times the parameters and has a MAPE nearly twice as high.
Eagle-p performs poorly, with a MAPE of 74.43\%, due to the inability of PMGNS features to accurately represent the models in our dataset.
Eagle-s, which uses our proposed features, performs better but still lags behind SeerNet.
Both PMGNS and Eagle-s outperform MLP-Node, achieving higher accuracy with fewer parameters, highlighting the ability of GNNs to capture execution dependencies.
Eagle-s also performs better than Eagle-p, demonstrating the effectiveness of our proposed features. SeerNet outperforms the other models, offering the best representation and prediction.

\emph{Baseline Comparison on the Dataset of nn-Meter.} % Comparison with nn-Meter
Table \ref{tab:comparison nnm} shows SeerNet outperforms nn-Meter with half the RMSPE, 21\% higher at 5\%Acc, and 5\% higher at 10\%Acc.
For models like VGG, ResNet, and AlexNet on the CPU, SeerNet is slightly less accurate than nn-Meter, likely due to the simple structures of these models and their predictable execution patterns on the CPU.
On VPU, nn-Meter achieves 50.6\% at 5\%Acc, while SeerNet reaches 85.7\%, 35\% higher. 
This is because the execution of VPU is more complex, and nn-Meter fails to design effective detection functions. 

\begin{table}[!htb]

\belowrulesep=0pt
\aboverulesep=0pt
\renewcommand{\arraystretch}{1}
    \resizebox{1.0\columnwidth}{!}{\begin{tabular}{lcccccccc}
    
    \specialrule{1.5pt}{0pt}{0pt}
    % summury title
    \multirow{3}{*}{Method} & \multicolumn{7}{c}{Accuracy (MAPE[\%]$\downarrow$)} & \multirow{3}{*}{\makecell{Params$\downarrow$ \\ {[}M{]}}} \\
    \cmidrule(lr){2-8}
    
    & \multicolumn{3}{c}{Training} & \multicolumn{3}{c}{Inference} & \multirow{2}{*}{\makecell{\textit{Mean} \\ (6 metrics)}}  \\
    % half mid line
    \cmidrule(lr){2-4} \cmidrule(lr){5-7}

    % title
    & {Util} & {Mem} & {Time} & {Util} & {Mem} & {Time} & \multicolumn{1}{c}{} \\
    \specialrule{1.0pt}{0pt}{0pt}

    PMGNS-Multi & 38.7 & 10.6 & 21.7 & 83.4 & 48.7 & 99.8 & \textit{50.5} & 3.45 \\ % 3,449,347
    SeerNet (×3) & 4.94 & 2.47 & 6.71 & 4.37 & 3.46 & 8.91 & \textit{5.14} & 3.05 \\ % 3,047,493
    SeerNet-Multi (w/o PCGrad) & 17.60 & 3.00 & 22.10 & 18.90 & 3.70 & 29.90 & \textit{15.85} & 1.15 \\ % 1,148,953
    \textbf{SeerNet-Multi (w/ PCGrad)} & \textbf{6.90} & \textbf{3.30} & \textbf{9.10} & \textbf{8.70} & \textbf{3.60} & \textbf{15.20} & \textbf{\textit{7.75}} & \textbf{1.15} \\ % 1,148,953
    \specialrule{1.5pt}{0pt}{0pt}
    
    \end{tabular}}
\caption{Evaluation result of SeerNet-Multi.}
\label{tab:seernet multi}
\end{table}

\begin{table}[!htb]

\belowrulesep=0pt
\aboverulesep=0pt
\newcolumntype{C}{>{\centering\arraybackslash}p{30pt}}
\renewcommand{\arraystretch}{1}
    \resizebox{1.0\columnwidth}{!}{\begin{tabular}{rccccccc}
    
    \specialrule{1.5pt}{0pt}{0pt}
    % summury title
    \multirow{3}{*}{\makecell{Dataset \\ scale}} & \multicolumn{7}{c}{Accuracy (MAPE[\%]$\downarrow$)} \\
    \cmidrule(lr){2-8}
    
    & \multicolumn{3}{c}{Training} & \multicolumn{3}{c}{Inference} & \multirow{2}{*}{\makecell{Mean \\ (6 metrics)}} \\
    \cmidrule(lr){2-4} \cmidrule(lr){5-7}
    
    % title
    & {Util} & {Mem} & {Time} & {Util} & {Mem} & {Time} \\
    \specialrule{1.0pt}{0pt}{0pt}

    \textbf{overall} & \textbf{4.94} & \textbf{2.47} & \textbf{6.71} & \textbf{4.37} & \textbf{3.46} & \textbf{8.91} & \textbf{\textit{5.14}} \\ % 128
    20000 & 5.49 & 2.49 & 8.58 & 5.04 & 3.83 & 13.65 & \textit{6.51} \\ % 128
    % 15000 & 5.76 & 3.23 & 9.25 & 5.40 & 3.93 & 13.32 & \textit{6.82} \\ % 128
    10000 & 6.01 & 2.91 & 10.05 & 5.31 & 5.37 & 14.49 & \textit{7.36} \\ % 128
    5000 & 7.61 & 4.24 & 10.78 & 6.32 & 4.79 & 17.89 & \textit{8.61} \\ % 64
    2000 & 10.24 & 6.39 & 12.33 & 8.26 & 7.49 & 19.25 & \textit{10.66} \\ % 32
    1000 & 18.70 & 5.53 & 15.68 & 14.86 & 6.94 & 28.19 & \textit{14.98} \\ % 8
    
  \specialrule{1.5pt}{0pt}{0pt}
  
  \end{tabular}}
\caption{Data dependency of SeerNet.}
\label{tab:data dependency}
\end{table}


\subsection{Effectiveness of SeerNet-Multi (RQ3)}\label{sec:eva seernet multi}
Table \ref{tab:comparison nnm} shows that PMGNS-Multi (PMGNS with multiple prediction heads) performed poorly with a MAPE of 50.5\%, while SeerNet-Multi (without PCGrad) had a MAPE of 15.85\%, indicating that predicting multiple metrics simultaneously reduces accuracy.
%
However, with PCGrad, SeerNet-Multi halved its MAPE from 15.85\% to 7.75\% without increasing parameter overhead. This demonstrates that PCGrad effectively mitigates conflicting gradient directions across tasks, enabling SeerNet-Multi to predict multiple metrics efficiently with minimal accuracy loss.
%
Furthermore, SeerNet-Multi has about one-third the parameters of SeerNet, with only a 2.61\% increase in MAPE, balancing parameter efficiency and prediction accuracy. This makes SeerNet-Multi ideal for scenarios requiring rapid predictions with limited resources and lower accuracy demands.

\subsection{Further Discussion (RQ4)}\label{sec:extension experiments}
\subsubsection{Application Scope.}
\emph{Multi-Model, Multi-Metric Support.}
SeerPerf provides accurate predictions for execution time, memory usage, and SM utilization during both training and inference across various architectures, including GoogLeNet, VGG, ResNe(X)t, MobileNet, and DenseNet.

\emph{Multi-Device Support.}
PerfSeer provides accurate predictions across various devices, including mobile CPUs, mobile GPUs, desktop GPUs, and Intel VPUs, as shown in Table \ref{tab:comparison nnm}. In contrast, nn-Meter exhibits poor prediction accuracy on Intel VPUs.

\emph{Multi-Platform Support.}
The representation of PerfSeer is based on ONNX, so SeerPerf supports any DL framework convertible to ONNX.
\emph{Overall}, our performance predictor, PerfSeer, has a wide application scope, making it suitable for most common applications.

\subsubsection{Overhead.}
\emph{Data dependency and deployment overhead.}
To evaluate the data dependency of SeerNet, we analyze the relationship between dataset scale and prediction accuracy, keeping the test set size fixed. 
Results (Table \ref{tab:data dependency}) show that accuracy decreases as the dataset scale shrinks. Nevertheless, SeerNet achieves a mean MAPE of 14.98 with only 1,000 samples, demonstrating low data dependency.
The deployment overhead includes 16.67 GPU hours for data collection and 0.05 GPU hours for training per 1,000 samples, resulting in low deployment overhead.

\emph{Usuage overhead.}
We evaluated the overhead of SeerPerf on an Intel i7-11700 CPU, which includes representation and prediction. 
The average representation latency is 248 ms, with prediction latencies of 2.0 ms for SeerNet and 2.1 ms for SeerNet-Multi. The total overhead of approximately 250 ms is acceptable for most applications. \emph{Overall}, SeerPerf demonstrates low overhead in both deployment and usage.

% \emph{The experiments conducted in this section conclude that PerfSeer is an efficient and accurate performance predictor, characterized by its broad application scope, low construction and usage overhead, and high prediction accuracy.}
