% ICCV 2025 Paper Template; see https://github.com/cvpr-org/author-kit

\documentclass[10pt,twocolumn,letterpaper]{article}

%%%%%%%%% PAPER TYPE  - PLEASE UPDATE FOR FINAL VERSION
% \usepackage{iccv}              % To produce the CAMERA-READY version
% \usepackage[review]{iccv}      % To produce the REVIEW version
\usepackage[pagenumbers]{iccv} % To force page numbers, e.g. for an arXiv version

% Import additional packages in the preamble file, before hyperref
\newcommand{\CG}{\mathcal{G}\xspace}
\newcommand{\CV}{\mathcal{V}\xspace}
\newcommand{\CE}{\mathcal{E}\xspace}
\newcommand{\CA}{\mathcal{A}\xspace}
\newcommand{\CF}{\mathcal{F}\xspace}
\newcommand{\CR}{\mathcal{R}\xspace}
\newcommand{\CB}{\mathcal{B}\xspace}
\newcommand{\CX}{\mathcal{X}\xspace}
\newcommand{\CK}{\mathcal{K}\xspace}
\newcommand{\CM}{\mathcal{M}\xspace}
\newcommand{\CC}{\mathcal{C}\xspace}
\newcommand{\CL}{\mathcal{L}\xspace}
\newcommand{\CI}{\mathcal{I}\xspace}
\newcommand{\CQ}{\mathcal{Q}\xspace}
\newcommand{\CO}{\mathcal{O}\xspace}
\newcommand{\CP}{\mathcal{P}\xspace}
\newcommand{\CS}{\mathcal{S}\xspace}
\newcommand{\CT}{\mathcal{T}\xspace}
\newcommand{\CJ}{\mathcal{J}\xspace}
\usepackage[para]{footmisc}
\usepackage{subfig}
% \usepackage{subcaption}
% \usepackage{array}
% \usepackage{colortbl}



% It is strongly recommended to use hyperref, especially for the review version.
% hyperref with option pagebackref eases the reviewers' job.
% Please disable hyperref *only* if you encounter grave issues, 
% e.g. with the file validation for the camera-ready version.
%
% If you comment hyperref and then uncomment it, you should delete *.aux before re-running LaTeX.
% (Or just hit 'q' on the first LaTeX run, let it finish, and you should be clear).
\definecolor{iccvblue}{rgb}{0.21,0.49,0.74}
\usepackage[pagebackref,breaklinks,colorlinks,allcolors=iccvblue]{hyperref}

%%%%%%%%% PAPER ID  - PLEASE UPDATE
\def\paperID{1059} % *** Enter the Paper ID here
\def\confName{ICCV}
\def\confYear{2025}

\usepackage{algorithm}
\usepackage{algpseudocode}
\usepackage{multirow}
% \usepackage{subfig}
\usepackage{booktabs}
\usepackage{amsmath}
\usepackage{amssymb}
\usepackage{mathtools}
\usepackage{amsthm}

% Algorithms packages
% \usepackage{algorithm}
% \usepackage{algorithmicx} % This is for better control over algorithms
\usepackage{algpseudocode}

% Graphics and Figures packages
\usepackage{graphicx}
% \usepackage{subfig}
\usepackage{booktabs}  % for professional tables
\usepackage{arydshln}  % for dashed lines in tables

% Hyperlinks and referencing packages
\usepackage[capitalize,noabbrev]{cleveref} % Cleveref must come after hyperref

% Miscellaneous useful packages
\usepackage{microtype}  % for better typography
\usepackage{bbm}        % for blackboard bold math symbols
\usepackage{colortbl}   % for coloring rows in tables
\usepackage{pifont}     % for special symbols
\definecolor{color3}{gray}{0.95}
\definecolor{color4}{HTML}{00b050}

\providecommand{\yulun}[1]{\textcolor{red}{[{\bf #1}]}}

\renewcommand{\ttdefault}{lmtt}
\newcommand{\ours}{\texttt{AdaSVD}}

%%%%%%%%% TITLE - PLEASE UPDATE
\title{$\ours$: Adaptive Singular Value Decomposition for Large Language Models}

%%%%%%%%% AUTHORS - PLEASE UPDATE
\author{
	Zhiteng Li$^{1}$\thanks{Equal contribution}~,\enspace
	Mingyuan Xia$^{1}$\footnotemark[1]~,\enspace
	Jingyuan Zhang$^{1}$,\enspace \\
	Zheng Hui$^{2}$,\enspace
	Linghe Kong$^{1}$\footnotemark[2]~,\enspace
	Yulun Zhang$^{1}$\thanks{Corresponding authors: Linghe Kong,  linghe.kong@sjtu.edu.cn, Yulun Zhang, yulun100@gmail.com}~, \enspace
	Xiaokang Yang$^{1}$\\
	\textsuperscript{1}Shanghai Jiao Tong University,\enspace
	\textsuperscript{2}MGTV, Shanhai Academy\\
	\vspace{-8mm}
}

\begin{document}
	\maketitle
	
	%% Don't use 'sec' 
	% \begin{abstract}

% Recent works to jointly reconstruct 3D human and object from a single RGB image, are mostly model-based, that fail to capture the fine details of the clothed human body and object surface. In this paper, we introduce ReCHOR, a novel, model-free, first-method to produce realistic clothed human-object reconstructions from a monocular view. This is extremely challenging due to human-object occlusions, diverse interactions and depth ambiguity, as it needs to infer both 3D spatial awareness and high resolution details. Our core idea is based on estimating neural implicit representations for human and object respectively by an attention-based neural implicit model that attends to pixel-aligned features from both the global human-object image for spatial awareness and  the local separate view of human and object images for high quality details. Additionally, the network is conditioned on semantic features from an initial estimated human-object pose prior and a generative diffusion model that inpaints occluded regions, thus enabling the retrieval of details from them.
% We also propose a synthetic dataset with rendered scenes of diverse, inter-occluded 3D human and object scans, to train our network. We evaluate our method on the synthetic and real world BEHAVE dataset. Our experiments show that our method outperforms the SOTA in achieving realistic clothed human-object reconstructions.
Recent approaches to jointly reconstruct 3D humans and objects from a single RGB image represent 3D shapes with template-based or coarse models, which fail to capture details of loose clothing on human bodies. In this paper, we introduce a novel implicit approach for jointly reconstructing realistic 3D clothed humans and objects from a monocular view. For the first time, we model both the human and the object with an implicit representation, allowing to capture more realistic details such as clothing. This task is extremely challenging due to human-object occlusions and the lack of 3D information in 2D images, often leading to poor detail reconstruction and depth ambiguity. To address these problems, we propose a novel attention-based neural implicit model that leverages image pixel alignment from both the input human-object image for a global understanding of the human-object scene and from local separate views of the human and object images to improve realism with, for example, clothing details. Additionally, the network is conditioned on semantic features derived from an estimated human-object pose prior, which provides 3D spatial information about the shared space of humans and objects. To handle human occlusion caused by objects, we use a generative diffusion model that inpaints the occluded regions, recovering otherwise lost details. For training and evaluation, we introduce a synthetic dataset featuring rendered scenes of inter-occluded 3D human scans and diverse objects. Extensive evaluation on both synthetic and real-world datasets demonstrates the superior quality of the proposed human-object reconstructions over competitive methods.
\end{abstract}    
	% \section{Introduction}
\label{sec:intro}
% Image editing methods in diffusion models depend on user-defined control directions - users can unlock their creativity using these methods by specifying the desired manipulation through prompts~\cite{gandikota2023concept}, reference images~\cite{ruiz2022dreambooth, kumari2022customdiffusion, gal2022image, chen2024trainingfreeregionalpromptingdiffusion}, or attribute vectors~\cite{parmar2023zero,hertz2022prompt}. In this work, we ask a fundamentally different question: \emph{Can we automatically discover the underlying visual structure of a concept within diffusion model's knowledge?} %Rather than requiring user-specified controls, we aim to decompose the model's internal knowledge into meaningful directions.

% This question touches on a fundamental limitation in how we interact with diffusion models. Current control methods ~\cite{zhang2023addingconditionalcontroltexttoimage, gandikota2023concept, ye2023ipadaptertextcompatibleimage,ye2023ipadaptertextcompatibleimage, hertz2024stylealignedimagegeneration, li2023photomaker, shi2024instantbooth, chen2024trainingfreeregionalpromptingdiffusion} require users to specify their desired manipulations in advance, limiting interactive creativity. This contrasts with natural human artistic workflows, where creators dynamically explore creative ideas while jointly refining them toward meaningful artistic outcomes~\cite{hoffmann2016modeling}. This synergy between specification and exploration is not new to generative models. Early GAN architectures naturally developed disentangled latent spaces that enabled continuous\cite{harkonen2020ganspace,radford2015unsupervised, wu2021stylespace, shen2020interfacegan}, compositional control over generated images. Users could explore these spaces to discover interesting variations that would be difficult to describe in words~\cite{wu2021stylespace}, then combine them to achieve their creative goals~\cite{grabe2022towards}. 


% While diffusion models have largely superseded GANs in conditional image synthesis~\cite{dhariwal2021diffusion},  their underlying structure remains less understood. Diffusion models achieve remarkable diversity through high-dimensional latents, unlike GANs' compact latent spaces.  With a single prompt, diffusion models can generate radically different variations through different random initializations of input noise. We ask - Is it possible to discover interpretable structure within this vast space of variations?

Text-to-image diffusion models are capable of generating remarkable visual variations from a single prompt through different random initializations. However, this vast creative potential remains largely opaque to users---while we can generate diverse images, we lack understanding of the underlying structure of these variations. This presents a fundamental challenge: how can we discover and expose the latent visual capabilities encoded within these models?

\let\thefootnote\relax \footnote{$^{*}$Correspondence to \texttt{gandikota.ro@northeastern.edu}}

The challenge touches on a key limitation in how we interact with diffusion models today. Current control methods require users to explicitly specify their desired edits in advance through prompts~\cite{gandikota2023concept}, reference images~\cite{zhang2023addingconditionalcontroltexttoimage, chen2024trainingfreeregionalpromptingdiffusion, ruiz2022dreambooth,kumari2022customdiffusion, Ryu_lora, hu2021lora}, or attribute vectors~\cite{ye2023ipadaptertextcompatibleimage, hertz2024stylealignedimagegeneration, li2023photomaker, shi2024instantbooth,parmar2023zero,hertz2022prompt}. That contrasts sharply with natural human creative workflows, where artists dynamically explore creative ideas and jointly refine them toward meaningful artistic outcomes~\cite{hoffmann2016modeling}. The need for pre-specified controls creates a barrier between users and the full creative potential of these models.

Interestingly, earlier generative models like GANs~\cite{gans,karras2019style,brock2018large} naturally developed more interpretable internal structures. Their compact latent spaces often exhibited emergent disentanglement~\cite{harkonen2020ganspace,radford2015unsupervised, wu2021stylespace, shen2020interfacegan}, enabling continuous and compositional control over generated images. Users could explore these spaces to discover interesting variations that would be difficult to describe in words~\cite{wu2021stylespace}, then combine them to achieve their creative goals~\cite{grabe2022towards}.

Diffusion models have largely superseded GANs in conditional image synthesis~\cite{dhariwal2021diffusion}, achieving greater diversity through much higher-dimensional latents. And yet an understanding of the underlying structure of these larger latent spaces has remained elusive. In this work, we ask a fundamental question: \emph{Can we automatically discover the visual structure within a diffusion model's knowledge of a concept?} Rather than requiring user-specified controls, we aim to decompose the model's internal representations into expressive directions that users can explore and combine.

To address these needs, we present \textbf{SliderSpace}, a framework that brings systematic explorability to diffusion models. Given just a text prompt, SliderSpace discovers a canonical set of meaningful, diverse, and controllable directions within the model's knowledge of that concept. Each direction is implemented as a low-rank adapter~\cite{hu2021lora} that can be scaled and composed with others, allowing users to explore and smoothly combine different aspects of variation, as shown in Figure~\ref{fig:intro}.

We ground SliderSpace discovery in three key requirements for meaningful decomposition of a diffusion model's visual manifold: 
\begin{enumerate}
    \item \textbf{Unsupervised Discovery:} The decomposition process should emerge from the intrinsic structure of the model's learned representation, rather than being guided by predefined attributes. This ensures we capture the true topology of the model's knowledge space rather than projecting our assumptions onto it.
    
    \item \textbf{Semantic Orthogonality:} Each discovered control must represent a distinct semantic direction. This is enforced in a semantic feature space, like CLIP, where every slider has an orthogonal effect in embeddings. This prevents discovering multiple controls that create similar semantic effects, making the system more efficient and easier.
    
    \item \textbf{Distribution Consistency:} Directions must induce consistent transformations across both random seeds and prompt variations. 
\end{enumerate}

These requirements naturally lead to our proposed framework, which we formalize in Section~\ref{sec:method}. As we show in our experiments, SliderSpace is architecture-agnostic, working with both conventional U-Net based models like Stable Diffusion~\cite{rombach2022high, rombach2022sd20, podell2023sdxl, turbo, dmd} and recent transformer-based architectures like Flux~\cite{flux}.

We demonstrate the expressiveness of SliderSpace through three applications: First, we show how SliderSpace can decompose high-level concepts into diverse and expressive components, revealing the natural axes of variation in the model's understanding. Second, we explore artistic style variation, where SliderSpace discovers directions that match or exceed the diversity of manually curated artist lists while being judged more useful by human evaluators. Finally, we show how SliderSpace can help reverse the mode collapse commonly observed in distilled diffusion models, restoring diversity while maintaining generation speed.

Beyond providing practical creative control, SliderSpace opens new avenues for understanding and utilizing the latent capabilities of diffusion models. By mapping these models' visual potential into intuitive, composable directions, we take a step toward making their creative possibilities more accessible and interpretable to users.

% Image editing methods in diffusion models unlock the creativity of users. In this work we ask an alternate question: \emph{Can we organize and expose what of the diffusion model is already capable of?}.
% Existing methods for controlling image generation typically require users to manually specify edit directions for desired changes. This process is time-consuming, requires technical expertise, and limits the spontaneity of the creative process. For instance, if a user wants to adjust the smile of a generated person, they must explicitly request this edit, often through imprecise prompt engineering or model fine-tuning. This approach of predefined controls or manual specifications restricts users from fully exploring the latent capabilities of the model. There may be interesting stylistic variations or attributes that the model can generate, but users have no easy way to discover or utilize these.

% Natural visual disentanglement was an emergent property in the latent space of Generative Adversarial Models (GANs) \cite{harkonen2020ganspace,radford2015unsupervised, wu2021stylespace, shen2020interfacegan}. In particular, it has been observed that StyleGAN~\cite{karras2019style} stylespace neurons offer detailed control over many meaningful aspects of images that would be difficult to describe in words~\cite{wu2021stylespace}. However, diffusion models do not share such a compact latent space~\cite{park2023unsupervised}; and efforts to uncover such a space in the semantic embeddings of the text conditioning have met with limited success \nik{Nick - is there a specific citation you were thinking about?}.

% In this work we introduce \textbf{SliderSpace}, which takes a step towards uncovering an analogous low dimensional representation of diffusion models' visual breadth; in essence treating the diffusion model as many generators sharing parameters, where a particular generator is defined by a specific prompt. For a given prompt we sample many random seeds (and optionally prompt expansions using an LLM), generate the corresponding images, and apply an off the shelf feature extractor (in this work CLIP, but our method can be applied to any differentiable feature extractor). We use PCA to analyze these features, and for each of the leading $k$ principal components we train a LoRA \cite{} which causes the diffusion model to produces images which increase the feature magnitude along that component when passed back through the same feature extractor. This leads to a 'Slider' for each principal component, because each LoRA can be scaled and applied to the original diffusion model, continuously varying those visual features in the generated results (as measured, in our case, by CLIP).

% There are many other works that enhance the controllability of diffusion models. One common approach is enabling users to add spatial constraints to a generation either manually, or via a reference image \cite{zhang2023addingconditionalcontroltexttoimage, chen2024trainingfreeregionalpromptingdiffusion}, a second is leveraging more abstract embeddings (e.g. identity, style) extracted from a reference image \cite{ye2023ipadaptertextcompatibleimage, hertz2024stylealignedimagegeneration, li2023photomaker, shi2024instantbooth}, a third is finetuning a foundation model to better generate a concept important to the user \cite{ruiz2022dreambooth, kumari2022customdiffusion, Ryu_lora, hu2021lora}, and a fourth (most relevant to this work) is finding low-rank adaptors of the model based on a prompt or small training set which can be scaled to provide continous control over one aspect of generated image (e.g. night vs day, basic vs luxury, etc.) \cite{gandikota2023concept}. SliderSpace is complementary to all of these methods and offers something distinct. All of the other methods we are aware require the user (and / or model designer) to know in advance what type of control they want. In contrast SliderSpace assists users in discovering and controlling hidden capabilities present in the diffusion model's distribution of possible generations.

%We propose that truly intuitive creative control in a text-to-image model should meet three key criteria: \emph{discoverability}, \emph{intuitiveness}, and \emph{specificity}. The model should reveal controllable attributes that may not be immediately obvious, offer controls that are easy to understand and manipulate, and ensure each control affects a distinct attribute of the generated image.

% We demonstrate the utility and power of SliderSpace using three applications built on top of SDXL-DMD \cite{dmd}, because its fast generation speed lends itself well to the continuous control offered by SliderSpace.

% First, we study concept decomposition (Section \ref{sec:concept_exp}), where we learn sliders for a specific concept (e.g. 'monster', 'waterfall', 'car'). Through quantitative metrics of diversity and text alignment we demonstrate that the learned sliders dramatically boost the diversity of generations when randomly applied without harming text alignment; we also ask humans to qualitatively judge these results in a user study where they find the SliderSpace results to be more 'Diverse', 'Useful', and 'Creative' than our baselines.

% Second, we attempt to compare the automatic discoveries of SliderSpace to a large scale manual study of artistic styles (Section \ref{sec:art_exp}), open-sourced by ParrotZone \cite{parrotzone}. In this study SDXL was prompted with over 4300 artist names,  and based on visual inspection the cases of successful stylistic mimicry recorded. Quantitatively SliderSpace more closely matches the distribution of artistic variation discovered by ParrotZone than other baselines, and in our user studies was judged to be significantly more 'Diverse' and 'Useful' than the baselines. To our surprise humans even judged SliderSpace results to be slightly more 'Diverse' than the results generated by the manually discovered artist names of \cite{parrotzone}.

% Third, we attempt to use SliderSpace to reverse the mode collapse commonly observed in distilled few-step diffusion models relative to the original teacher model (Section \ref{sec:diverse_exp}). We quantitatively demonstrate that applying SliderSpace to SDXL-DMD leads to more closely matching the distribution of images by the original teacher, SDXL.

%Through extensive experiments on various state-of-the-art text-to-image models, we demonstrate that SliderSpace significantly enhances user control and creative expression in AI-assisted image generation tasks. Our method enables a range of applications, including concept decomposition and control, diversity improvement in generated images, customization dissection and edits, and the exploration of artistic styles inherent in the model.

% SliderSpace goes beyond providing a practical tool for enhanced creative control. By mapping the visual potential of diffusion models it can open new avenues for generative creativity and deepens our understanding of each model's hidden potential.
	% \section{Related work}
\label{sec:formatting}

\subsection{Text-to-Video Generation}

T2V generation has made notable progress, evolving from early GAN-based models \cite{saito2017temporal,tulyakov2018mocogan,fu2023tell,li2018video,wu2022nuwa,yu2022generating} to newer transformer \cite{yan2021videogpt,arnab2021vivit,esser2021taming,ramesh2021zero,yu2022scaling} and diffusion models \cite{kirkpatrick2017overcoming,sohl2015deep,song2020denoising,zhang2022gddim}. Early efforts like MoCoGAN~\cite{tulyakov2018mocogan} focused on short video clips but faced issues with stability and coherence. The introduction of transformers improved sequential data handling, enhancing video generation, while diffusion models further improved video quality by progressively denoising the input. 
Despite these advances, T2V models still struggle to reflect human preferences, with the generated videos generally lacking aesthetic quality. Additionally, the scarcity of paired video preference data hinders effective model training and may lead to insufficient flexibility and poor quality in the generated videos.


\subsection{RLHF}

\iffalse
Aligning LLMs \cite{dai1901transformer,radford2019language,zhang2023opt} typically involves two steps: supervised fine-tuning followed by Reinforcement Learning with Human Feedback (RLHF) \cite{gao2023scaling,stiennon2020learning,rafailov2024direct}. Although effective, RLHF is computationally expensive and can lead to issues like reward hacking. Methods like DPO have streamlined alignment by leveraging feedback data directly, improving efficiency.

In contrast, diffusion model alignment is still evolving, focusing mainly on enhancing visual quality through curated datasets. Techniques like DOODL \cite{wallace2023end} and AlignProp \cite{prabhudesai2023aligning} target aesthetic improvements but face challenges with complex tasks such as text-image alignment. Reinforcement learning methods like DPOK \cite{fan2024reinforcement} and DDPO \cite{black2023training} improve reward optimization but struggle with scalability. DPO-SDXL integrates DPO into T2I generation, boosting both alignment and aesthetics.

However, aligning video generation remains a largely unaddressed challenge, especially when dealing with motion consistency and semantic coherence across frames.
\fi

RLHF \cite{gao2023scaling,stiennon2020learning,rafailov2024direct} is a method that utilizes human feedback to guide machine learning models. Early RLHF algorithms, such as DDPG~\cite{lillicrap2015continuous} and PPO~\cite{schulman2017proximal}, typically relied on complex reward models to quantify human feedback. These reward models require a large amount of annotated data and face challenges during tuning. As research has progressed, more efficient preference learning methods have emerged, among which DPO has become a new framework. DPO does not depend on a separate reward model; instead, it obtains human preferences through pairwise comparisons and directly optimizes these preferences. This shift not only simplifies the application of RLHF but also enhances the alignment of models with human values. Furthermore, DPO has been successfully introduced into T2I tasks~\cite{wallace2024diffusion,yang2024using}, providing new insights for generative models in addressing the alignment of human preferences and showcasing DPO's potential in the field of AIGC~\cite{shi2024instantbooth,
qing2024hierarchical,menapace2024snap,koley2024s}. However, there remains a gap in current research regarding the application of DPO strategies to T2V tasks. Effectively integrating DPO into T2V tasks presents a challenging endeavor.


	% \section{Preliminary}
\label{sec:preliminary}
In this section, we first introduce the mathematical formulation of flow-based text-to-image generative models~\cite{Xingchao_2022,Lipman_2022}, which forms the foundation of modern T2I systems~\cite{sd3,sdxl,imagen3,imagen}. We then describe classifier-free guidance~\cite{ho2022classifier}, a key technique to control the generation process through text conditioning.

\subsection{Flow-based text-to-image generative models}
In state-of-the-art T2I models~\cite{sd3}, the image generation process is modeled by learning, through a neural network, a flow $\psi$ that generates a probability path $(p_t)_{0\le t\le 1}$ bridging the source distribution $p_0$ and the target distribution $p_1$ ~\cite{Xingchao_2022,Lipman_2022}. This framework encompasses diffusion models~\cite{sohl2015deep,ddpm} as a special case. In particular, a commonly used formulation sets a Gaussian distribution as the source: $p_0 = \mathcal{N}(\mathbf{0}, \mathbf{I})$ and a delta distribution centered on a sample $\mathbf{x}_1$ from the data distribution $q$ as the target: $p_1 = \delta_{\mathbf{x}_1}$.
Then, it incorporates an affine conditional flow $\psi_t(\mathbf{x} | \mathbf{x}_1) = a_t \mathbf{x}_1 + b_t \mathbf{x}$ with the boundary condition $(a_0, b_0) = (0, 1),\ (a_1, b_1) = (1, 0)$ to bridge them. The neural network typically approximates quantities such as velocity fields, $x_0$ prediction or $x_1$ prediction. In this modeling, these quantities can be viewed as affine transformations of the marginal probability path score $\nabla_{\mathbf{x}} \log p_t(\mathbf{x})$.

\subsection{Classifier-free guidance in flow-based models}
Classifier-free guidance~\cite{ho2022classifier} is a method for sampling from a model conditioned by a text input $\mathbf{y}$ by guiding an unconditional image generation model modeled using the score $\nabla_{\mathbf{x}} \log p_t(\mathbf{x})$. This enables the sampling from
\[
q_w(\mathbf{x}, \mathbf{y}) \propto q(\mathbf{x})q(\mathbf{y}|\mathbf{x})^w \propto q(\mathbf{x})^{1-w}q(\mathbf{x}|\mathbf{y})^w
\]
where $w \in \mathbb{R}$ is the guidance scale typically used with $w > 1$. The score satisfies
\[
\nabla_{\mathbf{x}} \log q_w(\mathbf{x}, \mathbf{y}) = (1-w)\nabla_{\mathbf{x}} \log q(\mathbf{x}) + w\nabla_{\mathbf{x}} \log q(\mathbf{x}|\mathbf{y})
\]
so by training the network to learn both the unconditional score $\nabla_{\mathbf{x}} \log q(\mathbf{x})$ and conditional score $\nabla_{\mathbf{x}} \log q(\mathbf{x}|\mathbf{y})$, flexible sampling from the conditional distribution can be achieved through a weighted sum of the network outputs.
	
	\begin{abstract}
		Large language models (LLMs) have achieved remarkable success in natural language processing (NLP) tasks, yet their substantial memory requirements present significant challenges for deployment on resource-constrained devices. Singular Value Decomposition (SVD) has emerged as a promising compression technique for LLMs, offering considerable reductions in memory overhead. However, existing SVD-based methods often struggle to effectively mitigate the errors introduced by SVD truncation, leading to a noticeable performance gap when compared to the original models. Furthermore, applying a uniform compression ratio across all transformer layers fails to account for the varying importance of different layers. To address these challenges, we propose $\ours$, an adaptive SVD-based LLM compression approach. Specifically, $\ours$ introduces \textbf{adaComp}, which adaptively compensates for SVD truncation errors by alternately updating the singular matrices $\mathcal{U}$ and $\mathcal{V}^\top$. Additionally, $\ours$ introduces \textbf{adaCR}, which adaptively assigns layer-specific compression ratios based on the relative importance of each layer. Extensive experiments across multiple LLM/VLM families and evaluation metrics demonstrate that $\ours$ consistently outperforms state-of-the-art (SOTA) SVD-based methods, achieving superior performance with significantly reduced memory requirements. Code and models of $\ours$ will be available at \url{https://github.com/ZHITENGLI/AdaSVD}.% for $\ours$ to facilitate further research.
	\end{abstract}
	
	%% narrow the gap between equations and sentences
	\setlength{\abovedisplayskip}{2pt}
	\setlength{\belowdisplayskip}{2pt}
	
	\vspace{-3mm}
	\section{Introduction}
	\vspace{-1mm}
	
	\begin{figure}[t]
		\centering
		\includegraphics[width=\linewidth]{figs/Figure_0_ours.pdf} % Use a placeholder image (LaTeX comes with an example image)
		\vspace{-7mm}
		\caption{Comparison between vanilla SVD, FWSVD~\cite{hsu2022fwsvd}, ASVD~\cite{yuan2024asvd}, SVD-LLM~\cite{wang2024svdllm}, and our $\ours$ on WikiText2.}
		\vspace{-7mm}
	\end{figure}
	
	\begin{figure*}[t]
		\centering
		\includegraphics[width=1\textwidth]{figs/overview.pdf}
		\vspace{-7mm}
		\caption{Overview of the proposed $\ours$ method: (a) SVD decomposition and truncation for linear layer weights; (b) Stack-of-batch strategy for efficient use of calibration data under limited GPU memory; (c) Adaptive compression ratio assignment (\textbf{adaCR}) based on layer-wise importance; (d) Adaptive compensation (\textbf{adaComp}) through alternating updates of $\mathcal{U}$ and $\mathcal{V}^\top$.}
		\vspace{-5mm}
		\label{fig:overview}
	\end{figure*}
	
	Recently, large language models (LLMs) based on the Transformer architecture~\cite{vaswani2017attention} have shown remarkable potential across a wide range of natural language processing (NLP) tasks. However, their success is largely driven by their massive scale, with models such as the LLaMA family~\cite{touvron2023llama} and the Open Pre-trained Transformer (OPT) series~\cite{zhang2022opt} containing up to 70B and 66B parameters, respectively. The substantial memory requirements of these models present significant challenges for deploying them on mobile devices. Consequently, the widespread adoption of LLMs remains limited by their immense resource demands~\cite{wan2023efficient, wang2024iot, zhou2024survey}.
	
	
	
	Recent research on large language model (LLM) compression has explored various techniques, including weight quantization~\cite{lin2024awq, frantar2022gptq}, network pruning~\cite{sun2023simple, frantar2023sparsegpt}, low-rank factorization~\cite{wang2024svdllm, zhang2023loraprune, yuan2024asvd}, and knowledge distillation~\cite{zhong2024revisiting, gu2023knowledge}. Among these methods, low-rank factorization using Singular Value Decomposition (SVD)~\cite{hsu2022fwsvd,yuan2024asvd,wang2024svdllm} stands out as a powerful approach for reducing both model size and computational cost. SVD achieves this by decomposing large weight matrices into smaller, low-rank components while preserving model performance. Since LLMs are often memory-bound during inference~\cite{dao2022flashattention,dao2023flashattention}, SVD compression can effectively accelerate model inference by reducing the memory requirements, even when applied solely to the weights. This approach does not require specialized hardware or custom operators, unlike weight quantization, making SVD more versatile across different platforms. Additionally, SVD is orthogonal to other compression techniques~\cite{wang2024svdllm}, allowing it to be combined with methods like weight quantization or network pruning for even greater efficiency, enabling more scalable and adaptable solutions for deploying LLMs.
	
	
	Recent advancements in SVD-based LLM compression, including FWSVD~\cite{hsu2022fwsvd}, ASVD~\cite{yuan2024asvd}, and SVD-LLM~\cite{wang2024svdllm}, have significantly improved the low-rank factorization approach, enhancing the overall effectiveness of SVD compression. For example, FWSVD introduces Fisher information to prioritize the importance of parameters, while ASVD accounts for the impact of activation distribution on compression error. SVD-LLM establishes a relationship between singular values and compression loss through the data whitening techniques. While these methods have led to notable improvements in SVD compression, they still face challenges when applied at high compression ratios.
	
	
	To bridge the performance gap between compressed and original models at both low and high compression ratios, we revisit SOTA solutions for LLM compression using SVD decomposition. Our analysis highlights two key observations:
	\textbf{First,} low-rank weight compensation after truncating the smallest singular vectors has been largely overlooked or insufficiently explored in prior methods. When truncating parts of the matrices $\mathcal{U}$ and $\mathcal{V}^\top$, the remaining components should be adjusted accordingly to minimize the SVD compression error.
	\textbf{Second,} previous methods typically apply a uniform compression ratio across all transformer layers, failing to account for their varying relative importance. To address this, an importance-aware approach for assigning appropriate compression ratios is necessary.
	
	
	
	
	
	
	
	
	
	
	
	
	To tackle the challenges outlined above, we propose $\ours$, an adaptive SVD-based LLM compression method. \textbf{First,} $\ours$ proposes \textbf{adaComp}, an adaptive compensation technique designed to adjust the weights of $\mathcal{U}$ and $\mathcal{V}^\top$ after SVD truncation. By alternately updating the matrices $\mathcal{U}$ and $\mathcal{V}^\top$, \textbf{adaComp} effectively reduces compression errors in a stable and efficient manner. To optimize the use of calibration data with limited GPU memory, we also introduce a stack-of-batch technique when applying \textbf{adaComp}.
	\textbf{Second,} $\ours$ proposes \textbf{adaCR}, a method that assigns adaptive compression ratios to different layers based on their importance. With the target compression ratio fixed, this strategy significantly improves performance compared to using a uniform compression ratio across all layers.
	
	
	Our key contributions are summarized as follows:
	% \vspace{-2.5mm}
	\begin{itemize}
		\item We propose \textbf{adaComp}, a novel adaptive compensation method for SVD truncation. By alternately updating $\mathcal{U}$ and $\mathcal{V}^\top$ and employing the stack-of-batch technique, we effectively and stably minimize compression error.
		
		\item We propose \textbf{adaCR}, an adaptive compression ratio method that assigns layer-specific compression ratios according to their relative importance in LLMs. This importance-aware approach outperforms the previously used uniform compression ratio method.
		
		\item Extensive experiments on LLMs/VLMs demonstrate that our method, $\ours$, significantly outperforms the previous SOTA SVD-based LLM compression method, SVD-LLM, effectively narrowing the performance gap between compressed and original models.
		
		
		
	\end{itemize}
	
	% \begin{algorithm*}[t]
		% \caption{Pseudocode of \ours}
		% \begin{algorithmic}[1] % The number [1] tells LaTeX to number each line
			% \State \textbf{Input:} $\mathcal M$: Original LLM
			% \State \textbf{Output:} $\mathcal M'$: Updated Model by \ours
			% \Procedure{\ours}{$\mathcal M$} % Algorithm name and parameters
			%     \State Randomly collect several sentences as the calibration data $\mathcal{C}$ 
			%     \State \textbf{Shuffle} $\mathcal{C}$, randomly sample $m$ buckets and utilize mean value \Comment{Stack-of-batch strategy}
			%     \State $\text{Set}_\mathcal{S} \gets \textproc{Whitening}(\mathcal M, \mathcal C)$
			%     \State $\text{Set}_\mathcal{SVD} \gets \emptyset$ \Comment{Initialize the set of decomposed matrices for the weight to compress} 
			%     \State $\text{Set}_\mathcal{W} \gets \mathcal{M}$ \Comment{Obtain the set of weights in $M$ to compress} 
			%     \State $\text{Set}_\mathcal{CR} \gets$ \textproc{Layerwise Compression Ratio Calculation($\mathcal{M}$)}
			%     \For{$\mathcal{W}$ \textbf{in} $\text{Set}_\mathcal{W}$}
			%             \State $\mathcal{S} \gets \text{Set}_\mathcal{S}(\mathcal{W})$  \Comment{Extract the whitening matrix of current weight $\mathcal{W}$} 
			%             \State $\mathcal{U}, \Sigma, \mathcal{V} \gets \text{SVD}(\mathcal{WS})$, $\Sigma_1 \gets \text{Trunc.}(\Sigma)$, $\text{Set}_{\mathcal{SVD}} \gets (\mathcal{U}, \Sigma_1, \mathcal{V}) \cap \text{Set}_{\mathcal{SVD}}$      
			%             \State \Comment{Apply SVD and truncation. Notice that for each layer, $\mathcal{CR}(W_i) ~\text{equals}~\frac{\#\text{params of }U_k^\sigma + \#\text{params of }{V_k^\sigma}^\top}{\#\text{params of }W_i}$}
			%     \EndFor
			%     \State $\mathcal{M} \gets \textproc{Adaptive Layer-Wise Adaptive Compensation Update}(\mathcal{M, C, \text{Set}_S, \text{Set}_{SVD}})$
			%     \State \Return{$\mathcal{M'}$}
			% \EndProcedure
			% \end{algorithmic}
		% \label{algo:framework}
		% \end{algorithm*}
	
	
	\begin{algorithm*}[t]
		\caption{Pseudocode of \ours}
		\begin{algorithmic}[1] % The number [1] tells LaTeX to number each line
			\State \textbf{Inputs:} LLM $\mathcal M$, Calib Data $\mathcal C$, Bucket Size $M$, Target Retention Ratio $trr$, Min Retention Ratio $mrr$, Update Iteration $k$
			\State \textbf{Outputs:} Updated Model $\mathcal M'$ by \ours
			\Procedure{\ours}{$\mathcal{M, C}, trr, mrr, k$} % Algorithm name and parameters
			\State $\mathcal{X} \gets$ \textproc{Get\_calib($\mathcal{C}$)}  \Comment{Randomly collect samples as calibration data}
			
			\State $\mathcal{X}'[1],\mathcal{X}'[2],...,\mathcal{X}'[M] \gets $ \textproc{SOB($\mathcal{X}, M$)} \Comment{Shuffle samples and utilize stack-of-batch (SOB) strategy}
			
			\State $\text{Set}_\mathcal{S} \gets \textproc{Whitening}(\mathcal M, \mathcal{X}')$, $\text{Set}_\mathcal{SVD} \gets \emptyset$, $\text{Set}_\mathcal{W} \gets \mathcal{M}$ \Comment{Initialize sets of decomposed matrices and weights}
			
			\State $\text{Set}_\mathcal{CR} \gets$ \textproc{Layer\_CR($\mathcal{M}, \mathcal{X}', trr, mrr$)}
			\Comment{Calculate layerwise importance and compression ratio}
			
			\For{layer $i$ \textbf{in} language model $\mathcal{M}$}
			\State $\mathcal{W}_i \gets \text{Set}_\mathcal{W}(i)$, $\mathcal{S}_i \gets \text{Set}_\mathcal{S}(\mathcal{W}_i)$  \Comment{Extract the whitening matrix of current weight $\mathcal{W}_i$} 
			
			\State $\mathcal{U}_i, \Sigma_i, \mathcal{V}_i \gets$ \textproc{SVD($\mathcal{W}_i\mathcal{S}_i$)}
			\Comment{Apply Singular Value Decomposition}
			
			\State $\Sigma' \gets$ \textproc{Trunc($\Sigma_i$)}, ($\mathcal{U}_i', \mathcal{V}_i') \gets$ \textproc{Trunc\_UV($\mathcal{U,V},\Sigma'$)}
			\Comment{Apply adaptive compression ratio and truncation}
			
			\State $\text{Set}_{\mathcal{SVD}} \gets (\mathcal{U}_i', \mathcal{V}_i') \cup \text{Set}_{\mathcal{SVD}}$ 
			
			\EndFor
			\State $\mathcal{M}' \gets$ \textproc{Ada\_Update$(\mathcal{M, \mathcal{X}', \text{Set}_{SVD}}, k)$}
			\Comment{Utilize alternate update for $\mathcal{U}_i', \mathcal{V}_i'$ with iteration $k$}
			
			\State \Return{$\mathcal{M'}$}
			\EndProcedure
		\end{algorithmic}
		\label{algo:framework}
	\end{algorithm*}
	
	
	\vspace{-2mm}
	\section{Related Works}
	\vspace{-2mm}
	\subsection{LLM Compression Techniques}
	\vspace{-2mm}
	Recent advancements in model compression techniques have significantly enhanced the efficiency of deploying LLMs while maintaining their performance. Widely explored approaches include weight quantization~\cite{frantar2022gptq, lin2024awq}, network pruning~\cite{frantar2023sparsegpt, ma2023llmpruner, yang2024laco, gromov2024unreasonable, ashkboos2024slicegpt}, and hybrid methods~\cite{dong2024stbllm}.
	In unstructured pruning, SparseGPT~\cite{frantar2023sparsegpt} prunes weights based on their importance, as determined by the Hessian matrix. However, it faces challenges in achieving optimal speedup, particularly due to hardware compatibility issues. Structured pruning methods, in contrast, are more hardware-friendly. LLM-Pruner~\cite{ma2023llmpruner} selectively removes non-critical coupled structures using gradient information. LaCo~\cite{yang2024laco} introduces a layer-wise pruning strategy, where subsequent layers collapse into preceding ones. ~\citet{gromov2024unreasonable} explores the effectiveness of basic layer-pruning techniques combined with parameter-efficient fine-tuning (PEFT). Additionally, SliceGPT~\cite{ashkboos2024slicegpt} has pioneered post-training sparsification, emphasizing the importance of layer removal order for optimal performance.
	Quantization techniques offer another significant avenue for compression. GPTQ~\cite{frantar2022gptq} applies layer-wise quantization and reduces quantization errors through second-order error compensation. AWQ~\cite{lin2024awq} introduces activation-aware weight quantization, employing a scale transformation between weights and activations. Moreover, BiLLM~\cite{huang2024billm} and ARB-LLM~\cite{li2024arb} achieve further compression to 1-bit while maintaining remarkable performance. More recently, STB-LLM~\cite{dong2024stbllm} combines 1-bit quantization with pruning to achieve even greater memory reduction for LLMs.
	However, many of these compression techniques face challenges related to hardware compatibility, often requiring custom CUDA kernels~\cite{dong2024stbllm} to enable real-time inference speedup.
	
	
	
	\vspace{-1.2mm}
	\subsection{SVD-based LLM Compression}
	\vspace{-1.2mm}
	Singular Value Decomposition (SVD) is a widely used technique for reducing matrix size by approximating a matrix with two smaller, low-rank matrices~\cite{GOLUB1987317}. Although SVD-based methods have demonstrated potential in compressing LLMs, their full capabilities remain underexplored. Standard SVD typically focuses on compressing the original weight matrix without considering the significance of individual parameters, which can lead to considerable compression errors. To address this, \citet{hsu2022languagemodel} introduced FWSVD, which incorporates Fisher information to weight the importance of parameters. However, this method requires complex gradient calculations, making it resource-intensive. Another limitation of standard SVD is the impact of activation distribution on compression errors. To mitigate this, \citet{yuan2024asvd} proposed ASVD, which scales the weight matrix with a diagonal matrix that accounts for the influence of input channels on the weights. Subsequently, \citet{wang2024svdllm} introduced SVD-LLM, which establishes a connection between singular values and compression loss. This work demonstrates that truncating the smallest singular values after data whitening effectively minimizes compression loss. Despite these advancements, existing methods still exhibit significant accuracy loss at higher compression ratios and lack a comprehensive approach for compensating compressed weights after SVD truncation. Furthermore, most methods apply a uniform compression ratio across all transformer layers, overlooking the varying importance of different layers. $\ours$ seeks to address these limitations by proposing an adaptive compensation method (\textbf{adaComp}) and an importance-aware adaptive compression ratio method (\textbf{adaCR}).
	
	
	\vspace{-2.5mm}
	\section{Method}
	\vspace{-2.5mm}
	\textbf{Overview.\quad} As illustrated in~\cref{fig:overview}, our $\ours$ integrates adaptive compensation for SVD truncation (\textbf{adaComp}) with an adaptive importance-aware compression ratio method (\textbf{adaCR}). In~\cref{sec:adacom}, we first describe how \textbf{adaComp} compensates for SVD truncation. Next, in~\cref{sec:adacr}, we detail how \textbf{adaCR} determines the compression ratio based on layer importance. The pseudocode of $\ours$ is shown in~\cref{algo:framework}, and pseudocodes for \textbf{adaComp} and \textbf{adaCR} are provided in the supplementary file.
	
	\vspace{-2mm}
	\subsection{Adaptive Compensation for SVD Truncation}
	\label{sec:adacom}
	\vspace{-1.5mm}
	SVD compression first applies SVD decomposition for matrix $\mathcal{W}$, and then truncates the smallest singular values:
	\begin{align}
		\mathcal{W} = \mathcal{U}\Sigma \mathcal{V}^\top \approx \mathcal{U}_k\Sigma_k\mathcal{V}_k^\top = \widehat{\mathcal{W}},
	\end{align}
	where $\Sigma_k$ indicates the retaining top-k largest singular values, $\mathcal{U}_k$ and $\mathcal{V}_k^\top$ represent the corresponding retaining singular vectors. Moreover, the diagonal matrix $\Sigma_k$ can be further absorbed into $\mathcal{U}_k$ and $\mathcal{V}_k^\top$ by
	\begin{align}
		\mathcal{U}_k^\sigma &= \mathcal{U}_k\Sigma_k^\frac{1}{2}, \ \mathcal{V}_k^\sigma = \mathcal{V}_k\Sigma_k^\frac{1}{2},\\ \widehat{\mathcal{W}}&=\mathcal{U}_k\Sigma_k\mathcal{V}_k^\top = \mathcal{U}_k^\sigma(\mathcal{V}_k^\sigma)^\top.
	\end{align}
	The truncation of the smallest singular values minimizes the compression error with respect to $\mathcal{W}$, ensuring that $||\mathcal{U}_k^\sigma(\mathcal{V}_k^\sigma)^\top-\mathcal{W}||_F^2$ is minimized, which we refer to as the vanilla SVD method. However, this approach does not fully account for the practical effects of $\mathcal{X}$. To address this limitation, we introduce a more application-relevant metric for the SVD compression error, defined as follows:
	\begin{align}
		\mathcal{L}_\text{SVD}&=||\widehat{\mathcal{W}}\mathcal{X}-\mathcal{WX}||_F^2 \notag\\
		&=||\mathcal{U}_k^\sigma(\mathcal{V}_k^\sigma)^\top \mathcal{X}-\mathcal{WX}||_F^2.
	\end{align}
	Previous works~\cite{hsu2022languagemodel, yuan2024asvd, wang2024svdllm} have made significant efforts to minimize $\mathcal{L}_\text{SVD}$. However, some of them involve complex and time-consuming preprocessing steps. Furthermore, they still face substantial challenges in effectively mitigating the large errors that arise under high compression ratios, particularly when truncating 60\% or more of the parameters.
	
	\begin{figure}[t]
		\centering
		\includegraphics[width=1\linewidth]{figs/3.1-v4.pdf}
		\vspace{-7mm}
		\caption{Adaptive compensation for SVD truncation (\textbf{adaComp}). (a) Comparison between naive (NU) and Moore-Penrose pseudoinverse update (MPPU). (b) Comparison between naive (NC) and stack-of-batch calibration strategy (SobC). (c) Distribution comparison before and after applying \textbf{adaComp}. }
		\vspace{-6mm}
		\label{fig:3.1}
	\end{figure}
	
	To compensate for the error attributed to SVD truncation, we need to optimize the following objective:
	\begin{align}
		\mathcal{U}_k^\sigma, {\mathcal{V}_k^\sigma}^\top &= \arg\min_{\mathcal{U}_k^\sigma, {\mathcal{V}_k^\sigma}^\top} \| \mathcal{U}_k^\sigma {\mathcal{V}_k^\sigma}^\top \mathcal{X} - \mathcal{WX} \|_F^2.
	\end{align}
	A straightforward approach is to compute the partial derivatives of the SVD compression objective with respect to $\mathcal{U}_k^\sigma$ and ${\mathcal{V}_k^\sigma}^\top$, resulting in the following expressions (additional details can be found in the supplementary file):
	\begin{align}
		&\frac{\partial \mathcal{L}_\text{SVD}}{\partial \mathcal{U}_k^\sigma} = 0 \notag\\
		&\quad\Rightarrow \mathcal{U}_k^\sigma = \mathcal{WX} \mathcal{X}^\top \mathcal{V}_k^\sigma((\mathcal{V}_k^\sigma)^\top \mathcal{X} \mathcal{X}^\top \mathcal{V}_k^\sigma)^{-1}, \\
		&\frac{\partial \mathcal{L}_\text{SVD}}{\partial {\mathcal{V}_k^\sigma}^\top} = 0 \notag\\
		&\quad\Rightarrow {\mathcal{V}_k^\sigma}^\top = ((\mathcal{U}_k^\sigma)^\top \mathcal{U}_k^\sigma)^{-1}(\mathcal{U}_k^\sigma)^\top \mathcal{W}.
	\end{align}
	However, this method involves computing the matrix inverse, which can lead to unstable updates and significant compression errors, as shown in~\cref{fig:3.1} (a). To mitigate the issue of numerical instability, we propose a two-fold strategy to enhance the update quality of $\mathcal{U}_k^\sigma$ and ${\mathcal{V}_k^\sigma}^\top$.
	
	\begin{figure*}[htbp]
		\centering
		\includegraphics[width=1\textwidth]{figs/3.2-v4.pdf}
		\vspace{-7mm}
		\caption{Layer-wise relative importance of different LLMs. The importance across different layers varies significantly, and the first layer always weight most importance. More layer-wise importance visualization can be found in the supplementary file.}
		\vspace{-3.5mm}
		\label{fig:layer_importance}
	\end{figure*}
	
	\textbf{First}, the optimization objective for $\mathcal{U}_k^\sigma$ is reformulated as a Least Squares Estimation (LSE) problem, where ${\mathcal{V}_k^\sigma}^\top \mathcal{X}$ is treated as the input and $\mathcal{WX}$ as the output:
	\begin{align}
		\mathcal{U}_k^\sigma &= \arg\min_{\mathcal{U}_k^\sigma} \| \mathcal{A}(\mathcal{U}_k^\sigma)^\top  - \mathcal{B} \|_F^2,
	\end{align}
	where $\mathcal{A}=\mathcal{X}^\top \mathcal{V}_k^\sigma$ and $\mathcal{B}=(\mathcal{WX})^\top$. Since 
	$\mathcal{A}$ is typically not a square matrix and may not be full rank, we first apply SVD to 
	$\mathcal{A}$ to enhance numerical stability:
	\begin{align}
		\mathcal{A} = \mathcal{U}_\mathcal{A}\Sigma_\mathcal{A}\mathcal{V}_\mathcal{A}^\top,
	\end{align}
	and then obtain the solution for $\mathcal{U}_k^\sigma$ by using the Moore-Penrose pseudoinverse~\cite{penrose1955generalized} of $\mathcal{A}$:
	\begin{align}
		\mathcal{U}_k^\sigma = (\mathcal{A}^+\mathcal{B})^\top = (\mathcal{V}_\mathcal{A}\Sigma_\mathcal{A}^+\mathcal{U}_\mathcal{A}^\top \mathcal{B})^\top,
	\end{align}
	where $\Sigma_A^+$ denotes the Moore-Penrose pseudoinverse of $\Sigma_A$:
	\begin{align}
		\Sigma_A &= \text{diag}(\sigma_1, \sigma_2, \dots, \sigma_n), \\
		\Sigma_A^+ &= \text{diag} \left( \sigma_1^{-1} \mathbbm{1}_{\sigma_1 \neq 0}, \sigma_2^{-1} \mathbbm{1}_{\sigma_2 \neq 0}, \dots, \sigma_n^{-1} \mathbbm{1}_{\sigma_n \neq 0} \right).
	\end{align}
	Similarly, we update ${\mathcal{V}_k^\sigma}^\top$ using the Moore-Penrose pseudoinverse of $\mathcal{U}_k^\sigma$ to handle numerical instability:
	\begin{align}
		{\mathcal{V}_k^\sigma}^\top &= \arg\min_{{\mathcal{V}_k^\sigma}^\top} \| \mathcal{U}_k^\sigma {\mathcal{V}_k^\sigma}^\top \mathcal{X} - \mathcal{WX} \|_F^2 \notag\\
		&= {\Big((\mathcal{U}_k^\sigma)^+\Big)}^\top \mathcal{W}.
	\end{align}
	As shown in~\cref{fig:3.1} (a), by reformulating the optimization objective as an LSE problem and solving for $\mathcal{U}$ and $\mathcal{V}^\top$ using the Moore-Penrose pseudoinverse, we achieve a smooth curve that consistently reduces compression error stably.
	
	\textbf{Second}, since the update rule incorporates the calibration data $\mathcal{X}$, ideally, a large volume of $\mathcal{X}$ would yield better results. However, during our experiments, we found that extending $\mathcal{X}$ to just 32 samples on an 80GB GPU is challenging. To address this, we propose a \textbf{stack-of-batch} strategy that enables the utilization of more calibration data without increasing memory overhead. Specifically, given $N$ calibration samples and a bucket size $M$ (the maximum number of samples that can fit within the fixed GPU memory), we randomly sample $mini\_bsz=\lceil\frac{N}{M}\rceil$ samples into one bucket by taking their mean value as follows:
	\begin{align}
		\mathcal{X}_{\text{rand}} &= \textit{Shuffle}(\mathcal{X}), \\
		\mathcal{X}'[k] &= \frac{1}{mini\_bsz}\sum_{i=1}^{mini\_bsz} \mathcal{X}_{\text{rand}}[(k-1) \cdot mini\_bsz + i],
	\end{align}
	where $k = 1, 2, \dots, M$, and cardinality $|\mathcal{X}'|=M$.
	As shown in~\cref{fig:3.1} (b), integrating the \textbf{stack-of-batch} strategy further reduces the compression error.
	
	
	As shown in~\cref{fig:overview}, 
	to compensate for the error attributed to SVD truncation, we propose an adaptive method to subsequently update $\mathcal{U}_k^\sigma$ and $\mathcal{V}_k^\sigma$ with the above update rules.
	Moreover, the adaptation of $\mathcal{U}_k^\sigma$ and $\mathcal{V}_k^\sigma$ can be alternatively applied until convergence, where the update sequence over $\tau$ iterations can be expressed as
	\begin{align}
		\boxed{(\mathcal{U}_k^\sigma)^\mathbf{1}
			\rightarrow ({\mathcal{V}_k^\sigma}^\top)^\mathbf{1}} &\rightarrow \boxed{(\mathcal{U}_k^\sigma)^\mathbf{2} \rightarrow ({\mathcal{V}_k^\sigma}^\top)^\mathbf{2}} \notag\\
		\rightarrow \cdots &\rightarrow \boxed{(\mathcal{U}_k^\sigma)^{\boldsymbol{\tau}} \rightarrow ({\mathcal{V}_k^\sigma}^\top)^{\boldsymbol{\tau}}},
	\end{align}
	where $(\mathcal{U}_k^\sigma)^{\boldsymbol{\tau}}$ and $({\mathcal{V}_k^\sigma}^\top)^{\boldsymbol{\tau}}$ denote the updated singular matrices after $\tau$-th iteration, respectively, while the region bounded by $\boxed{\phantom{0000}}$ corresponding to one iteration of alternative update.
	As shown in~\cref{fig:3.1} (c), the gap between the outputs of the compressed and original models narrows after alternative updates. The overlapping area rapidly increases after just a few iterations. More visual comparisons are shown in the supplementary file.
	
	Notably, our adaptive compensation can be integrated with data whitening proposed by~\citet{wang2024svdllm} and~\citet{liu2024eora}, further reducing the SVD truncation error.
	
	\begin{table*}[t]
		\centering
		% \vspace{-2mm}
		
		
		\resizebox{1\textwidth}{!}{%}
		% \tiny
		\begin{tabular}{c|c|c|c|c|c|c|c|c|c|c
				% >{\centering\arraybackslash}p{1cm}|>{\centering\arraybackslash}p{2.2cm}|>{\centering\arraybackslash}p{2.45cm}|>{\centering\arraybackslash}p{2.45cm}|>{\centering\arraybackslash}p{2.45cm}|>{\centering\arraybackslash}p{1.4cm}|>{\centering\arraybackslash}p{1.4cm}|>{\centering\arraybackslash}p{1.4cm}|>{\centering\arraybackslash}p{1.4cm}|>{\centering\arraybackslash}p{1.4cm}|>{\centering\arraybackslash}p{1.4cm}
			}
			% \toprule[0.01pt]
			\toprule
			{\textsc{Ratio}}       & {\textsc{Method}}    & ~~~~WikiText-2{$\downarrow$}~~~~ & PTB{$\downarrow$} & ~~{C4{$\downarrow$}}~~ & ~~Mmlu~~ & ~ARC\_e~ & ~WinoG.~ & ~HellaS.~  & ~~PIQA~~ & ~\textbf{Average{$\uparrow$}}~        \\ \midrule
			{\color[HTML]{9B9B9B}0\%}  & {\color[HTML]{9B9B9B}Original}  & {\color[HTML]{9B9B9B}5.68}   & {\color[HTML]{9B9B9B}8.35}     & {\color[HTML]{9B9B9B}7.34}      & {\color[HTML]{9B9B9B} 45.30} & {\color[HTML]{9B9B9B} 74.62} & {\color[HTML]{9B9B9B} 69.22} & {\color[HTML]{9B9B9B} 76.00}  & {\color[HTML]{9B9B9B} 79.11} & {\color[HTML]{9B9B9B} 68.85}       \\ \midrule
			{\multirow{5}{*}{40\%}}   & {SVD}      & 39,661.03           & 69,493.00          & {56,954.00} & \textbf{26.51} &  26.39 & 48.62  &  25.64 & 52.99    & 36.03 \\ 
			
			{} & {FWSVD~\cite{hsu2022languagemodel}}      & 8,060.35           & 9,684.10          & 7,955.21              & 25.74  &26.05   &50.20   &25.70      &52.39     &36.01  \\
			
			{} & {ASVD~\cite{yuan2024asvd}}      & 1,609.32           & 7,319.49          & 1,271.85              &24.35   &26.81   &49.49   &25.83     &53.81     &36.06  \\
			
			{} & {SVD-LLM~\cite{wang2024svdllm}}      &16.11  &719.44          & {61.95}              & 22.97  & 36.99  & 56.04  & 30.49   & 56.96  & 40.69 \\
			\cmidrule{2-11} 
			{}                     & \cellcolor{purple!10}{\textbf{$\ours$}}  & \cellcolor{purple!10}\textbf{14.76} ~\footnotesize($\downarrow$8\%)  & \cellcolor{purple!10}\textbf{304.62}~\footnotesize($\downarrow$58\%) & \cellcolor{purple!10}{\textbf{56.98}}~\footnotesize($\downarrow$8\%)   & \cellcolor{purple!10}23.63               & \cellcolor{purple!10}\textbf{41.12}               & \cellcolor{purple!10}\textbf{58.17}               & \cellcolor{purple!10}\textbf{31.75}                  & \cellcolor{purple!10}\textbf{58.49}               %& \cellcolor{purple!10}\textbf{38.62}               
			& \cellcolor{purple!10}\textbf{42.63}    \\ 
			
			
			\midrule
			{\multirow{5}{*}{50\%}}                     & {SVD}      & 53,999.48           & 39,207.00          & {58,558.00} & \textbf{25.43}  & 25.80  & 47.36  & 25.55    & 52.67    & 35.36 \\
			
			{} & {FWSVD~\cite{hsu2022languagemodel}}      & 8,173.21           & 8,615.71          & 8,024.67              &24.83   &25.84   &48.70   &25.64      &52.83    &35.57  \\
			
			{} & {ASVD~\cite{yuan2024asvd}}      & 6,977.57           & 15,539.44          & 4,785.15              &24.52   &25.13   &49.17   &25.48     &52.94    &35.45  \\
			
			{} & {SVD-LLM~\cite{wang2024svdllm}}      & 27.19           & 1,772.91        & {129.66}              & 23.44  & 31.65  & 51.14  & 28.38    & 54.57    & 37.83 \\ 
			
			\cmidrule{2-11} 
			{}                     & \cellcolor{purple!10}{\textbf{$\ours$}}  &\cellcolor{purple!10}\textbf{25.58}~\footnotesize($\downarrow$6\%)   & \cellcolor{purple!10}\textbf{593.14}~\footnotesize($\downarrow$67\%) & \cellcolor{purple!10}{\textbf{113.84}}~\footnotesize($\downarrow$12\%)   & \cellcolor{purple!10}23.24               & \cellcolor{purple!10}\textbf{34.18}               & \cellcolor{purple!10}\textbf{54.06}               & \cellcolor{purple!10}\textbf{28.88}                        & \cellcolor{purple!10}\textbf{55.50}             % & \cellcolor{purple!10}\textbf{37.83}               
			& \cellcolor{purple!10}\textbf{39.17}    \\ 
			
			
			\midrule
			{\multirow{5}{*}{60\%}} & {SVD}      & 65,186.67           & 79,164.00          & {70,381.00} & 22.94  & 24.49  & \textbf{51.85}  & 25.40   & 53.16   & 35.57 \\
			
			{} & {FWSVD~\cite{hsu2022languagemodel}}      & 27,213.30           & 24,962.80          & 47,284.87              &\textbf{26.91}   &25.38   &48.46   &25.61     & 51.96    & 35.66 \\
			
			{} & {ASVD~\cite{yuan2024asvd}}      & 10,003.57           & 15,530.19          & 9,983.83              &26.89   &26.68   &48.86   &25.76      & 51.80   &36.00  \\
			
			{} & {SVD-LLM~\cite{wang2024svdllm}}      & 89.90           & 2,052.89         & {561.00}              & 22.88  & 26.73  & 47.43  & 26.89    & \textbf{53.48}    & 35.48 \\ 
			\cmidrule{2-11} 
			{}                     & \cellcolor{purple!10}{\textbf{$\ours$}}  & \cellcolor{purple!10}\textbf{50.33}~\footnotesize($\downarrow$44\%)   & \cellcolor{purple!10}\textbf{1,216.95}~\footnotesize($\downarrow$41\%) & \cellcolor{purple!10}\textbf{239.18}~\footnotesize($\downarrow$57\%)  & \cellcolor{purple!10}24.69              & 
			\cellcolor{purple!10}\textbf{28.20}               & 
			\cellcolor{purple!10}51.22               & 
			\cellcolor{purple!10}\textbf{27.36}               & \cellcolor{purple!10}52.83                            & \cellcolor{purple!10}\textbf{36.87}   \\ 
			\bottomrule
			
			
			% \bottomrule[0.01pt]
			% \vspace{-15mm}
		\end{tabular}
	}
	\label{tab:dataset_acc}
	\vspace{-2mm}
	\caption{Zero-shot performance comparison of LLaMA2-7B between $\ours$ and previous SVD compressed methods under 40\% to 60\% compression ratios. Evaluation on three language modeling datasets (measured by perplexity  ({$\downarrow$})) and five common sense reasoning datasets (measured by both individual and average accuracy ({$\uparrow$})) demonstrate the effectiveness of $\ours$.
	}
	\vspace{-4mm}
\end{table*}

% adaCR
\subsection{Adaptive SVD Compression Ratio}
\label{sec:adacr}


Previous studies on SVD compression typically apply a uniform compression ratio across all transformer layers of LLMs, overlooking the varying importance of different layers.
Inspired by~\citet{men2024shortgpt} and~\citet{dumitru2024change}, we propose \textbf{adaCR}, which adaptively determines the SVD compression ratio for each transformer layer, considering each layer's distinct impact on activations.

The importance of $\mathcal{W}$ can be measured by its impact on the input, which is quantified as the similarity between the input $\mathcal{X}$ and the output $\mathcal{Y}$ after passing through $\mathcal{W}$.
\begin{align}
	\mathcal{Y} &= \mathcal{WX}, \\
	\mathcal{I}(\mathcal{W}) &= \text{similarity}(\mathcal{X,Y}),
\end{align}
where $\mathcal{I}(\mathcal{W})$ denotes the layer-wise importance of $\mathcal{W}$. The similarity metric used can vary, and for simplicity, we adopt cosine similarity in our method.

Then, we normalize $\mathcal{I}(\mathcal{W})$ through mean centering to obtain the relative importance of $\mathcal{W}$:
\begin{align}
	\mathcal{I}_n(\mathcal{W}) = \mathcal{I}(\mathcal{W}) / \text{mean}(\mathcal{I}(\mathcal{W})).
\end{align}
After mean normalization, the average importance is 1. A value of $\mathcal{I}_n(\mathcal{W})$ greater than 1 indicates greater importance, while a value lower than 1 indicates lesser importance. The compression ratio of each layer will be adaptively adjusted based on the relative importance:
\begin{align}
	\mathcal{CR}(\mathcal{W}) = mrr + \mathcal{I}_n(\mathcal{W}) \cdot (trr - mrr),
\end{align}
where $mrr$ and $trr$ are the minimum and target retention ratios, respectively. Notably, $\mathcal{CR}(\mathcal{W}) = mrr$ when $\mathcal{I}_n(\mathcal{W})=0$, and $\mathcal{CR}(\mathcal{W}) = trr$ when $\mathcal{I}_n(\mathcal{W}) = 1$.

Given the compression ratio for the $i$-th layer by \textbf{adaCR}, we truncate the vectors of least singular values from both $\mathcal{U}_k^\sigma$ and ${\mathcal{V}_k^\sigma}^\top$ so that 
\begin{align}
	\mathcal{CR}(\mathcal{W}_i)=\frac{\#\text{params of }\mathcal{U}_k^\sigma + \#\text{params of }{\mathcal{V}_k^\sigma}^\top}{\#\text{params of }\mathcal{W}_i}.
\end{align}
As shown in~\cref{fig:layer_importance}, the importance of different layers varies. It can be observed that the first layer always weighs the most importance, suggesting that we should retain more weight on it. For the Llama family, the relative importance curve approximates a bowl shape, highlighting the significance of both the initial and final layers.



\section{Experiments}
\subsection{Setup}

We compare our $\ours$ with four baselines,  including vanilla SVD and SOTA SVD-based LLM compression methods FWSVD~\cite{hsu2022languagemodel}, ASVD~\cite{yuan2024asvd}, and SVD-LLM~\cite{wang2024svdllm}.

\begin{figure*}[htbp]
	\centering
	\includegraphics[width=1\textwidth]{figs/LLaVA-main.pdf}
	\vspace{-6mm}
	\caption{We perform image captioning by applying SVD, SVD-LLM~\cite{wang2024svdllm}, and our $\ours$ to LLaVA-7B model on the COCO dataset respectively, highlighting the \textcolor{color4}{correct} captions and \textcolor{red}{wrong} captions in different colors.}
	\vspace{-4.5mm}
	\label{fig:llava}
\end{figure*}

\begin{table}[t]
	\centering
	\vspace{1.5mm}
	%The relative performance gain compared to the best-performing baseline is marked in green color inside bracket.
	% \vspace{-2mm}
	\resizebox{\linewidth}{!}{%
		% \tiny
		\begin{tabular}{c|cccc}
			% \toprule[0.01pt]
			\toprule
			\textsc{Method} &OPT-6.7B   & LLaMA2-7B    & Mistral-7B        & Vicuna-7B \\ \midrule
			SVD       & 18,607.24          & 65,186.67         & 30,378.35            & 78,704.50         \\
			FWSVD~\cite{hsu2022languagemodel}     & 8,569.56      & 27,213.30          & 5,481.24              & 8,185.66       \\
			ASVD~\cite{yuan2024asvd}      & 10,326.48         & 10,003.57         & 22,705.51             & 20,241.17          \\
			SVD-LLM~\cite{wang2024svdllm}      &   92.10      &   89.90       &  72.17            &  64.06        \\
			\midrule
			\rowcolor{purple!10}\textbf{$\ours$}  & \textbf{86.64}~\footnotesize($\downarrow$6\%)         & \textbf{50.33}~\footnotesize($\downarrow$44\%)           &  \textbf{67.22}~\footnotesize($\downarrow$7\%)          & \textbf{56.97}~\footnotesize($\downarrow$11\%)         \\ \bottomrule
			% \bottomrule[0.01pt]
		\end{tabular}
	}
	\vspace{-2mm}
	\caption{Perplexity ($\downarrow$) of four different LLMs -- OPT-6.7B, LLaMA 2-7B, Mistral-7B, and Vicuna-7B -- under 60\% compression ratio on WikiText-2, where $\ours$ shows consistent improvements. \label{tab:different_llm_acc}}
	\vspace{-4mm}
\end{table}

\noindent\textbf{Models and Datasets.$\quad$} 
%
To demonstrate the generalizability of our method, we evaluate the performance of $\ours$ and the baselines on four models from three different LLM families, including LLaMA2-7B~\cite{touvron2023llama2}, OPT-6.7B~\cite{zhang2022opt}, Mistral-7B~\cite{jiang2023mistral}, and Vicuna-7B~\cite{chiang2023vicuna}. We benchmark on eight datasets, including three language modeling datasets (WikiText-2~\cite{merity2016pointersentinelmixturemodels}, PTB~\cite{marcus1993building}, and C4~\cite{raffel2023exploring}) and five common-sense reasoning datasets (WinoGrande~\cite{sakaguchi2019winograndeadversarialwinogradschema}, HellaSwag~\cite{zellers2019hellaswagmachinereallyfinish}, PIQA~\cite{bisk2019piqareasoningphysicalcommonsense}, ARC-e~\cite{clark2018thinksolvedquestionanswering}, and Mmlu~\cite{hendryckstest2021}).
We use the LM-Evaluation-Harness framework~\cite{eval-harness} to evaluate the model performance on these zero-shot Question-Answering (QA) datasets.

\noindent\textbf{Implementation Details.$\quad$} 
%
To ensure a fair comparison, we followed ASVD~\cite{yuan2024asvd} and SVD-LLM~\cite{wang2024svdllm} to randomly select 256 samples from WikiText-2 as the calibration data and conduct data whitening before SVD truncation. 
All the experiments are conducted with PyTorch~\cite{paszke2019pytorch} and Huggingface~\cite{paszke1912imperative} on a single NVIDIA A100-80GB GPU. 


\subsection{Main Results}

We evaluate the overall performance of $\ours$ from three aspects: \textbf{(1)} performance under different compression ratios \textbf{(40\%, 50\%, 60\%, 70\%, and 80\%)}, \textbf{(2)} performance on different LLMs. \textbf{(3)} performance on visual language models. Some performance evaluation results and generated contents by the compressed LLMs are included in the supplementary file to provide a more straightforward comparison.


\noindent\textbf{Performance under Different Compression Ratios.$\quad$}
First, we evaluate the performance of LLaMA2-7B compressed by $\ours$, vanilla SVD and the SOTA method SVD-LLM~\cite{wang2024svdllm} under compression ratios ranging from 40\% to 80\% on all $8$ datasets, as shown in~\cref{tab:dataset_acc}. 
On the three language modeling datasets, $\ours$ consistently outperforms vanilla SVD, and SVD-LLM across all the compression ratios. 
More importantly, $\ours$ exhibits significant advantages over the baselines under higher compression ratios. 
These results indicate that $\ours$ is more effective in compressing LLMs for more resource-constrained devices such as smartphones and IoT devices, which often have limited memory and processing capabilities.
On the five common sense reasoning datasets, 
$\ours$ also maintains its edge and performs better than the best-performing baseline on most of the datasets and consistently achieves higher average accuracy across all the compression ratios. Due to page limitations, comparisons for 70\% and 80\% compression ratios are provided in the supplementary file.



\noindent\textbf{Performance on Different LLMs.$\quad$}
To demonstrate the generability of $\ours$ across different LLMs, we compare $\ours$ and the baselines on four different models OPT-6.7B, LLaMA2-7B, Vicuna-7B, and Mistral-7B -- under 60\% compression ratio on WikiText-2.
As shown in~\cref{tab:different_llm_acc},  
$\ours$ consistently outperforms vanilla SVD, FWSVD, ASVD and SVD-LLM on all LLMs, and exhibits more stable performance across different LLMs, especially compared to vanilla SVD and FWSVD. We reproduce FWSVD, ASVD, and SVD-LLM using their official GitHub repositories. FWSVD and ASVD fail on these LLMs with compression ratios under 60\%, whereas SVD-LLM and $\ours$ maintain reasonable perplexity in such cases.


\noindent\textbf{Performance on Visual Language Models.$\quad$} Note that our $\ours$ can also be applied to visual language models (VLMs) like LLaVA~\cite{liu2023visual}. Following~\citet{lin2024awq}, we apply SVD compression to the language part of the VLMs since it dominates the model size. As shown in~\cref{fig:llava}, $\ours$ shows better image captioning results than vanilla SVD and SVD-LLM on COCO dataset~\cite{chen2015microsoft} under 40\% compression ratio.
More image captioning comparisons with various compression ratios can be found in supplementary file.




\subsection{Ablation Study}

We provide extensive ablation study results in ~\cref{tab:ablations} to show the effect of some key components in our work. 

\begin{table*}[t]
	% \vspace{-1.5mm}
	% subfloat c - BackBone Architecture
	\vspace{-2mm}
	
	
	\subfloat[\small Effectiveness of Adaptive Compensation \label{tab:adacomp}]{
		\scalebox{0.8}{\begin{tabular}{l@{\hskip 9pt}c@{\hskip 12pt}c@{\hskip 12pt}c@{\hskip 12pt}c@{\hskip 12pt}c}
				\toprule
				\rowcolor{color3}
				\textbf{Method} & \textbf{Tgt. CR} & \textbf{adaComp}  &\textbf{WikiText2 $\downarrow$} & \textbf{PTB $\downarrow$} & \textbf{C4 $\downarrow$}   \\
				\midrule
				SVD-LLM & 40\% & \ding{55} &16.11  &719.44   &61.95 \\
				\cdashline{1-6} \addlinespace[0.2em]
				$\ours$ & 40\% & \ding{55} & 15.47  &406.83  &66.29  \\
				\rowcolor{purple!10}$\ours$ & 40\% &  \ding{51} &14.76  &304.62  &56.98  \\
				\midrule
				SVD-LLM & 50\% & \ding{55} & 27.19&1,772.91 &129.66 \\
				\cdashline{1-6} \addlinespace[0.2em]
				$\ours$ & 50\% & \ding{55} & 30.00 & 1101.15 &166.02  \\
				\rowcolor{purple!10}$\ours$ & 50\% &  \ding{51} &25.58  & 593.14 &113.84  \\
				\midrule
				SVD-LLM & 60\% & \ding{55} &89.90 &2,052.89 &561.00 \\
				\cdashline{1-6} \addlinespace[0.2em]
				$\ours$ & 60\% & \ding{55} &78.82  &6,929.39  &339.31  \\
				\rowcolor{purple!10}$\ours$ & 60\% &  \ding{51} &50.33  &1,216.95 &239.18  \\
				\bottomrule
	\end{tabular}}}\hfill
	\subfloat[\small Effectiveness of Adaptive Compression Ratio\label{tab:adacr}]{ 
		\scalebox{0.8}{
			\begin{tabular}{l@{\hskip 9pt}c@{\hskip 12pt}c@{\hskip 12pt}c@{\hskip 12pt}c@{\hskip 12pt}c}
				%\small
				\toprule
				\rowcolor{color3}\textbf{Method} &\textbf{Tgt. CR} &\textbf{CR} & \textbf{WikiText2 $\downarrow$} & \textbf{PTB $\downarrow$} & \textbf{C4 $\downarrow$} \\
				\midrule
				SVD-LLM & 40\% & Const &16.11  &719.44   &61.95  \\
				\cdashline{1-6} \addlinespace[0.2em]
				$\ours$ & 40\% & Const & 15.38  &617.11  &60.43  \\
				\rowcolor{purple!10}$\ours$ & 40\% & Adapt &14.76  &304.62  &56.98  \\
				\midrule
				SVD-LLM & 50\% & Const & 27.19&1,772.91 &129.66 \\
				\cdashline{1-6} \addlinespace[0.2em]
				$\ours$ & 50\% & Const & 27.33 & 1,177.53 &126.85  \\
				\rowcolor{purple!10}$\ours$ & 50\% & Adapt &25.58  & 593.14 &113.84  \\
				\midrule
				SVD-LLM & 60\% & Const &89.90 &2,052.89 &561.00 \\
				\cdashline{1-6} \addlinespace[0.2em]
				$\ours$ & 60\% & Const &69.46  &2,670.20  &336.90 \\
				\rowcolor{purple!10}$\ours$ & 60\% &  Adapt &50.33  &1,216.95 &239.18   \\
				
				\bottomrule
	\end{tabular}}} \\
	% subfloat d - Multinomial vs Independent Masks
	% subfloat b - mask representation
	\subfloat[\small Iteration Number for Adaptive Compression \label{tab:num_iter}]{
		\scalebox{0.8}{\begin{tabular}{l@{\hskip 9pt}c@{\hskip 12pt}c@{\hskip 12pt}c@{\hskip 12pt}c@{\hskip 12pt}c}
				\toprule
				\rowcolor{color3}
				\textbf{Method} &\textbf{Tgt. CR} & \textbf{\#Iteration} &\textbf{WikiText2 $\downarrow$} &\textbf{PTB $\downarrow$} &\textbf{C4 $\downarrow$}   \\
				\midrule
				SVD-LLM & 40\% & - &16.11  &719.44   &61.95  \\
				\cdashline{1-6} \addlinespace[0.2em]
				\rowcolor{purple!10}$\ours$ & 40\% & 1 &14.76  &304.62  &56.98   \\
				$\ours$ &  40\% &  3 & 15.47  &249.41   &57.28   \\
				$\ours$ & 40\% & 15  & 15.84 &257.96  & 57.39 \\
				\midrule
				SVD-LLM & 50\% & - & 27.19&1,772.91 &129.66  \\
				\cdashline{1-6} \addlinespace[0.2em]
				\rowcolor{purple!10}$\ours$ & 50\% & 1 &25.58  & 593.14 &113.84  \\
				$\ours$ & 50\% & 3  &27.11  &844.09 &115.51\\
				$\ours$ & 50\% & 15 &27.45  & 812.21 & 110.35  \\
				\midrule
				SVD-LLM & 60\% & - &89.90 &2,052.89 &561.00  \\
				\cdashline{1-6} \addlinespace[0.2em]
				\rowcolor{purple!10}$\ours$ & 60\% & 1 &50.33  &1,216.95 &239.18  \\
				$\ours$ & 60\% & 3 &64.12  &3,546.45 &301.19  \\
				$\ours$ & 60\% & 15 &62.34  &4,293.79 &267.29  \\
				\bottomrule
	\end{tabular}}}\hfill
	\subfloat[\small Minimum Retention Ratio for Adaptive CR\label{tab:mrr}]{ 
		\scalebox{0.8}{\begin{tabular}{l@{\hskip 9pt}c@{\hskip 12pt}c@{\hskip 12pt}c@{\hskip 12pt}c@{\hskip 12pt}c}
				\toprule
				\rowcolor{color3}
				\textbf{Method} &\textbf{Tgt. CR} & \textbf{MRR} &\textbf{WikiText2 $\downarrow$} &\textbf{PTB $\downarrow$} &\textbf{C4 $\downarrow$}   \\
				\midrule
				SVD-LLM & 40\% & - &16.11  &719.44   &61.95  \\
				\cdashline{1-6} \addlinespace[0.2em]
				$\ours$ & 40\% & 0.40 &15.01  &223.19 &57.17  \\
				$\ours$ & 40\% & 0.45 &14.85  &241.90  &57.08   \\
				\rowcolor{purple!10}$\ours$ & 40\% & 0.50 &14.76  &304.62 &56.98  \\
				\midrule
				SVD-LLM & 50\% & - & 27.19&1,772.91 &129.66  \\
				\cdashline{1-6} \addlinespace[0.2em]
				\rowcolor{purple!10}$\ours$ & 50\% & 0.40 &25.58  &593.14 &113.84  \\
				$\ours$ & 50\% & 0.45 &26.01  & 814.63  &117.58  \\
				$\ours$ & 50\% & 0.50 &27.33  &1,177.53 &126.85  \\
				\midrule
				SVD-LLM & 60\% & - &89.90 &2,052.89 &561.00  \\
				\cdashline{1-6} \addlinespace[0.2em]
				\rowcolor{purple!10}$\ours$ & 60\% & 0.30 & 50.33 &1,216.95 &239.18  \\
				$\ours$ & 60\% & 0.35 & 53.17 &1,608.19 &256.66  \\
				$\ours$ & 60\%  & 0.40 & 60.08 &2,137.29 &294.26 \\
				\bottomrule
	\end{tabular}}}
	% subfloat d - Multinomial vs Independent Masks
	% subfloat b - mask representation
	% \vspace{-1mm}
	\vspace{-2mm}
	\caption{Ablation studies on LLaMA-2-7B. Results are measured by perplexity, with best results highlighted in \colorbox{purple!10}{\phantom{0000}}.\label{tab:ablations}}
	\vspace{-3mm}
\end{table*}


\noindent\textbf{Effectiveness of Adaptive Compensation. $\quad$} To validate the effectiveness of the proposed \textbf{adaComp}, we compare the PPL results of Llama2-7B with and without \textbf{adaComp} on Wikitest-2, PTB, and C4 datasets in~\cref{tab:adacomp}. Results of 70\% and 80\% compression ratios can be found in the supplementary file.
It can be observed that $\ours$ consistently outperforms SVD-LLM after applying \textbf{adaComp}, and the performance gap is more significant under high compression ratios (\textit{i.e.}, 60\%, 70\%, and 80\%).




\noindent\textbf{Iteration Number. $\quad$} 
To investigate the impact of the number of \textbf{adaComp} iterations under different compression ratios, we perform an ablation study with 1, 3, and 15 iterations, as shown in~\cref{tab:num_iter}. Results for 70\% and 80\% compression ratios are provided in the supplementary file. At lower compression ratios (\textit{e.g.}, 40\%, 50\%, and 60\%), it is observed that just 1 iteration of \textbf{adaComp} already outperforms the state-of-the-art method, SVD-LLM. However, increasing the number of iterations may lead to overfitting due to the limited calibration data, resulting in a performance drop. In contrast, at higher compression ratios (\textit{e.g.}, 70\% and 80\%), additional iterations lead to performance improvements, indicating that $\ours$ is more effective in high compression ratio scenarios where previous methods still struggle. This highlights the importance of balancing the number of iterations with the available data to avoid over-optimization, especially in low compression scales.



\begin{table}[t]
	\centering
	% \vspace{-2.5mm}
	
	\resizebox{1\linewidth}{!}{%}
	% \tiny
	\begin{tabular}{c|c|c|c|c|c}
		% \toprule[0.01pt]
		\toprule
		{\textsc{Ratio}}       & {\textsc{Method}}   & {\textsc{GPTQ-INT4}} & WikiText-2{$\downarrow$} & PTB{$\downarrow$} & {C4{$\downarrow$}}  \\ \midrule
		{\color[HTML]{9B9B9B}0\%}  & {\color[HTML]{9B9B9B}Original} & \ding{55}  & {\color[HTML]{9B9B9B}5.68}   & {\color[HTML]{9B9B9B}8.35}     & {\color[HTML]{9B9B9B}7.34}    \\ \midrule
		{\multirow{4}{*}{40\%}} & {SVD-LLM}  & \ding{55}     &16.11  &719.44            & {61.95}               \\
		& {SVD-LLM}  & \ding{51}     & 33.56          & 1,887.50          & {184.61}               \\
		% \cmidrule{2-13} 
		& \cellcolor{purple!10}{\textbf{$\ours$}} & \cellcolor{purple!10}\ding{55}  & \cellcolor{purple!10}\textbf{14.76}   & \cellcolor{purple!10}\textbf{304.62} & \cellcolor{purple!10}{\textbf{56.98} }      \\ 
		& \cellcolor{purple!10}{\textbf{$\ours$}} & \cellcolor{purple!10}\ding{51}  & \cellcolor{purple!10}\textbf{22.55}   & \cellcolor{purple!10}\textbf{844.21} & \cellcolor{purple!10}{\textbf{106.41} }      \\
		%%%%%%%%%%%%%%%%%%%%%%%%%%%%%%%%%%%%%%%%%
		\midrule
		
		{\multirow{4}{*}{50\%}} & {SVD-LLM}  & \ding{55}     & 27.19           & 1,772.91  & {129.66}                     \\ 
		& {SVD-LLM}  & \ding{51}     & 41.70           & 2,335.65             & {291.62}         \\ 
		
		% \cmidrule{2-13} 
		& \cellcolor{purple!10}{\textbf{$\ours$}} & \cellcolor{purple!10}\ding{55}  &\cellcolor{purple!10}\textbf{25.58}   & \cellcolor{purple!10}\textbf{593.14}    & \cellcolor{purple!10}{\textbf{113.84} }  \\ 
		& \cellcolor{purple!10}{\textbf{$\ours$}} & \cellcolor{purple!10}\ding{51}  &\cellcolor{purple!10}\textbf{37.34}   & \cellcolor{purple!10}\textbf{1,326.55}    & \cellcolor{purple!10}{\textbf{203.11} }  \\ 
		
		
		\midrule
		
		{\multirow{4}{*}{60\%}} & {SVD-LLM}  & \ding{55}     & 89.90           & 2,052.89          & {561.00}              \\ 
		& {SVD-LLM}  & \ding{51}     & 119.46           & 3,136.60              & {723.80}         \\ 
		% \cmidrule{2-13} 
		& \cellcolor{purple!10}{\textbf{$\ours$}} & \cellcolor{purple!10}\ding{55}  & \cellcolor{purple!10}\textbf{60.08}   & \cellcolor{purple!10}\textbf{2,137.28}   & \cellcolor{purple!10}{\textbf{294.26} }    \\ 
		& \cellcolor{purple!10}{\textbf{$\ours$}} & \cellcolor{purple!10}\ding{51}  & \cellcolor{purple!10}\textbf{82.08}   & \cellcolor{purple!10}\textbf{1,705.19} & \cellcolor{purple!10}{\textbf{379.96} }     \\ 
		
		
		
		\midrule
		
		{\multirow{4}{*}{70\%}} & {SVD-LLM}   & \ding{55}    & 125.16          & 6,139.78        & {677.38}                \\ 
		& {SVD-LLM}   & \ding{51}    & 159.53         & 2,115.44           & {848.24}            \\ 
		% \cmidrule{2-13} 
		& \cellcolor{purple!10}{\textbf{$\ours$}} & \cellcolor{purple!10}\ding{55}  & \cellcolor{purple!10}\textbf{107.90}   & \cellcolor{purple!10}\textbf{5,027.62}  & \cellcolor{purple!10}{\textbf{441.33} }  
		\\  
		& \cellcolor{purple!10}{\textbf{$\ours$}} & \cellcolor{purple!10}\ding{51}  & \cellcolor{purple!10}\textbf{118.75}   & \cellcolor{purple!10}\textbf{1,606.94}   & \cellcolor{purple!10}{\textbf{466.64} }  
		\\  
		
		
		
		
		\midrule
		
		{\multirow{4}{*}{80\%}} & {SVD-LLM}   & \ding{55}    & 372.48           & 6,268.53     & {1,688.78}                 \\ 
		& {SVD-LLM}  & \ding{51}     & 420.25           & 3,716.08 & {1,996.42}                     \\ 
		% \cmidrule{2-13} 
		& \cellcolor{purple!10}{\textbf{$\ours$}} & \cellcolor{purple!10}\ding{55}  & \cellcolor{purple!10}\textbf{206.51}   & \cellcolor{purple!10}\textbf{6,613.44}   & \cellcolor{purple!10}{\textbf{679.66} }   \\ 
		& \cellcolor{purple!10}{\textbf{$\ours$}} & \cellcolor{purple!10}\ding{51}  & \cellcolor{purple!10}\textbf{214.51}   & \cellcolor{purple!10}\textbf{2,728.78} & \cellcolor{purple!10}{\textbf{654.79} }   \\  \bottomrule
		
		
		
		
		
		% \bottomrule[0.01pt]
		% \vspace{-15mm}
	\end{tabular}
}
\vspace{-2mm}
\caption{$\ours$ with weight quantization method GPTQ.
}
\vspace{-5mm}
\label{tab:svd+quant}
\end{table}



\noindent\textbf{Effectiveness of Adaptive Compression Ratio. $\quad$} 
To validate the effectiveness of our \textbf{adaCR}, we compared the results after removing \textbf{adaCR} (\textit{i.e.}, using constant compression ratios for all layers) from $\ours$. As shown in~\cref{tab:adacr}, $\ours$ already outperforms SOTA SVD-LLM without using \textbf{adaCR}, while integrating \textbf{adaCR} can further enhance the performance across all compression ratios.





\noindent\textbf{Minimum Retention Ratio. $\quad$} The minimum retention ratio ($mrr$) in \textbf{adaCR} is also crucial, and we investigate the impact of different $mrr$ values in~\cref{tab:mrr} for 40\%, 50\%, and 60\% compression ratios (70\% and 80\% in supplementary file). It can be observed that $mrr$ remains relatively robust at lower compression ratios (40\% and 50\%), while contributing more at higher compression ratios (60\%). 




%%%%%%%%%%%%%%%%%%%%%%%%%%%%



% \vspace{-1mm}
\subsection{Integrate with Weight Quantization}
% \vspace{-1mm}
Similar to previous SVD-based compression methods~\cite{hsu2022fwsvd,yuan2024asvd,wang2024svdllm}, our $\ours$ is orthogonal to other types of compression techniques. Following~\citet{wang2024svdllm}, we integrate $\ours$ with the widely used weight quantization method GPTQ~\cite{frantar2022gptq}. As shown in~\cref{tab:svd+quant}, we compare $\ours$ with SVD-LLM~\cite{wang2024svdllm} on the LLaMA2-7B model, using different compression ratios (40\%, 50\%, 60\%, 70\%, and 80\%) across the WikiText-2, PTB, and C4 datasets. The results demonstrate that, when combined with the 4-bit weight quantization method GPTQ, $\ours$ also consistently outperforms SOTA SVD-LLM across all compression ratios. Under high compression ratios (\ie, 60\%, 70\%, and 80\%), $\ours$ + GPTQ-INT4 even surpasses SVD-LLM.

\vspace{-2mm}
\section{Conclusion}
\vspace{-1.5mm}
In this work, we propose $\ours$, an adaptive SVD-based compression method for LLMs. $\ours$ 
first proposes \textbf{adaComp}, which adaptively compensates for the error caused by the truncation of singular matrices, efficiently reducing compression error without requiring additional training. Furthermore, $\ours$ proposes \textbf{adaCR}, which adaptively assigns compression ratios based on the importance of each layer, further enhancing performance while maintaining the same target compression rate. Both strategies effectively minimize SVD compression errors, particularly at high compression ratios. Our experiments on multiple open-source LLM and VLM families demonstrate that $\ours$ pushes the performance boundary beyond the current state-of-the-art SVD-based LLM compression methods. 






{
\small
\bibliographystyle{ieeenat_fullname}
\bibliography{main}
}




% WARNING: do not forget to delete the supplementary pages from your submission 
% 
\clearpage
% \setcounter{page}{1}
% \maketitlesupplementary
\begin{center}
Supplementary Material
\end{center}

% {
%     \onecolumn
%     \centering
%     \Large
%     \textbf{\thetitle}\\
%     \vspace{0.5em}Supplementary Material \\
%     \vspace{1.0em}
% }

\section{Proof of \cref{theorem:dr}}
We require some additional regularity assumptions:
\begin{assumption} 1) The number of classes $C$ is bounded w.r.t the number of samples $N$, 2) the missingness mechanism $P(A=1|Y,\theta)$, as well as its estimated counterpart $P(A=1|Y,\theta)$, are bounded below by some constant $\epsilon > 0$, 3) the quantities $P(Y|X,\theta)$ and $P(A|Y,\theta)$ are estimated using auxiliary samples independent of samples used for the sample averaging.
\label{assumption:extra}
\end{assumption}
Assumptions 1 and 2 are natural. For the missingness mechanism, the ground truth being bounded means that there is a non-vanishing proportion of samples for every class. The boundedness of the estimate can be enforced by clipping the estimate. Assumption 3 is called sample splitting in \cite{kennedy-dr}.

For convenience we use operator $\E_N$ to denote the average of $N$ samples i.e. $\frac{1}{N}\sum_{i=1}^N$. Note that this is by itself a random variable, in contrast to $\E$ which is a fixed number.

\begin{proof}[Proof of \cref{theorem:dr}] Because $C$ is bounded (assumption \ref{assumption:extra}), we can fix a class $c$ and prove the theorem.
Let us define the influence function $\phi$, parameterized by $\theta$, as
\begin{equation}
\phi(O | \theta)(c) = P(Y=c|X,\theta) + \frac{\one(A=1)}{P(A=1|Y,\theta)} (\one(Y=c) - P(Y=c|X,\theta)) - P(Y=c)
\end{equation}
As we have done in the main text, we use $\phi(O)$ to denote the same function but all estimated quantities are replaced with their truths. In other words, we use $\phi(O)$ for $\phi(O|\theta_0)$ where $\theta_0$ is the truth, given that our model contains $\theta_0$ e.g. when the model is consistent.

Recall that:
\begin{equation}
\begin{aligned}
\Psi_{dr}(\theta)(c) &= \frac{1}{N}\sum_{i=1}^N \left\{P(Y=c|X,\theta) + \frac{\one(A=1)}{P(A=1|Y,\theta)} (\one(Y=c) - P(Y=c|X,\theta))\right\}\\
&= \E_N [\phi(O|\theta)(c)] + P(Y=c)
\end{aligned}
\end{equation}

We will show that:
\begin{equation}
\Psi_{dr}(\theta)(c) - P(Y=c) = (\E_N - \E)[\phi(O)(c)] + o_P(N^{-1/2})
\label{eq:proof-linearity}
\end{equation}
To do that, we use the following decomposition
\begin{equation}
\begin{aligned}
\Psi_{dr}(\theta)(c) - P(Y=c) &= \E_N [\phi(O|\theta)(c)] \\
&= (\E_N - \E)[\phi(O)(c)] + (\E_N - \E)[\phi(O|\theta)(c) - \phi(O)(c)] + \E[\phi(O|\theta)(c)]
% &+ (\E_n - \E)[\phi(O;\theta) - \phi(O)]\\
% &+ \E[P(Y=c|X,\theta)] - \E[P(Y=c|X)] + \E[\phi(O,\theta)]
\end{aligned}
\end{equation}
and analyze the second and third term. The third term is:
\begin{equation}
\begin{aligned}
\E[\phi(O|\theta)(c)] &= \E[P(Y=c|X,\theta)] + \E\left[\frac{\one(A=1)}{P(A=1|Y,\theta)}(\one(Y=c) - P(Y=c|X,\theta))\right]- P(Y=c) \\
&= \E\left[P(Y=c|X,\theta) + \frac{P(A=1|Y)}{P(A=1|Y,\theta)}(P(Y=c|X) - P(Y=c|X,\theta))\right] - \E[P(Y=c|X)]\\
&= \E\left[(P(Y=c|X,\theta) - P(Y=c|X)) (P(A=1|Y,\theta) -P(A=1|Y)) \frac{1}{P(A=1|Y,\theta)}\right]\\
\end{aligned}
\end{equation}
by Cauchy-Schwarz inequality:
\begin{equation}
\begin{aligned}
\E[\phi(O|\theta)(c)] &\le \frac{1}{\epsilon} \|P(A=1|Y,\theta) - P(A=1|Y)\|_2 \|P(Y=c|X,\theta) - P(Y=c|X)\|_{L_2(P)}\\
&= \frac{1}{\epsilon} o_P(N^{-1/4} N^{-1/4}) = o_P(N^{-1/2})
\end{aligned}
\end{equation}
by assumption \ref{assumption:4th-root-n} and that $P(A=1|Y,\theta) > \epsilon$ (assumption \ref{assumption:extra}). The second term can be bounded by Chebyshev inequality
% \begin{equation}
% \begin{aligned}
% \E[\E_N[\phi(O|\theta)(c) - \phi(O)(c)]] &= \E[\phi(O|\theta)(c) - \phi(O)(c)]\\
% \var[\E_N[\phi(O|\theta)(c) - \phi(O)(c)]] &= \frac{1}{N}\var[\phi(O|\theta)(c) - \phi(O)(c)] \le 
% \end{aligned}
% \end{equation}
\begin{equation}
P(|(\E_N - \E)[\phi(O|\theta)(c) - \phi(O)(c)]| \ge t) \le \frac{\var[\E_N[\phi(O|\theta)(c) - \phi(O)(c)]]}{t^2} = \frac{\var[\phi(O|\theta)(c) - \phi(O)(c)]}{Nt^2}
\end{equation}
note here that $\theta$ is independent of the samples used for $\E_N$ by assumption \ref{assumption:extra}. For any $\varepsilon > 0$, by picking $t = \frac{1}{\sqrt{N\varepsilon}}$ we get
\begin{equation}
P\left(\left|\frac{(\E_N - \E)[\phi(O|\theta)(c) - \phi(O)(c)]}{N^{-1/2}}\right| \ge \frac{1}{\sqrt{\varepsilon}}\right) \le \varepsilon \var[\phi(O|\theta)(c) - \phi(O)(c)]
\end{equation}
by the definition of $O_P$, we then get
\begin{equation}
(\E_N - \E)[\phi(O|\theta)(c) - \phi(O)(c)] = O_P(N^{-1/2}\var[\phi(O|\theta)(c) - \phi(O)(c)])
\end{equation}
Because $\phi$ is a continuous function of $P(Y|X,\theta)$ and $P(A|Y,\theta)$ (given $P(A|Y,\theta) > \epsilon$, assumption \ref{assumption:extra}), by the continuous mapping theorem and the fact that $P(Y|X,\theta)$ and $P(A|Y,\theta)$ are convergent in probability (assumption \ref{assumption:4th-root-n}), we get $\var[\phi(O|\theta)(c) - \phi(O)(c)] = o_P(1)$. This gives
\begin{equation}
(\E_N - \E)[\phi(O|\theta)(c) - \phi(O)(c)] = o_P(N^{-1/2})
\end{equation}
Therefore, we have shown that the second and third term are both $o_P(N^{-1/2})$, proving \cref{eq:proof-linearity}. As the final step, multiply both sides of this equation by $\sqrt{N}$ we get:
\begin{equation}
\sqrt{N}(\Psi_{dr}(\theta)(c) - P(Y=c)) = \sqrt{N} (\E_N - \E)[\phi(O)(c)] + o_P(1) \rightsquigarrow \mathcal{N}(0, \var[\phi(O)(c)])
\end{equation}
by the central limit theorem, and $\var[\phi(O)(c)] = \E[\phi(O)(c)^2]$ because $\E[\phi(O)(c)] = 0$.
\end{proof}

While we started with the definition of $\phi$, \cref{eq:proof-linearity} shows that $\phi$ is indeed an influence function. Now we show that $\phi$ is also the efficient influence function, by using the characterization of the model's tangent space \cite{tsiatis-missingdata}. Note that the joint probability factorizes as $P(X,A,Y) = P(X)P(Y|X)P(A|Y)$, therefore the tangent space $\mathcal{T}$ factorizes as $\mathcal{T} = \mathcal{T}_{X} \oplus \mathcal{T}_{Y|X} \oplus \mathcal{T}_{A|Y}$ where $\mathcal{T}_X = \{h(X): \E[h] = 0\}$, $\mathcal{T}_{Y|X} = \{h(X,Y): \E[h|X] = 0\}$, $\mathcal{T}_{A|Y} = \{h(A,Y): \E[h|Y] = 0\}$, and the 3 subspaces are pairwise orthogonal. All influence functions are orthogonal to the tangent space, but the influence function that is also in the tangent space has the smallest variance and is called the efficient influence function. As $\phi$ is already an influence function, we need only show that $\phi$ is in $\mathcal{T}$. We write $\phi$ as
\begin{equation}
\phi(O)(c) = (P(Y=c|X) - P(Y=c)) + \left[\frac{\one(A=1)}{P(A=1|Y)} - 1\right](\one(Y=c) - P(Y=c|X)) + (\one(Y=c) - P(Y=c|X))
\end{equation}
and note that the first, second and third term are in $\mathcal{T}_X$, $\mathcal{T}_{A|Y}$ and $\mathcal{T}_{Y|X}$ respectively. Therefore, $\phi$ is indeed in $\mathcal{T}$. The efficient influence function has the smallest variance of all influence function, and therefore our estimator being asymptotically linear in $\phi$ (\cref{eq:proof-linearity}) has the smallest mean squared error in a local asymptotic minimax sense \cite{kennedy-dr, asymptoticstatistics}

\section{Further background and related work}
\paragraph{Discussion on semi-supervised EM.}
It appears that semi-supervised EM was first used for parameter estimation when the missingness mechanism is non-ignorable in \cite{ibrahim1996parameter}, but has not been used for label shift estimation.
Perhaps this is because the semi-supervised situation where additional unlabeled data is available during training is rarer than the test-time adaptation case. EM is well suited to take advantage of the extra unlabeled data to improve the classifier under very scarce and long-tailed labeled data. While the connection between pseudo-labeling and EM has been explored before \cite{entropyminimization}, the situation with label shift has not until recently \cite{simpro}. Here the application of EM is much more interesting, because other than simply giving pseudo-labeling a rigorous formulation, EM also estimates the missingness mechanism (equivalently the label distribution shift), which is important for shift correction and thus high-quality pseudo-labels \cite{acr}. The application of confidence thresholding can be seen as a sparse variant of EM \cite{neal1998view}.

\paragraph{The doubly-robust risk.} 
\label{subsec:dr-risk}
A technique that also derives from the theory of semi-parametric efficiency is orthogonal statistical learning \citep{foster2023orthogonal}. The idea is to minimize the doubly-robust risk:
\label{subsec:method-dr-risk}
\begin{equation}
\label{eq:dr-risk}
\mathcal{R}(\theta_2) = \frac{1}{N} \sum_{i=1}^N \Bigg[ l(x_i, \hat y_i|\theta_2) + \frac{\one(a_i=1)}{P(A=a_i|Y=y_i, \theta_1)} (l(x_i, y_i | \theta_2) - l(x_i, \hat y_i | \theta_2))\Bigg]
\end{equation}
where $l(x,y|\theta) = -\sum_{c=1}^C [y]_c \log P(Y=c|X=x,\theta)$ is the negative cross-entropy. 
The notation $[y]_c$ means that we are using the $c$-entry in a C-dimension probability vector $y$. 
Thus, $y_i$ denotes the one-hot label of observation $i$, while $\hat y_i$ denotes the pseudo-label, which can be one-hot or all-zero. 
Finally, we use $\theta_1$ to denote that $P(a|y,\theta_1)$ is an estimation from a previous stage, but it can be estimated with $\theta_2$ as well. 
The risk $\mathcal{R}(\theta_2)$ can be used as a training loss in a straightforward fashion. 
Similar to the doubly robust estimation of $P(Y)$, the doubly robust risk provides approximately unbiased estimation of the risk. 
This property has been used in \citep{arelabelsinformative, onnonrandommissinglabels, drst} also in the semi-supervised learning setting.
More broadly, it is at the heart of one of the core techniques in heterogenous treatment effect estimation in causal estimation \cite{kennedy2023towards, foster2023orthogonal, wager2018estimation}. 
The focus here is not the estimation of $\mathcal{R}(\theta_2)$ per se, but the quality of the learned model \cite{foster2023orthogonal}.
By using the doubly-robust risk, we can achieve an optimality result similar in spirit to our theorem \cref{theorem:dr}, but for the generalization error.
While this is appealing, in practice there are 2 problems with this approach. First, the inverse probability weight $P(A=a_i|Y=y_i,\theta_1)$ can be very large if the class ratio is highly unlabeled, making training unstable \cite{kallus2020deepmatch, pham2023stable}. 
This problem exists for our estimation as well. However, it is much easier to control for estimation than for training because of the iterative nature of model update. Secondly, we can further write $\mathcal{R}$ as:
\begin{equation}
\mathcal{R}(\theta_2) = \frac{1}{N}\sum_{i=1}^N l\left(x_i, \hat y_i + \frac{\one(a_i=1)}{P(A=a_i|Y=y_i,\theta_1)} (y_i - \hat y_i)\Bigg\vert\theta_2\right)
\end{equation}
which is a cross-entropy loss with new meta-pseudo-labels. However, these labels are not meant to be learned exactly, and furthermore they can be negative. Thus, theoretical works have to put stringent assumptions on the models. In \cref{subsec:ablation-1}, we show that experimentally that the instability problem makes doubly-robust risk performance worse than our 2-stage approach.

\section{Training and hyperparameter settings.}
\label{subsec:training-setting}
For neural network training, we follow the implementation and hyperparameter settings of \cite{simpro}. In particular, we adapt the core code of SimPro for Supervised, MLE and EM. For MLE, we update $P(A|Y)$ using the Adam optimizer with learning rate 1e-3, while for EM we use a momentum update similar to SimPro's update of $P(Y|A)$ because it has a a closed-form solution at each mini-batch. We use Wide ResNet-28-2 on all methods and all datasets in this section, including Imagenet-127, because we are motivated by the fact that stage-1's goal is not classification accuracy but the estimation of a finite-dimensional parameter. When using Wide ResNet-28-2 for Imagenet-127, we use the hyperparameters of CIFAR-100, except we lower the batch size of unlabeled data to 2 times that of labeled data instead of 8 for memory reason. We do not perform additional hyperparameter tuning. All experiments can be performed on 1 A6000 RTX GPU, and are run 3 times. We report the total variation distance between the estimated and the ground truth unlabeled class distribution, similar to its usage in Theorem 3.1 of \cite{lsc}, and the top-1 classification accuracy.

In the second stage of our algorithm, we freeze our estimation and plug it in SimPro and BOAT.
We keep exactly the same hyperparameter settings that SimPro and BOAT use. In particular, for Imagenet-127, we now use ResNet-50 and run each experiment once.
In SimPro, we set the unlabeled class distribution $P(Y|A=0)$ at the E-step;  however, we still keep a running estimate of the class distribution $P(Y)$ in the logit adjustment loss \cref{eq:simpro-la-loss}. While it is possible to use the first stage estimate in the logit adjustment loss, we observe that doing so results in lower accuracy than using the the running average. This is conceptually consistent with the role of the running average - serving not as an accurate estimate of $P(Y)$ but to make the classifier's class distribution uniform through the logit adjustment loss, which is good for the test set. Similarly, in BOAT, we only replace $\Delta_c = \log P(Y|A=1) - \log P(Y|A=0)$ in equation (4) of \cite{boat}, which is adjusting a classifier's predictions from the labeled to the unlabeled class distribution, with our SimPro + DR estimate instead of their on-the-fly estimate. 


% \section{Additional experiments}
% % \begin{table*}[t]
\centering
\caption{Total Variation Distance on CIFAR-10-LT ($N_l = 500$, $M_l = 4000$) with different class imbalance ratios $\gamma_l$ and $\gamma_u$ under five different unlabeled class distributions.}
\label{tab:cifar10-tv}
\resizebox{\textwidth}{!}{
\begin{tabular}{lccccccccccc}
\toprule
& & \multicolumn{2}{c}{consistent} & \multicolumn{2}{c}{uniform} & \multicolumn{2}{c}{reversed} & \multicolumn{2}{c}{middle} & \multicolumn{2}{c}{head-tail} \\
\cmidrule(lr){3-4} \cmidrule(lr){5-6} \cmidrule(lr){7-8} \cmidrule(lr){9-10} \cmidrule(lr){11-12}
& & $\gamma_l = 150$ & $\gamma_l = 100$ & $\gamma_l = 150$ & $\gamma_l = 100$ & $\gamma_l = 150$ & $\gamma_l = 100$ & $\gamma_l = 150$ & $\gamma_l = 100$ & $\gamma_l = 150$ & $\gamma_l = 100$ \\
Model & Estimator & $\gamma_u = 150$ & $\gamma_u = 100$ & $\gamma_u = 1$ & $\gamma_u = 1$ & $\gamma_u = 1/150$ & $\gamma_u = 1/100$ & $\gamma_u = 150$ & $\gamma_u = 100$ & $\gamma_u = 150$ & $\gamma_u = 100$ \\
\midrule
Supervised & MLLS & 0.269 ± 0.252 & 0.038 ± 0.006 & 0.251 ± 0.046 & 0.255 ± 0.060 & 0.429 ± 0.028 & 0.493 ± 0.050 & 0.333 ± 0.042 & 0.320 ± 0.009 & 0.457 ± 0.034 & 0.444 ± 0.043 \\
Supervised & RLLS & 0.043 ± 0.001 & 0.044 ± 0.010 & 0.348 ± 0.034 & 0.305 ± 0.068 & 0.769 ± 0.016 & 0.678 ± 0.028 & 0.430 ± 0.008 & 0.368 ± 0.013 & 0.539 ± 0.018 & 0.503 ± 0.020 \\
\midrule
MLE & IPW & 0.027 ± 0.001 & 0.027 ± 0.000 & 0.319 ± 0.072 & 0.243 ± 0.010 & 0.674 ± 0.020 & 0.646 ± 0.041 & 0.438 ± 0.020 & 0.454 ± 0.026 & 0.547 ± 0.049 & 0.491 ± 0.059 \\
MLE & OR & 0.045 ± 0.004 & 0.042 ± 0.000 & 0.215 ± 0.026 & 0.203 ± 0.032 & 0.433 ± 0.017 & 0.395 ± 0.033 & 0.193 ± 0.006 & 0.209 ± 0.037 & 0.307 ± 0.147 & 0.249 ± 0.130 \\
MLE & DR & 0.090 ± 0.002 & 0.079 ± 0.000 & 0.407 ± 0.027 & 0.360 ± 0.007 & 0.425 ± 0.007 & 0.421 ± 0.029 & 0.256 ± 0.001 & 0.286 ± 0.031 & 0.435 ± 0.136 & 0.362 ± 0.122 \\
\midrule
EM & IPW & 0.035 ± 0.002 & 0.040 ± 0.001 & 0.021 ± 0.001 & 0.029 ± 0.015 & 0.303 ± 0.187 & 0.091 ± 0.010 & 0.119 ± 0.011 & 0.105 ± 0.022 & 0.104 ± 0.026 & 0.104 ± 0.051 \\
EM & OR & 0.037 ± 0.003 & 0.042 ± 0.002 & 0.016 ± 0.001 & 0.024 ± 0.012 & 0.269 ± 0.183 & 0.090 ± 0.008 & 0.122 ± 0.012 & 0.103 ± 0.022 & 0.072 ± 0.012 & 0.073 ± 0.024 \\
EM & DR & 0.034 ± 0.004 & 0.037 ± 0.001 & 0.014 ± 0.001 & 0.027 ± 0.020 & 0.264 ± 0.191 & 0.092 ± 0.005 & 0.111 ± 0.019 & 0.097 ± 0.026 & 0.077 ± 0.016 & 0.073 ± 0.028 \\
\midrule
SimPro & IPW & 0.070 ± 0.011 & 0.058 ± 0.000 & 0.046 ± 0.001 & 0.049 ± 0.005 & 0.254 ± 0.074 & 0.223 ± 0.098 & 0.097 ± 0.025 & 0.067 ± 0.002 & 0.105 ± 0.066 & 0.110 ± 0.079 \\
SimPro & OR & 0.071 ± 0.012 & 0.058 ± 0.000 & 0.045 ± 0.001 & 0.049 ± 0.006 & 0.040 ± 0.003 & 0.059 ± 0.017 & 0.074 ± 0.006 & 0.075 ± 0.002 & 0.033 ± 0.003 & 0.033 ± 0.003 \\
SimPro & DR & 0.017 ± 0.004 & 0.026 ± 0.001 & 0.019 ± 0.002 & 0.018 ± 0.003 & 0.039 ± 0.003 & 0.058 ± 0.025 & 0.091 ± 0.007 & 0.031 ± 0.001 & 0.015 ± 0.003 & 0.019 ± 0.007 \\
\bottomrule
\end{tabular}
}
\end{table*}
% 

\begin{table*}[t]
\centering
\caption{Total Variation Distance on CIFAR-100-LT ($N_l = 50$, $M_l = 400$) with different class imbalance ratios $\gamma_l$ and $\gamma_u$ under five different unlabeled class distributions.}
\label{tab:cifar100-tv}
\resizebox{\textwidth}{!}{
\begin{tabular}{lccccccccccc}
\toprule
& & \multicolumn{2}{c}{consistent} & \multicolumn{2}{c}{uniform} & \multicolumn{2}{c}{reversed} & \multicolumn{2}{c}{middle} & \multicolumn{2}{c}{head-tail} \\
\cmidrule(lr){3-4} \cmidrule(lr){5-6} \cmidrule(lr){7-8} \cmidrule(lr){9-10} \cmidrule(lr){11-12}
& & $\gamma_l = 20$ & $\gamma_l = 10$ & $\gamma_l = 20$ & $\gamma_l = 10$ & $\gamma_l = 20$ & $\gamma_l = 10$ & $\gamma_l = 20$ & $\gamma_l = 10$ & $\gamma_l = 20$ & $\gamma_l = 10$ \\
Model & Estimator & $\gamma_u = 20$ & $\gamma_u = 10$ & $\gamma_u = 1$ & $\gamma_u = 1$ & $\gamma_u = 1/20$ & $\gamma_u = 1/10$ & $\gamma_u = 20$ & $\gamma_u = 10$ & $\gamma_u = 20$ & $\gamma_u = 10$ \\
\midrule
Supervised & MLLS & 0.707 ± 0.016 & 0.313 ± 0.100 & 0.445 ± 0.172 & 0.309 ± 0.119 & 0.383 ± 0.075 & 0.397 ± 0.006 & 0.570 ± 0.001 & 0.373 ± 0.107 & 0.543 ± 0.009 & 0.231 ± 0.057 \\
Supervised & RLLS & 0.520 ± 0.007 & 0.133 ± 0.003 & 0.337 ± 0.125 & 0.253 ± 0.082 & 0.424 ± 0.060 & 0.463 ± 0.003 & 0.454 ± 0.021 & 0.306 ± 0.074 & 0.460 ± 0.028 & 0.241 ± 0.040 \\
\midrule
MLE & IPW & 0.075 ± 0.000 & 0.071 ± 0.001 & 0.229 ± 0.001 & 0.167 ± 0.002 & 0.565 ± 0.005 & 0.443 ± 0.007 & 0.415 ± 0.000 & 0.311 ± 0.005 & 0.343 ± 0.000 & 0.280 ± 0.001 \\
MLE & OR & 0.065 ± 0.002 & 0.061 ± 0.001 & 0.200 ± 0.007 & 0.143 ± 0.001 & 0.526 ± 0.011 & 0.399 ± 0.023 & 0.360 ± 0.003 & 0.256 ± 0.012 & 0.328 ± 0.003 & 0.266 ± 0.005 \\
MLE & DR & 0.149 ± 0.019 & 0.145 ± 0.010 & 0.243 ± 0.004 & 0.214 ± 0.019 & 0.568 ± 0.005 & 0.464 ± 0.014 & 0.403 ± 0.014 & 0.309 ± 0.012 & 0.365 ± 0.007 & 0.320 ± 0.004 \\
\midrule
EM & IPW & 0.097 ± 0.008 & 0.092 ± 0.004 & 0.239 ± 0.007 & 0.179 ± 0.003 & 0.478 ± 0.012 & 0.329 ± 0.020 & 0.262 ± 0.016 & 0.202 ± 0.003 & 0.312 ± 0.002 & 0.227 ± 0.001 \\
EM & OR & 0.121 ± 0.007 & 0.108 ± 0.005 & 0.261 ± 0.007 & 0.189 ± 0.004 & 0.489 ± 0.013 & 0.335 ± 0.020 & 0.274 ± 0.016 & 0.211 ± 0.004 & 0.336 ± 0.003 & 0.235 ± 0.001 \\
EM & DR & 0.125 ± 0.005 & 0.111 ± 0.004 & 0.269 ± 0.007 & 0.194 ± 0.005 & 0.497 ± 0.010 & 0.336 ± 0.024 & 0.281 ± 0.019 & 0.219 ± 0.008 & 0.336 ± 0.007 & 0.233 ± 0.004 \\
\midrule
SimPro & IPW & 0.125 ± 0.001 & 0.100 ± 0.005 & 0.166 ± 0.007 & 0.141 ± 0.009 & 0.353 ± 0.023 & 0.261 ± 0.008 & 0.202 ± 0.003 & 0.158 ± 0.005 & 0.277 ± 0.009 & 0.197 ± 0.003 \\
SimPro & OR & 0.133 ± 0.005 & 0.100 ± 0.004 & 0.160 ± 0.007 & 0.138 ± 0.010 & 0.322 ± 0.014 & 0.253 ± 0.008 & 0.202 ± 0.003 & 0.156 ± 0.005 & 0.269 ± 0.006 & 0.191 ± 0.004 \\
SimPro & DR & 0.122 ± 0.003 & 0.106 ± 0.006 & 0.188 ± 0.009 & 0.149 ± 0.006 & 0.343 ± 0.023 & 0.257 ± 0.007 & 0.219 ± 0.010 & 0.172 ± 0.002 & 0.279 ± 0.007 & 0.198 ± 0.004 \\
\bottomrule
\end{tabular}
}
\end{table*}

\end{document}
