% ICCV 2025 Paper Template; see https://github.com/cvpr-org/author-kit

\documentclass[10pt,twocolumn,letterpaper]{article}

%%%%%%%%% PAPER TYPE  - PLEASE UPDATE FOR FINAL VERSION
% \usepackage{iccv}              % To produce the CAMERA-READY version
% \usepackage[review]{iccv}      % To produce the REVIEW version
\usepackage[pagenumbers]{iccv} % To force page numbers, e.g. for an arXiv version

% Import additional packages in the preamble file, before hyperref
%
% --- inline annotations
%
\newcommand{\red}[1]{{\color{red}#1}}
\newcommand{\todo}[1]{{\color{red}#1}}
\newcommand{\TODO}[1]{\textbf{\color{red}[TODO: #1]}}
% --- disable by uncommenting  
% \renewcommand{\TODO}[1]{}
% \renewcommand{\todo}[1]{#1}



\newcommand{\VLM}{LVLM\xspace} 
\newcommand{\ours}{PeKit\xspace}
\newcommand{\yollava}{Yo’LLaVA\xspace}

\newcommand{\thisismy}{This-Is-My-Img\xspace}
\newcommand{\myparagraph}[1]{\noindent\textbf{#1}}
\newcommand{\vdoro}[1]{{\color[rgb]{0.4, 0.18, 0.78} {[V] #1}}}
% --- disable by uncommenting  
% \renewcommand{\TODO}[1]{}
% \renewcommand{\todo}[1]{#1}
\usepackage{slashbox}
% Vectors
\newcommand{\bB}{\mathcal{B}}
\newcommand{\bw}{\mathbf{w}}
\newcommand{\bs}{\mathbf{s}}
\newcommand{\bo}{\mathbf{o}}
\newcommand{\bn}{\mathbf{n}}
\newcommand{\bc}{\mathbf{c}}
\newcommand{\bp}{\mathbf{p}}
\newcommand{\bS}{\mathbf{S}}
\newcommand{\bk}{\mathbf{k}}
\newcommand{\bmu}{\boldsymbol{\mu}}
\newcommand{\bx}{\mathbf{x}}
\newcommand{\bg}{\mathbf{g}}
\newcommand{\be}{\mathbf{e}}
\newcommand{\bX}{\mathbf{X}}
\newcommand{\by}{\mathbf{y}}
\newcommand{\bv}{\mathbf{v}}
\newcommand{\bz}{\mathbf{z}}
\newcommand{\bq}{\mathbf{q}}
\newcommand{\bff}{\mathbf{f}}
\newcommand{\bu}{\mathbf{u}}
\newcommand{\bh}{\mathbf{h}}
\newcommand{\bb}{\mathbf{b}}

\newcommand{\rone}{\textcolor{green}{R1}}
\newcommand{\rtwo}{\textcolor{orange}{R2}}
\newcommand{\rthree}{\textcolor{red}{R3}}
\usepackage{amsmath}
%\usepackage{arydshln}
\DeclareMathOperator{\similarity}{sim}
\DeclareMathOperator{\AvgPool}{AvgPool}

\newcommand{\argmax}{\mathop{\mathrm{argmax}}}     



% It is strongly recommended to use hyperref, especially for the review version.
% hyperref with option pagebackref eases the reviewers' job.
% Please disable hyperref *only* if you encounter grave issues, 
% e.g. with the file validation for the camera-ready version.
%
% If you comment hyperref and then uncomment it, you should delete *.aux before re-running LaTeX.
% (Or just hit 'q' on the first LaTeX run, let it finish, and you should be clear).
\definecolor{iccvblue}{rgb}{0.21,0.49,0.74}
\usepackage[pagebackref,breaklinks,colorlinks,allcolors=iccvblue]{hyperref}

%%%%%%%%% PAPER ID  - PLEASE UPDATE
\def\paperID{1059} % *** Enter the Paper ID here
\def\confName{ICCV}
\def\confYear{2025}

\usepackage{algorithm}
\usepackage{algpseudocode}
\usepackage{multirow}
% \usepackage{subfig}
\usepackage{booktabs}
\usepackage{amsmath}
\usepackage{amssymb}
\usepackage{mathtools}
\usepackage{amsthm}

% Algorithms packages
% \usepackage{algorithm}
% \usepackage{algorithmicx} % This is for better control over algorithms
\usepackage{algpseudocode}

% Graphics and Figures packages
\usepackage{graphicx}
% \usepackage{subfig}
\usepackage{booktabs}  % for professional tables
\usepackage{arydshln}  % for dashed lines in tables

% Hyperlinks and referencing packages
\usepackage[capitalize,noabbrev]{cleveref} % Cleveref must come after hyperref

% Miscellaneous useful packages
\usepackage{microtype}  % for better typography
\usepackage{bbm}        % for blackboard bold math symbols
\usepackage{colortbl}   % for coloring rows in tables
\usepackage{pifont}     % for special symbols
\definecolor{color3}{gray}{0.95}
\definecolor{color4}{HTML}{00b050}

\providecommand{\yulun}[1]{\textcolor{red}{[{\bf #1}]}}

\renewcommand{\ttdefault}{lmtt}
\newcommand{\ours}{\texttt{AdaSVD}}

%%%%%%%%% TITLE - PLEASE UPDATE
\title{$\ours$: Adaptive Singular Value Decomposition for Large Language Models}

%%%%%%%%% AUTHORS - PLEASE UPDATE
\author{
	Zhiteng Li$^{1}$\thanks{Equal contribution}~,\enspace
	Mingyuan Xia$^{1}$\footnotemark[1]~,\enspace
	Jingyuan Zhang$^{1}$,\enspace \\
	Zheng Hui$^{2}$,\enspace
	Linghe Kong$^{1}$\footnotemark[2]~,\enspace
	Yulun Zhang$^{1}$\thanks{Corresponding authors: Linghe Kong,  linghe.kong@sjtu.edu.cn, Yulun Zhang, yulun100@gmail.com}~, \enspace
	Xiaokang Yang$^{1}$\\
	\textsuperscript{1}Shanghai Jiao Tong University,\enspace
	\textsuperscript{2}MGTV, Shanhai Academy\\
	\vspace{-8mm}
}

\begin{document}
	\maketitle
	
	%% Don't use 'sec' 
	% \begin{abstract}


The choice of representation for geographic location significantly impacts the accuracy of models for a broad range of geospatial tasks, including fine-grained species classification, population density estimation, and biome classification. Recent works like SatCLIP and GeoCLIP learn such representations by contrastively aligning geolocation with co-located images. While these methods work exceptionally well, in this paper, we posit that the current training strategies fail to fully capture the important visual features. We provide an information theoretic perspective on why the resulting embeddings from these methods discard crucial visual information that is important for many downstream tasks. To solve this problem, we propose a novel retrieval-augmented strategy called RANGE. We build our method on the intuition that the visual features of a location can be estimated by combining the visual features from multiple similar-looking locations. We evaluate our method across a wide variety of tasks. Our results show that RANGE outperforms the existing state-of-the-art models with significant margins in most tasks. We show gains of up to 13.1\% on classification tasks and 0.145 $R^2$ on regression tasks. All our code and models will be made available at: \href{https://github.com/mvrl/RANGE}{https://github.com/mvrl/RANGE}.

\end{abstract}

    
	% \section{Introduction}
Backdoor attacks pose a concealed yet profound security risk to machine learning (ML) models, for which the adversaries can inject a stealth backdoor into the model during training, enabling them to illicitly control the model's output upon encountering predefined inputs. These attacks can even occur without the knowledge of developers or end-users, thereby undermining the trust in ML systems. As ML becomes more deeply embedded in critical sectors like finance, healthcare, and autonomous driving \citep{he2016deep, liu2020computing, tournier2019mrtrix3, adjabi2020past}, the potential damage from backdoor attacks grows, underscoring the emergency for developing robust defense mechanisms against backdoor attacks.

To address the threat of backdoor attacks, researchers have developed a variety of strategies \cite{liu2018fine,wu2021adversarial,wang2019neural,zeng2022adversarial,zhu2023neural,Zhu_2023_ICCV, wei2024shared,wei2024d3}, aimed at purifying backdoors within victim models. These methods are designed to integrate with current deployment workflows seamlessly and have demonstrated significant success in mitigating the effects of backdoor triggers \cite{wubackdoorbench, wu2023defenses, wu2024backdoorbench,dunnett2024countering}.  However, most state-of-the-art (SOTA) backdoor purification methods operate under the assumption that a small clean dataset, often referred to as \textbf{auxiliary dataset}, is available for purification. Such an assumption poses practical challenges, especially in scenarios where data is scarce. To tackle this challenge, efforts have been made to reduce the size of the required auxiliary dataset~\cite{chai2022oneshot,li2023reconstructive, Zhu_2023_ICCV} and even explore dataset-free purification techniques~\cite{zheng2022data,hong2023revisiting,lin2024fusing}. Although these approaches offer some improvements, recent evaluations \cite{dunnett2024countering, wu2024backdoorbench} continue to highlight the importance of sufficient auxiliary data for achieving robust defenses against backdoor attacks.

While significant progress has been made in reducing the size of auxiliary datasets, an equally critical yet underexplored question remains: \emph{how does the nature of the auxiliary dataset affect purification effectiveness?} In  real-world  applications, auxiliary datasets can vary widely, encompassing in-distribution data, synthetic data, or external data from different sources. Understanding how each type of auxiliary dataset influences the purification effectiveness is vital for selecting or constructing the most suitable auxiliary dataset and the corresponding technique. For instance, when multiple datasets are available, understanding how different datasets contribute to purification can guide defenders in selecting or crafting the most appropriate dataset. Conversely, when only limited auxiliary data is accessible, knowing which purification technique works best under those constraints is critical. Therefore, there is an urgent need for a thorough investigation into the impact of auxiliary datasets on purification effectiveness to guide defenders in  enhancing the security of ML systems. 

In this paper, we systematically investigate the critical role of auxiliary datasets in backdoor purification, aiming to bridge the gap between idealized and practical purification scenarios.  Specifically, we first construct a diverse set of auxiliary datasets to emulate real-world conditions, as summarized in Table~\ref{overall}. These datasets include in-distribution data, synthetic data, and external data from other sources. Through an evaluation of SOTA backdoor purification methods across these datasets, we uncover several critical insights: \textbf{1)} In-distribution datasets, particularly those carefully filtered from the original training data of the victim model, effectively preserve the model’s utility for its intended tasks but may fall short in eliminating backdoors. \textbf{2)} Incorporating OOD datasets can help the model forget backdoors but also bring the risk of forgetting critical learned knowledge, significantly degrading its overall performance. Building on these findings, we propose Guided Input Calibration (GIC), a novel technique that enhances backdoor purification by adaptively transforming auxiliary data to better align with the victim model’s learned representations. By leveraging the victim model itself to guide this transformation, GIC optimizes the purification process, striking a balance between preserving model utility and mitigating backdoor threats. Extensive experiments demonstrate that GIC significantly improves the effectiveness of backdoor purification across diverse auxiliary datasets, providing a practical and robust defense solution.

Our main contributions are threefold:
\textbf{1) Impact analysis of auxiliary datasets:} We take the \textbf{first step}  in systematically investigating how different types of auxiliary datasets influence backdoor purification effectiveness. Our findings provide novel insights and serve as a foundation for future research on optimizing dataset selection and construction for enhanced backdoor defense.
%
\textbf{2) Compilation and evaluation of diverse auxiliary datasets:}  We have compiled and rigorously evaluated a diverse set of auxiliary datasets using SOTA purification methods, making our datasets and code publicly available to facilitate and support future research on practical backdoor defense strategies.
%
\textbf{3) Introduction of GIC:} We introduce GIC, the \textbf{first} dedicated solution designed to align auxiliary datasets with the model’s learned representations, significantly enhancing backdoor mitigation across various dataset types. Our approach sets a new benchmark for practical and effective backdoor defense.



	% \section{Related work}
\label{sec:formatting}

\subsection{Text-to-Video Generation}

T2V generation has made notable progress, evolving from early GAN-based models \cite{saito2017temporal,tulyakov2018mocogan,fu2023tell,li2018video,wu2022nuwa,yu2022generating} to newer transformer \cite{yan2021videogpt,arnab2021vivit,esser2021taming,ramesh2021zero,yu2022scaling} and diffusion models \cite{kirkpatrick2017overcoming,sohl2015deep,song2020denoising,zhang2022gddim}. Early efforts like MoCoGAN~\cite{tulyakov2018mocogan} focused on short video clips but faced issues with stability and coherence. The introduction of transformers improved sequential data handling, enhancing video generation, while diffusion models further improved video quality by progressively denoising the input. 
Despite these advances, T2V models still struggle to reflect human preferences, with the generated videos generally lacking aesthetic quality. Additionally, the scarcity of paired video preference data hinders effective model training and may lead to insufficient flexibility and poor quality in the generated videos.


\subsection{RLHF}

\iffalse
Aligning LLMs \cite{dai1901transformer,radford2019language,zhang2023opt} typically involves two steps: supervised fine-tuning followed by Reinforcement Learning with Human Feedback (RLHF) \cite{gao2023scaling,stiennon2020learning,rafailov2024direct}. Although effective, RLHF is computationally expensive and can lead to issues like reward hacking. Methods like DPO have streamlined alignment by leveraging feedback data directly, improving efficiency.

In contrast, diffusion model alignment is still evolving, focusing mainly on enhancing visual quality through curated datasets. Techniques like DOODL \cite{wallace2023end} and AlignProp \cite{prabhudesai2023aligning} target aesthetic improvements but face challenges with complex tasks such as text-image alignment. Reinforcement learning methods like DPOK \cite{fan2024reinforcement} and DDPO \cite{black2023training} improve reward optimization but struggle with scalability. DPO-SDXL integrates DPO into T2I generation, boosting both alignment and aesthetics.

However, aligning video generation remains a largely unaddressed challenge, especially when dealing with motion consistency and semantic coherence across frames.
\fi

RLHF \cite{gao2023scaling,stiennon2020learning,rafailov2024direct} is a method that utilizes human feedback to guide machine learning models. Early RLHF algorithms, such as DDPG~\cite{lillicrap2015continuous} and PPO~\cite{schulman2017proximal}, typically relied on complex reward models to quantify human feedback. These reward models require a large amount of annotated data and face challenges during tuning. As research has progressed, more efficient preference learning methods have emerged, among which DPO has become a new framework. DPO does not depend on a separate reward model; instead, it obtains human preferences through pairwise comparisons and directly optimizes these preferences. This shift not only simplifies the application of RLHF but also enhances the alignment of models with human values. Furthermore, DPO has been successfully introduced into T2I tasks~\cite{wallace2024diffusion,yang2024using}, providing new insights for generative models in addressing the alignment of human preferences and showcasing DPO's potential in the field of AIGC~\cite{shi2024instantbooth,
qing2024hierarchical,menapace2024snap,koley2024s}. However, there remains a gap in current research regarding the application of DPO strategies to T2V tasks. Effectively integrating DPO into T2V tasks presents a challenging endeavor.


	% \section{Preliminary}
\label{sec:preliminary}
In this section, we first introduce the mathematical formulation of flow-based text-to-image generative models~\cite{Xingchao_2022,Lipman_2022}, which forms the foundation of modern T2I systems~\cite{sd3,sdxl,imagen3,imagen}. We then describe classifier-free guidance~\cite{ho2022classifier}, a key technique to control the generation process through text conditioning.

\subsection{Flow-based text-to-image generative models}
In state-of-the-art T2I models~\cite{sd3}, the image generation process is modeled by learning, through a neural network, a flow $\psi$ that generates a probability path $(p_t)_{0\le t\le 1}$ bridging the source distribution $p_0$ and the target distribution $p_1$ ~\cite{Xingchao_2022,Lipman_2022}. This framework encompasses diffusion models~\cite{sohl2015deep,ddpm} as a special case. In particular, a commonly used formulation sets a Gaussian distribution as the source: $p_0 = \mathcal{N}(\mathbf{0}, \mathbf{I})$ and a delta distribution centered on a sample $\mathbf{x}_1$ from the data distribution $q$ as the target: $p_1 = \delta_{\mathbf{x}_1}$.
Then, it incorporates an affine conditional flow $\psi_t(\mathbf{x} | \mathbf{x}_1) = a_t \mathbf{x}_1 + b_t \mathbf{x}$ with the boundary condition $(a_0, b_0) = (0, 1),\ (a_1, b_1) = (1, 0)$ to bridge them. The neural network typically approximates quantities such as velocity fields, $x_0$ prediction or $x_1$ prediction. In this modeling, these quantities can be viewed as affine transformations of the marginal probability path score $\nabla_{\mathbf{x}} \log p_t(\mathbf{x})$.

\subsection{Classifier-free guidance in flow-based models}
Classifier-free guidance~\cite{ho2022classifier} is a method for sampling from a model conditioned by a text input $\mathbf{y}$ by guiding an unconditional image generation model modeled using the score $\nabla_{\mathbf{x}} \log p_t(\mathbf{x})$. This enables the sampling from
\[
q_w(\mathbf{x}, \mathbf{y}) \propto q(\mathbf{x})q(\mathbf{y}|\mathbf{x})^w \propto q(\mathbf{x})^{1-w}q(\mathbf{x}|\mathbf{y})^w
\]
where $w \in \mathbb{R}$ is the guidance scale typically used with $w > 1$. The score satisfies
\[
\nabla_{\mathbf{x}} \log q_w(\mathbf{x}, \mathbf{y}) = (1-w)\nabla_{\mathbf{x}} \log q(\mathbf{x}) + w\nabla_{\mathbf{x}} \log q(\mathbf{x}|\mathbf{y})
\]
so by training the network to learn both the unconditional score $\nabla_{\mathbf{x}} \log q(\mathbf{x})$ and conditional score $\nabla_{\mathbf{x}} \log q(\mathbf{x}|\mathbf{y})$, flexible sampling from the conditional distribution can be achieved through a weighted sum of the network outputs.
	
	\begin{abstract}
		Large language models (LLMs) have achieved remarkable success in natural language processing (NLP) tasks, yet their substantial memory requirements present significant challenges for deployment on resource-constrained devices. Singular Value Decomposition (SVD) has emerged as a promising compression technique for LLMs, offering considerable reductions in memory overhead. However, existing SVD-based methods often struggle to effectively mitigate the errors introduced by SVD truncation, leading to a noticeable performance gap when compared to the original models. Furthermore, applying a uniform compression ratio across all transformer layers fails to account for the varying importance of different layers. To address these challenges, we propose $\ours$, an adaptive SVD-based LLM compression approach. Specifically, $\ours$ introduces \textbf{adaComp}, which adaptively compensates for SVD truncation errors by alternately updating the singular matrices $\mathcal{U}$ and $\mathcal{V}^\top$. Additionally, $\ours$ introduces \textbf{adaCR}, which adaptively assigns layer-specific compression ratios based on the relative importance of each layer. Extensive experiments across multiple LLM/VLM families and evaluation metrics demonstrate that $\ours$ consistently outperforms state-of-the-art (SOTA) SVD-based methods, achieving superior performance with significantly reduced memory requirements. Code and models of $\ours$ will be available at \url{https://github.com/ZHITENGLI/AdaSVD}.% for $\ours$ to facilitate further research.
	\end{abstract}
	
	%% narrow the gap between equations and sentences
	\setlength{\abovedisplayskip}{2pt}
	\setlength{\belowdisplayskip}{2pt}
	
	\vspace{-3mm}
	\section{Introduction}
	\vspace{-1mm}
	
	\begin{figure}[t]
		\centering
		\includegraphics[width=\linewidth]{figs/Figure_0_ours.pdf} % Use a placeholder image (LaTeX comes with an example image)
		\vspace{-7mm}
		\caption{Comparison between vanilla SVD, FWSVD~\cite{hsu2022fwsvd}, ASVD~\cite{yuan2024asvd}, SVD-LLM~\cite{wang2024svdllm}, and our $\ours$ on WikiText2.}
		\vspace{-7mm}
	\end{figure}
	
	\begin{figure*}[t]
		\centering
		\includegraphics[width=1\textwidth]{figs/overview.pdf}
		\vspace{-7mm}
		\caption{Overview of the proposed $\ours$ method: (a) SVD decomposition and truncation for linear layer weights; (b) Stack-of-batch strategy for efficient use of calibration data under limited GPU memory; (c) Adaptive compression ratio assignment (\textbf{adaCR}) based on layer-wise importance; (d) Adaptive compensation (\textbf{adaComp}) through alternating updates of $\mathcal{U}$ and $\mathcal{V}^\top$.}
		\vspace{-5mm}
		\label{fig:overview}
	\end{figure*}
	
	Recently, large language models (LLMs) based on the Transformer architecture~\cite{vaswani2017attention} have shown remarkable potential across a wide range of natural language processing (NLP) tasks. However, their success is largely driven by their massive scale, with models such as the LLaMA family~\cite{touvron2023llama} and the Open Pre-trained Transformer (OPT) series~\cite{zhang2022opt} containing up to 70B and 66B parameters, respectively. The substantial memory requirements of these models present significant challenges for deploying them on mobile devices. Consequently, the widespread adoption of LLMs remains limited by their immense resource demands~\cite{wan2023efficient, wang2024iot, zhou2024survey}.
	
	
	
	Recent research on large language model (LLM) compression has explored various techniques, including weight quantization~\cite{lin2024awq, frantar2022gptq}, network pruning~\cite{sun2023simple, frantar2023sparsegpt}, low-rank factorization~\cite{wang2024svdllm, zhang2023loraprune, yuan2024asvd}, and knowledge distillation~\cite{zhong2024revisiting, gu2023knowledge}. Among these methods, low-rank factorization using Singular Value Decomposition (SVD)~\cite{hsu2022fwsvd,yuan2024asvd,wang2024svdllm} stands out as a powerful approach for reducing both model size and computational cost. SVD achieves this by decomposing large weight matrices into smaller, low-rank components while preserving model performance. Since LLMs are often memory-bound during inference~\cite{dao2022flashattention,dao2023flashattention}, SVD compression can effectively accelerate model inference by reducing the memory requirements, even when applied solely to the weights. This approach does not require specialized hardware or custom operators, unlike weight quantization, making SVD more versatile across different platforms. Additionally, SVD is orthogonal to other compression techniques~\cite{wang2024svdllm}, allowing it to be combined with methods like weight quantization or network pruning for even greater efficiency, enabling more scalable and adaptable solutions for deploying LLMs.
	
	
	Recent advancements in SVD-based LLM compression, including FWSVD~\cite{hsu2022fwsvd}, ASVD~\cite{yuan2024asvd}, and SVD-LLM~\cite{wang2024svdllm}, have significantly improved the low-rank factorization approach, enhancing the overall effectiveness of SVD compression. For example, FWSVD introduces Fisher information to prioritize the importance of parameters, while ASVD accounts for the impact of activation distribution on compression error. SVD-LLM establishes a relationship between singular values and compression loss through the data whitening techniques. While these methods have led to notable improvements in SVD compression, they still face challenges when applied at high compression ratios.
	
	
	To bridge the performance gap between compressed and original models at both low and high compression ratios, we revisit SOTA solutions for LLM compression using SVD decomposition. Our analysis highlights two key observations:
	\textbf{First,} low-rank weight compensation after truncating the smallest singular vectors has been largely overlooked or insufficiently explored in prior methods. When truncating parts of the matrices $\mathcal{U}$ and $\mathcal{V}^\top$, the remaining components should be adjusted accordingly to minimize the SVD compression error.
	\textbf{Second,} previous methods typically apply a uniform compression ratio across all transformer layers, failing to account for their varying relative importance. To address this, an importance-aware approach for assigning appropriate compression ratios is necessary.
	
	
	
	
	
	
	
	
	
	
	
	
	To tackle the challenges outlined above, we propose $\ours$, an adaptive SVD-based LLM compression method. \textbf{First,} $\ours$ proposes \textbf{adaComp}, an adaptive compensation technique designed to adjust the weights of $\mathcal{U}$ and $\mathcal{V}^\top$ after SVD truncation. By alternately updating the matrices $\mathcal{U}$ and $\mathcal{V}^\top$, \textbf{adaComp} effectively reduces compression errors in a stable and efficient manner. To optimize the use of calibration data with limited GPU memory, we also introduce a stack-of-batch technique when applying \textbf{adaComp}.
	\textbf{Second,} $\ours$ proposes \textbf{adaCR}, a method that assigns adaptive compression ratios to different layers based on their importance. With the target compression ratio fixed, this strategy significantly improves performance compared to using a uniform compression ratio across all layers.
	
	
	Our key contributions are summarized as follows:
	% \vspace{-2.5mm}
	\begin{itemize}
		\item We propose \textbf{adaComp}, a novel adaptive compensation method for SVD truncation. By alternately updating $\mathcal{U}$ and $\mathcal{V}^\top$ and employing the stack-of-batch technique, we effectively and stably minimize compression error.
		
		\item We propose \textbf{adaCR}, an adaptive compression ratio method that assigns layer-specific compression ratios according to their relative importance in LLMs. This importance-aware approach outperforms the previously used uniform compression ratio method.
		
		\item Extensive experiments on LLMs/VLMs demonstrate that our method, $\ours$, significantly outperforms the previous SOTA SVD-based LLM compression method, SVD-LLM, effectively narrowing the performance gap between compressed and original models.
		
		
		
	\end{itemize}
	
	% \begin{algorithm*}[t]
		% \caption{Pseudocode of \ours}
		% \begin{algorithmic}[1] % The number [1] tells LaTeX to number each line
			% \State \textbf{Input:} $\mathcal M$: Original LLM
			% \State \textbf{Output:} $\mathcal M'$: Updated Model by \ours
			% \Procedure{\ours}{$\mathcal M$} % Algorithm name and parameters
			%     \State Randomly collect several sentences as the calibration data $\mathcal{C}$ 
			%     \State \textbf{Shuffle} $\mathcal{C}$, randomly sample $m$ buckets and utilize mean value \Comment{Stack-of-batch strategy}
			%     \State $\text{Set}_\mathcal{S} \gets \textproc{Whitening}(\mathcal M, \mathcal C)$
			%     \State $\text{Set}_\mathcal{SVD} \gets \emptyset$ \Comment{Initialize the set of decomposed matrices for the weight to compress} 
			%     \State $\text{Set}_\mathcal{W} \gets \mathcal{M}$ \Comment{Obtain the set of weights in $M$ to compress} 
			%     \State $\text{Set}_\mathcal{CR} \gets$ \textproc{Layerwise Compression Ratio Calculation($\mathcal{M}$)}
			%     \For{$\mathcal{W}$ \textbf{in} $\text{Set}_\mathcal{W}$}
			%             \State $\mathcal{S} \gets \text{Set}_\mathcal{S}(\mathcal{W})$  \Comment{Extract the whitening matrix of current weight $\mathcal{W}$} 
			%             \State $\mathcal{U}, \Sigma, \mathcal{V} \gets \text{SVD}(\mathcal{WS})$, $\Sigma_1 \gets \text{Trunc.}(\Sigma)$, $\text{Set}_{\mathcal{SVD}} \gets (\mathcal{U}, \Sigma_1, \mathcal{V}) \cap \text{Set}_{\mathcal{SVD}}$      
			%             \State \Comment{Apply SVD and truncation. Notice that for each layer, $\mathcal{CR}(W_i) ~\text{equals}~\frac{\#\text{params of }U_k^\sigma + \#\text{params of }{V_k^\sigma}^\top}{\#\text{params of }W_i}$}
			%     \EndFor
			%     \State $\mathcal{M} \gets \textproc{Adaptive Layer-Wise Adaptive Compensation Update}(\mathcal{M, C, \text{Set}_S, \text{Set}_{SVD}})$
			%     \State \Return{$\mathcal{M'}$}
			% \EndProcedure
			% \end{algorithmic}
		% \label{algo:framework}
		% \end{algorithm*}
	
	
	\begin{algorithm*}[t]
		\caption{Pseudocode of \ours}
		\begin{algorithmic}[1] % The number [1] tells LaTeX to number each line
			\State \textbf{Inputs:} LLM $\mathcal M$, Calib Data $\mathcal C$, Bucket Size $M$, Target Retention Ratio $trr$, Min Retention Ratio $mrr$, Update Iteration $k$
			\State \textbf{Outputs:} Updated Model $\mathcal M'$ by \ours
			\Procedure{\ours}{$\mathcal{M, C}, trr, mrr, k$} % Algorithm name and parameters
			\State $\mathcal{X} \gets$ \textproc{Get\_calib($\mathcal{C}$)}  \Comment{Randomly collect samples as calibration data}
			
			\State $\mathcal{X}'[1],\mathcal{X}'[2],...,\mathcal{X}'[M] \gets $ \textproc{SOB($\mathcal{X}, M$)} \Comment{Shuffle samples and utilize stack-of-batch (SOB) strategy}
			
			\State $\text{Set}_\mathcal{S} \gets \textproc{Whitening}(\mathcal M, \mathcal{X}')$, $\text{Set}_\mathcal{SVD} \gets \emptyset$, $\text{Set}_\mathcal{W} \gets \mathcal{M}$ \Comment{Initialize sets of decomposed matrices and weights}
			
			\State $\text{Set}_\mathcal{CR} \gets$ \textproc{Layer\_CR($\mathcal{M}, \mathcal{X}', trr, mrr$)}
			\Comment{Calculate layerwise importance and compression ratio}
			
			\For{layer $i$ \textbf{in} language model $\mathcal{M}$}
			\State $\mathcal{W}_i \gets \text{Set}_\mathcal{W}(i)$, $\mathcal{S}_i \gets \text{Set}_\mathcal{S}(\mathcal{W}_i)$  \Comment{Extract the whitening matrix of current weight $\mathcal{W}_i$} 
			
			\State $\mathcal{U}_i, \Sigma_i, \mathcal{V}_i \gets$ \textproc{SVD($\mathcal{W}_i\mathcal{S}_i$)}
			\Comment{Apply Singular Value Decomposition}
			
			\State $\Sigma' \gets$ \textproc{Trunc($\Sigma_i$)}, ($\mathcal{U}_i', \mathcal{V}_i') \gets$ \textproc{Trunc\_UV($\mathcal{U,V},\Sigma'$)}
			\Comment{Apply adaptive compression ratio and truncation}
			
			\State $\text{Set}_{\mathcal{SVD}} \gets (\mathcal{U}_i', \mathcal{V}_i') \cup \text{Set}_{\mathcal{SVD}}$ 
			
			\EndFor
			\State $\mathcal{M}' \gets$ \textproc{Ada\_Update$(\mathcal{M, \mathcal{X}', \text{Set}_{SVD}}, k)$}
			\Comment{Utilize alternate update for $\mathcal{U}_i', \mathcal{V}_i'$ with iteration $k$}
			
			\State \Return{$\mathcal{M'}$}
			\EndProcedure
		\end{algorithmic}
		\label{algo:framework}
	\end{algorithm*}
	
	
	\vspace{-2mm}
	\section{Related Works}
	\vspace{-2mm}
	\subsection{LLM Compression Techniques}
	\vspace{-2mm}
	Recent advancements in model compression techniques have significantly enhanced the efficiency of deploying LLMs while maintaining their performance. Widely explored approaches include weight quantization~\cite{frantar2022gptq, lin2024awq}, network pruning~\cite{frantar2023sparsegpt, ma2023llmpruner, yang2024laco, gromov2024unreasonable, ashkboos2024slicegpt}, and hybrid methods~\cite{dong2024stbllm}.
	In unstructured pruning, SparseGPT~\cite{frantar2023sparsegpt} prunes weights based on their importance, as determined by the Hessian matrix. However, it faces challenges in achieving optimal speedup, particularly due to hardware compatibility issues. Structured pruning methods, in contrast, are more hardware-friendly. LLM-Pruner~\cite{ma2023llmpruner} selectively removes non-critical coupled structures using gradient information. LaCo~\cite{yang2024laco} introduces a layer-wise pruning strategy, where subsequent layers collapse into preceding ones. ~\citet{gromov2024unreasonable} explores the effectiveness of basic layer-pruning techniques combined with parameter-efficient fine-tuning (PEFT). Additionally, SliceGPT~\cite{ashkboos2024slicegpt} has pioneered post-training sparsification, emphasizing the importance of layer removal order for optimal performance.
	Quantization techniques offer another significant avenue for compression. GPTQ~\cite{frantar2022gptq} applies layer-wise quantization and reduces quantization errors through second-order error compensation. AWQ~\cite{lin2024awq} introduces activation-aware weight quantization, employing a scale transformation between weights and activations. Moreover, BiLLM~\cite{huang2024billm} and ARB-LLM~\cite{li2024arb} achieve further compression to 1-bit while maintaining remarkable performance. More recently, STB-LLM~\cite{dong2024stbllm} combines 1-bit quantization with pruning to achieve even greater memory reduction for LLMs.
	However, many of these compression techniques face challenges related to hardware compatibility, often requiring custom CUDA kernels~\cite{dong2024stbllm} to enable real-time inference speedup.
	
	
	
	\vspace{-1.2mm}
	\subsection{SVD-based LLM Compression}
	\vspace{-1.2mm}
	Singular Value Decomposition (SVD) is a widely used technique for reducing matrix size by approximating a matrix with two smaller, low-rank matrices~\cite{GOLUB1987317}. Although SVD-based methods have demonstrated potential in compressing LLMs, their full capabilities remain underexplored. Standard SVD typically focuses on compressing the original weight matrix without considering the significance of individual parameters, which can lead to considerable compression errors. To address this, \citet{hsu2022languagemodel} introduced FWSVD, which incorporates Fisher information to weight the importance of parameters. However, this method requires complex gradient calculations, making it resource-intensive. Another limitation of standard SVD is the impact of activation distribution on compression errors. To mitigate this, \citet{yuan2024asvd} proposed ASVD, which scales the weight matrix with a diagonal matrix that accounts for the influence of input channels on the weights. Subsequently, \citet{wang2024svdllm} introduced SVD-LLM, which establishes a connection between singular values and compression loss. This work demonstrates that truncating the smallest singular values after data whitening effectively minimizes compression loss. Despite these advancements, existing methods still exhibit significant accuracy loss at higher compression ratios and lack a comprehensive approach for compensating compressed weights after SVD truncation. Furthermore, most methods apply a uniform compression ratio across all transformer layers, overlooking the varying importance of different layers. $\ours$ seeks to address these limitations by proposing an adaptive compensation method (\textbf{adaComp}) and an importance-aware adaptive compression ratio method (\textbf{adaCR}).
	
	
	\vspace{-2.5mm}
	\section{Method}
	\vspace{-2.5mm}
	\textbf{Overview.\quad} As illustrated in~\cref{fig:overview}, our $\ours$ integrates adaptive compensation for SVD truncation (\textbf{adaComp}) with an adaptive importance-aware compression ratio method (\textbf{adaCR}). In~\cref{sec:adacom}, we first describe how \textbf{adaComp} compensates for SVD truncation. Next, in~\cref{sec:adacr}, we detail how \textbf{adaCR} determines the compression ratio based on layer importance. The pseudocode of $\ours$ is shown in~\cref{algo:framework}, and pseudocodes for \textbf{adaComp} and \textbf{adaCR} are provided in the supplementary file.
	
	\vspace{-2mm}
	\subsection{Adaptive Compensation for SVD Truncation}
	\label{sec:adacom}
	\vspace{-1.5mm}
	SVD compression first applies SVD decomposition for matrix $\mathcal{W}$, and then truncates the smallest singular values:
	\begin{align}
		\mathcal{W} = \mathcal{U}\Sigma \mathcal{V}^\top \approx \mathcal{U}_k\Sigma_k\mathcal{V}_k^\top = \widehat{\mathcal{W}},
	\end{align}
	where $\Sigma_k$ indicates the retaining top-k largest singular values, $\mathcal{U}_k$ and $\mathcal{V}_k^\top$ represent the corresponding retaining singular vectors. Moreover, the diagonal matrix $\Sigma_k$ can be further absorbed into $\mathcal{U}_k$ and $\mathcal{V}_k^\top$ by
	\begin{align}
		\mathcal{U}_k^\sigma &= \mathcal{U}_k\Sigma_k^\frac{1}{2}, \ \mathcal{V}_k^\sigma = \mathcal{V}_k\Sigma_k^\frac{1}{2},\\ \widehat{\mathcal{W}}&=\mathcal{U}_k\Sigma_k\mathcal{V}_k^\top = \mathcal{U}_k^\sigma(\mathcal{V}_k^\sigma)^\top.
	\end{align}
	The truncation of the smallest singular values minimizes the compression error with respect to $\mathcal{W}$, ensuring that $||\mathcal{U}_k^\sigma(\mathcal{V}_k^\sigma)^\top-\mathcal{W}||_F^2$ is minimized, which we refer to as the vanilla SVD method. However, this approach does not fully account for the practical effects of $\mathcal{X}$. To address this limitation, we introduce a more application-relevant metric for the SVD compression error, defined as follows:
	\begin{align}
		\mathcal{L}_\text{SVD}&=||\widehat{\mathcal{W}}\mathcal{X}-\mathcal{WX}||_F^2 \notag\\
		&=||\mathcal{U}_k^\sigma(\mathcal{V}_k^\sigma)^\top \mathcal{X}-\mathcal{WX}||_F^2.
	\end{align}
	Previous works~\cite{hsu2022languagemodel, yuan2024asvd, wang2024svdllm} have made significant efforts to minimize $\mathcal{L}_\text{SVD}$. However, some of them involve complex and time-consuming preprocessing steps. Furthermore, they still face substantial challenges in effectively mitigating the large errors that arise under high compression ratios, particularly when truncating 60\% or more of the parameters.
	
	\begin{figure}[t]
		\centering
		\includegraphics[width=1\linewidth]{figs/3.1-v4.pdf}
		\vspace{-7mm}
		\caption{Adaptive compensation for SVD truncation (\textbf{adaComp}). (a) Comparison between naive (NU) and Moore-Penrose pseudoinverse update (MPPU). (b) Comparison between naive (NC) and stack-of-batch calibration strategy (SobC). (c) Distribution comparison before and after applying \textbf{adaComp}. }
		\vspace{-6mm}
		\label{fig:3.1}
	\end{figure}
	
	To compensate for the error attributed to SVD truncation, we need to optimize the following objective:
	\begin{align}
		\mathcal{U}_k^\sigma, {\mathcal{V}_k^\sigma}^\top &= \arg\min_{\mathcal{U}_k^\sigma, {\mathcal{V}_k^\sigma}^\top} \| \mathcal{U}_k^\sigma {\mathcal{V}_k^\sigma}^\top \mathcal{X} - \mathcal{WX} \|_F^2.
	\end{align}
	A straightforward approach is to compute the partial derivatives of the SVD compression objective with respect to $\mathcal{U}_k^\sigma$ and ${\mathcal{V}_k^\sigma}^\top$, resulting in the following expressions (additional details can be found in the supplementary file):
	\begin{align}
		&\frac{\partial \mathcal{L}_\text{SVD}}{\partial \mathcal{U}_k^\sigma} = 0 \notag\\
		&\quad\Rightarrow \mathcal{U}_k^\sigma = \mathcal{WX} \mathcal{X}^\top \mathcal{V}_k^\sigma((\mathcal{V}_k^\sigma)^\top \mathcal{X} \mathcal{X}^\top \mathcal{V}_k^\sigma)^{-1}, \\
		&\frac{\partial \mathcal{L}_\text{SVD}}{\partial {\mathcal{V}_k^\sigma}^\top} = 0 \notag\\
		&\quad\Rightarrow {\mathcal{V}_k^\sigma}^\top = ((\mathcal{U}_k^\sigma)^\top \mathcal{U}_k^\sigma)^{-1}(\mathcal{U}_k^\sigma)^\top \mathcal{W}.
	\end{align}
	However, this method involves computing the matrix inverse, which can lead to unstable updates and significant compression errors, as shown in~\cref{fig:3.1} (a). To mitigate the issue of numerical instability, we propose a two-fold strategy to enhance the update quality of $\mathcal{U}_k^\sigma$ and ${\mathcal{V}_k^\sigma}^\top$.
	
	\begin{figure*}[htbp]
		\centering
		\includegraphics[width=1\textwidth]{figs/3.2-v4.pdf}
		\vspace{-7mm}
		\caption{Layer-wise relative importance of different LLMs. The importance across different layers varies significantly, and the first layer always weight most importance. More layer-wise importance visualization can be found in the supplementary file.}
		\vspace{-3.5mm}
		\label{fig:layer_importance}
	\end{figure*}
	
	\textbf{First}, the optimization objective for $\mathcal{U}_k^\sigma$ is reformulated as a Least Squares Estimation (LSE) problem, where ${\mathcal{V}_k^\sigma}^\top \mathcal{X}$ is treated as the input and $\mathcal{WX}$ as the output:
	\begin{align}
		\mathcal{U}_k^\sigma &= \arg\min_{\mathcal{U}_k^\sigma} \| \mathcal{A}(\mathcal{U}_k^\sigma)^\top  - \mathcal{B} \|_F^2,
	\end{align}
	where $\mathcal{A}=\mathcal{X}^\top \mathcal{V}_k^\sigma$ and $\mathcal{B}=(\mathcal{WX})^\top$. Since 
	$\mathcal{A}$ is typically not a square matrix and may not be full rank, we first apply SVD to 
	$\mathcal{A}$ to enhance numerical stability:
	\begin{align}
		\mathcal{A} = \mathcal{U}_\mathcal{A}\Sigma_\mathcal{A}\mathcal{V}_\mathcal{A}^\top,
	\end{align}
	and then obtain the solution for $\mathcal{U}_k^\sigma$ by using the Moore-Penrose pseudoinverse~\cite{penrose1955generalized} of $\mathcal{A}$:
	\begin{align}
		\mathcal{U}_k^\sigma = (\mathcal{A}^+\mathcal{B})^\top = (\mathcal{V}_\mathcal{A}\Sigma_\mathcal{A}^+\mathcal{U}_\mathcal{A}^\top \mathcal{B})^\top,
	\end{align}
	where $\Sigma_A^+$ denotes the Moore-Penrose pseudoinverse of $\Sigma_A$:
	\begin{align}
		\Sigma_A &= \text{diag}(\sigma_1, \sigma_2, \dots, \sigma_n), \\
		\Sigma_A^+ &= \text{diag} \left( \sigma_1^{-1} \mathbbm{1}_{\sigma_1 \neq 0}, \sigma_2^{-1} \mathbbm{1}_{\sigma_2 \neq 0}, \dots, \sigma_n^{-1} \mathbbm{1}_{\sigma_n \neq 0} \right).
	\end{align}
	Similarly, we update ${\mathcal{V}_k^\sigma}^\top$ using the Moore-Penrose pseudoinverse of $\mathcal{U}_k^\sigma$ to handle numerical instability:
	\begin{align}
		{\mathcal{V}_k^\sigma}^\top &= \arg\min_{{\mathcal{V}_k^\sigma}^\top} \| \mathcal{U}_k^\sigma {\mathcal{V}_k^\sigma}^\top \mathcal{X} - \mathcal{WX} \|_F^2 \notag\\
		&= {\Big((\mathcal{U}_k^\sigma)^+\Big)}^\top \mathcal{W}.
	\end{align}
	As shown in~\cref{fig:3.1} (a), by reformulating the optimization objective as an LSE problem and solving for $\mathcal{U}$ and $\mathcal{V}^\top$ using the Moore-Penrose pseudoinverse, we achieve a smooth curve that consistently reduces compression error stably.
	
	\textbf{Second}, since the update rule incorporates the calibration data $\mathcal{X}$, ideally, a large volume of $\mathcal{X}$ would yield better results. However, during our experiments, we found that extending $\mathcal{X}$ to just 32 samples on an 80GB GPU is challenging. To address this, we propose a \textbf{stack-of-batch} strategy that enables the utilization of more calibration data without increasing memory overhead. Specifically, given $N$ calibration samples and a bucket size $M$ (the maximum number of samples that can fit within the fixed GPU memory), we randomly sample $mini\_bsz=\lceil\frac{N}{M}\rceil$ samples into one bucket by taking their mean value as follows:
	\begin{align}
		\mathcal{X}_{\text{rand}} &= \textit{Shuffle}(\mathcal{X}), \\
		\mathcal{X}'[k] &= \frac{1}{mini\_bsz}\sum_{i=1}^{mini\_bsz} \mathcal{X}_{\text{rand}}[(k-1) \cdot mini\_bsz + i],
	\end{align}
	where $k = 1, 2, \dots, M$, and cardinality $|\mathcal{X}'|=M$.
	As shown in~\cref{fig:3.1} (b), integrating the \textbf{stack-of-batch} strategy further reduces the compression error.
	
	
	As shown in~\cref{fig:overview}, 
	to compensate for the error attributed to SVD truncation, we propose an adaptive method to subsequently update $\mathcal{U}_k^\sigma$ and $\mathcal{V}_k^\sigma$ with the above update rules.
	Moreover, the adaptation of $\mathcal{U}_k^\sigma$ and $\mathcal{V}_k^\sigma$ can be alternatively applied until convergence, where the update sequence over $\tau$ iterations can be expressed as
	\begin{align}
		\boxed{(\mathcal{U}_k^\sigma)^\mathbf{1}
			\rightarrow ({\mathcal{V}_k^\sigma}^\top)^\mathbf{1}} &\rightarrow \boxed{(\mathcal{U}_k^\sigma)^\mathbf{2} \rightarrow ({\mathcal{V}_k^\sigma}^\top)^\mathbf{2}} \notag\\
		\rightarrow \cdots &\rightarrow \boxed{(\mathcal{U}_k^\sigma)^{\boldsymbol{\tau}} \rightarrow ({\mathcal{V}_k^\sigma}^\top)^{\boldsymbol{\tau}}},
	\end{align}
	where $(\mathcal{U}_k^\sigma)^{\boldsymbol{\tau}}$ and $({\mathcal{V}_k^\sigma}^\top)^{\boldsymbol{\tau}}$ denote the updated singular matrices after $\tau$-th iteration, respectively, while the region bounded by $\boxed{\phantom{0000}}$ corresponding to one iteration of alternative update.
	As shown in~\cref{fig:3.1} (c), the gap between the outputs of the compressed and original models narrows after alternative updates. The overlapping area rapidly increases after just a few iterations. More visual comparisons are shown in the supplementary file.
	
	Notably, our adaptive compensation can be integrated with data whitening proposed by~\citet{wang2024svdllm} and~\citet{liu2024eora}, further reducing the SVD truncation error.
	
	\begin{table*}[t]
		\centering
		% \vspace{-2mm}
		
		
		\resizebox{1\textwidth}{!}{%}
		% \tiny
		\begin{tabular}{c|c|c|c|c|c|c|c|c|c|c
				% >{\centering\arraybackslash}p{1cm}|>{\centering\arraybackslash}p{2.2cm}|>{\centering\arraybackslash}p{2.45cm}|>{\centering\arraybackslash}p{2.45cm}|>{\centering\arraybackslash}p{2.45cm}|>{\centering\arraybackslash}p{1.4cm}|>{\centering\arraybackslash}p{1.4cm}|>{\centering\arraybackslash}p{1.4cm}|>{\centering\arraybackslash}p{1.4cm}|>{\centering\arraybackslash}p{1.4cm}|>{\centering\arraybackslash}p{1.4cm}
			}
			% \toprule[0.01pt]
			\toprule
			{\textsc{Ratio}}       & {\textsc{Method}}    & ~~~~WikiText-2{$\downarrow$}~~~~ & PTB{$\downarrow$} & ~~{C4{$\downarrow$}}~~ & ~~Mmlu~~ & ~ARC\_e~ & ~WinoG.~ & ~HellaS.~  & ~~PIQA~~ & ~\textbf{Average{$\uparrow$}}~        \\ \midrule
			{\color[HTML]{9B9B9B}0\%}  & {\color[HTML]{9B9B9B}Original}  & {\color[HTML]{9B9B9B}5.68}   & {\color[HTML]{9B9B9B}8.35}     & {\color[HTML]{9B9B9B}7.34}      & {\color[HTML]{9B9B9B} 45.30} & {\color[HTML]{9B9B9B} 74.62} & {\color[HTML]{9B9B9B} 69.22} & {\color[HTML]{9B9B9B} 76.00}  & {\color[HTML]{9B9B9B} 79.11} & {\color[HTML]{9B9B9B} 68.85}       \\ \midrule
			{\multirow{5}{*}{40\%}}   & {SVD}      & 39,661.03           & 69,493.00          & {56,954.00} & \textbf{26.51} &  26.39 & 48.62  &  25.64 & 52.99    & 36.03 \\ 
			
			{} & {FWSVD~\cite{hsu2022languagemodel}}      & 8,060.35           & 9,684.10          & 7,955.21              & 25.74  &26.05   &50.20   &25.70      &52.39     &36.01  \\
			
			{} & {ASVD~\cite{yuan2024asvd}}      & 1,609.32           & 7,319.49          & 1,271.85              &24.35   &26.81   &49.49   &25.83     &53.81     &36.06  \\
			
			{} & {SVD-LLM~\cite{wang2024svdllm}}      &16.11  &719.44          & {61.95}              & 22.97  & 36.99  & 56.04  & 30.49   & 56.96  & 40.69 \\
			\cmidrule{2-11} 
			{}                     & \cellcolor{purple!10}{\textbf{$\ours$}}  & \cellcolor{purple!10}\textbf{14.76} ~\footnotesize($\downarrow$8\%)  & \cellcolor{purple!10}\textbf{304.62}~\footnotesize($\downarrow$58\%) & \cellcolor{purple!10}{\textbf{56.98}}~\footnotesize($\downarrow$8\%)   & \cellcolor{purple!10}23.63               & \cellcolor{purple!10}\textbf{41.12}               & \cellcolor{purple!10}\textbf{58.17}               & \cellcolor{purple!10}\textbf{31.75}                  & \cellcolor{purple!10}\textbf{58.49}               %& \cellcolor{purple!10}\textbf{38.62}               
			& \cellcolor{purple!10}\textbf{42.63}    \\ 
			
			
			\midrule
			{\multirow{5}{*}{50\%}}                     & {SVD}      & 53,999.48           & 39,207.00          & {58,558.00} & \textbf{25.43}  & 25.80  & 47.36  & 25.55    & 52.67    & 35.36 \\
			
			{} & {FWSVD~\cite{hsu2022languagemodel}}      & 8,173.21           & 8,615.71          & 8,024.67              &24.83   &25.84   &48.70   &25.64      &52.83    &35.57  \\
			
			{} & {ASVD~\cite{yuan2024asvd}}      & 6,977.57           & 15,539.44          & 4,785.15              &24.52   &25.13   &49.17   &25.48     &52.94    &35.45  \\
			
			{} & {SVD-LLM~\cite{wang2024svdllm}}      & 27.19           & 1,772.91        & {129.66}              & 23.44  & 31.65  & 51.14  & 28.38    & 54.57    & 37.83 \\ 
			
			\cmidrule{2-11} 
			{}                     & \cellcolor{purple!10}{\textbf{$\ours$}}  &\cellcolor{purple!10}\textbf{25.58}~\footnotesize($\downarrow$6\%)   & \cellcolor{purple!10}\textbf{593.14}~\footnotesize($\downarrow$67\%) & \cellcolor{purple!10}{\textbf{113.84}}~\footnotesize($\downarrow$12\%)   & \cellcolor{purple!10}23.24               & \cellcolor{purple!10}\textbf{34.18}               & \cellcolor{purple!10}\textbf{54.06}               & \cellcolor{purple!10}\textbf{28.88}                        & \cellcolor{purple!10}\textbf{55.50}             % & \cellcolor{purple!10}\textbf{37.83}               
			& \cellcolor{purple!10}\textbf{39.17}    \\ 
			
			
			\midrule
			{\multirow{5}{*}{60\%}} & {SVD}      & 65,186.67           & 79,164.00          & {70,381.00} & 22.94  & 24.49  & \textbf{51.85}  & 25.40   & 53.16   & 35.57 \\
			
			{} & {FWSVD~\cite{hsu2022languagemodel}}      & 27,213.30           & 24,962.80          & 47,284.87              &\textbf{26.91}   &25.38   &48.46   &25.61     & 51.96    & 35.66 \\
			
			{} & {ASVD~\cite{yuan2024asvd}}      & 10,003.57           & 15,530.19          & 9,983.83              &26.89   &26.68   &48.86   &25.76      & 51.80   &36.00  \\
			
			{} & {SVD-LLM~\cite{wang2024svdllm}}      & 89.90           & 2,052.89         & {561.00}              & 22.88  & 26.73  & 47.43  & 26.89    & \textbf{53.48}    & 35.48 \\ 
			\cmidrule{2-11} 
			{}                     & \cellcolor{purple!10}{\textbf{$\ours$}}  & \cellcolor{purple!10}\textbf{50.33}~\footnotesize($\downarrow$44\%)   & \cellcolor{purple!10}\textbf{1,216.95}~\footnotesize($\downarrow$41\%) & \cellcolor{purple!10}\textbf{239.18}~\footnotesize($\downarrow$57\%)  & \cellcolor{purple!10}24.69              & 
			\cellcolor{purple!10}\textbf{28.20}               & 
			\cellcolor{purple!10}51.22               & 
			\cellcolor{purple!10}\textbf{27.36}               & \cellcolor{purple!10}52.83                            & \cellcolor{purple!10}\textbf{36.87}   \\ 
			\bottomrule
			
			
			% \bottomrule[0.01pt]
			% \vspace{-15mm}
		\end{tabular}
	}
	\label{tab:dataset_acc}
	\vspace{-2mm}
	\caption{Zero-shot performance comparison of LLaMA2-7B between $\ours$ and previous SVD compressed methods under 40\% to 60\% compression ratios. Evaluation on three language modeling datasets (measured by perplexity  ({$\downarrow$})) and five common sense reasoning datasets (measured by both individual and average accuracy ({$\uparrow$})) demonstrate the effectiveness of $\ours$.
	}
	\vspace{-4mm}
\end{table*}

% adaCR
\subsection{Adaptive SVD Compression Ratio}
\label{sec:adacr}


Previous studies on SVD compression typically apply a uniform compression ratio across all transformer layers of LLMs, overlooking the varying importance of different layers.
Inspired by~\citet{men2024shortgpt} and~\citet{dumitru2024change}, we propose \textbf{adaCR}, which adaptively determines the SVD compression ratio for each transformer layer, considering each layer's distinct impact on activations.

The importance of $\mathcal{W}$ can be measured by its impact on the input, which is quantified as the similarity between the input $\mathcal{X}$ and the output $\mathcal{Y}$ after passing through $\mathcal{W}$.
\begin{align}
	\mathcal{Y} &= \mathcal{WX}, \\
	\mathcal{I}(\mathcal{W}) &= \text{similarity}(\mathcal{X,Y}),
\end{align}
where $\mathcal{I}(\mathcal{W})$ denotes the layer-wise importance of $\mathcal{W}$. The similarity metric used can vary, and for simplicity, we adopt cosine similarity in our method.

Then, we normalize $\mathcal{I}(\mathcal{W})$ through mean centering to obtain the relative importance of $\mathcal{W}$:
\begin{align}
	\mathcal{I}_n(\mathcal{W}) = \mathcal{I}(\mathcal{W}) / \text{mean}(\mathcal{I}(\mathcal{W})).
\end{align}
After mean normalization, the average importance is 1. A value of $\mathcal{I}_n(\mathcal{W})$ greater than 1 indicates greater importance, while a value lower than 1 indicates lesser importance. The compression ratio of each layer will be adaptively adjusted based on the relative importance:
\begin{align}
	\mathcal{CR}(\mathcal{W}) = mrr + \mathcal{I}_n(\mathcal{W}) \cdot (trr - mrr),
\end{align}
where $mrr$ and $trr$ are the minimum and target retention ratios, respectively. Notably, $\mathcal{CR}(\mathcal{W}) = mrr$ when $\mathcal{I}_n(\mathcal{W})=0$, and $\mathcal{CR}(\mathcal{W}) = trr$ when $\mathcal{I}_n(\mathcal{W}) = 1$.

Given the compression ratio for the $i$-th layer by \textbf{adaCR}, we truncate the vectors of least singular values from both $\mathcal{U}_k^\sigma$ and ${\mathcal{V}_k^\sigma}^\top$ so that 
\begin{align}
	\mathcal{CR}(\mathcal{W}_i)=\frac{\#\text{params of }\mathcal{U}_k^\sigma + \#\text{params of }{\mathcal{V}_k^\sigma}^\top}{\#\text{params of }\mathcal{W}_i}.
\end{align}
As shown in~\cref{fig:layer_importance}, the importance of different layers varies. It can be observed that the first layer always weighs the most importance, suggesting that we should retain more weight on it. For the Llama family, the relative importance curve approximates a bowl shape, highlighting the significance of both the initial and final layers.



\section{Experiments}
\subsection{Setup}

We compare our $\ours$ with four baselines,  including vanilla SVD and SOTA SVD-based LLM compression methods FWSVD~\cite{hsu2022languagemodel}, ASVD~\cite{yuan2024asvd}, and SVD-LLM~\cite{wang2024svdllm}.

\begin{figure*}[htbp]
	\centering
	\includegraphics[width=1\textwidth]{figs/LLaVA-main.pdf}
	\vspace{-6mm}
	\caption{We perform image captioning by applying SVD, SVD-LLM~\cite{wang2024svdllm}, and our $\ours$ to LLaVA-7B model on the COCO dataset respectively, highlighting the \textcolor{color4}{correct} captions and \textcolor{red}{wrong} captions in different colors.}
	\vspace{-4.5mm}
	\label{fig:llava}
\end{figure*}

\begin{table}[t]
	\centering
	\vspace{1.5mm}
	%The relative performance gain compared to the best-performing baseline is marked in green color inside bracket.
	% \vspace{-2mm}
	\resizebox{\linewidth}{!}{%
		% \tiny
		\begin{tabular}{c|cccc}
			% \toprule[0.01pt]
			\toprule
			\textsc{Method} &OPT-6.7B   & LLaMA2-7B    & Mistral-7B        & Vicuna-7B \\ \midrule
			SVD       & 18,607.24          & 65,186.67         & 30,378.35            & 78,704.50         \\
			FWSVD~\cite{hsu2022languagemodel}     & 8,569.56      & 27,213.30          & 5,481.24              & 8,185.66       \\
			ASVD~\cite{yuan2024asvd}      & 10,326.48         & 10,003.57         & 22,705.51             & 20,241.17          \\
			SVD-LLM~\cite{wang2024svdllm}      &   92.10      &   89.90       &  72.17            &  64.06        \\
			\midrule
			\rowcolor{purple!10}\textbf{$\ours$}  & \textbf{86.64}~\footnotesize($\downarrow$6\%)         & \textbf{50.33}~\footnotesize($\downarrow$44\%)           &  \textbf{67.22}~\footnotesize($\downarrow$7\%)          & \textbf{56.97}~\footnotesize($\downarrow$11\%)         \\ \bottomrule
			% \bottomrule[0.01pt]
		\end{tabular}
	}
	\vspace{-2mm}
	\caption{Perplexity ($\downarrow$) of four different LLMs -- OPT-6.7B, LLaMA 2-7B, Mistral-7B, and Vicuna-7B -- under 60\% compression ratio on WikiText-2, where $\ours$ shows consistent improvements. \label{tab:different_llm_acc}}
	\vspace{-4mm}
\end{table}

\noindent\textbf{Models and Datasets.$\quad$} 
%
To demonstrate the generalizability of our method, we evaluate the performance of $\ours$ and the baselines on four models from three different LLM families, including LLaMA2-7B~\cite{touvron2023llama2}, OPT-6.7B~\cite{zhang2022opt}, Mistral-7B~\cite{jiang2023mistral}, and Vicuna-7B~\cite{chiang2023vicuna}. We benchmark on eight datasets, including three language modeling datasets (WikiText-2~\cite{merity2016pointersentinelmixturemodels}, PTB~\cite{marcus1993building}, and C4~\cite{raffel2023exploring}) and five common-sense reasoning datasets (WinoGrande~\cite{sakaguchi2019winograndeadversarialwinogradschema}, HellaSwag~\cite{zellers2019hellaswagmachinereallyfinish}, PIQA~\cite{bisk2019piqareasoningphysicalcommonsense}, ARC-e~\cite{clark2018thinksolvedquestionanswering}, and Mmlu~\cite{hendryckstest2021}).
We use the LM-Evaluation-Harness framework~\cite{eval-harness} to evaluate the model performance on these zero-shot Question-Answering (QA) datasets.

\noindent\textbf{Implementation Details.$\quad$} 
%
To ensure a fair comparison, we followed ASVD~\cite{yuan2024asvd} and SVD-LLM~\cite{wang2024svdllm} to randomly select 256 samples from WikiText-2 as the calibration data and conduct data whitening before SVD truncation. 
All the experiments are conducted with PyTorch~\cite{paszke2019pytorch} and Huggingface~\cite{paszke1912imperative} on a single NVIDIA A100-80GB GPU. 


\subsection{Main Results}

We evaluate the overall performance of $\ours$ from three aspects: \textbf{(1)} performance under different compression ratios \textbf{(40\%, 50\%, 60\%, 70\%, and 80\%)}, \textbf{(2)} performance on different LLMs. \textbf{(3)} performance on visual language models. Some performance evaluation results and generated contents by the compressed LLMs are included in the supplementary file to provide a more straightforward comparison.


\noindent\textbf{Performance under Different Compression Ratios.$\quad$}
First, we evaluate the performance of LLaMA2-7B compressed by $\ours$, vanilla SVD and the SOTA method SVD-LLM~\cite{wang2024svdllm} under compression ratios ranging from 40\% to 80\% on all $8$ datasets, as shown in~\cref{tab:dataset_acc}. 
On the three language modeling datasets, $\ours$ consistently outperforms vanilla SVD, and SVD-LLM across all the compression ratios. 
More importantly, $\ours$ exhibits significant advantages over the baselines under higher compression ratios. 
These results indicate that $\ours$ is more effective in compressing LLMs for more resource-constrained devices such as smartphones and IoT devices, which often have limited memory and processing capabilities.
On the five common sense reasoning datasets, 
$\ours$ also maintains its edge and performs better than the best-performing baseline on most of the datasets and consistently achieves higher average accuracy across all the compression ratios. Due to page limitations, comparisons for 70\% and 80\% compression ratios are provided in the supplementary file.



\noindent\textbf{Performance on Different LLMs.$\quad$}
To demonstrate the generability of $\ours$ across different LLMs, we compare $\ours$ and the baselines on four different models OPT-6.7B, LLaMA2-7B, Vicuna-7B, and Mistral-7B -- under 60\% compression ratio on WikiText-2.
As shown in~\cref{tab:different_llm_acc},  
$\ours$ consistently outperforms vanilla SVD, FWSVD, ASVD and SVD-LLM on all LLMs, and exhibits more stable performance across different LLMs, especially compared to vanilla SVD and FWSVD. We reproduce FWSVD, ASVD, and SVD-LLM using their official GitHub repositories. FWSVD and ASVD fail on these LLMs with compression ratios under 60\%, whereas SVD-LLM and $\ours$ maintain reasonable perplexity in such cases.


\noindent\textbf{Performance on Visual Language Models.$\quad$} Note that our $\ours$ can also be applied to visual language models (VLMs) like LLaVA~\cite{liu2023visual}. Following~\citet{lin2024awq}, we apply SVD compression to the language part of the VLMs since it dominates the model size. As shown in~\cref{fig:llava}, $\ours$ shows better image captioning results than vanilla SVD and SVD-LLM on COCO dataset~\cite{chen2015microsoft} under 40\% compression ratio.
More image captioning comparisons with various compression ratios can be found in supplementary file.




\subsection{Ablation Study}

We provide extensive ablation study results in ~\cref{tab:ablations} to show the effect of some key components in our work. 

\begin{table*}[t]
	% \vspace{-1.5mm}
	% subfloat c - BackBone Architecture
	\vspace{-2mm}
	
	
	\subfloat[\small Effectiveness of Adaptive Compensation \label{tab:adacomp}]{
		\scalebox{0.8}{\begin{tabular}{l@{\hskip 9pt}c@{\hskip 12pt}c@{\hskip 12pt}c@{\hskip 12pt}c@{\hskip 12pt}c}
				\toprule
				\rowcolor{color3}
				\textbf{Method} & \textbf{Tgt. CR} & \textbf{adaComp}  &\textbf{WikiText2 $\downarrow$} & \textbf{PTB $\downarrow$} & \textbf{C4 $\downarrow$}   \\
				\midrule
				SVD-LLM & 40\% & \ding{55} &16.11  &719.44   &61.95 \\
				\cdashline{1-6} \addlinespace[0.2em]
				$\ours$ & 40\% & \ding{55} & 15.47  &406.83  &66.29  \\
				\rowcolor{purple!10}$\ours$ & 40\% &  \ding{51} &14.76  &304.62  &56.98  \\
				\midrule
				SVD-LLM & 50\% & \ding{55} & 27.19&1,772.91 &129.66 \\
				\cdashline{1-6} \addlinespace[0.2em]
				$\ours$ & 50\% & \ding{55} & 30.00 & 1101.15 &166.02  \\
				\rowcolor{purple!10}$\ours$ & 50\% &  \ding{51} &25.58  & 593.14 &113.84  \\
				\midrule
				SVD-LLM & 60\% & \ding{55} &89.90 &2,052.89 &561.00 \\
				\cdashline{1-6} \addlinespace[0.2em]
				$\ours$ & 60\% & \ding{55} &78.82  &6,929.39  &339.31  \\
				\rowcolor{purple!10}$\ours$ & 60\% &  \ding{51} &50.33  &1,216.95 &239.18  \\
				\bottomrule
	\end{tabular}}}\hfill
	\subfloat[\small Effectiveness of Adaptive Compression Ratio\label{tab:adacr}]{ 
		\scalebox{0.8}{
			\begin{tabular}{l@{\hskip 9pt}c@{\hskip 12pt}c@{\hskip 12pt}c@{\hskip 12pt}c@{\hskip 12pt}c}
				%\small
				\toprule
				\rowcolor{color3}\textbf{Method} &\textbf{Tgt. CR} &\textbf{CR} & \textbf{WikiText2 $\downarrow$} & \textbf{PTB $\downarrow$} & \textbf{C4 $\downarrow$} \\
				\midrule
				SVD-LLM & 40\% & Const &16.11  &719.44   &61.95  \\
				\cdashline{1-6} \addlinespace[0.2em]
				$\ours$ & 40\% & Const & 15.38  &617.11  &60.43  \\
				\rowcolor{purple!10}$\ours$ & 40\% & Adapt &14.76  &304.62  &56.98  \\
				\midrule
				SVD-LLM & 50\% & Const & 27.19&1,772.91 &129.66 \\
				\cdashline{1-6} \addlinespace[0.2em]
				$\ours$ & 50\% & Const & 27.33 & 1,177.53 &126.85  \\
				\rowcolor{purple!10}$\ours$ & 50\% & Adapt &25.58  & 593.14 &113.84  \\
				\midrule
				SVD-LLM & 60\% & Const &89.90 &2,052.89 &561.00 \\
				\cdashline{1-6} \addlinespace[0.2em]
				$\ours$ & 60\% & Const &69.46  &2,670.20  &336.90 \\
				\rowcolor{purple!10}$\ours$ & 60\% &  Adapt &50.33  &1,216.95 &239.18   \\
				
				\bottomrule
	\end{tabular}}} \\
	% subfloat d - Multinomial vs Independent Masks
	% subfloat b - mask representation
	\subfloat[\small Iteration Number for Adaptive Compression \label{tab:num_iter}]{
		\scalebox{0.8}{\begin{tabular}{l@{\hskip 9pt}c@{\hskip 12pt}c@{\hskip 12pt}c@{\hskip 12pt}c@{\hskip 12pt}c}
				\toprule
				\rowcolor{color3}
				\textbf{Method} &\textbf{Tgt. CR} & \textbf{\#Iteration} &\textbf{WikiText2 $\downarrow$} &\textbf{PTB $\downarrow$} &\textbf{C4 $\downarrow$}   \\
				\midrule
				SVD-LLM & 40\% & - &16.11  &719.44   &61.95  \\
				\cdashline{1-6} \addlinespace[0.2em]
				\rowcolor{purple!10}$\ours$ & 40\% & 1 &14.76  &304.62  &56.98   \\
				$\ours$ &  40\% &  3 & 15.47  &249.41   &57.28   \\
				$\ours$ & 40\% & 15  & 15.84 &257.96  & 57.39 \\
				\midrule
				SVD-LLM & 50\% & - & 27.19&1,772.91 &129.66  \\
				\cdashline{1-6} \addlinespace[0.2em]
				\rowcolor{purple!10}$\ours$ & 50\% & 1 &25.58  & 593.14 &113.84  \\
				$\ours$ & 50\% & 3  &27.11  &844.09 &115.51\\
				$\ours$ & 50\% & 15 &27.45  & 812.21 & 110.35  \\
				\midrule
				SVD-LLM & 60\% & - &89.90 &2,052.89 &561.00  \\
				\cdashline{1-6} \addlinespace[0.2em]
				\rowcolor{purple!10}$\ours$ & 60\% & 1 &50.33  &1,216.95 &239.18  \\
				$\ours$ & 60\% & 3 &64.12  &3,546.45 &301.19  \\
				$\ours$ & 60\% & 15 &62.34  &4,293.79 &267.29  \\
				\bottomrule
	\end{tabular}}}\hfill
	\subfloat[\small Minimum Retention Ratio for Adaptive CR\label{tab:mrr}]{ 
		\scalebox{0.8}{\begin{tabular}{l@{\hskip 9pt}c@{\hskip 12pt}c@{\hskip 12pt}c@{\hskip 12pt}c@{\hskip 12pt}c}
				\toprule
				\rowcolor{color3}
				\textbf{Method} &\textbf{Tgt. CR} & \textbf{MRR} &\textbf{WikiText2 $\downarrow$} &\textbf{PTB $\downarrow$} &\textbf{C4 $\downarrow$}   \\
				\midrule
				SVD-LLM & 40\% & - &16.11  &719.44   &61.95  \\
				\cdashline{1-6} \addlinespace[0.2em]
				$\ours$ & 40\% & 0.40 &15.01  &223.19 &57.17  \\
				$\ours$ & 40\% & 0.45 &14.85  &241.90  &57.08   \\
				\rowcolor{purple!10}$\ours$ & 40\% & 0.50 &14.76  &304.62 &56.98  \\
				\midrule
				SVD-LLM & 50\% & - & 27.19&1,772.91 &129.66  \\
				\cdashline{1-6} \addlinespace[0.2em]
				\rowcolor{purple!10}$\ours$ & 50\% & 0.40 &25.58  &593.14 &113.84  \\
				$\ours$ & 50\% & 0.45 &26.01  & 814.63  &117.58  \\
				$\ours$ & 50\% & 0.50 &27.33  &1,177.53 &126.85  \\
				\midrule
				SVD-LLM & 60\% & - &89.90 &2,052.89 &561.00  \\
				\cdashline{1-6} \addlinespace[0.2em]
				\rowcolor{purple!10}$\ours$ & 60\% & 0.30 & 50.33 &1,216.95 &239.18  \\
				$\ours$ & 60\% & 0.35 & 53.17 &1,608.19 &256.66  \\
				$\ours$ & 60\%  & 0.40 & 60.08 &2,137.29 &294.26 \\
				\bottomrule
	\end{tabular}}}
	% subfloat d - Multinomial vs Independent Masks
	% subfloat b - mask representation
	% \vspace{-1mm}
	\vspace{-2mm}
	\caption{Ablation studies on LLaMA-2-7B. Results are measured by perplexity, with best results highlighted in \colorbox{purple!10}{\phantom{0000}}.\label{tab:ablations}}
	\vspace{-3mm}
\end{table*}


\noindent\textbf{Effectiveness of Adaptive Compensation. $\quad$} To validate the effectiveness of the proposed \textbf{adaComp}, we compare the PPL results of Llama2-7B with and without \textbf{adaComp} on Wikitest-2, PTB, and C4 datasets in~\cref{tab:adacomp}. Results of 70\% and 80\% compression ratios can be found in the supplementary file.
It can be observed that $\ours$ consistently outperforms SVD-LLM after applying \textbf{adaComp}, and the performance gap is more significant under high compression ratios (\textit{i.e.}, 60\%, 70\%, and 80\%).




\noindent\textbf{Iteration Number. $\quad$} 
To investigate the impact of the number of \textbf{adaComp} iterations under different compression ratios, we perform an ablation study with 1, 3, and 15 iterations, as shown in~\cref{tab:num_iter}. Results for 70\% and 80\% compression ratios are provided in the supplementary file. At lower compression ratios (\textit{e.g.}, 40\%, 50\%, and 60\%), it is observed that just 1 iteration of \textbf{adaComp} already outperforms the state-of-the-art method, SVD-LLM. However, increasing the number of iterations may lead to overfitting due to the limited calibration data, resulting in a performance drop. In contrast, at higher compression ratios (\textit{e.g.}, 70\% and 80\%), additional iterations lead to performance improvements, indicating that $\ours$ is more effective in high compression ratio scenarios where previous methods still struggle. This highlights the importance of balancing the number of iterations with the available data to avoid over-optimization, especially in low compression scales.



\begin{table}[t]
	\centering
	% \vspace{-2.5mm}
	
	\resizebox{1\linewidth}{!}{%}
	% \tiny
	\begin{tabular}{c|c|c|c|c|c}
		% \toprule[0.01pt]
		\toprule
		{\textsc{Ratio}}       & {\textsc{Method}}   & {\textsc{GPTQ-INT4}} & WikiText-2{$\downarrow$} & PTB{$\downarrow$} & {C4{$\downarrow$}}  \\ \midrule
		{\color[HTML]{9B9B9B}0\%}  & {\color[HTML]{9B9B9B}Original} & \ding{55}  & {\color[HTML]{9B9B9B}5.68}   & {\color[HTML]{9B9B9B}8.35}     & {\color[HTML]{9B9B9B}7.34}    \\ \midrule
		{\multirow{4}{*}{40\%}} & {SVD-LLM}  & \ding{55}     &16.11  &719.44            & {61.95}               \\
		& {SVD-LLM}  & \ding{51}     & 33.56          & 1,887.50          & {184.61}               \\
		% \cmidrule{2-13} 
		& \cellcolor{purple!10}{\textbf{$\ours$}} & \cellcolor{purple!10}\ding{55}  & \cellcolor{purple!10}\textbf{14.76}   & \cellcolor{purple!10}\textbf{304.62} & \cellcolor{purple!10}{\textbf{56.98} }      \\ 
		& \cellcolor{purple!10}{\textbf{$\ours$}} & \cellcolor{purple!10}\ding{51}  & \cellcolor{purple!10}\textbf{22.55}   & \cellcolor{purple!10}\textbf{844.21} & \cellcolor{purple!10}{\textbf{106.41} }      \\
		%%%%%%%%%%%%%%%%%%%%%%%%%%%%%%%%%%%%%%%%%
		\midrule
		
		{\multirow{4}{*}{50\%}} & {SVD-LLM}  & \ding{55}     & 27.19           & 1,772.91  & {129.66}                     \\ 
		& {SVD-LLM}  & \ding{51}     & 41.70           & 2,335.65             & {291.62}         \\ 
		
		% \cmidrule{2-13} 
		& \cellcolor{purple!10}{\textbf{$\ours$}} & \cellcolor{purple!10}\ding{55}  &\cellcolor{purple!10}\textbf{25.58}   & \cellcolor{purple!10}\textbf{593.14}    & \cellcolor{purple!10}{\textbf{113.84} }  \\ 
		& \cellcolor{purple!10}{\textbf{$\ours$}} & \cellcolor{purple!10}\ding{51}  &\cellcolor{purple!10}\textbf{37.34}   & \cellcolor{purple!10}\textbf{1,326.55}    & \cellcolor{purple!10}{\textbf{203.11} }  \\ 
		
		
		\midrule
		
		{\multirow{4}{*}{60\%}} & {SVD-LLM}  & \ding{55}     & 89.90           & 2,052.89          & {561.00}              \\ 
		& {SVD-LLM}  & \ding{51}     & 119.46           & 3,136.60              & {723.80}         \\ 
		% \cmidrule{2-13} 
		& \cellcolor{purple!10}{\textbf{$\ours$}} & \cellcolor{purple!10}\ding{55}  & \cellcolor{purple!10}\textbf{60.08}   & \cellcolor{purple!10}\textbf{2,137.28}   & \cellcolor{purple!10}{\textbf{294.26} }    \\ 
		& \cellcolor{purple!10}{\textbf{$\ours$}} & \cellcolor{purple!10}\ding{51}  & \cellcolor{purple!10}\textbf{82.08}   & \cellcolor{purple!10}\textbf{1,705.19} & \cellcolor{purple!10}{\textbf{379.96} }     \\ 
		
		
		
		\midrule
		
		{\multirow{4}{*}{70\%}} & {SVD-LLM}   & \ding{55}    & 125.16          & 6,139.78        & {677.38}                \\ 
		& {SVD-LLM}   & \ding{51}    & 159.53         & 2,115.44           & {848.24}            \\ 
		% \cmidrule{2-13} 
		& \cellcolor{purple!10}{\textbf{$\ours$}} & \cellcolor{purple!10}\ding{55}  & \cellcolor{purple!10}\textbf{107.90}   & \cellcolor{purple!10}\textbf{5,027.62}  & \cellcolor{purple!10}{\textbf{441.33} }  
		\\  
		& \cellcolor{purple!10}{\textbf{$\ours$}} & \cellcolor{purple!10}\ding{51}  & \cellcolor{purple!10}\textbf{118.75}   & \cellcolor{purple!10}\textbf{1,606.94}   & \cellcolor{purple!10}{\textbf{466.64} }  
		\\  
		
		
		
		
		\midrule
		
		{\multirow{4}{*}{80\%}} & {SVD-LLM}   & \ding{55}    & 372.48           & 6,268.53     & {1,688.78}                 \\ 
		& {SVD-LLM}  & \ding{51}     & 420.25           & 3,716.08 & {1,996.42}                     \\ 
		% \cmidrule{2-13} 
		& \cellcolor{purple!10}{\textbf{$\ours$}} & \cellcolor{purple!10}\ding{55}  & \cellcolor{purple!10}\textbf{206.51}   & \cellcolor{purple!10}\textbf{6,613.44}   & \cellcolor{purple!10}{\textbf{679.66} }   \\ 
		& \cellcolor{purple!10}{\textbf{$\ours$}} & \cellcolor{purple!10}\ding{51}  & \cellcolor{purple!10}\textbf{214.51}   & \cellcolor{purple!10}\textbf{2,728.78} & \cellcolor{purple!10}{\textbf{654.79} }   \\  \bottomrule
		
		
		
		
		
		% \bottomrule[0.01pt]
		% \vspace{-15mm}
	\end{tabular}
}
\vspace{-2mm}
\caption{$\ours$ with weight quantization method GPTQ.
}
\vspace{-5mm}
\label{tab:svd+quant}
\end{table}



\noindent\textbf{Effectiveness of Adaptive Compression Ratio. $\quad$} 
To validate the effectiveness of our \textbf{adaCR}, we compared the results after removing \textbf{adaCR} (\textit{i.e.}, using constant compression ratios for all layers) from $\ours$. As shown in~\cref{tab:adacr}, $\ours$ already outperforms SOTA SVD-LLM without using \textbf{adaCR}, while integrating \textbf{adaCR} can further enhance the performance across all compression ratios.





\noindent\textbf{Minimum Retention Ratio. $\quad$} The minimum retention ratio ($mrr$) in \textbf{adaCR} is also crucial, and we investigate the impact of different $mrr$ values in~\cref{tab:mrr} for 40\%, 50\%, and 60\% compression ratios (70\% and 80\% in supplementary file). It can be observed that $mrr$ remains relatively robust at lower compression ratios (40\% and 50\%), while contributing more at higher compression ratios (60\%). 




%%%%%%%%%%%%%%%%%%%%%%%%%%%%



% \vspace{-1mm}
\subsection{Integrate with Weight Quantization}
% \vspace{-1mm}
Similar to previous SVD-based compression methods~\cite{hsu2022fwsvd,yuan2024asvd,wang2024svdllm}, our $\ours$ is orthogonal to other types of compression techniques. Following~\citet{wang2024svdllm}, we integrate $\ours$ with the widely used weight quantization method GPTQ~\cite{frantar2022gptq}. As shown in~\cref{tab:svd+quant}, we compare $\ours$ with SVD-LLM~\cite{wang2024svdllm} on the LLaMA2-7B model, using different compression ratios (40\%, 50\%, 60\%, 70\%, and 80\%) across the WikiText-2, PTB, and C4 datasets. The results demonstrate that, when combined with the 4-bit weight quantization method GPTQ, $\ours$ also consistently outperforms SOTA SVD-LLM across all compression ratios. Under high compression ratios (\ie, 60\%, 70\%, and 80\%), $\ours$ + GPTQ-INT4 even surpasses SVD-LLM.

\vspace{-2mm}
\section{Conclusion}
\vspace{-1.5mm}
In this work, we propose $\ours$, an adaptive SVD-based compression method for LLMs. $\ours$ 
first proposes \textbf{adaComp}, which adaptively compensates for the error caused by the truncation of singular matrices, efficiently reducing compression error without requiring additional training. Furthermore, $\ours$ proposes \textbf{adaCR}, which adaptively assigns compression ratios based on the importance of each layer, further enhancing performance while maintaining the same target compression rate. Both strategies effectively minimize SVD compression errors, particularly at high compression ratios. Our experiments on multiple open-source LLM and VLM families demonstrate that $\ours$ pushes the performance boundary beyond the current state-of-the-art SVD-based LLM compression methods. 






{
\small
\bibliographystyle{ieeenat_fullname}
\bibliography{main}
}




% WARNING: do not forget to delete the supplementary pages from your submission 
% \clearpage
\pagenumbering{gobble}
\maketitlesupplementary

\section{Additional Results on Embodied Tasks}

To evaluate the broader applicability of our EgoAgent's learned representation beyond video-conditioned 3D human motion prediction, we test its ability to improve visual policy learning for embodiments other than the human skeleton.
Following the methodology in~\cite{majumdar2023we}, we conduct experiments on the TriFinger benchmark~\cite{wuthrich2020trifinger}, which involves a three-finger robot performing two tasks: reach cube and move cube. 
We freeze the pretrained representations and use a 3-layer MLP as the policy network, training each task with 100 demonstrations.

\begin{table}[h]
\centering
\caption{Success rate (\%) on the TriFinger benchmark, where each model's pretrained representation is fixed, and additional linear layers are trained as the policy network.}
\label{tab:trifinger}
\resizebox{\linewidth}{!}{%
\begin{tabular}{llcc}
\toprule
Methods       & Training Dataset & Reach Cube & Move Cube \\
\midrule
DINO~\cite{caron2021emerging}         & WT Venice        & 78.03     & 47.42     \\
DoRA~\cite{venkataramanan2023imagenet}          & WT Venice        & 81.62     & 53.76     \\
DoRA~\cite{venkataramanan2023imagenet}          & WT All           & 82.40     & 48.13     \\
\midrule
EgoAgent-300M & WT+Ego-Exo4D      & 82.61    & 54.21      \\
EgoAgent-1B   & WT+Ego-Exo4D      & \textbf{85.72}      & \textbf{57.66}   \\
\bottomrule
\end{tabular}%
}
\end{table}

As shown in Table~\ref{tab:trifinger}, EgoAgent achieves the highest success rates on both tasks, outperforming the best models from DoRA~\cite{venkataramanan2023imagenet} with increases of +3.32\% and +3.9\% respectively.
This result shows that by incorporating human action prediction into the learning process, EgoAgent demonstrates the ability to learn more effective representations that benefit both image classification and embodied manipulation tasks.
This highlights the potential of leveraging human-centric motion data to bridge the gap between visual understanding and actionable policy learning.



\section{Additional Results on Egocentric Future State Prediction}

In this section, we provide additional qualitative results on the egocentric future state prediction task. Additionally, we describe our approach to finetune video diffusion model on the Ego-Exo4D dataset~\cite{grauman2024ego} and generate future video frames conditioned on initial frames as shown in Figure~\ref{fig:opensora_finetune}.

\begin{figure}[b]
    \centering
    \includegraphics[width=\linewidth]{figures/opensora_finetune.pdf}
    \caption{Comparison of OpenSora V1.1 first-frame-conditioned video generation results before and after finetuning on Ego-Exo4D. Fine-tuning enhances temporal consistency, but the predicted pixel-space future states still exhibit errors, such as inaccuracies in the basketball's trajectory.}
    \label{fig:opensora_finetune}
\end{figure}

\subsection{Visualizations and Comparisons}

More visualizations of our method, DoRA, and OpenSora in different scenes (as shown in Figure~\ref{fig:supp pred}). For OpenSora, when predicting the states of $t_k$, we use all the ground truth frames from $t_{0}$ to $t_{k-1}$ as conditions. As OpenSora takes only past observations as input and neglects human motion, it performs well only when the human has relatively small motions (see top cases in Figure~\ref{fig:supp pred}), but can not adjust to large movements of the human body or quick viewpoint changes (see bottom cases in Figure~\ref{fig:supp pred}).

\begin{figure*}
    \centering
    \includegraphics[width=\linewidth]{figures/supp_pred.pdf}
    \caption{Retrieval and generation results for egocentric future state prediction. Correct and wrong retrieval images are marked with green and red boundaries, respectively.}
    \label{fig:supp pred}
\end{figure*}

\begin{figure*}[t]
    \centering
    \includegraphics[width=0.9\linewidth]{figures/motion_prediction.pdf}
    \vspace{-0.5mm}
    \caption{Motion prediction results in scenes with minor changes in observation.}
    \vspace{-1.5mm}
    \label{fig:motion_prediction}
\end{figure*}

\subsection{Finetuning OpenSora on Ego-Exo4D}

OpenSora V1.1~\cite{opensora}, initially trained on internet videos and images, produces severely inconsistent results when directly applied to infer future videos on the Ego-Exo4D dataset, as illustrated in Figure~\ref{fig:opensora_finetune}.
To address the gap between general internet content and egocentric video data, we fine-tune the official checkpoint on the Ego-Exo4D training set for 50 epochs.
OpenSora V1.1 proposed a random mask strategy during training to enable video generation by image and video conditioning. We adopted the default masking rate, which applies: 75\% with no masking, 2.5\% with random masking of 1 frame to 1/4 of the total frames, 2.5\% with masking at either the beginning or the end for 1 frame to 1/4 of the total frames, and 5\% with random masking spanning 1 frame to 1/4 of the total frames at both the beginning and the end.

As shown in Fig.~\ref{fig:opensora_finetune}, despite being trained on a large dataset, OpenSora struggles to generalize to the Ego-Exo4D dataset, producing future video frames with minimal consistency relative to the conditioning frame. While fine-tuning improves temporal consistency, the moving trajectories of objects like the basketball and soccer ball still deviate from realistic physical laws. Compared with our feature space prediction results, this suggests that training world models in a reconstructive latent space is more challenging than training them in a feature space.


\section{Additional Results on 3D Human Motion Prediction}

We present additional qualitative results for the 3D human motion prediction task, highlighting a particularly challenging scenario where egocentric observations exhibit minimal variation. This scenario poses significant difficulties for video-conditioned motion prediction, as the model must effectively capture and interpret subtle changes. As demonstrated in Fig.~\ref{fig:motion_prediction}, EgoAgent successfully generates accurate predictions that closely align with the ground truth motion, showcasing its ability to handle fine-grained temporal dynamics and nuanced contextual cues.

\section{OpenSora for Image Classification}

In this section, we detail the process of extracting features from OpenSora V1.1~\cite{opensora} (without fine-tuning) for an image classification task. Following the approach of~\cite{xiang2023denoising}, we leverage the insight that diffusion models can be interpreted as multi-level denoising autoencoders. These models inherently learn linearly separable representations within their intermediate layers, without relying on auxiliary encoders. The quality of the extracted features depends on both the layer depth and the noise level applied during extraction.


\begin{table}[h]
\centering
\caption{$k$-NN evaluation results of OpenSora V1.1 features from different layer depths and noising scales on ImageNet-100. Top1 and Top5 accuracy (\%) are reported.}
\label{tab:opensora-knn}
\resizebox{0.95\linewidth}{!}{%
\begin{tabular}{lcccccc}
\toprule
\multirow{2}{*}{Timesteps} & \multicolumn{2}{c}{First Layer} & \multicolumn{2}{c}{Middle Layer} & \multicolumn{2}{c}{Last Layer} \\
\cmidrule(r){2-3}   \cmidrule(r){4-5}  \cmidrule(r){6-7}  & Top1           & Top5           & Top1            & Top5           & Top1           & Top5          \\
\midrule
32        &  6.10           & 18.20             & 34.04               & 59.50             & 30.40             & 55.74             \\
64        & 6.12              & 18.48              & 36.04               & 61.84              & 31.80         & 57.06         \\
128       & 5.84             & 18.14             & 38.08               & 64.16              & 33.44       & 58.42 \\
256       & 5.60             & 16.58              & 30.34               & 56.38              &28.14          & 52.32        \\
512       & 3.66              & 11.70            & 6.24              & 17.62              & 7.24              & 19.44  \\ 
\bottomrule
\end{tabular}%
}
\end{table}

As shown in Table~\ref{tab:opensora-knn}, we first evaluate $k$-NN classification performance on the ImageNet-100 dataset using three intermediate layers and five different noise scales. We find that a noise timestep of 128 yields the best results, with the middle and last layers performing significantly better than the first layer.
We then test this optimal configuration on ImageNet-1K and find that the last layer with 128 noising timesteps achieves the best classification accuracy.

\section{Data Preprocess}
For egocentric video sequences, we utilize videos from the Ego-Exo4D~\cite{grauman2024ego} and WT~\cite{venkataramanan2023imagenet} datasets.
The original resolution of Ego-Exo4D videos is 1408×1408, captured at 30 fps. We sample one frame every five frames and use the original resolution to crop local views (224×224) for computing the self-supervised representation loss. For computing the prediction and action loss, the videos are downsampled to 224×224 resolution.
WT primarily consists of 4K videos (3840×2160) recorded at 60 or 30 fps. Similar to Ego-Exo4D, we use the original resolution and downsample the frame rate to 6 fps for representation loss computation.
As Ego-Exo4D employs fisheye cameras, we undistort the images to a pinhole camera model using the official Project Aria Tools to align them with the WT videos.

For motion sequences, the Ego-Exo4D dataset provides synchronized 3D motion annotations and camera extrinsic parameters for various tasks and scenes. While some annotations are manually labeled, others are automatically generated using 3D motion estimation algorithms from multiple exocentric views. To maximize data utility and maintain high-quality annotations, manual labels are prioritized wherever available, and automated annotations are used only when manual labels are absent.
Each pose is converted into the egocentric camera's coordinate system using transformation matrices derived from the camera extrinsics. These matrices also enable the computation of trajectory vectors for each frame in a sequence. Beyond the x, y, z coordinates, a visibility dimension is appended to account for keypoints invisible to all exocentric views. Finally, a sliding window approach segments sequences into fixed-size windows to serve as input for the model. Note that we do not downsample the frame rate of 3D motions.

\section{Training Details}
\subsection{Architecture Configurations}
In Table~\ref{tab:arch}, we provide detailed architecture configurations for EgoAgent following the scaling-up strategy of InternLM~\cite{team2023internlm}. To ensure the generalization, we do not modify the internal modules in InternML, \emph{i.e.}, we adopt the RMSNorm and 1D RoPE. We show that, without specific modules designed for vision tasks, EgoAgent can perform well on vision and action tasks.

\begin{table}[ht]
  \centering
  \caption{Architecture configurations of EgoAgent.}
  \resizebox{0.8\linewidth}{!}{%
    \begin{tabular}{lcc}
    \toprule
          & EgoAgent-300M & EgoAgent-1B \\
          \midrule
    Depth & 22    & 22 \\
    Embedding dim & 1024  & 2048 \\
    Number of heads & 8     & 16 \\
    MLP ratio &    8/3   & 8/3 \\
    $\#$param.  & 284M & 1.13B \\
    \bottomrule
    \end{tabular}%
    }
  \label{tab:arch}%
\end{table}%

Table~\ref{tab:io_structure} presents the detailed configuration of the embedding and prediction modules in EgoAgent, including the image projector ($\text{Proj}_i$), representation head/state prediction head ($\text{MLP}_i$), action projector ($\text{Proj}_a$) and action prediction head ($\text{MLP}_a$).
Note that the representation head and the state prediction head share the same architecture but have distinct weights.

\begin{table}[t]
\centering
\caption{Architecture of the embedding ($\text{Proj}_i$, $\text{Proj}_a$) and prediction ($\text{MLP}_i$, $\text{MLP}_a$) modules in EgoAgent. For details on module connections and functions, please refer to Fig.~2 in the main paper.}
\label{tab:io_structure}
\resizebox{\linewidth}{!}{%
\begin{tabular}{lcl}
\toprule
       & \multicolumn{1}{c}{Norm \& Activation} & \multicolumn{1}{c}{Output Shape}  \\
\midrule
\multicolumn{3}{l}{$\text{Proj}_i$ (\textit{Image projector})} \\
\midrule
Input image  & -          & 3$\times$224$\times$224 \\
Conv 2D (16$\times$16) & -       & Embedding dim$\times$14$\times$14    \\
\midrule
\multicolumn{3}{l}{$\text{MLP}_i$ (\textit{State prediction head} \& \textit{Representation head)}} \\
\midrule
Input embedding  & -          & Embedding dim \\
Linear & GELU       & 2048          \\
Linear & GELU       & 2048          \\
Linear & -          & 256           \\
Linear & -          & 65536     \\
\midrule
\multicolumn{3}{l}{$\text{Proj}_a$ (\textit{Action projector})} \\
\midrule
Input pose sequence  & -          & 4$\times$5$\times$17 \\
Conv 2D (5$\times$17) & LN, GELU   & Embedding dim$\times$1$\times$1    \\
\midrule
\multicolumn{3}{l}{$\text{MLP}_a$ (\textit{Action prediction head})} \\
\midrule
Input embedding  & -          & Embedding dim$\times$1$\times$1 \\
Linear & -          & 4$\times$5$\times$17     \\
\bottomrule
\end{tabular}%
}
\end{table}


\subsection{Training Configurations}
In Table~\ref{tab:training hyper}, we provide the detailed training hyper-parameters for experiments in the main manuscripts.

\begin{table}[ht]
  \centering
  \caption{Hyper-parameters for training EgoAgent.}
  \resizebox{0.86\linewidth}{!}{%
    \begin{tabular}{lc}
    \toprule
    Training Configuration & EgoAgent-300M/1B \\
    \midrule
    Training recipe: &  \\
    optimizer & AdamW~\cite{loshchilov2017decoupled} \\
    optimizer momentum & $\beta_1=0.9, \beta_2=0.999$ \\
    \midrule
    Learning hyper-parameters: &  \\
    base learning rate & 6.0E-04 \\
    learning rate schedule & cosine \\
    base weight decay & 0.04 \\
    end weight decay & 0.4 \\
    batch size & 1920 \\
    training iters & 72,000 \\
    lr warmup iters & 1,800 \\
    warmup schedule & linear \\
    gradient clip & 1.0 \\
    data type & float16 \\
    norm epsilon & 1.0E-06 \\
    \midrule
    EMA hyper-parameters: &  \\
    momentum & 0.996 \\
    \bottomrule
    \end{tabular}%
    }
  \label{tab:training hyper}%
\end{table}%

\clearpage


\end{document}
