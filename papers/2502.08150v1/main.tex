\def\isarxiv{1} %%% for icml submission version, we comment this line

\ifdefined\isarxiv
\documentclass[11pt]{article}
\usepackage[numbers]{natbib}

\else
% \documentclass{article}
\documentclass[sigconf,anonymous,review]{acmart}
\fi


\usepackage{amsmath}
\usepackage{amsthm}
\ifdefined\isarxiv %%%Zhao: I added the following lines
\usepackage{amssymb}
\else
\fi
\usepackage{algorithm}
\usepackage{subfig}
\usepackage{algpseudocode}
\usepackage{graphicx}
\usepackage{grffile}
\usepackage{wrapfig,epsfig}
\usepackage{url}
\usepackage{xcolor}
\usepackage{epstopdf}


\usepackage{bbm}
\usepackage{dsfont}
 
\allowdisplaybreaks
 

\ifdefined\isarxiv

\let\C\relax
\usepackage{tikz}
\usepackage{hyperref}  %%% arxiv don't allow this.
\hypersetup{colorlinks=true,citecolor=blue,linkcolor=blue} %%% Zhao : maybe we should comment this in submission.
\usetikzlibrary{arrows}
\usepackage[margin=1in]{geometry}

\else

\usepackage{microtype}
\usepackage{hyperref}
\definecolor{mydarkblue}{rgb}{0,0.08,0.45}
\hypersetup{colorlinks=true, citecolor=mydarkblue,linkcolor=mydarkblue}
 

\fi
 
\graphicspath{{./figs/}}

\theoremstyle{plain}
\newtheorem{theorem}{Theorem}[section]
\newtheorem{lemma}[theorem]{Lemma}
\newtheorem{definition}[theorem]{Definition}
\newtheorem{notation}[theorem]{Notation}
%\newtheorem{proof}[theorem]{Proof}
\newtheorem{proposition}[theorem]{Proposition}
\newtheorem{corollary}[theorem]{Corollary}
\newtheorem{conjecture}[theorem]{Conjecture}
\newtheorem{assumption}[theorem]{Assumption}
\newtheorem{observation}[theorem]{Observation}
\newtheorem{fact}[theorem]{Fact}
\newtheorem{remark}[theorem]{Remark}
\newtheorem{claim}[theorem]{Claim}
\newtheorem{example}[theorem]{Example}
\newtheorem{problem}[theorem]{Problem}
\newtheorem{open}[theorem]{Open Problem}
\newtheorem{property}[theorem]{Property}
\newtheorem{hypothesis}[theorem]{Hypothesis}

\newcommand{\wh}{\widehat}
\newcommand{\wt}{\widetilde}
\newcommand{\ov}{\overline}
\newcommand{\N}{\mathcal{N}}
\newcommand{\R}{\mathbb{R}}
\newcommand{\M}{\mathcal{M}}
\newcommand{\Q}{\mathcal{Q}}
\newcommand{\PP}{\mathcal{P}}
\newcommand{\KL}{\mathrm{KL}}
\newcommand{\RHS}{\mathrm{RHS}}
\newcommand{\LHS}{\mathrm{LHS}}
\renewcommand{\d}{\mathrm{d}}
\renewcommand{\i}{\mathbf{i}}
\renewcommand{\tilde}{\wt}
\renewcommand{\hat}{\wh}
\newcommand{\Tmat}{{\cal T}_{\mathrm{mat}}}
\newcommand{\True}{\mathrm{true}}


\DeclareMathOperator*{\E}{{\mathbb{E}}}
\DeclareMathOperator*{\var}{\mathrm{Var}}
\DeclareMathOperator*{\Z}{\mathbb{Z}}
\DeclareMathOperator*{\C}{\mathbb{C}}
\DeclareMathOperator*{\D}{\mathcal{D}}
\DeclareMathOperator*{\median}{median}
\DeclareMathOperator*{\mean}{mean}
\DeclareMathOperator{\OPT}{OPT}
\DeclareMathOperator{\supp}{supp}
\DeclareMathOperator{\poly}{poly}

\DeclareMathOperator{\nnz}{nnz}
\DeclareMathOperator{\sparsity}{sparsity}
\DeclareMathOperator{\rank}{rank}
\DeclareMathOperator{\diag}{diag}
\DeclareMathOperator{\dist}{dist}
\DeclareMathOperator{\cost}{cost}
\DeclareMathOperator{\vect}{vec}
\DeclareMathOperator{\tr}{tr}
\DeclareMathOperator{\dis}{dis}
\DeclareMathOperator{\cts}{cts}



\makeatletter
\newcommand*{\RN}[1]{\expandafter\@slowromancap\romannumeral #1@}
\makeatother
% \newcommand{\Zhao}[1]{{\color{red}[Zhao: #1]}}
% \newcommand{\Yang}[1]{{\color{orange}[Yang: #1]}}
% \newcommand{\Bo}[1]{{\color{blue}[Bo: #1]}}
% \newcommand{\Xiaoyu}[1]{{\color{purple}[Xiaoyu: #1]}}
% \newcommand{\Zhizhou}[1]{{\color{violet}[Zhizhou: #1]}}
% \newcommand{\Zhenmei}[1]{{\color{purple}[Zhenmei: #1]}}
% \newcommand{\Mingda}[1]{{\color{brown}[Mingda: #1]}}
% \newcommand{\InernNameB}[1]{{\color{blue}[InternNameB: #1]}} %%%Change to intern name


\usepackage{lineno}
% \def\linenumberfont{\normalfont\small}


\begin{document}

\ifdefined\isarxiv

\date{}

\title{Force Matching with Relativistic Constraints: A Physics-Inspired Approach to Stable and Efficient Generative Modeling}

% \author{}

% \iffalse
\author{
Yang Cao\thanks{\texttt{ ycao4@wyomingseminary.org}. Wyoming Seminary}
\and
Bo Chen\thanks{\texttt{ bc7b@mtmail.mtsu.edu}. Middle Tennessee State University.}
\and
Xiaoyu Li\thanks{\texttt{
xli216@stevens.edu}. Stevens Institute of Technology.}
\and
Yingyu Liang\thanks{\texttt{
yingyul@hku.hk}. The University of Hong Kong. \texttt{
yliang@cs.wisc.edu}. University of Wisconsin-Madison.} 
\and
Zhizhou Sha\thanks{\texttt{
shazz20@mails.tsinghua.edu.cn}. Tsinghua University.}
\and
Zhenmei Shi\thanks{\texttt{
zhmeishi@cs.wisc.edu}. University of Wisconsin-Madison.}
\and
Zhao Song\thanks{\texttt{ magic.linuxkde@gmail.com}. Simons Institute for the Theory of Computing, University of California, Berkeley.}
\and
Mingda Wan\thanks{\texttt{
dylan.r.mathison@gmail.com}. Anhui University.}
}
% \fi

\else

\title{Force Matching with Relativistic Constraints: A Physics-Inspired Approach to Stable and Efficient Generative Modeling}
% Newtonian and Relativistic Force Matching: Towards Stable and Efficient Generative Models
% Beyond Flow Matching: A Relativistic Force-Based Framework for Generative Modeling
% Force Matching for Generative Models: A Relativistic Approach to Stability and Efficiency
% Bridging Generative Modeling and Physics: Force Matching with Relativistic Constraints
% Force Matching: A Relativistic Approach to Stable and Efficient Generative Modeling
% Force Matching with Relativistic Constraints: Towards Stable andEfficient Cenerative Models

\author{%
}

\fi


\ifdefined\isarxiv
\begin{titlepage}
  \maketitle
  \begin{abstract}
\begin{abstract}

% Recent works to jointly reconstruct 3D human and object from a single RGB image, are mostly model-based, that fail to capture the fine details of the clothed human body and object surface. In this paper, we introduce ReCHOR, a novel, model-free, first-method to produce realistic clothed human-object reconstructions from a monocular view. This is extremely challenging due to human-object occlusions, diverse interactions and depth ambiguity, as it needs to infer both 3D spatial awareness and high resolution details. Our core idea is based on estimating neural implicit representations for human and object respectively by an attention-based neural implicit model that attends to pixel-aligned features from both the global human-object image for spatial awareness and  the local separate view of human and object images for high quality details. Additionally, the network is conditioned on semantic features from an initial estimated human-object pose prior and a generative diffusion model that inpaints occluded regions, thus enabling the retrieval of details from them.
% We also propose a synthetic dataset with rendered scenes of diverse, inter-occluded 3D human and object scans, to train our network. We evaluate our method on the synthetic and real world BEHAVE dataset. Our experiments show that our method outperforms the SOTA in achieving realistic clothed human-object reconstructions.
Recent approaches to jointly reconstruct 3D humans and objects from a single RGB image represent 3D shapes with template-based or coarse models, which fail to capture details of loose clothing on human bodies. In this paper, we introduce a novel implicit approach for jointly reconstructing realistic 3D clothed humans and objects from a monocular view. For the first time, we model both the human and the object with an implicit representation, allowing to capture more realistic details such as clothing. This task is extremely challenging due to human-object occlusions and the lack of 3D information in 2D images, often leading to poor detail reconstruction and depth ambiguity. To address these problems, we propose a novel attention-based neural implicit model that leverages image pixel alignment from both the input human-object image for a global understanding of the human-object scene and from local separate views of the human and object images to improve realism with, for example, clothing details. Additionally, the network is conditioned on semantic features derived from an estimated human-object pose prior, which provides 3D spatial information about the shared space of humans and objects. To handle human occlusion caused by objects, we use a generative diffusion model that inpaints the occluded regions, recovering otherwise lost details. For training and evaluation, we introduce a synthetic dataset featuring rendered scenes of inter-occluded 3D human scans and diverse objects. Extensive evaluation on both synthetic and real-world datasets demonstrates the superior quality of the proposed human-object reconstructions over competitive methods.
\end{abstract}

  \end{abstract}
  \thispagestyle{empty}
\end{titlepage}

{\hypersetup{linkcolor=black}
\tableofcontents
}
\newpage

\else

\begin{abstract}
\begin{abstract}

% Recent works to jointly reconstruct 3D human and object from a single RGB image, are mostly model-based, that fail to capture the fine details of the clothed human body and object surface. In this paper, we introduce ReCHOR, a novel, model-free, first-method to produce realistic clothed human-object reconstructions from a monocular view. This is extremely challenging due to human-object occlusions, diverse interactions and depth ambiguity, as it needs to infer both 3D spatial awareness and high resolution details. Our core idea is based on estimating neural implicit representations for human and object respectively by an attention-based neural implicit model that attends to pixel-aligned features from both the global human-object image for spatial awareness and  the local separate view of human and object images for high quality details. Additionally, the network is conditioned on semantic features from an initial estimated human-object pose prior and a generative diffusion model that inpaints occluded regions, thus enabling the retrieval of details from them.
% We also propose a synthetic dataset with rendered scenes of diverse, inter-occluded 3D human and object scans, to train our network. We evaluate our method on the synthetic and real world BEHAVE dataset. Our experiments show that our method outperforms the SOTA in achieving realistic clothed human-object reconstructions.
Recent approaches to jointly reconstruct 3D humans and objects from a single RGB image represent 3D shapes with template-based or coarse models, which fail to capture details of loose clothing on human bodies. In this paper, we introduce a novel implicit approach for jointly reconstructing realistic 3D clothed humans and objects from a monocular view. For the first time, we model both the human and the object with an implicit representation, allowing to capture more realistic details such as clothing. This task is extremely challenging due to human-object occlusions and the lack of 3D information in 2D images, often leading to poor detail reconstruction and depth ambiguity. To address these problems, we propose a novel attention-based neural implicit model that leverages image pixel alignment from both the input human-object image for a global understanding of the human-object scene and from local separate views of the human and object images to improve realism with, for example, clothing details. Additionally, the network is conditioned on semantic features derived from an estimated human-object pose prior, which provides 3D spatial information about the shared space of humans and objects. To handle human occlusion caused by objects, we use a generative diffusion model that inpaints the occluded regions, recovering otherwise lost details. For training and evaluation, we introduce a synthetic dataset featuring rendered scenes of inter-occluded 3D human scans and diverse objects. Extensive evaluation on both synthetic and real-world datasets demonstrates the superior quality of the proposed human-object reconstructions over competitive methods.
\end{abstract}
\end{abstract}

\begin{CCSXML}
<ccs2012>
   <concept>
       <concept_id>10010147.10010257</concept_id>
       <concept_desc>Computing methodologies~Machine learning</concept_desc>
       <concept_significance>500</concept_significance>
       </concept>
   <concept>
       <concept_id>10010147.10010178</concept_id>
       <concept_desc>Computing methodologies~Artificial intelligence</concept_desc>
       <concept_significance>500</concept_significance>
       </concept>
 </ccs2012>
\end{CCSXML}

\ccsdesc[500]{Computing methodologies~Machine learning}
\ccsdesc[500]{Computing methodologies~Artificial intelligence}


% \begin{CCSXML}
% <ccs2012>
%  <concept>
%   <concept_id>00000000.0000000.0000000</concept_id>
%   <concept_desc>Do Not Use This Code, Generate the Correct Terms for Your Paper</concept_desc>
%   <concept_significance>500</concept_significance>
%  </concept>
%  <concept>
%   <concept_id>00000000.00000000.00000000</concept_id>
%   <concept_desc>Do Not Use This Code, Generate the Correct Terms for Your Paper</concept_desc>
%   <concept_significance>300</concept_significance>
%  </concept>
%  <concept>
%   <concept_id>00000000.00000000.00000000</concept_id>
%   <concept_desc>Do Not Use This Code, Generate the Correct Terms for Your Paper</concept_desc>
%   <concept_significance>100</concept_significance>
%  </concept>
%  <concept>
%   <concept_id>00000000.00000000.00000000</concept_id>
%   <concept_desc>Do Not Use This Code, Generate the Correct Terms for Your Paper</concept_desc>
%   <concept_significance>100</concept_significance>
%  </concept>
% </ccs2012>
% \end{CCSXML}

% \ccsdesc[500]{Do Not Use This Code~Generate the Correct Terms for Your Paper}
% \ccsdesc[300]{Do Not Use This Code~Generate the Correct Terms for Your Paper}
% \ccsdesc{Do Not Use This Code~Generate the Correct Terms for Your Paper}
% \ccsdesc[100]{Do Not Use This Code~Generate the Correct Terms for Your Paper}

%%
%% Keywords. The author(s) should pick words that accurately describe
%% the work being presented. Separate the keywords with commas.
\keywords{Flow Matching, Diffusion Model, Generative Model}
%% A "teaser" image appears between the author and affiliation
%% information and the body of the document, and typically spans the
%% page.
\maketitle % Bo: error in KDD change template: abstract must be defined before maketitle command, so I move \maketitle after abstract.
\fi


\section{Introduction}
\label{sec:intro}
% Image editing methods in diffusion models depend on user-defined control directions - users can unlock their creativity using these methods by specifying the desired manipulation through prompts~\cite{gandikota2023concept}, reference images~\cite{ruiz2022dreambooth, kumari2022customdiffusion, gal2022image, chen2024trainingfreeregionalpromptingdiffusion}, or attribute vectors~\cite{parmar2023zero,hertz2022prompt}. In this work, we ask a fundamentally different question: \emph{Can we automatically discover the underlying visual structure of a concept within diffusion model's knowledge?} %Rather than requiring user-specified controls, we aim to decompose the model's internal knowledge into meaningful directions.

% This question touches on a fundamental limitation in how we interact with diffusion models. Current control methods ~\cite{zhang2023addingconditionalcontroltexttoimage, gandikota2023concept, ye2023ipadaptertextcompatibleimage,ye2023ipadaptertextcompatibleimage, hertz2024stylealignedimagegeneration, li2023photomaker, shi2024instantbooth, chen2024trainingfreeregionalpromptingdiffusion} require users to specify their desired manipulations in advance, limiting interactive creativity. This contrasts with natural human artistic workflows, where creators dynamically explore creative ideas while jointly refining them toward meaningful artistic outcomes~\cite{hoffmann2016modeling}. This synergy between specification and exploration is not new to generative models. Early GAN architectures naturally developed disentangled latent spaces that enabled continuous\cite{harkonen2020ganspace,radford2015unsupervised, wu2021stylespace, shen2020interfacegan}, compositional control over generated images. Users could explore these spaces to discover interesting variations that would be difficult to describe in words~\cite{wu2021stylespace}, then combine them to achieve their creative goals~\cite{grabe2022towards}. 


% While diffusion models have largely superseded GANs in conditional image synthesis~\cite{dhariwal2021diffusion},  their underlying structure remains less understood. Diffusion models achieve remarkable diversity through high-dimensional latents, unlike GANs' compact latent spaces.  With a single prompt, diffusion models can generate radically different variations through different random initializations of input noise. We ask - Is it possible to discover interpretable structure within this vast space of variations?

Text-to-image diffusion models are capable of generating remarkable visual variations from a single prompt through different random initializations. However, this vast creative potential remains largely opaque to users---while we can generate diverse images, we lack understanding of the underlying structure of these variations. This presents a fundamental challenge: how can we discover and expose the latent visual capabilities encoded within these models?

\let\thefootnote\relax \footnote{$^{*}$Correspondence to \texttt{gandikota.ro@northeastern.edu}}

The challenge touches on a key limitation in how we interact with diffusion models today. Current control methods require users to explicitly specify their desired edits in advance through prompts~\cite{gandikota2023concept}, reference images~\cite{zhang2023addingconditionalcontroltexttoimage, chen2024trainingfreeregionalpromptingdiffusion, ruiz2022dreambooth,kumari2022customdiffusion, Ryu_lora, hu2021lora}, or attribute vectors~\cite{ye2023ipadaptertextcompatibleimage, hertz2024stylealignedimagegeneration, li2023photomaker, shi2024instantbooth,parmar2023zero,hertz2022prompt}. That contrasts sharply with natural human creative workflows, where artists dynamically explore creative ideas and jointly refine them toward meaningful artistic outcomes~\cite{hoffmann2016modeling}. The need for pre-specified controls creates a barrier between users and the full creative potential of these models.

Interestingly, earlier generative models like GANs~\cite{gans,karras2019style,brock2018large} naturally developed more interpretable internal structures. Their compact latent spaces often exhibited emergent disentanglement~\cite{harkonen2020ganspace,radford2015unsupervised, wu2021stylespace, shen2020interfacegan}, enabling continuous and compositional control over generated images. Users could explore these spaces to discover interesting variations that would be difficult to describe in words~\cite{wu2021stylespace}, then combine them to achieve their creative goals~\cite{grabe2022towards}.

Diffusion models have largely superseded GANs in conditional image synthesis~\cite{dhariwal2021diffusion}, achieving greater diversity through much higher-dimensional latents. And yet an understanding of the underlying structure of these larger latent spaces has remained elusive. In this work, we ask a fundamental question: \emph{Can we automatically discover the visual structure within a diffusion model's knowledge of a concept?} Rather than requiring user-specified controls, we aim to decompose the model's internal representations into expressive directions that users can explore and combine.

To address these needs, we present \textbf{SliderSpace}, a framework that brings systematic explorability to diffusion models. Given just a text prompt, SliderSpace discovers a canonical set of meaningful, diverse, and controllable directions within the model's knowledge of that concept. Each direction is implemented as a low-rank adapter~\cite{hu2021lora} that can be scaled and composed with others, allowing users to explore and smoothly combine different aspects of variation, as shown in Figure~\ref{fig:intro}.

We ground SliderSpace discovery in three key requirements for meaningful decomposition of a diffusion model's visual manifold: 
\begin{enumerate}
    \item \textbf{Unsupervised Discovery:} The decomposition process should emerge from the intrinsic structure of the model's learned representation, rather than being guided by predefined attributes. This ensures we capture the true topology of the model's knowledge space rather than projecting our assumptions onto it.
    
    \item \textbf{Semantic Orthogonality:} Each discovered control must represent a distinct semantic direction. This is enforced in a semantic feature space, like CLIP, where every slider has an orthogonal effect in embeddings. This prevents discovering multiple controls that create similar semantic effects, making the system more efficient and easier.
    
    \item \textbf{Distribution Consistency:} Directions must induce consistent transformations across both random seeds and prompt variations. 
\end{enumerate}

These requirements naturally lead to our proposed framework, which we formalize in Section~\ref{sec:method}. As we show in our experiments, SliderSpace is architecture-agnostic, working with both conventional U-Net based models like Stable Diffusion~\cite{rombach2022high, rombach2022sd20, podell2023sdxl, turbo, dmd} and recent transformer-based architectures like Flux~\cite{flux}.

We demonstrate the expressiveness of SliderSpace through three applications: First, we show how SliderSpace can decompose high-level concepts into diverse and expressive components, revealing the natural axes of variation in the model's understanding. Second, we explore artistic style variation, where SliderSpace discovers directions that match or exceed the diversity of manually curated artist lists while being judged more useful by human evaluators. Finally, we show how SliderSpace can help reverse the mode collapse commonly observed in distilled diffusion models, restoring diversity while maintaining generation speed.

Beyond providing practical creative control, SliderSpace opens new avenues for understanding and utilizing the latent capabilities of diffusion models. By mapping these models' visual potential into intuitive, composable directions, we take a step toward making their creative possibilities more accessible and interpretable to users.

% Image editing methods in diffusion models unlock the creativity of users. In this work we ask an alternate question: \emph{Can we organize and expose what of the diffusion model is already capable of?}.
% Existing methods for controlling image generation typically require users to manually specify edit directions for desired changes. This process is time-consuming, requires technical expertise, and limits the spontaneity of the creative process. For instance, if a user wants to adjust the smile of a generated person, they must explicitly request this edit, often through imprecise prompt engineering or model fine-tuning. This approach of predefined controls or manual specifications restricts users from fully exploring the latent capabilities of the model. There may be interesting stylistic variations or attributes that the model can generate, but users have no easy way to discover or utilize these.

% Natural visual disentanglement was an emergent property in the latent space of Generative Adversarial Models (GANs) \cite{harkonen2020ganspace,radford2015unsupervised, wu2021stylespace, shen2020interfacegan}. In particular, it has been observed that StyleGAN~\cite{karras2019style} stylespace neurons offer detailed control over many meaningful aspects of images that would be difficult to describe in words~\cite{wu2021stylespace}. However, diffusion models do not share such a compact latent space~\cite{park2023unsupervised}; and efforts to uncover such a space in the semantic embeddings of the text conditioning have met with limited success \nik{Nick - is there a specific citation you were thinking about?}.

% In this work we introduce \textbf{SliderSpace}, which takes a step towards uncovering an analogous low dimensional representation of diffusion models' visual breadth; in essence treating the diffusion model as many generators sharing parameters, where a particular generator is defined by a specific prompt. For a given prompt we sample many random seeds (and optionally prompt expansions using an LLM), generate the corresponding images, and apply an off the shelf feature extractor (in this work CLIP, but our method can be applied to any differentiable feature extractor). We use PCA to analyze these features, and for each of the leading $k$ principal components we train a LoRA \cite{} which causes the diffusion model to produces images which increase the feature magnitude along that component when passed back through the same feature extractor. This leads to a 'Slider' for each principal component, because each LoRA can be scaled and applied to the original diffusion model, continuously varying those visual features in the generated results (as measured, in our case, by CLIP).

% There are many other works that enhance the controllability of diffusion models. One common approach is enabling users to add spatial constraints to a generation either manually, or via a reference image \cite{zhang2023addingconditionalcontroltexttoimage, chen2024trainingfreeregionalpromptingdiffusion}, a second is leveraging more abstract embeddings (e.g. identity, style) extracted from a reference image \cite{ye2023ipadaptertextcompatibleimage, hertz2024stylealignedimagegeneration, li2023photomaker, shi2024instantbooth}, a third is finetuning a foundation model to better generate a concept important to the user \cite{ruiz2022dreambooth, kumari2022customdiffusion, Ryu_lora, hu2021lora}, and a fourth (most relevant to this work) is finding low-rank adaptors of the model based on a prompt or small training set which can be scaled to provide continous control over one aspect of generated image (e.g. night vs day, basic vs luxury, etc.) \cite{gandikota2023concept}. SliderSpace is complementary to all of these methods and offers something distinct. All of the other methods we are aware require the user (and / or model designer) to know in advance what type of control they want. In contrast SliderSpace assists users in discovering and controlling hidden capabilities present in the diffusion model's distribution of possible generations.

%We propose that truly intuitive creative control in a text-to-image model should meet three key criteria: \emph{discoverability}, \emph{intuitiveness}, and \emph{specificity}. The model should reveal controllable attributes that may not be immediately obvious, offer controls that are easy to understand and manipulate, and ensure each control affects a distinct attribute of the generated image.

% We demonstrate the utility and power of SliderSpace using three applications built on top of SDXL-DMD \cite{dmd}, because its fast generation speed lends itself well to the continuous control offered by SliderSpace.

% First, we study concept decomposition (Section \ref{sec:concept_exp}), where we learn sliders for a specific concept (e.g. 'monster', 'waterfall', 'car'). Through quantitative metrics of diversity and text alignment we demonstrate that the learned sliders dramatically boost the diversity of generations when randomly applied without harming text alignment; we also ask humans to qualitatively judge these results in a user study where they find the SliderSpace results to be more 'Diverse', 'Useful', and 'Creative' than our baselines.

% Second, we attempt to compare the automatic discoveries of SliderSpace to a large scale manual study of artistic styles (Section \ref{sec:art_exp}), open-sourced by ParrotZone \cite{parrotzone}. In this study SDXL was prompted with over 4300 artist names,  and based on visual inspection the cases of successful stylistic mimicry recorded. Quantitatively SliderSpace more closely matches the distribution of artistic variation discovered by ParrotZone than other baselines, and in our user studies was judged to be significantly more 'Diverse' and 'Useful' than the baselines. To our surprise humans even judged SliderSpace results to be slightly more 'Diverse' than the results generated by the manually discovered artist names of \cite{parrotzone}.

% Third, we attempt to use SliderSpace to reverse the mode collapse commonly observed in distilled few-step diffusion models relative to the original teacher model (Section \ref{sec:diverse_exp}). We quantitatively demonstrate that applying SliderSpace to SDXL-DMD leads to more closely matching the distribution of images by the original teacher, SDXL.

%Through extensive experiments on various state-of-the-art text-to-image models, we demonstrate that SliderSpace significantly enhances user control and creative expression in AI-assisted image generation tasks. Our method enables a range of applications, including concept decomposition and control, diversity improvement in generated images, customization dissection and edits, and the exploration of artistic styles inherent in the model.

% SliderSpace goes beyond providing a practical tool for enhanced creative control. By mapping the visual potential of diffusion models it can open new avenues for generative creativity and deepens our understanding of each model's hidden potential. %%% Section 1. Introduction
\section{Related Work}
\label{sec:related_work}

The original investigation \cite{gibson1979ecological} on the relationship between visual perception and human action defines \emph{affordance} as the opportunities for interaction with the surrounding environment. Behavioral studies on regular and cognitively impaired persons have shown evidence that perception results in both visual and motor signals in the human brain. An extended study \cite{anderson2002attentional} shows that visual attention to the spatial characteristics of the perceived objects initiates automatic motor signals for different actions. In computer vision, human affordance learning involves novel pose prediction such that the estimated pose represents a valid human action within the scene context. The task is fundamental to many problems requiring robust semantic reasoning about the environment, such as human motion synthesis \cite{wang2021scene} and scene-aware human pose generation \cite{wang2017binge, roy2016multi, zhang2022inpaint, yao2023scene}.

Earlier methods of affordance learning have explored knowledge mining \cite{zhu2014reasoning} and multimodal feature cues \cite{roy2016multi} to address the problem. In \cite{zhu2014reasoning}, the authors use a Markov Logic Network for constructing a knowledge base by extracting several object attributes from different image and metadata sources, which can perform various downstream visual inference tasks without any additional classifier, including zero-shot affordance prediction. In \cite{roy2016multi}, the authors use depth map, surface normals, and segmentation map as multimodal cues to train a multi-scale convolutional neural network (CNN) for scene-level semantic label assignment associated with specific human actions. In \cite{do2018affordancenet}, the authors design a multi-branch end-to-end CNN with two separate pathways for object detection and affordance label assignment to achieve high real-time inference throughput. Researchers \cite{chuang2018learning} have also explored socially imposed constraints for affordance learning. In \cite{chuang2018learning}, the authors propose a graph neural network (GNN) to propagate contextual scene information from egocentric views for action-object affordance reasoning.

Probabilistic modeling of scene-aware human motion generation also involves semantic reasoning of human interaction with the environment. Initial works on human motion synthesis have taken different architectural approaches, such as sequence-to-sequence models \cite{barsoum2018hp}, generative adversarial networks (GAN) \cite{barsoum2018hp, cai2018deep, yang2018pose}, graph convolutional networks (GCN) \cite{yan2019convolutional}, and variational autoencoders (VAE) \cite{guo2020action2motion}. However, these methods have mostly ignored the role of environmental semantics. Due to potential uncertainty in human motion, in a recent approach \cite{wang2021scene}, the authors address such motion synthesis with a GAN conditioned on scene attributes and motion trajectory to predict probable body pose dynamics.

One key challenge of human affordance generation in 2D scenes is the lack of large-scale datasets with rich pose annotations. In \cite{wang2017binge}, the authors compile the only public dataset of annotated human body poses in complex 2D indoor scenes by extracting frames from sitcom videos. Aiming to generate a contextually valid human affordance at a user-defined location, the authors propose sampling the scale and deformation parameters for an existing human pose template using a VAE conditioned on the localized image patches as scene context. In \cite{zhang2022inpaint}, the authors introduce a two-stage GAN architecture for achieving a similar goal by estimating the affine bounding box parameters to localize a probable human in the scene and then generating a potential body pose at that location. The method uses the input scene, corresponding depth, and segmentation maps as semantic guidance. In \cite{yao2023scene}, the authors propose a transformer-based approach with knowledge distillation for generating human affordances in 2D indoor scenes.


\section{Preliminary} \label{sec:preli}

In Section~\ref{sub:notation}, we introduce all the notations we used in our paper. Then, in Section~\ref{sub:flow_matching}, we show the basic facts about flow matching. In Section~\ref{sub:special_relativity}, we present the basic background of special relativity and define the relativistic force.

\subsection{Notations} \label{sub:notation}

For any positive integer $n$, we use $[n]$ to denote set $\{1,2,\cdots, n\}$. 
For two vectors $x \in \R^n$ and $y \in \R^n$, we use $\langle x, y \rangle$ to denote the inner product between $x,y$.
For a vector $v \in \R^n$, we use $\|v\|_2$ to denote the $\ell_2$-norm of $v$.
We use ${\bf 1}_n$ to denote a length-$n$ vector where all the entries are ones.
We use the symbol $ \perp $ to represent a component that is perpendicular to the direction of velocity, as exemplified by $ a_{\perp t} $, which denotes the perpendicular acceleration. Similarly, the symbol $ \parallel $ is employed to indicate a component parallel to the direction of velocity, such as $ f_{\parallel t} $, which represents the parallel force. We use $\dot{x}_t$ to denote $\frac{\d x_t}{\d t}$, and $\ddot{x}_t$ to denote $\frac{\d^2 x_t}{\d t^2}$.


\subsection{Flow Matching} \label{sub:flow_matching}

Flow Matching (FM) \cite{lcb+22,lgl22} is a generative modeling technique that constructs a smooth, invertible (i.e., diffeomorphic) mapping from a simple prior distribution to a complex target distribution. In FM, a time-dependent mapping $Z_t$ is defined to evolve according to an ordinary differential equation (ODE) driven by a vector field:
\begin{align*}
    \frac{\d x_t}{\d t} = V_t(x_t), \quad t \in [0, T].
\end{align*}
The goal is to ensure that, at the terminal time $T$, the ODE transforms a sample $x_0$ from a simple distribution (e.g., a Gaussian) into a sample $x_T$ from the target data distribution $\mathcal{D}$.

To achieve this, Flow Matching (FM) constructs a stochastic interpolation between a sample $x_1 \sim \mathcal{D}$ and a sample $x_0$ drawn from a known prior distribution, typically $\N(0,I)$. The interpolation is defined as
\begin{align*}
    x_t := \alpha_t x_1 + \sigma_t x_0, \quad t\in [0,T],
\end{align*}
where the time-dependent coefficients $\alpha_t$ and $\sigma_t$ are chosen so that
\begin{align*}
    \alpha_0 = 0,\quad \sigma_0 = 1,\quad \alpha_T = 1,\quad \sigma_T = 0.
\end{align*}
Thus, at $t=0$ the interpolated sample is purely the prior ($x_0$), and at $t=T$ it becomes a data sample ($x_1$).

The instantaneous change of $x$ is obtained by differentiating the interpolation:
\begin{align*}
    \frac{\d x_t}{\d t} = \frac{\d \alpha_t}{\d t} x_1 + \frac{\d \sigma_t}{\d t} x_0.
\end{align*}

The vector field is approximated by a neural network $V_t(x_t)$ with learnable parameters $\theta$. The FM training objective is then given by
\begin{align*}
    \mathcal{L}_\mathrm{FM}(\theta) := \E_{t\sim {\sf Uniform}[0,T], x_1 \sim \mathcal{D}} [\| V_t(x_t) - v_t(x_t) \|_2^2 ].
\end{align*}
This loss ensures that the learned velocity field $V_t(x_t)$ closely tracks the conditional dynamics $v_t(x_t)$ along the interpolation path.

After training, samples are generated by solving the ODE
\begin{align*}
    \frac{\d x_t}{\d t} = V_t(x_t),
\end{align*}
starting from an initial sample $x_0 \sim \N(0,I)$. Integrating this ODE from $t=0$ to $t=T$ yields a sample $x_T$ that approximates a draw from the target distribution. This ODE-based formulation offers a flexible and powerful framework for modeling complex data distributions while naturally incorporating conditional sampling.

\subsection{Background on Special Relativity} \label{sub:special_relativity}

We first introduce several essential ideas of special relativity \cite{e+05}.

\begin{definition}[Lorentz Factor]
\label{def:LorentzFactor}
According to special relativity~\cite{e+05}, the Lorentz factor at lab time $t$ is given by
\begin{align*}
\gamma_t := \frac{1}{\sqrt{1 - {\|v_t^{\rm lab}\|_2^2}/{c^2}}},
\end{align*}
where $v_t^{\rm lab}$ is the velocity at lab frame of reference, $c = 3 \times 10^8$ is the speed of light in vacuum.
\end{definition}

Then, we introduce the proper time of special relativity.

\begin{definition}[Proper Time]
\label{def:ProperTime}
The proper time is defined as the time interval measured in the rest frame of a moving object according to special relativity~\cite{e+05}. The differential form of the proper time is given by
\begin{align*}
    \d \tau = \frac{\d t}{\gamma_t},
\end{align*}
where $\d t$ is the time interval in the laboratory frame of reference, and $\gamma_t$ is the Lorentz factor at time lab time $t$ as defined in Definition~\ref{def:LorentzFactor}.
\end{definition}

Next, we define the force under special relativity here.

\begin{definition}[Relativistic Force]
\label{def:RelativisticForce}
In the framework of special relativity, the \emph{local force} (i.e., the force measured in the instantaneous rest frame of the particle) denoted as $f^{\rm local}$ has
\begin{align}
    f^{\rm local} := \frac{\d p^{\rm lab}}{\d \tau}, \label{eq:f_local}
\end{align}
where $p^{\rm lab}$ is the momentum at lab frame of reference, $\tau$ denotes the proper time defined in Definition~\ref{def:ProperTime}.

The momentum in the lab frame is defined as
\begin{align}
    p^{\rm lab} := m^{\rm lab} v_t^{\rm lab}, \label{eq:p}
\end{align}
where $m^{\rm lab}$ is the mass at lab frame of reference, and $v_t^{\rm lab}$ is the velocity at lab frame of reference.
\end{definition}

We state an equivalence lemma. Due to the space limitation, we delayed the proofs into the appendix.
\begin{lemma}[Equivalent Form of Relativistic Force, informal version of Lemma~\ref{lem:equiv_relativistic_force:formal}]\label{lem:equiv_relativistic_force:informal}
Let $p^{\rm lab}$ be the momentum defined in Eq.~\eqref{eq:p}, $\gamma_t$ be the Lorentz factor at lab time $t$ defined in Definition~\ref{def:LorentzFactor}, $\tau$ denotes the proper time, $v_t^{\rm lab} = \dot{x}_t$ denotes the velocity, 
$a_t^{\rm lab} = \ddot{x}_t$ denotes the acceleration.
The relativistic force, defined as the time derivative of the momentum in the lab frame, can be written as
\begin{align*}
f^{\rm local} =  m^{\rm lab}  (\gamma_t a_t^{\rm lab} + \gamma_t^3 \frac{ \langle v_t^{\rm lab}, a_t^{\rm lab} \rangle}{c^2} v_t^{\rm lab}).
\end{align*}

\end{lemma} 
\section{Force Matching} \label{sec:form}

In this section, we introduce Force Matching (ForM), a new architecture for generative models, and provide its theoretical analysis. In Section~\ref{sub:obj}, we introduce the training objective of Force Matching. Then, in Section~\ref{sub:samp_ode}. In Section~\ref{sub:speed_limit}, we illustrate and discuss the speed limitation of ForM. In Section~\ref{sub:form_trig}, We show the interpolation path for ForM.


\subsection{Definition of Force Matching Objective} \label{sub:obj}

Next, we define the training objective of Force Matching.

\begin{definition}[Force Matching Objective]
\label{def:FormObjective}
The training objective of Force Matching (ForM) is defined by
\begin{align*}
     \mathcal{L}_{\rm ForM}(\theta) := \E_{t \sim {\sf Uniform}[0,T], x_1 \sim \mathcal{D}} 
    [\| F_t(x_t) - f_t(x_t)\|_2^2],   
\end{align*}
where $\mathcal{D}$ is the target data distribution, $f_t(x_t)$ is the target relativistic force defined in Definition~\ref{def:RelativisticForce}, and $F_t(x)$ is a trainable neural network parameterized with $\theta$.
\end{definition}

\subsection{Our Result I: Sampling ODE} \label{sub:samp_ode}

We define an ordinary differential equation (ODE) in order to get the position based on a given relativistic force. 

\begin{theorem}[Sampling ODE, informal version of Theorem~\ref{thm:ode_form:formal}]\label{thm:ode_form:informal}
    Giving the force at position $x_t$ denoted as $f_t(x_t)$, we could solve for ForM sampling path $x_t$ by the following ODE
    \begin{align*}
    \ddot{x}_t = \frac{1}{m^{\rm lab} \gamma_t}(f_t^{\rm local} - \frac{\langle v_t^{\rm lab}, f_t^{\rm local} \rangle}{c^2} v_t^{\rm lab}),
    \end{align*}
    where $x_0 \sim \N(0,I)$, $\dot{x}_0 = 0$.
\end{theorem}

Theorem~\ref{thm:ode_form:informal} shows how to derive the position $x_t$ from the relativistic force field $f_t(x_t)$. Unlike first-order flow-based methods, ForM naturally involves a second-order ODE because it encodes the evolution of both position and velocity under relativistic constraints. This allows for more expressive and physically-motivated sampling trajectories, where velocity constraints can help stabilize the generative process. Once a neural network $F_t(x)$ is trained to approximate $f_t(x)$, the sampling procedure integrates this second-order ODE to produce samples consistent with the target distribution.


\subsection{Our Result II: Speed Limit} \label{sub:speed_limit}

One property of relativistic mechanics is the velocity will always be under the constant $c$, which is the speed of light.

In reality, the speed of light is $c \approx 3 \times 10^8$. For any $v_t$, the speed $ \|v_t\|_2$ can approach but never exceed $c$. 
This property stabilizes the generating process. We formalize and prove this in Theorem~\ref{thm:vel:informal}.

\begin{theorem}[Speed Limit, informal version of Theorem~\ref{thm:vel:formal}] \label{thm:vel:informal}
For a ForM model with sampling path $x : [0,T) \to \R^n$, the velocity satisfies
\begin{align*}
\| \dot{x}_t \|_2 < c \quad, \forall t \in [0,T).
\end{align*}
\end{theorem}


It shows that the sample velocity remains strictly below $c$ at all times under relativistic constraints. Practically, this upper bound on velocity helps mitigate risks of numerical instability or ``exploding gradients" that can sometimes arise in diffusion- or flow-based generative models. By capping the speed of samples, ForM maintains a controlled and stable evolution in high-dimensional spaces. This provides a theoretical guarantee of safety against runaway behaviors, making the sampling process more robust.



\subsection{Our Result III: ForM with TrigFlow} \label{sub:form_trig}

The interpolation path of ForM is given by the following theorem.

\begin{theorem}[ForM with TrigFlow, informal version of Theorem~\ref{thm:form_trig:formal}] \label{thm:form_trig:informal}
We let $m = 1$ for simplicity in ForM. Giving a the interpolation $x_t = \alpha_t x_1 + \sigma_t x_0$, where $\alpha_T = 1$, $\alpha_0 = 0$, $\sigma_T = 0$, $\sigma_0 = 1$. We let $F_t(x_t)$ denote a vector map of force, a trainable neuron network parameterized with $\theta$. We select the $\alpha_t$ and $\sigma_t$ identical with TrigFlow \cite{ls24}, where $\alpha_t = \sin(t)$ and $\sigma_t = \cos(t)$, $T = \frac{\pi}{2}$. Then, force interpolation could be simplified to 
\begin{align*}
    f_t(x_t) =
    & ~ \frac{(\cos(t)x_1 - \sin(t)x_0) \cdot (-\sin(t)x_1 - \cos(t)x_0)}{c^2 - (\cos(t)x_1 - \sin(t)x_0)^2}\\
    & ~ (\cos(t)x_1 - \sin(t)x_0)).
\end{align*}
\end{theorem}

Theorem~\ref{thm:form_trig:informal} highlights how the ForM framework can be directly coupled with trigonometric interpolation paths. By choosing $\alpha_t$ and $\sigma_t$ as sine and cosine, respectively, we obtain a closed-form expression for the relativistic force that governs the sample evolution. This synergy suggests that ForM can not only unify different flow-based or diffusion-based models but also inherit the continuous-time advantages of the TrigFlow parameterization. Consequently, one can design more flexible interpolation strategies while still enjoying the stability benefits of relativistic velocity constraints.



\section{Experiment}
In this section, we conduct extensive experiments to evaluate the performance of various LLMs on our Hellaswag-Pro benchmark. Our study is guided by three key research questions:
\textbf{RQ1}: How do different LLMs perform across all variants?
\textbf{RQ2}: What is the relative difficulty of different variants?
\textbf{RQ3}: How robust are LLMs to diverse prompts during evaluation?

\subsection{Experiment Setup} 
\subsubsection{Model Selection and Implementation Details}
We select 41 representative commercial and open-source models, including English LLMs, such as GPT-4o, Claude-3.5-Sonnet, Gemini-1.5-Pro,Mistral series, Llama3 series and Chinese LLMs, like Qwen-Max,  Qwen2.5 series, InternLM-2.5 series, Yi-1.5 series, Baichuan-2 series and DeepSeek series.

We integrate both Chinese HellaSwag and HellaSwagPro into the lm-evaluation-harness platform. For the open-source models, we use the default settings of lm-evaluation-harness: do\_sample is set to false and the temperature is set to the default value of the hugging-face library. For the closed-source models, we set the temperature to 0.7. In addition, we set the maximum output length to 1024.

\subsubsection{Prompt Strategy}
Taking into account the influence of language and shot, we design 9 prompting strategies, including Direct, CN-CoT, EN-CoT, CN-XLT and EN-XLT. The last four setups include both zero-shot and few-shot variants.\footnote{
For open-source models, Direct adopts an approach similar to the official implementation of HellaSwag, computing the log-likelihood for each option and selecting the one with the highest log-likelihood. And we report normalized accuracy that accounts for the impact of option length. Other prompting strategies use a generation setup and report accuracy based on exact match.}
\textbf {(1)Direct}: LLMs makes the selection directly without any CoT process.
\textbf{(2)CN-CoT}: LLMs performs CoT in Chinese, regardless of dataset language.
\textbf{(3)EN-CoT}: Similar to CN-CoT, but CoT is conducted in English. 
\textbf{(4)CN-XLT}: LLMs are instructed to first translate English questions and options to Chinese, and then reason in Chinese.
\textbf{(5)EN-XLT}: Similar to CN-XLT, but translates from Chinese dataset to English and reasons in English. 

%\textbf {CN-CoT}: LLMs perform Chinese reasoning and then output the answer and 3 shots are provided.
%\textbf {CN-CoT}: Similar as CNCoTFewShot without any shots.
%\textbf {EN-CoT}: The reasoning process in English is executed and then the answer is output and 3 shots are provided.
%\textbf {CN-XLT}: Inspired by this, we instruct LLMs to translate questions in Chinese and then output the answer after performing reasoning in Chinese too. And 3 shots are provided.
%\textbf {EN-XLT}: Inspired by this, we instruct LLMs to translate questions in Englsih and then output the answer after performing reasoning in Englsih too. Three shots are provided.

\subsubsection{Evaluation metric}

To comprehensively evaluate the robustness of each LLM, we consider four metrics: 
% Original Accuracy (\textbf{OA}), Average Robust Accuracy (\textbf{ARA}), Robust Loss Accuracy (\textbf{RLA}), and  Consistent Robust Accuracy (\textbf{CRA}).
\noindent %
\textbf{- Original Accuracy (OA)} measures accuracy on original problems.
\begin{equation}\label{eq1}
OA=\frac{\sum_{(x, y) \in D} \mathds{1}[L M(x), y]}{|D|}.
\end{equation}
\noindent %
\textbf{- Average Robust Accuracy  (ARA)} represents average accuracy across all variants, gauging overall performance on the robustness tasks.
\begin{equation}\label{eq2}
ARA=\frac{\sum_{\left(x^{\prime}, y^{\prime}\right) \in D_{R}} \mathds{1}\left(L M\left(x^{\prime}, y^{\prime}\right)\right.}{\left|D_{R}\right|}.
\end{equation}

\noindent %
\textbf{- Robust Loss Accuracy (RLA)} is the difference between ARA and OA, indicating performance degradation on robustness data versus original data.
%\begin{tiny}
%\begin{equation}\label{eq3}
%RLA=\frac{\sum_{\left(x^{\prime}, y^{\prime}\right) \in D_{R}} %\mathds{1}\left(L M\left(x^{\prime}, y^{\prime}\right)\right.}{\left|D_{R}\right|}-\frac{\sum_{(x, y) \in D}\mathds{1}[L M(x), y]}{|D|}
%\end{equation}
%\end{tiny}
\begin{equation}\label{eq3}
RLA= OA - ARA.
\end{equation}
\noindent %
\textbf{- Consistent Robust Accuracy (CRA)} shows accuracy when the model correctly answers both original and variant data, reflecting the model do understand the problem.
% consistency in problem-solving.
\begin{equation}\label{eq4}
CRA=\frac{\sum_{x, y, x^{\prime}, y^{\prime}}\mathds{1}[L M(x), y] \cdot \mathds{1}[L M(x^{\prime}), y^{\prime}]}{\left|D_{R}\right|}.
\end{equation}
For all equation above, $D$ denotes the original dataset, where $x$ represents the input question and options, and $y$ represents the correct label, while $D_{R}$ is the robust dataset with $x^{\prime}$ and $y^{\prime}$ representing similar to $x$ and $y$.


\begin{table*}[ht]
\centering
\setlength{\tabcolsep}{5pt}
% \footnotesize
\scalebox{0.6}{
% Please add the following required packages to your document preamble:
% \usepackage{multirow}
% \usepackage[table,xcdraw]{xcolor}
% Beamer presentation requires \usepackage{colortbl} instead of \usepackage[table,xcdraw]{xcolor}
% Please add the following required packages to your document preamble:
% \usepackage{multirow}
% \usepackage[table,xcdraw]{xcolor}
% Beamer presentation requires \usepackage{colortbl} instead of \usepackage[table,xcdraw]{xcolor}
\begin{tabular}{ccccccccccccc}
\hline
\multicolumn{1}{c|}{{ }}& \multicolumn{4}{c|}{Chinese}& \multicolumn{4}{c|}{English}& \multicolumn{4}{c}{AVG}\\ \cline{2-13} 
\multicolumn{1}{c|}{\multirow{-2}{*}{{ Model}}} & { OA(\%)$\uparrow$}& { ARA(\%)$\uparrow$} & {RLA(\%)$\downarrow$}& \multicolumn{1}{l|}{{CRA(\%)$\uparrow$}} & { OA(\%)$\uparrow$}& { ARA(\%)$\uparrow$} & { RLA(\%)$\downarrow$}& \multicolumn{1}{l|}{{CRA(\%)$\uparrow$}} & {OA(\%)$\uparrow$}& { ARA(\%)$\uparrow$} & {RLA(\%)$\downarrow$}& { CRA(\%)$\uparrow$} \\ \hline
\multicolumn{1}{c|}{{ Human}} & 96.41& 97.79& -1.38 & \multicolumn{1}{l|}{92.03}& 95.56& 96.04& -0.48 & \multicolumn{1}{l|}{90.02}& 95.99 & 96.92 & -0.93& 91.03 \\ \hline
\multicolumn{13}{c}{\textit{Close-source LLMs}}\\ 
\multicolumn{1}{c|}{{ GPT-4o}}& { 91.37} & { 81.97} & { 9.40}& \multicolumn{1}{l|}{{ 75.55}} & { \textbf{88.63}} & { \textbf{70.17}} & { \textbf{18.46}} & \multicolumn{1}{l|}{{ \textbf{63.06}}} & { 90.00} & { \textbf{76.07}} & { \textbf{13.93}} & { \textbf{69.31}} \\
\multicolumn{1}{c|}{{ Claude3.5}}& { \textbf{95.37}} & { 80.15} & { 15.22} & \multicolumn{1}{l|}{{ 75.04}} & { 85.11} & { 66.02} & { 19.08} & \multicolumn{1}{l|}{{ 57.20}} & { 90.24} & { 73.09} & { 17.15} & { 66.12} \\
\multicolumn{1}{c|}{{ Gemini-1.5-Pro}}& { 90.62} & { 78.36} & { 12.26} & \multicolumn{1}{l|}{{ 70.48}} & { 87.75} & { 60.74} & { 27.01} & \multicolumn{1}{l|}{{ 58.27}} & { 89.19} & { 69.55} & { 19.63} & { 64.38} \\
\multicolumn{1}{c|}{{ Qwen-Max}}& { 93.50} & { \textbf{84.82}} & { \textbf{8.68}}& \multicolumn{1}{l|}{{ \textbf{78.91}}} & { 87.60} & { 62.61} & { 24.99} & \multicolumn{1}{l|}{{ 59.65}} & { \textbf{90.55}} & { 73.72} & { 16.83} & { 69.28} \\ \hline
\multicolumn{13}{c}{\textit{Chinese open-source LLMs}} \\ 
\multicolumn{1}{c|}{{ Qwen2.5-0.5B}}& { 60.75} & { 45.18} & { \textbf{15.57}} & \multicolumn{1}{l|}{{ 28.70}} & { 49.50} & { 38.21} & { \textbf{11.29}} & \multicolumn{1}{l|}{{ 20.57}} & { 55.13} & { 41.70} & { \textbf{13.43}} & { 24.64} \\
\multicolumn{1}{c|}{{ Qwen2.5-1.5B}}& { 63.25} & { 46.16} & { 17.09} & \multicolumn{1}{l|}{{ 29.89}} & { 56.88} & { 39.57} & { 17.30} & \multicolumn{1}{l|}{{ 23.48}} & { 60.06} & { 42.87} & { 17.20} & { 26.69} \\
\multicolumn{1}{c|}{{ Qwen2.5-3B}}& { 67.50} & { 48.75} & { 18.75} & \multicolumn{1}{l|}{{ 33.79}} & { 61.75} & { 39.98} & { 21.77} & \multicolumn{1}{l|}{{ 25.75}} & { 64.63} & { 44.37} & { 20.26} & { 29.77} \\
\multicolumn{1}{c|}{{ Qwen2.5-7B}}& { 67.63} & { 50.59} & { 17.04} & \multicolumn{1}{l|}{{ 35.62}} & { 65.63} & { 43.93} & { 21.70} & \multicolumn{1}{l|}{{ 30.77}} & { 66.63} & { 47.26} & { 19.37} & { 33.20} \\
\multicolumn{1}{c|}{{ Qwen2.5-14B}} & { 69.00} & { 51.41} & { 17.59} & \multicolumn{1}{l|}{{ 35.84}} & { 68.50} & { 45.20} & { 23.30} & \multicolumn{1}{l|}{{ 32.12}} & { 68.75} & { 48.30} & { 20.45} & { 33.98} \\
\multicolumn{1}{c|}{{ Qwen2.5-32B}} & { 69.75} & { 53.11} & { 16.64} & \multicolumn{1}{l|}{{ 37.54}} & { 70.00} & { 46.10} & { 23.90} & \multicolumn{1}{l|}{{ 32.68}} & { 69.88} & { 49.61} & { 20.27} & { 35.11} \\
\multicolumn{1}{c|}{{ Qwen2.5-72B}} & { \textbf{70.87}} & { \textbf{54.75}} & { 16.12} & \multicolumn{1}{l|}{{ \textbf{39.64}}} & { \textbf{72.00}} & { \textbf{47.75}} & { 24.25} & \multicolumn{1}{l|}{{\textbf{ 35.12}}} & { \textbf{71.44}} & { \textbf{51.25}} & {20.19} & { \textbf{37.38}} \\ \hdashline[0.5pt/5pt]
\multicolumn{1}{c|}{{ Baichuan2-7B}}& { 67.00} & { 46.16} & { 20.84} & \multicolumn{1}{l|}{{ 31.50}} & { 60.62} & { 39.04} & { 21.58} & \multicolumn{1}{l|}{{ 25.21}} & { 63.81} & { 42.60} & { 21.21} & { 28.36} \\
\multicolumn{1}{c|}{{ Baichua2-13B}}& { 69.13} & { 46.98} & { 22.15} & \multicolumn{1}{l|}{{ 33.45}} & { 64.62} & { 38.82} & { 25.80} & \multicolumn{1}{l|}{{ 26.07}} & { 66.88} & { 42.90} & { 23.97} & { 29.76} \\ \hdashline[0.5pt/5pt]
\multicolumn{1}{c|}{{ DeepSeek-7B}} & { 68.13} & { 47.96} & { 20.17} & \multicolumn{1}{l|}{{ 33.30}} & { 63.38} & { 40.39} & { 22.99} & \multicolumn{1}{l|}{{ 26.70}} & { 65.76} & { 44.18} & { 21.58} & { 30.00} \\
\multicolumn{1}{c|}{{ DeepSeek-67B}}& { 71.50} & { 49.21} & { 22.29} & \multicolumn{1}{l|}{{ 35.89}} & { 71.37} & { 40.63} & { 30.75} & \multicolumn{1}{l|}{{ 29.71}} & { 71.44} & { 44.92} & { 26.52} & { 32.80} \\ \hdashline[0.5pt/5pt]
\multicolumn{1}{c|}{{ InternLM2.5-1.8B}}& { 61.62} & { 42.07} & { 19.55} & \multicolumn{1}{l|}{{ 26.99}} & { 55.37} & { 38.46} & { 16.91} & \multicolumn{1}{l|}{{ 22.61}} & { 58.50} & { 40.27} & { 18.23} & { 24.80} \\
\multicolumn{1}{c|}{{ InternLM2.5-7B}}& { 67.25} & { 49.77} & { 17.48} & \multicolumn{1}{l|}{{ 34.57}} & { 69.50} & { 40.89} & { 28.61} & \multicolumn{1}{l|}{{ 29.75}} & { 68.38} & { 45.33} & { 23.04} & { 32.16} \\
\multicolumn{1}{c|}{{ InternLM2.5-20B}} & { 67.37} & { 48.08} & { 19.29} & \multicolumn{1}{l|}{{ 33.21}} & { 73.62} & { 41.11} & { 32.51} & \multicolumn{1}{l|}{{ 31.23}} & { 70.50} & { 44.60} & { 25.90} & { 32.22} \\ \hdashline[0.5pt/5pt]
\multicolumn{1}{c|}{{ Yi-1.5-6B}} & { 67.00} & { 49.59} & { 17.41} & \multicolumn{1}{l|}{{ 34.27}} & { 64.38} & { 39.37} & { 25.01} & \multicolumn{1}{l|}{{ 26.62}} & { 65.69} & { 44.48} & { 21.21} & { 30.45} \\
\multicolumn{1}{c|}{{ Yi-1.5-9B}} & { 68.50} & { 50.18} & { 18.32} & \multicolumn{1}{l|}{{ 35.55}} & { 66.37} & { 39.58} & { 26.79} & \multicolumn{1}{l|}{{ 27.48}} & { 67.44} & { 44.88} & { 22.56} & { 31.52} \\
\multicolumn{1}{c|}{{ Yi-1.5-34B}}& { 71.00} & { 52.23} & { 18.77} & \multicolumn{1}{l|}{{ 38.09}} & { 71.00} & { 40.75} & { 30.25} & \multicolumn{1}{l|}{{ 29.91}} & { 71.00} & { 46.49} & { 24.51} & { 34.00} \\ \hline
\multicolumn{13}{c}{\textit{English open-source LLMs}} \\ 
\multicolumn{1}{c|}{{ Llama3-8B}} & { 59.13} & { 46.62} & { 12.51} & \multicolumn{1}{l|}{{ 28.23}} & { 66.25} & { 40.21} & { 26.04} & \multicolumn{1}{l|}{{ 27.34}} & { 62.69} & { 43.42} & { 19.27} & { 27.79} \\
\multicolumn{1}{c|}{{ Llama3-70B}}& { 65.75} & { 48.63} & { 17.12} & \multicolumn{1}{l|}{{ 32.70}} & { \textbf{72.50}} & { 41.27} & { 31.23} & \multicolumn{1}{l|}{{\textbf{ 30.63}}} & {\textbf{ 69.13}} & { 44.95} & { 24.18} & { 31.67} \\ \hdashline[0.5pt/5pt]
\multicolumn{1}{c|}{{ Mistral-7B-v0.2}} & { 57.75} & { 46.25} & { \textbf{11.50}} & \multicolumn{1}{l|}{{ 27.57}} & { 67.50} & { \textbf{41.52}} & { 25.98} & \multicolumn{1}{l|}{{ 28.93}} & { 62.63} & { 43.88} & { 18.74} & { 28.25} \\
\multicolumn{1}{c|}{{ Mixtral-8x7B-v0.1}} & { 63.62} & { 46.80} & { 16.82} & \multicolumn{1}{l|}{{ 30.82}} & { 69.75} & { 41.21} & { 28.54} & \multicolumn{1}{l|}{{ 29.39}} & { 66.69} & { 44.01} & { 22.68} & { 30.11} \\
\multicolumn{1}{c|}{{ Mixtral-8x22B-v0.1}}& { 66.00} & {\textbf{ 50.73}} & { 15.27} & \multicolumn{1}{l|}{{ \textbf{34.32}}} & { 72.12} & { 41.25} & { 30.87} & \multicolumn{1}{l|}{{ 30.61}} & { 69.06} & { \textbf{45.99}} & { 23.07} & { \textbf{32.47}} \\ \hdashline[0.5pt/5pt]
\multicolumn{1}{c|}{{ Gemma-2-2B}}& { 61.88} & { 45.38} & { 16.51} & \multicolumn{1}{l|}{{ 29.02}} & { 59.62} & { 39.13} & { \textbf{20.50}} & \multicolumn{1}{l|}{{ 24.88}} & { 60.75} & { 42.25} & {\textbf{ 18.50}} & { 26.95} \\
\multicolumn{1}{c|}{{ Gemma-2-9B}}& { \textbf{69.13}} & { 46.75} & { 22.38} & \multicolumn{1}{l|}{{ 33.29}} & { 64.88} & { 39.80} & { 25.08} & \multicolumn{1}{l|}{{ 26.91}} & { 67.01} & { 43.28} & { 23.73} & { 30.10} \\
\multicolumn{1}{c|}{{ Gemma-2-27B}} & { 63.38} & { 48.52} & { 14.86} & \multicolumn{1}{l|}{{ 31.96}} & { 71.88} & { 40.91} & { 30.97} & \multicolumn{1}{l|}{{ 30.25}} & { 67.63} & { 44.71} & { 22.92} & { 31.11} \\ \hline
\end{tabular}
}
\caption{TODO: bolded is not result. Results of existing LLMs on our HellaSwag-Pro dataset using \textbf{Direct} prompt. ``AVG'' indicates the average performance of each model on Chinese and English parts of the dataset.
The best results for each metric in each model category are \textbf{bolded}. }
\label{tab:main experiment.}
\end{table*}

\subsection{Model Performance (RQ1)}
\paragraph{Overall Performance}
Table \ref{tab:main experiment.} provides a comprehensive evaluation of various LLMs across four performance metrics\footnote{The results of instruct/chat models of Qwen2.5, Llama3 and Mixtral latest series are shown in Appendix.}. The main observations are as follow:
\begin{itemize}[leftmargin=*,topsep=0pt]
% \setlength{}{0}
    \item Upon evaluating all available models, we found that all performed well in overall accuracy (e.g., GPT-4 scored 90.00 in AVG OA, Claude 3.5 scored 90.24 in AVG OA). However, all models struggled with variations of the questions, as evidenced by a positive RLA value for each model. In contrast, humans received a negative RLA value, suggesting that the question variants were not more challenging than the originals. This disparity further illustrates that current LLMs lack a true understanding of the reasoning process and can easily be misled by question variants.
    \item When comparing open-source and close-source models, the close-source models demonstrate stronger capabilities in both OA and ARA scores, similar to most existing benchmarks. Overall, the RLA values for close-source models are also smaller, indicating that they are more robust in commonsense reasoning tasks compared to open-source models.
    \item When we compare models within the same series (e.g., Qwen, Llama), we observe that larger models often achieve higher scores on OA, ARA, and CRA. However, they are also more susceptible to variations, i.e., they have higher RLA values, a phenomenon particularly evident in English datasets. We attribute this phenomenon to the fact that larger models, compared to smaller ones, may have memorized more data, allowing them to rely on memorization to solve some problems more easily and making them more prone to the influence of variations~\cite{}.
\end{itemize}
% 1. When evaluating all available models, We find although 
% 2. When comparing the opensource LLMs and close source LLMs, 
% 3. When looking into each serious details
% \noindent
% \textbf{Overall Model Performance.}
% 1. close-source > open-source 2. the large the better 3. all have a performance decline when meeting varients.

% To evaluate the performance of various models, we observed patterns consistent with current mainstream trends: closed-source models generally outperform open-source models across metrics. 
% For instance, the closed-source model GPT-4o achieved scores of 90.00 in OA, 76.07 in ARA, and 69.31 in CRA, whereas the open-source model Qwen2.5-72B scored 71.44, 51.25, and 37.38, respectively. 
% Furthermore, within each model series, performance tends to improve with larger model sizes. 
% Nevertheless, even the strongest closed-source models struggle with variations in questions, as indicated by positive values in RLA for all models. In contrast, human performance yields a negative RLA value, highlighting that current LLMs do not genuinely grasp the reasoning process and are prone to falling into traps set by question variants. 
% This suggests that there is still significant room for improvement in developing models that can robustly understand and reason through complex linguistic challenges.
% It reveals a consistent pattern across Chinese, English, and average scores, with close-sourced LLMs generally outperforming open-sourced models. 
% However, all models exhibit a significant drop in performance when faced with robust variants, as indicated by RLA and CRA. Among closed-source models, GPT-4o demonstrates the highest ARA of 76.07\% in average scores, demonstrating its overwhelming superiority. Among open-sourced models, larger models tend to perform better, with Qwen2.5-72B achieving the highest OA (71.44\%) and ARA (51.25\%) in the average scores. However, even these top performers still struggle with robustness, as evidenced by the substantial RLA of 13.93\% for GPT-4o and 20.19\% for Qwen2.5-72B. Interestingly, some English open-sourced models, such as Llama3-70B and Mixtral-8x22B-v0.1, show competitive performance in English tasks but lag in Chinese tasks, highlighting the importance of language-specific training.

% \noindent
% \textbf{Chinese Models vs English Models.}
% Chinese models generally demonstrate higher OA in Chinese tasks compared to English tasks, with Qwen-Max achieving 93.50\% OA in Chinese versus 87.60\% in English. Conversely, English models tend to perform better in English tasks, exemplified by Llama3-70B's 72.50\% OA in English compared to 65.75\% in Chinese. 
% However, both Chinese and English models exhibit important drops in ARA across languages, indicating challenges in maintaining performance when faced with variations. This trend suggests that while models may excel in their primary language, they struggle with robustness across linguistic boundaries. 
% Notably, larger models tend to achieve higher ARA scores but also experience more substantial RLA, as seen with Qwen2.5-0.5B (41.70\% ARA, 13.43\% RLA in total) and Qwen2.5-72B (51.25\% ARA, 20.19\% RLA in total). 
% This pattern indicates that while increased model size enhances overall performance, it doesn't necessarily improve robustness proportionally. 
% The discrepancy between OA and ARA across languages underscores the need for improved cross-lingual robustness in language models, particularly as they scale in size and capability.


% \noindent
% \textbf{Comparison between Chinese and English datasets.}
% Generally, models demonstrate higher accuracy on the Chinese dataset compared to the English one, as evidenced by the consistently higher OA, ARA and CRA scores. For instance, GPT-4o achieves an OA of 91.37\%, an ARA of 81.97\% , an CRA of 75.55\% on the Chinese dataset, compared to 88.63\% and 70.17\% respectively on the English dataset. This trend is observed across most models, suggesting that the Chinese dataset is easier than English one. Moreover, the RLA values are typically lower for Chinese, indicating smaller performance drops when dealing with robust variants of Chinese questions. For example, Qwen-Max shows an RLA of 8.68\% for Chinese versus 24.99\% for English, highlighting a more consistent performance in Chinese. The CRA scores further reinforce this observation, with models generally maintaining higher consistency in correct answers for both original and variant Chinese questions.
% We attribute this phenomenon to the fact that blablabla

\noindent
\textbf{Reasoning Transferable Capability.}
% 为了进一步
To further analyze whether the model can transfer reasoning ability from the original question to its variant, Figure \ref{consis} presents the distribution of model performance on the original question and variant pairs. For all models, the pairs of (HellaSwag \ding{51} HellaSwag-Pro \ding{55}) occupy a significant proportion, indicating a challenge in transferring reasoning capabilities for current LLMs to more complex scenarios. Looking deeply, closed-source models like GPT-4 and Qwen-Max achieve around a 69\% portion of (HellaSwag \ding{51} HellaSwag-Pro \ding{51}) and a 3\% portion of (HellaSwag \ding{55} HellaSwag-Pro \ding{55}), while in contrast, open-source models struggle with around a 30\% portion of (HellaSwag \ding{51} HellaSwag-Pro \ding{51}) and a 20\% portion of (HellaSwag \ding{55} HellaSwag-Pro \ding{55}), further indicating the robustness of reasoning abilities in closed-source models.
% If a model can get both the original question and the variant right, we consider it to have transferable reasoning ability. Table \ref{consis} presents the distribution of model performance on the original question and variant pairs. Among all models, the pairs of (HellaSwag \ding{51}HellaSwag-Pro \ding{55}) account for a considerable proportion, i 
% The closed-source models like GPT-4o and Qwen-Max achieve around 69\% portion of (HellaSwag \ding{51}HellaSwag-Pro \ding{51}) and 3\% portion of (HellaSwag \ding{55} HellaSwag-Pro \ding{55}), indicating stronger reasoning transfer ability than other models. In contrast, open-source models struggle more, with around 30\% portion of (HellaSwag \ding{51}HellaSwag-Pro \ding{51}) and 20\% portion of (HellaSwag \ding{55} HellaSwag-Pro \ding{55}). 
% A notable trend is observed among the Qwen2.5 series, where increasing model size from 7B to 72B parameters correlates with improved performance on correct answers for both datasets (33.20\% to 37.38\%) and decreased failure rates (17.69\% to 14.7\%). It underscores the importance of model size in commonsense reasoning tasks.

\begin{figure}[t]
\centering
\setlength{\abovecaptionskip}{0.1cm}
\setlength{\belowcaptionskip}{0cm}
\includegraphics[width=\linewidth,scale=1.00]{images/consis.pdf}
\caption{Analysis of the transferable ability of model reasoning based on question pair performance. The green part, where both the original and the variant data are right, represents the transferable performance of model reasoning.}
\label{consis}
\vspace{-15pt}
\end{figure}

\begin{figure*}[ht]
\centering
\setlength{\abovecaptionskip}{0.1cm}
\setlength{\belowcaptionskip}{0cm}
\includegraphics[width=\linewidth,scale=1.00]{images/xing.pdf}
\caption{The impact of different few-shot prompts on model performance. With - as the separator, the first two parts of the legend represent the prompt name, and the third part represents the language of the dataset.}
\label{xing}
\vspace{-15pt}
\end{figure*}

\begin{figure}[ht]
\centering
\setlength{\abovecaptionskip}{0.1cm}
\setlength{\belowcaptionskip}{0cm}
\includegraphics[width=1.05\linewidth,scale=1.05]{images/zhu.pdf}
\caption{The RLA Distribution for 7 variants of commonsense reasoning. Parts below the 0 axis indicate that the model’s performance on the variant is improved compared to the original problem.}
\label{fig:zhu}
\vspace{-15pt}
\end{figure}


\subsection{Variant Analysis (RQ2)}
To further analyze the impact of different variants, we assessed the contribution of each variant to the RLA score. A higher contribution indicates that the model is more likely to make errors in that type. Figure~\ref{fig:zhu} presents the overall results, and the key observations are as follows:
\begin{itemize}[leftmargin=*]
    \item For problem restatement, causal inference, and sentence ordering, these three categories are the least challenging. Almost all models, particularly the close-source and Qwen series models, perform well on these variants, indicating that current LLMs can effectively handle these forms and we do not pay more attention on this kind of varients.
    \item For reverse conversion and critical testing, these two varients each contribute about 10\% to the RLA score. This indicates that current LLMs struggle to fully generalize to these simple scenarios, possibly because these types of questions are not commonly encountered, and reaserchers should pay some attention to this type of varients.
    \item For negative transformation and scenario refinement, this are the two most difficult tasks, with negative transformation being particularly challenging. For almost all models, these two varients accounts for more than 50\% of the RLA score. This may be due to intuitively counterintuitive questions—such as the use of "will not"  or counterfactual scenarios in scenario refinement. These setups are less common in LLM training data and cannot be easily tackled through memory alone. Only those LLMs which truely understand the question could answer the varient correctly, wihch better reflect the true performance of the model.. In the future, researchers should focus more on enhancing LLM's capability to address such types of questions.
\end{itemize}

% 1. Problem restCausal Inference 
% To further analysis the impact of different varients, we further 
% Figure \ref{fig: zhu} presents a comprehensive analysis of various LLMs' performance across different variant types. Negative transformation emerges as the most challenging task for all models, with scores consistently above 50.00\% and peaking at 78.38\% for Gemini-1.5-Pro. Conversely, problem restatement appears to be the least challenging, with most models scoring in the negative range. Intriguingly, smaller models like Qwen2.5-0.5B demonstrate unexpected strengths in certain areas, such as sentence sorting (7.75\%), outperforming some larger counterparts. A detailed analysis of each variant type follows.

% \noindent
% \textbf{Causal inference.} In this category, scores vary widely from -4.73\% for Qwen-Max to 12.25\% for Baichuan2-13B, illustrating differing degrees of sensitivity to causal reasoning among the models. Smaller models, such as Qwen2.5-0.5B and Qwen2.5-1.5B, achieve better scores, indicating relatively stronger robustness in causal reasoning. Conversely, larger models, like Baichuan2-13B, have higher scores, suggesting greater sensitivity to the challenges of inferring causality.

% \noindent
% \textbf{Critical testing.} Larger models, including Qwen2.5-72B and DeepSeek-67B, exhibit higher RLA scores of 30.50\% and 31.37\%, respectively, suggesting increased sensitivity when dealing with incomplete key information. In contrast, GPT-4o achieves the lowest score, highlighting its superior robustness in critical reasoning. This trend indicates that more complex models might struggle to handle incomplete contexts, underscoring potential areas for improvement in sophisticated architectures.

% \noindent
% \textbf{Negative transformation.} This aspect remains consistently challenging for all models, with scores ranging from 48.88\% to 78.38\%. Advanced commercial models like Gemini-1.5-Pro and Claude-3.5 also score higher (78.38\% and 76.43\%, respectively), indicating a prevalent sensitivity issue in reasoning processes when handling negations, irrespective of model size or architecture.

% \noindent
% \textbf{Problem restatement.} The negative values in this category for nearly all models suggest it is not particularly challenging. This is surprising, given that previous models were quite sensitive to sentence representation.

% \noindent
% \textbf{Reverse conversion.} This variation, which involves swapping the roles of the question and answer, seems to specifically impact larger models. For example, Qwen2.5-72B and DeepSeek-67B exhibit higher RLA scores of 24.38\% and 27.43\%, respectively, indicating heightened sensitivity to reverse reasoning compared to their performance on original questions.

% \noindent
% \textbf{Scenario refinement.} The scores range from 16.06\% for Gemma-2-2B to 32.56\% for Qwen2.5-72B, with larger models displaying more sensitivity in adapting to counterfactual predictions. This suggests that larger models may rely more heavily on general commonsense rather than flexibly adapting to specific contexts. Consequently, increased model complexity might adversely affect adaptability to scenario changes, underscoring the need for enhanced flexibility in advanced models.

% \noindent
% \textbf{Sentence sorting.} This category exhibits the most varied results across models. Some larger models like DeepSeek-67B and InternLM2.5-20B display higher scores (26.69\% and 26.68\%), indicating sensitivity, while others like Qwen2.5-72B and Gemini-1.5-Pro excel with lower scores (-9.88\% and -1.07\%, respectively). This suggests that sentence sorting ability may depend more on specific training approaches rather than being solely contingent on model size.


\subsection{Prompt Robustness (RQ3)}
% To investigate how prompt  influence our benchmark, we apply sereral prompt strategy on our datasets and showcase the average performance of all models on different kind of prompt strategies.
% Table~\ref{prompt} illustrates the final results. For both Chinese and English datasets, CN LLMs achieve the highest performance using CN-CoT-Few-Shot, followed closely by EN-CoT-Few-Shot, with overall performance scores of 67.36\% and 67.03\%, respectively. In contrast, English LLMs perform best with the EN-CoT-Few-Shot, reaching 67.55\% on the Chinese dataset and 60.36\% on the English dataset.
% Contrary to previous findings, translating the dataset to the model's advantage language before performing reasoning does not enhance performance. Moreover, Figure~\ref{xing} also shows the similar phenomenon. Conducting CoT reasoning in the model’s advantage language generally leads to better outcomes compared to Direct. Additionally, increasing the number of shots consistently improves performance across most configurations, highlighting the benefits of exposing models to multiple examples. 
To explore the impact of various prompt strategies on our benchmarks, we evaluated several approaches across our datasets and present the average performance of all models using different prompting techniques. Table~\ref{prompt} summarizes the results. For both Chinese and English datasets, Chinese LLMs performed best with the CN-CoT-Few-Shot strategy, followed closely by EN-CoT-Few-Shot, achieving overall scores of 67.36\% and 67.03\%, respectively. Conversely, English LLMs showed optimal performance with the EN-CoT-Few-Shot approach, attaining 67.55\% on the Chinese dataset and 60.36\% on the English dataset.
Besides, translating datasets into the model's native language before reasoning did not enhance performance. This phenomenon is further illustrated in Figure~\ref{xing}. Conducting CoT reasoning in the model's native language generally yields better results compared to direct reasoning. Furthermore, increasing the number of examples (shots) consistently boosts performance across most configurations, emphasizing the advantages of exposing models to multiple examples.
% Overall, the interaction between question language, prompt language, and the number of shots underscores the importance of aligning these factors to optimize task performance and robustness in LLMs.



% Please add the following required packages to your document preamble:
% \usepackage{multirow}
% Please add the following required packages to your document preamble:
% \usepackage{multirow}
\begin{table}[t]
\setlength{\tabcolsep}{8pt}
% \footnotesize
\scalebox{0.65}{
\begin{tabular}{c|l|lll}
\hline
\multicolumn{1}{l|}{Dataset}  & Prompt  & CN LLMs & EN LLMs &  LLMs \\ \hline
\multirow{7}{*}{\begin{tabular}[c]{@{}c@{}}Chinese\\ HellaSwag-Pro\end{tabular}} & Direct  & 48.95& 41.16& 45.06  \\
& CN-CoT-Few  & \textbf{71.04}& 51.90& 61.47  \\
& EN-CoT-Few  & 70.95& \textbf{67.55}& \textbf{69.25}  \\
& EN-XLT-Few  & 41.48& 28.69& 35.09  \\
& CN-CoT-Zero & 44.82& 23.89& 34.36  \\
& EN-CoT-Zero & 45.38& 31.39& 38.39  \\
& EN-XLT-Zero & 28.57& 12.93& 20.75  \\ \hline
\multirow{7}{*}{\begin{tabular}[c]{@{}c@{}}English\\ HellaSwag-Pro\end{tabular}} & Direct  & 47.46& 40.66& 44.06  \\
& CN-CoT-Few  & \textbf{63.67}& 47.24& 55.46  \\
& EN-CoT-Few  & 63.12& \textbf{60.36}& \textbf{61.74}  \\
& CN-XLT-Few  & 48.77& 16.61& 32.69  \\
& CN-CoT-Zero & 34.89& 18.25& 26.57  \\
& EN-CoT-Zero & 42.41& 31.03& 36.72  \\
& CN-XLT-Zero & 16.36& 11.22& 13.79  \\ \hline
\multirow{9}{*}{HellaSwag-Pro}& Direct  & 48.21& 40.91& 44.83  \\
& CN-CoT-Few  & \textbf{67.36}& 49.57& 58.46  \\
& EN-CoT-Few  & 67.03& \textbf{63.95}& \textbf{65.49}  \\
& CN-XLT-Few  & 59.91& 34.26& 47.08  \\
& EN-XLT-Few  & 52.30& 44.52& 48.41  \\
& CN-CoT-Zero & 39.86& 21.07& 30.46  \\
& EN-CoT-Zero & 43.90& 31.21& 37.55  \\
& CN-XLT-Zero & 30.59& 17.55& 24.07  \\
& EN-XLT-Zero & 35.49& 21.98& 28.74  \\ \hline
\end{tabular}
}
\caption{Average ARA of all open-source models on different prompts. CN-LLMs contains 17 LLMs, and EN-LLMs contains 7 LLMs. The bast results for each dataset are \textbf{bolded}.}
\label{prompt}
\end{table}




\begin{table*}[ht]
    \centering
    \caption{Performance comparison of different methods supplemented with nnU-Net augmentations. Colored numbers show an improvement (or a decline, respectively) over a non-augmented method. GIN and MIND were only trained with nnU-Net augmentations.}
    
    % Add these color definitions to your preamble
    \definecolor{darkGreen}{RGB}{0, 102, 0}     % For improvements > 0.15
    \definecolor{medGreen}{RGB}{0, 153, 0}      % For improvements 0.05 to 0.15
    \definecolor{lightGreen}{RGB}{144, 238, 144} % For small improvements 0 to 0.05
    \definecolor{lightRed}{RGB}{255, 200, 200}   % For negative values
    
    \resizebox{\textwidth}{!}{%
    \begin{tabular}{lcccccccccc}
        \toprule
        & MR$\rightarrow$CT & CT$\rightarrow$MR & CT$\rightarrow$LDCT & CE CT$\rightarrow$CT & T1 CE$\rightarrow$T1 & T1 F & T1 Sc & T1 Mix & \textbf{avg DSC} & \textbf{avg gap} \\
        
        \midrule
        
        % GIN & 0.589 & 0.637 & 0.722 & 0.163 & 0.382 & 0.837 & 0.709 & 0.804 & 0.605 & 33.6\% \\
        
        CycleGAN 3D & 0.364 \textcolor{lightGreen}{$\uparrow$0.031} & 0.464 \textcolor{darkGreen}{$\uparrow$0.200} & 0.679 \textcolor{darkGreen}{$\uparrow$0.353} & 0.221 \textcolor{medGreen}{$\uparrow$0.091} & 0.379 \textcolor{lightGreen}{$\uparrow$0.034} & 0.825 \textcolor{lightGreen}{$\uparrow$0.034} & 0.810 \textcolor{medGreen}{$\uparrow$0.097} & 0.779 \textcolor{lightGreen}{$\uparrow$0.017} & 0.565 \textcolor{medGreen}{$\uparrow$0.107} & 34.1\% \textcolor{darkGreen}{$\uparrow$24.6\%} \\
        CycleGAN 2D & 0.301 \textcolor{medGreen}{$\uparrow$0.096} & 0.461 \textcolor{medGreen}{$\uparrow$0.055} & 0.666 \textcolor{medGreen}{$\uparrow$0.136} & 0.333 \textcolor{medGreen}{$\uparrow$0.117} & 0.416 \textcolor{lightGreen}{$\uparrow$0.018} & 0.865 \textcolor{lightGreen}{$\uparrow$0.013} & 0.850 \textcolor{lightGreen}{$\uparrow$0.049} & 0.815 \textcolor{lightGreen}{$\uparrow$0.020} & 0.588 \textcolor{medGreen}{$\uparrow$0.063} & 45.5\% \textcolor{medGreen}{$\uparrow$15.3\%} \\
        
        % MIND & 0.560 & 0.588 & 0.237 & 0.425 & 0.335 & 0.865 & 0.869 & 0.845 & 0.590 & 45.9\% \\
        
        Baseline (nnAugm) & 0.166 \textcolor{medGreen}{$\uparrow$0.134} & 0.102 \textcolor{medGreen}{$\uparrow$0.070} & 0.779 \textcolor{darkGreen}{$\uparrow$0.646} & 0.392 \textcolor{darkGreen}{$\uparrow$0.164} & 0.446 \textcolor{lightGreen}{$\uparrow$0.020} & 0.910 \textcolor{darkGreen}{$\uparrow$0.169} & 0.897 \textcolor{medGreen}{$\uparrow$0.131} & 0.889 \textcolor{darkGreen}{$\uparrow$0.329} & 0.573 \textcolor{darkGreen}{$\uparrow$0.208} & 51.9\% \textcolor{darkGreen}{$\uparrow$51.9\%} \\  % nnAugm (Baseline)
        DANN        & 0.414 \textcolor{medGreen}{$\uparrow$0.118} & 0.349 \textcolor{medGreen}{$\uparrow$0.071} & 0.809 \textcolor{medGreen}{$\uparrow$0.110} & 0.411 \textcolor{lightGreen}{$\uparrow$0.002} & 0.403 \textcolor{lightRed}{$\downarrow$-0.013} & 0.899 \textcolor{darkGreen}{$\uparrow$0.169} & 0.848 \textcolor{lightGreen}{$\uparrow$0.015} & 0.885 \textcolor{medGreen}{$\uparrow$0.109} & 0.627 \textcolor{medGreen}{$\uparrow$0.072} & 54.9\% \textcolor{darkGreen}{$\uparrow$23.3\%} \\

        IN          & 0.422 \textcolor{medGreen}{$\uparrow$0.119} & 0.471 \textcolor{darkGreen}{$\uparrow$0.163} & 0.796 \textcolor{medGreen}{$\uparrow$0.128} & 0.410 \textcolor{lightRed}{$\downarrow$-0.017} & 0.416 \textcolor{lightRed}{$\downarrow$-0.012} & 0.907 \textcolor{darkGreen}{$\uparrow$0.151} & 0.854 \textcolor{lightGreen}{$\uparrow$0.016} & 0.883 \textcolor{medGreen}{$\uparrow$0.099} & 0.645 \textcolor{medGreen}{$\uparrow$0.081} & 58.1\% \textcolor{darkGreen}{$\uparrow$26.6\%} \\
        AdaBN       & 0.495 \textcolor{darkGreen}{$\uparrow$0.173} & 0.532 \textcolor{darkGreen}{$\uparrow$0.179} & 0.604 \textcolor{lightGreen}{$\uparrow$0.017} & 0.365 \textcolor{medGreen}{$\uparrow$0.070} & 0.454 \textcolor{lightGreen}{$\uparrow$0.021} & 0.907 \textcolor{medGreen}{$\uparrow$0.129} & 0.890 \textcolor{medGreen}{$\uparrow$0.057} & 0.892 \textcolor{medGreen}{$\uparrow$0.096} & 0.642 \textcolor{medGreen}{$\uparrow$0.092} & 59.2\% \textcolor{darkGreen}{$\uparrow$24.2\%} \\

        SE          & 0.459 \textcolor{medGreen}{$\uparrow$0.068} & 0.571 \textcolor{darkGreen}{$\uparrow$0.183} & 0.768 \textcolor{darkGreen}{$\uparrow$0.165} & 0.389 \textcolor{medGreen}{$\uparrow$0.057} & 0.374 \textcolor{lightRed}{$\downarrow$-0.014} & 0.902 \textcolor{lightRed}{$\downarrow$-0.004} & 0.907 \textcolor{lightGreen}{$\uparrow$0.014} & 0.888 \textcolor{lightRed}{$\downarrow$-0.030} & 0.657 \textcolor{medGreen}{$\uparrow$0.055} & 60.1\% \enspace \textcolor{medGreen}{$\uparrow$8.4\%} \\
        MinEnt      & 0.388 \textcolor{darkGreen}{$\uparrow$0.248} & 0.362 \textcolor{darkGreen}{$\uparrow$0.190} & 0.788 \textcolor{darkGreen}{$\uparrow$0.283} & 0.449 \textcolor{medGreen}{$\uparrow$0.057} & 0.448 \textcolor{lightGreen}{$\uparrow$0.019} & 0.903 \textcolor{medGreen}{$\uparrow$0.133} & 0.901 \textcolor{medGreen}{$\uparrow$0.103} & 0.892 \textcolor{medGreen}{$\uparrow$0.116} & 0.641 \textcolor{medGreen}{$\uparrow$0.143} & 62.0\% \textcolor{darkGreen}{$\uparrow$33.5\%} \\





        
        \midrule
        % \textbf{Average improvement} & 0.376 \textcolor{medGreen}{$\uparrow$0.123} & 0.414 \textcolor{darkGreen}{$\uparrow$0.139} & 0.736 \textcolor{darkGreen}{$\uparrow$0.230} & 0.371 \textcolor{medGreen}{$\uparrow$0.068} & 0.417 \textcolor{lightGreen}{$\uparrow$0.009} & 0.890 \textcolor{medGreen}{$\uparrow$0.099} & 0.870 \textcolor{medGreen}{$\uparrow$0.060} & 0.865 \textcolor{medGreen}{$\uparrow$0.095} & 0.617 \textcolor{medGreen}{$\uparrow$0.103} \\
        \textbf{average} &\textcolor{medGreen}{$\uparrow$0.123} & \textcolor{darkGreen}{$\uparrow$0.139} & \textcolor{darkGreen}{$\uparrow$0.230} & \textcolor{medGreen}{$\uparrow$0.068} & \textcolor{lightGreen}{$\uparrow$0.009} & \textcolor{medGreen}{$\uparrow$0.099} & \textcolor{medGreen}{$\uparrow$0.060} & \textcolor{medGreen}{$\uparrow$0.095} & \textcolor{medGreen}{$\uparrow$0.103} & \textcolor{darkGreen}{$\uparrow$26.0\%} \\  % 53.2\%
        
        \bottomrule
    \end{tabular}}
    \label{tab:ablation_aug}
\end{table*}




% \begin{table*}[ht]
%     \centering
%     \caption{Ablation of selected methods on adding augmentations from the nnUNet pipeline.}%during training

%     \resizebox{\textwidth}{!}{%
%     \begin{tabular}{lcccccccccc}
%         \toprule
%         & nnUnet augm & MR$\rightarrow$CT & CT$\rightarrow$MR & CT$\rightarrow$LDCT & CE CT$\rightarrow$CT & T1 CE$\rightarrow$T1 & T1 F & T1 Sc & T1 Mix & \textit{avg DSC} \\
%         % & \textit{avg gap} \\
        
%         \midrule
%         Baseline    & \xmark & 0.032 & 0.032 & 0.133 & 0.228 & 0.426 & 0.741 & 0.766 & 0.560 & 0.365 \\ % & 0.0\% \\
%         nnAugm      & \cmark & 0.166 & 0.102 & 0.779 & 0.392 & 0.446 & 0.910 & 0.897 & 0.889 & 0.573 \\ % 48.9\% \\

%         % \rowcolor{lightgray}    
%         CycleGAN 3D & \xmark & 0.333 & 0.264 & 0.326 & 0.130 & 0.345 & 0.791 & 0.713 & 0.762 & 0.458 \\ % & 9.5\% \\
%         CycleGAN 3D & \cmark & 0.364 & 0.464 & 0.679 & 0.221 & 0.379 & 0.825 & 0.810 & 0.779 & 0.565 \\ % & 34.1\% \\

%         MinEnt      & \xmark & 0.140 & 0.172 & 0.505 & 0.392 & 0.429 & 0.770 & 0.798 & 0.776 & 0.498 \\ % & 28.5\% \\
%         MinEnt      & \cmark & 0.388 & 0.362 & 0.788 & 0.449 & 0.448 & 0.903 & 0.901 & 0.892 & 0.641 \\ % & 62.0\% \\

%         CycleGAN 2D & \xmark & 0.205 & 0.406 & 0.530 & 0.216 & 0.398 & 0.852 & 0.801 & 0.795 & 0.525 \\
%         CycleGAN 2D & \cmark & 0.301 & 0.461 & 0.666 & 0.333 & 0.416 & 0.865 & 0.850 & 0.815 & 0.588 \\ % & 45.5\% \\

%         AdaBN       & \xmark & 0.322 & 0.353 & 0.587 & 0.295 & 0.433 & 0.778 & 0.833 & 0.796 & 0.550 \\
%         AdaBN       & \cmark & 0.495 & 0.532 & 0.604 & 0.365 & 0.454 & 0.907 & 0.890 & 0.892 & 0.642 \\ % & 59.2\% \\
        
%         DANN        & \xmark & 0.296 & 0.278 & 0.699 & 0.409 & 0.416 & 0.730 & 0.833 & 0.776 & 0.555 \\ % & 31.6\% \\
%         DANN        & \cmark & 0.414 & 0.349 & 0.809 & 0.411 & 0.403 & 0.899 & 0.848 & 0.885 & 0.627 \\ % & 54.9\% \\

%         IN          & \xmark & 0.303 & 0.308 & 0.668 & 0.427 & 0.428 & 0.756 & 0.838 & 0.784 & 0.564 \\ % & 31.5\% \\
%         IN          & \cmark & 0.422 & 0.471 & 0.796 & 0.410 & 0.416 & 0.907 & 0.854 & 0.883 & 0.645 \\ % & 58.1\% \\

%         SE          & \xmark & 0.391 & 0.388 & 0.603 & 0.332 & 0.388 & 0.906 & 0.893 & 0.918 & 0.602 \\ % & 51.7% \\
%         SE          & \cmark & 0.459 & 0.571 & 0.768 & 0.389 & 0.374 & 0.902 & 0.907 & 0.888 & 0.657 \\ % & 60.1% \\
        
%         % nnUNet          & 0.397 & 0.355 & 0.750 & 0.373 & 0.330 & 0.923 & 0.914 & 0.907 & 0.619 & 54.9\% \\
        
%         % \midrule
        
%         % \textit{Oracle} & \xmark & 0.842 & 0.825 & 0.814 & 0.519 & 0.686 & 0.954 & 0.957 & 0.958 & 0.819 & 100\% \\
        
%         \bottomrule
         
%     \end{tabular}}
%     \label{tab:ablation_aug}
% \end{table*}
\section{Conclusion and future directions} \label{sec:conclusion}

In this paper we proposed a nested MLMC framework that offers important computational savings by performing most calculations in low precision and exploiting approximate random normal variables for the low precision path calculations. The low precision calculations could be performed in fixed precision on an FPGA for greater efficiency, and we suggested a procedure to optimise the bit-widths of every variable at each Monte Carlo level. This is an important improvement over previous mixed precision MLMC frameworks which held the lower precision fixed \cite{Rounding_error_oliver} or defined uniform bit-width at every level heuristically \cite{brugger2014mixed}. Our numerical results suggest that for the first levels our procedure reduces the cost at these levels by a factor 5 or 7. Hence the overall savings are significant since most paths are calculated on the first levels. Our approach would be even more efficient for the Milstein scheme because its higher order strong convergence leads to a greater proportion of the computational costs being on the coarsest levels.

The next stage of the research project will be to implement the RNG methods and the nested framework on FPGAs to determine the hardware requirements and confirm the extent of the computational savings. It would also be good to compare the performance benefits to using half-precision floating point arithmetic on GPUs or CPUs for the low-accuracy computations.





\newpage
\ifdefined\isarxiv
%\section*{Acknowledgments}
\bibliographystyle{alpha}
\bibliography{ref}
\else
\bibliography{ref}
\bibliographystyle{ACM-Reference-Format} %Bo: this is for KDD 2025
\fi



\newpage 
% \onecolumn % Bo: the KDD official template didn't use newp page and oneclumn
\appendix
\ifdefined\isarxiv
\begin{center}
    \textbf{\LARGE Appendix}
\end{center}
\else
\section*{Appendix}
\fi
\paragraph{Roadmap.}
In Section~\ref{sec:miss_proof}, we provide a formal version of theoretical analysis and proofs.

\section{Theoretical Analysis} \label{sec:miss_proof}

In this section, we first provide the formal theorem and proof for the sampling ODE in Section~\ref{sub:app:samp_ode}. Then, we formally proved the speed limit of ForM's sampling ODE in Section~\ref{sub:app:speed_limit}. In Section~\ref{sub:app:form_trig}, we formally prove the derivation of the interpolation path of ForM with TrigFlow. Last, we illustrate the formal proof for relativistic force in Section~\ref{sub:app:force}.


\subsection{Sampling ODE} \label{sub:app:samp_ode}

Here, we restate the Theorem~\ref{thm:ode_form:informal} and state its proof.

\begin{theorem}[Sampling ODE, formal version of Theorem~\ref{thm:ode_form:informal}]\label{thm:ode_form:formal}
    Giving the force at position $x_t$ denoted as $f_t(x_t)$, we could solve for ForM sampling path $x_t$ by the following ODE
    \begin{align*}
    \ddot{x}_t = \frac{1}{m^{\rm lab} \gamma_t}(f_t^{\rm local} - \frac{\langle v_t^{\rm lab}, f_t^{\rm local} \rangle}{c^2} v_t^{\rm lab})
    \end{align*}
    where $x_0 \sim \N(0,I)$, $\dot{x}_0 = 0$.
\end{theorem}

\begin{proof}
Recall $f^{\rm local}$ from Lemma~\ref{lem:equiv_relativistic_force:formal}
\begin{align*}
    f_t^{\rm local} = m^{\rm lab}  (\gamma_t a_t^{\rm lab} + \gamma_t^3 \frac{ \langle v_t^{\rm lab}, a_t^{\rm lab} \rangle}{c^2} v_t^{\rm lab}),
\end{align*}
where $\gamma_t$ is the Lorentz factor defined in Definition~\ref{def:LorentzFactor}.

To solve for $a_t^{\rm lab}$, we could first decompose $a_t^{\rm lab}$ by
\begin{align*}
    a_t^{\rm lab} = a_{t,\parallel}^{\rm lab} + a_{t,\perp}^{\rm lab},
\end{align*}
where $a_{t,\parallel}^{\rm lab}$ denotes the component of $a_t^{\rm lab}$ parallel with $v_t^{\rm lab}$, and $a_{t,\perp}^{\rm lab}$ denotes the component of $a_t^{\rm lab}$ perpendicular with $v_t^{\rm lab}$.

According to the definition of parallel and perpendicular, we have
\begin{align*}
    a_{t,\parallel}^{\rm lab} = & ~ \frac{ \langle v_t^{\rm lab}, a_t^{\rm lab} \rangle}{\|v_t^{\rm lab}\|_2^2} v_t^{\rm lab}, \\
    a_{t,\perp}^{\rm lab} = & ~ a_t^{\rm lab} - a_{t,\parallel}^{\rm lab}.
\end{align*}

Then we have
\begin{align}
    f_t^{\rm local} = & ~ m^{\rm lab}  (\gamma_t a_t^{\rm lab} + \gamma_t^3 \frac{ \langle v_t^{\rm lab}, a_{t}^{\rm lab} \rangle}{c^2} v_t^{\rm lab}) \notag \\
    = & ~ m^{\rm lab}  (\gamma_t (a_{t,\parallel}^{\rm lab} + a_{t,\perp}^{\rm lab}) + \gamma_t^3 \frac{ \langle v_t^{\rm lab}, a_{t,\parallel}^{\rm lab} + a_{t,\perp}^{\rm lab} \rangle}{c^2} v_t^{\rm lab}) \notag \\
    = & ~ m^{\rm lab}  (\gamma_t (a_{t,\parallel}^{\rm lab} + a_{t,\perp}^{\rm lab}) + \gamma_t^3 \frac{ \langle v_t^{\rm lab}, a_{t,\parallel}^{\rm lab} \rangle}{c^2} v_t^{\rm lab}), \label{eq:f_split}
\end{align}
where the first step follows Lemma~\ref{lem:equiv_relativistic_force:formal}, the second step decomposes $a_t^{\rm lab}$, and the last step follows from the simple fact that $\langle v_t^{\rm lab}, a_{t,\perp}^{\rm lab} \rangle = 0$.

Then we decompose the $f_t^{\rm local}$ to $f_{t, \parallel}^{\rm local}$ and $f_{t, \perp}^{\rm local}$, where $f_{t, \parallel}^{\rm local}$ denotes the component of $f_t^{\rm local}$ parallel with $v_t^{\rm lab}$, and $f_{t,\perp}^{\rm local}$ denotes the component of $f_t^{\rm local}$ perpendicular with $v_t^{\rm lab}$.

For the perpendicular component, we have
\begin{align*}
    f_{t, \perp}^{\rm local} = & ~ m^{\rm lab}\gamma_t  a_{t,\perp}^{\rm lab} \\
    a_{t,\perp}^{\rm lab} =  & ~ \frac{f_{t, \perp}^{\rm local}}{m^{\rm lab}\gamma_t},
\end{align*}
where the first step uses the perpendicular part from Eq.~\ref{eq:f_split}, and the second step rewrites the equation to get a closed-form solution for $a_{t,\perp}^{\rm lab}$.

For the parallel component, we have
\begin{align*}
    f_{t, \parallel}^{\rm local} = & ~ m^{\rm lab} (\gamma_t  a_{t,\parallel}^{\rm lab} + \gamma_t^3 a_{t,\parallel}^{\rm lab} \frac{\|v_t^{\rm lab}\|_2^2}{c^2}) \\
    = & ~ m^{\rm lab} a_{t,\parallel}^{\rm lab}(\gamma_t + \gamma_t^3 \frac{\|v_t^{\rm lab}\|_2^2}{c^2}) \\
    a_{t,\parallel}^{\rm lab} = & ~ \frac{f_{t, \parallel}^{\rm local}}{m^{\rm lab}(\gamma_t + \gamma_t^3 \frac{\|v_t^{\rm lab}\|_2^2}{c^2})},
\end{align*}
where the first step uses the parallel part from Eq.~\ref{eq:f_split}, the second step factors out the $a_{t,\parallel}^{\rm lab}$, and the last step rewrites the equation to get a closed-form solution for $a_{t,\parallel}^{\rm lab}$.

Then, we can combine these two components
\begin{align*}
    a_t^{\rm lab} = & ~\frac{f_{t, \perp}^{\rm local}}{m^{\rm lab}\gamma_t} + \frac{f_{t, \parallel}^{\rm local}}{m^{\rm lab}(\gamma_t + \gamma_t^3 \frac{\|v_t^{\rm lab}\|_2^2}{c^2})} \\
    = & ~\frac{f_t^{\rm local} - \frac{\langle v_t^{\rm lab}, f_t^{\rm local} \rangle}{\|v_t^{\rm lab}\|_2^2} v_t^{\rm lab}}{m^{\rm lab}\gamma_t} + \frac{\frac{ \langle v_t^{\rm lab}, f_t^{\rm local} \rangle}{\|v_t^{\rm lab}\|_2^2} v_t^{\rm lab}}{m^{\rm lab}(\gamma_t + \gamma_t^3 \frac{\|v_t^{\rm lab}\|_2^2}{c^2})} \\
    = & ~\frac{1}{m^{\rm lab}\gamma_t} f_t^{\rm local} + \frac{1}{m^{\rm lab}} ( \frac{1}{\gamma_t(1 + \gamma_t^2 \frac{\|v_t^{\rm lab}\|_2^2}{c^2})} - \frac{1}{\gamma_t} ) \frac{ \langle v_t^{\rm lab}, f_t^{\rm local} \rangle}{\|v_t^{\rm lab}\|_2^2} v_t^{\rm lab} \\
    = & ~\frac{1}{m^{\rm lab}\gamma_t} f_t^{\rm local} + \frac{1}{m^{\rm lab}} ( \frac{1}{\gamma_t \frac{c^2}{c^2 - \|v_t^{\rm lab}\|_2^2} } - \frac{1}{\gamma_t} ) \frac{ \langle v_t^{\rm lab}, f_t^{\rm local} \rangle}{\|v_t^{\rm lab}\|_2^2} v_t^{\rm lab} \\
    = & ~\frac{1}{m^{\rm lab}\gamma_t} f_t^{\rm local} + \frac{1}{m^{\rm lab}} ( \frac{c^2 - \|v_t^{\rm lab}\|_2^2}{\gamma_t c^2} - \frac{1}{\gamma_t} ) \frac{ \langle v_t^{\rm lab}, f_t^{\rm local} \rangle}{\|v_t^{\rm lab}\|_2^2} v_t^{\rm lab} \\
    = & ~\frac{1}{m^{\rm lab}\gamma_t} f_t^{\rm local} - \frac{1}{m^{\rm lab} \gamma_t} \frac{\|v_t^{\rm lab}\|_2^2}{c^2} \frac{ \langle v_t^{\rm lab}, f_t^{\rm local} \rangle}{\|v_t^{\rm lab}\|_2^2} v_t^{\rm lab} \\
    = & ~ \frac{1}{m^{\rm lab} \gamma_t}(f_t^{\rm local} - \frac{\langle v_t^{\rm lab}, f_t^{\rm local} \rangle}{c^2} v_t^{\rm lab}),
\end{align*}
where the first step combines the two terms, the second step decomposes the components, the third step factors out the $\gamma_t$ in denominator, the forth uses the fact that $\gamma_t^2 = \frac{1}{1 - \|v_t^{\rm lab}\|_2^2/c^2}$, the fifth step moves the denominator into numerator, the sixth step merges two terms, and the last step factors out the $\frac{1}{m^{\rm lab} \gamma_t}$.
\end{proof}

\subsection{Speed Limit} \label{sub:app:speed_limit}

In this subsection, we first calculate the derivative of the squared norm of velocity, then restate the Theorem~\ref{thm:vel:informal} and provide its proof.

\begin{lemma}[Derivative of the squared norm of velocity]\label{lem:velocity_derivative}
    Let $X(t) := \frac{1}{2}\|v_t^{\rm lab}\|_2^2$. Then we have
    \begin{align*}
        \frac{\d X(t)}{\d t} = \frac{ \langle f_t^{\rm local}, v_t^{\rm lab} \rangle}{m^{\rm lab} \gamma_t}(1 - \frac{\|v_t^{\rm lab}\|_2^2}{c^2}). 
    \end{align*}
\end{lemma}
\begin{proof}
Recall the sampling ODE
\begin{align}\label{eq:tmp}
    a_t^{\rm lab} = \frac{1}{m^{\rm lab} \gamma_t}(f_t^{\rm local} - \frac{\langle v_t^{\rm lab}, f_t^{\rm local} \rangle}{c^2} v_t^{\rm lab}),
\end{align}
where $\gamma_t$ is the Lorentz factor at lab time $t$ defined in Definition~\ref{def:LorentzFactor}.

We can show that
\begin{align*}
    \frac{\d X(t)}{\d t} = &~ \frac{\d}{\d t} \frac{1}{2}\|v_t^{\rm lab} \|_2^2 \\
    = &~ \langle v_t^{\rm lab}, \frac{\d}{\d t} v_t^{\rm lab} \rangle \\
    = &~ \langle v_t^{\rm lab}, a_t^{\rm lab} \rangle \\
    = &~ \langle v_t^{\rm lab}, \frac{1}{m^{\rm lab} \gamma_t}(f_t^{\rm local} - \frac{\langle v_t^{\rm lab}, f_t^{\rm local} \rangle}{c^2} v_t^{\rm lab}) \rangle \\
    = &~ \frac{1}{m^{\rm lab} \gamma_t} \langle f_t^{\rm local}, v_t^{\rm lab} \rangle - \frac{\langle v_t^{\rm lab}, f_t^{\rm local} \rangle}{m^{\rm lab} \gamma_t c^2} \| v_t^{\rm lab} \|_2^2 \\
    = &~ \frac{ \langle f_t^{\rm local}, v_t^{\rm lab} \rangle}{m^{\rm lab} \gamma_t}(1 - \frac{\|v_t^{\rm lab}\|_2^2}{c^2}) 
\end{align*}
where the first step follows from the definition of $X(t)$, the second step follows from the chain rule, the third step follows from the basic fact, the fourth step follows from the Eq.~\eqref{eq:tmp}, the fifth and last step follows from basic algebra.
\end{proof}

Here, we restate the Theorem~\ref{thm:vel:informal} and state its proof.

\begin{theorem}[Speed Limit, formal version of Theorem~\ref{thm:vel:informal}] \label{thm:vel:formal}
For a ForM model with sampling path $x : [0,T) \to \R^n$, the velocity satisfies
\begin{align*}
\| \dot{x}_t \|_2 < c \quad \text{for all } t \in [0,T).
\end{align*}
\end{theorem}
\begin{proof}
Let $X(t) := \tfrac12 \|v_t^{\rm lab}\|_2^2$. By Lemma~\ref{lem:velocity_derivative}, we have
    \begin{align*}
        \frac{\d X(t)}{\d t} 
        \;=\; \frac{1}{m^{\rm lab} \gamma_t}
        \,\langle f_t^{\rm local}, v_t^{\rm lab} \rangle
        \,\Bigl(1 - \tfrac{\|v_t^{\rm lab}\|_2^2}{c^2}\Bigr).
    \end{align*}
    Observe that the factor $\bigl(1 - \|v_t^{\rm lab}\|_2^2 / c^2 \bigr)$ becomes negative if ever $\|v_t^{\rm lab}\|_2 > c$, and it is zero when $\|v_t^{\rm lab}\|_2 = c$. Thus, if the velocity norm were to exceed $c$ at some time, the derivative of $X(t)$ at that moment would be negative, forcing $X(t)$ (i.e., $\|v_t^{\rm lab}\|_2^2$) to decrease rather than increase. In particular, once $\|v_t^{\rm lab}\|_2^2$ reaches $c^2$, it cannot increase further.

    Consequently, for all $t \in [0,T)$ we must have $\|v_t^{\rm lab}\|_2 < c$, which proves the speed limit. Equivalently, since $\dot{x}_t = v_t^{\rm lab}$ in our notation, we conclude
    \begin{align*}
        \|\dot{x}_t\|_2 < c,
        \quad \forall t \in [0,T).
    \end{align*}
    Thus, we complete the proof.
    \end{proof}

\subsection{ForM with TrigFlow} \label{sub:app:form_trig}

We restate the Theorem~\ref{thm:form_trig:informal} and provide its proof.

\begin{theorem}[ForM with TrigFlow, formal version of Theorem~\ref{thm:form_trig:informal}] \label{thm:form_trig:formal}
We let $m = 1$ for simplicity in ForM. Giving a the interpolation $x_t = \alpha_t x_1 + \sigma_t x_0$, where $\alpha_T = 1$, $\alpha_0 = 0$, $\sigma_T = 0$, $\sigma_0 = 1$. We let $F_t(x_t)$ denote a vector map of force, a trainable neuron network parameterized with $\theta$. We select the $\alpha_t$ and $\sigma_t$ identical with TrigFlow \cite{ls24}, where $\alpha_t = \sin(t)$ and $\sigma_t = \cos(t)$, $T = \frac{\pi}{2}$. Then, force interpolation could be simplified to 
\begin{align*}
    f_t(x_t) =
    & ~ \frac{(\cos(t)x_1 - \sin(t)x_0) \cdot (-\sin(t)x_1 - \cos(t)x_0)}{c^2 - (\cos(t)x_1 - \sin(t)x_0)^2}\\
    & ~ (\cos(t)x_1 - \sin(t)x_0)).
\end{align*}
\end{theorem}
\begin{proof}
    We can show that
    \begin{align*}
        f_t(x_t) = & ~ \gamma_t \ddot{x}_t + \gamma_t^3 \frac{\langle \dot{x}_t, \ddot{x}_t \rangle}{c^2}\dot{x}_t \\
        = & ~ (1 - \frac{(\cos(t)x_1 - \sin(t)x_0)^2}{c^2})^{-\frac{1}{2}} (-\sin(t)x_1 - \cos(t)x_0) + \\
        & ~ \frac{(\cos(t)x_1 - \sin(t)x_0) \cdot (-\sin(t)x_1 - \cos(t)x_0)}{c^2 - (\cos(t)x_1 - \sin(t)x_0)^2}\\
        & ~ (\cos(t)x_1 - \sin(t)x_0)),
    \end{align*}
    where we use Definition~\ref{def:RelativisticForce} and substitute acceleration $a_t^{\rm lab}$ with $\ddot{x}_t$ and $v_t^{\rm lab}$ with $\dot{x}_t$ in the first step, and substitute $x_t = \sin(t) x_1 + \cos(t) x_0$ and calculate it's derivative in the second step.
\end{proof}

\subsection{Relativistic Force Property} \label{sub:app:force}

In this subsection, we restate Lemma~\ref{lem:equiv_relativistic_force:informal}, and show its proof.

\begin{lemma}[Equivalent Form of Relativistic Force, formal version of Lemma~\ref{lem:equiv_relativistic_force:informal}]\label{lem:equiv_relativistic_force:formal}
Let $p^{\rm lab}$ be the momentum defined in Eq.~\eqref{eq:p}, $\gamma_t$ be the Lorentz factor at lab time $t$ defined in Definition~\ref{def:LorentzFactor}, $\tau$ denotes the proper time, $v_t^{\rm lab} = \dot{x}_t$ denotes the velocity, 
$a_t^{\rm lab} = \ddot{x}_t$ denotes the acceleration.
The relativistic force, defined as the time derivative of the momentum in the lab frame, can be written as
\begin{align*}
f^{\rm local} =  m^{\rm lab}  (\gamma_t a_t^{\rm lab} + \gamma_t^3 \frac{ \langle v_t^{\rm lab}, a_t^{\rm lab} \rangle}{c^2} v_t^{\rm lab}).
\end{align*}

\end{lemma}
\begin{proof}
We can show that
\begin{align*}
f^{\rm local}  = & ~ \frac{\d p^{\rm lab}}{\d \tau}  \\
= & ~ \frac{\d m^{\rm lab} v_t^{\rm lab} \gamma_t}{\d t} \\
= & ~ m^{\rm lab} \frac{\d v_t^{\rm lab} \gamma_t}{\d t} \\
= & ~ m^{\rm lab}  (\gamma_t a_t^{\rm lab} + \gamma_t^3 \frac{ \langle v_t^{\rm lab}, a_t^{\rm lab} \rangle}{c^2} v_t^{\rm lab}).
\end{align*}
where the first step follows from Eq.~\eqref{eq:f_local}, the second step follows Eq.~\eqref{eq:p} and Definition~\ref{def:ProperTime}, the third step is true because $m^{\rm lab}$ is a constant, and the last step takes the derivative.
\end{proof}



%%%% Cut-line between first 10 pages and appendix







%%% some writing rules

%% Writing rule for creating tags.
%% Tags :
%% Theorem    \ref{thm:bla_bla}
%% Lemma      \ref{lem:bla_bla}
%% Claim      \ref{cla:bla_bla}
%% Corollary  \ref{cor:bla_bla}
%% Fact       \ref{fac:bla_bla}
%% Definition \ref{def:bla_bla}
%% Section    \ref{sec:bla_bla}
%% Subsection \ref{sub:bla_bla}
%% Equation   \ref{eq:bla_bla}



\end{document}



%%%%%%%%%%%%%%%%%%%%%%%%%%%%%%%%%%%%%%%%%%%%%%%%%%%%%%%%%%%%%%%%%%%%%%%%%%%%%%%%%%%%%%%%%%%%%%%%%%%%%%%%%%%%%%%%%%%%%%%%%%%%%%%%%%%%%%%%%%%%%%%%%%%%%%%%%%%%%%%%%%%%%%%%%%%%%%%%%%%%%%%%%%%%%%%%%%%%%%%%%%%%%%%%%%%%%%%%%%%%%%%%%%%%%%%%%%%%%%%%%%%%%%%%%%%%%%%%%%%%%%%%%%%%%%%%%%%%%%%%%%%%%%%%%%%%%%%%%%%%%%%%%%%%%%%%%%%%%%%%%%%%%%%%%%%%%%%%%%%%%%%%%%%%%%%%%%%%%%%%%%%%%%%%%%%%%%%%%%%%%%%%%%%%%%%%%%%%%%%%%%%%%%%%%%%%%%%%%%%%%%%%%%%%%%%%%%%%%%%%%%%%%%%%%%%%%%%%%%%%%%
