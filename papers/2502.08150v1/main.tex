\def\isarxiv{1} %%% for icml submission version, we comment this line

\ifdefined\isarxiv
\documentclass[11pt]{article}
\usepackage[numbers]{natbib}

\else
% \documentclass{article}
\documentclass[sigconf,anonymous,review]{acmart}
\fi


\usepackage{amsmath}
\usepackage{amsthm}
\ifdefined\isarxiv %%%Zhao: I added the following lines
\usepackage{amssymb}
\else
\fi
\usepackage{algorithm}
\usepackage{subfig}
\usepackage{algpseudocode}
\usepackage{graphicx}
\usepackage{grffile}
\usepackage{wrapfig,epsfig}
\usepackage{url}
\usepackage{xcolor}
\usepackage{epstopdf}


\usepackage{bbm}
\usepackage{dsfont}
 
\allowdisplaybreaks
 

\ifdefined\isarxiv

\let\C\relax
\usepackage{tikz}
\usepackage{hyperref}  %%% arxiv don't allow this.
\hypersetup{colorlinks=true,citecolor=blue,linkcolor=blue} %%% Zhao : maybe we should comment this in submission.
\usetikzlibrary{arrows}
\usepackage[margin=1in]{geometry}

\else

\usepackage{microtype}
\usepackage{hyperref}
\definecolor{mydarkblue}{rgb}{0,0.08,0.45}
\hypersetup{colorlinks=true, citecolor=mydarkblue,linkcolor=mydarkblue}
 

\fi
 
\graphicspath{{./figs/}}

\theoremstyle{plain}
\newtheorem{theorem}{Theorem}[section]
\newtheorem{lemma}[theorem]{Lemma}
\newtheorem{definition}[theorem]{Definition}
\newtheorem{notation}[theorem]{Notation}
%\newtheorem{proof}[theorem]{Proof}
\newtheorem{proposition}[theorem]{Proposition}
\newtheorem{corollary}[theorem]{Corollary}
\newtheorem{conjecture}[theorem]{Conjecture}
\newtheorem{assumption}[theorem]{Assumption}
\newtheorem{observation}[theorem]{Observation}
\newtheorem{fact}[theorem]{Fact}
\newtheorem{remark}[theorem]{Remark}
\newtheorem{claim}[theorem]{Claim}
\newtheorem{example}[theorem]{Example}
\newtheorem{problem}[theorem]{Problem}
\newtheorem{open}[theorem]{Open Problem}
\newtheorem{property}[theorem]{Property}
\newtheorem{hypothesis}[theorem]{Hypothesis}

\newcommand{\wh}{\widehat}
\newcommand{\wt}{\widetilde}
\newcommand{\ov}{\overline}
\newcommand{\N}{\mathcal{N}}
\newcommand{\R}{\mathbb{R}}
\newcommand{\M}{\mathcal{M}}
\newcommand{\Q}{\mathcal{Q}}
\newcommand{\PP}{\mathcal{P}}
\newcommand{\KL}{\mathrm{KL}}
\newcommand{\RHS}{\mathrm{RHS}}
\newcommand{\LHS}{\mathrm{LHS}}
\renewcommand{\d}{\mathrm{d}}
\renewcommand{\i}{\mathbf{i}}
\renewcommand{\tilde}{\wt}
\renewcommand{\hat}{\wh}
\newcommand{\Tmat}{{\cal T}_{\mathrm{mat}}}
\newcommand{\True}{\mathrm{true}}


\DeclareMathOperator*{\E}{{\mathbb{E}}}
\DeclareMathOperator*{\var}{\mathrm{Var}}
\DeclareMathOperator*{\Z}{\mathbb{Z}}
\DeclareMathOperator*{\C}{\mathbb{C}}
\DeclareMathOperator*{\D}{\mathcal{D}}
\DeclareMathOperator*{\median}{median}
\DeclareMathOperator*{\mean}{mean}
\DeclareMathOperator{\OPT}{OPT}
\DeclareMathOperator{\supp}{supp}
\DeclareMathOperator{\poly}{poly}

\DeclareMathOperator{\nnz}{nnz}
\DeclareMathOperator{\sparsity}{sparsity}
\DeclareMathOperator{\rank}{rank}
\DeclareMathOperator{\diag}{diag}
\DeclareMathOperator{\dist}{dist}
\DeclareMathOperator{\cost}{cost}
\DeclareMathOperator{\vect}{vec}
\DeclareMathOperator{\tr}{tr}
\DeclareMathOperator{\dis}{dis}
\DeclareMathOperator{\cts}{cts}



\makeatletter
\newcommand*{\RN}[1]{\expandafter\@slowromancap\romannumeral #1@}
\makeatother
% \newcommand{\Zhao}[1]{{\color{red}[Zhao: #1]}}
% \newcommand{\Yang}[1]{{\color{orange}[Yang: #1]}}
% \newcommand{\Bo}[1]{{\color{blue}[Bo: #1]}}
% \newcommand{\Xiaoyu}[1]{{\color{purple}[Xiaoyu: #1]}}
% \newcommand{\Zhizhou}[1]{{\color{violet}[Zhizhou: #1]}}
% \newcommand{\Zhenmei}[1]{{\color{purple}[Zhenmei: #1]}}
% \newcommand{\Mingda}[1]{{\color{brown}[Mingda: #1]}}
% \newcommand{\InernNameB}[1]{{\color{blue}[InternNameB: #1]}} %%%Change to intern name


\usepackage{lineno}
% \def\linenumberfont{\normalfont\small}


\begin{document}

\ifdefined\isarxiv

\date{}

\title{Force Matching with Relativistic Constraints: A Physics-Inspired Approach to Stable and Efficient Generative Modeling}

% \author{}

% \iffalse
\author{
Yang Cao\thanks{\texttt{ ycao4@wyomingseminary.org}. Wyoming Seminary}
\and
Bo Chen\thanks{\texttt{ bc7b@mtmail.mtsu.edu}. Middle Tennessee State University.}
\and
Xiaoyu Li\thanks{\texttt{
xli216@stevens.edu}. Stevens Institute of Technology.}
\and
Yingyu Liang\thanks{\texttt{
yingyul@hku.hk}. The University of Hong Kong. \texttt{
yliang@cs.wisc.edu}. University of Wisconsin-Madison.} 
\and
Zhizhou Sha\thanks{\texttt{
shazz20@mails.tsinghua.edu.cn}. Tsinghua University.}
\and
Zhenmei Shi\thanks{\texttt{
zhmeishi@cs.wisc.edu}. University of Wisconsin-Madison.}
\and
Zhao Song\thanks{\texttt{ magic.linuxkde@gmail.com}. Simons Institute for the Theory of Computing, University of California, Berkeley.}
\and
Mingda Wan\thanks{\texttt{
dylan.r.mathison@gmail.com}. Anhui University.}
}
% \fi

\else

\title{Force Matching with Relativistic Constraints: A Physics-Inspired Approach to Stable and Efficient Generative Modeling}
% Newtonian and Relativistic Force Matching: Towards Stable and Efficient Generative Models
% Beyond Flow Matching: A Relativistic Force-Based Framework for Generative Modeling
% Force Matching for Generative Models: A Relativistic Approach to Stability and Efficiency
% Bridging Generative Modeling and Physics: Force Matching with Relativistic Constraints
% Force Matching: A Relativistic Approach to Stable and Efficient Generative Modeling
% Force Matching with Relativistic Constraints: Towards Stable andEfficient Cenerative Models

\author{%
}

\fi


\ifdefined\isarxiv
\begin{titlepage}
  \maketitle
  \begin{abstract}
\begin{abstract}


The choice of representation for geographic location significantly impacts the accuracy of models for a broad range of geospatial tasks, including fine-grained species classification, population density estimation, and biome classification. Recent works like SatCLIP and GeoCLIP learn such representations by contrastively aligning geolocation with co-located images. While these methods work exceptionally well, in this paper, we posit that the current training strategies fail to fully capture the important visual features. We provide an information theoretic perspective on why the resulting embeddings from these methods discard crucial visual information that is important for many downstream tasks. To solve this problem, we propose a novel retrieval-augmented strategy called RANGE. We build our method on the intuition that the visual features of a location can be estimated by combining the visual features from multiple similar-looking locations. We evaluate our method across a wide variety of tasks. Our results show that RANGE outperforms the existing state-of-the-art models with significant margins in most tasks. We show gains of up to 13.1\% on classification tasks and 0.145 $R^2$ on regression tasks. All our code and models will be made available at: \href{https://github.com/mvrl/RANGE}{https://github.com/mvrl/RANGE}.

\end{abstract}



  \end{abstract}
  \thispagestyle{empty}
\end{titlepage}

{\hypersetup{linkcolor=black}
\tableofcontents
}
\newpage

\else

\begin{abstract}
\begin{abstract}


The choice of representation for geographic location significantly impacts the accuracy of models for a broad range of geospatial tasks, including fine-grained species classification, population density estimation, and biome classification. Recent works like SatCLIP and GeoCLIP learn such representations by contrastively aligning geolocation with co-located images. While these methods work exceptionally well, in this paper, we posit that the current training strategies fail to fully capture the important visual features. We provide an information theoretic perspective on why the resulting embeddings from these methods discard crucial visual information that is important for many downstream tasks. To solve this problem, we propose a novel retrieval-augmented strategy called RANGE. We build our method on the intuition that the visual features of a location can be estimated by combining the visual features from multiple similar-looking locations. We evaluate our method across a wide variety of tasks. Our results show that RANGE outperforms the existing state-of-the-art models with significant margins in most tasks. We show gains of up to 13.1\% on classification tasks and 0.145 $R^2$ on regression tasks. All our code and models will be made available at: \href{https://github.com/mvrl/RANGE}{https://github.com/mvrl/RANGE}.

\end{abstract}


\end{abstract}

\begin{CCSXML}
<ccs2012>
   <concept>
       <concept_id>10010147.10010257</concept_id>
       <concept_desc>Computing methodologies~Machine learning</concept_desc>
       <concept_significance>500</concept_significance>
       </concept>
   <concept>
       <concept_id>10010147.10010178</concept_id>
       <concept_desc>Computing methodologies~Artificial intelligence</concept_desc>
       <concept_significance>500</concept_significance>
       </concept>
 </ccs2012>
\end{CCSXML}

\ccsdesc[500]{Computing methodologies~Machine learning}
\ccsdesc[500]{Computing methodologies~Artificial intelligence}


% \begin{CCSXML}
% <ccs2012>
%  <concept>
%   <concept_id>00000000.0000000.0000000</concept_id>
%   <concept_desc>Do Not Use This Code, Generate the Correct Terms for Your Paper</concept_desc>
%   <concept_significance>500</concept_significance>
%  </concept>
%  <concept>
%   <concept_id>00000000.00000000.00000000</concept_id>
%   <concept_desc>Do Not Use This Code, Generate the Correct Terms for Your Paper</concept_desc>
%   <concept_significance>300</concept_significance>
%  </concept>
%  <concept>
%   <concept_id>00000000.00000000.00000000</concept_id>
%   <concept_desc>Do Not Use This Code, Generate the Correct Terms for Your Paper</concept_desc>
%   <concept_significance>100</concept_significance>
%  </concept>
%  <concept>
%   <concept_id>00000000.00000000.00000000</concept_id>
%   <concept_desc>Do Not Use This Code, Generate the Correct Terms for Your Paper</concept_desc>
%   <concept_significance>100</concept_significance>
%  </concept>
% </ccs2012>
% \end{CCSXML}

% \ccsdesc[500]{Do Not Use This Code~Generate the Correct Terms for Your Paper}
% \ccsdesc[300]{Do Not Use This Code~Generate the Correct Terms for Your Paper}
% \ccsdesc{Do Not Use This Code~Generate the Correct Terms for Your Paper}
% \ccsdesc[100]{Do Not Use This Code~Generate the Correct Terms for Your Paper}

%%
%% Keywords. The author(s) should pick words that accurately describe
%% the work being presented. Separate the keywords with commas.
\keywords{Flow Matching, Diffusion Model, Generative Model}
%% A "teaser" image appears between the author and affiliation
%% information and the body of the document, and typically spans the
%% page.
\maketitle % Bo: error in KDD change template: abstract must be defined before maketitle command, so I move \maketitle after abstract.
\fi


\section{Introduction}
Backdoor attacks pose a concealed yet profound security risk to machine learning (ML) models, for which the adversaries can inject a stealth backdoor into the model during training, enabling them to illicitly control the model's output upon encountering predefined inputs. These attacks can even occur without the knowledge of developers or end-users, thereby undermining the trust in ML systems. As ML becomes more deeply embedded in critical sectors like finance, healthcare, and autonomous driving \citep{he2016deep, liu2020computing, tournier2019mrtrix3, adjabi2020past}, the potential damage from backdoor attacks grows, underscoring the emergency for developing robust defense mechanisms against backdoor attacks.

To address the threat of backdoor attacks, researchers have developed a variety of strategies \cite{liu2018fine,wu2021adversarial,wang2019neural,zeng2022adversarial,zhu2023neural,Zhu_2023_ICCV, wei2024shared,wei2024d3}, aimed at purifying backdoors within victim models. These methods are designed to integrate with current deployment workflows seamlessly and have demonstrated significant success in mitigating the effects of backdoor triggers \cite{wubackdoorbench, wu2023defenses, wu2024backdoorbench,dunnett2024countering}.  However, most state-of-the-art (SOTA) backdoor purification methods operate under the assumption that a small clean dataset, often referred to as \textbf{auxiliary dataset}, is available for purification. Such an assumption poses practical challenges, especially in scenarios where data is scarce. To tackle this challenge, efforts have been made to reduce the size of the required auxiliary dataset~\cite{chai2022oneshot,li2023reconstructive, Zhu_2023_ICCV} and even explore dataset-free purification techniques~\cite{zheng2022data,hong2023revisiting,lin2024fusing}. Although these approaches offer some improvements, recent evaluations \cite{dunnett2024countering, wu2024backdoorbench} continue to highlight the importance of sufficient auxiliary data for achieving robust defenses against backdoor attacks.

While significant progress has been made in reducing the size of auxiliary datasets, an equally critical yet underexplored question remains: \emph{how does the nature of the auxiliary dataset affect purification effectiveness?} In  real-world  applications, auxiliary datasets can vary widely, encompassing in-distribution data, synthetic data, or external data from different sources. Understanding how each type of auxiliary dataset influences the purification effectiveness is vital for selecting or constructing the most suitable auxiliary dataset and the corresponding technique. For instance, when multiple datasets are available, understanding how different datasets contribute to purification can guide defenders in selecting or crafting the most appropriate dataset. Conversely, when only limited auxiliary data is accessible, knowing which purification technique works best under those constraints is critical. Therefore, there is an urgent need for a thorough investigation into the impact of auxiliary datasets on purification effectiveness to guide defenders in  enhancing the security of ML systems. 

In this paper, we systematically investigate the critical role of auxiliary datasets in backdoor purification, aiming to bridge the gap between idealized and practical purification scenarios.  Specifically, we first construct a diverse set of auxiliary datasets to emulate real-world conditions, as summarized in Table~\ref{overall}. These datasets include in-distribution data, synthetic data, and external data from other sources. Through an evaluation of SOTA backdoor purification methods across these datasets, we uncover several critical insights: \textbf{1)} In-distribution datasets, particularly those carefully filtered from the original training data of the victim model, effectively preserve the model’s utility for its intended tasks but may fall short in eliminating backdoors. \textbf{2)} Incorporating OOD datasets can help the model forget backdoors but also bring the risk of forgetting critical learned knowledge, significantly degrading its overall performance. Building on these findings, we propose Guided Input Calibration (GIC), a novel technique that enhances backdoor purification by adaptively transforming auxiliary data to better align with the victim model’s learned representations. By leveraging the victim model itself to guide this transformation, GIC optimizes the purification process, striking a balance between preserving model utility and mitigating backdoor threats. Extensive experiments demonstrate that GIC significantly improves the effectiveness of backdoor purification across diverse auxiliary datasets, providing a practical and robust defense solution.

Our main contributions are threefold:
\textbf{1) Impact analysis of auxiliary datasets:} We take the \textbf{first step}  in systematically investigating how different types of auxiliary datasets influence backdoor purification effectiveness. Our findings provide novel insights and serve as a foundation for future research on optimizing dataset selection and construction for enhanced backdoor defense.
%
\textbf{2) Compilation and evaluation of diverse auxiliary datasets:}  We have compiled and rigorously evaluated a diverse set of auxiliary datasets using SOTA purification methods, making our datasets and code publicly available to facilitate and support future research on practical backdoor defense strategies.
%
\textbf{3) Introduction of GIC:} We introduce GIC, the \textbf{first} dedicated solution designed to align auxiliary datasets with the model’s learned representations, significantly enhancing backdoor mitigation across various dataset types. Our approach sets a new benchmark for practical and effective backdoor defense.


 %%% Section 1. Introduction
\section{Related Work}

\subsection{Large 3D Reconstruction Models}
Recently, generalized feed-forward models for 3D reconstruction from sparse input views have garnered considerable attention due to their applicability in heavily under-constrained scenarios. The Large Reconstruction Model (LRM)~\cite{hong2023lrm} uses a transformer-based encoder-decoder pipeline to infer a NeRF reconstruction from just a single image. Newer iterations have shifted the focus towards generating 3D Gaussian representations from four input images~\cite{tang2025lgm, xu2024grm, zhang2025gslrm, charatan2024pixelsplat, chen2025mvsplat, liu2025mvsgaussian}, showing remarkable novel view synthesis results. The paradigm of transformer-based sparse 3D reconstruction has also successfully been applied to lifting monocular videos to 4D~\cite{ren2024l4gm}. \\
Yet, none of the existing works in the domain have studied the use-case of inferring \textit{animatable} 3D representations from sparse input images, which is the focus of our work. To this end, we build on top of the Large Gaussian Reconstruction Model (GRM)~\cite{xu2024grm}.

\subsection{3D-aware Portrait Animation}
A different line of work focuses on animating portraits in a 3D-aware manner.
MegaPortraits~\cite{drobyshev2022megaportraits} builds a 3D Volume given a source and driving image, and renders the animated source actor via orthographic projection with subsequent 2D neural rendering.
3D morphable models (3DMMs)~\cite{blanz19993dmm} are extensively used to obtain more interpretable control over the portrait animation. For example, StyleRig~\cite{tewari2020stylerig} demonstrates how a 3DMM can be used to control the data generated from a pre-trained StyleGAN~\cite{karras2019stylegan} network. ROME~\cite{khakhulin2022rome} predicts vertex offsets and texture of a FLAME~\cite{li2017flame} mesh from the input image.
A TriPlane representation is inferred and animated via FLAME~\cite{li2017flame} in multiple methods like Portrait4D~\cite{deng2024portrait4d}, Portrait4D-v2~\cite{deng2024portrait4dv2}, and GPAvatar~\cite{chu2024gpavatar}.
Others, such as VOODOO 3D~\cite{tran2024voodoo3d} and VOODOO XP~\cite{tran2024voodooxp}, learn their own expression encoder to drive the source person in a more detailed manner. \\
All of the aforementioned methods require nothing more than a single image of a person to animate it. This allows them to train on large monocular video datasets to infer a very generic motion prior that even translates to paintings or cartoon characters. However, due to their task formulation, these methods mostly focus on image synthesis from a frontal camera, often trading 3D consistency for better image quality by using 2D screen-space neural renderers. In contrast, our work aims to produce a truthful and complete 3D avatar representation from the input images that can be viewed from any angle.  

\subsection{Photo-realistic 3D Face Models}
The increasing availability of large-scale multi-view face datasets~\cite{kirschstein2023nersemble, ava256, pan2024renderme360, yang2020facescape} has enabled building photo-realistic 3D face models that learn a detailed prior over both geometry and appearance of human faces. HeadNeRF~\cite{hong2022headnerf} conditions a Neural Radiance Field (NeRF)~\cite{mildenhall2021nerf} on identity, expression, albedo, and illumination codes. VRMM~\cite{yang2024vrmm} builds a high-quality and relightable 3D face model using volumetric primitives~\cite{lombardi2021mvp}. One2Avatar~\cite{yu2024one2avatar} extends a 3DMM by anchoring a radiance field to its surface. More recently, GPHM~\cite{xu2025gphm} and HeadGAP~\cite{zheng2024headgap} have adopted 3D Gaussians to build a photo-realistic 3D face model. \\
Photo-realistic 3D face models learn a powerful prior over human facial appearance and geometry, which can be fitted to a single or multiple images of a person, effectively inferring a 3D head avatar. However, the fitting procedure itself is non-trivial and often requires expensive test-time optimization, impeding casual use-cases on consumer-grade devices. While this limitation may be circumvented by learning a generalized encoder that maps images into the 3D face model's latent space, another fundamental limitation remains. Even with more multi-view face datasets being published, the number of available training subjects rarely exceeds the thousands, making it hard to truly learn the full distibution of human facial appearance. Instead, our approach avoids generalizing over the identity axis by conditioning on some images of a person, and only generalizes over the expression axis for which plenty of data is available. 

A similar motivation has inspired recent work on codec avatars where a generalized network infers an animatable 3D representation given a registered mesh of a person~\cite{cao2022authentic, li2024uravatar}.
The resulting avatars exhibit excellent quality at the cost of several minutes of video capture per subject and expensive test-time optimization.
For example, URAvatar~\cite{li2024uravatar} finetunes their network on the given video recording for 3 hours on 8 A100 GPUs, making inference on consumer-grade devices impossible. In contrast, our approach directly regresses the final 3D head avatar from just four input images without the need for expensive test-time fine-tuning.




\section{Preliminary} \label{sec:preliminary}
In this section, we first introduce the notations in Section~\ref{sec:notations}. Then, we present the general problem formulation in Section~\ref{sec:problem_formulation}.

\subsection{Notations}\label{sec:notations}
For any positive integer $n$, we use $[n]$ to denote the set $\{1, 2, \ldots, n\}$. We use $\mathbb{N}_+$ to represent the set of all positive integers. For two sets $\mathcal{B}$ and $\mathcal{C}$, we denote the set difference as $\mathcal{B} \setminus \mathcal{C}:=\{x\in \mathcal{B}:x\notin\mathcal{C}\}$. For a vector $x \in \mathbb{R}^d$, $\Diag(d)$ denotes a diagonal matrix $X \in \mathbb{R}^{d \times d}$, where the diagonal entries satisfy $X_{i,i} = x_i$ for all $i \in [d]$, and all off-diagonal entries are zero. We use $\mathbf{1}_n$ to denote an $n$-dimensional column vector with all entries equal to one.


\begin{figure}[!ht]
    \centering
    \includegraphics[width=0.95\linewidth]{our_research.pdf}
    \caption{
    Our research objective. This figure presents the goal of our study: creating a more equitable desk-rejection system. Consider Professor A, who has carelessly submitted numerous papers exceeding the submission limit, collaborating with another senior researcher (Professor B) with many submissions, and a young student with only one paper. Our proposed system prioritizes desk-rejecting papers from authors with a large number of submissions first, thereby increasing the student’s chances of having their paper accepted. This approach aims to mitigate the disparity in the impact of desk rejections and promote fairness.
    }
    \label{fig:our_research}
\end{figure}

\subsection{Problem Formulation} \label{sec:problem_formulation}

In this section, we further introduce the actual problem we will investigate in this paper, where we begin with introducing the definition for three kinds of authors that will appear later in our discussion. 


\begin{definition}[Submission Limit Problem]\label{def:submit_limit_problem}
    Let $\mathcal{A} = \{a_1, a_2, \dots, a_n\}$ denote the set of $n$ authors, and let $\mathcal{P} = \{p_1, p_2, \dots, p_m\}$ denote the set of $m$ papers. Each author $a_i \in \mathcal{A}$ has a subset of papers $P_i \subseteq \mathcal{P}$, and each paper $p_j \in \mathcal{P}$ is authored by a subset of authors $A_j \subseteq \mathcal{A}$. For each author, $a_i \in \mathcal{A}$, let $C_i$ denote the set of all coauthors of $a_i$ and let $x \in \mathbb{N}_+$ denote the maximum number of papers each author can submit. 

    The goal is to find a subset $S \subseteq \mathcal{P}$ of papers (to keep) such that for every $a_i\in\mathcal{A},$
    \begin{align*}
        \underbrace{|\{p_j \in  S : a_i \in A_j\}|}_{\#\mathrm{remained~papers~of~author}~a_i} \leq x.  
    \end{align*}
    or equivalently find a subset $\ov{S} \subseteq \mathcal{P}$ of papers (to reject) such that for every $a_i \in \mathcal{A}$,
        \begin{align*}
        |P_i| - \underbrace{|\{j \in  \ov{S} : i \in A_j\}|}_{\#{\mathrm{rejected~papers~of~author}~}a_i} \leq x.  
    \end{align*}
\end{definition}

We now present several fundamental facts related to Definition~\ref{def:submit_limit_problem}, which can be easily verified through basic set theory. 
\begin{fact}
    For any author $a_i \in \mathcal{A}$ and paper $p_j \in \mathcal{P}$, $a_i \in A_j$ if and only if $p_j \in P_i$.
\end{fact}

\begin{fact}
    For each author $a_i \in \mathcal{A}$, the number of papers submitted by the author can be formulated as:
    \begin{align*}
        |P_i| = |\{p_j \in \mathcal{P} : a_i \in A_j\}|.
    \end{align*}
\end{fact}

\begin{fact}
    For each paper $j \in [m]$, the number of authors of this paper can be formulated as:
    \begin{align*}
        |A_j| = |\{a_i \in \mathcal{A} : p_j \in P_i\}|.
    \end{align*}
\end{fact}

\begin{fact}
    For each author $a_i \in \mathcal{A}$, the set of coauthors for author $a_i$ can be formulated as:
    \begin{align*}
        C_i = (\bigcup_{p_j \in P_i} A_j) \setminus \{ a_i \}.
    \end{align*}
\end{fact}


\section{Force Matching} \label{sec:form}

In this section, we introduce Force Matching (ForM), a new architecture for generative models, and provide its theoretical analysis. In Section~\ref{sub:obj}, we introduce the training objective of Force Matching. Then, in Section~\ref{sub:samp_ode}. In Section~\ref{sub:speed_limit}, we illustrate and discuss the speed limitation of ForM. In Section~\ref{sub:form_trig}, We show the interpolation path for ForM.


\subsection{Definition of Force Matching Objective} \label{sub:obj}

Next, we define the training objective of Force Matching.

\begin{definition}[Force Matching Objective]
\label{def:FormObjective}
The training objective of Force Matching (ForM) is defined by
\begin{align*}
     \mathcal{L}_{\rm ForM}(\theta) := \E_{t \sim {\sf Uniform}[0,T], x_1 \sim \mathcal{D}} 
    [\| F_t(x_t) - f_t(x_t)\|_2^2],   
\end{align*}
where $\mathcal{D}$ is the target data distribution, $f_t(x_t)$ is the target relativistic force defined in Definition~\ref{def:RelativisticForce}, and $F_t(x)$ is a trainable neural network parameterized with $\theta$.
\end{definition}

\subsection{Our Result I: Sampling ODE} \label{sub:samp_ode}

We define an ordinary differential equation (ODE) in order to get the position based on a given relativistic force. 

\begin{theorem}[Sampling ODE, informal version of Theorem~\ref{thm:ode_form:formal}]\label{thm:ode_form:informal}
    Giving the force at position $x_t$ denoted as $f_t(x_t)$, we could solve for ForM sampling path $x_t$ by the following ODE
    \begin{align*}
    \ddot{x}_t = \frac{1}{m^{\rm lab} \gamma_t}(f_t^{\rm local} - \frac{\langle v_t^{\rm lab}, f_t^{\rm local} \rangle}{c^2} v_t^{\rm lab}),
    \end{align*}
    where $x_0 \sim \N(0,I)$, $\dot{x}_0 = 0$.
\end{theorem}

Theorem~\ref{thm:ode_form:informal} shows how to derive the position $x_t$ from the relativistic force field $f_t(x_t)$. Unlike first-order flow-based methods, ForM naturally involves a second-order ODE because it encodes the evolution of both position and velocity under relativistic constraints. This allows for more expressive and physically-motivated sampling trajectories, where velocity constraints can help stabilize the generative process. Once a neural network $F_t(x)$ is trained to approximate $f_t(x)$, the sampling procedure integrates this second-order ODE to produce samples consistent with the target distribution.


\subsection{Our Result II: Speed Limit} \label{sub:speed_limit}

One property of relativistic mechanics is the velocity will always be under the constant $c$, which is the speed of light.

In reality, the speed of light is $c \approx 3 \times 10^8$. For any $v_t$, the speed $ \|v_t\|_2$ can approach but never exceed $c$. 
This property stabilizes the generating process. We formalize and prove this in Theorem~\ref{thm:vel:informal}.

\begin{theorem}[Speed Limit, informal version of Theorem~\ref{thm:vel:formal}] \label{thm:vel:informal}
For a ForM model with sampling path $x : [0,T) \to \R^n$, the velocity satisfies
\begin{align*}
\| \dot{x}_t \|_2 < c \quad, \forall t \in [0,T).
\end{align*}
\end{theorem}


It shows that the sample velocity remains strictly below $c$ at all times under relativistic constraints. Practically, this upper bound on velocity helps mitigate risks of numerical instability or ``exploding gradients" that can sometimes arise in diffusion- or flow-based generative models. By capping the speed of samples, ForM maintains a controlled and stable evolution in high-dimensional spaces. This provides a theoretical guarantee of safety against runaway behaviors, making the sampling process more robust.



\subsection{Our Result III: ForM with TrigFlow} \label{sub:form_trig}

The interpolation path of ForM is given by the following theorem.

\begin{theorem}[ForM with TrigFlow, informal version of Theorem~\ref{thm:form_trig:formal}] \label{thm:form_trig:informal}
We let $m = 1$ for simplicity in ForM. Giving a the interpolation $x_t = \alpha_t x_1 + \sigma_t x_0$, where $\alpha_T = 1$, $\alpha_0 = 0$, $\sigma_T = 0$, $\sigma_0 = 1$. We let $F_t(x_t)$ denote a vector map of force, a trainable neuron network parameterized with $\theta$. We select the $\alpha_t$ and $\sigma_t$ identical with TrigFlow \cite{ls24}, where $\alpha_t = \sin(t)$ and $\sigma_t = \cos(t)$, $T = \frac{\pi}{2}$. Then, force interpolation could be simplified to 
\begin{align*}
    f_t(x_t) =
    & ~ \frac{(\cos(t)x_1 - \sin(t)x_0) \cdot (-\sin(t)x_1 - \cos(t)x_0)}{c^2 - (\cos(t)x_1 - \sin(t)x_0)^2}\\
    & ~ (\cos(t)x_1 - \sin(t)x_0)).
\end{align*}
\end{theorem}

Theorem~\ref{thm:form_trig:informal} highlights how the ForM framework can be directly coupled with trigonometric interpolation paths. By choosing $\alpha_t$ and $\sigma_t$ as sine and cosine, respectively, we obtain a closed-form expression for the relativistic force that governs the sample evolution. This synergy suggests that ForM can not only unify different flow-based or diffusion-based models but also inherit the continuous-time advantages of the TrigFlow parameterization. Consequently, one can design more flexible interpolation strategies while still enjoying the stability benefits of relativistic velocity constraints.



\section{Experiments}
\label{section5}

In this section, we conduct extensive experiments to show that \ourmethod~can significantly speed up the sampling of existing MR Diffusion. To rigorously validate the effectiveness of our method, we follow the settings and checkpoints from \cite{luo2024daclip} and only modify the sampling part. Our experiment is divided into three parts. Section \ref{mainresult} compares the sampling results for different NFE cases. Section \ref{effects} studies the effects of different parameter settings on our algorithm, including network parameterizations and solver types. In Section \ref{analysis}, we visualize the sampling trajectories to show the speedup achieved by \ourmethod~and analyze why noise prediction gets obviously worse when NFE is less than 20.


\subsection{Main results}\label{mainresult}

Following \cite{luo2024daclip}, we conduct experiments with ten different types of image degradation: blurry, hazy, JPEG-compression, low-light, noisy, raindrop, rainy, shadowed, snowy, and inpainting (see Appendix \ref{appd1} for details). We adopt LPIPS \citep{zhang2018lpips} and FID \citep{heusel2017fid} as main metrics for perceptual evaluation, and also report PSNR and SSIM \citep{wang2004ssim} for reference. We compare \ourmethod~with other sampling methods, including posterior sampling \citep{luo2024posterior} and Euler-Maruyama discretization \citep{kloeden1992sde}. We take two tasks as examples and the metrics are shown in Figure \ref{fig:main}. Unless explicitly mentioned, we always use \ourmethod~based on SDE solver, with data prediction and uniform $\lambda$. The complete experimental results can be found in Appendix \ref{appd3}. The results demonstrate that \ourmethod~converges in a few (5 or 10) steps and produces samples with stable quality. Our algorithm significantly reduces the time cost without compromising sampling performance, which is of great practical value for MR Diffusion.


\begin{figure}[!ht]
    \centering
    \begin{minipage}[b]{0.45\textwidth}
        \centering
        \includegraphics[width=1\textwidth, trim=0 20 0 0]{figs/main_result/7_lowlight_fid.pdf}
        \subcaption{FID on \textit{low-light} dataset}
        \label{fig:main(a)}
    \end{minipage}
    \begin{minipage}[b]{0.45\textwidth}
        \centering
        \includegraphics[width=1\textwidth, trim=0 20 0 0]{figs/main_result/7_lowlight_lpips.pdf}
        \subcaption{LPIPS on \textit{low-light} dataset}
        \label{fig:main(b)}
    \end{minipage}
    \begin{minipage}[b]{0.45\textwidth}
        \centering
        \includegraphics[width=1\textwidth, trim=0 20 0 0]{figs/main_result/10_motion_fid.pdf}
        \subcaption{FID on \textit{motion-blurry} dataset}
        \label{fig:main(c)}
    \end{minipage}
    \begin{minipage}[b]{0.45\textwidth}
        \centering
        \includegraphics[width=1\textwidth, trim=0 20 0 0]{figs/main_result/10_motion_lpips.pdf}
        \subcaption{LPIPS on \textit{motion-blurry} dataset}
        \label{fig:main(d)}
    \end{minipage}
    \caption{\textbf{Perceptual evaluations on \textit{low-light} and \textit{motion-blurry} datasets.}}
    \label{fig:main}
\end{figure}

\subsection{Effects of parameter choice}\label{effects}

In Table \ref{tab:ablat_param}, we compare the results of two network parameterizations. The data prediction shows stable performance across different NFEs. The noise prediction performs similarly to data prediction with large NFEs, but its performance deteriorates significantly with smaller NFEs. The detailed analysis can be found in Section \ref{section5.3}. In Table \ref{tab:ablat_solver}, we compare \ourmethod-ODE-d-2 and \ourmethod-SDE-d-2 on the \textit{inpainting} task, which are derived from PF-ODE and reverse-time SDE respectively. SDE-based solver works better with a large NFE, whereas ODE-based solver is more effective with a small NFE. In general, neither solver type is inherently better.


% In Table \ref{tab:hazy}, we study the impact of two step size schedules on the results. On the whole, uniform $\lambda$ performs slightly better than uniform $t$. Our algorithm follows the method of \cite{lu2022dpmsolverplus} to estimate the integral part of the solution, while the analytical part does not affect the error.  Consequently, our algorithm has the same global truncation error, that is $\mathcal{O}\left(h_{max}^{k}\right)$. Note that the initial and final values of $\lambda$ depend on noise schedule and are fixed. Therefore, uniform $\lambda$ scheduling leads to the smallest $h_{max}$ and works better.

\begin{table}[ht]
    \centering
    \begin{minipage}{0.5\textwidth}
    \small
    \renewcommand{\arraystretch}{1}
    \centering
    \caption{Ablation study of network parameterizations on the Rain100H dataset.}
    % \vspace{8pt}
    \resizebox{1\textwidth}{!}{
        \begin{tabular}{cccccc}
			\toprule[1.5pt]
            % \multicolumn{6}{c}{Rainy} \\
            % \cmidrule(lr){1-6}
             NFE & Parameterization      & LPIPS\textdownarrow & FID\textdownarrow &  PSNR\textuparrow & SSIM\textuparrow  \\
            \midrule[1pt]
            \multirow{2}{*}{50}
             & Noise Prediction & \textbf{0.0606}     & \textbf{27.28}   & \textbf{28.89}     & \textbf{0.8615}    \\
             & Data Prediction & 0.0620     & 27.65   & 28.85     & 0.8602    \\
            \cmidrule(lr){1-6}
            \multirow{2}{*}{20}
              & Noise Prediction & 0.1429     & 47.31   & 27.68     & 0.7954    \\
              & Data Prediction & \textbf{0.0635}     & \textbf{27.79}   & \textbf{28.60}     & \textbf{0.8559}    \\
            \cmidrule(lr){1-6}
            \multirow{2}{*}{10}
              & Noise Prediction & 1.376     & 402.3   & 6.623     & 0.0114    \\
              & Data Prediction & \textbf{0.0678}     & \textbf{29.54}   & \textbf{28.09}     & \textbf{0.8483}    \\
            \cmidrule(lr){1-6}
            \multirow{2}{*}{5}
              & Noise Prediction & 1.416     & 447.0   & 5.755     & 0.0051    \\
              & Data Prediction & \textbf{0.0637}     & \textbf{26.92}   & \textbf{28.82}     & \textbf{0.8685}    \\       
            \bottomrule[1.5pt]
        \end{tabular}}
        \label{tab:ablat_param}
    \end{minipage}
    \hspace{0.01\textwidth}
    \begin{minipage}{0.46\textwidth}
    \small
    \renewcommand{\arraystretch}{1}
    \centering
    \caption{Ablation study of solver types on the CelebA-HQ dataset.}
    % \vspace{8pt}
        \resizebox{1\textwidth}{!}{
        \begin{tabular}{cccccc}
			\toprule[1.5pt]
            % \multicolumn{6}{c}{Raindrop} \\     
            % \cmidrule(lr){1-6}
             NFE & Solver Type     & LPIPS\textdownarrow & FID\textdownarrow &  PSNR\textuparrow & SSIM\textuparrow  \\
            \midrule[1pt]
            \multirow{2}{*}{50}
             & ODE & 0.0499     & 22.91   & 28.49     & 0.8921    \\
             & SDE & \textbf{0.0402}     & \textbf{19.09}   & \textbf{29.15}     & \textbf{0.9046}    \\
            \cmidrule(lr){1-6}
            \multirow{2}{*}{20}
              & ODE & 0.0475    & 21.35   & 28.51     & 0.8940    \\
              & SDE & \textbf{0.0408}     & \textbf{19.13}   & \textbf{28.98}    & \textbf{0.9032}    \\
            \cmidrule(lr){1-6}
            \multirow{2}{*}{10}
              & ODE & \textbf{0.0417}    & 19.44   & \textbf{28.94}     & \textbf{0.9048}    \\
              & SDE & 0.0437     & \textbf{19.29}   & 28.48     & 0.8996    \\
            \cmidrule(lr){1-6}
            \multirow{2}{*}{5}
              & ODE & \textbf{0.0526}     & 27.44   & \textbf{31.02}     & \textbf{0.9335}    \\
              & SDE & 0.0529    & \textbf{24.02}   & 28.35     & 0.8930    \\
            \bottomrule[1.5pt]
        \end{tabular}}
        \label{tab:ablat_solver}
    \end{minipage}
\end{table}


% \renewcommand{\arraystretch}{1}
%     \centering
%     \caption{Ablation study of step size schedule on the RESIDE-6k dataset.}
%     % \vspace{8pt}
%         \resizebox{1\textwidth}{!}{
%         \begin{tabular}{cccccc}
% 			\toprule[1.5pt]
%             % \multicolumn{6}{c}{Raindrop} \\     
%             % \cmidrule(lr){1-6}
%              NFE & Schedule      & LPIPS\textdownarrow & FID\textdownarrow &  PSNR\textuparrow & SSIM\textuparrow  \\
%             \midrule[1pt]
%             \multirow{2}{*}{50}
%              & uniform $t$ & 0.0271     & 5.539   & 30.00     & 0.9351    \\
%              & uniform $\lambda$ & \textbf{0.0233}     & \textbf{4.993}   & \textbf{30.19}     & \textbf{0.9427}    \\
%             \cmidrule(lr){1-6}
%             \multirow{2}{*}{20}
%               & uniform $t$ & 0.0313     & 6.000   & 29.73     & 0.9270    \\
%               & uniform $\lambda$ & \textbf{0.0240}     & \textbf{5.077}   & \textbf{30.06}    & \textbf{0.9409}    \\
%             \cmidrule(lr){1-6}
%             \multirow{2}{*}{10}
%               & uniform $t$ & 0.0309     & 6.094   & 29.42     & 0.9274    \\
%               & uniform $\lambda$ & \textbf{0.0246}     & \textbf{5.228}   & \textbf{29.65}     & \textbf{0.9372}    \\
%             \cmidrule(lr){1-6}
%             \multirow{2}{*}{5}
%               & uniform $t$ & 0.0256     & 5.477   & \textbf{29.91}     & 0.9342    \\
%               & uniform $\lambda$ & \textbf{0.0228}     & \textbf{5.174}   & 29.65     & \textbf{0.9416}    \\
%             \bottomrule[1.5pt]
%         \end{tabular}}
%         \label{tab:ablat_schedule}



\subsection{Analysis}\label{analysis}
\label{section5.3}

\begin{figure}[ht!]
    \centering
    \begin{minipage}[t]{0.6\linewidth}
        \centering
        \includegraphics[width=\linewidth, trim=0 20 10 0]{figs/trajectory_a.pdf} %trim左下右上
        \subcaption{Sampling results.}
        \label{fig:traj(a)}
    \end{minipage}
    \begin{minipage}[t]{0.35\linewidth}
        \centering
        \includegraphics[width=\linewidth, trim=0 0 0 0]{figs/trajectory_b.pdf} %trim左下右上
        \subcaption{Trajectory.}
        \label{fig:traj(b)}
    \end{minipage}
    \caption{\textbf{Sampling trajectories.} In (a), we compare our method (with order 1 and order 2) and previous sampling methods (i.e., posterior sampling and Euler discretization) on a motion blurry image. The numbers in parentheses indicate the NFE. In (b), we illustrate trajectories of each sampling method. Previous methods need to take many unnecessary paths to converge. With few NFEs, they fail to reach the ground truth (i.e., the location of $\boldsymbol{x}_0$). Our methods follow a more direct trajectory.}
    \label{fig:traj}
\end{figure}

\textbf{Sampling trajectory.}~ Inspired by the design idea of NCSN \citep{song2019ncsn}, we provide a new perspective of diffusion sampling process. \cite{song2019ncsn} consider each data point (e.g., an image) as a point in high-dimensional space. During the diffusion process, noise is added to each point $\boldsymbol{x}_0$, causing it to spread throughout the space, while the score function (a neural network) \textit{remembers} the direction towards $\boldsymbol{x}_0$. In the sampling process, we start from a random point by sampling a Gaussian distribution and follow the guidance of the reverse-time SDE (or PF-ODE) and the score function to locate $\boldsymbol{x}_0$. By connecting each intermediate state $\boldsymbol{x}_t$, we obtain a sampling trajectory. However, this trajectory exists in a high-dimensional space, making it difficult to visualize. Therefore, we use Principal Component Analysis (PCA) to reduce $\boldsymbol{x}_t$ to two dimensions, obtaining the projection of the sampling trajectory in 2D space. As shown in Figure \ref{fig:traj}, we present an example. Previous sampling methods \citep{luo2024posterior} often require a long path to find $\boldsymbol{x}_0$, and reducing NFE can lead to cumulative errors, making it impossible to locate $\boldsymbol{x}_0$. In contrast, our algorithm produces more direct trajectories, allowing us to find $\boldsymbol{x}_0$ with fewer NFEs.

\begin{figure*}[ht]
    \centering
    \begin{minipage}[t]{0.45\linewidth}
        \centering
        \includegraphics[width=\linewidth, trim=0 0 0 0]{figs/convergence_a.pdf} %trim左下右上
        \subcaption{Sampling results.}
        \label{fig:convergence(a)}
    \end{minipage}
    \begin{minipage}[t]{0.43\linewidth}
        \centering
        \includegraphics[width=\linewidth, trim=0 20 0 0]{figs/convergence_b.pdf} %trim左下右上
        \subcaption{Ratio of convergence.}
        \label{fig:convergence(b)}
    \end{minipage}
    \caption{\textbf{Convergence of noise prediction and data prediction.} In (a), we choose a low-light image for example. The numbers in parentheses indicate the NFE. In (b), we illustrate the ratio of components of neural network output that satisfy the Taylor expansion convergence requirement.}
    \label{fig:converge}
\end{figure*}

\textbf{Numerical stability of parameterizations.}~ From Table 1, we observe poor sampling results for noise prediction in the case of few NFEs. The reason may be that the neural network parameterized by noise prediction is numerically unstable. Recall that we used Taylor expansion in Eq.(\ref{14}), and the condition for the equality to hold is $|\lambda-\lambda_s|<\boldsymbol{R}(s)$. And the radius of convergence $\boldsymbol{R}(t)$ can be calculated by
\begin{equation}
\frac{1}{\boldsymbol{R}(t)}=\lim_{n\rightarrow\infty}\left|\frac{\boldsymbol{c}_{n+1}(t)}{\boldsymbol{c}_n(t)}\right|,
\end{equation}
where $\boldsymbol{c}_n(t)$ is the coefficient of the $n$-th term in Taylor expansion. We are unable to compute this limit and can only compute the $n=0$ case as an approximation. The output of the neural network can be viewed as a vector, with each component corresponding to a radius of convergence. At each time step, we count the ratio of components that satisfy $\boldsymbol{R}_i(s)>|\lambda-\lambda_s|$ as a criterion for judging the convergence, where $i$ denotes the $i$-th component. As shown in Figure \ref{fig:converge}, the neural network parameterized by data prediction meets the convergence criteria at almost every step. However, the neural network parameterized by noise prediction always has components that cannot converge, which will lead to large errors and failed sampling. Therefore, data prediction has better numerical stability and is a more recommended choice.


\begin{table*}[ht]
    \centering
    \caption{Performance comparison of different methods supplemented with nnU-Net augmentations. Colored numbers show an improvement (or a decline, respectively) over a non-augmented method. GIN and MIND were only trained with nnU-Net augmentations.}
    
    % Add these color definitions to your preamble
    \definecolor{darkGreen}{RGB}{0, 102, 0}     % For improvements > 0.15
    \definecolor{medGreen}{RGB}{0, 153, 0}      % For improvements 0.05 to 0.15
    \definecolor{lightGreen}{RGB}{144, 238, 144} % For small improvements 0 to 0.05
    \definecolor{lightRed}{RGB}{255, 200, 200}   % For negative values
    
    \resizebox{\textwidth}{!}{%
    \begin{tabular}{lcccccccccc}
        \toprule
        & MR$\rightarrow$CT & CT$\rightarrow$MR & CT$\rightarrow$LDCT & CE CT$\rightarrow$CT & T1 CE$\rightarrow$T1 & T1 F & T1 Sc & T1 Mix & \textbf{avg DSC} & \textbf{avg gap} \\
        
        \midrule
        
        % GIN & 0.589 & 0.637 & 0.722 & 0.163 & 0.382 & 0.837 & 0.709 & 0.804 & 0.605 & 33.6\% \\
        
        CycleGAN 3D & 0.364 \textcolor{lightGreen}{$\uparrow$0.031} & 0.464 \textcolor{darkGreen}{$\uparrow$0.200} & 0.679 \textcolor{darkGreen}{$\uparrow$0.353} & 0.221 \textcolor{medGreen}{$\uparrow$0.091} & 0.379 \textcolor{lightGreen}{$\uparrow$0.034} & 0.825 \textcolor{lightGreen}{$\uparrow$0.034} & 0.810 \textcolor{medGreen}{$\uparrow$0.097} & 0.779 \textcolor{lightGreen}{$\uparrow$0.017} & 0.565 \textcolor{medGreen}{$\uparrow$0.107} & 34.1\% \textcolor{darkGreen}{$\uparrow$24.6\%} \\
        CycleGAN 2D & 0.301 \textcolor{medGreen}{$\uparrow$0.096} & 0.461 \textcolor{medGreen}{$\uparrow$0.055} & 0.666 \textcolor{medGreen}{$\uparrow$0.136} & 0.333 \textcolor{medGreen}{$\uparrow$0.117} & 0.416 \textcolor{lightGreen}{$\uparrow$0.018} & 0.865 \textcolor{lightGreen}{$\uparrow$0.013} & 0.850 \textcolor{lightGreen}{$\uparrow$0.049} & 0.815 \textcolor{lightGreen}{$\uparrow$0.020} & 0.588 \textcolor{medGreen}{$\uparrow$0.063} & 45.5\% \textcolor{medGreen}{$\uparrow$15.3\%} \\
        
        % MIND & 0.560 & 0.588 & 0.237 & 0.425 & 0.335 & 0.865 & 0.869 & 0.845 & 0.590 & 45.9\% \\
        
        Baseline (nnAugm) & 0.166 \textcolor{medGreen}{$\uparrow$0.134} & 0.102 \textcolor{medGreen}{$\uparrow$0.070} & 0.779 \textcolor{darkGreen}{$\uparrow$0.646} & 0.392 \textcolor{darkGreen}{$\uparrow$0.164} & 0.446 \textcolor{lightGreen}{$\uparrow$0.020} & 0.910 \textcolor{darkGreen}{$\uparrow$0.169} & 0.897 \textcolor{medGreen}{$\uparrow$0.131} & 0.889 \textcolor{darkGreen}{$\uparrow$0.329} & 0.573 \textcolor{darkGreen}{$\uparrow$0.208} & 51.9\% \textcolor{darkGreen}{$\uparrow$51.9\%} \\  % nnAugm (Baseline)
        DANN        & 0.414 \textcolor{medGreen}{$\uparrow$0.118} & 0.349 \textcolor{medGreen}{$\uparrow$0.071} & 0.809 \textcolor{medGreen}{$\uparrow$0.110} & 0.411 \textcolor{lightGreen}{$\uparrow$0.002} & 0.403 \textcolor{lightRed}{$\downarrow$-0.013} & 0.899 \textcolor{darkGreen}{$\uparrow$0.169} & 0.848 \textcolor{lightGreen}{$\uparrow$0.015} & 0.885 \textcolor{medGreen}{$\uparrow$0.109} & 0.627 \textcolor{medGreen}{$\uparrow$0.072} & 54.9\% \textcolor{darkGreen}{$\uparrow$23.3\%} \\

        IN          & 0.422 \textcolor{medGreen}{$\uparrow$0.119} & 0.471 \textcolor{darkGreen}{$\uparrow$0.163} & 0.796 \textcolor{medGreen}{$\uparrow$0.128} & 0.410 \textcolor{lightRed}{$\downarrow$-0.017} & 0.416 \textcolor{lightRed}{$\downarrow$-0.012} & 0.907 \textcolor{darkGreen}{$\uparrow$0.151} & 0.854 \textcolor{lightGreen}{$\uparrow$0.016} & 0.883 \textcolor{medGreen}{$\uparrow$0.099} & 0.645 \textcolor{medGreen}{$\uparrow$0.081} & 58.1\% \textcolor{darkGreen}{$\uparrow$26.6\%} \\
        AdaBN       & 0.495 \textcolor{darkGreen}{$\uparrow$0.173} & 0.532 \textcolor{darkGreen}{$\uparrow$0.179} & 0.604 \textcolor{lightGreen}{$\uparrow$0.017} & 0.365 \textcolor{medGreen}{$\uparrow$0.070} & 0.454 \textcolor{lightGreen}{$\uparrow$0.021} & 0.907 \textcolor{medGreen}{$\uparrow$0.129} & 0.890 \textcolor{medGreen}{$\uparrow$0.057} & 0.892 \textcolor{medGreen}{$\uparrow$0.096} & 0.642 \textcolor{medGreen}{$\uparrow$0.092} & 59.2\% \textcolor{darkGreen}{$\uparrow$24.2\%} \\

        SE          & 0.459 \textcolor{medGreen}{$\uparrow$0.068} & 0.571 \textcolor{darkGreen}{$\uparrow$0.183} & 0.768 \textcolor{darkGreen}{$\uparrow$0.165} & 0.389 \textcolor{medGreen}{$\uparrow$0.057} & 0.374 \textcolor{lightRed}{$\downarrow$-0.014} & 0.902 \textcolor{lightRed}{$\downarrow$-0.004} & 0.907 \textcolor{lightGreen}{$\uparrow$0.014} & 0.888 \textcolor{lightRed}{$\downarrow$-0.030} & 0.657 \textcolor{medGreen}{$\uparrow$0.055} & 60.1\% \enspace \textcolor{medGreen}{$\uparrow$8.4\%} \\
        MinEnt      & 0.388 \textcolor{darkGreen}{$\uparrow$0.248} & 0.362 \textcolor{darkGreen}{$\uparrow$0.190} & 0.788 \textcolor{darkGreen}{$\uparrow$0.283} & 0.449 \textcolor{medGreen}{$\uparrow$0.057} & 0.448 \textcolor{lightGreen}{$\uparrow$0.019} & 0.903 \textcolor{medGreen}{$\uparrow$0.133} & 0.901 \textcolor{medGreen}{$\uparrow$0.103} & 0.892 \textcolor{medGreen}{$\uparrow$0.116} & 0.641 \textcolor{medGreen}{$\uparrow$0.143} & 62.0\% \textcolor{darkGreen}{$\uparrow$33.5\%} \\





        
        \midrule
        % \textbf{Average improvement} & 0.376 \textcolor{medGreen}{$\uparrow$0.123} & 0.414 \textcolor{darkGreen}{$\uparrow$0.139} & 0.736 \textcolor{darkGreen}{$\uparrow$0.230} & 0.371 \textcolor{medGreen}{$\uparrow$0.068} & 0.417 \textcolor{lightGreen}{$\uparrow$0.009} & 0.890 \textcolor{medGreen}{$\uparrow$0.099} & 0.870 \textcolor{medGreen}{$\uparrow$0.060} & 0.865 \textcolor{medGreen}{$\uparrow$0.095} & 0.617 \textcolor{medGreen}{$\uparrow$0.103} \\
        \textbf{average} &\textcolor{medGreen}{$\uparrow$0.123} & \textcolor{darkGreen}{$\uparrow$0.139} & \textcolor{darkGreen}{$\uparrow$0.230} & \textcolor{medGreen}{$\uparrow$0.068} & \textcolor{lightGreen}{$\uparrow$0.009} & \textcolor{medGreen}{$\uparrow$0.099} & \textcolor{medGreen}{$\uparrow$0.060} & \textcolor{medGreen}{$\uparrow$0.095} & \textcolor{medGreen}{$\uparrow$0.103} & \textcolor{darkGreen}{$\uparrow$26.0\%} \\  % 53.2\%
        
        \bottomrule
    \end{tabular}}
    \label{tab:ablation_aug}
\end{table*}




% \begin{table*}[ht]
%     \centering
%     \caption{Ablation of selected methods on adding augmentations from the nnUNet pipeline.}%during training

%     \resizebox{\textwidth}{!}{%
%     \begin{tabular}{lcccccccccc}
%         \toprule
%         & nnUnet augm & MR$\rightarrow$CT & CT$\rightarrow$MR & CT$\rightarrow$LDCT & CE CT$\rightarrow$CT & T1 CE$\rightarrow$T1 & T1 F & T1 Sc & T1 Mix & \textit{avg DSC} \\
%         % & \textit{avg gap} \\
        
%         \midrule
%         Baseline    & \xmark & 0.032 & 0.032 & 0.133 & 0.228 & 0.426 & 0.741 & 0.766 & 0.560 & 0.365 \\ % & 0.0\% \\
%         nnAugm      & \cmark & 0.166 & 0.102 & 0.779 & 0.392 & 0.446 & 0.910 & 0.897 & 0.889 & 0.573 \\ % 48.9\% \\

%         % \rowcolor{lightgray}    
%         CycleGAN 3D & \xmark & 0.333 & 0.264 & 0.326 & 0.130 & 0.345 & 0.791 & 0.713 & 0.762 & 0.458 \\ % & 9.5\% \\
%         CycleGAN 3D & \cmark & 0.364 & 0.464 & 0.679 & 0.221 & 0.379 & 0.825 & 0.810 & 0.779 & 0.565 \\ % & 34.1\% \\

%         MinEnt      & \xmark & 0.140 & 0.172 & 0.505 & 0.392 & 0.429 & 0.770 & 0.798 & 0.776 & 0.498 \\ % & 28.5\% \\
%         MinEnt      & \cmark & 0.388 & 0.362 & 0.788 & 0.449 & 0.448 & 0.903 & 0.901 & 0.892 & 0.641 \\ % & 62.0\% \\

%         CycleGAN 2D & \xmark & 0.205 & 0.406 & 0.530 & 0.216 & 0.398 & 0.852 & 0.801 & 0.795 & 0.525 \\
%         CycleGAN 2D & \cmark & 0.301 & 0.461 & 0.666 & 0.333 & 0.416 & 0.865 & 0.850 & 0.815 & 0.588 \\ % & 45.5\% \\

%         AdaBN       & \xmark & 0.322 & 0.353 & 0.587 & 0.295 & 0.433 & 0.778 & 0.833 & 0.796 & 0.550 \\
%         AdaBN       & \cmark & 0.495 & 0.532 & 0.604 & 0.365 & 0.454 & 0.907 & 0.890 & 0.892 & 0.642 \\ % & 59.2\% \\
        
%         DANN        & \xmark & 0.296 & 0.278 & 0.699 & 0.409 & 0.416 & 0.730 & 0.833 & 0.776 & 0.555 \\ % & 31.6\% \\
%         DANN        & \cmark & 0.414 & 0.349 & 0.809 & 0.411 & 0.403 & 0.899 & 0.848 & 0.885 & 0.627 \\ % & 54.9\% \\

%         IN          & \xmark & 0.303 & 0.308 & 0.668 & 0.427 & 0.428 & 0.756 & 0.838 & 0.784 & 0.564 \\ % & 31.5\% \\
%         IN          & \cmark & 0.422 & 0.471 & 0.796 & 0.410 & 0.416 & 0.907 & 0.854 & 0.883 & 0.645 \\ % & 58.1\% \\

%         SE          & \xmark & 0.391 & 0.388 & 0.603 & 0.332 & 0.388 & 0.906 & 0.893 & 0.918 & 0.602 \\ % & 51.7% \\
%         SE          & \cmark & 0.459 & 0.571 & 0.768 & 0.389 & 0.374 & 0.902 & 0.907 & 0.888 & 0.657 \\ % & 60.1% \\
        
%         % nnUNet          & 0.397 & 0.355 & 0.750 & 0.373 & 0.330 & 0.923 & 0.914 & 0.907 & 0.619 & 54.9\% \\
        
%         % \midrule
        
%         % \textit{Oracle} & \xmark & 0.842 & 0.825 & 0.814 & 0.519 & 0.686 & 0.954 & 0.957 & 0.958 & 0.819 & 100\% \\
        
%         \bottomrule
         
%     \end{tabular}}
%     \label{tab:ablation_aug}
% \end{table*}
\section{Discussion}\label{sec:discussion}



\subsection{From Interactive Prompting to Interactive Multi-modal Prompting}
The rapid advancements of large pre-trained generative models including large language models and text-to-image generation models, have inspired many HCI researchers to develop interactive tools to support users in crafting appropriate prompts.
% Studies on this topic in last two years' HCI conferences are predominantly focused on helping users refine single-modality textual prompts.
Many previous studies are focused on helping users refine single-modality textual prompts.
However, for many real-world applications concerning data beyond text modality, such as multi-modal AI and embodied intelligence, information from other modalities is essential in constructing sophisticated multi-modal prompts that fully convey users' instruction.
This demand inspires some researchers to develop multimodal prompting interactions to facilitate generation tasks ranging from visual modality image generation~\cite{wang2024promptcharm, promptpaint} to textual modality story generation~\cite{chung2022tale}.
% Some previous studies contributed relevant findings on this topic. 
Specifically, for the image generation task, recent studies have contributed some relevant findings on multi-modal prompting.
For example, PromptCharm~\cite{wang2024promptcharm} discovers the importance of multimodal feedback in refining initial text-based prompting in diffusion models.
However, the multi-modal interactions in PromptCharm are mainly focused on the feedback empowered the inpainting function, instead of supporting initial multimodal sketch-prompt control. 

\begin{figure*}[t]
    \centering
    \includegraphics[width=0.9\textwidth]{src/img/novice_expert.pdf}
    \vspace{-2mm}
    \caption{The comparison between novice and expert participants in painting reveals that experts produce more accurate and fine-grained sketches, resulting in closer alignment with reference images in close-ended tasks. Conversely, in open-ended tasks, expert fine-grained strokes fail to generate precise results due to \tool's lack of control at the thin stroke level.}
    \Description{The comparison between novice and expert participants in painting reveals that experts produce more accurate and fine-grained sketches, resulting in closer alignment with reference images in close-ended tasks. Novice users create rougher sketches with less accuracy in shape. Conversely, in open-ended tasks, expert fine-grained strokes fail to generate precise results due to \tool's lack of control at the thin stroke level, while novice users' broader strokes yield results more aligned with their sketches.}
    \label{fig:novice_expert}
    % \vspace{-3mm}
\end{figure*}


% In particular, in the initial control input, users are unable to explicitly specify multi-modal generation intents.
In another example, PromptPaint~\cite{promptpaint} stresses the importance of paint-medium-like interactions and introduces Prompt stencil functions that allow users to perform fine-grained controls with localized image generation. 
However, insufficient spatial control (\eg, PromptPaint only allows for single-object prompt stencil at a time) and unstable models can still leave some users feeling the uncertainty of AI and a varying degree of ownership of the generated artwork~\cite{promptpaint}.
% As a result, the gap between intuitive multi-modal or paint-medium-like control and the current prompting interface still exists, which requires further research on multi-modal prompting interactions.
From this perspective, our work seeks to further enhance multi-object spatial-semantic prompting control by users' natural sketching.
However, there are still some challenges to be resolved, such as consistent multi-object generation in multiple rounds to increase stability and improved understanding of user sketches.   


% \new{
% From this perspective, our work is a step forward in this direction by allowing multi-object spatial-semantic prompting control by users' natural sketching, which considers the interplay between multiple sketch regions.
% % To further advance the multi-modal prompting experience, there are some aspects we identify to be important.
% % One of the important aspects is enhancing the consistency and stability of multiple rounds of generation to reduce the uncertainty and loss of control on users' part.
% % For this purpose, we need to develop techniques to incorporate consistent generation~\cite{tewel2024training} into multi-modal prompting framework.}
% % Another important aspect is improving generative models' understanding of the implicit user intents \new{implied by the paint-medium-like or sketch-based input (\eg, sketch of two people with their hands slightly overlapping indicates holding hand without needing explicit prompt).
% % This can facilitate more natural control and alleviate users' effort in tuning the textual prompt.
% % In addition, it can increase users' sense of ownership as the generated results can be more aligned with their sketching intents.
% }
% For example, when users draw sketches of two people with their hands slightly overlapping, current region-based models cannot automatically infer users' implicit intention that the two people are holding hands.
% Instead, they still require users to explicitly specify in the prompt such relationship.
% \tool addresses this through sketch-aware prompt recommendation to fill in the necessary semantic information, alleviating users' workload.
% However, some users want the generative AI in the future to be able to directly infer this natural implicit intentions from the sketches without additional prompting since prompt recommendation can still be unstable sometimes.


% \new{
% Besides visual generation, 
% }
% For example, one of the important aspect is referring~\cite{he2024multi}, linking specific text semantics with specific spatial object, which is partly what we do in our sketch-aware prompt recommendation.
% Analogously, in natural communication between humans, text or audio alone often cannot suffice in expressing the speakers' intentions, and speakers often need to refer to an existing spatial object or draw out an illustration of her ideas for better explanation.
% Philosophically, we HCI researchers are mostly concerned about the human-end experience in human-AI communications.
% However, studies on prompting is unique in that we should not just care about the human-end interaction, but also make sure that AI can really get what the human means and produce intention-aligned output.
% Such consideration can drastically impact the design of prompting interactions in human-AI collaboration applications.
% On this note, although studies on multi-modal interactions is a well-established topic in HCI community, it remains a challenging problem what kind of multi-modal information is really effective in helping humans convey their ideas to current and next generation large AI models.




\subsection{Novice Performance vs. Expert Performance}\label{sec:nVe}
In this section we discuss the performance difference between novice and expert regarding experience in painting and prompting.
First, regarding painting skills, some participants with experience (4/12) preferred to draw accurate and fine-grained shapes at the beginning. 
All novice users (5/12) draw rough and less accurate shapes, while some participants with basic painting skills (3/12) also favored sketching rough areas of objects, as exemplified in Figure~\ref{fig:novice_expert}.
The experienced participants using fine-grained strokes (4/12, none of whom were experienced in prompting) achieved higher IoU scores (0.557) in the close-ended task (0.535) when using \tool. 
This is because their sketches were closer in shape and location to the reference, making the single object decomposition result more accurate.
Also, experienced participants are better at arranging spatial location and size of objects than novice participants.
However, some experienced participants (3/12) have mentioned that the fine-grained stroke sometimes makes them frustrated.
As P1's comment for his result in open-ended task: "\emph{It seems it cannot understand thin strokes; even if the shape is accurate, it can only generate content roughly around the area, especially when there is overlapping.}" 
This suggests that while \tool\ provides rough control to produce reasonably fine results from less accurate sketches for novice users, it may disappoint experienced users seeking more precise control through finer strokes. 
As shown in the last column in Figure~\ref{fig:novice_expert}, the dragon hovering in the sky was wrongly turned into a standing large dragon by \tool.

Second, regarding prompting skills, 3 out of 12 participants had one or more years of experience in T2I prompting. These participants used more modifiers than others during both T2I and R2I tasks.
Their performance in the T2I (0.335) and R2I (0.469) tasks showed higher scores than the average T2I (0.314) and R2I (0.418), but there was no performance improvement with \tool\ between their results (0.508) and the overall average score (0.528). 
This indicates that \tool\ can assist novice users in prompting, enabling them to produce satisfactory images similar to those created by users with prompting expertise.



\subsection{Applicability of \tool}
The feedback from user study highlighted several potential applications for our system. 
Three participants (P2, P6, P8) mentioned its possible use in commercial advertising design, emphasizing the importance of controllability for such work. 
They noted that the system's flexibility allows designers to quickly experiment with different settings.
Some participants (N = 3) also mentioned its potential for digital asset creation, particularly for game asset design. 
P7, a game mod developer, found the system highly useful for mod development. 
He explained: "\emph{Mods often require a series of images with a consistent theme and specific spatial requirements. 
For example, in a sacrifice scene, how the objects are arranged is closely tied to the mod's background. It would be difficult for a developer without professional skills, but with this system, it is possible to quickly construct such images}."
A few participants expressed similar thoughts regarding its use in scene construction, such as in film production. 
An interesting suggestion came from participant P4, who proposed its application in crime scene description. 
She pointed out that witnesses are often not skilled artists, and typically describe crime scenes verbally while someone else illustrates their account. 
With this system, witnesses could more easily express what they saw themselves, potentially producing depictions closer to the real events. "\emph{Details like object locations and distances from buildings can be easily conveyed using the system}," she added.

% \subsection{Model Understanding of Users' Implicit Intents}
% In region-sketch-based control of generative models, a significant gap between interaction design and actual implementation is the model's failure in understanding users' naturally expressed intentions.
% For example, when users draw sketches of two people with their hands slightly overlapping, current region-based models cannot automatically infer users' implicit intention that the two people are holding hands.
% Instead, they still require users to explicitly specify in the prompt such relationship.
% \tool addresses this through sketch-aware prompt recommendation to fill in the necessary semantic information, alleviating users' workload.
% However, some users want the generative AI in the future to be able to directly infer this natural implicit intentions from the sketches without additional prompting since prompt recommendation can still be unstable sometimes.
% This problem reflects a more general dilemma, which ubiquitously exists in all forms of conditioned control for generative models such as canny or scribble control.
% This is because all the control models are trained on pairs of explicit control signal and target image, which is lacking further interpretation or customization of the user intentions behind the seemingly straightforward input.
% For another example, the generative models cannot understand what abstraction level the user has in mind for her personal scribbles.
% Such problems leave more challenges to be addressed by future human-AI co-creation research.
% One possible direction is fine-tuning the conditioned models on individual user's conditioned control data to provide more customized interpretation. 

% \subsection{Balance between recommendation and autonomy}
% AIGC tools are a typical example of 
\subsection{Progressive Sketching}
Currently \tool is mainly aimed at novice users who are only capable of creating very rough sketches by themselves.
However, more accomplished painters or even professional artists typically have a coarse-to-fine creative process. 
Such a process is most evident in painting styles like traditional oil painting or digital impasto painting, where artists first quickly lay down large color patches to outline the most primitive proportion and structure of visual elements.
After that, the artists will progressively add layers of finer color strokes to the canvas to gradually refine the painting to an exquisite piece of artwork.
One participant in our user study (P1) , as a professional painter, has mentioned a similar point "\emph{
I think it is useful for laying out the big picture, give some inspirations for the initial drawing stage}."
Therefore, rough sketch also plays a part in the professional artists' creation process, yet it is more challenging to integrate AI into this more complex coarse-to-fine procedure.
Particularly, artists would like to preserve some of their finer strokes in later progression, not just the shape of the initial sketch.
In addition, instead of requiring the tool to generate a finished piece of artwork, some artists may prefer a model that can generate another more accurate sketch based on the initial one, and leave the final coloring and refining to the artists themselves.
To accommodate these diverse progressive sketching requirements, a more advanced sketch-based AI-assisted creation tool should be developed that can seamlessly enable artist intervention at any stage of the sketch and maximally preserve their creative intents to the finest level. 

\subsection{Ethical Issues}
Intellectual property and unethical misuse are two potential ethical concerns of AI-assisted creative tools, particularly those targeting novice users.
In terms of intellectual property, \tool hands over to novice users more control, giving them a higher sense of ownership of the creation.
However, the question still remains: how much contribution from the user's part constitutes full authorship of the artwork?
As \tool still relies on backbone generative models which may be trained on uncopyrighted data largely responsible for turning the sketch into finished artwork, we should design some mechanisms to circumvent this risk.
For example, we can allow artists to upload backbone models trained on their own artworks to integrate with our sketch control.
Regarding unethical misuse, \tool makes fine-grained spatial control more accessible to novice users, who may maliciously generate inappropriate content such as more realistic deepfake with specific postures they want or other explicit content.
To address this issue, we plan to incorporate a more sophisticated filtering mechanism that can detect and screen unethical content with more complex spatial-semantic conditions. 
% In the future, we plan to enable artists to upload their own style model

% \subsection{From interactive prompting to interactive spatial prompting}


\subsection{Limitations and Future work}

    \textbf{User Study Design}. Our open-ended task assesses the usability of \tool's system features in general use cases. To further examine aspects such as creativity and controllability across different methods, the open-ended task could be improved by incorporating baselines to provide more insightful comparative analysis. 
    Besides, in close-ended tasks, while the fixing order of tool usage prevents prior knowledge leakage, it might introduce learning effects. In our study, we include practice sessions for the three systems before the formal task to mitigate these effects. In the future, utilizing parallel tests (\textit{e.g.} different content with the same difficulty) or adding a control group could further reduce the learning effects.

    \textbf{Failure Cases}. There are certain failure cases with \tool that can limit its usability. 
    Firstly, when there are three or more objects with similar semantics, objects may still be missing despite prompt recommendations. 
    Secondly, if an object's stroke is thin, \tool may incorrectly interpret it as a full area, as demonstrated in the expert results of the open-ended task in Figure~\ref{fig:novice_expert}. 
    Finally, sometimes inclusion relationships (\textit{e.g.} inside) between objects cannot be generated correctly, partially due to biases in the base model that lack training samples with such relationship. 

    \textbf{More support for single object adjustment}.
    Participants (N=4) suggested that additional control features should be introduced, beyond just adjusting size and location. They noted that when objects overlap, they cannot freely control which object appears on top or which should be covered, and overlapping areas are currently not allowed.
    They proposed adding features such as layer control and depth control within the single-object mask manipulation. Currently, the system assigns layers based on color order, but future versions should allow users to adjust the layer of each object freely, while considering weighted prompts for overlapping areas.

    \textbf{More customized generation ability}.
    Our current system is built around a single model $ColorfulXL-Lightning$, which limits its ability to fully support the diverse creative needs of users. Feedback from participants has indicated a strong desire for more flexibility in style and personalization, such as integrating fine-tuned models that cater to specific artistic styles or individual preferences. 
    This limitation restricts the ability to adapt to varied creative intents across different users and contexts.
    In future iterations, we plan to address this by embedding a model selection feature, allowing users to choose from a variety of pre-trained or custom fine-tuned models that better align with their stylistic preferences. 
    
    \textbf{Integrate other model functions}.
    Our current system is compatible with many existing tools, such as Promptist~\cite{hao2024optimizing} and Magic Prompt, allowing users to iteratively generate prompts for single objects. However, the integration of these functions is somewhat limited in scope, and users may benefit from a broader range of interactive options, especially for more complex generation tasks. Additionally, for multimodal large models, users can currently explore using affordable or open-source models like Qwen2-VL~\cite{qwen} and InternVL2-Llama3~\cite{llama}, which have demonstrated solid inference performance in our tests. While GPT-4o remains a leading choice, alternative models also offer competitive results.
    Moving forward, we aim to integrate more multimodal large models into the system, giving users the flexibility to choose the models that best fit their needs. 
    


\section{Conclusion}\label{sec:conclusion}
In this paper, we present \tool, an interactive system designed to help novice users create high-quality, fine-grained images that align with their intentions based on rough sketches. 
The system first refines the user's initial prompt into a complete and coherent one that matches the rough sketch, ensuring the generated results are both stable, coherent and high quality.
To further support users in achieving fine-grained alignment between the generated image and their creative intent without requiring professional skills, we introduce a decompose-and-recompose strategy. 
This allows users to select desired, refined object shapes for individual decomposed objects and then recombine them, providing flexible mask manipulation for precise spatial control.
The framework operates through a coarse-to-fine process, enabling iterative and fine-grained control that is not possible with traditional end-to-end generation methods. 
Our user study demonstrates that \tool offers novice users enhanced flexibility in control and fine-grained alignment between their intentions and the generated images.



\newpage
\ifdefined\isarxiv
%\section*{Acknowledgments}
\bibliographystyle{alpha}
\bibliography{ref}
\else
\bibliography{ref}
\bibliographystyle{ACM-Reference-Format} %Bo: this is for KDD 2025
\fi



\newpage 
% \onecolumn % Bo: the KDD official template didn't use newp page and oneclumn
\appendix
\ifdefined\isarxiv
\begin{center}
    \textbf{\LARGE Appendix}
\end{center}
\else
\section*{Appendix}
\fi
\paragraph{Roadmap.}
In Section~\ref{sec:miss_proof}, we provide a formal version of theoretical analysis and proofs.

\section{Theoretical Analysis} \label{sec:miss_proof}

In this section, we first provide the formal theorem and proof for the sampling ODE in Section~\ref{sub:app:samp_ode}. Then, we formally proved the speed limit of ForM's sampling ODE in Section~\ref{sub:app:speed_limit}. In Section~\ref{sub:app:form_trig}, we formally prove the derivation of the interpolation path of ForM with TrigFlow. Last, we illustrate the formal proof for relativistic force in Section~\ref{sub:app:force}.


\subsection{Sampling ODE} \label{sub:app:samp_ode}

Here, we restate the Theorem~\ref{thm:ode_form:informal} and state its proof.

\begin{theorem}[Sampling ODE, formal version of Theorem~\ref{thm:ode_form:informal}]\label{thm:ode_form:formal}
    Giving the force at position $x_t$ denoted as $f_t(x_t)$, we could solve for ForM sampling path $x_t$ by the following ODE
    \begin{align*}
    \ddot{x}_t = \frac{1}{m^{\rm lab} \gamma_t}(f_t^{\rm local} - \frac{\langle v_t^{\rm lab}, f_t^{\rm local} \rangle}{c^2} v_t^{\rm lab})
    \end{align*}
    where $x_0 \sim \N(0,I)$, $\dot{x}_0 = 0$.
\end{theorem}

\begin{proof}
Recall $f^{\rm local}$ from Lemma~\ref{lem:equiv_relativistic_force:formal}
\begin{align*}
    f_t^{\rm local} = m^{\rm lab}  (\gamma_t a_t^{\rm lab} + \gamma_t^3 \frac{ \langle v_t^{\rm lab}, a_t^{\rm lab} \rangle}{c^2} v_t^{\rm lab}),
\end{align*}
where $\gamma_t$ is the Lorentz factor defined in Definition~\ref{def:LorentzFactor}.

To solve for $a_t^{\rm lab}$, we could first decompose $a_t^{\rm lab}$ by
\begin{align*}
    a_t^{\rm lab} = a_{t,\parallel}^{\rm lab} + a_{t,\perp}^{\rm lab},
\end{align*}
where $a_{t,\parallel}^{\rm lab}$ denotes the component of $a_t^{\rm lab}$ parallel with $v_t^{\rm lab}$, and $a_{t,\perp}^{\rm lab}$ denotes the component of $a_t^{\rm lab}$ perpendicular with $v_t^{\rm lab}$.

According to the definition of parallel and perpendicular, we have
\begin{align*}
    a_{t,\parallel}^{\rm lab} = & ~ \frac{ \langle v_t^{\rm lab}, a_t^{\rm lab} \rangle}{\|v_t^{\rm lab}\|_2^2} v_t^{\rm lab}, \\
    a_{t,\perp}^{\rm lab} = & ~ a_t^{\rm lab} - a_{t,\parallel}^{\rm lab}.
\end{align*}

Then we have
\begin{align}
    f_t^{\rm local} = & ~ m^{\rm lab}  (\gamma_t a_t^{\rm lab} + \gamma_t^3 \frac{ \langle v_t^{\rm lab}, a_{t}^{\rm lab} \rangle}{c^2} v_t^{\rm lab}) \notag \\
    = & ~ m^{\rm lab}  (\gamma_t (a_{t,\parallel}^{\rm lab} + a_{t,\perp}^{\rm lab}) + \gamma_t^3 \frac{ \langle v_t^{\rm lab}, a_{t,\parallel}^{\rm lab} + a_{t,\perp}^{\rm lab} \rangle}{c^2} v_t^{\rm lab}) \notag \\
    = & ~ m^{\rm lab}  (\gamma_t (a_{t,\parallel}^{\rm lab} + a_{t,\perp}^{\rm lab}) + \gamma_t^3 \frac{ \langle v_t^{\rm lab}, a_{t,\parallel}^{\rm lab} \rangle}{c^2} v_t^{\rm lab}), \label{eq:f_split}
\end{align}
where the first step follows Lemma~\ref{lem:equiv_relativistic_force:formal}, the second step decomposes $a_t^{\rm lab}$, and the last step follows from the simple fact that $\langle v_t^{\rm lab}, a_{t,\perp}^{\rm lab} \rangle = 0$.

Then we decompose the $f_t^{\rm local}$ to $f_{t, \parallel}^{\rm local}$ and $f_{t, \perp}^{\rm local}$, where $f_{t, \parallel}^{\rm local}$ denotes the component of $f_t^{\rm local}$ parallel with $v_t^{\rm lab}$, and $f_{t,\perp}^{\rm local}$ denotes the component of $f_t^{\rm local}$ perpendicular with $v_t^{\rm lab}$.

For the perpendicular component, we have
\begin{align*}
    f_{t, \perp}^{\rm local} = & ~ m^{\rm lab}\gamma_t  a_{t,\perp}^{\rm lab} \\
    a_{t,\perp}^{\rm lab} =  & ~ \frac{f_{t, \perp}^{\rm local}}{m^{\rm lab}\gamma_t},
\end{align*}
where the first step uses the perpendicular part from Eq.~\ref{eq:f_split}, and the second step rewrites the equation to get a closed-form solution for $a_{t,\perp}^{\rm lab}$.

For the parallel component, we have
\begin{align*}
    f_{t, \parallel}^{\rm local} = & ~ m^{\rm lab} (\gamma_t  a_{t,\parallel}^{\rm lab} + \gamma_t^3 a_{t,\parallel}^{\rm lab} \frac{\|v_t^{\rm lab}\|_2^2}{c^2}) \\
    = & ~ m^{\rm lab} a_{t,\parallel}^{\rm lab}(\gamma_t + \gamma_t^3 \frac{\|v_t^{\rm lab}\|_2^2}{c^2}) \\
    a_{t,\parallel}^{\rm lab} = & ~ \frac{f_{t, \parallel}^{\rm local}}{m^{\rm lab}(\gamma_t + \gamma_t^3 \frac{\|v_t^{\rm lab}\|_2^2}{c^2})},
\end{align*}
where the first step uses the parallel part from Eq.~\ref{eq:f_split}, the second step factors out the $a_{t,\parallel}^{\rm lab}$, and the last step rewrites the equation to get a closed-form solution for $a_{t,\parallel}^{\rm lab}$.

Then, we can combine these two components
\begin{align*}
    a_t^{\rm lab} = & ~\frac{f_{t, \perp}^{\rm local}}{m^{\rm lab}\gamma_t} + \frac{f_{t, \parallel}^{\rm local}}{m^{\rm lab}(\gamma_t + \gamma_t^3 \frac{\|v_t^{\rm lab}\|_2^2}{c^2})} \\
    = & ~\frac{f_t^{\rm local} - \frac{\langle v_t^{\rm lab}, f_t^{\rm local} \rangle}{\|v_t^{\rm lab}\|_2^2} v_t^{\rm lab}}{m^{\rm lab}\gamma_t} + \frac{\frac{ \langle v_t^{\rm lab}, f_t^{\rm local} \rangle}{\|v_t^{\rm lab}\|_2^2} v_t^{\rm lab}}{m^{\rm lab}(\gamma_t + \gamma_t^3 \frac{\|v_t^{\rm lab}\|_2^2}{c^2})} \\
    = & ~\frac{1}{m^{\rm lab}\gamma_t} f_t^{\rm local} + \frac{1}{m^{\rm lab}} ( \frac{1}{\gamma_t(1 + \gamma_t^2 \frac{\|v_t^{\rm lab}\|_2^2}{c^2})} - \frac{1}{\gamma_t} ) \frac{ \langle v_t^{\rm lab}, f_t^{\rm local} \rangle}{\|v_t^{\rm lab}\|_2^2} v_t^{\rm lab} \\
    = & ~\frac{1}{m^{\rm lab}\gamma_t} f_t^{\rm local} + \frac{1}{m^{\rm lab}} ( \frac{1}{\gamma_t \frac{c^2}{c^2 - \|v_t^{\rm lab}\|_2^2} } - \frac{1}{\gamma_t} ) \frac{ \langle v_t^{\rm lab}, f_t^{\rm local} \rangle}{\|v_t^{\rm lab}\|_2^2} v_t^{\rm lab} \\
    = & ~\frac{1}{m^{\rm lab}\gamma_t} f_t^{\rm local} + \frac{1}{m^{\rm lab}} ( \frac{c^2 - \|v_t^{\rm lab}\|_2^2}{\gamma_t c^2} - \frac{1}{\gamma_t} ) \frac{ \langle v_t^{\rm lab}, f_t^{\rm local} \rangle}{\|v_t^{\rm lab}\|_2^2} v_t^{\rm lab} \\
    = & ~\frac{1}{m^{\rm lab}\gamma_t} f_t^{\rm local} - \frac{1}{m^{\rm lab} \gamma_t} \frac{\|v_t^{\rm lab}\|_2^2}{c^2} \frac{ \langle v_t^{\rm lab}, f_t^{\rm local} \rangle}{\|v_t^{\rm lab}\|_2^2} v_t^{\rm lab} \\
    = & ~ \frac{1}{m^{\rm lab} \gamma_t}(f_t^{\rm local} - \frac{\langle v_t^{\rm lab}, f_t^{\rm local} \rangle}{c^2} v_t^{\rm lab}),
\end{align*}
where the first step combines the two terms, the second step decomposes the components, the third step factors out the $\gamma_t$ in denominator, the forth uses the fact that $\gamma_t^2 = \frac{1}{1 - \|v_t^{\rm lab}\|_2^2/c^2}$, the fifth step moves the denominator into numerator, the sixth step merges two terms, and the last step factors out the $\frac{1}{m^{\rm lab} \gamma_t}$.
\end{proof}

\subsection{Speed Limit} \label{sub:app:speed_limit}

In this subsection, we first calculate the derivative of the squared norm of velocity, then restate the Theorem~\ref{thm:vel:informal} and provide its proof.

\begin{lemma}[Derivative of the squared norm of velocity]\label{lem:velocity_derivative}
    Let $X(t) := \frac{1}{2}\|v_t^{\rm lab}\|_2^2$. Then we have
    \begin{align*}
        \frac{\d X(t)}{\d t} = \frac{ \langle f_t^{\rm local}, v_t^{\rm lab} \rangle}{m^{\rm lab} \gamma_t}(1 - \frac{\|v_t^{\rm lab}\|_2^2}{c^2}). 
    \end{align*}
\end{lemma}
\begin{proof}
Recall the sampling ODE
\begin{align}\label{eq:tmp}
    a_t^{\rm lab} = \frac{1}{m^{\rm lab} \gamma_t}(f_t^{\rm local} - \frac{\langle v_t^{\rm lab}, f_t^{\rm local} \rangle}{c^2} v_t^{\rm lab}),
\end{align}
where $\gamma_t$ is the Lorentz factor at lab time $t$ defined in Definition~\ref{def:LorentzFactor}.

We can show that
\begin{align*}
    \frac{\d X(t)}{\d t} = &~ \frac{\d}{\d t} \frac{1}{2}\|v_t^{\rm lab} \|_2^2 \\
    = &~ \langle v_t^{\rm lab}, \frac{\d}{\d t} v_t^{\rm lab} \rangle \\
    = &~ \langle v_t^{\rm lab}, a_t^{\rm lab} \rangle \\
    = &~ \langle v_t^{\rm lab}, \frac{1}{m^{\rm lab} \gamma_t}(f_t^{\rm local} - \frac{\langle v_t^{\rm lab}, f_t^{\rm local} \rangle}{c^2} v_t^{\rm lab}) \rangle \\
    = &~ \frac{1}{m^{\rm lab} \gamma_t} \langle f_t^{\rm local}, v_t^{\rm lab} \rangle - \frac{\langle v_t^{\rm lab}, f_t^{\rm local} \rangle}{m^{\rm lab} \gamma_t c^2} \| v_t^{\rm lab} \|_2^2 \\
    = &~ \frac{ \langle f_t^{\rm local}, v_t^{\rm lab} \rangle}{m^{\rm lab} \gamma_t}(1 - \frac{\|v_t^{\rm lab}\|_2^2}{c^2}) 
\end{align*}
where the first step follows from the definition of $X(t)$, the second step follows from the chain rule, the third step follows from the basic fact, the fourth step follows from the Eq.~\eqref{eq:tmp}, the fifth and last step follows from basic algebra.
\end{proof}

Here, we restate the Theorem~\ref{thm:vel:informal} and state its proof.

\begin{theorem}[Speed Limit, formal version of Theorem~\ref{thm:vel:informal}] \label{thm:vel:formal}
For a ForM model with sampling path $x : [0,T) \to \R^n$, the velocity satisfies
\begin{align*}
\| \dot{x}_t \|_2 < c \quad \text{for all } t \in [0,T).
\end{align*}
\end{theorem}
\begin{proof}
Let $X(t) := \tfrac12 \|v_t^{\rm lab}\|_2^2$. By Lemma~\ref{lem:velocity_derivative}, we have
    \begin{align*}
        \frac{\d X(t)}{\d t} 
        \;=\; \frac{1}{m^{\rm lab} \gamma_t}
        \,\langle f_t^{\rm local}, v_t^{\rm lab} \rangle
        \,\Bigl(1 - \tfrac{\|v_t^{\rm lab}\|_2^2}{c^2}\Bigr).
    \end{align*}
    Observe that the factor $\bigl(1 - \|v_t^{\rm lab}\|_2^2 / c^2 \bigr)$ becomes negative if ever $\|v_t^{\rm lab}\|_2 > c$, and it is zero when $\|v_t^{\rm lab}\|_2 = c$. Thus, if the velocity norm were to exceed $c$ at some time, the derivative of $X(t)$ at that moment would be negative, forcing $X(t)$ (i.e., $\|v_t^{\rm lab}\|_2^2$) to decrease rather than increase. In particular, once $\|v_t^{\rm lab}\|_2^2$ reaches $c^2$, it cannot increase further.

    Consequently, for all $t \in [0,T)$ we must have $\|v_t^{\rm lab}\|_2 < c$, which proves the speed limit. Equivalently, since $\dot{x}_t = v_t^{\rm lab}$ in our notation, we conclude
    \begin{align*}
        \|\dot{x}_t\|_2 < c,
        \quad \forall t \in [0,T).
    \end{align*}
    Thus, we complete the proof.
    \end{proof}

\subsection{ForM with TrigFlow} \label{sub:app:form_trig}

We restate the Theorem~\ref{thm:form_trig:informal} and provide its proof.

\begin{theorem}[ForM with TrigFlow, formal version of Theorem~\ref{thm:form_trig:informal}] \label{thm:form_trig:formal}
We let $m = 1$ for simplicity in ForM. Giving a the interpolation $x_t = \alpha_t x_1 + \sigma_t x_0$, where $\alpha_T = 1$, $\alpha_0 = 0$, $\sigma_T = 0$, $\sigma_0 = 1$. We let $F_t(x_t)$ denote a vector map of force, a trainable neuron network parameterized with $\theta$. We select the $\alpha_t$ and $\sigma_t$ identical with TrigFlow \cite{ls24}, where $\alpha_t = \sin(t)$ and $\sigma_t = \cos(t)$, $T = \frac{\pi}{2}$. Then, force interpolation could be simplified to 
\begin{align*}
    f_t(x_t) =
    & ~ \frac{(\cos(t)x_1 - \sin(t)x_0) \cdot (-\sin(t)x_1 - \cos(t)x_0)}{c^2 - (\cos(t)x_1 - \sin(t)x_0)^2}\\
    & ~ (\cos(t)x_1 - \sin(t)x_0)).
\end{align*}
\end{theorem}
\begin{proof}
    We can show that
    \begin{align*}
        f_t(x_t) = & ~ \gamma_t \ddot{x}_t + \gamma_t^3 \frac{\langle \dot{x}_t, \ddot{x}_t \rangle}{c^2}\dot{x}_t \\
        = & ~ (1 - \frac{(\cos(t)x_1 - \sin(t)x_0)^2}{c^2})^{-\frac{1}{2}} (-\sin(t)x_1 - \cos(t)x_0) + \\
        & ~ \frac{(\cos(t)x_1 - \sin(t)x_0) \cdot (-\sin(t)x_1 - \cos(t)x_0)}{c^2 - (\cos(t)x_1 - \sin(t)x_0)^2}\\
        & ~ (\cos(t)x_1 - \sin(t)x_0)),
    \end{align*}
    where we use Definition~\ref{def:RelativisticForce} and substitute acceleration $a_t^{\rm lab}$ with $\ddot{x}_t$ and $v_t^{\rm lab}$ with $\dot{x}_t$ in the first step, and substitute $x_t = \sin(t) x_1 + \cos(t) x_0$ and calculate it's derivative in the second step.
\end{proof}

\subsection{Relativistic Force Property} \label{sub:app:force}

In this subsection, we restate Lemma~\ref{lem:equiv_relativistic_force:informal}, and show its proof.

\begin{lemma}[Equivalent Form of Relativistic Force, formal version of Lemma~\ref{lem:equiv_relativistic_force:informal}]\label{lem:equiv_relativistic_force:formal}
Let $p^{\rm lab}$ be the momentum defined in Eq.~\eqref{eq:p}, $\gamma_t$ be the Lorentz factor at lab time $t$ defined in Definition~\ref{def:LorentzFactor}, $\tau$ denotes the proper time, $v_t^{\rm lab} = \dot{x}_t$ denotes the velocity, 
$a_t^{\rm lab} = \ddot{x}_t$ denotes the acceleration.
The relativistic force, defined as the time derivative of the momentum in the lab frame, can be written as
\begin{align*}
f^{\rm local} =  m^{\rm lab}  (\gamma_t a_t^{\rm lab} + \gamma_t^3 \frac{ \langle v_t^{\rm lab}, a_t^{\rm lab} \rangle}{c^2} v_t^{\rm lab}).
\end{align*}

\end{lemma}
\begin{proof}
We can show that
\begin{align*}
f^{\rm local}  = & ~ \frac{\d p^{\rm lab}}{\d \tau}  \\
= & ~ \frac{\d m^{\rm lab} v_t^{\rm lab} \gamma_t}{\d t} \\
= & ~ m^{\rm lab} \frac{\d v_t^{\rm lab} \gamma_t}{\d t} \\
= & ~ m^{\rm lab}  (\gamma_t a_t^{\rm lab} + \gamma_t^3 \frac{ \langle v_t^{\rm lab}, a_t^{\rm lab} \rangle}{c^2} v_t^{\rm lab}).
\end{align*}
where the first step follows from Eq.~\eqref{eq:f_local}, the second step follows Eq.~\eqref{eq:p} and Definition~\ref{def:ProperTime}, the third step is true because $m^{\rm lab}$ is a constant, and the last step takes the derivative.
\end{proof}



%%%% Cut-line between first 10 pages and appendix







%%% some writing rules

%% Writing rule for creating tags.
%% Tags :
%% Theorem    \ref{thm:bla_bla}
%% Lemma      \ref{lem:bla_bla}
%% Claim      \ref{cla:bla_bla}
%% Corollary  \ref{cor:bla_bla}
%% Fact       \ref{fac:bla_bla}
%% Definition \ref{def:bla_bla}
%% Section    \ref{sec:bla_bla}
%% Subsection \ref{sub:bla_bla}
%% Equation   \ref{eq:bla_bla}



\end{document}



%%%%%%%%%%%%%%%%%%%%%%%%%%%%%%%%%%%%%%%%%%%%%%%%%%%%%%%%%%%%%%%%%%%%%%%%%%%%%%%%%%%%%%%%%%%%%%%%%%%%%%%%%%%%%%%%%%%%%%%%%%%%%%%%%%%%%%%%%%%%%%%%%%%%%%%%%%%%%%%%%%%%%%%%%%%%%%%%%%%%%%%%%%%%%%%%%%%%%%%%%%%%%%%%%%%%%%%%%%%%%%%%%%%%%%%%%%%%%%%%%%%%%%%%%%%%%%%%%%%%%%%%%%%%%%%%%%%%%%%%%%%%%%%%%%%%%%%%%%%%%%%%%%%%%%%%%%%%%%%%%%%%%%%%%%%%%%%%%%%%%%%%%%%%%%%%%%%%%%%%%%%%%%%%%%%%%%%%%%%%%%%%%%%%%%%%%%%%%%%%%%%%%%%%%%%%%%%%%%%%%%%%%%%%%%%%%%%%%%%%%%%%%%%%%%%%%%%%%%%%%%
