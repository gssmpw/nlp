\paragraph{Roadmap.}
In Section~\ref{sec:miss_proof}, we provide a formal version of theoretical analysis and proofs.

\section{Theoretical Analysis} \label{sec:miss_proof}

In this section, we first provide the formal theorem and proof for the sampling ODE in Section~\ref{sub:app:samp_ode}. Then, we formally proved the speed limit of ForM's sampling ODE in Section~\ref{sub:app:speed_limit}. In Section~\ref{sub:app:form_trig}, we formally prove the derivation of the interpolation path of ForM with TrigFlow. Last, we illustrate the formal proof for relativistic force in Section~\ref{sub:app:force}.


\subsection{Sampling ODE} \label{sub:app:samp_ode}

Here, we restate the Theorem~\ref{thm:ode_form:informal} and state its proof.

\begin{theorem}[Sampling ODE, formal version of Theorem~\ref{thm:ode_form:informal}]\label{thm:ode_form:formal}
    Giving the force at position $x_t$ denoted as $f_t(x_t)$, we could solve for ForM sampling path $x_t$ by the following ODE
    \begin{align*}
    \ddot{x}_t = \frac{1}{m^{\rm lab} \gamma_t}(f_t^{\rm local} - \frac{\langle v_t^{\rm lab}, f_t^{\rm local} \rangle}{c^2} v_t^{\rm lab})
    \end{align*}
    where $x_0 \sim \N(0,I)$, $\dot{x}_0 = 0$.
\end{theorem}

\begin{proof}
Recall $f^{\rm local}$ from Lemma~\ref{lem:equiv_relativistic_force:formal}
\begin{align*}
    f_t^{\rm local} = m^{\rm lab}  (\gamma_t a_t^{\rm lab} + \gamma_t^3 \frac{ \langle v_t^{\rm lab}, a_t^{\rm lab} \rangle}{c^2} v_t^{\rm lab}),
\end{align*}
where $\gamma_t$ is the Lorentz factor defined in Definition~\ref{def:LorentzFactor}.

To solve for $a_t^{\rm lab}$, we could first decompose $a_t^{\rm lab}$ by
\begin{align*}
    a_t^{\rm lab} = a_{t,\parallel}^{\rm lab} + a_{t,\perp}^{\rm lab},
\end{align*}
where $a_{t,\parallel}^{\rm lab}$ denotes the component of $a_t^{\rm lab}$ parallel with $v_t^{\rm lab}$, and $a_{t,\perp}^{\rm lab}$ denotes the component of $a_t^{\rm lab}$ perpendicular with $v_t^{\rm lab}$.

According to the definition of parallel and perpendicular, we have
\begin{align*}
    a_{t,\parallel}^{\rm lab} = & ~ \frac{ \langle v_t^{\rm lab}, a_t^{\rm lab} \rangle}{\|v_t^{\rm lab}\|_2^2} v_t^{\rm lab}, \\
    a_{t,\perp}^{\rm lab} = & ~ a_t^{\rm lab} - a_{t,\parallel}^{\rm lab}.
\end{align*}

Then we have
\begin{align}
    f_t^{\rm local} = & ~ m^{\rm lab}  (\gamma_t a_t^{\rm lab} + \gamma_t^3 \frac{ \langle v_t^{\rm lab}, a_{t}^{\rm lab} \rangle}{c^2} v_t^{\rm lab}) \notag \\
    = & ~ m^{\rm lab}  (\gamma_t (a_{t,\parallel}^{\rm lab} + a_{t,\perp}^{\rm lab}) + \gamma_t^3 \frac{ \langle v_t^{\rm lab}, a_{t,\parallel}^{\rm lab} + a_{t,\perp}^{\rm lab} \rangle}{c^2} v_t^{\rm lab}) \notag \\
    = & ~ m^{\rm lab}  (\gamma_t (a_{t,\parallel}^{\rm lab} + a_{t,\perp}^{\rm lab}) + \gamma_t^3 \frac{ \langle v_t^{\rm lab}, a_{t,\parallel}^{\rm lab} \rangle}{c^2} v_t^{\rm lab}), \label{eq:f_split}
\end{align}
where the first step follows Lemma~\ref{lem:equiv_relativistic_force:formal}, the second step decomposes $a_t^{\rm lab}$, and the last step follows from the simple fact that $\langle v_t^{\rm lab}, a_{t,\perp}^{\rm lab} \rangle = 0$.

Then we decompose the $f_t^{\rm local}$ to $f_{t, \parallel}^{\rm local}$ and $f_{t, \perp}^{\rm local}$, where $f_{t, \parallel}^{\rm local}$ denotes the component of $f_t^{\rm local}$ parallel with $v_t^{\rm lab}$, and $f_{t,\perp}^{\rm local}$ denotes the component of $f_t^{\rm local}$ perpendicular with $v_t^{\rm lab}$.

For the perpendicular component, we have
\begin{align*}
    f_{t, \perp}^{\rm local} = & ~ m^{\rm lab}\gamma_t  a_{t,\perp}^{\rm lab} \\
    a_{t,\perp}^{\rm lab} =  & ~ \frac{f_{t, \perp}^{\rm local}}{m^{\rm lab}\gamma_t},
\end{align*}
where the first step uses the perpendicular part from Eq.~\ref{eq:f_split}, and the second step rewrites the equation to get a closed-form solution for $a_{t,\perp}^{\rm lab}$.

For the parallel component, we have
\begin{align*}
    f_{t, \parallel}^{\rm local} = & ~ m^{\rm lab} (\gamma_t  a_{t,\parallel}^{\rm lab} + \gamma_t^3 a_{t,\parallel}^{\rm lab} \frac{\|v_t^{\rm lab}\|_2^2}{c^2}) \\
    = & ~ m^{\rm lab} a_{t,\parallel}^{\rm lab}(\gamma_t + \gamma_t^3 \frac{\|v_t^{\rm lab}\|_2^2}{c^2}) \\
    a_{t,\parallel}^{\rm lab} = & ~ \frac{f_{t, \parallel}^{\rm local}}{m^{\rm lab}(\gamma_t + \gamma_t^3 \frac{\|v_t^{\rm lab}\|_2^2}{c^2})},
\end{align*}
where the first step uses the parallel part from Eq.~\ref{eq:f_split}, the second step factors out the $a_{t,\parallel}^{\rm lab}$, and the last step rewrites the equation to get a closed-form solution for $a_{t,\parallel}^{\rm lab}$.

Then, we can combine these two components
\begin{align*}
    a_t^{\rm lab} = & ~\frac{f_{t, \perp}^{\rm local}}{m^{\rm lab}\gamma_t} + \frac{f_{t, \parallel}^{\rm local}}{m^{\rm lab}(\gamma_t + \gamma_t^3 \frac{\|v_t^{\rm lab}\|_2^2}{c^2})} \\
    = & ~\frac{f_t^{\rm local} - \frac{\langle v_t^{\rm lab}, f_t^{\rm local} \rangle}{\|v_t^{\rm lab}\|_2^2} v_t^{\rm lab}}{m^{\rm lab}\gamma_t} + \frac{\frac{ \langle v_t^{\rm lab}, f_t^{\rm local} \rangle}{\|v_t^{\rm lab}\|_2^2} v_t^{\rm lab}}{m^{\rm lab}(\gamma_t + \gamma_t^3 \frac{\|v_t^{\rm lab}\|_2^2}{c^2})} \\
    = & ~\frac{1}{m^{\rm lab}\gamma_t} f_t^{\rm local} + \frac{1}{m^{\rm lab}} ( \frac{1}{\gamma_t(1 + \gamma_t^2 \frac{\|v_t^{\rm lab}\|_2^2}{c^2})} - \frac{1}{\gamma_t} ) \frac{ \langle v_t^{\rm lab}, f_t^{\rm local} \rangle}{\|v_t^{\rm lab}\|_2^2} v_t^{\rm lab} \\
    = & ~\frac{1}{m^{\rm lab}\gamma_t} f_t^{\rm local} + \frac{1}{m^{\rm lab}} ( \frac{1}{\gamma_t \frac{c^2}{c^2 - \|v_t^{\rm lab}\|_2^2} } - \frac{1}{\gamma_t} ) \frac{ \langle v_t^{\rm lab}, f_t^{\rm local} \rangle}{\|v_t^{\rm lab}\|_2^2} v_t^{\rm lab} \\
    = & ~\frac{1}{m^{\rm lab}\gamma_t} f_t^{\rm local} + \frac{1}{m^{\rm lab}} ( \frac{c^2 - \|v_t^{\rm lab}\|_2^2}{\gamma_t c^2} - \frac{1}{\gamma_t} ) \frac{ \langle v_t^{\rm lab}, f_t^{\rm local} \rangle}{\|v_t^{\rm lab}\|_2^2} v_t^{\rm lab} \\
    = & ~\frac{1}{m^{\rm lab}\gamma_t} f_t^{\rm local} - \frac{1}{m^{\rm lab} \gamma_t} \frac{\|v_t^{\rm lab}\|_2^2}{c^2} \frac{ \langle v_t^{\rm lab}, f_t^{\rm local} \rangle}{\|v_t^{\rm lab}\|_2^2} v_t^{\rm lab} \\
    = & ~ \frac{1}{m^{\rm lab} \gamma_t}(f_t^{\rm local} - \frac{\langle v_t^{\rm lab}, f_t^{\rm local} \rangle}{c^2} v_t^{\rm lab}),
\end{align*}
where the first step combines the two terms, the second step decomposes the components, the third step factors out the $\gamma_t$ in denominator, the forth uses the fact that $\gamma_t^2 = \frac{1}{1 - \|v_t^{\rm lab}\|_2^2/c^2}$, the fifth step moves the denominator into numerator, the sixth step merges two terms, and the last step factors out the $\frac{1}{m^{\rm lab} \gamma_t}$.
\end{proof}

\subsection{Speed Limit} \label{sub:app:speed_limit}

In this subsection, we first calculate the derivative of the squared norm of velocity, then restate the Theorem~\ref{thm:vel:informal} and provide its proof.

\begin{lemma}[Derivative of the squared norm of velocity]\label{lem:velocity_derivative}
    Let $X(t) := \frac{1}{2}\|v_t^{\rm lab}\|_2^2$. Then we have
    \begin{align*}
        \frac{\d X(t)}{\d t} = \frac{ \langle f_t^{\rm local}, v_t^{\rm lab} \rangle}{m^{\rm lab} \gamma_t}(1 - \frac{\|v_t^{\rm lab}\|_2^2}{c^2}). 
    \end{align*}
\end{lemma}
\begin{proof}
Recall the sampling ODE
\begin{align}\label{eq:tmp}
    a_t^{\rm lab} = \frac{1}{m^{\rm lab} \gamma_t}(f_t^{\rm local} - \frac{\langle v_t^{\rm lab}, f_t^{\rm local} \rangle}{c^2} v_t^{\rm lab}),
\end{align}
where $\gamma_t$ is the Lorentz factor at lab time $t$ defined in Definition~\ref{def:LorentzFactor}.

We can show that
\begin{align*}
    \frac{\d X(t)}{\d t} = &~ \frac{\d}{\d t} \frac{1}{2}\|v_t^{\rm lab} \|_2^2 \\
    = &~ \langle v_t^{\rm lab}, \frac{\d}{\d t} v_t^{\rm lab} \rangle \\
    = &~ \langle v_t^{\rm lab}, a_t^{\rm lab} \rangle \\
    = &~ \langle v_t^{\rm lab}, \frac{1}{m^{\rm lab} \gamma_t}(f_t^{\rm local} - \frac{\langle v_t^{\rm lab}, f_t^{\rm local} \rangle}{c^2} v_t^{\rm lab}) \rangle \\
    = &~ \frac{1}{m^{\rm lab} \gamma_t} \langle f_t^{\rm local}, v_t^{\rm lab} \rangle - \frac{\langle v_t^{\rm lab}, f_t^{\rm local} \rangle}{m^{\rm lab} \gamma_t c^2} \| v_t^{\rm lab} \|_2^2 \\
    = &~ \frac{ \langle f_t^{\rm local}, v_t^{\rm lab} \rangle}{m^{\rm lab} \gamma_t}(1 - \frac{\|v_t^{\rm lab}\|_2^2}{c^2}) 
\end{align*}
where the first step follows from the definition of $X(t)$, the second step follows from the chain rule, the third step follows from the basic fact, the fourth step follows from the Eq.~\eqref{eq:tmp}, the fifth and last step follows from basic algebra.
\end{proof}

Here, we restate the Theorem~\ref{thm:vel:informal} and state its proof.

\begin{theorem}[Speed Limit, formal version of Theorem~\ref{thm:vel:informal}] \label{thm:vel:formal}
For a ForM model with sampling path $x : [0,T) \to \R^n$, the velocity satisfies
\begin{align*}
\| \dot{x}_t \|_2 < c \quad \text{for all } t \in [0,T).
\end{align*}
\end{theorem}
\begin{proof}
Let $X(t) := \tfrac12 \|v_t^{\rm lab}\|_2^2$. By Lemma~\ref{lem:velocity_derivative}, we have
    \begin{align*}
        \frac{\d X(t)}{\d t} 
        \;=\; \frac{1}{m^{\rm lab} \gamma_t}
        \,\langle f_t^{\rm local}, v_t^{\rm lab} \rangle
        \,\Bigl(1 - \tfrac{\|v_t^{\rm lab}\|_2^2}{c^2}\Bigr).
    \end{align*}
    Observe that the factor $\bigl(1 - \|v_t^{\rm lab}\|_2^2 / c^2 \bigr)$ becomes negative if ever $\|v_t^{\rm lab}\|_2 > c$, and it is zero when $\|v_t^{\rm lab}\|_2 = c$. Thus, if the velocity norm were to exceed $c$ at some time, the derivative of $X(t)$ at that moment would be negative, forcing $X(t)$ (i.e., $\|v_t^{\rm lab}\|_2^2$) to decrease rather than increase. In particular, once $\|v_t^{\rm lab}\|_2^2$ reaches $c^2$, it cannot increase further.

    Consequently, for all $t \in [0,T)$ we must have $\|v_t^{\rm lab}\|_2 < c$, which proves the speed limit. Equivalently, since $\dot{x}_t = v_t^{\rm lab}$ in our notation, we conclude
    \begin{align*}
        \|\dot{x}_t\|_2 < c,
        \quad \forall t \in [0,T).
    \end{align*}
    Thus, we complete the proof.
    \end{proof}

\subsection{ForM with TrigFlow} \label{sub:app:form_trig}

We restate the Theorem~\ref{thm:form_trig:informal} and provide its proof.

\begin{theorem}[ForM with TrigFlow, formal version of Theorem~\ref{thm:form_trig:informal}] \label{thm:form_trig:formal}
We let $m = 1$ for simplicity in ForM. Giving a the interpolation $x_t = \alpha_t x_1 + \sigma_t x_0$, where $\alpha_T = 1$, $\alpha_0 = 0$, $\sigma_T = 0$, $\sigma_0 = 1$. We let $F_t(x_t)$ denote a vector map of force, a trainable neuron network parameterized with $\theta$. We select the $\alpha_t$ and $\sigma_t$ identical with TrigFlow \cite{ls24}, where $\alpha_t = \sin(t)$ and $\sigma_t = \cos(t)$, $T = \frac{\pi}{2}$. Then, force interpolation could be simplified to 
\begin{align*}
    f_t(x_t) =
    & ~ \frac{(\cos(t)x_1 - \sin(t)x_0) \cdot (-\sin(t)x_1 - \cos(t)x_0)}{c^2 - (\cos(t)x_1 - \sin(t)x_0)^2}\\
    & ~ (\cos(t)x_1 - \sin(t)x_0)).
\end{align*}
\end{theorem}
\begin{proof}
    We can show that
    \begin{align*}
        f_t(x_t) = & ~ \gamma_t \ddot{x}_t + \gamma_t^3 \frac{\langle \dot{x}_t, \ddot{x}_t \rangle}{c^2}\dot{x}_t \\
        = & ~ (1 - \frac{(\cos(t)x_1 - \sin(t)x_0)^2}{c^2})^{-\frac{1}{2}} (-\sin(t)x_1 - \cos(t)x_0) + \\
        & ~ \frac{(\cos(t)x_1 - \sin(t)x_0) \cdot (-\sin(t)x_1 - \cos(t)x_0)}{c^2 - (\cos(t)x_1 - \sin(t)x_0)^2}\\
        & ~ (\cos(t)x_1 - \sin(t)x_0)),
    \end{align*}
    where we use Definition~\ref{def:RelativisticForce} and substitute acceleration $a_t^{\rm lab}$ with $\ddot{x}_t$ and $v_t^{\rm lab}$ with $\dot{x}_t$ in the first step, and substitute $x_t = \sin(t) x_1 + \cos(t) x_0$ and calculate it's derivative in the second step.
\end{proof}

\subsection{Relativistic Force Property} \label{sub:app:force}

In this subsection, we restate Lemma~\ref{lem:equiv_relativistic_force:informal}, and show its proof.

\begin{lemma}[Equivalent Form of Relativistic Force, formal version of Lemma~\ref{lem:equiv_relativistic_force:informal}]\label{lem:equiv_relativistic_force:formal}
Let $p^{\rm lab}$ be the momentum defined in Eq.~\eqref{eq:p}, $\gamma_t$ be the Lorentz factor at lab time $t$ defined in Definition~\ref{def:LorentzFactor}, $\tau$ denotes the proper time, $v_t^{\rm lab} = \dot{x}_t$ denotes the velocity, 
$a_t^{\rm lab} = \ddot{x}_t$ denotes the acceleration.
The relativistic force, defined as the time derivative of the momentum in the lab frame, can be written as
\begin{align*}
f^{\rm local} =  m^{\rm lab}  (\gamma_t a_t^{\rm lab} + \gamma_t^3 \frac{ \langle v_t^{\rm lab}, a_t^{\rm lab} \rangle}{c^2} v_t^{\rm lab}).
\end{align*}

\end{lemma}
\begin{proof}
We can show that
\begin{align*}
f^{\rm local}  = & ~ \frac{\d p^{\rm lab}}{\d \tau}  \\
= & ~ \frac{\d m^{\rm lab} v_t^{\rm lab} \gamma_t}{\d t} \\
= & ~ m^{\rm lab} \frac{\d v_t^{\rm lab} \gamma_t}{\d t} \\
= & ~ m^{\rm lab}  (\gamma_t a_t^{\rm lab} + \gamma_t^3 \frac{ \langle v_t^{\rm lab}, a_t^{\rm lab} \rangle}{c^2} v_t^{\rm lab}).
\end{align*}
where the first step follows from Eq.~\eqref{eq:f_local}, the second step follows Eq.~\eqref{eq:p} and Definition~\ref{def:ProperTime}, the third step is true because $m^{\rm lab}$ is a constant, and the last step takes the derivative.
\end{proof}
