\section{Introduction}
The field of generative modeling has witnessed significant progress with the advent of sophisticated techniques that leverage neural networks to synthesize high-quality data. Recent methods such as Diffusion Models (DM) \cite{swm+15,hja20,dn21,rbl+22,zlcz23,zlke23}, Flow Matching (FM) \cite{lcb+22,lgl22,ekb+24}, and the more recent Equilibrium Dynamics Model (EDM) \cite{kaal22, kal+24} have emerged as prominent approaches, each exploring distinct generative paradigms. These techniques differ fundamentally in how they utilize neural networks to evolve data representations over time: while Diffusion Models rely on iterative transformations of a Gaussian-initialized distribution, EDM employs an Ordinary Differential Equation (ODE) for continuous evolution, and FM directly predicts the data's velocity using a neural-network-based velocity field. Such diversity in generative approaches has motivated the need for a unified perspective that can bridge these conceptual differences.

In pursuit of this unification, TrigFlow \cite{ls24} was proposed as a generalized framework that provides a continuous generative process capable of transitioning between the behaviors of EDM and FM. By leveraging trigonometric parameterization, TrigFlow formulates data generation through a combination of trigonometric components, allowing for a flexible representation that captures the strengths of both paradigms. The framework introduces a trigonometric-based parameterization of the generative process, a loss function aligning with diffusion methods, and a probability flow ODE, thereby offering a more comprehensive understanding of generative modeling techniques and providing a foundation for further advancements in this domain.

Building on this unification perspective, this paper introduces Force Matching (ForM) as a novel generative framework inspired by principles of relativistic mechanics to stabilize the sampling process. By incorporating relativistic constraints through the Lorentz factor, ForM ensures stable sampling dynamics, limiting the velocity of generated samples to avoid instability. 
We establish that ForM is well-aligned with consistency models, suggesting its potential to enhance scalable generative modeling solutions.
Our contributions can be summarized as follows:
\begin{itemize}
    \item We propose \textbf{Force Matching (ForM)}, a novel generative modeling framework inspired by relativistic mechanics, which ensures stable sampling by constraining sample velocities through the Lorentz factor.
    \item We establish theoretical foundations for ForM, 
    highlighting its flexibility and scalability.
    \item We conduct extensive empirical evaluations, showing that ForM outperforms baseline flow matching methods in generative tasks and validating the effectiveness of its velocity constraint through ablation studies.
\end{itemize}

These contributions illustrate the promise of force-based methods in generative modeling, emphasizing their capability for stable, efficient, and flexible sampling. This work not only extends our understanding of generative techniques but also lays the foundation for exploring novel high-dimensional generative frameworks that effectively integrate stability and efficiency.


{\bf Roadmap.} In Section~\ref{sec:related_work}, we introduce related work of generative models and flow matching. Then, Section~\ref{sec:preli} introduces the preliminary of Force Matching. We then propose the Force Matching architecture in Section~\ref{sec:form}. Section~\ref{sec:exp} demonstrates empirical experiments of Force Matching, and Section~\ref{sec:ablation} performs ablation study of Force Matching. Finally, we conclude this paper in Section~\ref{sec:conclusion}.
