\section{Experiment} \label{sec:exp}

\begin{algorithm}[!ht]
\caption{First-order Training, \cite{lgl22}}
\label{alg:first_order_training}
\begin{algorithmic}[1]
\Procedure{1stTraining}{$\theta, D$}
\State \Comment{Parameter $\theta$ for the model $u_1$}
\State \Comment{Training dataset $D$}
\While{not converged}
\State {\color{blue} /* Sample a trajectory from dataset. */ }
\State {\color{blue} /*$x_t$ denotes the position at time $t$. */ }
\State {\color{blue} /*$\dot x_t$ denotes the velocity at time $t$. */ }
\State $\{x_t\}_{t \in [0, 1]}, \{\dot x_t\}_{t \in [0, 1]} \sim D$
\State {\color{blue} /* Random sample a timestep $t$. */ }
\State $t \sim \mathcal{U}(0, 1)$
\State {\color{blue} /* Update model parameter $\theta$. */ }
\State $\theta \gets \nabla_\theta ( \| u_1(x_t, t) - \dot x_t \|_2^2)$
\EndWhile
\State \Return{$\theta$}
\EndProcedure
\end{algorithmic}
\end{algorithm}

\begin{algorithm}
[!ht]
\caption{First-order Sampling, \cite{lgl22}}
\label{alg:first_order_sampling}
\begin{algorithmic}[1]
\Procedure{1stdSampling}{$\theta, M$}
\State \Comment{Parameter $\theta$ for the model $u_1$}
\State \Comment{The number of sampling steps $M$}
\State $x \sim \N (0, I)$
\State $d \gets 1 / M$
\State $t \gets 0$
\For{$n \in [0, \dots, M - 1]$}

\State $x \gets x + d \cdot u_1 (x, t)$
\State $t \gets t + d$
\EndFor
\State \textbf{return} $x$
    \EndProcedure
\end{algorithmic}
\end{algorithm}



\begin{algorithm}[!ht]
\caption{First and Second-order Training}
\label{alg:first_second_order_training}
\begin{algorithmic}[1]
\Procedure{2ndTraining}{$\theta, D$}
\State \Comment{Parameter $\theta$ for the models $u_1$ and $u_2$} 
\State \Comment{Training dataset $D$}
\While{not converged}
\State {\color{blue} /* Random sample a trajectory from dataset. */ }
\State {\color{blue} /*$x_t$ denotes the position at time $t$. */ }
\State {\color{blue} /*$\dot x_t$ denotes the velocity at time $t$. */ }
\State {\color{blue} /*$\ddot x_t$ denotes the acceleration at time $t$. */ }
\State $\{x_t\}_{t \in [0, 1]}, \{\dot x_t\}_{t \in [0, 1]}, \{\ddot x_t\}_{t \in [0, 1]} \sim D$
\State {\color{blue} /* Random sample a timestep $t$. */ }
\State $t \sim \mathcal{U}(0, 1)$
\State {\color{blue} /* Update model parameter $\theta$. */ }
\State $\theta \gets \nabla_\theta ( \| u_1(x_t, t) - \dot x_t \|_2^2$
\Statex \hspace{4.2em} $ + \| u_2 (u_1 (x_t, t), x_t, t) - \ddot x_t \|_2^2)$
\EndWhile
\State \Return{$\theta$}
\EndProcedure
\end{algorithmic}
\end{algorithm}

\begin{algorithm}
[!ht]
\caption{First and Second-order Sampling}
\label{alg:first_second_order_sampling}
\begin{algorithmic}[1]
\Procedure{2ndSampling}{$\theta, M$}
\State \Comment{Parameter $\theta$ for the models $u_1$ and $u_2$}
\State \Comment{The number of sampling steps $M$}
\State $x \sim \N (0, I)$
\State $d \gets 1 / M$
\State $t \gets 0$
\For{$n \in [0, \dots, M - 1]$}

\State $x \gets x + d \cdot u_1 (x, t) + \frac{d^2}{2} \cdot u_2 (u_1 (x, t), x, t)$
\State $t \gets t + d$
\EndFor
\State \textbf{return} $x$
    \EndProcedure
\end{algorithmic}
\end{algorithm}


\begin{algorithm}[!ht]
\caption{ForM Training, (Ours)}
\label{alg:form_training}
\begin{algorithmic}[1]
\Procedure{ForMTraining}{$\theta, D, p, k$}
\State \Comment{Parameter $\theta$ for the ForM model $F_\theta$}
\State \Comment{Training dataset $D$}
\State \Comment{Stepsize and time index distribution $p$}
\State \Comment{Batch size $k$}
\While{not converged}
\State {\color{blue} /* Random sample a trajectory from dataset. */ }
\State {\color{blue} /*$x_t$ denotes the position at time $t$. */ }
\State {\color{blue} /*$f_t$ denotes the Lorentz force at time $t$. */ }
\State $\{x_t\}_{t \in [0, 1]}, \{ f_t \}_{t \in [0, 1]} \sim D$
\State {\color{blue} /* Random sample a timestep $t$. */ }
\State $t \sim \mathcal{U}(0, 1)$
\State {\color{blue} /* Update model parameter $\theta$. */ }
\State $\theta \gets \nabla_\theta ( \| F_\theta(t) - f_t \|_2^2)$
\EndWhile
\State \Return{$\theta$}
\EndProcedure
\end{algorithmic}
\end{algorithm}

\begin{algorithm}
[!ht]
\caption{ForM Sampling, (Ours)}
\label{alg:form_sampling}
\begin{algorithmic}[1]
\Procedure{ForMSampling}{$\theta, M$}
\State \Comment{Parameter $\theta$ for the ForM model $F_\theta$}
\State \Comment{The number of sampling steps $M$}
\State $x \sim \N (0, I)$
\State $d \gets 1 / M$
\State $t \gets 0$
\For{$n \in [0, \dots, M - 1]$}
\State {\color{blue}/* 
Get Lorentz force $f_L$ via model $F_\theta$. */} 
\State $f_L \gets F_\theta(t)$
\State {\color{blue}/* 
Calculate acceleration $a$ via Theorem~\ref{thm:ode_form:informal}. */} 
\State $a \gets \phi (v_\mathrm{prev}, f_L)$ 
\State {\color{blue}/* 
Calculate velocity $v$. */} 
\State $v \gets v_\mathrm{prev} + d 
~ a_t$
\State {\color{blue}/* 
Calculate $x$. */} 
\State $x \gets x_\mathrm{prev} + d ~ (v_t + v_{\mathrm{prev}}) / 2$
\State {\color{blue}/* 
Update $t$. */} 
\State $t \gets t + d$
\State {\color{blue}/* Store variables for next iteration. */}
\State $x_{\mathrm{prev}} \gets x$
\State $v_{\mathrm{prev}} \gets v$
\EndFor
\State \textbf{return} $x$
    \EndProcedure
\end{algorithmic}
\end{algorithm}


\begin{algorithm}
[!ht]
\caption{Numerical ODE Solver}
\begin{algorithmic}[1]
\Procedure{NumericalODESolver}{$\theta, M$}
\State \Comment{Parameter $\theta$ for the ForM model $F_t$}
\State \Comment{The number of sampling steps $M$}
\State $x \sim \N (0, I)$
\State $d \gets 1 / M$
\State $t \gets 0$
\For{$n \in [0, \dots, M - 1]$}
\State {\color{blue}/* Solve the ODE in Theorem~\ref{thm:ode_form:informal}. */} 
\State $x \gets \mathrm{ODE Solver}(x, \frac{\d x}{\d t}, F(t), d)$ 
\State $t \gets t + d$
\EndFor
\State \textbf{return} $x$
    \EndProcedure
\end{algorithmic}
\end{algorithm}


In this section, we conduct a systematic evaluation of Force Matching (ForM) effectiveness through extensive experimentation, emphasizing its significant role in advancing distribution generation. Section~\ref{sec:exp:experiment_setup} describes the experimental framework, encompassing the Onedot, Halfmoons, and Spiral datasets, force dynamics, and a comparative analysis of various trajectory evolution methods. Section~\ref{sec:exp:result_anapysis} examines qualitative outcomes, demonstrating how second-order terms refine trajectory smoothness and highlighting ForM’s superiority in transport dynamics modeling. 

\subsection{Experiment setup} \label{sec:exp:experiment_setup}
For an in-depth trajectory evolution analysis, we assess three approaches: the standard flow matching~\cite{lcb+22} method utilizing only the first-order term, an improved approach integrating both first-order and second-order terms, and our proposed ForM model on three kinds of complex and challenging datasets.

\paragraph{Datasets.} (1) Onedot dataset: As illustrated in Figure~\ref{fig:onedot_dataset}, the Onedot dataset comprises 200 points sampled from a Gaussian distribution with variance $0.3$ to establish the central source distribution. The target distribution is generated via a Lorentz field, where each point's initial velocity matches its initial position vector. The parallel force is defined as $\gamma^3 m_0 a_x$, while the perpendicular force follows $\gamma m_0 a_x$, where $\gamma = (1 - v^2 / c^2)^{-\frac{1}{2}}$, and the speed of light is set to \(c = 3 \times 10^8\,\text{m/s}\). In the Onedot dataset, both forces are set to $1.5 \times 10^8$, with a duration of $1s$. 

(2) Halfmoons dataset:
Then, as illustrated in Figure~\ref{fig:halfmoons_dataset}, we sampled 1000 dots from a central source distribution, and the target distribution is generated by a Lorentz field. For the dots above the $x$-axis, we set the direction of their velocity parallel to the $x$-axis to the right, and the dots below the $x$-axis are parallel to the $x$-axis to the left. And the parallel force denoted as $1 \times 10^7 \cdot \sin( t )$, the perpendicular force follows $7 \times 10^8 \cdot \sin( 8 t )$. 

(3) Spiral dataset: 
With reference to the Spiral dataset as shown in Figure~\ref{fig:spiral_dataset}, 1000 data points were extracted from a central source distribution, and the target distribution was similarly synthesized via a Lorentz field. In this instance, the source distribution was segmented into two components—a circular core and an annular ring—by equally partitioning the original radius. The initial velocity assigned to each point was determined by both its designated segment and the angular coordinate of its starting position. As a result, points residing in the ring attain comparatively greater initial speeds, and those with larger angular coordinates likewise manifest enhanced velocities. Finally, the parallel force is specified by $1 \times 10^7 \cdot \sin( t )$, while the perpendicular force is given by $7 \times 10^8 \cdot \sin( t )$.

\paragraph{Baselines.} We compare our ForM model with two traditional yet powerful baselines. The first baseline is standard flow matching, which models the transfer trajectory between distributions using the first-order velocity of the trajectory. We denote this method as O1 (Algorithm~\ref{alg:first_order_training} and \ref{alg:first_order_sampling}). The second baseline is an improved version of the first-order approach, incorporating both first-order velocity and second-order acceleration to model the trajectory, which we denote as O1+O2 (Algorithm~\ref{alg:first_second_order_training} and \ref{alg:first_second_order_sampling}). Both baselines rely on modeling the higher-order derivatives of the trajectory. In contrast, our ForM model (Algorithm~\ref{alg:form_training} and \ref{alg:form_sampling}) introduces a novel approach by modeling the Lorentz force acting on the trajectory within a field, offering a fundamentally different perspective on learning the target distribution. This represents a new modeling paradigm that extends beyond conventional flow-based methods.


\begin{figure}[!ht]
\centering
\includegraphics[width=0.235\textwidth]{dataset_onedot.pdf} 
\includegraphics[width=0.235\textwidth]{trajectory_form_onedot.pdf}
\caption{Left: Onedot Dataset. The objective is to train the ForM model to learn a transport trajectory from distribution $\pi_0$ ({\textbf{blue}}) to distribution $\pi_1$ ({\textbf{pink}}). Right: The transportation trajectory generated by the ForM model.}
\label{fig:onedot_dataset}
\ifdefined\isarxiv
\else
\Description{}
\fi
\end{figure}


\begin{figure}[!ht]
\centering
\includegraphics[width=0.235\textwidth]{dataset_halfmoons.pdf}
\includegraphics[width=0.235\textwidth]{trajectory_form_halfmoons.pdf} 
\caption{Left: Halfmoons Dataset. The objective is to train ForM to learn a transport trajectory from distribution $\pi_0$ ({\textbf{blue}}) to distribution $\pi_1$ ({\textbf{pink}}). Right: The transportation trajectory generated by the ForM model.}
\label{fig:halfmoons_dataset}
\ifdefined\isarxiv
\else
\Description{}
\fi
\end{figure}

\begin{figure}[!ht]
\centering
\includegraphics[width=0.235\textwidth]{dataset_spiral.pdf}
\includegraphics[width=0.235\textwidth]{trajectory_form_spiral.pdf}
\caption{Left: Spiral Dataset. The objective is to train the ForM model to learn a transport trajectory from distribution $\pi_0$ ({\textbf{blue}}) to distribution $\pi_1$ ({\textbf{pink}}). Right: The transportation trajectory generated by the ForM model.}
\label{fig:spiral_dataset}
\ifdefined\isarxiv
\else
\Description{}
\fi
\end{figure}


\begin{figure*}[!ht]
\centering
\subfloat[O1, \cite{lgl22}]{\includegraphics[width=0.3\textwidth]{1_onedot.pdf}}
\subfloat[O1+O2]{\includegraphics[width=0.3\textwidth]{12_onedot.pdf}}
\subfloat[ForM, (Ours)]{\includegraphics[width=0.3\textwidth]{form_onedot.pdf}}
\\
\subfloat[O1, \cite{lgl22}]{\includegraphics[width=0.3\textwidth]{1_halfmoons.pdf}}
\subfloat[O1+O2]{\includegraphics[width=0.3\textwidth]{12_halfmoons.pdf}}
\subfloat[ForM, (Ours)]{\includegraphics[width=0.3\textwidth]{form_halfmoons.pdf}} \\
\subfloat[O1, \cite{lgl22}]{\includegraphics[width=0.3\textwidth]{1_spiral.pdf}}
\subfloat[O1+O2]{\includegraphics[width=0.3\textwidth]{12_spiral.pdf}}
\subfloat[ForM, (Ours)]{\includegraphics[width=0.3\textwidth]{form_spiral.pdf}}
\caption{
\textbf{Left:} Flow matching~\cite{lcb+22} using only the first-order term.
\textbf{Middle:} An improved method that incorporates both first- and second-order terms.
\textbf{Right:} Our proposed ForM model applied to the Onedot, Halfmoons, and Spiral datasets.  
Note that the first-order method (O1) fails to capture the target distribution, and although the second-order method (O2) exhibits slight improvement, it still does not adequately model the target distribution. In contrast, by leveraging the Lorentz force to guide the trajectory evolution, the ForM model significantly enhances the accuracy of the target distribution, as further evidenced by the quantitative results in Table~\ref{tab:euclidean_distance_complex_datasets_new}.
}
\label{fig:onedot}
\ifdefined\isarxiv
\else
\Description{}
\fi
\end{figure*}


\subsection{Results analysis} \label{sec:exp:result_anapysis}

This section presents a comparative analysis of trajectory evolution on the \textit{Onedot} and \textit{Halfmoons} datasets. Figure~\ref{fig:onedot} illustrates the trajectories generated by three approaches: (i) the original flow matching method that utilizes only the first-order term (denote as O1), (ii) an enhanced variant that incorporates both first- and second-order terms (denote as O1+O2), and (iii) our proposed Force Matching model (denote as ForM) which explicitly integrates the influence of a Lorentz force field.

Although the inclusion of the second-order term introduces additional dynamic information, the enhanced variant still fails to capture the complex transportation trajectories required by these datasets. In contrast, by modeling the potential Lorentz force, the ForM model markedly improves performance. This integration acts as a guiding mechanism, effectively steering the model toward a more accurate approximation of the target distribution. In other words, the physics-inspired addition of the Lorentz force facilitates the learning process, enabling the model to navigate the intricacies of the complex transport dynamics more efficiently.

This qualitative improvement is further substantiated by the quantitative results reported in Table~\ref{tab:euclidean_distance_complex_datasets_new}, where the ForM model achieves the lowest Euclidean distance loss among the three methods. These findings confirm that incorporating domain-specific forces, such as the Lorentz force, can significantly enhance the modeling of complex transport phenomena.

\subsection{Euclidean distance loss}\label{sec:exp:euclidean_distance_loss}
This section quantitatively assesses different approaches using Euclidean distance loss, which quantifies the deviation between transported and target distributions. Table~\ref{tab:euclidean_distance_complex_datasets_new} presents the loss values across three methods on the Onedot dataset and Halfmoons dataset, the unit we use is 0.1 light seconds. Lower values indicate higher precision in distribution transfer. The results demonstrate that incorporating the second-order term (O1 + O2) enhances performance beyond the first-order term (O1) alone. 
Notably, ForM surpasses both baselines, achieving the lowest Euclidean distance loss, reinforcing its effectiveness in modeling transport trajectories. 

\begin{table}[!ht]
\begin{center}
\caption{ \textbf{Euclidean distance loss across three methods: Flow matching~\cite{lcb+22} with the first order term ({\textbf{O1}}), an advanced version integrating both first order and second order terms ({\textbf{O1 + O2}}), and our proposed ForM on the Onedot dataset and Halfmoons dataset.} Lower values represent greater accuracy in distribution transfer. The optimal values are displayed in \textbf{bold}, while the second-best values are \underline{underlined}. The qualitative results are provided in Figure~\ref{fig:onedot}. }
\label{tab:euclidean_distance_complex_datasets_new}
\begin{tabular}{|l|c|c|c|}
\hline
\textbf{Loss terms}  & \textbf{Onedot} & \textbf{Halfmoons} & \textbf{Spiral} \\
\hline
O1, \cite{lgl22}             & 2.146 & 5.853 & 1.666 \\
\hline
O1 + O2          & \underline{2.048} & \underline{5.793} & \underline{1.578} \\
\hline
ForM, (Ours)      & \textbf{0.509} & \textbf{0.714} & \textbf{0.124} \\
\hline
\end{tabular}
\end{center}
\end{table}


