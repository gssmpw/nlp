This paper introduces Force Matching (ForM), a novel framework for generative modeling that represents an initial exploration into leveraging special relativistic mechanics to enhance the stability of the sampling process. By incorporating the Lorentz factor, ForM imposes a velocity constraint, ensuring that sample velocities remain bounded within a constant limit. This constraint serves as a fundamental mechanism for stabilizing the generative dynamics, leading to a more robust and controlled sampling process. 
We provide a rigorous theoretical analysis demonstrating that the velocity constraint is preserved throughout the sampling procedure within the ForM framework. To validate the effectiveness of our approach, we conduct extensive empirical evaluations. On the \textit{half-moons} dataset, ForM significantly outperforms baseline methods, achieving the lowest Euclidean distance loss of \textbf{0.714}, in contrast to vanilla first-order flow matching (5.853) and first- and second-order flow matching (5.793). Additionally, we perform an ablation study to further investigate the impact of our velocity constraint, reaffirming the superiority of ForM in stabilizing the generative process.
The theoretical guarantees and empirical results underscore the potential of integrating special relativity principles into generative modeling. Our findings suggest that ForM provides a promising pathway toward achieving stable, efficient, and flexible generative processes. This work lays the foundation for future advancements in high-dimensional generative modeling, opening new avenues for the application of physical principles in machine learning.

