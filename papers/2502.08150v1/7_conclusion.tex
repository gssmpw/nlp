\section{Discussion}


The core motivation behind the Force Matching (ForM) model stems from the need to stabilize generative modeling processes, particularly in flow-based methods where uncontrolled velocity magnitudes can lead to instability during sampling. Traditional approaches, such as Flow Matching (FM), offer different perspectives on modeling data evolution, yet they often lack explicit constraints that regulate sample movement, which can hinder both stability and efficiency. Inspired by special relativistic mechanics, we introduce a principled way to control velocity magnitudes through the Lorentz factor, ensuring that sample velocities remain bounded during the generative process. This formulation draws a parallel between relativistic motion and generative trajectories, where limiting sample speed prevents erratic behaviors and enhances robustness. By enforcing this constraint, ForM provides a novel perspective on generative modeling that aligns with both theoretical principles and practical stability considerations. 

While ForM establishes a strong foundation for stable generative modeling, several exciting directions remain open for further exploration. Extending ForM to high-dimensional image and video generation tasks would be a crucial next step, requiring efficient implementations of relativistic constraints in large-scale neural networks. Additionally, investigating how ForM’s velocity constraint can be integrated into score-based generative models could lead to hybrid approaches that combine the strengths of both paradigms. Another promising direction is the development of adaptive or learnable velocity constraints that dynamically regulate sample movement based on data complexity, potentially enhancing flexibility. More broadly, the incorporation of relativistic principles into generative modeling raises questions about the role of physics-inspired constraints in deep learning.
Finally, the implicit connection between ForM and optimal transport theory suggests that deeper theoretical investigations in this direction could lead to new generative frameworks grounded in optimal transport principles. 
By leveraging insights from physics and generative modeling, ForM paves the way for designing more stable, efficient, and interpretable generative models, inspiring further research into force-based approaches and their applications to complex data synthesis tasks.


\section{Conclusion} \label{sec:conclusion}

In this work, we introduce Force Matching (ForM), an innovative and comprehensive framework for generative modeling. ForM incorporates the principles of relativistic mechanics, higher-order flow matching, and TrigFlow, forming a unique and powerful synergy. The core idea behind ForM is to model the generative process in a way that accounts for both the geometric and dynamic aspects of flow, inspired by relativistic mechanics. We demonstrate that ForM not only preserves the structure of the data but also introduces an additional layer of stability to the generative process. Specifically, we theoretically prove that ForM bounds the velocity of the generative process under a hyperparameter $c$ during the sampling procedure, which leads to improved control and stabilization of the process. This stabilization mechanism mitigates issues such as mode collapse and sampling instability that often plague other generative models. 
Through extensive empirical experiments, we demonstrate that ForM outperforms both Flow Matching and second-order Flow Matching in terms of generative quality and sample diversity. These results highlight the effectiveness of incorporating higher-order dynamics and relativistic principles into the generative process. 
Additionally, we conduct an ablation study to evaluate the individual components of ForM, further demonstrating its superiority over existing methods. 
This analysis confirms the contribution of each aspect of the framework, such as the higher-order flow matching and relativistic dynamics, to its overall performance. 
ForM offers a fresh perspective on the understanding of Flow Matching within the context of relativistic mechanics, presenting a new paradigm in the field of generative modeling. 
By redefining the conceptual and practical foundations of generative processes, ForM sets a new benchmark for the future development of generative models.
