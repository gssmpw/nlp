\begin{figure}[ht!]
\centering
\includegraphics[width=\textwidth]{images/main.png}
\caption{Oriented Panel Detection Architecture.}
\label{fig:architecture}
\end{figure}

\section{Introduction}

The adoption of renewable energy resources is paramount to fighting the climate change crisis. Over the last decade renewable energy production has nearly quadrupled, 26\% of which is contributed by solar energy \citep{renewables}. To ensure that solar power plants are operating at maximum capacity, inspections utilizing remotely sensed imagery are important in identifying anomalies in damaged panels. However, with this rapid increase in solar production, it has become increasingly difficult to scale these inspections efficiently. The initial step in many inspection pipelines is to localize and georeference individual panels for downstream evaluation tasks. This can be prohibitively time-consuming for commercial sites, any of which can easily contain tens of thousands of solar panels. Therefore, this requires an automated approach for detecting and georeferencing individual solar panels from large scale aerial imagery. To be specific, we wish to create a model that directly predicts the coordinates of the vertices of each solar panel. We also require the approach to generalize to images of solar arrays in any orientation across a wide variety of landscapes and environments.

In this paper we introduce a rotated object detection framework for localizing individual solar panels with arbitrary orientation. We preprocess large-scale orthomosaics into patches which are then processed in batches. The predictions are stitched back together and projected from pixel coordinates to geospatial coordinates, creating an accurate mapping of individual panels in each site. To the best of our knowledge this is the first study to model solar panel detection as a rotated object detection problem which allows us to efficiently map individual solar panels in an end-to-end fashion. 