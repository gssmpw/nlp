\section{Related Work}
Most recent works build algorithms for mapping solar panels from aerial or satellite imagery \citep{DeepSolar, SegNet, MujtabaAndWani, GolovkoEtAl} with the goal of estimating surface area as a proxy for energy capacity, location density, and other potential insights. These studies have focused on both small-scale residential solar sites \citep{CrossNets, SolarDK} as well as large-scale commercial sites \citep{DeepSolar}. Most approaches use a two-stage architecture for mapping solar panels from aerial imagery \citep{SolarNet, HyperionSolarNet}. The first stage consists of a convolutional neural network (CNN) classifier for predicting the probability of the given image containing solar panels or not. The second stage uses a segmentation model for segmenting out solar panels, contingent upon the first stage predicting the given image to contain solar panels. While these methods have shown promise in mapping the surface area of solar arrays across a diverse set of geographies, it remains nontrivial to extract the coordinates of individual solar panel vertices at scale. 