\section{Methodology}
In this section, we describe the proposed framework's architecture and the modeling of detection targets for solar panels.

\subsection{Rotated Bounding Boxes}
\label{section:bounding_boxes}
Our key desideratum is to accurately localize individual solar panels, regardless of the orientation of the panels in the image, and to use this localization to extract the coordinates of the vertices of each panel. Accordingly, we model the problem as an arbitrarily oriented object detection task where the ground truth annotations of the solar panels are represented as rotated bounding boxes. Each box is comprised of the 5-dimensional tuple $(x, y, w, h, \theta)$ where $(x, y)$ is the centroid pixel coordinate of the box, $w$ is the width defined as the shorter side of the box, $h$ is the height defined as the longer side of the box, and $\theta$ is the rotation angle in degrees required to obtain the rotated box. This bounding box formulation provides a compact representation which allows us to easily convert to the complete set of vertex coordinates $(x_1, y_1, x_2, y_2, x_3, y_3, x_4, y_4)$ using equation \ref{eq:vertices} in Appendix \ref{appendix:implementation}.

\subsection{Rotated Anchors}
Following R-CNN style object detection architectures, we define a set of proposal anchors as a strong basis for regression to ground truth rotated bounding boxes. In addition to scale and aspect ratio, we also provide rotation angle anchors. We specify seven different rotation anchors: $-90^{\circ}$, $-60^{\circ}$, $-30^{\circ}$, $0^{\circ}$, $30^{\circ}$, $60^{\circ}$, and $90^{\circ}$. We use standard anchors for scale and aspect ratio parameters which are defined as $[32, 64, 128, 256, 512]$ and [1:2, 1:1, 2:1], respectively. This produces a set of 105 anchors in total (7 rotation anchors, 5 scale anchors, and 3 aspect ratio anchors).

\subsection{Rotated Object Detection}
We adopt a general Faster R-CNN architecture \citep{FasterRCNN} with a CNN backbone, region proposal network (RPN), classification and box regression head, and a region of interest (RoI) pooling layer. We use the oriented object detection framework from \citep{RRPN} in which the authors present rotated RPN (RRPN) and rotated RoI (RRoI) modifications. The RRPN performs two functions: it produces rotated bounding box proposals from the specified box anchors defined in Section \ref{section:bounding_boxes} and regresses the box proposals to accurately localize panel instances for each feature map. From the RRPN, the proposals and feature maps are fed into the RRoI pooling layer which projects the arbitrarily oriented proposals onto each feature map and subsequently performs a max pooling operation. The pooled features from the RRoI layer are fed into a classification head and a box regression head which produce class labels, class scores, and box coordinates 
for each solar panel instance. The architecture can be seen in Figure. \ref{fig:architecture}.