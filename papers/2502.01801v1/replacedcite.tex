\section{Related Work}
Our work is inspired by previous work on wearable assistants for older adults as well as object-finding and lifelogging tools. Since memory aids are typically underutilized in older adults than in younger adults ____, we explore the interaction of a multimodal user interface within the older adult population to assist with memory. 
% reframe to (1) overview of memory aids for older adults- all types. Then talk about how younger vs older people approach memory aids - https://onlinelibrary.wiley.com/doi/abs/10.1002/acp.2946?casa_token=a5HTMdaIx80AAAAA%3AR1iSMU16PA3BgxD26399vRlZiTH6_q5Z-dHSYt4QWvQjP5fKEzBO9JRASPsfxHo3Mvajj0k-Loc77do 

% clearly state the novelty in the work- automated diary, voice based, AI context based and home tour (limited onboarding) 
    

\subsection{Voice-based, Camera-based and Wearable Assistants for Older Adults}
\label{relatedworks_assistants}
%voice-based 

Prior works showed positive perceptions towards voice user interfaces (VUIs) by older adults and benefits of VUIs (e.g., Google Home, Amazon Alexa, social robots) in their daily lives more so than younger adults ____. These benefits include being assistants for various functions such as seeking information and controlling home devices____, companions to alleviate loneliness____, helping users develop skills____, encouraging physical activity____, \textcolor{black}{and providing personalized reminders through AR____}. Wearable conversational agents have been used to provide memory training for older adults____ and monitor their daily physical activity____ but these agents do not use visual data for \textcolor{black}{general} memory assistance. 

%It finds that optimal trigger times for these voice based prompts/reminders are during lower cognitive loads and idle times, suggesting that assisting with prospective memory loss should take voice-notification timing into consideration. 
%As MemPal aims to provide just-in-time safety reminders based on location context, we investigate the potential timing for these prompts during tasks. 
%The language of voice queries is closer to natural language than typed queries____, thus, voice-based assistants can be highly usable compared to screen-based/typing-based interfaces. 
With the integration of language models in voice assistants and advancements in language understanding, users can converse more naturally with these agents. LLMs now have an increased ability to process larger unstructured text and provide a contextualized response. In this work, we implement a multi-modal real-time LLM voice-based assistant that includes image input as context and uses vision language models for intelligent understanding. \textcolor{black}{Although the use of LLMs on older adults has recently been explored for fostering conversations such as A-CONECT ____ and Mindtalker ____, the use of multimodal LLMs that can process greater context has not.}


%Camera-based [research lifelogging literature] Remote tracking of the older adults using c
Previous Human-Computer Interaction (HCI) studies have mainly explored camera-based systems to support remote tracking in older adults' homes and could reduce caregiver burden via a smart-connected home with camera sensors. These systems could provide health monitoring____, detecting anomalies in activities of daily living____ and fall detection____.
%Prior Human-Computer Interaction (HCI) studies and commercial products aim to solve this issue via a smart connected home with camera sensors or wearable camera systems. 
%Since 2002, the proposed idea of the aware-home has been an ongoing study to transform the home of an older adults individual into one that is integrated with sensors for better patient monitoring ____. 
%Applications for in home camera sensors primarily include fall detection systems ____ or activity recognition to detect anomalies in activities of daily living ____. 
Despite this, the technology readiness level of smart home and health monitoring technologies for older adults has been considered low and there is limited evidence that they help improve quality of life____. Other limitations include the need to deploy multiple cameras to broaden monitoring coverage____. %The extent of tracking is directly proportional to the number of installed cameras, with limited cameras potentially leading to gaps in activity identification.  
%Recent wearable camera systems have become more prevalent in the market, suggesting wider spread adoption of these technologies among the public. Examples include the Humane AI clip, Narrative clip camera, and smart glasses like RayBan Meta smart glasses. These devices are more lightweight and useful in tracking egocentric activity but not currently designed for older adults and especially people with memory struggles.  
Wearable camera-based systems have primarily been used as lifelogging devices for supporting retrospective memory in older adults (i.e. remembering past events or people's names), such as the SenseCam____ and Autographer plus Flo____ (both of which are neck-worn devices), but store images as memory, and approaches to text-based memory diaries require integration with smart homes for activity monitoring ____.
%Smart glasses, like Google Glass, could support older adults in checking if they performed tasks by reviewing previous captured image on the heads-up display and for providing time-based reminders____.% Devices like lifelogging clips (e.g., Narrative Clip____) are more likely to be worn by older adults compared to head-worn form factors (e.g., smart glasses, VR headsets, LED glasses)____.

Building upon literature, we present a combined wearable camera-based and voice-based system that helps older adults with object tracking and activity monitoring. To allow for both modalities, our proposed system would store text-only data as memory unlike other systems and prioritize older adult preferences for simplistic interfaces ____ such as voice-based and limited onboarding. %Building upon literature, we use a neck-worn instead of a head-worn form factor.

%for audio reminders + QA 
%voice interactions.

%Kidd, Cory D., et al. "The aware home: A living laboratory for ubiquitous computing research." Cooperative Buildings. Integrating Information, Organizations, and Architecture: Second International Workshop, CoBuild’99, Pittsburgh, PA, USA, October 1-2, 1999. Proceedings 2. Springer Berlin Heidelberg, 1999. [… signifying our intent to produce an environment that is capable of knowing information about itself and the whereabouts and activities of its inhabitants.]

%Meyer, Sven, and Andry Rakotonirainy. "A survey of research on context-aware homes." Proceedings of the Australasian information security workshop conference on ACSW frontiers 2003-Volume 21. 2003. [presented examples of how people could benefit from living in context-aware homes and outlined issues to keep in mind during their development. ]

%White, Carla, and James N. Gilmore. "Imagining the thoughtful home: Google Nest and logics of domestic recording." Critical Studies in Media Communication 40.1 (2023): 6-19. [The Google Nest home security system offers an array of cameras, sensors, and Internet-connected devices to allow homeowners to monitor and record the exterior and interior of their home and automate various functions of heating and cooling, lights, and other appliances through smartphone application control panels.]

%Fleck, Sven, and Wolfgang Straßer. "Smart camera based monitoring system and its application to assisted living." Proceedings of the IEEE 96.10 (2008): 1698-1714. [It covers georeferenced person tracking and activity recognition (falling person detection). A prototype system installed in a home for assisted living has been running 24/7 for several months now and shows quite promising performance.]

%Nasution, Arie Hans, and Sabu Emmanuel. "Intelligent video surveillance for monitoring older adults in home environments." 2007 IEEE 9th Workshop on Multimedia Signal Processing. IEEE, 2007. [a novel method to detect and record various posture-based events of interest in a typical older adults monitoring application in a home surveillance scenario.]

%Molesworth, Sue, and Lisa Sharrock. "An Evaluation of" Autographer plus Flo." (2016). []
%Hoisko, Jyrki. "Context triggered visual episodic memory prosthesis." Digest of Papers. Fourth International Symposium on Wearable Computers. IEEE, 2000. [Presenting visual content about a user's life, in mobile use, should help memory to recall other facts about the user's context.]

%Klebbe, R., Steinert, A., & Müller-Werdan, U. (2019). Wearables for older adults: requirements, design, and user experience. Perspectives on Wearable Enhanced Learning (WELL) Current Trends, Research, and Practice, 313-332.

%Kunze, K., Henze, N., & Kise, K. (2014, September). Wearable computing for older adults: initial insights into head-mounted display usage. In Proceedings of the 2014 ACM International Joint Conference on Pervasive and Ubiquitous Computing: Adjunct Publication (pp. 83-86).

%devices like the lifelogging clip (ex. Narrative clip) are more likely to be worn by older adults compared to other form factors that could sense camera input (smart glasses, VR headsets, LED glasses) ____.

%[key gaps: safety reminders are location based, wearable, proactive, passive tracking and synced to caregivers→ helping both patient and caregiver] 

%One possible concern with wearable devices is the social and perceived acceptability depending on body-location, size, and visibility. A study published by BMC geriatrics proves that specifically lifelogging tools for the older adults are considered acceptable unless individuals are using the camera in public outside of the home ____. The FMT and Go-finder paper also performed acceptability studies and found that individuals were more concerned with the functionality and how it could assist them rather than social acceptability. We evaluated possible form factors that would provide ease of comfort, accurate tracking, and limited interference includes glasses. Using the stereotype content model (SCM) that predicts a device's stereotypical perception and social acceptability, a study shows that devices like the lifelogging clip (ex. Narrative clip) are more likely to be worn by older adults compared to other form factors that could sense camera input (smart glasses, VR headsets, LED glasses) ____.

\subsection{Object Finding Systems}
 
There are several wearable camera systems that help users find misplaced objects but all provide visual aid either on a tablet or AR display and do not store long term memory to allow for chat-based interfaces. Audio based assistance has yet to be explored. Fiducial Marker Tracker (FMT) had a neck-worn camera and was tested with older adults; it required manual registration of objects (placing marker tags on objects) and captured videos anytime the objects are interacted with____. To recall the object's last state (e.g., light: on/off), users sorted through video footage grouped by object, which might not provide a seamless experience for object tracking. FMT also required installation and markers around the house, and marker detection was heavily dependent on user’s height (limited personalization). GoFinder also had a neck-worn camera but was registration-free and allowed users to review automatically-grouped images to find objects____ but not categorizing the name of the object. It was, however, only tested with young adults (18-28 years old) and has not yet been studied with older adults. Our proposed solution would not store any image data and provide a more seamless user experience for older adults by only voice based querying.  %It allows users to explore	video clips that capture objects of interest and can help user determine if they have completed certain actions with these objects. 
%Yagi, Takuma, et al. "GO-finder: a registration-free wearable system for assisting users in finding lost objects via hand-held object discovery." 26th International Conference on Intelligent User Interfaces. 2021.[Go-Finder]
Systems like LocatAR ____ or Overthere ____ either simplified or removed the object registration process entirely based on gesturing or user motions but required an AR headset system for continuous use posing questions about social acceptability which our solution would avoid. %was an augmented reality headset system that automatically registered objects and object movement based on user's grasping and placing motions, and then visually displayed directions to find the objects. 
%Oshimi, H., Perusquía-Hernández, M., Isoyama, N., Uchiyama, H., & Kiyokawa, K. (2023, March). LocatAR: An AR Object Search Assistance System for a Shared Space. In Proceedings of the Augmented Humans International Conference 2023 (pp. 66-76).
%The automatic registration is performed by image-based item movement recognition from the user's grasping and placing motions. Tracks item movement
%Another notable wearable system____ showed that displaying the last seen image of the object on head-mounted displays was more effective at helping users find objects compared to map view of the objects' location.

There are other camera-based systems, like CamFi, an AI-driven system to help find lost objects in multi-user scenarios that uses stationary camera and displays objects and last-seen users of the objects on a smartphone____. It has also only been tested with younger adults who are generally more technologically savvy ____. 
Other commercially-available sensor systems, like RFID tags____, Ultra-Wideband (UWB) Systems ____ or Wi-Fi Fingerprint-Based Indoor Positioning ____, can track objects but require manually tagging of these devices to the objects.

%Yan, Ge, et al. "CamFi: An AI-driven and Camera-based System for Assisting Users in Finding Lost Objects in Multi-Person Scenarios." CHI Conference on Human Factors in Computing Systems Extended Abstracts. 2022. [Lost objects found with multi-person scenario] 
%[Other commercially-available systems: Rfid tags + airtags] 
%Gofinder 
%Li, Franklin Mingzhe, et al. "FMT: A wearable camera-based object tracking memory aid for older adults." Proceedings of the ACM on Interactive, Mobile, Wearable and Ubiquitous Technologies 3.3 (2019): 1-25. [explore whether	video clips captures from a body-worn camera every time objects	 of interest are	 found within its field of view can help older adults determine if they have completed certain actions with these objects and	 what their states are]

While LLMs have been used to identify the attributes of objects being found by a mobile robot in the FindThis system____, none of the works, to date, have used LLMs for helping users themselves to find the objects and enabled a voice-based chat interface for querying and response. This work presents a system which uses multi-modal LLM AI capabilities to support one of its key features of object finding designed specifically for older adults allowing them to also ask follow-up questions on the whereabouts and action of the user during misplacement. It uses visual inputs as context to the LLM so that no explicit object markers or tags are required. It stores object location information as text instead of images to be mindful of users' privacy and to enable audio descriptions of where any hand-held object is located. We also build upon Go-Finder's study design____ to test our proposed object-finding feature.


%[key gaps: system design (multimodal LLM AI tool), no explicit object markers, no images stored-> enabling audio descriptions  )], build open go-finder study design and used visual Visual condition 




%\subsection{Context-based Reminder and Tracking Systems}
%RQ2 Context based reminder systems: Context-based vs strictly time based reminders allow for flexible scheduling. 

%Retrospective memory loss (e.g., forgetting past events), affects older adults individuals with slight cognitive failures and progressively becomes worse with people with dementia and Alzheimer’s. Currently, to help older adults with this issue, speech-language pathologists have implemented ongoing diaries to practice recall. However, the task of continuous logging is quite tiresome and requires patients to accurately remember previous tasks. Additionally, prospective memory, forgetting future plans or intentions, also greatly affects older adults individuals. Clinch shows that it accounted for 72 percent of reported memory failures among their studied population of elders ____. 

%Most common solutions on the market for aiding prospective memory rely on time based context to trigger reminders that must be specified in advance such as pill medication alarms____ and stove top alerts____ which may be useful for specific circumstances but not general purpose for a variety of custom safety reminders. Therefore, more comprehensive context-aware reminding systems are necessary to provide maximum effectiveness in ensuring general home safety in older adults____ which visual contexts provide.

%The majority of these reminder systems use time as the main context like stove top alarms____ and smart pill systems. %(proactive + time but no visual context- location + activity). 

%Visual contexts could include location contexts____ or a combination of contexts, such as enhancing home safety using location and object as contexts____ and using time, location and users as contexts, like in Cybreminder____. Cybreminder's contextual architecture still requires users to manually denote whether a task is completed, which our proposed solution would passively track, and also would store memory of previous activities for assisting with retrospective memory loss.

%Reminder systems for older adults focus more on using activity as context. For example, Autominder____ uses older adults' plans and schedules to determine when to issue reminders. 
%Other systems use planned activities and current activities to deliver reminders____. Camera-based____ and wireless sensor tag-based____ activity tracking systems have been used to remind older adults with dementia of what they had to do next to complete activities of daily living such as hand washing. Many other systems that monitor activities of older adults in home settings focus on providing alerts to caregivers and family members____ instead of reminders or alerts to older adult users at home. 
%Lin, Chi-yau, et al. "Object reminder and safety alarm." Embedded and Ubiquitous Computing–EUC 2005 Workshops: EUC 2005 Workshops: UISW, NCUS, SecUbiq, USN, and TAUES, Nagasaki, Japan, December 6-9, 2005. Proceedings. Springer Berlin Heidelberg, 2005. [“Where did I leave my keys?”, “Did I turn off the stove?”, or “Did I close all the windows in my house?” etc.]
%Dey, Anind K., and Gregory D. Abowd. "Cybreminder: A context-aware system for supporting reminders." Handheld and Ubiquitous Computing: Second International Symposium, HUC 2000 Bristol, UK, September 25–27, 2000 Proceedings 2. Springer Berlin Heidelberg, 2000. [We describe CybreMinder, a prototype context-aware tool that supports users in sending and receiving reminders that can be associated to richly described situations involving time, place and more sophisticated pieces of context]
%While there are a range of systems that track and monitor activities of older adults in home settings____ (as also mentioned in Section~\ref{relatedworks_assistants}), many do not focus on providing reminders using activity as a context. 

%Our system distinguishes itself by integrating camera input with advanced language models for activity recognition and delivering both real-time and proactive safety reminders for older adults. This approach leverages visual data to understand the context of location and activities and uses the same infrastructure of text-based diary for memory that enables the object finding feature. Building on prior research, our system also includes passive activity tracking and provides caregivers with timely alerts, enhancing overall safety and support.

%[key gaps: safety reminders are location based, wearable, proactive/real time, passive tracking and synced to caregivers] 

%% Potential and need for recognizing crisis: falls, broken heaters in winter -- systems need to recognise potential problems and notify home occupants and outside support. 
% Mynatt, E. D., Essa, I., & Rogers, W. (2000, November). Increasing the opportunities for aging in place. In Proceedings on the 2000 conference on Universal Usability (pp. 65-71).



%Activity-based: Aware home, sensecam

%____, detecting anomalies in activities of daily living____ and fall detection____

%[ADDED] Kidd, Cory D., et al. "The aware home: A living laboratory for ubiquitous computing research." Cooperative Buildings. Integrating Information, Organizations, and Architecture: Second International Workshop, CoBuild’99, Pittsburgh, PA, USA, October 1-2, 1999. Proceedings 2. Springer Berlin Heidelberg, 1999. [… signifying our intent to produce an environment that is capable of knowing information about itself and the whereabouts and activities of its inhabitants.]

%[ADDED] Meyer, Sven, and Andry Rakotonirainy. "A survey of research on context-aware homes." Proceedings of the Australasian information security workshop conference on ACSW frontiers 2003-Volume 21. 2003. [presented examples of how people could benefit from living in context-aware homes and outlined issues to keep in mind during their development. ]
%%  (Mynatt, Essa and Rogers 2000) (Lines and Hone 2002).  (Stanford 2002). 


%White, Carla, and James N. Gilmore. "Imagining the thoughtful home: Google Nest and logics of domestic recording." Critical Studies in Media Communication 40.1 (2023): 6-19. [The Google Nest home security system offers an array of cameras, sensors, and Internet-connected devices to allow homeowners to monitor and record the exterior and interior of their home and automate various functions of heating and cooling, lights, and other appliances through smartphone application control panels.]

%[ADDED] Fleck, Sven, and Wolfgang Straßer. "Smart camera based monitoring system and its application to assisted living." Proceedings of the IEEE 96.10 (2008): 1698-1714. [It covers georeferenced person tracking and activity recognition (falling person detection). A prototype system installed in a home for assisted living has been running 24/7 for several months now and shows quite promising performance.]

%[ADDED] Nasution, Arie Hans, and Sabu Emmanuel. "Intelligent video surveillance for monitoring older adults in home environments." 2007 IEEE 9th Workshop on Multimedia Signal Processing. IEEE, 2007. [a novel method to detect and record various posture-based events of interest in a typical older adults monitoring application in a home surveillance scenario.]



%Add more 
%Ubicomp recording technology: Nguyen, David H., et al. "Encountering SenseCam: personal recording technologies in everyday life." Proceedings of the 11th international conference on Ubiquitous computing. 2009. [sensecam only activity tracking not reminders]

%Kidd, Cory D., et al. "The aware home: A living laboratory for ubiquitous computing research." Cooperative Buildings. Integrating Information, Organizations, and Architecture: Second International Workshop, CoBuild’99, Pittsburgh, PA, USA, October 1-2, 1999. Proceedings 2. Springer Berlin Heidelberg, 1999. [… signifying our intent to produce an environment that is capable of knowing information about itself and the whereabouts and activities of its inhabitants.]

%Meyer, Sven, and Andry Rakotonirainy. "A survey of research on context-aware homes." Proceedings of the Australasian information security workshop conference on ACSW frontiers 2003-Volume 21. 2003. [presented examples of how people could benefit from living in context-aware homes and outlined issues to keep in mind during their development. ]

%White, Carla, and James N. Gilmore. "Imagining the thoughtful home: Google Nest and logics of domestic recording." Critical Studies in Media Communication 40.1 (2023): 6-19. [The Google Nest home security system offers an array of cameras, sensors, and Internet-connected devices to allow homeowners to monitor and record the exterior and interior of their home and automate various functions of heating and cooling, lights, and other appliances through smartphone application control panels.]

%Fleck, Sven, and Wolfgang Straßer. "Smart camera based monitoring system and its application to assisted living." Proceedings of the IEEE 96.10 (2008): 1698-1714. [It covers georeferenced person tracking and activity recognition (falling person detection). A prototype system installed in a home for assisted living has been running 24/7 for several months now and shows quite promising performance.]


%Since 2002, the proposed idea of the aware-home has been an ongoing study to transform the home of an older adults individual into one that is integrated with sensors for better patient monitoring ____. Applications for in home camera sensors primarily include fall detection systems ____ or activity recognition to detect anomalies in activities of daily living ____. However, studies show that the technology-readiness level of smart-homes is low ____ and there is limited evidence on the ability to help address health related quality of life. Other limitations include the range and coverage of patient activity tracking. The extent of tracking is directly proportional to the number of cameras installed, potentially leading to incomplete activity identification. 





\iffalse

\subsection{Cognitive Assistants for the older adults}

Retrospective memory loss (e.g., forgetting past events), affects older adults individuals with slight cognitive failures and progressively becomes worse with people with dementia and Alzheimer’s. Currently, to help older adults with this issue, speech-language pathologists have implemented ongoing diaries to practice recall. However, the task of continuous logging is quite tiresome and requires patients to accurately remember previous tasks. Additionally, prospective memory, forgetting future plans or intentions, also greatly affects older adults individuals. Clinch shows that it accounted for 72 percent of reported memory failures among their studied population of elders ____. Current solutions on the market for aiding with prospective and retrospective memory loss include: pill medication alarms, stove top alerts, but although useful for specific circumstances these smart sensors are not general purpose. Therefore, context aware reminding systems are necessary in order to provide maximum effectiveness ____.   

\subsection{Early memory augmentation systems}

\subsection{Camera Based Technologies for older adults }

Remote tracking of the older adults using camera based systems within a user's home reduces caregiver burden. Prior HCI studies and products on the market aim to solve this issue via a smart connected home with camera sensors or wearable camera systems. Since 2002, the proposed idea of the aware-home has been an ongoing study to transform the home of an older adults individual into one that is integrated with sensors for better patient monitoring ____. Applications for in home camera sensors primarily include fall detection systems ____ or activity recognition to detect anomalies in activities of daily living ____. However, studies show that the technology-readiness level of smart-homes is low ____ and there is limited evidence on the ability to help address health related quality of life. Other limitations include the range and coverage of patient activity tracking. The extent of tracking is directly proportional to the number of cameras installed, potentially leading to incomplete activity identification. 

In terms of wearable camera systems, there are two notable devices that focused on object tracking and helping users find misplaced objects: FMT and Go-Finder. The fiducial marker tracker (FMT) ____ is a wearable camera-based object tracking memory aid, that requires manual registration of objects and captures video footage anytime the object is interacted with. In order to recall the object's last state, users have to manually sort through video footage grouped by object, which does not provide a seamless experience for object tracking. GoFinder, another object finding wearable device, does not require object registration but instead requires manual lookup and image storage ____. Recent wearable camera systems have become more prevalent in the market, suggesting wider spread adoption of these technologies among the public. Examples include the Humane AI clip, Narrative clip camera, and smart glasses like RayBan Meta smart glasses. These devices are more lightweight and useful in tracking egocentric activity but not currently designed for the older adults especially people with memory problems.  

\subsection{Social Acceptability of Wearable Cameras}

One possible concern with wearable devices is the social and perceived acceptability depending on body-location, size, and visibility. A study published by BMC geriatrics proves that specifically lifelogging tools for the older adults are considered acceptable unless individuals are using the camera in public outside of the home ____. The FMT and Go-finder paper also performed acceptability studies and found that individuals were more concerned with the functionality and how it could assist them rather than social acceptability. We evaluated possible form factors that would provide ease of comfort, accurate tracking, and limited interference includes glasses. Using the stereotype content model (SCM) that predicts a device's stereotypical perception and social acceptability, a study shows that devices like the lifelogging clip (ex. Narrative clip) are more likely to be worn by older adults compared to other form factors that could sense camera input (smart glasses, VR headsets, LED glasses) ____.


\subsection{Chat based Voice Interfaces}

Relational agents, which refer to competent computer based conversationalists, have shown to be effective for the older adults population ____. More specifically, prior studies analyze the efficacy of voice user interfaces (VUI) for the older adults (ex. Google Home, Alexa, etc.). A systematic review addressed the perception and use of VUIs by older adults, showing benefits of these interfaces on daily lives of these individuals ____. Guy showed that the language of voice queries is closer to natural language than typed queries ____. These conversational assistants can be used for question answering or initiate reminders and prompts. For example, Prompto is a conversational memory coach using cognitive contexts for providing prospective memory training sessions ____. It finds that optimal trigger times for these voice based prompts/reminders are during lower cognitive loads and idle times, suggesting that assisting with prospective memory loss should take voice-notification timing into consideration. As MemPal aims to provide just-in-time safety reminders based on location context, we investigate the potential timing for these prompts during tasks. 

Additionally, with the integration of language models in voice assistants, users can converse more naturally with these agents, due to the advancements in language understanding. LLMs now have an increased ability to process larger unstructured text and provide a contextualized response. However, multimodal real-time voice assistants that include image and video input as context have not been explored in assisting older adults with memory conditions.

\fi