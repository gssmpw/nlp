\begin{table*}
\centering
\begin{tabular}{|l|p{3cm}|p{4cm}|p{6cm}|}
\hline
\textbf{Issue} & \textbf{Description} & \textbf{Examples} & \textbf{Quotes} \\ \hline
\textbf{Issue 1: Object-finding} & Finding lost objects is frustrating and time-consuming. & Losing a purse in the home, concerns about not recognizing personal possessions, frequently lost objects. & OC8: ``My MIL was always missing her purse and was sure someone had stolen it. She would ransack her room all night long. It would be in the closet. This happened often.'' \\
& & & OC9: ``Hearing aids and purse. Sometimes hearing aids are in the purse. Drives my poor dad nuts.'' \\
& & & OC10: ``The phone chargers! Always misplaces her iPhone cord, looks for another one, takes someone else's type C, then replaces that with some random micro USB or whatever... ugh... The issue here is that she doesn't know what is hers and what isn't, and ends up taking and keeping or throwing away other people's belongings.'' \\
& & & OC11: ``Glasses and dentures'' \\ \hline

\textbf{Issue 2: Safety Reminders} & Safety reminders for daily tasks that may be forgotten and cause safety risks. & Forgetting important items when leaving the home, safety concerns related to short-term memory loss, feedback from speech pathologists or doctors on safety reminders. & \textbf{Anecdotal Scenarios} \\
& & & D1: ``You have the most leverage when developing systems to ensure safety, especially during activities like cooking.'' \\
& & & E6: ``Didn't have a wallet or purse when I left the house.'' \\
& & & E7: ``I forgot to put milk back in the fridge and my wife got mad.'' \\
& & & E8: ``I have issues with short term memory—when going to the doctor if you’re asked what did you do today being like I have no idea, it’s hard to remember the day to day.'' \\ 
& & & \textbf{Current Safety Monitoring} \\
& & & SP1: ``Speech pathologists use sequence boards to help with safety trained tasks so a log of safety related tasks can reduce our burden as well as caregiver burden.'' \\
& & & C10: ``Extra burden of decision making is being deferred to caregivers.'' \\
& & & C12: ``Use scraps of paper to write a schedule, then I got a portable whiteboard and big letters that I print out first thing in the morning.'' \\
& & & OC1: ``I put up 'reminders' for my husband suffering from dementia and they didn't do a damn bit of good.'' \\
& & & OC2: ``My written reminders got ignored as their reality says it’s not needed...'' \\
& & & OC3: ``I found it necessary to remove things, place things out of reach, and supervise with cameras and movement alarms to keep my person safe.'' \\
& & & OC4: ``The medication was also a problem for my person who forgets what was taken...'' \\
& & & OC5: ``Too many signs or signs up too long become white noise...'' \\
& & & OC6: ``We have notes everywhere, but he doesn't pay attention to any of them...'' \\
& & & OC7: ``Had a whiteboard directly in front of the chair where my mom slept...'' \\
& & & OC8: ``Some other things might be to lock the door, make sure faucets are turned off...'' \\ \hline

\textbf{Issue 3: Confabulation of Past Events} & Difficulty remembering longer sequences of past events, leading to confabulation. & Passive remote tracking as a solution, caregivers and speech pathologists dealing with confabulation, anecdotes of ways to manage this. & D1: ``Remote patient monitoring is very helpful and tracking is helpful since we use a lot of cognitive assessment.'' \\
& & & D2: ``Passive sensors are better since caregivers are usually filling out subjective questionnaires of activities of daily living form.'' \\
& & & E6: ``My biggest issue with his memory is that he'll tell us he has something when he doesn't. He lies daily.'' \\
& & & E7: ``While it is technically lying, the term is confabulation. He likely doesn't remember...'' \\
& & & E8: ``When I left for short trips, I would give her a printed page of instructions...'' \\
& & & SP1: ``A memory book is a diary of daily activities that can help patients remember events...'' \\ \hline
\end{tabular}
\caption{Summary of Issues, Descriptions, Examples, and Quotes}
\label{tab:issues}
\end{table*}
