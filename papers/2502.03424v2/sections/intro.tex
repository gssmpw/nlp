\section{Introduction}
\label{introduction}
With the rapid urbanization and increasing complexity of building structures, fire safety has become a critical global issue. Fires pose significant challenges to structural integrity and stability of buildings, making it essential to incorporate fire safety measures during the design phase to enhance resilience and prevent catastrophic failures. The devastating impact of recent wildfires in California highlights the urgency of improving fire risk assessment and mitigation measures in the built environment \cite{cal_fire2023}.

\revise{
In designing a building structure, there are many requirements to be met, e.g., requirements related to gravity load stability, earthquake resistance, wind effects, and fire safety. Before a final check with multiple load combinations, e.g., combination of gravity, earthquake and fire loads, each individual requirement should be satisfied in a pre-check phase. 
} A standard fire safety requirement mandates that a building must withstand collapse for at least one hour under fire conditions at any specific point within the structure. \revise{Current methods to ensure this requirement involve computational simulations, e.g., using Finite Element Analysis (FEA) tools, where fire scenarios are simulated at all potential locations within the building structure.} This exhaustive approach is computationally expensive and time-consuming, increasing the cost and duration of the structural design process. 

\revise{To address this inefficiency, we first propose the concept of the Most Fire-Sensitive Point (MFSP) --the location where fire has the greatest impact on a structure's lateral stability. The MFSP corresponds to the worst-case scenario when a fire occurs. By identifying the MFSP, designers may only need to simulate only its corresponding fire scenario to evaluate structural compliance in the pre-check phase, significantly reducing computational costs and effectively streamlining the design process. However, identifying the MFSP is not a trivial task --conventionally, brute-force methods with computational simulations are still required. Fortunately, recent Machine Learning (ML) advancements provide new opportunities to efficiently identify the MFSP as conducted in this study. Regarding the fire scenarios, \cite{nan_structuralfire_2023} and \cite{wang_graph_2024} employed Recurrent Neural Network (RNN) for real-time displacement prediction, focusing on post-fire collapse warning systems by forecasting future outcomes through early-stage fire features. On the other hand, \cite{chen_tempnet_2024} utilized Graph Neural Network (GNN) for post-fire concrete temperature field mapping with material-level assessment. Despite these works, using ML techniques to assess the overall structural lateral stability in the design stage without real-time information of fires remains a major challenge. 
}

\revise{Moreover, even with a Neural Network (NN) predictor, it is difficult to directly train it for MFSP prediction. Traditional supervised learning requires large amount of of labeled data, i.e., structures with ground truth MFSP labels. However, obtaining such labeled data is impractical due to the required brute-force simulations, which is both labor-intensive and computationally prohibitive. To overcome this challenge, we propose a framework with two Graph Neural Network (GNN)-based predictors: the Maximum Interstory Drift Ratio (MIDR) predictor as the differentiable agent of fire simulations to assess the overall structural lateral stability, and the MFSP predictor as the ``argmaxer`` of the MIDR predictor to predict the MFSP. 
}
% The efficiency and differentiability of this agent are fully utilized by pseudo labeling dataset and directly guiding the argmaxer's training. A dataset generator is also included in this framework.

The workflow begins with FEA simulations for a limited set of fire scenarios, which are used to train the MIDR predictor. 
\revise{In traditional FEA simulations, the complex operations due to nonlinear material behavior, iterative solvers, and discrete time stepping make traditional FEA tools non-differentiable as well as computationally expensive. In contrast, our differentiable agent uses a GNN with differentiable operations (e.g., a ReLU-activated multilayer perceptron), ensuring computations of the gradients for end-to-end training. Compared to traditional physics-based simulations, the GNN-based MIDR predictor offers two major advantages: (1) the differentiability enables the agent to work as an indicator to directly guide the training of the MFSP predictor to be an ``argmaxer'' of the agent, and (2) the computational efficiency enables it to generate synthetic data to assist in the training, allowing it to efficiently identify the MFSP without requiring extensive real-world data.
}

\revise{This study focuses on framed steel building structures, which are widely used in modern construction, including commercial facilities. However, the proposed research method is not limited to steel structures and can be extended to other types of buildings with similar structural configurations.} The primary contributions of this research are as follows:
\begin{itemize}
    \item {\bf{Data creation}}: \revise{We generated a large-scale dataset of building structures, encompassing their geometry, material properties, and nodal displacements under gravity loads. After filtering out unreasonable structures in the whole dataset, 16,050 structures form the {\em{unlabeled}} dataset, and another 1,573 structures were simulated for 30 fire scenarios, forming the {\em{labeled}} dataset, which contains the nodal displacements for each of the $1,573 \times 30 = 47,190$ fire cases, obtained using a FEA tool, namely OpenSeesRT \cite{perez2024openseesrt}.}
    \item {\bf{Prediction framework}}: We propose a novel framework combining a differentiable agent for fire dynamics and FEA simulation with an MFSP predictor to identify the most fire-sensitive point.
    \item {\bf{GNNs}}: We employ GNNs to model building structures and utilize Transfer Learning (TL) for efficient training. An Edge Update (EU) mechanism is introduced to account for the changes in the structural properties under fire scenarios.
    \item {\bf{Open-source resources}}: We provide the dataset and implementation details via a public GitHub repository: \url{https://github.com/STAIRlab/MFSP_Prediction}.
\end{itemize}

\revise{The remaining parts of this paper is organized as follows: \secref{sec:literature} reviews relevant literature on GNN applications, agents, and fire analysis of building structures. \secref{sec:system_overview} provides an overview of the proposed system architecture. Sections \ref{sec:mdrp}, \ref{sec:mfspp}, and \ref{sec:dataset} detail the MIDR predictor, the MSFP predictor, and the dataset generation, respectively. Implementation details and evaluation results are presented in \secref{sec:evaluation}. We discuss the limitations and future work in \secref{sec:discussions}. Finally, \secref{sec:conclusion_future} is a concise summary of the main conclusions of the study.}
Through this research, we aim to enhance fire risk assessment accuracy and efficiency, providing a robust tool to support the design of safer, more resilient building structures.
