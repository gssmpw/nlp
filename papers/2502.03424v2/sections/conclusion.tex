\section{Conclusions}
\label{sec:conclusion_future}

This study introduces a neural network-based framework for predicting the Most Fire-Sensitive Point (MFSP) in building structures, advancing fire safety assessment through a novel integration of data-driven and simulation-based approaches. The proposed framework leverages a differentiable agent model to efficiently simulate Finite Element Analysis (FEA) and predict the Maximum Interstory Drift Ratio (MIDR) under fire conditions. This agent enables the rapid generation of MFSP-labeled data, directly supporting the training of an MFSP predictor. A rule-based temperature generation method was developed to facilitate the large-scale structural and fire simulation datasets, forming a robust pipeline for data generation and neural network training. Key innovations of this work include: 
\begin{enumerate}
    \item The application of Graph Neural Networks (GNNs) to process structural data \revise{in fire scenarios}.
    \item The integration of Transfer Learning (TL) to improve prediction performance.
    \item The introduction of an Edge Update (EU) mechanism to account for property changes in beams and columns during fire exposure. 
\end{enumerate}
Framework evaluations demonstrated its effectiveness: 
\begin{itemize}
\item The MIDR predictor achieved a maximum average Spearman's rank correlation coefficient of $0.74$, with an average correlation exceeding $0.91$ for severe fire scenarios.
\item The MFSP predictor, at the room level, attained Top-5 and Top-10 accuracies of up to $74.1\%$ and $83.2\%$, respectively.
\end{itemize}
The results highlight the proposed framework potential to enhance building fire safety assessment by efficiently identifying critical structural vulnerabilities under fire conditions.
