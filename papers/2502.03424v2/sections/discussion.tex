{\blockRevise
\section{Limitations \& Future Work}
\label{sec:discussions}
\subsection{Performance Improvement} 
The proposed framework demonstrates promising results, yet there are opportunities to enhance its predictive capabilities. From a modeling perspective, the constraint of depth of GNN layers to match the number of building stories effectively mitigates over-smoothing but may limit the ability to model horizontal fire spread modeling in low-rise, large-area buildings. Future work could explore alternative architectures, such as introducing dummy nodes or improving message-passing mechanisms, to enhance cross-node communication in such scenarios. Additionally, the current implementation employs shallow MLPs for key components of the message-passing network, which may restrict the model's ability to capture complex interaction patterns. Exploring more advanced GNN variants, such as GraphSAGE \cite{hamilton2017inductive}, Graph Attention Networks \cite{velickovic2018graph}, could improve prediction accuracy by better handling multi-scale spatial dependencies. A deeper analysis of the trade-off between MSE loss and MIDR loss could also further refine the model's performance.

Beyond a purely data-driven approach, integrating Physics-Informed Neural Networks (PINNs) could leverage physical laws to refine predictions. While PINNs have shown great promise in modeling physical systems, their effectiveness depends on well-formulated Partial Differential Equations (PDEs) --a challenge in multi-physics problems like fire scenarios. In our case, the problem involves not only generating spatial thermal loads but also solving for the building's structural response under these loads. These complexities complicate the PDE formulation, limiting the immediate application of PINNs. However, as hybrid methods and improved PDE modeling techniques evolve, incorporating PINNs into fire modeling remains a valuable research direction. Hybrid approaches that combine data-driven and physics-based methods could provide a more balanced and robust solution. 

Investigating the theoretical connections between GNN performance and FEA tools could strengthen the foundation for applying GNNs in structural analysis. The proposed framework is also compatible with other advanced ML techniques. For instance, ensemble learning \cite{alam_dynamic_2020} could integrate multimodal data, such as images or OpenSeesRT scripts of the building structure, and self-supervised learning --proven effective for sequential data like electroencephalograms \cite{rafiei_selfsupervised_2024a, rafiei_selfsupervised_2024}-- could offer insights into temporal thermal analysis. Furthermore, general techniques such as neural architecture search and hyperparameter optimization could  be incorporated to enhance both accuracy and practicality.

\subsection{Dataset Generation}
The lack of comprehensive datasets is a significant challenge in applying ML methods to fire-related research. Our work introduced a simplified rule-based approach for efficient fire dataset generation, but it may not fully capture the complexity of real-world structural geometries or thermal field fields. Incorporating experimental measurements or advanced tools such as  the FDS could improve fidelity. 

Balancing fidelity, efficiency, and cost remains a key consideration. The proposed rule-based method could be refined to accommodate different structural configurations or fire patterns. For example, beyond tuning the parameters as discussed in \secref{subsec:thermal_load_generation}, the ISO 834 curve could be replaced with a parametric time-temperature fire curve that accounts for additional factors such as ventilation conditions and fuel load density. 

While physics-based simulations like OpenSeesRT facilitates large-scale training data generation, their accuracy depends on assumptions about fire load evolution and structural responses. Rigorous experimental validation and cross-comparison studies are essential to verify simulation fidelity, particularly for complex events like fires. Future research should prioritize such validation efforts to enhance confidence in simulation outputs and downstream predictive models. Additionally, addressing data scarcity through techniques such as TL, domain adaptation, or synthetic data augmentation could further improve model robustness. Leveraging pretrained models from this study as a foundation could enable adaptation to diverse structural configurations and fire scenarios.

\subsection{Beyond Structural Analysis}
The framework's differentiable agent and modular design offer transformative potential beyond fire scenario predictions, unlocking new applications across interdisciplinary domains. A key advantage of the differentiable agent is its ability to support training auxiliary predictors (e.g., ``argmaxers'') while also providing direct access to gradients that quantify how structural elements or environmental factors influence system-level responses. A natural extension of this work is structural analysis with combined loads rather gravity and fire scenarios only --for instance, assessing the impact of both earthquakes and the subsequent fire-induced structural degradation. Beyond structural analysis, the framework's ability to identify worst-case scenarios can be generalized to other domains. For example, training a differentiable agent to simulate molecular dispersion (e.g., airborne viruses or chemical leaks) could identify the ``Most Diffusion-Sensitive Point'' (MDSP) --the location where contaminants spread most rapidly. Such insight could inform critical design decisions, such as ensuring chemical laboratories or quarantine rooms are positioned away from MDSPs, or optimizing ventilation systems to minimize cross-contamination risks.

Moreover, the agent’s gradient information can directly map element-level properties (e.g., material stiffness, airflow resistance) to global performance metrics (e.g., structural drift, contaminant concentration). This enables physics-aware sensitivity analysis, where engineers can prioritize elements with the highest gradient magnitudes --those that have the greatest impact on system safety-- when making decisions about retrofitting, material selection, or structural requirements.

By addressing these limitations and exploring the outlined directions, future work can refine the framework’s accuracy, expand its scope, and unlock new possibilities for data-driven structural engineering.

} 