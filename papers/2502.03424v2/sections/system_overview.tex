\section{System Overview}
\label{sec:system_overview}
The primary goal of this work is to accurately predict the MFSP in building structures subjected to various fire scenarios \revise{by utilizing the MIDR as a metric for the overall lateral stability.} To achieve this, we propose an integrated framework that combines GNNs and FEA. The system architecture is illustrated in \figref{fig:system_overview} and comprises two key components: the MIDR predictor and the MFSP predictor. \revise{There are two stages corresponding to the MIDR predictor's different modes. In stage 1, we use data with ground truth MIDR values obtained from FEA simulations to train the MIDR predictor. Then, in stage 2, with the well-trained MIDR predictor working as a differentiable agent of fire simulation, we train the MFSP predictor to determine the point that maximize the MIDR.}
This framework seamlessly integrates physics-based simulations, GNN-driven predictive modeling, and data-driven techniques, enabling efficient and accurate MFSP prediction to provide a powerful tool for proactive fire safety analysis and risk mitigation in building structures.
\begin{figure*} [h!]
    \centering
    \includegraphics[width=0.8\textwidth]{figures/system_overview.pdf}
    \caption{\revise{Proposed framework for predicting the MFSP in building structures. Trapezoids  and gray rectangles represent the NN predictor and processing, respectively. Dashed and solid lines indicate that MIDR predictor is in the respective training mode and evaluation mode with parameters fixed, acting as a differentiable agent.}}
    \label{fig:system_overview}
\end{figure*}

{\blockRevise
\subsection{Structural Stability Metrics} 
The Interstory Drift Ratio (IDR) of each node serves as a critical parameter for evaluating structural lateral stability and deformation under external forces. IDR quantifies the relative displacement between two consecutive floors (interstory displacement) as a percentage of the floor height. Mathematically, the IDR for a given node $i$ is defined as follows:
\begin{equation}
    d_i = \left.\sqrt{\left(\Delta x_i\right)^2 + \left(\Delta y_i\right)^2}\right/ H \times 100 \%,
\end{equation}
where the numerator represents the relative displacement (in the horizontal plane $xy$) of node $i$ with respect to the corresponding node on the floor below, and $H$ denotes the story height between these two floors. Excessively high drift ratios can indicate significant structural deformation, potentially leading to significant damage or collapse due to lateral instability. The Maximum IDR among all the nodes of a structure, i.e., MIDR, is chosen to be a representative example metric for assessing the overall structural stability performance during fire events. Although MIDR may not capture all possible failure modes, such as local collapses due to midspan softening, local buckling, or loss of vertical elements, we emphasize that the proposed method is not limited to MIDR. The framework can be easily adapted to other performance indicators of interest, by simply replacing the MIDR with the desired metric.

\textit{Remark:} Selecting MIDR as the primary metric for assessing the structural integrity under fire conditions is motivated by its effectiveness in quantifying global deformation patterns. In fire-induced scenarios, thermal expansion, stiffness degradation, and gravity-induced deformations contribute to structural instability, which can be captured through relative floor displacements. MIDR provides a direct and interpretable measure of structural vulnerability by identifying floors experiencing excessive lateral deformations that may lead to global instability or loss of vertical load-carrying capacity. While MIDR is commonly used for seismic and wind-induced responses, its application to fire scenarios is justified as a fire-driven thermal effect also leads to large-scale deformation patterns, particularly in multi-story steel structures. Importantly, MIDR serves as an effective proxy for overall building stability in computational frameworks where parameterizing every potential failure mode (e.g., local buckling, connection failure, progressive collapse) is infeasible. However, fire-induced failure mechanisms extend beyond interstory drift, and localized effects are not explicitly captured by MIDR. These mechanisms typically develop locally and may not always translate into immediate global structural instability. While our current framework focuses on identifying the MFSP based on a worst-case drift metric, future work could integrate alternative failure criteria to further refine the fire vulnerability predictions.

Note that ``M'' in MIDR and MFSP represents different concepts. In the case of MIDR, for a given structure and fire source point, the IDR is computed at each node, and the MIDR is defined as the maximum IDR among all nodes. In contrast, MFSP refers to the fire source location that results in the highest MIDR across all possible fire source points within the structure. In summary, an MIDR is associated with a specific structure and fire source point pair, whereas an MFSP characterizes an entire structure by identifying the most critical fire source location.

}

\subsection{MIDR \& MFSP Predictors}

The MIDR predictor is a GNN-based model designed to estimate the MIDR of a building under a given fire scenario. The inputs to this model include:
\begin{itemize}
    \item {\bf{Structural configuration}}: Building geometry, material property, and gravity loads. 
    \item {\bf{Fire location}}: The specific point where the fire is initiated within the building.
\end{itemize}
A GNN processes this input to represent the structural configuration and fire location as a graph. The MIDR predictor is trained on labeled data generated using OpenSeesRT, a robust open-source FEA framework \cite{perez2024openseesrt}. These labels represent detailed structural responses under various fire conditions. Once trained, the MIDR predictor functions as a {\em{differentiable agent}}, offering computationally efficient MIDR estimates. Its capabilities include:
\begin{itemize}
    \item {\bf{Annotating datasets}}: Assigning  MIDR values to support subsequent analyses.
    \item {\bf{Integrating with NNs}}: Reducing the computational cost typically associated with simulation-based methods.
\end{itemize}

% \subsection{MFSP Predictor}
The MFSP predictor acts as an ``argmaxer module'' for the MIDR predictor, identifing the fire location that results in the highest MIDR. This location corresponds to the point of the greatest structural vulnerability. By leveraging the structural graph as input and utilizing the MIDR predictor's outputs, the MFSP predictor efficiently pinpoints the critical fire location.

\subsection{Data Generation and Training Pipeline}
To ensure robustness and generalizability, we introduce a comprehensive data generator pipeline:
\begin{enumerate}
    \item {\bf{Structure data generator}}: This component creates synthetic datasets for diverse building configurations, including geometry, material, and gravity loads.
    \item {\bf{FEA simulations}}: With the high-fidelity FEA simulation software, OpenSeesRT, the gravity simulation is first conducted to confirm the rationality of the synthetic dataset. Further, a subset of the generated configurations undergoes fire scenario simulations using OpenSeesRT based on a rule-based thermal load generation method. These simulations produce \textbf{labeled data} detailing  the structural responses to various fire locations, forming training and testing sets for the MIDR predictor.
    \item {\bf{Unlabeled data utilization}}: The remaining configurations, without MIDR labels from the FEA simulation, are also used to train and test the MFSP predictor, leveraging the MIDR predictor as a computationally efficient, yet accurate, {\em{surrogate}} model. Although the structural configurations with unlabeled data do not undergo FEA simulations, they can be rapidly and efficiently  \emph{pseudo labeled} using this surrogate model.
\end{enumerate}

