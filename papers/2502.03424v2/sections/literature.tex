\section{Literature Review}
\label{sec:literature}
Understanding the interaction between fire dynamics and building structures is fundamental to designing safer, more resilient systems. Resilience, in this context, encompasses robustness, adaptability, recoverability, and redundancy, ensuring systems can withstand, adapt to, and recover from adverse events. Recent advances in FEA, ML, and physics-informed simulations have provided powerful tools for analyzing and predicting structural performance under fire conditions. This section reviews key contributions in these domains, identifies gaps, and motivates the integration of differentiable agents with thermal simulators for enhanced fire analysis in buildings.

{\blockRevise
\subsection{GNN Applications and Agents}
Recently, GNN has shown successful applications in many areas in civil engineering. For example, \cite{huang_dynamic_2024} employed GNN for predicting network-wide metro passenger flow with dynamic propagation graphs, and \cite{sheng_egoplanningguided_2024} introduced multi-graph convolutional network designed to predict the future trajectories of traffic agents near an autonomous vehicle by incorporating various interaction perspectives and EGO-planning information, where EGO stands for {\bf{E}}uclidean Signed Distance Field-free {\bf{G}}radient-based l{\bf{O}}cal. Moreover, \cite{gao_urban_2024} utilizes physics-informed GNN-assisted auto-encoder to reconstruct high-resolution urban wind fields based on sparse sensor data. \cite{jia_graph_2023a} reviews the application of GNNs in construction. All these successful applications demonstrate the strong ability of GNN to capture spatial relationships and also demonstrate the potential for applying GNN in structural analysis. In predicting structural responses, \cite{chou_structgnn_2024} applied GNN to a static problem to predict responses such as nodal displacements and shear forces, showing very high accuracy. GNN was further combined with RNN in \cite{jia_temporal_2024} to predict building thermal load in general scenarios, focusing on capturing spatial interactions between multiple thermal zones and temporal dependencies. 

Despite these successful applications, using GNN to predict the structural response in fire scenarios has not been well studied. \cite{wang_graph_2024} proposed a GNN-RNN hybrid model for real-time displacement prediction of steel frames under fire, yet the work focused on post-fire collapse warning systems by forecasting future outcomes through early-stage fire features, rather than evaluating the stability at the design phase. Meanwhile, \cite{chen_tempnet_2024} developed a GNN-based model for post-fire concrete temperature field mapping, but their approach remained limited to material-level assessment and did not address the response at the structural level. Moreover, existing works merely use GNN for inference, i.e., prediction, neglecting that it could also work as an \emph{agent} for other functions. The concept of agent has been widely used in civil engineering scenarios. For example \cite{soto_multiagent_2017} proposes a multi-agent replicator controller for vibration control of smart structures, and \cite{yao_multiagent_2024} introduced a multi-agent Reinforcement Learning (RL) model for optimizing maintenance decisions in interdependent highway pavement networks. However, these works mainly treat agents as the final objective, i.e., the well-trained agents are directly used to generate the desired decisions. We distinguish the presented study from these works by the role of the agent in the framework, where herein the agent works as a {\em{surrogate model}} and is an {\em{intermediate module}} in the proposed framework. The agent's differentiability and high efficiency are fully utilized in the presented work.

\subsection{Thermal Simulation}
Fire event simulation encompasses several critical aspects, including heat transfer modeling, fire spread dynamics, and the impact of thermal loads on structural elements.
The interplay between thermal loads and structural elements during fire events has been extensively studied. Early work in \cite{anderberg1988modelling} introduced a method for simulating heat transfer in steel structures. This approach was later enhanced to incorporate transient thermal responses and load redistribution effects \cite{kodur2010response}, which remain vital for predicting collapse mechanisms under high-temperature conditions \cite{franssen2002fire}.

Beyond heat transfer analysis, fire simulation also involves assessing the deterioration of structural resistance. This includes temperature-induced material degradation, progressive weakening of load-bearing components, and failure mechanisms under prolonged fire exposure. Notably, steel structures experience reductions in Young’s modulus and yield strength at elevated temperatures, which affect their ability to sustain applied loads. Thermal expansion and differential heating further contribute to internal stress redistribution, potentially leading to local or global instability. Although phenomena such as buckling of slender members, loss of composite action in beam-column connections, and creep effects under sustained high temperatures are integral considerations in advanced fire analysis models, they are beyond the scope of the present paper and should be considered in future studies. 

Recently, \cite{wang_optimizationimproved_2024} developed a computational framework to enhance the simulation of welding residual stresses using ML and optimization techniques, validated through experimental measurements. \cite{wan_thermal_2024} studied the thermal contraction coordination behavior between an unbound aggregate layer and an asphalt mixture overlay using a finite difference and discrete element coupling method. Most past works focused on local behavior of system components under thermal loads instead of the overall structural stability.

Computational Fluid Dynamics (CFD)-based methods, such as Fire Dynamics Simulator (FDS) \cite{mcgrattan_fire_2000}, provide high-fidelity modeling of temperature evolution, flame spread, and gas-phase combustion. However, the computational cost remains a significant limitation, restricting large-scale structural assessments. FDS requires not only a large computing time for the simulation, but also significant modeling time for the building structures. This limits the use of FDS to case studies with necessary building structural details input, as well as information on combustible materials inside the building. For example, \cite{manea_fds_2022} employed FDS modeling to investigate two ignition scenarios in a 2014 Romanian restaurant fire, determining the most likely cause of rapid fire spread and lethal carbon monoxide exposure that resulted in four fatalities.

Real-world fire loads often display substantial spatial variability across different rooms in a building \cite{dundar_fire_2023}, resulting in scenario-specific temperature fields with limited generalizability. Furthermore, even within identical scenarios, variations in fire modeling methodologies can produce distinctly different temperature fields, as demonstrated by \cite{zhang_temperature_2020} using a full-scale experiment to develop a new temperature field prediction model for large space fires, and by \cite{du_new_2012} through creating a new temperature-time curve to better represent the non-uniform temperature distribution in large space fires compared to standard curves. Moreover, studies on bridge fires \cite{he_study_2024} have demonstrated that environmental factors, such as wind speeds, can significantly influence temperature distributions. These challenges emphasize the need for efficient and adaptive methods to generate fire temperature data, where less important factors might be neglected. 

While existing studies have extensively explored post-fire structural behavior, integrating material deterioration mechanisms into a ML-based predictive framework remains an open challenge. The proposed approach in this study aims to bridge this gap by efficiently approximating fire-induced structural responses without relying on computationally intensive physics-based simulations. To overcome the above difficulties, we propose a {\em{rule-based}} temperature generation method, which is efficient and scalable for serving large-scale dataset simulations. With a generated temperature field in a structure, FEA tools e.g., \cite{yu_objectoriented_1993}, can be used to obtain the structural response under thermal loads. For steel structures, previous studies \cite{nan_structuralfire_2023} have demonstrated that they rapidly equilibrate with surrounding gas temperatures due to efficient heat exchange. Consequently, gas temperatures can be directly used as input for FEA. This inspired us to generate a large dataset using FEA for thermal analysis in fire scenarios to form the foundation of NN training and testing. 

%%%%%%} % \color{red}

%Fire event simulation encompasses several critical aspects, including heat transfer modeling, fire spread dynamics, and the impact of thermal loads on structural elements. Computational Fluid Dynamics (CFD)-based methods, such as Fire Dynamics Simulator (FDS), provide high-fidelity modeling of temperature evolution, flame spread, and gas-phase combustion. However, the computational cost remains a significant limitation, restricting large-scale structural assessments.

%Beyond heat transfer analysis, fire simulation also involves assessing the deterioration of structural resistance. This includes temperature-induced material degradation, progressive weakening of load-bearing components, and failure mechanisms under prolonged fire exposure. Notably, steel structures experience reductions in Young’s modulus and yield strength at elevated temperatures, which affect their ability to sustain applied loads. Thermal expansion and differential heating further contribute to internal stress redistribution, potentially leading to local or global instability. Additionally, phenomena such as buckling of slender members, loss of composite action in beam-column connections, and creep effects under sustained high temperatures are integral considerations in advanced fire analysis models.

%While existing studies have extensively explored post-fire structural behavior, integrating these deterioration mechanisms into ML-based predictive frameworks remains an open challenge. The proposed approach in this study aims to bridge this gap by efficiently approximating fire-induced structural responses without relying on computationally intensive physics-based simulations.

%One major difficulty in performing analysis on overall structural stability is to obtain the temperature field in the structure level. FDS \cite{mcgrattan_fire_2000} gave a CFD-based solution to provide precise temperature predictions. However, the CFD-based method inherently incurs high computational cost as well as time consumption. It requires not only a large amount of time for general-purpose computers to run the simulation, but also significant amount of time for modeling the building structures. This makes FDS mostly used for case studies with given detailed structures, as well as information on combustible materials inside buildings. For example,\cite{lai_fire_2024} employed FDS to validate fire accident report as a case study in Chiayi City. FDS's prohibitively high cost limits its scalability and makes it unsuitable for large-scale dataset generation and analysis.

%Furthermore, real-world fire loads often display substantial spatial variability across different rooms \cite{dundar_fire_2023}, resulting in scenario-specific temperature fields with limited generalizability. For example, studies on bridge fires \cite{he_study_2024} have demonstrated that environmental factors, such as wind speeds, can significantly influence temperature distributions. Furthermore, even within identical scenarios, variations in fire modeling methodologies can produce distinctly different temperature fields, as demonstrated by \cite{zhang_temperature_2020}, who conducted a full-scale experiment and developed a new temperature field prediction model for large space fires, and \cite{du_new_2012}, who created a new temperature-time curve to better represent the non-uniform temperature distribution in large space fires compared to traditional standard curves. These challenges emphasize the need for efficient and adaptive methods to generate fire temperature data, where less important factors might be neglected.

%To overcome these difficulties, we propose a rule-based temperature generation method, which is efficient and scalable for serving large-scale dataset simulations. With temperature field in a structure, Finite Element Analysis (FEA) tools e.g.,\cite{yu_objectoriented_1993}, are used to obtain structural response under thermal loads. For steel structures of interest, previous studies \cite{nan_structuralfire_2023} have demonstrated that steel structures rapidly equilibrate with surrounding gas temperatures due to efficient heat exchange. Consequently, gas temperatures can be directly used as input for FEA tools. This inspires us to generate a large dataset using FEA tools for thermal analysis in fire scenarios, to form the foundation of neural network training and testing. To be specific, we will use the maximum inter-story drift ratio to measure the overall stability of building structures.

} %\blockRevise
