
%\documentclass[preprint,12pt]{elsarticle}

%% Use the option review to obtain double line spacing
%% \documentclass[authoryear,preprint,review,12pt]{elsarticle}

%% Use the options 1p,twocolumn; 3p; 3p,twocolumn; 5p; or 5p,twocolumn
%% for a journal layout:
%% \documentclass[final,1p,times]{elsarticle}
%% \documentclass[final,1p,times,twocolumn]{elsarticle}
%% \documentclass[final,3p,times]{elsarticle}
%% \documentclass[final,3p,times,twocolumn]{elsarticle}
%% \documentclass[final,5p,times]{elsarticle}
  \documentclass[final,5p,times,twocolumn]{elsarticle}

%% For including figures, graphicx.sty has been loaded in
%% elsarticle.cls. If you prefer to use the old commands
%% please give \usepackage{epsfig}

%% The amssymb package provides various useful mathematical symbols
\usepackage{amssymb}
\usepackage{url}
\usepackage{hyperref}
\usepackage{placeins}
\usepackage{graphicx}
\usepackage{tcolorbox}
\usepackage{subcaption}
\usepackage{amsmath}
\usepackage{algorithm}
\usepackage{algorithmic}
\usepackage{colortbl}
\usepackage{xcolor}
%\usepackage{multirow}


\UseRawInputEncoding

%% The amsthm package provides extended theorem environments
%% \usepackage{amsthm}

%% The lineno packages adds line numbers. Start line numbering with
%% \begin{linenumbers}, end it with \end{linenumbers}. Or switch it on
%% for the whole article with \linenumbers.
%% \usepackage{lineno}
\FloatBarrier
\journal{.}

\begin{document}

\begin{frontmatter}

%% Title, authors and addresses

%% use the tnoteref command within \title for footnotes;
%% use the tnotetext command for theassociated footnote;
%% use the fnref command within \author or \address for footnotes;
%% use the fntext command for theassociated footnote;
%% use the corref command within \author for corresponding author footnotes;
%% use the cortext command for theassociated footnote;
%% use the ead command for the email address,
%% and the form \ead[url] for the home page:
%% \title{Title\tnoteref{label1}}
%% \tnotetext[label1]{}
%% \author{Name\corref{cor1}\fnref{label2}}
%% \ead{email address}
%% \ead[url]{home page}
%% \fntext[label2]{}
%% \cortext[cor1]{}
%% \affiliation{organization={},
%%             addressline={},
%%             city={},
%%             postcode={},
%%             state={},
%%             country={}}
%% \fntext[label3]{}

\title{Global Ease of Living Index: a machine learning framework  for longitudinal analysis of major economies}

%% use optional labels to link authors explicitly to addresses:
%% \author[label1,label2]{}
%% \affiliation[label1]{organization={},
%%             addressline={},
%%             city={},
%%             postcode={},
%%             state={},
%%             country={}}
%%
%% \affiliation[label2]{organization={},
%%             addressline={},
%%             city={},
%%             postcode={},
%%             state={},
%%             country={}}


\author[inst1,inst2]{Tanay Panat}

\author[inst1]{Rohitash Chandra}

\affiliation[inst1]{organization={Transitional Artificial Intelligence Research Group, School of Mathematics and Statistics},%Department and Organization
            addressline={University of New South Wales}, 
            city={Sydney},
            postcode={2052}, 
            state={NSW},
            country={Australia}}

            
\affiliation[inst2]{organization={Centre for Artificial Intelligence and Innovation}, 
            addressline={Pingla Institute}, 
            city={Sydney}, 
            country={Australia}}
 

\begin{abstract}
%% Text of abstract
The drastic changes in the global economy, geopolitical conditions, and disruptions such as the COVID-19 pandemic have impacted the cost of living and quality of life. It is important to understand the long-term nature of the cost of living and quality of life in major economies. A transparent and comprehensive living index must include multiple dimensions of living conditions. 
In this study, we present an approach to quantifying the quality of life through the  Global Ease of Living Index that combines various socio-economic and infrastructural factors into a single composite score. Our index utilises economic indicators that define living standards, which could help in targeted interventions to improve specific areas. We present a machine learning framework for addressing the problem of missing data for some of the economic indicators for specific countries. We then curate and update the data and use a dimensionality reduction approach (principal component analysis) to create the Ease of Living Index for major economies since 1970. Our work significantly adds to the literature by offering a practical tool for policymakers to identify areas needing improvement, such as healthcare systems, employment opportunities, and public safety. Our approach with open data and code can be easily reproduced and applied to various contexts. This transparency and accessibility make our work a valuable resource for ongoing research and policy development in quality-of-life assessment.


\end{abstract}
  

\begin{keyword}
%% keywords here, in the form: keyword \sep keyword
 Quality of Life\sep Socio-Economic Factors\sep Composite Score\sep Living Conditions\sep Data transparency\sep Reproducibility 
\end{keyword}

\end{frontmatter}

%% \linenumbers

%% main text
\section{Introduction}
\label{sec:sample1}
The assessment of the quality of life has evolved significantly over the years, incorporating a wide range of socioeconomic and infrastructural factors \cite{multifac_index}. Quantitative indexes, such as the World Happiness Index, rank countries based on various quality of life factors\cite{helliwell2012world}. Measuring happiness and quality of life is complex, current methods for assessing happiness might not be directly comparable, thus questioning the validity of averaging happiness and quality of life \cite{Ng}. Existing indices are often highly biased towards economic factors such as Gross Domestic Product (GDP) per capita with a direct link to the overall quality of life \cite{Arbor}. 
The Human Development Index, World Happiness Index  \cite{HDI_dependency_GDP} and  Global Competitive Index  \cite{GCI} use the GDP per capita as a key indicator of national well-being, employment and economic prosperity. 

Recently, there has been a lot of criticism towards existing indexes for their biases towards subjective measures. Research has shown that variables such as GDP per capita can significantly and negatively impact the happiness index \cite{Impact_macro_factors}, and quality of life index \cite{Quality_dependence}, suggesting that economic metrics heavily influence reported happiness levels \cite{supa}. Factors such as leisure, inequality, mortality, morbidity, crime, and the natural environment play significant roles in determining living standards but are not adequately reflected in GDP measurements \cite{Jones}. In addition to these challenges, the multidimensional nature of well-being \cite{well_being} necessitates a more comprehensive approach. Previous studies have often isolated economic factors, neglecting the interplay between socio-economic conditions and infrastructural elements. This oversight limits the understanding of how diverse factors collectively influence living standards \cite{Zanon}.


Several major indexes for ranking countries in the domain of quality of life and ease of living exist, but they have major limitations. The Global Livability Index \cite{Livebility_Index} primarily focuses on stability, healthcare, culture, and environment, but it frequently lacks detailed data on social cohesion and community engagement, which are vital for understanding the quality of life in urban areas. Moreover, the Quality of Life Index \cite{Quality_life_index_gaps}, while encompassing various dimensions of urban living, often relies on aggregated data that may not reflect local realities. This is particularly evident in regions where data collection mechanisms are underdeveloped, leading to significant gaps in information on critical aspects such as public safety \cite{yu2014assessing}, environmental sustainability \cite{environmental_sustainability_gaps},  and economic inclusiveness. 

The World Happiness Report \cite{World_Happiness_Report} utilises data from surveys that ask people to respondents to evaluate their current lives on a scale, often referred to as the "happiness ladder". A notable flaw in this index is its reliance on self-reported data, which can be influenced by cultural biases and social desirability, leading to discrepancies between reported happiness and actual well-being \cite{Flaw_HI}. The Happiness Index developed by the Happiness Alliance employs a comprehensive survey instrument that assesses multiple dimensions of well-being, yet it does not explicitly address how each dimension is weighted in the overall index calculation\cite{HI_Issue}. This lack of clarity can lead to criticisms regarding the transparency and validity of the index. Another critical issue is the potential for bias in the data collection process. The Quality of Life Index like the happiness index is calculated using a combination of health, economic, and social indicators. It often includes metrics such as life expectancy, education levels, and income. Quality of Life \cite{QOL_Issues} assessments rely on self-reported data, which can introduce self-selection bias and affect the generalisability of the findings. There is also a predominant issue in the selection of dimensions and indicators that would truly represent a dynamic Quality of Life Index \cite{QOLI_EU_Issue}. The European Quality of Life Index, for instance, attempts to address this complexity by incorporating a wide range of indicators, but this can also lead to difficulties in ensuring that all relevant aspects of quality of life are adequately represented\cite{EU_QOLI}.



The Indian Ministry of Housing and Urban Affairs (MoHUA) launched the Ease of Living Index (EoLI) \cite{mohua2018ease} in 2018, which serves as a comprehensive tool to measure the quality of life in Indian cities. The EoLI is designed to assess the performance of 111 cities across institutional, social, economic, and physical pillars. These pillars are further divided into 15 categories, incorporating 78 indicators, of which 56 are core indicators and 22 are supporting. The institutional pillar evaluates governance, public services, and participatory governance. The social pillar focuses on key societal aspects such as education, health, identity, culture, and safety. The economic pillar assesses growth potential, employment opportunities, and inclusiveness of economic development, while the physical pillar evaluates city infrastructure, including housing, transportation, waste management, and environmental sustainability \cite{mohua2018ease}. The EoLI employs a robust model that can be adapted to various global settings.
Additionally, the application of the EoLI framework globally could facilitate the identification of best practices and lessons learned from different urban contexts. For example, countries facing rapid urbanisation \cite{zhang2016trends}, such as those in Southeast Asia, could benefit from the EoLI's structured approach to assessing infrastructure needs and social services \cite{Global_implementation_ELOI}. 

 



The Cost of Living Index \cite{abraham2003toward}, the Pollution Index \cite{Junger_Leon}, Healthcare Index \cite{kabir2022non}, Crime Index \cite{targonski2012ucr} and Carbon-dioxide Emission \cite{macknick2009energy} are examples of indices that suffer from incomplete datasets, particularly in developing nations where data collection mechanisms may be underdeveloped\cite{Missing_imp_index_values}. The absence of indicators related to social aspects, such as community engagement and cultural identity, further limits the effectiveness of these indices in capturing the full spectrum of urban living conditions \cite{Missing_imp_index_values}. Data imputation is a crucial technique for addressing the challenges posed by missing data \cite{chen2024deep}. Machine learning models such as random forests and neural networks have been prominent for data imputation \cite{RFR_2}, including time series and tabular datasets.  

Kulkarni and Chandra introduced Bayes-CATSI \cite{kulkarniChandra2024bayes}, a variational Bayesian approach tailored for medical time series data imputation that leveraged the probabilistic framework to project uncertainties in the data imputation. Furthermore, studies have explored the imputation of missing values in economic and financial time series data using principal component analysis (PCA) approaches. They assessed five different PCA-based methods to determine their effectiveness in handling missing data under various scenarios\cite{Econ_imputation}.
Data imputation via machine learning can be useful, particularly in index creation, where complete datasets are essential for accurate assessments. We will apply machine learning to imputing economic indicators in our study. Another major challenge of creating an index is determining which factors are more important than others for determining ease of living and quality of life. After the data is acquired and imputed as needed, we need to decide based on the literature and expert knowledge on the weighting of the different sectors such as economic, employment, crime and health indicators to create an index. These processes are crucial in decision-making frameworks, especially in multi-criteria decision-making (MCDM) \cite{omid}, where unequal weight assignment is often applied to reflect the varying importance of different criteria in real-world problems.  

Dimensional reduction methods such as Principal Component Analysis (PCA) and Factor Analysis can be utilised for index creation\cite{PCA_Index}.  Muratov \cite{Index_FA_Agri} employed factor analysis to evaluate the impact of innovations on farm productivity and construct an index reflecting the use of innovative practices in agricultural activities. Factor Analysis has also been deployed to create ecological factors such as sustainable development index for river basin in Ronda, Brazil \cite{FA_Ecological_Index} and integrate environmental, economic, social, and institutional performance indicators\cite{FA_Ecological_Index}. It has also been applied to develop a comprehensive evaluation index that captures various dimensions of educational performance \cite{Edu_FA}. This helps illustrate how factor analysis can help narrow the gap in educational attainment across different regions by providing a structured framework for evaluating and comparing educational institutions \cite{Edu_FA}. Similarly, PCA  has been used to create a Multidimensional Poverty Index (MPI) \cite{Poverty_Index_PCA} and analyze various poverty-related indicators, allowing researchers to derive weights that reflect the relative importance of each indicator in contributing to the overall poverty measurement. PCA has also been used to construct economic indexes,  such as the cryptocurrency index \cite{shah2021principal}, where PCA normalised the data to construct the cryptocurrency index by aggregating the selected principal components based on their explained variance, ensuring that more significant components have a larger influence on the final index. Wang et al. \cite{wang2022sgl}  presented a  modified PCA for healthcare index creation from high-dimensional sensor data that incorporates sparsity and group lasso techniques to enhance the interpretability of the principal components. 


In this study, we present a framework for constructing a global  Ease of Living Index (EoLI) utilising machine learning for data imputation and index weightage computation. We evaluate the various factors such as health and economic indicators weight assignments to compute the final EoLI from its underlying sub-indices. Using factor analysis to identify the most significant components contributing to each dimension, we compute the sub-indices such as economic, institutional, quality of life, and physical infrastructure.  We compare different methods for calculating and validating the index based on the sub-indices. The sub-indices are then assigned different weights to reflect their relative importance before aggregating them into the comprehensive EoLI.
In data imputation, we utilise a machine learning approach using Random Forests \cite{jaafar2023imputation}, which effectively handles missing values based on the patterns in the observed data. This technique has been selected due to its ability to maintain the variability and correlation structure of the dataset \cite{jaafar2023imputation}, ensuring robust imputation results.
The dataset used in this study spans the period from 1970 to 2022, covering a wide range of socioeconomic indicators for all available countries. However, our primary goal is the creation of the EoLI for countries such as the United States, China, India, and Australia, while also examining global trends of G20 countries. We aim to deliver a reliable and meaningful index that captures the multidimensional aspects of living standards across countries, offering insights into temporal and regional patterns.

% including the Weighted Sum Method \cite{}, Factor Analysis\cite{FA_tool} and PCA \cite{subx_PCA}

The rest of the paper is presented as follows. Section 2 reviews related indices such as the World Happiness Index, Quality of Life Index, and Gini Index, highlighting their methodologies and limitations. Section 3 presents the methodology covering data collection and imputation methods. Section 4 presents the results and analysis, comparing the index across different regions. Section 5 provides a discussion, and Section 6 provides directions for future research directions.


\section{Related Work}

\subsection{Machine Learning for Data Imputation }
 

Machine learning has emerged as a powerful tool for addressing the challenge of missing data imputation, particularly in time series datasets. Traditional imputation methods often struggle with the complexities and dependencies inherent in such data \cite{chen2024deep}.  In recent years, Generative Adversarial Networks (GANs) \cite{GAN} have gained prominence due to their ability to model complex data distributions. An implementation of GAN, the Generative Adversarial Imputation Network (GAIN) has been shown to outperform conventional methods by effectively capturing the underlying data structure and generating plausible imputations for missing values in various datasets, including clinical and environmental data\cite{Dong_GAIN,Yang_GAIN}. Recent developments in deep learning have introduced models such as Bidirectional Recurrent Imputation for Time Series (BRITS) \cite{BRITS}, which specifically address the temporal dependencies in time series data. BRITS employ a recurrent neural network (RNNs) \cite{yu2019review} architecture to handle multiple correlated missing values and adaptively update imputed values.

In addition to GANs, other machine learning models have also been explored for imputation tasks. For example, Random Forests have been combined with generative models \cite{GAN_RF} to enhance the robustness of imputations, particularly in scenarios with large gaps in data. 

%MICE: Azur, M. J., Stuart, E. A., Frangakis, C., & Leaf, P. J. (2011). Multiple imputation by chained equations: what is it and how does it work?. International journal of methods in psychiatric research, 20(1), 40-49.

%CATSI :Yin, K., Feng, L., & Cheung, W. K. (2020). Context-aware time series imputation for multi-analyte clinical data. Journal of Healthcare Informatics Research, 4, 411-426.

%See section  2.1 here and write a paragraph citing above : https://arxiv.org/pdf/2410.01847

The data can be multifaceted, representing complex interactions between economic, institutional, social, and environmental dimensions. EOLI exhibits intricate temporal dynamics due to the diverse nature of factors like governance, public health, infrastructure, and quality of life indicators, each influenced by various external and internal processes over time. Due to such nature of our data, we require more sophisticated approaches to accurately capture and model these complexities, ensuring that the evolving trends in living standards are well-represented and analysed. Multiple Imputation by Chained Equations (MICE) is a widely adopted method for handling missing data in statistical analyses\cite{MICE_imputation}. This method allows us to account for the uncertainty associated with missing data by generating several plausible imputed datasets, which can then be analysed separately and combined to produce estimates that reflect this uncertainty.  MICE works under the assumption that data is missing completely at random \cite{little1988test}, this assumption may not be entirely appropriate for EoLI as data can be missing due to underdeveloped data collection mechanisms in developing countries, due to variables related to governance such as crime or public safety may be systematically under-reported in regions where governments are less transparent and due to inconsistencies in data reporting due to several factors such as wars etc.    

%xxxx

 
A recent study proposed a variational Bayesian deep learning framework for medical data imputation \cite{kulkarniChandra2024bayes}, allowing for a more nuanced understanding of the data's structure and dependencies using CATSI. This approach not only improves the accuracy of the imputations but also enhances the overall predictive performance of the models used for forecasting groundwater levels. Furthermore, the use of Bayesian inference such as MCMC (Markov Chain Mote Carlo) sampling has also been prominent in data imputation \cite{takahashi2017statistical}. Richardson et al \cite{richardson2020mcflow}  presented a novel framework designed for data imputation that leverages the capabilities of Monte Carlo sampling and normalizing flow generative models. They proposed an innovative iterative learning scheme that alternates between updating the density estimates of the data and imputing the missing values, thereby enhancing the robustness of the imputation process. Chen et al.  \cite{chen2024deep}  utilised MCMC sampling via a linear model for imputing data for groundwater modelling via deep learning models. 

Furthermore,  Denoising Autoencoder models \cite{Chen_Shi} have shown promise in learning latent representations of data, which has been leveraged to impute missing values effectively \cite{Lall_Robin}. Gondara and Wang \cite{Gondara_Wang} introduced MIDA, to leverage  DAEs to perform multiple imputations of missing data by reconstructing clean outputs from noisy inputs, which is particularly relevant for missing data scenarios where the absence of values can be viewed as a form of noise. Similarly, Pereira et al. \cite{Pereira_Santos} explored the potential of DAEs for missing data imputation, emphasising their capacity to learn from corrupted data. 
The flexibility of deep learning models allows them to adapt to different types of data and missing patterns, providing a significant advantage over traditional statistical methods.

\subsection{Machine learning for Index creation}
 

Machine learning has become an invaluable tool in developing and refining various indices that measure socio-economic factors \cite{keys2021machine}. Machine learning framework for  Doing Business Index \cite{DBI}  involves clustering countries and applying a multivariate Composite I-distance Indicator (CIDI) to derive data-driven weights for the index components. Furthermore, the application of machine learning techniques extends to other indices that assess economic performance and quality of life. For instance, a study demonstrated how machine learning can be utilised to predict stock market movements and stock price indices by employing trend deterministic data preparation methods\cite{Stock_Index}. This predictive capability is crucial for investors and policymakers alike, as it enables them to make informed decisions based on anticipated market trends. Machine learning techniques such as PCA have been employed to distil multiple indicators into a single composite score, allowing for a more nuanced understanding of different factors \cite{Reddy_reduction}.  PCA not only aids in reducing dimensionality but also enhances the interpretability of the data by highlighting the most significant variables \cite{Hiles_Dutcher}. The adaptability of PCA across various data types and its robustness against noise make it a preferred choice across fields, including social sciences\cite{Belay_Wondimu} and health studies\cite{Nowroozi_Roshani}. 
 

 Tiwari et al.  \cite{Abtahi} presented the COVID-19 Vulnerability Index (C19VI), which integrated diverse data from public health databases, including demographic statistics, and socioeconomic indicators. The study used machine learning models, specifically Random Forest and Support Vector Machines (SVM), for the identification of regions vulnerable to COVID-19. Furthermore,   Kheirati and Khan \cite{Khan_Ali} developed a Pavement Condition Index (PCI) using data from pavement condition surveys, incorporating visual inspections and surface distress measurements. The study applied machine learning models such as Decision Trees and Neural Networks, employing feature selection to enhance model accuracy and support effective maintenance planning. Keys et al. \cite{Isa} estimated the Human Footprint Index using ensemble learning techniques such as Gradient Boosting and Random Forests. The study integrated datasets on land use, population density, and ecological factors to capture the intricate relationships between human activities and environmental impacts, validated through spatial analysis techniques. 

%In general, these studies demonstrate that machine learning can advance public health, infrastructure maintenance, and environmental sustainability by offering precise, data-driven insights.  


\subsection{Background on Related Indexes}

\subsubsection{Quality of Life Index}

%Blomquist in his study stated that life is good when the quality of life is high \cite{QOLI_1}. 

A number of researchers rank quality of life based on compensating differentials in labour, housing, and consumption markets. The Index of Urban Quality of Life is an example that measures the value of local amenities that vary from one urban area to another. These amenities are attractive features of locations, such as sunny, smog-free days, safety from violent crime, and well-staffed, effective schools. This index assesses the monetary value of the set of amenities that households gain by living and working in a particular area \cite{rosen1979wage}. In this respect, observation can be drawn about consumers' locational choices based on their evaluation of an area's amenities, which ultimately determine their overall well-being. Similarly, the Quality of Life Index (QoLI) %\cite{}   
serves as a pivotal instrument in assessing the multidimensional aspects of well-being across various populations, particularly in the context of health-related quality-of-life assessments. QoLI provides a structured framework for evaluating how different aspects of life, beyond economic indicators, contribute to the overall quality of living in a given region. Building upon this concept, Numbeo \footnote{Numbeo website: \url{numbeo.com}} has developed its Quality of Life Index \cite{numbeo_qol}, which incorporates several critical factors. These include the "cost of living" and "purchasing power," which are essential for evaluating economic well-being, as they indicate the affordability of goods and services within a region. Numbeo includes the affordability in housing, pollution, crime rate, health system, and traffic in their index. 
 
\subsubsection{World Happiness Index}

The World Happiness Index (WHI) \cite{carlsen2018happiness} evaluates individuals' subjective well-being across different countries, reflecting their overall happiness and life satisfaction. This index has gained prominence as a tool for understanding the factors that contribute to happiness on a global scale, transcending traditional economic indicators such as Gross Domestic Product (GDP). The WHI is based on survey data that captures individuals' perceptions of their lives, encompassing various dimensions such as emotional well-being, social support, and economic stability. The WHI employs five key indicators including life ladders, social support, freedom to make life choices, generosity, and perceptions of corruption to create a comprehensive picture of happiness across different countries.  
In addition to these primary indicators, the WHI also considers macroeconomic factors such as GDP per capita, which can provide context for understanding happiness levels. Over the years, there has been a significant backlash in the media about the biases in data collection for this index, which seem to penalise developing countries based on their GDP \cite{GDP_Bias} and cultural measurement bias \cite{Cultural_bias}. For instance, a war-hit country such as Ukraine was ranked to be happier than India in 2023 \cite{newindianexpress2023happiness}. Such indexes are based on surveys that may not be properly sampled in large countries such as India and influenced by political narratives. In the study conducted by Bond and Lang, they critically examine the validity and reliability of happiness scales, arguing that the current methodologies used in happiness research may not accurately capture the constructs they intend to measure\cite{Bond_Lang}. They highlight that happiness scales often lack a solid theoretical foundation and can yield results that are reversible, meaning that the conclusions drawn from these scales can vary significantly depending on the context or the specific outcomes being measured.

\subsubsection{Ease of Living Index: India}


As mentioned earlier \cite{mohua2018ease}. the index uses a mix of data from government sources and citizen perception surveys, gathering over 60,000 responses to gauge satisfaction with urban services. The index employs a weight of 45\% to the physical pillar, 25\% each to the institutional and social pillars, and 5\% to the economic pillar and the core indicators are assigned a higher weight (70\%) than supporting indicators (30\%). The ranking system follows a dimensional index methodology, where cities are evaluated relative to national or international benchmarks where available, or the best-performing city is used as a benchmark.

%discuss these studies

%Kumar, S. (2023). Review of Performance Indicators of Smart Cities in India–Ease of Living Index: a case of Lucknow Smart City. Sustainability, Agri, Food and Environmental Research, 11.

%Jha, S., & Kumar, M. (2020). Augmenting ease of living in India: A critique of union budget 2020-21. Journal of Politics & Governance, 8(2).

%Puri, P. (22). Ease of Living Index and the Master Plan of Delhi 2041. Horizon J. Hum. Soc. Sci. Res, 4(1), 37-50.

Several studies have highlighted the critical role of the Ease of Living Index (EoLI) in shaping urban development and governance in India. Kumar et al.  \cite{Kumar} focused on Lucknow Smart City, demonstrating how EoLI is used to evaluate the impact of smart city initiatives, emphasising technology and citizen engagement as key factors for success. The study highlights the integration of ICT in urban management\cite{Renukappa} and stresses the importance of inclusive growth to address socio-economic inequalities. Similarly, Jha and Kumar (2020) critique the Union Budget 2020-21, calling for a strategic approach to fiscal policies that prioritise health, education, and social welfare to improve the ease of living \cite{jha2020augmenting}. They advocate for budget alignment with smart city goals, stressing investments in infrastructure and gender-responsive budgeting to ensure equity and inclusivity \cite{Tandon}.

Puri (2022) examines the EoLI’s relevance to the Master Plan of Delhi 2041, arguing for its integration to guide sustainable urban planning \cite{Puri}. The study identifies gaps in existing policies and stresses the importance of aligning urban development strategies with EoLI metrics to address inclusivity and accessibility across various dimensions such as transportation, health, education, and environmental sustainability \cite{Puri}. Puri emphasises a participatory approach, incorporating citizen feedback and continuous monitoring to ensure data-driven governance. Collectively, these studies underline the EoLI’s significance in informing urban policies, advocating for a holistic approach that combines technology, strategic budgeting, and citizen engagement to enhance the quality of life in Indian cities.

\subsubsection{Gini Index}

In economics, the Gini coefficient, also referred to as the Gini index, is a widely used metric for assessing statistical dispersion \cite{lerman1984note}. It is used primarily in the context of income, wealth, or consumption inequality within a population or social group. This measure was introduced by the Italian statistician and sociologist Corrado Gini \cite{Ceriani2012} to quantify inequality by analysing the distribution of variables, such as income levels. A Gini coefficient of 0 indicates complete equality, where all individuals have identical income or wealth, whereas a coefficient of 1 (or 100\%) signifies extreme inequality, where one individual possesses all the income or wealth, and the rest have none. This metric is instrumental in understanding the degree of inequality within a given group or nation\cite{WorldBank_Gini}. The Gini coefficient is calculated using the Lorenz curve \cite{gastwirth1972estimation}, which represents the cumulative share of income or wealth of a population.  

\subsubsection{Human Development Index}

The Human Development Index (HDI) is designed to measure the average achievement in three core dimensions of human development: longevity and health, education, and standard of living. The HDI is calculated as the geometric mean of normalised indices for each of these dimensions. Specifically, the health component is evaluated by life expectancy at birth, the education component is assessed through the mean years of schooling for adults (25 years and older) and the expected years of schooling for children entering school. The standard of living is measured using gross national income (GNI) per capita, where a logarithmic scale is applied to account for the diminishing returns of higher income levels. The three dimensions are combined into a single composite index through the geometric mean. The HDI can be a valuable tool in evaluating national policies by highlighting how countries with similar GNI per capita can achieve varying levels of human development \cite{ivanova1999assessment}. This discrepancy can help spark discussions on the effectiveness of government priorities and policies. However, the HDI is limited in scope and does not encompass the full range of human development aspects, and other forms of measures are being explored \cite{comim2016beyond}. It does not account for factors like inequality, poverty, human security, or empowerment.




\section{Methodology}
\label{sec:sample2}

\subsection{Data Collection}




In order to construct the Global Ease of Living Index, we collected data from multiple but credible sources, including the International Monetary Fund (IMF), World Health Organisation (WHO),  and World Bank \cite{IMF2023,WHO2023,WorldBank2023}, and others as described in the Appendix. These data sources provided diverse socio-economic, healthcare, institutional, and quality-of-life indicators required to assess living standards across countries. Our study focuses on collected data from 1970 to 2021 to develop the Ease of Living Index and provide an assessment during this period. %give some more information about motivation about the data sources and vision 







\subsection{Data Imputation }

%what is mice 

%what is RF 



We encountered several challenges related to missing data in the data extraction process.  In the case of the cost of living index, the historical data was missing from 2012 to 1970, and the unemployment rate was missing from 1970 to 1990. The data for the gender development index was missing from 1970-1990, and we observed a pattern of \textit{missingness} for multiple indices as shown in Table \ref{tab:missingdata}. Furthermore, imputing missing data using mean or mode values could misrepresent the underlying data patterns, and simply eliminating the rows with missing values would have resulted in a severe loss of information \cite{data_imputation_1}. 

Time series data imputation is important because missing values are common in datasets from a variety of industries, including banking, healthcare, and environmental monitoring. Among the several major methods available for addressing missing data, we selected the Random Forest Regressor (RFR) and Multiple Imputation by Chained Equations (MICE) since they have demonstrated effectiveness in the literature and also provide flexibility in implementation for our problem. 

\begin{table}[ht]
\centering
\small
\caption{Percentage of Missing Values by Sub-Index}
\begin{tabular}{|l|l|c|}
\label{tab:missingdata}
%\hline
\textbf{Sub-Index} & \textbf{Attributes} & \textbf{\% Missing} \\
\hline
 Economic Index & GDP Growth Rate & 2.82\% \\
                                 & Inflation Rate  & 21.83\% \\
                                 & GDP Per Capita & 4.80\% \\
                                 & Unemployment Rate & 41.29\% \\
                                 & Cost of Living Index & 87.58\% \\
                                 & Local Purchasing Power Index & 87.58\% \\
\hline
 Institutional Index & Control of Corruption & 4.01\% \\
                                     & Government Effectiveness & 4.48\% \\
                                     & Political Stability & 3.31\% \\
                                     & Regulatory Quality & 4.44\% \\
                                     & Rule of Law & 2.26\% \\
                                     & Voice and Accountability & 3.13\% \\
\hline
 Quality of Life Index & Life Expectancy at Birth & 6.83\% \\
                                       & Doctors per 10,000 & 67.91\% \\
                                       & Access to Electricity & 47.84\% \\
                                       & CO$_2$   per Capita   & 57.84\% \\
                                       & Gender Development Index & 66.47\% \\
                                       & Gender Inequality Index & 66.50\% \\
                                       & Human Development Index & 61.83\% \\
                                       & Health Care Index & 92.44\% \\
                                       & Crime Index & 88.78\% \\
\hline
Sustainability Index & CO$_2$ Emissions & 44.21\% \\
                                      & Non-renewable Electricity  & 45.30\% \\
                                      & Renewable Electricity   & 45.30\% \\
                                      & Micro Air Pollution & 55.78\% \\
                                      & Greenhouse Emissions & 43.07\% \\
\hline
\end{tabular}
\end{table}


Table \ref{tab:missingdata} shows the percentage of missing values for each variable within the different sub-indices in our dataset. The identified missing values in each column underscore the need for robust imputation techniques to ensure data completeness. Specifically, we use the Random Forest Regressor \cite{rf_3}, which accounts for the relationships between variables when filling in missing values. Additionally, we use  MICE data imputation \cite{RFR}, ensuring more reliable results for subsequent analysis.

\subsubsection{Random Forest Regressor}

A Random Forest \cite{segal2004machine} is an ensemble learning method used for classification and regression problems. Random Forests employ an ensemble of decision trees to decide by voting or aggregation. The Random Forest Regressor (RFR) \cite{rf_3} is particularly effective in capturing complex relationships within the data, as it does not rely on the assumption of a linear relationship between variables.  Khan et al. \cite{RFR} reported that the RFR outperformed conventional data imputation methods, especially when dealing with complicated and high-dimensional data. García et al. \cite{RFR_2}  demonstrated that RFR improved imputation accuracy on multivariate time series datasets and efficiently filled in missing values by leveraging correlations between numerous variables. Further studies have shown that RFR outperforms traditional imputation methods \cite{RF_Imputer}, such as mean or median imputation, especially under conditions where data is missing completely at random (MCAR) or missing at random (MAR) \cite{RF_Imputer}. RFR implementation, in missForest \cite{missForest}, allows for the refinement of imputed values through repeated predictions until convergence is achieved \cite{Imputation_accuracy_RF}. This iterative method is better than single imputation techniques because it increases accuracy while lowering bias in the imputed values \cite{comparison_RF}.


%\subsubsection{Random Forest Regressor}


\subsubsection{Multiple Imputation by Chained Equations}

MICE \cite{MICE_imputation} is a robust statistical technique that operates by creating multiple datasets with imputed values, allowing for errors in the imputation process to be included in later analyses. MICE is particularly advantageous for time series data, where missing values can disrupt the continuity and temporal relationships inherent in the data.  

 MICE  handles missing data by iteratively imputing each variable with missing values, conditioned on the other variables. MICE operates by updating these imputations over multiple cycles to ensure the imputed values reflect the underlying relationships in the data. This iterative process helps preserve the structure and variability of the data.
 
\begin{algorithm}
\caption{MICE  Algorithm}
\label{alg:mice}
\begin{algorithmic}[1]
    \STATE \textbf{Step 1:} For each missing value in the dataset, perform a basic imputation by  imputing the mean, which serves as "placeholders."
    
    \STATE \textbf{Step 2:} Set the placeholder mean imputations for one variable, denoted as "var," back to missing.
    
    \STATE \textbf{Step 3:} Treating "var" as the dependent variable and the other variables as independent variables, regress the observed values of "var" on the other variables in the imputation model. The assumptions of linear, logistic, or Poisson regression are all met by this regression model.
    
    \STATE \textbf{Step 4:} Substitute the regression model's predictions (imputations) for the missing values of "var". Both observed and imputed values are included when "var" is used as an independent variable in regression models for other variables.
    
    \STATE \textbf{Step 5:} Repeat Steps 2--4 for each variable with missing data, completing one full iteration or "cycle." At the end of each cycle, all missing values have been replaced with predictions that reflect relationships in the data.
    
    \STATE \textbf{Step 6:} Repeat Steps 2--4 for multiple cycles, updating imputations at each cycle.
    
\end{algorithmic}
\end{algorithm}


Algorithm \ref{alg:mice} shows how MICE operates by updating the data imputations over multiple cycles to ensure the imputed values reflect the underlying relationships in the data. This approach is appropriate for complicated datasets frequently found in time series analysis, MICE is adaptable and can handle a variety of data formats, including continuous and categorical variables \cite{MICE}. MICE is designed to handle MCAR and MAR missing data cases by leveraging the correlations between observed and missing data points \cite{MICE_2}. MICE can also accommodate both sporadically and systematically missing data, making it a versatile tool in various research contexts \cite{MICE_3}. Another useful ability of the MICE is to maintain the variance structure of the data during imputation, which is important for statistical analysis \cite{MICE_4}. 

 



\subsection{Index creation using PCA and Factor Analysis}
 
\subsubsection{Factor Analysis}

  Factor Analysis is a linear statistical approach used to capture and explain the variability among observed variables by grouping them into underlying, unobserved variables known as factors. This technique simplifies the observed variables into fewer latent factors, highlighting relationships and reducing dimensionality\cite{Engelhardt2013}. The purpose of factor rotation is to increase the overall interpretability by converting components into uncorrelated factors. Although there are other approaches, we implement Factor Analysis using varimax rotation, The co-variation in the observed variables is due to the presence of one or more latent variables (factors) that exert causal inference on observed variables, and we wish to infer them. 
 


 Factor Analysis has been used on several instances to create an index, Pervaiz et al.  \cite{Factor_Analysis_1} illustrated the application of factor analysis in constructing a Social Exclusion Index in Pakistan, utilising a multidimensional deprivation score derived from several indicators related to living standards and education. Luzzi et al. \cite{Factor_Analysis_2} highlighted the importance of Factor Analysis in constructing poverty indicators, showing that it can uncover insights into multidimensional poverty by merging multiple underlying variables into common factors that represent key dimensions of poverty. Similarly,  Alkire and Santos \cite{Factor_Analysis_3} developed the Multidimensional Poverty Index using a dual-cutoff approach combined with factor analysis to recognise and assess poverty across various dimensions. We need to first perform an adequacy test to check if we can find the factors in our dataset before we run Factor Analysis. We implemented the Kaiser-Meyer-Olkin (KMO) test, which measures the suitability of data for factor analysis. KMO estimates the proportion of variance among all the observed variables. KMO values range between 0 and 1, where a value of less than 0.5 is considered inadequate \cite{FA}.  


\subsubsection{Principal Component Analysis}

 PCA  is a statistical method commonly used for reducing dimensionality, transforming a high-dimensional dataset into a lower-dimensional version while preserving as much variance as feasible\cite{abdi2010principal}.   PCA  uses principal components that are orthogonal linear combinations of the original features, capturing maximum variance without overlapping information \cite{Principle_Component}.  PCA has been successfully applied in different fields to create composite indices. For example, Singh \cite{Principle_Component_1} used PCA to create India's food consumption index, emphasising its ability to manage multicollinearity among variables—a frequent challenge in complex surveys. Choi et al.  \cite{Principle_Component_2}   used PCA in creating an Aggregate Air Quality Index, offering a comparative assessment of air quality across various modes of transport.  Ali et al. \cite{Principle_Component_3} used PCA for a water quality index, highlighting the method’s strength in managing missing data and outliers. 
 
\subsection{Sub-indices}


We need to create the sub-indices using available data and indexes to create the Global Ease of Living Index where we review the quality of life over time for selected countries,   Therefore, we create four sub-indices as follows. 
 

\subsubsection{Economic Sub-Index}

 In the Economic sub-index, we include the "cost of living" since it provides an accurate representation of economic well-being \cite{Ogura}. The GDP growth rate is an important macroeconomic variable that affects stock exchange performance, underscoring its importance in evaluating the economic health of a country\cite{Verma}. There is a strong relationship between GDP per capita growth and financial growth and quality of life  \cite{Shapoval}. We include factors such as "inflation fate" and "unemployment rate" because of the trade-off between unemployment and inflation rates, indicating that changes in these economic indicators can significantly impact individuals' well-being \cite{Blanchflower}. 

\subsubsection{Quality of Life Sub-Index}

The  Quality of Life sub-index covers dimensions such as daily activities and physiological aspects, which are vital for evaluating overall well-being \cite{Behzadifar}. This indicator contains 7 sub-indicators, such as the "healthcare accessibility", which is directly related to the general well-being and enables us to observe health inequalities within and between countries, emphasising the importance of addressing disparities in healthcare access and quality \cite{Ona}. We include the  "crime rate" since it affects quality of life, and policymakers and law enforcement officials can have more effective strategies for crime prevention and response \cite{Sakip}. We include the average years of schooling, in conjunction with other factors such as expected years of schooling, and life expectancy they impact the overall quality of life  \cite{Jalil}. We also include factors such as the "poverty rate" to cover the impact of financial inclusion on quality of life  \cite{Ullah}.

\subsubsection{Institution Sub-Index}

Our  Institutional sub-index comprises the "corruption perception index", "freedom of speech index", "fundamental rights", "order and security", "regulatory enforcement", "civil justice" and "criminal justice". These factors are very important as they tell us about the relationship between the institutions and income inequality, which ultimately affects the ease of living in a country \cite{Behnezhad}. Factors such as the "human development index" (HDI) \cite{sagar1998human} measure development by considering factors such as living standards education and sustainable development, it is important in assessing a country's progress in improving the quality of life of its population \cite{Jones}.  

\subsubsection{Sustainability Sub-Index}

Our Sustainability sub-index incorporates key indicators such as "carbon dioxide emissions," "electricity production from renewable sources," "micro air pollution", and "greenhouse gas emission" which are critical for evaluating environmental sustainability and guiding policy decisions.  The carbon dioxide emission is a fundamental metric in assessing environmental impact, as it is directly linked to industrial activities and economic growth. Studies have shown that economic expansion often correlates with increased carbon dioxide emission, necessitating their inclusion in sustainability assessments to accurately reflect ecological consequences \cite{Sustainability_Index_1}. In addition to carbon dioxide emission, the percentage of electricity generated from renewable sources is vital for both developed and developing nations. Transitioning to renewable energy is essential for achieving Sustainable Development Goals (SDGs) and ensuring long-term energy sustainability\cite{Sustainability_Index_2}. Moreover, micro-level air pollution has been increasingly recognised for its detrimental effects on public health, making it a crucial factor in sustainability assessments. Studies have demonstrated that air pollution directly impacts health outcomes, underscoring the necessity of including this metric in sustainability indices \cite{Sustainability_Index_3}. The integration of air quality indicators alongside carbon dioxide emission and renewable energy production provides a holistic view of sustainability efforts, allowing for more informed decision-making and resource allocation\cite{Sustainability_Index_4}. We believe this area of our research has the highest potential to grow in future and by including multiple other factors, we can have a better estimate of the sustainable development of a country.
 




\subsection{Framework of Global Ease of Living Index}

 

Our Ease of Living Index aims to address the limitations of existing related indices (such as the Ease of Living index) by providing a more holistic view of living conditions by including factors such as healthcare access, crime rate, and freedom of speech and implementing them on a global scale. Therefore,  we offer policymakers a practical tool to identify and address specific areas needing improvement. Our contribution significantly enhances the literature on the quality of life assessment, providing a robust, reproducible methodology that can be applied across different contexts to inform targeted interventions.  We designed our index to be openly accessible, with all data and methodologies available for further research purposes ensuring transparency and reproducibility. Moreover, our approach avoids specific biases towards any single economic factor, offering a balanced assessment that can guide policymakers in implementing well-rounded improvements in living standards. This perspective is essential for capturing the true complexity of well-being and fostering informed, effective policy decisions.

Our Ease of Living Index consists of 4 major pillars namely Economic Index, Institutional Index, Quality of Life Index and Sustainability Index. The dataset created can be accessed on Google Drive \footnote{Global Ease of Living Index data: \url{https://drive.google.com/drive/folders/1S95vmsnH1RHQh_DPruKuKglaIksi0cpF?usp=sharing}}. We create the sub-indices using Factor Analysis, which captures the variance of the whole dataset into one factor. We subsequently assign weights to each factor based on their perceived importance \cite{weighted}, summing up to one, and compute the Ease of Living Index as the weighted sum of these 4 sub-indices.

\begin{figure*}[htbp!]
\centering
\includegraphics[width=0.75\textwidth]{Images/IndexCreation.png}
\caption{Ease of Living Index Creation}  
\label{fig:ease_of_living_index}
\end{figure*}

To reflect the varying levels of importance of each category in assessing living standards, we apply different weights to each sub-index. Specifically, we weighted the Economic Index as 0.25, the Institutional Index as 0.25, the Quality of Life Index as 0.35, and 0.15 for the Sustainability Index \cite{mohua2018ease}. We take inspiration from the Ease of Living Index created by the Government of India \cite{mohua2018ease}, which has given 0.25 to the institutional and social sub-index, 0.05 to the economic sub-index and 0.45 to the physical infrastructure sub-index. Similar to this, we also focus on an index having less reliance on Economic factors and focusing on the institutional and quality of life indices, which have factors such as healthcare and government effectiveness. 
 
%\xxxxxxxxxxxxxxxxxxxxxxxxxxxxxxxxxxxxxxxxxxxxx






\begin{figure*}[htbp!]  %label different steps and describe each step - see my other papers
\centering
\includegraphics[width = 1\textwidth]{Images/FrameworkDiagram.jpg}
\caption{Ease of Living Index Framework.}
\label{fig:framework}  %paste link of ppt slide - then I can also edit your framework diagram
\end{figure*}


The framework (Figure \ref{fig:framework}) for creating the Global Ease of Living Index is centred around the collection of various socio-economic and infrastructural data points, pre-processing them for consistency, and applying advanced statistical techniques to generate a reliable index.

In Step 1 (Figure \ref{fig:framework}),  the data collection process involved both web scraping and direct downloads from reputable sources. We utilised web scraping to extract data from several websites, with BeautifulSoup \footnote{\url{https://pypi.org/project/beautifulsoup4/}} as the primary library for this task \cite{webscraping}. Additionally, we downloaded some of the datasets from the respective websites and archived them in our GitHub repository \footnote{ \url{  }}. 




In Step 2, we merge the data from various datasets to create a single dataset having data from all 4 sub-indices. We note that the values of the indices within the data vary in different ranges with their units. Therefore, in Step 3, we apply standardisation to overcome the different ranges and convert all the units to standard units. To achieve uniformity across the numerous sources and for strong cross-national comparisons, we grouped the data by country and year, allowing for the smooth integration of indicators from different periods. We ensured that certain variables (GDP) were measured on uniform scales during preprocessing by standardising units to a common currency (such as USD).  This standardisation was essential for later statistical analysis to ensure that the data could be compared across countries without distortion.

In Step 4, after preprocessing and data cleaning, we present the data for imputation of missing values. We deploy two robust data imputation methods, namely RFR and MICE, and use an evaluation metric to find the best model for imputation. We keep a subset of data as the test case, knowing its true value, in order to evaluate the effectiveness of the different models. 

In Step 5, after obtaining the complete data with the best imputation method, we proceed with sub-index creation where we deploy data reduction methods such as PCA and Factor Analysis which enables the reduction of a vast number of indicators into a manageable set of factors that best represent the ease of living across different countries. In Step 6, we obtain our Ease of Living Index which is a weighted sum of four indices: Economic Index, Institutional Index, Quality of Life Index and Sustainability Index. Each of these indices was derived through data reduction techniques and assigned weights based on their relative importance. In the next step, we perform a statistical and longitudinal analysis of our Ease of Living Index.

Finally, in Step 7,  we compare the countries based on their Global Ease of Living Index. We ranked the countries for each year from 1970 till 2021, our main focus few of the G20 countries. We also compare countries that are currently in their developing phase, nearing the developed status and developed countries among the G20 members. This comparison between the Ease of Living Index gives a robust evaluation of the progress a country has made over the past five decades. 


%what is done next, you compare countries, mention all that



\subsection{Technical details}

We used MICE data imputation within each sub-index, including the Economic, Institutional, Quality of Life, and Sustainability indices. We select the hyperparameters in trial experiments, including a gradient-boosting decision tree for the ensemble method with a selected maximum depth of trees (5),   and learning rate (0.02). Similarly, in the case of RFR, we selected trees in the ensemble, with default hyperparameters for minimum samples/split, minimum samples/ leaf and maximum depth of the tree. 

\section{Results}
 

%These configurations were chosen to optimize the imputation process, with each set of hyperparameters tailored to the unique characteristics of the data within each sub-index.

%\begin{table*}[ht]
%\small
%\centering
%\caption{Hyperparameters for MICE data imputation by sub-index}
%\begin{tabular}{|l|c|c|c|c|}
%\hline
%\textbf{Sub-Index} & \textbf{Boosting} & \textbf{Max Depth} & \textbf{Num Leaves} & \textbf{Learning Rate} \\
%\hline
%Economic Index & gbdt & 4 & 15 & 0.02 \\
%Institutional Index & gbdt & 5 & 19 & 0.02 \\
%Quality of Life Index & gbdt & 5 & 18 & 0.02 \\
%Sustainability Index & gbdt & 4 & 15 & 0.02 \\
%\hline
%\end{tabular}
%\end{table*}

%Table 4 presents the hyperparameters used for the Random Forest Imputer across different sub-indices: Economic, Institutional, Quality of Life, and Sustainability. For each sub-index, specific values for n\_estimators (the number of trees in the forest), min\_samples\_split (the minimum number of samples required to split an internal node), min\_samples\_leaf (the minimum number of samples required to be at a leaf node), max\_features (the fraction of features considered for splitting), and max\_depth (the maximum depth of the tree) were selected. These tailored hyperparameters aim to enhance the accuracy of the imputation process by aligning with the data characteristics of each sub-index.

%\begin{table*}[ht]
%\centering
%\caption{RFR hyperparameters by Sub-Index}
%\begin{tabular}{|l|c|c|c|c|c|}
%\hline
%\textbf{Sub-Index} & \textbf{n\_estimators} & \textbf{min\_samples\_split} & \textbf{min\_samples\_leaf} & \textbf{max\_features} & \textbf{max\_depth} \\
%\hline
%Economic Index & 150 & 4 & 3 & 0.5 & 30 \\
%Institutional Index & 200 & 10 & 2 & 0.5 & 20 \\
%Quality of Life Index & 200 & 2 & 1 & 0.5 & 10 \\
%Sustainability Index & 200 & 2 & 1 & 0.5 & 10 \\
%\hline
%\end{tabular}
%\end{table*}

 

%Table 4.2 presents the hyperparameters used for the Random Forest Imputer across different sub-indices: Economic, Institutional, Quality of Life, and Sustainability. For each sub-index, specific values for n\_estimators (the number of trees in the forest), min\_samples\_split (the minimum number of samples required to split an internal node), min\_samples\_leaf (the minimum number of samples required to be at a leaf node), max\_features (the fraction of features considered for splitting), and max\_depth (the maximum depth of the tree) were selected. These tuned hyperparameters aim to enhance the accuracy of the imputation process by aligning with the characteristics of each sub-index.


 

%details about model implementation, hyperparameters used 


\subsection{Data imputation using MICE and RFR}
 

Next, we evaluate data imputation methods (MICE and RFR) to assess their ability to impute the missing values across key sub-indices. We assess their performance using the Root Mean Square Error (RMSE), and Mean Absolute Error (MAE).  We created synthetic data for the evaluation of the imputation model by using the columns from respective sub-indices that have the lowest percentage of missing (null) values.  To simulate the missing value situation, 40\% of the values were randomly deleted from the selected columns. We subdivided this data into training and testing datasets.  We run 30 independent model training experiments with random data initialisation, for each model and report the mean and standard deviation. 

Table \ref{tab:compareimp} compares  RMSE and MAE for MICE and RFR models across the Economic, Institutional, Quality of Life, and Sustainability sub-indices. In the Economic Index, MICE demonstrates a lower RMSE for GDP Growth Rate, highlighting its strength in variables with more straightforward relationships, while RFR achieves a smaller RMSE for both GDP growth rate and GDP per capita, reflecting its effectiveness in capturing more complex and straightforward data structures. Within the Institutional Index, both models exhibit similar RMSE scores across variables, with MICE performing moderately better than RFR indicating a linear relationship between the variables. The lower values for RMSE indicate consistent and reliable imputation performance for this sub-index. In the Quality of Life sub-index, RFR shows a slight advantage in variables such as Life Expectancy at Birth and Access to Electricity, with lower RMSE scores, suggesting a better fit for imputation in cases with more diverse data distributions. In the Sustainability sub-index, RFR consistently outperforms MICE, achieving lower RMSE for all the attributes,  which underscores its capability in handling complex, non-linear environmental data. 
The MAE scores provide additional insight into model performance, particularly emphasising the absolute magnitude of errors.


\begin{table*}[htbp!]
\centering
\caption{ Comparison of   MICE and RFR for given sub-indices showing mean and standard decision (in brackets) for RMSE and MAE from 30 independent model training runs.}
\label{tab:compareimp}
\resizebox{\textwidth}{!}{%
\begin{tabular}{|l|l|cc|cc|}
\hline
\textbf{Sub-Index} & \textbf{Attribute} &   \textbf{MICE} & &  \textbf{RFR} &\\
 \hline
 & & \textbf{RMSE} & \textbf{MAE} & \textbf{RMSE} & \textbf{MAE} \\
\hline
Economic Index & GDP Growth Rate & 4.51 [0.17] & 1.92 [0.03] & 3.83 [0.23] & 1.37 [0.13] \\
                              & GDP Per Capita  & 12004.46 [355.89] & 5408.87 [56.89] & 10003.01 [249.46] & 3390.69 [305.83] \\
\hline
 Quality of Life Index & Access to Electricity & 11.40 [0.44] & 3.38 [0.15] & 6.25 [0.11] & 1.84 [0.02] \\
                                     & Gender Development Index & 0.03 [0.0009] & 0.01 [0.0004] & 0.02 [0.0002] & 0.008 [0.00007] \\
                                     & Gender Inequality Index & 0.08 [0.002] & 0.03 [0.0007] & 0.05 [0.0005] & 0.017 [0.0002] \\
                                    & Human Development Index & 0.06 [0.0015] & 0.02 [0.0006] & 0.03 [0.0003] & 0.01 [0.0001] \\
                                    & LE at Birth            & 3.54 [0.12] & 1.40 [0.04] & 2.12 [0.03] & 0.85 [0.01] \\
\hline
Institutional Index & Control of Corruption & 0.32 [0.0087] & 0.13 [0.0033] & 0.49 [0.006] & 0.29 [0.0015] \\
                                & Government Effectiveness & 0.32 [0.0084] & 0.13 [0.0026] & 0.49 [0.004] & 0.29 [0.0013] \\
                                   & Political Stability    & 0.46 [0.0105] & 0.19 [0.0043] & 0.63 [0.008] & 0.39 [0.0057] \\
                               & Regulatory Quality     & 0.35 [0.0082] & 0.14 [0.0031] & 0.45 [0.007] & 0.32 [0.0026] \\
                                 & Rule of Law            & 0.30 [0.0089] & 0.12 [0.0033] & 0.47 [0.004] & 0.28 [0.0014] \\
                                 & Voice and Accountability & 0.44 [0.0070] & 0.18 [0.0031] & 0.50 [0.011] & 0.38 [0.0068] \\
\hline
Sustainability Index & Carbon Dioxide Emissions & 72.93 [3.47] & 21.65 [0.74] & 68.86 [0.58] & 15.63 [0.31] \\
                                  & Non-Renewable Electricity Production & 23.18 [0.45] & 9.91 [0.25] & 18.17 [0.17] & 8.65 [0.08] \\
                                  & Renewable Electricity Production & 4.04 [0.17] & 1.15 [0.05] & 3.08 [0.06] & 0.95 [0.01] \\
                              & Micro Air Pollution      & 10.56 [0.29] & 4.19 [0.12] & 7.57 [0.09] & 2.94 [0.03] \\
                                  & Greenhouse Emissions     & 30.84 [0.29] & 11.33 [0.12] & 24.02 [0.37] & 7.40 [0.11] \\
\hline
\end{tabular}%
}
\end{table*}

%The MAE scores provide additional insight into model performance, particularly emphasizing the absolute magnitude of errors. In the Economic Index, Random Forest achieved substantially lower MAE values compared to MICE, reflecting more precise imputation for GDP Per Capita and GDP Growth Rate. This trend is observed across several Quality of Life variables, such as Access to Electricity and Life Expectancy at Birth, where Random Forest's lower MAE scores point to better accuracy and consistency. In contrast, MICE displays lower MAE scores for Institutional variables like Control of Corruption and Government Effectiveness, suggesting strong imputation reliability for these metrics. For the Sustainability Index, Random Forest's consistently lower MAE values further emphasize its superior performance in handling interdependent complex data, offering a clear advantage in reducing error magnitudes.

We can summarise that RFR offers a more robust and effective solution for data imputation. MICE  excels with simpler linear relationships, while RFR demonstrated superior performance in handling data imputation for complex and non-linear data structures. Therefore, we will prioritise RFR, taking advantage of its ability to capture complex relationships and achieve greater accuracy in our data imputation process.

% KDE plots for MICE algorith for Institutional Index.
%\subsubsection{Kernel Density Estimation }

We next present the Kernel Density Estimation (KDE) plots obtained from MICE (Figure \ref{fig:kde_plots}) to illustrate the distribution of six key governance indicators including Control of Corruption, Government Effectiveness, Political Stability, Regulatory Quality, Rule of Law, and Voice and Accountability.  The imputed data distributions generally align well with the original distributions, indicating that the imputation process has effectively captured the underlying patterns in the data. This is further reinforced by the observation that both the imputed and original values exhibit peaks at similar points, indicating consistency in the underlying data distribution. 

\begin{figure*}[htbp!]
\centering
\includegraphics[width=0.8\textwidth]{Images/KDEPlotsInstitutionalIndex.png}
\caption{KDE Plots for sub-indices showing governance indicators. The black curves represent the distribution of the original data, while the red curves depict the distribution of the data that has been imputed using   MICE.}
\label{fig:kde_plots}
\end{figure*}

Some discrepancies are visible between the original and imputed distributions, suggesting areas where MICE  has made adjustments to fill in missing values. These differences highlight the overall structure and variability of the data, which is crucial for the reliability of the Institutional Index in subsequent analyses.
The KDE plots for the other three indices, including the Economic Index, Quality of Life Index, and Ease of Living Index—have been provided in the Appendix. These additional plots follow similar trends, reinforcing the consistency and reliability of the imputation process across different indices.


\subsection{Results for index creation using PCA and FA}

%The number of factors to retain in a model can be determined using two primary methods: an analytical approach and a graphical approach \cite{abdi2010principal}. Eigenvalues serve as an effective analytical criterion for selecting factors in PCA, with the general rule being to retain components that have an eigenvalue greater than 1 \cite{abdi2010principal}.

%\subsubsection{Principal Component Analysis}

%The graphical method, known as a scree plot, provides a visual representation of the eigenvalues for each principal component. By examining the scree plot, one can identify the "elbow" point where the curve starts to flatten\cite{jackson1993stopping}, indicating the optimal number of factors to retain. This method helps to visually confirm the results from the analytical approach.


%\begin{figure}[h]
%\centering
%\includegraphics[width=0.45\textwidth]{Images/Scree Plot.png}
%\caption{Scree Plot of Eigenvalues for Principal Component Analysis Across Sub-Indices}
%\label{fig:ease_of_living_index}
%\end{figure}

%From the provided scree plot, we can observe that for each sub-index (Economic, Institutional, Sustainability, and Quality of Life), there is a clear distinction where eigenvalues greater than 1 can be identified\cite{jackson1993stopping}. For example, in the Economic Index, the first two principal components have eigenvalues greater than 1, suggesting that two factors could be retained. The Institutional Index has a dominant first component explaining 85.17\% of the variance, with subsequent components contributing minimally.

Table \ref{tab:pca} provides a summary of the percentage of variance explained by each principal component (PC) across the four sub-indices. In the Economic Index, PC-1 explains 44.43\% of the variance, with PC-2 and PC-3 contributing substantially. In contrast, the Institutional Index is heavily dominated by PC-1, which accounts for a significant 85.17\% of the variance. The Quality of Life Index has PC1 explaining 65.80\% of the variance, while the Sustainability Index distributes the variance more evenly across its components, with PC-1 and PC-2 together explaining over 58\%.
 
\begin{table}[htbp!]
\centering
\small
\caption{Percentage of variance explained by   PCA}
\label{tab:pca}
\begin{tabular}{|l|l|c|}
\hline
\textbf{Sub-Index} & \textbf{PC} & \textbf{\% Variance} \\
\hline
 Economic Index & PC-1 & 44.43\% \\
                                 & PC-2 & 19.35\% \\
                                 & PC-3 & 18.05\% \\
                                 & PC-4 & 12.16\% \\
                                 & PC-5 & 3.90\% \\
                                 & PC-6 & 2.11\% \\
\hline
 Institutional Index & PC-1 & 85.17\% \\
                                     & PC-2 & 6.35\% \\
                                     & PC-3 & 4.74\% \\
                                     & PC-4 & 2.02\% \\
                                     & PC-5 & 0.87\% \\
                                     & PC-6 & 0.85\% \\
\hline
 Quality of Life Index & PC-1 & 65.80\% \\
                                       & PC-2 & 10.85\% \\
                                       & PC-3 & 6.94\% \\
                                       & PC-4 & 5.44\% \\
                                       & PC-5 & 4.38\% \\
                                       & PC-6 & 3.27\% \\
\hline
 Sustainability Index & PC-1 & 37.50\% \\
                                      & PC-2 & 20.93\% \\
                                      & PC-3 & 17.76\% \\
                                      & PC-4 & 13.27\% \\
                                      & PC-5 & 10.54\% \\
\hline
\end{tabular}
\end{table}

We then select the top principal components for each sub-index and then use factor analysis to effectively construct four sub-indices. This approach ensured a comprehensive understanding of the variance structure and justified the use of multiple components where necessary.

%\subsubsection{Factor Loadings for sub-indices}
 
%xxy


\begin{table}[htbp!]
\centering 
\resizebox{\columnwidth}{!}{ 

\begin{tabular}{|l|c|c|c|c|}
\hline
\textbf{Indicators} & \textbf{FL-1} & \textbf{FL-2} & \textbf{FL-3} & \textbf{FL-4} \\
\hline
GDP Growth Rate & -0.043 & 0.037 & \fbox{0.726} & -0.131 \\
Inflation Rate & -0.086 & 0.144 & 0.008 & 0.043 \\
%GDP Per Capita & \0.519 & -0.275 & \fbox{0.74} & 0.042 \\
%Unemployment Rate & -0.036 & -0.111 & -0.221 & 0.095 \\
%Cost of Living Index & \0.541 & -0.238 & \fbox{0.681} & 0.055 \\
%Local Purchasing Power Index & \0.567 & -0.388 & \fbox{0.586} & -0.009 \\
%Control of Corruption & \fbox{0.912} & -0.249 & 0.186 & 0.058 \\
%Government Effectiveness & \fbox{0.869} & -0.401 & 0.203 & -0.048 \\
%Political Stability & \fbox{0.811} & -0.206 & 0.089 & 0.237 \\
%Regulatory Quality & \fbox{0.857} & -0.378 & 0.183 & -0.07 \\
%Rule of Law & \fbox{0.924} & -0.303 & 0.158 & 0.039 \\
% Voice and Accountability & \fbox{0.842} & -0.231 & 0.032 & 0.192 \\
%Life Expectancy at birth & 0.363 & 0.234 & 0.116 & -0.025 \\
%Doctors Per 10,000 & 0.346 & \fbox{0.412} & 0.205 & 0.067 \\
%Access to Electricity & 0.286 & \fbox{-0.857} & 0.007 & -0.006 \\
%%C02 per capita  & 0.299 & -0.428 & 0.311 & -0.105 \\
%Gender Development Index & 0.226 & \fbox{0.49} & -0.01 & 0.206 \\
%Gender Inequality Index & -0.467 & \fbox{0.716} & -0.241 & -0.05 \\
%Human Development Index & 0.443 & \fbox{0.866} & 0.172 & 0.037 \\
%Health Care Index & 0.31 & \fbox{0.367} & 0.117 & 0.013 \\
%Crime Index & -0.375 & \fbox{0.65} & -0.173 & -0.03 \\
%CO2 emissions & 0.025 & 0.043 & 0.11 & \fbox{0.667} \\
%Electricity production: non-renewable & -0.071 & -0.136 & -0.043 & \fbox{0.46} \\
%Electricity production: renewable & 0.267 & -0.049 & 0.106 & \fbox{0.533} \\
%Micro air pollution & -0.328 & 0.164 & 0.007 & \fbox{-0.549} \\
%Greenhouse emission & -0.079 & 0.009 & 0.055 & \fbox{0.757} \\
\hline
\end{tabular}
}
\caption{Factor Loadings (FL) for various indicators in sub-index}
\label{tab:factor}
\end{table}

Factor Analysis reduces the dimensions by identifying the patterns of correlation among the variables, and in our case, it generated 4 distinct factors that capture key dimensions of the dataset. These factors represent the underlying sub-indices, and the factor loadings help clarify their structure, reinforcing our assumption that only four factors are needed to effectively explain the data.

As shown in Table \ref{tab:factor},   Factor 1 includes high loadings on indicators such as  Control of Corruption, Government Effectiveness, and Rule of Law, encapsulating governance quality, suggesting a strong association with institutional index. Factor 2 encompasses social and human development aspects, as observed by significant loadings on Gender Inequality Index, Human Development Index, and Access to Electricity. This factor highlights areas related to social equity and basic infrastructure, this corresponds highest with our Quality of Life Index. Factor 3 captures economic prosperity, with high loadings on GDP Growth Rate, GDP Per Capita, and Cost of Living Index, emphasising indicators tied to wealth and living standards, which are similar to our Economic Index. Lastly, Factor 4 reflects environmental sustainability, focusing on emissions, renewable energy, and pollution, with high loadings on Carbon Dioxide Emissions and Greenhouse Emission, which corresponds to the Sustainability Index. We used these factors to build 4 sub-indices that make up our Global Ease of Living Index. 

%\subsubsection{Kaiser–Meyer–Olkin (KMO) test results}



\begin{table}[htbp!]
\centering
\small
\caption{ Results for  KMO test in  Factor Analysis by two data imputation methods}
\label{tab:kmo}
\begin{tabular}{|l|c|c|}
\hline
\textbf{Sub-Index} & \textbf{MICE} & \textbf{Random Forest} \\
\hline
Economic Index         & 0.72 & 0.71 \\
Institutional Index    & 0.90 & 0.90 \\
Quality of Life Index  & 0.89 & 0.85 \\
Sustainability Index   & 0.62 & 0.49 \\
\hline
\end{tabular}
\end{table}

The KMO test assesses whether the data is adequate for Factor Analysis, with values above 0.6 indicating suitability and values above 0.8 suggesting excellent sampling adequacy.   Table \ref{tab:kmo} shows KMO test \cite{shrestha2021factor} results for Factor Analysis across sub-indices, with scores for MICE and RFR. In the case of the Economic Index, both methods produce KMO  test scores above 0.7, supporting the applicability of Factor Analysis. The Institutional and Quality of Life sub-indices show high KMO values, both with MICE and RFR, indicating substantial common variance, ideal for extracting latent factors. Although the Sustainability Index yields a lower KMO score with RFR (0.49), the MICE score of 0.62 meets the threshold, suggesting that Factor Analysis remains feasible with MICE-imputed data. Overall, these results favour Factor Analysis, especially for indices where shared variance is high, enabling effective latent factor representation.

\subsection{Data Analysis of Sub-Indices}


We used a series of methods, including data imputation (MICE and RFR) and dimensionality reduction  (PCA and RFR) to create data for sub-indices that provide a comprehensive view of global living conditions, spanning from 1970 to 2021. It includes data for 70 countries, capturing a wide range of economic, social, and environmental factors that influence the ease of living across different regions. The dataset is structured around four key sub-indices including the Economic Index, which measures economic performance and growth; the Institutional Index, reflecting the governance, policy, and institutional frameworks of the countries; the Quality of Life Index, which captures dimensions of well-being such as healthcare, education, and social stability; and the Sustainability Index, focused on environmental sustainability and green practices, including carbon emissions and renewable energy use. We publish our sub-indices using GitHub repository. %\footnote{Sub-index: \url{}}.



\begin{figure*}[htbp!]
\centering
\includegraphics[width=0.65\textwidth]{Images/Top10economies.png}
\begin{center}
\caption{ The plot highlights the GDP of the world's top 10 economies in 2024.} 
\end{center}
%cite IMF 2024
\label{fig:ease_of_living_index}
\end{figure*}

We then focus on selected countries from the top 10 largest economies in the world as of 2024, which includes the United States, China, Japan, Germany, India, the United Kingdom, and Canada \cite{forbes2024top}. These economies play a pivotal role in shaping global trends and offer valuable insights into how different nations approach the challenge of improving living standards amidst varying levels of institutional strength, and environmental policies. This pool of countries gives us diversity in the context of the development of a nation. Countries such as the United States, Germany, Canada and Japan are considered as developed countries and countries such as India and China are considered as developing nations. We have also focused on some of the emerging economies of the world such as Malaysia. They are on track to become a high-income nation by the end of 2028 \cite{worldbank2021malaysia}. We have also included Australia in our analysis as it has one of the highest living standards, including the human capital index and high GDP per capita \cite{worldbank_australia_data}. 

\subsubsection{Analysis of Quality of Life sub-index}
 


We present a longitudinal study of the Quality of Life sub-index  for the United States, China, India, and Australia.  Figure \ref{fig:QOLI_over_time} illustrates how developed nations (United States and Australia), have maintained consistently high living standards, while emerging economies   (India, China and Malaysia) have seen significant improvements over the years. These results reinforce that countries with better socio-economic factors such as healthcare facilities \cite{QOLI_Results_1},   gender development \cite{QOLI_Results_2},  and HDI\cite{QOLI_Results_3} have better living conditions. On the other hand, developing countries such as China have shown a steady rise in their Quality of Life Index, reflecting improvements in infrastructure, healthcare, and education. This is all a result of the policies that the Chinese government have implemented in recent years\cite{QOLI_Results_4}. India’s Quality of Life has risen more sharply since the 1990s, as the country began to focus more on social and economic reforms\cite{QOLI_Results_5}. This analysis highlights the evolution of living standards globally and the efforts made by developing countries to catch up with established economies.

%xx

\begin{figure*}[htbp!]
\centering
\includegraphics[width=0.75\textwidth]{Images/QOLIndexComparision.png}
\caption{Quality of Life sub-index over time.}
\label{fig:QOLI_over_time}
\end{figure*}

Figure \ref{fig:healthcare_index} represents the choropleth map, which offers a geographical perspective on healthcare quality across different countries. The darker shades on the map represent nations with stronger healthcare systems, such as Australia, Canada, and many parts of Europe \cite{barber2017haq}, while lighter shades reflect countries with weaker healthcare infrastructures, particularly in regions such as  Sub-Saharan Africa and parts of Asia. The map visually highlights global disparities in healthcare access and quality, emphasising how economically stronger countries can provide better healthcare services and the practices \cite{joumard2010health} and policies that these countries have implemented can become a guideline for developing nations to improve their healthcare infrastructure \cite{QOLI_Results_6}. It also underscores the importance of healthcare in determining a nation’s quality of life, as countries with higher healthcare scores generally exhibit better living standards overall.

%xxx

\begin{figure*}[htbp!]
\centering
\includegraphics[width=0.95\textwidth]{Images/HealthcareIndex.png}
\caption{Healthcare Index by Country (2022). }
\label{fig:healthcare_index}
\end{figure*}

\subsubsection{Economic sub-index}
%xx

Figure \ref{fig:GDP_growth} depicts the GDP growth rate graph, which highlights the average economic growth across countries in our study. We can observe that China leads the group, with an average growth rate of 6.11\%, followed by India at 5.73\% which indicates that emerging economies have been experiencing rapid economic expansion. This is contrary to established economies \cite{Economic_results_1} such as  Germany  (1.12\%) and Japan (0.77). Countries such as the United States and Australia show moderate growth rates, reflecting their mature but stable economies. The differences in growth rates reflect each country’s stage of economic development, with emerging economies growing faster due to industrialisation and modernisation \cite{Economic_results_2}, while developed countries had slower and steady growth \cite{Economic_results_3}.

\begin{figure}[htbp!]
\centering
\includegraphics[width=0.55\textwidth]{Images/GDPGrowthdecade.png}
\caption{Average GDP growth rate over last decade. %\cite{}.
}
\label{fig:GDP_growth}
\end{figure}

Figure \ref{fig:cost_of_living} represents the Cost of Living sub-index for the year 2021, which compares the relative living costs across different countries. We can observe that the United States (0.581) and Australia (0.58) have the highest cost of living, which aligns with their high GDP per capita and advanced economies. In contrast, India (0.039), has the lowest cost of living followed by Malaysia.In the case of India, the ratio of GDP per capita to the cost of living is the highest among all nations which suggests that even though absolute earnings for India is much lower,  the purchasing power remains relatively high \cite{Economic_results_4}. The lower cost of living in India reflects more affordable goods, services, and housing. It also indicates that individuals can achieve a relatively comfortable standard of living, with lower income levels when compared to developed countries such as the United States and Australia, where higher incomes are offset by the elevated cost of living. This affordability is one of the key reasons for India's labour force and growth of foreign direct investment, especially for investors and expats  seeking a high quality of life at a lower cost \cite{Economic_results_5}. Developed economies such as  Germany, Japan, and the United Kingdom also show moderate cost-of-living levels, reflecting the balance between higher wages and more affordable services. This further highlights how economic development affects the affordability of goods and services, with wealthier nations tending to have higher living costs due to higher wages, property values, and general expenses.

\begin{figure}[htbp!]
\centering
\includegraphics[width=0.45\textwidth]{Images/CostofLiving2022.png}
\caption{ The 2021 Cost of Living sub-index for selected countries, illustrating variations in living expenses across different regions.}
\label{fig:cost_of_living}
\end{figure}

%Figure \ref{fig:COL} represents the Cost of Living Index which compares the relative living costs across different countries. The United States (0.58) and Australia (0.58) have the highest cost of living, which is a byproduct of their high GDP per capita and advanced economies. In contrast, India, with a Cost of Living Index of just 0.04, has the lowest living costs among the countries compared. What stands out for India is the ratio of GDP per capita to the cost of living, which is the highest among all nation even thought India has lowest GDP per capita among the countries at study. This suggests that even though absolute earnings for India might be lower but the purchasing power remains relatively high \cite{Economic_results_4}. While the lower cost of living in India reflects more affordable goods, services, and housing, it also indicates that individuals can achieve a relatively comfortable standard of living with lower income levels compared to developed countries like the United States and Australia, where higher incomes are offset by the high cost of living. This affordability is one of the key reasons for India's attractiveness, especially for expats and digital nomads seeking a high quality of life at a lower cost\cite{Economic_results_5}. Middle-tier economies like Germany, Japan, and the United Kingdom also show moderate cost-of-living levels, reflecting the balance between higher wages and more affordable services. This comparison highlights how economic development affects the affordability of goods and services, with wealthier nations tending to have higher living costs due to higher wages, property values, and general expenses.


Figure \ref{fig:Econ_Index} shows the Economic sub-index which compares the overall economic strength of these nations over time, combining factors like GDP, employment, inflation, and other economic indicators. China and India show rising economic stability, with China demonstrating the most significant growth, particularly around 2010, reflecting its rapid industrialisation and global influence \cite{Economic_results_6}. Australia, Canada, and Germany also show consistent performance, though at a more moderate level, reflecting their stable, well-developed economies. The United States has been one of the most consistently performing economies. India has shown improvement after 2005, signalling its growing presence on the global economic stage. This comparison highlights the dynamic nature of economic growth, where emerging economies such as India and China are catching up with the developed countries.


\begin{figure}[h]
\centering
\includegraphics[width=0.45\textwidth]{Images/EconomicIndex.png}
\caption{Economic Index over past two decades.}
\label{fig:Econ_Index}
\end{figure}

\subsubsection{Institutional sub-index}

The Institutional sub-index provides a comprehensive view of various governance-related factors, such as the effectiveness of rule enforcement, freedom of speech, crime rates, and other institutional structures that contribute to the overall stability and prosperity of a nation.

Figure \ref{fig:crime_index} presents the crime  index that provides average crime rates across the selected countries (2000-2024). We can see that Malaysia has the highest crime index \cite{Institutional_results_1}, followed by India and the United States.  The United States has the highest gun ownership in the world, with estimates suggesting there are more guns than people in the country, at around 120.5 guns per 100 residents \cite{Institutional_results_2}. The high availability of firearms has been strongly linked to the elevated rates of gun-related crimes, such as homicides, assaults, and mass shootings. China and the United Kingdom follow closely, showing relatively high crime levels but still lower than  Malaysia, India, and the United States.

\begin{figure}[htbp!]
\centering
\includegraphics[width=0.45\textwidth]{Images/CrimeIndex.png}
\caption{ The crime index from 2000 to 2024 for selected countries, shows Malaysia and India having higher index values compared to Japan and Germany.} 
\label{fig:crime_index}
\end{figure}

 Australia, Canada, and the United Kingdom fall into the mid-level category which likely benefits from better institutional frameworks and governance but still face challenges in fully minimising crime rates \cite{Institutional_results_3}. On the lower end, Germany and Japan exhibit the lowest crime indices among the selected countries. These nations are known for their strong law enforcement, efficient legal systems, and social stability \cite{Institutional_results_4}.

Figure \ref{fig:speech} shows the Freedom of Speech Index \cite{}, which measures the extent to which citizens in different countries can express themselves freely without fear of retribution. We observe a clear distinction between  Australia, Canada, and Germany, where the index remains consistently high, reflecting strong protections for freedom of speech \cite{Institutional_results_5, Institutional_results_6}. On the other hand, China and Malaysia consistently show lower values, highlighting restricted freedom of expression. China's index hovers around -1.5 which aligns with its authoritarian governance structure \cite{Institutional_results_7}. India occupies a middle ground, showing a stable but lower freedom of speech index than the developed nations. %This trend emphasises the importance of institutional freedoms in maintaining democracy and guaranteeing that people can engage in any political or non-political conversation without fear of retaliation.

\begin{figure}[h]
\centering
\includegraphics[width=0.45\textwidth]{Images/FreedomofSpeech.png}
\caption{  Freedom of Speech Index   (2000 to 2020) illustrating China's getting consistently lower values when compared to countries like the United States and Australia. }
\label{fig:speech}
\end{figure}




 Figure \ref{fig:inst}  shows that   Australia, Canada, and the United States have consistently maintained high institutional index scores \cite{}, reflecting stable governance, strong rule of law, and effective administration. India and Malaysia, on the other hand, show moderate but improving scores over time, indicating progress in strengthening governance structures. China presents a relatively lower Institutional Index, which reflects its centralised governance system and weaker rule of law. Notably, Germany and the United Kingdom remain at the top of the Institutional Index, demonstrating strong, consistent governance and regulatory frameworks. This overall analysis shows that stronger institutions often correlate with higher economic and social stability, while weaker institutions might struggle to enforce laws, protect freedoms, and maintain low crime rates.

\begin{figure}[htbp!]
\centering
\includegraphics[width=0.45\textwidth]{Images/InstitutionalIndex.png}
\caption{ Institutional Index trends for selected countries (2000 to 2020), highlighting stable high performance in  Japan and gradual improvements in India.}
\label{fig:inst}
\end{figure}



\subsubsection{Sustainability Index}

Figure \ref{fig:sustainability_index} presents the sustainability index \cite{} for the last two decades, which highlights a general upward trend for most countries, indicating progressive improvements in sustainable practices over the years. Developed nations such as Germany and the United Kingdom show significant gains, particularly from 2010 onwards \cite{Sustainability_results_1}, which can be attributed to rigorous environmental policies, widespread investment in renewable energy, and a societal shift towards sustainable living practices. For example, Germany's Energiewende (energy transition) initiative %\cite{}
has accelerated the adoption of renewables, while the United Kingdom has committed to reducing carbon emissions through stringent regulations and clean energy incentives \cite{Sustainability_results_2}. 

\begin{figure*}[htbp]
\centering
\includegraphics[width=0.65\textwidth]{Images/SustainabilityIndex.png}
\caption{Sustainability Index over last two decades.}
\label{fig:sustainability_index}
\end{figure*}

On the contrary,  India and China have shown slower progress, largely because they relied on industrial growth fuelled by non-renewable energy sources and slower policy adoption in sustainability. The lower sustainability index for India can be attributed to its overreliance on coal as the primary energy source, accounting for approximately 71\% electricity generation, further complicating the transition to a more sustainable energy model \cite{Sustainability_results_3}. The rapid rise observed in Malaysia around 2015 likely reflects targeted government programs to boost environmental standards and embrace renewable energy, possibly driven by international pressure and economic incentives \cite{Sustainability_results_4}. Overall, this index underscores the varying levels of commitment to sustainability across nations, with developed countries setting the pace and emerging economies making steady but slower advancements.
 

Figure \ref{fig:green_house} presents carbon emissions \cite{} by selected countries, which reveals a critical challenge for many countries, like restricting pollution while striving for economic growth.  China and India have emission levels that align with their heavy reliance on fossil fuels for industrial output and energy production \cite{Sustainability_results_5}. %Their rapid economic development, fuelled by non-renewable energy sources, has led to high greenhouse gas emissions, which, in turn, adversely affect their Sustainability Index scores.

\begin{figure*}[htbp]
\centering
\includegraphics[width=0.5\textwidth]{Images/GreenHouseGasEmission.png}

\caption{Greenhouse gas emissions (in metric tonnes). Malaysia and China leading in emissions when compared to lower values observed in Germany and the United Kingdom.} 
\label{fig:green_house}
\end{figure*}


In contrast, the United Kingdom and Germany have successfully implemented measures to curb emissions, thanks to their transition towards renewable energy and policies for reducing the carbon footprint  \cite{}. These policies include the decommissioning of coal-fired power facilities, more stringent automobile laws, and emissions trading programs \cite{}. The importance of energy and environmental policy is shown by the inverse connection between emissions and sustainability initiatives. To improve their Sustainability Index rankings, high-emission countries must invest in cleaner technologies and adopt more aggressive environmental regulations\cite{Sustainability_results_6}.



\subsection{Global Ease of Living Index}

We finally review the four sub-indices obtained from previous investigations and then develop the Global Ease of Living Index.
Figure \ref{fig:heatmap_EOLI} presents a correlation matrix of all the sub-indices, highlighting the relationships between the Economic Index, Institutional Index, Quality of Life Index, and Sustainability Index. The correlation values range between -1 and 1, where values close to 1 indicate a strong positive correlation, while values near 0 suggest little to no relationship \cite{stewart1968general}. 
A notable finding is the strong correlation between the Quality of Life and the Sustainability sub-indices, suggesting that countries with better social well-being and living standards also tend to have strong sustainability practices. Additionally, the Economic Index and Institutional Index exhibit a moderate correlation, implying that stronger economies are often supported by robust governance structures. 
The Year and Quality of Life sub-index show a moderate correlation, indicating that global living standards have generally improved over time. Similarly, the Economic and Quality of Life sub-indices are moderately correlated, which suggests that factors beyond economic strength, such as institutional quality and environmental sustainability, also significantly impact a country's quality of life.


\begin{figure}[htbp!]
\centering
\includegraphics[width=0.45\textwidth]{Images/subindexheatmap.png}
\caption{\small \textit{The correlation   among sub-indices within the Global Ease of Living Index, highlighting strong correlations between the Quality of Life and Sustainability sub-indices.}}
\label{fig:heatmap_EOLI}
\end{figure}
%xxx



\begin{figure}[htbp!]
\centering
\includegraphics[width=0.45\textwidth]{Images/ELOIViolinPlot.png}
\caption{\small \textit{The violin plot illustrates the distribution of the Ease of Living Index for selected countries over the last three decades, highlighting variations in data spread and central tendencies across regions.}}
\label{fig:violin_plots}
\end{figure}

Figure \ref{fig:violin_plots} illustrates the violin plot of the distribution of our Global Ease of Living Index across the selected countries, highlighting both the range and density of the data.  We observe that the developed exhibit consistently high living standards, which can be deducted from their narrow violin and having higher values. Canada and Germany show stable, high median values with a bit more variability compared to the United States. In contrast, China, India, and Malaysia have lower median values, with India showing a particularly tight distribution around lower scores, reflecting relatively uniform, but lower living conditions compared to more developed nations. This is natural since developing countries have a series of challenges given resources, infrastructure, government services, and employment that highly influence the quality of life. 

We next present a map-based visualisation of the selected countries in our study. 
The choropleth maps present the global distribution of the Ease of Living Index representing the beginning of the given decades with years 1991, 2001, 2011 and 2021, respectively. We categorise the selected countries into four major levels, as shown in the Appendix.
 %These categories are visually represented using a gradient of colors, from light yellow (Low) to dark red (High), offering a clear depiction of how living conditions have changed globally over two decades.


We note that there has been media hype about the World Happiness Index \cite{helliwell2015geography}, which has attracted attention in social media and the government of countries such as India questioning the methodology  and arguing that the rank should have been 48 rather than 126 in 2024 according to SBI report \footnote{\url{https://timesofindia.indiatimes.com/india/world-happiness-index-flawed-indias-rank-should-have-been-48-not-126-sbi-ecowrap/articleshow/99337121.cms}}. We review and compare the World Happiness Index, in comparison to our Global Ease of Living Index. 

\begin{table}[htbp!]
\centering 

\begin{tabular}{|l|c|c|}
\hline
\textbf{Country} & \textbf{EoLI Rank} & \textbf{Happiness Index Rank} \\
\hline
Malaysia & \textbf{47} & \textbf{81} \\
United States & 18 & 14 \\
China & \textbf{41} & \textbf{84} \\
Canada & 17 & 15 \\
Germany & 19 & 13 \\
Australia & 4 & 12 \\
India & \textbf{51} & \textbf{139} \\
Japan & 8 & 56 \\
United Kingdom & 21 & 17 \\
\hline
\end{tabular}

\caption{Ease of Living Indian (EoLI) and Happiness Index Rank for 2021.}
\label{tab:happiness}
\end{table}

%Due to biases and limitations in data gathering methods, the comparative study of the 2021 Ease of Living Index and Happiness Index rankings shows notable disparities between nations. 


Table \ref{tab:happiness} shows a comparison of Global Ease of Living rankings and Happiness Index rankings for 2021 showing notable disparities between the countries. For instance, the United States and Canada achieve relatively high rankings on both indices (18th and 17th for Ease of Living, 14th and 15th for Happiness, respectively), reflecting their strong economic foundations and extensive social welfare systems. In contrast, India displays a notable disparity: ranking 51st on the Ease of Living Index but plummeting to 139th on the Happiness Index. This gap is partly attributable to the Happiness Index’s emphasis on economic measures like GDP per capita, which can unfairly disadvantage developing nations \cite{Akgün}. Moreover, the limited sampling by the World Happiness Index which is based on collecting only 2,000 - 3,000 survey responses to represent large countries, such as India and China that have  populations exceeding 1.4 billion, which fails to accurately reflect the complexity and diversity of various economic and social factors in different regions   \cite{Ko_Lee}. Malaysia similarly suffers from inadequate sample representation, potentially skewing results across its diverse regions. Furthermore, happiness outcomes across countries are strongly influenced by differences in ecological variables, such as access to natural resources, environmental quality, and income levels. In contrast, Japan ranks lower in happiness (56th), while having a great Ease of Living Index ranking (8th). This might be because of sociocultural variables including ageing populations and work-related stress. This suggests that the World Happiness Index is a context-dependent metric that takes into account the particular socioeconomic and environmental characteristics of each nation rather than being a universal one \cite{Radkevich}. These limitations highlight the necessity for more representative, equitable and transparent data collection methods, and emphasising the importance of developing more nuanced indices that account for varying global contexts.

\begin{table*}[htbp!]
    \centering
    \caption{\small{\textit{Average Rankings of Sub-Index Over the Last Decade}}}
    \resizebox{\textwidth}{!}{%
    \begin{tabular}{|l|c|c|c|c|c|}
        \hline
        \textbf{Country} & \textbf{Economic Rank} & \textbf{Institutional Rank} & \textbf{Quality of Life Rank} & \textbf{Sustainability Rank} & \textbf{Ease of Living Rank} \\
        \hline
        Australia & 7.4 & 10.1 & 7.5 & 6.0 & 4.2 \\
        Canada & 17.8 & 9.1 & 12.4 & 9.4 & 9.1 \\
        China & 37.6 & 63.3 & 71.8 & 37.6 & 52.9 \\
        Germany & 23.6 & 12.6 & 4.4 & 24.8 & 17.3 \\
        India & 49.7 & 58.9 & 80.0 & 77.5 & 66.4 \\
        Japan & 7.7 & 16.9 & 17.6 & 13.5 & 9.1 \\
        Malaysia & 56.7 & 39.8 & 54.6 & 19.2 & 46.4 \\
        United Kingdom & 17.4 & 13.4 & 15.1 & 29.8 & 16.1 \\
        United States & 16.1 & 18.3 & 15.9 & 18.0 & 12.8 \\
        \hline
    \end{tabular}%
    }

    \label{tab:average_rankings}
\end{table*}
We finally review the sub-indices and ranking for the selected countries for the last decade in relation to the Global Ease of Living Index. 
Table \ref{tab:average_rankings} shows significant differences in rankings across economic, institutional, quality of life, sustainability, and ease of living indices. Germany, Canada, and Australia continuously do well in every category, which is indicative of their strong economies, efficient political systems, excellent standards of life, and aggressive environmental programs. We notice that good governance and economic stability are demonstrated by Japan's good performance, especially in economic and institutional rankings. However, despite advancements in economic growth, China and India have higher average scores, indicating continued difficulties in enhancing sustainability and quality of life. Similar issues confront Malaysia, which performs poorly on sustainability and institutional indexes. In relation to their high economic and living standards, the United States and the United Kingdom have balanced but somewhat lower sustainability and institutional rankings. These insights emphasise the need for a multifaceted approach to national development that not only focuses on economic growth but also strengthens governance, social welfare, and environmental sustainability to improve overall ease of living. %Figure \ref{} presents the Global Ease of Living Index for the selected countries from 1970-2024 where which finds a close relationship with the Quality of Life sub-index as shown in Figure 5 \ref{}. %say more. 


\section{Discussion}

%\subsection{Overview of Results}

%Our results section highlights the methodologies, hyperparameter optimization, and outcomes of developing and evaluating the Global Ease of Living Index and its sub-indices: Economic, Institutional, Quality of Life, and Sustainability. Data imputation played a critical role, with MICE (Multiple Imputation by Chained Equations) and Random Forest serving as the primary methods. Because of its prowess in managing intricate, non-linear data structures and a high volume of missing values—many of which were not completely absent at random—Random Forest typically performed better than MICE. Although MICE performed well as a statistical model when applied to structured data, it showed shortcomings when dealing with significant missing value unpredictability. Key themes emerged from analyses of each sub-index, with the Institutional Index showing how governance systems affect stability, the Quality of Life Index revealing differences between nations, and the Economic Index displaying macroeconomic stability and growth patterns. The Sustainability Index emphasized differing levels of environmental commitment, with developed nations leading in emissions reduction and renewable energy use.A thorough Global Ease of Living Index was produced by identifying latent variables influencing each sub-index using Principal Component Analysis (PCA) and Factor Analysis. Correlation analysis revealed a strong link between Quality of Life Index and Sustainability Index, comparisons with the Happiness Index revealed shortcomings of Happiness Index with regard to data representativeness and socioeconomic biases, especially in developing countries. This analysis underscores the diverse factors shaping global living conditions and the need for targeted, equitable improvements in quality of life.

%\subsection{Results in the Context of Current Literature}

Multiple imputation is widely regarded as a robust method for addressing missing data, as it allows for the creation of several complete datasets by imputing missing values based on observed data patterns \cite{Mutubuki_Alili}. This technique helps to preserve the statistical power of the analysis and reduces the bias that can arise from simpler methods such as mean imputation or last observation carried forward \cite{Alili_Dongen}. Many studies have implemented this approach for the imputation of socioeconomic and environmental data, demonstrating its effectiveness and flexibility. In alignment with these findings, we have utilised MICE, which employs multiple imputations across all sub-indices in our analysis. This reflects consistency with the established practices in the literature and underscores its suitability as a good statistical model for data imputation. RFR  demonstrated strong performance, even when up to 50\% of values are missing \cite{Kakaï}, with their effectiveness particularly noted in ecological data \cite{Kakaï}. According to a number of studies comparing different imputation strategies, models like missForest frequently offer strong performance across a range of missing data mechanisms  \cite{Miss_Forest}. According to our investigation, the RFR performed better than MICE, which is consistent with studies in the literature. % showing that this method is capable of managing complicated missing data scenarios and providing improved imputation accuracy across data types. 

We have taken cues from well-established approaches in the literature while creating our Global Ease of Living Index. In particular, we have used a weighted average mean approach to develop our score, which is in line with the four sub-pillars that the Indian government has established for their Ease of Living score \cite{mohua2018ease}. Furthermore, in our Quality of Life sub-index, we have incorporated various factors, encompassing standard of living, education, and health. This strategy draws inspiration from the  HDI \cite{undp_hdi}, one of the most well-known composite indices that assesses human development along three main dimensions: standard of living (Gross National Income per capita), education (mean and expected years of schooling), and health (life expectancy) \cite{undp_hdi}. We have utilised  PCA to capture latent variables that encapsulate the maximum amount of information from our data, resulting in composite scores for the respective sub-indices. A large body of research \cite{Srivastava,Delalić_Šikalo} showed the effectiveness of PCA in creating composite indexes, which has motivated  our framework.  

Even though our Global Ease of Living Index  has yielded insightful information about living standards around the world thanks to thorough data analysis and rigorous methodology, there are still areas ripe for future enhancement. Complex and incomplete data patterns could be handled much better by integrating sophisticated imputation techniques based on data augmentation \cite{khan2024review}
with Random Forest models. Furthermore, utilizing Bayesian deep learning \cite{kulkarniChandra2024bayes} for imputation would provide an improved method of handling uncertainty, enhancing data precision and resilience in general. By integrating Bayesian ranking techniques \cite{rendle2012bpr} into index creation, comparative analysis can be improved, and fair rank can be established.

Broadening the scope of parameters within the Sustainability Index—by incorporating metrics like biodiversity, waste management practices, and climate resilience—would yield a more comprehensive view of environmental and societal health. Furthermore, expanding empirical validation through extensive physical surveys-for example, conducting surveys in major Australian cities —could improve our comprehension of living situations and strengthen data reliability. Creating an interactive dashboard for EoLI based on GitHub.io would allow users to interact with the index in real-time and promote dynamic data additions. These efforts when combined will better guarantee that the index stays relevant, and flexible for the changing global conditions, encouraging a group effort toward sustainable growth and improved living standards.


%[overview of general results]


%[how results relate to literature - data imputation and other indexes ]

%[limitations of your study ]


%[Bayesian deep learning for imputation - uncertainiaty]
%[Bayeain ranking]


\section{Conclusion}
 
We presented the Global Ease of Living Index    to reflect the complex relationship between living conditions in selected countries. We used machine learning techniques for data imputation, notably MICE and Random Forest models, to fill large gaps in socioeconomic data. The Random Forest imputer provided a more robust and dependable strategy for dealing with missing information. We constructed four sub-indices including Economic, Institutional, Quality of Life, and Sustainability  PCA and Factor Analysis,  contributing to a holistic understanding of global living conditions. The analysis revealed stark contrasts between developed economies, such as Australia, the United States, Japan and Germany, which consistently rank high due to their stable institutions and economic prosperity, and emerging nations such as India, China and Malaysia, which demonstrated impressive economic growth but continue to face challenges in social and environmental areas.

 Additionally, our comparison with the notable Happiness Index revealed notable biases, particularly the index's excessive dependence on GDP and small sample size based on surveys that fail to effectively reflect large populations in  China and India. This exposes the limitations of using a one-size-fits-all approach to measure well-being. Our index, on the other hand, acknowledges the context-dependent nature of quality of life by using a more diverse range of indicators. Our research highlights the value of reproducible and transparent data, providing policymakers with a useful framework to pinpoint areas that need improvement, ranging from environmental sustainability to healthcare access. By presenting a balanced and data-driven assessment, this research paves the way for more nuanced and equitable strategies for enhancing global living standards, encouraging ongoing research and policy innovation to address the complexities of modern well-being.

\section*{Data and Code Avaialbaility}

Code: \url{https://github.com/pinglainstitute/ease-of-living-index}

%https://github.com/TanayPanat28/Ease-of-Living-Index

 %The original code with data \footnote{Global Ease of Living Index data: \url{https://github.com/TanayPanat28/Ease-of-Living-Index}}. 

%% The Appendices part is started with the command \appendix;
%% appendix sections are then done as normal sections

%% If you have bibdatabase file and want bibtex to generate the
%% bibitems, please use
%%
 \bibliographystyle{elsarticle-num} 
 \bibliography{cas-refs}

%% else use the following coding to input the bibitems directly in the
%% TeX file.

% \begin{thebibliography}{00}

% %% \bibitem{label}
% %% Text of bibliographic item

% \bibitem{}
\appendix

\subsection{Map: Global Ease of Living Index}

\begin{figure*}[htbp]
    \centering
    \includegraphics[width=0.9\textwidth]{Images/1991EOLI.png}
    \begin{center}
    \caption{\textit{Global Distribution of Ease of Living Index for the Year 1991}}
    \end{center}
    \label{fig:eoli1991}
\end{figure*}

\begin{figure*}[htbp]
    \centering
    \includegraphics[width=0.9\textwidth]{Images/2000EOLI.png}
    \begin{center}
    \caption{\textit{Global Distribution of Ease of Living Index for the Year 2000}}
    \end{center}
    \label{fig:eoli2000}
\end{figure*}

\begin{figure*}[htbp]
    \centering
    \includegraphics[width=0.9\textwidth]{Images/2011EOLI.png}
    \begin{center}
    \caption{\textit{Global Distribution of Ease of Living Index for the Year 2011}}
    \end{center}
    \label{fig:eoli2011}
\end{figure*}

\begin{figure*}[htbp]
    \centering
    \includegraphics[width=0.9\textwidth]{Images/2021EOLI.png}
    \begin{center}
    \caption{\textit{Global Distribution of Ease of Living Index for the Year 2021}}
    \end{center}
    \label{fig:eoli2021}
\end{figure*}



Figure 17 (1991) and Figure 18 (2001) highlight significant disparities in living standards, with many countries in Africa, South Asia, and Latin America classified as Low or Medium Low. In contrast, regions such as  North America, Western Europe, and Australia are predominantly categorised as High, reflecting well-established infrastructure, strong governance, and superior quality of life in these developed areas.


Figure 20 (2021) shows notable improvements from Figure 19 (2011), particularly in East Asia, the Middle East, and parts of Eastern Europe, where countries have shifted from lower to higher categories. This progress underscores the impact of economic growth, improved governance, and infrastructure development. Nevertheless, regions such as sub-Saharan Africa and parts of South Asia continue to struggle with lower scores, highlighting persistent challenges in achieving widespread development. Furthermore, countries such as India and China can be seen improving throughout all 4 maps. Overall, comparison across different years, from 1991 through 2021, reveals both progress and persistent inequality. While many emerging economies have experienced advancements, developed regions still maintain a significant lead in living standards. These findings emphasise the ongoing need for targeted development strategies to reduce global disparities and promote sustainable improvements in living conditions worldwide.



\subsection{Source of data}



\begin{table*}[htbp] %This table needs more columns - as discussed and various data sources needed to be grouped into sub-indices 
\centering
\scriptsize
\begin{tabular}{|p{7cm}|p{8cm}|}
\hline
\textbf{Factors} & \textbf{Source} \\ \hline
Access to electricity & \url{https://data.worldbank.org/indicator/EG.ELC.ACCS.ZS} \\ \hline
Annual healthcare expenditure & \url{https://apps.who.int/nha/database} \\ \hline
Cost of Living Index & \url{https://www.numbeo.com/cost-of-living/rankings_by_country.jsp} \\ \hline
Crime Index & \url{https://www.numbeo.com/crime/rankings_by_country.jsp} \\ \hline
Corruption Perception Index & \url{https://www.transparency.org/en/cpi/2021} \\ \hline
Ease of Doing Business Index & \url{https://archive.doingbusiness.org/en/rankings} \\ \hline
Freedom of Speech & \url{https://worldbank.org} \\ \hline
GDP Based on PPP & \url{https://www.imf.org/external/datamapper/PPPSH@WEO/OEMDC/ADVEC/WEOWORLD?year=2024} \\ \hline
GDP per Capita & \url{https://www.imf.org/external/datamapper/NGDPDPC@WEO/OEMDC/ADVEC/WEOWORLD} \\ \hline
Global Terrorism Index & \url{https://prosperitydata360.worldbank.org/en/indicator/QOG+BD+voh_gti} \\ \hline
Healthcare Index & \url{https://www.numbeo.com/health-care/rankings_by_country.jsp} \\ \hline
Inflation Rate, Average Consumer Price & \url{https://www.imf.org/external/datamapper/PCPIPCH@WEO/WEOWORLD?year=2024} \\ \hline
Life expectancy & \url{https://data.worldbank.org/indicator/SP.DYN.LE00.IN} \\ \hline
MYSE index & \url{https://genderdata.worldbank.org/en/indicator/se-sch-life} \\ \hline
Population in Slums & \url{https://data.unhabitat.org/pages/housing-slums-and-informal-settlements} \\ \hline
Property Price Ratio to Income & \url{https://www.numbeo.com/property-investment/rankings_by_country.jsp} \\ \hline
Real GDP Growth & \url{https://www.imf.org/external/datamapper/NGDP_RPCH@WEO/WEOWORLD?year=2024} \\ \hline
Rule of Law Index & \url{https://worldjusticeproject.org/rule-of-law-index/} \\ \hline
Unemployment Rate (ILO data) & \url{https://ilostat.ilo.org/data/#} \\ \hline
Unemployment Rate (World Bank data) & \url{https://data.worldbank.org/indicator/SL.UEM.TOTL.ZS} \\ \hline
\end{tabular}
\caption{Data Sources for Global Ease of Living Index}
\label{tab:data_sources}
\end{table*}

%\subsection{Kernel Density Plots}

%\begin{figure*}[htbp!]
%\centering
%\includegraphics[width=0.8\textwidth]{Images/KDE_Plots_Eonomic_Index.png}
%\caption{KDE Plots for Economic Index}
%\label{fig:kde_plots}
%%\end{figure*}


%\begin{figure*}[htbp!]
%\centering
%\includegraphics[width=0.8\textwidth]{Images/KDE_Plots_QOLI_Index.png}
%\caption{KDE Plots for Quality of Life Index}
%\label{fig:kde_plots}
%\end{figure*}

%\begin{figure*}[htbp!]
%\centering
%\includegraphics[width=0.8\textwidth]{Images/KDE_Plots_Sustainability_Index.png}
%\caption{KDE Plots for Sustanability Index}
%\label{fig:kde_plots}
%\end{figure*}
 
% \end{thebibliography}
\end{document}
\endinput
%%
%% End of file `elsarticle-template-num.tex'.
