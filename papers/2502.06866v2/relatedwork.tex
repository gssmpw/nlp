\section{Related Work}
\subsection{Machine Learning for Data Imputation }
 

Machine learning has emerged as a powerful tool for addressing the challenge of missing data imputation, particularly in time series datasets. Traditional imputation methods often struggle with the complexities and dependencies inherent in such data \cite{chen2024deep}.  In recent years, Generative Adversarial Networks (GANs) \cite{GAN} have gained prominence due to their ability to model complex data distributions. An implementation of GAN, the Generative Adversarial Imputation Network (GAIN) has been shown to outperform conventional methods by effectively capturing the underlying data structure and generating plausible imputations for missing values in various datasets, including clinical and environmental data\cite{Dong_GAIN,Yang_GAIN}. Recent developments in deep learning have introduced models such as Bidirectional Recurrent Imputation for Time Series (BRITS) \cite{BRITS}, which specifically address the temporal dependencies in time series data. BRITS employ a recurrent neural network (RNNs) \cite{yu2019review} architecture to handle multiple correlated missing values and adaptively update imputed values.

In addition to GANs, other machine learning models have also been explored for imputation tasks. For example, Random Forests have been combined with generative models \cite{GAN_RF} to enhance the robustness of imputations, particularly in scenarios with large gaps in data. 

%MICE: Azur, M. J., Stuart, E. A., Frangakis, C., & Leaf, P. J. (2011). Multiple imputation by chained equations: what is it and how does it work?. International journal of methods in psychiatric research, 20(1), 40-49.

%CATSI :Yin, K., Feng, L., & Cheung, W. K. (2020). Context-aware time series imputation for multi-analyte clinical data. Journal of Healthcare Informatics Research, 4, 411-426.

%See section  2.1 here and write a paragraph citing above : https://arxiv.org/pdf/2410.01847

The data can be multifaceted, representing complex interactions between economic, institutional, social, and environmental dimensions. EOLI exhibits intricate temporal dynamics due to the diverse nature of factors like governance, public health, infrastructure, and quality of life indicators, each influenced by various external and internal processes over time. Due to such nature of our data, we require more sophisticated approaches to accurately capture and model these complexities, ensuring that the evolving trends in living standards are well-represented and analysed. Multiple Imputation by Chained Equations (MICE) is a widely adopted method for handling missing data in statistical analyses\cite{MICE_imputation}. This method allows us to account for the uncertainty associated with missing data by generating several plausible imputed datasets, which can then be analysed separately and combined to produce estimates that reflect this uncertainty.  MICE works under the assumption that data is missing completely at random \cite{little1988test}, this assumption may not be entirely appropriate for EoLI as data can be missing due to underdeveloped data collection mechanisms in developing countries, due to variables related to governance such as crime or public safety may be systematically under-reported in regions where governments are less transparent and due to inconsistencies in data reporting due to several factors such as wars etc.    

%xxxx

 
A recent study proposed a variational Bayesian deep learning framework for medical data imputation \cite{kulkarniChandra2024bayes}, allowing for a more nuanced understanding of the data's structure and dependencies using CATSI. This approach not only improves the accuracy of the imputations but also enhances the overall predictive performance of the models used for forecasting groundwater levels. Furthermore, the use of Bayesian inference such as MCMC (Markov Chain Mote Carlo) sampling has also been prominent in data imputation \cite{takahashi2017statistical}. Richardson et al \cite{richardson2020mcflow}  presented a novel framework designed for data imputation that leverages the capabilities of Monte Carlo sampling and normalizing flow generative models. They proposed an innovative iterative learning scheme that alternates between updating the density estimates of the data and imputing the missing values, thereby enhancing the robustness of the imputation process. Chen et al.  \cite{chen2024deep}  utilised MCMC sampling via a linear model for imputing data for groundwater modelling via deep learning models. 

Furthermore,  Denoising Autoencoder models \cite{Chen_Shi} have shown promise in learning latent representations of data, which has been leveraged to impute missing values effectively \cite{Lall_Robin}. Gondara and Wang \cite{Gondara_Wang} introduced MIDA, to leverage  DAEs to perform multiple imputations of missing data by reconstructing clean outputs from noisy inputs, which is particularly relevant for missing data scenarios where the absence of values can be viewed as a form of noise. Similarly, Pereira et al. \cite{Pereira_Santos} explored the potential of DAEs for missing data imputation, emphasising their capacity to learn from corrupted data. 
The flexibility of deep learning models allows them to adapt to different types of data and missing patterns, providing a significant advantage over traditional statistical methods.

\subsection{Machine learning for Index creation}
 

Machine learning has become an invaluable tool in developing and refining various indices that measure socio-economic factors \cite{keys2021machine}. Machine learning framework for  Doing Business Index \cite{DBI}  involves clustering countries and applying a multivariate Composite I-distance Indicator (CIDI) to derive data-driven weights for the index components. Furthermore, the application of machine learning techniques extends to other indices that assess economic performance and quality of life. For instance, a study demonstrated how machine learning can be utilised to predict stock market movements and stock price indices by employing trend deterministic data preparation methods\cite{Stock_Index}. This predictive capability is crucial for investors and policymakers alike, as it enables them to make informed decisions based on anticipated market trends. Machine learning techniques such as PCA have been employed to distil multiple indicators into a single composite score, allowing for a more nuanced understanding of different factors \cite{Reddy_reduction}.  PCA not only aids in reducing dimensionality but also enhances the interpretability of the data by highlighting the most significant variables \cite{Hiles_Dutcher}. The adaptability of PCA across various data types and its robustness against noise make it a preferred choice across fields, including social sciences\cite{Belay_Wondimu} and health studies\cite{Nowroozi_Roshani}. 
 

 Tiwari et al.  \cite{Abtahi} presented the COVID-19 Vulnerability Index (C19VI), which integrated diverse data from public health databases, including demographic statistics, and socioeconomic indicators. The study used machine learning models, specifically Random Forest and Support Vector Machines (SVM), for the identification of regions vulnerable to COVID-19. Furthermore,   Kheirati and Khan \cite{Khan_Ali} developed a Pavement Condition Index (PCI) using data from pavement condition surveys, incorporating visual inspections and surface distress measurements. The study applied machine learning models such as Decision Trees and Neural Networks, employing feature selection to enhance model accuracy and support effective maintenance planning. Keys et al. \cite{Isa} estimated the Human Footprint Index using ensemble learning techniques such as Gradient Boosting and Random Forests. The study integrated datasets on land use, population density, and ecological factors to capture the intricate relationships between human activities and environmental impacts, validated through spatial analysis techniques. 

%In general, these studies demonstrate that machine learning can advance public health, infrastructure maintenance, and environmental sustainability by offering precise, data-driven insights.  


\subsection{Background on Related Indexes}

\subsubsection{Quality of Life Index}

%Blomquist in his study stated that life is good when the quality of life is high \cite{QOLI_1}. 

A number of researchers rank quality of life based on compensating differentials in labour, housing, and consumption markets. The Index of Urban Quality of Life is an example that measures the value of local amenities that vary from one urban area to another. These amenities are attractive features of locations, such as sunny, smog-free days, safety from violent crime, and well-staffed, effective schools. This index assesses the monetary value of the set of amenities that households gain by living and working in a particular area \cite{rosen1979wage}. In this respect, observation can be drawn about consumers' locational choices based on their evaluation of an area's amenities, which ultimately determine their overall well-being. Similarly, the Quality of Life Index (QoLI) %\cite{}   
serves as a pivotal instrument in assessing the multidimensional aspects of well-being across various populations, particularly in the context of health-related quality-of-life assessments. QoLI provides a structured framework for evaluating how different aspects of life, beyond economic indicators, contribute to the overall quality of living in a given region. Building upon this concept, Numbeo \footnote{Numbeo website: \url{numbeo.com}} has developed its Quality of Life Index \cite{numbeo_qol}, which incorporates several critical factors. These include the "cost of living" and "purchasing power," which are essential for evaluating economic well-being, as they indicate the affordability of goods and services within a region. Numbeo includes the affordability in housing, pollution, crime rate, health system, and traffic in their index. 
 
\subsubsection{World Happiness Index}

The World Happiness Index (WHI) \cite{carlsen2018happiness} evaluates individuals' subjective well-being across different countries, reflecting their overall happiness and life satisfaction. This index has gained prominence as a tool for understanding the factors that contribute to happiness on a global scale, transcending traditional economic indicators such as Gross Domestic Product (GDP). The WHI is based on survey data that captures individuals' perceptions of their lives, encompassing various dimensions such as emotional well-being, social support, and economic stability. The WHI employs five key indicators including life ladders, social support, freedom to make life choices, generosity, and perceptions of corruption to create a comprehensive picture of happiness across different countries.  
In addition to these primary indicators, the WHI also considers macroeconomic factors such as GDP per capita, which can provide context for understanding happiness levels. Over the years, there has been a significant backlash in the media about the biases in data collection for this index, which seem to penalise developing countries based on their GDP \cite{GDP_Bias} and cultural measurement bias \cite{Cultural_bias}. For instance, a war-hit country such as Ukraine was ranked to be happier than India in 2023 \cite{newindianexpress2023happiness}. Such indexes are based on surveys that may not be properly sampled in large countries such as India and influenced by political narratives. In the study conducted by Bond and Lang, they critically examine the validity and reliability of happiness scales, arguing that the current methodologies used in happiness research may not accurately capture the constructs they intend to measure\cite{Bond_Lang}. They highlight that happiness scales often lack a solid theoretical foundation and can yield results that are reversible, meaning that the conclusions drawn from these scales can vary significantly depending on the context or the specific outcomes being measured.

\subsubsection{Ease of Living Index: India}


As mentioned earlier \cite{mohua2018ease}. the index uses a mix of data from government sources and citizen perception surveys, gathering over 60,000 responses to gauge satisfaction with urban services. The index employs a weight of 45\% to the physical pillar, 25\% each to the institutional and social pillars, and 5\% to the economic pillar and the core indicators are assigned a higher weight (70\%) than supporting indicators (30\%). The ranking system follows a dimensional index methodology, where cities are evaluated relative to national or international benchmarks where available, or the best-performing city is used as a benchmark.

%discuss these studies

%Kumar, S. (2023). Review of Performance Indicators of Smart Cities in India–Ease of Living Index: a case of Lucknow Smart City. Sustainability, Agri, Food and Environmental Research, 11.

%Jha, S., & Kumar, M. (2020). Augmenting ease of living in India: A critique of union budget 2020-21. Journal of Politics & Governance, 8(2).

%Puri, P. (22). Ease of Living Index and the Master Plan of Delhi 2041. Horizon J. Hum. Soc. Sci. Res, 4(1), 37-50.

Several studies have highlighted the critical role of the Ease of Living Index (EoLI) in shaping urban development and governance in India. Kumar et al.  \cite{Kumar} focused on Lucknow Smart City, demonstrating how EoLI is used to evaluate the impact of smart city initiatives, emphasising technology and citizen engagement as key factors for success. The study highlights the integration of ICT in urban management\cite{Renukappa} and stresses the importance of inclusive growth to address socio-economic inequalities. Similarly, Jha and Kumar (2020) critique the Union Budget 2020-21, calling for a strategic approach to fiscal policies that prioritise health, education, and social welfare to improve the ease of living \cite{jha2020augmenting}. They advocate for budget alignment with smart city goals, stressing investments in infrastructure and gender-responsive budgeting to ensure equity and inclusivity \cite{Tandon}.

Puri (2022) examines the EoLI’s relevance to the Master Plan of Delhi 2041, arguing for its integration to guide sustainable urban planning \cite{Puri}. The study identifies gaps in existing policies and stresses the importance of aligning urban development strategies with EoLI metrics to address inclusivity and accessibility across various dimensions such as transportation, health, education, and environmental sustainability \cite{Puri}. Puri emphasises a participatory approach, incorporating citizen feedback and continuous monitoring to ensure data-driven governance. Collectively, these studies underline the EoLI’s significance in informing urban policies, advocating for a holistic approach that combines technology, strategic budgeting, and citizen engagement to enhance the quality of life in Indian cities.

\subsubsection{Gini Index}

In economics, the Gini coefficient, also referred to as the Gini index, is a widely used metric for assessing statistical dispersion \cite{lerman1984note}. It is used primarily in the context of income, wealth, or consumption inequality within a population or social group. This measure was introduced by the Italian statistician and sociologist Corrado Gini \cite{Ceriani2012} to quantify inequality by analysing the distribution of variables, such as income levels. A Gini coefficient of 0 indicates complete equality, where all individuals have identical income or wealth, whereas a coefficient of 1 (or 100\%) signifies extreme inequality, where one individual possesses all the income or wealth, and the rest have none. This metric is instrumental in understanding the degree of inequality within a given group or nation\cite{WorldBank_Gini}. The Gini coefficient is calculated using the Lorenz curve \cite{gastwirth1972estimation}, which represents the cumulative share of income or wealth of a population.  

\subsubsection{Human Development Index}

The Human Development Index (HDI) is designed to measure the average achievement in three core dimensions of human development: longevity and health, education, and standard of living. The HDI is calculated as the geometric mean of normalised indices for each of these dimensions. Specifically, the health component is evaluated by life expectancy at birth, the education component is assessed through the mean years of schooling for adults (25 years and older) and the expected years of schooling for children entering school. The standard of living is measured using gross national income (GNI) per capita, where a logarithmic scale is applied to account for the diminishing returns of higher income levels. The three dimensions are combined into a single composite index through the geometric mean. The HDI can be a valuable tool in evaluating national policies by highlighting how countries with similar GNI per capita can achieve varying levels of human development \cite{ivanova1999assessment}. This discrepancy can help spark discussions on the effectiveness of government priorities and policies. However, the HDI is limited in scope and does not encompass the full range of human development aspects, and other forms of measures are being explored \cite{comim2016beyond}. It does not account for factors like inequality, poverty, human security, or empowerment.