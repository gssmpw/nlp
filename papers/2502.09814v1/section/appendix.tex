\newpage
\section{\dataset{} Dataset}
\label{app:dataset}

\subsection{Statistics of Corpus and Slot Entities}
\label{app:stat}
\begin{table}[!hbt]
  \centering
  \resizebox{0.5\textwidth}{!}{%
  \begin{tabular}{cc|ccc|cccccc}
  \toprule[1pt]
    \multicolumn{2}{c|}{\textbf{Language}} & \multicolumn{3}{c|}{\textbf{Corpus Statistics in Token}} & \multicolumn{6}{c}{\textbf{Slot Entities}} \\
    Code & Name & Total & Avg. & Unique & Total & Avg. & Unique & Un. Fleiss'$\kappa$ & Fleiss'$\kappa$ & $\delta$ \\ \midrule
  \lang{amh} & Amharic & 24233 & 7.573 & 5270 & 10748 & 3.36 & 33 & 0.836 & 0.933 & +0.096 \\
  \lang{ewe} & Ewe & 33210 & 10.378 & 4422 & 11563 & 3.61 & 34 & 0.854 & 1.000 & +0.146 \\
  \lang{hau} & Hausa & 32330 & 10.103 & 1896 & 11792 & 3.69 & 33 & 0.863 & 0.996 & +0.133 \\
  \lang{ibo} & Igbo & 35036 & 10.928 & 3860 & 12639 & 3.94 & 33 & 0.798 & 0.973 & +0.175 \\
  \lang{kin} & Kinyarwanda & 30216 & 9.443 & 6112 & 10753 & 3.36 & 34 & 0.712 & 0.959 & +0.247 \\
  \lang{lin} & Lingala & 29571 & 9.241 & 2672 & 11400 & 3.56 & 33 & 0.798 & 0.990 & +0.192 \\
  \lang{lug} & Luganda & 33368 & 10.418 & 6589 & 12262 & 3.83 & 33 & 0.864 & 0.990 & +0.126 \\
  \lang{orm} & Oromo & 29429 & 9.197 & 5706 & 12570 & 3.93 & 33 & 0.844 & 0.992 & +0.148 \\
  \lang{sna} & Shona & 32901 & 10.282 & 8206 & 15779 & 4.93 & 33 & 0.934 & 0.976 & +0.042 \\
  \lang{sot} & Sotho & 29515 & 9.223 & 3323 & 6699 & 2.09 & 34 & 0.670 & 0.997 & +0.327 \\
  \lang{swa} & Swahili & 38822 & 12.132 & 4603 & 14750 & 4.61 & 34 & 0.864 & 0.985 & +0.121 \\
  \lang{twi} & Twi & 44303 & 13.845 & 4775 & 14881 & 4.65 & 34 & 0.913 & 0.986 & +0.074 \\
  \lang{wol} & Wolof & 37120 & 11.600 & 3460 & 11265 & 3.52 & 33 & 0.726 & 0.941 & +0.215 \\
  \lang{xho} & Xhosa & 26118 & 8.162 & 5086 & 12673 & 3.96 & 33 & 0.804 & 0.936 & +0.132 \\
  \lang{yor} & Yoruba & 43319 & 13.537 & 3103 & 13886 & 4.34 & 34 & 0.847 & 0.988 & +0.141 \\
  \lang{zul} & Zulu & 26496 & 8.285 & 7742 & 12330 & 3.86 & 34 & 0.618 & 0.912 & +0.294 \\
  \lang{eng} & English & 20266 & 10.861 & 3097 & -- & -- & -- & -- & -- & -- \\ \bottomrule[1pt]
  \end{tabular}
  }
  \caption{Statistics of the \dataset{} dataset across 17 languages, including corpus statistics (token counts and distributions) and slot entity analysis (entity counts, averages, and inter-annotator agreement measures) with \textbf{unmerged} slot types.}
  \label{tab:dataset-stats}
\end{table}


\subsection{Categories of Intent Detection}
\label{sec:intent-label}
The following are the intent labels used in the \dataset{} dataset. These are a total of 40 categories across 5 domains (Banking, Kitchen and Dining, Travel, Utility, and Home).

\begin{table}[h!]
\centering
\resizebox{\columnwidth}{!}{%
\begin{tabular}{c|l}
\toprule
\textbf{Domain} & \textbf{Intent} \\ \midrule
 & freeze\_account, pin\_change, pay\_bill, interest\_rate, \\ Banking &  min\_payment, bill\_balance, balance, spending\_history, \\ &  transactions, transfer \\ \hline % 11
Kitchen & food\_last, confirm\_reservation, ingredients\_list, cook\_time, \\ 
and & restaurant\_reviews, meal\_suggestion, restaurant\_suggestion, \\ 
Dining & restaurant\_reservation, cancel\_reservation, recipe \\ \hline % 10
Home & play\_music, calendar\_update, update\_playlist, \\ &  shopping\_list\_update \\ \hline % 4
 & plug\_type, travel\_notification, translate, international\_visa, \\ Travel &  exchange\_rate, travel\_suggestion, book\_flight, book\_hotel, \\ & car\_rental \\ \hline % 9
Utility & weather, alarm, share\_location, make\_call, time, text \\ \bottomrule % 6
\end{tabular}
}
\caption{Grouped intents categories by five domains.}
\label{tab:categorized-intents}
\end{table}


\subsection{Categories of Slot Filling}
\label{sec:slot-label}
\autoref{tab:slot-types} shows the original slot types and their final status after merging similar or low-frequency types during preprocessing. The ``Original Slot Type''are used during the dataset annotation phase, which contained 34 slot types.
After merging similar or low-frequency types during data preprocessing in Section \ref{sec:label-merge}, it was reduced to 23 distinct slot types as shown in the ``Final Merged Type'' column.

\begin{table}[h!]
\centering
\resizebox{0.9\columnwidth}{!}{%
\begin{tabular}{l|l|l}
\toprule
\textbf{Original Slot Type} & \textbf{Status} & \textbf{Final Merged Type} \\ \midrule
account type & kept & account type \\
artist name & kept & artist name \\
bank name & kept & bank name \\
bill type & kept & bill type \\
calendar event & kept & calendar event \\
country & kept & country \\
currency & kept & currency \\
date & kept & date \\
hotel name & kept & hotel name \\
language name & kept & language name \\
meal period & kept & meal period \\
money & kept & money \\
music genre & kept & music genre \\
number & kept & number \\
payment company & kept & payment company \\
personal name & kept & personal name \\
place name & kept & place name \\
restaurant name & kept & restaurant name \\
shopping item & kept & shopping item \\
song name & kept & song name \\
time & kept & time \\ \midrule
airline & deleted & -- \\
airport name & deleted & -- \\
car rental company & deleted & -- \\
car type & deleted & -- \\
continent & deleted & -- \\
nationality & deleted & -- \\
plug type & deleted & -- \\
supermarket name & deleted & -- \\
timezone & deleted & -- \\ \midrule
city name & merged & \multirow{2}{*}{city or province} \\
state or province & merged & \\
dish name & merged & \multirow{2}{*}{dish or food} \\
food item & merged & \\ \bottomrule
\end{tabular}
}
\caption{Original and final slot types in the \dataset{} dataset. ``kept'' indicates the slot type was retained, while ``deleted'' indicates the slot type was removed. ``merged'' indicates the slot type was combined with another similar type.}
\label{tab:slot-types}
\end{table}

\section{Experiments Setup}
\label{sec:exp-more}

\subsection{Training Configuration}
To ensure equitable comparison across architectures, we implement a standardized training protocol. All SLMs are finetuned using the AdamW optimizer in 20 epochs with a learning rate schedule incorporating 10\% linear warmup steps followed by linear decay. Early stopping (patience=5) is adopted, and the dev set performance is monitored. Learning rates are carefully calibrated for each architecture type as detailed in \autoref{tab:lr-choice}. Our empirical investigations demonstrate that slot filling tasks consistently require higher learning rates compared to intent detection tasks specifically, encoder-only models utilize $1\times10^{-5}$/$3\times10^{-5}$ for intent detection/slot filling respectively, while encoder-decoder architectures necessitate elevated rates of $5\times10^{-5}$/$1\times10^{-4}$.

Given the computational constraints of finetuning LLMs, Fully Supervised Fine-Tuning (FSFT) is exclusively performed on the Gemma 2 9B model with 5 epochs. Based on established SFT practices and task-specific requirements, we use learning rates of $2\times10^{-5}$ and $2.5\times10^{-5}$ for intent detection and slot filling respectively. Training data is constructed from the combined train splits of \dataset{} dataset across all 17 languages, with prompts randomly sampled from a pool of 5 predefined templates.

All experiments are conducted using full precision (FP32) on NVIDIA H100/L40S GPUs with a consistent batch size of 32, achieved through gradient accumulation when necessary. 

\subsection{Learning Rate Choice}
\label{sec:lr-impact}
Before the final model training, we conducted a comprehensive analysis of learning rate variations to understand their effect on model performance across Intent Detection and Slot Filling tasks. This investigation helped determine optimal learning rates for different model architectures. \autoref{tab:lr-optimize} presents detailed results, extending the findings from \autoref{tab:lr-choice}.

\begin{table}[h]
\centering
\resizebox{0.5\textwidth}{!}{%
\begin{tabular}{l|rcc}
\toprule
Task & \multicolumn{1}{l}{Encoder Only} & \multicolumn{1}{l}{Encoder-Decoder} & \multicolumn{1}{l}{NLLB LLM2Vec} \\ \midrule
Intent Detection & $1\times10^{-5}$ & $5\times10^{-5}$ & $1\times10^{-4}$ \\
Slot Filling & $3\times10^{-5}$ & $1\times10^{-4}$ & $3\times10^{-4}$ \\
\bottomrule
\end{tabular}%
}
\caption{Selected learning rates for different architectures of both tasks of intent detection and slot filling.}
\label{tab:lr-choice}
\end{table}
\vspace{-1em}


\subsection{Language Coverage of Baselines}
\label{app:language}
The table below briefly introduces the baseline models along with the languages they were trained on in the \dataset{} dataset.

\begin{table}[h]
\centering
\resizebox{0.49\textwidth}{!}{%
\begin{tabular}{l|l}
\toprule
\textbf{Model} & \textbf{Languages} \\ \midrule
AfroXLMR Large (550M) & amh, hau, ibo, kin, orm, sna, sot, swa, xho, yor, zul \\ \midrule
AfroXLMR Large 76L (550M) & amh, ewe, hau, ibo, kin, lin, lug, orm, sna, sot, swa, twi, wol, xho, yor, zul \\ \midrule
XLM-RoBERTa Large (550M) & amh, orm, swa, xho \\ \midrule
AfriBERTa V2 Large (187M) & amh, hau, ibo, sna, sot, swa, xho, yor \\ \midrule
AfriTeVa V2 Large (1.2B) & amh, hau, ibo, sna, sot, swa, xho, yor \\ \midrule
mT5-Large (1.2B) & amh, hau, ibo, sot, swa, yor, zul \\ \bottomrule
\end{tabular}
}
\caption{Baseline models with their corresponding language coverage in \dataset{}.}
\label{tab:model_languages}
\end{table}

% \subsection{Overall Ranking of Baseline Models}
% in Mono-lingual Training
% \autoref{tab:rank} summarizes the overall ranking of baseline models in intent detection and slot filling tasks. The final ranking of each model is determined by averaging its ranks across all tasks.

% \begin{table}[h]
% \centering
% \resizebox{0.45\textwidth}{!}{%
% \begin{tabular}{l|cccc}
% \toprule
% \textbf{Model} & \begin{tabular}[c]{@{}c@{}}\textbf{Intent}\\\textbf{Detection}\end{tabular} & \begin{tabular}[c]{@{}c@{}}\textbf{Slot}\\\textbf{Filling}\end{tabular} & \textbf{Average} & \begin{tabular}[c]{@{}c@{}}\textbf{Final}\\\textbf{Rank}\end{tabular} \\ \midrule
% mT5-Large & 6& 7& 6.5& 7\\
% AfriTeVa V2 Large & 4& 3& 3.5& 4\\
% XLM-RoBERTa Large & 7& 4& 5.5& 6\\
% AfroXLMR-large & 3& 2& 2.5& 2\\
% \textbf{AfroXLMR-large 76L} & \textbf{1}& \textbf{1}& \textbf{1}& \textbf{1}\\
% AfriBERTa V2 Large & 5& 4& 4& 5\\
% NLLB LLM2Vec & 2& 4& 3& 3\\ \bottomrule
% \end{tabular}%
% }
% \caption{Model rankings based on intent detection and slot filling task performance.}
% \label{tab:rank}
% \end{table}


\begin{table*}[h]
\centering
\resizebox{\textwidth}{!}{%
\begin{tabular}{c|cl|cccccccc}
\toprule[1pt]
\multirow{2}{*}{\textbf{Task}} & \multirow{2}{*}{\begin{tabular}[c]{@{}c@{}}\textbf{Model}\\\textbf{Type}\end{tabular}} & \multirow{2}{*}{\textbf{Model}} & \multicolumn{8}{c}{\textbf{Learning Rate}} \\
  &  &  & $1\times10^{-5}$ & $2\times10^{-5}$ &$3\times10^{-5}$& $5\times10^{-5}$& $1\times10^{-4}$& $2\times10^{-4}$ & $3\times10^{-4}$ & $5\times10^{-4}$ \\ \midrule
\multirow{10}{*}{\begin{tabular}[c]{@{}c@{}}\textsc{Intent}\\\textsc{Detection}\end{tabular}} & \multirow{5}{*}{Encoder} & AfriBERTa V2 Large & 97.50 & 98.13 & 98.13 & 98.13 & 97.81 & 97.81 & 95.00 & 2.50 \\
  &  & AfroXLMR-large & 98.13 & 98.13 & 98.75 & 2.50 & 2.50 & 2.50 & 2.50 & 2.50 \\
  &  & AfroXLMR-large 76L & 98.75 & 98.75 & 99.06 & 98.75 & 2.50 & 2.50 & 2.50 & 2.50 \\
  &  & XLM-RoBERTa Large & 98.75 & 2.50 & 13.44 & 6.25 & 2.50 & 2.50 & 4.06 & 2.50 \\
  &  & \textit{Average} & \textbf{98.28} & 74.38 & 77.34 & 51.41 & 26.33 & 26.33 & 26.02 & 2.50 \\ \cline{2-11} 
  & \multirow{3}{*}{\begin{tabular}[c]{@{}c@{}}Encoder-\\Decoder\end{tabular}} & AfriTeVa V2 Large & 0.00 & 0.00 & 96.88 & 97.81 & 97.19 & 97.81 & 96.56 & 97.50 \\
  &  & mT5-Large & 0.00 & 95.31 & 95.94 & 97.50 & 97.50 & 97.50 & 98.13 & 97.50 \\
  &  & \textit{Average} & 0.00 & 47.66 & 96.41 & \textbf{97.66} & 97.34 & \textbf{97.66} & 97.34 & 97.50 \\ \cline{2-11} 
  & \multirow{2}{*}{Other} & NLLB LLM2Vec & 97.19 & 97.50 & 96.88 & 95.94 & 98.44 & 97.81 & 98.44 & 97.19 \\
  &  & \textit{Average} & 97.19 & 97.50 & 96.88 & 95.94 & \textbf{98.44} & 97.81 & \textbf{98.44} & 97.19 \\ \midrule
\multirow{10}{*}{{\begin{tabular}[c]{@{}c@{}}\textsc{Slot}\\\textsc{Filling}\end{tabular}}} & \multirow{5}{*}{Encoder} & AfriBERTa V2 Large & 86.12 & 89.70 & 90.21 & 90.74 & 91.22 & 90.45 & 88.24 & 0.00 \\
  &  & AfroXLMR-large & 89.95 & 90.13 & 91.04 & 89.87 & 0.00 & 0.00 & 0.00 & 0.00 \\
  &  & AfroXLMR-large 76L & 90.04 & 90.91 & 90.96 & 90.58 & 91.28 & 0.00 & 0.00 & 0.00 \\
  &  & XLM-RoBERTa Large & 88.55 & 89.72 & 91.63 & 89.52 & 88.10 & 0.00 & 0.00 & 0.00 \\
  &  & \textit{Average} & 88.67 & 90.11 & \textbf{90.96} & 90.18 & 67.65 & 22.61 & 22.06 & 0.00 \\ \cline{2-11}  % \cmidrule{2-11}
  & \multirow{3}{*}{\begin{tabular}[c]{@{}c@{}}Encoder-\\Decoder\end{tabular}} & AfriTeVa V2 Large & 39.07 & 83.47 & 83.47 & 89.63 & 90.51 & 81.59 & 88.11 & 88.44 \\
  &  & mT5-Large & 22.31 & 59.61 & 89.16 & 82.71 & 88.54 & 89.67 & 89.16 & 90.40 \\
  &  & \textit{Average} & 30.69 & 71.54 & 86.32 & 86.17 & \textbf{89.53} & 85.63 & 88.64 & 89.42 \\ \cline{2-11} 
  & \multirow{2}{*}{Other} & NLLB LLM2Vec & 81.13 & 84.84 & 85.33 & 85.69 & 85.57 & 84.44 & 87.02 & 86.12 \\
  &  & \textit{Average} & 81.13 & 84.84 & 85.33 & 85.69 & 85.57 & 84.44 & \textbf{87.02} & 86.12 \\ 
\bottomrule[1pt]
\end{tabular}%
}
\caption{Comparative analysis of model performance across different learning rates for Intent Detection and Slot Filling tasks. Results are shown for various model architectures including Encoder-only, Encoder-Decoder, and other approaches. Bold values indicate the best performance for each model type.}
\label{tab:lr-optimize}
\end{table*}


\subsection{Results of Multi-lingual Training}
\label{app:multi-lingual-results}
\begin{table*}[h]
\centering
\resizebox{\textwidth}{!}{%
\begin{tabular}{c|l|cccccccccccccccccc}
\toprule[1pt]
\multicolumn{1}{l|}{\textbf{Task}} & \textbf{Model} & \lang{amh} & \lang{ewe} & \lang{hau} & \lang{ibo} & \lang{kin} & \lang{lin} & \lang{lug} & \lang{orm} & \lang{sna} & \lang{sot} & \lang{swa} & \lang{twi} & \lang{wol} & \lang{xho} & \lang{yor} & \lang{zul} & \lang{eng} & \textbf{AVG} \\ \midrule
% \begin{tabular}[c]{@{}c@{}}Language\\ eff\end{tabular}
  \multirow{4}{*}{\begin{tabular}[c]{@{}c@{}}\textsc{Intent} \\ \textsc{Detection}\end{tabular}} 
  & AfroXLMR-large 76L& 96.0& 92.6& 99.2& 96.6& 87.7& 95.9& 92.3& 92.9& 96.5& 87.6& 97.8& 94.2& 97.1& 97.3& 97.9& 89.2& 89.0& \textbf{94.4$_{\pm 3.6}$} \\
  & AfroXLMR-large& 96.1& 90.3& 99.3& 96.5& 86.8& 94.2& 91.6& 92.2& 96.0& 87.1& 97.9& 91.6& 96.1& 96.9& 97.4& 88.6& 89.7& 93.7$_{\pm 3.9}$ \\
  & NLLB LLM2Vec& 95.8& 90.2& 98.7& 96.5& 86.2& 95.4& 92.6& 87.9& 96.9& 86.8& 97.3& 93.9& 95.5& 96.9& 97.2& 88.6& 89.1& 93.5$_{\pm 4.1}$ \\
  & AfriTeVa V2 Large& 94.6& 85.8& 99.2& 96.5& 87.3& 93.6& 90.8& 88.6& 95.9& 85.3& 98.0& 89.6& 94.4& 97.3& 97.6& 88.3& 89.7& 92.7$_{\pm 4.6}$ \\ \midrule
  \multirow{4}{*}{{\begin{tabular}[c]{@{}c@{}}\textsc{Slot} \\\textsc{Filling}\end{tabular}}} 
  & AfroXLMR-large 76L& 88.2& 87.0& 96.3& 84.0& 79.3& 90.3& 89.2& 87.2& 86.1& 80.4& 90.5& 90.3& 83.3& 91.8& 90.2& 83.3& 82.4& \textbf{87.3$_{\pm 4.4}$} \\
  & AfroXLMR-large& 87.9& 84.0& 96.4& 83.6& 80.4& 89.5& 88.4& 88.2& 87.0& 82.0& 91.5& 87.7& 81.9& 91.7& 90.4& 84.2& 82.8& 87.2$_{\pm 4.2}$ \\
  & NLLB LLM2Vec& 84.3& 82.0& 94.6& 80.3& 72.3& 86.9& 85.1& 81.9& 82.0& 77.2& 87.3& 85.8& 80.0& 90.4& 87.1& 79.9& 80.8& 83.6$_{\pm 5.2}$ \\
  & AfriTeVa V2 Large& 78.9& 72.6& 92.0& 80.0& 75.7& 85.3& 81.8& 76.0& 79.8& 77.0& 88.2& 81.7& 76.5& 86.7& 86.3& 66.7& 78.9& 80.3$_{\pm 6.3}$ \\ \bottomrule[1pt]
\end{tabular}%
}
\caption{Multilingual Training: 4 model performance on Intent Detection and Slot Filling tasks across languages.}
\label{tab:combined-languages-sft-result}
\end{table*}

We selected the 4 top-performing models from the in-language training phase and evaluated them on the \dataset{} test set, comparing the performance of the models when trained on individual languages and when trained on the combined dataset. The results are shown in \autoref{tab:combined-languages-sft-result}.

% \begin{figure*}[!htbp]
%   \label{fig:finetune-intent}
%   \centering
%   \includegraphics[width=\textwidth]{figures/finetune_slm_seqc.pdf}
%   \caption{Accuracy of Intent Detection task with top finetuned models (AfroXLMR-76L, AfroXLMR, NLLB-LLM2Vec, AfriTeVa V2). Solid colors indicate models trained and tested per language, while hatched patterns show models finetuned once on the combined 17 languages training dataset.}
% \end{figure*}

% \begin{figure*}[!htbp]
%   \label{fig:finetune-slot}
%   \centering
%   \includegraphics[width=\textwidth]{figures/finetune_slm_tokenc.pdf}
%   \caption{F1 score of Slot Filling task with top finetuned models (AfroXLMR-76L, AfroXLMR, NLLB-LLM2Vec, AfriTeVa V2). Solid colors indicate models trained and tested per language, while hatched patterns show models finetuned once on the combined 17 languages training dataset.}
% \end{figure*}
\subsection{Results of Cross-lingual Transfer}

This section provides additional commentary on \autoref{tab:afroxlm76l_intent_detection} which reports the cross-lingual transfer performance of AfroXLMR-76L on the Intent Detection task under different shot conditions. The table compares two datasets (CLINC and Injongo) in both their original in-language and translate-test settings. For each dataset, results are presented at multiple shot levels (e.g., 5, 10, 25, 50, and 100 shots), with the average performance and corresponding standard deviation indicated. Notably, the results illustrate how performance progressively improves as the number of shots increases, and how the transfer capability is affected by the linguistic diversity of the datasets.

\begin{table*}[htbp]
\centering
\resizebox{\textwidth}{!}{%
\begin{tabular}{c*{17}{c}}
\toprule
\textbf{\# of Shots} & \lang{amh} & \lang{ewe} & \lang{hau} & \lang{ibo} & \lang{kin} & \lang{lin} & \lang{lug} & \lang{orm} & \lang{sna} & \lang{sot} & \lang{swa} & \lang{twi} & \lang{wol} & \lang{xho} & \lang{yor} & \lang{zul} & \textbf{AVG} \\
\midrule
\multicolumn{18}{l}{\textbf{CLINC Dataset}} \\
5 & 3.1 & 2.7 & 2.5 & 2.6 & 2.5 & 3.2 & 2.6 & 2.4 & 2.9 & 2.9 & 3.6 & 1.9 & 2.8 & 2.3 & 2.7 & 3.0 & 2.7$_{\pm 0.5}$ \\
10 & 6.7 & 2.8 & 5.5 & 4.1 & 3.5 & 5.1 & 3.2 & 3.2 & 3.1 & 4.2 & 7.0 & 2.7 & 3.1 & 2.6 & 3.4 & 3.6 & 4.0$_{\pm1.0}$ \\
25 & 80.3 & 24.4 & 69.6 & 51.9 & 49.9 & 57.2 & 37.6 & 29.8 & 53.0 & 42.9 & 78.0 & 38.4 & 26.4 & 59.6 & 35.6 & 55.1 & 49.4$_{\pm9.4}$ \\
50 & 83.8 & 36.1 & 78.4 & 61.9 & 55.3 & 63.4 & 45.7 & 39.4 & 59.8 & 50.3 & 82.6 & 47.1 & 34.9 & 65.3 & 48.7 & 60.0 & 57.1$_{\pm8.4}$ \\
100 & 84.7 & 37.8 & 80.9 & 62.6 & 55.7 & 65.2 & 47.2 & 39.6 & 63.2 & 50.9 & 85.2 & 48.8 & 36.0 & 66.7 & 52.5 & 62.1 & 58.7$_{\pm8.5}$ \\
\midrule
\multicolumn{18}{l}{\textbf{Injongo Dataset}} \\
5 & 3.6 & 2.6 & 3.3 & 2.6 & 3.1 & 3.0 & 2.8 & 2.3 & 3.2 & 3.3 & 3.9 & 2.2 & 2.8 & 2.8 & 2.5 & 2.8 & 2.9$_{\pm0.5}$ \\
10 & 29.7 & 7.3 & 27.0 & 15.3 & 13.9 & 19.4 & 9.3 & 6.3 & 13.4 & 16.2 & 36.7 & 9.4 & 7.1 & 16.5 & 10.0 & 20.3 & 16.1$_{\pm5.1}$ \\
25 & 76.1 & 24.2 & 70.2 & 53.9 & 50.1 & 55.2 & 41.2 & 28.4 & 53.9 & 45.1 & 78.5 & 41.1 & 27.8 & 59.7 & 38.8 & 55.1 & 50.0$_{\pm8.9}$ \\
\midrule
\multicolumn{18}{l}{\textbf{CLINC Dataset (translate  test)}} \\
5 & 4.2 & 3.6 & 3.5 & 4.0 & 3.9 & 4.0 & 3.6 & 3.5 & 3.7 & 3.8 & 4.3 & 3.2 & 3.6 & 3.9 & 3.6 & 4.0 & 3.8$_{\pm0.5}$ \\
10 & 14.9 & 11.2 & 15.9 & 13.7 & 11.9 & 15.3 & 11.0 & 4.8 & 10.2 & 12.4 & 13.8 & 9.4 & 10.2 & 12.4 & 10.3 & 12.5 & 11.9$_{\pm1.9}$ \\
25 & 83.2 & 58.8 & 85.7 & 78.3 & 67.6 & 76.7 & 71.4 & 32.1 & 80.2 & 67.4 & 84.6 & 67.8 & 56.8 & 82.8 & 75.3 & 71.0 & 71.2$_{\pm7.4}$ \\
50 & 86.2 & 61.7 & 89.9 & 81.9 & 69.1 & 78.8 & 74.2 & 33.2 & 82.7 & 71.7 & 85.8 & 70.1 & 58.1 & \textbf{86.7} & 79.2 & 74.4 & 74.0$_{\pm7.7}$ \\
100 & \textbf{86.7} & \textbf{62.5} & \textbf{91.0} & \textbf{83.3} & \textbf{69.5} & \textbf{80.5} & \textbf{75.4} & \textbf{34.4} & \textbf{84.9} & \textbf{72.2} & \textbf{87.7} & \textbf{71.1 }& \textbf{59.1} & 86.2 & \textbf{81.2} & \textbf{76.6} & \textbf{75.1}$_{\pm7.7}$ \\
\midrule
\multicolumn{18}{l}{\textbf{Injongo Dataset (translate test)}} \\
5 & 4.4 & 4.7 & 5.3 & 4.4 & 3.7 & 5.0 & 4.4 & 2.7 & 4.5 & 4.5 & 4.7 & 3.8 & 4.2 & 5.1 & 5.3 & 3.6 & 4.4$\pm$0.6 \\
10 & 45.4 & 32.2 & 52.0 & 44.7 & 39.7 & 43.9 & 42.8 & 16.3 & 44.2 & 38.6 & 50.6 & 37.1 & 32.7 & 48.0 & 45.3 & 40.8 & 40.9$\pm$4.9 \\
25 & 80.5 & 59.0 & 86.0 & 77.5 & 66.4 & 74.2 & 72.7 & 31.7 & 77.9 & 68.1 & 84.1 & 66.6 & 55.9 & 82.4 & 76.8 & 69.7 & 70.6$\pm$7.3 \\
\bottomrule
\end{tabular}%
}
\caption{AfroXLMR-76L Intent Detection performance under varying shot conditions.}
\label{tab:afroxlm76l_intent_detection}
\end{table*}


\subsection{Inference Setup of LLMs}

For closed‐source models (GPT-4o and Gemini 1.5 Pro), we utilize the API provided by the respective vendor for inference. For open‐source models, inference is performed using vLLM \cite{vllm}, except for Aya-101, where Text Generation Inference (TGI)\footnote{\href{https://huggingface.co/docs/text-generation-inference/index}{Text Generation Inference}} is employed.

% VLLM \cite{vllm} is used for inference in all experiments. The model is fine-tuned on the \dataset{} dataset for 5 epochs with a learning rate of 2e-5. The model is trained on the combined train splits of the \dataset{} dataset across all 17 languages. The model is evaluated on the test set of the \dataset{} dataset. The model is trained using the AdamW optimizer with a batch size of 32 and a learning rate schedule incorporating 10\% linear warmup steps followed by linear decay. The early stopping (patience=5) is adopted and monitors the dev set performance. The model is trained using full precision (FP32) on NVIDIA H100/L40S GPUs.


\subsection{Results of LLMs prompting}
% \caption{Zero-shot and few-shot performance comparison across languages on Intent Detection and Slot Filling tasks. For Intent Detection, shots refer to examples per domain (5 examples), per Intent Detection (1 shot), and 4 examples per Intent Detection (4 shots). For Slot Filling, shots refer to examples per domain (5 examples), per slot type (1 shot), and 4 examples per slot type (4 shots).}
% \label{tab:few-shot-results}
% \end{table*}
\label{app:llm-few-shot-results}
Across 5 LLMs, we evaluated the performance of zero-shot and few-shot learning on the Intent Detection and Slot Filling tasks. The complete results are presented in \autoref{tab:few-shot-results}. We only evaluate the performance of the models on the best prompt for each task. The 2nd prompt for Intent Detection and the 3rd prompt for Slot Filling are used for evaluation. 
% The complete prompt list can be found in the Appendix~\ref{app:prompts}.

\begin{table*}[htbp!]
\centering
\resizebox{\textwidth}{!}{%
\begin{tabular}{c|l|l|cccccccccccccccccc}
\toprule
\textbf{Task} & \textbf{Model} & \textbf{Setup} & \lang{eng} & \lang{amh} & \lang{ewe} & \lang{hau} & \lang{ibo} & \lang{kin} & \lang{lin} & \lang{lug} & \lang{orm} & \lang{sna} & \lang{sot} & \lang{swa} & \lang{twi} & \lang{wol} & \lang{xho} & \lang{yor} & \lang{zul} & \textbf{Avg} \\ \midrule
\multirow{20}{*}{\begin{tabular}[c]{@{}c@{}} \textsc{Intent} \\ \textsc{Detection}\end{tabular}} & \multirow{4}{*}{GPT-4o} & 0 shot & 81.2 & 76.2 & 14.5 & 80.8 & 71.6 & 64.4 & 55.9 & 68.1 & 58.6 & 75.6 & 58.6 & 85.2 & 58.3 & 43.1 & 78.6 & 76.1 & 70.3 & 64.7$_{\pm 16.8}$ \\
& & 5 examples & 81.8 & 75.9 & 21.2 & 85.3 & 76.6 & 69.8 & 74.8 & 74.7 & 69.1 & 80.8 & 69.2 & 82.2 & 68.9 & 63.3 & 82.0 & 78.4 & 72.7 & 71.6$_{\pm 14.2}$ \\
& & 1 shot & 82.6 & 83.0 & 37.8 & 88.8 & 82.0 & 76.2 & 82.2 & 83.3 & 78.6 & 85.0 & 71.2 & 85.8 & 76.7 & 72.6 & 85.2 & 82.0 & 77.3 & 78.0$_{\pm 11.4}$ \\
& & 4 shots & 83.3 & 85.2 & 64.5 & \textbf{91.7} & \textbf{86.7} & \textbf{79.4} & 85.0 & \textbf{85.3} & \textbf{83.3} & \textbf{89.7} & \textbf{75.0} & 87.8 & 82.2 & 81.1 & \textbf{87.2} & \textbf{87.7} & 79.2 & 83.2$_{\pm 6.4}$ \\ \cmidrule{2-21}
& \multirow{4}{*}{Gemini 1.5 Pro} & 0 shot & 82.3 & 80.2 & 26.9 & 78.8 & 69.5 & 66.2 & 58.1 & 64.4 & 42.8 & 71.4 & 55.5 & 85.0 & 50.5 & 27.9 & 79.4 & 71.6 & 71.9 & 62.5$_{\pm 17.2}$ \\
& & 5 examples & 81.8 & 81.1 & 52.3 & 86.4 & 77.3 & 71.4 & 76.4 & 76.1 & 67.0 & 80.2 & 69.2 & 85.5 & 71.2 & 49.1 & 81.1 & 77.3 & 72.8 & 73.4$_{\pm 10.1}$ \\
& & 1 shot & 81.2 & 85.8 & 69.2 & 90.2 & 80.3 & 75.6 & 82.5 & 82.8 & 74.7 & 85.2 & 73.3 & 87.3 & 80.8 & 71.2 & 84.7 & 84.5 & 77.5 & 80.3$_{\pm 5.9}$ \\
& & 4 shots & \textbf{83.8} & \textbf{85.9} & \textbf{78.3} & 90.9 & 86.2 & 79.1 & \textbf{85.6} & 83.6 & 78.0 & 87.7 & 73.6 & \textbf{88.9} & \textbf{84.2} & \textbf{81.6} & 86.9 & 85.6 & \textbf{80.3} & \textbf{83.5$_{\pm 4.5}$} \\ \cmidrule{2-21}
& \multirow{4}{*}{Gemma 2 IT 9B} & 0 shot & 78.9 & 51.4 & 7.0 & 43.1 & 33.4 & 26.1 & 24.5 & 25.6 & 8.6 & 30.5 & 21.2 & 73.1 & 23.1 & 14.9 & 44.4 & 33.8 & 40.5 & 31.3$_{\pm 16.2}$ \\
& & 5 examples & 79.1 & 58.3 & 13.4 & 71.2 & 58.9 & 44.1 & 41.6 & 40.0 & 18.3 & 54.1 & 41.4 & 79.1 & 39.2 & 29.8 & 61.1 & 48.4 & 53.3 & 47.0$_{\pm 17.0}$ \\
& & 1 shot & 78.9 & 54.7 & 15.5 & 76.7 & 58.8 & 43.3 & 50.3 & 42.2 & 28.4 & 54.7 & 43.1 & 82.0 & 46.7 & 30.3 & 68.3 & 60.5 & 58.8 & 50.9$_{\pm 16.9}$ \\
& & 4 shots & 78.5 & 68.6 & 44.8 & 84.7 & 73.4 & 60.9 & 72.0 & 70.0 & 55.0 & 75.2 & 60.3 & 82.8 & 66.4 & 66.9 & 77.0 & 67.7 & 67.7 & 68.3$_{\pm 9.7}$ \\ \cmidrule{2-21}
& \multirow{4}{*}{Gemma 2 IT 27B} & 0 shot & 80.2 & 48.4 & 6.6 & 49.8 & 40.2 & 27.8 & 31.6 & 28.6 & 6.4 & 38.0 & 27.3 & 77.5 & 23.0 & 18.7 & 51.7 & 35.5 & 47.3 & 34.9$_{\pm 17.5}$ \\
& & 5 examples & 78.3 & 59.5 & 13.6 & 80.0 & 70.0 & 52.2 & 55.5 & 52.5 & 22.5 & 68.3 & 55.3 & 84.7 & 54.5 & 37.0 & 73.9 & 63.3 & 64.2 & 56.7$_{\pm 18.5}$ \\
& & 1 shot & 80.5 & 61.6 & 26.2 & 85.2 & 74.2 & 59.8 & 68.3 & 65.9 & 49.2 & 77.8 & 59.5 & 86.7 & 63.7 & 58.4 & 77.0 & 75.9 & 68.8 & 66.2$_{\pm 14.3}$ \\
& & 4 shots & 81.5 & 76.4 & 57.7 & 87.7 & 80.6 & 65.9 & 78.1 & 74.1 & 65.3 & 83.4 & 68.1 & 85.3 & 74.7 & 70.9 & 81.2 & 80.2 & 75.0 & 75.3$_{\pm 7.9}$ \\ \cmidrule{2-21}
& \multirow{4}{*}{Llama 3.3 70B} & 0 shot & 80.9 & 57.3 & 10.5 & 53.3 & 53.0 & 35.5 & 38.0 & 39.7 & 13.8 & 32.8 & 31.7 & 81.4 & 31.7 & 21.0 & 44.7 & 41.6 & 44.8 & 39.4$_{\pm 16.8}$ \\
& & 5 examples & 82.3 & 56.6 & 12.0 & 79.2 & 69.1 & 51.1 & 45.6 & 48.3 & 28.9 & 51.1 & 47.7 & 84.2 & 50.2 & 31.0 & 62.2 & 63.7 & 58.0 & 52.4$_{\pm 17.7}$ \\
& & 1 shot & 82.2 & 75.3 & 37.0 & 84.7 & 77.8 & 59.2 & 59.7 & 72.8 & 54.1 & 72.3 & 61.3 & 86.7 & 69.8 & 60.3 & 76.4 & 77.8 & 70.3 & 68.5$_{\pm 12.2}$ \\
& & 4 shots & 83.3 & 81.4 & 57.0 & 88.1 & 83.6 & 69.7 & 75.8 & 77.7 & 65.8 & 81.4 & 68.6 & 88.0 & 77.0 & 74.5 & 80.3 & 84.8 & 74.8 & 76.8$_{\pm 8.1}$ \\ \midrule \midrule
\multirow{20}{*}{{\begin{tabular}[c]{@{}c@{}}\textsc{Slot} \\\textsc{Filling}\end{tabular}}} & \multirow{4}{*}{GPT-4o} & 0 shot & 55.1 & 23.8 & 20.3 & 38.8 & 38.9 & 37.3 & 33.6 & 37.9 & 12.7 & 41.4 & 42.6 & 44.5 & 39.0 & 41.3 & 9.1 & 41.7 & 36.9 & 33.7$_{\pm 10.7}$ \\
& & 5 examples & 63.9 & 39.3 & 43.5 & 60.8 & 59.8 & 46.8 & 61.0 & 51.0 & 36.6 & 60.6 & 58.8 & 62.4 & 61.8 & 58.5 & 50.7 & 59.4 & 40.8 & 53.3$_{\pm 8.8}$ \\
& & 1 shot & 71.3 & 53.3 & 50.0 & 66.2 & 59.9 & 54.3 & 63.3 & 60.3 & 54.9 & 64.7 & 56.7 & 67.4 & 61.5 & 56.3 & 67.3 & 67.8 & 50.4 & 59.6$_{\pm 5.9}$ \\
& & 4 shot & \textbf{75.4} & 64.2 & 57.2 & 71.1 & 70.6 & \textbf{62.8} & 74.0 & 74.1 & \textbf{66.8} & 71.3 & 63.5 & 77.1 & 74.4 & 68.4 & 75.8 & 77.8 & \textbf{58.6} & 69.2$_{\pm 6.3}$ \\ \cmidrule{2-21}
& \multirow{4}{*}{Gemini 1.5 Pro} & 0 shot & 48.4 & 20.3 & 18.0 & 30.2 & 34.5 & 33.3 & 33.2 & 34.7 & 14.4 & 33.8 & 40.4 & 33.7 & 34.3 & 33.1 & 7.9 & 35.9 & 36.5 & 29.6$_{\pm 8.9}$ \\
& & 5 examples & 64.7 & 52.6 & 42.3 & 61.3 & 59.2 & 47.7 & 56.8 & 63.1 & 36.5 & 65.6 & 62.4 & 66.1 & 61.8 & 55.4 & 46.1 & 59.6 & 49.7 & 55.4$_{\pm 8.5}$ \\
& & 1 shot & 64.8 & 62.1 & 51.0 & 67.3 & 61.5 & 52.1 & 61.5 & 66.2 & 47.2 & 66.0 & 57.2 & 70.4 & 68.4 & 56.0 & 64.8 & 67.3 & 52.4 & 60.7$_{\pm 7.0}$ \\
& & 4 shots & 75.2 & \textbf{69.0} & \textbf{67.6} & \textbf{72.2} & \textbf{73.4} & 62.5 & \textbf{77.4} & \textbf{77.4} & 66.6 & \textbf{77.0} & \textbf{65.9} & \textbf{79.9} & \textbf{77.2} & \textbf{69.8} & \textbf{80.0} & \textbf{81.0} & 57.4 & \textbf{72.1$_{\pm 6.7}$} \\ \cmidrule{2-21}
& \multirow{4}{*}{Gemma 2 IT 9B} & 0 shot & 27.0 & 0.3 & 0.0 & 3.2 & 5.6 & 1.4 & 3.9 & 2.1 & 0.0 & 2.4 & 2.6 & 13.7 & 0.2 & 0.5 & 0.0 & 0.2 & 3.0 & 2.4$_{\pm 3.3}$ \\
& & 5 examples & 55.0 & 26.4 & 20.5 & 39.8 & 40.3 & 29.7 & 32.0 & 37.7 & 11.9 & 42.7 & 31.5 & 57.1 & 50.0 & 36.6 & 33.8 & 37.5 & 41.0 & 35.5$_{\pm 10.4}$ \\
& & 1 shot & 55.6 & 37.5 & 26.2 & 50.6 & 44.0 & 38.4 & 34.2 & 44.8 & 20.5 & 46.4 & 37.6 & 59.3 & 47.1 & 41.3 & 50.2 & 49.7 & 38.2 & 41.6$_{\pm 9.3}$ \\
& & 4 shots & 59.6 & 45.1 & 38.8 & 61.3 & 51.1 & 41.8 & 47.8 & 51.4 & 32.2 & 56.2 & 37.9 & 65.6 & 51.6 & 49.4 & 58.7 & 54.5 & 41.0 & 49.0$_{\pm 8.9}$ \\ \cmidrule{2-21}
& \multirow{4}{*}{Gemma 2 IT 27B} & 0 shot & 54.0 & 24.1 & 21.9 & 34.8 & 38.3 & 32.4 & 25.6 & 37.6 & 5.2 & 37.8 & 42.1 & 44.7 & 37.8 & 38.2 & 5.9 & 39.4 & 39.4 & 31.6$_{\pm 11.6}$ \\
& & 5 examples & 58.4 & 37.4 & 32.3 & 54.1 & 53.6 & 37.2 & 39.2 & 48.0 & 22.0 & 52.2 & 21.8 & 64.6 & 51.9 & 48.3 & 40.2 & 49.5 & 40.5 & 43.3$_{\pm 11.3}$ \\
& & 1 shot & 41.7 & 21.8 & 37.4 & 60.8 & 58.6 & 45.3 & 45.9 & 54.8 & 27.1 & 55.5 & 46.0 & 67.0 & 56.0 & 49.3 & 48.4 & 58.4 & 47.1 & 48.7$_{\pm 11.6}$ \\
& & 4 shots & 69.8 & 47.2 & 45.0 & 64.4 & 61.5 & 46.8 & 58.3 & 60.8 & 38.8 & 64.5 & 48.6 & 70.8 & 62.2 & 56.2 & 69.3 & 63.3 & 42.7 & 56.3$_{\pm 9.7}$ \\ \cmidrule{2-21}
& \multirow{4}{*}{Llama 3.3 70B} & 0 shot & 55.2 & 28.7 & 24.2 & 28.2 & 35.6 & 30.9 & 22.7 & 32.0 & 11.4 & 29.2 & 31.5 & 43.2 & 34.3 & 37.2 & 7.3 & 33.2 & 30.7 & 28.8$_{\pm 8.7}$ \\
& & 5 examples & 54.0 & 30.7 & 9.5 & 33.2 & 49.6 & 32.1 & 36.7 & 41.5 & 23.5 & 45.2 & 0.0 & 54.4 & 27.6 & 28.5 & 38.7 & 43.3 & 37.7 & 33.3$_{\pm 13.5}$ \\
& & 1 shot & 61.7 & 32.7 & 29.3 & 38.6 & 52.6 & 33.7 & 36.9 & 44.1 & 28.6 & 45.8 & 28.8 & 64.0 & 51.7 & 43.7 & 53.0 & 53.6 & 35.5 & 42.0$_{\pm 10.4}$ \\
& & 4 shots & 62.3 & 35.6 & 20.5 & 28.0 & 32.4 & 36.7 & 53.5 & 38.7 & 34.8 & 40.4 & 22.6 & 63.2 & 53.3 & 46.5 & 45.3 & 59.6 & 32.9 & 40.3$_{\pm 12.1}$ \\
\bottomrule
\end{tabular}
}
\caption{Zero-shot and few-shot performance comparison across languages on and tasks. For Intent Detection, shots refer to examples per domain 5 examples, per (1 shot), and 4 examples per (4 shots). For Slot Filling, shots refer to examples per domain 5 examples, per slot type (1 shot), and 4 examples per slot type (4 shots).}
\label{tab:few-shot-results}
\end{table*}

% for GPT-4o and Gemini 1.5 Pro

% \subsection{LLM Zero-shot Results}
% We plot the bar charts showing the zero-shot performance of the LLM models on the Intent Detection and Slot Filling tasks in Figures \ref{fig:prompt-intent} and \ref{fig:prompt-slot}, visualizing the results presented in \autoref{tab:prompt-results} in the main text.
% \begin{figure*}[!htbp]
%   \label{fig:prompt-intent}
%   \centering
%   \includegraphics[width=0.9\textwidth]{figures/prompt_seqc.pdf}
%   \vspace{-1em}
%   \caption{Zero-shot Intent Detection results using different prompt-based methods across languages}
% \end{figure*}  

% \begin{figure*}[!htbp]
%   \label{fig:prompt-slot}
%   \centering
%   \includegraphics[width=0.9\textwidth]{figures/prompt_tokenc.pdf}
%   \vspace{-1em}
%   \caption{Zero-shot Slot Filling results using different prompt-based methods across languages}
% \end{figure*}

% \newpage
% \subsection{LLM Few-shot Results}
% We evaluated GPT-4o and Gemini 1.5 Pro models using few-shot learning on both tasks. Figures \ref{fig:gpt4-fewshot} and \ref{fig:gemini-fewshot} visualize the performance across languages, extending the results presented in \autoref{tab:shot-comparison}.

% \begin{figure*}[!htbp]
%   % \centering
%     \includegraphics[width=0.5\linewidth]{figures/prompt_fewshot_seqc_GPT-4o (2024-08-06).pdf}
%   \hfill
%     \includegraphics[width=0.5\linewidth]{figures/prompt_fewshot_tokenc_GPT-4o (2024-08-06).pdf}
%   \caption{Few-shot Intent Detection (left) and Slot Filling (right) performance of GPT-4o across languages}
%   \label{fig:gpt4-fewshot}
% \end{figure*}

% \begin{figure*}[!htbp]
% % % \vfill
%     \includegraphics[width=0.5\linewidth]{figures/prompt_fewshot_seqc_Gemini 1.5 Pro 002.pdf}
%   \hfill
%     \includegraphics[width=0.5\linewidth]{figures/prompt_fewshot_tokenc_Gemini 1.5 Pro 002.pdf}
%   \caption{Few-shot Intent Detection (left) and Slot Filling (right) performance of Gemini 1.5 Pro across languages}
%   \label{fig:gemini-fewshot}
% \end{figure*}

% \begin{figure*}[!htbp]
% % % \vfill
%     \includegraphics[width=0.5\linewidth]{figures/prompt_fewshot_seqc_Gemma 2 IT 9B.pdf}
%   \hfill
%     \includegraphics[width=0.5\linewidth]{figures/prompt_fewshot_tokenc_Gemma 2 IT 9B.pdf}
%   \caption{Few-shot Intent Detection (left) and Slot Filling (right) performance of Gemma 2 9B IT across languages}
%   \label{fig:gemini-fewshot}
% \end{figure*}


% \twocolumn
% \subsection{Cross-lingual Transfer Learning Results}
% The following heatmaps show the cross-lingual transfer learning results for Intent Detection and Slot Filling tasks with the NLLB LLM2Vec, AfroXLMR-76L, and AfriTeVa V2 models (one model per architecture). The vertical axis represents the source language, while the horizontal axis represents the target language. The colour intensity indicates the performance of the model when transferred from the source language to the target language.

% \subsubsection{Intent Detection}
% \begin{figure}[!htpb]
%   \centering
%   \includegraphics[width=0.5\textwidth]{figures/AfroXLMR-76L_seqc_crosslingual.pdf} % 0.45
%   \caption{Cross-lingual Transfer Learning Results for Intent Detection Using AfroXLMR-76L Model}
%   \label{fig:afro76l-intent-transfer}
% \end{figure}

% \begin{figure}[!htpb]
%   \centering
%   \includegraphics[width=0.5\textwidth]{figures/NLLB-LLM2Vec_seqc_crosslingual.pdf}
%   \caption{Cross-lingual Transfer Learning Results for Intent Detection Using NLLB LLM2Vec Model}
% %   \label{fig:nllb-intent-transfer}
% % \end{figure}  
      
% % \begin{figure}[h]
% %   \centering
%   \includegraphics[width=0.5\textwidth]{figures/AfriTeVa V2_seqc_crosslingual.pdf}
%   \caption{Cross-lingual Transfer Learning Results for Intent Detection Using AfriTeVa V2 Model}
%   \label{fig:afriteva-intent-transfer}
% \end{figure}
% \newpage
% \subsubsection{Slot Filling}

% \begin{figure}[!htpb]
%   \centering
%   \includegraphics[width=0.5\textwidth]{figures/AfroXLMR-76L_tokenc_crosslingual.pdf}
%   \caption{Cross-lingual Transfer Learning Results for Slot Filling Using AfroXLMR-76L Model}
% %   \label{fig:afro76l-slot-transfer}
% % \end{figure}  

% % \begin{figure}[h]
% %   \centering
%   \includegraphics[width=0.5\textwidth]{figures/NLLB-LLM2Vec_tokenc_crosslingual.pdf}
%   \caption{Cross-lingual Transfer Learning Results for Slot Filling Using NLLB LLM2Vec Model}
% %   \label{fig:nllb-slot-transfer}
% % \end{figure}  

% % \begin{figure}[h]
% %   \centering
%   \includegraphics[width=0.5\textwidth]{figures/AfriTeVa V2_tokenc_crosslingual.pdf}
%   \caption{Cross-lingual Transfer Learning Results for Slot Filling Using AfriTeVa V2 Model}
%   \label{fig:afriteva-slot-transfer}
% \end{figure}


% \subsection{Cultured English Bridge Results}
% The following heatmaps show the complete cross-lingual transfer learning results for Intent Detection and Slot Filling tasks with the AfroXLMR-76L model when training on \lang{eng} split of \dataset{} [A] and subset of CLINC [B] \cite{CLINC}. The vertical axis represents the source language, while the horizontal axis represents the different combinations of A, B dataset. The \textit{translate} label indicates we translated the target language data to English with NLLB. The colour intensity indicates the performance of the model when transferred from the source language to the target language. 
% % This experiment shows the cultured dataset are more effective and robust for cross-lingual transfer learning.
% \begin{figure*}[!htpb]
%   \centering
%   \includegraphics[width=\linewidth]{figures/AfroXLMR-76L_seqc_crosslingual_eng.pdf}
%   \caption{Cross-lingual Transfer Learning Results for Intent Detection Using AfroXLMR-76L Model with vairous English dataset combinations}
%   \label{fig:eng-bridge-seqc}
% \end{figure*}

\newpage
% \onecolumn
\section{Prompts for Large Language Models}
\label{app:prompts}

We provide the prompts in Jinja format \footnote{\href{https://jinja.palletsprojects.com/en/stable/}{Jinjia: A fast, expressive, extensible templating engine.}} used for the Intent Detection and Slot Filling tasks in the zero-shot and few-shot learning experiments. The prompts are designed to guide the model to perform the specific task on the given input text. The prompts are language-specific and tailored to the task requirements. The prompts 

The variables in the prompts are replaced with the actual input text during the model evaluation. Here is the list of variables used in the prompts:

\begin{itemize}
    % \item \texttt{round}: The ith prompt used currently.
    \item \texttt{shot\_count}: The number of examples provided to the model, if \texttt{shot\_count} is 0 zero, means zero-shot.
    \item \texttt{examples}: A list of examples provided to the model for few-shot learning.
    \item \texttt{text}: The sentence for which the model needs to predict the intent or slot.
\end{itemize}

\subsection{Intent Detection}
% \begin{lstlisting}[breaklines=true]
\texttt{\textbf{Prompt I}}
% 
% frame=lines, 
\begin{minted}[breaklines=true, bgcolor=promptColor1]{jinja}
Classify the given sentence by identifying its intent and selecting the most appropriate category from the provided list.

# Steps
1. Analyze the sentence to understand its primary intention or purpose.
2. Compare the identified intention against the possible intent categories.
3. Select the category that best matches the sentence's intent.

# Output Format
- Return the only one matching intent category from the list above. 
- No additional text or punctuation should be included in the output. 
\end{minted}
% 
\texttt{\textbf{Prompt II}}
% frame=lines, 
\begin{minted}[breaklines=true,bgcolor=promptColor2]{jinja}
Identify the intent of the provided text by selecting the most suitable category from the list of available options.

# Steps
1. Analyze the sentence to determine its primary purpose or intention.
2. Match the identified intention with the available intent categories.
3. Choose the category that best aligns with the sentence's intent.

# Output Format
- Return the selected intent category from the list above.
- Do not include any additional text or punctuation in the response.
\end{minted}
% 
\texttt{\textbf{Prompt III}}
% frame=lines, 
\begin{minted}[breaklines=true,bgcolor=promptColor3]{jinja}
Determine the intent of the provided text by selecting the most appropriate category from the given options.

# Steps
1. **Read the Text**: Carefully read the provided text to understand the context and main message.
2. **Identify Key Elements**: Identify the main action, subject, and any relevant details that indicate the overall purpose of the text.
3. **Consider Categories**: Review the list of available categories and consider which category best matches the text's intent.
4. **Reasoning**: Consider why you believe the text fits a certain category by assessing how the identified key elements align with the category's definition.
5. **Selection**: Select the category that most accurately represents the intent of the text.

# Output Format
- Provide the selected category as a plain text response. 
- Don't include any justification.
\end{minted}
% 
\texttt{\textbf{Prompt IV}}
% frame=lines, 
\begin{minted}[breaklines=true,bgcolor=promptColor4]{jinja}
Identify the intent of the provided text by selecting the most suitable category from the list of available options.

# Steps

1. Analyze the text to understand its primary purpose and context.
2. Consider the range of possible intents that the text might express, such as inquiry, statement, request, etc.
3. Match the text with the most appropriate category based on its content and purpose.

# Output Format
Provide the resulting intent category as a short, concise phrase or word that best represents the text's purpose from the available options.

# Notes
- Carefully evaluate any subtleties in the language to determine the intent accurately.
- Consider edge cases where texts might have multiple overlapping intents, and choose the most dominant one.
\end{minted}
% 
\texttt{\textbf{Prompt V}}
% frame=lines, 
\begin{minted}[breaklines=true,bgcolor=promptColor5]{jinja}
Identify the intent of the provided text by selecting the most suitable category from the list of available options.

Consider the subtleties in language and any overlapping intents to determine the most dominant intent category.

# Steps

1. **Analyze the Text**: Thoroughly read and understand the text to grasp its primary purpose and context.
2. **Consider Possible Intents**: Reflect on the range of potential intents the text could express, such as inquiry, statement, or request.
3. **Match with Category**: Align the text with the most appropriate category based on content, language subtleties, and dominant purpose.

# Output Format

Provide the resulting intent category as a short, concise phrase or word.

# Notes

- Pay attention to context and subtleties in the text.
- Evaluate texts with multiple intents, prioritizing the most dominant one.
\end{minted}
% 
\texttt{\textbf{Suffix for Zero-shot and Few-shot}}

% frame=lines, 
\begin{minted}[breaklines=true]{jinja}
# Intent Categories
alarm, balance, bill_balance, book_flight, book_hotel, calendar_update, cancel_reservation, car_rental, confirm_reservation, cook_time, exchange_rate, food_last, freeze_account, ingredients_list, interest_rate, international_visa, make_call, meal_suggestion, min_payment, pay_bill, pin_change, play_music, plug_type, recipe, restaurant_reservation, restaurant_reviews, restaurant_suggestion, share_location, shopping_list_update, spending_history, text, time, timezone, transactions, transfer, translate, travel_notification, travel_suggestion, update_playlist, weather


{# Zero-shot Suffix #}
# Format Example:
Sentence: Can you tell me the weather forecast for today?
Output: weather

{# Few-shot Suffix #}

Sentence: {{ example.text }}
Output: {{ example.intent }}


Based on the example, consider the following:

Sentence: {{ text }}
Output: 
\end{minted}

% \end{minted}
% 
% 
% # Format Example:
% Sentence: Can you tell me the weather forecast for today?
% Output: weather
% 
% Sentence: {{ example.text }}
% Output: {{ example.intent }}
% Based on the example, consider the following:
% Sentence: {{ text }}
% Output: 
% \texttt{\textbf{Zero-shot Suffix}}
% \begin{minted}[breaklines=true]{jinja}

\subsection{Slot Filling}
% \begin{lstlisting}[breaklines=true]linenos
% \begin{minted}[breaklines=true]{jinja}
% 
\texttt{\textbf{Prompt I}}
% 
% frame=lines, 
\begin{minted}[breaklines=true, bgcolor=promptColor1]{jinja}
Identify all named entities in the sentence provided according to the available entity types. Use `$$` as a separator between each pair of identified named entity types and corresponding content from the sentence. Only return the listed named entities without providing any additional commentary.

# Output Format
- List all the named entities found in the passage provided by the user. 
- Separate the paired named entities types and text using a `$$` symbol.
- Only return the entity list, without any prefix or explanation.
\end{minted}
\texttt{\textbf{Prompt II}}
\begin{minted}[breaklines=true, bgcolor=promptColor2]{jinja}
Identify and extract named entities from the provided sentence. Each identified entity pair (including entity type and content from the sentence) should be separated from their content using the "$$" delimiter.

# Steps
1. Analyze the sentence to identify named entities.
2. Extract each identified named entity and its content.
3. Concatenate the named entity type and its content with space as one pair.
4. Join all pairs of named entities using "$$" as a delimiter.
\end{minted}
\texttt{\textbf{Prompt III}}
\begin{minted}[breaklines=true, bgcolor=promptColor3]{jinja}
Extract named entities from the provided text and format the output by placing $$ between each entity type and its respective content. Ensure the output contains only the extracted entities and their labels, with no additional commentary or information.

# Steps
1. Analyze the provided text and identify named entities.
2. Categorize each identified entity by its correct type, careful to match the entity with the appropriate label.
3. Format the output by placing the entity type and its corresponding content, separated by $$.
\end{minted}
\texttt{\textbf{Prompt IV}}
\begin{minted}[breaklines=true, bgcolor=promptColor4]{jinja}
Identify named entities from the provided text. Format each entity and its content using $$ as a separator. 

# Steps
1. Parse the input text to identify all named entities. This includes proper nouns like names of people, places, organizations, dates, etc.
2. For each identified entity, extract the specific text corresponding to the entity.
3. Concatenate the name of the entity type and the associated text using space. 
4. Compile these formatted entries into a list with the $$ as a separator.

# Output Format
- A string joined by a " $$ " for each pair of the entity type and content, formatted as `EntityType EntityContent`.
\end{minted}
\texttt{\textbf{Prompt V}}
\begin{minted}[breaklines=true, bgcolor=promptColor5]{jinja}
Detect named entities in the supplied sentence. Use $$ as a separator between entities and their corresponding parts of the sentence. Limit the response strictly to the formatted list.

# Output Format
- Entities and their parts separated by $$
- Return a plain list with no additional context
- If no entities are present, return `$$`
\end{minted}
% 
\texttt{\textbf{Suffix for Zero-shot and Few-shot}}
\begin{minted}[breaklines=true]{jinja}
# Named Entities Types to Identify
ACCOUNT_TYPE, ARTIST_NAME, BANK_NAME, BILL_TYPE, CALENDAR_EVENT, CITY_OR_PROVINCE, COUNTRY, CURRENCY, DATE, DISH_OR_FOOD, HOTEL_NAME, LANGUAGE_NAME, MEAL_PERIOD, MONEY, MUSIC_GENRE, NUMBER, PAYMENT_COMPANY, PERSONAL_NAME, PLACE_NAME, RESTAURANT_NAME, SHOPPING_ITEM, SONG_NAME, TIME


{# Zero-shot Suffix #}
Please ensure that the entities match the listed types and that unstated entities should not be included in the response if no entities are found, return `$$` only.

# Format Example:
Sentence: John went to Paris and paid 100 dollars at an Awater restaurant.
Output: PERSONAL_NAME John $$ CITY_OR_PROVINCE Paris $$ MONEY 100 $$ RESTAURANT_NAME Awater

{# Few-shot Suffix #}
# Output Examples (Do not include in the response):

Sentence: {{ example.text }}
Output: {{ example.slot }}


Based on the example, consider the following:

Sentence: {{ text }}
Output: 
\end{minted}

% \end{lstlisting}



\section{Instruction for Annotators}
\label{app:annotation-instruction}

This section provides a brief introduction to the annotation guide for the Slot Filling task.
Follow the instructions below to annotate the input text accordingly.

% \begin{lstlisting}[breaklines=true]
% linenos
\begin{minted}[breaklines=true]{markdown}

A Slot Filling task is a natural language processing (NLP) task that involves extracting specific pieces of information (slots) from a given text. This task is commonly used in dialogue systems and information extraction applications where the goal is to identify and fill predefined categories or slots with relevant information from user inputs or text data.

### LANGUAGE_NAME
1. Spanish: A Romance language that originated in the Iberian Peninsula and is now the primary language of Spain and most Latin American countries.
2. Luganda: A Bantu language spoken primarily in Uganda, particularly by the Ganda people.
3. French: A Romance language spoken as a first language in France, parts of Belgium, and Switzerland, and in various communities worldwide.


### ACCOUNT_TYPE
1. Savings Account: A bank account that earns interest over time, typically used for long-term savings.
2. Checking Account: A bank account used for everyday transactions, such as deposits and withdrawals.
3. Student Account: A bank account designed for students, often with no monthly fees and special benefits.
Not to be confused with payment company. 
A credit card is NOT an account type.

### MONEY
1. $500: Five hundred dollars, often used to signify a substantial amount of money in various contexts.
2. 5 dollars: A small amount of money, typically used for minor purchases or expenses.
3. $1,000: One thousand dollars, indicating a significant sum, commonly used in transactions or savings.

### CURRENCY
1. Dollar: The currency of several countries, including the United States, Canada, and Australia.
2. Euro: The official currency of the Eurozone, used by 19 of the 27 European Union member states.
3. Yen: The official currency of Japan.

### CITY_NAME
1. London: The capital city of the United Kingdom, known for its historical landmarks and cultural diversity.
2. Kampala: The capital city of Uganda, known for its bustling markets and vibrant cultural scene.
3. New York: A major city in the United States, known for its skyscrapers and as a global financial and cultural center.
If you are not sure if a place is a City name (Town name) State/Province or Village name, please refer to a search engine for clarification.

### FOOD_ITEM
1. Sugar: A sweet substance commonly used in baking and cooking.
2. Orange: A citrus fruit known for its sweet and tangy flavor and high vitamin C content.
Not to be confused with Shopping item or Dish name.

### BANK_NAME
1. Ecobank: A pan-African banking conglomerate with operations in 36 African countries.
2. Wells Fargo: An American multinational financial services company headquartered in San Francisco, California.
3. HSBC: A British multinational banking and financial services organization with global operations.
When annotating Bank names, you do not need to include “bank” unless it is attached to the bank name, like seen above, with Ecobank.

### RESTAURANT_NAME
1. KFC: An American fast-food restaurant chain known for its fried chicken.
2. McDonald's: An American fast-food company famous for its hamburgers, fries, and other quick-serve meals.
3. Subway: An American fast-food franchise known for its submarine sandwiches (subs) and salads.

### DISH_NAME
1. Jollof Rice: A popular West African dish made with rice, tomatoes, onions, and various spices.
2. Paella: A Spanish rice dish originally from Valencia, featuring saffron, meat, seafood, and vegetables.
3. Sushi: A Japanese dish consisting of vinegared rice accompanied by various ingredients such as raw fish and vegetables.

### TIME
1. 2pm: A specific time in the afternoon.
2. Morning: The period from sunrise until noon.
3. Evening: The period of the day from the end of the afternoon to the beginning of night.
Anything that is less than one day should be annotated as TIME and not DATE, as seen in the above examples.

### TIMEZONE
1. Pacific Time (PT): A time zone covering parts of western Canada, the western United States, and western Mexico.
2. West Africa Time (WAT): A time zone used by countries in West Africa, one hour ahead of Coordinated Universal Time (UTC+1).
3. Eastern Standard Time (EST): A time zone covering parts of the eastern United States and parts of Canada, five hours behind Coordinated Universal Time (UTC-5).

### DATE
1. January: The first month of the year in the Gregorian calendar.
2. 2024: A specific year.
3. October: The tenth month of the year in the Gregorian calendar.
Anything that is more than one day must be annotated as DATE and not time, as seen above 

### BILL_TYPE
1. Internet Fees: Charges for the provision of internet services.
2. School Fees: Costs associated with attending an educational institution.
3. Electricity Bill: Charges for the consumption of electrical power.
4. Water Bill: 
You include “bill” as part of the annotation.

### PLUG_TYPE
1. Type A: A two-pronged plug commonly used in North America and Japan.
2. Type C: A two-pin plug used in Europe, South America, and Asia.
3. Type G: A three-pronged plug used in the United Kingdom and other countries.
Internet cable, extension cord are NOT  plug types. 

### COUNTRY
1. Germany: A country in Central Europe known for its rich history and economic strength.
2. Nigeria: A country in West Africa, known for its diverse cultures and large population.
3. Japan: An island nation in East Asia known for its technology and rich cultural heritage.

### PERSONAL_NAME
1. Dave: A common given name.
2. Maria: A common given name, often used in Spanish and Portuguese-speaking countries.
3. Akiko: A common Japanese given name.
4. Don’t annotate titles as personal names e.g Mr., Dr., Mrs. 
Mom, dad, aunt, sister is NOT a personal names

### MUSIC_GENRE
1. Fuji: A popular Nigerian musical genre that originated from the Yoruba people.
2. Gospel: A genre of Christian music.
3. Rock: A broad genre of popular music that originated as "rock and roll" in the United States in the late 1940s and early 1950s.
Old songs are not genres- Do not annotate them

### ARTIST_NAME
1. Fela: Refers to Fela Kuti, a Nigerian multi-instrumentalist and pioneer of Afrobeat music.
2. Beyoncé: An American singer, songwriter, and actress.
3. Mozart: Wolfgang Amadeus Mozart, an influential classical composer from Austria.

### HOTEL_NAME
1. Radisson: A global hotel chain known for its upscale accommodations and services.
2. Marriott: A worldwide hospitality company with a broad range of hotels and related services.
3. Hilton: A global brand of full-service hotels and resorts.
You can annotate Radisson Hotel as a whole.

### MEAL_PERIOD
1. Breakfast: The first meal of the day, typically eaten in the morning.
2. Lunch: A meal eaten around midday.
3. Dinner: The main meal of the day, usually eaten in the evening.

### PAYMENT_COMPANY
1. Paypal: An American company operating a worldwide online payments system.
2. Stripe: An Irish-American financial services and software as a service (SaaS) company.
3. Visa: A multinational financial services corporation known for its credit and debit cards.
Not to be confused with account type.

### CONTINENT
1. Africa: The second-largest and second-most-populous continent on Earth.
2. Europe: A continent located entirely in the Northern Hemisphere and mostly in the Eastern Hemisphere.
3. Asia: The largest and most populous continent, located primarily in the Eastern and Northern Hemispheres.

### AIRPORT_NAME
1. Bole Addis Ababa International Airport: The main international gateway to Addis Ababa, Ethiopia.
2. Heathrow Airport: A major international airport in London, United Kingdom.
3. John F. Kennedy International Airport: A major international airport in New York City, United States.

### SUPERMARKET
1. Shoprite: A leading food retailer in Africa with stores in several countries.
2. Walmart: A large multinational retail corporation operating a chain of hypermarkets.
3. Tesco: A British multinational groceries and general merchandise retailer.

### STATE/PROVINCE
1. Quebec Province: A province in eastern Canada, the largest in area and second-largest in population.
2. Ogun State: A state in southwestern Nigeria.
3. California: A state in the western United States, known for its diverse geography and large economy.

### NUMBER
1. 10: A numerical value, often used to denote quantity or ranking.
2. 20: A numerical value, commonly used to signify quantity or sequence.
3. Fifty-four: non-numeric should be annotated as a number.

### NATIONALITY
1. Nigerian: Pertaining to Nigeria or its people.
2. Kenyan: Pertaining to Kenya or its people.
3. American: Pertaining to the United States of America or its people.

### CALENDAR_EVENT
1. Football Match: A scheduled competitive game of football (soccer).
2. Concert: A live music performance.
3. Wedding: A ceremony where two people are united in marriage.
Christmas, Valentines day, birthdays, etc

### SHOPPING_ITEM
1. Shoe: A covering for the foot, typically made of leather, having a sturdy sole and not reaching above the ankle.
2. Shirt: A piece of clothing worn on the upper body, typically with sleeves and a collar.
3. Laptop: A portable personal computer with a screen and alphanumeric keyboard.
Not to be confused with Food items

### SONG_NAME
1. African Queen: A popular song by Nigerian artist 2Baba.
2. Thriller: A song by Michael Jackson from his album of the same name.
3. Shape of You: A song by Ed Sheeran.

### CAR_TYPE
1. BMW: A German multinational company that produces luxury vehicles and motorcycles.
2. Sedan: A passenger car in a three-box configuration with separate compartments for the engine, passenger, and cargo.
3.SUV: A sport utility vehicle, typically equipped with four-wheel drive for on- or off-road ability.
Ambulance, Fire truck are not  car types.

### PLACE
1.Tourist Attractions: Places of interest that draw visitors due to their cultural, historical, natural, or recreational significance. Examples include the Eiffel Tower in Paris, a global cultural icon of France, and the Grand Canyon in Arizona, known for its immense size and its intricate and colorful landscape.

2. Museums: Institutions that collect, preserve, and display objects of historical, cultural, artistic, or scientific importance. Examples include the Louvre Museum in Paris, which houses a vast collection of art, and the Smithsonian National Museum of Natural History in Washington, D.C., known for its exhibits on natural history and anthropology.

3. Mall:  A large indoor shopping complex featuring a variety of retail stores, restaurants, and entertainment facilities. Examples include the Mall of America in Minnesota, which is one of the largest malls in the United States, and the Dubai Mall in the UAE, known for its luxury shops and attractions like the Dubai Aquarium and Underwater Zoo.

4. Park:  A public area set aside for recreation and enjoyment, often featuring green spaces, playgrounds, and walking paths. Examples include Central Park in New York City, a vast urban park offering numerous recreational activities, and Hyde Park in London, known for its historical significance and open-air concerts.

With this entity, only annotate if entity is named explicitly, e,g Name of airport, museum or mall is not and nt just “mall”, “airport” etc

PS: Do not skip any annotations, if there is nothing to annotate, submit and go to the next one.
\end{minted}
% \end{lstlisting}