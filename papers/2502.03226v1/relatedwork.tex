\section{Related Work}
\subsection{Online Collaborative Learning for Children}

CSCW and Computer-Supported Collaborative Learning (CSCL) are communities significantly concerned with how Information and Communication Technology (ICT) might support collaborative learning\citep{ludvigsen2010computer,liaqat2020leveraging,yadav2019leap}. CSCL has researched the collaborative learning process that occurred when college students used computer-mediated communication as a means of small group and large group communication\citep{stacey1999collaborative}. In recent years, an increasing number of studies have begun to address children's online collaborative learning, exploring the effects of elementary school students' reading skills, peer interaction, and self-concept in CSCL environments\citep{tsuei2011development,bouta2013enhancing,hourcade2002kidpad}. Bitter et al.\citep{bitter2016effects} found that regular use of Sokikom, an online collaborative math program, significantly improves elementary students' math achievement and motivation. Specifically, students using Sokikom regularly scored 18\% higher on the CAASPP math test and demonstrated a greater increase in motivation and positive attitude towards learning mathematics, regardless of their teacher or school. Online learning systems can also support conversational language learning and digital collaborative storytelling\citep{sharma2019exploring}.

Nowadays, remote socializing such as learning, communication, and collaboration has become increasingly crucial for students\citep{odenwald2020tabletop,yuan2021tabletop,galdo2022pair,gui2021teacher}, with some studies specifically discussing the opportunities and challenges of remote collaboration \citep{galdo2022pair,gui2021teacher,yuan2021tabletop}. For example, Galdo et al.\citep{galdo2022pair} reported positive feedback on students' perceptions of remote collaboration, with some mentioning they felt more successful because they collaborated with their partner, suggesting the possibilities of remote collaboration as a new style of working for young learners. However, research on distant synchronous cooperation among youngsters is still in its infancy. First, some current remote collaboration platforms for children emphasize asynchronous collaboration on long-term projects, such as RALfie\citep{orwin2015using} and Scratch\citep{aragon2009tale}. They are more aimed at promoting creative content than helping children socialize and collaborate remotely \cite{orwin2015using,aragon2009tale}. Second, most remote collaboration systems, such as Tabletop Teleporter\cite{odenwald2020tabletop} and Sharetable\citep{yarosh2013almost}, mainly serve child-adult (such as parents or teachers) collaboration, and only a few focus on child-child collaboration. Angelia et al.\citep{angelia2015design} proposed a game involving remote collaboration among children but lacked a systematic description and specific assessment of remote collaboration. 

These studies have examined how online collaborative learning affects children, but the collaborative task objectives they have set are skill-level tasks such as math, reading, programming, and storytelling. Collaborative projects in the Scratch community, which are child-initiated and use community-generated learning materials\citep{cheng2022interest,dasgupta2016remixing}. However, they did not involve complex tasks such as project-based learning, problem-based learning, STEAM programs, and maker programs, which are more aligned with authentic and natural collaborative learning. Besides, there is a dearth of remote synchronous collaboration systems intended for child-child collaboration. Thus, in this study, we design a real-life project for children using a project-based approach to explore the opportunities for long-term collaborative learning online for children.

\subsection{Parental Involvement in Children's Online Collaborative Learning}

Parents play an important role in children's learning, with many studies developed to promote parent engagement and assist families\citep{mendez2018guilford,cano2021early}. Parents have filled different support and assistance roles during children's learning process at home. Previous work on remote learning has explored parent involvement in pre-planned and well-developed remote learning programs\citep{gui2021teacher}. Barron et al. investigated the role of parents in supporting their children's development of new media skills and technological fluency. Through interviews with eight middle school students, seven parental roles were identified: teacher, collaborator, learning broker, resource provider, nontechnical consultant, employer, and learner\citep{barron2009parents}. Yu et al. analyzed parents acting as spectators, scaffolders, and teachers in children's learning. Resources for parents to manage their children's learning process were also explored\citep{yu2020considering}. Parents also need to provide technical support to help children overcome technological barriers as they use a variety of online learning platforms\citep{cumbo2021exploring,anastasiades2008collaborative}.

Parents have encountered a variety of challenges during their children's learning-from-home period, which consumed much of their time and energy\citep{dong2020young}. As for parental facilitation patterns of children’s technology-based learning, parents need to design learning, find resources, manage, and teach\citep{yu2021parental}. Parents needed to re-learn a lot of new content to co-create with their children and guide them through the learning process\citep{kucirkova2015child, mills2021covid}. In addition, because online learning requires a lot of technical support and children lack fundamental digital literacy, they need more parental help and supervision. Parents, therefore, needed to play additional roles to support their children's online learning\citep{han2022factors,gelir2021children}.

As previously stated, the majority of research has concentrated on parental support for online learning, rather than parental influence on children's collaborative learning processes. Junnan Yu et al. explored how parents can support children's collaboration in learning AI by encouraging group discussions, facilitating collaborative projects, promoting peer learning, providing joint learning resources, organizing family challenges, seeking external collaborative opportunities, and serving as facilitators and mediators during group activities. Collaborative learning fosters teamwork, communication skills, and a deeper understanding of AI concepts\citep{druga2022family}. In order to explore parents' role in supporting children's project-based online collaboration, this study conducted a project-based online collaborative learning program and documented the entire parental support process in this paper.

\subsection{Design Considerations for Children's Long-term Online Collaborative Learning}
Drawing upon the combined pedagogical experiences of our interdisciplinary team, composed of three educators and three educational researchers, we have identified several challenges to successful online collaborative learning for children. These challenges include low retention rates, hard-to-maintain students' attention, inadequate real-time guidance and feedback, underdeveloped communication and collaborative skills, and the lack of child-friendly interfaces in online collaborative platforms. In this regard, our research team has embarked on a thorough exploration of relevant theories, including developmental psychology, constructivism, and social learning theory, to underpin the design of our Children's Long-term Online Collaborative Learning. After several rounds of discussions and iterative refinement, we converged a set of five design guidelines based on five relevant theories as follows. 

\subsubsection{Active Learning}
Active learning is "a method of learning in which students are actively or experientially involved in the learning process and where there are different levels of active learning, depending on student involvement." Bonwell \& Eison state that "students participate in active learning when they are doing something besides passively listening" \citep{bonwellactive}. According to Hanson \& Moser using active teaching techniques in the classroom creates better academic outcomes for students\cite{hanson2003reflections}. Scheyvens et al. further noted that “by utilizing learning strategies that can include small-group work, role-play and simulations, data collection, and analysis, active learning is purported to increase student interest and motivation and to build students ‘critical thinking, problem-solving, and social skills” \citep{scheyvens2008experimenting}.

There is a wide range of alternatives for the term active learning, such as learning through play, technology-based learning, activity-based learning, group work, project methods, etc. The common factors in these are some significant qualities and characteristics of active learning. Active learning is the opposite of passive learning. It is learner-centered, not teacher-centered, and requires more than just listening; the active participation of students is a necessary aspect of active learning. In active learning, children are encouraged to come up with creative solutions for projects. This fosters creativity and innovation, which are vital skills in the 21st century \citep{lamb2017key,luna2015futures}. Furthermore, online project-based learning often involves teamwork. Active learning promotes collaboration as children need to work together, share ideas, and cooperate to complete projects. This enhances their social skills and helps them learn to work effectively in teams. Based on the above, we developed our design guideline 1:
\begin{quote}
  Design Guideline 1: Design dynamic and interactive activities to enhance students‘  active learning.
\end{quote}

\subsubsection{Observational Learning}
Observational learning is learning that occurs through observing the behavior of others, as children observe their peers, they also learn the importance of teamwork, collaboration, and the sharing of ideas. They get to understand different perspectives and approaches to solving a problem or completing a project, which can enhance their own problem-solving skills, which is important for collaborative learning.  In humans, this form of learning seems to not need reinforcement to occur, but instead, requires a social model such as a parent, friend, or teacher with the surroundings. Particularly in childhood, a model is someone of authority or higher status in an environment. In animals, observational learning is often based on classical conditioning, in which an instinctive behavior is elicited by observing the behavior of another, but other processes may be involved as 
well\citep{shettleworth2009cognition}. Besides, Online learning requires certain behaviors and skills, such as digital literacy, time management, and self-discipline. By observing their teachers and peers, students can learn and adapt these behaviors for their own learning. It is a form of social learning that usually takes an in-person form. We believe that observational learning will also benefit online collaborative learning for children. Thus, we formulated our design guideline 2:

\begin{quote}
  Design Guideline 2: Build an online collaborative environment to support online observational learning and facilitate communication among students.
\end{quote}

\subsubsection{Parental Involvement}
In an online environment, parental guidance is often necessary to ensure that children stay on track with their projects. Parents can help create a conducive learning environment at home, supervise their children's online activities to ensure safety, and guide them when they encounter problems. Parents' involvement in their children's education and parental warmth has been linked to many positive child outcomes\citep{ogg2020process}. Educators consider parental involvement an important ingredient in the remedy for many problems in education. Parents can enhance their children's learning by providing additional resources, discussing concepts, or even learning together with their children. They can also help their children make connections between what they are learning and real-world applications. It is important to set a role for parents to be involved in their children's learning. Parents also need to provide technical support to help children overcome technological barriers as they use a variety of online learning platforms\citep{cumbo2021exploring,anastasiades2008collaborative}. Thus, we formulated our design guideline 3:
  
\begin{quote}  
  Design Guideline 3: Create different roles to involve parents in students' learning and form a learning community
\end{quote}
  
\subsubsection{Scaffolding}
Scaffolding is a pedagogical approach that involves providing learners with the necessary support to enable them to accomplish tasks and develop understandings that are beyond their immediate grasp\citep{hammond2005scaffolding}. Its theoretical underpinnings are rooted in Vygotsky’s socio-cultural theory\citep{hammond2005scaffolding}, which underscores the importance of social interaction and support in learning\citep{hourcade2008interaction}. The facilitator plays a pivotal role in shaping the learners’ experience, with the ability to modulate the pace of the activity, influence group dynamics, and guide learners towards the desired outcomes. Scaffolding is particularly crucial in educational settings, where teachers or more advanced peers assist learners by progressively tapering off support as the learner becomes more proficient. 

In the realm of online project-based learning for children, scaffolding is indispensable for fostering comprehension and tailoring the learning experience to meet individual needs. Various forms of scaffolding can be implemented, including but not limited to task modeling, provision of step-by-step instructions, feedback delivery, posing guiding questions, and employing digital tools to navigate learners through tasks. These strategies collectively enhance the impact of online project-based learning on children. Consequently, we have revised our fourth design guideline to reflect the importance of scaffolding in online project-based learning:

 \begin{quote}
  Design Guideline 4:Implement scaffolding strategies to foster students' deep learning
\end{quote}

\subsubsection{Gamification}
Gamification is a strategic attempt to enhance systems, services, organizations, and activities by creating similar experiences to those experienced when playing games in order to motivate and engage users\citep{koivisto2019rise}. The gamification of learning is an educational approach that seeks to motivate students by using video game design and game elements in learning environments\citep{shatz2015using,sailer2017gamification}. The goal is to maximize enjoyment and engagement by capturing the interest of learners and inspiring them to continue learning\citep{huang2013gamification}.

Gamified incentive mechanisms can effectively stimulate student engagement in online project-based learning (PBL) designed for children. These mechanisms can harness the natural playfulness and competitive spirit of children, transforming learning into a more enjoyable and interactive process. A crucial component of these mechanisms, such as leaderboards, offers children timely feedback on their progress. This instant feedback mechanism can sustain student interest and motivation over extended periods, critical for long-term online PBL projects. The incorporation of incentive mechanisms into children's online PBL not only encourages active participation but also fosters a sense of accomplishment and an environment conducive to peer learning and collaboration. Based on the above, we developed our design guideline 5.

\begin{quote}
    Design Guideline 5: Cultivate motivation and engagement through an incentive-based system
\end{quote}