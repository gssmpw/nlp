%%%%%%%% ICML 2023 EXAMPLE LATEX SUBMISSION FILE %%%%%%%%%%%%%%%%%

\documentclass{article}

% Recommended, but optional, packages for figures and better typesetting:
\usepackage{microtype}
\usepackage{graphicx}
\usepackage{subfigure}
\usepackage{booktabs} % for professional tables

% hyperref makes hyperlinks in the resulting PDF.
% If your build breaks (sometimes temporarily if a hyperlink spans a page)
% please comment out the following usepackage line and replace
% \usepackage{icml2023} with \usepackage[nohyperref]{icml2023} above.
\usepackage{hyperref}
\usepackage[table,xcdraw]{xcolor}
% \usepackage{colortbl}

% Attempt to make hyperref and algorithmic work together better:
\newcommand{\theHalgorithm}{\arabic{algorithm}}

% Use the following line for the initial blind version submitted for review:
\usepackage[accepted]{icml2023}

% If accepted, instead use the following line for the camera-ready submission:
% \usepackage[accepted]{icml2023}

% For theorems and such
\usepackage{amsmath}
\usepackage{amssymb}
\usepackage{mathtools}
\usepackage{amsthm}

% if you use cleveref..
\usepackage[capitalize,noabbrev]{cleveref}

%%%%%%%%%%%%%%%%%%%%%%%%%%%%%%%%
% THEOREMS
%%%%%%%%%%%%%%%%%%%%%%%%%%%%%%%%
\theoremstyle{plain}
\newtheorem{theorem}{Theorem}[section]
\newtheorem{proposition}[theorem]{Proposition}
\newtheorem{lemma}[theorem]{Lemma}
\newtheorem{corollary}[theorem]{Corollary}
\theoremstyle{definition}
\newtheorem{definition}[theorem]{Definition}
\newtheorem{assumption}[theorem]{Assumption}
\theoremstyle{remark}
\newtheorem{remark}[theorem]{Remark}

% Todonotes is useful during development; simply uncomment the next line
%    and comment out the line below the next line to turn off comments
%\usepackage[disable,textsize=tiny]{todonotes}
\usepackage[textsize=tiny]{todonotes}


% The \icmltitle you define below is probably too long as a header.
% Therefore, a short form for the running title is supplied here:
\icmltitlerunning{Heterogeneous Image GNN: Graph-Conditioned Diffusion for Image Synthesis}

\makeatletter
\renewcommand{\@copyrightspace}{}
\renewcommand{\Notice@String}{}
\renewcommand{\ICML@appearing}{}
\makeatother


\begin{document}

\twocolumn[
\icmltitle{Heterogeneous Image GNN: Graph-Conditioned Diffusion for Image Synthesis}

% It is OKAY to include author information, even for blind
% submissions: the style file will automatically remove it for you
% unless you've provided the [accepted] option to the icml2023
% package.

% List of affiliations: The first argument should be a (short)
% identifier you will use later to specify author affiliations
% Academic affiliations should list Department, University, City, Region, Country
% Industry affiliations should list Company, City, Region, Country

% You can specify symbols, otherwise they are numbered in order.
% Ideally, you should not use this facility. Affiliations will be numbered
% in order of appearance and this is the preferred way.
% \icmlsetsymbol{equal}{*}

\begin{icmlauthorlist}
\icmlauthor{Rupert Menneer\textsuperscript{*}}{cam}
\icmlauthor{Christos Margadji}{cam}
\icmlauthor{Sebastian W. Pattinson}{cam}
\end{icmlauthorlist}

\icmlaffiliation{cam}{Department of Engineering, University of Cambridge, Cambridge, United Kingdom}


\icmlcorrespondingauthor{Rupert Menneer}{rfsm2@cam.ac.uk}
\icmlcorrespondingauthor{Sebastian W. Pattinson}{swp29@cam.ac.uk}
% \icmlauthor{Christos Margadji}{equal,yyy,comp}
% \icmlauthor{Sebastian Pattinson}{comp}
% \icmlauthor{Firstname4 Lastname4}{sch}
% \icmlauthor{Firstname5 Lastname5}{yyy}
% \icmlauthor{Firstname6 Lastname6}{sch,yyy,comp}
% \icmlauthor{Firstname7 Lastname7}{comp}
% %\icmlauthor{}{sch}
% \icmlauthor{Firstname8 Lastname8}{sch}
% \icmlauthor{Firstname8 Lastname8}{yyy,comp}
% %\icmlauthor{}{sch}
% %\icmlauthor{}{sch}
% \end{icmlauthorlist}

% \icmlaffiliation{yyy}{Department of XXX, University of YYY, Location, Country}
% \icmlaffiliation{comp}{Company Name, Location, Country}
% \icmlaffiliation{sch}{School of ZZZ, Institute of WWW, Location, Country}

% \icmlcorrespondingauthor{Firstname1 Lastname1}{first1.last1@xxx.edu}
% \icmlcorrespondingauthor{Firstname2 Lastname2}{first2.last2@www.uk}

% You may provide any keywords that you
% find helpful for describing your paper; these are used to populate
% the "keywords" metadata in the PDF but will not be shown in the document
\icmlkeywords{Machine Learning, ICML}

\vskip 0.3in
]

% this must go after the closing bracket ] following \twocolumn[ ...

% This command actually creates the footnote in the first column
% listing the affiliations and the copyright notice.
% The command takes one argument, which is text to display at the start of the footnote.
% The \icmlEqualContribution command is standard text for equal contribution.
% Remove it (just {}) if you do not need this facility.

%\printAffiliationsAndNotice{}  % leave blank if no need to mention equal contribution
% \renewcommand{\Notice@String}{}
% \renewcommand{\ICML@appearing}{}
\printAffiliationsAndNotice{} % otherwise use the standard text.

\begin{abstract}
We present an image blending pipeline, \textit{IBURD}, that creates realistic synthetic images to assist in the training of deep detectors for use on underwater autonomous vehicles (AUVs) for marine debris detection tasks. 
Specifically, IBURD generates both images of underwater debris and their pixel-level annotations, using source images of debris objects, their annotations, and target background images of marine environments. 
With Poisson editing and style transfer techniques, IBURD is even able to robustly blend transparent objects into arbitrary backgrounds and automatically adjust the style of blended images using the blurriness metric of target background images. 
These generated images of marine debris in actual underwater backgrounds address the data scarcity and data variety problems faced by deep-learned vision algorithms in challenging underwater conditions, and can enable the use of AUVs for environmental cleanup missions. 
Both quantitative and robotic evaluations of IBURD demonstrate the efficacy of the proposed approach for robotic detection of marine debris. 
\end{abstract}



\IEEEPARstart{L}everaging advanced algorithms and neural network architectures like Transformers~\cite{vaswani2023attentionneed}, AI has been empowered with strong reasoning ability and made tremendous progress in recent years. Breakthroughs in model design and training methodologies have allowed machines to excel in complex tasks, including Natural Language Processing (NLP) applications such as language translation, sentiment analysis, and text generation, achieving high accuracy and fostering intuitive human-computer interactions. Similarly, advancements in Computer Vision (CV) have empowered AI to analyze and interpret images, videos, and audio sequences with remarkable precision. In healthcare, Artificial Intelligence (AI) is revolutionizing medicine by enabling data-driven insights, improving diagnostics, and personalizing treatments~\cite{topol2019,Esteva2017,KOUROU20158}. These innovations have enabled significant applications, such as medical imaging analysis, disease diagnosis, pathology, radiology workflow optimization, and surgical assistance, transforming patient care and clinical workflows~\cite{empeek2024,pmc2021}.


The medical field faces unique challenges in data interpretation and decision-making for healthcare specialists; they must analyze diverse types of information including medical imaging (X-rays, MRIs, pathology slides), clinical notes, patient histories, and real-time observations. Medical images are critical for diagnostic checks and measurements, such as identifying anatomical abnormalities, quantifying disease progression, or assessing treatment efficacy. On the other hand, textual data, such as clinical notes, nurse evaluations, and patient histories, provide essential context for screening, understanding symptoms, and documenting disease progression. Textual outputs, such as radiology reports or discharge summaries, are equally vital, as they synthesize findings into actionable insights for clinicians. The complexity and volume of this multi-modal medical data often lead to cognitive overload, impacting the speed and accuracy of diagnoses. Traditional single-modality approaches, which treat images and text separately, fail to capture the intricate relationships between visual findings and clinical context. This limitation underscores the need for integrated vision-language models (VLMs) that can bridge the gap between these modalities\cite{bordes2024introductionvisionlanguagemodeling}, enabling more comprehensive and accurate decision-making in healthcare. This integrated approach promises to enhance clinical decision-making by providing more contextually informed insights and reducing the cognitive burden on healthcare providers.

\begin{figure*}[ht]
    \centering
    \includegraphics[width=\linewidth]{images/methods2.png}
    \caption{\textbf{Comprehensive Framework for Medical Vision-Language Models (VLMs)}. \textbf{(a)} Training involves processing diverse inputs such as images, texts, metadata, and historical data, followed by pre-training. \textbf{(b)} Benchmarking is conducted on a variety of medical datasets including GMAI-MMBench, OmniMedVQA, RadBench, and others. \textbf{(c)} Advanced training strategies are employed, such as vision-text alignment, knowledge distillation, masked language modeling, contrastive learning, and parameter-efficient tuning. \textbf{(d)} Evaluation strategies encompass automated metrics like BLEU, ROUGE, BERTScore, and clinical-specific tools like CheXpert Labeler and RadGraph, alongside human evaluation. \textbf{(e)} Integration of VLMs into the medical workflow leverages contextual data to provide actionable insights and improve clinical decision-making.}
    \label{fig:method}
\end{figure*}

However, visual and language provide totally different modalities that are not trivial to be integrated directly. As illustrated in Fig.~\ref{fig:method}, existing works address this challenge through various strategies, including vision-text alignment (in MedViL\cite{devlin2019bertpretrainingdeepbidirectional}, MedCLIP\cite{radford2021learningtransferablevisualmodels}, BioMedCLIP\cite{zhang2024biomedclipmultimodalbiomedicalfoundation}, VividMed\cite{luo2024vividmedvisionlanguagemodel}), knowledge distillation with VividMed\cite{luo2024vividmedvisionlanguagemodel}, masked language modeling (in MedViL\cite{devlin2019bertpretrainingdeepbidirectional} and BioMedCLIP\cite{zhang2024biomedclipmultimodalbiomedicalfoundation}) and contrastive learning in MedCLIP\cite{radford2021learningtransferablevisualmodels}, BioViL\cite{Boecking2022} \& ConVIRT\cite{zhang2022contrastivelearningmedicalvisual}. More recent advancements have introduced additional approaches, such as frozen encoders and Q-Former (e.g., BLIP-2\cite{li2023blip2bootstrappinglanguageimagepretraining}, InstructBLIP\cite{dai2023instructblipgeneralpurposevisionlanguagemodels}), image-text pair learning and fine-tuning (e.g., LLaVA\cite{liu2023visualinstructiontuning}, LLaVA-Med\cite{li2023llavamedtraininglargelanguageandvision}, BiomedGPT\cite{Zhang2024}, MedVInT\cite{zhang2024pmcvqavisualinstructiontuning}), parameter-efficient tuning (e.g., LLaMA-Adapter-V2\cite{zhang2024llamaadapterefficientfinetuninglanguage}), two-stage training (e.g., MiniGPT-4\cite{zhu2023minigpt4enhancingvisionlanguageunderstanding}), and modular multimodal pre-training (e.g., mPLUG-Owl\cite{ye2024mplugowlmodularizationempowerslarge}, Otter\cite{li2023ottermultimodalmodelincontext}) and the Sigmoid Loss for Language-Image Pre-Training (SigLIP)\cite{zhai2023sigmoidlosslanguageimage}. SigLIP replaces traditional softmax-based contrastive learning with a simpler sigmoid loss approach, enabling more efficient and scalable training by treating image-text pair alignment as a binary classification task. These methods aim to establish coherent relationships between visual inputs and textual outputs, enabling models to effectively interpret and generate relevant information across modalities. As a comparison, the contrastive learning-based methods leverage the similarities and differences between paired visual and textual data to enhance model robustness and generalization.

This paper provides a comprehensive review of VLMs and their applications in healthcare. We first discuss how VLMs are constructed by integrating advancements in NLP and computer vision. Next, we summarize key methodologies and advancements in the field, including state-of-the-art models like Qwen-VL\cite{bai2023qwenvlversatilevisionlanguagemodel}, RadFM\cite{wu2023generalistfoundationmodelradiology}, and DeepSeek-VL\cite{lu2024deepseekvlrealworldvisionlanguageunderstanding}. We then explore how VLMs are applied in the medical domain, highlighting their potential to improve diagnostic accuracy, clinical decision-making, and other healthcare tasks. Finally, we conclude by outlining future directions and challenges in the integration of VLMs into healthcare practices.

\section{Related Work}
\label{sec:relatedworks}

% \begin{table*}[t]
% \centering 
% \renewcommand\arraystretch{0.98}
% \fontsize{8}{10}\selectfont \setlength{\tabcolsep}{0.4em}
% \begin{tabular}{@{}lc|cc|cc|cc@{}}
% \toprule
% \textbf{Methods}           & \begin{tabular}[c]{@{}c@{}}\textbf{Training}\\ \textbf{Paradigm}\end{tabular} & \begin{tabular}[c]{@{}c@{}}\textbf{$\#$ PT Data}\\ \textbf{(Tokens)}\end{tabular} & \begin{tabular}[c]{@{}c@{}}\textbf{$\#$ IFT Data}\\ \textbf{(Samples)}\end{tabular} & \textbf{Code}  & \begin{tabular}[c]{@{}c@{}}\textbf{Natural}\\ \textbf{Language}\end{tabular} & \begin{tabular}[c]{@{}c@{}}\textbf{Action}\\ \textbf{Trajectories}\end{tabular} & \begin{tabular}[c]{@{}c@{}}\textbf{API}\\ \textbf{Documentation}\end{tabular}\\ \midrule 
% NexusRaven~\citep{srinivasan2023nexusraven} & IFT & - & - & \textcolor{green}{\CheckmarkBold} & \textcolor{green}{\CheckmarkBold} &\textcolor{red}{\XSolidBrush}&\textcolor{red}{\XSolidBrush}\\
% AgentInstruct~\citep{zeng2023agenttuning} & IFT & - & 2k & \textcolor{green}{\CheckmarkBold} & \textcolor{green}{\CheckmarkBold} &\textcolor{red}{\XSolidBrush}&\textcolor{red}{\XSolidBrush} \\
% AgentEvol~\citep{xi2024agentgym} & IFT & - & 14.5k & \textcolor{green}{\CheckmarkBold} & \textcolor{green}{\CheckmarkBold} &\textcolor{green}{\CheckmarkBold}&\textcolor{red}{\XSolidBrush} \\
% Gorilla~\citep{patil2023gorilla}& IFT & - & 16k & \textcolor{green}{\CheckmarkBold} & \textcolor{green}{\CheckmarkBold} &\textcolor{red}{\XSolidBrush}&\textcolor{green}{\CheckmarkBold}\\
% OpenFunctions-v2~\citep{patil2023gorilla} & IFT & - & 65k & \textcolor{green}{\CheckmarkBold} & \textcolor{green}{\CheckmarkBold} &\textcolor{red}{\XSolidBrush}&\textcolor{green}{\CheckmarkBold}\\
% LAM~\citep{zhang2024agentohana} & IFT & - & 42.6k & \textcolor{green}{\CheckmarkBold} & \textcolor{green}{\CheckmarkBold} &\textcolor{green}{\CheckmarkBold}&\textcolor{red}{\XSolidBrush} \\
% xLAM~\citep{liu2024apigen} & IFT & - & 60k & \textcolor{green}{\CheckmarkBold} & \textcolor{green}{\CheckmarkBold} &\textcolor{green}{\CheckmarkBold}&\textcolor{red}{\XSolidBrush} \\\midrule
% LEMUR~\citep{xu2024lemur} & PT & 90B & 300k & \textcolor{green}{\CheckmarkBold} & \textcolor{green}{\CheckmarkBold} &\textcolor{green}{\CheckmarkBold}&\textcolor{red}{\XSolidBrush}\\
% \rowcolor{teal!12} \method & PT & 103B & 95k & \textcolor{green}{\CheckmarkBold} & \textcolor{green}{\CheckmarkBold} & \textcolor{green}{\CheckmarkBold} & \textcolor{green}{\CheckmarkBold} \\
% \bottomrule
% \end{tabular}
% \caption{Summary of existing tuning- and pretraining-based LLM agents with their training sample sizes. "PT" and "IFT" denote "Pre-Training" and "Instruction Fine-Tuning", respectively. }
% \label{tab:related}
% \end{table*}

\begin{table*}[ht]
\begin{threeparttable}
\centering 
\renewcommand\arraystretch{0.98}
\fontsize{7}{9}\selectfont \setlength{\tabcolsep}{0.2em}
\begin{tabular}{@{}l|c|c|ccc|cc|cc|cccc@{}}
\toprule
\textbf{Methods} & \textbf{Datasets}           & \begin{tabular}[c]{@{}c@{}}\textbf{Training}\\ \textbf{Paradigm}\end{tabular} & \begin{tabular}[c]{@{}c@{}}\textbf{\# PT Data}\\ \textbf{(Tokens)}\end{tabular} & \begin{tabular}[c]{@{}c@{}}\textbf{\# IFT Data}\\ \textbf{(Samples)}\end{tabular} & \textbf{\# APIs} & \textbf{Code}  & \begin{tabular}[c]{@{}c@{}}\textbf{Nat.}\\ \textbf{Lang.}\end{tabular} & \begin{tabular}[c]{@{}c@{}}\textbf{Action}\\ \textbf{Traj.}\end{tabular} & \begin{tabular}[c]{@{}c@{}}\textbf{API}\\ \textbf{Doc.}\end{tabular} & \begin{tabular}[c]{@{}c@{}}\textbf{Func.}\\ \textbf{Call}\end{tabular} & \begin{tabular}[c]{@{}c@{}}\textbf{Multi.}\\ \textbf{Step}\end{tabular}  & \begin{tabular}[c]{@{}c@{}}\textbf{Plan}\\ \textbf{Refine}\end{tabular}  & \begin{tabular}[c]{@{}c@{}}\textbf{Multi.}\\ \textbf{Turn}\end{tabular}\\ \midrule 
\multicolumn{13}{l}{\emph{Instruction Finetuning-based LLM Agents for Intrinsic Reasoning}}  \\ \midrule
FireAct~\cite{chen2023fireact} & FireAct & IFT & - & 2.1K & 10 & \textcolor{red}{\XSolidBrush} &\textcolor{green}{\CheckmarkBold} &\textcolor{green}{\CheckmarkBold}  & \textcolor{red}{\XSolidBrush} &\textcolor{green}{\CheckmarkBold} & \textcolor{red}{\XSolidBrush} &\textcolor{green}{\CheckmarkBold} & \textcolor{red}{\XSolidBrush} \\
ToolAlpaca~\cite{tang2023toolalpaca} & ToolAlpaca & IFT & - & 4.0K & 400 & \textcolor{red}{\XSolidBrush} &\textcolor{green}{\CheckmarkBold} &\textcolor{green}{\CheckmarkBold} & \textcolor{red}{\XSolidBrush} &\textcolor{green}{\CheckmarkBold} & \textcolor{red}{\XSolidBrush}  &\textcolor{green}{\CheckmarkBold} & \textcolor{red}{\XSolidBrush}  \\
ToolLLaMA~\cite{qin2023toolllm} & ToolBench & IFT & - & 12.7K & 16,464 & \textcolor{red}{\XSolidBrush} &\textcolor{green}{\CheckmarkBold} &\textcolor{green}{\CheckmarkBold} &\textcolor{red}{\XSolidBrush} &\textcolor{green}{\CheckmarkBold}&\textcolor{green}{\CheckmarkBold}&\textcolor{green}{\CheckmarkBold} &\textcolor{green}{\CheckmarkBold}\\
AgentEvol~\citep{xi2024agentgym} & AgentTraj-L & IFT & - & 14.5K & 24 &\textcolor{red}{\XSolidBrush} & \textcolor{green}{\CheckmarkBold} &\textcolor{green}{\CheckmarkBold}&\textcolor{red}{\XSolidBrush} &\textcolor{green}{\CheckmarkBold}&\textcolor{red}{\XSolidBrush} &\textcolor{red}{\XSolidBrush} &\textcolor{green}{\CheckmarkBold}\\
Lumos~\cite{yin2024agent} & Lumos & IFT  & - & 20.0K & 16 &\textcolor{red}{\XSolidBrush} & \textcolor{green}{\CheckmarkBold} & \textcolor{green}{\CheckmarkBold} &\textcolor{red}{\XSolidBrush} & \textcolor{green}{\CheckmarkBold} & \textcolor{green}{\CheckmarkBold} &\textcolor{red}{\XSolidBrush} & \textcolor{green}{\CheckmarkBold}\\
Agent-FLAN~\cite{chen2024agent} & Agent-FLAN & IFT & - & 24.7K & 20 &\textcolor{red}{\XSolidBrush} & \textcolor{green}{\CheckmarkBold} & \textcolor{green}{\CheckmarkBold} &\textcolor{red}{\XSolidBrush} & \textcolor{green}{\CheckmarkBold}& \textcolor{green}{\CheckmarkBold}&\textcolor{red}{\XSolidBrush} & \textcolor{green}{\CheckmarkBold}\\
AgentTuning~\citep{zeng2023agenttuning} & AgentInstruct & IFT & - & 35.0K & - &\textcolor{red}{\XSolidBrush} & \textcolor{green}{\CheckmarkBold} & \textcolor{green}{\CheckmarkBold} &\textcolor{red}{\XSolidBrush} & \textcolor{green}{\CheckmarkBold} &\textcolor{red}{\XSolidBrush} &\textcolor{red}{\XSolidBrush} & \textcolor{green}{\CheckmarkBold}\\\midrule
\multicolumn{13}{l}{\emph{Instruction Finetuning-based LLM Agents for Function Calling}} \\\midrule
NexusRaven~\citep{srinivasan2023nexusraven} & NexusRaven & IFT & - & - & 116 & \textcolor{green}{\CheckmarkBold} & \textcolor{green}{\CheckmarkBold}  & \textcolor{green}{\CheckmarkBold} &\textcolor{red}{\XSolidBrush} & \textcolor{green}{\CheckmarkBold} &\textcolor{red}{\XSolidBrush} &\textcolor{red}{\XSolidBrush}&\textcolor{red}{\XSolidBrush}\\
Gorilla~\citep{patil2023gorilla} & Gorilla & IFT & - & 16.0K & 1,645 & \textcolor{green}{\CheckmarkBold} &\textcolor{red}{\XSolidBrush} &\textcolor{red}{\XSolidBrush}&\textcolor{green}{\CheckmarkBold} &\textcolor{green}{\CheckmarkBold} &\textcolor{red}{\XSolidBrush} &\textcolor{red}{\XSolidBrush} &\textcolor{red}{\XSolidBrush}\\
OpenFunctions-v2~\citep{patil2023gorilla} & OpenFunctions-v2 & IFT & - & 65.0K & - & \textcolor{green}{\CheckmarkBold} & \textcolor{green}{\CheckmarkBold} &\textcolor{red}{\XSolidBrush} &\textcolor{green}{\CheckmarkBold} &\textcolor{green}{\CheckmarkBold} &\textcolor{red}{\XSolidBrush} &\textcolor{red}{\XSolidBrush} &\textcolor{red}{\XSolidBrush}\\
API Pack~\cite{guo2024api} & API Pack & IFT & - & 1.1M & 11,213 &\textcolor{green}{\CheckmarkBold} &\textcolor{red}{\XSolidBrush} &\textcolor{green}{\CheckmarkBold} &\textcolor{red}{\XSolidBrush} &\textcolor{green}{\CheckmarkBold} &\textcolor{red}{\XSolidBrush}&\textcolor{red}{\XSolidBrush}&\textcolor{red}{\XSolidBrush}\\ 
LAM~\citep{zhang2024agentohana} & AgentOhana & IFT & - & 42.6K & - & \textcolor{green}{\CheckmarkBold} & \textcolor{green}{\CheckmarkBold} &\textcolor{green}{\CheckmarkBold}&\textcolor{red}{\XSolidBrush} &\textcolor{green}{\CheckmarkBold}&\textcolor{red}{\XSolidBrush}&\textcolor{green}{\CheckmarkBold}&\textcolor{green}{\CheckmarkBold}\\
xLAM~\citep{liu2024apigen} & APIGen & IFT & - & 60.0K & 3,673 & \textcolor{green}{\CheckmarkBold} & \textcolor{green}{\CheckmarkBold} &\textcolor{green}{\CheckmarkBold}&\textcolor{red}{\XSolidBrush} &\textcolor{green}{\CheckmarkBold}&\textcolor{red}{\XSolidBrush}&\textcolor{green}{\CheckmarkBold}&\textcolor{green}{\CheckmarkBold}\\\midrule
\multicolumn{13}{l}{\emph{Pretraining-based LLM Agents}}  \\\midrule
% LEMUR~\citep{xu2024lemur} & PT & 90B & 300.0K & - & \textcolor{green}{\CheckmarkBold} & \textcolor{green}{\CheckmarkBold} &\textcolor{green}{\CheckmarkBold}&\textcolor{red}{\XSolidBrush} & \textcolor{red}{\XSolidBrush} &\textcolor{green}{\CheckmarkBold} &\textcolor{red}{\XSolidBrush}&\textcolor{red}{\XSolidBrush}\\
\rowcolor{teal!12} \method & \dataset & PT & 103B & 95.0K  & 76,537  & \textcolor{green}{\CheckmarkBold} & \textcolor{green}{\CheckmarkBold} & \textcolor{green}{\CheckmarkBold} & \textcolor{green}{\CheckmarkBold} & \textcolor{green}{\CheckmarkBold} & \textcolor{green}{\CheckmarkBold} & \textcolor{green}{\CheckmarkBold} & \textcolor{green}{\CheckmarkBold}\\
\bottomrule
\end{tabular}
% \begin{tablenotes}
%     \item $^*$ In addition, the StarCoder-API can offer 4.77M more APIs.
% \end{tablenotes}
\caption{Summary of existing instruction finetuning-based LLM agents for intrinsic reasoning and function calling, along with their training resources and sample sizes. "PT" and "IFT" denote "Pre-Training" and "Instruction Fine-Tuning", respectively.}
\vspace{-2ex}
\label{tab:related}
\end{threeparttable}
\end{table*}

\noindent \textbf{Prompting-based LLM Agents.} Due to the lack of agent-specific pre-training corpus, existing LLM agents rely on either prompt engineering~\cite{hsieh2023tool,lu2024chameleon,yao2022react,wang2023voyager} or instruction fine-tuning~\cite{chen2023fireact,zeng2023agenttuning} to understand human instructions, decompose high-level tasks, generate grounded plans, and execute multi-step actions. 
However, prompting-based methods mainly depend on the capabilities of backbone LLMs (usually commercial LLMs), failing to introduce new knowledge and struggling to generalize to unseen tasks~\cite{sun2024adaplanner,zhuang2023toolchain}. 

\noindent \textbf{Instruction Finetuning-based LLM Agents.} Considering the extensive diversity of APIs and the complexity of multi-tool instructions, tool learning inherently presents greater challenges than natural language tasks, such as text generation~\cite{qin2023toolllm}.
Post-training techniques focus more on instruction following and aligning output with specific formats~\cite{patil2023gorilla,hao2024toolkengpt,qin2023toolllm,schick2024toolformer}, rather than fundamentally improving model knowledge or capabilities. 
Moreover, heavy fine-tuning can hinder generalization or even degrade performance in non-agent use cases, potentially suppressing the original base model capabilities~\cite{ghosh2024a}.

\noindent \textbf{Pretraining-based LLM Agents.} While pre-training serves as an essential alternative, prior works~\cite{nijkamp2023codegen,roziere2023code,xu2024lemur,patil2023gorilla} have primarily focused on improving task-specific capabilities (\eg, code generation) instead of general-domain LLM agents, due to single-source, uni-type, small-scale, and poor-quality pre-training data. 
Existing tool documentation data for agent training either lacks diverse real-world APIs~\cite{patil2023gorilla, tang2023toolalpaca} or is constrained to single-tool or single-round tool execution. 
Furthermore, trajectory data mostly imitate expert behavior or follow function-calling rules with inferior planning and reasoning, failing to fully elicit LLMs' capabilities and handle complex instructions~\cite{qin2023toolllm}. 
Given a wide range of candidate API functions, each comprising various function names and parameters available at every planning step, identifying globally optimal solutions and generalizing across tasks remains highly challenging.



\section{Preliminaries}
\label{Preliminaries}
\begin{figure*}[t]
    \centering
    \includegraphics[width=0.95\linewidth]{fig/HealthGPT_Framework.png}
    \caption{The \ourmethod{} architecture integrates hierarchical visual perception and H-LoRA, employing a task-specific hard router to select visual features and H-LoRA plugins, ultimately generating outputs with an autoregressive manner.}
    \label{fig:architecture}
\end{figure*}
\noindent\textbf{Large Vision-Language Models.} 
The input to a LVLM typically consists of an image $x^{\text{img}}$ and a discrete text sequence $x^{\text{txt}}$. The visual encoder $\mathcal{E}^{\text{img}}$ converts the input image $x^{\text{img}}$ into a sequence of visual tokens $\mathcal{V} = [v_i]_{i=1}^{N_v}$, while the text sequence $x^{\text{txt}}$ is mapped into a sequence of text tokens $\mathcal{T} = [t_i]_{i=1}^{N_t}$ using an embedding function $\mathcal{E}^{\text{txt}}$. The LLM $\mathcal{M_\text{LLM}}(\cdot|\theta)$ models the joint probability of the token sequence $\mathcal{U} = \{\mathcal{V},\mathcal{T}\}$, which is expressed as:
\begin{equation}
    P_\theta(R | \mathcal{U}) = \prod_{i=1}^{N_r} P_\theta(r_i | \{\mathcal{U}, r_{<i}\}),
\end{equation}
where $R = [r_i]_{i=1}^{N_r}$ is the text response sequence. The LVLM iteratively generates the next token $r_i$ based on $r_{<i}$. The optimization objective is to minimize the cross-entropy loss of the response $\mathcal{R}$.
% \begin{equation}
%     \mathcal{L}_{\text{VLM}} = \mathbb{E}_{R|\mathcal{U}}\left[-\log P_\theta(R | \mathcal{U})\right]
% \end{equation}
It is worth noting that most LVLMs adopt a design paradigm based on ViT, alignment adapters, and pre-trained LLMs\cite{liu2023llava,liu2024improved}, enabling quick adaptation to downstream tasks.


\noindent\textbf{VQGAN.}
VQGAN~\cite{esser2021taming} employs latent space compression and indexing mechanisms to effectively learn a complete discrete representation of images. VQGAN first maps the input image $x^{\text{img}}$ to a latent representation $z = \mathcal{E}(x)$ through a encoder $\mathcal{E}$. Then, the latent representation is quantized using a codebook $\mathcal{Z} = \{z_k\}_{k=1}^K$, generating a discrete index sequence $\mathcal{I} = [i_m]_{m=1}^N$, where $i_m \in \mathcal{Z}$ represents the quantized code index:
\begin{equation}
    \mathcal{I} = \text{Quantize}(z|\mathcal{Z}) = \arg\min_{z_k \in \mathcal{Z}} \| z - z_k \|_2.
\end{equation}
In our approach, the discrete index sequence $\mathcal{I}$ serves as a supervisory signal for the generation task, enabling the model to predict the index sequence $\hat{\mathcal{I}}$ from input conditions such as text or other modality signals.  
Finally, the predicted index sequence $\hat{\mathcal{I}}$ is upsampled by the VQGAN decoder $G$, generating the high-quality image $\hat{x}^\text{img} = G(\hat{\mathcal{I}})$.



\noindent\textbf{Low Rank Adaptation.} 
LoRA\cite{hu2021lora} effectively captures the characteristics of downstream tasks by introducing low-rank adapters. The core idea is to decompose the bypass weight matrix $\Delta W\in\mathbb{R}^{d^{\text{in}} \times d^{\text{out}}}$ into two low-rank matrices $ \{A \in \mathbb{R}^{d^{\text{in}} \times r}, B \in \mathbb{R}^{r \times d^{\text{out}}} \}$, where $ r \ll \min\{d^{\text{in}}, d^{\text{out}}\} $, significantly reducing learnable parameters. The output with the LoRA adapter for the input $x$ is then given by:
\begin{equation}
    h = x W_0 + \alpha x \Delta W/r = x W_0 + \alpha xAB/r,
\end{equation}
where matrix $ A $ is initialized with a Gaussian distribution, while the matrix $ B $ is initialized as a zero matrix. The scaling factor $ \alpha/r $ controls the impact of $ \Delta W $ on the model.

\section{HealthGPT}
\label{Method}


\subsection{Unified Autoregressive Generation.}  
% As shown in Figure~\ref{fig:architecture}, 
\ourmethod{} (Figure~\ref{fig:architecture}) utilizes a discrete token representation that covers both text and visual outputs, unifying visual comprehension and generation as an autoregressive task. 
For comprehension, $\mathcal{M}_\text{llm}$ receives the input joint sequence $\mathcal{U}$ and outputs a series of text token $\mathcal{R} = [r_1, r_2, \dots, r_{N_r}]$, where $r_i \in \mathcal{V}_{\text{txt}}$, and $\mathcal{V}_{\text{txt}}$ represents the LLM's vocabulary:
\begin{equation}
    P_\theta(\mathcal{R} \mid \mathcal{U}) = \prod_{i=1}^{N_r} P_\theta(r_i \mid \mathcal{U}, r_{<i}).
\end{equation}
For generation, $\mathcal{M}_\text{llm}$ first receives a special start token $\langle \text{START\_IMG} \rangle$, then generates a series of tokens corresponding to the VQGAN indices $\mathcal{I} = [i_1, i_2, \dots, i_{N_i}]$, where $i_j \in \mathcal{V}_{\text{vq}}$, and $\mathcal{V}_{\text{vq}}$ represents the index range of VQGAN. Upon completion of generation, the LLM outputs an end token $\langle \text{END\_IMG} \rangle$:
\begin{equation}
    P_\theta(\mathcal{I} \mid \mathcal{U}) = \prod_{j=1}^{N_i} P_\theta(i_j \mid \mathcal{U}, i_{<j}).
\end{equation}
Finally, the generated index sequence $\mathcal{I}$ is fed into the decoder $G$, which reconstructs the target image $\hat{x}^{\text{img}} = G(\mathcal{I})$.

\subsection{Hierarchical Visual Perception}  
Given the differences in visual perception between comprehension and generation tasks—where the former focuses on abstract semantics and the latter emphasizes complete semantics—we employ ViT to compress the image into discrete visual tokens at multiple hierarchical levels.
Specifically, the image is converted into a series of features $\{f_1, f_2, \dots, f_L\}$ as it passes through $L$ ViT blocks.

To address the needs of various tasks, the hidden states are divided into two types: (i) \textit{Concrete-grained features} $\mathcal{F}^{\text{Con}} = \{f_1, f_2, \dots, f_k\}, k < L$, derived from the shallower layers of ViT, containing sufficient global features, suitable for generation tasks; 
(ii) \textit{Abstract-grained features} $\mathcal{F}^{\text{Abs}} = \{f_{k+1}, f_{k+2}, \dots, f_L\}$, derived from the deeper layers of ViT, which contain abstract semantic information closer to the text space, suitable for comprehension tasks.

The task type $T$ (comprehension or generation) determines which set of features is selected as the input for the downstream large language model:
\begin{equation}
    \mathcal{F}^{\text{img}}_T =
    \begin{cases}
        \mathcal{F}^{\text{Con}}, & \text{if } T = \text{generation task} \\
        \mathcal{F}^{\text{Abs}}, & \text{if } T = \text{comprehension task}
    \end{cases}
\end{equation}
We integrate the image features $\mathcal{F}^{\text{img}}_T$ and text features $\mathcal{T}$ into a joint sequence through simple concatenation, which is then fed into the LLM $\mathcal{M}_{\text{llm}}$ for autoregressive generation.
% :
% \begin{equation}
%     \mathcal{R} = \mathcal{M}_{\text{llm}}(\mathcal{U}|\theta), \quad \mathcal{U} = [\mathcal{F}^{\text{img}}_T; \mathcal{T}]
% \end{equation}
\subsection{Heterogeneous Knowledge Adaptation}
We devise H-LoRA, which stores heterogeneous knowledge from comprehension and generation tasks in separate modules and dynamically routes to extract task-relevant knowledge from these modules. 
At the task level, for each task type $ T $, we dynamically assign a dedicated H-LoRA submodule $ \theta^T $, which is expressed as:
\begin{equation}
    \mathcal{R} = \mathcal{M}_\text{LLM}(\mathcal{U}|\theta, \theta^T), \quad \theta^T = \{A^T, B^T, \mathcal{R}^T_\text{outer}\}.
\end{equation}
At the feature level for a single task, H-LoRA integrates the idea of Mixture of Experts (MoE)~\cite{masoudnia2014mixture} and designs an efficient matrix merging and routing weight allocation mechanism, thus avoiding the significant computational delay introduced by matrix splitting in existing MoELoRA~\cite{luo2024moelora}. Specifically, we first merge the low-rank matrices (rank = r) of $ k $ LoRA experts into a unified matrix:
\begin{equation}
    \mathbf{A}^{\text{merged}}, \mathbf{B}^{\text{merged}} = \text{Concat}(\{A_i\}_1^k), \text{Concat}(\{B_i\}_1^k),
\end{equation}
where $ \mathbf{A}^{\text{merged}} \in \mathbb{R}^{d^\text{in} \times rk} $ and $ \mathbf{B}^{\text{merged}} \in \mathbb{R}^{rk \times d^\text{out}} $. The $k$-dimension routing layer generates expert weights $ \mathcal{W} \in \mathbb{R}^{\text{token\_num} \times k} $ based on the input hidden state $ x $, and these are expanded to $ \mathbb{R}^{\text{token\_num} \times rk} $ as follows:
\begin{equation}
    \mathcal{W}^\text{expanded} = \alpha k \mathcal{W} / r \otimes \mathbf{1}_r,
\end{equation}
where $ \otimes $ denotes the replication operation.
The overall output of H-LoRA is computed as:
\begin{equation}
    \mathcal{O}^\text{H-LoRA} = (x \mathbf{A}^{\text{merged}} \odot \mathcal{W}^\text{expanded}) \mathbf{B}^{\text{merged}},
\end{equation}
where $ \odot $ represents element-wise multiplication. Finally, the output of H-LoRA is added to the frozen pre-trained weights to produce the final output:
\begin{equation}
    \mathcal{O} = x W_0 + \mathcal{O}^\text{H-LoRA}.
\end{equation}
% In summary, H-LoRA is a task-based dynamic PEFT method that achieves high efficiency in single-task fine-tuning.

\subsection{Training Pipeline}

\begin{figure}[t]
    \centering
    \hspace{-4mm}
    \includegraphics[width=0.94\linewidth]{fig/data.pdf}
    \caption{Data statistics of \texttt{VL-Health}. }
    \label{fig:data}
\end{figure}
\noindent \textbf{1st Stage: Multi-modal Alignment.} 
In the first stage, we design separate visual adapters and H-LoRA submodules for medical unified tasks. For the medical comprehension task, we train abstract-grained visual adapters using high-quality image-text pairs to align visual embeddings with textual embeddings, thereby enabling the model to accurately describe medical visual content. During this process, the pre-trained LLM and its corresponding H-LoRA submodules remain frozen. In contrast, the medical generation task requires training concrete-grained adapters and H-LoRA submodules while keeping the LLM frozen. Meanwhile, we extend the textual vocabulary to include multimodal tokens, enabling the support of additional VQGAN vector quantization indices. The model trains on image-VQ pairs, endowing the pre-trained LLM with the capability for image reconstruction. This design ensures pixel-level consistency of pre- and post-LVLM. The processes establish the initial alignment between the LLM’s outputs and the visual inputs.

\noindent \textbf{2nd Stage: Heterogeneous H-LoRA Plugin Adaptation.}  
The submodules of H-LoRA share the word embedding layer and output head but may encounter issues such as bias and scale inconsistencies during training across different tasks. To ensure that the multiple H-LoRA plugins seamlessly interface with the LLMs and form a unified base, we fine-tune the word embedding layer and output head using a small amount of mixed data to maintain consistency in the model weights. Specifically, during this stage, all H-LoRA submodules for different tasks are kept frozen, with only the word embedding layer and output head being optimized. Through this stage, the model accumulates foundational knowledge for unified tasks by adapting H-LoRA plugins.

\begin{table*}[!t]
\centering
\caption{Comparison of \ourmethod{} with other LVLMs and unified multi-modal models on medical visual comprehension tasks. \textbf{Bold} and \underline{underlined} text indicates the best performance and second-best performance, respectively.}
\resizebox{\textwidth}{!}{
\begin{tabular}{c|lcc|cccccccc|c}
\toprule
\rowcolor[HTML]{E9F3FE} &  &  &  & \multicolumn{2}{c}{\textbf{VQA-RAD \textuparrow}} & \multicolumn{2}{c}{\textbf{SLAKE \textuparrow}} & \multicolumn{2}{c}{\textbf{PathVQA \textuparrow}} &  &  &  \\ 
\cline{5-10}
\rowcolor[HTML]{E9F3FE}\multirow{-2}{*}{\textbf{Type}} & \multirow{-2}{*}{\textbf{Model}} & \multirow{-2}{*}{\textbf{\# Params}} & \multirow{-2}{*}{\makecell{\textbf{Medical} \\ \textbf{LVLM}}} & \textbf{close} & \textbf{all} & \textbf{close} & \textbf{all} & \textbf{close} & \textbf{all} & \multirow{-2}{*}{\makecell{\textbf{MMMU} \\ \textbf{-Med}}\textuparrow} & \multirow{-2}{*}{\textbf{OMVQA}\textuparrow} & \multirow{-2}{*}{\textbf{Avg. \textuparrow}} \\ 
\midrule \midrule
\multirow{9}{*}{\textbf{Comp. Only}} 
& Med-Flamingo & 8.3B & \Large \ding{51} & 58.6 & 43.0 & 47.0 & 25.5 & 61.9 & 31.3 & 28.7 & 34.9 & 41.4 \\
& LLaVA-Med & 7B & \Large \ding{51} & 60.2 & 48.1 & 58.4 & 44.8 & 62.3 & 35.7 & 30.0 & 41.3 & 47.6 \\
& HuatuoGPT-Vision & 7B & \Large \ding{51} & 66.9 & 53.0 & 59.8 & 49.1 & 52.9 & 32.0 & 42.0 & 50.0 & 50.7 \\
& BLIP-2 & 6.7B & \Large \ding{55} & 43.4 & 36.8 & 41.6 & 35.3 & 48.5 & 28.8 & 27.3 & 26.9 & 36.1 \\
& LLaVA-v1.5 & 7B & \Large \ding{55} & 51.8 & 42.8 & 37.1 & 37.7 & 53.5 & 31.4 & 32.7 & 44.7 & 41.5 \\
& InstructBLIP & 7B & \Large \ding{55} & 61.0 & 44.8 & 66.8 & 43.3 & 56.0 & 32.3 & 25.3 & 29.0 & 44.8 \\
& Yi-VL & 6B & \Large \ding{55} & 52.6 & 42.1 & 52.4 & 38.4 & 54.9 & 30.9 & 38.0 & 50.2 & 44.9 \\
& InternVL2 & 8B & \Large \ding{55} & 64.9 & 49.0 & 66.6 & 50.1 & 60.0 & 31.9 & \underline{43.3} & 54.5 & 52.5\\
& Llama-3.2 & 11B & \Large \ding{55} & 68.9 & 45.5 & 72.4 & 52.1 & 62.8 & 33.6 & 39.3 & 63.2 & 54.7 \\
\midrule
\multirow{5}{*}{\textbf{Comp. \& Gen.}} 
& Show-o & 1.3B & \Large \ding{55} & 50.6 & 33.9 & 31.5 & 17.9 & 52.9 & 28.2 & 22.7 & 45.7 & 42.6 \\
& Unified-IO 2 & 7B & \Large \ding{55} & 46.2 & 32.6 & 35.9 & 21.9 & 52.5 & 27.0 & 25.3 & 33.0 & 33.8 \\
& Janus & 1.3B & \Large \ding{55} & 70.9 & 52.8 & 34.7 & 26.9 & 51.9 & 27.9 & 30.0 & 26.8 & 33.5 \\
& \cellcolor[HTML]{DAE0FB}HealthGPT-M3 & \cellcolor[HTML]{DAE0FB}3.8B & \cellcolor[HTML]{DAE0FB}\Large \ding{51} & \cellcolor[HTML]{DAE0FB}\underline{73.7} & \cellcolor[HTML]{DAE0FB}\underline{55.9} & \cellcolor[HTML]{DAE0FB}\underline{74.6} & \cellcolor[HTML]{DAE0FB}\underline{56.4} & \cellcolor[HTML]{DAE0FB}\underline{78.7} & \cellcolor[HTML]{DAE0FB}\underline{39.7} & \cellcolor[HTML]{DAE0FB}\underline{43.3} & \cellcolor[HTML]{DAE0FB}\underline{68.5} & \cellcolor[HTML]{DAE0FB}\underline{61.3} \\
& \cellcolor[HTML]{DAE0FB}HealthGPT-L14 & \cellcolor[HTML]{DAE0FB}14B & \cellcolor[HTML]{DAE0FB}\Large \ding{51} & \cellcolor[HTML]{DAE0FB}\textbf{77.7} & \cellcolor[HTML]{DAE0FB}\textbf{58.3} & \cellcolor[HTML]{DAE0FB}\textbf{76.4} & \cellcolor[HTML]{DAE0FB}\textbf{64.5} & \cellcolor[HTML]{DAE0FB}\textbf{85.9} & \cellcolor[HTML]{DAE0FB}\textbf{44.4} & \cellcolor[HTML]{DAE0FB}\textbf{49.2} & \cellcolor[HTML]{DAE0FB}\textbf{74.4} & \cellcolor[HTML]{DAE0FB}\textbf{66.4} \\
\bottomrule
\end{tabular}
}
\label{tab:results}
\end{table*}
\begin{table*}[ht]
    \centering
    \caption{The experimental results for the four modality conversion tasks.}
    \resizebox{\textwidth}{!}{
    \begin{tabular}{l|ccc|ccc|ccc|ccc}
        \toprule
        \rowcolor[HTML]{E9F3FE} & \multicolumn{3}{c}{\textbf{CT to MRI (Brain)}} & \multicolumn{3}{c}{\textbf{CT to MRI (Pelvis)}} & \multicolumn{3}{c}{\textbf{MRI to CT (Brain)}} & \multicolumn{3}{c}{\textbf{MRI to CT (Pelvis)}} \\
        \cline{2-13}
        \rowcolor[HTML]{E9F3FE}\multirow{-2}{*}{\textbf{Model}}& \textbf{SSIM $\uparrow$} & \textbf{PSNR $\uparrow$} & \textbf{MSE $\downarrow$} & \textbf{SSIM $\uparrow$} & \textbf{PSNR $\uparrow$} & \textbf{MSE $\downarrow$} & \textbf{SSIM $\uparrow$} & \textbf{PSNR $\uparrow$} & \textbf{MSE $\downarrow$} & \textbf{SSIM $\uparrow$} & \textbf{PSNR $\uparrow$} & \textbf{MSE $\downarrow$} \\
        \midrule \midrule
        pix2pix & 71.09 & 32.65 & 36.85 & 59.17 & 31.02 & 51.91 & 78.79 & 33.85 & 28.33 & 72.31 & 32.98 & 36.19 \\
        CycleGAN & 54.76 & 32.23 & 40.56 & 54.54 & 30.77 & 55.00 & 63.75 & 31.02 & 52.78 & 50.54 & 29.89 & 67.78 \\
        BBDM & {71.69} & {32.91} & {34.44} & 57.37 & 31.37 & 48.06 & \textbf{86.40} & 34.12 & 26.61 & {79.26} & 33.15 & 33.60 \\
        Vmanba & 69.54 & 32.67 & 36.42 & {63.01} & {31.47} & {46.99} & 79.63 & 34.12 & 26.49 & 77.45 & 33.53 & 31.85 \\
        DiffMa & 71.47 & 32.74 & 35.77 & 62.56 & 31.43 & 47.38 & 79.00 & {34.13} & {26.45} & 78.53 & {33.68} & {30.51} \\
        \rowcolor[HTML]{DAE0FB}HealthGPT-M3 & \underline{79.38} & \underline{33.03} & \underline{33.48} & \underline{71.81} & \underline{31.83} & \underline{43.45} & {85.06} & \textbf{34.40} & \textbf{25.49} & \underline{84.23} & \textbf{34.29} & \textbf{27.99} \\
        \rowcolor[HTML]{DAE0FB}HealthGPT-L14 & \textbf{79.73} & \textbf{33.10} & \textbf{32.96} & \textbf{71.92} & \textbf{31.87} & \textbf{43.09} & \underline{85.31} & \underline{34.29} & \underline{26.20} & \textbf{84.96} & \underline{34.14} & \underline{28.13} \\
        \bottomrule
    \end{tabular}
    }
    \label{tab:conversion}
\end{table*}

\noindent \textbf{3rd Stage: Visual Instruction Fine-Tuning.}  
In the third stage, we introduce additional task-specific data to further optimize the model and enhance its adaptability to downstream tasks such as medical visual comprehension (e.g., medical QA, medical dialogues, and report generation) or generation tasks (e.g., super-resolution, denoising, and modality conversion). Notably, by this stage, the word embedding layer and output head have been fine-tuned, only the H-LoRA modules and adapter modules need to be trained. This strategy significantly improves the model's adaptability and flexibility across different tasks.


\section{Experiments}
\label{sec:experiments}

% \subsection{Empirically Verifying Magnitude Preservation}

% We

% As per prior work \cite{karras_analyzing_2024, rombach_high-resolution_2022} we use latent diffusion with a pre-trained autoencoder.
\subsection{Datasets}
We apply our method to two datasets: COCO-Stuff and Visual Genome both center cropped and resized to 512x512. We precompute latent images using the same autoencoder as in EDM2\footnote{\href{https://huggingface.co/stabilityai/sd-vae-ft-mse}{https://huggingface.co/stabilityai/sd-vae-ft-mse}}. We also employ random horizontal flipping and save our graph/image pairs to disk. For conditioning variables that involve natural language i.e. class labels, object relationships and image captions we opt to first encode them with a pretrained CLIP-ViT-Large\footnote{\href{https://huggingface.co/openai/clip-vit-large-patch14}{https://huggingface.co/openai/clip-vit-large-patch14}} model to get 768-d vectors for conditioning features. We precompute a vocabulary of class labels, attributes and relationships. This approach allows for generalised representations and for datasets to be fairly unfiltered when compared to previous work. Viewing all datasets through the lens of heterogeneous graphs allows us to jointly train on a variety of different conditioning signals, even when datasets contain different label modalities. Notably, these labels may be represented by vectors of variable length. 
% i.e. the visual genome dataset is typically filtered for object categories occurring at least 2000 times in the train set, along with other filters this results in only 62k of 108k images being used by previous methods.

\begin{table*}[t]
\label{table:metrics}
\centering
\caption{Results comparison between our method with guidance = 1.8 and LayoutDiffusion.}
\vspace{5bp}
\begin{tabular}{@{}lccccccc@{}}
\toprule & \multicolumn{4}{c}{\textbf{COCO-stuff}} & \multicolumn{3}{c}{\textbf{Visual Genome}} \\ 
\cmidrule(l){2-5} 
\cmidrule(l){6-8} 
\textbf{Methods} & \textbf{FID ↓} & \textbf{$\text{FID}_{\textit{DINOv2}}$ ↓} & \textbf{DS ↑} & \textbf{YOLOScore} ↑ & \textbf{FID ↓ }& \textbf{$\text{FID}_{\textit{DINOv2}}$ ↓}& \textbf{DS ↑} \\ \midrule

LayoutDiffusion (128$\times$128) &16.57&-&0.47±0.09&27.00&16.35&-&0.49±0.09\\
LayoutDiffusion (256$\times$256) &15.63&-&0.57±0.10&32.00&15.63&-&0.59±0.10\\
% \cmidrule[0.3pt]{2-4} \cmidrule[0.3pt]{5-6} \\
\textbf{HIG-Medium} (box only) & 11.63 &317.69&\textbf{0.67±0.10} &34.40& 8.99 & 367.89 & 0.68±0.10 \\
\textbf{HIG-Medium} & \textbf{11.42} & 256.93&0.59±0.11&\textbf{41.20}& N/A & N/A & N/A \\
\textbf{HIG-XXL} (box only) & 11.59 &  210.16 & 0.66±0.10 & 28.00 &\textbf{8.79}&213.80&0.64±0.10\\
\textbf{HIG-XXL} & 12.48 & \textbf{198.47}&0.56±0.12& 34.90&N/A&N/A& N/A \\
\bottomrule
\end{tabular}
\end{table*}

\textbf{COCO-stuff} consists of 118k training images. Each image is annotated with a semantic segmentation mask comprising of 183 classes (inc. an unlabeled class), instance bounding boxes and an image caption. For global caption conditioning, we use the same approach used for class labels in the original EDM2 paper but replace class labels with CLIP captions.
To construct graphs we create two types of nodes: mask nodes and instance nodes.
For mask nodes, we extract the number of unique classes present in the semantic mask. Each mask node is connected to the image via its corresponding class assignment (w.r.t the mask) at the latent resolution. Instance nodes are created per bounding box annotation and connected to pixels within the box. Each instance is connected to the class node of the same type. During training time we randomly drop out ($p=0.5$) mask nodes with uniform probability, ensuring the model does not rely too heavily on mask inputs and must infer the rest of the scene when partial masks are provided.
% Finally, we densely interconnect conditioning nodes of the same node type (i.e. class-to-class), to allow communication across higher level abstractions of the image.

\textbf{Visual Genome} consists of 108k training images with 3.8M object instances and 2.3M relationships between them. The dataset primarily comprises: object bounding boxes, attributes and relationships. Due to the volume of classes and different relationships between them, we step away from previous image synthesis research using scene graphs, and instead opt for a strategy of encoding classes, attributes and relationships through CLIP \cite{radford2021learningtransferablevisualmodels}, closer inline with prior text-to-image work. For example, to encode the relationship \textit{pulling(horse, carriage)} we extract the latent representation of ``\textit{pulling}'' from CLIP and use this as the directional edge attribute. We apply the same method to attributes via self-loops. Likewise, instance node features are populated in the same fashion via their class label, and are connected to the image w.r.t to their bounding box information. The HIG representation naturally lends itself to complex data and allows a method to account for overlapping regions and relationships to be naturally represented. We filter the attributes/relationships and objects to remove instances that occur less than 250 and 1000 times respectively, in contrast to previous methods that apply stricter thresholds of 500 and 2000.

\textbf{Implementation Details} 
We train models using 4 x A100 GPUs, using training recipes from EDM2, however, due to limited compute resources we reduce the batch-size from $2048$ to $256$ and half the learning-rate from the proposed values, although not optimal we find this to be quicker to train in practice. We train our medium and XXL base models for a combined total of 12 A100 GPU days each, yet do not witness convergence within this timeframe. Our medium model is comparable, in terms of trainable parameters, with previous SOTA, while the XXL is trained to demonstrate scalability. See Appendix Table \ref{table:model_hparams} for full details.

\subsection{Evaluation Metrics.}
We assess quality, diversity and controllability.

\textbf{Fréchet inception distance} (FID) is the primary evaluation metric for visual quality \cite{heusel2018ganstrainedtimescaleupdate}, measuring the distance between feature distributions of generated and real images. It computes the Fréchet distance using features from Inception-v3 \cite{szegedy2015rethinkinginceptionarchitecturecomputer}  with lower values indicating higher quality and realism. We also report $\text{FID}_{\textit{DINOv2}}$ \cite{oquab_dinov2_2024} which has been observed to align better with human preferences \cite{stein_exposing_2023}.
% Our FID is reported on the same validation splits as previous work ($\sim5$K samples for COCO, $\sim11$K samples for VG), however prior work has excessively filtered these sets to improve performance (i.e. selecting only images with 3-8 objects), a practice we do not adopt. FID is calculated between ground-truth validation splits and x5 generations using generated samples given the validation conditions. 

\textbf{Diversity Score} (DS), as introduced in \cite{zheng_layoutdiffusion_2024}, quantifies the perceptual similarity between two images of the same layout generated using different seeds. Specifically, it leverages the learned perceptual image patch similarity (LPIPS) metric \cite{zhang2018perceptual}. A higher LPIPS value corresponds to greater dissimilarity, meaning that a higher DS is desirable for our application.

% , which measures the distance between image patches in feature space as extracted by an AlexNet \cite{NIPS2012_c399862d}

\textbf{YOLOScore} \cite{LAMA} is used to assess the compliance of the generated samples against prior conditions. To do this, a YOLOv4 \cite{bochkovskiy2020yolov4optimalspeedaccuracy} is used to predict the bounding boxes for a generated sample which are then compared against the ground truth ones, previously used in the HIG for generation. YOLOScore is derived from a combination of several metrics,  including intersection over union, classification accuracy with respect to categorical labels, and confidence scores.

Our metrics are reported on the same validation splits as previous work (5K samples for COCO, 11K samples for VG), however prior work has excessively filtered the sets to improve performance, a practice we do not adopt. FID is calculated between ground-truth validation splits and $\times5$ generations given identical prior conditions. Reported DS values are calculated as the average between all available pair-wise comparisons, and error bars represent the standard deviation between them. Finally, YOLOScore is reported as the average mean precision (\%) of the scores calculated from single corresponding generations. For generations we use a default auto-guidance strength of 1.8, closely in line with the optimal value in EDM2 for $\text{FID}_{\textit{DINOv2}}$, however, as noted in their work, FID considers this optimal value to be a poor choice, and vice versa - see Appendix \ref{fig:fid_appendix}.

\subsection{Quantitative Results}

Quantitative comparisons of our work (at $512\times512$) and the previous SOTA (at $256\times256$) are shown in Table \ref{table:metrics}. For a fair comparison we present results on generated samples with and without mask inputs. We achieve superior performance in all metrics and experiments, despite adopting a less restrictive filtering strategy on validation data. 
% This suggests that our model not only breaks the previous record in terms of quality, but also in terms of controllability.

\begin{figure}
    \centering
    \includegraphics[width=1\linewidth]{icml2023/hig_fig3.pdf}
    \vspace{-20pt}
    \caption{Comparison with SOTA and reference on COCO-stuff.}
    \label{fig:3}    
\end{figure}

\begin{figure}[t]
    \centering
    \includegraphics[width=1\linewidth]{icml2023/hig_fig4.pdf}
    \vspace{-20pt}
    \caption{Generations from the same seed demonstrate HIG's ability to control size, quantity and position of objects.}
    \label{fig:size_editing}    
\end{figure}

Across our presented models there are interesting disparities. We see much higher DS for our bounding box models when compared to our mask models - we interpret this as the increase in spatial control decreases the diversity of the generated images. For YOLOScore, we find the performance improves when the mask input is available, likely due to less ambiguity in the conditioning and overlapping bounding boxes taking less effect. In essence, the outline of the object is given for free which likely improves classification accuracy. We see that our XXL model performs worse on this task, suggesting that our larger models may
% produce higher fidelity images but are
be less controllable. Given compute constraints, the XXL model remains under-trained and did not plateau during training (notably, the original model was pre-trained for 291 GPU days, whereas ours was trained for only 12.) Perhaps given more training time this scenario would change.

Finally, we observe interesting trends in the FID scores. We notice that standard FID prefers our smaller model, whereas $\text{FID}_{\textit{DINOv2}}$ has a much clearer trend and greatly prefers our XXL model. While standard FID is more sensitive to low-level distributional shifts, $\text{FID}_{\textit{DINOv2}}$ better captures high-level semantic consistency, as detailed in the original DINO work \cite{szegedy2015rethinkinginceptionarchitecturecomputer}. Thus, this preference suggests that our XXL model produces images more close to the ground truth in terms of semantic coherence and object fidelity. This is further supported from visual inspection in the head-to-head comparison shown in Appendix Figure \ref{fig:m_vs_xl_comparison}, where high $\text{FID}_{\textit{DINOv2}}$ seems to align more closely with human-perceived quality.

\subsection{Qualitative Results} 

Our comparison with the current state-of-the-art method \cite{zheng_layoutdiffusion_2024}, illustrated in Figure \ref{fig:3}, demonstrates significant improvements in image quality achieved by our approach. Specifically, HIG with bounding boxes generates highly realistic images, outperforming previous SOTA in terms of detail and coherence. Furthermore, HIG with masks excels in accurately preserving the original sub-structures of the image, as explicitly guided by the given masks. This capability makes HIG with masks particularly valuable in scenarios where precise spatial control is essential. Refer to Appendix Figure \ref{fig:appendix_best} for a selection of high quality generated outputs, and Appendix Figures \ref{fig:appendix_best_diversity}-\ref{fig:appendix_best_box} for diverse generations that demonstrate the difference between mask and bounding box generation.

\textbf{Layout and attribute modification} involves altering the HIG condition through semantic and/or spatial alterations. To evaluate the controllability of the model, we conduct a series of custom generations to guide the model to a desired layout. In Figure \ref{fig:size_editing} we test the model's capability in adjusting size, layout and position of objects in a scene. To make a fair test we keep the random generation seed across all runs the same, which also leads to the effect that the image remains mostly unchanged, a desirable effect in image editing. The model generation is highly controllable through adjusting the spatial edges on the HIG, in all three tests.

We also test the model's capability to adjust semantics in the scene, for example modifying attributes (e.g. colour), objects (e.g. category) and relationships (e.g. behind/in front.) Our results in Figure \ref{fig:colour_editing} show unprecedented control, enabling high-quality image generation with precise localized edits (e.g., changing the color of a specific cow.) Furthermore, our model exhibits strong generalizability by effectively generating out-of-distribution object combinations, such as a cat-shaped broccoli, positioned beneath a hat. While object-dependent spatial masks pose challenges when combined with semantic guidance, the model preserves consistency and quality, further demonstrated in Figure \ref{fig:fusion}. Lastly, the model respects relationships between two ambiguous overlapping bounding boxes (e.g., positioning one giraffe in front of another), establishing a new benchmark for structured and context-aware image generation, see Appendix Figures \ref{fig:relationships appendx}-\ref{fig:elephant_appendix} for more examples.

\begin{figure}[t]
    \centering
    \includegraphics[width=1\linewidth]{icml2023/hig_fig5.pdf}
    \vspace{-20pt}
    \caption{HIG enables precise, localized control over semantic conditions, including attributes, objects, and their relationships.}
    \label{fig:colour_editing}
\end{figure}

\begin{figure}[t]
    \centering
\includegraphics[width=1\linewidth]{icml2023/hig_fig6.pdf}
    \vspace{-20bp}
    \caption{We show HIG can effectively generate consistent images, adhering to underlying mask and semantic conditions. In this case we show `HIG' written in trees, food, windows and kites.}
    \label{fig:fusion}
\end{figure}

\subsection{Ablation Experiments}
We conduct a small ablation study on various architecture design choices. We launch smaller training runs to isolate factors of GNN depth, control integration and magnitude preservation. In Table \ref{table:ablation} we show that decreasing the number of HIG blocks degrades performance. We also find that applying magnitude preservation functions to incoming conditioning features is essential for the ControlNet, by only using regular addition (and zero-gain) we find NaNs start to appear in the training loss as feature magnitudes explode. We also investigate whether preserving magnitudes in the GNN is essential or if standard PixNorm suffices. Interestingly, our tests show performance comparable to our base MP-GNN approach. Finally, we experiment whether full-fining is a viable alternative to ControlNet, despite scoring the lowest $\text{FID}_{\textit{DINOv2}}$ we witness poor perceptual quality in comparison to the base model. We leave it to future research to develop upon these results further, for example by training to convergence.
\begin{table}[t]
\centering
\caption{Ablation study results on COCO-stuff at 12M images.} 

\begin{tabular}{@{}lcc@{}}
\toprule

& \textbf{FID ↓} & \textbf{$\text{FID}_{\textit{DINOv2}}$ ↓} \\ \cmidrule{2-3}
HIG-Medium & \textbf{18.34} & 332.00 \\
\midrule 
Depth (2 Layers) & 20.40 & 366.09 \\
Without MP ControlNet & \textit{NaN} & \textit{NaN} \\
Without MP GNN (w/ PixNorm) & 18.56 & 328.67 \\
% Alt. GNN Placement & 20.40 & 366.09 \\
Full Fine-tuning & 19.66 & \textbf{325.297} \\
\bottomrule
\end{tabular}
\label{table:ablation}
\end{table}

\section{Discussion, Limitations and Future Work}
\label{sec:discussion}

Our novel HIG representation enables highly controllable and complex conditioning, outperforming the state-of-the-art in image quality while remaining more computationally efficient by foregoing quadratic attention mechanisms. By structuring images and conditioning variables as a single heterogeneous graph, we incorporate all available data—including attributes and relationships—into a unified representation. This allows for unprecedented control over local attribute conditioning and complex spatial ambiguities, paving the way for larger and more diverse datasets in image synthesis. While we attempt to mitigate randomness by using consistent seeds across experiments, our model still exhibits notable failures in adhering to the prescribed relationships and attribute conditioning, which we attribute to factors such as inconsistent spatial labeling, under-training, and other contributing limitations.
Despite relying on basic graph convolutional operators, our model achieves strong results. Future work will explore more advanced architectures, such as attention-based message passing or higher-order graph operators, which could further enhance expressivity. Moreover, we only utilise models trained on ImageNet as our backbone, we believe a more generalised model would significantly improve our results. More broadly, we believe our approach has implications beyond diffusion models, offering new directions for structured representations in generative modeling.

\clearpage
\newpage

\section*{Impact Statement}

Large-scale image generation models present notable societal risks, including the spread of disinformation and the reinforcement of stereotypes \cite{eiras2024risksopportunitiesopensourcegenerative}. While the methods we introduce in this paper significantly improve the controllability of these models, they may also intensify these risks by making it easier to generate tailored, high-fidelity content that aligns precisely with specified scenes, faces, and events, potentially amplifying the spread of misleading or biased information.

The substantial computational demands required for training and sampling diffusion models should also be acknowledged, as they lead to considerable energy consumption and may further contribute to broader environmental challenges, including climate change.

\section*{Acknowledgments} We would like to acknowledge support from the Engineering and Physical Sciences Research Council (EPSRC) Ph.D. Studentship EP/N509620/1 and the UKRI access to high performance computing facilities program.

\section*{Conflicts of Interest} SWP is co-founder of a spin-out company called Matta that develops AI for Factories.

% E. Negative societal impacts Large-scale image generators such as DALL·E 3, Stable Diffusion XL, or MidJourney can have various negative societal effects, including types of disinformation or emphasizing sterotypes and harmful biases [52]. Our advances to the result quality can potentially further amplify some of these issues. Even with our efficiency improvements, the training and sampling of diffusion models continue to require a lot of electricity, potentially contributing to wider issues such as climate change.

\bibliography{icml}
\bibliographystyle{icml2023}

\renewcommand{\thesection}{A\arabic{section}}
\renewcommand{\thesubsection}{A\arabic{section}.\arabic{subsection}}

\appendix
\clearpage
\onecolumn
\setcounter{page}{1}  % Reset page number if needed
\setcounter{section}{0}  % Reset section numbering to start at A1
\section*{Appendix}


\section{Sum of Random Unit Vectors}
\label{appendix:sum_random}

First, recall that for random zero-mean vectors, their expected Euclidean norm $\mathbb{E}[\|\mathbf{a}\|^2] = \mathbb{E}[a_1^2 + a_2^2 + \dots + a_d^2] $ is related to the variance, since $
\text{Var}(a_i) = \mathbb{E}[a_i^2] - (\mathbb{E}[a_i])^2 = \mathbb{E}[a_i^2]$. Due to the independence assumptions, we can rewrite the expected norm in terms of the variance as $\mathbb{E}[\|\mathbf{a}\|^2] = \mathbb{E}[a_1^2] + \mathbb{E}[a_2^2] + \dots + \mathbb{E}[a_d^2].$

Let us consider the sum of \(N\) independent random \textit{unit} vectors \(\mathbf{c} = \sum_{i=1}^{N} \mathbf{a}_i\), where due to independence zero-mean expectation \(\mathbb{E}[\mathbf{a}_i \mathbf{a}_j] = \mathbb{E}[\mathbf{a}_i]\mathbb{E}[\mathbf{a}_j] = 0\) for \(i \neq j\). Then,

\begin{equation}
\mathbb{E}[\|\mathbf{c}\|^2] = \frac{1}{N_c} \sum_{i=1}^{N_c} \mathbb{E} \left[ \left( \sum_{i=1}^{N}  \mathbf{a}_i \right)^2 \right]
\end{equation}

\begin{equation}
= \frac{1}{N_c} \sum_{i=1}^{N_c} \mathbb{E} \left[ \sum_{i=1}^{N} \mathbf{a}_i^2 + 2 \sum_{i=1}^{N}\sum^{N}_{\substack{j=1 \\ j < i}} \mathbf{a}_i \mathbf{a}_j \right]
\end{equation}

\begin{equation}
= \frac{1}{N_c} \sum_{i=1}^{N_c} \left[ \sum_{i=1}^{N}  \mathbb{E}[\mathbf{a}_i^2] 
+ 2 \sum^{N}_{i=1}\sum^{N}_{\substack{j=1 \\ j < i}} \underbrace{\mathbb{E}[\mathbf{a}_i \mathbf{a}_j]}_{= 0} \right]
\end{equation}


\begin{equation}
= \sum_{i=1}^{N} \mathbb{E}[\mathbf{a}_i^2].
\end{equation}

If the inputs are standardised, this further simplifies to:

\begin{equation}
\mathbb{E}[\|\mathbf{c}\|] = \sqrt{\sum_{i=1}^{N} \mathbb{E}[\mathbf{a}_i^2]} = \sqrt{N}.
\end{equation}

A normalised version of \(\mathbf{c}\) is therefore: 

\begin{equation}
\hat{\mathbf{c}} = \frac{\mathbf{c}}{\mathbb{E}[\|\mathbf{c}\|]} = \frac{\sum_{i=1}^{N} \mathbf{a}_i}{\sqrt{N}}.
\label{eq:normalised_c}
\end{equation}

We use this simple magnitude preserving summation formula for the aggregations across both meta paths ($\Phi$) and for our neighbourhood pooling operation. Since this is trivial to compute on the fly, this allows variable size neighbourhoods to be included into our operator in Equation \ref{eq:hignn_operator} without increasing its magnitude.

\section{EDM2 Mathematical Preliminaries}
\label{appendix:edm2_preliminaries}

We apply a brief summary of relevant equations for reader convenience. For a complete overview, readers should reference \cite{karras_analyzing_2024} and their exhaustive appendix.

We use forced weight magnitude preservation, as per their Equation 47:

\begin{equation}
\hat{w_i} = \frac{w_i}{\|w\|_2 + \epsilon}.
 \label{eq:karras_47}
\end{equation}

Where $\epsilon$ is a small delta to avoid numerical issues. Readers should reference Karras et al. Algorithm 1 for an implementation of forced weight normalisation. Additionally, we utilise magnitude preserving sum (their Equation 88) at several locations in the network, including integration with the control net:

\begin{equation}
\text{MP-Sum}(\mathbf{a}, \mathbf{b}, t) = \frac{(1 - t) \mathbf{a} + t \mathbf{b}}{\sqrt{(1 - t)^2 + t^2}},
 \label{eq:mp_sum}
\end{equation}

and magnitude preserving concatenation (their Equation 103):

\begin{equation}
\text{MP-Cat}(\mathbf{a}, \mathbf{b}, t) = \sqrt{\frac{N_a + N_b}{(1 - t)^2 + t^2}} \cdot \left[ \frac{1 - t}{\sqrt{N_a}} \mathbf{a} \oplus \frac{t}{\sqrt{N_b}} \mathbf{b} \right].
 \label{eq:mp_cat}
\end{equation}

\section{Implementation parameters}
\label{appendix:parameter_table}

% Computing restraints restrict us from doing full scale hyper-parameter tuning, as a rough guide we use default values chosen from the EDM2 paper. However, it is highly likely that these values are not sensible given the very different nature of our training task, for example, EMA length/tuning can contribute significantly to FID - however, with the base parameters we are unable to utilise power EMA, likely due to undertraining. 

\begin{table}[h]
\centering
\caption{Training details for HIGnn diffusion models.}
\begin{tabular}{@{}lcc@{}}
\toprule
\textbf{Model details}                      &\textbf{ M} & \textbf{XXL} \\ \midrule
Number of GPUs                     &  4 &  4   \\
Minibatch size                     &  256 & 256    \\
Duration                           & 40M &  15M   \\
Channel multiplier                 & 256 & 448    \\
Number of HIG blocks  &  4 &  4    \\
Dropout probability                & 10\% &  10\%   \\
Learning rate max ($\alpha_{ref}$)   & 0.0045 & 0.003    \\
Learning rate decay ($t_{ref}$)      & 10000 & 10000   \\
Noise distribution mean ($P_{mean}$) &  -0.4 &  -0.4   \\
Noise distribution std. ($P_{std}$)  &  1.0 &  1.0    \\ 
Base Model capacity (Mparams)  &  497.8 &  1523.2    \\
ControlNet capacity (Mparams)  &  163.2 &  495.3   \\
HIGnn capacity (Mparams)  &  7.0 &  18.9    \\
Total capacity (Mparams)  &  668.0 &  2037.5    \\
Training time (days) &  3.0 &  3.0    \\
EMA &  None &  None    \\
\bottomrule
\end{tabular}
\label{table:model_hparams}
\end{table}


\newpage

\begin{figure}
    \centering
    \includegraphics[width=1\linewidth]{icml2023/appendix_selected.pdf}
    \caption{Selected generated samples from HIG using COCO validation conditions.}
    \label{fig:appendix_best}
\end{figure}

\begin{figure}
    \centering
    \includegraphics[width=0.5\linewidth]{icml2023/hig_fid_guidance.pdf}
    \caption{\text{FID} and $\text{FID}_{\textit{DINOv2}}$ vs guidance. To understand the relationship between auto-guidance strength and FID values we generate 5K images for coco validation (i.e. one seed). The relationship we witness is similiar to that reported in the original \cite{karras_analyzing_2024} work, they report optimal FID at 1.4, and optimal $\text{FID}_{\textit{DINOv2}}$ at 1.9. Our optimal values are both slightly lower than this.}
    \label{fig:fid_appendix}
\end{figure}

\newpage

\begin{figure}
    \centering
    \includegraphics[width=1\linewidth]{icml2023/appendix_div_mask.pdf}
    \caption{Diverse HIG samples using mask and bounding boxes from COCO validation set.}
    \label{fig:appendix_best_diversity}
\end{figure}

\newpage

\begin{figure}
    \centering
    \includegraphics[width=1\linewidth]{icml2023/best_selected_bounding_box.pdf}
    \caption{Diverse HIG samples using only bounding boxes from the COCO validation set. We naturally observe increased diversity in images when compared to Figure \ref{fig:appendix_best_diversity} that uses mask inputs.}
    \label{fig:appendix_best_box}
\end{figure}

\clearpage
\newpage

\begin{figure*}
    \centering
    \includegraphics[width=1\linewidth]{icml2023/hig_a1.pdf}
    \caption{Head-to-head comparisons between generations from HIG-Medium and HIG-XXL. There is a notable enhancement in realism in the latter, which exhibit richer color depth and more convincing surface textures.}
    \label{fig:m_vs_xl_comparison}
\end{figure*}

\begin{figure*}
    \centering
    \includegraphics[width=1\linewidth]{icml2023/appendix_animal_closeup.pdf}
    \caption{Selected generations using a single bounding box with an animal class label.}
    \label{fig:animal_appendix}
\end{figure*}   


\clearpage
\newpage

\begin{figure*}
    \centering
    \includegraphics[width=1\linewidth]{icml2023/appendix_tree_rel.pdf}
    \caption{We showcase the ability of the HIG to disambiguate overlapping bounding boxes. In this figure we show bear, horse, elephant and cow bounding boxes overlapped with a tree bounding box,  with the relationship `in front' or 'behind'. We also show the bounding box when the tree is removed.}
    \label{fig:relationships appendx}
\end{figure*}   

\begin{figure*}
    \centering
    \includegraphics[width=1\linewidth]{icml2023/appendix_color_animal.pdf}
    \caption{We showcase the ability to locally edit the attribute of an object i.e. the colour of a specific animal in an image.}
    \label{fig:animal_color_appendix}
\end{figure*}   


\begin{figure*}
    \centering
    \includegraphics[width=1\linewidth]{icml2023/appendix_editing_local.pdf}
    \caption{We showcase the ability to locally edit the attribute of an object i.e. the colour of a specific bus or flower in a generation.}
    \label{fig:color_appendix_1}
\end{figure*}   

\begin{figure*}
    \centering
    \includegraphics[width=1\linewidth]{icml2023/appendix_color_obj.pdf}
    \caption{Generations with different colour attributes. }
    \label{fig:color_appendix_2}
\end{figure*}   

\begin{figure*}
    \centering
    \includegraphics[width=1\linewidth]{icml2023/appendix_donut.pdf}
    \caption{Editing a scenes layout. We observe that when generating images with the same seed that object semantics stay relatively consistent as we add or remove objects from a scene. }
    \label{fig:donut_appendix}
\end{figure*}   

\begin{figure*}
    \centering
    \includegraphics[width=1\linewidth]{icml2023/appendix_elephant.pdf}
    \caption{We showcase the ability to change the size and position of an object in a scene.}
    \label{fig:elephant_appendix}
\end{figure*}  






\end{document}


% This document was modified from the file originally made available by
% Pat Langley and Andrea Danyluk for ICML-2K. This version was created
% by Iain Murray in 2018, and modified by Alexandre Bouchard in
% 2019 and 2021 and by Csaba Szepesvari, Gang Niu and Sivan Sabato in 2022.
% Modified again in 2023 by Sivan Sabato and Jonathan Scarlett.
% Previous contributors include Dan Roy, Lise Getoor and Tobias
% Scheffer, which was slightly modified from the 2010 version by
% Thorsten Joachims & Johannes Fuernkranz, slightly modified from the
% 2009 version by Kiri Wagstaff and Sam Roweis's 2008 version, which is
% slightly modified from Prasad Tadepalli's 2007 version which is a
% lightly changed version of the previous year's version by Andrew
% Moore, which was in turn edited from those of Kristian Kersting and
% Codrina Lauth. Alex Smola contributed to the algorithmic style files.
