\begin{abstract}
Deploying large language models (LLMs) in real-world applications requires robust safety guard models to detect and block harmful user prompts. While large safety guard models achieve strong performance, their computational cost is substantial. To mitigate this, smaller distilled models are used, but they often underperform on ``hard'' examples where the larger model provides accurate predictions. We observe that many inputs can be reliably handled by the smaller model, while only a small fraction require the larger model’s capacity. Motivated by this, we propose \textbf{SafeRoute}, a binary router that distinguishes hard examples from easy ones.  Our method selectively applies the larger safety guard model to the data that the router considers hard, improving efficiency while maintaining accuracy compared to solely using the larger safety guard model. Experimental results on multiple benchmark datasets demonstrate that our adaptive model selection significantly enhances the trade-off between computational cost and safety performance, outperforming relevant baselines.

\vspace{-0.2em}
\centering\textcolor{red}{\textbf{Warning: This paper contains potentially harmful language model outputs.}}
\end{abstract}

