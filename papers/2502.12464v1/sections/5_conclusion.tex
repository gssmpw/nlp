\section{Conclusion}
In this work, we proposed training a binary router, SafeRoute, that adaptively selects either a larger or smaller safety guard model based on the difficulty of the input data. This approach improved the trade-off between computational overhead and accuracy gains compared to other relevant baselines on several benchmark datasets. While we focused on the dynamic selection of safety guard models with different sizes, our approach is not limited to prompt-response pair classification. An interesting direction for future work is extending this method to other tasks, such as reasoning or programming.

% \newpage
\section*{Limitations}
Although our proposed adaptive selection between a smaller and a larger safety guard model significantly improves the trade-off between accuracy gains and computational overhead compared to other baselines, it has some limitations. First, the current parameterization of the binary classifier $f_\theta$ does not encode what the larger model knows, limiting its generalization performance. In our preliminary experiments, we incorporated representations of the larger model as part of the classifier’s input. While this improved accuracy, it introduced significant computational overhead, making the approach even slower than using the larger model alone. Approximating the larger model's features in an efficient manner would be an interesting direction as future work. Another limitation is that the performance of our selection mechanism is highly dependent on the quality and representativeness of the training data for $f_\theta$. If the training dataset does not adequately capture the diversity of prompt-response pairs --- particularly those at the boundary between easy and hard instances --- the classifier may make suboptimal decisions. Steering LLMs to generate diverse and high-quality data is another promising avenue for future work.


\section*{Ethics Statement}
Our proposed method, SafeRoute, aims to improve the trade-off between efficiency and accuracy gains of safety guard models in large language model (LLM) deployment. We do not foresee any direct ethical concerns arising from the use of SafeRoute, as it functions solely as an adaptive mechanism for selecting between smaller and larger models based on their predictive performance across different input types. By doing so, it ensures a more efficient deployment while maintaining high safety performance, reducing computational overhead without compromising the ability to detect harmful inputs. We are committed to the responsible use of LLMs and the enhancement of safety mechanisms, ensuring that no additional harm is introduced by our approach. All experiments were conducted with publicly available benchmark datasets. 