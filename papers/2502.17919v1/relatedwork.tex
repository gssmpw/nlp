\section{Related Work}
\label{related_work}
% In the recent years, many domains have been integrated with machine learning methods and forecasting air pollutions is one of them. Though physics based technies have been present for the said task, commonly are WRF-Chem, CMAQ. And due to complex nature of air pollution variables, it is very hard to model with physics based models, inducing high uncertaintis and low predcition accuracy. With more effective principles and characteristics, data driven machine learning models are more effective. Due to non linear features in air pollution variables, data driven models are popular choice among the available options.
% Physics based models are grounded with the principles of atmospheric chemistry and physics. Capturing the weather patterns along with air pollutions has been notable things to make air pollution forecasting more effectuve but it is challenging to set accurate initial conditions and along with that computational expense is high for high resolution and complex environments.

In recent years, the integration of machine learning methods into various scientific domains has gained significant attention, with air pollution forecasting a notable example. 
Traditionally, physics-based models like the WRF-Chem \citep{ojha2020widespread} and CMAQ \citep{zhang2012source} model have been employed to predict air pollution levels. 
These models are grounded in the fundamental principles of atmospheric chemistry and physics to simulate complex interactions within the atmosphere.
However, the highly nonlinear and complex nature of air pollution variables poses substantial challenges for these physics-based models. 
Modeling air pollution's complexity often leads to high uncertainties and reduced prediction accuracy \citep{hao2020spatiotemporal, li2019air}. 
Additionally, running these models at high resolutions in complex environments demands significant computational resources, which can be challenging for real-time forecasting.

In contrast, data-driven machine learning models \citep{yu2022deep, cai2023forecasting, bodnar2024aurora} have emerged as a more effective alternative for air pollution forecasting. These models excel at capturing nonlinear relationships and patterns within large datasets, allowing them to handle the complexities of air pollution variables more adeptly. Machine learning approaches can provide more accurate and efficient predictions without requiring detailed physical simulations by learning directly from the data. 
% Consequently, they have become a popular choice aiming to improve air quality forecasts. 
In \cite{yu2022deep}, a deep ensemble-based approach is introduced for estimating daily PM2.5 concentrations. This framework leverages machine learning base models, such as XGBoost, which are used to train meta-models in the second stage, with an optimization algorithm applied in the third stage.
In \cite{cai2023forecasting}, authors proposed a framework to enhance the prediction of hourly PM2.5 concentrations. Their method involves breaking down complex data into simpler components, each representing different frequency levels. These components are then modeled using a combination of autoregressive and CNN-based methods to capture patterns in data, to improve the accuracy of the predictions.


\begin{table*}[t]
\caption{A list of all the weather and air quality variables present in our dataset. Furthermore, for variables that contain data at different pressure levels, we collect 7 of them.}
\label{tab:variables}
\adjustbox{max width=\linewidth}{
\begin{tabular}{@{}llll@{}}
\toprule
& Variable (short name)                                    & Description                                                & Pressure Levels       \\ \midrule
\multirow{9}{*}{\rotatebox[origin=c]{90}{Weather Variables}} &
geopotential (z)                              & Varies with the height of a pressure level                 & 7 levels     \\&
temperature (t)                               & Temperature                                                & 7 levels     \\&
specific humidity (q)                         & Mixing ratio of water vapor                                & 7 levels     \\&
relative humidity (r)                         & Humidity relative to saturation                            & 7 levels     \\&
u component of wind (u)                       & Wind in longitude direction                                & 7 levels     \\&
v component of wind (v)                       & Wind in latitude direction                                 & 7 levels     \\&
2m temperature (t2m)                          & Temperature at 2m height above surface                     & Single level \\&
10m u component of wind (u10)                 & Wind in longitude direction at 10m height                  & Single level \\&
10m v component of wind (v10)                 & Wind in latitude direction at 10m height                   & Single level \\ \midrule
\multirow{12}{*}{\rotatebox[origin=c]{90}{Air Quality Variables}} &
carbon monoxide (co)                          & Carbon monoxide concentrations                             & 7 levels     \\&
ozone (go3)                                   & Ozone concentrations                                       & 7 levels     \\&
Nitrogen monoxide (no)                        & Nitrogen monoxide concentrations                           & 7 levels     \\&
Nitrogen dioxide (no2)                        & Nitrogen dioxide concentrations                            & 7 levels     \\&
Sulphur dioxide (so2)                         & Sulphur dioxide concentrations                             & 7 levels     \\&
Particulate matter d \textless 1 µm (pm1)     & Particulate matter with diameter less than 1 µm            & Single level \\&
Particulate matter d \textless 10 µm (pm10)   & Particulate matter with diameter less than 10 µm           & Single level \\&
Particulate matter d \textless 2.5 µm (pm2.5) & Particulate matter with diameter less than 2.5 µm          & Single level \\&
Total column carbon monoxide (tcco)           & Total amount overall levels & Single level \\&
Total column nitrogen monoxide (tc\_no)       & Total amount overall levels & Single level \\&
Total column nitrogen dioxide (tcno2)         & Total amount overall levels & Single level \\&
Total column ozone (gtco3)                    & Total amount overall levels & Single level \\ \bottomrule
\end{tabular}
}
\end{table*}


Recent advancements in neural network architectures have significantly improved the integration of air quality variables with weather variables in forecasting models. 
For instance, the Aurora model \citep{bodnar2024aurora} primarily trains on weather data and then forecasts air pollution levels as a downstream task.
Similarly, ClimaX \citep{nguyen2023climaxfoundationmodelweather} is an open-source weather model that leverages the vision transformer architecture \citep{dosovitskiy2021imageworth16x16words}. 
Trained on large-scale datasets, it serves as a foundational model by employing a pretext task focused on predicting future time steps randomly sampled within a specified range. 
In its downstream applications, ClimaX handles a variety of tasks across different spatial and temporal scales, including regional weather forecasting. Due to its adaptable and efficient design, we have selected ClimaX to demonstrate our proposed approach. 
A recent study \citep{munir2024efficient}, has explored enhancing ClimaX for MENA weather forecasting using parameter-efficient fine-tuning like LoRA. This underscores the potential of adapting foundational models to meet the challenges of specific regions.
However, it's important to note that ClimaX currently operates at a lower spatial resolution compared to models like Aurora \citep{bodnar2024aurora} and CAMS \citep{CAMS23}. In contrast to large-scale foundational models, existing work on PM forecasting typically employs fewer variables, limiting itself to smaller capacity models or statistical approaches \citep{CabelloTorres2022, Masood2023-mc}. A notable exception is the work of \cite{Sarafian2023-mx}, which uses a transformer-based model to forecast PM10 concentrations using several weather variables, demonstrating their importance in the process. However, most studies in this field, while valuable, often focus on predicting a single PM concentration variable and utilize only a subset of available predictors. Our work aims to address these limitations by adopting a more comprehensive, multi-variable approach.

% In AirCast, we integrate air quality variables directly with weather variables. This integration allows us to overcome the limitations of traditional physics-based models, which may not fully capture complex interactions and data-driven methods that consider only one type of data. 
% By combining both air quality and weather data, our approach is better equipped to model the complex environmental dynamics. This is particularly significant for regions like the MENA, where complex environmental factors demand models capable of effectively handling a diverse set of variables.
In AirCast, we combine air quality and weather data to better capture complex environmental dynamics, addressing the limitations of models that rely on one type of data. This approach is important for regions like the MENA, where diverse environmental factors require models capable of handling intricate interactions effectively.

\begin{figure*}[t]
  \centering
    \includegraphics[width=\textwidth]{figures/model_diagram_vaf1.pdf}
  \caption{This illustrates the architecture of the \textbf{AirCast} model, an extension of~\cite{nguyen2023climaxfoundationmodelweather}. The model integrates weather data from the ERA5 dataset and air quality data from the CAMS EAC4 dataset. The model is trained using regional data from the MENA region. The input variables are tokenized and aggregated, with a Vision Transformer (ViT) encoder, processing the combined weather and air quality inputs. 
  A dual decoder head is employed, with one predicting weather variables and the other forecasting air quality variables. The predictions are compared with the ground truth at a certain lead time using the Frequency-Weighted MAE loss function. }
  % \vspace{-0.9em}
  \label{fig:method}
\end{figure*}