The analysis in Section 2 provides the motivation for a focused investigation into the use of generalisable stylistic features as well as the novel social-monetisation features that may lend themselves to improved generalisability (the ability of a model to perform well when tested on a different dataset than the one on which it was trained) of fake news detection models. As such, the objectives of this study can be formalised into the following three research questions:


\begin{itemize}
    \item \textbf{RQ1.} How well do fake news detection methods using token-\\representations/LLMs generalise? 
    \item \textbf{RQ2.} Do fake news detection methods using stylistic features generalise better than fake news detection models using token-representations/LLMs?
    \item \textbf{RQ3.} Do fake news detection methods using stylistic features and the proposed social-monetisation features generalise better than models using stylistic features only?  
\end{itemize}

