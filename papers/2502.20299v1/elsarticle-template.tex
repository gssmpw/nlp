%% 
%% Copyright 2007-2020 Elsevier Ltd
%% 
%% This file is part of the 'Elsarticle Bundle'.
%% ---------------------------------------------
%% 
%% It may be distributed under the conditions of the LaTeX Project Public
%% License, either version 1.2 of this license or (at your option) any
%% later version.  The latest version of this license is in
%%    http://www.latex-project.org/lppl.txt
%% and version 1.2 or later is part of all distributions of LaTeX
%% version 1999/12/01 or later.
%% 
%% The list of all files belonging to the 'Elsarticle Bundle' is
%% given in the file `manifest.txt'.
%% 
%% Template article for Elsevier's document class `elsarticle'
%% with harvard style bibliographic references

\documentclass[preprint,12pt,authoryear]{elsarticle}

%% Use the option review to obtain double line spacing
%% \documentclass[authoryear,preprint,review,12pt]{elsarticle}

%% Use the options 1p,twocolumn; 3p; 3p,twocolumn; 5p; or 5p,twocolumn
%% for a journal layout:
%% \documentclass[final,1p,times,authoryear]{elsarticle}
%% \documentclass[final,1p,times,twocolumn,authoryear]{elsarticle}
%% \documentclass[final,3p,times,authoryear]{elsarticle}
%% \documentclass[final,3p,times,twocolumn,authoryear]{elsarticle}
%% \documentclass[final,5p,times,authoryear]{elsarticle}
%% \documentclass[final,5p,times,twocolumn,authoryear]{elsarticle}

%% For including figures, graphicx.sty has been loaded in
%% elsarticle.cls. If you prefer to use the old commands
%% please give \usepackage{epsfig}

%% The amssymb package provides various useful mathematical symbols
\usepackage{amssymb}
%% The amsthm package provides extended theorem environments
%% \usepackage{amsthm}

%% The lineno packages adds line numbers. Start line numbering with
%% \begin{linenumbers}, end it with \end{linenumbers}. Or switch it on
%% for the whole article with \linenumbers.
%% \usepackage{lineno}

%%Package for merging cells in tables

\usepackage{multirow}

\usepackage{adjustbox}
\usepackage[table]{xcolor}
\usepackage{chngpage}
\usepackage{xcolor}
\usepackage{appendix}

\journal{Expert Systems with Applications}

\begin{document}

\begin{frontmatter}

%% Title, authors and addresses

%% use the tnoteref command within \title for footnotes;
%% use the tnotetext command for theassociated footnote;
%% use the fnref command within \author or \affiliation for footnotes;
%% use the fntext command for theassociated footnote;
%% use the corref command within \author for corresponding author footnotes;
%% use the cortext command for theassociated footnote;
%% use the ead command for the email address,
%% and the form \ead[url] for the home page:
%% \title{Title\tnoteref{label1}}
%% \tnotetext[label1]{}
%% \author{Name\corref{cor1}\fnref{label2}}
%% \ead{email address}
%% \ead[url]{home page}
%% \fntext[label2]{}
%% \cortext[cor1]{}
%% \affiliation{organization={},
%%            addressline={}, 
%%            city={},
%%            postcode={}, 
%%            state={},
%%            country={}}
%% \fntext[label3]{}

\title{An Exploration of Features to Improve the Generalisability of Fake News Detection Models}

%% use optional labels to link authors explicitly to addresses:
%% \author[label1,label2]{}
%% \affiliation[label1]{organization={},
%%             addressline={},
%%             city={},
%%             postcode={},
%%             state={},
%%             country={}}
%%
%% \affiliation[label2]{organization={},
%%             addressline={},
%%             city={},
%%             postcode={},
%%             state={},
%%             country={}}



\author[inst1]{Mr. Nathaniel Hoy\corref{cor1}}
\ead{nathaniel.hoy2@brunel.ac.uk}

\author[inst1]{Dr. Theodora Koulouri}
\ead{theodora.koulouri@brunel.ac.uk}

\affiliation[inst1]{organization={Department Of Computer Science, Brunel University London},%Department and Organization
            addressline={Kingston Lane}, 
            city={Uxbridge},
            postcode={UB83PH}, 
            state={Middlesex},
            country={United Kingdom}}

\cortext[cor1]{Corresponding author}

\begin{abstract}
%% Text of abstract
Fake news poses significant global risks by influencing elections and spreading misinformation, making its detection a critical area of research. Existing approaches, primarily using Natural Language Processing (NLP) and supervised Machine Learning, achieve strong results under cross-validation and hold-out testing but struggle to generalise to other datasets, even those within the same domain. This limitation stems from reliance on coarsely labelled training datasets, where articles are often labelled based on their publisher, introducing biases that token-based representations such as TF-IDF and BERT are sensitive to. While Large Language Models (LLMs) represent a promising development in NLP, their application to fake news detection remains limited. This study demonstrates that meaningful features can still be extracted from coarsely labelled datasets to improve model robustness for real-world scenarios. Stylistic features, including lexical, syntactic, and semantic attributes, are explored as an alternative due to their reduced sensitivity to dataset biases. In addition, novel `social-monetisation' features are introduced, capturing economic incentives behind fake news, such as the presence of advertisements, external links, and social media sharing elements. The study employs the coarsely labelled NELA 2020-21 dataset for training and the manually labelled Facebook URLs dataset for external validation, representing a gold standard for evaluating model generalisability. The results highlight the limitations of token-based models, when trained on coarsely labelled data. Additionally, this study contributes to the limited evidence on the performance of LLMs such as LLaMa in this domain. The findings indicate that stylistic features, complemented by social-monetisation attributes, provide more generalisable predictions for real-world scenarios in comparison to token-based methods and LLMs. Statistical and permutation feature importance analyses further reveal the potential of these features to enhance performance and address dataset biases, offering a path forward for improving fake news detection models.
\end{abstract}

\begin{highlights}

\item Demonstrated poor generalisability of token-representations/LLMs on real-world data

\item Established stylistic features lead to more generalisable and balanced models

\item Proposed social-monetisation features increased accuracy across datasets

\item Established a simplified feature-set matching comprehensive set's performance

\end{highlights}

\begin{keyword}
%% keywords here, in the form: keyword \sep keyword
fake news \sep misinformation \sep machine learning \sep natural language processing \sep feature engineering \sep generalizability
%% PACS codes here, in the form: \PACS code \sep code
%% MSC codes here, in the form: \MSC code \sep code
%% or \MSC[2008] code \sep code (2000 is the default)
\end{keyword}

\end{frontmatter}

%% \linenumbers

%% For citations use: 
%%       \citet{<label>} ==> Jones et al. (2015)
%%       \citep{<label>} ==> (Jones et al., 2015)

%% main text
\section{Introduction}
\documentclass[../main.tex]{subfiles}
\graphicspath{{../images/}}
\makeatletter
\def\input@path{{../images/}}
\makeatother
\begin{document}
\section{Introduction}
\begin{figure}
\centering
\begin{tikzpicture}
\node[inner sep=0pt] (ws) at (0, 0) {
\includegraphics[height=.4\textwidth, trim={10cm 0 10cm 0},clip]{world_space.png}};
\node[inner sep=0pt] (cs) at (6,0) {\includegraphics[height=.4\textwidth, trim={10cm 1cm 10cm 4cm},clip]{conf_space.png}};
\end{tikzpicture}
\vspace{-5pt}
\label{fig:pbrm_intro}
\caption{\textbf{Left}: Shows world space obstacles as grey spheres. Robots start and goal configuration is colored red and green, respectively. Configurations along the computed path are colored transparent blue. \textbf{Right:} Mapped world space scenario to configuration space. Obstacle region is the grey mesh. Red spheres are collision-free regions computed by the neural SCDF. The optimized shortest path in the convex corridor is the blue curve.}
\vspace{-25pt}
\end{figure}
Motion planning is the problem of finding a collision-free trajectory that connects a given start and goal configuration. The planning takes place in the configuration space of the robot. For single body robots, like mobile robots or drones, the configuration space and the world space are usually the same. This simplifies the planning, since explicit obstacle representations are available which enables geometrical tools like separating hyperplanes, smallest distance to obstacles etc., to be used when designing motion planning algorithms. For multi-body robots like manipulators, the situation is completely different. The world space obstacles are usually mapped to non-convex regions, and to make the problem even harder, the mapping is usually not known. Forming explicit representations of the obstacle region in the configuration space is usually too expensive or intractable. Despite all of this, sampling based planners are used with great success, which mainly is due to their use of implicit representations of the obstacle region. The basic idea is to construct a graph in the configuration space that covers and connects the collision-free region. From this graph, a path can be extracted that connects a given start and goal configuration. The approach is computationally expensive, since the graph is constructed with the smallest geometrical building block available, points, which represents a collision-check. Furthermore, the extracted paths from the graph are non-smooth and jagged due to the stochastic nature of the approach. This adds an additional post-processing step to the process, where the paths are shortcutted and smoothened, before the path can be used for tracking. Clearly a lot of time is invested to form this graph and produce smooth paths. Thus, if the obstacles start to move, then all of this work is done in no use, since all points that make up this graph need to be re-verified, which is simply too time consuming to be done in real time.
\\\\
In this work, we want to address the existing drawbacks of the sampling based planners. Our main contribution is an improved motion planner where each vertex in the graph covers a collision-free region in the form of a sphere instead of a point and where the edges are formed with neighboring intersecting spheres. This representation has the advantage of instead of returning piecewise linear paths, returning a sequence of overlapping spheres, i.e. a convex corridor, that connects a given start and goal configuration, illustrated in Figure \ref{fig:pbrm_intro}. This convex corridor allows us to use convex optimization to produce smooth trajectories, instead of computationally expensive post-processing methods. The representation further allows us to estimate the coverage of the collision-free space, which gives us awareness and feedback in the offline roadmap construction phase. Finally, our representation is simple to adapt to moving obstacles, simply requery for the new radii and recheck for intersections. 
\\\\
The spherical collision-free regions are formed using a signed distance function (SDF), which is a function that returns the smallest distance from an arbitrary point to the boundary of an obstacle. As the name implies, the distance is signed, thus if the point is inside the obstacle it is negative otherwise positive. If the distance is positive, a sphere with radius equal to the distance is guaranteed to cover a collision-free region. Using an SDF in motion planning is not new, but what is novel about our approach is that we express the distance in the configuration space instead of the world space and by doing so allows us to form these convex collision-free regions. We refer to the resulting SDF as a signed configuration distance function (SCDF). Computing an SCDF analytically is non-trivial, our approach is therefore to parameterize the SCDF with a deep neural network and learn the mapping by supervised learning. Our resulting neural SCDF can compute distances for different parameter values of obstacle shapes and we also show how multiple distances can be combined, thus making our approach flexible.
\section{Related work}
Motion planning algorithms can roughly be divided into three families, grid-based, sampling based and optimization based methods. Grid-based methods (GBM) discretize the planning space from which a graph is then compiled. A standard search method is A$^\star$ \citep{a_star}, which is classified as an \textit{informed} search method, since it employs a heuristic function to speed up the search. A$^\star$ guarantees to return an optimal path at the level of discretization used. GBMs usually discretize the planning space by a regular lattice and this limits the GBMs to problems with low dimensionality due to the curse of dimensionality. Thus, GBMs are usually limited to single-body robots where the degrees of freedom (DOF) are low. To overcome the inherent scaling problem with the GBMs, stochastic methods are usually used for multi-body robots. These methods are termed as sampling-based methods (SBM) and core members within this family are the rapidly-exploring random trees (RRT) \citep{rrt} and the probabilistic roadmap (PRM) \citep{prm}. RRT grows a tree from the start configuration and explores the collision-free region in a rapid way until it is able to connect to the goal region. RRT is usually improved by bi-directional planning \citep{rrt_connect}, i.e. an additional tree is grown from the goal configuration and the trees are tested for connection after any tree has been expanded. RRT is a single-query method, thus it searches for a path from scratch each time it is queried. Contrary to this, PRM is a multi-query method, which solves for multiple queries without starting from scratch. PRM does this by creating a roadmap (graph) that covers the collision-free space as an offline step. The graph is then used to solve for multiple queries. PRMs are used in cases where the environment does not change since the extra offline step is too computationally costly and needs to be re-done if the environment is changed. In our work, we address this inherent issue by using a different roadmap representation. Our vertices in the graph cover a collision-free region in the form of spheres and we form the edges by checking for intersecting spheres. If something in the environment changes, we recompute the spheres radii and recheck the intersections, without relying on collision detection. We use a trained neural network to compute the sphere radius, therefore querying for the radius can be done fast, hence our representation enables the PRM for dynamic environments.
\\\\
In the recent decades, optimization based methods (OBM) \citep{chomp, schulman, itomp, stomp} have been introduced as an alternative to SBM for multi-body robots. Like the SBM, the OBMs scale well to higher dimensional problems and produce smoother motion. It is common to use a SDF in the optimization since it is a smooth function, thus enabling gradient-based methods. However, the standard way of expressing the SDF is in world space. The distance therefore needs to be mapped to the configuration space by the forward kinematics. This mapping makes the optimization problem a non-linear program (NLP), which is computationally expensive to solve. Recently, a different approach has been proposed. In \cite{mp_gcs} motion planning is formulated as a convex optimization problem by using the graph of convex sets framework \citep{gcs}. The underlying idea is to decompose the collision-free space into intersecting convex sets from which a convex optimization problem is formulated. In cases where an explicit representation of the obstacles in the configuration space exists, like for single-body robots, creating collision-free convex regions can be done fast \citep{iris}. For multi-body robots, this is non-trivial. Existing work does this successfully \citep{iris_nlp, iris_c} by an optimization based approach, but the methods are still too time consuming to be used in the presence of moving obstacles. Our approach is instead to use deep learning to learn an SDF expressed in the configuration space. With this, we can query for shortest distances to the collision boundary, which allows us to expand spherical regions which are collision-free. Our approach is fast and therefore enables our suggested roadmap planner to be used in dynamic environments.
\\\\
Recent research has focused on learning collision detection \citep{fk_kernel_distance, diffco, graphdistnet} by predicting the signed distance between the robot links and the surrounding obstacles in the world space. The learned SDF is used in trajectory optimization but since the distance is expressed in the world space, the problem becomes an NLP and therefore takes a long time to solve. We take a novel approach and suggest to instead express the signed distance in the configuration space. This allows us to improve the PRM at the same time as it enables convex optimization for trajectory optimization, which runs faster and is more reliable than NLP solvers. In \cite{cspf} a learned signed distance function in the configuration space is proposed similar to our approach. However, their approach is restricted to point cloud representations, while we propose to represent the obstacles as parameterized geometric shapes, e.g. spheres. Furthermore, we also show how to use our learned SCDF to improve an existing roadmap planner.
\section{Problem formulation}
A robot is located in the world space, $\W \subset \R^3 $. The unique location of the robot is given by its configuration $\q \in \C$, where $\C$ is the configuration space. The set of points covered by the robots bodies at a certain configuration is expressed as $\B(\q) \subset \W$. The robot is surrounded by $\NrObst$ obstacles $\O = \bigcup_{i=1}^{\NrObst} \O_i$, where  $\O_i \subset \W$. The representation of the obstacle in the configuration space is the set $\C\O_i = \{\q \in \C \: |\: \B(\q) \cap \O_i \neq \emptyset \}$. The obstacle space is formed as $\Co = \bigcup_{i=1}^{\NrObst} \C \O_i$. The complement is referred to as the free space, $\Cf = \C \setminus \Co$. The path planning problem is a tuple, ($\Cf$, $\qStart$, $\qGoal$), where we want to connect a query pair, consisting of a start, $\qStart$, and goal configuration, $\qGoal$, with a geometric path, $\q(s): [0, 1] \mapsto \Cf$, such that $\q(0)=\qStart$ and $\q(1)=\qGoal$, or report correctly when such a path does not exist.
\end{document}


\section{Related Work}
\section{Related Work}
% \subsection{Vision Language Model}
% 시각장애인에서 상황을 설명할 DB가 없으니 만들었다. 그리고 이를 VLM에 튜닝했다.
\subsection{Technical approaches for assisting the visually-impaired}


\subsection{Datasets for visual instruction tuning}


\section{Research Questions}
The analysis in Section 2 provides the motivation for a focused investigation into the use of generalisable stylistic features as well as the novel social-monetisation features that may lend themselves to improved generalisability (the ability of a model to perform well when tested on a different dataset than the one on which it was trained) of fake news detection models. As such, the objectives of this study can be formalised into the following three research questions:


\begin{itemize}
    \item \textbf{RQ1.} How well do fake news detection methods using token-\\representations/LLMs generalise? 
    \item \textbf{RQ2.} Do fake news detection methods using stylistic features generalise better than fake news detection models using token-representations/LLMs?
    \item \textbf{RQ3.} Do fake news detection methods using stylistic features and the proposed social-monetisation features generalise better than models using stylistic features only?  
\end{itemize}



\section{Methodology}

\section{\label{sec:method}Methodology}

Each SIEM system uses its own RDL to define threat detection rules, and each RDL has its own schema.
For example, the Splunk SIEM uses the SPL to define its threat detection rules.
The task of understanding threat detection rules and recommending relevant MITRE ATT\&CK techniques (or sub-techniques) requires complex reasoning skills.
In the case of LLMs, this can be achieved with a technique called prompt chaining in which each task is divided into multiple sub-tasks in order to understand the complex reasoning behind the task.
Therefore, we employ a multi-phase architecture based on prompt chaining that leverages the power of LLMs to take a SIEM rule defined in any RDL and map it to relevant MITRE ATT\&CK techniques using the power of LLMs.
Our approach is based on the following intuitions:
\begin{itemize}[nosep,leftmargin=*]
    \item \textit{LLMs' implicit knowledge}: LLMs possess deep understanding of diverse RDLs. This enables them to interpret any rule, regardless of the RDL it is defined in, and convert it into comprehensible natural language text.
    \item \textit{LLMs' similarity comparison capability}: LLMs are adept at analyzing and comparing textual descriptions. 
    They can intelligently assess the similarity between two textual inputs to establish a meaningful connection.
\end{itemize}

\methodName has two main phases: (1) the rule to text translation phase, and (2) the MITRE ATT\&CK techniques recommendation phase.
These two phases in the pipeline include six key steps to determine relevant TTPs, as illustrated in Figure~\ref{fig:r2t}.

Although LLMs excel at translating SIEM rules into natural language, they often lack critical domain-specific contextual information related to IoCs in the rules.
To overcome this limitation, the \textit{rule to text translation} phase includes three steps: IoC extraction, contextual information retrieval, and natural language translation.

The workflow begins with the extraction of IoCs from the rules (for example, processes, log source, event codes, and file names) that the rule searches for in the logs (step (1)).In the next sstep a web search agent performs the task of obtaining additional contextual information about the IoCs discovered ((step 2)).
By incorporating this additional domain-specific information, the pipeline enhances the language translation, resulting in a more accurate and meaningful interpretation of SIEM rules.
The rule itself and the IoCs' contextual additional information from the previous stage are then used to translate the rule from RDL to natural language (step (3)).

The \textit{MITRE ATT\&CK techniques} recommendation phase of the pipeline includes the following three steps.
The rule in processed in data source identification step in which the probable origin of the data is identified (step (4)).
The description of the rule is then used to determine probable MITRE ATT\&CK techniques based on the implicit knowledge of the LLM (step (5)).
Finally, using chain-of-thought~\cite{wei2022chain} prompting, the most relevant techniques are extracted from the list of probable techniques (step (6)).
Each step of our method is further described in detail below.


% [bb=0 0 1440 900,width=1.43\linewidth,height=0.9\textwidth]
\begin{figure*}[htbp]
   \includegraphics[width=\textwidth]{Images/stages.jpg}
    
   \caption{An illustration of the different steps in \methodName.}
   \label{fig:stages}
\end{figure*} 

\subsection{IoC Extraction}
The context associated with a SIEM detection rule is crucial for its accurate interpretation and effective application. 
Obtaining this contextual understanding requires comprehensive analysis of the embedded IoCs in the SIEM rule.
In the first step, \methodName systematically identifies and extracts all IoCs, identifying the types of IoCs and their corresponding values that form the foundational elements of the detection rules. 
Leveraging the LLM's inherent understanding of rule structures and IoCs, we employ a zero-shot prompting approach for this task. 
Zero-shot prompting enables the direct extraction of IoCs from the rules without requiring extensive pre-training on specific datasets.

\noindent The result of this stage is a dictionary structure, where:
\begin{itemize}[nosep,leftmargin=*]
    \item Keys represent types of IoC, such as processes, files, IP addresses, and log sources.
    \item Values are lists containing specific IoC details, such as process names, file names, IP addresses, and log source identifiers.
\end{itemize}

In the example depicted in Figure~\ref{fig:stages}(a), the pipeline processes a rule for which relevant MITRE ATT\&CK techniques need to be recommended. 
The IoC extractor LLM produces a dictionary structure as output, organizing the IoCs in a structured format to support subsequent stages in the analysis pipeline. 



\subsection{Contextual Information Retrieval}
In this step, an LLM agent is employed to retrieve relevant information pertaining to the IoCs extracted from the rule.
A REACT agent~\cite{react} was used in this case to generate both reasoning traces and task-specific actions in an interleaved manner.
REACT agents interact with external tools to retrieve additional information that leads to more factual and reliable responses.
The LLM agent conducts a systematic search across web resources to gather additional contextual information for each IoC value present in the rule. 
This step addresses LLMS' lack of up-to-date knowledge or specialized domain expertise (which is critical to understanding the role and significance of the IoCs in the rule), without the need for retraining or fine-tuning.
Figure~\ref{fig:stages}(b) presents an example in which the rule includes the process name \texttt{soaphound.exe} as an IoC.
As can be seen, the web search results indicate that \texttt{soaphound.exe} is being used for active directory (AD) enumeration, which is important for the understanding of the attack. 

\subsection{Natural Language Translation}

The translation of detection rules into natural language textual descriptions fulfills three key objectives:
\begin{enumerate}[nosep,leftmargin=*]
    \item \textbf{Ensures that \methodName is format-agnostic}: It converts rules defined in various RDL formats into a generic, unstructured text format, ensuring compatibility with different SIEM systems, regardless of the specific rule format.
    \item \textbf{Provides contextual explanation}: It includes all relevant contextual information to produce a concise and comprehensible explanation of the rule.
    \item \textbf{Enhances the comprehension for LLMs}: It enables LLMs to more effectively compare the translated rule with descriptions in the MITRE ATT\&CK framework by providing a unified textual representation.
\end{enumerate}
To achieve these objectives, a zero-shot prompting technique is employed. 
The input to the LLM comprises two components:
\begin{itemize}
    \item \textbf{Syntactical information}: The rule itself, providing the structural and operational details.
    \item \textbf{Contextual information}: Details of the IoCs extracted from the rule, providing semantic insights into the rule's intent and function.
\end{itemize}
The LLM utilizes these inputs to generate a natural language textual description of the rule. 
This transformation not only ensures a more interpretable representation but also facilitates further steps of analysis and comparison, particularly in aligning the rule with MITRE ATT\&CK techniques and sub-techniques.



\subsection{Data Source or Mitigation Identification}
Identifying the most relevant data component or mitigation associated with the rule description in this step is critical for filtering out irrelevant MITRE ATT\&CK techniques (or sub-techniques) in subsequent steps of the pipeline.
In the MITRE ATT\&CK framework, data sources represent various categories of information that can be gathered from sensors or logs. 
These data sources include \textit{data components}, which are specific attributes or properties within a data source that are directly relevant to detecting a particular technique or sub-technique~. 
For example, in the context of the rule described in Figure~\ref{fig:stages}(a), the term \texttt{Endpoint.Processes} indicates that the activity is happening on an endpoint. 
Presence of the terms such as, \texttt{soaphound.exe}, \texttt{--buildcache}, \texttt{--certdump} and etc. indicate that the rule searches for command line execution of an executable named \texttt{soaphound.exe} with specific parameters. 
Therefore, the appropriate data source in this example is \textit{Command}, with the corresponding data component being \textit{Command Execution}.
Additionally, \textit{mitigations} are defined as categories of technologies or strategies that can prevent or reduce the impact of specific techniques or sub-techniques. 
The MITRE ATT\&CK framework explicitly establishes relationships between data components, mitigations, and techniques (or sub-techniques), enabling a systematic approach for identifying relevant elements.

To identify the most relevant data component or mitigation associated with a given rule description, we utilize agentic retrieval augmented generation (RAG), which incorporates an AI Agent-based implementation of the RAG framework.
Data from the MITRE ATT\&CK framework, specifically related to data components and mitigations, is stored in a vector database (e.g., ChromaDB). 
The process begins with the rule description from the previous stage, which serves as the input to the AI Agent. 
The LLM-powered agent automatically generates a search query tailored to retrieve relevant information from the RAG database.

For each query, the system retrieves the five most similar documents from the database, each containing contextual information about data components or mitigations. 
These documents are then utilized by the LLM agent to contextualize the rule description. 
By comparing the content of these retrieved documents with the rule description, the LLM agent determines and outputs the most relevant data component or mitigation along with a chain-of-thought as to why the data component or mitigation is related to the rule.


\subsection{Probable Technique Recommendation}

In this step, an LM agent is utilized to propose probable MITRE ATT\&CK techniques (and sub-techniques) that may be relevant to the description of the provided rule.
We used a REACT agent in this step as well to utilize both implicit and explicit knowledge during reasoning.
For explicit knowledge, the agent searches the MITRE ATT\&CK framework to obtain the list of probable techniques (and sub-techniques).
The natural language description of the rule from the previous step serves as input to the LLM agent.
The output of this stage consists of a list of JSON objects, each containing the MITRE technique ID, technique name, and technique description as seen in Figure~\ref{fig:stages}(c).

Throughout our experiments, we observed that as the number of recommendations ($k$) increases, both the framework's average recall and precision initially improve, however beyond a certain threshold of $k$, the %average 
precision begins to decline.
Based on these observations(please refer Table~\ref{tab:results3}), we selected a $k$-value of 11 to ensure a high recall.



\subsection{Relevant Technique Extraction}
In this step, \methodName refines the set of probable MITRE ATT\&CK techniques identified in the previous stage by eliminating irrelevant entries.
This step in the pipeline serves two primary purposes: (1) to enhance precision while maintaining recall achieved in previous step, and (2) to provide a clear rationale for the selection of the labels, ensuring transparency and interpretability of the mapping process.
This refinement process is grounded in the assumption that LLMs are effective for text similarity matching tasks.

The process comprises two key steps:
\begin{itemize}
    \item \textit{Rule-technique comparison}: The description of each technique in the set of probable techniques is compared with the rule description. 
    A chain-of-thought technique is then applied to elucidate the reasoning behind the association of each technique with the rule.
    \item \textit{Confidence calculation}: The generated chain-of-thought rationale for each technique (or sub-technique) is compared with the rule description to compute a relevance (or confidence) score, as done in prior work~\cite{freitas2024ai}.
    % \item \textbf{Reasoning}: \new{Add here the reasoning that it provides - explaining in NLP why it was selected...}
\end{itemize}

Techniques with higher confidence scores are deemed more relevant to the rule. 
Conversely, techniques with scores falling below a predefined threshold are excluded.
The techniques retained after this filtering step represent the most relevant techniques corresponding to the given rule's description. 


The chain-of-thought (CoT) rationale generated during the comparison of each rule to its probable technique is also provided as an output in this step.
This rationale offers a detailed natural language explanation, articulating why a particular technique is relevant to the given rule. 
Such explanations are highly valuable for security analysts, as they provide clear and transparent reasoning behind the mapping, enabling analysts to better understand and validate the association between the rule and the technique.
Other classification models proposed in previous works within this domain also suffer from the limitation of being black-box models, which lack the ability to provide clear reasoning or explanations. 
Unlike \methodName, these models fail to generate transparent, CoT rationales that explain why a particular rule is mapped to a specific technique, making them less interpretable and less useful for security analysts.

\section{Results}
\begin{table}[ht!]
\centering
\caption{\textbf{Super Resolution Performance Results.} Our proposed WGAN EEG Spatial Upsampling method significantly outperforms a baseline of Bicubic Interpolation commonly used in EEG upsampling pipelines.}
\label{tab:results}
\resizebox{0.8\linewidth}{!}{%
\begin{tabular}{@{}cccccc@{}}
\toprule
\multirow{2}{*}{\textbf{Dataset}} & \multirow{2}{*}{\textbf{Scale}} & \multicolumn{2}{c}{\textbf{Bicubic}} & \multicolumn{2}{c}{\textbf{WGAN}} \\ \cmidrule(l){3-6} 
                      &   & \textbf{MSE} & \textbf{MAE} & \textbf{MSE}    & \textbf{MAE}   \\
\toprule
\multirow{2}{*}{Val}  & 2 & 3.71E7       & 3.89E3       & \textbf{2.01E3} & \textbf{24.38} \\
                      & 4 & 7.23E7       & 6.42E3       & \textbf{8.53E3} & \textbf{63.83} \\
\midrule
\multirow{2}{*}{Test} & 2 & 3.75E7       & 3.91E3       & \textbf{2.06E3} & \textbf{24.66} \\
                      & 4 & 7.30E7       & 6.45E3       & \textbf{8.68E3} & \textbf{64.39} \\
\bottomrule
\end{tabular}%
}
\end{table}

\section{Discussion}
This work identifies signal collapse as a critical bottleneck in one-shot neural network pruning. Performance loss in pruned networks is due to \textbf{signal collapse} in addition to the removal of critical parameters. We propose \textbf{REFLOW} (\textbf{Re}storing \textbf{F}low of \textbf{Low}-variance signals), a simple yet effective method that mitigates signal collapse without computationally expensive weight updates. By focusing on signal preservation, REFLOW highlights the importance of mitigating signal collapse in sparse networks and enables magnitude pruning to match or surpass state-of-the-art one-shot pruning methods such as CHITA, CBS, and WF.

REFLOW consistently achieves state-of-the-art accuracy across diverse architectures, restoring ResNeXt-101 from under 4.1\% to 78.9\% top-1 accuracy at 80\% sparsity on ImageNet. Its lightweight design makes it a practical solution for both research and deployment, delivering high-quality sparse models without the overhead of traditional approaches. These findings challenge the traditional emphasis on weight selection strategies and underscore the critical role of signal propagation for achieving high-quality sparse networks in the context of one-shot pruning.




\section{Future Work and Conclusions}
While this study makes a significant contribution to the field of fake news detection by investigating the seldom addressed issue of generalisability using ‘real-world’ data, a number of limitations and therefore opportunities for future work have been identified.

Firstly, while this paper has explored the use of LLMs through the LLaMA model, providing initial insights into their application in this domain, future work may further investigate the potential of other fine-tuned LLMs to enhance generalisability and performance. Although this study, \cite{pavlyshenko2023analysis} and \cite{kumar2024silver} indicate that even advanced LLMs face challenges in achieving robust generalisability, continued research could examine integrating LLMs with other feature sets to address these limitations and improve performance in fake news detection tasks.

Secondly, although the inclusion of stylistic and social-monetisation features enhanced model performance and balance, the study was constrained to specific datasets. Future research should investigate the effectiveness of these features across a wider range of datasets, including those focused on different types of news topics, to better understand their generalisability and robustness across different domains. However, training and testing on coarsely labelled datasets can lead to misleading results that show high levels of performance in cross-validation or hold-out testing on unseen portions of training datasets, but not on real-world data. Given the limited availability of manually labelled real-world data like the Facebook URLs dataset used in this study, more effort is needed to produce granularly labelled datasets that can serve as robust benchmarks for evaluating fake news detection models. However, it is crucial to ensure user privacy and safety when creating these datasets, especially when derived from social media platforms such as Facebook or X/Twitter.

Moreover, while this study aimed to tackle the issue of poor generalisability from a feature engineering perspective, future work should focus on optimising model hyperparameters to further enhance the performance and robustness of fake news detection models, alongside the features proposed in this study. Fine-tuning hyperparameters such as learning rate, batch size and regularisation techniques could potentially improve model accuracy, recall, and specificity across different datasets and domains. Additionally, while promising results were produced in this study with the Gradient Boosting and Random Forest algorithms, it is also important to acknowledge the potential biases introduced by the models themselves. Algorithms such as Gradient Boosting prioritise features (such as exclamations and ads seen in the PFI analysis) that reduce the loss function. This prioritisation can potentially introduce biases if these features are not equally relevant across different datasets. Future work should therefore investigate other potential sources of bias, beyond dataset bias, to ensure the model's fairness and robustness. This includes a further examination of how features, especially the newly introduced 'social-monetisation' features, might inadvertently influence model predictions and contribute to biases. By addressing these biases, we can develop more reliable and equitable fake news detection models.

Finally, despite the comprehensive exploration of various stylistic features in this study, it is important to acknowledge that there are many other stylistic features that are yet to be explored in this context. Future work, therefore, should seek to identify other generalisable features, similar to the proposed social monetisation features, exclamations and all-caps words, as identified by this study. Additionally, given the advantages demonstrated through the four proposed novel features, future work should also try to identify such features that are available in the broader context of the whole webpage and not exclusively the article text. Further investigation into the computational efficiency of these features, compared to other approaches, should also be a priority in future research. This will ensure that the developed models can be efficiently deployed in real-time systems where computational resources and rapid response times are critical.

\bibliographystyle{apalike} 
\bibliography{cas-refs}

\newpage
\appendix
\section{Feature Tables}
\section{Proofs from Section~\ref{sec:gammaok}} \label{app:gamma}

\subsection{On the girth of locally \texorpdfstring{$\gamma$}{gamma}-sparse graphs}
\begin{lemma}\label{lemma:girth_rev}
    Let $G = (V,E)$ be an undirected graph with girth $g(G)$.
    Then $G$ is \ok{0} if and only if $g(G) \geq 5$.
\end{lemma}
\begin{proof}
    We first prove that if $G$ is \ok{0} then $g(G)$ must be at least $5$.
    In order to prove that, we simply negate the statement and prove that if $G$ has girth $<5$ then $G$ can not be \ok{0}.
    Without loss of generality, assume that $g(G) = 4$ (the case $g(G) = 3$ is similar).
    Then there must exist a cycle $C = (u_1, u_2, u_3, u_4)$ of $4$ vertices.
    It is simple to see that $u_2,u_4 \in \lset_1(u_1)$ and $u_3 \in \lset_2(u_1)$.
    Since $u_3$ is a neighbor of both $u_2$ and $u_4$, the degree of $u_3$ in the subgraph $G\left[\lset_1(u_1) \cup \{u_4\} \right]$ is at least $2$, hence $G$ is not \ok{0} (see \Cref{subfig:girth1}).
    
    We now prove that if $g(G) \geq 5$ then $G$ must be \ok{0}.
    Again, we negate this statement and prove that if $G$ is not \ok{0} then the girth of $G$ must be less then $5$.
    Let us assume that $G$ is \gammaok, for any $\gamma > 0$, thus it is not \ok{0}.
    Since $G$ is not \ok{0} there exists a vertex $v \in V$ such that at least one of the following properties holds (see \Cref{subfig:girth2}):
    \begin{enumerate}
        \item $\exists u \in \lset_1(v)$ such that the degree of $u$ in $G\left[ \lset_1(v) \right]$ is greater then $0$, or;
        \item $\exists w \in \lset_2(v)$ such that the degree of $w$ in $G\left[ \lset_1(v) \cup \{ w \} \right]$ is greater then $1$.
    \end{enumerate}
    In the first case, we have a cycle of $3$ vertices, then $g(G) = 3$.
    In the second case, we have a cycle of $4$ vertices, then $g(G) = 4$.
    In both cases $g(G) < 5$.
\end{proof}
\begin{figure}[h]
    \centering
    \begin{subfigure}[b]{0.35\linewidth}
            \centering
            \includegraphics[width=\linewidth]{img/girth-1.pdf}
            \caption{}
            \label{subfig:girth1}
    \end{subfigure}
    \begin{subfigure}[b]{0.6\linewidth}
            \centering
            \includegraphics[width=\linewidth]{img/girth-2.pdf}
            \caption{}
            \label{subfig:girth2}
    \end{subfigure}%
    \caption{}
    \label{fig:example_girth}
\end{figure}

\subsection{Deterministic lazy-update on \texorpdfstring{$\gamma$}{gamma}-sparse graphs}\label{apx:gamma-ok-deterministic}

\begin{theorem}\label{lemma:gamma-ok-error-bound-balls}
    
Let $\varepsilon \in (0,1)$, and let $G^{(0)}$ be an initial graph. Consider any sequence of edge insertions that yields a final graph $G$. If $G$ is \gammaok, \lazyscheme$(\varphi = \frac{\varepsilon}{1 - \varepsilon},k=0)$ has an approximation ratio of  $\frac{\gamma + 1}{1-\varepsilon}$ and amortized update cost $O(1/\varepsilon)$. 
    
\end{theorem}
\begin{proof}
Recall that $\bd_u$ denotes the black degree of $u$, and that  \Cref{alg:det_thresh} guarantees that $\deg_u$ is at most $(1+\varphi)\bd_u$.
    Then, it is simple to give an upper bound to the size of $\ball_2(u)$, that is $\vert \ball_2(u) \vert \leq 1+ \sum_{v \in \lset_1(u)} (1 + \varphi)\bd_v$.Consider a vertex $v \in \lset_1(u)$. Since $G$ is \gammaok, the number of neighbors of $v$ belonging to $\lset_2(u)$ is at lest $\deg_v - (\gamma+1)$ of which $\bd_v - (\gamma+1)$ must belong to $\apxball_2(u)$. Moreover, a vertex in $\lset_2(u)$ has at most $\gamma+1$ neighbors in $\lset_1(u)$. Therefore: 
    \begin{align*}
    \vert \apxball_2(u) \vert
    &\geq  \bd_u + 1 + \frac{1}{\gamma + 1}\sum_{v \in \lset_1(u)}(\bd_v - (\gamma + 1))\\
    &= \bd_u + 1 + \frac{1}{\gamma + 1}\sum_{v \in \lset_1(u)}\bd_v - \underbrace{\frac{1}{\gamma + 1}\sum_{v \in \lset_1(u)}(\gamma + 1)}_{= \bd_u}\\
    &= 1+ \frac{1}{\gamma + 1}\sum_{v \in \lset_1(u)}\bd_v.
    \end{align*}
  
    As a consequence, $\vert \apxball_2(u) \vert/\vert \ball_2(u) \vert \ge \frac{1}{(1+\varphi)(\gamma+1)}$. By setting $\varphi = \frac{\varepsilon}{1 - \varepsilon}$, and by using \Cref{lm:amortized_det_alg},  the claim follows.
\end{proof}

\subsection{Proof of \Cref{le:gamma_ok_expect_lowerbound}}\label{apx:proof_gamma_ok_expect_lowerbound}
\begin{proof}
Let $e_1, \dots, e_{\ell_v}$ be the \emph{red edges} between $v$ and $\lset_2(u)$, and define the binary random variable $\lrdr_v(i)$ that is equal to $1$ if $e_i$ is a \emph{quasi-black edge} for $u$, $0$ otherwise, for $i = 1, \dots, \lrd_v$. Thus we can express $\lrdr_v = \sum_{i=1}^{\lrd_v} \lrdr_v(i)$, with expectation

\begin{equation}\label{eq:gamma_ok_lb_fact_eq_1}
\begin{aligned}
  \Expec{}{\lrdr_v} & = \sum_{i=1}^{\lrd_v}{\Prob{}{\lrdr_v(i)=1}} = \lrd_v - \sum_{i=1}^{\lrd_v} {\Prob{}{\lrdr_v(i)=0}}.
\end{aligned}
\end{equation}

Without loss of generality, assume that the edges $e_1, \dots, e_{\lrd_v}$ have been inserted at times $t_1 < \dots < t_{\lrd_v}$, respectively.
If $e_i$ is not a quasi-black edge for $u$, then it must be that $u$ is not selected by $v$ at \Cref{line:random_selection} of \Cref{alg:det_thresh}, at times $t_i, t_{i+1},\dots, t_{\lrd_v}$.
This holds with probability 
\begin{equation}\label{eq:gamma_ok_lb_fact_eq_2}
\begin{aligned} 
    &\Prob{}{\lrdr_v(i) = 0}
    \leq \prod_{j=i}^{\lrd_v} \left( 1-\frac{k}{\deg_v^{(t_j)}} \right)
    \leq \prod_{j=i}^{\lrd_v} \left( 1 - \frac{k}{\deg_{v}^{(t_{\lrd_v})}} \right) \\
    &\leq \left( 1-\frac{k}{\lbdd_v + \lrd_v + \gamma + 1}\right)^{\lrd_v - i + 1} 
    \leq \left(1-\frac{k}{2(\lbdd_v + \gamma + 1)}\right)^{\lrd_v - i}.
\end{aligned}
\end{equation}
The third inequality holds since the edges incident to $v$ having endpoints in $L_1(u)$ are at most $\gamma$, while those having endpoints in $L_2(u)$ are exactly $\lbdd_v+ \lrd_v$. Moreover, the last inequality holds because $\lrd_v \leq \rd_v \leq \bd_v \leq \lbdd_v + \gamma + 1$, given the assumption $\varphi = 1$.

By plugging in \eqref{eq:gamma_ok_lb_fact_eq_2} into   \eqref{eq:gamma_ok_lb_fact_eq_1} and we obtain
\begin{align*}
    &\Expec{}{\lrdr_v} \geq \lrd_v - \sum_{i=1}^{\lrd_v}\left( 1-\frac{k}{2(\lbdd_v + \gamma + 1)}\right)^{\lrd_v - i} \\
    &= \lrd_v - \sum_{i=0}^{\lrd_v-1} \left(1-\frac{k}{2(\lbdd_v + \gamma + 1)}\right)^i 
    \leq \lrd_v - \frac{1-\left(1-\frac{k}{2(\lbdd_v+\gamma+1)}\right)^{\lrd_v}}{1-\left(1-\frac{k}{2(\lbdd_v + \gamma + 1)}\right)} \\
    &\geq \lrd_v - \frac{1}{1-\left(1-\frac{k}{2(\lbdd_v + \gamma + 1)}\right)}
    \geq \lrd_v - \frac{2(\lbdd_v + \gamma + 1)}{k}.
\end{align*}
\end{proof}
\clearpage
\section{Feature Importance}
\subsection{Model Rank and Pairwise Comparisons}\label{apd:cd_diagrams}
\begin{figure*}[htbp]
    \centering
   \begin{minipage}{0.48\textwidth}
        \centering
        \includegraphics[width=\textwidth]{images/cd_baselines_aggregate.pdf}
        \subcaption{Model choice for in-distribution series (p-value: 2.71e-8)}
        \label{fig:cd_baselines_aggregate}
    \end{minipage}%
    \hfill
    \begin{minipage}{0.48\textwidth}
        \centering
        \includegraphics[width=\textwidth]{images/cd_baselines_component.pdf}
        \subcaption{Model choice for out-of-distribution series (p-value: 3.52e-8)}
        \label{fig:cd_baselines_component}
    \end{minipage}

\caption{Critical Difference (CD) diagrams illustrate model ranks and pairwise statistical comparisons of model performance on compositional reasoning tasks across all datasets. Lower ranks indicate better performance. A thick horizontal line groups models that are not significantly different. The statistical tests used to generate the CD diagrams are detailed in Section \ref{section:evaluation}. \textbf{(a, b)} The patch-based Transformer models and MLP-based models outperform other models in both traditional and compositional reasoning forecasting paradigms. The Friedman p-value is included in the subcaptions.}
    \label{fig:cd_diagrams_baselines_apd}
\end{figure*}



\begin{figure*}[htbp]
    \centering

    \vspace{1ex} % Vertical space between rows

    \begin{minipage}{0.48\textwidth}
        \centering
        \includegraphics[width=\textwidth]{images/cd_tokenization_ablation_aggregate.pdf}
        \subcaption{Tokenization for in-distribution series (p-value=2.91e-3)}
        \label{fig:cd_tokenization_ablation_aggregate}
    \end{minipage}%
    \hfill
    \begin{minipage}{0.48\textwidth}
        \centering
        \includegraphics[width=\textwidth]{images/cd_tokenization_ablation_component.pdf}
        \subcaption{Tokenization for out-of-distribution series (p-value=1.17e-2)}
        \label{fig:cd_tokenization_ablation_component}
    \end{minipage}

    \vspace{1ex} % Vertical space between rows

    \begin{minipage}{0.48\textwidth}
        \centering
        \includegraphics[width=\textwidth]{images/cd_size_ablation_aggregate.pdf}
        \subcaption{Model size for in-distribution series (p-value: 6.58e-2)}
        \label{fig:cd_size_ablation_aggregate}
    \end{minipage}%
    \hfill
    \begin{minipage}{0.48\textwidth}
        \centering
        \includegraphics[width=\textwidth]{images/cd_size_ablation_component.pdf}
        \subcaption{Model Size for out-of-distribution series (p-value=8.58e-2)}
        \label{fig:cd_size_ablation_component}
    \end{minipage}

    \vspace{1ex} % Vertical space between rows

    \begin{minipage}{0.48\textwidth}
        \centering
        \includegraphics[width=\textwidth]{images/cd_attn_ablation_aggregate.pdf}
        \subcaption{Attn. type for in-distribution series}
        \label{fig:cd_attn_ablation_aggregate}
    \end{minipage}%
    \hfill
    \begin{minipage}{0.48\textwidth}
        \centering
        \includegraphics[width=\textwidth]{images/cd_attn_ablation_component.pdf}
        \subcaption{Attn. type for out-of-distribution series}
        \label{fig:cd_attn_ablation_component}
    \end{minipage}

    \vspace{1ex} % Vertical space between rows

    \begin{minipage}{0.48\textwidth}
        \centering
        \includegraphics[width=\textwidth]{images/cd_proj_ablation_aggregate.pdf}
        \subcaption{Projection layer for in-distribution series}
        \label{fig:cd_proj_ablation_aggregate}
    \end{minipage}%
    \hfill
    \begin{minipage}{0.48\textwidth}
        \centering
        \includegraphics[width=\textwidth]{images/cd_proj_ablation_component.pdf}
        \subcaption{Projection layer for out-of-distribution series}
        \label{fig:cd_proj_ablation_component}
    \end{minipage}

    \vspace{1ex} % Vertical space between rows

    \begin{minipage}{0.48\textwidth}
        \centering
        \includegraphics[width=\textwidth]{images/cd_tokenlen_ablation_aggregate.pdf}
        \subcaption{Token length for in-distribution series (p-value=0.22)}
        \label{fig:cd_tokenlen_ablation_aggregate}
    \end{minipage}%
    \hfill
    \begin{minipage}{0.48\textwidth}
        \centering
        \includegraphics[width=\textwidth]{images/cd_tokenlen_ablation_component.pdf}
        \subcaption{Token length for out-of-distribution series (p-value=8.62e-4)}
        \label{fig:cd_tokenlen_ablation_component}
    \end{minipage}

    \vspace{1ex} % Vertical space between rows

    \begin{minipage}{0.48\textwidth}
        \centering
        \includegraphics[width=\textwidth]{images/cd_pe_ablation_aggregate.pdf}
        \subcaption{Positional encoding for in-distribution series (p-value=0.71)}
        \label{fig:cd_pe_ablation_aggregate}
    \end{minipage}%
    \hfill
    \begin{minipage}{0.48\textwidth}
        \centering
        \includegraphics[width=\textwidth]{images/cd_pe_ablation_component.pdf}
        \subcaption{Positional encoding for out-of-distribution series (p-value=0.46)}
        \label{fig:cd_pe_ablation_component}
    \end{minipage}
    
    \vspace{1ex} % Vertical space between rows

    \begin{minipage}{0.48\textwidth}
        \centering
        \includegraphics[width=\textwidth]{images/cd_loss_ablation_aggregate.pdf}
        \subcaption{Loss function for in-distribution series (p-value=0.24)}
        \label{fig:cd_loss_ablation_aggregate}
    \end{minipage}%
    \hfill
    \begin{minipage}{0.48\textwidth}
        \centering
        \includegraphics[width=\textwidth]{images/cd_loss_ablation_component.pdf}
        \subcaption{Loss function for out-of-distribution series (p-value=0.14)}
        \label{fig:cd_loss_ablation_component}
    \end{minipage}

    \vspace{1ex} % Vertical space between rows

    \begin{minipage}{0.48\textwidth}
        \centering
        \includegraphics[width=\textwidth]{images/cd_scaler_ablation_aggregate.pdf}
        \subcaption{Scaler for in-distribution series}
        \label{fig:cd_scaler_ablation_aggregate}
    \end{minipage}%
    \hfill
    \begin{minipage}{0.48\textwidth}
        \centering
        \includegraphics[width=\textwidth]{images/cd_scaler_ablation_component.pdf}
        \subcaption{Scaler function for out-of-distribution series}
        \label{fig:cd_scaler_ablation_component}
    \end{minipage}

    \vspace{1ex} % Vertical space between rows

    \begin{minipage}{0.48\textwidth}
        \centering
        \includegraphics[width=\textwidth]{images/cd_contextlen_ablation_aggregate.pdf}
        \subcaption{Context length function for in-distribution series}
        \label{fig:cd_contextlen_ablation_aggregate}
    \end{minipage}%
    \hfill
    \begin{minipage}{0.48\textwidth}
        \centering
        \includegraphics[width=\textwidth]{images/cd_contextlen_ablation_component.pdf}
        \subcaption{Context length for out-of-distribution series}
        \label{fig:cd_contextlen_ablation_component}
    \end{minipage}

    \vspace{1ex} % Vertical space between rows

    \begin{minipage}{0.48\textwidth}
        \centering
        \includegraphics[width=\textwidth]{images/cd_decomp_ablation_aggregate.pdf}
        \subcaption{Input decomposition for in-distribution series}
        \label{fig:cd_decomp_ablation_aggregate}
    \end{minipage}%
    \hfill
    \begin{minipage}{0.48\textwidth}
        \centering
        \includegraphics[width=\textwidth]{images/cd_decomp_ablation_component.pdf}
        \subcaption{Input decomposition function for out-of-distribution series}
        \label{fig:cd_decomp_ablation_component}
    \end{minipage}

    \caption{Critical Difference (CD) diagrams illustrate model ranks and pairwise statistical comparisons of model performance on compositional reasoning tasks across all datasets. Lower ranks indicate better performance. A thick horizontal line groups models that are not significantly different. The statistical tests used to generate the CD diagrams are detailed in Section \ref{section:evaluation}. For analyses comparing three or more methods, the Friedman p-value is included in the subcaptions.}
    \label{fig:cd_diagrams_apd}
\end{figure*}

\newpage
\subsection{Model Forecasts}\label{apd:forecast_examples}
We include example forecasts for each of the 6 datasets used in this study.


\begin{figure*}[ht!]
    \centering

    \begin{minipage}{0.411\textwidth}
        \centering
        \includegraphics[width=\textwidth, trim=0 0 310 0, clip]{images/sine_id_forecast_example.pdf}
        \subcaption{Model forecasts for in-distribution Sinusoid series}
        \label{fig:sine_id_forecast_example}
    \end{minipage}
    \hfill
    \begin{minipage}{0.584\textwidth}
        \centering
        \includegraphics[width=\textwidth]{images/sine_ood_forecast_example.pdf}
        \subcaption{Model forecasts for out-of-distribution Sinusoid series}
        \label{fig:sine_ood_forecast_example.pdf}
    \end{minipage}

    \vspace{2.9em}
    
    \begin{minipage}{0.411\textwidth}
        \centering
        \includegraphics[width=\textwidth, trim=0 0 310 0, clip]{images/ecl_id_forecast_example.pdf}
        \subcaption{Model forecasts for in-distribution ECL series
        }
        \label{fig:ecl_id_forecast_example}
    \end{minipage}
    \hfill
    \begin{minipage}{0.584\textwidth}
        \centering
        \includegraphics[width=\textwidth]{images/ecl_ood_forecast_example.pdf}
        \subcaption{Model forecasts for out-of-distribution ECL series}
        \label{fig:ecl_ood_forecast_example.pdf}
    \end{minipage}

    \vspace{2.9em}
    
    \begin{minipage}{0.411\textwidth}
        \centering
        \includegraphics[width=\textwidth, trim=0 0 310 0, clip]{images/ettm2_id_forecast_example.pdf}
        \subcaption{Model forecasts for in-distribution ETTm2 series
        }
        \label{fig:ettm2_id_forecast_example}
    \end{minipage}
    \hfill
    \begin{minipage}{0.584\textwidth}
        \centering
        \includegraphics[width=\textwidth]{images/ettm2_ood_forecast_example.pdf}
        \subcaption{Model forecasts for out-of-distribution ETTm2 series}
        \label{fig:ettm2_ood_forecast_example.pdf}
    \end{minipage}

    \caption{\textbf{(a, c, e)} Forecasts for a ground truth series $\mathbf{y}(t)$ for the Sinusoid, ETTm2, and ECL datasets for models trained using the traditional forecasting paradigm. \textbf{(b, d, f)} Forecasts for the Sinusoid, ETTm2, and ECL datasets for models trained using the compositional reasoning forecasting paradigm. Patch-based Transformer models and MLP-based models (top), which rank among the top-performing models, demonstrate generalization to out-of-distribution time series, whereas other Transformer variants and linear models struggle to do so (bottom).}
    \label{fig:forecast_examples1_apd}
\end{figure*}

\newpage
\begin{figure*}[ht!]
    \centering
    \begin{minipage}{0.411\textwidth}
        \centering
        \includegraphics[width=\textwidth, trim=0 0 310 0, clip]{images/solar_id_forecast_example.pdf}
        \subcaption{Model forecasts for in-distribution Solar series
        }
        \label{fig:solar_id_forecast_example}
    \end{minipage}
    \hfill
    \begin{minipage}{0.584\textwidth}
        \centering
        \includegraphics[width=\textwidth]{images/solar_ood_forecast_example.pdf}
        \subcaption{Model forecasts for out-of-distribution Solar series}
        \label{fig:solar_ood_forecast_example.pdf}
    \end{minipage}

    \vspace{2.9em}
    
    \begin{minipage}{0.411\textwidth}
        \centering
        \includegraphics[width=\textwidth, trim=0 0 310 0, clip]{images/subseasonal_id_forecast_example.pdf}
        \subcaption{Model forecasts for in-distribution Subseasonal series
        }
        \label{fig:subseasonal_id_forecast_example}
    \end{minipage}
    \hfill
    \begin{minipage}{0.584\textwidth}
        \centering
        \includegraphics[width=\textwidth]{images/subseasonal_ood_forecast_example.pdf}
        \subcaption{Model forecasts for out-of-distribution Subseasonal series}
        \label{fig:subseasonal_ood_forecast_example.pdf}
    \end{minipage}

    \vspace{2.9em}

    \begin{minipage}{0.411\textwidth}
        \centering
        \includegraphics[width=\textwidth, trim=0 0 310 0, clip]{images/loopseattle_id_forecast_example.pdf}
        \subcaption{Model forecasts for in-distribution Loop Seattle series
        }
        \label{fig:loopseattle_id_forecast_example}
    \end{minipage}
    \hfill
    \begin{minipage}{0.584\textwidth}
        \centering
        \includegraphics[width=\textwidth]{images/loopseattle_ood_forecast_example.pdf}
        \subcaption{Model forecasts for out-of-distribution Loop Seattle series}
        \label{fig:loopseattle_ood_forecast_example.pdf}
    \end{minipage}

    \caption{\textbf{(a, c, e)} Forecasts for a ground truth series $\mathbf{y}(t)$ for the Solar, Subseasonal, and Loop Seattle datasets for models trained using the traditional forecasting paradigm. \textbf{(b, d, f)} Forecasts for a ground truth series $\mathbf{y}(t)$ for the Solar, Subseasonal, and Loop Seattle datasets for models trained using the compositional reasoning forecasting paradigm. Patch-based Transformer models and MLP-based models (top), which rank among the top-performing models, demonstrate generalization to out-of-distribution time series, whereas other Transformer variants and linear models struggle to do so (bottom).}
    \label{fig:forecast_examples2_apd}
\end{figure*}


\newpage
\subsection{Model Performance, Efficiency, and Size Comparison}\label{apd:flops_plots}
We rank model performance across datasets and compare rank with model computational complexity in terms of floating-point operations per second (FLOPs) and model size in terms of the total number of trainable parameters. 

\begin{figure*}[htbp]
    \centering

    \begin{minipage}{0.495\textwidth}
        \centering
        \includegraphics[width=\textwidth]{images/flops_aggregate.pdf}
        \subcaption{In-distribution series results (excluding \TimesNet)}
        \label{fig:flops_aggregate}
    \end{minipage}%
    \hfill
    \begin{minipage}{0.495\textwidth}
        \centering
        \includegraphics[width=\textwidth]{images/flops_component.pdf}
        \subcaption{Out-of-distribution series results (excluding \TimesNet)}
        \label{fig:flops_component}
    \end{minipage}

    \vspace{1ex} % Vertical space between rows

    \begin{minipage}{0.495\textwidth}
        \centering
        \includegraphics[width=\textwidth]{images/flops_aggregate_with_timesnet.pdf}
        \subcaption{In-distribution series results (including \TimesNet)}
        \label{fig:flops_aggregate_with_timesnet}
    \end{minipage}%
    \hfill
    \begin{minipage}{0.495\textwidth}
        \centering
        \includegraphics[width=\textwidth]{images/flops_component_with_timesnet.pdf}
        \subcaption{Out-of-distribution series results (including \TimesNet)}
        \label{fig:flops_component_with_timesnet}
    \end{minipage}

    \caption{Comparison of average rank across datasets and random seeds versus model computational complexity, measured by floating-point operations per second (FLOPs). The size of each point represents the number of trainable parameters, highlighting the trade-offs between model complexity and performance. \textbf{(a)} In-distribution and \textbf{(b)} out-of-distribution results for all models, excluding \TimesNet, are shown to provide a clearer comparison by mitigating the parameter size skew. \textbf{(c)} In-distribution and \textbf{(d)} out-of-distribution results for all models, including \TimesNet.}
    \label{fig:flops_model_comparison_apd}
\end{figure*}


\newpage
\subsection{Model Composition Reasoning Results}\label{apd:composition_full_table_results}
We include the complete table results with MAE error mean and standard deviation measured across three random seeds. The results of the compositional reasoning task for 16 widely adopted time series forecasting models are included in Table~\ref{tab:composition_baseline_results_table}. Composition reasoning task results for controlled ablations of architecture components used in TSFMs are shown in Table~\ref{tab:composition_t5_results_table}.

\begin{table}[!ht] 
\centering
\caption{Mean Absolute Error (MAE) averaged over 3 random seeds (with standard deviation in parentheses) for composition reasoning tasks. The out-of-distribution (OOD) column presents MAE results for models trained via the compositional reasoning forecasting paradigm. The in-distribution (ID) column presents MAE results for models trained via the traditional forecasting paradigm. The \Tfive\ with the best patch length (PL) from Table~\ref{tab:composition_t5_results_table} is included. Best results are highlighted in \textbf{bold}, second best results are \underline{underlined}. The count of instances across datasets where the model ranks in the top three for performance is shown in the second to last column with non-zero entries in \textcolor{blue}{blue}. The average number of top $k$ compositions the model can outperform over the datasets is shown in the last column with nonzero entries in \textcolor{purple}{purple}.}
\label{tab:composition_baseline_results_table}
\resizebox{1.0\textwidth}{!}{
\begin{tabular}{ll|cc|cc|cc|cc|cc|cc||cc|cc}
\toprule
\multicolumn{2}{c|}{\multirow{2}{*}{\textbf{Model}}} & \multicolumn{2}{c}{\textbf{Synthetic Sinusoid}} & \multicolumn{2}{c}{\textbf{ECL}} & \multicolumn{2}{c}{\textbf{ETTm2}} & \multicolumn{2}{c}{\textbf{Solar}} & \multicolumn{2}{c}{\textbf{Subseasonal}} & \multicolumn{2}{c||}{\textbf{Loop Seattle}} & \multicolumn{2}{c}{\textbf{\small{Top 3 Win Count}}} & \multicolumn{2}{c}{\textbf{\small{Top $k$ Basis Wins}}} \\
\cline{3-18}
{} & {} & \textbf{OOD} & \textbf{ID} & \textbf{OOD} & \textbf{ID} & \textbf{OOD} & \textbf{ID} & \textbf{OOD} & \textbf{ID} & \textbf{OOD} & \textbf{ID} & \textbf{OOD} & \textbf{ID} & \textbf{\small{OOD}} & \textbf{\small{ID}} & \textbf{\small{OOD}} & \textbf{\small{ID}} \\
\hline\hline
\multirow{4}{*}{\rotatebox[origin=c]{90}{\textbf{Statistical}}} & \multirow{2}{*}{\ARIMA} & -- & 15.538 & -- & 0.822 & -- & 0.332 & -- & 9.687 & -- & 7.855 & -- & 8.638 & \multirow{2}{*}{\small{0}} & \multirow{2}{*}{\small{0}} & \multirow{2}{*}{\small{--}} & \multirow{2}{*}{\small{0.5}} \\
                      {} & {} &
                      \small{(--)} & 
                      \small{(--)} & 
                      \small{(--)} & 
                      \small{(--)} & 
                      \small{(--)} & 
                      \small{(--)} & 
                      \small{(--)} & 
                      \small{(--)} &
                      \small{(--)} & 
                      \small{(--)} & 
                      \small{(--)} & 
                      {} &
                      {} &
                      {} \\
\cline{2-18}
{} & \multirow{2}{*}{\ETS} & -- & 16.075 & -- & 0.105 & -- & 0.211 & -- & 1.730 & -- & 2.067 & -- & 5.575 & \small{--} & \small{0} & \small{--} & \small{0.8} \\
                      {} & {} &
                      \small{(--)} & 
                      \small{(--)} & 
                      \small{(--)} & 
                      \small{(--)} & 
                      \small{(--)} & 
                      \small{(--)} & 
                      \small{(--)} & 
                      \small{(--)} &
                      \small{(--)} & 
                      \small{(--)} & 
                      \small{(--)} & 
                      \small{(--)} &
                      {} &
                      {} \\
\hline
\multirow{4}{*}{\rotatebox[origin=c]{90}{\textbf{Linear}}} & \multirow{2}{*}{\DLinear} & 12.991 & 12.460 & 0.820 & \underline{0.103} & 0.330 & 0.135 & 9.925 & \textbf{1.555} & 8.042 & 1.496 & 9.085 & 4.293 & \multirow{2}{*}{\small{0}} & \multirow{2}{*}{\small{\textcolor{blue}{2}}} & \multirow{2}{*}{\small{0.2}} & \multirow{2}{*}{\small{\textcolor{purple}{55.5}}} \\
                      {} & {} &
                      \small{(0.051)} & \small{(0.025)} & \small{(0.073)} & \small{(0.000)} & \small{(0.028)} & \small{(0.000)} & \small{(1.000)} & \small{(0.007)} & \small{(0.402)} & \small{(0.030)} &
                      \small{(0.327)} & 
                      \small{(0.033)} &
                      {} &
                      {} \\
\cline{2-18}
{} & \multirow{2}{*}{\NLinear} & 13.287 & 13.056 & 0.801 & 0.104 & 0.325 & 0.136 & 9.681 & \underline{1.569} & 8.436 & 1.509 & 9.026 & 4.307 & \multirow{2}{*}{\small{0}} & \multirow{2}{*}{\small{\textcolor{blue}{1}}} & \multirow{2}{*}{\small{0.3}} & \multirow{2}{*}{\small{\textcolor{purple}{53.7}}} \\
                      {} & {} &
                      \small{(0.098)} & \small{(0.060)} & \small{(0.026)} & \small{(0.001)} & \small{(0.007)} & \small{(0.002)} & \small{(1.203)} & \small{(0.010)} & \small{(0.552)} & \small{(0.011)} &
                      \small{(0.248)} & 
                      \small{(0.017)} &
                      {} &
                      {} \\
\hline
\multirow{8}{*}{\rotatebox[origin=c]{90}{\textbf{MLP-Based}}} & \multirow{2}{*}{\MLP} & \underline{8.647} & 2.475 & \underline{0.283} & 0.106 & 0.253 & 0.114 & \underline{4.826} & 1.559 & 1.886 & 1.456 & 7.839 & 3.864 & \multirow{2}{*}{\small{\textcolor{blue}{3}}} & \multirow{2}{*}{\small{\textcolor{blue}{1}}} & \multirow{2}{*}{\small{\textcolor{purple}{10.2}}} & \multirow{2}{*}{\small{\textcolor{purple}{61.2}}} \\
                      {} & {} &
                      \small{(0.258)} & \small{(0.011)} & \small{(0.020)} & \small{(0.004)} & \small{(0.011)} & \small{(0.002)} & \small{(0.043)} & \small{(0.028)} & \small{(0.118)} & \small{(0.046)} &
                      \small{(0.139)} & 
                      \small{(0.061)} &
                      {} &
                      {} \\
\cline{2-18}
{} & \multirow{2}{*}{\NHITS} & 8.924 & \textbf{1.106} & 0.295 & \textbf{0.101} & \textbf{0.214} & \textbf{0.100} & 5.682 & 1.592 & 1.858 & \underline{1.135} & \underline{7.747} & 3.448 & \multirow{2}{*}{\small{\textcolor{blue}{4}}} & \multirow{2}{*}{\small{\textcolor{blue}{4}}} & \multirow{2}{*}{\small{\textcolor{purple}{11.0}}} & \multirow{2}{*}{\small{\textcolor{purple}{64.3}}} \\
                      {} & {} &
                      \small{(0.044)} & \small{(0.036)} & \small{(0.019)} & \small{(0.001)} & \small{(0.006)} & \small{(0.003)} & \small{(0.651)} & \small{(0.026)} & \small{(0.060)} & \small{(0.008)} &
                      \small{(0.141)} & 
                      \small{(0.043)} &
                      {} &
                      {} \\
\cline{2-18}
{} & \multirow{2}{*}{\NBEATS} & 8.907 & 1.383 & 0.294 & \underline{0.103} & \underline{0.216} & \underline{0.102} & 5.852 & 1.599 & \underline{1.840} & 1.177 & 7.763 & 3.479 & \multirow{2}{*}{\small{\textcolor{blue}{5}}} & \multirow{2}{*}{\small{\textcolor{blue}{4}}} & \multirow{2}{*}{\small{\textcolor{purple}{11.2}}} & \multirow{2}{*}{\small{\textcolor{purple}{61.8}}} \\
                      {} & {} &
                      \small{(0.106)} & \small{(0.058)} & \small{(0.016)} & \small{(0.004)} & \small{(0.005)} & \small{(0.002)} & \small{(0.645)} & \small{(0.028)} & \small{(0.108)} & \small{(0.007)} &
                      \small{(0.097)} & 
                      \small{(0.034)} &
                      {} &
                      {} \\
\cline{2-18}
{} & \multirow{2}{*}{\TSMixer} & 14.466 & 15.090 & 0.799 & 0.129 & 0.335 & 0.182 & 9.877 & 1.979 & 7.770 & 1.602 & 8.865 & 5.565 & \multirow{2}{*}{\small{0}} & \multirow{2}{*}{\small{0}} & \multirow{2}{*}{\small{0.3}} & \multirow{2}{*}{\small{\textcolor{purple}{19.7}}} \\
                      {} & {} &
                      \small{(1.804)} & \small{(0.339)} & \small{(0.013)} & \small{(0.004)} & \small{(0.024)} & \small{(0.051)} & \small{(0.118)} & \small{(0.132)} & \small{(0.543)} & \small{(0.099)} &
                      \small{(0.249)} & 
                      \small{(1.038)} &
                      {} &
                      {} \\
\hline
\multirow{2}{*}{\rotatebox[origin=c]{90}{\textbf{RNN}}} & \multirow{2}{*}{\LSTM} & 13.410 & 4.238 & 0.835 & 0.110 & 0.337 & 0.135 & 10.241 & 1.717 & 8.095 & 1.545 & 9.465 & 3.703 & \multirow{2}{*}{\small{0}} & \multirow{2}{*}{\small{0}} & \multirow{2}{*}{\small{0}} & \multirow{2}{*}{\small{\textcolor{purple}{48.7}}} \\
                      {} & {} &
                      \small{(0.203)} & \small{(0.637)} & \small{(0.004)} & \small{(0.001)} & \small{(0.000)} & \small{(0.006)} & \small{(0.620)} & \small{(0.131)} & \small{(0.093)} & \small{(0.075)} &
                      \small{(1.093)} & 
                      \small{(0.054)} &
                      {} &
                      {} \\
\hline
\multirow{4}{*}{\rotatebox[origin=c]{90}{\textbf{CNN}}} & \multirow{2}{*}{\TCN} & 11.478 & 3.833 & 0.837 & 0.106 & 0.339 & 0.135 & 9.868 & 1.642 & 6.170 & 1.234 & 8.792 & \underline{3.422} & \multirow{2}{*}{\small{0}} & \multirow{2}{*}{\small{\textcolor{blue}{1}}} & \multirow{2}{*}{\small{0.7}} & \multirow{2}{*}{\small{\textcolor{purple}{55.7}}} \\
                      {} & {} &
                      \small{(0.410)} & \small{(0.193)} & \small{(0.001)} & \small{(0.002)} & \small{(0.004)} & \small{(0.001)} & \small{(0.016)} & \small{(0.112)} & \small{(3.256)} & \small{(0.031)} &
                      \small{(0.020)} & 
                      \small{(0.119)} &
                      {} &
                      {} \\
\cline{2-18}
{} & \multirow{2}{*}{\TimesNet} & 9.788 & \underline{2.451} & 0.518 & 0.104 & 0.313 & 0.109 & 9.914 & 1.714 & 4.109 & 1.500 & 9.872 & 2.970 & \multirow{2}{*}{\small{0}} & \multirow{2}{*}{\small{\textcolor{blue}{2}}} & \multirow{2}{*}{\small{\textcolor{purple}{2.3}}} & \multirow{2}{*}{\small{\textcolor{purple}{52.8}}} \\
                      {} & {} &
                      \small{(0.493)} & \small{(0.252)} & \small{(0.285)} & \small{(0.003)} & \small{(0.042)} & \small{(0.004)} & \small{(0.116)} & \small{(0.259)} & \small{(3.409)} & \small{(0.111)} &
                      \small{(0.991)} & 
                      \small{(0.360)} &
                      {} &
                      {} \\
\hline
\multirow{22}{*}{\rotatebox[origin=c]{90}{\textbf{Transformer}}} & \multirow{2}{*}{\VanillaTransformer} & 12.279 & 4.935 & 0.919 & 0.106 & 0.334 & 0.136 & 11.956 & 1.667 & 9.641 & 1.276 & 11.675 & 3.591 & \multirow{2}{*}{\small{0}} & \multirow{2}{*}{\small{0}} & \multirow{2}{*}{\small{0.2}} & \multirow{2}{*}{\small{\textcolor{purple}{54.8}}} \\
                      {} & {} &
                      \small{(0.660)} & \small{(0.080)} & \small{(0.033)} & \small{(0.002)} & \small{(0.038)} & \small{(0.005)} & \small{(2.659)} & \small{(0.032)} & \small{(0.363)} & \small{(0.071)} &
                      \small{(0.803)} & 
                      \small{(0.008)} &
                      {} &
                      {} \\
\cline{2-18}
{} & \multirow{2}{*}{\iTransformer} & 15.478 & 15.203 & 0.829 & 0.157 & 0.326 & 0.196 & 9.822 & 1.805 & 8.447 & 1.628 & 9.182 & 4.871 & \multirow{2}{*}{\small{0}} & \multirow{2}{*}{\small{0}} & \multirow{2}{*}{\small{0.2}} & \multirow{2}{*}{\small{\textcolor{purple}{23.5}}} \\
                      {} & {} &
                      \small{(0.351)} & \small{(0.782)} & \small{(0.007)} & \small{(0.007)} & \small{(0.003)} & \small{(0.001)} & \small{(0.055)} & \small{(0.093)} & \small{(0.503)} & \small{(0.027)} &
                      \small{(0.380)} & 
                      \small{(0.105)} &
                      {} &
                      {} \\
\cline{2-18}
{} & \multirow{2}{*}{\Autoformer} & 15.301 & 15.018 & 0.795 & 0.137 & 0.330 & 0.294 & 10.348 & 2.108 & 7.933 & 2.390 & 8.634 & 4.758 & \multirow{2}{*}{\small{0}} & \multirow{2}{*}{\small{0}} & \multirow{2}{*}{\small{0.3}} & \multirow{2}{*}{\small{\textcolor{purple}{15.8}}} \\
                      {} & {} &
                     \small{(0.184)} & \small{(0.130)} & \small{(0.014)} & \small{(0.009)} & \small{(0.008)} & \small{(0.046)} & \small{(0.945)} & \small{(0.638)} & \small{(0.410)} & \small{(0.551)} &
                      \small{(0.166)} & 
                      \small{(0.307)} &
                      {} &
                      {} \\
\cline{2-18}
{} & \multirow{2}{*}{\Informer} & 14.353 & 10.144 & 0.787 & 0.128 & 0.321 & 0.141 & 8.351 & 1.662 & 6.878 & 1.564 & 11.549 & 4.241 & \multirow{2}{*}{\small{0}} & \multirow{2}{*}{\small{0}} & \multirow{2}{*}{\small{0.5}} & \multirow{2}{*}{\small{\textcolor{purple}{41.0}}} \\
                      {} & {} &
                      \small{(0.174)} & \small{(2.257)} & \small{(0.082)} & \small{(0.006)} & \small{(0.016)} & \small{(0.007)} & \small{(1.560)} & \small{(0.051)} & \small{(1.340)} & \small{(0.094)} &
                      \small{(0.627)} & 
                      \small{(0.105)} &
                      {} &
                      {} \\
\cline{2-18}
{} & \multirow{2}{*}{\TFT} & 14.531 & 9.745 & 0.445 & 0.115 & 0.312 & 0.117 & 12.873 & 2.106 & 2.684 & 1.454 & 11.280 & 5.340 & \multirow{2}{*}{\small{0}} & \multirow{2}{*}{\small{0}} & \multirow{2}{*}{\small{\textcolor{purple}{4.7}}} & \multirow{2}{*}{\small{\textcolor{purple}{28.5}}} \\
                      {} & {} &
                      \small{(0.880)} & \small{(1.210)} & \small{(0.072)} & \small{(0.010)} & \small{(0.029)} & \small{(0.005)} & \small{(2.141)} & \small{(0.538)} & \small{(0.146)} & \small{(0.151)} &
                      \small{(1.182)} & 
                      \small{(0.855)} &
                      {} &
                      {} \\
\cline{2-18}
{} & \multirow{2}{*}{\PatchTST\ (PL=8)} & 13.808 & 12.036 & 0.713 & 0.428 & 0.309 & 0.156 & 9.081 & 2.350 & 6.242 & 2.007 & 11.440 & 4.755 & \multirow{2}{*}{\small{0}} & \multirow{2}{*}{\small{0}} & \multirow{2}{*}{\small{0.7}} & \multirow{2}{*}{\small{\textcolor{purple}{18.7}}} \\
                      {} & {} &
                      \small{(0.471)} & \small{(0.738)} & \small{(0.172)} & \small{(0.318)} & \small{(0.031)} & \small{(0.026)} & \small{(0.857)} & \small{(0.872)} & \small{(1.286)} & \small{(0.228)} &
                      \small{(1.564)} & 
                      \small{(0.774)} &
                      {} &
                      {} \\
\cline{2-18}
{} & \multirow{2}{*}{\PatchTST\ (PL=16)} & 14.133 & 13.633 & 0.666 & 0.279 & 0.253 & 0.150 & 10.788 & 1.633 & 5.877 & 1.904 & 9.855 & 4.989 & \multirow{2}{*}{\small{0}} & \multirow{2}{*}{\small{0}} & \multirow{2}{*}{\small{0.8}} & \multirow{2}{*}{\small{\textcolor{purple}{27.8}}} \\
                      {} & {} &
                      \small{(2.693)} & \small{(1.202)} & \small{(0.185)} & \small{(0.025)} & \small{(0.017)} & \small{(0.005)} & \small{(0.139)} & \small{(0.013)} & \small{(1.389)} & \small{(0.336)} &
                      \small{(0.618)} & 
                      \small{(1.000)} &
                      {} &
                      {} \\
\cline{2-18}
{} & \multirow{2}{*}{\PatchTST\ (PL=32)} & 13.316 & 13.412 & 0.736 & 0.273 & 0.271 & 0.168 & 8.267 & 2.721 & 4.904 & 2.385 & 9.422 & 4.924 & \multirow{2}{*}{\small{0}} & \multirow{2}{*}{\small{0}} & \multirow{2}{*}{\small{\textcolor{purple}{1.3}}} & \multirow{2}{*}{\small{\textcolor{purple}{14.3}}} \\
                      {} & {} &
                      \small{(1.751)} & \small{(2.499)} & \small{(0.248)} & \small{(0.106)} & \small{(0.010)} & \small{(0.002)} & \small{(1.302)} & \small{(1.625)} & \small{(2.018)} & \small{(0.639)} &
                      \small{(1.359)} & 
                      \small{(0.133)} &
                      {} &
                      {} \\
\cline{2-18}
{} & \multirow{2}{*}{\PatchTST\ (PL=64)} & 12.232 & 13.544 & 0.482 & 0.122 & 0.247 & 0.169 & 8.054 & 2.813 & 2.292 & 1.754 & 10.529 & 4.405 & \multirow{2}{*}{\small{\textcolor{blue}{1}}} & \multirow{2}{*}{\small{0}} & \multirow{2}{*}{\small{\textcolor{purple}{6.8}}} & \multirow{2}{*}{\small{\textcolor{purple}{27.2}}} \\
                      {} & {} &
                      \small{(0.252)} & \small{(1.055)} & \small{(0.306)} & \small{(0.011)} & \small{(0.007)} & \small{(0.040)} & \small{(2.248)} & \small{(0.724)} & \small{(0.152)} & \small{(0.353)} &
                      \small{(4.987)} & 
                      \small{(0.883)} &
                      {} &
                      {} \\
\cline{2-18}
{} & \multirow{2}{*}{\PatchTST\ (PL=96)} & 11.235 & 8.374 & 0.508 & 0.196 & 0.250 & 0.147 & 6.426 & 1.799 & 2.185 & 1.659 & 7.965 & 4.579 & \multirow{2}{*}{\small{0}} & \multirow{2}{*}{\small{0}} & \multirow{2}{*}{\small{\textcolor{purple}{7.7}}} & \multirow{2}{*}{\small{\textcolor{purple}{27.0}}} \\
                      {} & {} &
                      \small{(0.483)} & \small{(1.834)} & \small{(0.287)} & \small{(0.051)} & \small{(0.013)} & \small{(0.021)} & \small{(1.312)} & \small{(0.304)} & \small{(0.168)} & \small{(0.282)} &
                      \small{(0.355)} & 
                      \small{(0.978)} &
                      {} &
                      {} \\
\cline{2-18}
{} & \multirow{2}{*}{\PatchTST\ (PL=128)} & 10.696 & 6.959 & 0.832 & 0.161 & 0.323 & 0.138 & 5.726 & 1.964 & 2.448 & 1.678 & 9.378 & 3.653 & \multirow{2}{*}{\small{0}} & \multirow{2}{*}{\small{0}} & \multirow{2}{*}{\small{\textcolor{purple}{5.7}}} & \multirow{2}{*}{\small{\textcolor{purple}{32.7}}} \\
                      {} & {} &
                      \small{(0.907)} & \small{(2.126)} & \small{(0.010)} & \small{(0.053)} & \small{(0.037)} & \small{(0.018)} & \small{(0.567)} & \small{(0.266)} & \small{(0.467)} & \small{(0.100)} &
                      \small{(0.185)} & 
                      \small{(0.257)} &
                      {} &
                      {} \\
\cline{2-18}
{} & \multirow{2}{*}{\Tfive\ (Best PL)} & \textbf{7.177} & 2.480 & \textbf{0.239} & \underline{0.103} & 0.259 & 0.103 & \textbf{3.899} & 1.578 & \textbf{1.714} & \textbf{1.097} & \textbf{6.589} & \textbf{3.351} & \multirow{2}{*}{\small{\textcolor{blue}{5}}} & \multirow{2}{*}{\small{\textcolor{blue}{4}}} & \multirow{2}{*}{\small{\textcolor{purple}{12.2}}} & \multirow{2}{*}{\small{\textcolor{purple}{64.3}}} \\
                      {} & {} &
                      \small{(0.089)} & 
                      \small{(0.198)} & 
                      \small{(0.005)} & 
                      \small{(0.001)} & 
                      \small{(0.008)} & 
                      \small{(0.006)} & 
                      \small{(0.578)} & 
                      \small{(0.009)} &
                      \small{(0.040) } &
                      \small{0.039)} &
                      \small{(0.109)} & 
                      \small{(0.017)} &
                      {} &
                      {} \\
\bottomrule
% \multicolumn{13}{c}{\textbf{  }} \\
% \cline{1-14}
% \multirow{8}{*}{\rotatebox[origin=c]{90}{\textbf{Baseline}}} & \Fourier\ (topk=1) & 
%     \multicolumn{2}{c|}{13.032} & 
%     \multicolumn{2}{c|}{0.669} & 
%     \multicolumn{2}{c|}{0.325} &
%     \multicolumn{2}{c|}{9.878} & 
%     \multicolumn{2}{c|}{7.977} & 
%     \multicolumn{2}{c}{8.758} \\
% \cline{2-14}
% {} & \Fourier\ (topk=2) & 
%     \multicolumn{2}{c|}{\textcolor{purple}{6.495}} & 
%     \multicolumn{2}{c|}{0.321} & 
%     \multicolumn{2}{c|}{0.305} & 
%     \multicolumn{2}{c|}{7.840} & 
%     \multicolumn{2}{c|}{6.758} & 
%     \multicolumn{2}{c}{8.150} \\
% \cline{2-14}
% {} & \Fourier\ (topk=3) & 
%     \multicolumn{2}{c|}{5.078} & 
%     \multicolumn{2}{c|}{0.250} & 
%     \multicolumn{2}{c|}{0.252} &
%     \multicolumn{2}{c|}{4.218} & 
%     \multicolumn{2}{c|}{5.185} & 
%     \multicolumn{2}{c}{\textcolor{purple}{6.259}} \\
% \cline{2-14}
% {} & \Fourier\ (topk=4) & 
%     \multicolumn{2}{c|}{0.719} & 
%     \multicolumn{2}{c|}{\textcolor{purple}{0.218}} & 
%     \multicolumn{2}{c|}{0.249} & 
%     \multicolumn{2}{c|}{\textcolor{purple}{3.663}} & 
%     \multicolumn{2}{c|}{5.048} & 
%     \multicolumn{2}{c}{6.297} \\
% \cline{2-14}
% {} & \Fourier\ (topk=5) & 
%     \multicolumn{2}{c|}{0.729} & 
%     \multicolumn{2}{c|}{0.185} &
%     \multicolumn{2}{c|}{0.222} & 
%     \multicolumn{2}{c|}{2.058} & 
%     \multicolumn{2}{c|}{4.911} & 
%     \multicolumn{2}{c}{5.961} \\
% \cline{2-14}
% {} & \Fourier\ (topk=6) & 
%     \multicolumn{2}{c|}{0.738} & 
%     \multicolumn{2}{c|}{0.175} &
%     \multicolumn{2}{c|}{0.222} & 
%     \multicolumn{2}{c|}{1.972} & 
%     \multicolumn{2}{c|}{4.812} & 
%     \multicolumn{2}{c}{6.050} \\
% \cline{2-14}
% {} & \Fourier\ (topk=7) & 
%     \multicolumn{2}{c|}{0.737} & 
%     \multicolumn{2}{c|}{0.160} &
%     \multicolumn{2}{c|}{\textcolor{purple}{0.211}} & 
%     \multicolumn{2}{c|}{1.827} & 
%     \multicolumn{2}{c|}{4.662} & 
%     \multicolumn{2}{c}{5.863} \\
% \cline{2-14}
% {} & \Fourier\ (topk=63) & 
%     \multicolumn{2}{c|}{0.887} & 
%     \multicolumn{2}{c|}{0.100} &
%     \multicolumn{2}{c|}{0.075} & 
%     \multicolumn{2}{c|}{1.640} & 
%     \multicolumn{2}{c|}{\textcolor{purple}{1.702}} & 
%     \multicolumn{2}{c}{4.471} \\
% \cline{1-14}
\end{tabular}
}
\end{table}



\begin{table}[ht]
\centering
\caption{Mean Absolute Error (MAE) averaged over 3 random seeds (with standard deviation in parentheses) for composition reasoning tasks. The out-of-distribution (OOD) column presents MAE results for models trained via the compositional reasoning forecasting paradigm. The in-distribution (ID) column presents MAE results for models trained via the traditional forecasting paradigm. Best results are highlighted in \textbf{bold}. The count of instances across datasets where the model has the best performance is shown in the last column with non-zero entries in \textcolor{blue}{blue}.}
\label{tab:composition_t5_results_table}
\resizebox{1.0\textwidth}{!}{
\begin{tabular}{ll|cc|cc|cc|cc|cc|cc||cc}
\toprule
\multicolumn{2}{c|}{\multirow{2}{*}{\textbf{Transformer Model (T5 Backbone)}}} & \multicolumn{2}{c}{\textbf{Synthetic Sinusoid}} & \multicolumn{2}{c}{\textbf{ECL}} & \multicolumn{2}{c}{\textbf{ETTm2}} & \multicolumn{2}{c}{\textbf{Solar}} & \multicolumn{2}{c}{\textbf{Subseasonal}} & \multicolumn{2}{c||}{\textbf{Loop Seattle}} & \multicolumn{2}{c}{\textbf{\small{Win Count}}} \\
\cline{3-16}
{} & {} & \textbf{OOD} & \textbf{ID} & \textbf{OOD} & \textbf{ID} & \textbf{OOD} & \textbf{ID} & \textbf{OOD} & \textbf{ID} & \textbf{OOD} & \textbf{ID} & \textbf{OOD} & \textbf{ID} & \textbf{\small{OOD}} & \textbf{\small{ID}} \\
\hline\hline
\multirow{8}{*}{\rotatebox[origin=c]{90}{\textbf{Tokenization}}} & \multirow{2}{*}{None} & 14.032 & 4.894 & 0.685 & \textbf{0.106} & 0.369 & 0.121 & 9.969 & \textbf{1.635} & 6.401 & 1.572 & 14.327 & 3.766 & \multirow{2}{*}{\small{0}} & \multirow{2}{*}{\small{\textcolor{blue}{2}}}\\
                      {} & {} &
                      \small{(1.612)} & 
                      \small{(0.140)} & 
                      \small{(0.148)} & 
                      \small{(0.005)} & 
                      \small{(0.015)} & 
                      \small{(0.010)} & 
                      \small{(0.804)} & 
                      \small{(0.079)} &
                      \small{(1.739)} & 
                      \small{(0.049) } &
                      \small{(2.548)} & 
                      \small{(0.035)} \\
\cline{2-16}
{} & \multirow{2}{*}{Fixed Length Patches} & \textbf{8.648} & \textbf{2.611} & \textbf{0.266} & 0.107 & \textbf{0.268} & \textbf{0.100} & \textbf{3.908} & 1.663 & \textbf{1.729} & \textbf{1.154} & \textbf{7.658} & \textbf{3.118} & \multirow{2}{*}{\small{\textcolor{blue}{6}}} & \multirow{2}{*}{\small{\textcolor{blue}{4}}}\\
                      {} & {} &
                      \small{(0.072)} & 
                      \small{(0.158)} & 
                      \small{(0.019)} & 
                      \small{(0.002)} & 
                      \small{(0.003)} & 
                      \small{(0.003)} & 
                      \small{(0.204)} & 
                      \small{(0.023)} &
                      \small{(0.028)} & 
                      \small{(0.029)} &
                      \small{(0.100)} & 
                      \small{(0.026)} \\
\cline{2-16}
{} & \multirow{2}{*}{Binning} & 17.039 & 9.504 & 0.833 & 0.270 & 0.317 & 0.199 & 8.445 & 4.719 & 3.758 & 3.253 & 12.728 & 7.735 & \multirow{2}{*}{\small{0}} & \multirow{2}{*}{\small{0}}\\
                      {} & {} &
                      \small{(0.788)} & 
                      \small{(0.502)} & 
                      \small{(0.007)} & 
                      \small{(0.003)} & 
                      \small{(0.011)} & 
                      \small{(0.004)} & 
                      \small{(4.206)} & 
                      \small{(0.151)} &
                      \small{(0.365)} & 
                      \small{(0.532)} &
                      \small{(2.339)} & 
                      \small{(0.985)} \\
\cline{2-16}
{} & \multirow{2}{*}{Lags} & 13.442 & 4.599 & 0.820 & 0.120 & 0.415 & 0.126 & 10.156  & 1.669 & 4.022 & 1.376 & 11.638 & 3.897 & \multirow{2}{*}{\small{0}} & \multirow{2}{*}{\small{0}} \\
                      {} & {} &
                      \small{(0.254)} & 
                      \small{(0.328)} & 
                      \small{(0.178)} & 
                      \small{(0.005)} & 
                      \small{(0.046)} & 
                      \small{(0.006)} & 
                      \small{(1.364)} & 
                      \small{(0.030)} &
                      \small{(0.187)} & 
                      \small{(0.043)} &
                      \small{(1.556)} & 
                      \small{(0.074)} \\
\hline\hline
\multirow{8}{*}{\rotatebox[origin=c]{90}{\textbf{Model Size}}} & \multirow{2}{*}{Tiny} & \textbf{7.644} & 2.628 & \textbf{0.239} & 0.105 & 0.274 & 0.109 & 3.899 & \textbf{1.641} & 1.899 & 1.097 & \textbf{6.701} & 3.355 & \multirow{2}{*}{\small{\textcolor{blue}{3}}} & \multirow{2}{*}{\small{\textcolor{blue}{1}}} \\
                      {} & {} &
                      \small{(0.025)} & 
                      \small{(0.108)} & 
                      \small{(0.005)} & 
                      \small{(0.002)} & 
                      \small{(0.005)} & 
                      \small{(0.006)} & 
                      \small{(0.578)} & 
                      \small{(0.068)} &
                      \small{(0.203)} & 
                      \small{(0.039)} &
                      \small{(0.153)} & 
                      \small{(0.053)} \\
\cline{2-16}
{} & \multirow{2}{*}{Mini} & 7.882 & 2.318 & 0.242 & 0.107 & \textbf{0.273} & 0.103 & \textbf{3.810} & 1.663 & \textbf{1.769} & 1.104 & 6.888 & 3.018 & \multirow{2}{*}{\small{\textcolor{blue}{3}}} & \multirow{2}{*}{\small{0}}\\
                      {} & {} &
                      \small{(0.069)} & 
                      \small{(0.076)} & 
                      \small{(0.006)} & 
                      \small{(0.001)} & 
                      \small{(0.005)} & 
                      \small{(0.001)} & 
                      \small{(0.226)} & 
                      \small{(0.027)} &
                      \small{(0.108)} & 
                      \small{(0.008)} &
                      \small{(0.032)} & 
                      \small{(0.011)} \\
\cline{2-16}
{} & \multirow{2}{*}{Small} & 8.057 & 2.103 & 0.268 & 0.103 & 0.268 & 0.096 & 4.172 & 1.665 & 1.770 & 1.048 & 6.924 & 2.764 & \multirow{2}{*}{\small{0}} & \multirow{2}{*}{\small{0}}\\
                      {} & {} &
                      \small{(0.119)} & 
                      \small{(0.098)} & 
                      \small{(0.009)} & 
                      \small{(0.002)} & 
                      \small{(0.014)} & 
                      \small{(0.001)} & 
                      \small{(0.192)} & 
                      \small{(0.028)} &
                      \small{(0.197)} & 
                      \small{(0.016)} &
                      \small{(0.187)} & 
                      \small{(0.033)} \\
\cline{2-16}
{} & \multirow{2}{*}{Base} & 8.308 & \textbf{2.084} & 0.247 & \textbf{0.100} & 0.274 & \textbf{0.091} & 4.338 & 1.667 & 1.853 & \textbf{0.967} & 7.005 & \textbf{2.536} & \multirow{2}{*}{\small{0}} & \multirow{2}{*}{\small{\textcolor{blue}{5}}} \\
                      {} & {} &
                      \small{(0.228)} & 
                      \small{(0.072)} & 
                      \small{(0.014)} & 
                      \small{(0.006)} & 
                      \small{(0.007)} & 
                      \small{(0.001)} & 
                      \small{(0.123)} & 
                      \small{(0.065)} &
                      \small{(0.150)} & 
                      \small{(0.012)} &
                      \small{(0.255)} & 
                      \small{(0.061)} \\
\hline\hline
\multirow{4}{*}{\rotatebox[origin=c]{90}{\textbf{Attn. Type}}} & \multirow{2}{*}{Bidirectional Attn.} & \textbf{7.644} & \textbf{2.628} & \textbf{0.239} & \textbf{0.105} & 0.274 & 0.109 & \textbf{3.899} & 1.641 & 1.899 & \textbf{1.097} & \textbf{6.701} & 3.355 & \multirow{2}{*}{\small{\textcolor{blue}{4}}} & \multirow{2}{*}{\small{\textcolor{blue}{3}}}\\
                      {} & {} &
                      \small{(0.025)} & 
                      \small{(0.108)} & 
                      \small{(0.005)} & 
                      \small{(0.002)} & 
                      \small{(0.005)} & 
                      \small{(0.006)} & 
                      \small{(0.578)} & 
                      \small{(0.068)} &
                      \small{(0.203)} & 
                      \small{(0.039)} &
                      \small{(0.153)} & 
                      \small{(0.053)} \\
\cline{2-16}
{} & \multirow{2}{*}{Causal Attn.} & 7.978 & 2.828 & 0.248 & 0.106 & \textbf{0.267} & \textbf{0.105} & 4.307 & \textbf{1.589} & \textbf{1.820} & 1.170 & 6.891 & \textbf{3.337} & \multirow{2}{*}{\small{\textcolor{blue}{2}}} & \multirow{2}{*}{\small{\textcolor{blue}{3}}} \\
                      {} & {} &
                      \small{(0.02)} & 
                      \small{(0.233)} & 
                      \small{(0.005)} & 
                      \small{(0.004)} & 
                      \small{(0.010)} & 
                      \small{(0.003)} & 
                      \small{(0.144)} & 
                      \small{(0.042)} &
                      \small{(0.173)} & 
                      \small{(0.054)} &
                      \small{(0.133)} & 
                      \small{(0.044)} \\
\hline\hline
\multirow{4}{*}{\rotatebox[origin=c]{90}{\textbf{Proj./Head}}} & \multirow{2}{*}{Linear} & \textbf{7.644} & 2.628 & \textbf{0.239} & 0.105 & 0.274 & 0.109 & \textbf{3.899} & 1.641 & 1.899 & 1.097 & \textbf{6.701} & 3.355 & \multirow{2}{*}{\small{\textcolor{blue}{4}}} & \multirow{2}{*}{\small{0}} \\
                      {} & {} &
                      \small{(0.025)} & 
                      \small{(0.108)} & 
                      \small{(0.005)} & 
                      \small{(0.002)} & 
                      \small{(0.005)} & 
                      \small{(0.006)} & 
                      \small{(0.578)} & 
                      \small{(0.068)} &
                      \small{(0.203)} & 
                      \small{(0.039)} &
                      \small{(0.153)} & 
                      \small{(0.053)} \\
\cline{2-16}
{} & \multirow{2}{*}{Residual} & 8.537 & \textbf{2.617} & 0.311 & \textbf{0.102} & \textbf{0.256} & \textbf{0.085} & 4.880 & \textbf{1.595} & \textbf{1.871} & \textbf{0.924} & 7.931 & \textbf{2.524} & \multirow{2}{*}{\small{\textcolor{blue}{2}}} & \multirow{2}{*}{\small{\textcolor{blue}{6}}} \\
                      {} & {} &
                      \small{(0.184)} & 
                      \small{(0.043)} &
                      \small{(0.011)} & 
                      \small{(0.002)} & 
                      \small{(0.008)} & 
                      \small{(0.003)} & 
                      \small{(0.389)} &
                      \small{(0.039)} & 
                      \small{(0.296)} &
                      \small{(0.016)} &
                      \small{(0.655)} &
                      \small{(0.028)} \\
\hline\hline
\multirow{12}{*}{\rotatebox[origin=c]{90}{\textbf{Token (Patch) Length}}} & \multirow{2}{*}{8} & 9.327 & 3.183 & 0.346 & \textbf{0.103} & 0.300 & 0.111 & 7.721 & \textbf{1.578} & 2.581 & 1.532 & 8.048 & 3.573 & \multirow{2}{*}{\small{0}} & \multirow{2}{*}{\small{\textcolor{blue}{2}}} \\
                      {} & {} &
                      \small{(0.28)} & 
                      \small{(0.117)} & 
                      \small{(0.040)} & 
                      \small{(0.001)} & 
                      \small{(0.013)} & 
                      \small{(0.003)} & 
                      \small{(0.640)} & 
                      \small{(0.009)} &
                      \small{(0.335)} & 
                      \small{(0.002)} &
                      \small{(0.108)} & 
                      \small{(0.047)} \\
\cline{2-16}
{} & \multirow{2}{*}{16} & 9.007 & 3.573 & 0.418 & 0.105 & 0.283 & \textbf{0.103} & 9.139 & 1.673 & 2.731 & 1.362 & 8.499 & 3.510 & \multirow{2}{*}{\small{0}} & \multirow{2}{*}{\small{\textcolor{blue}{1}}} \\
                      {} & {} &
                      \small{(0.057)} & 
                      \small{(0.235)} & 
                      \small{(0.062)} & 
                      \small{(0.002)} & 
                      \small{(0.011)} & 
                      \small{(0.006)} & 
                      \small{(0.400)} & 
                      \small{(0.088)} &
                      \small{(0.939)} & 
                      \small{(0.25)} &
                      \small{(0.339)} & 
                      \small{(0.060)} \\
\cline{2-16}
{} & \multirow{2}{*}{32} & 10.11 & 3.719 & 0.244 & 0.108 & 0.289 & 0.109 & 5.256 & 1.652 & 2.059 & 1.318 & 7.014 & 3.463 & \multirow{2}{*}{\small{0}} & \multirow{2}{*}{\small{0}} \\
                      {} & {} &
                      \small{(0.07)} & 
                      \small{(0.190)} & 
                      \small{(0.011)} & 
                      \small{(0.004)} & 
                      \small{(0.007)} & 
                      \small{(0.002)} & 
                      \small{(0.591)} & 
                      \small{(0.071)} &
                      \small{(0.171)} & 
                      \small{(0.187)} &
                      \small{(0.167)} & 
                      \small{(0.015)} \\
\cline{2-16}
{} & \multirow{2}{*}{64} & 8.747 & 2.971 & \textbf{0.239 }& 0.106 & 0.285 & 0.107 & 4.201 & 1.651 & 1.819 & 1.265 & \textbf{6.589} & 3.408 & \multirow{2}{*}{\small{\textcolor{blue}{2}}} & \multirow{2}{*}{\small{0}} \\
                      {} & {} &
                      \small{(0.243)} & 
                      \small{(0.209)} & 
                      \small{(0.009)} & 
                      \small{(0.003)} & 
                      \small{(0.002)} & 
                      \small{(0.006)} & 
                      \small{(0.103)} & 
                      \small{(0.076)} &
                      \small{(0.032)} & 
                      \small{(0.198)} &
                      \small{(0.109)} & 
                      \small{(0.044)} \\
\cline{2-16}
{} & \multirow{2}{*}{96} & 7.644 & 2.628 & \textbf{0.239} & 0.105 & 0.274 & 0.109 & \textbf{3.899} & 1.641 & 1.899 & \textbf{1.097} & 6.701 & 3.355 & \multirow{2}{*}{\small{\textcolor{blue}{2}}} & \multirow{2}{*}{\small{\textcolor{blue}{1}}} \\
                      {} & {} &
                      \small{(0.025)} & 
                      \small{(0.108)} & 
                      \small{(0.005)} & 
                      \small{(0.002)} & 
                      \small{(0.005)} & 
                      \small{(0.006)} & 
                      \small{(0.578)} & 
                      \small{(0.068)} &
                      \small{(0.203)} & 
                      \small{(0.039)} &
                      \small{(0.153)} & 
                      \small{(0.053)} \\
\cline{2-16}
{} & \multirow{2}{*}{128} & \textbf{7.177} & \textbf{2.480} & 0.255 & 0.107 & \textbf{0.259} & 0.107 & 4.385 & 1.673 & \textbf{1.714} & 1.100 & 6.878 & \textbf{3.351} & \multirow{2}{*}{\small{\textcolor{blue}{3}}} & \multirow{2}{*}{\small{\textcolor{blue}{2}}} \\
                      {} & {} &
                      \small{(0.089)} & 
                      \small{(0.198)} & 
                      \small{(0.012)} & 
                      \small{(0.001)} & 
                      \small{(0.008)} & 
                      \small{(0.006)} & 
                      \small{(0.145)} & 
                      \small{(0.033)} &
                      \small{(0.040)} & 
                      \small{(0.069)} &
                      \small{(0.069)} & 
                      \small{(0.017)} \\
\hline\hline
\multirow{8}{*}{\rotatebox[origin=c]{90}{\textbf{Positional Encoding}}} & \multirow{2}{*}{Relative} & 7.751 & 2.750 & 0.284 & 0.105 & \textbf{0.268} & 0.111 & 4.373 & 1.640 & \textbf{1.826} & \textbf{1.093} & 6.883 & \textbf{3.337} & \multirow{2}{*}{\small{\textcolor{blue}{2}}} & \multirow{2}{*}{\small{\textcolor{blue}{2}}} \\
                      {} & {} &
                      \small{(0.163)} & 
                      \small{(0.262)} & 
                      \small{(0.059) } & 
                      \small{(0.002)} & 
                      \small{(0.015)} & 
                      \small{(0.002)} & 
                      \small{(0.254)} & 
                      \small{(0.033)} &
                      \small{(0.156)} & 
                      \small{(0.025)} &
                      \small{(0.162)} & 
                      \small{(0.020)} \\
\cline{2-16}
{} & \multirow{2}{*}{SinCos} & 7.881 & \textbf{2.456} & 0.247 & \textbf{0.104} & 0.270 & 0.112 & 4.357 & \textbf{1.624} & 1.839 & 1.464 & 6.818 & 3.366 & \multirow{2}{*}{\small{0}} & \multirow{2}{*}{\small{\textcolor{blue}{3}}} \\
                      {} & {} &
                      \small{(0.148)} & 
                      \small{(0.049)} & 
                      \small{(0.015)} & 
                      \small{(0.003)} & 
                      \small{(0.007)} & 
                      \small{(0.004)} & 
                      \small{(0.284)} & 
                      \small{(0.025)} &
                      \small{(0.195)} & 
                      \small{(0.051)} &
                      \small{(0.182} & 
                      \small{(0.055)} \\
\cline{2-16}
{} & \multirow{2}{*}{SinCos+Relative} & \textbf{7.644} & 2.628 & \textbf{0.239} & 0.105 & 0.274 & \textbf{0.109} & \textbf{3.899} & 1.641 & 1.899 & 1.097 & \textbf{6.701} & 3.355 & \multirow{2}{*}{\small{\textcolor{blue}{4}}} & \multirow{2}{*}{\small{\textcolor{blue}{1}}} \\
                      {} & {} &
                      \small{(0.025)} & 
                      \small{(0.108)} & 
                      \small{(0.005)} & 
                      \small{(0.002)} & 
                      \small{(0.005)} & 
                      \small{(0.006)} & 
                      \small{(0.578)} & 
                      \small{(0.068)} &
                      \small{(0.203)} & 
                      \small{(0.039)} &
                      \small{(0.153)} & 
                      \small{(0.053)} \\
\cline{2-16}
{} & \multirow{2}{*}{RoPE} & 7.921 & 2.701 & 0.255 & \textbf{0.104} & 0.274 & 0.112 & 4.608 & 1.647 & 1.872 & 1.106 & 6.721 & 3.378 & \multirow{2}{*}{\small{0}} & \multirow{2}{*}{\small{\textcolor{blue}{1}}} \\
                      {} & {} &
                      \small{(0.125)} & 
                      \small{(0.077)} & 
                      \small{(0.006)} & 
                      \small{(0.002)} & 
                      \small{(0.002)} & 
                      \small{(0.002)} & 
                      \small{(0.333)} & 
                      \small{(0.032)} &
                      \small{(0.125)} &
                      \small{(0.037)} & 
                      \small{(0.174)} &
                      \small{(0.033)} \\
\hline\hline
\multirow{8}{*}{\rotatebox[origin=c]{90}{\textbf{Loss Function}}} & \multirow{2}{*}{MAE} & \textbf{7.644} & 2.628 & 0.239 & 0.105 & 0.274 &\textbf{ 0.109} & 3.899 & \textbf{1.641} & 1.899 & \textbf{1.097} & 6.701 & 3.355 & \multirow{2}{*}{\small{\textcolor{blue}{1}}} & \multirow{2}{*}{\small{\textcolor{blue}{3}}} \\
                      {} & {} &
                      \small{(0.025)} & 
                      \small{(0.108)} & 
                      \small{(0.005)} & 
                      \small{(0.002)} & 
                      \small{(0.005)} & 
                      \small{(0.006)} & 
                      \small{(0.578)} & 
                      \small{(0.068)} &
                      \small{(0.203)} & 
                      \small{(0.039)} &
                      \small{(0.153)} & 
                      \small{(0.053)} \\
\cline{2-16}
{} & \multirow{2}{*}{MSE} & 7.694 & \textbf{2.618} & \textbf{0.229} & 0.109 & \textbf{0.262} & 0.116 & 3.633 & 1.752 & 1.891 & 1.411 & 6.613 & 3.302 & \multirow{2}{*}{\small{\textcolor{blue}{2}}} & \multirow{2}{*}{\small{\textcolor{blue}{1}}} \\
                      {} & {} &
                      \small{(0.178)} & 
                      \small{(0.088)} & 
                      \small{(0.006)} & 
                      \small{(0.004)} & 
                      \small{(0.007)} & 
                      \small{(0.010)} & 
                      \small{(0.466)} & 
                      \small{(0.030)} &
                      \small{(0.193)} & 
                      \small{(0.214)} &
                      \small{(0.118)} & 
                      \small{(0.045)} \\
\cline{2-16}
{} & \multirow{2}{*}{Huber} & 7.648 & 2.914 & 0.234 & 0.108 & 0.264 & 0.112 & \textbf{4.035} & 1.746 & 1.801 & 1.258 & \textbf{6.571} & \textbf{3.281} & \multirow{2}{*}{\small{\textcolor{blue}{2}}} & \multirow{2}{*}{\small{\textcolor{blue}{1}}} \\
                      {} & {} &
                      \small{(0.076)} & 
                      \small{(0.234)} & 
                      \small{(0.012)} & 
                      \small{(0.005)} & 
                      \small{(0.012)} & 
                      \small{(0.003)} & 
                      \small{(0.633)} & 
                      \small{(0.056)} &
                      \small{(0.084)} & 
                      \small{(0.194)} &
                      \small{(0.182)} & 
                      \small{(0.047)} \\
\cline{2-16}
{} & \multirow{2}{*}{StudentT} & 7.776 & 2.660 & 0.244 & \textbf{0.104} & 0.270 & 0.113 & 4.134 & 1.648 & \textbf{1.735} & 1.377 & 6.872 & 3.543 & \multirow{2}{*}{\small{\textcolor{blue}{1}}} & \multirow{2}{*}{\small{\textcolor{blue}{1}}} \\
                      {} & {} &
                      \small{(0.281)} & 
                      \small{(0.105)} & 
                      \small{(0.003)} & 
                      \small{(0.003)} & 
                      \small{(0.004)} & 
                      \small{(0.003)} & 
                      \small{(0.804)} & 
                      \small{(0.025)} &
                      \small{(0.037)} & 
                      \small{(0.024)} &
                      \small{(0.191)} & 
                      \small{(0.026)} \\
\hline\hline
\multirow{4}{*}{\rotatebox[origin=c]{90}{\textbf{Scaler}}} & \multirow{2}{*}{RevIN (Standard, non-learnable)} & \textbf{7.644} & \textbf{2.628} & \textbf{0.239} & 0.105 & 0.274 & 0.109 & \textbf{3.899} & \textbf{1.641} & 1.899 & \textbf{1.097} & \textbf{6.701} & \textbf{3.355} & \multirow{2}{*}{\small{\textcolor{blue}{4}}} & \multirow{2}{*}{\small{\textcolor{blue}{4}}} \\
                      {} & {} &
                      \small{(0.025)} & 
                      \small{(0.108)} & 
                      \small{(0.005)} & 
                      \small{(0.002)} & 
                      \small{(0.005)} & 
                      \small{(0.006)} & 
                      \small{(0.578)} & 
                      \small{(0.068)} &
                      \small{(0.203)} & 
                      \small{(0.039)} &
                      \small{(0.153)} & 
                      \small{(0.053)} \\
\cline{2-16}
% {} & \multirow{2}{*}{RevIN (Standard, learnable)} & 7.644 & 2.628 & 0.239 & 0.105 & 0.274 & 0.109 & 3.899 & 1.641 & 1.899 & 1.097 & 6.701 & 3.355 \\
%                       {} & {} &
%                       \small{(0.025)} & 
%                       \small{(0.108)} & 
%                       \small{(0.005)} & 
%                       \small{(0.002)} & 
%                       \small{(0.005)} & 
%                       \small{(0.006)} & 
%                       \small{(0.578)} & 
%                       \small{(0.068)} &
%                       \small{(0.203)} & 
%                       \small{(0.039)} &
%                       \small{(0.153)} & 
%                       \small{(0.053)} \\
% \cline{2-14}
{} & \multirow{2}{*}{Robust} & 7.931 & 2.725 & 0.326 & \textbf{0.103} & \textbf{0.270} & \textbf{0.106} & 8.170 & 1.734 & \textbf{1.736} & 1.232 & 6.982 & 3.412 & \multirow{2}{*}{\small{\textcolor{blue}{2}}} & \multirow{2}{*}{\small{\textcolor{blue}{2}}} \\
                      {} & {} &
                      \small{(0.026)} & 
                      \small{(0.234)} & 
                      \small{(0.006)} & 
                      \small{(0.003)} & 
                      \small{(0.007)} & 
                      \small{(0.007)} & 
                      \small{(0.389)} & 
                      \small{(0.093)} &
                      \small{(0.015)} & 
                      \small{(0.227)} &
                      \small{(0.271)} & 
                      \small{(0.043)} \\
\hline\hline
\multirow{4}{*}{\rotatebox[origin=c]{90}{\textbf{Context}}} & \multirow{2}{*}{256} & 7.644 & 2.628 & \textbf{0.239} & \textbf{0.105} & \textbf{0.274} & \textbf{0.109} & \textbf{3.899} & \textbf{1.641} & \textbf{1.899} & \textbf{1.097} & 6.701 & \textbf{3.355} & \multirow{2}{*}{\small{\textcolor{blue}{4}}} & \multirow{2}{*}{\small{\textcolor{blue}{5}}} \\
                      {} & {} &
                      \small{(0.025)} & 
                      \small{(0.108)} & 
                      \small{(0.005)} & 
                      \small{(0.002)} & 
                      \small{(0.005)} & 
                      \small{(0.006)} & 
                      \small{(0.578)} & 
                      \small{(0.068)} &
                      \small{(0.203)} & 
                      \small{(0.039)} &
                      \small{(0.153)} & 
                      \small{(0.053)} \\
\cline{2-16}
{} & \multirow{2}{*}{512} & \textbf{7.072} & \textbf{2.605} & 0.253 & 0.111 & 0.298 & 0.151 & 4.187 & 1.740 & 1.934 & 1.829 & \textbf{6.295} & 4.013 & \multirow{2}{*}{\small{\textcolor{blue}{2}}} & \multirow{2}{*}{\small{\textcolor{blue}{1}}} \\
                      {} & {} &
                      \small{(0.382)} & 
                      \small{(0.158)} & 
                      \small{(0.037)} & 
                      \small{(0.005)} & 
                      \small{(0.015)} & 
                      \small{(0.008)} & 
                      \small{(0.216)} & 
                      \small{(0.046)} &
                      \small{(0.083)} & 
                      \small{(0.094)} &
                      \small{(0.284)} & 
                      \small{(0.002)} \\
\hline\hline
\multirow{4}{*}{\rotatebox[origin=c]{90}{\textbf{Decomp.}}} & \multirow{2}{*}{None} & \textbf{7.644} & \textbf{2.628} & \textbf{0.239} & \textbf{0.105} & 0.274 & \textbf{0.109} & \textbf{3.899} & \textbf{1.641} & 1.899 & \textbf{1.097} & \textbf{6.701} & \textbf{3.355} & \multirow{2}{*}{\small{\textcolor{blue}{4}}} & \multirow{2}{*}{\small{\textcolor{blue}{6}}} \\
                      {} & {} &
                      \small{(0.025)} & 
                      \small{(0.108)} & 
                      \small{(0.005)} & 
                      \small{(0.002)} & 
                      \small{(0.005)} & 
                      \small{(0.006)} & 
                      \small{(0.578)} & 
                      \small{(0.068)} &
                      \small{(0.203)} & 
                      \small{(0.039)} &
                      \small{(0.153)} & 
                      \small{(0.053)} \\
\cline{2-16}
{} & \multirow{2}{*}{Moving Avg. Filter (DLinear, Autoformer)} & 6.726 & 2.860 & 0.265 & 0.106 & \textbf{0.264} & 0.112 & 4.018 & 1.663 & \textbf{1.769} & 1.252 & 6.942 & 3.470 & \multirow{2}{*}{\small{\textcolor{blue}{2}}} & \multirow{2}{*}{\small{0}} \\
                      {} & {} &
                      \small{(0.184)} & 
                      \small{(0.483)} & 
                      \small{(0.015)} & 
                      \small{(0.002)} & 
                      \small{(0.015)} & 
                      \small{(0.001)} & 
                      \small{(0.256)} & 
                      \small{(0.063)} &
                      \small{(0.132)} & 
                      \small{(0.243)} &
                      \small{(0.283)} & 
                      \small{(0.043)} \\
\bottomrule
\end{tabular}
}
\end{table}





%% The Appendices part is started with the command \appendix;
%% appendix sections are then done as normal sections

%% If you have bibdatabase file and want bibtex to generate the
%% bibitems, please use
%%

%% else use the following coding to input the bibitems directly in the
%% TeX file.

% \begin{thebibliography}{00}

% %% \bibitem[Author(year)]{label}
% %% Text of bibliographic item

% \bibitem[ ()]{}

% \end{thebibliography}
\end{document}

\endinput
%%
%% End of file `elsarticle-template-harv.tex'.
