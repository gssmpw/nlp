\section{Related Work}
%\subsection{Code-Switching and Multilingual ASR} Multilingual ASR, especially in low-resource languages such as those spoken in Africa, faces significant challenges, particularly with accented speech and code-switching. \cite{nguyen2020canvec} examined the difficulties of ASR in code-switching, working on the Vietnamese-English CanVEC dataset, showed that monolingual ASR models struggle with code-switched speech, resulting in significantly lower performance. The lack of high-quality code-switching corpora limits ASR models' generalization ability across multilingual contexts.  

%Similarly,  \cite{lyu2010seame} found high word error rates (WER) on the SEAME Mandarin-English dataset due to spontaneous speech and frequent language switching within individual sentences. This issue is not limited to a specific language pair, as similar challenges were observed in Chinese-English (ASCEND)  \cite{lovenia2021ascend} and Arabic-English with French datasets \cite{chowdhury2021towards}, where the multilingual ASR system outperformed SOTA monolingual models, achieving lower WERs on Arabic dialectal and code-switching test sets and performed comparably with English and French datasets.

%Extensive research further underscores the challenges of code-switching across multiple language pairs in text \cite{etori2024rideke} and speech modalities, such as Hindi-English, Bengali-English, Gujarati-English
%Tamil-English \cite{banerjee2018dataset}, Spanish-English and modern Standard Arabic-Egyptian \cite{aguilar2019named}, and Arabic-French \cite{chowdhury2021towards}, overall reveals difficulties in code-switched datasets.


\subsection{ASR in Medical Conversations and Summarization}
The role of ASR in medical documentation has grown significantly, particularly in telehealth and in-person patient-physician consultations \cite{korfiatis2022primock57, galloway2024impact, michalopoulos2022medicalsum, yim2023aci}. Accurate ASR in medical dialogue is critical, as transcription errors can lead to incorrect medical records. Several datasets have been developed to facilitate the study of ASR in medical contexts. PriMock57 \cite{korfiatis2022primock57} provides primary care mock consultations in European English accents; 
\citet{fareez2022dataset} evaluates medical ASR on simulated patient-physician medical interviews with a focus on respiratory cases; \citet{enarvi2020generating} experiments automatically generated medical reports from diarized doctor-patient surgery conversations. %Their unsupervised approach (mimics machine translation and summarization but differs in the homogeneity of the source-target language, reasoning over long span of source sentences, and occurrence of incomplete and irrelevant information) reveals better performance and scalability of Transformers compared to RNN models in ROGUE-L and medical fact extractors F1 relative error rate across each section of the dataset. 
\citet{le2024real} proposed a real-time speech summarization system for medical conversations with the ability to generate summaries for every (local) and end of (global) utterances, eliminating the need for continuous update and revision of the summary state. 

These datasets primarily focus on non-African accents and therefore do not account for the challenges specific to African-accented medical speech.

\subsection{Non-medical conversational ASR} 
\citet{pkezik2022diabiz} released DiaBiz, an Annotated Corpus of over 400hrs of Polish call center dialogs. Other conversational, parliamentary, or oratory datasets like AMI \cite{carletta2005ami}, Earnings22 \cite{del2022earnings}, Voxpopuli \cite{wang-etal-2021-voxpopuli} have gained popularity on public ASR benchmarks. Conversational ASR has also been explored in other domains such as call centers \citep{plaza2021call}, and robotics \citep{skantze2021turn}. However, these datasets lack representation of African-accented speech.

\subsection{African Accented ASR}
There has been growing interest in developing ASR systems that cater to African languages; for example, \citet{yilmaz2018building} developed a multilingual ASR system for code-switched South African speech. 
% East African ASR \cite{elamin2023multilingual}.
Multilingual ASR systems, such as EVI dataset \cite{spithourakis2022evi}, offer a strong foundation for developing similar models in African contexts where data scarcity hinders progress. \citet{olatunji2023afrispeech} released a pan-African accented English dataset for medical and general ASR. While these datasets focus on single-speaker speech recognition, AfriSpeech-Dialog is the first African-accented English conversational dataset spanning medical and non-medical domains, enabling additional tasks like diarization and summarization.


\subsection{Speaker Diarization in Multi-Speaker Conversations}
To increase the efficiency of NLP/ASR systems, enormous contributions were made to researching the integration of speaker diarization (SD) into its pipeline. \citet{serikov2024proceedings} provides a comparative analysis of SD models - Pyannote \cite{bredin2020pyannote}, CLEVER\footnote{https://www.oxfordwaveresearch.com/products/cleaver/}, and NVIDIA NeMo \cite{harper2019nemo} on 20 different German dialects for diarization and identification (DI) task. NVIDIA NeMo performs slightly better with a competitive performance due to its multiscale segmentation for identifying and removing shorter segments. On a similar DI task, \cite{chua2023merlion} also benchmarked the performance of multilingual ASR models in open and closed tracks on the challenging MERLIon CCS English - Mandarin datasets - an extracted spontaneous and codeswitching parent-child conversation speeches. However, SD for African-accented conversations remains underexplored. Benchmarking SOTA SD models on AfriSpeech-Dialog reveals their limitations in this setting.