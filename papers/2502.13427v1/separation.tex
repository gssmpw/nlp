\section{Separation between quantum one-way and two-way-LOCC SMP protocols}\label{sec:separation}

For hybrid SMP schemes for non-symmetric problems, it is important which party can send a quantum message while the other one can only send classical messages, and let us introduce more fine-grained hybrid SMP models. In this section, we will consider a hybrid SMP model in which Alice sends a quantum message and Bob sends a classical message, and let us denote by $\Q \R^{||}(R)$ the hybrid SMP communication complexity for $R$. We also consider a hybrid SMP model in which Bob sends a quantum message and Alice sends a classical message, and let us denote by $\R \Q^{||}(R)$ the hybrid SMP communication complexity for $R$. 
We also define $\Q^{||,\mathsf{LOCC}_1^{A \rightarrow B}}(R)$ as the complexity of a quantum one-way LOCC protocol for $R$ where $\rm{Ref}_A$ measures Alice's message first and sends a measurement result to $\rm{Ref}_B$, and then $\rm{Ref}_B$ measures Bob's message, $\Q^{||,\mathsf{LOCC}_1^{B \rightarrow A}}(R)$ as the complexity of a quantum one-way LOCC protocol for $R$ where the $\rm{Ref}_B$ measures Bob's message first and sends a measurement result to $\rm{Ref}_A$, and then $\rm{Ref}_A$ measures Alice's message.
Other notations of communication complexity are the same as defined in \cref{sec:prel}.

The Hidden Matching Problem was introduced by Bar-Yossef, Jayram, and Kerenidis \cite{BYJK08} and they showed that there exists a relation problem which exhibits an exponential separation between classical and quantum one-way communication complexity. For $k \in [n]$, let us denote by $x(k)$ the $(k-1)$th bit of $x \in \{0,1\}^n$.

\begin{definition}[The Hidden Matching Problem  ($\mathsf{HM}_n$) \cite{BYJK08}]
    Let $n$ be an even positive integer. In the Hidden Matching Problem ($\mathsf{HM}_n$), Alice is given $x \in \{0,1\}^n$ and Bob is given $M \in \mathcal{M}_n$ where $ \mathcal{M}_n$ is the family of all possible perfect matchings on $n$ nodes. They coordinate to output a tuple $\langle i,j,b \rangle$ such that the edge $(i,j)$ belongs to the matching $M$ and $b = x(i) \oplus x(j)$.
\end{definition}

\begin{theorem}[\cite{BYJK08}]
    $\Q^{1}(\mathsf{HM}_n) = O(\log n)$, $\R^{1}(\mathsf{HM}_n) = \Theta(\sqrt{n})$.
\end{theorem}

For completeness, let us describe a quantum protocol to solve $\mathsf{HM}_n$ efficiently.

\begin{algorithm}[H]
\caption{\, $O(\log n)$ qubit one-way communication protocol for $\mathsf{HM}_n$ \cite{BYJK08}}
\label{}
\begin{algorithmic}[1]
\State Alice sends $O(\log n)$-qubit state $\frac{1}{\sqrt{n}} \sum_{i=1}^n (-1)^{x(i)} \ket{i}$ to Bob. 
\State Bob regards $(i,j) \in M$ as a projector $P_{ij} = \ket{i}\bra{i} + \ket{j}\bra{j}$, and conducts the corresponding projective measurement on the quantum message. If the result is $(i,j) \in M$, Bob finally measures the residue state in a base $\{ \frac{1}{\sqrt{2}} (\ket{i}+\ket{j}), \frac{1}{\sqrt{2}}(\ket{i}-\ket{j}) \}$ and outputs the final measurement result.
\end{algorithmic}
\end{algorithm}

They also introduced a variant of the Hidden Matching Problem and showed a separation between classical SMP models and the hybrid SMP model.

\begin{definition}[Restricted Hidden Matching Problem ($\mathsf{RHM}_n$), Section 5 in \cite{BYJK08}]
    Let $n$ be an even positive integer. In the Restricted Hidden Matching Problem, fix $\mathcal{M}$ to be any set of $m = \Theta(n)$ pairwise edge-disjoint matchings. Alice is given $x \in \{0, 1\}^n$, and Bob is given $M \in \mathcal{M}$. Their goal is to output a tuple $\langle i, j, b \rangle$ such that the edge $(i, j)$ belongs to the matching $M$ and $b = x(i) \oplus x(j)$.
\end{definition}

\begin{lemma}[Section 5 in \cite{BYJK08}]\label{lem:rhm}
    $\R^{1}(\mathsf{RHM}_n) = \R^{||}(\mathsf{RHM}_n) = \Theta(\sqrt{n})$, $\Q\R^{||}(\mathsf{RHM}_n) = O(\log n)$.
\end{lemma}

Note that in the hybrid SMP protocol, Bob tells the referee which matching he received from $O(n)$ matchings by $O(\log n)$ bits.

We show that, in a quantum one-way LOCC protocol, if the referee measures Bob's message first, the problem is easy, and, if the referee measures Alice's message first, the problem is hard.

\begin{proposition}
    $\Q^{||,\mathsf{LOCC}_1^{B \rightarrow A}}(\mathsf{RHM}_n) = O(\log n)$, $\R\Q^{||}(\mathsf{RHM}_n) = \Q^{||,\mathsf{LOCC}_1^{A \rightarrow B}}(\mathsf{RHM}_n) = \Theta(\sqrt{n})$.
\end{proposition}

\begin{proof}
    $\Q^{||,\mathsf{LOCC}_1^{B \rightarrow A}}(\mathsf{RHM}_n) = O(\log n)$ follows from $\Q\R^{||}(\mathsf{RHM}_n) = O(\log n)$ by encoding and decoding the information of the matchings in the computational basis.
    
    When we regard Bob and the referee as one party and allow arbitrary communication between them, the communication model is the same as the classical one-way communication model. In the classical one-way communication model, $\Theta(\sqrt{n})$ bits communication is required from \cref{lem:rhm}. Therefore, in the hybrid SMP model as a weaker model than the classical one-way communication model, $\Theta(\sqrt{n})$ bits communication is also required (i.e, $\R\Q^{||}(\mathsf{RHM}_n) = \Theta (\sqrt{n})$).
    
    Finally, consider $\Q^{||,\mathsf{LOCC}_1^{A \rightarrow B}}(\mathsf{RHM}_n)$. From \cref{thm:one-way_LOCC}, we can replace Alice's quantum message with a classical message by a small overhead, and we have $\Q^{||,\mathsf{LOCC}_1^{A \rightarrow B}}(\mathsf{RHM}_n) =  \Theta (\sqrt{n})$.
\end{proof}

We next consider a further variant of the Restricted Hidden Matching Problem to make both parties send quantum messages.

\begin{definition}[Double Restricted Hidden Matching problem ($\mathsf{DRHM}_n$]
    Let $n$ be an even positive integer. Take two independent instances from the Restricted Hidden Matching Problem. Let $x_1 \in \{0, 1\}^n, M_1 \in \mathcal{M}_1$ be an input of the first instance and $x_2 \in \{0, 1\}^n, M_2 \in \mathcal{M}_2$ be an input of the second instance. Alice is given $x_1$ and $M_2$ and Bob is given $x_2$ and $M_1$. Their goal is to output two tuples $\langle i_1, j_1, b_1 \rangle$ and $\langle i_2, j_2, b_2 \rangle$ to solve the two Restricted Hidden Matching Problem.
\end{definition}

\begin{proposition}\label{prop:separation}
    $\Q^{||,\mathrm{LOCC}}(\mathsf{DRHM}_n) = O(\log n)$, $\R^{||} (\mathsf{DRHM}_n) = \R\Q^{||} (\mathsf{DRHM}_n) = \Q\R^{||} (\mathsf{DRHM}_n) = \Q^{||,\mathsf{LOCC}_1^{B \rightarrow A}}(\mathsf{DRHM}_n) = \Q^{||,\mathsf{LOCC}_1^{A \rightarrow B}}(\mathsf{DRHM}_n) =  \Theta(\sqrt{n})$.
\end{proposition}

\begin{proof}
    Let us first show $\Q^{||,\mathrm{LOCC}}(\mathsf{DRHM}_n) = O(\log n)$. Alice sends $\rm{Ref}_A$ $\frac{1}{\sqrt{n}} \sum_{i=1}^n (-1)^{x_{1}(i)} \ket{i}$ and $\ket{M_2}$ where $M_2$ is encoded in the computational basis. Bob sends $\rm{Ref}_B$ $\frac{1}{\sqrt{n}} \sum_{i=1}^n (-1)^{x_{2}(i)} \ket{i}$ and $\ket{M_1}$ where $M_1$ is encoded in the computational basis. Then, $\rm{Ref}_A$ measures $\ket{M_2}$ and sends $M_2$ to $\rm{Ref}_B$. $\rm{Ref}_B$ measures $\ket{M_1}$ and sends $M_1$ to $\rm{Ref}_A$. Based on $M_1$, $\rm{Ref}_A$ measures $\frac{1}{\sqrt{n}} \sum_{i=1}^n (-1)^{x_{1}(i)} \ket{i}$. Based on $M_2$, $\rm{Ref}_B$ measures $\frac{1}{\sqrt{n}} \sum_{i=1}^n (-1)^{x_{2}(i)} \ket{i}$. Note that each matching can be represented with $O(\log n)$ bits because the size of the set of matchings $|\mathcal{M}_1|$ and $|\mathcal{M}_2|$ is $\Theta(n)$.
    
    Let us next show $\R\Q^{||} (\mathsf{DRHM}_n) = \Theta(\sqrt{n})$. When we regard Bob and the referee as one party and allow arbitrary communication between them, the communication model is the same as the classical one-way communication model. The first and second instances are independently chosen, and to solve the first instance, $\Theta(\sqrt{n})$ bits communication is required in the model from \cref{lem:rhm}. Therefore, in the SMP model as a weaker model than the classical one-way communication model, $\Theta(\sqrt{n})$ bits communication from Alice to the referee is required. By a similar discussion, it is shown that $\Q\R^{||} (\mathsf{DRHM}_n) = \Theta(\sqrt{n})$.

    Finally, from \cref{thm:one-way_LOCC}, we have $\Q^{||,\mathsf{LOCC}_1^{B \rightarrow A}}(\mathsf{DRHM}_n) = \Q^{||,\mathsf{LOCC}_1^{A \rightarrow B}}(\mathsf{DRHM}_n) = \Theta(\sqrt{n})$.
\end{proof}

\begin{remark}\label{remark}
    In the quantum two-way-LOCC protocol, we can allow $\rm{Ref}_A$ and $\rm{Ref}_B$ to do several 2-value measurements in sequence (by making the other one do nothing in the turns). For $\mathsf{DRHM}_n$, we can solve it by $2 \log_2 n + O(\log n)$ rounds 2-value two-way-LOCC measurements. Let us describe the protocol. First, $\rm{Ref}_A$ measures $\ket{M_2}$ in the computational basis, which takes $O(\log n)$ rounds. Similarly, $\rm{Ref}_B$ measures $\ket{M_1}$ that takes $O(\log n)$ rounds. Then, based on $M_1$,  $\rm{Ref}_A$ puts half of ${(i,j) \in M_1}$ into one group and the other half into another, and measures $\frac{1}{\sqrt{n}} \sum_{i=1}^n (-1)^{x_{1}(i)} \ket{i}$ by a projector constructed from the grouping. Based on the measurement outcome, $\rm{Ref}_A$ repeats another 2-value measurement with a similar half/half grouping. $\rm{Ref}_B$ conducts a similar sequence of 2-value measurements on $\frac{1}{\sqrt{n}} \sum_{i=1}^n (-1)^{x_{2}(i)} \ket{i}$ based on $M_2$.

    However, the number of rounds is larger than the bound obtained from \cref{thm:two-way-LOCC_general_relation}, and thus there is no contradiction. Moreover, this implies that our analysis is tight up to some constant factor. Therefore, to prove a stronger lower bound toward \cref{conj}, we should develop another proof strategy which might be inherent to $\EQ_n$.
\end{remark}
