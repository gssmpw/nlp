\documentclass[11pt]{article}
\usepackage{fullpage}
\usepackage[utf8]{inputenc}
\usepackage{braket}
\usepackage{amsthm}
\usepackage{amsmath, amssymb}
\usepackage[pdftex]{graphicx}
\usepackage{xcolor}
\usepackage{algorithm}
\usepackage{mathtools}
\usepackage{algpseudocode}
\usepackage{latexsym}
\usepackage{optidef}

\usepackage[pagebackref]{hyperref}
\hypersetup{
    linktocpage,
    colorlinks=true,
    citecolor=blue,
    linkcolor=magenta,
    urlcolor=orange,
}

\newcommand{\BQP}{\mathsf{BQP}}
\newcommand{\NP}{\mathsf{NP}}
\newcommand{\MA}{\mathsf{MA}}
\newcommand{\QMA}{\mathsf{QMA}}
\newcommand{\Q}{\mathsf{Q}}
\newcommand{\C}{\mathsf{C}}
\newcommand{\D}{\mathsf{D}}
\newcommand{\R}{\mathsf{R}}
\newcommand{\EQ}{\mathsf{EQ}}
\newcommand{\poly}{\mathrm{poly}}
\newcommand{\GT}{\mathsf{GT}}
\newcommand{\MAX}{\mathsf{MAX}}
\newcommand{\tr}{\mathrm{tr}}
\newcommand{\TV}{\mathrm{TV}}
\newcommand{\HS}{\mathrm{HS}}
\newcommand{\rank}{\mathrm{rank}}
\newcommand{\U}{\mathbf{U}}
\newcommand{\M}{\mathbf{M}}
\newcommand{\Z}{\mathbf{Z}}
\newcommand{\E}{\mathbb{E}}
\newcommand{\I}{\mathrm{I}}
\newcommand{\Ent}{\mathrm{H}}
\newcommand{\Min}[2]{#1 \wedge #2}
\newcommand{\ceil}[1]{\left\lceil #1 \right\rceil}

\newcommand{\email}[1]{\href{mailto:#1}{\texttt{#1}}}

\usepackage[capitalise,nameinlink]{cleveref}
\Crefname{lemma}{Lemma}{Lemmas}
\Crefname{fact}{Fact}{Facts}
\Crefname{theorem}{Theorem}{Theorems}
\Crefname{corollary}{Corollary}{Corollaries}
\Crefname{claim}{Claim}{Claims}
\Crefname{example}{Example}{Examples}
\Crefname{problem}{Problem}{Problems}
\Crefname{definition}{Definition}{Definitions}
\Crefname{notation}{Notation}{Notations}
\Crefname{assumption}{Assumption}{Assumptions}
\Crefname{subsection}{Section}{Sections}
\Crefname{section}{Section}{Sections}
% \Crefname{equation}{Equation}{Equations}
\Crefformat{equation}{(#2#1#3)}
%\Crefname{figure}{Figure}{Figures}

\newtheorem{theorem}{Theorem}
\newtheorem{definition}{Definition}
\newtheorem{lemma}[theorem]{Lemma}
\newtheorem{claim}[theorem]{Claim}
\newtheorem{fact}{Fact}
\newtheorem{corollary}[theorem]{Corollary}
\newtheorem{conjecture}{Conjecture}
\newtheorem{proposition}[theorem]{Proposition}
\newtheorem{question}{Question}

\theoremstyle{remark}
\newtheorem{remark}{Remark}

\title{Does there exist a quantum fingerprinting protocol \\ without coherent measurements?}

\author{Atsuya Hasegawa\thanks{Graduate School of Mathematics, Nagoya University. \ Email: \email{atsuya.hasegawa@math.nagoya-u.ac.jp}} \and Srijita Kundu\thanks{Institute for Quantum Computing and Department of Combinatorics and Optimization, University of Waterloo. \ Email: \email{srijita.kundu@uwaterloo.ca}} \and Fran{\c{c}}ois Le Gall\thanks{Graduate School of Mathematics, Nagoya University. \ Email: \email{legall@math.nagoya-u.ac.jp}} \and Harumichi Nishimura\thanks{Graduate School of Informatics, Nagoya University. \ Email: \email{hnishimura@i.nagoya-u.ac.jp}} \and Qisheng Wang\thanks{School of Informatics, University of Edinburgh. \ Email: \email{QishengWang1994@gmail.com}}}

\date{}

\begin{document}

\maketitle

\begin{abstract}
Buhrman, Cleve, Watrous, and de Wolf (PRL 2001) discovered the quantum fingerprinting protocol, which is the quantum SMP protocol with $O(\log n)$ qubits communication for the equality problem. In the protocol, Alice and Bob create some quantum fingerprints of their inputs, and the referee conducts the SWAP tests for the quantum fingerprints. Since $\Omega(\sqrt{n})$ bits communication is required with the classical SMP scheme for the equality problem first shown by Newman and Szegedy (STOC 1996), there exists an exponential quantum advantage in the amount of communication.

In this paper, we consider a setting in which the referee can do only incoherent measurements rather than coherent measurements including the SWAP tests. We first show that, in the case of one-way LOCC measurements, $\Omega(\sqrt{n})$ qubits communication is required. To prove the result, we derive a new method to replace quantum messages by classical messages and consider a reduction to the optimal lower bound in the hybrid SMP model where one message is quantum and the other is classical, which was first shown by Klauck and Podder (MFCS 2014). Our method uses the result of Oszmaniec, Guerini, Wittek, and Ac{\'i}n (PRL 2017), who showed that general POVM measurements can be simulated by randomized projective measurements with small ancilla qubits, and Newman's theorem.

We further investigate the setting of quantum SMP protocols with two-way LOCC measurements, and derive a lower bound against some restricted two-way LOCC measurements. To prove it, we revisit the technique to replace quantum messages by classical deterministic messages introduced by Aaronson (ToC 2005) and generalized by Gavinsky, Regev, and de Wolf (CJTCS 2008), and show that, using the deterministic message, the referee can simulate the two-way LOCC measurements.
\end{abstract}

\clearpage

\tableofcontents

\section{Introduction}
\label{sec:introduction}
The business processes of organizations are experiencing ever-increasing complexity due to the large amount of data, high number of users, and high-tech devices involved \cite{martin2021pmopportunitieschallenges, beerepoot2023biggestbpmproblems}. This complexity may cause business processes to deviate from normal control flow due to unforeseen and disruptive anomalies \cite{adams2023proceddsriftdetection}. These control-flow anomalies manifest as unknown, skipped, and wrongly-ordered activities in the traces of event logs monitored from the execution of business processes \cite{ko2023adsystematicreview}. For the sake of clarity, let us consider an illustrative example of such anomalies. Figure \ref{FP_ANOMALIES} shows a so-called event log footprint, which captures the control flow relations of four activities of a hypothetical event log. In particular, this footprint captures the control-flow relations between activities \texttt{a}, \texttt{b}, \texttt{c} and \texttt{d}. These are the causal ($\rightarrow$) relation, concurrent ($\parallel$) relation, and other ($\#$) relations such as exclusivity or non-local dependency \cite{aalst2022pmhandbook}. In addition, on the right are six traces, of which five exhibit skipped, wrongly-ordered and unknown control-flow anomalies. For example, $\langle$\texttt{a b d}$\rangle$ has a skipped activity, which is \texttt{c}. Because of this skipped activity, the control-flow relation \texttt{b}$\,\#\,$\texttt{d} is violated, since \texttt{d} directly follows \texttt{b} in the anomalous trace.
\begin{figure}[!t]
\centering
\includegraphics[width=0.9\columnwidth]{images/FP_ANOMALIES.png}
\caption{An example event log footprint with six traces, of which five exhibit control-flow anomalies.}
\label{FP_ANOMALIES}
\end{figure}

\subsection{Control-flow anomaly detection}
Control-flow anomaly detection techniques aim to characterize the normal control flow from event logs and verify whether these deviations occur in new event logs \cite{ko2023adsystematicreview}. To develop control-flow anomaly detection techniques, \revision{process mining} has seen widespread adoption owing to process discovery and \revision{conformance checking}. On the one hand, process discovery is a set of algorithms that encode control-flow relations as a set of model elements and constraints according to a given modeling formalism \cite{aalst2022pmhandbook}; hereafter, we refer to the Petri net, a widespread modeling formalism. On the other hand, \revision{conformance checking} is an explainable set of algorithms that allows linking any deviations with the reference Petri net and providing the fitness measure, namely a measure of how much the Petri net fits the new event log \cite{aalst2022pmhandbook}. Many control-flow anomaly detection techniques based on \revision{conformance checking} (hereafter, \revision{conformance checking}-based techniques) use the fitness measure to determine whether an event log is anomalous \cite{bezerra2009pmad, bezerra2013adlogspais, myers2018icsadpm, pecchia2020applicationfailuresanalysispm}. 

The scientific literature also includes many \revision{conformance checking}-independent techniques for control-flow anomaly detection that combine specific types of trace encodings with machine/deep learning \cite{ko2023adsystematicreview, tavares2023pmtraceencoding}. Whereas these techniques are very effective, their explainability is challenging due to both the type of trace encoding employed and the machine/deep learning model used \cite{rawal2022trustworthyaiadvances,li2023explainablead}. Hence, in the following, we focus on the shortcomings of \revision{conformance checking}-based techniques to investigate whether it is possible to support the development of competitive control-flow anomaly detection techniques while maintaining the explainable nature of \revision{conformance checking}.
\begin{figure}[!t]
\centering
\includegraphics[width=\columnwidth]{images/HIGH_LEVEL_VIEW.png}
\caption{A high-level view of the proposed framework for combining \revision{process mining}-based feature extraction with dimensionality reduction for control-flow anomaly detection.}
\label{HIGH_LEVEL_VIEW}
\end{figure}

\subsection{Shortcomings of \revision{conformance checking}-based techniques}
Unfortunately, the detection effectiveness of \revision{conformance checking}-based techniques is affected by noisy data and low-quality Petri nets, which may be due to human errors in the modeling process or representational bias of process discovery algorithms \cite{bezerra2013adlogspais, pecchia2020applicationfailuresanalysispm, aalst2016pm}. Specifically, on the one hand, noisy data may introduce infrequent and deceptive control-flow relations that may result in inconsistent fitness measures, whereas, on the other hand, checking event logs against a low-quality Petri net could lead to an unreliable distribution of fitness measures. Nonetheless, such Petri nets can still be used as references to obtain insightful information for \revision{process mining}-based feature extraction, supporting the development of competitive and explainable \revision{conformance checking}-based techniques for control-flow anomaly detection despite the problems above. For example, a few works outline that token-based \revision{conformance checking} can be used for \revision{process mining}-based feature extraction to build tabular data and develop effective \revision{conformance checking}-based techniques for control-flow anomaly detection \cite{singh2022lapmsh, debenedictis2023dtadiiot}. However, to the best of our knowledge, the scientific literature lacks a structured proposal for \revision{process mining}-based feature extraction using the state-of-the-art \revision{conformance checking} variant, namely alignment-based \revision{conformance checking}.

\subsection{Contributions}
We propose a novel \revision{process mining}-based feature extraction approach with alignment-based \revision{conformance checking}. This variant aligns the deviating control flow with a reference Petri net; the resulting alignment can be inspected to extract additional statistics such as the number of times a given activity caused mismatches \cite{aalst2022pmhandbook}. We integrate this approach into a flexible and explainable framework for developing techniques for control-flow anomaly detection. The framework combines \revision{process mining}-based feature extraction and dimensionality reduction to handle high-dimensional feature sets, achieve detection effectiveness, and support explainability. Notably, in addition to our proposed \revision{process mining}-based feature extraction approach, the framework allows employing other approaches, enabling a fair comparison of multiple \revision{conformance checking}-based and \revision{conformance checking}-independent techniques for control-flow anomaly detection. Figure \ref{HIGH_LEVEL_VIEW} shows a high-level view of the framework. Business processes are monitored, and event logs obtained from the database of information systems. Subsequently, \revision{process mining}-based feature extraction is applied to these event logs and tabular data input to dimensionality reduction to identify control-flow anomalies. We apply several \revision{conformance checking}-based and \revision{conformance checking}-independent framework techniques to publicly available datasets, simulated data of a case study from railways, and real-world data of a case study from healthcare. We show that the framework techniques implementing our approach outperform the baseline \revision{conformance checking}-based techniques while maintaining the explainable nature of \revision{conformance checking}.

In summary, the contributions of this paper are as follows.
\begin{itemize}
    \item{
        A novel \revision{process mining}-based feature extraction approach to support the development of competitive and explainable \revision{conformance checking}-based techniques for control-flow anomaly detection.
    }
    \item{
        A flexible and explainable framework for developing techniques for control-flow anomaly detection using \revision{process mining}-based feature extraction and dimensionality reduction.
    }
    \item{
        Application to synthetic and real-world datasets of several \revision{conformance checking}-based and \revision{conformance checking}-independent framework techniques, evaluating their detection effectiveness and explainability.
    }
\end{itemize}

The rest of the paper is organized as follows.
\begin{itemize}
    \item Section \ref{sec:related_work} reviews the existing techniques for control-flow anomaly detection, categorizing them into \revision{conformance checking}-based and \revision{conformance checking}-independent techniques.
    \item Section \ref{sec:abccfe} provides the preliminaries of \revision{process mining} to establish the notation used throughout the paper, and delves into the details of the proposed \revision{process mining}-based feature extraction approach with alignment-based \revision{conformance checking}.
    \item Section \ref{sec:framework} describes the framework for developing \revision{conformance checking}-based and \revision{conformance checking}-independent techniques for control-flow anomaly detection that combine \revision{process mining}-based feature extraction and dimensionality reduction.
    \item Section \ref{sec:evaluation} presents the experiments conducted with multiple framework and baseline techniques using data from publicly available datasets and case studies.
    \item Section \ref{sec:conclusions} draws the conclusions and presents future work.
\end{itemize}
% Consider a lasso optimization procedure with potentially distinct regularization penalties:
% \begin{align}
%     \hat{\beta} = \arg\min_{\beta}\{\|y-X\beta\|^2_2+\sum_{i=1}^{N}\lambda_i|\beta_i|\}.
% \end{align}
\subsection{Supervised Data-Driven Learning}\label{subsec:supervised}
We consider a generic data-driven supervised learning procedure. Given a dataset \( \mathcal{D} \) consisting of \( n \) data points \( (x_i, y_i) \in \mathcal{X} \times \mathcal{Y} \) drawn from an underlying distribution \( p(\cdot|\theta) \), our goal is to estimate parameters \( \theta \in \Theta \) through a learning procedure, defined as \( f: (\mathcal{X} \times \mathcal{Y})^n \rightarrow \Theta \) 
that minimizes the predictive error on observed data. 
Specifically, the learning objective is defined as follows:
\begin{align}
\hat{\theta}_f := f(\mathcal{D}) = \arg\min_{\theta} \mathcal{L}(\theta, \mathcal{D}),
\end{align}
where \( \mathcal{L}(\cdot,\mathcal{D}) := \sum_{i=1}^{n} \mathcal{L}(\cdot, (x_i, y_i))\), and $\mathcal{L}$ is a loss function quantifying the error between predictions and true outcomes. 
Here, $\hat{\theta}_f$ is the parameter that best explains the observed data pairs \( (x_i, y_i) \) according to the chosen loss function \( \mathcal{L} (\cdot) \).

\paragraph{Feature Selection.}
Feature selection aims to improve model \( f \)'s predictive performance while minimizing redundancy. 
%Formally, given data \( X \), response \( y \), feature set \( \mathcal{F} \), loss function \( \mathcal{L}(\cdot) \), and a feature limit \( k \), the objective is:
% \begin{align}
% \mathcal{S}^* = \arg \min_{\mathcal{S} \subseteq \mathcal{F}, |\mathcal{S}| \leq k} \mathcal{L}(y, f(X_\mathcal{S})) + \lambda R(\mathcal{S}),
% \end{align}
% where \( X_\mathcal{S} \) is the submatrix of \( X \) for selected features \( \mathcal{S} \), \( \lambda \) is a regularization parameter, and \( R(\mathcal{S}) \) penalizes feature redundancy.
 State-of-the-art techniques fall into four categories: (i) filter methods, which rank features based on statistical properties like Fisher score \citep{duda2001pattern,song2012feature}; (ii) wrapper methods, which evaluate model performance on different feature subsets \citep{kohavi1997wrappers}; (iii) embedded methods, which integrate feature selection into the learning process using techniques like regularization \citep{tibshirani1996LASSO,lemhadri2021lassonet}; and (iv) hybrid methods, which combine elements of (i)-(iii) \citep{SINGH2021104396,li2022micq}. This paper focuses on embedded methods via Lasso, benchmarking against approaches from (i)-(iii).

\subsection{Language Modeling}
% The objective of language modeling is to learn a probability distribution \( p_{LM}(x) \) over sequences of text \( x = (X_1, \ldots, X_{|x|}) \), such that \( p_{LM}(x) \approx p_{text}(x) \), where \( p_{text}(x) \) represents the true distribution of natural language. This process involves estimating the likelihood of token sequences across variable lengths and diverse linguistic structures.
% Modern large language models (LLMs) are trained on vast datasets spanning encyclopedias, news, social media, books, and scientific papers \cite{gao2020pile}. This broad training enables them to generalize across domains, learn contextual knowledge, and perform zero-shot learning—tackling new tasks using only task descriptions without fine-tuning \cite{brown2020gpt3}.
Language modeling aims to approximate the true distribution of natural language \( p_{\text{text}}(x) \) by learning \( p_{\text{LM}}(x) \), a probability distribution over text sequences \( x = (X_1, \ldots, X_{|x|}) \). Modern large language models, trained on diverse datasets \citep{gao2020pile}, exhibit strong generalization across domains, acquire contextual knowledge, and perform zero-shot learning—solving new tasks using only task descriptions—or few-shot learning by leveraging a small number of demonstrations \citep{brown2020gpt3}.
\paragraph{Retrieval-Augmented Generation (RAG).} Retrieval-Augmented Generation (RAG) enhances the performance of generative language models by  integrating a domain-specific information retrieval process  \citep{lewis2020retrieval}. The RAG framework comprises two main components: \textit{retrieval}, which extracts relevant information from external knowledge sources, and \textit{generation}, where an LLM generates context-aware responses using the prompt combined with the retrieved context. Documents are indexed through various databases, such as relational, graph, or vector databases \citep{khattab2020colbert, douze2024faiss, peng2024graphretrievalaugmentedgenerationsurvey}, enabling efficient organization and retrieval via algorithms like semantic similarity search to match the prompt with relevant documents in the knowledge base. RAG has gained much traction recently due to its demonstrated ability to reduce incidence of hallucinations and boost LLMs' reliability as well as performance \citep{huang2023hallucination, zhang2023merging}. 
 
% image source: https://medium.com/@bindurani_22/retrieval-augmented-generation-815c1ae438d8
\begin{figure}
    \centering
\includegraphics[width=1.03\linewidth]{fig/fig1.pdf}
\vspace{-0.6cm}
\scriptsize 
    \caption{Retrieval Augmented Generation (RAG) based $\ell_1$-norm weights (penalty factors) for Lasso. Only feature names---no training data--- are included in LLM prompt.} 
    \label{fig:rag}
\end{figure}
% However, for the RAG model to be effective given the input token constraints of the LLM model used, we need to effectively process the retrieval documents through a procedure known as \textit{chunking}.

\subsection{Task-Specific Data-Driven Learning}
LLM-Lasso aims to bridge the gap between data-driven supervised learning and the predictive capabilities of LLMs trained on rich metadata. This fusion not only enhances traditional data-driven methods by incorporating key task-relevant contextual information often overlooked by such models, but can also be especially valuable in low-data regimes, where the learning algorithm $f:\mathcal{D}\rightarrow\Theta$ (seen as a map from datasets $\mathcal{D}$ to the space of decisions $\Theta$) is susceptible to overfitting.

The task-specific data-driven learning model $\tilde{f}:\mathcal{D}\times\mathcal{D}_\text{meta}\rightarrow\Theta$ can be described as a metadata-augmented version of $f$, where a link function $h(\cdot)$ integrates metadata (i.e. $\mathcal{D}_\text{meta}$) to refine the original learning process. This can be expressed as:
\[
\tilde{f}(\mathcal{D}, \mathcal{D}_\text{meta}) := \mathcal{T}(f(\mathcal{D}),  h(\mathcal{D}_{\text{meta}})),
\]
where the functional $\mathcal{T}$ takes the original learning algorithm $f(\mathcal{D})$ and transforms it into a task-specific learning algorithm $\tilde{f}(\mathcal{D}, \mathcal{D}_\text{meta})$ by incorporating the metadata $\mathcal{D}_\text{meta}$. 
% In particular, the link function $h(\mathcal{D}_{\text{meta}})$ provides a structured mechanism summarizing the contextual knowledge.

There are multiple approaches to formulate $\mathcal{T}$ and $h$.
%to ``inform" the data-driven model $f$ of %meta knowledge. 
For instance, LMPriors \citep{choi2022lmpriorspretrainedlanguagemodels} designed $h$ and $\mathcal{T}$ such that $h(\mathcal{D}_{\text{meta}})$ first specifies which features to retain (based on a probabilistic prior framework), and then $\mathcal{T}$ keeps the selected features and removes all the others from the original learning objective of $f$. 
Note that this approach inherently is restricted as it selects important features solely based on $\mathcal{D}_\text{meta}$ without seeing $\mathcal{D}$.

In contrast, we directly embed task-specific knowledge into the optimization landscape through regularization by introducing a structured inductive bias. This bias guides the learning process toward solutions that are consistent with metadata-informed insights, without relying on explicit probabilistic modeling. Abstractly, this can be expressed as:
\begin{align}
    \!\!\!\!\!\hat{\theta}_{\tilde{f}} := \tilde{f}(\mathcal{D},\mathcal{D}
    _\text{meta})= \arg\min_{\theta} \mathcal{L}(\theta, \mathcal{D}) + \lambda R(\theta, \mathcal{D}_{\text{meta}}),
\end{align}
where \( \lambda \) is a regularization parameter, \( R(\cdot) \) is a regularizer, and $\theta$ is the prediction parameter.
%We explain our framework with more details in the following section.


% Our research diverges from both aforementioned approaches by positioning the LLM not as a standalone feature selector but as an enhancement to data-driven models through an embedded feature selection method, L-LASSO. L-LASSO incorporates domain expertise—auxiliary natural language metadata about the task—via the LLM-informed LASSO penalty, which is then used in statistical models to enhance predictive performance. This method integrates the rich, context-sensitive insights of LLMs with the rigor and transparency of statistical modeling, bridging the gap between data-driven and knowledge-driven feature selection approaches. To approach this task, we need to tackle two key components: (i). train an LLM that is expert in the task-specific knowledge; (ii). inform data-driven feature selector LASSO with LLM knowledge.

% In practice, this involves combining techniques like prompt engineering and data engineering to develop an effective framework for integrating metadata into existing data-driven models. We will go through this in detail in Section \ref{mthd} and \ref{experiment}.


\section{Lower bound against one-way LOCC measurement}\label{sec:one-way_locc}

\subsection{Proof}\label{subsec:proof_1waylocc}

We first show that a variant of Newman's Theorem holds in our setting. We assume that in the hybrid SMP model, Alice sends a classical message and Bob sends a quantum message.
Let us denote by $\R\Q^{||,pub}(F)$ and $\R\Q^{||,pub}(R)$ the complexity in the hybrid SMP model of a function $F$ and a relation $R$, respectively, when Alice and the referee share unlimited randomness.

\begin{lemma}\label{lem:newman}
    Let $R \subseteq X \times Y \times Z$ be any relational problem. For every $\epsilon>0$ and $\delta>0$, $\R\Q_{\epsilon+\delta}^{||}(R) \leq \R\Q_\epsilon^{||,pub} (R) + O(\log (\log |X| + \log |Y|) + \log (\frac{1}{\delta}))$.
\end{lemma}

\begin{proof}
    We will show that any hybrid SMP protocol $\mathcal{P}$ with unlimited random bits between Alice and the referee can be transformed into another hybrid SMP protocol $\mathcal{P}'$ in which Alice and the referee share only $O(\log n + \log (\frac{1}{\delta}))$ random bits while increasing the error by only $\delta$. Since the amount of randomness is small, by making Alice send all the random bits to the referee, we have the desired protocol. Let $\Pi$ be the probabilistic distribution of the shared randomness of the protocol $\mathcal{P}$.
    % The proof then follows because Alice can send her private random bits to Bob and then the two players proceed as in $\mathcal{P}'$.

    Let $V(x, y, r)$ be a random variable which is defined as the probability that $\mathcal{P}$’s output $z \in Z$ on input $(x, y) \in X \times Y$ and random string $r$ shared by Alice and the referee satisfy $(x, y, z) \notin R$. Because $\mathcal{P}$ computes $R$ with $\epsilon$ error, we have $\mathbb{E}_{r \in \Pi} [V(x,y,r)] \leq \epsilon$ for all $(x,y)$. We will build a new protocol that uses fewer random bits, using the probabilistic method. Let $t$ be a parameter, and let $r_1,\ldots,r_t$ be $t$ strings. For such strings, let us define a protocol $\mathcal{P}_{r_1,\ldots,r_t}$ as follows: Alice and the referee choose $1 \leq i \leq t$ uniformly at random and then proceed as in $\mathcal{P}$ with $r_i$ as their common random string. We now show that there exist strings $r_1,\ldots,r_t$ such that $\frac{1}{t} \sum_{i=1}^t [V(x,y,r_i)] \leq \epsilon + \delta$ for all $(x,y)$. For this choice of strings, the protocol $\mathcal{P}_{r_1,\ldots,r_t}$ is the desired protocol.

    To do so, we choose the $t$ values $r_1,\ldots,r_t$ by sampling the distribution $\Pi$ $t$ times. Consider a particular input pair $(x,y)$ and compute the probability that $\frac{1}{t} \sum_{i=1}^t V(x,y,r_i) > \epsilon + \delta$. By Hoeffding's bound (\cref{fact:hoeffding}), since $\mathbb{E}_{r \in \Pi}[V(x,y,r)] \leq \epsilon$, we get
    \[
        \mathrm{Pr} \left[ \left( \frac{1}{t}  \sum_{i=1}^t V(x,y,r_i) - \epsilon \right) > \delta \right] \leq 2e^{-2\delta^2 t}.
    \]
    By choosing $t=O(\frac{ \log |X| + \log |Y| }{\delta^2})$, this is smaller than $|X|^{-1} |Y|^{-1}$. Thus, for a random choice of $r_1,\ldots,r_t$, the probability that for some input $(x,y)$, $\frac{1}{t} \sum_{i=1}^t [V(x,y,r_i)] > \epsilon + \delta$ is smaller than $|X||Y| |X|^{-1} |Y|^{-1} = 1$. This implies that there exists a choice of $r_1,\ldots,r_t$ where for every $(x,y)$ the error of the protocol $\mathcal{P}_{r_1,\ldots,r_t}$ is at most $\epsilon + \delta$. Finally, note that the number of random bits used by the protocol $\mathcal{P}_{r_1,\ldots,r_t}$ is $\log t = O(\log (\log |X| + \log |Y|) + \log (\frac{1}{\delta}))$.
\end{proof}

We also need the result by Oszmaniec, Guerini, Wittek, and Ac{\'i}n \cite{OGWA17}, who showed that any POVM can be simulated by projective measurements with randomization and postprocessing.

\begin{lemma}[Theorem 1 in \cite{OGWA17}]\label{lem:OGWA17}
    Let $\mathbf{S}\mathbb{P}(d^2,nd)$ be the set of PM simulable, $n$-outcome POVMs on $\mathcal{D}(\mathbb{C}^d \otimes \mathbb{C}^d)$. Let $\mathbf{M} \in \mathcal{P}(d,n)$ be an arbitrary $n$-outcome POVM on $\mathcal{D}(\mathbb{C}^d)$ and $\ket{\phi}$ be some fixed pure state on $\mathcal{B}(\mathbb{C}^d)$. Then there exists a PM-simulable POVM $\mathbf{N} \in \mathbf{S}\mathbb{P}(d^2,nd)$ such that $\mathrm{tr}(\rho M_i) = \mathrm{tr}((\rho \otimes \ket{\phi} \bra{\phi}) N_i)$ for $i=1,\ldots,n$ \footnote{Since the ancilla state is fixed, $\mathrm{tr}((\rho \otimes \ket{\phi} \bra{\phi}) N_i)=0$ for $i=n+1,\ldots,nd$. The number of the outcomes of the POVM $\mathbf{N}$ is $nd$ to make $\mathbf{N}$ valid (i.e., $\sum_{i=1}^{nd} N_i = I$).} and for all states $\rho \in \mathcal{D}(\mathbb{C}^d)$.
\end{lemma}

We are now ready to prove the main result of this section.

\begin{theorem}\label{thm:one-way_LOCC}
    Let $R \subseteq X \times Y \times Z$ be any relation. Suppose that there exists a quantum one-way-LOCC SMP protocol $\mathcal{P}$ for $R$ whose message sizes are $a$ and $b$, with $a$ being the size of Alice's message, on which the first measurement is performed by $\rm{Ref}_A$. Then, $\R\Q(R) \leq 2a + b + O(\log (\log |X| + \log |Y|))$.
\end{theorem}

\begin{proof}
    We give a new hybrid SMP protocol $\mathcal{P}'$ for $R$ with shared randomness between Alice and the referee constructed from the protocol $\mathcal{P}$. Let $\mathbf{M}$ be the $m$-outcome POVM performed by $\rm{Ref}_A$ and let $\rho_x$ be a quantum message from Alice to $\rm{Ref}_A$ in the protocol $\mathcal{P}$. From \cref{lem:OGWA17}, there exist a fixed pure state $\ket{\phi} \in \mathcal{B}(\mathbb{C}^{2^a})$ and a PM-simulable POVM $\mathbf{N} \in \mathbf{S}\mathbb{P}(2^{2a},m 2^a)$ such that $\mathrm{tr}(\rho_x M_i) = \mathrm{tr}((\rho_x \otimes \ket{\phi} \bra{\phi}) N_i)$ for $i=1,\ldots,m$. Since $\mathbf{N}$ is a PM-simulable POVM, there exist a set of projectors $\{\mathbf{N}_k\}$ and a probability distribution $\{p_k\}_k$ such that $\mathbf{N} = \sum_k p_k \mathbf{N}_k$. The measurement results of each $\mathbf{N}_k$ on $\rho_x \otimes \ket{\phi} \bra{\phi}$ are represented by $2a$ bits. Consider a hybrid SMP protocol $\mathcal{P}'$ in which Alice and the referee share the probability distribution $\{p_k\}_k$ and Alice performs $\sum_k p_k \mathbf{N}_k$ on $\rho_x \otimes \ket{\phi}\bra{\phi}$ and sends the measurement result of $2a$ bits. Then, the referee can recover the original result of $\mathbf{M}$ from the $2a$ bits message from Alice because the referee knows the index $k$ from the shared randomness with Alice, and thus simulates $\rm{Ref}_B$ exactly.
    
    Let us apply \cref{lem:newman} for the protocol $\mathcal{P}'$ by taking $\delta$ as a sufficiently small constant. We then obtain a private-coin hybrid SMP protocol $\mathcal{P}''$ to solve $R$ with bounded error and the complexity of the protocol $\mathcal{P}''$ is $2a + O(\log (\log |X| + \log |Y|)) + b$.
\end{proof}

Since for a Boolean function $F:\{0,1\}^n \times \{0,1\}^n \rightarrow \{0,1\}$, $|X|=|Y| = 2^n$, from the lower bound in the hybrid scheme for $\EQ_n$ (\cref{thm:lower_bound_hybrid}), we have a lower bound in quantum one-way-LOCC SMP protocols for $\EQ_n$.

\begin{corollary}\label{cor:EQ}
    $\Q^{||,\mathrm{LOCC}_1}(\EQ_n) = \Omega(\sqrt{n})$.
\end{corollary}

Since $\mathrm{BELL} \subseteq \mathrm{LOCC_1}$, we also have a lower bound with the BELL measurements.

\begin{corollary}
    $\Q^{||,\mathrm{BELL}}(\EQ_n) = \Omega(\sqrt{n})$.
\end{corollary}

Moreover, we can obtain a more general result for any Boolean function. For any Boolean function $F:\{0,1\}^n \times \{0,1\}^n \rightarrow \{0,1\}$, it is known that $\R^{||}(F) \leq \D^{||}(F)^2$ \cite{BK97}, and $\R\Q^{||}(F) \leq \R^{||}(F)^2$ \cite{GRdW08}. Combining this with \cref{thm:one-way_LOCC}, we have the following claim.

\begin{corollary}\label{cor:quartic}
    For any Boolean function $F:\{0,1\}^n \times \{0,1\}^n \rightarrow \{0,1\}$, if $\D^{||}(F) = \Omega(n^c)$ for a constant $c > 0$, $\Q^{||,\mathrm{LOCC_1}} (F) = \Omega(n^{\frac{c}{4}})$.
\end{corollary}

\subsection{Implication for quantum incoherent one-way communication complexity}

By applying the technique developed in \cref{subsec:proof_1waylocc}, we show that, in the setting of the quantum incoherent one-way communication protocols, quantum messages can be replaced by classical messages with very small overhead.
\begin{corollary}\label{cor:incoherent}
    Let $R \subseteq X \times Y \times Z$ be any relation. $\R^1(R) \leq 2 \Q^{1,\perp}(R) + O(\log (\log |X| + \log |Y|)$.
\end{corollary}
\begin{proof}
    Let $\mathcal{P}$ be an incoherent quantum one-way communication protocol for $R$. Let $a$ be the qubit size from Alice to Bob. Let $\mathbf{M} = \{M_i\}_{i \in [m]}$ be the $m$-outcome POVM performed by Bob and let $\rho_x$ be a quantum message from Alice to Bob in the protocol $\mathcal{P}$. From \cref{lem:OGWA17}, there exist a fixed pure state $\ket{\phi} \in \mathcal{B}(\mathbb{C}^{2^a})$ and a PM-simulable POVM $\mathbf{N} \in \mathbf{S}\mathbb{P}(2^2a,m2^a)$ such that $\mathrm{tr}(\rho_x M_i) = \mathrm{tr}((\rho_x \otimes \ket{\phi} \bra{\phi}) N_i)$ for $i=1,\ldots,m$. Since $\mathbf{N}$ is a PM-simulable POVM, there exist a set of projectors $\{\mathbf{N}_k\}$ and a probability distribution $\{p_k\}_k$ such that $\mathbf{N} = \sum_k p_k \mathbf{N}_k$. The measurement result of each $\mathbf{N}_k$ on $\rho_x \otimes \ket{\phi} \bra{\phi}$ is represented by $2a$ bits. Consider a classical one-way communication protocol $\mathcal{P}'$ in which Alice and Bob share the probability distribution $\{p_k\}_k$ and Alice performs $\sum_k p_k \mathbf{N}_k$ on $\rho_x \otimes \ket{\phi}\bra{\phi}$ and sends the measurement result with $2a$ bits. Then, Bob can recover the original result of $\mathbf{M}$ from the $2a$ bits message from Alice because Bob knows the index $k$ from the shared randomness with Alice.
    We then apply a variant of the Newman's theorem proven by the same way as \cref{lem:newman}, and we obtain a classical one-way communication protocol whose complexity is $2a + O(\log (\log |X| + \log |Y|))$, which concludes the proof.
\end{proof}
\section{Lower bound against two-way LOCC measurement}\label{sec:two-way_locc}

In \cref{subsec:replace}, we first describe how to replace quantum messages by deterministic messages. To illustrate our idea and analysis more clearly, we will give a proof of a lower bound in quantum two-value two-round LOCC SMP protocols as a first step in \cref{subsec:warm-up}. We then prove our lower bound in quantum two-value multiple-round LOCC SMP protocols in \cref{subsec:many_rounds}.

\subsection{Replacing quantum messages by classical messages}\label{subsec:replace}

\begin{lemma}\label{lem:replace}
    Suppose Alice has the classical description of an arbitrary $q$-qubit density matrix $\rho$, and Bob has $2^c$ $2$-value POVM operators $\{E_b\}_{b \in \{0,1\}^c}$. For any $\delta >0$, there exists a deterministic message of $O(\frac{q}{\delta^3} \log (\frac{q}{\delta}) (c+ \log \frac{1}{\delta}) )$ bits from Alice that enables Bob to output values $p'_b$ satisfying that $|p_b-p'_b| \leq \delta$ simultaneously for all $b \in \{0,1\}^c$ where $p_b = \mathrm{tr}(E_b \rho)$.
\end{lemma}

For completeness, we will give a proof of this theorem. See also Theorem 5 in \cite{GRdW08}, Theorem 4 in \cite{Aar18}, and Theorem 6 in \cite{ACH+18}. We will need the quantum union bound for the proof.

\begin{lemma}[Quantum union bound, see, e.g., Lemma 3.1 in \cite{Wil13}]\label{lem:quantum_union_bound}
    Suppose that for a quantum state $\rho$, we conduct a sequence of 2-value POVM operators $\{M_i\}_{i\in[k]}$ such that $\mathrm{tr}(M_i \rho) \geq 1 - \delta$. Then, the probability that all the measurements succeed is at least $1-2\sqrt{k\delta}$.
\end{lemma}

\begin{proof}[Proof of \cref{lem:replace}]
    Suppose that Alice sends $r$ many copies of her state.
    Let $\rho'=\rho^{\otimes r}$ be the state she sends, and $r q$ is the total number of qubits. Define the operator
    \[
        F_b=\frac{1}{r}\sum_{j=1}^r E_b^{(j)},
    \]
    where $E_b^{(j)}$ applies $E_b$ to the $j$th copy. This operator gives the fraction of successes if you separately measure each of the $r$ copies of $\rho$ with $E_b$. Hoeffding's bound (\cref{fact:hoeffding}) implies that the outcome $p_b'$ of this measurement applied to $\rho'$ will probably be close to its expectation $p_b=\mathrm{tr}(E_b\rho)$:
    \begin{equation}\label{eq:use_chernoff}
        \Pr[|p'_b-p_b|>\delta/4]\leq 2 e^{-\frac{\delta^2 r}{8}}.
    \end{equation}

    Let us show what is Alice's classical message. Consider all $b=1,\ldots,2^c$ in order. We will sequentially build $rq$-qubit density matrices $\rho_b$, one for each $E_b$.
    Alice's classical message will enable Bob to reconstruct this entire sequence. Let us say $b$ is \emph{good} if $|\mathrm{tr}(F_b \rho_b)-p_b| \leq \delta$, and say $b$ is \emph{bad} otherwise. Note that if Bob has a classical description of a good $\rho_b$, then he can approximate $p_b$ to within $\pm \delta$ (since he knows what $F_b$ is). We start with the completely mixed state: $\rho_1=\frac{I}{2^{rq}}$ and define the subsequent $\rho_b$ one by one, as follows. If $b$ is good, then define $\rho_{b+1}$ as equal to $\rho_b$. If $b$ is bad, Alice adds the pair $(b,\widetilde{p}_b)$ to her message, where $\widetilde{p}_b$ is the $\log(1/\delta)+O(1)$ most significant bits of $p_b$, and then $|\widetilde{p}_b-p_b| \ll \delta$. In this case, let $M_b$ be the projector on the subspace spanned by the eigenvectors of $F_b$ with eigenvalues in the interval $[\widetilde{p}_b-\delta/2,\widetilde{p}_b+\delta/2]$, and let $\rho_{b+1}$ be the renormalized projection of $\rho_b$ on this subspace. Note that we will later show $\mathrm{tr}(M_b \rho_b)$ is nonzero in \cref{eq:lower_bound} and thus this renormalized projection is well defined. Continuing all the way to $b=2^c$, we obtain a message $(b_1,\widetilde{p}_{b_1}),\ldots,(b_T,\widetilde{p}_{b_t})$ for some $t$. We need to show two things: (i) this message enables Bob to approximate all $p_b$ to within $\pm \delta$, and (ii) $t=O(rq)$, which implies that the message length is 
    \begin{equation}\label{eq:total_length}
        O\left(t \left( c+\log \frac{1}{\delta} \right) \right)
    \end{equation}    
    bits. We will show these two things in turn.

    First, we will show (i): the construction of the messages works. Note that Bob knows which $b\in[2^c]$ are bad, since those $b$ are exactly the ones in Alice's message. Bob can in fact compute the whole sequence $\rho_1,\ldots,\rho_{2^c}$ given the message: $\rho_1=\frac{I}{2^{rq}}$; if $b$ is good, then $\rho_{b+1}=\rho_b$; if $b$ is bad, then $(b,\widetilde{p}_b)$ is part of Alice's message and $\rho_{b+1}$ can be computed from this information. Suppose that Bob wants to approximate $p_b=\mathrm{tr}(E_b\rho)$. If $b$ is good then by definition $|\mathrm{tr}(F_b\rho_b)-p_b| \leq \delta$ and Bob can calculate $\mathrm{tr}(F_b\rho_b)$. If $b$ is bad, then the pair $(b,\widetilde{p}_b)$ is part of Alice's message, so Bob knows $p_b$ with sufficient precision. Hence Bob can approximate all $p_b$ up to $\pm \delta$, for all $b$ simultaneously.

    Second, we will show (ii): $t=O(rq)$. Define $\eta=1-\delta/4$ and assume $t \leq (\frac{q}{\delta})^{10}$ (this assumption will be justified later). We consider the sequence $b_1,\ldots,b_t$ of the first $t$ bad $b$'s.
    Let
    \[
        p=\mathrm{tr} \left( M_{b_t} \cdots M_{b_1} \frac{I}{2^{rq}} M_{b_1} \cdots M_{b_t} \right)
    \]
    be the probability that all $t$ measurements succeed if we start with the completely mixed state and sequentially measure $M_{b_1},\ldots,M_{b_t}$. 

    We will upper bound and lower bound $p$ and do the upper bound first. If we sequentially measure $M_{b_1},\ldots,M_{b_t}$, starting from the completely mixed state, and if all $t$ measurements succeed, then we exactly have the sequence of density matrices $\rho_{b_1}= \frac{I}{2^{rq}},\ldots,\rho_{b_t},\rho_{b_{t+1}}$. 
    We will show the claim that if $\rho_b$ is bad, then $\mathrm{tr}(M_b\rho_b)\leq \eta$. Let $X$ denote the random variable representing the outcome of measuring $\rho_b$ with the observable $F_b$. Note that $X$ takes values in $[0,1]$. Assume $\mathrm{tr}(M_b\rho_b)=\Pr[|X-\widetilde{p}_b|\leq\delta/2] > \eta$. Let us evaluate the value $\mathrm{tr}(F_b\rho_b) = \E[X]$. From the assumption, we have
    \[
        \Pr [ X \geq \widetilde{p}_b - \delta/2 ] \geq \Pr[|X-\widetilde{p}_b|\leq\delta/2] > \eta.
    \]
    Then, from Markov's inequality (\cref{fact:markov}), if $\widetilde{p}_b - \delta/2 >0$, we have
    \[
        \E[X] > \eta (\widetilde{p}_b - \delta/2).
    \]
    Otherwise, it is trivial that $\E[X] \geq 0 \geq \eta (\widetilde{p}_b - \delta/2)$. Since $X$ takes values in $[0,1]$, by a similar discussion, we have
    \[
        \E[X] \leq \eta (\widetilde{p}_b + \delta/2) + 1 - \eta.
    \]
    Since $|\widetilde{p}_b-p_b| \ll \delta$ and thus $\widetilde{p}_b \leq 1 + \delta/100$, $\mathrm{tr}(F_b\rho_b) = \E[X]$ must necessarily be in the range
    \[
        [\eta (\widetilde{p}_b - \delta/2), \eta (\widetilde{p}_b + \delta/2) + 1-\eta] \subseteq [\widetilde{p}_b - 3\delta/4, \widetilde{p}_b + 3\delta/4] \subseteq [p_b - \delta, p_b + \delta],
    \]
    and hence $\rho_b$ is good. Thus we have the claim by contraposition, and then the probability that all $t$ measurements succeed is 
    \begin{equation}\label{eq:upper_bound}
        p\leq \eta^t.
    \end{equation}

    Now we lower bound on $p$. Note that $M_b$ succeeds on $\rho'$ if and only if the outcome $p'_b$ of the observable $F_b$ is at most $\delta/2$ away from the number $\widetilde{p}_b$, which is the truncated version of $p_b=\mathrm{tr}(E_b\rho)$ (recall $|\widetilde{p}_b-p_b|\ll \delta$). Hence by \cref{eq:use_chernoff}, we have
    \begin{equation}\label{eq:nonzero}
        \mathrm{tr}(M_b\rho')=\Pr[|p'_b-\widetilde{p}_b|\leq\delta/2]\geq \Pr[|p'_b-p_b|\leq\delta/4]\geq 1-2 e^{-\frac{\delta^2 r}{8}}.
    \end{equation}
    This allows us to measure $\rho'$ with $M_b$ while disturbing the state by only an insignificant amount. If we measure each $M_b$, for the first $t$ bad $b$'s in sequence, starting from $\rho'$, then from \cref{lem:quantum_union_bound} with probability at least 
    \[
        1-2\sqrt{2t e^{-\frac{\delta^2 r}{8}}}
    \]
    all measurements will succeed. Set $r:=\frac{C}{\delta^2}\log\frac{q}{\delta}$ for a sufficiently large constant $C$. Then we have
    \[
        1-2\sqrt{2t e^{-\frac{\delta^2 r}{8}}} = 1-2\sqrt{t \frac{\delta^{10}}{400 q^{10}}} \geq \frac{1}{2},
    \]
    where we use the assumption $t \leq (\frac{q}{\delta})^{10}$. Moreover, the completely mixed state can be written as $\frac{I}{2^{rq}}=\frac{1}{2^{rq}}\rho'+(1-\frac{1}{2^{rq}})\rho''$ where $\rho''$ is orthogonal to $\rho'$. Hence, if we start from $\frac{I}{2^{rq}}$, then the probability of all measurements succeeding is 
    \begin{equation}\label{eq:lower_bound}
        p\geq \frac{1}{2^{rq+1}}.
    \end{equation}

    Combining the upper bound \cref{eq:upper_bound} and lower bound \cref{eq:lower_bound}, we have
    \[
        \frac{1}{2^{rq+1}} \leq \left(1-\frac{\delta}{4} \right)^t.
    \]
    Since $rq = O(\frac{q}{\delta^2} \log \frac{q}{\delta})$, we have
    \[
        t = O\left(\frac{q}{\delta^3} \log \frac{q}{\delta}\right).
    \]
    This bound also justifies the assumption $t \leq (\frac{q}{\delta})^{10}$. Therefore, from \cref{eq:total_length}, the total length of the deterministic message is
    \[
        O\left(\frac{q}{\delta^3} \log \left(\frac{q}{\delta}\right) \left(c+ \log \frac{1}{\delta} \right) \right)
    \]
    bits, as claimed.
\end{proof}

\subsection{Warm-up case: 2-round-LOCC SMP protocols}\label{subsec:warm-up}

\begin{proposition}\label{prop:warm-up}
    Assume that there exists a 2-value 2-round LOCC SMP protocol $\mathcal{P}$ to solve $\EQ_n$ with high probability. Then the number of qubits of Alice's message
    %$\rho_x$ 
    is $\Omega(n/\log n)$.
\end{proposition}

\begin{proof}
    Let $\{\rho_x\}_{x\in \{0,1\}^n}$ be a quantum encoding by Alice of her input $x$, and $\{\sigma_y\}_{y \in \{0,1\}^n}$ be a quantum encoding by Bob of his input $y$.
    Let $r$ be a number of rounds, and $a \in \{0,1\}^{r}$ be previous measurement results by $\rm{Ref}_A$, and $b \in \{0,1\}^{r}$ be previous measurement results by $\rm{Ref}_B$. Let $h$ be all previous measurement results $a_0b_0\cdots a_{r-1}b_{r-1}$. For $m \in \{0,1\}$, let $M_{m|h}$ be a measurement operator of the $(r+1)$th measurement by $\rm{Ref}_A$, for previous measurement results $h$. Without loss of generality, we assume that the first measurement is done by $\rm{Ref}_A$ for $\rho_x$, and let $M_0$ and $M_1$ be the first measurement operators. We also assume, without loss of generality, that the final measurement outcome by $\rm{Ref}_A$ is the output of the protocol because it depends on all the previous measurement outcomes.
    
    For each $r \in \mathbb{N}$, let us denote by $p^A_{m|h}$ the probability that, conditioned on all previous measurement outcomes $h = a_0 b_0\cdots a_{r-1} b_{r-1}$ which consists of measurement outcomes $a \in \{0,1\}^r$ by $\rm{Ref}_A$ and measurement outcomes $b \in \{0,1\}^r$ by $\rm{Ref}_B$, the $(r+1)$th measurement result by $\rm{Ref}_A$ is $m \in \{0,1\}$. Let us denote by $p^B_{m|h}$ the probability that, conditioned on all previous measurement outcomes $h = a_0 b_0\cdots a_{r-1} b_{r-1} a_r$ which consists of measurement outcomes $a \in \{0,1\}^{r+1}$ by $\rm{Ref}_A$ and measurement outcomes $b \in \{0,1\}^r$ by $\rm{Ref}_B$, the $(r+1)$th measurement result by $\rm{Ref}_A$ is $m \in \{0,1\}$. Let us abbreviate $p^A_{m|\emptyset}$ as $p^A_m$. 
     
    Let us also denote by $v_0 = \mathrm{tr}(M_0^\dagger M_0 \rho_x)$, $v_1 = \mathrm{tr}(M_1^\dagger M_1 \rho_x) = 1 - v_0$,
    \begin{align*}
    v_{1|00}=\mathrm{tr}(M^{ \dagger}_0 M_{1|00}^{\dagger} M_{1|00} M_0 \rho_x), \quad &  v_{1|01}=\mathrm{tr}(M_0^\dagger M_{1|01}^{\dagger} M_{1|01} M_0 \rho_x), \\
    v_{1|10}=\mathrm{tr}(M^{\dagger}_1 M_{1|10}^{\dagger} M_{1|10} M_1 \rho_x), \quad & v_{1|11}=\mathrm{tr}(M^{\dagger}_1 M_{1|11}^{\dagger} M_{1|11} M_1 \rho_x).
    \end{align*}
    By definition,
    \[
        p^A_0 = v_0,\ \ p^A_1 = v_1.
    \]
    For example, when the result of the first measurement by $\rm{Ref}_A$ is $1$ and that of the first measurement by $\rm{Ref}_B$ is $0$, the probability that the second measurement result by $\rm{Ref}_A$ is $1$ is 
    \begin{equation}\label{eq:fraction}
        p^A_{1|10} = \mathrm{tr}(M_{1|10}^\dagger M_{1|10} \frac{M_1 \rho_x M^\dagger_1}{\mathrm{tr}(M^\dagger_1 M_1 \rho_x)}) = \frac{\mathrm{tr}(M^\dagger_1 M_{1|10}^\dagger M_{1|10} M_1 \rho_x) }{\mathrm{tr}(M^\dagger_1 M_1 \rho_x)} = \frac{v_{1|10}}{v_1},
    \end{equation}
    where we use the cyclic property of the trace. By the same argument, we have 
     \begin{equation}
         p^A_{1|00} = \frac{v_{1|00}}{v_0},\ \ p^A_{1|01} = \frac{v_{1|01}}{v_0},\ \ p^A_{1|10} = \frac{v_{1|10}}{v_1},\ \ p^A_{1|11} = \frac{v_{1|11}}{v_1}.
     \end{equation}
    
    Since $ p^A_{0|00} +  p^A_{1|00} = 1$, we have
    \[
        \frac{v_{0|00}}{v_0} + \frac{v_{1|00}}{v_0} = 1,
    \]
    which implies 
    \[
         v_0 = v_{0|00} + v_{1|00}.
    \]
    By the same argument, we also have
    \[
         v_0 = v_{0|01} + v_{1|01},\ \ v_1 = v_{0|10} + v_{1|10},\ \ v_1 = v_{0|11} + v_{1|11}.
    \]
    
    Then, the probability that the original LOCC protocol accepts is
    \begin{align*}
        & p^A_{0} \cdot p_{0|0}^B \cdot p^A_{1|00} + p^A_{0} \cdot p_{1|0}^B \cdot p^A_{1|01} + p^A_{1} \cdot p_{0|1}^B \cdot p^A_{1|10} + p^A_{1} \cdot p_{1|1}^B \cdot p^A_{1|11} \\
        &= v_{0} \cdot p_{0|0}^B \cdot \frac{v_{1|00}}{v_0} + v_{0} \cdot p_{1|0}^B \cdot \frac{v_{1|01}}{v_0} + v_{1} \cdot p_{0|1}^B \cdot \frac{v_{1|10}}{v_1} +  v_{1} \cdot p_{1|1}^B \cdot \frac{v_{1|11}}{v_1}\\
        &= p_{0|0}^B \cdot v_{1|00} + p_{1|0}^B \cdot v_{1|01} + p_{0|1}^B \cdot v_{1|10} + p_{1|1}^B \cdot v_{1|11}.
    \end{align*}
    
    Next, we will replace quantum messages $\rho_x$ with classical messages $s_x$ using \cref{lem:replace}. Let $\delta$ be a sufficiently small constant, say $\frac{1}{10^6}$. Let us denote by $q$ the number of qubits of the message $\rho_x$. The operators we will consider are 
    \begin{align*}
        \{E_b\} = \{ M_0^\dagger M_0, M_0^\dagger M_{1|00}^\dagger M_{1|00} M_0, M_0^\dagger M_{1|01}^\dagger M_{1|01} M_0, M_1^\dagger M_{1|10}^\dagger M_{1|10} M_1,  M_1^\dagger M_{1|11}^\dagger M_{1|11} M_1 \}.
    \end{align*}
    From the definition, these operators $\{E_b\}$ satisfy $0 \leq E_b \leq I$.
    Since the number of operators we need to care is 5, the length of the classical string $s_x$ is $O(q \log q)$ bits. Let us consider a hybrid SMP scheme that uses $s_x$ and $\sigma_y$ as two messages where the referee simulates the original LOCC interactions. Let us denote by $v'_0, v'_{1|00}, v'_{1|01}, v'_{1|01}, v'_{1|11}$ the five values the referee guesses using $s_x$. We also define another set of variables $v''_0, \ldots, v''_{1|11}$ that the referee will use.
    \begin{itemize}
\item    If $v'_0 > 1$, $v''_0 = 1$. If $v'_0 < 0$, $v''_0 = 0$. Otherwise, $v''_0 = v'_0$. Let $v''_1 = 1-v''_0$.
    
\item    If $v'_{1|00} > v''_0$, $v''_{1|00} = v''_0$. If $v'_{1|00} < 0$, $v''_{1|00} = 0$. Otherwise, $v''_{1|00} = v'_{1|00}$.
\item    If $v'_{1|01} > v''_0$, $v''_{1|01} = v''_0$. If $v'_{1|01} < 0$, $v''_{1|01} = 0$. Otherwise, $v''_{1|01} = v'_{1|01}$.
\item    If $v'_{1|10} > v''_1$, $v''_{1|10} = v''_1$. If $v'_{1|10} < 0$, $v''_{1|10} = 0$. Otherwise, $v''_{1|10} = v'_{1|10}$.
\item   If $v'_{1|11} > v''_1$, $v''_{1|11} = v''_1$. If $v'_{1|11} < 0$, $v''_{1|11} = 0$. Otherwise, $v''_{1|11} = v'_{1|11}$.
    \end{itemize}
    This adjustment is required to make $v''_{0}$, $v''_{1}$, $\frac{v''_{1|00}}{v''_{0}}$, $\frac{v''_{1|01}}{v''_{0}}$, $\frac{v''_{1|10}}{v''_{1}}$, $\frac{v''_{1|11}}{v''_{1}}$ valid probabilities. 
    
    The referee first simulates the first measurement result by $\rm{Ref}_A$. With probability $v''_0$, the referee simulates the result to be $0$ and with probability $v''_1$ the result to be $1$. Next, the referee measures Bob's message based on the value of the first measurement result obtained by the simulation. Finally, the referee simulates the final measurement result by $\rm{Ref}_A$ using $v''_{1|00},v''_{1|01},v''_{1|10},v''_{1|11}$.
    The acceptance probability of the simulation is
    \begin{align*}
        &v''_{0} \cdot p_{0|0}^B \cdot \frac{v''_{1|00}}{v''_{0}} + v''_{0} \cdot p_{0|0}^B \cdot \frac{v''_{1|01}}{v''_{0}} + v''_{1} \cdot p_{1|0}^B \cdot \frac{v''_{1|10}}{v''_{1}} + v''_{1} \cdot p_{1|1}^B \cdot \frac{v''_{1|11}}{v''_{1}} \\
        &= p_{0|0}^B \cdot v''_{1|00} + p_{1|0}^B \cdot v''_{1|01} + p_{0|1}^B \cdot v''_{1|10} + p_{1|1}^B \cdot v''_{1|11}.
    \end{align*}
    If $v''_0=0$, $v''_{1|00} = 0$ and $v''_{1|01} = 0$. If $v''_1=0$, $v''_{1|10} = 0$ and $v''_{1|11} = 0$. Therefore, the quantity does not lose generality.

    If $v'_0 > 1$, $- \delta \leq v_0 - v'_0 \leq v_0 - v''_0 = v_0 - 1 \leq 0$, which implies $|v_0 - v''_0| \leq \delta$. If $v'_0 < 0$, $0 \leq v_0 - v''_0 = v_0 \leq v_0 - v'_0 \leq \delta$, which implies $|v_0 - v''_0| \leq \delta$. If $v''_0 = v'_0$, $|v_0 - v''_0| = |v_0 - v'_0| \leq \delta$.
    If $v'_0 > 1$, $|v_1 - v''_1| = v_1 = 1-v_0 \leq \delta$. If $v'_0 < 0$, $|v_1 - v''_1| = |v_1 - 1| = |v_0| \leq \delta$. If $v''_0 = v'_0$, $|v_1 - v''_1| = |(1-v_0) - (1-v''_0)| \leq \delta$. In summary, we have $|v_0 - v''_0| \leq \delta$ and $|v_1 - v''_1| \leq \delta$ in all cases.
    
    Let us show the quantity $|v_{1|00} - v''_{1|00}|$ is small.
    If $v'_{1|00} > v''_0$, $v''_{1|00} = v''_0$ and we have 
    \[
         - \delta \leq v_{1|00} - v'_{1|00} \leq v_{1|00} - v''_{0} \leq v_{0} - v''_{0} \leq \delta
    \]
    by the definitions, and it implies $|v_{1|00} - v''_{1|00}| = |v_{1|00} - v''_0| \leq \delta$. 
    If $v'_{1|00} < 0$, $v''_{1|00} = 0$ and we have
    \[
        |v_{1|00} - v''_{1|00}| =  v_{1|00} \leq |v_{1|00} - v'_{1|00}| \leq \delta
    \]
    by the definitions.
    Otherwise, $v''_{1|00} = v'_{1|00}$ and $|v_{1|00} - v''_{1|00}| = |v_{1|00} - v'_{1|00}| \leq \delta$. In summary, in all cases, we have $|v_{1|00} - v''_{1|00}| \leq \delta$. By similar discussions, we also have $|v_{1|01} - v''_{1|01}| \leq \delta$.

    Let us next show the quantity $|v_{1|10} - v''_{1|10}|$ is small. If $v'_{1|10} > v''_1$, $v''_{1|10} = v''_1$ and we have
    \[
         - \delta \leq v_{1|10} - v'_{1|10} \leq v_{1|10} - v''_{1} \leq v_{1} - v''_{1} \leq \delta
    \]
    by the definitions, and it implies $|v_{1|10} - v''_{1|10}| = |v_{1|10} - v''_1| \leq \delta$. If $v'_{1|00} < 0$, $v''_{1|00} = 0$ and we have
    \[
        |v_{1|00} - v''_{1|00}| =  v_{1|00} \leq |v_{1|00} - v'_{1|00}| \leq \delta
    \]
    by the definitions.
    If $v'_{1|10} < 0$, $v''_{1|10} = 0$ and we have
    \[
        |v_{1|10} - v''_{1|10}| =  v_{1|10} \leq |v_{1|10} - v'_{1|10}| \leq \delta
    \]
    by the definitions.
    Otherwise, $v''_{1|10} = v'_{1|10}$ and $|v_{1|10} - v''_{1|10}| = |v_{1|10} - v'_{1|10}| \leq \delta$ by the definition. In summary, in all cases, we have $|v_{1|10} - v''_{1|10}| \leq \delta$. By similar arguments, we also have $|v_{1|11} - v''_{1|11}| \leq \delta$.
    
    Therefore, the difference between the true value and the simulation value is
    \begin{align*}
        &|(p_{0|0}^B \cdot v_{1|00} + p_{1|0}^B \cdot v_{1|01} + p_{0|1}^B \cdot v_{1|10} + p_{1|1}^B \cdot v_{1|11}) - (p_{0|0}^B \cdot v''_{1|00} +p_{1|0}^B \cdot v''_{1|10} + p_{0|1}^B \cdot v''_{1|10} + p_{1|1}^B \cdot v''_{1|11})| \\
        & \leq p_{0|0}^B |v_{1|00} - v''_{1|00}| + p_{1|0}^B |v_{1|01} - v''_{1|01}| + p_{0|1}^B |v_{1|10} - v''_{1|10}| + p_{1|1}^B |v_{1|11} - v''_{1|11}| \\
        & \leq (p_{0|0}^B + p^B_{1|0} + p^B_{0|1} + p^B_{1|1}) \delta = 2 \delta,
    \end{align*}
    where we use $p_{0|0}^B + p^B_{1|0} = 1$ and $p^B_{0|1} + p^B_{1|1} = 1$.
    
    If classical messages of hybrid SMP schemes for $\EQ_n$ are deterministic, then the length of the classical message is $\Omega(n)$. This is because, for $2^n$ inputs, there are some conflicts of deterministic messages and the referee cannot distinguish them. We thus have $O(q \log q) = \Omega(n)$ and $q = \Omega(n/\log n)$, which concludes the proof.
\end{proof}

\subsection{Multiple-round LOCC SMP protocols}\label{subsec:many_rounds}

When we increase the number $r$ of rounds by $2$, the number of operators we will care is increased by $4^r$. This is because both the numbers of measurement outcomes by $\rm{Ref}_A$ and $\rm{Ref}_B$ are $2$. Therefore, the total number of the operators we will care for $2r$-round LOCC protocols is 
\begin{equation}\label{eq:num}
    \sum_{i=0}^{r-1} 4^r = \frac{4^r - 1}{4-1} \leq 4^r = 2^{2r}. 
\end{equation}

We first show that each measurement outcome can be simulated by taking ratios of $2^{2r}$ values to simulate quantum $2$-value $2r$-round quantum LOCC SMP protocol. Note that for measurement operators $M_i$ for $i\in[n]$, $0 \leq M_0^\dagger\cdots M_{n-1}^\dagger M_{n-1}\cdots M_0 \leq I$.

\begin{lemma}\label{lem:ratio}
    Let us consider the case where a quantum state $\rho$ is measured $n \in \mathbb{N}$ times, and let $M_i$ be a measurement operator of the $i$th 2-value measurement for $i \in [n]$. Let us denote $v_\emptyset:=1$ and $v_i := \mathrm{tr}(M_0^\dagger\cdots M_{i}^\dagger M_{i}\cdots M_0 \rho )$. Then, the probability that the $n$th measurement succeeds conditioned on the event that all $n-1$ previous measurements succeed, $p_{n-1|n-2}$, is $\frac{v_{n-1}}{v_{n-2}}$.
\end{lemma}
\begin{proof}
    We will show the statement by induction. When $n=1$, $p_{0|\emptyset} = \mathrm{tr}(M_0^\dagger M_0 \rho) = \frac{v_0}{v_\emptyset}$.
    
    For $k>1$, let us consider $p_{k|k-1}$. Let us denote by $\rho_i$ the state after the $(i-1)$th measurement succeeds for $i \in [k]$. In other words, we define
    \[
        \rho_i := \frac{M_i \rho_{i-1} M_i^\dagger}{\mathrm{tr}(M_i^\dagger M_i \rho_{i-1})}
    \]
    for $i \in [k]$. Using the notation, let us assume that 
    \[
        p_{i|i-1} = \mathrm{tr}(M_{i}^\dagger M_{i} \rho_{i-1}) = \frac{v_{i}}{v_{i-1}}
    \]
    for all $i \in [k-1]$.

    Then, we have
    \begin{align*}
        p_{k|k-1} &= \mathrm{tr}(M_k^\dagger M_k \rho_{k-1}) = \frac{\mathrm{tr}(M_k^\dagger M_k M_{k-1} \rho_{k-2} M_{k-1}^\dagger)}{\mathrm{tr}(M_{k-1}^\dagger M_{k-1} \rho_{k-2})} \\
        &= \frac{v_{k-2}}{v_{k-1}} \mathrm{tr}(M_k^\dagger M_k M_{k-1} \rho_{k-2} M_{k-1}^\dagger) = \frac{v_{k-2}}{v_{k-1}} \mathrm{tr}(M_{k-1}^\dagger M_k^\dagger M_k M_{k-1} \rho_{k-2}) \\
        &= \frac{v_{k-2}}{v_{k-1}} \mathrm{tr}(M_{k-1}^\dagger M_k^\dagger M_k M_{k-1} \frac{M_{k-2} \rho_{k-3} M_{k-2}^\dagger}{\mathrm{tr}(M_{k-2}^\dagger M_{k-2} \rho_{k-3})}) \\
        &= \frac{v_{k-2}}{v_{k-1}} \frac{v_{k-3}}{v_{k-2}} \mathrm{tr}(M_{k-2}^\dagger M_{k-1}^\dagger M_k^\dagger M_k M_{k-1} M_{k-2} \rho_{k-3}) \\
        &= \cdots \\
        &= \frac{v_{k-2}}{v_{k-1}} \frac{v_{k-3}}{v_{k-2}} \cdots \frac{v_1}{v_2}\mathrm{tr}(M_2^\dagger \cdot M_k^\dagger M_k \cdots M_2 \rho_1) \\
        &= \frac{v_{k-2}}{v_{k-1}} \frac{v_{k-3}}{v_{k-2}} \cdots \frac{v_1}{v_2} \frac{v_k}{v_1}\\
        &= \frac{v_k}{v_{k-1}}
    \end{align*}
    where we use the cyclic property of the trace.
    Therefore, by induction, we have the claim.
\end{proof}

\begin{lemma}\label{lem:error}
    Let $\mathcal{P}$ be a quantum $2$-value $2r$-round LOCC SMP protocol to solve $\EQ_n$. Using a deterministic message from \cref{lem:replace}, the referee can simulate $\mathcal{P}$ with error $2^r(r+1)\delta$.
\end{lemma}

\begin{proof}
We assume, without loss of generality, that the final measurement outcome by $\rm{Ref}_A$ is the output of the protocol because it depends on all the previous measurement outcomes.
For all previous measurement results $h = a_0 b_0 \cdots$, let us denote by $h[i]$ the $(i+1)$th bit of $h$ and by $h[i,j]$ $(i+1)$th to $(j+1)$th bits of $h$. 
For $h \in \{0,1\}^{2R-1}$, let us define 
\[
v_{m|h} = \mathrm{tr}(M_{h[0]}^\dagger \cdots M_{h[2R-2]|h[0,2R-3]}^\dagger M_{h[2R-2]|h[0,2R-3]} \cdots M_{h[0]} \rho_x).
\] 
We also define $v''_{m|h}$ to make the probabilities valid as in \cref{prop:warm-up} (see the proof of \cref{lem:error} for a formal definition).

The true and simulation probability is the sum of $2^{2r}$ values and, from \cref{lem:ratio}, the total error is
\[
    \left| \sum_{h \in \{0,1\}^{2r}} \prod_{i \in [r]} p^B_{h[2i-1]|h[0,2i-2]} (v_{1|h} - v''_{1|h}) \right| \leq \sum_{h \in \{0,1\}^{2r}} \prod_{i \in [r]} p^B_{h[2i-1]|h[0,2i-2]}  |v_{1|h} - v''_{1|h}|,
\]
where $h[2i-1]$ denotes the $(2i)$th bit of $h$ (i.e., $b_i$) and $h[0,2i-2]$ denotes 1st to $(2i-1)$th bits of $h$ (i.e., $a_0b_0\cdots a_{i-1} b_{i-1} a_i$).

Let us evaluate how much error each term causes.
\begin{lemma}
    $|v_{m|h} - v''_{m|h}| \leq (k+1) \delta$ for all $m \in \{0,1\}$, $h \in \{0,1\}^{2k}$.
\end{lemma}

\begin{proof}
    Let us first show the base case where $k=0$ and $h = \emptyset$. If $v'_0 > 1$, $v''_0 := 1$. If $v'_0 < 0$, $v''_0 := 0$. Otherwise, $v''_0 := v'_0$. Let $v''_1 := 1-v''_0$. 
    
    If $v'_0 > 1$, $- \delta \leq v_0 - v'_0 \leq v_0 - v''_0 = v_0 - 1 \leq 0$, which implies $|v_0 - v''_0| \leq \delta$. If $v'_0 < 0$, $0 \leq v_0 - v''_0 = v_0 \leq v_0 - v'_0 \leq \delta$, which implies $|v_0 - v''_0| \leq \delta$. If $v''_0 = v'_0$, $|v_0 - v''_0| = |v_0 - v'_0| \leq \delta$.
    If $v'_0 > 1$, $|v_1 - v''_1| = v_1 = 1-v_0 \leq \delta$. If $v'_0 < 0$, $|v_1 - v''_1| = |v_1 - 1| = |v_0| \leq \delta$. If $v''_0 = v'_0$, $|v_1 - v''_1| = |(1-v_0) - (1-v''_0)| \leq \delta$. In summary, we have $|v_0 - v''_0| \leq \delta$ and $|v_1 - v''_1| \leq \delta$ in all cases.
    
    Assume that $|v_{m|h} - v''_{m|h}| \leq (t+1) \delta$ for all $m \in \{0,1\}$ and $h \in \{0,1\}^{2t}$. If $v'_{0|hab} > v''_{a|h}$, $v''_{0|hab} = v''_{a|h}$. If $v'_{0|hab} <0$, $v''_{0|hab} = 0$. Otherwise, $v''_{0|hab}= v'_{0|hab}$. Let $v''_{1|hab} = v''_{a|h} - v''_{0|hab}$. We want to prove that $|v_{m|hab} - v''_{m|hab}| \leq (t+2) \delta$ for $m \in \{0,1\}$, $h \in \{0,1\}^t$, $a \in \{0,1\}$ and $b \in \{0,1\}$.

    For conciseness, let us fix $a=0$, $b=0$ and we will prove $|v''_{m|h00} - v_{m|h00}| \leq (t+2) \delta$ for $m \in \{0,1\}$. If $v'_{0|h00} > v''_{0|h}$,
    \[
        -\delta \leq v_{0|h00} - v'_{0|h00} \leq v_{0|h00} - v''_{0|h} \leq v_{0|h} - v''_{0|h} \leq (t+1) \delta,
    \]
    which implies $|v_{0|h00} - v''_{0|h00}| = |v_{0|h00} - v''_{0|h}| \leq (t+1) \delta$.
    If $v'_{0|h00} <0$, $|v_{0|h00} - v''_{0|h00}| = |v_{0|h00}| \leq |v_{0|h00} - v'_{0|h00}| \leq (t+1) \delta$. Otherwise, $|v_{0|h00} - v''_{0|h00}| = |v_{0|h00} - v'_{0|h00}| \leq \delta$. In summary, we have $|v_{0|h00} - v''_{0|h00}| \leq (t+1) \delta$.

    We next show $|v_{1|h00} - v''_{1|h00}| \leq (t+2) \delta$.  If $v'_{0|h00} > v''_{0|h}$, $|v_{1|h00} - v''_{1|h00}| = v_{1|h00} = v_{0|h} - v_{0|h00} \leq (v''_{0|h} + (t+1)\delta) - (v'_{0|h00} - \delta) \leq (t+2)\delta$. If $v'_{0|h00} <0$, $v''_{1|h00} = v''_{0|h}$ and $|v_{1|h00} - v''_{1|h00}| = |v_{1|h00} - v''_{0|h}| \leq |v_{1|h00} - v_{0|h}| + |v_{0|h} - v''_{0|h}| = |v_{0|h00}| + |v_{0|h} - v''_{0|h}| \leq (t+2)\delta$. Otherwise, $|v_{1|h00} - v''_{1|h00}| = |v_{1|h00} - (v''_{0|h} - v'_{0|h00})| = |(v_{0|h} - v_{0|h00}) - (v''_{0|h} - v'_{0|h00})| \leq |v_{0|h} - v''_{0|h}| + |v_{0|h00} - v'_{0|h00}| \leq (t+2)\delta$.

    Taking all the values $a,b \in \{0,1\}$ into account, we have $|v_{m|hab} - v''_{m|hab}| \leq (t+2) \delta$ for $m \in \{0,1\}$, $h \in \{0,1\}^t$, $a \in \{0,1\}$ and $b \in \{0,1\}$. By induction on $t$, we have the claim.
\end{proof}

For all $h$, $p^B_{0|h} + p^B_{1|h} = 1$. We thus have
\begin{align*}
    &\sum_{h \in \{0,1\}^{2r}} \prod_{i \in [r]} p^B_{h[2i-1]|h[0,2i-2]}  
    = \sum_{h[0] \in \{0,1\}} \sum_{h[1] \in \{0,1\}} \cdots \sum_{h[2r-1] \in \{0,1\}} \prod_{i \in [r]} p^B_{h[2i-1]|h[0,2i-2]} \\
    &= \sum_{h[0] \in \{0,1\}} \sum_{h[2] \in \{0,1\}} \cdots \sum_{h[2r-2] \in \{0,1\}} \sum_{h[1] \in \{0,1\}} p^B_{h[1]|h[0]}  \sum_{h[3] \in \{0,1\}} p^B_{h[3]|h[0,2]} \cdots \sum_{h[2r-1] \in \{0,1\}}  p^B_{h[2i-1]|h[0,2i-2]} \\
    &= \sum_{h[0] \in \{0,1\}} \sum_{h[2] \in \{0,1\}} \cdots \sum_{h[2r-2] \in \{0,1\}} 1 = 2^{r}
\end{align*}

Therefore, the error is at most
\begin{align*}
    &\sum_{h \in \{0,1\}^{2r}} \prod_{i \in [r]} p^B_{h[2i-1]|h[0,2i-2]} (r+1) \delta = 2^r(r+1)\delta,
\end{align*}
which concludes the proof.
\end{proof}

\begin{theorem}\label{thm:two-way_LOCC}
    For a sufficiently small constant $k>0$, $n \in\mathbb{N}$, let $\mathcal{P}$ be a quantum $2$-value $2k \log_2 n$-round LOCC protocol to solve $\EQ_n$. Then the size of Alice's message $|\rho_x|$ is $\omega(n^{1-3k-\epsilon})$ for an arbitrary small constant $\epsilon >0$.
\end{theorem}

\begin{proof}
    Let $\delta$ be $n^{-k-\epsilon_1}$ for a sufficiently small constant $\epsilon_1 >0$. Suppose that $|\rho_x|$ is $O(n^{1-3(k+\epsilon_1)-\epsilon_2})$ for a sufficiently small constant $\epsilon_2 >0$. From \cref{eq:num}, the number of values the referee guesses is $2^{2 k \log_2 n}$. Then, from \cref{lem:replace} and \cref{lem:error}, the length of the deterministic message is $O ( (n^{1-3(k+\epsilon_1)-\epsilon_2}) n^{3(k+\epsilon_1)} ((1-3(k+\epsilon_1)-\epsilon_2)\log n + (k+\epsilon_1) \log n) (2 k \log_2 n + (k+\epsilon_1)\log n) )= o(n)$ and the error is $n^k (k \log n + 1) (n^{-k-\epsilon_1}) = o(1)$. If one message is deterministic in the hybrid SMP model and the model solves $\EQ$ with high probability, the length of the message is $n$. Since we can take $\epsilon = 3\epsilon_1+\epsilon_2$ an arbitrary small value, we have $|\rho_x| = \omega(n^{1-3k-\epsilon})$ for an arbitrary small constant $\epsilon>0$.
\end{proof}

\subsection{General arguments}

In the proofs of \cref{subsec:warm-up} and \cref{subsec:many_rounds}, we use a property that is inherent to $\EQ_n$. The property is that, if one message is deterministic in SMP models for $\EQ_n$, the length of the deterministic message is $n$. We can show more general results by replacing both quantum messages.

\begin{lemma}\label{lem:error_replace_both}
    Let $\mathcal{P}$ be a $2$-value $2r$-round LOCC protocol to solve $F:\{0,1\}^n \times \{0,1\}^n \rightarrow \{0,1\}$. Using a deterministic message from \cref{lem:replace}, the referee can simulate the quantum LOCC SMP protocol with error $O(2^{2r}r^2\delta^2)$.
\end{lemma}

\begin{proof}
For $m \in \{0,1\}$ and $h \in \{0,1\}^*$, let us define $v^A_{m|h} := v_{m|h}, v''^A_{m|h} := v''_{m|h}$ and, in a similar way, define $v^B_{m|h}, v''^B_{m|h}$ from the replacement of Bob's message.
Then, the error by the simulation of the referee from the two deterministic strings is at most
\begin{align*}
    & \left| \sum_{h \in \{0,1\}^{2r}} v^A_{1|h} v^B_{h[2i-1]|h[0,2i-2]} - v''^A_{1|h} v''^B_{h[2i-1]|h[0,2i-2]} \right| \\
    & \leq \sum_{h \in \{0,1\}^{2r}} \left| v^A_{1|h} v^B_{h[2i-1]|h[0,2i-2]} - v''^A_{1|h} v''^B_{h[2i-1]|h[0,2i-2]} \right| \\
    & \leq \sum_{h \in \{0,1\}^{2r}} \left| v^A_{1|h} v^B_{h[2i-1]|h[0,2i-2]} - \left( v^A_{1|h} + O(r\delta) \right) \left( v^B_{h[2i-1]|h[0,2i-2]} + O(r\delta) \right) \right| = O(2^{2r}r^2\delta^2) 
\end{align*}
\end{proof}

\begin{proposition}\label{prop:two-way-LOCC_general}
    For a sufficiently small constant $k>0$, $n \in\mathbb{N}$, let $\mathcal{P}$ be a quantum $2$-value $2k \log_2 n$-round LOCC protocol to solve $F:\{0,1\}^n \times \{0,1\}^n \rightarrow \{0,1\}$ such that $\R^{||}(F) = \Omega(n^c)$ for a constant $c>0$. Then, the size of the quantum messages in $\mathcal{P}$ is $\omega(n^{c-6k-\epsilon})$ for an arbitrary small constant $\epsilon >0$.
\end{proposition}

\begin{proof}
    Let $\delta$ be $n^{-2k-\epsilon_1}$ for a sufficiently small constant $\epsilon_1 >0$. Let $\rho_x$ and $\sigma_y$ be quantum messages from Alice and Bob, respectively. Suppose that $|\rho_x|$ and $|\sigma_y|$ are $O(n^{c-3(2k+\epsilon_1)-\epsilon_2})$ for a sufficiently small constant $\epsilon_2 >0$. From \cref{eq:num}, the number of values the referee guesses is $2^{2 k \log_2 n}$ for each message. Then, from \cref{lem:replace} and \cref{lem:error_replace_both}, the sum of the lengths of the deterministic messages is $2 \times O(n^{c-3(2k+\epsilon_1)-\epsilon_2}) n^{3(2k+\epsilon_1)} ((c-3(2k+\epsilon_1)-\epsilon_2)\log n + (2k+\epsilon_1) \log n) (2 k \log_2 n + (2k+\epsilon_1)\log n)= o(n^c)$ and the error is $O(n^{2k}(k \log n)^2 n^{-2k-\epsilon_1}) = o(1)$. This contradicts $\R^{||}(F) = \Omega(n^c)$. Since we can take $\epsilon = 3\epsilon_1+\epsilon_2$ an arbitrary small value, we have $\max\{|\rho_x|,|\sigma_y|\} = \omega(n^{c-6k-\epsilon})$ for an arbitrary small constant $\epsilon>0$.
\end{proof}

Moreover, we can consider a lower bound for general relation problems.

\begin{lemma}
    Let $\mathcal{P}$ be a $2$-value $2r$-round LOCC protocol to solve $R \subseteq \{0,1\}^n \times \{0,1\}^n \times \{0,1\}^*$. Using a deterministic message from \cref{lem:replace}, the referee can simulate the output of the quantum LOCC SMP protocol with error $O(2^{2r}r^2\delta^2)$.
\end{lemma}

\begin{proof}
For $m \in \{0,1\}$ and $h \in \{0,1\}^*$, let us define $v^A_{m|h} := v_{m|h}, v''^A_{m|h} := v''_{m|h}$ and, in a similar way, define $v^B_{m|h}, v''^B_{m|h}$ from the replacement of Bob's message.
Then, the error by the simulation of all the measurement outcomes of the referee from the two deterministic strings is at most
\begin{align*}
    & \left| \sum_{m \in \{0,1\}, h \in \{0,1\}^{2r}} v^A_{m|h} v^B_{h[2i-1]|h[0,2i-2]} - v''^A_{m|h} v''^B_{h[2i-1]|h[0,2i-2]} \right| \\
    & \leq 2 \times \sum_{h \in \{0,1\}^{2r}} \left| v^A_{1|h} v^B_{h[2i-1]|h[0,2i-2]} - v''^A_{1|h} v''^B_{h[2i-1]|h[0,2i-2]} \right| \\
    & \leq 2 \times \sum_{h \in \{0,1\}^{2r}} \left| v^A_{1|h} v^B_{h[2i-1]|h[0,2i-2]} - \left( v^A_{1|h} + O(r\delta) \right) \left( v^B_{h[2i-1]|h[0,2i-2]} + O(r\delta) \right) \right| = O(2^{2r}r^2\delta^2) 
\end{align*}
If all the measurement outcomes are the same, the output of the protocol is also the same as the original protocol.
\end{proof}

Then, by the same proof as \cref{prop:two-way-LOCC_general}, we have the same argument for general relation problems.

\begin{theorem}\label{thm:two-way-LOCC_general_relation}
    For a sufficiently small constant $k>0$, $n \in\mathbb{N}$, let $\mathcal{P}$ be a $2$-value $2k \log_2 n$-round LOCC protocol to solve $R \subseteq \{0,1\}^n \times \{0,1\}^n \times \{0,1\}^*$ such that $\R^{||}(R) = \Omega(n^c)$ for a constant $c>0$. Then, the size of the quantum messages in $\mathcal{P}$ is $\omega(n^{c-6k-\epsilon})$ for an arbitrary small constant $\epsilon >0$.
\end{theorem}

\section{Separation between quantum one-way and two-way-LOCC SMP protocols}\label{sec:separation}

For hybrid SMP schemes for non-symmetric problems, it is important which party can send a quantum message while the other one can only send classical messages, and let us introduce more fine-grained hybrid SMP models. In this section, we will consider a hybrid SMP model in which Alice sends a quantum message and Bob sends a classical message, and let us denote by $\Q \R^{||}(R)$ the hybrid SMP communication complexity for $R$. We also consider a hybrid SMP model in which Bob sends a quantum message and Alice sends a classical message, and let us denote by $\R \Q^{||}(R)$ the hybrid SMP communication complexity for $R$. 
We also define $\Q^{||,\mathsf{LOCC}_1^{A \rightarrow B}}(R)$ as the complexity of a quantum one-way LOCC protocol for $R$ where $\rm{Ref}_A$ measures Alice's message first and sends a measurement result to $\rm{Ref}_B$, and then $\rm{Ref}_B$ measures Bob's message, $\Q^{||,\mathsf{LOCC}_1^{B \rightarrow A}}(R)$ as the complexity of a quantum one-way LOCC protocol for $R$ where the $\rm{Ref}_B$ measures Bob's message first and sends a measurement result to $\rm{Ref}_A$, and then $\rm{Ref}_A$ measures Alice's message.
Other notations of communication complexity are the same as defined in \cref{sec:prel}.

The Hidden Matching Problem was introduced by Bar-Yossef, Jayram, and Kerenidis \cite{BYJK08} and they showed that there exists a relation problem which exhibits an exponential separation between classical and quantum one-way communication complexity. For $k \in [n]$, let us denote by $x(k)$ the $(k-1)$th bit of $x \in \{0,1\}^n$.

\begin{definition}[The Hidden Matching Problem  ($\mathsf{HM}_n$) \cite{BYJK08}]
    Let $n$ be an even positive integer. In the Hidden Matching Problem ($\mathsf{HM}_n$), Alice is given $x \in \{0,1\}^n$ and Bob is given $M \in \mathcal{M}_n$ where $ \mathcal{M}_n$ is the family of all possible perfect matchings on $n$ nodes. They coordinate to output a tuple $\langle i,j,b \rangle$ such that the edge $(i,j)$ belongs to the matching $M$ and $b = x(i) \oplus x(j)$.
\end{definition}

\begin{theorem}[\cite{BYJK08}]
    $\Q^{1}(\mathsf{HM}_n) = O(\log n)$, $\R^{1}(\mathsf{HM}_n) = \Theta(\sqrt{n})$.
\end{theorem}

For completeness, let us describe a quantum protocol to solve $\mathsf{HM}_n$ efficiently.

\begin{algorithm}[H]
\caption{\, $O(\log n)$ qubit one-way communication protocol for $\mathsf{HM}_n$ \cite{BYJK08}}
\label{}
\begin{algorithmic}[1]
\State Alice sends $O(\log n)$-qubit state $\frac{1}{\sqrt{n}} \sum_{i=1}^n (-1)^{x(i)} \ket{i}$ to Bob. 
\State Bob regards $(i,j) \in M$ as a projector $P_{ij} = \ket{i}\bra{i} + \ket{j}\bra{j}$, and conducts the corresponding projective measurement on the quantum message. If the result is $(i,j) \in M$, Bob finally measures the residue state in a base $\{ \frac{1}{\sqrt{2}} (\ket{i}+\ket{j}), \frac{1}{\sqrt{2}}(\ket{i}-\ket{j}) \}$ and outputs the final measurement result.
\end{algorithmic}
\end{algorithm}

They also introduced a variant of the Hidden Matching Problem and showed a separation between classical SMP models and the hybrid SMP model.

\begin{definition}[Restricted Hidden Matching Problem ($\mathsf{RHM}_n$), Section 5 in \cite{BYJK08}]
    Let $n$ be an even positive integer. In the Restricted Hidden Matching Problem, fix $\mathcal{M}$ to be any set of $m = \Theta(n)$ pairwise edge-disjoint matchings. Alice is given $x \in \{0, 1\}^n$, and Bob is given $M \in \mathcal{M}$. Their goal is to output a tuple $\langle i, j, b \rangle$ such that the edge $(i, j)$ belongs to the matching $M$ and $b = x(i) \oplus x(j)$.
\end{definition}

\begin{lemma}[Section 5 in \cite{BYJK08}]\label{lem:rhm}
    $\R^{1}(\mathsf{RHM}_n) = \R^{||}(\mathsf{RHM}_n) = \Theta(\sqrt{n})$, $\Q\R^{||}(\mathsf{RHM}_n) = O(\log n)$.
\end{lemma}

Note that in the hybrid SMP protocol, Bob tells the referee which matching he received from $O(n)$ matchings by $O(\log n)$ bits.

We show that, in a quantum one-way LOCC protocol, if the referee measures Bob's message first, the problem is easy, and, if the referee measures Alice's message first, the problem is hard.

\begin{proposition}
    $\Q^{||,\mathsf{LOCC}_1^{B \rightarrow A}}(\mathsf{RHM}_n) = O(\log n)$, $\R\Q^{||}(\mathsf{RHM}_n) = \Q^{||,\mathsf{LOCC}_1^{A \rightarrow B}}(\mathsf{RHM}_n) = \Theta(\sqrt{n})$.
\end{proposition}

\begin{proof}
    $\Q^{||,\mathsf{LOCC}_1^{B \rightarrow A}}(\mathsf{RHM}_n) = O(\log n)$ follows from $\Q\R^{||}(\mathsf{RHM}_n) = O(\log n)$ by encoding and decoding the information of the matchings in the computational basis.
    
    When we regard Bob and the referee as one party and allow arbitrary communication between them, the communication model is the same as the classical one-way communication model. In the classical one-way communication model, $\Theta(\sqrt{n})$ bits communication is required from \cref{lem:rhm}. Therefore, in the hybrid SMP model as a weaker model than the classical one-way communication model, $\Theta(\sqrt{n})$ bits communication is also required (i.e, $\R\Q^{||}(\mathsf{RHM}_n) = \Theta (\sqrt{n})$).
    
    Finally, consider $\Q^{||,\mathsf{LOCC}_1^{A \rightarrow B}}(\mathsf{RHM}_n)$. From \cref{thm:one-way_LOCC}, we can replace Alice's quantum message with a classical message by a small overhead, and we have $\Q^{||,\mathsf{LOCC}_1^{A \rightarrow B}}(\mathsf{RHM}_n) =  \Theta (\sqrt{n})$.
\end{proof}

We next consider a further variant of the Restricted Hidden Matching Problem to make both parties send quantum messages.

\begin{definition}[Double Restricted Hidden Matching problem ($\mathsf{DRHM}_n$]
    Let $n$ be an even positive integer. Take two independent instances from the Restricted Hidden Matching Problem. Let $x_1 \in \{0, 1\}^n, M_1 \in \mathcal{M}_1$ be an input of the first instance and $x_2 \in \{0, 1\}^n, M_2 \in \mathcal{M}_2$ be an input of the second instance. Alice is given $x_1$ and $M_2$ and Bob is given $x_2$ and $M_1$. Their goal is to output two tuples $\langle i_1, j_1, b_1 \rangle$ and $\langle i_2, j_2, b_2 \rangle$ to solve the two Restricted Hidden Matching Problem.
\end{definition}

\begin{proposition}\label{prop:separation}
    $\Q^{||,\mathrm{LOCC}}(\mathsf{DRHM}_n) = O(\log n)$, $\R^{||} (\mathsf{DRHM}_n) = \R\Q^{||} (\mathsf{DRHM}_n) = \Q\R^{||} (\mathsf{DRHM}_n) = \Q^{||,\mathsf{LOCC}_1^{B \rightarrow A}}(\mathsf{DRHM}_n) = \Q^{||,\mathsf{LOCC}_1^{A \rightarrow B}}(\mathsf{DRHM}_n) =  \Theta(\sqrt{n})$.
\end{proposition}

\begin{proof}
    Let us first show $\Q^{||,\mathrm{LOCC}}(\mathsf{DRHM}_n) = O(\log n)$. Alice sends $\rm{Ref}_A$ $\frac{1}{\sqrt{n}} \sum_{i=1}^n (-1)^{x_{1}(i)} \ket{i}$ and $\ket{M_2}$ where $M_2$ is encoded in the computational basis. Bob sends $\rm{Ref}_B$ $\frac{1}{\sqrt{n}} \sum_{i=1}^n (-1)^{x_{2}(i)} \ket{i}$ and $\ket{M_1}$ where $M_1$ is encoded in the computational basis. Then, $\rm{Ref}_A$ measures $\ket{M_2}$ and sends $M_2$ to $\rm{Ref}_B$. $\rm{Ref}_B$ measures $\ket{M_1}$ and sends $M_1$ to $\rm{Ref}_A$. Based on $M_1$, $\rm{Ref}_A$ measures $\frac{1}{\sqrt{n}} \sum_{i=1}^n (-1)^{x_{1}(i)} \ket{i}$. Based on $M_2$, $\rm{Ref}_B$ measures $\frac{1}{\sqrt{n}} \sum_{i=1}^n (-1)^{x_{2}(i)} \ket{i}$. Note that each matching can be represented with $O(\log n)$ bits because the size of the set of matchings $|\mathcal{M}_1|$ and $|\mathcal{M}_2|$ is $\Theta(n)$.
    
    Let us next show $\R\Q^{||} (\mathsf{DRHM}_n) = \Theta(\sqrt{n})$. When we regard Bob and the referee as one party and allow arbitrary communication between them, the communication model is the same as the classical one-way communication model. The first and second instances are independently chosen, and to solve the first instance, $\Theta(\sqrt{n})$ bits communication is required in the model from \cref{lem:rhm}. Therefore, in the SMP model as a weaker model than the classical one-way communication model, $\Theta(\sqrt{n})$ bits communication from Alice to the referee is required. By a similar discussion, it is shown that $\Q\R^{||} (\mathsf{DRHM}_n) = \Theta(\sqrt{n})$.

    Finally, from \cref{thm:one-way_LOCC}, we have $\Q^{||,\mathsf{LOCC}_1^{B \rightarrow A}}(\mathsf{DRHM}_n) = \Q^{||,\mathsf{LOCC}_1^{A \rightarrow B}}(\mathsf{DRHM}_n) = \Theta(\sqrt{n})$.
\end{proof}

\begin{remark}\label{remark}
    In the quantum two-way-LOCC protocol, we can allow $\rm{Ref}_A$ and $\rm{Ref}_B$ to do several 2-value measurements in sequence (by making the other one do nothing in the turns). For $\mathsf{DRHM}_n$, we can solve it by $2 \log_2 n + O(\log n)$ rounds 2-value two-way-LOCC measurements. Let us describe the protocol. First, $\rm{Ref}_A$ measures $\ket{M_2}$ in the computational basis, which takes $O(\log n)$ rounds. Similarly, $\rm{Ref}_B$ measures $\ket{M_1}$ that takes $O(\log n)$ rounds. Then, based on $M_1$,  $\rm{Ref}_A$ puts half of ${(i,j) \in M_1}$ into one group and the other half into another, and measures $\frac{1}{\sqrt{n}} \sum_{i=1}^n (-1)^{x_{1}(i)} \ket{i}$ by a projector constructed from the grouping. Based on the measurement outcome, $\rm{Ref}_A$ repeats another 2-value measurement with a similar half/half grouping. $\rm{Ref}_B$ conducts a similar sequence of 2-value measurements on $\frac{1}{\sqrt{n}} \sum_{i=1}^n (-1)^{x_{2}(i)} \ket{i}$ based on $M_2$.

    However, the number of rounds is larger than the bound obtained from \cref{thm:two-way-LOCC_general_relation}, and thus there is no contradiction. Moreover, this implies that our analysis is tight up to some constant factor. Therefore, to prove a stronger lower bound toward \cref{conj}, we should develop another proof strategy which might be inherent to $\EQ_n$.
\end{remark}


\section*{Acknowledgment}
AH thanks Richard Cleve, Uma Girish, Alex B. Grilo, Kohdai Kuroiwa, Alex May, Masayuki Miyamoto, Ryuhei Mori, Ashwin Nayak, Daiki Suruga, Yuki Takeuchi, Eyuri Wakakuwa, Yibin Wang for helpful discussions. AH is also grateful to Richard Cleve and Ken-ichi Kawarabayashi for their generous support. Part of the work was done while AH was visiting University of Waterloo, and AH is grateful for their hospitality.

AH was supported by JSPS KAKENHI grants Nos.~JP22J22563, 24H00071, and  JST ASPIRE Grant No.~JPMJAP2302. SK was supported by the Natural Sciences and Engineering Research Council of Canada (NSERC) Discovery Grants Program, and Fujitsu Labs America. FLG was supported by JSPS KAKENHI grants Nos.~JP20H05966, 20H00579, 24H00071, MEXT Q-LEAP grant No.~JPMXS0120319794 and JST CREST grant No.~JPMJCR24I4. HN was supported by JSPS KAKENHI grants Nos.~JP20H05966, 22H00522, 24H00071, 24K22293, MEXT Q-LEAP grant No.~PMXS0120319794 and JST CREST grant No.~JPMJCR24I4. QW was supported by the Engineering and Physical Sciences Research Council under Grant No.~EP/X026167/1.

\bibliographystyle{alpha}
\bibliography{ref}

\end{document}
