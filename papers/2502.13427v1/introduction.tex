\section{Introduction}

\subsection{Background}

In 1979, the seminal paper by Yao \cite{Yao79} introduced the concept of communication complexity. The big picture behind this concept is, when inputs are distributed to some local processors and they coordinate to compute a function depending on distributed inputs, to identify the minimum amount of communication between them (rather than considering the computational resource in each local party). Yao introduced several communication models, including two-party communication protocols, and one important communication model is the Simultaneous Message Passing (SMP) model (referred to as $``1 \rightarrow 3 \leftarrow 2"$ in the paper). In the SMP model, Alice has a part of an input $x$, Bob has the other part of the input $y$, and a third party, whom we usually call the referee, calculates some Boolean function $F(x,y)$ by receiving two messages from Alice and Bob simultaneously. The complexity in the SMP protocols is the sum of the length of the two messages. One fundamental function is the equality function ($\EQ_n$ where $n$ stands for the size of inputs) to check if the inputs are the same or not (named ``the identification function" in the paper), and Yao asked what the complexity in the SMP model is for $\EQ_n$. There is a trivial upper bound of $O(n)$ for the complexity while it can be reduced much when Alice and Bob are allowed to share randomness by sending hashing values of their inputs. 
However, without shared randomness, the SMP model is a weak communication model, and the complexity was unknown.

In the 1990s, the question by Yao was solved. First, Ambainis \cite{Amb96} constructed an SMP protocol of cost $O(\sqrt{n})$ that exploits properties of good classical error correction codes. Subsequently, Newman and Szegedy \cite{NS96} proved a matching lower bound of $\Omega(\sqrt{n})$. Their results were simplified and generalized by Babai and Kimmel \cite{BK97}, who showed that the randomized and
deterministic complexities can be at most quadratically far apart for any function in this model. Let us denote by $\R^{||}(\EQ_n)$ the complexity in the classical (randomized) SMP model for $\EQ_n$, and the results are summarized as follows.

\begin{theorem}[\cite{Amb96,NS96,BK97}]
    $\R^{||}(\EQ_n) = \Theta (\sqrt{n})$.
\end{theorem}

Yao \cite{Yao93} also introduced a quantum analog of the two-party communication complexity, and subsequent pioneering works \cite{Kre95,CB97,BvDHT99,CvDNT98,BCW98,Raz99,BdW01} investigated the power and limit of quantum communication complexity. In 2001, Buhrman, Cleve, Watrous, and de Wolf \cite{BCWdW01} considered a quantum analog of the SMP model where Alice and Bob can send some quantum messages and the referee conducts some quantum algorithm over two messages to calculate a function. They discovered the quantum fingerprinting protocol, which is the quantum SMP protocol with $O(\log n)$ qubits of communication for the equality problem. Since $\Omega(\sqrt{n})$ bits of communication is required for the classical SMP model, there exists an exponential quantum advantage. In the \cite{BCWdW01} protocol, Alice and Bob create some quantum fingerprints of their inputs as a superposition of a codeword of their inputs with some good classical error correction codes, and the referee conducts a SWAP test \cite{BBD+97,BCWdW01} on the quantum fingerprints. Note that this bound is tight because the quantum one-way communication complexity for $\EQ_n$ is $\Theta(\log n)$ and the one-way communication is a stronger model than the SMP model \cite{BdW01,BCWdW01}. Let us denote by $\Q^{||}(\EQ_n)$ the complexity in the quantum SMP model for $\EQ_n$, and the result is denoted as follows.

\begin{theorem}[\cite{BCWdW01}]
    $\Q^{||}(\EQ_n) = \Theta(\log n)$.
\end{theorem}

Gavinsky, Regev, and de Wolf \cite{GRdW08} first considered a classical and quantum hybrid SMP protocol where one party can send a quantum message, while the other party can send only a classical message. They showed that $\Omega(\sqrt{n/\log n})$ qubits or bits communication is required for $\EQ_n$ in this model. Klauck and Podder \cite{KP14}, and Bottesch, Gavinsky, and Klauck \cite{BGK15} showed a tight lower bound to match the upper bound by Ambainis \cite{Amb96}. Let us denote by $\R \Q^{||}(\EQ_n)$ the complexity in the hybrid SMP model for the equality problem. Then, the tight lower bounds are described as follows.

\begin{theorem}[\cite{KP14,BGK15}]\label{thm:lower_bound_hybrid}
    $\R \Q^{||}(\EQ_n) = \Theta(\sqrt{n})$.
\end{theorem}

\subsection{Our setting and result}

With our current technology in quantum computing, it is still hard to implement some {\em coherent measurements} over two quantum states 
(i.e., measurements done jointly over them) with high accuracy, and the SWAP test is one of the canonical coherent measurements. In this paper, we pose the following question: 
\newline
\newline
\centerline{\emph{Does there exist an efficient quantum SMP protocol for $\EQ_n$ without coherent measurements?}}
\newline

Here \emph{efficient} means $\poly (\log n)$-qubit cost which exhibits an exponential advantage over the classical SMP models. Let us denote by $\Q^{||,\mathbb{M}}(R)$ the complexity in the quantum SMP for a relation $R$ where the referee is restricted to performing measurements from a measurement class $\mathbb{M}$. See \cref{sec:measurement} for formal definitions of some measurement classes. Note that a function is a special case of a relation in which there is a unique solution for each input, and we also denote by $\Q^{||,\mathbb{M}}(F)$ the complexity for a function $F$. It is known that the SWAP test is a projection to the symmetric subspace and therefore the corresponding measurement operator is in a measurement class SEP\footnote{Other works may denote by $\mathrm{SEP}$ the set of $M$ such that $M$ and $I-M$ can be decomposed as separable operators, and by $\mathrm{SEP}_\mathrm{YES}$ the set of $M$ such that only $M$ has to satisfy the operator is separable. See \cref{sec:measurement} for formal definitions.} \cite{HM13}. Therefore, we can rewrite the result of \cite{BCWdW01} as follows.

\begin{theorem}[\cite{BCWdW01}]
    $\Q^{||,\mathrm{SEP}}(\EQ_n) = \Theta(\log n)$. 
\end{theorem}

What is a measurement that is not coherent (or {\em incoherent}) one? Operationally, it is considered as the so-called {\em LOCC (Local Operations and Classical Communication)} measurements, which is well studied in the quantum information community (see, e.g., \cite{CLM+14}). This type of measurement over two states is a sequence of measurements over either of the states 
which can use the previous measurement results via classical communication between the states. 
Thus, our question can be described as follows:
\newline
\newline
\centerline{\emph{Does there exist an efficient quantum SMP protocol for $\EQ_n$ with LOCC measurements?}}
\newline

We first investigate the power of a quantum SMP model with a subclass of LOCC measurements, the quantum {\em one-way-LOCC} SMP model. In this model, the referee performs a measurement on one quantum message and then performs a measurement on the other quantum message based on the first measurement result. The final measurement outcome is expected to be a solution for each input. This is easier to implement than coherent measurements as well as general (two-way) LOCC measurements since we do not need any quantum memory for keeping the residue states after the measurements. Let us denote by $\Q^{||,\mathrm{LOCC}_1}(R)$ the message complexity in the model for $R$. Although it is a stronger model than the classical and quantum hybrid SMP model, we provide a new method to replace a quantum message by a classical message and prove the quantum advantage is small. Note that $\Q^{||,\mathrm{LOCC}_1}(R) \leq \R\Q^{||}(R)$ because a quantum one-way-LOCC SMP model is a stronger model than a hybrid SMP model.

\begin{theorem}[Informal version of \cref{thm:one-way_LOCC}]\label{thm:intro_one-way_LOCC}
    For any relation $R \subseteq \{0,1\}^n \times \{0,1\}^n \times \{0,1\}^*$,
    \[
        \R\Q^{||}(R) \leq 2 \Q^{||,\mathrm{LOCC}_1}(R) + O(\log n).
    \]
\end{theorem}

Combined with \cref{thm:lower_bound_hybrid}, the above theorem shows that there exists no efficient protocol in the quantum one-way-LOCC SMP model.

\begin{corollary}[\cref{cor:EQ}]
    $\Q^{||,\mathrm{LOCC}_1}(\EQ_n) = \Omega(\sqrt{n})$.
\end{corollary}

Moreover, we can describe a more general result for Boolean functions. For any Boolean function $F:\{0,1\}^n \times \{0,1\}^n \rightarrow \{0,1\}$, it is known that $\R^{||}(F) \leq \D^{||}(F)^2$ \cite{BK97} ($\D^{||}$ denotes the complexity in the deterministic SMP model), and $\R\Q^{||}(F) \leq \R^{||}(F)^2$ \cite{GRdW08}. Combining with \cref{thm:intro_one-way_LOCC}, we can claim that, for any Boolean function which is hard for classical deterministic SMP models, the advantage of quantum one-way-LOCC SMP models over classical deterministic SMP models is at most quartic.

\begin{corollary}[\cref{cor:quartic}]
    For any Boolean function $F:\{0,1\}^n \times \{0,1\}^n \rightarrow \{0,1\}$, if $\D^{||}(F) = \Omega(n^c)$ for a constant $c > 0$, then $\Q^{||,\mathrm{LOCC_1}} (F) = \Omega(n^{\frac{c}{4}})$.
\end{corollary}

The one-way LOCC measurement is a good intermediate setting to analyze because the direction of the information flow in the measurement is indeed one-way. In the more general setting of two-way LOCC measurements, the measurement results go back and forth between two parties, which gives a lot of freedom to do local measurements, and makes the analysis difficult to obtain lower bounds. For the complexity class $\QMA(2)$ \cite{KMY03,HM13}, which is an analogue of $\QMA$ but the quantum polynomial-time verifier receives two separate quantum states for certification, some results have been obtained in a similar direction to investigate what problems the verifier can verify when the measurement ability is restricted. Although the class $\QMA^{\mathbb{M}}(2)$ under the measurement restriction collapses to $\QMA$ when $\mathbb{M}$ is the class of one-way LOCC measurements \cite{BCY11,LS15} (namely, the restriction of one-way LOCC measurements is severe for $\QMA(2)$), to our knowledge, it remains unknown whether it holds or not when $\mathbb{M}$ is the class of LOCC measurements (see \cref{sec:related} for more detailed explanations) for more than 10 years. This situation demonstrates the difficulty of obtaining lower bounds involving two-way LOCC measurements.

In this very challenging setting, we succeed in proving a lower bound in quantum SMP protocols with some restricted two-way LOCC measurements. Here is our result for $\EQ_n$.

\begin{theorem}[Informal version of \cref{thm:two-way_LOCC}]
    Suppose that there exists a quantum two-way-LOCC SMP protocol for $\EQ_n$ whose measurement outcomes are $2$ values and the number of rounds is smaller than $k\log_2 n$ where $k$ is a constant such that $0<k<\frac{2}{3}$. Then, the message size is $n^{\Omega(1)}$.
\end{theorem}

We also prove a lower bound for general problems.

\begin{theorem}[Informal version of \cref{thm:two-way-LOCC_general_relation}]
    For $R \subseteq \{0,1\}^n \times \{0,1\}^n \times \{0,1\}^*$, suppose $\R^{||}(R) = \Omega(n^c)$. Also, suppose that there exists a quantum two-way-LOCC SMP protocol for $R$ whose measurement outcomes are $2$ values and the number of rounds is smaller than $k\log_2 n$ where $k$ is a constant such that $0<k<\frac{c}{3}$. Then, the message size of the quantum two-way-LOCC SMP protocol is $n^{\Omega(1)}$.
\end{theorem}

One may think that this quantum LOCC SMP model is just a weak model, but we show a quantum 2-round LOCC SMP protocol can be exponentially powerful than a quantum one-way-LOCC SMP protocol for a relation problem, albeit with measurements that are not 2-value\footnote{In fact, this protocol can be conducted by a sequence of 2-value measurements. See \cref{remark} for further information.}. The relation problem we considered is a variant of the Hidden Matching Problem introduced by Bar-Yossef, Jayram, and Kerenidis \cite{BYJK08}. 
Let us denote by $\Q^{||,\mathrm{LOCC}}(R)$ the complexity in quantum two-way-LOCC SMP protocols for a relation $R$.

\begin{proposition}[\cref{prop:separation}]
    There exists a relation problem $R$ whose input size is $n$ such that $\Q^{||,\mathrm{LOCC}}(R) = O(\log n)$ and $\Q^{||,\mathrm{LOCC}_1}(R) = \Theta(\sqrt{n})$.
\end{proposition}

We also consider a variant of quantum one-way communication protocols where measurements are not ideal. In the usual setting, a receiver's quantum computer is ``programmable" based on receiver's $2^n$ inputs (see \cref{def:Q_one-way} for a formal definition). In an actual setting, it is hard to implement a measurement from a large number of varieties with high accuracy. Instead of this situation, let us consider the case where we implement some complex and accurate measurement whose outcome can be a large classical string, but the measurement bases are fixed for any input, and we cannot keep residue states after the first measurement. In such a situation, are quantum messages more useful than classical messages? We show a negative answer for the question above by the same technique derived to have a lower bound in the quantum one-way-LOCC SMP models for $\EQ_n$. Let us call the protocol a quantum incoherent one-way communication complexity and denote by $\Q^{1,\perp}(R)$ the quantum incoherent one-way communication complexity for $R$ (see \cref{def:Q_one-way_incoherent} for a formal definition). Let us also denote by $\R^1(R)$ the randomized one-way communication complexity for $R$. Note that $\Q^{1,\perp}(R) \leq \R^1(R)$ from their definitions.

\begin{corollary}[\cref{cor:incoherent}]\label{cor:incoherent_intro}
    For any relation $R \subseteq \{0,1\}^n \times \{0,1\}^n \times \{0,1\}^*$, 
    \[
    \R^1(R) \leq 2 \Q^{1,\perp}(R) + O(\log n).
    \]
\end{corollary}

There are some known results to show that we can replace quantum messages by classical messages in one-way communication complexity. Here we consider a (possibly partial) function $F : X \times Y \rightarrow \{0,1\}$. Let us denote by $\Q^{1,*}(F)$ a variant of quantum one-way communication complexity where Alice and Bob share (unlimited) prior entanglement. We also consider classical one-way communication complexity and denote by $\D^1(F)$ (${\rm resp.}$ $\R^1(F)$) the classical deterministic (${\rm resp.}$ randomized) one-way communication complexity for $F$.

\begin{theorem}[Theorem 3.4 in \cite{Aar05}]
    $\D^1(F) = O(\log (|Y|) \Q^1(F) \log \Q^1(F) )$.
\end{theorem}

\begin{theorem}[Theorem 1.4 in \cite{Aar07}]
    $\R^1(F) = O( \log (|Y|) \Q^1(F))$.
\end{theorem}

\begin{theorem}[Theorem 3.1 in \cite{KP14}]
    $\D^1(F) = O( \log (|Y|) \Q^{1,*}(F))$.
\end{theorem}

By comparing \cref{cor:incoherent_intro} with these results, we can claim that the quantum incoherent one-way communication protocols are much weaker protocols than usual quantum one-way communication protocols.

\subsection{Our techniques}

\subsubsection*{Against one-way LOCC measurements}

For clarity, in quantum LOCC SMP protocols, let us decompose the referee into two referees $\rm{Ref}_A$ and $\rm{Ref}_B$, who receive quantum messages from Alice and Bob, respectively.
In the quantum one-way-LOCC SMP model, $\rm{Ref}_A$ first measures Alice's message and, based on the measurement result, $\rm{Ref}_B$ measures Bob's message. Our proof strategy is to make Alice do the measurement of $\rm{Ref}_A$ on Alice's message state and send it to the referee. Then, the SMP protocol is the classical and quantum hybrid SMP model.

However, there is an issue that comes up in implementing this strategy. The measurement by $\rm{Ref}_A$ can be done by adding many ancilla qubits to Alice's message state, and the measurement result can be a large classical string. Therefore, we cannot consider a direct reduction to the lower bound in the hybrid SMP models. To overcome this, we use a result of Oszmaniec, Guerini, Wittek, and Ac{\'i}n \cite{OGWA17}, who showed that general POVM measurements can be simulated by randomized projective measurements with few ancilla qubits (see \cref{lem:OGWA17} for a formal description). Since the number of outcomes is small if the measurements are projective measurements with a small number of ancilla qubits, the result may lead to a nice reduction.

However, there is still another issue that comes up. To recover the original measurement result and exactly simulate $\rm{Ref}_A$, 
%$\rm{Ref}_B$, 
the referee needs to know which measurement Alice made. If the number of measurements is limited, then we could make Alice send which measurement she did. The number of measurements corresponds to the number of unitary channels that must be in a convex combination to obtain any mixed unitary channel \cite{CD04,DPP05}. Moreover, the number is related to Carath{\'e}odory’s theorem \cite{Car11} and generally large (see Proposition 4.9 in \cite{Wat18}). Therefore, for a valid reduction, Alice and the referee need to share a probabilistic mixture over a large number of alphabets to know which measurement Alice did.

Fortunately, we can show that a variant of Newman's theorem \cite{New91} holds in this setting (\cref{lem:newman}), and we can remove the generally large shared randomness to a small ($O(\log n)$ bits) private randomness and make Alice send it to the referee. Therefore, we have a valid reduction from a quantum one-way-LOCC SMP protocol to the hybrid SMP protocol for any relation. Since $\EQ_n$ is hard for the hybrid SMP protocols, it is also hard for quantum one-way-LOCC SMP protocols.

\subsubsection*{Against two-way LOCC measurements}

Aaronson \cite{Aar05} introduced a method to replace quantum messages by classical deterministic messages in one-way communication protocols. His construction exploits the property that if a 2-value measurement accepts a state with high probability, after the measurement, one can recover a state that is close to the original state just before the measurement (see Lemma 2.2 in \cite{Aar05}). Theorem 3.4 in \cite{Aar05} assumes that the measurement probability is close to 0 or 1.  Gavinsky, Regev and de Wolf \cite{GRdW08} proved a similar argument without this assumption. See also Theorem 4 in \cite{Aar18}.

\begin{theorem}[Theorem 5 in \cite{GRdW08}]
    Suppose Alice has the classical description of an arbitrary $q$-qubit density matrix $\rho$, and Bob has $2^c$ $2$-value POVM operators $\{E_b\}_{b \in \{0,1\}^c}$. For any constant $\delta >0$, there exists a deterministic message of $O(qc\log q)$\footnote{This asymptotic order hides some large constant depending on $\delta$ and holds for any small constant $\delta$. See \cref{lem:replace} for a formal statement.} bits from Alice that enables Bob to output values $p'_b$ satisfying that $|p_b-p'_b| \leq \delta$ simultaneously for all $b \in \{0,1\}^c$ where $p_b = \mathrm{tr}(E_b \rho)$.
\end{theorem}

Our first observation is that the dependence of the message size on the number of measurements in this technique is very good: using a deterministic message, the referee can guess the acceptance probabilities for \emph{all} $2^n$ inputs (this property is also noted in Section 5 in \cite{GRdW08}). The receiver in one-way communication only needs to use one value to estimate acceptance probability. Our idea is to use all the values to simulate quantum two-way-LOCC SMP protocols with not too many rounds.

However, we cannot directly apply the technique because during an LOCC measurement, a number of 2-value measurements will be performed sequentially on the post-measurement state from the previous measurement. To make the idea work, we show that the measurement outcomes at each LOCC iteration can be estimated by taking some ratios of values the referee can estimate from deterministic messages (\cref{lem:ratio}). Then, by carefully evaluating the error caused by the simulation, we have the desired result.

\subsubsection*{Separation between quantum one-way-LOCC and two-way-LOCC SMP protocols}

Bar-Yossef, Jayram and Kerenidis \cite{BYJK08} introduced the Hidden Matching problem to separate classical and quantum one-way communication complexity (this was the first exponential separation between them). By limiting the number of matchings by $\Theta(n)$, they also introduced Restricted Hidden Matching problem and showed an exponential separation between classical and hybrid SMP models.

We introduce a further variant of the Restricted Hidden Matching problem we call Double Restricted Hidden Matching problem, and make both parties send quantum messages, i.e., the problem is hard for hybrid SMP schemes. It is shown that the problem is easy if both quantum messages can be measured adaptively based on the other message, i.e., for quantum two-way-LOCC SMP protocols. Moreover, since we know quantum one-way-LOCC SMP protocols and hybrid SMP protocols are almost equivalent (\cref{thm:one-way_LOCC}), we show that the problem is hard for quantum one-way-LOCC SMP protocols.

\subsection{Related works}\label{sec:related}

Variants with LOCC measurements have also been studied for the complexity class $\QMA(2)$, which was first introduced by Kobayashi, Matsumoto, and Yamakami \cite{KMY03}. Harrow and Montanaro \cite{HM13} showed that, by making the verifier conduct the product test, which is a variant of the SWAP test, $\QMA(2)$ contains the extended class $\QMA(k)$ for any $k\geq 2$ where the verifier receives $k$ separate quantum states for certification. Let us denote $\QMA^\mathbb{M}(k)$ as the class $\QMA(k)$ with the verifier restricted to performing measurements from $\mathbb{M}$. As already noted, the SWAP test is a projection to the symmetric subspace and therefore the corresponding measurement operator is in SEP.

\begin{theorem}[Informal version from \cite{HM13}]
For $k \in \mathbb{N}$,
\[
    \QMA^\mathrm{SEP}(2) \supseteq \QMA(k).
\]
\end{theorem}

In contrast, Brandao, Christandl and Yard \cite{BCY11}, and Li and Smith \cite{LS15} showed that if the verifier's ability is restricted to one-way LOCC measurements, the class is contained in $\QMA$. Note that, to our knowledge, it is still unknown whether $\QMA^{\mathrm{LOCC}}(k) \subseteq \QMA$ is or not.

\begin{theorem}[Informal version from \cite{BCY11,LS15}]
For $k \in \mathbb{N}$,
\[
    \QMA^{\mathrm{LOCC}_1}(k) \subseteq \QMA.
\]    
\end{theorem}

Aaronson, Beigi, Drucker, Fefferman and Shor \cite{ABD+09} showed that 3-SAT can be solved by a $\QMA(k)$ protocol with $k=\Tilde{O}(\sqrt{n})$ and messages of length $O(\log n)$ exploiting the property of the SWAP test. Chen and Drucker \cite{CD10} showed that the result can be obtained even when the verifier's ability is restricted to BELL measurements, a subclass of one-way LOCC measurements. 

Recently, the power of incoherent measurements has been explored from sample complexity in the literature of quantum state learning (see, e.g., \cite{BCL20,HKP21,ACQ22,CLO22,CCHL22,CHLL22,CHL+23,CLL24,CGY24}). The tasks they considered include quantum state tomography \cite{HHJ+16,OW16}, state certification \cite{BOW19}, shadow tomography \cite{Aar18,HKP20}, and purity testing \cite{MdW16}.

Anshu, Landau and Liu \cite{ALL22} considered a task where two players
Alice and Bob who have multiple copies of unknown $d$-dimensional states $\rho$ and $\sigma$ respectively, and they want to estimate the inner product of $\rho$ and $\sigma$ up to additive error $\epsilon$ by classical communication. They showed that for this task, which they call distributed quantum inner product estimation, $\Theta( \max \{ \frac{1}{\epsilon^2}, \frac{\sqrt{d}}{\epsilon} \})$ copies are sufficient and necessary in multiple measurement and communication settings. One may consider that their technique for the lower bound on sample complexity can be applied in our setting, but we cannot. To prove it, they derive a connection with optimal cloning by Chiribella \cite{Chi10} and exploit the property that the states $\rho$ and $\sigma$ can be any states over the Haar measure. Although the SWAP test in the quantum fingerprinting protocol is essentially inner product estimation, the encoding of quantum messages on their inputs is discrete. Hence there could be a cleverer encoding compared to Haar-random states, and we are unable to use techniques from \cite{ALL22}.

\subsection{Outlook}

In this paper, we introduce and investigate the setting in which the referee can conduct only incoherent measurements in the SMP model. We show that quantum one-way-LOCC SMP models are almost equivalent to classical and quantum hybrid SMP models and thus $\EQ_n$ is hard for quantum one-way-LOCC SMP models. Moreover, if we restrict the number of outcomes and rounds, it is also hard for quantum two-way-LOCC SMP models. We also present a separation between quantum one-way and two-way-LOCC SMP models by a relation problem.

However, despite our efforts, it is still unknown whether there exists a quantum fingerprinting protocol without coherent measurements. We believe that a lower bound would hold with more general two-way LOCC measurements. In quantum information theory, separable measurements are often used to obtain lower bounds for LOCC measurements (this is also the case in \cite{ALL22}). This is because separable measurements include LOCC measurements and they are mathematically simple (see \cref{sec:measurement}). We conjecture that equality remains hard even in the SMP setting where the referee can make measurements where both the accept and reject operators are separable. We use SEP-BOTH to denote this class of measurements.\footnote{This is different from the class of measurements SEP described earlier, in which only the accept operator is required to be separable; the SWAP test is in SEP, but not SEP-BOTH.}

\begin{conjecture}\label{conj}
    $\Q^{||,\mathrm{LOCC}}(\EQ_n) \geq \Q^{||,\mathrm{SEP\text{-}BOTH}}(\EQ_n) \geq \Omega(\sqrt{n})$.
\end{conjecture}

\subsection{Organization}

In \cref{sec:prel}, we introduce the preliminary concepts used in our results. In \cref{sec:one-way_locc}, we prove the lower bound against one-way LOCC measurements. In \cref{sec:two-way_locc}, we prove the lower bound against two-way LOCC measurements. In \cref{sec:separation}, we present a separation between quantum one-way-LOCC and two-way-LOCC SMP protocols by a relation problem. 