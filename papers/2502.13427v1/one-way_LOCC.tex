\section{Lower bound against one-way LOCC measurement}\label{sec:one-way_locc}

\subsection{Proof}\label{subsec:proof_1waylocc}

We first show that a variant of Newman's Theorem holds in our setting. We assume that in the hybrid SMP model, Alice sends a classical message and Bob sends a quantum message.
Let us denote by $\R\Q^{||,pub}(F)$ and $\R\Q^{||,pub}(R)$ the complexity in the hybrid SMP model of a function $F$ and a relation $R$, respectively, when Alice and the referee share unlimited randomness.

\begin{lemma}\label{lem:newman}
    Let $R \subseteq X \times Y \times Z$ be any relational problem. For every $\epsilon>0$ and $\delta>0$, $\R\Q_{\epsilon+\delta}^{||}(R) \leq \R\Q_\epsilon^{||,pub} (R) + O(\log (\log |X| + \log |Y|) + \log (\frac{1}{\delta}))$.
\end{lemma}

\begin{proof}
    We will show that any hybrid SMP protocol $\mathcal{P}$ with unlimited random bits between Alice and the referee can be transformed into another hybrid SMP protocol $\mathcal{P}'$ in which Alice and the referee share only $O(\log n + \log (\frac{1}{\delta}))$ random bits while increasing the error by only $\delta$. Since the amount of randomness is small, by making Alice send all the random bits to the referee, we have the desired protocol. Let $\Pi$ be the probabilistic distribution of the shared randomness of the protocol $\mathcal{P}$.
    % The proof then follows because Alice can send her private random bits to Bob and then the two players proceed as in $\mathcal{P}'$.

    Let $V(x, y, r)$ be a random variable which is defined as the probability that $\mathcal{P}$’s output $z \in Z$ on input $(x, y) \in X \times Y$ and random string $r$ shared by Alice and the referee satisfy $(x, y, z) \notin R$. Because $\mathcal{P}$ computes $R$ with $\epsilon$ error, we have $\mathbb{E}_{r \in \Pi} [V(x,y,r)] \leq \epsilon$ for all $(x,y)$. We will build a new protocol that uses fewer random bits, using the probabilistic method. Let $t$ be a parameter, and let $r_1,\ldots,r_t$ be $t$ strings. For such strings, let us define a protocol $\mathcal{P}_{r_1,\ldots,r_t}$ as follows: Alice and the referee choose $1 \leq i \leq t$ uniformly at random and then proceed as in $\mathcal{P}$ with $r_i$ as their common random string. We now show that there exist strings $r_1,\ldots,r_t$ such that $\frac{1}{t} \sum_{i=1}^t [V(x,y,r_i)] \leq \epsilon + \delta$ for all $(x,y)$. For this choice of strings, the protocol $\mathcal{P}_{r_1,\ldots,r_t}$ is the desired protocol.

    To do so, we choose the $t$ values $r_1,\ldots,r_t$ by sampling the distribution $\Pi$ $t$ times. Consider a particular input pair $(x,y)$ and compute the probability that $\frac{1}{t} \sum_{i=1}^t V(x,y,r_i) > \epsilon + \delta$. By Hoeffding's bound (\cref{fact:hoeffding}), since $\mathbb{E}_{r \in \Pi}[V(x,y,r)] \leq \epsilon$, we get
    \[
        \mathrm{Pr} \left[ \left( \frac{1}{t}  \sum_{i=1}^t V(x,y,r_i) - \epsilon \right) > \delta \right] \leq 2e^{-2\delta^2 t}.
    \]
    By choosing $t=O(\frac{ \log |X| + \log |Y| }{\delta^2})$, this is smaller than $|X|^{-1} |Y|^{-1}$. Thus, for a random choice of $r_1,\ldots,r_t$, the probability that for some input $(x,y)$, $\frac{1}{t} \sum_{i=1}^t [V(x,y,r_i)] > \epsilon + \delta$ is smaller than $|X||Y| |X|^{-1} |Y|^{-1} = 1$. This implies that there exists a choice of $r_1,\ldots,r_t$ where for every $(x,y)$ the error of the protocol $\mathcal{P}_{r_1,\ldots,r_t}$ is at most $\epsilon + \delta$. Finally, note that the number of random bits used by the protocol $\mathcal{P}_{r_1,\ldots,r_t}$ is $\log t = O(\log (\log |X| + \log |Y|) + \log (\frac{1}{\delta}))$.
\end{proof}

We also need the result by Oszmaniec, Guerini, Wittek, and Ac{\'i}n \cite{OGWA17}, who showed that any POVM can be simulated by projective measurements with randomization and postprocessing.

\begin{lemma}[Theorem 1 in \cite{OGWA17}]\label{lem:OGWA17}
    Let $\mathbf{S}\mathbb{P}(d^2,nd)$ be the set of PM simulable, $n$-outcome POVMs on $\mathcal{D}(\mathbb{C}^d \otimes \mathbb{C}^d)$. Let $\mathbf{M} \in \mathcal{P}(d,n)$ be an arbitrary $n$-outcome POVM on $\mathcal{D}(\mathbb{C}^d)$ and $\ket{\phi}$ be some fixed pure state on $\mathcal{B}(\mathbb{C}^d)$. Then there exists a PM-simulable POVM $\mathbf{N} \in \mathbf{S}\mathbb{P}(d^2,nd)$ such that $\mathrm{tr}(\rho M_i) = \mathrm{tr}((\rho \otimes \ket{\phi} \bra{\phi}) N_i)$ for $i=1,\ldots,n$ \footnote{Since the ancilla state is fixed, $\mathrm{tr}((\rho \otimes \ket{\phi} \bra{\phi}) N_i)=0$ for $i=n+1,\ldots,nd$. The number of the outcomes of the POVM $\mathbf{N}$ is $nd$ to make $\mathbf{N}$ valid (i.e., $\sum_{i=1}^{nd} N_i = I$).} and for all states $\rho \in \mathcal{D}(\mathbb{C}^d)$.
\end{lemma}

We are now ready to prove the main result of this section.

\begin{theorem}\label{thm:one-way_LOCC}
    Let $R \subseteq X \times Y \times Z$ be any relation. Suppose that there exists a quantum one-way-LOCC SMP protocol $\mathcal{P}$ for $R$ whose message sizes are $a$ and $b$, with $a$ being the size of Alice's message, on which the first measurement is performed by $\rm{Ref}_A$. Then, $\R\Q(R) \leq 2a + b + O(\log (\log |X| + \log |Y|))$.
\end{theorem}

\begin{proof}
    We give a new hybrid SMP protocol $\mathcal{P}'$ for $R$ with shared randomness between Alice and the referee constructed from the protocol $\mathcal{P}$. Let $\mathbf{M}$ be the $m$-outcome POVM performed by $\rm{Ref}_A$ and let $\rho_x$ be a quantum message from Alice to $\rm{Ref}_A$ in the protocol $\mathcal{P}$. From \cref{lem:OGWA17}, there exist a fixed pure state $\ket{\phi} \in \mathcal{B}(\mathbb{C}^{2^a})$ and a PM-simulable POVM $\mathbf{N} \in \mathbf{S}\mathbb{P}(2^{2a},m 2^a)$ such that $\mathrm{tr}(\rho_x M_i) = \mathrm{tr}((\rho_x \otimes \ket{\phi} \bra{\phi}) N_i)$ for $i=1,\ldots,m$. Since $\mathbf{N}$ is a PM-simulable POVM, there exist a set of projectors $\{\mathbf{N}_k\}$ and a probability distribution $\{p_k\}_k$ such that $\mathbf{N} = \sum_k p_k \mathbf{N}_k$. The measurement results of each $\mathbf{N}_k$ on $\rho_x \otimes \ket{\phi} \bra{\phi}$ are represented by $2a$ bits. Consider a hybrid SMP protocol $\mathcal{P}'$ in which Alice and the referee share the probability distribution $\{p_k\}_k$ and Alice performs $\sum_k p_k \mathbf{N}_k$ on $\rho_x \otimes \ket{\phi}\bra{\phi}$ and sends the measurement result of $2a$ bits. Then, the referee can recover the original result of $\mathbf{M}$ from the $2a$ bits message from Alice because the referee knows the index $k$ from the shared randomness with Alice, and thus simulates $\rm{Ref}_B$ exactly.
    
    Let us apply \cref{lem:newman} for the protocol $\mathcal{P}'$ by taking $\delta$ as a sufficiently small constant. We then obtain a private-coin hybrid SMP protocol $\mathcal{P}''$ to solve $R$ with bounded error and the complexity of the protocol $\mathcal{P}''$ is $2a + O(\log (\log |X| + \log |Y|)) + b$.
\end{proof}

Since for a Boolean function $F:\{0,1\}^n \times \{0,1\}^n \rightarrow \{0,1\}$, $|X|=|Y| = 2^n$, from the lower bound in the hybrid scheme for $\EQ_n$ (\cref{thm:lower_bound_hybrid}), we have a lower bound in quantum one-way-LOCC SMP protocols for $\EQ_n$.

\begin{corollary}\label{cor:EQ}
    $\Q^{||,\mathrm{LOCC}_1}(\EQ_n) = \Omega(\sqrt{n})$.
\end{corollary}

Since $\mathrm{BELL} \subseteq \mathrm{LOCC_1}$, we also have a lower bound with the BELL measurements.

\begin{corollary}
    $\Q^{||,\mathrm{BELL}}(\EQ_n) = \Omega(\sqrt{n})$.
\end{corollary}

Moreover, we can obtain a more general result for any Boolean function. For any Boolean function $F:\{0,1\}^n \times \{0,1\}^n \rightarrow \{0,1\}$, it is known that $\R^{||}(F) \leq \D^{||}(F)^2$ \cite{BK97}, and $\R\Q^{||}(F) \leq \R^{||}(F)^2$ \cite{GRdW08}. Combining this with \cref{thm:one-way_LOCC}, we have the following claim.

\begin{corollary}\label{cor:quartic}
    For any Boolean function $F:\{0,1\}^n \times \{0,1\}^n \rightarrow \{0,1\}$, if $\D^{||}(F) = \Omega(n^c)$ for a constant $c > 0$, $\Q^{||,\mathrm{LOCC_1}} (F) = \Omega(n^{\frac{c}{4}})$.
\end{corollary}

\subsection{Implication for quantum incoherent one-way communication complexity}

By applying the technique developed in \cref{subsec:proof_1waylocc}, we show that, in the setting of the quantum incoherent one-way communication protocols, quantum messages can be replaced by classical messages with very small overhead.
\begin{corollary}\label{cor:incoherent}
    Let $R \subseteq X \times Y \times Z$ be any relation. $\R^1(R) \leq 2 \Q^{1,\perp}(R) + O(\log (\log |X| + \log |Y|)$.
\end{corollary}
\begin{proof}
    Let $\mathcal{P}$ be an incoherent quantum one-way communication protocol for $R$. Let $a$ be the qubit size from Alice to Bob. Let $\mathbf{M} = \{M_i\}_{i \in [m]}$ be the $m$-outcome POVM performed by Bob and let $\rho_x$ be a quantum message from Alice to Bob in the protocol $\mathcal{P}$. From \cref{lem:OGWA17}, there exist a fixed pure state $\ket{\phi} \in \mathcal{B}(\mathbb{C}^{2^a})$ and a PM-simulable POVM $\mathbf{N} \in \mathbf{S}\mathbb{P}(2^2a,m2^a)$ such that $\mathrm{tr}(\rho_x M_i) = \mathrm{tr}((\rho_x \otimes \ket{\phi} \bra{\phi}) N_i)$ for $i=1,\ldots,m$. Since $\mathbf{N}$ is a PM-simulable POVM, there exist a set of projectors $\{\mathbf{N}_k\}$ and a probability distribution $\{p_k\}_k$ such that $\mathbf{N} = \sum_k p_k \mathbf{N}_k$. The measurement result of each $\mathbf{N}_k$ on $\rho_x \otimes \ket{\phi} \bra{\phi}$ is represented by $2a$ bits. Consider a classical one-way communication protocol $\mathcal{P}'$ in which Alice and Bob share the probability distribution $\{p_k\}_k$ and Alice performs $\sum_k p_k \mathbf{N}_k$ on $\rho_x \otimes \ket{\phi}\bra{\phi}$ and sends the measurement result with $2a$ bits. Then, Bob can recover the original result of $\mathbf{M}$ from the $2a$ bits message from Alice because Bob knows the index $k$ from the shared randomness with Alice.
    We then apply a variant of the Newman's theorem proven by the same way as \cref{lem:newman}, and we obtain a classical one-way communication protocol whose complexity is $2a + O(\log (\log |X| + \log |Y|))$, which concludes the proof.
\end{proof}