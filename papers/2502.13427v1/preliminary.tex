\section{Preliminaries}\label{sec:prel}

We assume that readers are familiar with basic notions of quantum computation and information. We refer to \cite{NC10,Wat18,dW19} for standard references.

For a Hilbert (finite-dimensional complex Euclidean) space $\mathcal{H}$ whose dimension is $d$, $\mathcal{B}(\mathbb{C}^d)$ and $\mathcal{D}(\mathbb{C}^d)$ denote the sets of pure and mixed states over $\mathcal{H}$ respectively.

Here are basic probabilistic inequalities.

\begin{fact}[Markov's inequality]\label{fact:markov}
    For a nonnegative random variable $X$ and $a>0$,
    \[
        \Pr[X \geq a] \leq \frac{\E[X]}{a}.
    \]
\end{fact}
\begin{fact}[Hoeffding's bound \cite{Hoe63}]\label{fact:hoeffding}
    Let $x_1,\ldots,x_t$ be independent random variables such that $0 \leq x_i \leq 1$ for $i \in [t]$. Then,
    \[
        \mathrm{Pr}\left[ \left| \frac{1}{t} \left(\sum_{i=1}^t x_i - \mathbb{E} \left[ \sum_{i=1}^t x_i \right] \right) \right| \geq \delta \right] \leq 2e^{-2 \delta^2 t}.
    \]
\end{fact}

\subsection{Communication complexity}
Let us recall some basic definitions of quantum communication complexity. As standard references, we refer to \cite{KN96,RY20} for classical communication complexity and \cite{BCMdW10} for quantum communication complexity and the simultaneous message passing (SMP) model.

The goal in communication complexity is for Alice and Bob to compute a function $F : \mathcal{X} \times \mathcal{Y} \to \{0,1,\perp\}$. We interpret $1$ as ``accept'' and $0$ as ``reject'' and mostly consider $\mathcal{X}=\mathcal{Y}=\{0,1\}^n$. In the computational model, Alice receives an input $x \in \mathcal{X}$ (unknown to Bob), and Bob receives an input $y \in \mathcal{Y}$ (unknown to Alice) promised that $(x,y) \in \mathsf{dom}(F)=F^{-1}(\{0,1\})$. We say $F$ is a total function if $F(x,y) \in \{0,1\}$ for all $x \in \mathcal{X}$ and $y \in \mathcal{Y}$, and $F$ is a partial function otherwise. The bounded error communication complexity of $F$ is defined as the minimum cost of classical or quantum communication protocols to compute $F(x,y)$ with high probability, say $\frac{2}{3}$.

In some settings, it is useful to consider multiple correct outputs for each input. We call such problems relational problems and denote them as $R \subseteq X \times Y \times Z$ where $(x,y) \in X \times Y$ is an input and $z \in Z$ such that $(x,y,z) \in R$ are correct outputs. We can consider a relational problem as a generalization of functions. This is because from a function $F : X \times Y \rightarrow \{0,1, \perp \}$, we can construct a relation $R = \{ (x,y,F(x,y)) \}$ for all $x \in X$ and $y \in Y$ such that $(x,y) \in F^{-1}(\{0,1\})$. Therefore, we state some of our statements for relations, and they also hold for functions.

The simultaneous message passing (SMP) model is a specific model of communication protocols. In this model, Alice and Bob send a single (possibly deterministic, randomized or quantum) message to a referee. The goal of the referee is to output $z$ such that $(x,y,z) \in R$ with high probability, say at least $\frac{2}{3}$. The complexity measure of the protocol is the total amount (bits or qubits) of messages the referee receives from Alice and Bob.  

In this paper, $\D^{||}(R)$ denotes the classical deterministic SMP communication complexity for $R$, $\R^{||}(R)$ denotes the classical randomized SMP communication complexity for $R$, and $\Q^{||}(R)$ denotes the quantum SMP communication complexity for $R$. We also consider a hybrid SMP model in which Alice or Bob sends a quantum message and the other party sends a classical message, and let us denote by $\R\Q^{||}(R)$ the hybrid SMP communication complexity for $R$. There is no shared randomness or entanglement between the parties in these definitions, although we will later introduce definitions with shared randomness. We also denote by $\Q^{||,\mathbb{M}}(R)$ the complexity of the quantum SMP for $R$ where the referee is restricted to performing measurements from $\mathbb{M}$. Several measurement operators will be formally defined in \cref{sec:measurement}.

A basic function considered in communication complexity is the equality function $\EQ_n:~\{0,1\}^n\times\{0,1\}^n\rightarrow \{0,1\}$, 
which is defined as $\EQ_n(x,y)=1$ if $x=y$ and $0$ otherwise.

Let us review the construction of the quantum fingerprint in \cite{BCWdW01}. For a constant $c$, let $E : \{0,1\}^n \to \{0,1\}^N$, where $N=O(n)$, be an error correcting code such that $E(x)$ and $E(y)$ have Hamming distance greater than $cN$ for any distinct $x$ and $y$. Let $E_i(x)$ be the $i$-th bit of $E(x)$. For $x \in \{0,1\}^n$, the quantum fingerprint is defined as $\ket{h_x} = \frac{1}{\sqrt{N}} \sum_i \ket{i}\ket{E_i(x)}$ \cite{BCWdW01}. We can alternatively regard $\ket{h_x}$ as a superposition of hash functions. Let us consider the protocol where Alice and Bob create $\ket{h_x}$ and $\ket{h_y}$ on inputs $x$ and $y$ respectively, and send them to the referee. Since the acceptance probability of the SWAP test is $\frac{1}{2} + \frac{|\braket{h_x | h_y}|^2}{2}$, the referee can distinguish $x=y$ or $x \neq y$ using constantly many copies of quantum fingerprints $\ket{h_x}$ and $\ket{h_y}$ with high probability \cite{BCWdW01}.

Let us recall the formal definition of the standard quantum one-way communication complexity.

\begin{definition}[Quantum one-way communication complexity, $\Q^1(R)$]\label{def:Q_one-way}
    Given a relation $R \subseteq X \times Y \times Z$, Alice has a part of the input $x \in X$ and Bob has the other part of the input $y \in Y$. Alice produces a quantum state $\rho_x$ and sends it to Bob. Then, Bob performs a POVM $\{M^y_{z}\}_{z \in Z}$ on $\rho_x$. We say that a quantum one-way communication protocol computes a relation $R \subseteq X \times Y \times Z$, if for all inputs $(x,y) \in X \times Y$ the measurement outputs $z$ such that $(x,y,z) \in R$ with high probability (say $\frac{2}{3}$). The cost of the protocol is the size of the message $\rho_x$ and we denote by $\Q^1(R)$ the minimum cost for all such protocols to compute $R$.
\end{definition}

We denote by $\Q^{1,*}(R)$ a variant of quantum one-way communication complexity where Alice and Bob share (unlimited) prior entanglement. We also consider classical one-way communication complexity and denote by $\D^1(R)$ (${\rm resp.}$ $\R^1(R)$) the classical deterministic (${\rm resp.}$ randomized) one-way communication complexity for $R$.

Newman \cite{New91} showed that any public-coin randomized communication protocol can be converted into a small private-coin randomized protocol allowing for a small constant error. For a function $F:X \times Y \rightarrow \{0,1\}$ and $c>0$, let us denote by $\R_\gamma(F)$ the communication complexity with private randomness and error $\gamma$, and let us denote by $\R_\gamma^{pub}(F)$ the communication complexity with public unlimited randomness and error $\gamma$. 

\begin{fact}[\cite{New91}]
    Let $F: X \times Y \rightarrow \{0,1\}$ be a function. For every $\epsilon>0$ and $\delta>0$, $\R_{\epsilon+\delta} (F) \leq \R_\epsilon^{pub} (F) + O(\log (\log |X| + \log |Y|) + \log (\frac{1}{\delta}))$.
\end{fact}

\subsection{Measurement in quantum mechanics}
In this subsection, let us recall the definitions of measurements on quantum states.

Quantum measurements are described by a collection of $\{M_m\}_m$ of measurement operators. These are operators acting on the state space of the system being measured. The index $m$ refers to the measurement outcomes that may occur in the experiment. If the state of the quantum system is $\rho$ immediately before the measurement, then the probability that the result $m$ occurs is given by $\mathrm{tr}(M_m^\dagger M_m \rho)$. The state after the measurement is $\frac{M_m \rho M_m^\dagger}{\mathrm{tr}(M_m^\dagger M_m \rho)}$. The measurement operators satisfy the completeness equation, $\sum_m M_m^\dagger M_m = I$.

When we do not care about residual states after measurements, we have a simpler description. Let us define $E_m:=M_m^\dagger M_m$. The collection $\{E_m\}$ is sufficient to determine the probabilities $\mathrm{tr}(E_m\rho)$ of measurement outcomes. This is called a POVM (Positive Operator-Valued Measurement): A POVM on $\mathcal{D}(\mathbb{C}^d)$ with $n$ outcomes is a collection $\mathbf{E}=\{E_1,\ldots,E_n\}$ of non-negative operators satisfying $\sum_i E_i = I$. We denote the set of POVMs on $\mathcal{D}(\mathbb{C}^d)$ with $n$ outcomes by $\mathcal{P}(d,n)$. Given two POVMs $\mathbf{E}=\{E_i\},\mathbf{F}=\{F_i\} \in \mathcal{P}(d,n)$, their convex combination $p \mathbf{E} + (1-p) \mathbf{F}$ is the POVM with the $i$th element given by $p E_i + (1-p) F_i$. A projective measurement is a POVM whose effects are orthogonal projectors. We denote by $\mathbb{P}(d,n)$ the set of projective POVM, which is a subset of $\mathcal{P}(d,n)$.

There are two techniques to simulate quantum measurements. The first is to manipulate quantum measurements randomly by considering a convex combination of POVMs $\{ \mathbf{N}_k \}$, $\mathbf{N}=\sum_k p_k \mathbf{N}_k$. Classical post-processing of a POVM $\mathbf{N}$ is a strategy in which, upon obtaining an output $j$, one produces the final output $i$ with probability $q(i|j)$. For a given post-processing strategy $q(i|j)$, this procedure gives a POVM $\mathcal{Q}(\mathbf{N})$ with elements given by $[\mathcal{Q}(\mathbf{N})]_i := \sum_j q(i|j)N_j$. We say that a POVM $\mathbf{M} \in \mathcal{P}(d,n)$ is PM-simulable if and only if it can be realized by classical randomization followed by classical postprocessing of some projective measurements.

We say $\{M_i\}_{i \in [N] }$ are 2-value POVM operators if $\{M_i,I-M_i\}$ is a valid 2-value POVM for $i \in [N]$, i.e., $0 \leq M_i \leq I$. For a 2-value POVM operator $M$, we say the corresponding 2-value POVM $\{M,I-M\}$ on some quantum state succeeds if the measurement outcome corresponds to $M$. 

\subsection{Classes of measurements on bipartite states}\label{sec:measurement}

Let us formally introduce some specific classes of 2-value POVMs acting on bipartite states. See Appendix C in \cite{HM13} for a reference. Each class of measurement operators describes operators on $\mathbb{C}^d \otimes \mathbb{C}^d$, and for a 2-value POVM $\{M, I-M\}$, $M$ corresponds to ``accept'' and $I-M$ corresponds to ``reject''.

\begin{definition}
    We say a 2-value POVM $\{M, I-M\} \in \mathrm{BELL}$ if $M$ can be expressed as 
    \[
        M = \sum_{(i,j)\in S} \alpha_i \otimes \beta_j,
    \]
    where $\sum_i \alpha_i = I$ and $\sum_j \beta_j = I$, and $S$ is a set of pairs of indices.
\end{definition}

In other words, with BELL measurements, systems are measured locally, and the verifier accepts based on the measurement results.

Let us next introduce some classes where adaptive measurements are allowed. LOCC is the abbreviation for local operations and classical communication.
\begin{definition}
     We say a 2-value POVM $\{M, I-M\} \in \mathrm{LOCC}_1$ if $M$ can be realized by measuring the first system and then choosing a measurement on the second system conditional on the result of the first measurement.  Such $M$ can be written as
     \[
        M = \sum_i \alpha_i \otimes M_i,
     \]
    where $\sum_i \alpha_i = I$ and $0\leq M_i \leq I$ for each $i$.
\end{definition}

\begin{definition}
     We say a 2-value POVM $\{M, I-M\} \in \mathrm{LOCC}$ is if $M$ that can be realized by alternating partial measurements on the two systems a finite number of times, choosing each measurement conditioned on the previous outcomes. An inductive definition is that M is in $\mathrm{LOCC}$ if there exist operators $\{E_i\}, \{M_i\}$, with $\sum_i E_i\leq I$ and each $M_i\in{\rm LOCC}$, such that either $M = \sum_i (\sqrt{E_i}\otimes I)M_i(\sqrt{E_i}\otimes I)$ or $M = \sum_i (I\otimes \sqrt{E_i})M_i(I\otimes \sqrt{E_i})$.   For the base case, it suffices to take $I\in {\rm LOCC}$.
\end{definition}

We finally introduce a set of operators that can be written as a sum of separable operators.

\begin{definition}
     We say a 2-value POVM $\{M, I-M\} \in \mathrm{SEP}$ if
    \begin{equation*}
        M = \sum_i \alpha_i \otimes \beta_i
    \end{equation*}
    for some positive semidefinite matrices $\{\alpha_i\},\{\beta_i\}$.
\end{definition}

\begin{definition}
    We say a 2-value POVM $\{M, I-M\} \in \mathrm{SEP\text{-}BOTH}$ if $M$ and $I-M$ both have the form $\sum_i\alpha_i\otimes\beta_i$.
\end{definition}

\begin{definition}
    $\mathrm{ALL}$ is any $M$ such that $0 \leq M \leq I$.
\end{definition}

Note that SEP-BOTH is a natural relaxation of LOCC
because they preserve the property that both $M$ and $I-M$ must be
realizable through local operations and classical communication.  On
the other hand, SEP is more natural when we consider $M$ by
itself and do not wish to consider additional constraints on $I-M$.

These sets satisfy the following inclusions

$$
\begin{array}{ccccccccccc}
{\rm BELL} &\subset& {\rm LOCC}_1 &\subset& {\rm LOCC}& \subset &
\text{SEP-BOTH} & \subset & {\rm SEP} & \subset & {\rm ALL}.
\end{array}
$$

\subsection{Quantum LOCC SMP protocols}

In this section, let us present the definitions of quantum one-way-LOCC and two-way-LOCC SMP protocols more explicitly. For clarity, let us decompose the referee into the two referees $\rm{Ref}_A$ and $\rm{Ref}_B$ who receive quantum messages from Alice and Bob respectively. In the definitions, we assume that $\rm{Ref}_A$ conducts the first measurement.

\begin{definition}[Quantum one-way-LOCC SMP protocols, $\Q^{||,\mathrm{LOCC}_1}(R)$]
    For a relation $R \subseteq X \times Y \times Z$, Alice has a part of the input $x \in X$ and Bob has the other part of the input $y \in Y$. Alice sends a quantum state $\rho_x$ to $\rm{Ref}_A$ and Bob sends a quantum state $\sigma_y$ to $\rm{Ref}_B$. $\rm{Ref}_A$ conducts some POVM measurement on $\rho_x$ and sends a measurement result to $\rm{Ref}_B$. $\rm{Ref}_B$ conducts some POVM measurement on $\sigma_y$ based on the classical message from $\rm{Ref}_A$. The measurement result of $\rm{Ref}_B$ is the output of the whole communication protocol. We call this protocol a quantum one-way-LOCC protocol for $R$, and say that a quantum one-way-LOCC SMP protocol computes a relation $R \subseteq X \times Y \times Z$, if for all inputs $(x,y) \in X \times Y$, $\rm{Ref}_B$ computes $z \in Z$ such that $(x,y,z) \in R$ with high probability (say $\frac{2}{3}$). The complexity $\Q^{||,\mathrm{LOCC}_1}(R)$ is the sum of sizes of the two messages from Alice and Bob, i.e., $|\rho_x|+|\sigma_y|$.
\end{definition}

\begin{definition}[Quantum two-way-LOCC SMP protocols, $\Q^{||,\mathrm{LOCC}}(R)$]
    For a relation $R \subseteq X \times Y \times Z$, Alice has a part of the input $x \in X$ and Bob has the other part of the input $y \in Y$. Alice sends a quantum state $\rho_x$ to $\rm{Ref}_A$ and Bob sends a quantum state $\sigma_y$ to $\rm{Ref}_B$. $\rm{Ref}_A$ measures $\rho_x$ and sends a measurement result to $\rm{Ref}_B$. Then, $\rm{Ref}_B$ measures $\sigma_y$ based on the measurement result of $\rho_x$ and sends a measurement result to $\rm{Ref}_A$. Then, $\rm{Ref}_A$ measures the residue state of $\rho_x$ based on all the previous measurement results. After iterating these measurements and communication, $\rm{Ref}_A$ or $\rm{Ref}_B$ finally outputs $z \in Z$ based on all the previous measurement outcomes. We call this protocol a quantum two-way-LOCC protocol for $R$, and say that a quantum two-way-LOCC SMP protocol computes a relation $R \subseteq X \times Y \times Z$, if for all inputs $(x,y) \in X \times Y$,  $\rm{Ref}_A$ or $\rm{Ref}_B$ outputs $z \in Z$ such that $(x,y,z) \in R$ with high probability (say $\frac{2}{3}$). The complexity $\Q^{||,\mathrm{LOCC}}(R)$ is the sum of two messages from Alice and Bob, i.e., $|\rho_x|+|\sigma_y|$.
\end{definition}

\subsection{Quantum incoherent one-way communication protocols}

In this paper, we introduce a new quantum one-way communication protocol. There is a quantum channel from Alice to Bob, and Bob's quantum computer can indeed receive quantum messages from Alice. It can perform some complex measurement on the message state, but the measurement cannot depend on his input and it cannot keep residue states after the first measurement. Let us call complexity in this scenario quantum incoherent one-way communication complexity, and denote it by $\Q^{1,\perp}(R)$.

\begin{definition}[Quantum incoherent one-way communication complexity, $\Q^{1,\perp}(R)$]\label{def:Q_one-way_incoherent}
    Given a relation $R \subseteq X \times Y \times Z$, Alice has a part of the input $x \in X$ and Bob has the other part of the input $y \in Y$. Alice produces a quantum state $\rho_x$ and sends it to Bob. Then, Bob performs some POVM $\{M_m\}_m$ on $\rho_x$ and then outputs $z \in Z$ based on $m$ and $y$. We say that a quantum one-way communication protocol computes a relation $R \subseteq X \times Y \times Z$, if for all inputs $(x,y) \in X \times Y$, Bob computes $z \in Z$ such that $(x,y,z) \in R$ with high probability (say $\frac{2}{3}$). The cost of the protocol is the size of the message $\rho_x$ and we denote by $\Q^{1,\perp}(R)$ the minimum cost for all such communication protocols to compute $R$.
\end{definition}

Note that $\Q^{1,\perp}(R) \leq \R^1(R)$ because classical messages can be encoded in quantum messages with the computational basis and Bob measures it with the basis.