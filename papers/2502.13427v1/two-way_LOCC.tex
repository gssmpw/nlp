\section{Lower bound against two-way LOCC measurement}\label{sec:two-way_locc}

In \cref{subsec:replace}, we first describe how to replace quantum messages by deterministic messages. To illustrate our idea and analysis more clearly, we will give a proof of a lower bound in quantum two-value two-round LOCC SMP protocols as a first step in \cref{subsec:warm-up}. We then prove our lower bound in quantum two-value multiple-round LOCC SMP protocols in \cref{subsec:many_rounds}.

\subsection{Replacing quantum messages by classical messages}\label{subsec:replace}

\begin{lemma}\label{lem:replace}
    Suppose Alice has the classical description of an arbitrary $q$-qubit density matrix $\rho$, and Bob has $2^c$ $2$-value POVM operators $\{E_b\}_{b \in \{0,1\}^c}$. For any $\delta >0$, there exists a deterministic message of $O(\frac{q}{\delta^3} \log (\frac{q}{\delta}) (c+ \log \frac{1}{\delta}) )$ bits from Alice that enables Bob to output values $p'_b$ satisfying that $|p_b-p'_b| \leq \delta$ simultaneously for all $b \in \{0,1\}^c$ where $p_b = \mathrm{tr}(E_b \rho)$.
\end{lemma}

For completeness, we will give a proof of this theorem. See also Theorem 5 in \cite{GRdW08}, Theorem 4 in \cite{Aar18}, and Theorem 6 in \cite{ACH+18}. We will need the quantum union bound for the proof.

\begin{lemma}[Quantum union bound, see, e.g., Lemma 3.1 in \cite{Wil13}]\label{lem:quantum_union_bound}
    Suppose that for a quantum state $\rho$, we conduct a sequence of 2-value POVM operators $\{M_i\}_{i\in[k]}$ such that $\mathrm{tr}(M_i \rho) \geq 1 - \delta$. Then, the probability that all the measurements succeed is at least $1-2\sqrt{k\delta}$.
\end{lemma}

\begin{proof}[Proof of \cref{lem:replace}]
    Suppose that Alice sends $r$ many copies of her state.
    Let $\rho'=\rho^{\otimes r}$ be the state she sends, and $r q$ is the total number of qubits. Define the operator
    \[
        F_b=\frac{1}{r}\sum_{j=1}^r E_b^{(j)},
    \]
    where $E_b^{(j)}$ applies $E_b$ to the $j$th copy. This operator gives the fraction of successes if you separately measure each of the $r$ copies of $\rho$ with $E_b$. Hoeffding's bound (\cref{fact:hoeffding}) implies that the outcome $p_b'$ of this measurement applied to $\rho'$ will probably be close to its expectation $p_b=\mathrm{tr}(E_b\rho)$:
    \begin{equation}\label{eq:use_chernoff}
        \Pr[|p'_b-p_b|>\delta/4]\leq 2 e^{-\frac{\delta^2 r}{8}}.
    \end{equation}

    Let us show what is Alice's classical message. Consider all $b=1,\ldots,2^c$ in order. We will sequentially build $rq$-qubit density matrices $\rho_b$, one for each $E_b$.
    Alice's classical message will enable Bob to reconstruct this entire sequence. Let us say $b$ is \emph{good} if $|\mathrm{tr}(F_b \rho_b)-p_b| \leq \delta$, and say $b$ is \emph{bad} otherwise. Note that if Bob has a classical description of a good $\rho_b$, then he can approximate $p_b$ to within $\pm \delta$ (since he knows what $F_b$ is). We start with the completely mixed state: $\rho_1=\frac{I}{2^{rq}}$ and define the subsequent $\rho_b$ one by one, as follows. If $b$ is good, then define $\rho_{b+1}$ as equal to $\rho_b$. If $b$ is bad, Alice adds the pair $(b,\widetilde{p}_b)$ to her message, where $\widetilde{p}_b$ is the $\log(1/\delta)+O(1)$ most significant bits of $p_b$, and then $|\widetilde{p}_b-p_b| \ll \delta$. In this case, let $M_b$ be the projector on the subspace spanned by the eigenvectors of $F_b$ with eigenvalues in the interval $[\widetilde{p}_b-\delta/2,\widetilde{p}_b+\delta/2]$, and let $\rho_{b+1}$ be the renormalized projection of $\rho_b$ on this subspace. Note that we will later show $\mathrm{tr}(M_b \rho_b)$ is nonzero in \cref{eq:lower_bound} and thus this renormalized projection is well defined. Continuing all the way to $b=2^c$, we obtain a message $(b_1,\widetilde{p}_{b_1}),\ldots,(b_T,\widetilde{p}_{b_t})$ for some $t$. We need to show two things: (i) this message enables Bob to approximate all $p_b$ to within $\pm \delta$, and (ii) $t=O(rq)$, which implies that the message length is 
    \begin{equation}\label{eq:total_length}
        O\left(t \left( c+\log \frac{1}{\delta} \right) \right)
    \end{equation}    
    bits. We will show these two things in turn.

    First, we will show (i): the construction of the messages works. Note that Bob knows which $b\in[2^c]$ are bad, since those $b$ are exactly the ones in Alice's message. Bob can in fact compute the whole sequence $\rho_1,\ldots,\rho_{2^c}$ given the message: $\rho_1=\frac{I}{2^{rq}}$; if $b$ is good, then $\rho_{b+1}=\rho_b$; if $b$ is bad, then $(b,\widetilde{p}_b)$ is part of Alice's message and $\rho_{b+1}$ can be computed from this information. Suppose that Bob wants to approximate $p_b=\mathrm{tr}(E_b\rho)$. If $b$ is good then by definition $|\mathrm{tr}(F_b\rho_b)-p_b| \leq \delta$ and Bob can calculate $\mathrm{tr}(F_b\rho_b)$. If $b$ is bad, then the pair $(b,\widetilde{p}_b)$ is part of Alice's message, so Bob knows $p_b$ with sufficient precision. Hence Bob can approximate all $p_b$ up to $\pm \delta$, for all $b$ simultaneously.

    Second, we will show (ii): $t=O(rq)$. Define $\eta=1-\delta/4$ and assume $t \leq (\frac{q}{\delta})^{10}$ (this assumption will be justified later). We consider the sequence $b_1,\ldots,b_t$ of the first $t$ bad $b$'s.
    Let
    \[
        p=\mathrm{tr} \left( M_{b_t} \cdots M_{b_1} \frac{I}{2^{rq}} M_{b_1} \cdots M_{b_t} \right)
    \]
    be the probability that all $t$ measurements succeed if we start with the completely mixed state and sequentially measure $M_{b_1},\ldots,M_{b_t}$. 

    We will upper bound and lower bound $p$ and do the upper bound first. If we sequentially measure $M_{b_1},\ldots,M_{b_t}$, starting from the completely mixed state, and if all $t$ measurements succeed, then we exactly have the sequence of density matrices $\rho_{b_1}= \frac{I}{2^{rq}},\ldots,\rho_{b_t},\rho_{b_{t+1}}$. 
    We will show the claim that if $\rho_b$ is bad, then $\mathrm{tr}(M_b\rho_b)\leq \eta$. Let $X$ denote the random variable representing the outcome of measuring $\rho_b$ with the observable $F_b$. Note that $X$ takes values in $[0,1]$. Assume $\mathrm{tr}(M_b\rho_b)=\Pr[|X-\widetilde{p}_b|\leq\delta/2] > \eta$. Let us evaluate the value $\mathrm{tr}(F_b\rho_b) = \E[X]$. From the assumption, we have
    \[
        \Pr [ X \geq \widetilde{p}_b - \delta/2 ] \geq \Pr[|X-\widetilde{p}_b|\leq\delta/2] > \eta.
    \]
    Then, from Markov's inequality (\cref{fact:markov}), if $\widetilde{p}_b - \delta/2 >0$, we have
    \[
        \E[X] > \eta (\widetilde{p}_b - \delta/2).
    \]
    Otherwise, it is trivial that $\E[X] \geq 0 \geq \eta (\widetilde{p}_b - \delta/2)$. Since $X$ takes values in $[0,1]$, by a similar discussion, we have
    \[
        \E[X] \leq \eta (\widetilde{p}_b + \delta/2) + 1 - \eta.
    \]
    Since $|\widetilde{p}_b-p_b| \ll \delta$ and thus $\widetilde{p}_b \leq 1 + \delta/100$, $\mathrm{tr}(F_b\rho_b) = \E[X]$ must necessarily be in the range
    \[
        [\eta (\widetilde{p}_b - \delta/2), \eta (\widetilde{p}_b + \delta/2) + 1-\eta] \subseteq [\widetilde{p}_b - 3\delta/4, \widetilde{p}_b + 3\delta/4] \subseteq [p_b - \delta, p_b + \delta],
    \]
    and hence $\rho_b$ is good. Thus we have the claim by contraposition, and then the probability that all $t$ measurements succeed is 
    \begin{equation}\label{eq:upper_bound}
        p\leq \eta^t.
    \end{equation}

    Now we lower bound on $p$. Note that $M_b$ succeeds on $\rho'$ if and only if the outcome $p'_b$ of the observable $F_b$ is at most $\delta/2$ away from the number $\widetilde{p}_b$, which is the truncated version of $p_b=\mathrm{tr}(E_b\rho)$ (recall $|\widetilde{p}_b-p_b|\ll \delta$). Hence by \cref{eq:use_chernoff}, we have
    \begin{equation}\label{eq:nonzero}
        \mathrm{tr}(M_b\rho')=\Pr[|p'_b-\widetilde{p}_b|\leq\delta/2]\geq \Pr[|p'_b-p_b|\leq\delta/4]\geq 1-2 e^{-\frac{\delta^2 r}{8}}.
    \end{equation}
    This allows us to measure $\rho'$ with $M_b$ while disturbing the state by only an insignificant amount. If we measure each $M_b$, for the first $t$ bad $b$'s in sequence, starting from $\rho'$, then from \cref{lem:quantum_union_bound} with probability at least 
    \[
        1-2\sqrt{2t e^{-\frac{\delta^2 r}{8}}}
    \]
    all measurements will succeed. Set $r:=\frac{C}{\delta^2}\log\frac{q}{\delta}$ for a sufficiently large constant $C$. Then we have
    \[
        1-2\sqrt{2t e^{-\frac{\delta^2 r}{8}}} = 1-2\sqrt{t \frac{\delta^{10}}{400 q^{10}}} \geq \frac{1}{2},
    \]
    where we use the assumption $t \leq (\frac{q}{\delta})^{10}$. Moreover, the completely mixed state can be written as $\frac{I}{2^{rq}}=\frac{1}{2^{rq}}\rho'+(1-\frac{1}{2^{rq}})\rho''$ where $\rho''$ is orthogonal to $\rho'$. Hence, if we start from $\frac{I}{2^{rq}}$, then the probability of all measurements succeeding is 
    \begin{equation}\label{eq:lower_bound}
        p\geq \frac{1}{2^{rq+1}}.
    \end{equation}

    Combining the upper bound \cref{eq:upper_bound} and lower bound \cref{eq:lower_bound}, we have
    \[
        \frac{1}{2^{rq+1}} \leq \left(1-\frac{\delta}{4} \right)^t.
    \]
    Since $rq = O(\frac{q}{\delta^2} \log \frac{q}{\delta})$, we have
    \[
        t = O\left(\frac{q}{\delta^3} \log \frac{q}{\delta}\right).
    \]
    This bound also justifies the assumption $t \leq (\frac{q}{\delta})^{10}$. Therefore, from \cref{eq:total_length}, the total length of the deterministic message is
    \[
        O\left(\frac{q}{\delta^3} \log \left(\frac{q}{\delta}\right) \left(c+ \log \frac{1}{\delta} \right) \right)
    \]
    bits, as claimed.
\end{proof}

\subsection{Warm-up case: 2-round-LOCC SMP protocols}\label{subsec:warm-up}

\begin{proposition}\label{prop:warm-up}
    Assume that there exists a 2-value 2-round LOCC SMP protocol $\mathcal{P}$ to solve $\EQ_n$ with high probability. Then the number of qubits of Alice's message
    %$\rho_x$ 
    is $\Omega(n/\log n)$.
\end{proposition}

\begin{proof}
    Let $\{\rho_x\}_{x\in \{0,1\}^n}$ be a quantum encoding by Alice of her input $x$, and $\{\sigma_y\}_{y \in \{0,1\}^n}$ be a quantum encoding by Bob of his input $y$.
    Let $r$ be a number of rounds, and $a \in \{0,1\}^{r}$ be previous measurement results by $\rm{Ref}_A$, and $b \in \{0,1\}^{r}$ be previous measurement results by $\rm{Ref}_B$. Let $h$ be all previous measurement results $a_0b_0\cdots a_{r-1}b_{r-1}$. For $m \in \{0,1\}$, let $M_{m|h}$ be a measurement operator of the $(r+1)$th measurement by $\rm{Ref}_A$, for previous measurement results $h$. Without loss of generality, we assume that the first measurement is done by $\rm{Ref}_A$ for $\rho_x$, and let $M_0$ and $M_1$ be the first measurement operators. We also assume, without loss of generality, that the final measurement outcome by $\rm{Ref}_A$ is the output of the protocol because it depends on all the previous measurement outcomes.
    
    For each $r \in \mathbb{N}$, let us denote by $p^A_{m|h}$ the probability that, conditioned on all previous measurement outcomes $h = a_0 b_0\cdots a_{r-1} b_{r-1}$ which consists of measurement outcomes $a \in \{0,1\}^r$ by $\rm{Ref}_A$ and measurement outcomes $b \in \{0,1\}^r$ by $\rm{Ref}_B$, the $(r+1)$th measurement result by $\rm{Ref}_A$ is $m \in \{0,1\}$. Let us denote by $p^B_{m|h}$ the probability that, conditioned on all previous measurement outcomes $h = a_0 b_0\cdots a_{r-1} b_{r-1} a_r$ which consists of measurement outcomes $a \in \{0,1\}^{r+1}$ by $\rm{Ref}_A$ and measurement outcomes $b \in \{0,1\}^r$ by $\rm{Ref}_B$, the $(r+1)$th measurement result by $\rm{Ref}_A$ is $m \in \{0,1\}$. Let us abbreviate $p^A_{m|\emptyset}$ as $p^A_m$. 
     
    Let us also denote by $v_0 = \mathrm{tr}(M_0^\dagger M_0 \rho_x)$, $v_1 = \mathrm{tr}(M_1^\dagger M_1 \rho_x) = 1 - v_0$,
    \begin{align*}
    v_{1|00}=\mathrm{tr}(M^{ \dagger}_0 M_{1|00}^{\dagger} M_{1|00} M_0 \rho_x), \quad &  v_{1|01}=\mathrm{tr}(M_0^\dagger M_{1|01}^{\dagger} M_{1|01} M_0 \rho_x), \\
    v_{1|10}=\mathrm{tr}(M^{\dagger}_1 M_{1|10}^{\dagger} M_{1|10} M_1 \rho_x), \quad & v_{1|11}=\mathrm{tr}(M^{\dagger}_1 M_{1|11}^{\dagger} M_{1|11} M_1 \rho_x).
    \end{align*}
    By definition,
    \[
        p^A_0 = v_0,\ \ p^A_1 = v_1.
    \]
    For example, when the result of the first measurement by $\rm{Ref}_A$ is $1$ and that of the first measurement by $\rm{Ref}_B$ is $0$, the probability that the second measurement result by $\rm{Ref}_A$ is $1$ is 
    \begin{equation}\label{eq:fraction}
        p^A_{1|10} = \mathrm{tr}(M_{1|10}^\dagger M_{1|10} \frac{M_1 \rho_x M^\dagger_1}{\mathrm{tr}(M^\dagger_1 M_1 \rho_x)}) = \frac{\mathrm{tr}(M^\dagger_1 M_{1|10}^\dagger M_{1|10} M_1 \rho_x) }{\mathrm{tr}(M^\dagger_1 M_1 \rho_x)} = \frac{v_{1|10}}{v_1},
    \end{equation}
    where we use the cyclic property of the trace. By the same argument, we have 
     \begin{equation}
         p^A_{1|00} = \frac{v_{1|00}}{v_0},\ \ p^A_{1|01} = \frac{v_{1|01}}{v_0},\ \ p^A_{1|10} = \frac{v_{1|10}}{v_1},\ \ p^A_{1|11} = \frac{v_{1|11}}{v_1}.
     \end{equation}
    
    Since $ p^A_{0|00} +  p^A_{1|00} = 1$, we have
    \[
        \frac{v_{0|00}}{v_0} + \frac{v_{1|00}}{v_0} = 1,
    \]
    which implies 
    \[
         v_0 = v_{0|00} + v_{1|00}.
    \]
    By the same argument, we also have
    \[
         v_0 = v_{0|01} + v_{1|01},\ \ v_1 = v_{0|10} + v_{1|10},\ \ v_1 = v_{0|11} + v_{1|11}.
    \]
    
    Then, the probability that the original LOCC protocol accepts is
    \begin{align*}
        & p^A_{0} \cdot p_{0|0}^B \cdot p^A_{1|00} + p^A_{0} \cdot p_{1|0}^B \cdot p^A_{1|01} + p^A_{1} \cdot p_{0|1}^B \cdot p^A_{1|10} + p^A_{1} \cdot p_{1|1}^B \cdot p^A_{1|11} \\
        &= v_{0} \cdot p_{0|0}^B \cdot \frac{v_{1|00}}{v_0} + v_{0} \cdot p_{1|0}^B \cdot \frac{v_{1|01}}{v_0} + v_{1} \cdot p_{0|1}^B \cdot \frac{v_{1|10}}{v_1} +  v_{1} \cdot p_{1|1}^B \cdot \frac{v_{1|11}}{v_1}\\
        &= p_{0|0}^B \cdot v_{1|00} + p_{1|0}^B \cdot v_{1|01} + p_{0|1}^B \cdot v_{1|10} + p_{1|1}^B \cdot v_{1|11}.
    \end{align*}
    
    Next, we will replace quantum messages $\rho_x$ with classical messages $s_x$ using \cref{lem:replace}. Let $\delta$ be a sufficiently small constant, say $\frac{1}{10^6}$. Let us denote by $q$ the number of qubits of the message $\rho_x$. The operators we will consider are 
    \begin{align*}
        \{E_b\} = \{ M_0^\dagger M_0, M_0^\dagger M_{1|00}^\dagger M_{1|00} M_0, M_0^\dagger M_{1|01}^\dagger M_{1|01} M_0, M_1^\dagger M_{1|10}^\dagger M_{1|10} M_1,  M_1^\dagger M_{1|11}^\dagger M_{1|11} M_1 \}.
    \end{align*}
    From the definition, these operators $\{E_b\}$ satisfy $0 \leq E_b \leq I$.
    Since the number of operators we need to care is 5, the length of the classical string $s_x$ is $O(q \log q)$ bits. Let us consider a hybrid SMP scheme that uses $s_x$ and $\sigma_y$ as two messages where the referee simulates the original LOCC interactions. Let us denote by $v'_0, v'_{1|00}, v'_{1|01}, v'_{1|01}, v'_{1|11}$ the five values the referee guesses using $s_x$. We also define another set of variables $v''_0, \ldots, v''_{1|11}$ that the referee will use.
    \begin{itemize}
\item    If $v'_0 > 1$, $v''_0 = 1$. If $v'_0 < 0$, $v''_0 = 0$. Otherwise, $v''_0 = v'_0$. Let $v''_1 = 1-v''_0$.
    
\item    If $v'_{1|00} > v''_0$, $v''_{1|00} = v''_0$. If $v'_{1|00} < 0$, $v''_{1|00} = 0$. Otherwise, $v''_{1|00} = v'_{1|00}$.
\item    If $v'_{1|01} > v''_0$, $v''_{1|01} = v''_0$. If $v'_{1|01} < 0$, $v''_{1|01} = 0$. Otherwise, $v''_{1|01} = v'_{1|01}$.
\item    If $v'_{1|10} > v''_1$, $v''_{1|10} = v''_1$. If $v'_{1|10} < 0$, $v''_{1|10} = 0$. Otherwise, $v''_{1|10} = v'_{1|10}$.
\item   If $v'_{1|11} > v''_1$, $v''_{1|11} = v''_1$. If $v'_{1|11} < 0$, $v''_{1|11} = 0$. Otherwise, $v''_{1|11} = v'_{1|11}$.
    \end{itemize}
    This adjustment is required to make $v''_{0}$, $v''_{1}$, $\frac{v''_{1|00}}{v''_{0}}$, $\frac{v''_{1|01}}{v''_{0}}$, $\frac{v''_{1|10}}{v''_{1}}$, $\frac{v''_{1|11}}{v''_{1}}$ valid probabilities. 
    
    The referee first simulates the first measurement result by $\rm{Ref}_A$. With probability $v''_0$, the referee simulates the result to be $0$ and with probability $v''_1$ the result to be $1$. Next, the referee measures Bob's message based on the value of the first measurement result obtained by the simulation. Finally, the referee simulates the final measurement result by $\rm{Ref}_A$ using $v''_{1|00},v''_{1|01},v''_{1|10},v''_{1|11}$.
    The acceptance probability of the simulation is
    \begin{align*}
        &v''_{0} \cdot p_{0|0}^B \cdot \frac{v''_{1|00}}{v''_{0}} + v''_{0} \cdot p_{0|0}^B \cdot \frac{v''_{1|01}}{v''_{0}} + v''_{1} \cdot p_{1|0}^B \cdot \frac{v''_{1|10}}{v''_{1}} + v''_{1} \cdot p_{1|1}^B \cdot \frac{v''_{1|11}}{v''_{1}} \\
        &= p_{0|0}^B \cdot v''_{1|00} + p_{1|0}^B \cdot v''_{1|01} + p_{0|1}^B \cdot v''_{1|10} + p_{1|1}^B \cdot v''_{1|11}.
    \end{align*}
    If $v''_0=0$, $v''_{1|00} = 0$ and $v''_{1|01} = 0$. If $v''_1=0$, $v''_{1|10} = 0$ and $v''_{1|11} = 0$. Therefore, the quantity does not lose generality.

    If $v'_0 > 1$, $- \delta \leq v_0 - v'_0 \leq v_0 - v''_0 = v_0 - 1 \leq 0$, which implies $|v_0 - v''_0| \leq \delta$. If $v'_0 < 0$, $0 \leq v_0 - v''_0 = v_0 \leq v_0 - v'_0 \leq \delta$, which implies $|v_0 - v''_0| \leq \delta$. If $v''_0 = v'_0$, $|v_0 - v''_0| = |v_0 - v'_0| \leq \delta$.
    If $v'_0 > 1$, $|v_1 - v''_1| = v_1 = 1-v_0 \leq \delta$. If $v'_0 < 0$, $|v_1 - v''_1| = |v_1 - 1| = |v_0| \leq \delta$. If $v''_0 = v'_0$, $|v_1 - v''_1| = |(1-v_0) - (1-v''_0)| \leq \delta$. In summary, we have $|v_0 - v''_0| \leq \delta$ and $|v_1 - v''_1| \leq \delta$ in all cases.
    
    Let us show the quantity $|v_{1|00} - v''_{1|00}|$ is small.
    If $v'_{1|00} > v''_0$, $v''_{1|00} = v''_0$ and we have 
    \[
         - \delta \leq v_{1|00} - v'_{1|00} \leq v_{1|00} - v''_{0} \leq v_{0} - v''_{0} \leq \delta
    \]
    by the definitions, and it implies $|v_{1|00} - v''_{1|00}| = |v_{1|00} - v''_0| \leq \delta$. 
    If $v'_{1|00} < 0$, $v''_{1|00} = 0$ and we have
    \[
        |v_{1|00} - v''_{1|00}| =  v_{1|00} \leq |v_{1|00} - v'_{1|00}| \leq \delta
    \]
    by the definitions.
    Otherwise, $v''_{1|00} = v'_{1|00}$ and $|v_{1|00} - v''_{1|00}| = |v_{1|00} - v'_{1|00}| \leq \delta$. In summary, in all cases, we have $|v_{1|00} - v''_{1|00}| \leq \delta$. By similar discussions, we also have $|v_{1|01} - v''_{1|01}| \leq \delta$.

    Let us next show the quantity $|v_{1|10} - v''_{1|10}|$ is small. If $v'_{1|10} > v''_1$, $v''_{1|10} = v''_1$ and we have
    \[
         - \delta \leq v_{1|10} - v'_{1|10} \leq v_{1|10} - v''_{1} \leq v_{1} - v''_{1} \leq \delta
    \]
    by the definitions, and it implies $|v_{1|10} - v''_{1|10}| = |v_{1|10} - v''_1| \leq \delta$. If $v'_{1|00} < 0$, $v''_{1|00} = 0$ and we have
    \[
        |v_{1|00} - v''_{1|00}| =  v_{1|00} \leq |v_{1|00} - v'_{1|00}| \leq \delta
    \]
    by the definitions.
    If $v'_{1|10} < 0$, $v''_{1|10} = 0$ and we have
    \[
        |v_{1|10} - v''_{1|10}| =  v_{1|10} \leq |v_{1|10} - v'_{1|10}| \leq \delta
    \]
    by the definitions.
    Otherwise, $v''_{1|10} = v'_{1|10}$ and $|v_{1|10} - v''_{1|10}| = |v_{1|10} - v'_{1|10}| \leq \delta$ by the definition. In summary, in all cases, we have $|v_{1|10} - v''_{1|10}| \leq \delta$. By similar arguments, we also have $|v_{1|11} - v''_{1|11}| \leq \delta$.
    
    Therefore, the difference between the true value and the simulation value is
    \begin{align*}
        &|(p_{0|0}^B \cdot v_{1|00} + p_{1|0}^B \cdot v_{1|01} + p_{0|1}^B \cdot v_{1|10} + p_{1|1}^B \cdot v_{1|11}) - (p_{0|0}^B \cdot v''_{1|00} +p_{1|0}^B \cdot v''_{1|10} + p_{0|1}^B \cdot v''_{1|10} + p_{1|1}^B \cdot v''_{1|11})| \\
        & \leq p_{0|0}^B |v_{1|00} - v''_{1|00}| + p_{1|0}^B |v_{1|01} - v''_{1|01}| + p_{0|1}^B |v_{1|10} - v''_{1|10}| + p_{1|1}^B |v_{1|11} - v''_{1|11}| \\
        & \leq (p_{0|0}^B + p^B_{1|0} + p^B_{0|1} + p^B_{1|1}) \delta = 2 \delta,
    \end{align*}
    where we use $p_{0|0}^B + p^B_{1|0} = 1$ and $p^B_{0|1} + p^B_{1|1} = 1$.
    
    If classical messages of hybrid SMP schemes for $\EQ_n$ are deterministic, then the length of the classical message is $\Omega(n)$. This is because, for $2^n$ inputs, there are some conflicts of deterministic messages and the referee cannot distinguish them. We thus have $O(q \log q) = \Omega(n)$ and $q = \Omega(n/\log n)$, which concludes the proof.
\end{proof}

\subsection{Multiple-round LOCC SMP protocols}\label{subsec:many_rounds}

When we increase the number $r$ of rounds by $2$, the number of operators we will care is increased by $4^r$. This is because both the numbers of measurement outcomes by $\rm{Ref}_A$ and $\rm{Ref}_B$ are $2$. Therefore, the total number of the operators we will care for $2r$-round LOCC protocols is 
\begin{equation}\label{eq:num}
    \sum_{i=0}^{r-1} 4^r = \frac{4^r - 1}{4-1} \leq 4^r = 2^{2r}. 
\end{equation}

We first show that each measurement outcome can be simulated by taking ratios of $2^{2r}$ values to simulate quantum $2$-value $2r$-round quantum LOCC SMP protocol. Note that for measurement operators $M_i$ for $i\in[n]$, $0 \leq M_0^\dagger\cdots M_{n-1}^\dagger M_{n-1}\cdots M_0 \leq I$.

\begin{lemma}\label{lem:ratio}
    Let us consider the case where a quantum state $\rho$ is measured $n \in \mathbb{N}$ times, and let $M_i$ be a measurement operator of the $i$th 2-value measurement for $i \in [n]$. Let us denote $v_\emptyset:=1$ and $v_i := \mathrm{tr}(M_0^\dagger\cdots M_{i}^\dagger M_{i}\cdots M_0 \rho )$. Then, the probability that the $n$th measurement succeeds conditioned on the event that all $n-1$ previous measurements succeed, $p_{n-1|n-2}$, is $\frac{v_{n-1}}{v_{n-2}}$.
\end{lemma}
\begin{proof}
    We will show the statement by induction. When $n=1$, $p_{0|\emptyset} = \mathrm{tr}(M_0^\dagger M_0 \rho) = \frac{v_0}{v_\emptyset}$.
    
    For $k>1$, let us consider $p_{k|k-1}$. Let us denote by $\rho_i$ the state after the $(i-1)$th measurement succeeds for $i \in [k]$. In other words, we define
    \[
        \rho_i := \frac{M_i \rho_{i-1} M_i^\dagger}{\mathrm{tr}(M_i^\dagger M_i \rho_{i-1})}
    \]
    for $i \in [k]$. Using the notation, let us assume that 
    \[
        p_{i|i-1} = \mathrm{tr}(M_{i}^\dagger M_{i} \rho_{i-1}) = \frac{v_{i}}{v_{i-1}}
    \]
    for all $i \in [k-1]$.

    Then, we have
    \begin{align*}
        p_{k|k-1} &= \mathrm{tr}(M_k^\dagger M_k \rho_{k-1}) = \frac{\mathrm{tr}(M_k^\dagger M_k M_{k-1} \rho_{k-2} M_{k-1}^\dagger)}{\mathrm{tr}(M_{k-1}^\dagger M_{k-1} \rho_{k-2})} \\
        &= \frac{v_{k-2}}{v_{k-1}} \mathrm{tr}(M_k^\dagger M_k M_{k-1} \rho_{k-2} M_{k-1}^\dagger) = \frac{v_{k-2}}{v_{k-1}} \mathrm{tr}(M_{k-1}^\dagger M_k^\dagger M_k M_{k-1} \rho_{k-2}) \\
        &= \frac{v_{k-2}}{v_{k-1}} \mathrm{tr}(M_{k-1}^\dagger M_k^\dagger M_k M_{k-1} \frac{M_{k-2} \rho_{k-3} M_{k-2}^\dagger}{\mathrm{tr}(M_{k-2}^\dagger M_{k-2} \rho_{k-3})}) \\
        &= \frac{v_{k-2}}{v_{k-1}} \frac{v_{k-3}}{v_{k-2}} \mathrm{tr}(M_{k-2}^\dagger M_{k-1}^\dagger M_k^\dagger M_k M_{k-1} M_{k-2} \rho_{k-3}) \\
        &= \cdots \\
        &= \frac{v_{k-2}}{v_{k-1}} \frac{v_{k-3}}{v_{k-2}} \cdots \frac{v_1}{v_2}\mathrm{tr}(M_2^\dagger \cdot M_k^\dagger M_k \cdots M_2 \rho_1) \\
        &= \frac{v_{k-2}}{v_{k-1}} \frac{v_{k-3}}{v_{k-2}} \cdots \frac{v_1}{v_2} \frac{v_k}{v_1}\\
        &= \frac{v_k}{v_{k-1}}
    \end{align*}
    where we use the cyclic property of the trace.
    Therefore, by induction, we have the claim.
\end{proof}

\begin{lemma}\label{lem:error}
    Let $\mathcal{P}$ be a quantum $2$-value $2r$-round LOCC SMP protocol to solve $\EQ_n$. Using a deterministic message from \cref{lem:replace}, the referee can simulate $\mathcal{P}$ with error $2^r(r+1)\delta$.
\end{lemma}

\begin{proof}
We assume, without loss of generality, that the final measurement outcome by $\rm{Ref}_A$ is the output of the protocol because it depends on all the previous measurement outcomes.
For all previous measurement results $h = a_0 b_0 \cdots$, let us denote by $h[i]$ the $(i+1)$th bit of $h$ and by $h[i,j]$ $(i+1)$th to $(j+1)$th bits of $h$. 
For $h \in \{0,1\}^{2R-1}$, let us define 
\[
v_{m|h} = \mathrm{tr}(M_{h[0]}^\dagger \cdots M_{h[2R-2]|h[0,2R-3]}^\dagger M_{h[2R-2]|h[0,2R-3]} \cdots M_{h[0]} \rho_x).
\] 
We also define $v''_{m|h}$ to make the probabilities valid as in \cref{prop:warm-up} (see the proof of \cref{lem:error} for a formal definition).

The true and simulation probability is the sum of $2^{2r}$ values and, from \cref{lem:ratio}, the total error is
\[
    \left| \sum_{h \in \{0,1\}^{2r}} \prod_{i \in [r]} p^B_{h[2i-1]|h[0,2i-2]} (v_{1|h} - v''_{1|h}) \right| \leq \sum_{h \in \{0,1\}^{2r}} \prod_{i \in [r]} p^B_{h[2i-1]|h[0,2i-2]}  |v_{1|h} - v''_{1|h}|,
\]
where $h[2i-1]$ denotes the $(2i)$th bit of $h$ (i.e., $b_i$) and $h[0,2i-2]$ denotes 1st to $(2i-1)$th bits of $h$ (i.e., $a_0b_0\cdots a_{i-1} b_{i-1} a_i$).

Let us evaluate how much error each term causes.
\begin{lemma}
    $|v_{m|h} - v''_{m|h}| \leq (k+1) \delta$ for all $m \in \{0,1\}$, $h \in \{0,1\}^{2k}$.
\end{lemma}

\begin{proof}
    Let us first show the base case where $k=0$ and $h = \emptyset$. If $v'_0 > 1$, $v''_0 := 1$. If $v'_0 < 0$, $v''_0 := 0$. Otherwise, $v''_0 := v'_0$. Let $v''_1 := 1-v''_0$. 
    
    If $v'_0 > 1$, $- \delta \leq v_0 - v'_0 \leq v_0 - v''_0 = v_0 - 1 \leq 0$, which implies $|v_0 - v''_0| \leq \delta$. If $v'_0 < 0$, $0 \leq v_0 - v''_0 = v_0 \leq v_0 - v'_0 \leq \delta$, which implies $|v_0 - v''_0| \leq \delta$. If $v''_0 = v'_0$, $|v_0 - v''_0| = |v_0 - v'_0| \leq \delta$.
    If $v'_0 > 1$, $|v_1 - v''_1| = v_1 = 1-v_0 \leq \delta$. If $v'_0 < 0$, $|v_1 - v''_1| = |v_1 - 1| = |v_0| \leq \delta$. If $v''_0 = v'_0$, $|v_1 - v''_1| = |(1-v_0) - (1-v''_0)| \leq \delta$. In summary, we have $|v_0 - v''_0| \leq \delta$ and $|v_1 - v''_1| \leq \delta$ in all cases.
    
    Assume that $|v_{m|h} - v''_{m|h}| \leq (t+1) \delta$ for all $m \in \{0,1\}$ and $h \in \{0,1\}^{2t}$. If $v'_{0|hab} > v''_{a|h}$, $v''_{0|hab} = v''_{a|h}$. If $v'_{0|hab} <0$, $v''_{0|hab} = 0$. Otherwise, $v''_{0|hab}= v'_{0|hab}$. Let $v''_{1|hab} = v''_{a|h} - v''_{0|hab}$. We want to prove that $|v_{m|hab} - v''_{m|hab}| \leq (t+2) \delta$ for $m \in \{0,1\}$, $h \in \{0,1\}^t$, $a \in \{0,1\}$ and $b \in \{0,1\}$.

    For conciseness, let us fix $a=0$, $b=0$ and we will prove $|v''_{m|h00} - v_{m|h00}| \leq (t+2) \delta$ for $m \in \{0,1\}$. If $v'_{0|h00} > v''_{0|h}$,
    \[
        -\delta \leq v_{0|h00} - v'_{0|h00} \leq v_{0|h00} - v''_{0|h} \leq v_{0|h} - v''_{0|h} \leq (t+1) \delta,
    \]
    which implies $|v_{0|h00} - v''_{0|h00}| = |v_{0|h00} - v''_{0|h}| \leq (t+1) \delta$.
    If $v'_{0|h00} <0$, $|v_{0|h00} - v''_{0|h00}| = |v_{0|h00}| \leq |v_{0|h00} - v'_{0|h00}| \leq (t+1) \delta$. Otherwise, $|v_{0|h00} - v''_{0|h00}| = |v_{0|h00} - v'_{0|h00}| \leq \delta$. In summary, we have $|v_{0|h00} - v''_{0|h00}| \leq (t+1) \delta$.

    We next show $|v_{1|h00} - v''_{1|h00}| \leq (t+2) \delta$.  If $v'_{0|h00} > v''_{0|h}$, $|v_{1|h00} - v''_{1|h00}| = v_{1|h00} = v_{0|h} - v_{0|h00} \leq (v''_{0|h} + (t+1)\delta) - (v'_{0|h00} - \delta) \leq (t+2)\delta$. If $v'_{0|h00} <0$, $v''_{1|h00} = v''_{0|h}$ and $|v_{1|h00} - v''_{1|h00}| = |v_{1|h00} - v''_{0|h}| \leq |v_{1|h00} - v_{0|h}| + |v_{0|h} - v''_{0|h}| = |v_{0|h00}| + |v_{0|h} - v''_{0|h}| \leq (t+2)\delta$. Otherwise, $|v_{1|h00} - v''_{1|h00}| = |v_{1|h00} - (v''_{0|h} - v'_{0|h00})| = |(v_{0|h} - v_{0|h00}) - (v''_{0|h} - v'_{0|h00})| \leq |v_{0|h} - v''_{0|h}| + |v_{0|h00} - v'_{0|h00}| \leq (t+2)\delta$.

    Taking all the values $a,b \in \{0,1\}$ into account, we have $|v_{m|hab} - v''_{m|hab}| \leq (t+2) \delta$ for $m \in \{0,1\}$, $h \in \{0,1\}^t$, $a \in \{0,1\}$ and $b \in \{0,1\}$. By induction on $t$, we have the claim.
\end{proof}

For all $h$, $p^B_{0|h} + p^B_{1|h} = 1$. We thus have
\begin{align*}
    &\sum_{h \in \{0,1\}^{2r}} \prod_{i \in [r]} p^B_{h[2i-1]|h[0,2i-2]}  
    = \sum_{h[0] \in \{0,1\}} \sum_{h[1] \in \{0,1\}} \cdots \sum_{h[2r-1] \in \{0,1\}} \prod_{i \in [r]} p^B_{h[2i-1]|h[0,2i-2]} \\
    &= \sum_{h[0] \in \{0,1\}} \sum_{h[2] \in \{0,1\}} \cdots \sum_{h[2r-2] \in \{0,1\}} \sum_{h[1] \in \{0,1\}} p^B_{h[1]|h[0]}  \sum_{h[3] \in \{0,1\}} p^B_{h[3]|h[0,2]} \cdots \sum_{h[2r-1] \in \{0,1\}}  p^B_{h[2i-1]|h[0,2i-2]} \\
    &= \sum_{h[0] \in \{0,1\}} \sum_{h[2] \in \{0,1\}} \cdots \sum_{h[2r-2] \in \{0,1\}} 1 = 2^{r}
\end{align*}

Therefore, the error is at most
\begin{align*}
    &\sum_{h \in \{0,1\}^{2r}} \prod_{i \in [r]} p^B_{h[2i-1]|h[0,2i-2]} (r+1) \delta = 2^r(r+1)\delta,
\end{align*}
which concludes the proof.
\end{proof}

\begin{theorem}\label{thm:two-way_LOCC}
    For a sufficiently small constant $k>0$, $n \in\mathbb{N}$, let $\mathcal{P}$ be a quantum $2$-value $2k \log_2 n$-round LOCC protocol to solve $\EQ_n$. Then the size of Alice's message $|\rho_x|$ is $\omega(n^{1-3k-\epsilon})$ for an arbitrary small constant $\epsilon >0$.
\end{theorem}

\begin{proof}
    Let $\delta$ be $n^{-k-\epsilon_1}$ for a sufficiently small constant $\epsilon_1 >0$. Suppose that $|\rho_x|$ is $O(n^{1-3(k+\epsilon_1)-\epsilon_2})$ for a sufficiently small constant $\epsilon_2 >0$. From \cref{eq:num}, the number of values the referee guesses is $2^{2 k \log_2 n}$. Then, from \cref{lem:replace} and \cref{lem:error}, the length of the deterministic message is $O ( (n^{1-3(k+\epsilon_1)-\epsilon_2}) n^{3(k+\epsilon_1)} ((1-3(k+\epsilon_1)-\epsilon_2)\log n + (k+\epsilon_1) \log n) (2 k \log_2 n + (k+\epsilon_1)\log n) )= o(n)$ and the error is $n^k (k \log n + 1) (n^{-k-\epsilon_1}) = o(1)$. If one message is deterministic in the hybrid SMP model and the model solves $\EQ$ with high probability, the length of the message is $n$. Since we can take $\epsilon = 3\epsilon_1+\epsilon_2$ an arbitrary small value, we have $|\rho_x| = \omega(n^{1-3k-\epsilon})$ for an arbitrary small constant $\epsilon>0$.
\end{proof}

\subsection{General arguments}

In the proofs of \cref{subsec:warm-up} and \cref{subsec:many_rounds}, we use a property that is inherent to $\EQ_n$. The property is that, if one message is deterministic in SMP models for $\EQ_n$, the length of the deterministic message is $n$. We can show more general results by replacing both quantum messages.

\begin{lemma}\label{lem:error_replace_both}
    Let $\mathcal{P}$ be a $2$-value $2r$-round LOCC protocol to solve $F:\{0,1\}^n \times \{0,1\}^n \rightarrow \{0,1\}$. Using a deterministic message from \cref{lem:replace}, the referee can simulate the quantum LOCC SMP protocol with error $O(2^{2r}r^2\delta^2)$.
\end{lemma}

\begin{proof}
For $m \in \{0,1\}$ and $h \in \{0,1\}^*$, let us define $v^A_{m|h} := v_{m|h}, v''^A_{m|h} := v''_{m|h}$ and, in a similar way, define $v^B_{m|h}, v''^B_{m|h}$ from the replacement of Bob's message.
Then, the error by the simulation of the referee from the two deterministic strings is at most
\begin{align*}
    & \left| \sum_{h \in \{0,1\}^{2r}} v^A_{1|h} v^B_{h[2i-1]|h[0,2i-2]} - v''^A_{1|h} v''^B_{h[2i-1]|h[0,2i-2]} \right| \\
    & \leq \sum_{h \in \{0,1\}^{2r}} \left| v^A_{1|h} v^B_{h[2i-1]|h[0,2i-2]} - v''^A_{1|h} v''^B_{h[2i-1]|h[0,2i-2]} \right| \\
    & \leq \sum_{h \in \{0,1\}^{2r}} \left| v^A_{1|h} v^B_{h[2i-1]|h[0,2i-2]} - \left( v^A_{1|h} + O(r\delta) \right) \left( v^B_{h[2i-1]|h[0,2i-2]} + O(r\delta) \right) \right| = O(2^{2r}r^2\delta^2) 
\end{align*}
\end{proof}

\begin{proposition}\label{prop:two-way-LOCC_general}
    For a sufficiently small constant $k>0$, $n \in\mathbb{N}$, let $\mathcal{P}$ be a quantum $2$-value $2k \log_2 n$-round LOCC protocol to solve $F:\{0,1\}^n \times \{0,1\}^n \rightarrow \{0,1\}$ such that $\R^{||}(F) = \Omega(n^c)$ for a constant $c>0$. Then, the size of the quantum messages in $\mathcal{P}$ is $\omega(n^{c-6k-\epsilon})$ for an arbitrary small constant $\epsilon >0$.
\end{proposition}

\begin{proof}
    Let $\delta$ be $n^{-2k-\epsilon_1}$ for a sufficiently small constant $\epsilon_1 >0$. Let $\rho_x$ and $\sigma_y$ be quantum messages from Alice and Bob, respectively. Suppose that $|\rho_x|$ and $|\sigma_y|$ are $O(n^{c-3(2k+\epsilon_1)-\epsilon_2})$ for a sufficiently small constant $\epsilon_2 >0$. From \cref{eq:num}, the number of values the referee guesses is $2^{2 k \log_2 n}$ for each message. Then, from \cref{lem:replace} and \cref{lem:error_replace_both}, the sum of the lengths of the deterministic messages is $2 \times O(n^{c-3(2k+\epsilon_1)-\epsilon_2}) n^{3(2k+\epsilon_1)} ((c-3(2k+\epsilon_1)-\epsilon_2)\log n + (2k+\epsilon_1) \log n) (2 k \log_2 n + (2k+\epsilon_1)\log n)= o(n^c)$ and the error is $O(n^{2k}(k \log n)^2 n^{-2k-\epsilon_1}) = o(1)$. This contradicts $\R^{||}(F) = \Omega(n^c)$. Since we can take $\epsilon = 3\epsilon_1+\epsilon_2$ an arbitrary small value, we have $\max\{|\rho_x|,|\sigma_y|\} = \omega(n^{c-6k-\epsilon})$ for an arbitrary small constant $\epsilon>0$.
\end{proof}

Moreover, we can consider a lower bound for general relation problems.

\begin{lemma}
    Let $\mathcal{P}$ be a $2$-value $2r$-round LOCC protocol to solve $R \subseteq \{0,1\}^n \times \{0,1\}^n \times \{0,1\}^*$. Using a deterministic message from \cref{lem:replace}, the referee can simulate the output of the quantum LOCC SMP protocol with error $O(2^{2r}r^2\delta^2)$.
\end{lemma}

\begin{proof}
For $m \in \{0,1\}$ and $h \in \{0,1\}^*$, let us define $v^A_{m|h} := v_{m|h}, v''^A_{m|h} := v''_{m|h}$ and, in a similar way, define $v^B_{m|h}, v''^B_{m|h}$ from the replacement of Bob's message.
Then, the error by the simulation of all the measurement outcomes of the referee from the two deterministic strings is at most
\begin{align*}
    & \left| \sum_{m \in \{0,1\}, h \in \{0,1\}^{2r}} v^A_{m|h} v^B_{h[2i-1]|h[0,2i-2]} - v''^A_{m|h} v''^B_{h[2i-1]|h[0,2i-2]} \right| \\
    & \leq 2 \times \sum_{h \in \{0,1\}^{2r}} \left| v^A_{1|h} v^B_{h[2i-1]|h[0,2i-2]} - v''^A_{1|h} v''^B_{h[2i-1]|h[0,2i-2]} \right| \\
    & \leq 2 \times \sum_{h \in \{0,1\}^{2r}} \left| v^A_{1|h} v^B_{h[2i-1]|h[0,2i-2]} - \left( v^A_{1|h} + O(r\delta) \right) \left( v^B_{h[2i-1]|h[0,2i-2]} + O(r\delta) \right) \right| = O(2^{2r}r^2\delta^2) 
\end{align*}
If all the measurement outcomes are the same, the output of the protocol is also the same as the original protocol.
\end{proof}

Then, by the same proof as \cref{prop:two-way-LOCC_general}, we have the same argument for general relation problems.

\begin{theorem}\label{thm:two-way-LOCC_general_relation}
    For a sufficiently small constant $k>0$, $n \in\mathbb{N}$, let $\mathcal{P}$ be a $2$-value $2k \log_2 n$-round LOCC protocol to solve $R \subseteq \{0,1\}^n \times \{0,1\}^n \times \{0,1\}^*$ such that $\R^{||}(R) = \Omega(n^c)$ for a constant $c>0$. Then, the size of the quantum messages in $\mathcal{P}$ is $\omega(n^{c-6k-\epsilon})$ for an arbitrary small constant $\epsilon >0$.
\end{theorem}
