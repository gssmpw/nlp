\begin{figure*}[t]
    \centering
    \def\figwidth{0.17\linewidth }
    \def\trimdim{5mm 5mm 5mm 5mm }
    \def\figdir{figures/barycenters_comparison_panels/}
    \foreach \index in {11,...,20}
    {
        \begin{subfigure}{\figwidth}
            \centering
            %[left, down, right, up]
           {\includegraphics[trim =11mm 6mm 11mm 6mm, clip, width=.98\linewidth]{\figdir\index.pdf}\label{fig:regfree:\index}}

        \end{subfigure}
    }
    \caption{The effect of the regularization HP. The figures shows 10 samples (gray) from the ECGFiveDays dataset with their estimated average (blue), and compares Euclidean averaging, DBA, SoftDTW, and several DTAN methods. DBA requires no HP but falls to poor local minima. SoftDTW's barycenter is severely affected by the choice of its smoothing HP, $\gamma$: $\gamma=0.1$ results in a visible `pinching' effect while $\gamma=10$ smoothens out  peaks/valleys. DBA and SoftDTW are computed per class and do not learn how to generalize to new data, unlike DTAN which is learning-based and requires a single model for all classes. The regularization often used with DTAN has 2 HPs, ($\lambda_{\sigma} ,\lambda_{\mathrm{smooth}}$), where a \emph{weak} regularization ($\lambda_{\sigma} ,\lambda_{\mathrm{smooth}}: .5, .01$) is insufficient and a \emph{strong} regularization ($\lambda_{\sigma},\lambda_{\mathrm{smooth}}:.001, .1$), is too restrictive. $\Lcal_{\mathrm{ICAE}}$ and $\Lcal_{\mathrm{ICAE-triplet}}$ are regularization-free, yet provide barycenters that represent the data well.}
    \label{fig:reg:ecg}
    \end{figure*}
