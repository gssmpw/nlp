The nuisance nonlinear misalignment distorts,
among other things, the sample mean~\cite{wigley:climate:1984:average,Gusfield:Cambridge:1997:Steiner}. As discussed in~\autoref{Sec:previous}, averaging under the DTW distance 
is a commonly-used solution to this issue~\cite{Petitjean:2011:global,Petitjean:2014:dynamic,cuturi:ICML:2014:fast,cuturi:2017:soft}; 
however, such non-learning DTW-based methods are computationally expensive.
This is especially problematic since,
as these methods do not generalize, each batch of new signals requires them to solve another optimization problem (\ie consider the assignment step in the K-means algorithm).
In contrast, as DTAN easily aligns new signals inexpensively and almost instantaneously via a forward pass, it also provides, in the single-class case, a mechanism 
for quickly averaging a new collection of previously-unseen signals.
In other words, this is nothing more than computing the sample mean
of the warped test data: 
\begin{align}
\label{eqn:average}
\mu = \tfrac{1}{N}\sum\nolimits_{i=1}^{N}v_i = \tfrac{1}{N}\sum\nolimits_{i=1}^N u_i \circ T^{\btheta_i}\, .
\end{align}