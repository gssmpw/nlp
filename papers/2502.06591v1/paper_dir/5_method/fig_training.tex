\begin{figure*}[ht]
\centering
 %[left, down, right, up]
\newcommand{\myWidth}{0.98\linewidth}
\newcommand{\mySubfigWidth}{.19\linewidth}
\newcommand{\mySmallWidth}{1.00\linewidth}
\newcommand{\mySmallSubfigWidth}{.0894\linewidth}
\newcommand{\myL}{0}

\begin{subfigure}{\mySubfigWidth}
  \centering
   %[left, down, right, up]
   \includegraphics[width=\myWidth,trim=5mm 3mm 100mm 23mm,clip]
   {figures/training/BeetleFly/BeetleFly-reference1.pdf}
\end{subfigure}
\foreach \index in {10,20,50,100}
{%
  \begin{subfigure}{\mySubfigWidth}
  \centering
   %[left, down, right, up]
   \includegraphics[width=\myWidth,trim=5mm 3mm 100mm 23mm,clip]
   {figures/training/BeetleFly/no_reg/\index.pdf}
  \end{subfigure}
}
%
  \begin{subfigure}{\mySubfigWidth}
  \centering
   %[left, down, right, up]
   \includegraphics[width=\myWidth,trim=5mm 3mm 100mm 23mm,clip]
   {figures/training/BeetleFly/icae/0.pdf}
  \caption{Input data}
  \end{subfigure}
%
\foreach \index in {10,20,50,100}
{%
  \begin{subfigure}{\mySubfigWidth}
  \centering
   %[left, down, right, up]
   \includegraphics[width=\myWidth,trim=5mm 3mm 100mm 23mm,clip]
   {figures/training/BeetleFly/icae/\index.pdf}
  \caption{Epoch \index}
  \end{subfigure}
}
%  
\caption{Training procedure on the \textit{BeetleFly} dataset. The first column depicts the input data (for better visualization, the top panel shows 3 random signals while the bottom 10 signals and their average are in blue). (\textbf{Top}) The Within-Class Sum of Squares (WCSS) loss reduces variance by applying an unrealistic deformation to the data, resulting in visible `pinching' effect (\ie, bad local minima). (\textbf{Bottom}) The proposed $\mathcal{L}_{\mathrm{ICAE}}$, while requiring no regularization, avoids such an undersired solution by maintaining consistency between the average sequence and its class members.}
\label{fig:training}
\end{figure*}
