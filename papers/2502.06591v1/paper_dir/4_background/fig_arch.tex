\begin{figure*}[t]
\begin{center}
\centering
\def\figwidth{0.99\linewidth}% - 1.2em}


 %[left, down, right, up]
% [trim = left, down, right, up]
\includegraphics[trim =0mm 6mm 0mm 0mm, clip, width=\figwidth]{figures/model/DTAN_arch.pdf}


\caption{DTAN joint alignment demonstrated on a class of the ``Trace" dataset~\cite{Chen:UCR:Archive:2015} with a simple 1D ConvNet backbone which was used in~\cite{Shapira:NIPS:2019:DTAN}.
Signals are denoted in gray and their average in blue. 
Each Convolution layer is followed by a ReLU, Batch Normalization, and a Max-Pooling layer. The final Fully-Connected 
layer (fc) predicts the warping parameters, $\btheta$, of the CPA velocity fields, $v^{\btheta}$, which is then integrated to form a CPAB warp, $T^{\btheta}$. The latter, in turn, is applied to the input signal ($u$) to create the output, $u\circ T^{\btheta}=v$. The loss consists of the empirical within-class variance ($\Lcal_{\mathrm{data}}$) and a regularization term on $\btheta$ ($\Lcal_{\mathrm{reg}}$).}

\label{fig:fig_main}
\end{center}
\end{figure*}
