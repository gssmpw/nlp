\subsection{Diffeomorphisms}\label{cpab}
 As mentioned in~\autoref{Sec:Introduction}, $\Tcal$ needs to be specified. 
In the context of time warping, \emph{diffeomorphisms} is a natural choice~\cite{Mumford:Book:2010:PT}.
  \begin{Definition}
 A ($C^1$) diffeomorphism is a differentiable invertible map with a differentiable inverse. 
  \end{Definition}
 Working with diffeomorphisms usually involves expensive computations.
 In our case, since the proposed method explicitly incorporates them 
in a DL architecture, it is even more important (than in traditional non-DL applications
of diffeomorphisms) to drastically reduce the computational difficulties. 
The reason is that during training, the quantities
$x\mapsto T^\btheta(x)$ and $x\mapsto \nabla_\btheta T^\btheta( x)$
are computed at multiple time points $x$ and for multiple values of $\btheta$. 

As mentioned in~\autoref{Sec:Introduction}, we have chosen to incorporate the CPAB transformation family into DTAN~\cite{Shapira:NIPS:2019:DTAN,Shapira:ICML:2023:RFDTAN}. These warps combine expressiveness and efficiency, making them a natural choice in a DL context~\cite{Hauberg:AISTATS:2016:DA,Skafte:CVPR:2018:DDTN}. 
Other efficient and expressive diffeomorphisms~(\eg, \cite{Zhang:IJCV:2018:fast,Arsigny:BIR:2006,Durrleman:IJCV:2013,Allassonniere:SIAM:2015})
can also be explored in the DTAN context, provided they also offer
an efficient and highly-accurate way to evaluate $x\mapsto\nabla_\btheta T^\btheta( x)$
as CPAB warps do~\cite{Freifeld:TR_CPAB_Derivaitive:2017}.
%
Below we briefly explain CPAB warps (restricting the discussion to 1D), and refer the reader to~\cite{Freifeld:ICCV:2015:CPAB,Freifeld:PAMI:2017:CPAB,Freifeld:TR_CPAB_Derivaitive:2017}
for more details. 

The name CPAB, short for CPA-Based, is due to the fact that these warps 
are based on Continuous Piecewise-Affine (CPA) velocity fields. 
The term ``piecewise'' is \wrt some partition, denoted by $\Omega$, of the signal's domain into subintervals.
Let $\Vcal$ denote the linear space of CPA velocity fields \wrt such a fixed $\Omega$,
let $d=\dim(\Vcal)$, and let $v^\btheta:\Omega\to\RR$, a velocity field parametrized
by $\btheta\in\Rd$, denote the generic element
of $\Vcal$, where $\btheta$ stands for the coefficient \wrt some basis of $\Vcal$.
%
The corresponding space of CPAB warps, obtained via integration of elements of $\Vcal$,  is 
\begin{align}
\hspace{-2mm}&\Tcal\triangleq
 %\set
  %\biggl
   \Bigl \{
 T^\btheta:
   x\mapsto \phi^\btheta( x;1)
  \text{ s.t. } \phi^\btheta( x;t) \text{ solves } \nonumber \\ 
\hspace{-2mm} & \phi^\btheta(x;t) = x+\int_{0}^t  v^\btheta(\phi^\btheta(x;\tau))\, 
 \mathrm{d}\tau \text{ where }  v^\btheta\in \Vcal\, 
 %\biggr
  \Bigr
 \}\, . \vspace{-5mm}
 \label{Eqn:IntegralEquation}
\end{align}
It can be shown that these warps are indeed ($C^1$) diffeomorphisms~\cite{Freifeld:ICCV:2015:CPAB,Freifeld:PAMI:2017:CPAB}.
See~\autoref{fig:cpab1}
 for a typical warp. While $v^\btheta$
is CPA, $T^\btheta:\Omega\to\Omega$ is not (\eg, $T^\btheta$ is differentiable, unlike $v^\theta$).
CPA velocity fields support an
integration method that is faster \emph{and} more accurate than typical 
velocity-field integration methods~\cite{Freifeld:ICCV:2015:CPAB,Freifeld:PAMI:2017:CPAB}.
The fineness of  $\Omega$ controls the trade-off between expressiveness of $\Tcal$
on the one hand and the computational complexity as well as the dimensionality (\ie, the value of $d=\dim(\btheta)$) on the other 
hand. 
%
%
%

\textbf{Initialization.} Since $\btheta = \bzero$ gives the identity map,
we initialize the final layer of the localization network by sampling the weights from a zero-mean normal
distribution (\ie $\bw \sim \Ncal(\bzero, 10^{-5}))$. 

\textbf{Optional zero-boundary conditions.} If of interest, one can easily restrict 
the CPA fields to vanish at the endpoints of the domain, implying these points will be fixed points
of the resulting warp, \ie $v[0]=v[n]=0$ (see~\citet{Freifeld:PAMI:2017:CPAB} more details). 

\textbf{Optional circularity constraint}. Alternatively, one can enforce a circularity constraint by adding a linear constraint on the CPA velocity field,
making it circularly continuous: $v[0]=v[n]$ (\ie enforcing the velocity at the starting point to be equal to the
velocity at the end-point). See~\cite{kaufman:icip:2021:cyclic} for details.

\textbf{The CPAB Gradient.}
Importantly, in 1D, the \emph{CPAB gradient},
$\nabla_\btheta T^\btheta( x)$, has a closed-form expression which was recently discovered in~\cite{Martinez:ICML:2022:closed}.
The latter is more efficient and stable than the numerical solution and facilitates much faster training and inference time.



