\section{Conclusion}\label{Sec:Conclusion}
In this work, we introduced DTAN, a novel deep learning framework for time-series Joint Alignment (JA), drawing upon a blend of contemporary and traditional concepts such as Spatial Transformer Networks (STN; ~\cite{Jaderberg:NIPS:2015:spatial,Skafte:CVPR:2018:DDTN}), efficient and highly-expressive diffeomorphisms~\cite{Freifeld:ICCV:2015:CPAB,Freifeld:PAMI:2017:CPAB}, and JA cost functions~\cite{Learned:PAMI:2006:align,Cox:CVPR:2008:LS,Erez:2022:ECCV:MCBM,Barel:ECCV:2024:spacejam}. DTAN facilitates unsupervised learning of alignments, with an extension to a weakly-supervised regime when class labels are available, enabling class-specific JA. The inherent challenges of unsupervised JA, particularly the risk of trivial solutions through excessive signal distortion, are mitigated through two distinct strategies: a regularization term for warps and the introduction of the Inverse Consistency Averaging Error (ICAE), a novel, regularization-free approach. Furthermore, we present RDTAN, an enhanced recurrent version of DTAN, which surpasses the original in terms of expressiveness and performance without an increase in parameter count. To augment class separation, we propose MT-DTAN, a multi-tasking extension of DTAN, optimized for simultaneous alignment and classification, alongside the Inverse Consistent Centroids Triplet Loss, derived from ICAE. Further evaluations demonstrate that architectures originally designed for time-series classification, when used as the backbone for the TTN, are equally effective in facilitating time-series alignment tasks. 
Finally, our findings underscore the efficacy of JA in enabling misalignbment-robust and efficient Principal Component Analysis (PCA) for time-series data.
