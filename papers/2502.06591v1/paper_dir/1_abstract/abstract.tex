In time-series analysis, nonlinear temporal misalignment remains a pivotal challenge that forestalls even
simple averaging. Since its introduction, the Diffeomorphic Temporal Alignment Net (DTAN), which we first introduced in~\cite{Shapira:NIPS:2019:DTAN} and further developed in~\cite{Shapira:ICML:2023:RFDTAN}, 
has proven itself as an effective solution for this problem (the conference papers~\cite{Shapira:NIPS:2019:DTAN} and~\cite{Shapira:ICML:2023:RFDTAN} are earlier partial versions of the current manuscript). DTAN predicts and applies diffeomorphic transformations in an input-dependent manner, thus facilitating the joint alignment (JA) and averaging of time-series ensembles in an unsupervised or a weakly-supervised manner. The inherent challenges of the weakly/unsupervised setting, particularly the risk of trivial solutions through excessive signal distortion, are mitigated using either one of two distinct strategies: 1) a regularization term for warps; 2) using the Inverse Consistency Averaging Error (ICAE).
The latter is a novel, regularization-free approach which also facilitates the JA of variable-length signals.
We also further extend our framework to incorporate multi-task learning (MT-DTAN), enabling simultaneous time-series alignment and classification. Additionally, we conduct a comprehensive evaluation of different backbone architectures, demonstrating their efficacy in time-series alignment tasks. Finally, we showcase the utility of our approach in enabling Principal Component Analysis (PCA) for misaligned time-series data. Extensive experiments across 128 UCR datasets validate the superiority of our approach over contemporary averaging methods, including both traditional and learning-based approaches, marking a significant advancement in the field of time-series analysis.