\begin{figure}[t!]
\centering
\def\figwidth{0.3\linewidth}% - 1.2em}

\begin{tikzpicture}
    % Nodes for the images in the first row
    \node[inner sep=0] (img1) at (0,0) {
        \includegraphics[trim = 9mm 9mm 12mm 15mm, clip, width=\figwidth]{figures/tsne/facesUCR/Original.pdf}
    };
    \node[inner sep=0, right=2mm of img1] (img2) {
        \includegraphics[trim = 9mm 9mm 12mm 15mm, clip, width=\figwidth]{figures/tsne/facesUCR/Embedding.pdf}
    };
    \node[inner sep=0, right=2mm of img2] (img3) {
        \includegraphics[trim = 9mm 9mm 12mm 15mm, clip, width=\figwidth]{figures/tsne/facesUCR/Aligned.pdf}
    };

    % Text for the first row
    \node[rotate=90, left=2mm of img1, anchor=center, font=\scriptsize] {DTAN};

    % Nodes for the images in the second row
    \node[inner sep=0, below=2mm of img1] (img4) {
        \includegraphics[trim = 9mm 9mm 12mm 15mm, clip, width=\figwidth]{figures/tsne/facesUCR/Original.pdf}
    };
    \node[inner sep=0, below=2mm of img2] (img5) {
        \includegraphics[trim = 9mm 9mm 12mm 15mm, clip, width=\figwidth]{figures/tsne/facesUCR/mt_Embedding.pdf}
    };
    \node[inner sep=0, below=2mm of img3] (img6) {
        \includegraphics[trim = 9mm 9mm 12mm 15mm, clip, width=\figwidth]{figures/tsne/facesUCR/mt_Aligned.pdf}
    };

    % Text for the second row
    \node[rotate=90, left=2mm of img4, anchor=center, font=\scriptsize] {MT-DTAN};
    % Text for columns
    \node[below=2mm of img4, anchor=center, font=\scriptsize] {Original data};
    \node[below=2mm of img5, anchor=center, font=\scriptsize] {Original data embedding};
    \node[below=2mm of img6, anchor=center, font=\scriptsize] {Aligned data};
\end{tikzpicture}
    \caption{t-SNE visualizations of the 14-class \textit{FacesUCR} dataset are shown for DTAN (top) and MT-DTAN (bottom) using the \emph{InceptionTime} backbone. DTAN effectively reduces the within-class variance in the original signal's domain but fails to achieve similar results for latent features, also known as the embedding. Conversely, the multi-tasking framework, MT-DTAN, demonstrates improved separation both in the original domain and in the embedding space.}
\label{fig:tsne}
\end{figure}