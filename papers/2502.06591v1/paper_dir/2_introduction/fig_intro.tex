\begin{figure}[t]
\centering
%\begin{center}
% \centerline{\includegraphics[width=\columnwidth]{icml_numpapers}}
\def\figwidth{0.98\linewidth } % Larger values screw up the THIRD page; go figure...O

% \begin{subfigure}{\figwidth}
%  \centering
%  %[left, down, right, up]
% {\includegraphics[trim = 6mm 2mm 7mm 0mm, clip, width=.95\linewidth]{../figures/Intro/intro_train.pdf}\label{fig:intro:a}}
% \caption{Train}
% \end{subfigure}

\begin{subfigure}{\figwidth}
 \centering
{\includegraphics[trim = 6mm 2mm 7mm 0mm, clip, width=.95\linewidth]{figures/Intro/intro_test.pdf}}%\label{fig:intro}}
% \caption{Test}
\end{subfigure}

\caption{An illustration of the joint-alignment problem in ECG data. 
The data shown is test data. 
    Top: temporal misalignment between ECG signals and its effect on the sample mean (\textit{ECGFiveDays} Dataset).
    Bottom: joint-alignment prediction by DTAN at test time.}
    %DTAN aligns the new test samples in an input-dependent manner with a single forward pass.}
\label{fig:intro}
%\end{center}
\end{figure}
%~\cite{Chen:UCR:Archive:2015}