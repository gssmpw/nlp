\begin{figure}[t]
    \subimport{../figures}{pca/Trace/explained_variance.tex}
    \centering
    \def\figwidth{0.92\linewidth} % Define the width for the images

    \begin{tikzpicture}
        % First subfigure
        \node[inner sep=0] (img1) at (0,0.2) {
            \includegraphics[trim = 1mm 1mm 1mm 10mm, clip, width=\figwidth]{figures/pca/Trace/original_PCs.pdf}
        };
        % Label for the first subfigure
        \node[rotate=90, left=1mm of img1, anchor=center, font=\scriptsize] {Original PCs};

        % Second subfigure, placed below the first
        \node[inner sep=0, below=-1mm of img1] (img2) {
            \includegraphics[trim = 1mm 1mm 1mm 10mm, clip, width=\figwidth]{figures/pca/Trace/aligned_PCs.pdf}
        };
        % Label for the second subfigure
        \node[rotate=90, left=1mm of img2, anchor=center, font=\scriptsize] {Aligned PCs};

        % Third subfigure, placed below the second
        \node[inner sep=0, below=0mm of img2] (img3) {
            \includegraphics[trim = 1mm 1mm 1mm 1mm, clip, width=\figwidth]{figures/pca/Trace/PC_reconstruction.pdf}
        };
        % Label for the third subfigure
        \node[rotate=90, left=1mm of img3, anchor=center, font=\scriptsize] {Reconstruction};
    \end{tikzpicture}
    
    \caption{The benefits of joint alignment for dimensionality reduction, evaluated on the \textit{Trace} dataset~\cite{Dau:2019:ucr} using Principal Component Analysis. The top panel shows the cumulative explained variance as a function of the number of Principal Components (PCs). The middle-top and middle-bottom panels depict the first 3 PCs of the original and DTAN-aligned data, respectively. The bottom panel illustrates the reconstruction of the original and (inverse warped) aligned data using the first 6 PCs.}
    
    \label{fig:pca}
    \end{figure}


