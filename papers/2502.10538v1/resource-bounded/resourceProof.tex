\begin{IEEEproof}
    The decoder given $L,R \geq 0$ such that $R - L + 1 \geq \kappaP,$ queries the word $\tilde{\YStar} \circ \tilde{\YP}$ in two steps. 
    First, it performs queries for $\LDCStar.\Dec^{\tilde{\YStar}}$ to recover the puzzle $\bZ$, and after solving the puzzle to generate secret key $\sk,$ it performs queries for $\paLDC.\Dec_{\sk}^{\tilde{\YP}}(L,R)$. 
    The total number of queries is at most $\ellStar + (R - L + 1) \alphaP$ so $\alpha$ can be derived as
    \[\alpha (R - L + 1) \leq \ellStar + (R - L + 1)\alphaP \implies \alpha \leq \alphaP + \ellStar / \kappaP.\]
    Next, the error tolerance $\delta$ can be derived as the weighted minimum error-tolerance between the \LDCStar\ and \paLDC\ codewords, $\delta = (1/n) \times \min\{\deltaStar \nStar, \deltaP \nP\}.$
    A channel can choose either $\YP$ or $\YStar$ to place all $\delta n$ errors, so the fraction of errors tolerated is the minimum of the error tolerances of the chosen \LDCStar \ and \paLDC. 
    
    The remainder of the proof of security and decoding error follows the proof of Theorem $6.8$ of \cite{SCN:AmeBloBlo22} with minor changes for notational differences.
    We repeat it for the sake of completeness.
    Suppose there exists an algorithm/adversarial channel $\calA \in \R$ that, given the hard puzzle \Puz, can cause a decoding error for some chosen range $[L,R]$ with some non-negligible probability $\eps(\lambda)$ for some $\lambda \in \N.$
    Then, we can construct an adversary $\calB \in \R$ breaking security of the $(\R,\epsPuz)$-hard puzzle by a two-phase hybrid argument.
    In the first phase, we define two encoders $\Enc_0$ and $\Enc_1$, where $\Enc_0$ is equal to the original encoder in Construction \ref{constr:resourcePaLDC} and $\Enc_1$ is defined to take in the secret key $\sk$ (rather than generating it as in $\Enc_0$) as follows.
    \begin{weirdFrame}{$\Enc_1(\bx,\sk)$}
                \begin{enumerate}
            \item Sample random seed $\bs \unif \{0,1\}^{\kP}$.
            \item Choose polynomial $t > t'$ and compute $\bZ \leftarrow \Puz.\Gen(1^\lambda,t(\lambda),s)$ where $\bZ \in \{0,1\}^{\kStar}.$
            \item Set $\YStar \leftarrow \LDCStar.\Enc(\bZ)$.
            \item Set $\YP \leftarrow \paLDC.\Enc_{\sk}(\bx)$
            \item Output $\YStar \circ \YP$.
        \end{enumerate}
    \end{weirdFrame}
    We proceed with constructing the two-phase hybrid distinguisher $(\calD_1,\calD_2)$.
    First, $\calD_1$ is given as input message $\bx$ and access to encoders $\Enc_0$ and $\Enc_1.$
    \begin{weirdFrame}{$\calD_1(\bx)$}
        \begin{enumerate}
            \item Compute $b \unif \{0,1\}$ and $\sk_1 \leftarrow \paLDC.\Gen(1^\lambda)$.
            \item If $b = 0$, output $\bY_b \leftarrow \Enc_0(\bx)$. 
            Otherwise, if $b = 1,$ output $\bY_b \leftarrow \Enc_1(\bx,\sk)$.
        \end{enumerate}
    \end{weirdFrame}
    Denote the secret key used in $\Enc_b$ as $\sk_b$, where $b \in \{0,1\}$. 
    $\sk_b$ is either generated by the distinguisher (when $b = 1)$ or the encoding algorithm $\Enc_0$ (when $b = 0)$.
    The output of $\calD_1(\bx)$ is given to the adversarial channel $\calA$, who outputs  $\bY_b' = \bY'_{\msf{P},b} \circ \YStar'$.
    Then, in phase two, $\calD_2$ is given as input the message $\bx$, the secret key $\sk_b$, and the corrupt \paLDC\ codeword $\bY'_{\msf{P},b},$ where $b$ is the bit computed by $\calD_1(\bx).$
    \begin{weirdFrame}{$\calD_2(\bx,\sk_b,\bY'_{\msf{P},b})$}
        \begin{enumerate}
            \item Sample $L,R \unif [k]$ such that $L \leq R$ and $R - L + 1 \geq \kappa.$
            \item Compute $\bx_i' \leftarrow \paLDC.\Dec^{\bY'_{\msf{P},b}}_{\sk_b}(L,R)$.
            \item Output $b' = 0$ if $\bx_i \neq \bx_i'$, otherwise output $b' = 1.$
        \end{enumerate}
    \end{weirdFrame}
    Now, we give the two-phase distinguisher which breaks the $(\R,\epsPuz)$-hard puzzle. 
    For puzzle solutions $s_0,s_1$ we construct an adversary $\calB \in \R$ which distinguishes $(\bZ_b,\bZ_{1 - b},s_0,s_1)$ with probability at least $\eps'$ for $b \unif \{0,1\}$.
    Fix a message $\bx$ and $\lambda \in \N.$
    \begin{weirdFrame}{$\calB(\bZ_b,\bZ_{1 -b},s_0,s_1)$}
        \begin{enumerate}
            \item  \textbf{Encoding.}
        \begin{enumerate}
            \item Generate $\sk \leftarrow \paLDC.\Gen(1^\lambda;s_0)$.
            \item Set $\YStar \leftarrow \LDCStar.\Enc(\bZ_b)$.
            \item Set $\YP \leftarrow \paLDC.\Enc_{\sk}(\bx)$.
            \item Set $\bY = \YStar \circ \YP$.
        \end{enumerate}
        \item Give $(\bx,\bY,\delta,\eps,k,n)$ to $\calA$ to obtain $\bY'$. 
        Set $\YP'$ as the substring corresponding to the corruption of $\YP$ above.
        \item Compute $x'_L,\dots, x'_R \leftarrow \paLDC.\Dec_{\sk}^{\YP'}$ and if there exists $x'_i \neq x_i$ for some $i \in [L,R],$ output $b' = 0$. 
        Else output $b' = 1.$
        \end{enumerate}
    \end{weirdFrame}
    First, by assumption, the adversary $\calB \in \R$ since each of its subroutines $\calA, \paLDC.\Enc, \paLDC.\Dec,\LDCStar \in \R.$
    We claim that $\calB$ distinguishes $(\bZ_b,\bZ_{1 -b},s_0,s_1)$ with noticeable advantage. 
    First, observe that $\sk$ is always generated by $\paLDC.\Gen(1^\lambda,s_0).$
    Next, for $b = 1$, $\YStar$ encodes puzzle $\bZ_1$ and the secret key $\sk$ is unrelated to the solution $s_1$ of puzzle $\bZ_1.$
    Since in this case, $\sk$ and $\YStar$ are uncorrelated, $\calA$ causes a decoding error with probability $\epsP$ where $\bY'$ is at most $\delta n$ distance away from $\bY.$

    In the case that $b = 0,$ puzzle $\bZ_0$ is encoded as $\YStar$ with solution $s_0$ that is used to generate $\sk.$ 
    Thus, in this case, the probability that the decoder outputs at least one incorrect $x_i$ is at least the advantage of $\calA$, $\eps.$

    $\calB$ outputs $b' = b$ when $b = 0$ with probability $\eps \epsStar (1/k)$ by the argument above. 
    For $b = 1,$ this probability is at least $1 - \epsP \epsStar$ since the probability that $b' = 0$ in this case is at most $\epsP \epsStar.$ 
    Thus, we have that 
    \[\Pr[\calB(\bZ_b,\bZ_{1 -b},s_0,s_1) = b] - \frac{1}{2} \geq \frac{1}{2}\left(\eps \epsStar (1/k) - \epsP (1 - \epsStar)\right),\]
    which is non-negligible, contradicting the security of the puzzle \Puz.
\end{IEEEproof}