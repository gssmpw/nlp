\section{RELATED WORK}
\subsection{Surgical Simulators and Surgical Robotics}
Surgical simulator development has mainly been influenced by two streams. The first one originates from the need for surgical training and skills assessment within a simulated environment, with the users being surgeons, residents, and medical researchers.
Such simulators are typically capable of simulating a wide range of surgical training tasks realistically, such as cutting, suturing, and cauterization.
Examples of these are the commercial ones such as da Vinci SimNow\footnote{\url{https://www.intuitive.com/products-and-services/da-vinci/learning/simnow/}}
% and VirtaMed LaparoS\footnote{\url{https://www.virtamed.com/products-and-solutions/simulators/laparos}},
and the open-source ones such as iMSTK\footnote{\url{https://www.imstk.org/}}.
However, they are less used by robotics researchers due to their proprietary nature, the lack of real-time performance, or difficulty in integrating with existing software, such as the Robot Operating System (ROS).

The second stream arises from the recent trend in surgical robotics research, particularly for surgical automation and autonomy, which relies on simulators for benchmarking, synthesizing data, and trial-and-error learning.
Existing studies such as AMBF \cite{munawar2019RealTimeDynamic}, ATAR \cite{enayati2018RoboticAssistanceasNeeded}, and V-Rep Simulator for the dVRK \cite{fontanelli2018VREPSimulator} provide diverse applications for surgical robotics research.
Other work such as ORBIT-Surgical \cite{yu2024OrbitSurgicalOpenSimulationa}, Surgical Gym \cite{schmidgall2024SurgicalGym}, FF-SRL \cite{dallalba2024FFSRLHigh}, LapGym \cite{scheikl2023LapGymOpen}, AMBF-RL \cite{varier2022AMBFRLRealtime}, SurRoL \cite{xu2021SurRoLOpensource}, dVRL \cite{richter2020OpenSourcedReinforcement}, and UnityFlexML \cite{tagliabue2020SoftTissue} has specific focus on machine learning applications.
These platforms typically rely on an existing physics engine, such as PhysX (used in Nvidia Omniverse), FleX, SOFA, and Bullet.
Therefore, their simulation capability largely depends on the physics solvers. For instance, Bullet is primarily a rigid body engine and only provides limited FEM soft body support.
PhysX 5 supports FEM, but interactions such as cutting and suturing in soft bodies have not been realistically achieved, possibly due to the limitations of FEM.
A Recent study \cite{yang2024efficient} integrates basic MPM support into SurRoL using MLS-MPM but does not address cutting and suturing contact methods.
In contrast, our work focuses on developing a comprehensive engine-level MPM library specifically designed for simulating surgical soft body manipulations.

\subsection{Real-Time Soft Body Simulation}
The commonly used soft body simulation methods for real-time applications include FEM, MPM, and PBD.
As discussed in Section~\ref{sec:introduction}, FEM has several disadvantages when used for large-scale simulations in robotic research, including computational expenses, the need for real-time re-meshing for topological change, and inaccuracies and instabilities under large deformation.

PBD \cite{muller2007PositionBaseda} and extended PBD (XPBD) \cite{macklin2016XPBDPositionbased} are efficient and numerically stable alternatives, which solve only positional constraints without relying on force integration, making the simulation highly stable.
However, this also means less physically accurate behavior and the realism of deformations depends on the formulation of constraints rather than the underlying physics.
Although this is acceptable for computer animation, the application in biomedical and robotics research poses additional challenges, as careful parameter tuning is needed to match the simulation with real-world tissue behavior, as in \cite{liu2021RealtoSimRegistration}.
Additionally, forming positional constraints can sometimes be challenging for certain types of materials, such as viscoelastic or elastoplastic models, which is common in biomedical applications.

MPM overcomes many disadvantages of FEM while preserving accuracy and versatility for simulating a wide range of material behaviors.
MPM is considered within the particle-in-cell (PIC) methods family, where particles carry physical quantities and a background grid is used for computations.
The initial PIC method \cite{francish1964ParticleincellComputing} was developed for fluid mechanics, with later improvements by the fluid implicit particle method (FLIP) \cite{brackbill1986FLIPMethod}.
MPM extends the methods to simulating solids and multiphase materials, including elastic and hyperelastic materials, granular materials (\eg snow and sand), and elastoplastic materials such as biological tissues.
It is particularly famous for snow simulation in Disney's film \textit{Frozen} \cite{stomakhin2013MaterialPoint}.
In addition, since MPM does not explicitly use a mesh, topological changes are not a concern, making it easier to simulate cutting and fracture.
The MPM and the PIC methods are closely related and have numerous variations such as generalized interpolation material point (GIMP) \cite{bardenhagen1970GeneralizedInterpolation}, convected particle domain interpolation (CPDI) \cite{sadeghirad2011ConvectedParticle}, affine particle-in-cell (APIC) \cite{jiang2015AffineParticleincell}, moving least squares MPM (MLS-MPM) \cite{hu2018MovingLeast}, and more recently position-based MPM (PB-MPM) \cite{lewin2024PositionBased}.
However, the method is less investigated and adopted for surgical simulation, possibly due to its recent emergence and the lack of a high-performance library that supports a range of MPM formulations.