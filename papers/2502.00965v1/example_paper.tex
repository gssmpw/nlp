%%%%%%%% ICML 2025 EXAMPLE LATEX SUBMISSION FILE %%%%%%%%%%%%%%%%%

\documentclass{article}

% Recommended, but optional, packages for figures and better typesetting:
\usepackage{microtype}
\usepackage{arydshln}
\usepackage[table]{xcolor}
\usepackage{graphicx}
\usepackage{subfigure}
\usepackage{enumitem}
\usepackage{booktabs} % for professional tables

% hyperref makes hyperlinks in the resulting PDF.
% If your build breaks (sometimes temporarily if a hyperlink spans a page)
% please comment out the following usepackage line and replace
% \usepackage{icml2025} with \usepackage[nohyperref]{icml2025} above.
\usepackage{hyperref}
%\PassOptionsToPackage{table}{xcolor}

% Attempt to make hyperref and algorithmic work together better:
\newcommand{\theHalgorithm}{\arabic{algorithm}}

% Use the following line for the initial blind version submitted for review:
%\usepackage{icml2025}

% If accepted, instead use the following line for the camera-ready submission:
\usepackage[accepted]{icml2025}

% For theorems and such
\usepackage{amsmath}
\usepackage{amssymb}
\usepackage{bbm}
\usepackage{mathtools}
\usepackage{amsthm}
\usepackage{xspace}
\usepackage{multirow}
\usepackage{multicol}

% if you use cleveref..
\usepackage[capitalize,noabbrev]{cleveref}

%%%%%%%%%%%%%%%%%%%%%%%%%%%%%%%%
% THEOREMS
%%%%%%%%%%%%%%%%%%%%%%%%%%%%%%%%
\theoremstyle{plain}
\newtheorem{theorem}{Theorem}[section]
\newtheorem{proposition}[theorem]{Proposition}
\newtheorem{lemma}[theorem]{Lemma}
\newtheorem{corollary}[theorem]{Corollary}
\theoremstyle{definition}
\newtheorem{definition}[theorem]{Definition}
\newtheorem{assumption}[theorem]{Assumption}
\theoremstyle{remark}
\newtheorem{remark}[theorem]{Remark}

% Todonotes is useful during development; simply uncomment the next line
%    and comment out the line below the next line to turn off comments
%\usepackage[disable,textsize=tiny]{todonotes}
\usepackage[textsize=tiny]{todonotes}
\newcommand{\name}{CLIP-UP\xspace}

% The \icmltitle you define below is probably too long as a header.
% Therefore, a short form for the running title is supplied here:
\icmltitlerunning{\name: A Simple and Efficient Mixture-of-Experts CLIP Training Recipe with Sparse Upcycling}
\pagestyle{fancy}
\fancyhead[L]{Preprint}

\begin{document}

\twocolumn[
\icmltitle{\name: A Simple and Efficient Mixture-of-Experts CLIP Training  Recipe with Sparse Upcycling}

% It is OKAY to include author information, even for blind
% submissions: the style file will automatically remove it for you
% unless you've provided the [accepted] option to the icml2025
% package.

% List of affiliations: The first argument should be a (short)
% identifier you will use later to specify author affiliations
% Academic affiliations should list Department, University, City, Region, Country
% Industry affiliations should list Company, City, Region, Country

% You can specify symbols, otherwise they are numbered in order.
% Ideally, you should not use this facility. Affiliations will be numbered
% in order of appearance and this is the preferred way.
\icmlsetsymbol{equal}{*}

\begin{icmlauthorlist}
%\icmlauthor{Xinze Wang}{comp}
%\icmlauthor{Chen Chen}{comp}
%\icmlauthor{Yinfei Yang}{comp}
%\icmlauthor{Hong-You Chen}{comp}
%\icmlauthor{Bowen Zhang}{comp}
%\icmlauthor{Aditya Pal}{comp}
%\icmlauthor{Xiangxin Zhu}{comp}
%\icmlauthor{Xianzhi Du}{comp}


\icmlauthor{Xinze Wang}{}
\icmlauthor{Chen Chen}{}
\icmlauthor{Yinfei Yang}{}
\icmlauthor{Hong-You Chen}{}
\icmlauthor{Bowen Zhang}{}

\icmlauthor{Aditya Pal}{}
\icmlauthor{Xiangxin Zhu}{}
\icmlauthor{Xianzhi Du}{}

\setlength{\parskip}{10pt}
Apple

\end{icmlauthorlist}

% \icmlaffiliation{yyy}{Department of XXX, University of YYY, Location, Country}
%\icmlaffiliation{comp}{Apple, Cupertino, USA}
% \icmlaffiliation{sch}{School of ZZZ, Institute of WWW, Location, Country}

\icmlcorrespondingauthor{Xinze Wang}{xinze\_wang@apple.com}
% \icmlcorrespondingauthor{Firstname2 Lastname2}{first2.last2@www.uk}

% You may provide any keywords that you
% find helpful for describing your paper; these are used to populate
% the "keywords" metadata in the PDF but will not be shown in the document
\icmlkeywords{Machine Learning, ICML}

\vskip 0.3in
]
% this must go after the closing bracket ] following \twocolumn[ ...

% This command actually creates the footnote in the first column
% listing the affiliations and the copyright notice.
% The command takes one argument, which is text to display at the start of the footnote.
% The \icmlEqualContribution command is standard text for equal contribution.
% Remove it (just {}) if you do not need this facility.
\printAffiliationsAndNotice{}  % leave blank if no need to mention equal contribution
% \printAffiliationsAndNotice{\icmlEqualContribution}
%\printAffiliationsAndNotice{}% otherwise use the standard text.

\begin{abstract}
Mixture-of-Experts (MoE) models are crucial for scaling model capacity while controlling inference costs. While integrating MoE into multimodal models like CLIP improves performance, training these models is notoriously challenging and expensive. We propose \textbf{CLIP-Up}cycling ( \textbf{\name}), an efficient alternative training strategy that converts a pre-trained dense CLIP model into a sparse MoE architecture. Through extensive experimentation with various settings and auxiliary losses, we demonstrate that \name significantly reduces training complexity and cost. Remarkably, our sparse CLIP B/16 model, trained with \name, outperforms its dense counterpart by 7.2\% and 6.6\% on COCO and Flickr30k text-to-image Recall@1 benchmarks respectively. It even surpasses the larger CLIP L/14 model on this task while using only 30\% of the inference FLOPs. We further demonstrate the generalizability of our training recipe across different scales, establishing sparse upcycling as a practical and scalable approach for building efficient, high-performance CLIP models.

\end{abstract}

\section{Introduction}
\label{submission}

CLIP~\cite{radford2021learningtransferablevisualmodels,jia2021scaling} has proven to be a transformative model across a wide range of domains, including image classification, multimodal retrieval, and AI-driven multimodality content generation~\cite{Zhou_2022,rao2022densecliplanguageguideddenseprediction,gan2022visionlanguagepretrainingbasicsrecent,ramesh2021zeroshottexttoimagegeneration,liu2023visualinstructiontuning}. However, as demands on CLIP grow across these domains, scaling up the model becomes increasingly important to achieve better performance. Recent efforts to scale up CLIP have primarily focused on increasing the size of dense models \cite{Cherti_2023}, which, while effective, result in significant computational overhead and high inference costs.

\begin{figure}[t!]
%\vskip -0.1in
\begin{center}
\centerline{\includegraphics[width=1.0\columnwidth]{images/recipe_study.pdf}}
\caption{\textbf{Our proposed Mixture-of-Experts (MoE) CLIP pre-training recipe.} We identify several key ingredients for efficient and effective training including shared vs. separated image-text backbones, training from scratch vs. sparse upcycling from a dense model, use of auxiliary losses, and etc. A detailed analysis is provided in Section~\ref{comparison-methodology}. We conclude a simple strategy that outperforms previous practice~\cite{mustafa2022multimodalcontrastivelearninglimoe} with even less overall training cost.
}
\label{fig:strategy-compare}
\end{center}
\vskip -0.3in
\end{figure}

\begin{figure*}[ht]
\vskip -0.05in
\begin{center}
\centerline{\includegraphics[width=1.0\textwidth]{images/workflow.pdf}}
\caption{\name overview (Left) and sparse upcycling  initialization (Right). To upcycle from a pre-trained dense checkpoint (e.g., single expert), the MLP layers in some transformer blocks are replaced with MoE layers. During initialization, the MoE MLP parameters are copied from the original dense checkpoint. The MoE router is randomly initialized.}
\label{fig:clip-moe-flow}
\end{center}
\vskip -0.3in
\end{figure*}

Another promising approach to scaling CLIP is through sparse modeling, particularly by introducing Mixture-of-Expert~(MoE) layers~\cite{mustafa2022multimodalcontrastivelearninglimoe}. MoE is a technique that activates only a subset of the model's experts (parameters) per input, which reduces inference cost~\cite{shazeer2017outrageouslylargeneuralnetworks} compared to the dense model with the same number of parameters. MoE models have demonstrated great success in various tasks, including multimodal tasks~\cite{fedus2022switchtransformersscalingtrillion,lepikhin2020gshardscalinggiantmodels}. 
Despite these advances, there remains key problems in prohibitively high training cost of CLIP like models from scratch and the need of complicated auxiliary losses to stabilize the MoE model training. 
\citet{mustafa2022multimodalcontrastivelearninglimoe} investigated training the LIMOE model – an MoE CLIP model with a shared backbone for text and image encoders – from scratch. While it significantly outperforms the dense model, LIMOE demands substantial computational resources and introduces newly defined auxiliary losses for enhanced robustness. For instance, LIMOE with CLIP B/16 requires 1.35 times the training FLOPs of its dense counterpart.

In this work, we explore an alternative training strategy: upcycling a pre-trained dense CLIP model to a sparse MoE architecture. Sparse upcycling~\cite{komatsuzaki2023sparseupcyclingtrainingmixtureofexperts} leverages a well-trained dense model to initialize a sparse model with MoE layers. Figure~\ref{fig:clip-moe-flow} illustrates the process in detail. Our experiments start with the CLIP-B/16 architecture from LIMOE, initially trained from scratch with a shared backbone. We then further evaluate the impact of LIMOE auxiliary loss, shared versus separated backbones, and training from scratch versus sparse upcycling. As shown in Figure \ref{fig:strategy-compare}, sparse upcycling with a separated backbone yields the best quality metrics while significantly reducing training costs, lowering ZFLOPS from 4.2 to 3.7 compared to training from scratch. Incorporating LIMOE’s local and global entropy loss ~\cite{mustafa2022multimodalcontrastivelearninglimoe} substantially improves the performance of shared backbone trained from scratch but still lags behind other configurations. Section \ref{comparison-methodology} provides a detailed analysis of training strategies and the impact of LIMOE auxiliary loss under different setups.

% In this work, we explore an alternative training strategy: upcycling a pre-trained dense CLIP model to a sparse MoE architecture. 
% Our experiments, comparing models with shared versus separated backbones and trained via sparse upcycling versus training from scratch, reveal that the separated backbone with sparse upcycling yields the best results. We also observe that inititliazing from a pre-trained dense model greatly stabilizes the training process. In particular, while incorporating the local and global entry loss introduced by LIMOE~\cite{mustafa2022multimodalcontrastivelearninglimoe} substantially improves the performance of the shared backbone trained from scratch, it still lags behind other configurations. Notably, this auxiliary loss is no longer necessary for the other settings, could leading to even stronger models. The key findings are summarized in Figure \ref{fig:strategy-compare}.

In particular, we propose \textbf{\name}, a sparse upcycling approach~\cite{komatsuzaki2023sparseupcyclingtrainingmixtureofexperts} for CLIP models with MoE. As illustrated in Figure~\ref{fig:clip-moe-flow}, this method upgrades a pretrained CLIP into a sparsely activated MoE model with relatively low additional training cost. By leveraging the pretrained dense model, \name benefits from a warm start, significantly improving training efficiency. We demonstrate that \name outperforms both continued training of the original dense model and training a sparse model from scratch across different scales.

In summary, our contributions are as follows:
\begin{enumerate}[nosep,topsep=0pt,parsep=0pt,partopsep=0pt, leftmargin=*]
  \item  We introduce \name, a simple yet effective training recipe for Mixture-of-Experts (MoE) CLIP models leverages sparse upcycling from pre-trained weights, eliminating the need for complex auxiliary losses. \name outperforms existing MoE CLIP training methods, regardless of whether the backbone is shared or separated.  
  
  \item  \name achieves substantial performance improvements across various Text-Image retrieval benchmarks compared to the corresponding dense models. 
  Notably, \name with a B/16 backbone surpasses dense CLIP by 7.2\% and 5.5\% on recall@1 of COCO~\cite{lin2014microsoft} and Flickr30K~\cite{flickr30k} text-to-image retrieval tasks respectively.
  
  \item  We show \name is scalable to different model sizes (from B/32 to L/14) and provide a comprehensive analysis of key factors and challenges, providing valuable insights into the design decisions.
\end{enumerate}


\section{Related Work}
\label{related-work}

\textbf{Mixture-of-Experts.} Sparsely-activated MoE models have emerged as a powerful approach for scaling model architectures while maintaining computational efficiency~\cite{shazeer2017outrageouslylargeneuralnetworks,fedus2022switchtransformersscalingtrillion}. By activating only a subset of experts during inference, MoE models significantly reduce the required computation compared to dense models, without sacrificing performance~\cite{jiang2024mixtralexperts,dai2024deepseekmoeultimateexpertspecialization,xue2024openmoeearlyeffortopen}. However, the additional expert layers and the complexity of routing mechanisms required to select active experts during training make MoE models more resource-intensive to train from scratch.

Early pioneer work like LIMOE~\cite{mustafa2022multimodalcontrastivelearninglimoe} applies the MoE framework to CLIP models, using a shared structure for both text and image modalities. This design enables efficient learning of shared representations, leading to significant performance improvements across various multimodal tasks and outperforming compute-matched dense models.

Despite its advantages, LIMOE shares a common limitation with other MoE models: the need for substantial computational resources during training~\cite{du2024revisitingmoedensespeedaccuracy}. 
% For instance, LIMOE with CLIP B/16 requires 1.35 times the training FLOPs compared to its dense counterpart. 
As shown in Figure \ref{fig:strategy-compare}, training with a shared backbone from scratch yields suboptimal results compared to alternative backbone selections and training strategies. While LIMOE auxiliary loss offers some improvement, the overall performance remains inferior, suggesting that the methodology may not be the most effective approach.

% Additionally, it introduces a more complex routing mechanism, Balanced Partition Routing (BPR) (\cite{riquelme2021scalingvisionsparsemixture}), and two more auxiliary losses, to improve training stability. BPR is more intricate than the simpler top-2 routing with "first-come-first-serve" logic commonly used in traditional MoE models (\cite{lepikhin2020gshardscalinggiantmodels}, \cite{fedus2022switchtransformersscalingtrillion}), as well as the more complicated auxiliary losses, increasing implementation complexity and training overhead.

\textbf{Sparse upcycling.} Sparse upcycling~\cite{komatsuzaki2023sparseupcyclingtrainingmixtureofexperts} is a technique that effectively reduce training cost of sparse models by reusing existing dense model checkpoints. By leveraging the prior knowledge from a well-trained dense model, the sparse version benefits from accelerated convergence and reduced computational overhead~\cite{he2024upcyclinglargelanguagemodels}.

The current applications of sparse upcycling have been primarily focused on single-modality models~\cite{komatsuzaki2023sparseupcyclingtrainingmixtureofexperts,he2024upcyclinglargelanguagemodels}. The effectiveness of this approach in the context of multimodal models remains largely unexplored. The concurrent work CLIP-MoE~\cite{zhang2024clipmoebuildingmixtureexperts} upcycles MoE layers by initializing them with fine-tuned MLP layers via cluster-and-contrast learning. Each expert requires an additional training stage. Scaling this method is challenging, as adding more experts increases complexity and necessitates additional training stages.
% The concurrent work of CLIP-MoE~\cite{zhang2024clipmoebuildingmixtureexperts} employs diversified multiplet upcycling for CLIP using MoE. Its training involves two main stages. First, the MLP layers of the base CLIP model are fine-tuned through a cluster-and-contrast process in Multistage Contrastive Learning. Second, the initialized experts from the MLP layers are used to fine-tune the resulting CLIP-MoE with both contrastive learning loss and router balance loss. However, scaling this method is challenging, as adding more experts increases complexity and necessitates additional training stages.

In this work, we extend sparse upcycling to CLIP with MoE using a single-stage training process. This approach significantly reduces training complexity and cost while achieving superior results, like shown in Figure \ref{fig:strategy-compare}. We also demonstrate its generalizability across shared or separated backbones 
% various architectures 
and model scales. This establishes sparse upcycling as a practical and scalable solution for building efficient, high-performance CLIP models.

% {\color{red} Concurrent work \cite{zhang2024clipmoebuildingmixtureexperts}.}

\section{Sparse Upcycling with Mixture-of-Experts}
\label{method}

Figure \ref{fig:clip-moe-flow} outlines the \name architecture and its upcycle training strategy. The left panel showcases the traditional CLIP contrastive training with expert layers replacing some MLP layers. The right panel demonstrates the innovative upcycle training approach. We'll dive into the specifics of each in this section.


\subsection{CLIP}
Given $n$ pairs of image and text captions $\{(\mathbf{I}_j, \mathbf{T}_j)\}_{j=1}^{n}$, CLIP~\cite{radford2021learningtransferablevisualmodels} learns image representations as $f(\mathbf{I}_j)$ and text representation as $g(\mathbf{T}_j)$ via contrastive learning objective within a training batch. Relevant pairs are positioned closer in the feature space, while irrelevant pairs are pushed farther apart. Given batch size of $B$ and a learned temperature parameter $\theta$, the image to text loss is
\[
\mathcal{L}_{I-T} = - \frac{1}{B} \sum_{j=1}^B \log \frac{e^{\text{sim}(f(\mathbf{I}_j), g(\mathbf{T}_j)) / \theta}}{\sum_{k=1}^{B} e^{\text{sim}(f(\mathbf{I}_j), g(\mathbf{T}_k)) / \theta}}
\tag{1}
\]

and text to image loss is
\[
\mathcal{L}_{T-I} = - \frac{1}{B} \sum_{j=1}^B \log \frac{e^{\text{sim}(f(\mathbf{I}_j), g(\mathbf{T}_j)) / \theta}}{\sum_{k=1}^{B} e^{\text{sim}(f(\mathbf{I}_k), g(\mathbf{T}_j)) / \theta}}
\tag{2}
\]

The contrastive loss is calculated by averaging the above two losses
\[
\mathcal{L}_\text{Contrastive} = \frac{1}{2} (L_{I-T} + L_{T-I})
\tag{3}
\]

\subsection{CLIP with Mixture-of-Experts upcycling} 
In \name, for half of the transformer layers, we replace the dense MLPs with sparsely-gated MoE layers as an alternating dense-sparse-dense structure, similar to Figure~\ref{fig:clip-moe-flow} (Left). This helps the stability of feedforwarding and prevents gradient oscillation, especially during initial training. 


Each MoE layer contains $E$ experts and a router. The experts are MLPs, selectively activated based on the input. The router selects the top-$K$ experts from $E$ total experts, by predicting normalized gating logits for each input token. Let $\mathbf{X}_j \in \mathbb{R}^{D}$ denotes the input of $j$-th token to an MoE layer,  $\mathbf{G}_j \in \mathbb{R}^{D \times E}$ is the gating logis for $E$ experts, $\mathbf{W}_e \in \mathbb{R}^{D}$ is the router weight matrix for expert $e$, the output $MoE(\mathbf{X}_j)$ is computed as follows:
\[
MoE(\mathbf{X}_j) = \mathbf{X}_j + \sum_{e \in \text{Top-K}}g_{e,j}MLP_e(\mathbf{X}_j)
\]
\[
g_{e,j} = 
\begin{cases}
    \mathbf{G}_{e,j}, &\mathbf{G}_{e,j} \in \text{Top-K}({\mathbf{G}_{e,j} | 1 \leq e \leq E}, K), \tag{4} \\
    0, & \text{otherwise},
\end{cases}
\]
\[
\mathbf{G}_{e,j} = Softmax(\mathbf{W}_e^T \mathbf{X}_j)
\]

% \[
% \hspace{-2.0 cm} MoE(\mathbf{X}_j) = \mathbf{X}_j + \sum_{e=1}^{E} g_{e,j}FFN_e(\mathbf{X}_j)
% \]
% \[
% \hspace{0 cm} g_{e,j} = 
% \begin{cases}
%     \mathbf{G}_{e,j}, &\mathbf{G}_{e,j} \in TopK({\mathbf{G}_{e,j} | 1 \leq e \leq E}, K), \tag{2} \\
%     0, & \text{otherwise},
% \end{cases}
% \]
% \[
% \hspace{-2.0 cm} \mathbf{G}_{e,j} = Softmax(\mathbf{W}_e^T \mathbf{X}_j)
% \]
Each expert is assigned a fixed buffer capacity~\cite{fedus2022switchtransformersscalingtrillion}, meaning it can process only a limited number of tokens at a time. With predefined capacity factor $C$, and $B_t$ denotes the number of tokens per batch, the capacity size of each expert $B_e$ is computed as $B_e = (B_t / E) \times C$. This constraint enhances computational efficiency by preventing experts from being overwhelmed, while also ensuring that hardware and memory resources are managed effectively.

Tokens are assigned to experts on a "first-come-first-serve" basis, meaning they are processed in the order they arrive~\cite{fedus2022switchtransformersscalingtrillion}. This simple and effective mechanism handles input streams effectively without introducing additional complexity or overhead from token prioritization or sorting, ensuring that the system remains efficient while distributing tokens across experts.

\textbf{Auxiliary loss.} While the simplified token assignment reduces overhead, it presents the challenge of maintaining a balanced token distribution across all experts. Poor distribution can result in significant token dropping~\cite{zeng2024turnwasteworthrectifying}, which can severely impact performance. Therefore, achieving balance is essential to avoid overloading or underutilizing certain experts. Overuse of some experts may lead to overfitting, while others may not receive enough data for effective training.


To mitigate this, we follow~\cite{zoph2022stmoedesigningstabletransferable} to introduce an auxiliary loss comprising load balance loss and router $z$-loss for each modality, promoting even token distribution. Specifically, the load balance loss encourages the gating mechanism to allocate tokens more uniformly over $E$ total experts. For a sequence of length $S$, the load balance loss is defined as:
\[
\mathcal{L}_\text{Balance} = \sum_{e=1}^E R_e \cdot P_e
\]
\[
R_e = \frac{E}{K \cdot S} \sum_{j=1}^S \mathbbm{1}(\text{Token $j$ dispatched to Expert $e$})
\]
\[
P_e = \frac{1}{S} \sum_{j=1}^S \mathbf{G}_{e,j}
 \tag{5}
\]
% \[
% \hspace{-4.2 cm} \mathcal{L}_{Balance} = \alpha \cdot \sum_{e=1}^E R_e \cdot P_e
% \]
% \[
% \hspace{1.0 cm} R_e = \frac{E}{K \cdot S} \sum_{j=1}^S \mathbbm{1}(\text{Token $j$ dispatched to Expert $e$}) \tag{5}
% \]
% \[
% \hspace{-3.7 cm} P_e = \frac{1}{S} \sum_{j=1}^S \mathbf{G}_{e,j}
% \]
% where $\mathbf{G}_{e,j}$ is the gating logits of Expert $e$ for Token $j$, 
where $R_e$ represents the token assignment ratio to expert $e$, and $P_e$ denotes the average router probability of expert $e$.
% , and $\alpha$ is a hyper-parameter controlling the weight of load balance loss.

The router $z$-loss is to stabilize the gating process by regularizing the output of the router logits, and it ensures the router outputs stay in a reasonable range. It is defined as:
\[
\mathcal{L}_\text{Router} = \frac{1}{S} \sum_{j=1}^{S} \left( \log \sum_{e=1}^{E} e^{\mathbf{G}_{e,j}} \right)^2
\tag{6}
\]
% where $\beta$ is the scaling factor of router-$z$ loss. 

The final training loss is the sum of the contrastive loss and auxiliary loss, with $\alpha$ and $\beta$ as scaling factors,
\[
\label{eq:final_loss}
\mathcal{L} = \mathcal{L}_\text{Contrastive} + \alpha \cdot \mathcal{L}_\text{Balance} + \beta \cdot \mathcal{L}_\text{Router}
\tag{7}
\]


\textbf{LIMOE auxiliary loss.} LIMOE~\cite{mustafa2022multimodalcontrastivelearninglimoe} proposes two new auxiliary losses, the local entropy loss and the global entropy loss, which are applied on a per-modality basis. The local entropy loss is defined as
\[
\mathcal{L}_\text{Local} = \frac{1}{S} \sum_{j=1}^{S} \left( -\sum_{e=1}^{E} \mathbf{G}_{e,j} \log \mathbf{G}_{e,j} \right)
\tag{8}
\]

This encourages concentrated router weights but may reduce the diversity of expert utilization. The global entropy loss is defined as
\[
\mathcal{L}_\text{Global} = -\sum_{e=1}^{E} P_e \log P_e
\tag{9}
\]

which promotes a balanced expert distribution, and enhance diversity in expert usage. To prevent a single modality from overusing too many experts, the global entropy loss is thresholded as:
\[
\mathcal{L}_\text{Global}^{\tau} = \max \{0, \tau + \mathcal{L}_\text{Global} \}
\tag{10}
\]

where $\tau = \log(m)$, ensuring that at least $m$ experts are utilized to minimize the loss. These auxiliary losses stabilize training and improve performance, as demonstrated by~\cite{mustafa2022multimodalcontrastivelearninglimoe}.

According to the best recipe of LIMOE, applying the total training loss becomes:

\begin{equation}
\begin{aligned}
\mathcal{L} = & \mathcal{L}_\text{Contrastive} + \alpha \cdot \mathcal{L}_\text{Balance} +  \beta \cdot \mathcal{L}_\text{Router} \\ 
 & + \gamma \cdot \mathcal{L}^\text{Text}_\text{Local} + \lambda \cdot \mathcal{L}^{\tau, \text{Image}}_\text{Global} + \delta \cdot \mathcal{L}^{\tau, \text{Text}}_\text{Global}
\end{aligned}
\tag{11}
\end{equation}

In our experiments, we replaced the auxiliary loss with the LIMOE training recipe and tuned the hyperparameters for the best perfomrance. While the results demonstrate that the LIMOE auxiliary loss significantly enhances the performance of the shared backbone trained from scratch, its performance still lags behind other configurations and the same recipe doesn't provide similar improvements in those settings. So we chose equation \ref{eq:final_loss} as our final loss. More details will be illustrated in Section \ref{sec:exp}.

% 
\begin{table*}[h]
\caption{Comparing performance of different combination of shared versus separated backbone and training from scratch versus sparse upcycling, and the impact of adding LIMOE auxiliary loss. We benchmark the models on ImageNet Zero-shot classification (Accuracy@1 $\times$\%), COCO and Flickr30k Image retrieval (Recall@1 $\times$\%). T2I: Text-to-Image retrieval; I2T: Image-to-Text retrieval. To isolate the impact of the model's inherent performance, we \textbf{bold} the best model in each task without using on the LIMOE auxiliary loss (the rows with gray background).}
\label{tab:methodology-compare}
\vskip 0.15in
\begin{center}
\begin{small}
\begin{sc}
% \setlength{\tabcolsep}{3.5 pt}
% \renewcommand{\arraystretch}{0.8}
\begin{tabular}{lccccc}
\toprule
\multirow{2}{*}{\multicolumn{1}{l}{Model}} &\multicolumn{1}{c}{Imagenet} &\multicolumn{2}{c}{COCO} &\multicolumn{2}{c}{Flickr30K} 
& &Accuray@1 &T2I R@1 &I2T R@1 &T2I R@1 &I2T R@1 \\
\midrule
\rowcolor{lightgray} 
Shared & 69.7 &46.7 &65.6 &71.4 &86.3 \rule{0pt}{3ex} \\
\quad+LIMOE Aux. Loss &73.1 &49.7 &69.7 &75.6 &87.9 \rule{0pt}{3ex} \\
\quad $\Delta$ & +3.4 & +3.0 & +4.1 & +4.2 & +1.6 \\
\cdashline{1-6}
\rowcolor{lightgray} 
Shared-UpCycle &75.2 &51.6 &\textbf{72.7} &78.0 &92.0 \rule{0pt}{3ex} \\
\quad+LIMOE Aux. Loss &73.9 &50.9 &70.1 &73.8 &85.5 \rule{0pt}{3ex} \\
\quad $\Delta$ & -1.3 & -0.7 & -2.6 & -4.2 & -6.5 \\
\cdashline{1-6}
\rowcolor{lightgray} 
Separated &74.5 &\textbf{53.1} &70.6 &78.3 &88.2 \rule{0pt}{3ex} \\
\quad+LIMOE Aux. Loss &72.6 &46.4 &62.6 &73.4 &85.2 \rule{0pt}{3ex} \\
\quad $\Delta$ & -2.0 & -6.7 & -8.0 & -4.9 & -3.0 \\
\cdashline{1-6}
\rowcolor{lightgray} 
Separated-UpCycle &\textbf{76.9} &52.1 &71.5 &\textbf{80.9} &\textbf{92.3} \rule{0pt}{3ex} \\
\quad+LIMOE Aux. Loss &75.9 &52.9 &73.5 &81.3 &92.5 \rule{0pt}{3ex} \\
\quad $\Delta$ & -1.0 & +0.8 & +2.0 & +0.4 & +0.2 \\
\bottomrule
\end{tabular}
\end{sc}
\end{small}
\end{center}
\end{table*}
% \begin{table*}[h]
% \caption{Comparing performance of different combination of shared versus separated backbone and training from scratch versus sparse upcycling. We benchmark the models on ImageNet Zero-shot classification (Accuracy@1 $\times$100), MSCOCO and Flickr30k Image retrieval (Recall@1 $\times$100). T2I: Text-to-Image retrieval; I2T: Image-to-Text retrieval.}
% \label{tab:methodology-compare-no-limoe}
% \vskip 0.15in
% \begin{center}
% \begin{small}
% \begin{sc}
% % \setlength{\tabcolsep}{3.5 pt}
% % \renewcommand{\arraystretch}{0.8}
% \begin{tabular}{lccccc}
% \toprule
% \multirow{2}{*}{\multicolumn{1}{l}{Model}} &\multicolumn{1}{c}{Imagenet} &\multicolumn{2}{c}{MSCOCO} &\multicolumn{2}{c}{Flickr30K} 
% & &Accuray@1 &T2I R@1 &I2T R@1 &T2I R@1 &I2T R@1 \\
% \midrule
% % \rowcolor{lightgray} 
% Shared & 69.7 &46.7 &65.6 &71.4 &86.3 \rule{0pt}{3ex} \\
% % \quad+LIMOE Aux. Loss &73.1 &49.7 &69.7 &75.6 &87.9 \rule{0pt}{3ex} \\
% % \quad $\Delta$ & +3.4 & +3.0 & +4.1 & +4.2 & +1.6 \\
% % \cdashline{1-6}
% % \rowcolor{lightgray} 
% Shared-UpCycle &75.2 &51.6 &\textbf{72.7} &78.0 &92.0 \rule{0pt}{3ex} \\
% % \quad+LIMOE Aux. Loss &73.9 &50.9 &70.1 &73.8 &85.5 \rule{0pt}{3ex} \\
% % \quad $\Delta$ & -1.3 & -0.7 & -2.6 & -4.2 & -6.5 \\
% % \cdashline{1-6}
% % \rowcolor{lightgray} 
% Separated &74.5 &\textbf{53.1} &70.6 &78.3 &88.2 \rule{0pt}{3ex} \\
% % \quad+LIMOE Aux. Loss &72.6 &46.4 &62.6 &73.4 &85.2 \rule{0pt}{3ex} \\
% % \quad $\Delta$ & -2.0 & -6.7 & -8.0 & -4.9 & -3.0 \\
% % \cdashline{1-6}
% % \rowcolor{lightgray} 
% Separated-UpCycle &\textbf{76.9} &52.1 &71.5 &\textbf{80.9} &\textbf{92.3} \rule{0pt}{3ex} \\
% % \quad+LIMOE Aux. Loss &75.9 &52.9 &73.5 &81.3 &92.5 \rule{0pt}{3ex} \\
% % \quad $\Delta$ & -1.0 & +0.8 & +2.0 & +0.4 & +0.2 \\
% \bottomrule
% \end{tabular}
% \end{sc}
% \end{small}
% \end{center}
% \end{table*}


\begin{table}[t]
%\vskip -0.1in
\caption{Ablation study of combinations of shared vs. separated backbone and training from scratch vs. sparse upcycling. We benchmark the models on ImageNet Zero-shot classification (Accuracy@1 \%); COCO and Flickr30k text-to-image (T2I) / image-to-text (I2T) retrieval (Recall@1 \%).}
\label{tab:methodology-compare-no-limoe}
\begin{center}
%\vskip -0.5pt
\begin{small}
\begin{sc}
\setlength{\tabcolsep}{1.4 pt}
% \renewcommand{\arraystretch}{0.8}
\begin{tabular}{lc|ccccc}
\toprule
\multirow{3}{*}{Backbone} & \multirow{3}{*}{Upcycle?} & Imagenet &\multicolumn{2}{c}{COCO} &\multicolumn{2}{c}{Flickr30K} \\ 
& & & T2I &I2T &T2I &I2T \\
& & Acc@1 &R@1 &R@1 &R@1 &R@1 \\
\midrule
% \rowcolor{lightgray} 
Shared & N & 69.7 &46.7 &65.6 &71.4 &86.3 \rule{0pt}{3ex} \\
% \quad+LIMOE Aux. Loss &73.1 &49.7 &69.7 &75.6 &87.9 \rule{0pt}{3ex} \\
% \quad $\Delta$ & +3.4 & +3.0 & +4.1 & +4.2 & +1.6 \\
% \cdashline{1-6}
% \rowcolor{lightgray} 
Shared & Y & 75.2 &51.6 &\textbf{72.7} &78.0 &92.0 \rule{0pt}{3ex} \\
% \quad+LIMOE Aux. Loss &73.9 &50.9 &70.1 &73.8 &85.5 \rule{0pt}{3ex} \\
% \quad $\Delta$ & -1.3 & -0.7 & -2.6 & -4.2 & -6.5 \\
% \cdashline{1-6}
% \rowcolor{lightgray} 
Separated & N & 74.5 &\textbf{53.1} &70.6 &78.3 &88.2 \rule{0pt}{3ex} \\
% \quad+LIMOE Aux. Loss &72.6 &46.4 &62.6 &73.4 &85.2 \rule{0pt}{3ex} \\
% \quad $\Delta$ & -2.0 & -6.7 & -8.0 & -4.9 & -3.0 \\
% \cdashline{1-6}
% \rowcolor{lightgray} 
Separated & Y & \textbf{76.9} &52.1 &71.5 &\textbf{80.9} &\textbf{92.3} \rule{0pt}{3ex} \\
% \quad+LIMOE Aux. Loss &75.9 &52.9 &73.5 &81.3 &92.5 \rule{0pt}{3ex} \\
% \quad $\Delta$ & -1.0 & +0.8 & +2.0 & +0.4 & +0.2 \\
\bottomrule
\end{tabular}
\vskip -0.2in
\end{sc}
\end{small}
\end{center}
\end{table}


\subsection{Sparse Upcycling Training}

The sparse upcycling process starts from a pre-trained dense CLIP model. The MoE encoders in the upcycled CLIP follow the architecture of the transformer used in~\cite{komatsuzaki2023sparseupcyclingtrainingmixtureofexperts} including the number of layers and feature size, similarly for our separated backbone setup.
% including the number of layers and Transformer blocks, 
A selective subset of MLP layers within the Transformer is expanded into MoE layers, with each expert initialized using the same MLP weights from the dense MLP layer of the dense CLIP model, while the router weights are randomly initialized. All other layers, such as attention and embedding layers, are copied directly from the original model. This approach ensures that only a sparse part of the model is modified, while most parameters remain unchanged, preserving the model’s core functionality. After initialization, the upcycled model undergoes additional training steps with the original hyperparameters like batch size, while slightly reducing the learning rate and weight decay to improve performance and stabilize training. The upcycle training process is illustrated in the right panel of figure \ref{fig:clip-moe-flow}.

\begin{figure}[h]  % Create a figure environment
    \centering  % Center the image
    \includegraphics[width=1.0\linewidth]{images/limoe-compare.pdf}
    \vspace{-10pt}
    \caption{Impact of LIMOE auxiliary loss on different training configurations. We observe adding LIMOE loss could sometimes lead to instability issue especially when the backbones are not shared, while our upcycling recipe are more robust.} 
    \label{fig:limoe-compare}  % Add a label for referencing
    \vskip -0.1in
\end{figure}

\section{Experiments}
\label{sec:exp}

\paragraph{Datasets.} For training, we used the same paired image-text dataset for the initial dense CLIP checkpoint training, \name, as well as the baseline dense CLIP model. The training data includes WIT-3000M~\citep{wu2024mofilearningimagerepresentations} and DFN-5B~\citep{fang2023datafilteringnetworks}. For evaluation, we used the standard ImageNet~\citep{deng2009imagenet,pmlr-v119-shankar20c} image classification task and COCO~\citep{lin2014microsoft} and Flickr30K~\citep{flickr30k} image-to-text and text-to-image retrieval tasks. The input image resolution is 224 for all of the datasets. 

\paragraph{Setup.}
We begin by training a dense CLIP model for 440k steps, which serves as the initial stage. Following this, we refine the dense model into an MoE version using the proposed upcycle training, then continuously training it for an additional 350k steps. Both stages use the AdamW optimizer with a batch size of 32k. For the dense model, we apply a learning rate of $5 \times 10^{-4}$ and a weight decay of 0.2. During the upcycle training phase, the learning rate is reduced to $5 \times 10^{-5}$, and the weight decay to 0.05. In the MoE model, half of the MLP layers in the Transformer are replaced with MoE layers, creating an alternating [dense, sparse] structure for both modalities. Each MoE layer includes 8 experts, with the router selecting the top 2 experts for activation. During training, we choose $\alpha = 0.01$ and $\beta = 0.001$ in the router auxiliary loss to ensure balanced expert distribution without dominating the primary cross-entropy objective. Hyper-parameters were fine-tuned based on preliminary experiments to improve performance and ensure training stability. 

\begin{table*}[t]
% \small
\vskip -0.1in
\caption{Comparing performance of \name and baseline models including LIMOE~\cite{mustafa2022multimodalcontrastivelearninglimoe} across different model sizes. The sparse upcycling leverages the CLIP dense checkpoint trained for 440k steps. \name is further trained for additional 350k steps (790k steps in total). We also train CLIP with 790k steps to have comparable training steps to \name.}
\label{zero-shot-metrics-flops-steps}
\begin{center}
%\vskip -10pt
\small
\begin{sc}
\setlength{\tabcolsep}{5pt}
%\renewcommand{\arraystretch}{1.15}
%\resizebox{1.0\textwidth}{!}{
\begin{tabular}{l|c|c|cc|cccc|cccc}
\toprule
\multirow{3}{*}{Model} &\multirow{3}{*}{Steps} &\multirow{3}{*}{Inference} &\multicolumn{2}{c|}{Imagenet} &\multicolumn{4}{c|}{COCO retrieval} &\multicolumn{4}{c}{Flickr30k retrieval} \\
& & & \multicolumn{2}{c|}{classification} &T2I &T2I &I2T &I2T &T2I &T2I &I2T &I2T \\
&(k) &GFLOPS &Acc@1 &Acc@5 &R@1 &R@5 &R@1 &R@5 &R@1 &R@5 &R@1 &R@5 \\
\midrule
\multicolumn{13}{c}{\bf \textit{B/32}} \\
\midrule
OpenAI-CLIP &- &14.8 &63.2 &88.8 &30.8 &55.9 &51.6 &75.7 &- &- &- &- \rule{0pt}{3ex} \\
LIMOE &- &22.3 &67.5 &- &31.0 &- &45.7 &- &- &- &- &- \rule{0pt}{3ex} \\
\cdashline{1-13}
CLIP (ours) &440 &14.8 &72.4 &92.8 &41.7 &68.2 &62.3 &84.3 &68.0 &90.0 &\textbf{86.5} &\textbf{97.7} \rule{0pt}{3ex} \\
CLIP (ours) &790 &14.8 &72.4 &92.8 &41.9 &67.8 &62.4 &84.1 &67.8 &88.7 &85.6 &96.4 \rule{0pt}{3ex} \\
\rowcolor{lightblue}
\name & 790 &19.6 &\textbf{73.2} &\textbf{93.3} &\textbf{47.3} &\textbf{74.0} &\textbf{66.6} &\textbf{86.7} &\textbf{72.9} &\textbf{91.9} &85.9 &96.9 \rule{0pt}{3ex} \\
\midrule
\multicolumn{13}{c}{\bf \textit{B/16}} \\
\midrule
OpenAI-CLIP &- &41.2 &68.4 &91.9 &33.1 &58.4 &53.8 &77.9 &- &- &- &- \rule{0pt}{3ex} \\
LIMOE &- &48.7 &73.7 &- &36.2 &- &51.3 &- &- &- &- &-\rule{0pt}{3ex} \\
\cdashline{1-13}
CLIP (ours) &440 &41.2 &76.0 &94.7 &44.4 &70.3 &65.7 &87.0 &73.6 &92.1 &88.0 &97.8 \rule{0pt}{3ex} \\
CLIP (ours) &790 &41.2 &76.8 &\textbf{95.1} &44.9 &70.8 &66.0 &86.6 &74.3 &92.7 &88.9 &98.0 \rule{0pt}{3ex} \\
\rowcolor{lightblue}
\name & 790 &54.3 &\textbf{76.9} &\textbf{95.1} &\textbf{52.1} &\textbf{77.6} &\textbf{71.5} &\textbf{89.2}  &\textbf{80.9} &\textbf{95.6} &\textbf{92.3} &\textbf{99.2} \rule{0pt}{3ex} \\
\midrule
\multicolumn{13}{c}{\bf \textit{L/14}} \\
\midrule
OpenAI-CLIP &- &175.5 &75.3 &94.5 &36.1 &60.8 &57.7 &79.1 &- &- &- &- \rule{0pt}{3ex} \\
\cdashline{1-13}
CLIP (ours) &440 &175.5 &81.1 &96.4 &49.6 &74.4 &70.9 &89.6 &78.4 &94.7 &91.9 &\textbf{99.2} \rule{0pt}{3ex} \\
CLIP (ours) &790 &175.5 &\textbf{81.6} &\textbf{96.6} &50.2 &75.2 &71.4 &89.9 &79.3 &94.9 &91.7 &99.0 \rule{0pt}{3ex} \\
\rowcolor{lightblue}
\name & 790 &231.7 &81.2 & \textbf{96.6} &\textbf{53.9} &\textbf{79.4} &\textbf{73.8} &\textbf{92.0}  &\textbf{82.0} &\textbf{96.1} &\textbf{92.4} &99.1 \rule{0pt}{3ex} \\
\bottomrule
\end{tabular}
%}
\end{sc}
\end{center}
\end{table*}


Additionally, we train another dense CLIP model for a total of 790k steps to enable a fair comparison. This model is trained with the same configuration as mentioned earlier, using a learning rate of  $5 \times 10^{-4}$ and a weight decay of 0.2 to optimize performance.

\subsection{Recipe Study}
\label{comparison-methodology}
We first investigate the impact of using shared versus separated backbones and the effect of training from scratch compared to employing sparse upcycling. For a fair comparison, the shared backbone configuration uses 16 experts, matching the total number of experts (8 each for text and image) in the separated backbone setup. All experiments are conducted with the CLIP-B/16 model. All results of these models on ImageNet zero-shot classification and COCO/Flickr30k image-text retrieval tasks are illustrated in table \ref{tab:methodology-compare-no-limoe}. 

% \paragraph{Results}
The results show that the separated backbone combined with sparse upcycling delivers the best overall performance, achieving the highest ImageNet Top-1 accuracy of 76.9\% , highest Flickr30K T2I and I2T recall@1 of 80.9\% and 92.3\%, and competitive results in COCO. Separated backbones generally outperform shared ones, likely because of the additional parameters dedicated to each modality. Notably, sparse upcycling provides greater improvements to shared backbones than to separated ones. Compared to shared models trained from scratch, sparse upcycling boosts ImageNet accuracy by 5.5\% and COCO/Flickr retrieval by up to 7.1\%, whereas the separated backbone achieves gains of 2.4\% on ImageNet and up to 4.1\% on retrieval tasks. Sparse upcycling proves effectiveness across both shared and separated architectures, consistently outperforming models trained from scratch. These results highlight the versatility and efficiency of \name across different backbone configurations.

\begin{figure*}[ht]  % Create a figure environment
%\vskip -0.1in
    \centering  % Center the image
    \includegraphics[width=0.95\linewidth]{images/scratch-compare.pdf}
%    \vskip -0.1in
    \caption{Performance comparison along with training EFLOPS of \name and training from scratch on CLIP B/16.}
    \label{fig:scratch-comparison}  % Add a label for referencing
\end{figure*} 

\begin{figure*}[ht]  % Create a figure environment
    \centering  % Center the image
    \vskip -0.1in
    \includegraphics[width=0.95\linewidth]{images/moe-modality.pdf}
%    \vskip -0.1in
    \caption{\name with MoE upcycling for only the text encoder, image encoder, or both. We observe upcycling both the image and text encoders into MoE generally helps, especially for retrieval tasks.} 
    \label{fig:moe-modality}  % Add a label for referencing
    \vskip -0.1in
\end{figure*}

We further explore the impact of adding LIMOE specific auxiliary loss described in Section \ref{method}. Specifically, we set $\tau = 6$ to incentivize at least 6 experts (uniformly) or more (not necessarily in a uniform way)~\cite{mustafa2022multimodalcontrastivelearninglimoe} are used for each modality and fine-tuned the loss weight under our experiment setup. As shown in Figure \ref{fig:limoe-compare}, applying the LIMOE auxiliary loss on the shared backbone trained from scratch significantly improves both ImageNet classification and COCO retrieval metrics, which is aligned with the finding from original LIMOE paper~\cite{mustafa2022multimodalcontrastivelearninglimoe}. However, its performance remains below other configurations without applying LIMOE specific auxiliary losses. 
Adding LIMOE specific loss to best configuration (Separated-Upcycle) achieves a slightly better text-image retrieval metrics but falls short on ImageNet zero-shot classficaition.


In addition, we also find adding these new auxiliary losses make the model hard to train as there are much more hyper-parameters to tune. To our best effort, we cannot make LIMOE specific auxiliary loss work in all configurations, e.g. in Shared-Upcycle and Separated settings. 
Due to the limit of resources, we decide to use the loss in equation \ref{eq:final_loss} as the final training objective to reduce the effort of finding best hyperparameters.

\subsection{Final Model Evaluation and Baselines}

According to the results from the previous section, we select the separted backbone with sparse upcycling as the default recipe and further study its effectiness by training \name models in different model sizes, from B/32 to L/14. Table \ref{zero-shot-metrics-flops-steps} compares the zero-shot image classification and retrieval performance of \name against baseline dense CLIP model trained for an equivalent number of steps across different scales. Extending training steps for dense CLIP from 440k to 790k yields only minor gains. In contrast, \name demonstrates substantial improvements across all metrics. These gains are consistent across model scales, particularly for image-text retrieval tasks. Remarkably, \name B/32, using only 47\% of the inference GFLOPS, outperforms the baseline dense CLIP B/16 by 2.4\% in COCO T2I recall@1 and matches its I2T recall@1. Similarly, \name B/16, requiring just 31\% of the inference GFLOPS, surpasses the baseline dense CLIP L/14 by 1.9\% in COCO T2I recall@1 while matching its I2T recall@1. This highlights the effectiveness of training high-performance CLIP models by leveraging pretrained dense models through sparse upcycling with MoE.

\subsection{Training Efficiency}

To better illustrate the effectiveness of upcycle training, we also show the training process from dense pretraining + upcycling compared to training from scratch in Figure \ref{fig:scratch-comparison}. The initial dense model training provides a strong foundation for subsequent upcycling. In contrast, training from scratch without leveraging the pretrained dense checkpoint, results in much higher computational costs to reach the performance level of \name. This difference is especially noticeable for COCO image-to-text and ImageNet classification tasks. While sparse upcycling introduces an initial quality drop on ImageNet due to model reconfiguration~\cite{komatsuzaki2023sparseupcyclingtrainingmixtureofexperts}, \name consistently outperforms the model trained from scratch, highlighting its efficiency in reducing training resources and improving overall performance.

\section{Discussion and More Ablations} \label{ablations}

In this section, we discuss important architecture and training ablations relative to the configuration used during experiment. We use the CLIP B/16 with 8 experts as the default configuration here, expert capacity factor $C = 2.0$, and 6 MoE layers interspersed in the Transformer for both modalities. The hyper-parameters are same as Section \ref{sec:exp}.

% \subsection{Training Efficiency}

% To better illustrate the effectiveness of upcycle training, we also show the training process from dense pretraining + upcycling compared to training from scratch in Figure \ref{fig:scratch-comparison}. The initial dense model training provides a strong foundation for subsequent upcycling. In contrast, training from scratch without leveraging the pretrained dense checkpoint, results in much higher computational costs to reach the performance level of \name. This difference is especially noticeable for COCO image-to-text and ImageNet classification tasks. While sparse upcycling introduces an initial quality drop on ImageNet due to model reconfiguration~\cite{komatsuzaki2023sparseupcyclingtrainingmixtureofexperts}, \name consistently outperforms the model trained from scratch, highlighting its efficiency in reducing training resources and improving overall performance.


% \subsection{Shared architecture across modalities} 

% Table \ref{tab:methodology-compare} demonstrates that \name performs effectively with both shared and separated architectures for image and text modalities. This section further compares \name under shared or separated backbone with dense and sparse model trained from scratch with equivalent steps. 

% \begin{figure}[ht]  % Create a figure environment
%     \centering  % Center the image
%     \includegraphics[width=1.0\linewidth]{images/share-sep-clip-compare.pdf}
%     \caption{Model performance comparison between shared and separated architecture across modalities.} 
%     \label{fig:shared-architecture}  % Add a label for referencing
% \end{figure}

% As shown in Figure \ref{fig:shared-architecture}, the dense and sparse model is trained from scratch with equivalent 790k steps under shared or separated backbones. \name is trained for 350k steps, initialized from a dense checkpoint trained for 440k steps. The separated backbone delivers better results on all metrics for both dense or sparse models trained from scratch, due to the increased model parameters. However, with sparse upcycling, the shared architecture significantly improves, outperforming the separated architecture in COCO image-to-text top-1 retrieval. Moreover, the shared backbone achieves greater improvement over the baseline dense CLIP model trained for the same steps. Compared to the dense CLIP, \name with a shared backbone improves COCO image-to-text top-1 retrieval by 7.82\% and text-to-image by 8.9\%, while the separated backbone achieves 5.54\% and 7.15\% improvements, respectively. These results confirm that \name is particularly beneficial for shared architectures, offering a viable solution for use cases with limited inference resources and memory.

% \name can perform effectively with both shared and separate architectures for image and text modalities as shown in \ref{comparison-methodology}. In this section, we did some detailed analysis of the impact regarding to shared or separated architecture across modalities. In terms of final performance as shown in Figure \ref{fig:shared-architecture}, \name with a shared architecture outperforms the separate architecture in COCO image-to-text top-1 retrieval. However, the separate architecture yields better results in text-to-image retrieval and ImageNet top-1 accuracy. Although the separate architecture generally achieves higher metrics than the shared one, the shared architecture demonstrates greater improvement over the baseline dense CLIP model trained for the same number of steps. Compared to the dense CLIP, \name with a shared backbone improves COCO image-to-text top-1 retrieval by 7.82\% and text-to-image by 8.9\%, while the separate backbone achieves improvements of 5.54\% and 7.15\%, respectively. These results confirm the versatility and effectiveness of \name across both shared and separate architectures.

% \subsection{Impact of Adding LIMOE Specific Auxiliary Loss}

% \begin{figure}[ht]  % Create a figure environment
%     \centering  % Center the image
%     \includegraphics[width=0.8\linewidth]{images/limoe-compare.pdf}
%     \caption{Comparison of LIMOE auxiliary loss impact under different training configurations} 
%     \label{fig:limoe-compare}  % Add a label for referencing
% \end{figure}

% We further explore the impact of adding LIMOE specific auxiliary loss described in Section \ref{method}. Specifically, we set $\tau = 6$ to ensure at least 6 experts are used for each modality and fine-tuned the loss weight under our experiment setup. As shown in Figure \ref{fig:limoe-compare}, applying the LIMOE auxiliary loss on the shared backbone trained from scratch significantly improves both ImageNet classification and MSCOCO retrieval metrics, which is aligned with the finding from original LIMOE paper~\cite{mustafa2022multimodalcontrastivelearninglimoe}. However, its performance remains below other configurations without applying LIMOE specific auxiliary losses. 
% Adding LIMOE specific loss to best configuration (Separated-UpCycle) achieves a slightly better text-image retrieval metrics but falls short on ImageNet zero-shot classficaition.


% In addition, we also find adding these new auxiliary losses make the model hard to train as there are much more hyper-parameters to tune. To our best effort, we cannot make LIMOE specific auxiliary loss work in all configurations.
% % , e.g. in Shared-Upcycle and Separated settings. 
% Due to the limit of resources, we decide to use the loss in equation \ref{eq:final_loss} as the final training objective to reduce the effort of finding best hyperparameters.
% % in the following experiments.

\subsection{MoE added to single or multiple modalities} 

In the \name model, the MoE setup is applied to both the text encoder and image encoder. We also explore the effect of adding MoE layers to only one modality while keeping the other modality fully dense. The COCO retrieval and ImageNet results for these configurations are shown in Figure \ref{fig:moe-modality}, measured across different training steps.

We observe that the initial performance of newly upcycled models tends to decline compared to the starting dense model, regardless of where the MoE setup is applied. However, all configurations recover after approximately 5k training steps. Applying MoE layers to both modalities leads to a more significant initial drop on average. When MoE layers are applied to only one modality (either image or text), the final performance remains comparable. Notably, the initial performance drop is more pronounced when MoE layers are applied to the image modality, suggesting that the model is more sensitive to changes in image representations.

\subsection{Expert capacity factor}

\begin{figure}[h]  % Create a figure environment
    \centering  % Center the image
    \includegraphics[width=1.0\linewidth]{images/capacity-compare.pdf}
    \vskip -0.1in
    \caption{Effects of different expert capacity $C$.} 
    \label{fig:capacity-compare}  % Add a label for referencing
    %\vskip -3pt
\end{figure}

Intuitively, the number of tokens processed by each expert plays a crucial role in determining model quality. In \name, this is controlled by the expert capacity factor, denoted as $C$. A higher $C$ results in less token dropping, thereby reducing the initial quality drop. However, this doesn't necessarily guarantee a higher final model quality. By default, we set the capacity factor to 2.0 for both modalities. As shown in Figure \ref{fig:capacity-compare}, increasing $C_{image}$ to 4.0 significantly boosts the ImageNet zero-shot metrics. However, this adjustment results in a noticeable drop in performance on COCO and Flickr30k retrieval tasks.
% We leave the study of capacity factor for different tasks to a future work.

\begin{figure}[ht]  % Create a figure environment
    %\vskip -0.1in
    \centering  % Center the image
    \begin{subfigure}
        \centering
        \includegraphics[width=1.0\linewidth]{images/coco_dropped_image_token.pdf}
    \end{subfigure}
    \begin{subfigure}
        \centering
        \includegraphics[width=1.0\linewidth]{images/imagenet_dropped_image_token.pdf}
    \end{subfigure}
    \vskip -0.1in
    \caption{Visualization of image tokens dropped by the router (i.e., not being assigned to any expert due to capacity constraint) on COCO (Top) and ImageNet (Bottom).}
    %\vskip -0.05in
    \label{fig:imagenet-dropped-image-token}  % Add a label for referencing
\end{figure}

Figure \ref{fig:imagenet-dropped-image-token} visualizes the behavior of dropped image tokens under different capacity settings of COCO and Imagenet. The red squares represent the dropped image tokens. With a higher $C_{image}$, fewer image tokens are discarded. This benefits ImageNet performance since the dataset primarily consists of single-object images. Retaining more tokens allows the model to focus on the key features. In contrast, COCO images often depict complex scenes with multiple objects, but not all of which are relevant to the paired captions. Dropping less important image tokens helps the model concentrate on the most important objects, which explains the drop in COCO performance when fewer tokens are discarded.
We leave the further study of capacity factor for different tasks to a future work.

\section{Conclusions}
\label{conclusion}
In this work, we introduce \name, a simple yet efficient training strategy for CLIP that integrates MoE with sparse upcycling. Through extensive experiments on various configurations and auxiliary losses, we show that \name significantly reduces training costs while consistently improving retrieval and classification performance across multiple model scales. Notably, \name outperforms larger dense models in some tasks while drastically cutting inference FLOPs, highlighting sparse upcycling as a practical and scalable approach for building efficient, high-performance CLIP models. Additionally, ablation studies provide comprehensive insights into key design decisions.

% In this work, we introduce \name, a simple and efficient training strategy for CLIP that combines MoE with sparse upcycling. Through extensive experiments across various configurations and auxiliary losses, we show that \name significantly reduces training costs while consistently enhancing performance on retrieval and classification tasks across multiple model scales. Remarkably, \name outperforms larger dense models in certain tasks while drastically reducing inference FLOPs. 

% For instance, the sparse CLIP B/16 backbone surpasses the dense CLIP L/14 by 1.9\% in COCO T2I R@1 retrieval, using only 30\% of the inference FLOPs. Compared to a sparse model with the same architecture trained from scratch, our method achieves superior results with significantly lower training costs. 
% % Ablation studies on different MoE configurations further confirm consistent performance gains across all setups.

\clearpage

\section*{Acknowledgement}
We thank Wentao Wu, Haotian Zhang, and many others for their invaluable help and feedback. 

%\section{Impact Statement}
%
%Our work introduces \name, a simple and efficient strategy to convert dense CLIP models into sparse MoE architectures. \name significantly reduces training complexity and inference costs while simultaneously improving model performance. We have demonstrated its generalizability across different model scales, establishing sparse upcycling as a practical approach for building efficient CLIP models.
%
%This paper advances the field of efficient training strategies for CLIP models. By reducing computational demands, our method contributes to more sustainable pretraining, leading to lower energy consumption and reduced environmental impact. Additionally, by enabling high-performance models to be trained and deployed with fewer resources, our work has the potential to democratize Machine Learning research, making powerful visual foundation models accessible to a wider range of institutions and users with limited computational capabilities.
%
%Overall, our work represents a step forward in sustainable Machine Learning model training, with broad implications for both research and industry adoption of efficient multimodal systems.

% \subsection{Submitting Papers}

% \textbf{Anonymous Submission:} ICML uses double-blind review: no identifying
% author information may appear on the title page or in the paper
% itself. \cref{author info} gives further details.

% \medskip

% Authors must provide their manuscripts in \textbf{PDF} format.
% Furthermore, please make sure that files contain only embedded Type-1 fonts
% (e.g.,~using the program \texttt{pdffonts} in linux or using
% File/DocumentProperties/Fonts in Acrobat). Other fonts (like Type-3)
% might come from graphics files imported into the document.

% Authors using \textbf{Word} must convert their document to PDF\@. Most
% of the latest versions of Word have the facility to do this
% automatically. Submissions will not be accepted in Word format or any
% format other than PDF\@. Really. We're not joking. Don't send Word.

% Those who use \textbf{\LaTeX} should avoid including Type-3 fonts.
% Those using \texttt{latex} and \texttt{dvips} may need the following
% two commands:

% {\footnotesize
% \begin{verbatim}
% dvips -Ppdf -tletter -G0 -o paper.ps paper.dvi
% ps2pdf paper.ps
% \end{verbatim}}
% It is a zero following the ``-G'', which tells dvips to use
% the config.pdf file. Newer \TeX\ distributions don't always need this
% option.

% Using \texttt{pdflatex} rather than \texttt{latex}, often gives better
% results. This program avoids the Type-3 font problem, and supports more
% advanced features in the \texttt{microtype} package.

% \textbf{Graphics files} should be a reasonable size, and included from
% an appropriate format. Use vector formats (.eps/.pdf) for plots,
% lossless bitmap formats (.png) for raster graphics with sharp lines, and
% jpeg for photo-like images.

% The style file uses the \texttt{hyperref} package to make clickable
% links in documents. If this causes problems for you, add
% \texttt{nohyperref} as one of the options to the \texttt{icml2025}
% usepackage statement.


% \subsection{Submitting Final Camera-Ready Copy}

% The final versions of papers accepted for publication should follow the
% same format and naming convention as initial submissions, except that
% author information (names and affiliations) should be given. See
% \cref{final author} for formatting instructions.

% The footnote, ``Preliminary work. Under review by the International
% Conference on Machine Learning (ICML). Do not distribute.'' must be
% modified to ``\textit{Proceedings of the
% $\mathit{42}^{nd}$ International Conference on Machine Learning},
% Vancouver, Canada, PMLR 267, 2025.
% Copyright 2025 by the author(s).''

% For those using the \textbf{\LaTeX} style file, this change (and others) is
% handled automatically by simply changing
% $\mathtt{\backslash usepackage\{icml2025\}}$ to
% $$\mathtt{\backslash usepackage[accepted]\{icml2025\}}$$
% Authors using \textbf{Word} must edit the
% footnote on the first page of the document themselves.

% Camera-ready copies should have the title of the paper as running head
% on each page except the first one. The running title consists of a
% single line centered above a horizontal rule which is $1$~point thick.
% The running head should be centered, bold and in $9$~point type. The
% rule should be $10$~points above the main text. For those using the
% \textbf{\LaTeX} style file, the original title is automatically set as running
% head using the \texttt{fancyhdr} package which is included in the ICML
% 2025 style file package. In case that the original title exceeds the
% size restrictions, a shorter form can be supplied by using

% \verb|\icmltitlerunning{...}|

% just before $\mathtt{\backslash begin\{document\}}$.
% Authors using \textbf{Word} must edit the header of the document themselves.

% \section{Format of the Paper}

% All submissions must follow the specified format.

% \subsection{Dimensions}




% The text of the paper should be formatted in two columns, with an
% overall width of 6.75~inches, height of 9.0~inches, and 0.25~inches
% between the columns. The left margin should be 0.75~inches and the top
% margin 1.0~inch (2.54~cm). The right and bottom margins will depend on
% whether you print on US letter or A4 paper, but all final versions
% must be produced for US letter size.
% Do not write anything on the margins.

% The paper body should be set in 10~point type with a vertical spacing
% of 11~points. Please use Times typeface throughout the text.

% \subsection{Title}

% The paper title should be set in 14~point bold type and centered
% between two horizontal rules that are 1~point thick, with 1.0~inch
% between the top rule and the top edge of the page. Capitalize the
% first letter of content words and put the rest of the title in lower
% case.

% \subsection{Author Information for Submission}
% \label{author info}

% ICML uses double-blind review, so author information must not appear. If
% you are using \LaTeX\/ and the \texttt{icml2025.sty} file, use
% \verb+\icmlauthor{...}+ to specify authors and \verb+\icmlaffiliation{...}+ to specify affiliations. (Read the TeX code used to produce this document for an example usage.) The author information
% will not be printed unless \texttt{accepted} is passed as an argument to the
% style file.
% Submissions that include the author information will not
% be reviewed.

% \subsubsection{Self-Citations}

% If you are citing published papers for which you are an author, refer
% to yourself in the third person. In particular, do not use phrases
% that reveal your identity (e.g., ``in previous work \cite{langley00}, we
% have shown \ldots'').

% Do not anonymize citations in the reference section. The only exception are manuscripts that are
% not yet published (e.g., under submission). If you choose to refer to
% such unpublished manuscripts \cite{anonymous}, anonymized copies have
% to be submitted
% as Supplementary Material via OpenReview\@. However, keep in mind that an ICML
% paper should be self contained and should contain sufficient detail
% for the reviewers to evaluate the work. In particular, reviewers are
% not required to look at the Supplementary Material when writing their
% review (they are not required to look at more than the first $8$ pages of the submitted document).

% \subsubsection{Camera-Ready Author Information}
% \label{final author}

% If a paper is accepted, a final camera-ready copy must be prepared.
% %
% For camera-ready papers, author information should start 0.3~inches below the
% bottom rule surrounding the title. The authors' names should appear in 10~point
% bold type, in a row, separated by white space, and centered. Author names should
% not be broken across lines. Unbolded superscripted numbers, starting 1, should
% be used to refer to affiliations.

% Affiliations should be numbered in the order of appearance. A single footnote
% block of text should be used to list all the affiliations. (Academic
% affiliations should list Department, University, City, State/Region, Country.
% Similarly for industrial affiliations.)

% Each distinct affiliations should be listed once. If an author has multiple
% affiliations, multiple superscripts should be placed after the name, separated
% by thin spaces. If the authors would like to highlight equal contribution by
% multiple first authors, those authors should have an asterisk placed after their
% name in superscript, and the term ``\textsuperscript{*}Equal contribution"
% should be placed in the footnote block ahead of the list of affiliations. A
% list of corresponding authors and their emails (in the format Full Name
% \textless{}email@domain.com\textgreater{}) can follow the list of affiliations.
% Ideally only one or two names should be listed.

% A sample file with author names is included in the ICML2025 style file
% package. Turn on the \texttt{[accepted]} option to the stylefile to
% see the names rendered. All of the guidelines above are implemented
% by the \LaTeX\ style file.

% \subsection{Abstract}

% The paper abstract should begin in the left column, 0.4~inches below the final
% address. The heading `Abstract' should be centered, bold, and in 11~point type.
% The abstract body should use 10~point type, with a vertical spacing of
% 11~points, and should be indented 0.25~inches more than normal on left-hand and
% right-hand margins. Insert 0.4~inches of blank space after the body. Keep your
% abstract brief and self-contained, limiting it to one paragraph and roughly 4--6
% sentences. Gross violations will require correction at the camera-ready phase.

% \subsection{Partitioning the Text}

% You should organize your paper into sections and paragraphs to help
% readers place a structure on the material and understand its
% contributions.

% \subsubsection{Sections and Subsections}

% Section headings should be numbered, flush left, and set in 11~pt bold
% type with the content words capitalized. Leave 0.25~inches of space
% before the heading and 0.15~inches after the heading.

% Similarly, subsection headings should be numbered, flush left, and set
% in 10~pt bold type with the content words capitalized. Leave
% 0.2~inches of space before the heading and 0.13~inches afterward.

% Finally, subsubsection headings should be numbered, flush left, and
% set in 10~pt small caps with the content words capitalized. Leave
% 0.18~inches of space before the heading and 0.1~inches after the
% heading.

% Please use no more than three levels of headings.

% \subsubsection{Paragraphs and Footnotes}

% Within each section or subsection, you should further partition the
% paper into paragraphs. Do not indent the first line of a given
% paragraph, but insert a blank line between succeeding ones.

% You can use footnotes\footnote{Footnotes
% should be complete sentences.} to provide readers with additional
% information about a topic without interrupting the flow of the paper.
% Indicate footnotes with a number in the text where the point is most
% relevant. Place the footnote in 9~point type at the bottom of the
% column in which it appears. Precede the first footnote in a column
% with a horizontal rule of 0.8~inches.\footnote{Multiple footnotes can
% appear in each column, in the same order as they appear in the text,
% but spread them across columns and pages if possible.}

% \begin{figure}[ht]
% \vskip 0.2in
% \begin{center}
% \centerline{\includegraphics[width=\columnwidth]{icml_numpapers}}
% \caption{Historical locations and number of accepted papers for International
% Machine Learning Conferences (ICML 1993 -- ICML 2008) and International
% Workshops on Machine Learning (ML 1988 -- ML 1992). At the time this figure was
% produced, the number of accepted papers for ICML 2008 was unknown and instead
% estimated.}
% \label{icml-historical}
% \end{center}
% \vskip -0.2in
% \end{figure}

% \subsection{Figures}

% You may want to include figures in the paper to illustrate
% your approach and results. Such artwork should be centered,
% legible, and separated from the text. Lines should be dark and at
% least 0.5~points thick for purposes of reproduction, and text should
% not appear on a gray background.

% Label all distinct components of each figure. If the figure takes the
% form of a graph, then give a name for each axis and include a legend
% that briefly describes each curve. Do not include a title inside the
% figure; instead, the caption should serve this function.

% Number figures sequentially, placing the figure number and caption
% \emph{after} the graphics, with at least 0.1~inches of space before
% the caption and 0.1~inches after it, as in
% \cref{icml-historical}. The figure caption should be set in
% 9~point type and centered unless it runs two or more lines, in which
% case it should be flush left. You may float figures to the top or
% bottom of a column, and you may set wide figures across both columns
% (use the environment \texttt{figure*} in \LaTeX). Always place
% two-column figures at the top or bottom of the page.

% \subsection{Algorithms}

% If you are using \LaTeX, please use the ``algorithm'' and ``algorithmic''
% environments to format pseudocode. These require
% the corresponding stylefiles, algorithm.sty and
% algorithmic.sty, which are supplied with this package.
% \cref{alg:example} shows an example.

% \begin{algorithm}[tb]
%    \caption{Bubble Sort}
%    \label{alg:example}
% \begin{algorithmic}
%    \STATE {\bfseries Input:} data $x_i$, size $m$
%    \REPEAT
%    \STATE Initialize $noChange = true$.
%    \FOR{$i=1$ {\bfseries to} $m-1$}
%    \IF{$x_i > x_{i+1}$}
%    \STATE Swap $x_i$ and $x_{i+1}$
%    \STATE $noChange = false$
%    \ENDIF
%    \ENDFOR
%    \UNTIL{$noChange$ is $true$}
% \end{algorithmic}
% \end{algorithm}

% \subsection{Tables}

% You may also want to include tables that summarize material. Like
% figures, these should be centered, legible, and numbered consecutively.
% However, place the title \emph{above} the table with at least
% 0.1~inches of space before the title and the same after it, as in
% \cref{sample-table}. The table title should be set in 9~point
% type and centered unless it runs two or more lines, in which case it
% should be flush left.

% % Note use of \abovespace and \belowspace to get reasonable spacing
% % above and below tabular lines.

% \begin{table}[t]
% \caption{Classification accuracies for naive Bayes and flexible
% Bayes on various data sets.}
% \label{sample-table}
% \vskip 0.15in
% \begin{center}
% \begin{small}
% \begin{sc}
% \begin{tabular}{lcccr}
% \toprule
% Data set & Naive & Flexible & Better? \\
% \midrule
% Breast    & 95.9$\pm$ 0.2& 96.7$\pm$ 0.2& $\surd$ \\
% Cleveland & 83.3$\pm$ 0.6& 80.0$\pm$ 0.6& $\times$\\
% Glass2    & 61.9$\pm$ 1.4& 83.8$\pm$ 0.7& $\surd$ \\
% Credit    & 74.8$\pm$ 0.5& 78.3$\pm$ 0.6&         \\
% Horse     & 73.3$\pm$ 0.9& 69.7$\pm$ 1.0& $\times$\\
% Meta      & 67.1$\pm$ 0.6& 76.5$\pm$ 0.5& $\surd$ \\
% Pima      & 75.1$\pm$ 0.6& 73.9$\pm$ 0.5&         \\
% Vehicle   & 44.9$\pm$ 0.6& 61.5$\pm$ 0.4& $\surd$ \\
% \bottomrule
% \end{tabular}
% \end{sc}
% \end{small}
% \end{center}
% \vskip -0.1in
% \end{table}

% Tables contain textual material, whereas figures contain graphical material.
% Specify the contents of each row and column in the table's topmost
% row. Again, you may float tables to a column's top or bottom, and set
% wide tables across both columns. Place two-column tables at the
% top or bottom of the page.

% \subsection{Theorems and such}
% The preferred way is to number definitions, propositions, lemmas, etc. consecutively, within sections, as shown below.
% \begin{definition}
% \label{def:inj}
% A function $f:X \to Y$ is injective if for any $x,y\in X$ different, $f(x)\ne f(y)$.
% \end{definition}
% Using \cref{def:inj} we immediate get the following result:
% \begin{proposition}
% If $f$ is injective mapping a set $X$ to another set $Y$, 
% the cardinality of $Y$ is at least as large as that of $X$
% \end{proposition}
% \begin{proof} 
% Left as an exercise to the reader. 
% \end{proof}
% \cref{lem:usefullemma} stated next will prove to be useful.
% \begin{lemma}
% \label{lem:usefullemma}
% For any $f:X \to Y$ and $g:Y\to Z$ injective functions, $f \circ g$ is injective.
% \end{lemma}
% \begin{theorem}
% \label{thm:bigtheorem}
% If $f:X\to Y$ is bijective, the cardinality of $X$ and $Y$ are the same.
% \end{theorem}
% An easy corollary of \cref{thm:bigtheorem} is the following:
% \begin{corollary}
% If $f:X\to Y$ is bijective, 
% the cardinality of $X$ is at least as large as that of $Y$.
% \end{corollary}
% \begin{assumption}
% The set $X$ is finite.
% \label{ass:xfinite}
% \end{assumption}
% \begin{remark}
% According to some, it is only the finite case (cf. \cref{ass:xfinite}) that is interesting.
% \end{remark}
% %restatable

% \subsection{Citations and References}

% Please use APA reference format regardless of your formatter
% or word processor. If you rely on the \LaTeX\/ bibliographic
% facility, use \texttt{natbib.sty} and \texttt{icml2025.bst}
% included in the style-file package to obtain this format.

% Citations within the text should include the authors' last names and
% year. If the authors' names are included in the sentence, place only
% the year in parentheses, for example when referencing Arthur Samuel's
% pioneering work \yrcite{Samuel59}. Otherwise place the entire
% reference in parentheses with the authors and year separated by a
% comma \cite{Samuel59}. List multiple references separated by
% semicolons \cite{kearns89,Samuel59,mitchell80}. Use the `et~al.'
% construct only for citations with three or more authors or after
% listing all authors to a publication in an earlier reference \cite{MachineLearningI}.

% Authors should cite their own work in the third person
% in the initial version of their paper submitted for blind review.
% Please refer to \cref{author info} for detailed instructions on how to
% cite your own papers.

% Use an unnumbered first-level section heading for the references, and use a
% hanging indent style, with the first line of the reference flush against the
% left margin and subsequent lines indented by 10 points. The references at the
% end of this document give examples for journal articles \cite{Samuel59},
% conference publications \cite{langley00}, book chapters \cite{Newell81}, books
% \cite{DudaHart2nd}, edited volumes \cite{MachineLearningI}, technical reports
% \cite{mitchell80}, and dissertations \cite{kearns89}.

% Alphabetize references by the surnames of the first authors, with
% single author entries preceding multiple author entries. Order
% references for the same authors by year of publication, with the
% earliest first. Make sure that each reference includes all relevant
% information (e.g., page numbers).

% Please put some effort into making references complete, presentable, and
% consistent, e.g. use the actual current name of authors.
% If using bibtex, please protect capital letters of names and
% abbreviations in titles, for example, use \{B\}ayesian or \{L\}ipschitz
% in your .bib file.

% \section*{Accessibility}
% Authors are kindly asked to make their submissions as accessible as possible for everyone including people with disabilities and sensory or neurological differences.
% Tips of how to achieve this and what to pay attention to will be provided on the conference website \url{http://icml.cc/}.

% \section*{Software and Data}

% If a paper is accepted, we strongly encourage the publication of software and data with the
% camera-ready version of the paper whenever appropriate. This can be
% done by including a URL in the camera-ready copy. However, \textbf{do not}
% include URLs that reveal your institution or identity in your
% submission for review. Instead, provide an anonymous URL or upload
% the material as ``Supplementary Material'' into the OpenReview reviewing
% system. Note that reviewers are not required to look at this material
% when writing their review.

% % Acknowledgements should only appear in the accepted version.
% \section*{Acknowledgements}

% \textbf{Do not} include acknowledgements in the initial version of
% the paper submitted for blind review.

% If a paper is accepted, the final camera-ready version can (and
% usually should) include acknowledgements.  Such acknowledgements
% should be placed at the end of the section, in an unnumbered section
% that does not count towards the paper page limit. Typically, this will 
% include thanks to reviewers who gave useful comments, to colleagues 
% who contributed to the ideas, and to funding agencies and corporate 
% sponsors that provided financial support.

% \section*{Impact Statement}

% Authors are \textbf{required} to include a statement of the potential 
% broader impact of their work, including its ethical aspects and future 
% societal consequences. This statement should be in an unnumbered 
% section at the end of the paper (co-located with Acknowledgements -- 
% the two may appear in either order, but both must be before References), 
% and does not count toward the paper page limit. In many cases, where 
% the ethical impacts and expected societal implications are those that 
% are well established when advancing the field of Machine Learning, 
% substantial discussion is not required, and a simple statement such 
% as the following will suffice:

% ``This paper presents work whose goal is to advance the field of 
% Machine Learning. There are many potential societal consequences 
% of our work, none which we feel must be specifically highlighted here.''

% The above statement can be used verbatim in such cases, but we 
% encourage authors to think about whether there is content which does 
% warrant further discussion, as this statement will be apparent if the 
% paper is later flagged for ethics review.


% In the unusual situation where you want a paper to appear in the
% references without citing it in the main text, use \nocite
\nocite{langley00}

%\bibliography{example_paper}
%\bibliographystyle{icml2025}

%%%%%%%% ICML 2025 EXAMPLE LATEX SUBMISSION FILE %%%%%%%%%%%%%%%%%

\documentclass{article}

% Recommended, but optional, packages for figures and better typesetting:
\usepackage{microtype}
%\usepackage{geometry}
\usepackage{graphicx}
%\usepackage{subfigure}
\usepackage[subrefformat=parens]{subcaption}
\usepackage{float}
\usepackage{booktabs} % for professional tables

% hyperref makes hyperlinks in the resulting PDF.
% If your build breaks (sometimes temporarily if a hyperlink spans a page)
% please comment out the following usepackage line and replace
% \usepackage{icml2025} with \usepackage[nohyperref]{icml2025} above.
\usepackage{hyperref}


% Attempt to make hyperref and algorithmic work together better:
\newcommand{\theHalgorithm}{\arabic{algorithm}}

% Use the following line for the initial blind version submitted for review:
%\usepackage{icml2025}

% If accepted, instead use the following line for the camera-ready submission:
\usepackage[accepted]{icml2025}

% For theorems and such
\usepackage{amsmath}
\usepackage{amssymb}
\usepackage{mathtools}
\usepackage{amsthm}
\usepackage{url}
\usepackage{booktabs}       % professional-quality tables
\usepackage{amsfonts}       % blackboard math symbols
\usepackage{nicefrac}       % compact symbols for 1/2, etc.
\usepackage{microtype}      % microtypography
\usepackage{xcolor}         % colors
\usepackage{longtable}
\usepackage{afterpage}
\usepackage{makecell}
\usepackage{booktabs}
\usepackage{tabularx}
\usepackage{authblk}
%\usepackage{subcaption}
\usepackage{wrapfig}
\usepackage{multirow}
\usepackage{multicol}
\usepackage{utfsym}
\usepackage{arydshln}
\usepackage{listings}


%\usepackage[ruled, vlined]{algorithm2e}
%\usepackage[linesnumbered,ruled,vlined]{algorithm2e}
%\usepackage{algorithmic}
%\usepackage{algpseudocode}
% if you use cleveref..
\usepackage[capitalize,noabbrev]{cleveref}

%%%%%%%%%%%%%%%%%%%%%%%%%%%%%%%%
% THEOREMS
%%%%%%%%%%%%%%%%%%%%%%%%%%%%%%%%
\theoremstyle{plain}
\newtheorem{theorem}{Theorem}[section]
\newtheorem{proposition}[theorem]{Proposition}
\newtheorem{lemma}[theorem]{Lemma}
\newtheorem{corollary}[theorem]{Corollary}
\theoremstyle{definition}
\newtheorem{definition}[theorem]{Definition}
\newtheorem{assumption}[theorem]{Assumption}
\theoremstyle{remark}
\newtheorem{remark}[theorem]{Remark}
\newcommand{\benchmark}{{UGPhysics}}
\newcommand{\judge}{{MARJ}}
\newcommand{\change}[1]{\textcolor{blue}{#1}}
%\NewDocumentCommand{\todo}{ mO{} }{\textcolor{blue}{\textsuperscript{\textit{todo}}\textsf{\textbf{\small[#1]}}}}
% Todonotes is useful during development; simply uncomment the next line
%    and comment out the line below the next line to turn off comments
%\usepackage[disable,textsize=tiny]{todonotes}
\usepackage[textsize=tiny]{todonotes}

\newcommand{\shizhe}[1]{\textcolor{orange}{\textbf{[Shizhe: #1]}}}

% The \icmltitle you define below is probably too long as a header.
% Therefore, a short form for the running title is supplied here:
\icmltitlerunning{{\benchmark}: A Comprehensive Benchmark for Undergraduate Physics Reasoning with Large Language Models}

\begin{document}

\twocolumn[
\icmltitle{{\benchmark}: A Comprehensive Benchmark for Undergraduate Physics Reasoning with Large Language Models}

% It is OKAY to include author information, even for blind
% submissions: the style file will automatically remove it for you
% unless you've provided the [accepted] option to the icml2025
% package.

% List of affiliations: The first argument should be a (short)
% identifier you will use later to specify author affiliations
% Academic affiliations should list Department, University, City, Region, Country
% Industry affiliations should list Company, City, Region, Country

% You can specify symbols, otherwise they are numbered in order.
% Ideally, you should not use this facility. Affiliations will be numbered
% in order of appearance and this is the preferred way.
\icmlsetsymbol{equal}{*}

\begin{icmlauthorlist}
\icmlauthor{Xin Xu}{equal,xxx}
\icmlauthor{Qiyun Xu}{equal,yyy}
\icmlauthor{Tong Xiao}{zzz}
\icmlauthor{Tianhao Chen}{xxx}
\icmlauthor{Yuchen Yan}{hhh}
\icmlauthor{Jiaxin Zhang}{xxx}
\icmlauthor{Shizhe Diao}{comp}
%\icmlauthor{}{sch}
\icmlauthor{Can Yang}{xxx}
\icmlauthor{Yang Wang}{xxx}
%\icmlauthor{}{sch}
%\icmlauthor{}{sch}
\end{icmlauthorlist}

\icmlaffiliation{xxx}{The Hong Kong University of Science and Technology}
\icmlaffiliation{yyy}{Tsinghua University}
\icmlaffiliation{zzz}{The Universify of Science and Technology of China}
\icmlaffiliation{hhh}{Zhejiang University}
\icmlaffiliation{comp}{NVIDIA}

\icmlcorrespondingauthor{Can Yang}{macyang@ust.hk}
%\icmlcorrespondingauthor{Firstname2 Lastname2}{first2.last2@www.uk}

% You may provide any keywords that you
% find helpful for describing your paper; these are used to populate
% the "keywords" metadata in the PDF but will not be shown in the document
\icmlkeywords{Large Language Models, Physics Reasoning, Dataset and Benchmark}

\vskip 0.3in
]

% this must go after the closing bracket ] following \twocolumn[ ...

% This command actually creates the footnote in the first column
% listing the affiliations and the copyright notice.
% The command takes one argument, which is text to display at the start of the footnote.
% The \icmlEqualContribution command is standard text for equal contribution.
% Remove it (just {}) if you do not need this facility.

%\printAffiliationsAndNotice{}  % leave blank if no need to mention equal contribution
\printAffiliationsAndNotice{\icmlEqualContribution} % otherwise use the standard text.

\begin{abstract}
Large language models (LLMs) have demonstrated remarkable capabilities in solving complex reasoning tasks, particularly in mathematics.
However, the domain of physics reasoning %, foundational to the natural sciences,
presents unique challenges that have received significantly less attention. Existing benchmarks often fall short in evaluating LLMs’ abilities on the breadth and depth of undergraduate-level physics, underscoring the need for a comprehensive evaluation.
To fill this gap, we introduce {\benchmark}, a large-scale and comprehensive benchmark specifically designed to evaluate \textbf{U}nder\textbf{G}raduate-level \textbf{Physics} (\textbf{\benchmark}) reasoning with LLMs.
{\benchmark} includes 5,520 undergraduate-level physics problems in both English and Chinese, covering 13 subjects with seven different answer types and four distinct physics reasoning skills, all rigorously screened for data leakage.
Additionally, we develop a \textbf{M}odel-\textbf{A}ssistant \textbf{R}ule-based \textbf{J}udgment (\textbf{{\judge}}) pipeline specifically tailored for assessing answer correctness of physics problems, ensuring accurate evaluation.
%To better understand the skill sets required for solving different physics problems, we categorize them into four distinct difficulty levels based on the abilities necessary to solve them.
Our evaluation of 31 leading LLMs shows that the highest overall accuracy, 49.8\% (achieved by OpenAI-o1-mini), emphasizes the necessity for models with stronger physics reasoning skills, beyond math abilities.
We hope {\benchmark}, along with {\judge}, will drive future advancements in AI for physics reasoning.
Codes and data are available at \href{https://github.com/YangLabHKUST/UGPhysics}{this link}.
\end{abstract}


\section{Introduction}


\begin{figure}[t]
\centering
\includegraphics[width=0.6\columnwidth]{figures/evaluation_desiderata_V5.pdf}
\vspace{-0.5cm}
\caption{\systemName is a platform for conducting realistic evaluations of code LLMs, collecting human preferences of coding models with real users, real tasks, and in realistic environments, aimed at addressing the limitations of existing evaluations.
}
\label{fig:motivation}
\end{figure}

\begin{figure*}[t]
\centering
\includegraphics[width=\textwidth]{figures/system_design_v2.png}
\caption{We introduce \systemName, a VSCode extension to collect human preferences of code directly in a developer's IDE. \systemName enables developers to use code completions from various models. The system comprises a) the interface in the user's IDE which presents paired completions to users (left), b) a sampling strategy that picks model pairs to reduce latency (right, top), and c) a prompting scheme that allows diverse LLMs to perform code completions with high fidelity.
Users can select between the top completion (green box) using \texttt{tab} or the bottom completion (blue box) using \texttt{shift+tab}.}
\label{fig:overview}
\end{figure*}

As model capabilities improve, large language models (LLMs) are increasingly integrated into user environments and workflows.
For example, software developers code with AI in integrated developer environments (IDEs)~\citep{peng2023impact}, doctors rely on notes generated through ambient listening~\citep{oberst2024science}, and lawyers consider case evidence identified by electronic discovery systems~\citep{yang2024beyond}.
Increasing deployment of models in productivity tools demands evaluation that more closely reflects real-world circumstances~\citep{hutchinson2022evaluation, saxon2024benchmarks, kapoor2024ai}.
While newer benchmarks and live platforms incorporate human feedback to capture real-world usage, they almost exclusively focus on evaluating LLMs in chat conversations~\citep{zheng2023judging,dubois2023alpacafarm,chiang2024chatbot, kirk2024the}.
Model evaluation must move beyond chat-based interactions and into specialized user environments.



 

In this work, we focus on evaluating LLM-based coding assistants. 
Despite the popularity of these tools---millions of developers use Github Copilot~\citep{Copilot}---existing
evaluations of the coding capabilities of new models exhibit multiple limitations (Figure~\ref{fig:motivation}, bottom).
Traditional ML benchmarks evaluate LLM capabilities by measuring how well a model can complete static, interview-style coding tasks~\citep{chen2021evaluating,austin2021program,jain2024livecodebench, white2024livebench} and lack \emph{real users}. 
User studies recruit real users to evaluate the effectiveness of LLMs as coding assistants, but are often limited to simple programming tasks as opposed to \emph{real tasks}~\citep{vaithilingam2022expectation,ross2023programmer, mozannar2024realhumaneval}.
Recent efforts to collect human feedback such as Chatbot Arena~\citep{chiang2024chatbot} are still removed from a \emph{realistic environment}, resulting in users and data that deviate from typical software development processes.
We introduce \systemName to address these limitations (Figure~\ref{fig:motivation}, top), and we describe our three main contributions below.


\textbf{We deploy \systemName in-the-wild to collect human preferences on code.} 
\systemName is a Visual Studio Code extension, collecting preferences directly in a developer's IDE within their actual workflow (Figure~\ref{fig:overview}).
\systemName provides developers with code completions, akin to the type of support provided by Github Copilot~\citep{Copilot}. 
Over the past 3 months, \systemName has served over~\completions suggestions from 10 state-of-the-art LLMs, 
gathering \sampleCount~votes from \userCount~users.
To collect user preferences,
\systemName presents a novel interface that shows users paired code completions from two different LLMs, which are determined based on a sampling strategy that aims to 
mitigate latency while preserving coverage across model comparisons.
Additionally, we devise a prompting scheme that allows a diverse set of models to perform code completions with high fidelity.
See Section~\ref{sec:system} and Section~\ref{sec:deployment} for details about system design and deployment respectively.



\textbf{We construct a leaderboard of user preferences and find notable differences from existing static benchmarks and human preference leaderboards.}
In general, we observe that smaller models seem to overperform in static benchmarks compared to our leaderboard, while performance among larger models is mixed (Section~\ref{sec:leaderboard_calculation}).
We attribute these differences to the fact that \systemName is exposed to users and tasks that differ drastically from code evaluations in the past. 
Our data spans 103 programming languages and 24 natural languages as well as a variety of real-world applications and code structures, while static benchmarks tend to focus on a specific programming and natural language and task (e.g. coding competition problems).
Additionally, while all of \systemName interactions contain code contexts and the majority involve infilling tasks, a much smaller fraction of Chatbot Arena's coding tasks contain code context, with infilling tasks appearing even more rarely. 
We analyze our data in depth in Section~\ref{subsec:comparison}.



\textbf{We derive new insights into user preferences of code by analyzing \systemName's diverse and distinct data distribution.}
We compare user preferences across different stratifications of input data (e.g., common versus rare languages) and observe which affect observed preferences most (Section~\ref{sec:analysis}).
For example, while user preferences stay relatively consistent across various programming languages, they differ drastically between different task categories (e.g. frontend/backend versus algorithm design).
We also observe variations in user preference due to different features related to code structure 
(e.g., context length and completion patterns).
We open-source \systemName and release a curated subset of code contexts.
Altogether, our results highlight the necessity of model evaluation in realistic and domain-specific settings.






\section{RELATED WORK}
\label{sec:relatedwork}
In this section, we describe the previous works related to our proposal, which are divided into two parts. In Section~\ref{sec:relatedwork_exoplanet}, we present a review of approaches based on machine learning techniques for the detection of planetary transit signals. Section~\ref{sec:relatedwork_attention} provides an account of the approaches based on attention mechanisms applied in Astronomy.\par

\subsection{Exoplanet detection}
\label{sec:relatedwork_exoplanet}
Machine learning methods have achieved great performance for the automatic selection of exoplanet transit signals. One of the earliest applications of machine learning is a model named Autovetter \citep{MCcauliff}, which is a random forest (RF) model based on characteristics derived from Kepler pipeline statistics to classify exoplanet and false positive signals. Then, other studies emerged that also used supervised learning. \cite{mislis2016sidra} also used a RF, but unlike the work by \citet{MCcauliff}, they used simulated light curves and a box least square \citep[BLS;][]{kovacs2002box}-based periodogram to search for transiting exoplanets. \citet{thompson2015machine} proposed a k-nearest neighbors model for Kepler data to determine if a given signal has similarity to known transits. Unsupervised learning techniques were also applied, such as self-organizing maps (SOM), proposed \citet{armstrong2016transit}; which implements an architecture to segment similar light curves. In the same way, \citet{armstrong2018automatic} developed a combination of supervised and unsupervised learning, including RF and SOM models. In general, these approaches require a previous phase of feature engineering for each light curve. \par

%DL is a modern data-driven technology that automatically extracts characteristics, and that has been successful in classification problems from a variety of application domains. The architecture relies on several layers of NNs of simple interconnected units and uses layers to build increasingly complex and useful features by means of linear and non-linear transformation. This family of models is capable of generating increasingly high-level representations \citep{lecun2015deep}.

The application of DL for exoplanetary signal detection has evolved rapidly in recent years and has become very popular in planetary science.  \citet{pearson2018} and \citet{zucker2018shallow} developed CNN-based algorithms that learn from synthetic data to search for exoplanets. Perhaps one of the most successful applications of the DL models in transit detection was that of \citet{Shallue_2018}; who, in collaboration with Google, proposed a CNN named AstroNet that recognizes exoplanet signals in real data from Kepler. AstroNet uses the training set of labelled TCEs from the Autovetter planet candidate catalog of Q1–Q17 data release 24 (DR24) of the Kepler mission \citep{catanzarite2015autovetter}. AstroNet analyses the data in two views: a ``global view'', and ``local view'' \citep{Shallue_2018}. \par


% The global view shows the characteristics of the light curve over an orbital period, and a local view shows the moment at occurring the transit in detail

%different = space-based

Based on AstroNet, researchers have modified the original AstroNet model to rank candidates from different surveys, specifically for Kepler and TESS missions. \citet{ansdell2018scientific} developed a CNN trained on Kepler data, and included for the first time the information on the centroids, showing that the model improves performance considerably. Then, \citet{osborn2020rapid} and \citet{yu2019identifying} also included the centroids information, but in addition, \citet{osborn2020rapid} included information of the stellar and transit parameters. Finally, \citet{rao2021nigraha} proposed a pipeline that includes a new ``half-phase'' view of the transit signal. This half-phase view represents a transit view with a different time and phase. The purpose of this view is to recover any possible secondary eclipse (the object hiding behind the disk of the primary star).


%last pipeline applies a procedure after the prediction of the model to obtain new candidates, this process is carried out through a series of steps that include the evaluation with Discovery and Validation of Exoplanets (DAVE) \citet{kostov2019discovery} that was adapted for the TESS telescope.\par
%



\subsection{Attention mechanisms in astronomy}
\label{sec:relatedwork_attention}
Despite the remarkable success of attention mechanisms in sequential data, few papers have exploited their advantages in astronomy. In particular, there are no models based on attention mechanisms for detecting planets. Below we present a summary of the main applications of this modeling approach to astronomy, based on two points of view; performance and interpretability of the model.\par
%Attention mechanisms have not yet been explored in all sub-areas of astronomy. However, recent works show a successful application of the mechanism.
%performance

The application of attention mechanisms has shown improvements in the performance of some regression and classification tasks compared to previous approaches. One of the first implementations of the attention mechanism was to find gravitational lenses proposed by \citet{thuruthipilly2021finding}. They designed 21 self-attention-based encoder models, where each model was trained separately with 18,000 simulated images, demonstrating that the model based on the Transformer has a better performance and uses fewer trainable parameters compared to CNN. A novel application was proposed by \citet{lin2021galaxy} for the morphological classification of galaxies, who used an architecture derived from the Transformer, named Vision Transformer (VIT) \citep{dosovitskiy2020image}. \citet{lin2021galaxy} demonstrated competitive results compared to CNNs. Another application with successful results was proposed by \citet{zerveas2021transformer}; which first proposed a transformer-based framework for learning unsupervised representations of multivariate time series. Their methodology takes advantage of unlabeled data to train an encoder and extract dense vector representations of time series. Subsequently, they evaluate the model for regression and classification tasks, demonstrating better performance than other state-of-the-art supervised methods, even with data sets with limited samples.

%interpretation
Regarding the interpretability of the model, a recent contribution that analyses the attention maps was presented by \citet{bowles20212}, which explored the use of group-equivariant self-attention for radio astronomy classification. Compared to other approaches, this model analysed the attention maps of the predictions and showed that the mechanism extracts the brightest spots and jets of the radio source more clearly. This indicates that attention maps for prediction interpretation could help experts see patterns that the human eye often misses. \par

In the field of variable stars, \citet{allam2021paying} employed the mechanism for classifying multivariate time series in variable stars. And additionally, \citet{allam2021paying} showed that the activation weights are accommodated according to the variation in brightness of the star, achieving a more interpretable model. And finally, related to the TESS telescope, \citet{morvan2022don} proposed a model that removes the noise from the light curves through the distribution of attention weights. \citet{morvan2022don} showed that the use of the attention mechanism is excellent for removing noise and outliers in time series datasets compared with other approaches. In addition, the use of attention maps allowed them to show the representations learned from the model. \par

Recent attention mechanism approaches in astronomy demonstrate comparable results with earlier approaches, such as CNNs. At the same time, they offer interpretability of their results, which allows a post-prediction analysis. \par



\begin{table}[t]
    \centering
    \caption{The performance of different pre-trained models on ImageNet and infrared semantic segmentation datasets. The \textit{Scratch} means the performance of randomly initialized models. The \textit{PT Epochs} denotes the pre-training epochs while the \textit{IN1K FT epochs} represents the fine-tuning epochs on ImageNet \citep{imagenet}. $^\dag$ denotes models reproduced using official codes. $^\star$ refers to the effective epochs used in \citet{iBOT}. The top two results are marked in \textbf{bold} and \underline{underlined} format. Supervised and CL methods, MIM methods, and UNIP models are colored in \colorbox{orange!15}{\rule[-0.2ex]{0pt}{1.5ex}orange}, \colorbox{gray!15}{\rule[-0.2ex]{0pt}{1.5ex}gray}, and \colorbox{cyan!15}{\rule[-0.2ex]{0pt}{1.5ex}cyan}, respectively.}
    \label{tab:benchmark}
    \centering
    \scriptsize
    \setlength{\tabcolsep}{1.0mm}{
    \scalebox{1.0}{
    \begin{tabular}{l c c c c  c c c c c c c c}
        \toprule
         \multirow{2}{*}{Methods} & \multirow{2}{*}{\makecell[c]{PT \\ Epochs}} & \multicolumn{2}{c}{IN1K FT} & \multicolumn{4}{c}{Fine-tuning (FT)} & \multicolumn{4}{c}{Linear Probing (LP)} \\
         \cmidrule{3-4} \cmidrule(lr){5-8} \cmidrule(lr){9-12} 
         & & Epochs & Acc & SODA & MFNet-T & SCUT-Seg & Mean & SODA & MFNet-T & SCUT-Seg & Mean \\
         \midrule
         \textcolor{gray}{ViT-Tiny/16} & & &  & & & & & & & & \\
         Scratch & - & - & - & 31.34 & 19.50 & 41.09 & 30.64 & - & - & - & - \\
         \rowcolor{gray!15} MAE$^\dag$ \citep{mae} & 800 & 200 & \underline{71.8} & 52.85 & 35.93 & 51.31 & 46.70 & 23.75 & 15.79 & 27.18 & 22.24 \\
         \rowcolor{orange!15} DeiT \citep{deit} & 300 & - & \textbf{72.2} & 63.14 & 44.60 & 61.36 & 56.37 & 42.29 & 21.78 & 31.96 & 32.01 \\
         \rowcolor{cyan!15} UNIP (MAE-L) & 100 & - & - & \underline{64.83} & \textbf{48.77} & \underline{67.22} & \underline{60.27} & \underline{44.12} & \underline{28.26} & \underline{35.09} & \underline{35.82} \\
         \rowcolor{cyan!15} UNIP (iBOT-L) & 100 & - & - & \textbf{65.54} & \underline{48.45} & \textbf{67.73} & \textbf{60.57} & \textbf{52.95} & \textbf{30.10} & \textbf{40.12} & \textbf{41.06}  \\
         \midrule
         \textcolor{gray}{ViT-Small/16} & & & & & & & & & & & \\
         Scratch & - & - & - & 41.70 & 22.49 & 46.28 & 36.82 & - & - & - & - \\
         \rowcolor{gray!15} MAE$^\dag$ \citep{mae} & 800 & 200 & 80.0 & 63.36 & 42.44 & 60.38 & 55.39 & 38.17 & 21.14 & 34.15 & 31.15 \\
         \rowcolor{gray!15} CrossMAE \citep{crossmae} & 800 & 200 & 80.5 & 63.95 & 43.99 & 63.53 & 57.16 & 39.40 & 23.87 & 34.01 & 32.43 \\
         \rowcolor{orange!15} DeiT \citep{deit} & 300 & - & 79.9 & 68.08 & 45.91 & 66.17 & 60.05 & 44.88 & 28.53 & 38.92 & 37.44 \\
         \rowcolor{orange!15} DeiT III \citep{deit3} & 800 & - & 81.4 & 69.35 & 47.73 & 67.32 & 61.47 & 54.17 & 32.01 & 43.54 & 43.24 \\
         \rowcolor{orange!15} DINO \citep{dino} & 3200$^\star$ & 200 & \underline{82.0} & 68.56 & 47.98 & 68.74 & 61.76 & 56.02 & 32.94 & 45.94 & 44.97 \\
         \rowcolor{orange!15} iBOT \citep{iBOT} & 3200$^\star$ & 200 & \textbf{82.3} & 69.33 & 47.15 & 69.80 & 62.09 & 57.10 & 33.87 & 45.82 & 45.60 \\
         \rowcolor{cyan!15} UNIP (DINO-B) & 100 & - & - & 69.35 & 49.95 & 69.70 & 63.00 & \underline{57.76} & \underline{34.15} & \underline{46.37} & \underline{46.09} \\
         \rowcolor{cyan!15} UNIP (MAE-L) & 100 & - & - & \textbf{70.99} & \underline{51.32} & \underline{70.79} & \underline{64.37} & 55.25 & 33.49 & 43.37 & 44.04 \\
         \rowcolor{cyan!15} UNIP (iBOT-L) & 100 & - & - & \underline{70.75} & \textbf{51.81} & \textbf{71.55} & \textbf{64.70} & \textbf{60.28} & \textbf{37.16} & \textbf{47.68} & \textbf{48.37} \\ 
        \midrule
        \textcolor{gray}{ViT-Base/16} & & & & & & & & & & & \\
        Scratch & - & - & - & 44.25 & 23.72 & 49.44 & 39.14 & - & - & - & - \\
        \rowcolor{gray!15} MAE \citep{mae} & 1600 & 100 & 83.6 & 68.18 & 46.78 & 67.86 & 60.94 & 43.01 & 23.42 & 37.48 & 34.64 \\
        \rowcolor{gray!15} CrossMAE \citep{crossmae} & 800 & 100 & 83.7 & 68.29 & 47.85 & 68.39 & 61.51 & 43.35 & 26.03 & 38.36 & 35.91 \\
        \rowcolor{orange!15} DeiT \citep{deit} & 300 & - & 81.8 & 69.73 & 48.59 & 69.35 & 62.56 & 57.40 & 34.82 & 46.44 & 46.22 \\
        \rowcolor{orange!15} DeiT III \citep{deit3} & 800 & 20 & \underline{83.8} & 71.09 & 49.62 & 70.19 & 63.63 & 59.01 & \underline{35.34} & 48.01 & 47.45 \\
        \rowcolor{orange!15} DINO \citep{dino} & 1600$^\star$ & 100 & 83.6 & 69.79 & 48.54 & 69.82 & 62.72 & 59.33 & 34.86 & 47.23 & 47.14 \\
        \rowcolor{orange!15} iBOT \citep{iBOT} & 1600$^\star$ & 100 & \textbf{84.0} & 71.15 & 48.98 & 71.26 & 63.80 & \underline{60.05} & 34.34 & \underline{49.12} & \underline{47.84} \\
        \rowcolor{cyan!15} UNIP (MAE-L) & 100 & - & - & \underline{71.47} & \textbf{52.55} & \underline{71.82} & \textbf{65.28} & 58.82 & 34.75 & 48.74 & 47.43 \\
        \rowcolor{cyan!15} UNIP (iBOT-L) & 100 & - & - & \textbf{71.75} & \underline{51.46} & \textbf{72.00} & \underline{65.07} & \textbf{63.14} & \textbf{39.08} & \textbf{52.53} & \textbf{51.58} \\
        \midrule
        \textcolor{gray}{ViT-Large/16} & & & & & & & & & & & \\
        Scratch & - & - & - & 44.70 & 23.68 & 49.55 & 39.31 & - & - & - & - \\
        \rowcolor{gray!15} MAE \citep{mae} & 1600 & 50 & \textbf{85.9} & 71.04 & \underline{51.17} & 70.83 & 64.35 & 52.20 & 31.21 & 43.71 & 42.37 \\
        \rowcolor{gray!15} CrossMAE \citep{crossmae} & 800 & 50 & 85.4 & 70.48 & 50.97 & 70.24 & 63.90 & 53.29 & 33.09 & 45.01 & 43.80 \\
        \rowcolor{orange!15} DeiT3 \citep{deit3} & 800 & 20 & \underline{84.9} & \underline{71.67} & 50.78 & \textbf{71.54} & \underline{64.66} & \underline{59.42} & \textbf{37.57} & \textbf{50.27} & \underline{49.09} \\
        \rowcolor{orange!15} iBOT \citep{iBOT} & 1000$^\star$ & 50 & 84.8 & \textbf{71.75} & \textbf{51.66} & \underline{71.49} & \textbf{64.97} & \textbf{61.73} & \underline{36.68} & \underline{50.12} & \textbf{49.51} \\
        \bottomrule
    \end{tabular}}}
    \vspace{-2mm}
\end{table}

\section{Experiments}
\label{sec:experiments}
The experiments are designed to address two key research questions.
First, \textbf{RQ1} evaluates whether the average $L_2$-norm of the counterfactual perturbation vectors ($\overline{||\perturb||}$) decreases as the model overfits the data, thereby providing further empirical validation for our hypothesis.
Second, \textbf{RQ2} evaluates the ability of the proposed counterfactual regularized loss, as defined in (\ref{eq:regularized_loss2}), to mitigate overfitting when compared to existing regularization techniques.

% The experiments are designed to address three key research questions. First, \textbf{RQ1} investigates whether the mean perturbation vector norm decreases as the model overfits the data, aiming to further validate our intuition. Second, \textbf{RQ2} explores whether the mean perturbation vector norm can be effectively leveraged as a regularization term during training, offering insights into its potential role in mitigating overfitting. Finally, \textbf{RQ3} examines whether our counterfactual regularizer enables the model to achieve superior performance compared to existing regularization methods, thus highlighting its practical advantage.

\subsection{Experimental Setup}
\textbf{\textit{Datasets, Models, and Tasks.}}
The experiments are conducted on three datasets: \textit{Water Potability}~\cite{kadiwal2020waterpotability}, \textit{Phomene}~\cite{phomene}, and \textit{CIFAR-10}~\cite{krizhevsky2009learning}. For \textit{Water Potability} and \textit{Phomene}, we randomly select $80\%$ of the samples for the training set, and the remaining $20\%$ for the test set, \textit{CIFAR-10} comes already split. Furthermore, we consider the following models: Logistic Regression, Multi-Layer Perceptron (MLP) with 100 and 30 neurons on each hidden layer, and PreactResNet-18~\cite{he2016cvecvv} as a Convolutional Neural Network (CNN) architecture.
We focus on binary classification tasks and leave the extension to multiclass scenarios for future work. However, for datasets that are inherently multiclass, we transform the problem into a binary classification task by selecting two classes, aligning with our assumption.

\smallskip
\noindent\textbf{\textit{Evaluation Measures.}} To characterize the degree of overfitting, we use the test loss, as it serves as a reliable indicator of the model's generalization capability to unseen data. Additionally, we evaluate the predictive performance of each model using the test accuracy.

\smallskip
\noindent\textbf{\textit{Baselines.}} We compare CF-Reg with the following regularization techniques: L1 (``Lasso''), L2 (``Ridge''), and Dropout.

\smallskip
\noindent\textbf{\textit{Configurations.}}
For each model, we adopt specific configurations as follows.
\begin{itemize}
\item \textit{Logistic Regression:} To induce overfitting in the model, we artificially increase the dimensionality of the data beyond the number of training samples by applying a polynomial feature expansion. This approach ensures that the model has enough capacity to overfit the training data, allowing us to analyze the impact of our counterfactual regularizer. The degree of the polynomial is chosen as the smallest degree that makes the number of features greater than the number of data.
\item \textit{Neural Networks (MLP and CNN):} To take advantage of the closed-form solution for computing the optimal perturbation vector as defined in (\ref{eq:opt-delta}), we use a local linear approximation of the neural network models. Hence, given an instance $\inst_i$, we consider the (optimal) counterfactual not with respect to $\model$ but with respect to:
\begin{equation}
\label{eq:taylor}
    \model^{lin}(\inst) = \model(\inst_i) + \nabla_{\inst}\model(\inst_i)(\inst - \inst_i),
\end{equation}
where $\model^{lin}$ represents the first-order Taylor approximation of $\model$ at $\inst_i$.
Note that this step is unnecessary for Logistic Regression, as it is inherently a linear model.
\end{itemize}

\smallskip
\noindent \textbf{\textit{Implementation Details.}} We run all experiments on a machine equipped with an AMD Ryzen 9 7900 12-Core Processor and an NVIDIA GeForce RTX 4090 GPU. Our implementation is based on the PyTorch Lightning framework. We use stochastic gradient descent as the optimizer with a learning rate of $\eta = 0.001$ and no weight decay. We use a batch size of $128$. The training and test steps are conducted for $6000$ epochs on the \textit{Water Potability} and \textit{Phoneme} datasets, while for the \textit{CIFAR-10} dataset, they are performed for $200$ epochs.
Finally, the contribution $w_i^{\varepsilon}$ of each training point $\inst_i$ is uniformly set as $w_i^{\varepsilon} = 1~\forall i\in \{1,\ldots,m\}$.

The source code implementation for our experiments is available at the following GitHub repository: \url{https://anonymous.4open.science/r/COCE-80B4/README.md} 

\subsection{RQ1: Counterfactual Perturbation vs. Overfitting}
To address \textbf{RQ1}, we analyze the relationship between the test loss and the average $L_2$-norm of the counterfactual perturbation vectors ($\overline{||\perturb||}$) over training epochs.

In particular, Figure~\ref{fig:delta_loss_epochs} depicts the evolution of $\overline{||\perturb||}$ alongside the test loss for an MLP trained \textit{without} regularization on the \textit{Water Potability} dataset. 
\begin{figure}[ht]
    \centering
    \includegraphics[width=0.85\linewidth]{img/delta_loss_epochs.png}
    \caption{The average counterfactual perturbation vector $\overline{||\perturb||}$ (left $y$-axis) and the cross-entropy test loss (right $y$-axis) over training epochs ($x$-axis) for an MLP trained on the \textit{Water Potability} dataset \textit{without} regularization.}
    \label{fig:delta_loss_epochs}
\end{figure}

The plot shows a clear trend as the model starts to overfit the data (evidenced by an increase in test loss). 
Notably, $\overline{||\perturb||}$ begins to decrease, which aligns with the hypothesis that the average distance to the optimal counterfactual example gets smaller as the model's decision boundary becomes increasingly adherent to the training data.

It is worth noting that this trend is heavily influenced by the choice of the counterfactual generator model. In particular, the relationship between $\overline{||\perturb||}$ and the degree of overfitting may become even more pronounced when leveraging more accurate counterfactual generators. However, these models often come at the cost of higher computational complexity, and their exploration is left to future work.

Nonetheless, we expect that $\overline{||\perturb||}$ will eventually stabilize at a plateau, as the average $L_2$-norm of the optimal counterfactual perturbations cannot vanish to zero.

% Additionally, the choice of employing the score-based counterfactual explanation framework to generate counterfactuals was driven to promote computational efficiency.

% Future enhancements to the framework may involve adopting models capable of generating more precise counterfactuals. While such approaches may yield to performance improvements, they are likely to come at the cost of increased computational complexity.


\subsection{RQ2: Counterfactual Regularization Performance}
To answer \textbf{RQ2}, we evaluate the effectiveness of the proposed counterfactual regularization (CF-Reg) by comparing its performance against existing baselines: unregularized training loss (No-Reg), L1 regularization (L1-Reg), L2 regularization (L2-Reg), and Dropout.
Specifically, for each model and dataset combination, Table~\ref{tab:regularization_comparison} presents the mean value and standard deviation of test accuracy achieved by each method across 5 random initialization. 

The table illustrates that our regularization technique consistently delivers better results than existing methods across all evaluated scenarios, except for one case -- i.e., Logistic Regression on the \textit{Phomene} dataset. 
However, this setting exhibits an unusual pattern, as the highest model accuracy is achieved without any regularization. Even in this case, CF-Reg still surpasses other regularization baselines.

From the results above, we derive the following key insights. First, CF-Reg proves to be effective across various model types, ranging from simple linear models (Logistic Regression) to deep architectures like MLPs and CNNs, and across diverse datasets, including both tabular and image data. 
Second, CF-Reg's strong performance on the \textit{Water} dataset with Logistic Regression suggests that its benefits may be more pronounced when applied to simpler models. However, the unexpected outcome on the \textit{Phoneme} dataset calls for further investigation into this phenomenon.


\begin{table*}[h!]
    \centering
    \caption{Mean value and standard deviation of test accuracy across 5 random initializations for different model, dataset, and regularization method. The best results are highlighted in \textbf{bold}.}
    \label{tab:regularization_comparison}
    \begin{tabular}{|c|c|c|c|c|c|c|}
        \hline
        \textbf{Model} & \textbf{Dataset} & \textbf{No-Reg} & \textbf{L1-Reg} & \textbf{L2-Reg} & \textbf{Dropout} & \textbf{CF-Reg (ours)} \\ \hline
        Logistic Regression   & \textit{Water}   & $0.6595 \pm 0.0038$   & $0.6729 \pm 0.0056$   & $0.6756 \pm 0.0046$  & N/A    & $\mathbf{0.6918 \pm 0.0036}$                     \\ \hline
        MLP   & \textit{Water}   & $0.6756 \pm 0.0042$   & $0.6790 \pm 0.0058$   & $0.6790 \pm 0.0023$  & $0.6750 \pm 0.0036$    & $\mathbf{0.6802 \pm 0.0046}$                    \\ \hline
%        MLP   & \textit{Adult}   & $0.8404 \pm 0.0010$   & $\mathbf{0.8495 \pm 0.0007}$   & $0.8489 \pm 0.0014$  & $\mathbf{0.8495 \pm 0.0016}$     & $0.8449 \pm 0.0019$                    \\ \hline
        Logistic Regression   & \textit{Phomene}   & $\mathbf{0.8148 \pm 0.0020}$   & $0.8041 \pm 0.0028$   & $0.7835 \pm 0.0176$  & N/A    & $0.8098 \pm 0.0055$                     \\ \hline
        MLP   & \textit{Phomene}   & $0.8677 \pm 0.0033$   & $0.8374 \pm 0.0080$   & $0.8673 \pm 0.0045$  & $0.8672 \pm 0.0042$     & $\mathbf{0.8718 \pm 0.0040}$                    \\ \hline
        CNN   & \textit{CIFAR-10} & $0.6670 \pm 0.0233$   & $0.6229 \pm 0.0850$   & $0.7348 \pm 0.0365$   & N/A    & $\mathbf{0.7427 \pm 0.0571}$                     \\ \hline
    \end{tabular}
\end{table*}

\begin{table*}[htb!]
    \centering
    \caption{Hyperparameter configurations utilized for the generation of Table \ref{tab:regularization_comparison}. For our regularization the hyperparameters are reported as $\mathbf{\alpha/\beta}$.}
    \label{tab:performance_parameters}
    \begin{tabular}{|c|c|c|c|c|c|c|}
        \hline
        \textbf{Model} & \textbf{Dataset} & \textbf{No-Reg} & \textbf{L1-Reg} & \textbf{L2-Reg} & \textbf{Dropout} & \textbf{CF-Reg (ours)} \\ \hline
        Logistic Regression   & \textit{Water}   & N/A   & $0.0093$   & $0.6927$  & N/A    & $0.3791/1.0355$                     \\ \hline
        MLP   & \textit{Water}   & N/A   & $0.0007$   & $0.0022$  & $0.0002$    & $0.2567/1.9775$                    \\ \hline
        Logistic Regression   &
        \textit{Phomene}   & N/A   & $0.0097$   & $0.7979$  & N/A    & $0.0571/1.8516$                     \\ \hline
        MLP   & \textit{Phomene}   & N/A   & $0.0007$   & $4.24\cdot10^{-5}$  & $0.0015$    & $0.0516/2.2700$                    \\ \hline
       % MLP   & \textit{Adult}   & N/A   & $0.0018$   & $0.0018$  & $0.0601$     & $0.0764/2.2068$                    \\ \hline
        CNN   & \textit{CIFAR-10} & N/A   & $0.0050$   & $0.0864$ & N/A    & $0.3018/
        2.1502$                     \\ \hline
    \end{tabular}
\end{table*}

\begin{table*}[htb!]
    \centering
    \caption{Mean value and standard deviation of training time across 5 different runs. The reported time (in seconds) corresponds to the generation of each entry in Table \ref{tab:regularization_comparison}. Times are }
    \label{tab:times}
    \begin{tabular}{|c|c|c|c|c|c|c|}
        \hline
        \textbf{Model} & \textbf{Dataset} & \textbf{No-Reg} & \textbf{L1-Reg} & \textbf{L2-Reg} & \textbf{Dropout} & \textbf{CF-Reg (ours)} \\ \hline
        Logistic Regression   & \textit{Water}   & $222.98 \pm 1.07$   & $239.94 \pm 2.59$   & $241.60 \pm 1.88$  & N/A    & $251.50 \pm 1.93$                     \\ \hline
        MLP   & \textit{Water}   & $225.71 \pm 3.85$   & $250.13 \pm 4.44$   & $255.78 \pm 2.38$  & $237.83 \pm 3.45$    & $266.48 \pm 3.46$                    \\ \hline
        Logistic Regression   & \textit{Phomene}   & $266.39 \pm 0.82$ & $367.52 \pm 6.85$   & $361.69 \pm 4.04$  & N/A   & $310.48 \pm 0.76$                    \\ \hline
        MLP   &
        \textit{Phomene} & $335.62 \pm 1.77$   & $390.86 \pm 2.11$   & $393.96 \pm 1.95$ & $363.51 \pm 5.07$    & $403.14 \pm 1.92$                     \\ \hline
       % MLP   & \textit{Adult}   & N/A   & $0.0018$   & $0.0018$  & $0.0601$     & $0.0764/2.2068$                    \\ \hline
        CNN   & \textit{CIFAR-10} & $370.09 \pm 0.18$   & $395.71 \pm 0.55$   & $401.38 \pm 0.16$ & N/A    & $1287.8 \pm 0.26$                     \\ \hline
    \end{tabular}
\end{table*}

\subsection{Feasibility of our Method}
A crucial requirement for any regularization technique is that it should impose minimal impact on the overall training process.
In this respect, CF-Reg introduces an overhead that depends on the time required to find the optimal counterfactual example for each training instance. 
As such, the more sophisticated the counterfactual generator model probed during training the higher would be the time required. However, a more advanced counterfactual generator might provide a more effective regularization. We discuss this trade-off in more details in Section~\ref{sec:discussion}.

Table~\ref{tab:times} presents the average training time ($\pm$ standard deviation) for each model and dataset combination listed in Table~\ref{tab:regularization_comparison}.
We can observe that the higher accuracy achieved by CF-Reg using the score-based counterfactual generator comes with only minimal overhead. However, when applied to deep neural networks with many hidden layers, such as \textit{PreactResNet-18}, the forward derivative computation required for the linearization of the network introduces a more noticeable computational cost, explaining the longer training times in the table.

\subsection{Hyperparameter Sensitivity Analysis}
The proposed counterfactual regularization technique relies on two key hyperparameters: $\alpha$ and $\beta$. The former is intrinsic to the loss formulation defined in (\ref{eq:cf-train}), while the latter is closely tied to the choice of the score-based counterfactual explanation method used.

Figure~\ref{fig:test_alpha_beta} illustrates how the test accuracy of an MLP trained on the \textit{Water Potability} dataset changes for different combinations of $\alpha$ and $\beta$.

\begin{figure}[ht]
    \centering
    \includegraphics[width=0.85\linewidth]{img/test_acc_alpha_beta.png}
    \caption{The test accuracy of an MLP trained on the \textit{Water Potability} dataset, evaluated while varying the weight of our counterfactual regularizer ($\alpha$) for different values of $\beta$.}
    \label{fig:test_alpha_beta}
\end{figure}

We observe that, for a fixed $\beta$, increasing the weight of our counterfactual regularizer ($\alpha$) can slightly improve test accuracy until a sudden drop is noticed for $\alpha > 0.1$.
This behavior was expected, as the impact of our penalty, like any regularization term, can be disruptive if not properly controlled.

Moreover, this finding further demonstrates that our regularization method, CF-Reg, is inherently data-driven. Therefore, it requires specific fine-tuning based on the combination of the model and dataset at hand.

\section{Analysis}
\label{sec:analysis}
In the following sections, we will analyze European type approval regulation\footnote{Strictly speaking, the German enabling act (AFGBV) does not regulate type-approval, but how test \& operating permits are issued for SAE-Level-4 systems. Type-approval regulation for SAE-Level-3 systems follows UN Regulation No. 157 (UN-ECE-ALKS) \parencite{un157}.} regarding the underlying notions of ``safety'' and ``risk''.
We will classify these notions according to their absolute or relative character, underlying risk sources, or underlying concepts of harm.

\subsection{Classification of Safety Notions}
\label{sec:safety-notions}
We will refer to \emph{absolute} notions of safety as conceptualizations that assume the complete absence of any kind of risk.
Opposed to this, \emph{relative} notions of safety are based on a conceptualization that specifically includes risk acceptance criteria, e.g., in terms of ``tolerable'' risk or ``sufficient'' safety.

For classifying notions of safety by their underlying risk (or rather ``hazard'') sources, and different concepts of harm, \Cref{fig:hazard-sources} provides an overview of our reasoning, which is closely in line with the argumentation provided by Waymo in \parencite{favaro2023}.
We prefer ``hazard sources'' over ``risk sources'', as a risk must always be related to a \emph{cause} or \emph{source of harm} (i.e., a hazard \parencite[p.~1, def. 3.2]{iso51}).
Without a concrete (scenario) context that the system is operating in, a hazard is \emph{latent}: E.g., when operating in public traffic, there is a fundamental possibility that a \emph{collision with a pedestrian} leads to (physical) harm for that pedestrian. 
However, only if an automated vehicle shows (potentially) hazardous behavior (e.g., not decelerating properly) \emph{and} is located near a pedestrian (context), the hazard is instantiated and leads to a hazardous event.
\begin{figure*}
    \includeimg[width=.9\textwidth]{hazard-sources0.pdf}
    \caption{Graphical summary of a taxonomy of risk related to automated vehicles, extended based on ISO 21448 (\parencite{iso21448}) and \parencite{favaro2023}. Top: Causal chain from hazard sources to actual harm; bottom: summary of the individual elements' contributions to a resulting risk. Graphic translated from \parencite{nolte2024} \label{fig:hazard-sources}}
\end{figure*}
If the hazardous event cannot be mitigated or controlled, we see a loss event in which the pedestrian's health is harmed.
Note that this hypothetical chain of events is summarized in the definition of risk:
The probability of occurrence of harm is determined by a) the frequency with which hazard sources manifest, b) the time for which the system operates in a context that exposes the possibility of harm, and c) by the probability with which a hazardous event can be controlled.
A risk can then be determined as a function of the probability of harm and the severity of the harm potentially inflicted on the pedestrian.

In the following, we will apply this general model to introduce different types of hazard sources and also different types of harm.
\cref{fig:hazard-sources} shows two distinct hazard sources, i.e., functional insufficiencies and E/E-failures that can lead to hazardous behavior.
ISO~21488 \parencite{iso21448} defines functional insufficiencies as insufficiencies that stem from an incomplete or faulty system specification (specification insufficiencies).
In addition, the standard considers insufficiencies that stem from insufficient technical capability to operate inside the targeted Operational Design Domain (performance insufficiencies).
Functional insufficiencies are related to the ``Safety of the Intended Functionality (SOTIF)'' (according to ISO~21448), ``Behavioral Safety'' (according to Waymo \parencite{waymo2018}), or ``Operational Safety'' (according to UN Regulation No. 157 \parencite{un157}).
E/E-Failures are related to classic functional safety and are covered exhaustively by ISO~26262 \parencite{iso2018}.
Additional hazard sources can, e.g., be related to malicious security attacks (ISO~21434), or even to mechanical failures that should be covered (in the US) in the Federal Motor Vehicle Safety Standards (FMVSS).

For the classification of notions of safety by the related harm, in \parencite{salem2024, nolte2024}, we take a different approach compared to \parencite{koopman2024}:
We extend the concept of harm to the violation of stakeholder \emph{values}, where values are considered to be a ``standard of varying importance among other such standards that, when combined, form a value pattern that reduces complexity for stakeholders [\ldots] [and] determines situational actions [\ldots].'' \parencite{albert2008}
In this sense, values are profound, personal determinants for individual or collective behavior.
The notion of values being organized in a weighted value pattern shows that values can be ranked according to importance.
For automated vehicles, \emph{physical wellbeing} and \emph{mobility} can, e.g., be considered values which need to be balanced to achieve societal acceptance, in line with the discussion of required tradeoffs in \cref{sec:terminology}.
For the analysis of the following regulatory frameworks, we will evaluate if the given safety or risk notions allow tradeoffs regarding underlying stakeholder values. 

\subsection{UN Regulation No. 157 \& European Implementing Regulation (EU) 2022/1426}
\label{sec:enabling-act}
UN Regulation No. 157 \parencite{un157} and the European Implementing Regulation 2022/1426 \parencite{eu1426} provide type approval regulation for automated vehicles equipped with SAE-Level-3 (UN Reg. 157) and Level 4 (EU 2022/1426) systems on an international (UN Reg. 157) and European (EU 2022/1426) level.

Generally, EU type approval considers UN ECE regulations mandatory for its member states ((EU) 2018/858, \parencite{eu858}), while the EU largely forgoes implementing EU-specific type approval rules, it maintains the right to alter or to amend UN ECE regulation \parencite{eu858}.

In this respect, the terminology and conceptualizations in the EU Implementing Act closely follow those in UN Reg. No. 157.
The EU Implementing Act gives a clear reference to UN Reg. No. 157 \parencite[][Preamble,  Paragraph 1]{eu1426}.
Hence, the documents can be assessed in parallel.
Differences will be pointed out as necessary.

Both acts are written in rather technical language, including the formulation of technical requirements (e.g., regarding deceleration values or speeds in certain scenarios).
While providing exhaustive definitions and terminology, neither of both documents provide an actual definition of risk or safety.
The definition of ``unreasonable'' risk in both documents does not define risk, but only what is considered \emph{unreasonable}. It states that the ``overall level of risk for [the driver, (only in UN Reg. 157)] vehicle occupants and other road users which is increased compared to a competently and carefully driven manual vehicle.''
The pertaining notions of safety and risk can hence only be derived from the context in which they are used.

\subsubsection{Absolute vs. Relative Notions of Safety}
In line with the technical detail provided in the acts, both clearly imply a \emph{relative} notion of safety and refer to the absence of \emph{unreasonable} risk throughout, which is typical for technical safety definitions.

Both acts require sufficient proof and documentation that the to-be-approved automated driving systems are ``free of unreasonable safety risks to vehicle occupants and other road users'' for type approval.\footnote{As it targets SAE-Level-3 systems, UN Reg. 157 also refers to the driver, where applicable.}
In this respect, both acts demand that the manufacturers perform verification and validation activities for performance requirements that include ``[\ldots] the conclusion that the system is designed in such a way that it is free from unreasonable risks [\ldots]''.
Additionally, \emph{risk minimization} is a recurring theme when it comes to the definition of Minimum Risk Maneuvers (MRM).

Finally, supporting the relative notions of safety and risk, UN Reg. 157 introduces the concept of ``reasonable foreseeable and preventable'' \parencite[Article 1, Clause 5.1.1.]{un157} collisions, which implies that a residual risk will remain with the introduction of automated vehicles.
\parencite[][Appendix 3, Clause 3.1.]{un157} explicitly states that only \emph{some} scenarios that are unpreventable for a competent human driver can actually be prevented by an automated driving system.
While this concept is not applied throughout the EU Implementing Act, both documents explicitly refer to \emph{residual} risks that are related to the operation of automated driving systems (\parencite[][Annex I, Clause 1]{un157}, \parencite[][Annex II, Clause 7.1.1.]{eu1426}).

\subsubsection{Hazard Sources}
Hazard sources that are explicitly differentiated in UN Reg. 157 and (EU) 2022/1426 are E/E-failures that are in scope of functional safety (ISO~26262) and functional insufficiencies that are in scope of behavioral (or ``operational'') safety (ISO~21448).
Both documents consistently differentiate both sources when formulating requirements.

While the acts share a common definition of ``operational'' safety (\parencite[][Article 2, def. 30.]{eu1426}, \parencite[][Annex 4, def. 2.15.]{un157}), the definitions for functional safety differ.
\parencite{un157} defines functional safety as the ``absence of unreasonable risk under the occurrence of hazards caused by a malfunctioning behaviour of electric/electronic systems [\ldots]'', \parencite{eu1426} drops the specification of ``electric/electronic systems'' from the definition.
When taken at face value, this definition would mean that functional safety included all possible hazard sources, regardless of their origin, which is a deviation from the otherwise precise usage of safety-related terminology.

\subsubsection{Harm Types}
As the acts lack explicit definitions of safety and risk, there is no consistent and explicit notion of different harm types that could be differentiated.

\parencite{un157} gives little hints regarding different considered harm types.
``The absence of unreasonable risk'' in terms of human driving performance could hence be related to any chosen performance metric that allows a comparison with a competent careful human driver including, e.g., accident statistics, statistics about rule violations, or changes in traffic flow.

In \parencite{eu1426}, ``safety'' is, implicitly, attributed to the absence of unreasonable risk to life and limb of humans.
This is supported by the performance requirements that are formulated:
\parencite[][Annex II, Clause 1.1.2. (d)]{eu1426} demands that an automated driving system can adapt the vehicle behavior in a way that it minimizes risk and prioritizes the protection of human life.

Both acts demand the adherence to traffic rules (\parencite[][Annex 2, Clause 1.3.]{eu1426}, \parencite[][Clause 5.1.2.]{un157}).
\parencite[][Annex II, Clause 1.1.2. (c)]{eu1426} also demands that an automated driving system shall adapt its behavior to surrounding traffic conditions, such as the current traffic flow.
With the relative notion of risk in both acts, the unspecific clear statement that there may be unpreventable accidents \parencite{un157}, and a demand of prioritization of human life in \parencite{eu1426}, both acts could be interpreted to allow developers to make tradeoffs as discussed in \cref{sec:terminology}.


\subsubsection{Conclusion}
To summarize, the UN Reg. 157 and the (EU) 2022/1426 both clearly support the technical notion of safety as the absence of unreasonable risk.
The notion is used consistently throughout both documents, providing a sufficiently clear terminology for the developers of automated vehicles.
Uncertainty remains when it comes to considered harm types: Both acts do not explicitly allow for broader notions of safety, in the sense of \parencite{koopman2024} or \parencite{salem2024}.
Finally, a minor weak spot can be seen in the definition of risk acceptance criteria: Both acts take the human driving performance as a baseline.
While (EU) 2022/1426 specifies that these criteria are specific to the systems' Operational Design Domain \parencite[][Annex II, Clause 7.1.1.]{eu1426}, the reference to the concrete Operational Design Domain is missing in UN Reg. 157.
Without a clearly defined notion of safety, however, it remains unclear, how aspects beyond net accident statistics (which are given as an example in \parencite[][Annex II, Clause 7.1.1.]{eu1426}), can be addressed practically, as demanded by \parencite{koopman2024}.

\subsection{German Regulation (StVG \& AFGBV)}
\label{sec:afgbv}
The German L3 (Automated Driving Act) and L4 (Act on Autonomous Driving) Acts from 2017 and 2021,\footnote{Formally, these are amendments to the German Road Traffic Act (StVG): 06/21/2017, BGBl. I p. 1648, 07/12/2021 BGBl. I p. 3108.} respectively, provide enabling regulation for the operation of SAE-Level-3 and 4 vehicles on German roads.
The German Implementing Regulation (\parencite{afgbv}, AFGBV) defines how this enabling regulation is to be implemented for granting testing permits for SAE-Level-3 and -4 and driving permits for SAE-Level-3 and -4 automated driving systems.\footnote{Note that these permits do not grant EU-wide type approval, but serve as a special solution for German roads only. At the same time, the AFGBV has the same scope as (EU) 2022/1426.}
With all three acts, Germany was the first country to regulate the approval of automated vehicles for a domestic market.
All acts are subject to (repeated) evaluation until the year 2030 regarding their impact on the development of automated driving technology.
An assessment of the German AFGBV and comparisons to (EU) 2022/1426 have been given in \cite{steininger2022} in German.

Just as for UN Reg. 157 and (EU) 2022/1426, neither the StVG nor the AFGBV provide a clear definition of ``safety'' or ``risk'' -- even though the "safety" of the road traffic is one major goal of the StVG and StVO.
Again, different implicit notions of both concepts can only be interpreted from the context of existing wording.
An additional complication that is related to the German language is that ``safety'' and ``security'' can both be addressed as ``Sicherheit'', adding another potential source of unclarity.
Literal Quotations in this section are our translations from the German act.

\subsubsection{Absolute vs. Relative Notions of Safety}
For assessing absolute vs. relative notions of safety in German regulation, it should be mentioned that the main goal of the German StVO is to ensure the ``safety and ease of traffic flow'' -- an already diametral goal that requires human drivers to make tradeoffs.\footnote{For human drivers, this also creates legal uncertainty which can sometimes only be settled in a-posteriori court cases.}
While UN and EU regulation clearly shows a relative notion of safety\footnote{And even the StVG contains sections that use wording such as ``best possible safety for vehicle occupants'' (§1d (4) StVG) and acknowledges that there are unavoidable hazards to human life (§1e (2) No. 2c)).}, the German AFGBV contains ambiguous statements in this respect:
Several paragraphs contain a demand for a hazard free operation of automated vehicles.
§4 (1) No. 4 AFGBV, e.g., states that ``the operation of vehicles with autonomous driving functions must neither negatively impact road traffic safety or traffic flow, nor endanger the life and limb of persons.''
Additionally, §6 (1) AFGBV states that the permits for testing and operation have to be revoked, if it becomes apparent that a ``negative impact on road traffic safety or traffic flow, or hazards to the life and limb of persons cannot be ruled out''.
The same wording is used for the approval of operational design domains regulated in §10 (1) No. 1.
A particularly misleading statement is made regarding the requirements for technical supervision instances which are regulated in §14 (3) AFGBV which states that an automated vehicle has to be  ``immediately removed from the public traffic space if a risk minimal state leads to hazards to road traffic safety or traffic flow''.
Considering the argumentation in \cref{sec:terminology}, that residual risks related to the operation of automated driving systems are inevitable, these are strong statements which, if taken at face value, technically prohibit the operation of automated vehicles.
It suggests an \emph{absolute} notion of safety that requires the complete absence of risk.  
The last statement above is particularly contradictory in itself, considering that a risk \emph{minimal} state always implies a residual risk.

In addition to these absolute safety notions, there are passages which suggest a relative notion of safety:
The approval for Operational Design Domains is coupled to the proof that the operation of an automated vehicle ``neither negatively impacts road traffic safety or traffic flow, nor significantly endangers the life and limb of persons beyond the general risk of an impact that is typical of local road traffic'' (§9 (2) No. 3 AFGBV).
The addition of a relative risk measure ``beyond the general risk of an impact'' provides a relaxation (cf. also \cite{steininger2022}, who criticizes the aforementioned absolute safety notion) that also yields an implicit acceptance criterion (\emph{statistically as good as} human drivers) similar to the requirements stated in UN Reg. 157 and (EU) 2022/1426.

Additional hints for a relative notion of safety can be found in Annex 1, Part 1, No. 1.1 and Annex 1, Part 2, No. 10.
Part 1, No 1.1 specifies collision-avoidance requirements and acknowledges that not all collisions can be avoided.\footnote{The same is true for Part 2, No. 10, Clause 10.2.5.}
Part 2, No. 10 specifies requirements for test cases.
It demands that test cases are suitable to provide evidence that the ``safety of a vehicle with an autonomous driving function is increased compared to the safety of human-driven vehicles''.
This does not only acknowledge residual risks, but also yields an acceptance criterion (\emph{better} than human drivers) that is different from the implied acceptance criterion given in §9 (2) No. 3 AFGBV.

\subsubsection{Hazard Sources}
Regarding hazard sources, Annex 1 and 3 AFGBV explicitly refer to ISO~26262 and ISO~21448 (or rather its predecessor ISO/PAS~21448:2019).
However, regarding the discussion of actual hazard sources, the context in which both standards are mentioned is partially unclear:
Annex 1, Clause 1.3 discusses requirements for path and speed planning.
Clause 1.3 d) demands that in intersections, a Time to Collision (TTC) greater than 3 seconds must be guaranteed.
If manufacturers deviate from this, it is demanded that ``state-of-the-art, systematic safety evaluations'' are performed.
Fulfillment of the state of the art is assumed if ``the guidelines of ISO~26262:2018-12 Road Vehicles -- Functional Safety are fulfilled''.
Technically, ISO~26262 is not suitable to define the state of the art in this context, as the requirements discussed fall in the scope of operational (or behavioral) safety (ISO~21448).
A hazard source ``violated minimal time to collision'' is clearly a functional insufficiency, not an E/E-failure.

Similar unclarity presents itself in Annex 3, Clause 1 AFGBV: 
Clause 1 specifies the contents of the ``functional specification''.
The ``specification of the functionality'' is an artifact which is demanded in ISO~21448:2022 (Clause 5.3) \parencite{iso21448}.
However, Annex 3, Clause 1 AFGBV states that the ``functional specification'' is considered to comply to the state of the art, if the ``functional specification'' adheres to ISO~26262-3:2018 (Concept Phase).
Again, this assumes SOTIF-related contents as part of ISO~26262, which introduces the ``Item Definition'' as an artifact, which is significantly different from the ``specification of the functionality'' which is demanded by ISO~21448.
Finally, Annex 3, Clause 3 AFGBV demands a ``documentation of the safety concept'' which ``allows a functional safety assessment''.
A safety concept that is related to operational / behavioral safety is not demanded.
Technically, the unclarity with respect to the addressed harm types lead to the fact that the requirements provided by the AFGBV do not comply with the state of the art in the field, providing questionable regulation.

\subsubsection{Harm Types}
Just like UN Reg. 157 and (EU) 2022/1426, the German StVG and AFGBV do not explicitly differentiate concrete harm types for their notions of safety.
However, the AFGBV mentions three main concerns for the operation of automated vehicles which are \emph{traffic flow} (e.g., §4 (1) No. 4 AFGBV), compliance to \emph{traffic law} (e.g., §1e (2) No. 2 StVG), and the \emph{life and limb of humans} (e.g., §4 (1) No. 4 AFGBV).

Again, there is some ambiguity in the chosen wording:
The conflict between traffic flow and safety has already been argued in \cref{sec:terminology}.
The wording given in §4 (1) No. 4 and §6 (1) AFGBV  demand to ensure (absolute) safety \emph{and} traffic flow at the same time, which is impossible (cf. \cref{sec:terminology}) from an engineering perspective.
§1e (2) No. 2 StVG defines that ``vehicles with an autonomous driving function must [\ldots] be capable to comply to [\ldots] traffic rules in a self-contained manner''.
Taken at face value, this wording implies that an automated driving system could lose its testing or operating permit as soon as it violates a traffic rule.
A way out could be provided by §1 of the German Traffic Act (StVO) which demands careful and considerate behavior of all traffic participants and by that allows judgement calls for human drivers.
However, if §1 is applicable in certain situations is often settled in court cases. 
For developers, the application of §1 StVO during system design hence remains a legal risk.

While there are rather absolute statements as mentioned above, sections of the AFGBV and StVG can be interpreted to allow tradeoffs:
§1e (2) No. 2 b) demands that a system,  ``in case of an inevitable, alternative harm to legal objectives, considers the significance of the legal objectives, where the protection of human life has highest priority''.
This exact wording \emph{could} provide some slack for the absolute demands in other parts of the acts, enabling tradeoffs between (tolerable) risk and mobility as discussed in \cref{sec:terminology}.
However, it remains unclear if this interpretation is legally possible.

\subsubsection{Conclusion}
Compared to UN Reg. 157 and (EU) 2022/1426, the German StVG and AFGBV introduce openly inconsistent notions of safety and risk which are partially directly contradictory:
The wording partially implies absolute and relative notions of safety and risk at the same time.
The implied validation targets (``better'' or ``as good as'' human drivers) are equally contradictory. 
The partially implied absolute notions of safety, when taken at face value, prohibit engineers from making the tradeoffs required to develop a system that is safe and provides customer benefit at the same time. 
In consequence, the wording in the acts is prone to introducing legal uncertainty.
This uncertainty creates additional clarification need and effort for manufacturers and engineers who design and develop SAE-Level-3 and -4 automated driving systems. The use of undefined legal terms not only makes it more difficult for engineers to comply with the law, but also complicates the interpretation of the law and leads to legal uncertainty.

\subsection{UK Automated Vehicles Act 2024 (2024 c. 10)}
The UK has issued a national enabling act for regulating the approval of automated vehicles on the roads in the UK.
To the best of our knowledge, concrete implementing regulation has not been issued yet.
Regarding terminology, the act begins with a dedicated terminology section to clarify the terms used in the act \parencite[Part 1, Chapter 1, Section 1]{ukav2024}.
In that regard, the act defines a vehicle to drive ```autonomously' if --- (a)
it is being controlled not by an individual but by equipment of the vehicle, and (b) neither the vehicle nor its surroundings are being monitored by an individual with a view to immediate intervention in the driving of the vehicle.''
The act hence covers SAE-Level-3 to SAE-Level-5 automated driving systems.

\subsubsection{Absolute vs. Relative Notions of Safety}
While not providing an explicit definition of safety and risk, the UK Automated Vehicles Act (``UK AV Act'') \parencite{ukav2024} explicitly refers to a relative notion of safety.
Part~1, Chapter~1, Section~1, Clause (7)~(a) defines that an automated vehicle travels ```safely' if it travels to an acceptably safe standard''.
This clarifies that absolute safety is not achievable and that acceptance criteria to prove the acceptability of residual risk are required, even though a concrete safety definition is not given.
The act explicitly tasks the UK Secretary of State\footnote{Which means, that concrete implementation regulation needs to be enacted.} to install safety principles to determine the ``acceptably safe standard'' in Part~1, Chapter~1, Section~1, Clause (7)~(a).
In this respect, the act also provides one general validation target as it demands that the safety principles must ensure that ``authorized automated vehicles will achieve a level of safety equivalent to, or higher than, that of careful and competent human drivers''.
Hence, the top-level validation risk acceptance criterion assumed for UK regulation is ``\emph{at least as good} as human drivers''.

\subsubsection{Hazard Sources}
The UK AV Act contains no statements that could be directly related to different hazard sources.
Note that, in contrast to the rest of the analyzed documents, the UK AV Act is enabling rather than implementing regulation.
It is hence comparable to the German StVG, which does not refer to concrete hazard sources as well.

\subsubsection{Types of Harm}
Even though providing a clear relative safety notion, the missing definition of risk also implies a lack of explicitly differentiable types of harm.
Implicitly, three different types of harm can be derived from the wording in the act.
This includes the harm to life and limb of humans\footnote{Part~1, Chapter~3, Section~25 defines ``aggravated offence where death or serious injury occurs'' \parencite{ukav2024}.}, the violation of traffic rules\footnote{Part~1, Chapter~1, Clause~(7)~(b) defines that an automated vehicle travels ```legally' if it travels with an acceptably low risk of committing a traffic infraction''}, and the cause of inconvenience to the public \parencite[Part~1, Chapter~1, Section~58, Clause (2)~(d)]{ukav2024}.

The act connects all the aforementioned types of harm to ``risk'' or ``acceptable safety''.
While the act generally defines criminal offenses for providing ``false or misleading information about safety'', it also acknowledges possible defenses if it can be proven that ``reasonable precautions'' were taken and that ``due diligence'' was exercised to ``avoid the commission of the offence''.
This statement could enable tradeoffs within the scope of ``reasonable risk'' to the life and limb of humans, the violation of traffic rules, or to the cause of inconvenience to the public, as we argued in \cref{sec:terminology}.

\subsubsection{Conclusion}
From the set of reviewed documents, the current UK AV Act is the one with the most obvious relative notions of safety and risk and the one that seems to provide a legal framework for permitting tradeoffs.
In our review, we did not spot major inconsistency beyond a missing definitions of safety and risk\footnote{Note that with the Office for Product Safety and Standards (OPSS), there is a British government agency that maintains an exhaustive and widely focussed ``Risk Lexicon'' that provides suitable risk definitions. For us, it remains unclear, to what extent this terminology is assumed general knowledge in British legislation.}.
The general, relative notion of safety and the related alleged ability for designers to argue well-founded development tradeoffs within the legal framework could prove beneficial for the actual implementation of automated driving systems.
While the act thus appears as a solid foundation for the market introduction of automated vehicles, without accompanying implementing regulation, it is too early to draw definite conclusions.

\section{Conclusion}
In this work, we propose a simple yet effective approach, called SMILE, for graph few-shot learning with fewer tasks. Specifically, we introduce a novel dual-level mixup strategy, including within-task and across-task mixup, for enriching the diversity of nodes within each task and the diversity of tasks. Also, we incorporate the degree-based prior information to learn expressive node embeddings. Theoretically, we prove that SMILE effectively enhances the model's generalization performance. Empirically, we conduct extensive experiments on multiple benchmarks and the results suggest that SMILE significantly outperforms other baselines, including both in-domain and cross-domain few-shot settings.


\section*{Impact Statement}
This paper introduces an undergraduate-level physics benchmark aimed at advancing AI capabilities in physics problem-solving. 
Future directions include incorporating problems with images to enable multi-modal evaluation or more language to facilitate multi-lingual assessment. 

\section*{Acknowledgments}
This work was partially supported by a grant from the Research Grants Council of the Hong Kong Special Administrative Region, China (Project Reference Number: AoE/E-601/24-N).

% In the unusual situation where you want a paper to appear in the
% references without citing it in the main text, use \nocite
%\nocite{langley00}
\clearpage
\bibliography{example_paper}
\bibliographystyle{icml2025}


%%%%%%%%%%%%%%%%%%%%%%%%%%%%%%%%%%%%%%%%%%%%%%%%%%%%%%%%%%%%%%%%%%%%%%%%%%%%%%%
%%%%%%%%%%%%%%%%%%%%%%%%%%%%%%%%%%%%%%%%%%%%%%%%%%%%%%%%%%%%%%%%%%%%%%%%%%%%%%%
% APPENDIX
%%%%%%%%%%%%%%%%%%%%%%%%%%%%%%%%%%%%%%%%%%%%%%%%%%%%%%%%%%%%%%%%%%%%%%%%%%%%%%%
%%%%%%%%%%%%%%%%%%%%%%%%%%%%%%%%%%%%%%%%%%%%%%%%%%%%%%%%%%%%%%%%%%%%%%%%%%%%%%%
\newpage
\clearpage
\appendix
\onecolumn
% \begin{figure*}[htbp]
%     % 左侧图片
%     \begin{minipage}{0.77\linewidth}  % 调整宽度
%         \centering
%         \includegraphics[width=\linewidth]{images/benchmark_construction.pdf}
%     \end{minipage}%
%     % 间隔
%     \hfill
%     % 右侧表格
%     \begin{minipage}{0.23\linewidth}  % 调整宽度
%         \centering
%         \resizebox{\linewidth}{!}{  % 调整表格至合适的宽度
%             \begin{tabular}{lcc}
%                 \toprule
%                 \textbf{Statistic} & \textbf{Number} \\
%                 \midrule
%                 \rowcolor[HTML]{F2F2F2} 
%                 \textit{Domain Count} &  \\
%                 \midrule
%                 Domain & 103 \\
%                 Requirement & 8 \\
%                 \midrule
%                 \rowcolor[HTML]{F2F2F2} 
%                 \textit{Token Count} &  \\
%                 \midrule
%                 Description & 851.6 $\pm$ 515.2 \\
%                 - Min/Max & [159, 2814] \\
%                 Domain & 1187.2 $\pm$ 1212.1 \\
%                 - Min/Max & [85, 7514] \\
%                 \midrule
%                 \rowcolor[HTML]{F2F2F2} 
%                 \textit{Line Count} &  \\
%                 \midrule
%                 Domain & 75.4 $\pm$ 62.9 \\
%                 - Min/Max & [9, 394] \\
%                 \midrule
%                 \rowcolor[HTML]{F2F2F2} 
%                 \textit{Component Count} &  \\
%                 \midrule
%                 Actions & 4.5 $\pm$ 2.8 \\
%                 - Min/Max & [1, 16] \\
%                 Predicates & 8.1 $\pm$ 4.8 \\
%                 - Min/Max & [1, 25] \\
%                 Types & 1.1 $\pm$ 1.3 \\
%                 - Min/Max & [1, 8] \\
%                 \bottomrule
%             \end{tabular}
%         }
%     \end{minipage}
%     % 公共标题
%     \caption{Dataset construction process (left) and key statistics (right) of the \texttt{\benchmark} dataset.     Dataset construction process including: (a) \textit{Data Acquisition} (\S\ref{sec:data_acquisition}); (b) \textit{Data Filtering and Manual Selection} (\S\ref{sec:data_filtering}); (c) \textit{Data Annotation and Quality Assurance}(\S\ref{sec:data_annotation} and \S\ref{sec:quality_assurance}). Tokens are counted by GPT-2~\cite{openai2019gpt2} tokenizer.}
%     \label{fig:combined}
% \end{figure*}


\newcommand{\tabincell}[2]{\begin{tabular}{@{}#1@{}}#2\end{tabular}}
\newcommand{\rowstyle}[1]{\gdef\currentrowstyle{#1}%
	#1\ignorespaces
}

\newcommand{\className}[1]{\textbf{\sf #1}}
\newcommand{\functionName}[1]{\textbf{\sf #1}}
\newcommand{\objectName}[1]{\textbf{\sf #1}}
\newcommand{\annotation}[1]{\textbf{#1}}
\newcommand{\todo}[1]{\textcolor{blue}{\textbf{[[TODO: #1]]}}}
\newcommand{\change}[1]{\textcolor{blue}{#1}}
\newcommand{\fetch}[1]{\textbf{\em #1}}
\newcommand{\phead}[1]{\vspace{1mm} \noindent {\bf #1}}
\newcommand{\wei}[1]{\textcolor{blue}{{\it [Wei says: #1]}}}
\newcommand{\peter}[1]{\textcolor{red}{{\it [Peter says: #1]}}}
\newcommand{\zhenhao}[1]{\textcolor{dkblue}{{\it [Zhenhao says: #1]}}}
\newcommand{\feng}[1]{\textcolor{magenta}{{\it [Feng says: #1]}}}
\newcommand{\jinqiu}[1]{\textcolor{red}{{\it [Jinqiu says: #1]}}}
\newcommand{\shouvick}[1]{\textcolor{violet(ryb)}{{\it [Shouvick says: #1]}}}
\newcommand{\pattern}[1]{\emph{#1}}
%\newcommand{\tool}{{{DectGUILag}}\xspace}
\newcommand{\tool}{{{GUIWatcher}}\xspace}


\newcommand{\guo}[1]{\textcolor{yellow}{{\it [Linqiang says: #1]}}}

\newcommand{\rqbox}[1]{\begin{tcolorbox}[left=4pt,right=4pt,top=4pt,bottom=4pt,colback=gray!5,colframe=gray!40!black,before skip=2pt,after skip=2pt]#1\end{tcolorbox}}



\begin{table*}[t]
\centering
\fontsize{11pt}{11pt}\selectfont
\begin{tabular}{lllllllllllll}
\toprule
\multicolumn{1}{c}{\textbf{task}} & \multicolumn{2}{c}{\textbf{Mir}} & \multicolumn{2}{c}{\textbf{Lai}} & \multicolumn{2}{c}{\textbf{Ziegen.}} & \multicolumn{2}{c}{\textbf{Cao}} & \multicolumn{2}{c}{\textbf{Alva-Man.}} & \multicolumn{1}{c}{\textbf{avg.}} & \textbf{\begin{tabular}[c]{@{}l@{}}avg.\\ rank\end{tabular}} \\
\multicolumn{1}{c}{\textbf{metrics}} & \multicolumn{1}{c}{\textbf{cor.}} & \multicolumn{1}{c}{\textbf{p-v.}} & \multicolumn{1}{c}{\textbf{cor.}} & \multicolumn{1}{c}{\textbf{p-v.}} & \multicolumn{1}{c}{\textbf{cor.}} & \multicolumn{1}{c}{\textbf{p-v.}} & \multicolumn{1}{c}{\textbf{cor.}} & \multicolumn{1}{c}{\textbf{p-v.}} & \multicolumn{1}{c}{\textbf{cor.}} & \multicolumn{1}{c}{\textbf{p-v.}} &  &  \\ \midrule
\textbf{S-Bleu} & 0.50 & 0.0 & 0.47 & 0.0 & 0.59 & 0.0 & 0.58 & 0.0 & 0.68 & 0.0 & 0.57 & 5.8 \\
\textbf{R-Bleu} & -- & -- & 0.27 & 0.0 & 0.30 & 0.0 & -- & -- & -- & -- & - &  \\
\textbf{S-Meteor} & 0.49 & 0.0 & 0.48 & 0.0 & 0.61 & 0.0 & 0.57 & 0.0 & 0.64 & 0.0 & 0.56 & 6.1 \\
\textbf{R-Meteor} & -- & -- & 0.34 & 0.0 & 0.26 & 0.0 & -- & -- & -- & -- & - &  \\
\textbf{S-Bertscore} & \textbf{0.53} & 0.0 & {\ul 0.80} & 0.0 & \textbf{0.70} & 0.0 & {\ul 0.66} & 0.0 & {\ul0.78} & 0.0 & \textbf{0.69} & \textbf{1.7} \\
\textbf{R-Bertscore} & -- & -- & 0.51 & 0.0 & 0.38 & 0.0 & -- & -- & -- & -- & - &  \\
\textbf{S-Bleurt} & {\ul 0.52} & 0.0 & {\ul 0.80} & 0.0 & 0.60 & 0.0 & \textbf{0.70} & 0.0 & \textbf{0.80} & 0.0 & {\ul 0.68} & {\ul 2.3} \\
\textbf{R-Bleurt} & -- & -- & 0.59 & 0.0 & -0.05 & 0.13 & -- & -- & -- & -- & - &  \\
\textbf{S-Cosine} & 0.51 & 0.0 & 0.69 & 0.0 & {\ul 0.62} & 0.0 & 0.61 & 0.0 & 0.65 & 0.0 & 0.62 & 4.4 \\
\textbf{R-Cosine} & -- & -- & 0.40 & 0.0 & 0.29 & 0.0 & -- & -- & -- & -- & - & \\ \midrule
\textbf{QuestEval} & 0.23 & 0.0 & 0.25 & 0.0 & 0.49 & 0.0 & 0.47 & 0.0 & 0.62 & 0.0 & 0.41 & 9.0 \\
\textbf{LLaMa3} & 0.36 & 0.0 & \textbf{0.84} & 0.0 & {\ul{0.62}} & 0.0 & 0.61 & 0.0 &  0.76 & 0.0 & 0.64 & 3.6 \\
\textbf{our (3b)} & 0.49 & 0.0 & 0.73 & 0.0 & 0.54 & 0.0 & 0.53 & 0.0 & 0.7 & 0.0 & 0.60 & 5.8 \\
\textbf{our (8b)} & 0.48 & 0.0 & 0.73 & 0.0 & 0.52 & 0.0 & 0.53 & 0.0 & 0.7 & 0.0 & 0.59 & 6.3 \\  \bottomrule
\end{tabular}
\caption{Pearson correlation on human evaluation on system output. `R-': reference-based. `S-': source-based.}
\label{tab:sys}
\end{table*}



\begin{table}%[]
\centering
\fontsize{11pt}{11pt}\selectfont
\begin{tabular}{llllll}
\toprule
\multicolumn{1}{c}{\textbf{task}} & \multicolumn{1}{c}{\textbf{Lai}} & \multicolumn{1}{c}{\textbf{Zei.}} & \multicolumn{1}{c}{\textbf{Scia.}} & \textbf{} & \textbf{} \\ 
\multicolumn{1}{c}{\textbf{metrics}} & \multicolumn{1}{c}{\textbf{cor.}} & \multicolumn{1}{c}{\textbf{cor.}} & \multicolumn{1}{c}{\textbf{cor.}} & \textbf{avg.} & \textbf{\begin{tabular}[c]{@{}l@{}}avg.\\ rank\end{tabular}} \\ \midrule
\textbf{S-Bleu} & 0.40 & 0.40 & 0.19* & 0.33 & 7.67 \\
\textbf{S-Meteor} & 0.41 & 0.42 & 0.16* & 0.33 & 7.33 \\
\textbf{S-BertS.} & {\ul0.58} & 0.47 & 0.31 & 0.45 & 3.67 \\
\textbf{S-Bleurt} & 0.45 & {\ul 0.54} & {\ul 0.37} & 0.45 & {\ul 3.33} \\
\textbf{S-Cosine} & 0.56 & 0.52 & 0.3 & {\ul 0.46} & {\ul 3.33} \\ \midrule
\textbf{QuestE.} & 0.27 & 0.35 & 0.06* & 0.23 & 9.00 \\
\textbf{LlaMA3} & \textbf{0.6} & \textbf{0.67} & \textbf{0.51} & \textbf{0.59} & \textbf{1.0} \\
\textbf{Our (3b)} & 0.51 & 0.49 & 0.23* & 0.39 & 4.83 \\
\textbf{Our (8b)} & 0.52 & 0.49 & 0.22* & 0.43 & 4.83 \\ \bottomrule
\end{tabular}
\caption{Pearson correlation on human ratings on reference output. *not significant; we cannot reject the null hypothesis of zero correlation}
\label{tab:ref}
\end{table}


\begin{table*}%[]
\centering
\fontsize{11pt}{11pt}\selectfont
\begin{tabular}{lllllllll}
\toprule
\textbf{task} & \multicolumn{1}{c}{\textbf{ALL}} & \multicolumn{1}{c}{\textbf{sentiment}} & \multicolumn{1}{c}{\textbf{detoxify}} & \multicolumn{1}{c}{\textbf{catchy}} & \multicolumn{1}{c}{\textbf{polite}} & \multicolumn{1}{c}{\textbf{persuasive}} & \multicolumn{1}{c}{\textbf{formal}} & \textbf{\begin{tabular}[c]{@{}l@{}}avg. \\ rank\end{tabular}} \\
\textbf{metrics} & \multicolumn{1}{c}{\textbf{cor.}} & \multicolumn{1}{c}{\textbf{cor.}} & \multicolumn{1}{c}{\textbf{cor.}} & \multicolumn{1}{c}{\textbf{cor.}} & \multicolumn{1}{c}{\textbf{cor.}} & \multicolumn{1}{c}{\textbf{cor.}} & \multicolumn{1}{c}{\textbf{cor.}} &  \\ \midrule
\textbf{S-Bleu} & -0.17 & -0.82 & -0.45 & -0.12* & -0.1* & -0.05 & -0.21 & 8.42 \\
\textbf{R-Bleu} & - & -0.5 & -0.45 &  &  &  &  &  \\
\textbf{S-Meteor} & -0.07* & -0.55 & -0.4 & -0.01* & 0.1* & -0.16 & -0.04* & 7.67 \\
\textbf{R-Meteor} & - & -0.17* & -0.39 & - & - & - & - & - \\
\textbf{S-BertScore} & 0.11 & -0.38 & -0.07* & -0.17* & 0.28 & 0.12 & 0.25 & 6.0 \\
\textbf{R-BertScore} & - & -0.02* & -0.21* & - & - & - & - & - \\
\textbf{S-Bleurt} & 0.29 & 0.05* & 0.45 & 0.06* & 0.29 & 0.23 & 0.46 & 4.2 \\
\textbf{R-Bleurt} & - &  0.21 & 0.38 & - & - & - & - & - \\
\textbf{S-Cosine} & 0.01* & -0.5 & -0.13* & -0.19* & 0.05* & -0.05* & 0.15* & 7.42 \\
\textbf{R-Cosine} & - & -0.11* & -0.16* & - & - & - & - & - \\ \midrule
\textbf{QuestEval} & 0.21 & {\ul{0.29}} & 0.23 & 0.37 & 0.19* & 0.35 & 0.14* & 4.67 \\
\textbf{LlaMA3} & \textbf{0.82} & \textbf{0.80} & \textbf{0.72} & \textbf{0.84} & \textbf{0.84} & \textbf{0.90} & \textbf{0.88} & \textbf{1.00} \\
\textbf{Our (3b)} & 0.47 & -0.11* & 0.37 & 0.61 & 0.53 & 0.54 & 0.66 & 3.5 \\
\textbf{Our (8b)} & {\ul{0.57}} & 0.09* & {\ul 0.49} & {\ul 0.72} & {\ul 0.64} & {\ul 0.62} & {\ul 0.67} & {\ul 2.17} \\ \bottomrule
\end{tabular}
\caption{Pearson correlation on human ratings on our constructed test set. 'R-': reference-based. 'S-': source-based. *not significant; we cannot reject the null hypothesis of zero correlation}
\label{tab:con}
\end{table*}

\section{Results}
We benchmark the different metrics on the different datasets using correlation to human judgement. For content preservation, we show results split on data with system output, reference output and our constructed test set: we show that the data source for evaluation leads to different conclusions on the metrics. In addition, we examine whether the metrics can rank style transfer systems similar to humans. On style strength, we likewise show correlations between human judgment and zero-shot evaluation approaches. When applicable, we summarize results by reporting the average correlation. And the average ranking of the metric per dataset (by ranking which metric obtains the highest correlation to human judgement per dataset). 

\subsection{Content preservation}
\paragraph{How do data sources affect the conclusion on best metric?}
The conclusions about the metrics' performance change radically depending on whether we use system output data, reference output, or our constructed test set. Ideally, a good metric correlates highly with humans on any data source. Ideally, for meta-evaluation, a metric should correlate consistently across all data sources, but the following shows that the correlations indicate different things, and the conclusion on the best metric should be drawn carefully.

Looking at the metrics correlations with humans on the data source with system output (Table~\ref{tab:sys}), we see a relatively high correlation for many of the metrics on many tasks. The overall best metrics are S-BertScore and S-BLEURT (avg+avg rank). We see no notable difference in our method of using the 3B or 8B model as the backbone.

Examining the average correlations based on data with reference output (Table~\ref{tab:ref}), now the zero-shoot prompting with LlaMA3 70B is the best-performing approach ($0.59$ avg). Tied for second place are source-based cosine embedding ($0.46$ avg), BLEURT ($0.45$ avg) and BertScore ($0.45$ avg). Our method follows on a 5. place: here, the 8b version (($0.43$ avg)) shows a bit stronger results than 3b ($0.39$ avg). The fact that the conclusions change, whether looking at reference or system output, confirms the observations made by \citet{scialom-etal-2021-questeval} on simplicity transfer.   

Now consider the results on our test set (Table~\ref{tab:con}): Several metrics show low or no correlation; we even see a significantly negative correlation for some metrics on ALL (BLEU) and for specific subparts of our test set for BLEU, Meteor, BertScore, Cosine. On the other end, LlaMA3 70B is again performing best, showing strong results ($0.82$ in ALL). The runner-up is now our 8B method, with a gap to the 3B version ($0.57$ vs $0.47$ in ALL). Note our method still shows zero correlation for the sentiment task. After, ranks BLEURT ($0.29$), QuestEval ($0.21$), BertScore ($0.11$), Cosine ($0.01$).  

On our test set, we find that some metrics that correlate relatively well on the other datasets, now exhibit low correlation. Hence, with our test set, we can now support the logical reasoning with data evidence: Evaluation of content preservation for style transfer needs to take the style shift into account. This conclusion could not be drawn using the existing data sources: We hypothesise that for the data with system-based output, successful output happens to be very similar to the source sentence and vice versa, and reference-based output might not contain server mistakes as they are gold references. Thus, none of the existing data sources tests the limits of the metrics.  


\paragraph{How do reference-based metrics compare to source-based ones?} Reference-based metrics show a lower correlation than the source-based counterpart for all metrics on both datasets with ratings on references (Table~\ref{tab:sys}). As discussed previously, reference-based metrics for style transfer have the drawback that many different good solutions on a rewrite might exist and not only one similar to a reference.


\paragraph{How well can the metrics rank the performance of style transfer methods?}
We compare the metrics' ability to judge the best style transfer methods w.r.t. the human annotations: Several of the data sources contain samples from different style transfer systems. In order to use metrics to assess the quality of the style transfer system, metrics should correctly find the best-performing system. Hence, we evaluate whether the metrics for content preservation provide the same system ranking as human evaluators. We take the mean of the score for every output on each system and the mean of the human annotations; we compare the systems using the Kendall's Tau correlation. 

We find only the evaluation using the dataset Mir, Lai, and Ziegen to result in significant correlations, probably because of sparsity in a number of system tests (App.~\ref{app:dataset}). Our method (8b) is the only metric providing a perfect ranking of the style transfer system on the Lai data, and Llama3 70B the only one on the Ziegen data. Results in App.~\ref{app:results}. 


\subsection{Style strength results}
%Evaluating style strengths is a challenging task. 
Llama3 70B shows better overall results than our method. However, our method scores higher than Llama3 70B on 2 out of 6 datasets, but it also exhibits zero correlation on one task (Table~\ref{tab:styleresults}).%More work i s needed on evaluating style strengths. 
 
\begin{table}%[]
\fontsize{11pt}{11pt}\selectfont
\begin{tabular}{lccc}
\toprule
\multicolumn{1}{c}{\textbf{}} & \textbf{LlaMA3} & \textbf{Our (3b)} & \textbf{Our (8b)} \\ \midrule
\textbf{Mir} & 0.46 & 0.54 & \textbf{0.57} \\
\textbf{Lai} & \textbf{0.57} & 0.18 & 0.19 \\
\textbf{Ziegen.} & 0.25 & 0.27 & \textbf{0.32} \\
\textbf{Alva-M.} & \textbf{0.59} & 0.03* & 0.02* \\
\textbf{Scialom} & \textbf{0.62} & 0.45 & 0.44 \\
\textbf{\begin{tabular}[c]{@{}l@{}}Our Test\end{tabular}} & \textbf{0.63} & 0.46 & 0.48 \\ \bottomrule
\end{tabular}
\caption{Style strength: Pearson correlation to human ratings. *not significant; we cannot reject the null hypothesis of zero corelation}
\label{tab:styleresults}
\end{table}

\subsection{Ablation}
We conduct several runs of the methods using LLMs with variations in instructions/prompts (App.~\ref{app:method}). We observe that the lower the correlation on a task, the higher the variation between the different runs. For our method, we only observe low variance between the runs.
None of the variations leads to different conclusions of the meta-evaluation. Results in App.~\ref{app:results}.


\begin{lemma}\label{Lemma:multi1} 
   Fixing the number of data contributor $i$ collects $n_i$, and others' strategies $\strategy_{-i}$, $\hat{\mu}\left(X_i\right)$ is the minimax estimator for the Normal distribution class $\Normaldistrib := \left\{\mathcal{N}(\mu,\sigma^2) \;\middle|\; \mu \in \mathbb{R}\right\}$,
    \begin{align*}
       \hat{\mu}(X_i)  = \underset{\hat{\mu}}{\arg\min} \sbr{\sup _\mu \mathbb{E}\left[(\hat{\mu}( Y_i)- \hat{\mu}( Y_{-i}) )^2 \;\middle|\;  \mu \right] }
    \end{align*} 
     
\end{lemma}


\begin{proof}

\begin{align*}
    & \ \mathbb{E}\left[ \left( \hat{\mu}\left( Y_i \right)-\hat{\mu}\left( Y_{-i} \right)  \right)^2 \right] \\ =  & \ \mathbb{E}\left[ \left( (\hat{\mu}\left(  Y_i \right)-\mu) -(\hat{\mu}\left(  Y_{-i} \right) -\mu) \right)^2   \right] \\ =  & \ A_0 + \mathbb{E}\left[ (\hat{\mu}\left(  Y_i \right)-\mu)^2  \right]
\end{align*}
where $A_0$ is a positive coefficient.

Thus the maximum risk can be written as:

\begin{align*}
    \sup _\mu \mathbb{E}\left[A_0 + \left(\hat{\mu}\left( Y_i\right)-\mu\right)^{2} \;\middle|\;  \mu \right]
\end{align*}


We construct a lower bound on the maximum risk using a sequence of Bayesian risks. Let $\Lambda_{\ell}:=\mathcal{N}\left(0, \ell^2\right), \ell=1,2, \ldots$ be a sequence of prior for $\mu$. For fixed $\ell$, the posterior distribution is:
$$
\begin{aligned}
p\left(\mu \;\middle|\;  X_i\right) & \propto p\left(X_i \;\middle|\;  \mu\right) p(\mu) \\ & \propto \exp \left(-\frac{1}{2 \sigma^2} \sum_{x \in X_i}(x-\mu)^2\right) \exp \left(-\frac{1}{2 \ell^2} \mu^2\right) \\
& \propto \exp \left(-\frac{1}{2}\left(\frac{n_i}{\sigma^2}+\frac{1}{\ell^2}\right) \mu^2+\frac{1}{2} 2 \frac{\sum_{x \in X_i} x}{\sigma^2} \mu\right) .
\end{aligned}
$$

This means the posterior of $\mu$ given $X_i$ is Gaussian with:

\begin{align*}
    \mu \lvert\, X_i & \sim \mathcal{N}\left(\frac{n_i \hat{\mu}\left(X_i\right) / \sigma^2}{n_i / \sigma^2+1 / \ell^2}, \frac{1}{n_i / \sigma^2+1 / \ell^2}\right) 
    \\ & =: \mathcal{N}\left(\mu_{\ell}, \sigma_{\ell}^2\right).
\end{align*}



Therefore, the posterior risk is: 
$$
\begin{aligned}
&   \mathbb{E}\left[A_0 + \left(\hat{\mu}\left( Y_i\right)-\mu\right)^{2}  \;\middle|\;  X_i\right] \\ = &  \mathbb{E}\left[A_0 +  \left(\left(\hat{\mu}\left( Y_i\right)-\mu_{\ell}\right)-\left(\mu-\mu_{\ell}\right)\right)^{2 j} \;\middle|\;  X_i\right] \\ =
& A_0+\int_{-\infty}^{\infty} \underbrace{\left(e-\left(\hat{\mu}\left( Y_i\right)-\mu_{\ell}\right)\right)^2}_{=: F_1\left(e-\left(\hat{\mu}\left(Y_i\right)-\mu_{\ell}\right)\right)} \underbrace{\frac{1}{\sigma_{\ell} \sqrt{2 \pi}} \exp \left(-\frac{e^2}{2 \sigma_{\ell}^2}\right)}_{=: F_2(e)} d e
\end{aligned}
$$

Because:
\begin{itemize}
    \item $F_1(\cdot)$ is even function and increases on $[0, \infty)$;
    \item $F_2(\cdot)$ is even function and decreases on $\left[0, \infty \right)$, and $\int_{\mathbb{R}} F_2(e) de<\infty$
    \item For any $a \in \mathbb{R}, \int_{\mathbb{R}} F_1(e-a) F_2(e) de<\infty$
\end{itemize}

By the corollary of Hardy-Littlewood inequality in Lemma \ref{lemmaHardy},
$$
\int_{\mathbb{R}} F_1(e-a) F_2(e) d e \geq \int_{\mathbb{R}} F_1(e) F_2(e) d e
$$
which means the posterior risk is minimized when $\hat{\mu}\left(Y_i\right)=\mu_{\ell}$. We then write the Bayes risk as, the Bayes risk is minimized by the posterior mean $\mu_{\ell}$:

\begin{align*}    
R_{\ell}:= & \mathbb{E}\left[ A_0+\mathbb{E}\left[\left(\mu-\mu_{\ell}\right)^{2 } \;\middle|\;  X_i\right]\right] \\ = & A_0 + \sigma_{\ell}^{2}
\end{align*}

and the limit of Bayesian risk as $\ell \rightarrow \infty$ is
$$
R_{\infty}:= A_0 + \frac{\sigma^{2}}{n_i}.
$$

When $\hat{\mu}\left(Y_i\right)=\hat{\mu}\left(X_i\right)$, i.e, the contributor submit a set of size $n_i$ with each element equal to $ \hat{\mu}\left(X_i\right)$, the maximum risk is:

\begin{align*}
& \sup _\mu \mathbb{E}\left[A_0+\left(\mu- 
\hat{\mu}\left(Y_i\right) \right)^{2 } \;\middle|\;  \mu \right] \\
= & \sup _\mu \mathbb{E}\left[A_0+\left(\mu- 
\hat{\mu}\left(X_i\right) \right)^{2 } \;\middle|\;  \mu \right]  \\
= & A_0+ \sigma^{2 } n_i^{-1}  \\
= &  R_{\infty}.
\end{align*}

This implies that,
\begin{align*}
    & \underset{\mu}{\sup}\; \mathbb{E} \sbr{ \rbr{\hat{\mu}\left( Y_i \right)-\hat{\mu}\left( Y_{-i} \right)  }^2 \;\middle|\;  \mu }  \\ \geq & \; R_{\infty} =  \sup _\mu \; \mathbb{E}\left[A_0+\left(\mu- 
\hat{\mu}\left(X_i\right) \right)^{2 } \;\middle|\;  \mu \right]
\end{align*}

Therefore, the recommended strategy $\hat{\mu}(Y_i) =\hat{\mu}( X_i)$ has a smaller maximum risk than other strategies. 

\end{proof}




1. The payment from the buyer a constant $v(n^{\star})$.


2. If the payment for every seller is a fixed constant, then sellers can fabricate data without actually collecting data.\\


%$p_1 = b/2 +(\hat{\mu}(Y_1)- \hat{\mu}(Y_2))^2$, $p_2 = b/2 -(\hat{\mu}(Y_1)- \hat{\mu}(Y_2))^2$, seller 1 can choose ${\mu}' = u + \epsilon$, expected payment for seller 1 is larger than $b/2$. %NIC for seller 1: $g({\mu}',\mu ) < b/2$ for all ${\mu}' \neq \mu$. NIC for seller 2: $g( \mu, {\mu}') >  b/2 $ for all ${\mu}' \neq \mu$.

To demonstrate that no truthful mechanism (NIC) satisfies all desired properties in a two-seller setting, we use proof by contradiction.  

Suppose that there is a NIC mechanism $M$ satisfying property 1-5. Under this mechanism, the best strategy for each seller is to collect $N_i^{\star}$ amount of data and submit truthfully, where $N_1^{\star}+N_2^{\star} = n^{\star} $. Since $M$ is NIC for strategy space $\left\{ (f_i,N_i)\right\}_{i=1,2}$, it must be NIC for the sub strategy space $\left\{ (f_i, N_i^{\star})\right\}_{i=1,2}$. 


Consider the case in which everyone collects $N_i^{\star}$ data point and submits $N_i^{\star}$ data point. Assume that the true mean is $\mu$, seller $1$ submit $N({\mu}', \sigma^2),\ {\mu}' = f(\mu) $ while seller 2 submit $N({\mu}, \sigma^2) $. We denote seller 1's expected payment as  $\mathbb{E}\left[ p_1(M,\strategy) \right] = g({\mu}', \mu)$.
Seller 1's utility is then:
\[ u_1(M,f) = \underset{\mu}{\inf} \ g({\mu}', \mu) -c\times N_1^{\star}\] where $c$ is the cost for collecting one data point.


The total payment from the buyer is $v(n^{\star})$, hence by budget balance, \[p_2 (M,f)  = v(n^{\star}) -  p_1(M,f), \ \mathbb{E}\left[ p_2(M,f) \right] = v(n^{\star}) - g({\mu}', \mu) \]

By NIC, we have,
\[ \underset{\mu}{\inf} \ g({\mu}', \mu) -c\times N_1^{\star} \leq \underset{\mu}{\inf} \ g({\mu}, \mu) -c\times N_1^{\star} \] \[ \underset{\mu}{\inf} \ (v(n^{\star})- g( \mu, {\mu}')) -c\times N_2^{\star} \leq \underset{\mu}{\inf} \ (v(n^{\star})-g( \mu, {\mu})) -c\times N_2^{\star}  \]

%Using the fact that $\underset{\mu}{\inf} \ g({\mu}, \mu) = \underset{\mu}{\sup} \ g({\mu}, \mu) = {v(n^{\star})}/2 $.
We obtain that for any ${\mu}'$ and $\mu$,
\[  \underset{\mu}{\inf} \ g({\mu}', \mu) \leq \underset{\mu}{\inf} \ g({\mu}, \mu)   \] \[  \underset{\mu}{\sup} \  g( \mu, {\mu})  \leq \underset{\mu}{\sup } \ g( \mu, {\mu}')  \]

We next show that the inequalities are strict. Assume, for contradiction there exists ${\mu}'$, for any $\mu$, $g({\mu}', 
\mu) \geq \underset{\mu}{\inf} \ g({\mu}, \mu)$. It then follows that $ \underset{\mu}{\inf} \ g({\mu}', \mu) \geq \underset{\mu}{\inf} \ g({\mu}, \mu)$. Under this assumption, seller 1 could fabricate data by submitting $N({\mu}', \sigma^2)$ without collecting any actual data. This contradicts with the fact that $(f_1 = I, N_1 = N_1^{\star})$ is the best strategy for seller 1. Hence, for any ${\mu}'$, there exists some $\mu$ such that $g({\mu}', 
\mu) < \underset{\mu}{\inf} \ g({\mu}, \mu)$. Therefore, for any ${\mu}'$, \[ \underset{\mu}{\inf} \ g({\mu}', \mu) < \underset{\mu}{\inf} \ g({\mu}, \mu). \]Similarly, we also have \[ \underset{\mu}{\sup} \  g( \mu, {\mu})  < \underset{\mu}{\sup } \ g( \mu, {\mu}').  \] 


For any $ {\mu}'$, let $f({\mu}') =  \underset{\mu}{\arg\sup}\, g(\mu, {\mu}')$, then we have for any ${\mu}'$,  $f({\mu}') \neq {\mu}'$ and $ g(f({\mu}'), {\mu}') > \underset{\mu}{\sup} \  g( \mu, {\mu}) \geq  \underset{\mu}{\inf} \  g( \mu, {\mu})$. This implies that seller 1 could fabricate data based on function $f$, this contradicts with the fact that the mechanism is NIC.


pay the seller $v(n^{\star})/2 - \beta (\hat{\mu}(Y_1)-\hat{\mu}(Y_2))^2$, charge buyer $v(n^{\star}) - 2\beta (\hat{\mu}(Y_1)-\hat{\mu}(Y_2))^2$


Buyer utility: $v(n^*)$-payment
Seller utility: payment - $cn^*$.

Sellers utility is positive?

Seller payment $(v(n^*)/2)-\beta (\hat{\mu}(Y_1)-\hat{\mu}(Y_2))^2 $, $\beta = (v(n^*)-cn^*)c(n^*)^2 / 4\sigma^2$, 



\section{Multiple buyers}
\subsection{}
Question 1: Do we fix the amount of data for sale ahead of time?


Assume we fix $N$, the amount of data for sale. The goal of mechanism is to maximize the sellers' revenue. According to previous paper, there exists at least one type who purchase at the amount $N$. Suppose that in offline setting, i.e., when the mechanism knows the buyer valuation and type distribution, the optimal revenue is $\text{OPT} $. 


We ask $d$ sellers to collect $N$ data points, and split $\text{OPT} $ revenue among sellers. (data can be duplicated). 

\[ p_i(M,s) =\mathbb{I}\left( \left| Y_i \right| = \frac{N}{d} \right) \rbr{\frac{\text{OPT}}{d}+d_i \frac{\sigma^2}{N_{-i}^{\star}} +d_i \frac{\sigma^2}{N_i^{\star}} }- d_i \rbr{\hat{\mu}(Y_i)-\hat{\mu}(Y_{-i}) }^2  \]

Buyer's expected utility is non negative. Next, we discuss sellers' expected utility $T\mathbb{E}[p_i]- cn_i$ (over $T$ roundsm\, maybe $T$ is fixed). Let $N_i^{\star} = \frac{N}{d}$.

\[ u_i(M,s) = \mathbb{I}\left( \left| Y_i \right| = \frac{N}{d} \right) \rbr{\frac{\text{OPT}}{d}+d_i \frac{\sigma^2}{N_{-i}^{\star}} +d_i \frac{\sigma^2}{N_i^{\star}} }T- Td_i \mathbb{E}\rbr{\hat{\mu}(Y_i)-\hat{\mu}(Y_{-i}) }^2 
 -  cn_i \]
Choose $d_i = \frac{c(N_i^{\star})^2}{T\sigma^2}$.


If we do not fix \( T \) in advance, let \( T_0 \) represent the time at which the cumulative utility over at least \( T_0 \) rounds is non-negative. We can select \( d_i \) such that \( d_i \geq \frac{c(N/d)^2}{T_0 \sigma^2} \). This ensures that the seller will never choose to collect less than \( N/d \) amount of data.








\section{Single buyer} \label{section: singlebuyer}


Each contributor \( i \) incurs a cost \( c_i \) to collect data,  without loss of generality, we assume \( c_1 \leq c_2 \leq \dots \leq c_d \). The broker is assumed to have full knowledge of the buyer's valuation curve \( \val(n) \), as well as the contributors’ costs $ c_{ i \in \contributors}$  for collecting each data point.

The maximum total profit for the contributors, assuming no constraints on truthful submissions, is given by:
\[
\mathrm{profit}^\star = \underset{\datanum_1, \dots, \datanum_d}{\max} \left( \val \left(\sum_{i=1}^{d} \datanum_i\right) - \sum_{i=1}^{d} c_i \datanum_i \right),
\]

where \( \datanum_i \) represents the number of data points collected by contributor \( i \). In this unconstrained scenario, since contributor 1 has the lowest collection cost, the optimal strategy is for contributor 1 to collect all the required data points while other contributors collect none. This approach maximizes total profit without considering the incentive for truthful submissions.

However, when truthful submission is taken into account, at least two contributors are needed because we need to use one contributor's data to verify the other's. We demonstrate that the maximum profit achievable under Nash Equilibrium is:
\[
\mathrm{profit}^\star + (c_1 - c_2),
\]
where \( c_1 - c_2 \) represents the additional cost differential caused by enforcing truthful behavior among contributors. 

\begin{algorithm}[H]
    \caption{Process of mechanism.}
    \begin{algorithmic}
        \STATE {\bfseries Input:} A population of buyers $\buyers$.
        \STATE The broker chooses the optimal data allocation to maximize contributors' profit:
        $$
        \{ \datanum_i^{\star} \}_{i=1}^d = \underset{\datanum_1,\dots,\datanum_d}{\arg\max}\  \rbr{v\rbr{\sum_i \datanum_i}-\sum_i \cost_i \datanum_i  }
      $$
       
        \STATE The broker recommend a strategy to each contributor: $\strategy_i^{\star} = (\datanum_i^{\star}, \mathbf{I})$.
        \STATE Each contributor selects a strategy $\strati = (\datanum_i, f_i)$, collects $\datanum_i$ data points $X_i$, and submits $Y_i = f_i(X_i)$.
        \STATE The mechanism generates an estimator $\hat{\mu}(M,\strategy)$ for the buyer, and charge her $\price_{j \in \buyers}$. \COMMENT{See (\ref{eq:buyer_pay}) }
        \STATE Each contributor is paid $\payi$.    \COMMENT{See (\ref{eq:seller_pay}) }
    \end{algorithmic}   
\end{algorithm}




\begin{theorem}
    there exists NIC mechanism satisfying the following properties (1) $\strategy^{\star}$ is Nash equilibrium. (2) The mechanism is individually rational at $\strategy^{\star}$ for both buyers and sellers. (3) Budget balance. (4) Under strategy $\strategy^{\star}$, the expected profit of buyers approximates the optimal profit $ \mathrm{profit}^{\star}$ within an additive error $\cost_2 - \cost_1$. 
\end{theorem}


Let $n^{\star}$ denote the optimal total number of data to be collected, $\datanum_1^{\star}=n^{\star}-1$, and $\datanum_2^{\star}=1$. Let $w=\val(n^{\star})-cn^{\star}$ denote the social welfare. One option for payment function is

\begin{align*}
    &\; \pay_i(M,\strategy^{\star}) \\  
    = & \;\mathbb{I}\left( \left| Y_i \right| = \datanum_i^{\star} \right) \rbr{\frac{\datanum_1^{\star}}{n^{\star}}\val(n^{\star})+d_i \frac{\sigma^2}{\datanum_{-i}^{\star}} +d_i \frac{\sigma^2}{\datanum_i^{\star}} } \\ & - d_i \rbr{\hat{\mu}(Y_i)-\hat{\mu}(Y_{-i}) }^2, \\[20pt] % Adds vertical space between equations
    &\; \price(M,\strategy^{\star}) \\  
    = & \; \sum_{i=1}^{2}\mathbb{I}\left(  \left| Y_i \right| = \datanum_i^{\star} \right) \rbr{\frac{\datanum_1^{\star}}{n^{\star}}\val(n^{\star}) +d_i \frac{\sigma^2}{\datanum_{-i}^{\star}} +d_i \frac{\sigma^2}{\datanum_i^{\star}} } \\ 
    & - \sum_{i=1}^{2} d_i \rbr{\hat{\mu}(Y_i)-\hat{\mu}(Y_{-i}) }^2, \\[20pt] % Adds vertical space between equations
    &\;  \utilityb (M,\strategy^{\star}) \\ 
   = & \; v(\datanum^{\star}) -\mathbb{E}[\price(M,\strategy^{\star})] \\ 
    = & \; - \sum_{i=1}^{d} \rbr{d_i \frac{\sigma^2}{\datanum_{-i}^{\star}} +d_i \frac{\sigma^2}{\datanum_i^{\star}}  } + \sum_{i=1}^{d}d_i \mathbb{E} \rbr{\hat{\mu}(Y_i)-\hat{\mu}(Y_{-i}) }^2 \\ 
    = & \; 0.
\end{align*}



Then contributors i's expected ptofit under strategy $\strategy^{\star}$ is 

\begin{align*}
& \; \utilci \rbr{\mechspace, \strategy^{\star} } \\ = &  \; \mathbb{E}\sbr{\pay_i(M,\strategy^{\star})} - \cost_i n_i^{\star} \\ = &  \; \mathbb{I}\left(\left| Y_i \right| = \datanum_i^{\star} \right) \rbr{\frac{\datanum_1^{\star}}{n^{\star}}\val(n^{\star})  +d_i \frac{\sigma^2}{\datanum_{-i}^{\star}} +d_i \frac{\sigma^2}{\datanum_i^{\star}} }\\  & -  d_i \mathbb{E}\rbr{\hat{\mu}(Y_i)-\hat{\mu}(Y_{-i}) }^2  -\cost n_i^{\star} \\ = &  \;  \frac{w}{\numcontributors}
\end{align*}
where $d_i = c(\datanum_i^*/d)^2 $, 

%\textcolor{red}{Buyer payment $\pi(M,s)$ can ve negative? Can it be interpreted as when the quality of data is bad, the mechanism pays money to the buyer as compensate, the contributor pays money to the mechanism as a penalty. }\textcolor{red}{Buyer pays $v(n^*)$, $p_i = v(n^*) \frac{\rbr{\hat{\mu}(Y_i)-\hat{\mu}(Y_{-i}) }^{-2}}{\sum{\rbr{\hat{\mu}(Y_i)-\hat{\mu}(Y_{-i}) }^{-2}}}$ }


We prove the NIC in three steps.


\textbf{First step} \textcolor{red}{to be fixed}: Giving others submitting truthfully, we know that when fixing $n_i$, submitting $\left| Y_i \right| = \frac{n^{\star}}{d}$ is the best strategy, otherwise, $\pay_i <0$ when $\left| Y_i \right| \neq \frac{n^{\star}}{d}$. Therefore, for any $n_i$ and $f_i$, we have for any $\mu$,
\begin{align*}
& u_i\rbr{\mechspace, (n_i,f_i, \left| Y_i \right| =\datanum_i^{\star}),\strategy_{-i}^{\star} } \\  \geq \  & u_i\rbr{\mechspace, \strategy_{-i}^{\star}} 
\end{align*}

\textbf{Second step}: Fixing $n_i$ and $\left| Y_i \right|$, sample mean $\hat{\mu}(X_i)$ is minimax estimator of $\mathbb{E}\rbr{\rbr{\hat{\mu}(Y_i)-\hat{\mu}(Y_{-i})}^2 \;\middle|\; P} $, i.e., \[ \hat{\mu}(X_i) = \underset{\hat{\mu}}{\inf} \  \underset{\distrifamily}{\sup}\ \mathbb{E}\sbr{\rbr{\hat{\mu}(Y_i)-\hat{\mu}(Y_{-i})}^2  \;\middle|\; \distri} \]Therefore we have 
\begin{align*}
     & \underset{\distrifamily}{\inf}\;u_i\rbr{\mechspace, (n_i,\hat{\mu}(Y_i)=\hat{\mu}(X_i), \left| Y_i \right|),\strategy_{-i}^{\star} } \\  \geq \  & \underset{\distrifamily}{\inf}\;u_i\rbr{\mechspace, (n_i,\hat{\mu}(Y_i), \left| Y_i \right|),\strategy_{-i}^{\star} } 
\end{align*}


\textbf{Third step}: By setting constant $d_i= c\rbr{\frac{\datanum_i^{\star}}{\sigma}}^2$, when fixing $\hat{\mu}(Y_i)=\hat{\mu}(X_i)$ and $\left| Y_i \right| = \datanum_i^{\star}$, collecting $n_i = \datanum_i^{\star}$ amount of data maximize the contributor utility 
\begin{align*}
    \underset{\distrifamily}{\inf}\;u_i\rbr{\mechspace, (n_i= \datanum_i^{\star},\hat{\mu}(Y_i)=\hat{\mu}(X_i), \left| Y_i \right|=\strategy_{-i}^{\star} } \\ \geq \underset{\distrifamily}{\inf}\;u_i\rbr{\mechspace, (n_i,\hat{\mu}(Y_i)=\hat{\mu}(X_i), \left| Y_i \right|=s_{-i}^{\star}}  
\end{align*}
 

\begin{align*}
    & u_i\rbr{\mechspace, (n_i,\hat{\mu}(Y_i)=\hat{\mu}(X_i), \left| Y_i \right|=\datanum_i^{\star}),\strategy_{-i}^{\star} }  \\ = & \rbr{\frac{w}{\numcontributors}+ c \datanum_i^{\star} +d_i \frac{\sigma^2}{\datanum_{-i}^{\star}} +d_i \frac{\sigma^2}{\datanum_i^{\star}} } - d_i \rbr{ \frac{\sigma^2}{\datanum_{-i}^{\star}} +\frac{\sigma^2}{\datanum_i^{\star}} } - cn_i
\end{align*}

Therefore, we have 
\begin{align*}
    &\underset{\distrifamily}{\inf}\;u_i \rbr{\mechspace, (n_i= \datanum_i^{\star},\hat{\mu}(Y_i)=\hat{\mu}(X_i), \left| Y_i \right|=\datanum_i^{\star}),\strategy_{-i}^{\star} } \\  = & \underset{\distrifamily}{\inf}\;u_i\rbr{\mechspace, (\datanum_i^{\star},f_i^{\star}),\strategy_{-i}^{\star} } \\ \geq &   \underset{\distrifamily}{\inf}\; u_i\rbr{\mechspace, (n_i,f_i ),\strategy_{-i}^{\star}} 
\end{align*}

When following the best strategy, properties 1-5 are all satisfied.



%%%%%%%%%%%%%%%%%%%%%%%%%%%%%%%%%%%%%%%%%%%%%%%%%%%%%%%%%%%%%%%%%%%%%%%%%%%%%%%
%%%%%%%%%%%%%%%%%%%%%%%%%%%%%%%%%%%%%%%%%%%%%%%%%%%%%%%%%%%%%%%%%%%%%%%%%%%%%%%


\end{document}


% This document was modified from the file originally made available by
% Pat Langley and Andrea Danyluk for ICML-2K. This version was created
% by Iain Murray in 2018, and modified by Alexandre Bouchard in
% 2019 and 2021 and by Csaba Szepesvari, Gang Niu and Sivan Sabato in 2022.
% Modified again in 2023 and 2024 by Sivan Sabato and Jonathan Scarlett.
% Previous contributors include Dan Roy, Lise Getoor and Tobias
% Scheffer, which was slightly modified from the 2010 version by
% Thorsten Joachims & Johannes Fuernkranz, slightly modified from the
% 2009 version by Kiri Wagstaff and Sam Roweis's 2008 version, which is
% slightly modified from Prasad Tadepalli's 2007 version which is a
% lightly changed version of the previous year's version by Andrew
% Moore, which was in turn edited from those of Kristian Kersting and
% Codrina Lauth. Alex Smola contributed to the algorithmic style files.


%%%%%%%%%%%%%%%%%%%%%%%%%%%%%%%%%%%%%%%%%%%%%%%%%%%%%%%%%%%%%%%%%%%%%%%%%%%%%%%
%%%%%%%%%%%%%%%%%%%%%%%%%%%%%%%%%%%%%%%%%%%%%%%%%%%%%%%%%%%%%%%%%%%%%%%%%%%%%%%
% APPENDIX
%%%%%%%%%%%%%%%%%%%%%%%%%%%%%%%%%%%%%%%%%%%%%%%%%%%%%%%%%%%%%%%%%%%%%%%%%%%%%%%
%%%%%%%%%%%%%%%%%%%%%%%%%%%%%%%%%%%%%%%%%%%%%%%%%%%%%%%%%%%%%%%%%%%%%%%%%%%%%%%
\newpage
\appendix
\onecolumn

\section{Training Details}
Below, we provide detailed training hyper-parameters and setups for dense CLIP (weights are used for sparse upcycling), sparse CLIP trained from scratch, and \name.

\subsection{Training hyper-parameters}
We primarily follow \cite{radford2021learningtransferablevisualmodels} for hyper-parameter selection, using the WIT-3000M~\cite{wu2024mofilearningimagerepresentations} and DFN-5B~\cite{fang2023datafilteringnetworks} training datasets. Table \ref{tab:hyper-parameter} summarizes the hyper-parameters for all experiments, including MoE-specific configurations and parameters for dense CLIP, sparse CLIP, and \name.

\begin{table}[h]
\caption{Training hyper-parameters and settings for dense CLIP used for sparse upcycling and \name}
\label{tab:hyper-parameter}
\vskip 0.15in
\begin{center}
\begin{small}
\begin{sc}
% \setlength{\tabcolsep}{4.0 pt}
\renewcommand{\arraystretch}{1.5}
\resizebox{0.8\textwidth}{!}{
\begin{tabular}{lc}
\toprule
\multicolumn{2}{c}{\textbf{General}} \\
\midrule
Batch size & 32768 \\
\rowcolor{lightgray} 
Image size & $224 \times 224$ \\
Text tokenizer & T5~\cite{raffel2023exploringlimitstransferlearning}, lowercase \\
\rowcolor{lightgray} 
Text maximum length & 77 tokens \\
% Steps for dense model & 439087 (\textit{i.e.,} $\sim$ 14B examples seen) \\
% \rowcolor{lightgray} 
% Steps for upcycling & 351269 (\textit{i.e.,} $\sim$ 11B examples seen) \\
Optimizer & AdamW ($\beta_1 = 0.9, \beta_2 = 0.98$) \\
% \rowcolor{lightgray} 
% Peak learning rate (LR) & $5e^{-4}$ (dense CLIP), $5e^{-5}$ (\name) \\
\rowcolor{lightgray} 
LR schedule & cosine decays with linear warm-up (first 2k steps) \\
% \rowcolor{lightgray} 
% Weight decay & 0.2 (dense CLIP), 0.05 (\name) \\
Dropout rate & 0.0 \\
\midrule

\multicolumn{2}{c}{\textbf{MoE}} \\
\midrule
% Expert count & 8 (for both text and image seprately), 16 (for shared backbone) \\
Inner structure & Pre-Layer Normalization~\cite{xiong2020layernormalizationtransformerarchitecture} \\
\rowcolor{lightgray} 
Router type & Top-2 routing \\
Expert capacity factor ($C$) & 2.0 (both text and image) \\
\rowcolor{lightgray} 
MoE position & [dense, sparse] (half of MLP layers replaced by MoE layers) \\
Load balance loss weight &0.01 \\
\rowcolor{lightgray} 
Router-z loss weight & 0.0001 \\
\midrule

\multicolumn{2}{c}{\textbf{Dense Model}} \\
\midrule
Steps & 439087 (\textit{i.e.,} $\sim$ 14B examples seen) \\
\rowcolor{lightgray} 
Peak learning rate (LR) & $5e^{-4}$ \\
Weight decay & 0.2 \\
\midrule

\multicolumn{2}{c}{\textbf{\name}} \\
\midrule
Steps & 351269 (\textit{i.e.,} $\sim$ 11B examples seen) \\
\rowcolor{lightgray} 
Peak learning rate (LR) & $5e^{-5}$ \\
Weight decay & 0.05 \\
\rowcolor{lightgray} 
Expert count & 8 (for text and image separately) \\
\midrule

\multicolumn{2}{c}{\textbf{Sparse Model}} \\
\midrule
Steps & 790356 (\textit{i.e.,} $\sim$ 25B examples seen) \\
\rowcolor{lightgray} 
Peak learning rate (LR) & $5e^{-4}$ \\
Weight decay & 0.2 \\
\rowcolor{lightgray} 
Expert count & 8 \\
% \midrule

\bottomrule
\end{tabular}}
\end{sc}
\end{small}
\end{center}
\end{table}



\newpage
\section{Ablation study}

\subsection{Normalize gating weights before or after routing.} 
To mitigate the initial quality drop observed when applying sparse upcycling, we experimented with normalizing the router output logits after routing. This ensures the remaining gating weights are normalized to sum to 1, even when some tokens are dropped due to expert capacity constraints. The intuition behind this approach is that in the dense model, each token was previously processed by a single expert MLP.

\begin{figure}[ht]  % Create a figure environment
    \centering  % Center the image
    \includegraphics[width=1.0\linewidth]{images/normalization-compare.pdf}
    \caption{Model performance for gating normalization applied before or after routing} 
    \label{fig:normalization-compare}  % Add a label for referencing
\end{figure}

As shown in Figure \ref{fig:normalization-compare}, normalizing gating weights post-routing helps reduce the initial quality drop. However, in terms of final model performance, this approach shows improved results in image-to-text retrieval, but performs worse in text-to-image retrieval. 

A possible explanation for this discrepancy is that post-routing normalization maintains the magnitude of all remaining tokens, which benefits the image encoder, as most image tokens are informative. In contrast, text encoder often deals with padding tokens, and reducing the magnitude of these tokens can enhance the text encoder's ability to focus on meaningful content. It also aligns with the finding that the initial quality drop is the biggest when adding MoE layers into image modality only.

\newpage
\section{Tabular results}
\subsection{Comparison of model architectures and impact of LIMOE auxiliary loss} 

All results for different model architectures, with and without LIMOE auxiliary loss, as discussed in Section \ref{comparison-methodology}.

% \begin{table}[h]
% \caption{All results as discussed in Section \ref{comparison-methodology}}
% \label{tab:appendix-recipe-study}
% \vskip 0.15in
% \begin{center}
% \begin{small}
% \begin{sc}
% % \setlength{\tabcolsep}{4.0 pt}
% % \renewcommand{\arraystretch}{0.8}
% \begin{tabular}{lccccc}
% \toprule
% \multirow{3}{*}{\multicolumn{1}{l}{Model}} &\multicolumn{1}{c}{Imagenet} &\multicolumn{2}{c}{MSCOCO} &\multicolumn{2}{c}{Flickr30K} 
% & & &T2I &I2T &T2I &I2T \\
% &Acc@1 &R@1 &R@1 &R@1 &R@1 \\

% \midrule
% % \rowcolor{lightgray} 
% Shared & 69.7 &46.7 &65.6 &71.4 &86.3 \rule{0pt}{3ex} \\
% % \quad+LIMOE Aux. Loss &73.1 &49.7 &69.7 &75.6 &87.9 \rule{0pt}{3ex} \\
% % \quad $\Delta$ & +3.4 & +3.0 & +4.1 & +4.2 & +1.6 \\
% % \cdashline{1-6}
% % \rowcolor{lightgray} 
% Shared-Up &75.2 &51.6 &\textbf{72.7} &78.0 &92.0 \rule{0pt}{3ex} \\
% % \quad+LIMOE Aux. Loss &73.9 &50.9 &70.1 &73.8 &85.5 \rule{0pt}{3ex} \\
% % \quad $\Delta$ & -1.3 & -0.7 & -2.6 & -4.2 & -6.5 \\
% % \cdashline{1-6}
% % \rowcolor{lightgray} 
% Separated &74.5 &\textbf{53.1} &70.6 &78.3 &88.2 \rule{0pt}{3ex} \\
% % \quad+LIMOE Aux. Loss &72.6 &46.4 &62.6 &73.4 &85.2 \rule{0pt}{3ex} \\
% % \quad $\Delta$ & -2.0 & -6.7 & -8.0 & -4.9 & -3.0 \\
% % \cdashline{1-6}
% % \rowcolor{lightgray} 
% Separated-Up &\textbf{76.9} &52.1 &71.5 &\textbf{80.9} &\textbf{92.3} \rule{0pt}{3ex} \\
% % \quad+LIMOE Aux. Loss &75.9 &52.9 &73.5 &81.3 &92.5 \rule{0pt}{3ex} \\
% % \quad $\Delta$ & -1.0 & +0.8 & +2.0 & +0.4 & +0.2 \\
% \bottomrule
% \end{tabular}
% \end{sc}
% \end{small}
% \end{center}
% \end{table}




\begin{table*}[h]
\caption{All results from Table \ref{tab:methodology-compare-no-limoe} and Figure \ref{fig:limoe-compare} as discussed in Section \ref{comparison-methodology}}
\label{tab:appendix-recipe-study}
\vskip 0.15in
\begin{center}
\begin{small}
\begin{sc}
% \setlength{\tabcolsep}{3.5 pt}
% \renewcommand{\arraystretch}{0.8}
\begin{tabular}{lccccc}
\toprule
\multirow{2}{*}{Model} &\multicolumn{1}{c}{Imagenet} &\multicolumn{2}{c}{COCO} &\multicolumn{2}{c}{Flickr30K} \\
&Accuray@1 &T2I R@1 &I2T R@1 &T2I R@1 &I2T R@1 \\
\midrule
\rowcolor{lightgray} 
Shared & 69.7 &46.7 &65.6 &71.4 &86.3 \rule{0pt}{3ex} \\
\quad+LIMOE Aux. Loss &73.1 &49.7 &69.7 &75.6 &87.9 \rule{0pt}{3ex} \\
\quad $\Delta$ & +3.4 & +3.0 & +4.1 & +4.2 & +1.6 \\
\cdashline{1-6}
\rowcolor{lightgray} 
Shared-UpCycle &75.2 &51.6 &\textbf{72.7} &78.0 &92.0 \rule{0pt}{3ex} \\
\quad+LIMOE Aux. Loss &73.9 &50.9 &70.1 &73.8 &85.5 \rule{0pt}{3ex} \\
\quad $\Delta$ & -1.3 & -0.7 & -2.6 & -4.2 & -6.5 \\
\cdashline{1-6}
\rowcolor{lightgray} 
Separated &74.5 &\textbf{53.1} &70.6 &78.3 &88.2 \rule{0pt}{3ex} \\
\quad+LIMOE Aux. Loss &72.6 &46.4 &62.6 &73.4 &85.2 \rule{0pt}{3ex} \\
\quad $\Delta$ & -2.0 & -6.7 & -8.0 & -4.9 & -3.0 \\
\cdashline{1-6}
\rowcolor{lightgray} 
Separated-UpCycle &\textbf{76.9} &52.1 &71.5 &\textbf{80.9} &\textbf{92.3} \rule{0pt}{3ex} \\
\quad+LIMOE Aux. Loss &75.9 &52.9 &73.5 &81.3 &92.5 \rule{0pt}{3ex} \\
\quad $\Delta$ & -1.0 & +0.8 & +2.0 & +0.4 & +0.2 \\
\bottomrule
\end{tabular}
\end{sc}
\end{small}
\end{center}
\end{table*}


\newpage
\subsection{Comparison of MoE added to single modality or both modalities} 

All results from Figure \ref{fig:moe-modality} to compare MoE added into different modalities. 

\begin{table*}[h]
\caption{All results from Figure \ref{fig:moe-modality}. MoE-text: MoE layers are added into text modality only. MoE-image: MoE layers are added into image modality only. MoE-both: MoE layers are added into both text and image modalities.}
\label{tab:appendix-modality}
\vskip 0.15in
\begin{center}
\begin{small}
\begin{sc}
% \setlength{\tabcolsep}{4.0 pt}
% \renewcommand{\arraystretch}{0.8}
\begin{tabular}{lcccccc}
\toprule
\multirow{3}{*}{Model} &\multirow{3}{*}{Steps} &\multicolumn{1}{c}{Imagenet} &\multicolumn{2}{c}{COCO} &\multicolumn{2}{c}{Flickr30K} \\
& &Accuracy@1 &T2I R@1 &I2T R@1 &T2I R@1 &I2T R@1 \\
% & &Acc@1 &R@1 &R@1 &R@1 &R@1 \\

\midrule
% \rowcolor{lightgray} 
\multirow{5}{*}{MoE-text} 
&0 &70.2 &42.7 &52.5 &72.1 &79.4 \rule{0pt}{3ex} \\
% \cdashline{2-7}
&5 & 72.3 &42.2 &63.2 &70.0 &86.9 \rule{0pt}{3ex} \\
% \rowcolor{lightgray}
&100 & 73.3 &42.9 &63.1 &70.4 &88.0 \rule{0pt}{3ex} \\
% \rowcolor{lightgray} 
&200 & 75.4 &42.8 &64.1 &72.7 &88.6 \rule{0pt}{3ex} \\
&350 & 77.2 &45.5 &66.0 &74.2 &89.6 \rule{0pt}{3ex} \\
% \quad+LIMOE Aux. Loss &73.1 &49.7 &69.7 &75.6 &87.9 \rule{0pt}{3ex} \\
% \quad $\Delta$ & +3.4 & +3.0 & +4.1 & +4.2 & +1.6 \\
% \cdashline{1-6}
% \rowcolor{lightgray} 
\midrule
\multirow{5}{*}{MoE-image} &0 &62.6 &29.1 &50.9 &56.1 &73.7 \rule{0pt}{3ex} \\
% \rowcolor{lightgray} 
&5 & 72.4 &41.4 &62.5 &70.0 &84.8 \rule{0pt}{3ex} \\
&100 & 73.8 &42.2 &63.6 &71.7 &88.2 \rule{0pt}{3ex} \\
% \rowcolor{lightgray} 
&200 & 75.6 &43.8 &65.0 &72.2 &88.4 \rule{0pt}{3ex} \\
&350 & 77.6 &45.5 &66.3 &74.5 &89.4 \rule{0pt}{3ex} \\

\midrule
\multirow{5}{*}{MoE-both} &0 &57.2 &33.0 &45.3 &62.7 &73.4 \rule{0pt}{3ex} \\
% \rowcolor{lightgray} 
&5 & 71.4 &44.9 &67.9 &71.6 &87.8 \rule{0pt}{3ex} \\
&100 & 72.8 &49.7 &71.4 &74.9 &89.1 \rule{0pt}{3ex} \\
% \rowcolor{lightgray} 
&200 & 75.1 &49.0 &68.5 &73.1 &85.1 \rule{0pt}{3ex} \\
&350 & 76.9 &52.1 &71.5 &80.9 &92.3 \rule{0pt}{3ex} \\

\bottomrule
\end{tabular}
\end{sc}
\end{small}
\end{center}
\end{table*}

\newpage
\subsection{Comparison of capacity factor} 

All results from Figure \ref{fig:capacity-compare} to compare the impact of capacity factor $C$.

\begin{table*}[h]
\caption{All results from Figure \ref{fig:capacity-compare}}
\label{tab:appendix-capacity-compare}
\vskip 0.15in
\begin{center}
\begin{small}
\begin{sc}
% \setlength{\tabcolsep}{4.0 pt}
% \renewcommand{\arraystretch}{0.8}
\begin{tabular}{lccccc}
\toprule
\multirow{3}{*}{Model} &\multicolumn{1}{c}{Imagenet} &\multicolumn{2}{c}{COCO} &\multicolumn{2}{c}{Flickr30K} \\
&Accuracy@1 &T2I R@1 &I2T R@1 &T2I R@1 &I2T R@1 \\

\midrule
% \rowcolor{lightgray} 
$C_{image}=2, C_{text}=2$ & 76.9 &52.1 &71.5 &80.9 &92.3 \rule{0pt}{3ex} \\
% \quad+LIMOE Aux. Loss &73.1 &49.7 &69.7 &75.6 &87.9 \rule{0pt}{3ex} \\
% \quad $\Delta$ & +3.4 & +3.0 & +4.1 & +4.2 & +1.6 \\
% \cdashline{1-6}
% \rowcolor{lightgray} 
$C_{image}=4, C_{text}=2$ &78.4 &46.3 &66.9 &75.5 &90.2 \rule{0pt}{3ex} \\
% \quad+LIMOE Aux. Loss &73.9 &50.9 &70.1 &73.8 &85.5 \rule{0pt}{3ex} \\
% \quad $\Delta$ & -1.3 & -0.7 & -2.6 & -4.2 & -6.5 \\
% \cdashline{1-6}
% \rowcolor{lightgray} 
% Separated &74.5 &\textbf{53.1} &70.6 &78.3 &88.2 \rule{0pt}{3ex} \\
% % \quad+LIMOE Aux. Loss &72.6 &46.4 &62.6 &73.4 &85.2 \rule{0pt}{3ex} \\
% % \quad $\Delta$ & -2.0 & -6.7 & -8.0 & -4.9 & -3.0 \\
% % \cdashline{1-6}
% % \rowcolor{lightgray} 
% Separated-Up &\textbf{76.9} &52.1 &71.5 &\textbf{80.9} &\textbf{92.3} \rule{0pt}{3ex} \\
% \quad+LIMOE Aux. Loss &75.9 &52.9 &73.5 &81.3 &92.5 \rule{0pt}{3ex} \\
% \quad $\Delta$ & -1.0 & +0.8 & +2.0 & +0.4 & +0.2 \\
\bottomrule
\end{tabular}
\end{sc}
\end{small}
\end{center}
\end{table*}

% You can have as much text here as you want. The main body must be at most $8$ pages long.
% For the final version, one more page can be added.
% If you want, you can use an appendix like this one.  

% The $\mathtt{\backslash onecolumn}$ command above can be kept in place if you prefer a one-column appendix, or can be removed if you prefer a two-column appendix.  Apart from this possible change, the style (font size, spacing, margins, page numbering, etc.) should be kept the same as the main body.
%%%%%%%%%%%%%%%%%%%%%%%%%%%%%%%%%%%%%%%%%%%%%%%%%%%%%%%%%%%%%%%%%%%%%%%%%%%%%%%
%%%%%%%%%%%%%%%%%%%%%%%%%%%%%%%%%%%%%%%%%%%%%%%%%%%%%%%%%%%%%%%%%%%%%%%%%%%%%%%

\end{document}


% This document was modified from the file originally made available by
% Pat Langley and Andrea Danyluk for ICML-2K. This version was created
% by Iain Murray in 2018, and modified by Alexandre Bouchard in
% 2019 and 2021 and by Csaba Szepesvari, Gang Niu and Sivan Sabato in 2022.
% Modified again in 2023 and 2024 by Sivan Sabato and Jonathan Scarlett.
% Previous contributors include Dan Roy, Lise Getoor and Tobias
% Scheffer, which was slightly modified from the 2010 version by
% Thorsten Joachims & Johannes Fuernkranz, slightly modified from the
% 2009 version by Kiri Wagstaff and Sam Roweis's 2008 version, which is
% slightly modified from Prasad Tadepalli's 2007 version which is a
% lightly changed version of the previous year's version by Andrew
% Moore, which was in turn edited from those of Kristian Kersting and
% Codrina Lauth. Alex Smola contributed to the algorithmic style files.
