%%%% ijcai25.tex

\typeout{IJCAI--25 Instructions for Authors}

% These are the instructions for authors for IJCAI-25.

\documentclass{article}
\pdfpagewidth=8.5in
\pdfpageheight=11in

% The file ijcai25.sty is a copy from ijcai22.sty
% The file ijcai22.sty is NOT the same as previous years'
\usepackage{ijcai-template/ijcai25}


% Use the postscript times font!
\usepackage{times}
\usepackage{soul}
\usepackage{url}
\usepackage[hidelinks]{hyperref}
\usepackage[utf8]{inputenc}
\usepackage[small]{caption}
\usepackage{graphicx}
\usepackage{amsmath}
\usepackage{amsthm}
\usepackage{booktabs}
\usepackage{algorithm}
\usepackage{algorithmic}
\usepackage[switch]{lineno}

\usepackage{adjustbox}
\usepackage{booktabs}
% Comment out this line in the camera-ready submission
%\linenumbers

\urlstyle{same}

% the following package is optional:
% \usepackage{latexsym}

% packages added by me
\usepackage{comment}
\usepackage{xcolor}
\usepackage{amsmath}
\usepackage{amssymb}

% See https://www.overleaf.com/learn/latex/theorems_and_proofs
% for a nice explanation of how to define new theorems, but keep
% in mind that the amsthm package is already included in this
% template and that you must *not* alter the styling.
\newtheorem{example}{Example}
\newtheorem{theorem}{Theorem}

% Following comment is from ijcai97-submit.tex:
% The preparation of these files was supported by Schlumberger Palo Alto
% Research, AT\&T Bell Laboratories, and Morgan Kaufmann Publishers.
% Shirley Jowell, of Morgan Kaufmann Publishers, and Peter F.
% Patel-Schneider, of AT\&T Bell Laboratories collaborated on their
% preparation.

% These instructions can be modified and used in other conferences as long
% as credit to the authors and supporting agencies is retained, this notice
% is not changed, and further modification or reuse is not restricted.
% Neither Shirley Jowell nor Peter F. Patel-Schneider can be listed as
% contacts for providing assistance without their prior permission.

% To use for other conferences, change references to files and the
% conference appropriate and use other authors, contacts, publishers, and
% organizations.
% Also change the deadline and address for returning papers and the length and
% page charge instructions.
% Put where the files are available in the appropriate places.


% PDF Info Is REQUIRED.

% Please leave this \pdfinfo block untouched both for the submission and
% Camera Ready Copy. Do not include Title and Author information in the pdfinfo section
\pdfinfo{
/TemplateVersion (IJCAI.2025.0)
}

%%%%%%%%%%%%%%%%%%%%%%%%%%%%%%%%%%%%%%%%%%%%%%%%%%%%%%%%%%%%%%%%%%%%%%%%

\title{Decentralised multi-agent coordination for real-time railway traffic management}

%%%%%%%%%%%%%%%%%%%%%%%%%%%%%%%%%%%%%%%%%%%%%%%%%%%%%%%%%%%%%%%%%%%%%%%%

\author{
Leo D'Amato$^{1,3}$
\and
Paola Pellegrini$^2$\and
Vito Trianni$^3$\\
\affiliations
$^1$Politecnico di Torino, DAUIN, Corso Castelfidardo, 34/d, Turin 10128, Italy.\\
$^2$Univ Gustave Eiffel, COSYS-ESTAS, F-59650 Villeneuve d’Ascq, France\\
$^3$Institute for Cognitive Sciences and Technologies, CNR, Via G. D. Romagnosi 18/A, 00196 Rome, Italy.\\
\emails
%\{first, second\}@example.com,
leo.damato@polito.it
%second@other.example.com,
%third@example.com
}


%%%%%%%%%%%%%%%%%%%%%%%%%%%%%%%%%%%%%%%%%%%%%%%%%%%%%%%%%%%%%%%%%%%%%%%%

\begin{document}

%%%%%%%%%%%%%%%%%%%%%%%%%%%%%%%%%%%%%%%%%%%%%%%%%%%%%%%%%%%%%%%%%%%%%%%%

\maketitle

%%%%%%%%%%%%%%%%%%%%%%%%%%%%%%%%%%%%%%%%%%%%%%%%%%%%%%%%%%%%%%%%%%%%%%%%

\begin{abstract}
    \begin{abstract}  
Test time scaling is currently one of the most active research areas that shows promise after training time scaling has reached its limits.
Deep-thinking (DT) models are a class of recurrent models that can perform easy-to-hard generalization by assigning more compute to harder test samples.
However, due to their inability to determine the complexity of a test sample, DT models have to use a large amount of computation for both easy and hard test samples.
Excessive test time computation is wasteful and can cause the ``overthinking'' problem where more test time computation leads to worse results.
In this paper, we introduce a test time training method for determining the optimal amount of computation needed for each sample during test time.
We also propose Conv-LiGRU, a novel recurrent architecture for efficient and robust visual reasoning. 
Extensive experiments demonstrate that Conv-LiGRU is more stable than DT, effectively mitigates the ``overthinking'' phenomenon, and achieves superior accuracy.
\end{abstract}  
\end{abstract}

%%%%%%%%%%%%%%%%%%%%%%%%%%%%%%%%%%%%%%%%%%%%%%%%%%%%%%%%%%%%%%%%%%%%%%%%

\section{Introduction}
\label{sec:introduction}
The business processes of organizations are experiencing ever-increasing complexity due to the large amount of data, high number of users, and high-tech devices involved \cite{martin2021pmopportunitieschallenges, beerepoot2023biggestbpmproblems}. This complexity may cause business processes to deviate from normal control flow due to unforeseen and disruptive anomalies \cite{adams2023proceddsriftdetection}. These control-flow anomalies manifest as unknown, skipped, and wrongly-ordered activities in the traces of event logs monitored from the execution of business processes \cite{ko2023adsystematicreview}. For the sake of clarity, let us consider an illustrative example of such anomalies. Figure \ref{FP_ANOMALIES} shows a so-called event log footprint, which captures the control flow relations of four activities of a hypothetical event log. In particular, this footprint captures the control-flow relations between activities \texttt{a}, \texttt{b}, \texttt{c} and \texttt{d}. These are the causal ($\rightarrow$) relation, concurrent ($\parallel$) relation, and other ($\#$) relations such as exclusivity or non-local dependency \cite{aalst2022pmhandbook}. In addition, on the right are six traces, of which five exhibit skipped, wrongly-ordered and unknown control-flow anomalies. For example, $\langle$\texttt{a b d}$\rangle$ has a skipped activity, which is \texttt{c}. Because of this skipped activity, the control-flow relation \texttt{b}$\,\#\,$\texttt{d} is violated, since \texttt{d} directly follows \texttt{b} in the anomalous trace.
\begin{figure}[!t]
\centering
\includegraphics[width=0.9\columnwidth]{images/FP_ANOMALIES.png}
\caption{An example event log footprint with six traces, of which five exhibit control-flow anomalies.}
\label{FP_ANOMALIES}
\end{figure}

\subsection{Control-flow anomaly detection}
Control-flow anomaly detection techniques aim to characterize the normal control flow from event logs and verify whether these deviations occur in new event logs \cite{ko2023adsystematicreview}. To develop control-flow anomaly detection techniques, \revision{process mining} has seen widespread adoption owing to process discovery and \revision{conformance checking}. On the one hand, process discovery is a set of algorithms that encode control-flow relations as a set of model elements and constraints according to a given modeling formalism \cite{aalst2022pmhandbook}; hereafter, we refer to the Petri net, a widespread modeling formalism. On the other hand, \revision{conformance checking} is an explainable set of algorithms that allows linking any deviations with the reference Petri net and providing the fitness measure, namely a measure of how much the Petri net fits the new event log \cite{aalst2022pmhandbook}. Many control-flow anomaly detection techniques based on \revision{conformance checking} (hereafter, \revision{conformance checking}-based techniques) use the fitness measure to determine whether an event log is anomalous \cite{bezerra2009pmad, bezerra2013adlogspais, myers2018icsadpm, pecchia2020applicationfailuresanalysispm}. 

The scientific literature also includes many \revision{conformance checking}-independent techniques for control-flow anomaly detection that combine specific types of trace encodings with machine/deep learning \cite{ko2023adsystematicreview, tavares2023pmtraceencoding}. Whereas these techniques are very effective, their explainability is challenging due to both the type of trace encoding employed and the machine/deep learning model used \cite{rawal2022trustworthyaiadvances,li2023explainablead}. Hence, in the following, we focus on the shortcomings of \revision{conformance checking}-based techniques to investigate whether it is possible to support the development of competitive control-flow anomaly detection techniques while maintaining the explainable nature of \revision{conformance checking}.
\begin{figure}[!t]
\centering
\includegraphics[width=\columnwidth]{images/HIGH_LEVEL_VIEW.png}
\caption{A high-level view of the proposed framework for combining \revision{process mining}-based feature extraction with dimensionality reduction for control-flow anomaly detection.}
\label{HIGH_LEVEL_VIEW}
\end{figure}

\subsection{Shortcomings of \revision{conformance checking}-based techniques}
Unfortunately, the detection effectiveness of \revision{conformance checking}-based techniques is affected by noisy data and low-quality Petri nets, which may be due to human errors in the modeling process or representational bias of process discovery algorithms \cite{bezerra2013adlogspais, pecchia2020applicationfailuresanalysispm, aalst2016pm}. Specifically, on the one hand, noisy data may introduce infrequent and deceptive control-flow relations that may result in inconsistent fitness measures, whereas, on the other hand, checking event logs against a low-quality Petri net could lead to an unreliable distribution of fitness measures. Nonetheless, such Petri nets can still be used as references to obtain insightful information for \revision{process mining}-based feature extraction, supporting the development of competitive and explainable \revision{conformance checking}-based techniques for control-flow anomaly detection despite the problems above. For example, a few works outline that token-based \revision{conformance checking} can be used for \revision{process mining}-based feature extraction to build tabular data and develop effective \revision{conformance checking}-based techniques for control-flow anomaly detection \cite{singh2022lapmsh, debenedictis2023dtadiiot}. However, to the best of our knowledge, the scientific literature lacks a structured proposal for \revision{process mining}-based feature extraction using the state-of-the-art \revision{conformance checking} variant, namely alignment-based \revision{conformance checking}.

\subsection{Contributions}
We propose a novel \revision{process mining}-based feature extraction approach with alignment-based \revision{conformance checking}. This variant aligns the deviating control flow with a reference Petri net; the resulting alignment can be inspected to extract additional statistics such as the number of times a given activity caused mismatches \cite{aalst2022pmhandbook}. We integrate this approach into a flexible and explainable framework for developing techniques for control-flow anomaly detection. The framework combines \revision{process mining}-based feature extraction and dimensionality reduction to handle high-dimensional feature sets, achieve detection effectiveness, and support explainability. Notably, in addition to our proposed \revision{process mining}-based feature extraction approach, the framework allows employing other approaches, enabling a fair comparison of multiple \revision{conformance checking}-based and \revision{conformance checking}-independent techniques for control-flow anomaly detection. Figure \ref{HIGH_LEVEL_VIEW} shows a high-level view of the framework. Business processes are monitored, and event logs obtained from the database of information systems. Subsequently, \revision{process mining}-based feature extraction is applied to these event logs and tabular data input to dimensionality reduction to identify control-flow anomalies. We apply several \revision{conformance checking}-based and \revision{conformance checking}-independent framework techniques to publicly available datasets, simulated data of a case study from railways, and real-world data of a case study from healthcare. We show that the framework techniques implementing our approach outperform the baseline \revision{conformance checking}-based techniques while maintaining the explainable nature of \revision{conformance checking}.

In summary, the contributions of this paper are as follows.
\begin{itemize}
    \item{
        A novel \revision{process mining}-based feature extraction approach to support the development of competitive and explainable \revision{conformance checking}-based techniques for control-flow anomaly detection.
    }
    \item{
        A flexible and explainable framework for developing techniques for control-flow anomaly detection using \revision{process mining}-based feature extraction and dimensionality reduction.
    }
    \item{
        Application to synthetic and real-world datasets of several \revision{conformance checking}-based and \revision{conformance checking}-independent framework techniques, evaluating their detection effectiveness and explainability.
    }
\end{itemize}

The rest of the paper is organized as follows.
\begin{itemize}
    \item Section \ref{sec:related_work} reviews the existing techniques for control-flow anomaly detection, categorizing them into \revision{conformance checking}-based and \revision{conformance checking}-independent techniques.
    \item Section \ref{sec:abccfe} provides the preliminaries of \revision{process mining} to establish the notation used throughout the paper, and delves into the details of the proposed \revision{process mining}-based feature extraction approach with alignment-based \revision{conformance checking}.
    \item Section \ref{sec:framework} describes the framework for developing \revision{conformance checking}-based and \revision{conformance checking}-independent techniques for control-flow anomaly detection that combine \revision{process mining}-based feature extraction and dimensionality reduction.
    \item Section \ref{sec:evaluation} presents the experiments conducted with multiple framework and baseline techniques using data from publicly available datasets and case studies.
    \item Section \ref{sec:conclusions} draws the conclusions and presents future work.
\end{itemize}

%%%%%%%%%%%%%%%%%%%%%%%%%%%%%%%%%%%%%%%%%%%%%%%%%%%%%%%%%%%%%%%%%%%%%%%%

\section{Viewer-provider two-sided systems}

This section models the dynamics of viewer and provider populations on a recommendation platform. 
Specifically, we consider sub-group dynamics where viewers and providers are categorized into $K$ and $L$ subgroups\footnote{We can consider a ``subgroup'' of size 1. In such cases, the viewer ``population'' corresponds to the time spent by an individual viewer, while the provider ``population'' can be the amount of content produced by an individual provider.
}. Then, we model the populations, recommendation policy, payoffs, and social welfare as follows.

\begin{enumerate}[leftmargin=12pt]
    \item (Viewer/provider population)  
    Let $\lambda_{k} \in \mathbb{R}_{\geq 0}$ be the population of the viewer group $k \in [K]$ and $\lambda_{l}$ be that of the provider group $l \in [L]$. Also let $\blambda := (\lambda_{1}, \lambda_{2}, \cdots, \lambda_{K},
    \lambda_{1}, \lambda_{2}, \cdots, \lambda_{L})$ be the joint population vector of viewers and providers.
    \item (Platform's recommendation policy) 
    The platform matches each viewer group $k$ to a provider group $l$ with a recommendation policy denoted by a $K$-by-$L$ matrix $\bpi$. Specifically, its $(k,l)$-th element $\pi_{k,l}$ represents the probability of allocating the provider group $l$ to the viewer group $k$. 
    Thus $\sum_{l=1}^L \pi_{k,l} = 1, \forall k \in [K]$. For example, the uniform random policy, which assigns equal exposure probability across all provider groups is represented as given by $\bpi=\frac{1}{L}\1_{L\times K}$.
    \item (Viewer/provider payoffs) After viewer and provider groups are matched by the policy $\bpi$, their perceived payoffs can be quantified by the following metrics:
    \begin{align}\label{eq:user_satisfaction}
    \text{Viewer Satisfaction: \quad } & s_k = \textstyle \sum_{l=1}^L \pi_{k,l} q_{k,l} \,  , \\\label{eq:content_exposure}
    \text{Provider Exposure: \quad} & e_l = \textstyle\sum_{k=1}^K \pi_{k,l}\lambda_k,
    \end{align}
    where $q_{k,l}$ is the (expected) utility that viewers $k$ receive from the provider groups $l$. Eqs.~\eqref{eq:user_satisfaction} and~\eqref{eq:content_exposure} define viewer satisfaction as determined by the total utility they receive from recommendations, while providers care about the total amount of exposure they receive by recommendation. This model is prevalent is prior works including \citep{singh2018fairness, mladenov2020optimizing}.
    \item (Social welfare) Finally, we consider the following total viewer welfare as the global metric of the platform:
    \begin{align*}
        R(\bpi; \blambda) := \textstyle\sum_{k=1}^{K} \lambda_{k} s_k
    \end{align*}
    Note that here we consider the sum of viewer-side satisfaction as the social welfare, a formulation prevalent in related works~\citep{mladenov2020optimizing, huttenlocher2023matching}.
    The sum of content-side exposure simplifies to the total size of the viewer population.
\end{enumerate}

\subsection{Interaction dynamics and ``population effects''}\label{sec:dynamic_formulation}

We have so far seen a typical formulation in two-sided platforms. However, a key limitation of such formulation is to ignore potential non-stationarity in the viewer and provider populations, which is common in many real-world two-sided systems~\citep{boutilier2023modeling,  deffayet2024sardine}. 

First, consider the impact of provider population growth on the utility experience by viewers, which we call \textit{``population effects''}.
An increase in provider population naturally leads to more high-quality content. 
For example, consider a two-stage recommendation policy, where our higher-level policy $\bpi$ decides the matching between viewer and provider groups, and a second-stage policy selects individual providers among the selected group. 
Any reasonable second stage policy should be able to select a better provider from a larger provider pool~\citep{su2023value, evnine2024achieving}. 
To model such ``population effects'', we introduce the following utility decomposition:
\begin{align}
    q_{k,l} = b_{k,l} + f_{k,l}(\lambda_{l}) \label{eq:reward_decomposition}
\end{align}
where $b_{k,l}$ is the \textit{base} utility, which may indicates the matching between the preference of viewer and provider groups (e.g., this viewer group likes sports articles). In contrast, $f_{k,l}(\cdot)$ represents the quality of the provider which improves as the provider population increases. $f_{k,l}$ might be heterogeneous among different viewer and provider groups because quality might be multi-dimensional (e.g., visuals, intellects, novelty), viewers may have different preferences, and providers may have different abilities. 
We take $f_{k,l}$ to be a monotonically increasing function.

Next, consider the impact of viewer and provider payoffs on the population.
The number of viewers that a platform can maintain is related to the level of satisfaction, similarly the number of providers is related to the exposure.
We assume that viewer and provider subgroups have 
some \textit{``reference''} population $\bar{\lambda}_{k}(s_{k})$ and $\bar{\lambda}_{l}(e_{l})$ given the level of viewer satisfaction $s_k$ and provider exposure $e_l$. We assume that $\bar{\lambda}$ is a monotonically increasing function, so higher viewer satisfaction and provider exposure result in increased populations. 
Based on this, we model the viewer and provider population dynamics as:
\begin{align}
    \text{Viewer: \,}  \lambda_{t+1,k} = (1 - \eta_k) \lambda_{t,k} + \eta_k \bar{\lambda}_{k}(s_{t,k}), \label{eq:user_dynamics} \\
    \text{Content: \,}  \lambda_{t+1,l} = (1 - \eta_l) \lambda_{t,l} + \eta_l \bar{\lambda}_{l}(e_{t,l}), \label{eq:content_dynamics}
\end{align}
where $\eta \in [0, 1]$ are the \textit{reactiveness} hyperparams, determining how fast the population changes. Note that similar models are widely adopted in performative predictions~\citep{perdomo2020performative, brown2022performative}. 
We thus have that the viewer satisfaction $s_k$ depends on the provider population via ``population effects'' $f_{k,l}$, while the provider exposure directly depends on the viewer population.
The two-sided platform has complex dynamics between viewers and providers. 
Our goal will be to consider long-term objectives under such co-evolving and two-sided dynamics.

\subsection{Game-theoretic interpretation}\label{sec:game_formulation}

Next, we provide a further justification of and insight into the dynamics model by introducing a game-theoretic formulation that is equivalent to Eqs. \eqref{eq:user_dynamics} and \eqref{eq:content_dynamics}.

Consider a $(K+L)$-player game involving $K$ viewer groups and $L$ provider groups. Each viewer group selects a pure strategy $\lambda_k \in \RR_{\geq 0}$, and each provider group chooses a pure strategy $\lambda_l \in \RR_{\geq 0}$. The utility functions for the viewer and provider groups, denoted by $\{u_k\}_{k=1}^K$ and $\{v_l\}_{l=1}^L$ are defined as follows:
\begin{align}\label{eq:util_user}
    & u_k(\blambda)= \lambda_k \cdot \bar{\lambda}_k \left(\sum_{l=1}^L \pi_{k,l}\left(b_{k,l}+f_{k,l}(\lambda_l)\right)\right)-\frac{\lambda_k^2}{2}, \\ \label{eq:util_creator}
    & v_l(\blambda)= \lambda_l\cdot \bar{\lambda}_l \left(\textstyle\sum_{k=1}^K \pi_{k,l}\lambda_k\right)-\frac{\lambda_l^2}{2},
\end{align}
We denote this game as $\G(\bpi, B, f, \bar{\lambda})$, where $B$ is a $K$-by-$L$ matrix whose $(k,l)$-element is $b_{k,l}$. Proposition \ref{prop:dynamics_equivalence} establishes a connection between the game instance $\G$ and the 
formulation presented in Section \ref{sec:dynamic_formulation}.

\begin{proposition}\label{prop:dynamics_equivalence}
    If all players in $\G$ apply gradient ascent to optimize their utility functions with learning rates $\{\eta_k\}_{k=1}^K$ and $\{\eta_l\}_{l=1}^L$, the resulting joint strategy evolving dynamics are exactly given by Eqs.~\eqref{eq:user_dynamics} and \eqref{eq:content_dynamics}.
\end{proposition}

Through Proposition \ref{prop:dynamics_equivalence}, our game-theoretic formulation provides a first-principles perspective for understanding the dynamical formulation in Eqs.~\eqref{eq:user_dynamics} and \eqref{eq:content_dynamics}.\footnote{The game $\G$ resembles the Cournot Duopoly competition \cite{cournot1838recherches}. When $K = L = 1$ and $\bar{\lambda}(\mu) = a - b\mu$ and $\bar{\mu}(\lambda) = a - b\lambda$ for some positive constants $a$ and $b$, the game $\G$ corresponds exactly to the Cournot Duopoly model. The key distinction in ours is that $\bar{\mu}$ and $\bar{\lambda}$ are generic increasing functions.} 
That is, 
we can interpret $\bar{\lambda}(\cdot)$ as the marginal gain from increasing the size of a viewer or provider group by one unit. Consequently, the first terms $\lambda_k \cdot \bar{\lambda}_k(\cdot)$ and $\lambda_l \cdot \bar{\lambda}_l(\cdot)$ represent the collective payoffs for viewer and provider groups of sizes $\lambda_k$ and $\lambda_l$. 
The quadratic terms $-\frac{\lambda_k^2}{2}$ and $-\frac{\lambda_l^2}{2}$ capture the congestion costs associated with maintaining larger populations (e.g., if a provider group becomes too large, providers within the group may face intensified competition and thus reduce their productivity due to diminished marginal gains). This suggests that Eqs.~\eqref{eq:user_dynamics} and \eqref{eq:content_dynamics} are quite reasonable formulation to capture real-world interactions.

%%%%%%%%%%%%%%%%%%%%%%%%%%%%%%%%%%%%%%%%%%%%%%%%%%%%%%%%%%%%%%%%%%%%%%%%

% insert this in supplementary materials
% \section{Centralised solution to the coordination problem}\label{sec:centralised_appr}

In this section, we aim at determining if a given problem instance $(\mathcal{G}_I, \mathcal{G}_C)$ admits a solution and, in case it does, we want to compute all possible solutions while identifying the optimal ones.
%\textcolor{blue}{forse conviene specificare che non è possibile farlo in maniera decentralizzata ? }
To accomplish this, we can reformulate the problem as an integer linear programming (ILP) task and utilize widely available software such as CPLEX \footnote{https://www.ibm.com/it-it/products/ilog-cplex-optimization-studio} as solvers.
%
%Given a problem instance $(\mathcal{G}_I, \mathcal{G}_C)$, we would like to know whether it admits a solution or not and, in case it does, what are all the possible solutions and which ones are optimal. A method to achieve this goal is by reformulating it as an integer linear programming (ILP) problem and solve it by means of standard software like CPLEX. 
%
%The output of our decentralised algorithm at convergence is a solution to the problem instance being solved. A way to assess the performance of the our algorithm is to compare the value of the solution found with the value of the optimal solution to the problem being solved. A method to compute the optimal solution of a problem instance is by reformulating it as an integer linear programming (ILP) problem and solve it by means of tools like CPLEX. 
%
Such a centralised approach provides a reference for the evaluation of the decentralised solutions in our experiments, as discussed in the main text. 

%The output of our decentralised algorithm at convergence is a solution to the problem instance being solved. In order to assess the performance of the consensus procedure we must access all the solutions of the problem instance. One way to compute the entire set of solutions of a problem instance is by reformulating the rtRTMP as an integer linear programming (ILP) problem and solve it by means of a solver like CPLEX. Such an approach is no longer decentralised and serves as a baseline for our experiments in section \ref{sec:exps}. 
To define the ILP forumulation, we first introduce the mapping $\alpha : \mathcal{V}_C \mapsto \{1, \ldots, n\}$ associating to each path $d$ the index of the agent it belongs to. Then, we define the binary decision variables $y_d$ as follows:
%
\begin{equation}
\label{eq:binary}
y_d = \left\{
                \begin{array}{ll}
                  1 \quad \mbox{if $v_{\alpha(d)}=d$}     \\
                  0	\quad \mbox{otherwise}\\
                \end{array}
                 \right. \quad \forall d \in \mathcal{V}_C
\end{equation}
where the notation $v_{\alpha(d)}=d$ means that the path $d$ has been assigned to the variable $v_{\alpha(d)}$ controlled by the agent $a_{\alpha(d)}$. 
Recall also that each node $d \in \mathcal{V}_C$ has a value $u(d)$. 
Then, the centralized optimization problem consists in finding the appropriate complete assignment of paths to maximise the following objective function: 
%
\begin{equation}\label{eq:objective}
\max\sum_{d \in \mathcal{V}_C} u(d) \ y_d 
\end{equation}
provided that the following constraints are satisfied:
\begin{align}
\sum_{d \in D_i} y_d = 1,\quad      &\forall i = 1 \ldots n  \label{eq:eq3} \\[10pt] 
\sum_{d^{\prime} \in D_j : (d, d^{\prime}) \in \mathcal{E}_C} y_{d^{\prime}}  \geq y_d \quad &\forall d \in \mathcal{V}_C, \nonumber \\[-15pt] 
                                                                                       &\forall j=1 \ldots n : A_j  \in \mathcal{N}_{\alpha(d)}, j \neq \alpha(d) \label{eq:eq4}
\end{align}
Constraints~(\ref{eq:eq3}) ensure that exactly one path per train is selected (each agent can only assign one value to the variable it controls). Constraints~(\ref{eq:eq4}) state that, if path $d$ is selected and it is associated to agent $\alpha(d)$, then each neighbouring agent in $\mathcal{N}_{\alpha(d)}$ must select a path that is compatible with $d$.

By exploiting CPLEX as solver for this ILP formulation of the agent coordination in the dec-rtRTMP, we are able to find all possible solutions to a given problem instance, and among them, to identify the optimal ones as well.
%and among these to also identify the optimal solution.

%%%%%%%%%%%%%%%%%%%%%%%%%%%%%%%%%%%%%%%%%%%%%%%%%%%%%%%%%%%%%%%%%%%%%%%%

\begin{figure*}[t]
    \hfill
    \begin{minipage}{0.2425\textwidth}
        \includegraphics[trim=0.5cm 0.5cm 0.5cm 0.5cm,height=4cm]{figures/consensus/ar.pdf}
    \end{minipage}
    \hfill
    \begin{minipage}{0.2425\textwidth}
        \includegraphics[trim=0.5cm 0.5cm 0.5cm 0.5cm,height=4cm]{figures/consensus/oscc.pdf}
    \end{minipage}
    \hfill
    \begin{minipage}{0.2425\textwidth}
        \includegraphics[trim=0.5cm 0.5cm 0.5cm 0.5cm,height=4cm]{figures/consensus/lta.pdf}
    \end{minipage}
    \hfill
    \begin{minipage}{0.2425\textwidth}
        \includegraphics[trim=0.5cm 0.5cm 0cm 0.5cm,height=4cm]{figures/consensus/mq.pdf}
    \end{minipage}
    \hfill
    \caption{\textbf{Activations consensus for different \emph{novel tasks}.} Activations consensus between two \emph{support tasks} is defined as the percentage of their prototypes corresponding to the same label activated by the two tasks. Fine-grained tasks, \ie, AR, OSCC and LTA, have higher average consensus. On the contrary, MQ has lower average consensus.}
    \label{fig:consensus}
    \vspace{-0.3cm}
\end{figure*}

%%%%%%%%%%%%%%%%%%%%%%%%%%%%%%%%%%%%%%%%%%%%%%%%%%%%%%%%%%%%%%%%%%%%%%%%

\section{Experiments}
\label{sec:experiments}
The experiments are designed to address two key research questions.
First, \textbf{RQ1} evaluates whether the average $L_2$-norm of the counterfactual perturbation vectors ($\overline{||\perturb||}$) decreases as the model overfits the data, thereby providing further empirical validation for our hypothesis.
Second, \textbf{RQ2} evaluates the ability of the proposed counterfactual regularized loss, as defined in (\ref{eq:regularized_loss2}), to mitigate overfitting when compared to existing regularization techniques.

% The experiments are designed to address three key research questions. First, \textbf{RQ1} investigates whether the mean perturbation vector norm decreases as the model overfits the data, aiming to further validate our intuition. Second, \textbf{RQ2} explores whether the mean perturbation vector norm can be effectively leveraged as a regularization term during training, offering insights into its potential role in mitigating overfitting. Finally, \textbf{RQ3} examines whether our counterfactual regularizer enables the model to achieve superior performance compared to existing regularization methods, thus highlighting its practical advantage.

\subsection{Experimental Setup}
\textbf{\textit{Datasets, Models, and Tasks.}}
The experiments are conducted on three datasets: \textit{Water Potability}~\cite{kadiwal2020waterpotability}, \textit{Phomene}~\cite{phomene}, and \textit{CIFAR-10}~\cite{krizhevsky2009learning}. For \textit{Water Potability} and \textit{Phomene}, we randomly select $80\%$ of the samples for the training set, and the remaining $20\%$ for the test set, \textit{CIFAR-10} comes already split. Furthermore, we consider the following models: Logistic Regression, Multi-Layer Perceptron (MLP) with 100 and 30 neurons on each hidden layer, and PreactResNet-18~\cite{he2016cvecvv} as a Convolutional Neural Network (CNN) architecture.
We focus on binary classification tasks and leave the extension to multiclass scenarios for future work. However, for datasets that are inherently multiclass, we transform the problem into a binary classification task by selecting two classes, aligning with our assumption.

\smallskip
\noindent\textbf{\textit{Evaluation Measures.}} To characterize the degree of overfitting, we use the test loss, as it serves as a reliable indicator of the model's generalization capability to unseen data. Additionally, we evaluate the predictive performance of each model using the test accuracy.

\smallskip
\noindent\textbf{\textit{Baselines.}} We compare CF-Reg with the following regularization techniques: L1 (``Lasso''), L2 (``Ridge''), and Dropout.

\smallskip
\noindent\textbf{\textit{Configurations.}}
For each model, we adopt specific configurations as follows.
\begin{itemize}
\item \textit{Logistic Regression:} To induce overfitting in the model, we artificially increase the dimensionality of the data beyond the number of training samples by applying a polynomial feature expansion. This approach ensures that the model has enough capacity to overfit the training data, allowing us to analyze the impact of our counterfactual regularizer. The degree of the polynomial is chosen as the smallest degree that makes the number of features greater than the number of data.
\item \textit{Neural Networks (MLP and CNN):} To take advantage of the closed-form solution for computing the optimal perturbation vector as defined in (\ref{eq:opt-delta}), we use a local linear approximation of the neural network models. Hence, given an instance $\inst_i$, we consider the (optimal) counterfactual not with respect to $\model$ but with respect to:
\begin{equation}
\label{eq:taylor}
    \model^{lin}(\inst) = \model(\inst_i) + \nabla_{\inst}\model(\inst_i)(\inst - \inst_i),
\end{equation}
where $\model^{lin}$ represents the first-order Taylor approximation of $\model$ at $\inst_i$.
Note that this step is unnecessary for Logistic Regression, as it is inherently a linear model.
\end{itemize}

\smallskip
\noindent \textbf{\textit{Implementation Details.}} We run all experiments on a machine equipped with an AMD Ryzen 9 7900 12-Core Processor and an NVIDIA GeForce RTX 4090 GPU. Our implementation is based on the PyTorch Lightning framework. We use stochastic gradient descent as the optimizer with a learning rate of $\eta = 0.001$ and no weight decay. We use a batch size of $128$. The training and test steps are conducted for $6000$ epochs on the \textit{Water Potability} and \textit{Phoneme} datasets, while for the \textit{CIFAR-10} dataset, they are performed for $200$ epochs.
Finally, the contribution $w_i^{\varepsilon}$ of each training point $\inst_i$ is uniformly set as $w_i^{\varepsilon} = 1~\forall i\in \{1,\ldots,m\}$.

The source code implementation for our experiments is available at the following GitHub repository: \url{https://anonymous.4open.science/r/COCE-80B4/README.md} 

\subsection{RQ1: Counterfactual Perturbation vs. Overfitting}
To address \textbf{RQ1}, we analyze the relationship between the test loss and the average $L_2$-norm of the counterfactual perturbation vectors ($\overline{||\perturb||}$) over training epochs.

In particular, Figure~\ref{fig:delta_loss_epochs} depicts the evolution of $\overline{||\perturb||}$ alongside the test loss for an MLP trained \textit{without} regularization on the \textit{Water Potability} dataset. 
\begin{figure}[ht]
    \centering
    \includegraphics[width=0.85\linewidth]{img/delta_loss_epochs.png}
    \caption{The average counterfactual perturbation vector $\overline{||\perturb||}$ (left $y$-axis) and the cross-entropy test loss (right $y$-axis) over training epochs ($x$-axis) for an MLP trained on the \textit{Water Potability} dataset \textit{without} regularization.}
    \label{fig:delta_loss_epochs}
\end{figure}

The plot shows a clear trend as the model starts to overfit the data (evidenced by an increase in test loss). 
Notably, $\overline{||\perturb||}$ begins to decrease, which aligns with the hypothesis that the average distance to the optimal counterfactual example gets smaller as the model's decision boundary becomes increasingly adherent to the training data.

It is worth noting that this trend is heavily influenced by the choice of the counterfactual generator model. In particular, the relationship between $\overline{||\perturb||}$ and the degree of overfitting may become even more pronounced when leveraging more accurate counterfactual generators. However, these models often come at the cost of higher computational complexity, and their exploration is left to future work.

Nonetheless, we expect that $\overline{||\perturb||}$ will eventually stabilize at a plateau, as the average $L_2$-norm of the optimal counterfactual perturbations cannot vanish to zero.

% Additionally, the choice of employing the score-based counterfactual explanation framework to generate counterfactuals was driven to promote computational efficiency.

% Future enhancements to the framework may involve adopting models capable of generating more precise counterfactuals. While such approaches may yield to performance improvements, they are likely to come at the cost of increased computational complexity.


\subsection{RQ2: Counterfactual Regularization Performance}
To answer \textbf{RQ2}, we evaluate the effectiveness of the proposed counterfactual regularization (CF-Reg) by comparing its performance against existing baselines: unregularized training loss (No-Reg), L1 regularization (L1-Reg), L2 regularization (L2-Reg), and Dropout.
Specifically, for each model and dataset combination, Table~\ref{tab:regularization_comparison} presents the mean value and standard deviation of test accuracy achieved by each method across 5 random initialization. 

The table illustrates that our regularization technique consistently delivers better results than existing methods across all evaluated scenarios, except for one case -- i.e., Logistic Regression on the \textit{Phomene} dataset. 
However, this setting exhibits an unusual pattern, as the highest model accuracy is achieved without any regularization. Even in this case, CF-Reg still surpasses other regularization baselines.

From the results above, we derive the following key insights. First, CF-Reg proves to be effective across various model types, ranging from simple linear models (Logistic Regression) to deep architectures like MLPs and CNNs, and across diverse datasets, including both tabular and image data. 
Second, CF-Reg's strong performance on the \textit{Water} dataset with Logistic Regression suggests that its benefits may be more pronounced when applied to simpler models. However, the unexpected outcome on the \textit{Phoneme} dataset calls for further investigation into this phenomenon.


\begin{table*}[h!]
    \centering
    \caption{Mean value and standard deviation of test accuracy across 5 random initializations for different model, dataset, and regularization method. The best results are highlighted in \textbf{bold}.}
    \label{tab:regularization_comparison}
    \begin{tabular}{|c|c|c|c|c|c|c|}
        \hline
        \textbf{Model} & \textbf{Dataset} & \textbf{No-Reg} & \textbf{L1-Reg} & \textbf{L2-Reg} & \textbf{Dropout} & \textbf{CF-Reg (ours)} \\ \hline
        Logistic Regression   & \textit{Water}   & $0.6595 \pm 0.0038$   & $0.6729 \pm 0.0056$   & $0.6756 \pm 0.0046$  & N/A    & $\mathbf{0.6918 \pm 0.0036}$                     \\ \hline
        MLP   & \textit{Water}   & $0.6756 \pm 0.0042$   & $0.6790 \pm 0.0058$   & $0.6790 \pm 0.0023$  & $0.6750 \pm 0.0036$    & $\mathbf{0.6802 \pm 0.0046}$                    \\ \hline
%        MLP   & \textit{Adult}   & $0.8404 \pm 0.0010$   & $\mathbf{0.8495 \pm 0.0007}$   & $0.8489 \pm 0.0014$  & $\mathbf{0.8495 \pm 0.0016}$     & $0.8449 \pm 0.0019$                    \\ \hline
        Logistic Regression   & \textit{Phomene}   & $\mathbf{0.8148 \pm 0.0020}$   & $0.8041 \pm 0.0028$   & $0.7835 \pm 0.0176$  & N/A    & $0.8098 \pm 0.0055$                     \\ \hline
        MLP   & \textit{Phomene}   & $0.8677 \pm 0.0033$   & $0.8374 \pm 0.0080$   & $0.8673 \pm 0.0045$  & $0.8672 \pm 0.0042$     & $\mathbf{0.8718 \pm 0.0040}$                    \\ \hline
        CNN   & \textit{CIFAR-10} & $0.6670 \pm 0.0233$   & $0.6229 \pm 0.0850$   & $0.7348 \pm 0.0365$   & N/A    & $\mathbf{0.7427 \pm 0.0571}$                     \\ \hline
    \end{tabular}
\end{table*}

\begin{table*}[htb!]
    \centering
    \caption{Hyperparameter configurations utilized for the generation of Table \ref{tab:regularization_comparison}. For our regularization the hyperparameters are reported as $\mathbf{\alpha/\beta}$.}
    \label{tab:performance_parameters}
    \begin{tabular}{|c|c|c|c|c|c|c|}
        \hline
        \textbf{Model} & \textbf{Dataset} & \textbf{No-Reg} & \textbf{L1-Reg} & \textbf{L2-Reg} & \textbf{Dropout} & \textbf{CF-Reg (ours)} \\ \hline
        Logistic Regression   & \textit{Water}   & N/A   & $0.0093$   & $0.6927$  & N/A    & $0.3791/1.0355$                     \\ \hline
        MLP   & \textit{Water}   & N/A   & $0.0007$   & $0.0022$  & $0.0002$    & $0.2567/1.9775$                    \\ \hline
        Logistic Regression   &
        \textit{Phomene}   & N/A   & $0.0097$   & $0.7979$  & N/A    & $0.0571/1.8516$                     \\ \hline
        MLP   & \textit{Phomene}   & N/A   & $0.0007$   & $4.24\cdot10^{-5}$  & $0.0015$    & $0.0516/2.2700$                    \\ \hline
       % MLP   & \textit{Adult}   & N/A   & $0.0018$   & $0.0018$  & $0.0601$     & $0.0764/2.2068$                    \\ \hline
        CNN   & \textit{CIFAR-10} & N/A   & $0.0050$   & $0.0864$ & N/A    & $0.3018/
        2.1502$                     \\ \hline
    \end{tabular}
\end{table*}

\begin{table*}[htb!]
    \centering
    \caption{Mean value and standard deviation of training time across 5 different runs. The reported time (in seconds) corresponds to the generation of each entry in Table \ref{tab:regularization_comparison}. Times are }
    \label{tab:times}
    \begin{tabular}{|c|c|c|c|c|c|c|}
        \hline
        \textbf{Model} & \textbf{Dataset} & \textbf{No-Reg} & \textbf{L1-Reg} & \textbf{L2-Reg} & \textbf{Dropout} & \textbf{CF-Reg (ours)} \\ \hline
        Logistic Regression   & \textit{Water}   & $222.98 \pm 1.07$   & $239.94 \pm 2.59$   & $241.60 \pm 1.88$  & N/A    & $251.50 \pm 1.93$                     \\ \hline
        MLP   & \textit{Water}   & $225.71 \pm 3.85$   & $250.13 \pm 4.44$   & $255.78 \pm 2.38$  & $237.83 \pm 3.45$    & $266.48 \pm 3.46$                    \\ \hline
        Logistic Regression   & \textit{Phomene}   & $266.39 \pm 0.82$ & $367.52 \pm 6.85$   & $361.69 \pm 4.04$  & N/A   & $310.48 \pm 0.76$                    \\ \hline
        MLP   &
        \textit{Phomene} & $335.62 \pm 1.77$   & $390.86 \pm 2.11$   & $393.96 \pm 1.95$ & $363.51 \pm 5.07$    & $403.14 \pm 1.92$                     \\ \hline
       % MLP   & \textit{Adult}   & N/A   & $0.0018$   & $0.0018$  & $0.0601$     & $0.0764/2.2068$                    \\ \hline
        CNN   & \textit{CIFAR-10} & $370.09 \pm 0.18$   & $395.71 \pm 0.55$   & $401.38 \pm 0.16$ & N/A    & $1287.8 \pm 0.26$                     \\ \hline
    \end{tabular}
\end{table*}

\subsection{Feasibility of our Method}
A crucial requirement for any regularization technique is that it should impose minimal impact on the overall training process.
In this respect, CF-Reg introduces an overhead that depends on the time required to find the optimal counterfactual example for each training instance. 
As such, the more sophisticated the counterfactual generator model probed during training the higher would be the time required. However, a more advanced counterfactual generator might provide a more effective regularization. We discuss this trade-off in more details in Section~\ref{sec:discussion}.

Table~\ref{tab:times} presents the average training time ($\pm$ standard deviation) for each model and dataset combination listed in Table~\ref{tab:regularization_comparison}.
We can observe that the higher accuracy achieved by CF-Reg using the score-based counterfactual generator comes with only minimal overhead. However, when applied to deep neural networks with many hidden layers, such as \textit{PreactResNet-18}, the forward derivative computation required for the linearization of the network introduces a more noticeable computational cost, explaining the longer training times in the table.

\subsection{Hyperparameter Sensitivity Analysis}
The proposed counterfactual regularization technique relies on two key hyperparameters: $\alpha$ and $\beta$. The former is intrinsic to the loss formulation defined in (\ref{eq:cf-train}), while the latter is closely tied to the choice of the score-based counterfactual explanation method used.

Figure~\ref{fig:test_alpha_beta} illustrates how the test accuracy of an MLP trained on the \textit{Water Potability} dataset changes for different combinations of $\alpha$ and $\beta$.

\begin{figure}[ht]
    \centering
    \includegraphics[width=0.85\linewidth]{img/test_acc_alpha_beta.png}
    \caption{The test accuracy of an MLP trained on the \textit{Water Potability} dataset, evaluated while varying the weight of our counterfactual regularizer ($\alpha$) for different values of $\beta$.}
    \label{fig:test_alpha_beta}
\end{figure}

We observe that, for a fixed $\beta$, increasing the weight of our counterfactual regularizer ($\alpha$) can slightly improve test accuracy until a sudden drop is noticed for $\alpha > 0.1$.
This behavior was expected, as the impact of our penalty, like any regularization term, can be disruptive if not properly controlled.

Moreover, this finding further demonstrates that our regularization method, CF-Reg, is inherently data-driven. Therefore, it requires specific fine-tuning based on the combination of the model and dataset at hand.

%%%%%%%%%%%%%%%%%%%%%%%%%%%%%%%%%%%%%%%%%%%%%%%%%%%%%%%%%%%%%%%%%%%%%%%%

\vspace{-0.2cm}
\section{Impact: Why Free Scientific Knowledge?}
\vspace{-0.1cm}

Historically, making knowledge widely available has driven transformative progress. Gutenberg’s printing press broke medieval monopolies on information, increasing literacy and contributing to the Renaissance and Scientific Revolution. In today's world, open source projects such as GNU/Linux and Wikipedia show that freely accessible and modifiable knowledge fosters innovation while ensuring creators are credited through copyleft licenses. These examples highlight a key idea: \textit{access to essential knowledge supports overall advancement.} 

This aligns with the arguments made by Prabhakaran et al. \cite{humanrightsbasedapproachresponsible}, who specifically highlight the \textbf{ human right to participate in scientific advancement} as enshrined in the Universal Declaration of Human Rights. They emphasize that this right underscores the importance of \textit{ equal access to the benefits of scientific progress for all}, a principle directly supported by our proposal for Knowledge Units. The UN Special Rapporteur on Cultural Rights further reinforces this, advocating for the expansion of copyright exceptions to broaden access to scientific knowledge as a crucial component of the right to science and culture \cite{scienceright}. 

However, current intellectual property regimes often create ``patently unfair" barriers to this knowledge, preventing innovation and access, especially in areas critical to human rights, as Hale compellingly argues \cite{patentlyunfair}. Finding a solution requires carefully balancing the imperative of open access with the legitimate rights of authors. As Austin and Ginsburg remind us, authors' rights are also human rights, necessitating robust protection \cite{authorhumanrights}. Shareable knowledge entities like Knowledge Units offer a potential mechanism to achieve this delicate balance in the scientific domain, enabling wider dissemination of research findings while respecting authors' fundamental rights.

\vspace{-0.2cm}
\subsection{Impact Across Sectors}

\textbf{Researchers:} Collaboration across different fields becomes easier when knowledge is shared openly. For instance, combining machine learning with biology or applying quantum principles to cryptography can lead to important breakthroughs. Removing copyright restrictions allows researchers to freely use data and methods, speeding up discoveries while respecting original contributions.

\textbf{Practitioners:} Professionals, especially in healthcare, benefit from immediate access to the latest research. Quick access to newer insights on the effectiveness of drugs, and alternative treatments speeds up adoption and awareness, potentially saving lives. Additionally, open knowledge helps developing countries gain access to health innovations.

\textbf{Education:} Education becomes more accessible when teachers use the latest research to create up-to-date curricula without prohibitive costs. Students can access high-quality research materials and use LM assistance to better understand complex topics, enhancing their learning experience and making high-quality education more accessible.

\textbf{Public Trust:} When information is transparent and accessible, the public can better understand and trust decision-making processes. Open access to government policies and industry practices allows people to review and verify information, helping to reduce misinformation. This transparency encourages critical thinking and builds trust in scientific and governmental institutions.

Overall, making scientific knowledge accessible supports global fairness. By viewing knowledge as a common resource rather than a product to be sold, we can speed up innovation, encourage critical thinking, and empower communities to address important challenges.

\vspace{-0.2cm}
\section{Open Problems}
\vspace{-0.1cm}

Moving forward, we identify key research directions to further exploit the potential of converting original texts into shareable knowledge entities such as demonstrated by the conversion into Knowledge Units in this work:


\textbf{1. Enhancing Factual Accuracy and Reliability:}  Refining KUs through cross-referencing with source texts and incorporating community-driven correction mechanisms, similar to Wikipedia, can minimize hallucinations and ensure the long-term accuracy of knowledge-based datasets at scale.

\textbf{2. Developing Applications for Education and Research:}  Using KU-based conversion for datasets to be employed in practical tools, such as search interfaces and learning platforms, can ensure rapid dissemination of any new knowledge into shareable downstream resources, significantly improving the accessibility, spread, and impact of KUs.

\textbf{3. Establishing Standards for Knowledge Interoperability and Reuse:}  Future research should focus on defining standardized formats for entities like KU and knowledge graph layouts \citep{lenat1990cyc}. These standards are essential to unlock seamless interoperability, facilitate reuse across diverse platforms, and foster a vibrant ecosystem of open scientific knowledge. 

\textbf{4. Interconnecting Shareable Knowledge for Scientific Workflow Assistance and Automation:} There might be further potential in constructing a semantic web that interconnects publicly shared knowledge, together with mechanisms that continually update and validate all shareable knowledge units. This can be starting point for a platform that uses all collected knowledge to assist scientific workflows, for instance by feeding such a semantic web into recently developed reasoning models equipped with retrieval augmented generation. Such assistance could assemble knowledge across multiple scientific papers, guiding scientists more efficiently through vast research landscapes. Given further progress in model capabilities, validation, self-repair and evolving new knowledge from already existing vast collection in the semantic web can lead to automation of scientific discovery, assuming that knowledge data in the semantic web can be freely shared.

We open-source our code and encourage collaboration to improve extraction pipelines, enhance Knowledge Unit capabilities, and expand coverage to additional fields.

\vspace{-0.2cm}
\section{Conclusion}
\vspace{-0.1cm}

In this paper, we highlight the potential of systematically separating factual scientific knowledge from protected artistic or stylistic expression. By representing scientific insights as structured facts and relationships, prototypes like Knowledge Units (KUs) offer a pathway to broaden access to scientific knowledge without infringing copyright, aligning with legal principles like German \S 24(1) UrhG and U.S. fair use standards. Extensive testing across a range of domains and models shows evidence that Knowledge Units (KUs) can feasibly retain core information. These findings offer a promising way forward for openly disseminating scientific information while respecting copyright constraints.

\section*{Author Contributions}

Christoph conceived the project and led organization. Christoph and Gollam led all the experiments. Nick and Huu led the legal aspects. Tawsif led the data collection. Ameya and Andreas led the manuscript writing. Ludwig, Sören, Robert, Jenia and Matthias provided feedback. advice and scientific supervision throughout the project. 

\section*{Acknowledgements}

The authors would like to thank (in alphabetical order): Sebastian Dziadzio, Kristof Meding, Tea Mustać, Shantanu Prabhat for insightful feedback and suggestions. Special thanks to Andrej Radonjic for help in scaling up data collection. GR and SA acknowledge financial support by the German Research Foundation (DFG) for the NFDI4DataScience Initiative (project number 460234259). AP and MB acknowledge financial support by the Federal Ministry of Education and Research (BMBF), FKZ: 011524085B and Open Philanthropy Foundation funded by the Good Ventures Foundation. AH acknowledges financial support by the Federal Ministry of Education and Research (BMBF), FKZ: 01IS24079A and the Carl Zeiss Foundation through the project "Certification and Foundations of Safe ML Systems" as well as the support from the International Max Planck Research School for Intelligent Systems (IMPRS-IS). JJ acknowledges funding by the Federal Ministry of Education and Research of Germany (BMBF) under grant no. 01IS22094B (WestAI - AI Service Center West), under grant no. 01IS24085C (OPENHAFM) and under the grant DE002571 (MINERVA), as well as co-funding by EU from EuroHPC Joint Undertaking programm under grant no. 101182737 (MINERVA) and from Digital Europe Programme under grant no. 101195233 (openEuroLLM) 

%%%%%%%%%%%%%%%%%%%%%%%%%%%%%%%%%%%%%%%%%%%%%%%%%%%%%%%%%%%%%%%%%%%%%%%%

%\section*{Acknowledgments}
%The work presented in this paper has been carried out in the context of the SORTEDMOBILITY project. This project is supported by the European Commission and funded under the Horizon 2020 ERA-NET Cofund scheme under grant agreement N 875022. Vito Trianni and Leo D'Amato acknowledge partial support by TAILOR, a project funded by EU Horizon 2020 research and innovation program under GA No 952215.

%%%%%%%%%%%%%%%%%%%%%%%%%%%%%%%%%%%%%%%%%%%%%%%%%%%%%%%%%%%%%%%%%%%%%%%%

\bibliographystyle{ijcai-template/named}
\bibliography{src/references}

%%%%%%%%%%%%%%%%%%%%%%%%%%%%%%%%%%%%%%%%%%%%%%%%%%%%%%%%%%%%%%%%%%%%%%%%

\newpage
\appendix
%reset the counter
\setcounter{figure}{0} 
\renewcommand{\thefigure}{S\arabic{figure}}
\renewcommand{\theHfigure}{S\arabic{figure}}

\setcounter{section}{0} 
\renewcommand{\thesection}{S\arabic{section}}
% Custom appendix autoref labels
\preto\appendix{%
   \renewcommand{\sectionautorefname}{Appendix}%
   \renewcommand{\subsectionautorefname}{Appendix Subsection}%
}

\begin{center}
    \makebox[1\width]{
    \begin{tabular}{c}
        \huge \textbf{Supplementary Materials}
    \end{tabular}
    }
\end{center}

\section{Introduction}

The structure of supplementary material is as follows:
Section \ref{sec:data_gen} illustrates the generation process of a problem instance.
Section~\ref{sec:centralised_appr} describes an integer linear-programming formulation of the problem, useful to compute optimal solutions in a centralised way, as a benchmark for decentralised approaches.
Section \ref{sec:suppl_dsa} shows the effect of the activation probability $\alpha$ by running experiments with $\alpha=1$, $0.9$ and $0.7$.

Recall from the main text that an instance of dec-rtRTMP can be compactly represented as a pair of graphs $(\mathcal{G}_I, \mathcal{G}_C)$ where $\mathcal{G}_I = (\mathcal{V}_I, \mathcal{E}_I)$ is the \emph{interaction graph}, i.e. a graph whose nodes $\mathcal{V}_I$ are agents and links $\mathcal{E}_I$ represents neighbouring agents, and $\mathcal{G}_C = (\mathcal{V}_C, \mathcal{E}_C)$ is the constraint graph, i.e. a $n$-partite graph whose nodes $\mathcal{V}_C$ are all the possible paths $d \in \bigcup_{D \in \mathfrak{D}} D$ generated by the agents and the links $\mathcal{E}_C$ denote compatible paths. An example of problem instance is reported in Figure \ref{fig:pi_example}.

\begin{figure}[ht!]
    \centering
    \includegraphics[width=1\linewidth]{assets/imgs/pi_formulation.pdf}
    \caption{An example of problem instance in our DCOP-based formulation of the dec-rtRTMP. 
        \textbf{(A)} Interaction graph $\mathcal{G}_I$.
        \textbf{(B)} Constraint graph $\mathcal{G}_C$.
        \textbf{(C)} Two possible solutions of the problem instance.
        }
    \label{fig:pi_example}
\end{figure}

%Since a path is an hypothesis formulated by the train about future evolution of railway traffic, from here on we refer to it simply as \emph{hypothesis}.

\section{Data Generation}\label{sec:data_gen}

% The dataset contains synthetically generated problem instances. 
To generate a problem instance, we specify (i) the number of agents $n$, (ii) the interaction rate between agents $p_{\operatorname{int}}$, (iii) the maximum number of paths each agent can generate $n_d$, (iv) the minimum number of solutions $n_{sol}$ we want the problem instance to have, and (v) a random seed $\sigma$ since the generation process is not deterministic.

The parameters $n$ and $p_{\operatorname{int}}$ determine the stochastic generation of the interaction graph $\mathcal{G}_I$: $n$ corresponds to the number of nodes of $\mathcal{V}_I$ while $p_{\operatorname{int}}$ corresponds to the probability of an edge between two nodes, and therefore it varies between 0 and 1. $\mathcal{G}_I$ is generated as a connected graph because, when it is not connected, our algorithm can be applied to each connected component independently. To generate $\mathcal{G}_I$ as random connected graph, we operate the following steps: we first create a tree with nodes $\mathcal{V}_I$ and then we add random links between nodes with probability $p_{\operatorname{int}}$.
%is the same as the one for generating a Erd\H{o}s-R\'enyi graph \cite{Erdos:1959:pmd}, except that we enforce the resulting graph to be connected by adding one random edge for each node before adding any other edge.

The parameters $n_d$ and $n_{sol}$ are instead used to generate the constraint graph $\mathcal{G}_C$. Each agent in $\mathcal{G}_I$ generates a random number of paths between $1$ and $n_d$. Each path $d$ is associated with a utility value $u(d)$. We assume that each agent has one preferred path and other less desirable ones. The former has a utility value equal to 1, while the latter have a small utility value equal to 0.1. This choice follows a standard practice for experimentation in multi-alternative decisions in which the less-preferred options are considered distractors with an equally low value~\cite{Reina:2017jl}.
%Possible values for $u_h$ are either $0.1$ or $1$, with the constraint that exactly one path among the generated ones has a value of $1$. The rationale behind this design choice...  } 
The set of paths of all the agents correspond to the nodes $\mathcal{V}_C$ of $\mathcal{G}_C$.
The links in $\mathcal{G}_C$ are inserted according to the following procedure: we first construct $n_{sol}$ different solutions (see below), and the links constituting such solutions are then added to $\mathcal{G}_C$. At this point not all the nodes in $\mathcal{G}_C$ have a link and thus we randomly add one link for each node with degree $0$ to avoid having paths that are not compatible with any other path. When adding random links, we must take into account the interactions in $\mathcal{G}_I$. Indeed, if two agents do not interact, there is no reason to check the compatibility of their paths and thus we only randomly add links between paths of neighbouring agents.

The construction of a solution to the problem is straightforward: it is enough to randomly select one path per agent and then add all the possible links between paths of neighbouring agents in $\mathcal{G}_C$. Note that even if we add $n_{sol}$ solutions to a problem instance, the total number of solutions can be greater than $n_{sol}$ because the union of the links constituting two solutions can generate several additional solutions, especially if the number of nodes in $\mathcal{V}_C$ is small. This means that the total number of solutions in a problem instance cannot be controlled at generation time but must be computed after the generation, in a centralised fashion, by means of the CPLEX solver, as described in Section~\ref{sec:centralised_appr}. Figure 1 in the main text shows the distribution of the number of solutions in the problem instances of our dataset.  Note that 
%$n_{sol}$ determines the minimum number of solutions, but not the total number, as shown in Figure~\ref{fig:sol_distr}. Indeed, 
the highest number of solutions is found in instances in which $n$ is not much larger than $n_{sol}$, with instances in the group determined by $n=20$ and $n_{sol}=10$ having up to $10^4$ possible solutions. Instead, when $n$ is much larger than $n_{sol}$ (e.g., $n=100$ and $n_{sol}=10$), only the minimum number of solutions $n_{sol}$ is present.

We generated problem instances by fixing the interaction rate $p_{\operatorname{int}}=0.3$ and the maximum number of paths per train $n_d=8$. These values have been selected empirically to obtain a sufficiently rich topology for $\mathcal{G}_I$ and $\mathcal{G}_C$. Then, we generate problem instances by varying $n \in \{10, 20, 50, 100\}$ and $n_{sol}\in\{3, 5, 10\}$. For each combination, we generate 100 instances by varying the random seed $\sigma\in\{0,\ldots,99\}$. The seed $\sigma$ is  used to make the generation process reproducible. Overall, the total number of problem instances in our dataset is $1200$.\footnote{The dataset will be publicly available with the aim of stimulating research on dec-rtRTMP.}


\section{Centralised solution to the coordination problem}\label{sec:centralised_appr}

In this section, we aim at determining if a given problem instance $(\mathcal{G}_I, \mathcal{G}_C)$ admits a solution and, in case it does, we want to compute all possible solutions while identifying the optimal ones.
%\textcolor{blue}{forse conviene specificare che non è possibile farlo in maniera decentralizzata ? }
To accomplish this, we can reformulate the problem as an integer linear programming (ILP) task and utilize widely available software such as CPLEX \footnote{https://www.ibm.com/it-it/products/ilog-cplex-optimization-studio} as solvers.
%
%Given a problem instance $(\mathcal{G}_I, \mathcal{G}_C)$, we would like to know whether it admits a solution or not and, in case it does, what are all the possible solutions and which ones are optimal. A method to achieve this goal is by reformulating it as an integer linear programming (ILP) problem and solve it by means of standard software like CPLEX. 
%
%The output of our decentralised algorithm at convergence is a solution to the problem instance being solved. A way to assess the performance of the our algorithm is to compare the value of the solution found with the value of the optimal solution to the problem being solved. A method to compute the optimal solution of a problem instance is by reformulating it as an integer linear programming (ILP) problem and solve it by means of tools like CPLEX. 
%
Such a centralised approach provides a reference for the evaluation of the decentralised solutions in our experiments, as discussed in the main text. 

%The output of our decentralised algorithm at convergence is a solution to the problem instance being solved. In order to assess the performance of the consensus procedure we must access all the solutions of the problem instance. One way to compute the entire set of solutions of a problem instance is by reformulating the rtRTMP as an integer linear programming (ILP) problem and solve it by means of a solver like CPLEX. Such an approach is no longer decentralised and serves as a baseline for our experiments in section \ref{sec:exps}. 
To define the ILP forumulation, we first introduce the mapping $\alpha : \mathcal{V}_C \mapsto \{1, \ldots, n\}$ associating to each path $d$ the index of the agent it belongs to. Then, we define the binary decision variables $y_d$ as follows:
%
\begin{equation}
\label{eq:binary}
y_d = \left\{
                \begin{array}{ll}
                  1 \quad \mbox{if $v_{\alpha(d)}=d$}     \\
                  0	\quad \mbox{otherwise}\\
                \end{array}
                 \right. \quad \forall d \in \mathcal{V}_C
\end{equation}
where the notation $v_{\alpha(d)}=d$ means that the path $d$ has been assigned to the variable $v_{\alpha(d)}$ controlled by the agent $a_{\alpha(d)}$. 
Recall also that each node $d \in \mathcal{V}_C$ has a value $u(d)$. 
Then, the centralized optimization problem consists in finding the appropriate complete assignment of paths to maximise the following objective function: 
%
\begin{equation}\label{eq:objective}
\max\sum_{d \in \mathcal{V}_C} u(d) \ y_d 
\end{equation}
provided that the following constraints are satisfied:
\begin{align}
\sum_{d \in D_i} y_d = 1,\quad      &\forall i = 1 \ldots n  \label{eq:eq3} \\[10pt] 
\sum_{d^{\prime} \in D_j : (d, d^{\prime}) \in \mathcal{E}_C} y_{d^{\prime}}  \geq y_d \quad &\forall d \in \mathcal{V}_C, \nonumber \\[-15pt] 
                                                                                       &\forall j=1 \ldots n : A_j  \in \mathcal{N}_{\alpha(d)}, j \neq \alpha(d) \label{eq:eq4}
\end{align}
Constraints~(\ref{eq:eq3}) ensure that exactly one path per train is selected (each agent can only assign one value to the variable it controls). Constraints~(\ref{eq:eq4}) state that, if path $d$ is selected and it is associated to agent $\alpha(d)$, then each neighbouring agent in $\mathcal{N}_{\alpha(d)}$ must select a path that is compatible with $d$.

By exploiting CPLEX as solver for this ILP formulation of the agent coordination in the dec-rtRTMP, we are able to find all possible solutions to a given problem instance, and among them, to identify the optimal ones as well.
%and among these to also identify the optimal solution.

\section{DSA}
\label{sec:suppl_dsa}

We run the DSA algorithm with $\alpha=1, 0.9, 0.7$ and evaluate it performance as in the main text. See Figures \ref{fig:suppl_dsa_ranking}, \ref{fig:suppl_dsa_regret} and \ref{fig:suppl_dsa_convTime}.

\begin{figure*}[!t]
    \centering
    \includegraphics[width=\textwidth]{assets/imgs/suppl_dsa_ranking_h.pdf}
    \caption{Ranking of the solutions given by DSA algorithms. Data are grouped by the type of agent, by number of agents $n$ and by minimum number of solutions $n_{sol}$ we require the problem instance to have. Bars with label ``$1$'' represent the fraction of executions that converged to an optimal solution. Bars with label ``$\geq 10$'' represent the fraction of executions that converged to a solution in position greater than 10 in the ranking. Bars with label ``Fail'' represent the fraction of executions that exceeded the upper bound of $10^5$ iterations.}
    \label{fig:suppl_dsa_ranking}
\end{figure*}

\begin{figure*}[!t]
    \centering
    \includegraphics[width=\textwidth]{assets/imgs/suppl_dsa_regret_top3.pdf}
    \caption{Top: Regret (percentage loss from the optimal solution value) of the solutions given by DSA algorithm in position 2 or 3 of the ranking. Data are grouped by type of agent, by number of agents $n$ and by minimum number of solutions $n_{sol}$ we require the problem instance to have.
    Bottom: Average convergence rate to a top-3 solution.}
    \label{fig:suppl_dsa_regret}
\end{figure*}

\begin{figure*}[!t]
    \centering
    \includegraphics[width=\textwidth]{assets/imgs/suppl_dsa_convTime_h.pdf}
    \caption{Distribution of convergence times (in terms of number of iterations) per type of agent on problem instances grouped by number of agents $n$ and by minimum number of solutions $n_{sol}$ we require the problem instance to have. For each of the 100 problem instances characterised by $n$ and $n_{sol}$, we performed $100$ executions of DSA algorithm for each agent type. Each execution has an upper bound of $10^5$ iterations, beyond which it fails.}
    \label{fig:suppl_dsa_convTime}
\end{figure*}



%%%%%%%%%%%%%%%%%%%%%%%%%%%%%%%%%%%%%%%%%%%%%%%%%%%%%%%%%%%%%%%%%%%%%%%%

\end{document}