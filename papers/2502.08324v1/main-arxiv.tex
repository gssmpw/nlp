%%%% ijcai25.tex

\typeout{IJCAI--25 Instructions for Authors}

% These are the instructions for authors for IJCAI-25.

\documentclass{article}
\pdfpagewidth=8.5in
\pdfpageheight=11in

% The file ijcai25.sty is a copy from ijcai22.sty
% The file ijcai22.sty is NOT the same as previous years'
\usepackage{ijcai-template/ijcai25}


% Use the postscript times font!
\usepackage{times}
\usepackage{soul}
\usepackage{url}
\usepackage[hidelinks]{hyperref}
\usepackage[utf8]{inputenc}
\usepackage[small]{caption}
\usepackage{graphicx}
\usepackage{amsmath}
\usepackage{amsthm}
\usepackage{booktabs}
\usepackage{algorithm}
\usepackage{algorithmic}
\usepackage[switch]{lineno}

\usepackage{adjustbox}
\usepackage{booktabs}
% Comment out this line in the camera-ready submission
%\linenumbers

\urlstyle{same}

% the following package is optional:
% \usepackage{latexsym}

% packages added by me
\usepackage{comment}
\usepackage{xcolor}
\usepackage{amsmath}
\usepackage{amssymb}

% See https://www.overleaf.com/learn/latex/theorems_and_proofs
% for a nice explanation of how to define new theorems, but keep
% in mind that the amsthm package is already included in this
% template and that you must *not* alter the styling.
\newtheorem{example}{Example}
\newtheorem{theorem}{Theorem}

% Following comment is from ijcai97-submit.tex:
% The preparation of these files was supported by Schlumberger Palo Alto
% Research, AT\&T Bell Laboratories, and Morgan Kaufmann Publishers.
% Shirley Jowell, of Morgan Kaufmann Publishers, and Peter F.
% Patel-Schneider, of AT\&T Bell Laboratories collaborated on their
% preparation.

% These instructions can be modified and used in other conferences as long
% as credit to the authors and supporting agencies is retained, this notice
% is not changed, and further modification or reuse is not restricted.
% Neither Shirley Jowell nor Peter F. Patel-Schneider can be listed as
% contacts for providing assistance without their prior permission.

% To use for other conferences, change references to files and the
% conference appropriate and use other authors, contacts, publishers, and
% organizations.
% Also change the deadline and address for returning papers and the length and
% page charge instructions.
% Put where the files are available in the appropriate places.


% PDF Info Is REQUIRED.

% Please leave this \pdfinfo block untouched both for the submission and
% Camera Ready Copy. Do not include Title and Author information in the pdfinfo section
\pdfinfo{
/TemplateVersion (IJCAI.2025.0)
}

%%%%%%%%%%%%%%%%%%%%%%%%%%%%%%%%%%%%%%%%%%%%%%%%%%%%%%%%%%%%%%%%%%%%%%%%

\title{Decentralised multi-agent coordination for real-time railway traffic management}

%%%%%%%%%%%%%%%%%%%%%%%%%%%%%%%%%%%%%%%%%%%%%%%%%%%%%%%%%%%%%%%%%%%%%%%%

\author{
Leo D'Amato$^{1,3}$
\and
Paola Pellegrini$^2$\and
Vito Trianni$^3$\\
\affiliations
$^1$Politecnico di Torino, DAUIN, Corso Castelfidardo, 34/d, Turin 10128, Italy.\\
$^2$Univ Gustave Eiffel, COSYS-ESTAS, F-59650 Villeneuve d’Ascq, France\\
$^3$Institute for Cognitive Sciences and Technologies, CNR, Via G. D. Romagnosi 18/A, 00196 Rome, Italy.\\
\emails
%\{first, second\}@example.com,
leo.damato@polito.it
%second@other.example.com,
%third@example.com
}


%%%%%%%%%%%%%%%%%%%%%%%%%%%%%%%%%%%%%%%%%%%%%%%%%%%%%%%%%%%%%%%%%%%%%%%%

\begin{document}

%%%%%%%%%%%%%%%%%%%%%%%%%%%%%%%%%%%%%%%%%%%%%%%%%%%%%%%%%%%%%%%%%%%%%%%%

\maketitle

%%%%%%%%%%%%%%%%%%%%%%%%%%%%%%%%%%%%%%%%%%%%%%%%%%%%%%%%%%%%%%%%%%%%%%%%

\begin{abstract}
    \begin{abstract}
Retrieval-Augmented Generation (RAG) is often used with Large Language Models (LLMs) to infuse domain knowledge or user-specific information. In RAG, given a user query, a retriever extracts chunks of relevant text from a knowledge base. These chunks are sent to an LLM as part of the input prompt. Typically, any given chunk is repeatedly retrieved across user questions. However, currently, for every question, attention-layers in LLMs fully compute the key values (KVs) repeatedly for the input chunks, as state-of-the-art methods cannot reuse KV-caches when chunks appear at arbitrary locations with arbitrary contexts. Naive reuse leads to output quality degradation.  This leads to potentially redundant computations on expensive GPUs and increases latency. In this work, we propose \sys, a system for managing and reusing precomputed KVs corresponding to the text chunks (we call \textit{chunk-caches}) in RAG-based systems. We present how to identify \hl{\textit{chunk-caches} that are reusable}, how to efficiently perform a small fraction of recomputation to \textit{fix} the cache to maintain output quality, and how to efficiently store and evict \textit{chunk-caches} in the hardware for maximizing reuse while masking any overheads. With real production workloads as well as synthetic datasets, we show that \sys reduces redundant computation by \textbf{51\%} over SOTA prefix-caching and \textbf{75\%} over full recomputation.
\hl{Additionally, with continuous batching on a real production workload, we get a \textbf{1.6$\times$} speedup in throughput and a \textbf{2$\times$} reduction in end-to-end response latency over prefix-caching while maintaining quality, for both the \llama-3-8B and \llama-3-70B models. 
}
\end{abstract}





\end{abstract}

%%%%%%%%%%%%%%%%%%%%%%%%%%%%%%%%%%%%%%%%%%%%%%%%%%%%%%%%%%%%%%%%%%%%%%%%

\documentclass[../main.tex]{subfiles}
\graphicspath{{../images/}}
\makeatletter
\def\input@path{{../images/}}
\makeatother
\begin{document}
\section{Introduction}
\begin{figure}
\centering
\begin{tikzpicture}
\node[inner sep=0pt] (ws) at (0, 0) {
\includegraphics[height=.4\textwidth, trim={10cm 0 10cm 0},clip]{world_space.png}};
\node[inner sep=0pt] (cs) at (6,0) {\includegraphics[height=.4\textwidth, trim={10cm 1cm 10cm 4cm},clip]{conf_space.png}};
\end{tikzpicture}
\vspace{-5pt}
\label{fig:pbrm_intro}
\caption{\textbf{Left}: Shows world space obstacles as grey spheres. Robots start and goal configuration is colored red and green, respectively. Configurations along the computed path are colored transparent blue. \textbf{Right:} Mapped world space scenario to configuration space. Obstacle region is the grey mesh. Red spheres are collision-free regions computed by the neural SCDF. The optimized shortest path in the convex corridor is the blue curve.}
\vspace{-25pt}
\end{figure}
Motion planning is the problem of finding a collision-free trajectory that connects a given start and goal configuration. The planning takes place in the configuration space of the robot. For single body robots, like mobile robots or drones, the configuration space and the world space are usually the same. This simplifies the planning, since explicit obstacle representations are available which enables geometrical tools like separating hyperplanes, smallest distance to obstacles etc., to be used when designing motion planning algorithms. For multi-body robots like manipulators, the situation is completely different. The world space obstacles are usually mapped to non-convex regions, and to make the problem even harder, the mapping is usually not known. Forming explicit representations of the obstacle region in the configuration space is usually too expensive or intractable. Despite all of this, sampling based planners are used with great success, which mainly is due to their use of implicit representations of the obstacle region. The basic idea is to construct a graph in the configuration space that covers and connects the collision-free region. From this graph, a path can be extracted that connects a given start and goal configuration. The approach is computationally expensive, since the graph is constructed with the smallest geometrical building block available, points, which represents a collision-check. Furthermore, the extracted paths from the graph are non-smooth and jagged due to the stochastic nature of the approach. This adds an additional post-processing step to the process, where the paths are shortcutted and smoothened, before the path can be used for tracking. Clearly a lot of time is invested to form this graph and produce smooth paths. Thus, if the obstacles start to move, then all of this work is done in no use, since all points that make up this graph need to be re-verified, which is simply too time consuming to be done in real time.
\\\\
In this work, we want to address the existing drawbacks of the sampling based planners. Our main contribution is an improved motion planner where each vertex in the graph covers a collision-free region in the form of a sphere instead of a point and where the edges are formed with neighboring intersecting spheres. This representation has the advantage of instead of returning piecewise linear paths, returning a sequence of overlapping spheres, i.e. a convex corridor, that connects a given start and goal configuration, illustrated in Figure \ref{fig:pbrm_intro}. This convex corridor allows us to use convex optimization to produce smooth trajectories, instead of computationally expensive post-processing methods. The representation further allows us to estimate the coverage of the collision-free space, which gives us awareness and feedback in the offline roadmap construction phase. Finally, our representation is simple to adapt to moving obstacles, simply requery for the new radii and recheck for intersections. 
\\\\
The spherical collision-free regions are formed using a signed distance function (SDF), which is a function that returns the smallest distance from an arbitrary point to the boundary of an obstacle. As the name implies, the distance is signed, thus if the point is inside the obstacle it is negative otherwise positive. If the distance is positive, a sphere with radius equal to the distance is guaranteed to cover a collision-free region. Using an SDF in motion planning is not new, but what is novel about our approach is that we express the distance in the configuration space instead of the world space and by doing so allows us to form these convex collision-free regions. We refer to the resulting SDF as a signed configuration distance function (SCDF). Computing an SCDF analytically is non-trivial, our approach is therefore to parameterize the SCDF with a deep neural network and learn the mapping by supervised learning. Our resulting neural SCDF can compute distances for different parameter values of obstacle shapes and we also show how multiple distances can be combined, thus making our approach flexible.
\section{Related work}
Motion planning algorithms can roughly be divided into three families, grid-based, sampling based and optimization based methods. Grid-based methods (GBM) discretize the planning space from which a graph is then compiled. A standard search method is A$^\star$ \citep{a_star}, which is classified as an \textit{informed} search method, since it employs a heuristic function to speed up the search. A$^\star$ guarantees to return an optimal path at the level of discretization used. GBMs usually discretize the planning space by a regular lattice and this limits the GBMs to problems with low dimensionality due to the curse of dimensionality. Thus, GBMs are usually limited to single-body robots where the degrees of freedom (DOF) are low. To overcome the inherent scaling problem with the GBMs, stochastic methods are usually used for multi-body robots. These methods are termed as sampling-based methods (SBM) and core members within this family are the rapidly-exploring random trees (RRT) \citep{rrt} and the probabilistic roadmap (PRM) \citep{prm}. RRT grows a tree from the start configuration and explores the collision-free region in a rapid way until it is able to connect to the goal region. RRT is usually improved by bi-directional planning \citep{rrt_connect}, i.e. an additional tree is grown from the goal configuration and the trees are tested for connection after any tree has been expanded. RRT is a single-query method, thus it searches for a path from scratch each time it is queried. Contrary to this, PRM is a multi-query method, which solves for multiple queries without starting from scratch. PRM does this by creating a roadmap (graph) that covers the collision-free space as an offline step. The graph is then used to solve for multiple queries. PRMs are used in cases where the environment does not change since the extra offline step is too computationally costly and needs to be re-done if the environment is changed. In our work, we address this inherent issue by using a different roadmap representation. Our vertices in the graph cover a collision-free region in the form of spheres and we form the edges by checking for intersecting spheres. If something in the environment changes, we recompute the spheres radii and recheck the intersections, without relying on collision detection. We use a trained neural network to compute the sphere radius, therefore querying for the radius can be done fast, hence our representation enables the PRM for dynamic environments.
\\\\
In the recent decades, optimization based methods (OBM) \citep{chomp, schulman, itomp, stomp} have been introduced as an alternative to SBM for multi-body robots. Like the SBM, the OBMs scale well to higher dimensional problems and produce smoother motion. It is common to use a SDF in the optimization since it is a smooth function, thus enabling gradient-based methods. However, the standard way of expressing the SDF is in world space. The distance therefore needs to be mapped to the configuration space by the forward kinematics. This mapping makes the optimization problem a non-linear program (NLP), which is computationally expensive to solve. Recently, a different approach has been proposed. In \cite{mp_gcs} motion planning is formulated as a convex optimization problem by using the graph of convex sets framework \citep{gcs}. The underlying idea is to decompose the collision-free space into intersecting convex sets from which a convex optimization problem is formulated. In cases where an explicit representation of the obstacles in the configuration space exists, like for single-body robots, creating collision-free convex regions can be done fast \citep{iris}. For multi-body robots, this is non-trivial. Existing work does this successfully \citep{iris_nlp, iris_c} by an optimization based approach, but the methods are still too time consuming to be used in the presence of moving obstacles. Our approach is instead to use deep learning to learn an SDF expressed in the configuration space. With this, we can query for shortest distances to the collision boundary, which allows us to expand spherical regions which are collision-free. Our approach is fast and therefore enables our suggested roadmap planner to be used in dynamic environments.
\\\\
Recent research has focused on learning collision detection \citep{fk_kernel_distance, diffco, graphdistnet} by predicting the signed distance between the robot links and the surrounding obstacles in the world space. The learned SDF is used in trajectory optimization but since the distance is expressed in the world space, the problem becomes an NLP and therefore takes a long time to solve. We take a novel approach and suggest to instead express the signed distance in the configuration space. This allows us to improve the PRM at the same time as it enables convex optimization for trajectory optimization, which runs faster and is more reliable than NLP solvers. In \cite{cspf} a learned signed distance function in the configuration space is proposed similar to our approach. However, their approach is restricted to point cloud representations, while we propose to represent the obstacles as parameterized geometric shapes, e.g. spheres. Furthermore, we also show how to use our learned SCDF to improve an existing roadmap planner.
\section{Problem formulation}
A robot is located in the world space, $\W \subset \R^3 $. The unique location of the robot is given by its configuration $\q \in \C$, where $\C$ is the configuration space. The set of points covered by the robots bodies at a certain configuration is expressed as $\B(\q) \subset \W$. The robot is surrounded by $\NrObst$ obstacles $\O = \bigcup_{i=1}^{\NrObst} \O_i$, where  $\O_i \subset \W$. The representation of the obstacle in the configuration space is the set $\C\O_i = \{\q \in \C \: |\: \B(\q) \cap \O_i \neq \emptyset \}$. The obstacle space is formed as $\Co = \bigcup_{i=1}^{\NrObst} \C \O_i$. The complement is referred to as the free space, $\Cf = \C \setminus \Co$. The path planning problem is a tuple, ($\Cf$, $\qStart$, $\qGoal$), where we want to connect a query pair, consisting of a start, $\qStart$, and goal configuration, $\qGoal$, with a geometric path, $\q(s): [0, 1] \mapsto \Cf$, such that $\q(0)=\qStart$ and $\q(1)=\qGoal$, or report correctly when such a path does not exist.
\end{document}


%%%%%%%%%%%%%%%%%%%%%%%%%%%%%%%%%%%%%%%%%%%%%%%%%%%%%%%%%%%%%%%%%%%%%%%%

\section{Problem Formulation}
\label{sec:problem formulation}

An HASN graph is denoted as $G(V, E)$, where $\forall v \in V$ is a set of vertices comprising the sets $H$ (human users) and $AI$ (AI entities), such that $|V| = |H| + |AI|$, and $\forall e \in E$ represents the set of edges between humans, AIs, and human-AI connections. 

\textbf{The \problem\ clustering problem} aims to partition an HASN graph into $K$ disjoint subgraphs $C_i(V_i, E_i)$, where $\bigcup_{i=1}^K V_i \subseteq V$ (since AI nodes and their connected edges may be removed during the clustering process) and $V_i \bigcap V_j = \emptyset$, with prior knowledge of which nodes in the network are AI nodes. The goal of \problem\ is to discover a set of clusters (subgraphs) $P = \{ C_i \}_1^K = \{ C_1, C_2, \ldots, C_K \}$ that can maximize human closeness with minimal AI presence. Concretely, a desirable clustering result of an HASN should achieve two key objectives simultaneously: (1) maximizing human closeness and (2) minimizing the presence of AI nodes for each cluster. 

\subsection{Objective Function of \problem}
\label{subsec:objective_function}

To achieve the goal of \problem, we employ a modularity function introduced in a seminal work by Newman as our objective function \cite{newman2004finding}:

\begin{equation}
Q(P=\{C_i\}_{i=1}^K) = \frac{1}{2|E|} \left( \sum_{i=1}^K \sum_{v_p, v_q \in C_i}\left( A_{pq} - \frac{d_p d_q}{2|E|} \right) \right)
\end{equation}
\vspace{0.5em}

Modularity $Q$ measures clustering quality in networks by comparing the density within clusters to the density between clusters. It ranges from -0.5 to 1, with higher scores indicating better clustering. Here, $A$ is the adjacency matrix, $A_{pq}$ indicates the presence of a connection between nodes $p$ and $q$, and $d_p$ is the degree of node $p$. 

To encourage the clustering algorithm to generate cohesive communities with minimal AI presence, we modify the vanilla modularity by infusing a reward-penalty function. This function reweights the clustering quality based on the ratio of humans (and AIs) presence in each cluster $C_i$, defined by:

\begin{equation}
W(C_i) = \beta \cdot \frac{\sum_{v \in C_i} L_v}{|C_i|} - \gamma \cdot \frac{\sum_{v \in C_i} (1 - L_v)}{|C_i|}
\end{equation}
\vspace{0.5em}

\noindent where 
\begin{equation}
L_v =
\begin{cases} 
1, & \text{if node } v \in H \\
0, & \text{if node } v \in AI
\end{cases}
\end{equation}
\vspace{0.5em}

\noindent This leads to a human-centric modularity $HQ$:

\begin{equation}
HQ(P) = \frac{1}{2|E|} \left( \sum_{i=1}^K \alpha \cdot W(C_i) \cdot \left( \sum_{v_p, v_q \in C_i}\left( A_{pq} - \frac{d_p d_q}{2|E|} \right) \right) \right)
\end{equation}
\vspace{0.5em}

\noindent Note that $\beta$ is the weight for rewarding human nodes, $\gamma$ is the weight for penalizing AI nodes, and $\alpha$ is the weight for adjusting the emphasis on human nodes in the objective function \footnote{For simplicity, we set $\alpha$, $\beta$, and $\gamma$ to 1 in our experiments to observe the proposed algorithm’s core behavior without the added complexity of multiple parameters.}. Accordingly, the purpose of \problem\ is to discover a set of clusters (subgraphs) $P = \{ C_i \}_1^K$ that maximizes $HQ$:

\begin{equation}
P^* = \arg \max_{\{C_i\}_{i=1}^k} HQ(\{C_i\}_{i=1}^K)
\end{equation}
\vspace{0.5em}

This objective function promotes the generation of tight-knit communities with minimal AI presence. Since certain AI entities can aid in the formation of these human-centric communities, it is crucial to identify and preserve AI nodes that can promote human closeness while removing those that can not.



%%%%%%%%%%%%%%%%%%%%%%%%%%%%%%%%%%%%%%%%%%%%%%%%%%%%%%%%%%%%%%%%%%%%%%%%

% insert this in supplementary materials
% \section{Centralised solution to the coordination problem}\label{sec:centralised_appr}

In this section, we aim at determining if a given problem instance $(\mathcal{G}_I, \mathcal{G}_C)$ admits a solution and, in case it does, we want to compute all possible solutions while identifying the optimal ones.
%\textcolor{blue}{forse conviene specificare che non è possibile farlo in maniera decentralizzata ? }
To accomplish this, we can reformulate the problem as an integer linear programming (ILP) task and utilize widely available software such as CPLEX \footnote{https://www.ibm.com/it-it/products/ilog-cplex-optimization-studio} as solvers.
%
%Given a problem instance $(\mathcal{G}_I, \mathcal{G}_C)$, we would like to know whether it admits a solution or not and, in case it does, what are all the possible solutions and which ones are optimal. A method to achieve this goal is by reformulating it as an integer linear programming (ILP) problem and solve it by means of standard software like CPLEX. 
%
%The output of our decentralised algorithm at convergence is a solution to the problem instance being solved. A way to assess the performance of the our algorithm is to compare the value of the solution found with the value of the optimal solution to the problem being solved. A method to compute the optimal solution of a problem instance is by reformulating it as an integer linear programming (ILP) problem and solve it by means of tools like CPLEX. 
%
Such a centralised approach provides a reference for the evaluation of the decentralised solutions in our experiments, as discussed in the main text. 

%The output of our decentralised algorithm at convergence is a solution to the problem instance being solved. In order to assess the performance of the consensus procedure we must access all the solutions of the problem instance. One way to compute the entire set of solutions of a problem instance is by reformulating the rtRTMP as an integer linear programming (ILP) problem and solve it by means of a solver like CPLEX. Such an approach is no longer decentralised and serves as a baseline for our experiments in section \ref{sec:exps}. 
To define the ILP forumulation, we first introduce the mapping $\alpha : \mathcal{V}_C \mapsto \{1, \ldots, n\}$ associating to each path $d$ the index of the agent it belongs to. Then, we define the binary decision variables $y_d$ as follows:
%
\begin{equation}
\label{eq:binary}
y_d = \left\{
                \begin{array}{ll}
                  1 \quad \mbox{if $v_{\alpha(d)}=d$}     \\
                  0	\quad \mbox{otherwise}\\
                \end{array}
                 \right. \quad \forall d \in \mathcal{V}_C
\end{equation}
where the notation $v_{\alpha(d)}=d$ means that the path $d$ has been assigned to the variable $v_{\alpha(d)}$ controlled by the agent $a_{\alpha(d)}$. 
Recall also that each node $d \in \mathcal{V}_C$ has a value $u(d)$. 
Then, the centralized optimization problem consists in finding the appropriate complete assignment of paths to maximise the following objective function: 
%
\begin{equation}\label{eq:objective}
\max\sum_{d \in \mathcal{V}_C} u(d) \ y_d 
\end{equation}
provided that the following constraints are satisfied:
\begin{align}
\sum_{d \in D_i} y_d = 1,\quad      &\forall i = 1 \ldots n  \label{eq:eq3} \\[10pt] 
\sum_{d^{\prime} \in D_j : (d, d^{\prime}) \in \mathcal{E}_C} y_{d^{\prime}}  \geq y_d \quad &\forall d \in \mathcal{V}_C, \nonumber \\[-15pt] 
                                                                                       &\forall j=1 \ldots n : A_j  \in \mathcal{N}_{\alpha(d)}, j \neq \alpha(d) \label{eq:eq4}
\end{align}
Constraints~(\ref{eq:eq3}) ensure that exactly one path per train is selected (each agent can only assign one value to the variable it controls). Constraints~(\ref{eq:eq4}) state that, if path $d$ is selected and it is associated to agent $\alpha(d)$, then each neighbouring agent in $\mathcal{N}_{\alpha(d)}$ must select a path that is compatible with $d$.

By exploiting CPLEX as solver for this ILP formulation of the agent coordination in the dec-rtRTMP, we are able to find all possible solutions to a given problem instance, and among them, to identify the optimal ones as well.
%and among these to also identify the optimal solution.

%%%%%%%%%%%%%%%%%%%%%%%%%%%%%%%%%%%%%%%%%%%%%%%%%%%%%%%%%%%%%%%%%%%%%%%%

\section{A decentralised multi-agent coordination algorithm}\label{sec:consensus}

This section describes the self-organization process at the heart of our decentralised coordination approach. 
Inspired by the classical Decentralised Stochastic Algorithm (DSA) \cite{fitzpatrick2003distributed} from the DCOP literature, coordination is achieved through an iterative procedure, through which agents try to reach a (possibly optimal) solution to a given problem instance $(\mathcal{G}_I, \mathcal{G}_C)$ by only exploiting local information about their respective neighbours. 

At each iteration, the agent $a_i$ decides which value $d \in D_i$ to assign to the variable $v_i$ it controls. The algorithm starts with the agents performing a greedy assignment, i.e. $v_i=d^*$, where $d^* = \operatorname{argmax}_{d \in D_i} u_r(v_i,d)$. At each subsequent iteration, $a_i$ can decide either to keep its current assignment for $v_i$ or switch to another value $d^{\prime} \in D_i$. This choice depends solely on the local information available to the agent, that is, the path utility of the paths in $D_i$ and the degree of compatibility of $(v_i, d)$ with the current assignments $S(N_i)$ of its neighbours, $\forall d \in D_i$. 
Note that the path utility of neighbours' paths is not known, as this information is private to each agent.

More specifically, at each iteration $t$, the agent $a_i$, whose current assignment is $(v_i, d)$, operates the following steps:
\begin{enumerate}
    \item It randomly selects at most $k\geq1$ neighbours from $N_i$. We denote by $K_i(t)$ the subset of neighbours selected by the agent $a_i$ at time $t$. If $k\geq|N_i|$, then $K_i(t)=N_i$.
    \item It observes the current assignments of the selected neighbours, i.e. the set $S(K_i(t))$.  
    \item It creates a ranking over the set $D_i$. For each $d \in D_i$, the agent computes its rank $r(d,t)$ as the number of binary constraints satisfied by a potential assignment of $d$ to $v_i$ with respect to the current assignments $S(K_i(t))$ of the selected neighbours, i.e.     
    $$
    r(d,t) = \sum_{(v_j, d_j) \in S(K_i(t))} u_c \left((v_i, d),  (v_j, d_j)\right)
    $$
    %
    % \begin{align*} &r(d) =  \\ & \left| \left\{ (v_j, d_j) :  a_j \in K_i, v_j=d_j, u \left( \{ (v_i, d) , (v_j, d_j) \} \right) = 1 \right\} \right|. \\ \end{align*}
    %
    Hence, $r(d,t)$ represents the degree of compatibility of the value $d$ with the current assignments of the selected neighbours.
    \item It decides to keep its current assignment $(v_i, d)$ or to switch to a more compatible value $d^{\prime} \in D_i$ according to the following policy: 
    \begin{enumerate}
        \item if $r(d,t) = k$  (i.e. $d$ is compatible with the assignments of all the selected neighbours in $K_i(t)$), then the agent keeps $v_i=d$ as its current assignment.
        \item if $r(d,t) < k$, then the agent selects a more compatible value by sampling a value $d^{\prime}$ from the set of values $D_i$ satisfying the property $d^{\prime} = \operatorname{argmax}_{d \in D_i} r(d,t)$. In case of multiple values $d^{\prime}$ satisfying this property, a probabilistic choice is made with probability proportional to their respective path utility $u_r(v_i, d^{\prime})$.
    \end{enumerate}
\end{enumerate} 
The policy described in step 4 allows the agents to gradually adjust their assignments towards a configuration (solution to the problem) in which all neighbouring agents hold compatible values, while prioritising values with the highest possible utility score.
The parameter $k$ acts as a sort of learning rate for the algorithm.
With high values of the parameter $k$, the agent considers multiple neighbours during the decision making. This can lead the agent to seek compatibility with more neighbours at the same time, hence possibly increasing the speed of convergence towards a shared solution. Conversely, when $k$ is small (possibly, $k=1$) the agent only considers a few neighbours or just one at the time, and therefore the speed of convergence may be slower.
%\textcolor{blue}{
%[Qui mi avevi chiesto di spiegare l'effetto di K ma mi sembra un po' uno spoiler di quello che viene dopo nei risultati.]
%The speed of convergence to a solution is influenced by the hyperparameter $K$. When $K>1$, the agent considers multiple neighbours during the decision making and it can move faster towards a hypothesis that is compatible with the one selected by all its neighbours. The drawback is that agents with large value of $K$ can be trapped into deadlocks. This happens when the decision maker has an hypothesis that is compatible with most its neighbours but not all of them (hence such hypothesis is not part of any solution to the problem). In this scenario, the agent will select this hypothesis, influencing the decisions of all its neighbours at subsequent iterations. As a result, the system starts oscillating between configurations that are not solutions to the problem and it is not able to converge. Deadlocks never happen when $K=1$. In this case, the agent coordinates with one neighbour at a time and one incompatibility is enough to escape eventual deadlocks configurations, provided a sufficient number of iterations.
%}

\paragraph{Convergence Criteria.} 
We implemented an asynchronous multi-agent simulation to emulate real-world operation. Specifically, at each iteration, one single agent is selected randomly and updates its assignment following the default policy discussed above. This ensures that agents take turn with an approximate period of $n$ iterations.
This iterative process continues until one of the following conditions is met: (i) \emph{Convergence}: all agents achieve a state where their hypotheses are compatible with all their neighbours; (ii) \emph{Maximum Iterations}: a predefined maximum number of iterations is reached.

\paragraph{DSA.} Our algorithm differs from the classical DSA on three aspects:
\begin{itemize}
    \item A DSA agent \emph{always} interacts with all its neighbours (it cannot select a subset of them according to $k$).
    \item A DSA agent, with probability $1-\alpha$, can decide to keep it assignment a priori, regardless of the assignments of the neighbours. The parameter $\alpha$ is known as \emph{activation probability}.
    \item A DSA agent implements a different policy. In our implementation of DSA, the agent $a_i$ assigns a score $r(d)$ to each value $d \in D_i$ as follows:
    $$
    r(d) = u_r(v_i, d) + \sum_{(v_j, d_j) \in S(N_i)} u_c \left((v_i, d),  (v_j, d_j)\right)
    $$
    %given by the sum of the utilities of all the constraints (unary and binary) involving the assignment $(v_i, d)$. 
    Then, the agent greedily assigns to $v_i$ the value $d^*$ with the highest score $r$. In some variants of DSA \cite{Zhang_Wang_Xing_Wittenburg_2005}, the assignment can be $\epsilon$-greedy, i.e. with a (usually small) probability $\epsilon$, the agent assigns a random value $d$ to the variable $v_i$ instead of being always greedy. This helps escaping local maxima.
\end{itemize}
Hence, our algorithm improves over DSA by adding a more flexible interaction scheme and proposing a slightly different policy for the decision-making phase. 


%%%%%%%%%%%%%%%%%%%%%%%%%%%%%%%%%%%%%%%%%%%%%%%%%%%%%%%%%%%%%%%%%%%%%%%%

\section{Experiments: Planning outperforms Heuristics}
\label{sec:experiment}

We begin our empirical demonstrations by showcasing the effectiveness of our planning framework on both synthetic and real datasets. We focus on the simplest planning algorithm, 1-step lookaheads (Algorithm~\ref{alg:complete}), and show that even basic planning can hold great promise. 
We illustrate our framework using two uncertainty quantification modules---GPs and 
\ensembles/ \ensembleplus. 

Throughout this section, we focus on evaluating the mean squared error of 
a regression model $\model$,  and develop adaptive policies that minimize uncertainty on $g(f)$ defined in~\eqref{eqn:l2-g-f}.
When GPs provide a valid model of uncertainty, 
our experiments show that our planning framework significantly outperforms other baselines. 
We further demonstrate that our conceptual framework extends to deep learning-based uncertainty quantification methods such as  \ensembleplus while highlighting computational challenges that need to be resolved in order to scale our ideas. 
For simplicity, we assume a naive predictor, i.e., $\psi(\cdot) \equiv 0$. However, we emphasize that this problem is just as complex as if we were using a sophisticated model $\psi(.)$. The performance gap between the algorithms 
primarily depends
on the level  of uncertainty in our prior beliefs.

To evaluate the performance of our algorithm, we benchmark it against several baselines. 
%Active learning baselines use an acquisition function $\ac$ to select points that have the highest   function value: $X\opt_t \in \argmax_{X \in \xpoolj{t}} \ac({X})$ at every step $t$. These methods may also need an UQ module, which we simply use the same UQ module as in our algorithm, and it  outputs $V(X)$ that measures the the uncertainty of each point $X \in \xpoolj{t}$.
Our first set of baselines are from active learning~\citep{AggarwalKoGuHaPh14}:
\\ % \noindent\textbf{Active Learning Heuristics:} 
\textbf{(1)} 
\textsf{Uncertainty Sampling (Static):}  In this approach, we query the samples for which the model is least certain about. Specifically, we estimate the variance of the latent output $f(X)$ for each $X \in \xpool$ using the UQ module and select the top-$K$ points with the highest uncertainty. \\
\textbf{(2)} \textsf{Uncertainty Sampling (Sequential):} This is a greedy heuristic that sequentially selects the points with the highest uncertainty within a batch, while updating the posterior beliefs using pseudo labels from the current posterior state. Unlike \textsf{Uncertainty Sampling (Static)}, this method takes into account the information gained from each point within batch, and hence tries to diversify the selected points within a batch. 

 
We also compare our approach to the  \textbf{(3)} \textsf{Random Sampling}, which selects each batch uniformly at random from the pool. Additionally, we compare solving the planning problem using  \textsf{REINFORCE}-based policy gradients with   $\mathsf{Smoothed\text{-}Autodiff}$ policy gradients.\footnote{Our code repository is available at
  \url{https://github.com/namkoong-lab/adaptive-labeling}.}
%Detailed experimental setups are provided in Section \ref{sec:details-experiments}.

%We repeat all experiments with 10 random seeds.




\begin{figure}[t]
\centering
\begin{minipage}[b]{0.49\textwidth}
\centering
\includegraphics[width=\textwidth, height=5cm]{figures/original_scale/Var_of_l_2_loss.pdf}
\caption{(Synthetic data) Variance of mean squared loss evaluated through the posterior belief $\mu_t$ at each horizon $t$. This is the objective that policy gradient methods like \textsf{REINFORCE} and $\ouralgo$ optimizes. 1-step lookaheads are surprisingly effective even in long horizons.}
\label{fig:var-l2-sim}
\end{minipage}
\hfill
\begin{minipage}[b]{0.49\textwidth}
\centering \includegraphics[width=\textwidth, height=5cm]{figures/original_scale/Error_of_estimated_model_l_2_loss.pdf}
\caption{(Synthetic data) Error between MSE calculated based on collected data $\mc{D}^{0:T}$ vs. population oracle MSE over $\mc{D}_{\rm eval} \sim P_X$. Reducing uncertainty over posteriors directly leads to better OOD evaluations. 1-step lookaheads significantly outperform active learning heuristics in small horizons.}
\label{fig:mean-l2-sim}
\end{minipage}
%\caption{Simulated data for GPs}
%\label{fig:both_plots}
\end{figure}

\subsection{Planning with Gaussian processes}
\label{sec:experiment-plan-GP}
We now briefly describe the data generation process for the GP experiments,  deferring a more detailed discussion of the dataset generation to Section~\ref{sec:details-experiments}. 
We use both the synthetic data and the real data to test our methodology.
For the \emph{simulated data},  we construct a setting where the general population is distributed across \emph{51 non-overlapping clusters} while the initial labeled data $\dtrain$ just comes from one cluster. In contrast, both $\dpool \defeq (\xpool,\ypool),\deval \defeq (\xeval,\yeval)$ are generated   from all the clusters. 
We begin with a low-dimensional scenario, generating a one-dimensional regression setting using a GP. %Gaussian Process (GP).
Although the data-generating process is not known to the algorithms,  we assume that the GP hyperparameters are known to all the algorithms
to ensure fair comparisons. This can be viewed as a setting where our prior is well-specified, allowing us to isolate the effects
of different policy optimization approaches
 without any concerns about the misspecified priors. We select $10$ batches, each of size $K=5$ across $T = 10$ time horizons.

To examine the robustness of our method against the distributional assumptions made  in the simulated case, we then move to a real dataset where the correct prior is not known. We simulate selection bias from the eICU dataset~\citep{PollardJoRaCeMaBa18}, which contains real-world patient data with in-hospital mortality outcomes. 
We conduct a $k$-means clustering to generate 51 clusters and then select data from those clusters. We view this to be a credible replication of practice, as severe distribution shifts are common due to selection bias in clinical labels.  To convert the binary mortality labels into a regression setting, we train a  random forest classifier and fit a GP on predicted scores, which serves as the UQ module for all the algorithms. As before, the task is to select 10 batches, each consisting of 5 samples, across 10 time horizons.

 In Figures~\ref{fig:var-l2-sim} and~\ref{fig:mean-l2-sim}, we present results for the simulated data. 
Figure~\ref{fig:var-l2-sim} shows the variance of $\ell_2$ loss, and Figure~\ref{fig:mean-l2-sim} presents the error in the estimated $\ell_2$ loss using $\mu_t$ (relative to true $\ell_2$ loss, that is unknown to the algorithm). 
As we can see from these plots, our method one-step lookahead  gives substantial improvements  over active learning baselines and random sampling. In addition,
compared to the one-step lookahead planning approach using \textsf{REINFORCE}-based policy gradients, 
we observe that $\mathsf{Smoothed\text{-}Autodiff}$-based policy gradients provide significantly more robust performance over all horizons.

In Figures~\ref{fig:var-l2-real}~and~\ref{fig:mean-l2-real}, we observe similar findings on the eICU data. We see that planning policies (\textsf{REINFORCE} and $\mathsf{Smoothed\text{-}Autodiff}$) consistently outperform other heuristics by a large margin.  Active learning baselines perform poorly in these small-horizon batched problems and can sometimes be even worse than the random search baselines.  Overall, our results show the importance of careful planning in adaptive labeling for reliable model evaluation. 

We offer some intuition as to why one-step lookahead planning may outperform other heuristic algorithms. 
 First,  \textsf{Uncertainty sampling (Static)} while myopically selects the
 top-$K$ inputs with the highest uncertainty, it fails to consider 
the overlap in information content among the ``best” instances; see \citep{AggarwalKoGuHaPh14} for more details. 
In other words,  it might acquire points from the same region with high uncertainty while failing to induce diversity among the batch.
Although \textsf{Uncertainty Sampling (Sequential)} somewhat addresses the issue of information overlap, a significant drawback of 
this algorithm
is the disconnect between the objective we aim to optimize and the algorithm. For example, it might sample from a region with high uncertainty but very low density. 

\begin{figure}[t]
\centering
\begin{minipage}[b]{0.48\textwidth}
\centering
\includegraphics[width=\textwidth, height=5cm]{figures/original_scale/Var_of_l_2_loss_real.pdf}
\caption{(Real-world eICU data) Variance of mean squared loss evaluated through the posterior belief $\mu_t$ at each horizon $t$. Even 1-step lookaheads are extremely effective planners, and auto-differentiation-based pathwise policy gradients provide a reliable optimization algorithm based on low-variance gradient estimates.}
\label{fig:var-l2-real}
\end{minipage}
\hfill
\begin{minipage}[b]{0.48\textwidth}
\centering \includegraphics[width=\textwidth, height=5cm]{figures/original_scale/Error_of_estimated_model_l_2_loss_real.pdf}
\caption{(Real-world eICU data) Error between MSE calculated based on collected data $\mc{D}^{0:T}$ vs. population oracle MSE over $\mc{D}_{\rm eval} \sim P_X$. Reducing uncertainty over posteriors directly leads to better OOD evaluations. Our method significantly outperforms active learning-based heuristics, and random sampling.}
\label{fig:mean-l2-real}
\end{minipage}
%\caption{Real data for GPs}
\end{figure}
 
%\vspace{-1.5cm}
% \begin{wrapfigure}{r}{.32\columnwidth}
%   \vspace{-.5cm} 
%   \centering
% \includegraphics[scale=.29]{figures/Var of l2l_2 loss.pdf}
%   \vspace{-0.2cm}
%   \caption{Results of GP}
% \label{fig:var-l2-gp}
%   \vspace{-0.1cm}
% \end{wrapfigure}


% Attempts have been made  in the past to address these  drawbacks heuristically  (see \citep{AggarwalKoGuHaPh14}). We give a unified computational framework while approaching the problem in a more principled manner and solving it more optimally.




\subsection{Planning with  neural network-based uncertainty quantification methods ($\ensembleplus$)}


We now provide a proof-of-concept that shows the generalizability of our conceptual framework  to the deep learning-based UQ modules, specifically focusing on $\ensembleplus$ due to their previously observed superior performance~\citep{OsbandWenAsDwIbLuRo23}. Recall that implementing our framework with deep learning-based UQ modules  requires us to retrain the model across multiple possible random actions $\bm{a}(\theta)$ sampled from the current policy $\pi_\theta$.
This requires significant computational resources, in sharp contrast to the GPs where the posteriors are in closed form and can be readily updated and differentiated. 

Due to the computational constraints, we test $\ensembleplus$ on a toy setting to demonstrate the generalizability of our framework. We consider a setting where the general population consists of four clusters, while the initial labeled data only comes from one cluster. Again we generate data using GPs.  The task is to select a batch of 2 points in one horizon. We detail the $\ensembleplus$ architecture in Section \ref{sec:details-experiments}, and we assume prior uncertainty to be large (depends on the scaling of the prior generating functions). 
The results are summarized in the Table~\ref{tab:UQ_ensemble}.

% \begin{table}[H]
% \vspace{-10pt}
% \caption{Performance under \ensembleplus as UQ module}
%     \centering
%     \begin{tabular}{|m{3cm}|m{2.5cm}|m{2cm}|} 
%     \hline
%       Algorithm   & Variance of $\loss_2$ loss estimate & Error of $\loss_2$ loss estimate  \\ \hline Random Sampling 
%          & $1710.9 \pm 1352.1$ & $8.67\pm6.62$ 
%       \\ \hline \ouralgo & $1.30 \pm 0.68$ & $0.91\pm0.25$ \\ \hline
%     \end{tabular}
%     \label{tab:UQ_ensemble}
%     %\vspace{-10pt}
% \end{table}




\begin{table}[h]
\vspace{-10pt}
\caption{Performance under \ensembleplus as the UQ module}
\centering
\begin{tabular}{|l|l|l|}
\hline
Algorithm   & Variance of $\loss_2$ loss estimate & Error of $\loss_2$ loss estimate  \\
\hline
\textsf{Random sampling} & 7129.8 $\pm$ 1027.0 & 136.2 $\pm$ 8.28 \\ \hline
\textsf{Uncertainty sampling (Static)} & 10852 $\pm$ 0.0 & 162.156 $\pm$ 0.0 \\ \hline
\textsf{Uncertainty sampling (Sequential)} & 8585.5 $\pm$ 898.9 & 144 $\pm$ 6.93 \\ \hline
\textsf{REINFORCE} & 1697.1 $\pm$ 0.0 & 45.27 $\pm$ 0.0 \\ \hline
\ouralgo & 1697.1 $\pm$ 0.0 & 45.27 $\pm$ 0.0 \\ \hline
\end{tabular}
%\caption{Comparison of different algorithms based on variance   and   error in $\ell_2$ loss estimation with Ensemble $+$ as the UQ module. Our results demonstrate that {\ouralgo} and REINFORCE outperformthe other active learning based heuristics, confirming the benefits of our MDP formulation for the adaptive labeling problem, as also demonstrated in Section 4.\\
%\footnotesize{Experimental details: We use Gaussian Processes as our data generating process, GP parameters are the same as in Section D.3.  The task is to select a batch of 2 points along one horizon.The marginal distribution $p_X$ has 4 \textit{non-overlapping} clusters. Initial data comes from one cluster, while pool and evaluation points comes from all the clusters. We have $20$ initial labeled data points, $10$ pool points, and $252$ evaluation points.  Training procedures are similar to the one in Section D.3.} }
\label{tab:UQ_ensemble}
\end{table}



% We faced  issues in scaling up these experiments which will be our focus in the future. 





% \begin{itemize}
%     \item Posteriors should be consistent. Two dimensions: even with less training,  
%     \item the inference should be  fast enough
% \end{itemize}


% Potential research directions for uncertainty quantification

% In this section we consider a simple setting We consider a simpler setting and 


% For synthetic dataset generation, we use ...... For real datasets, we use ...... We compare our methodolgy to several baselines ()    This Section is structured as follows:
% \begin{itemize}
%     \item \textbf{GPs, square loss objective} (Section \ref{}): 
%     %the broad aim of the experiments  in this section is to isolate the performance of our methodology without any concerns for the inefficiencies induced due to a mis-specified prior or imperfect posterior inference. To accomplish this we generate synthetic datasets using GPs (detailed later). We use the well specified prior (GPs - with same hyperparameter setting) as our UQ module.   
%      As GPs provide differentaible posterior inference - any errors induced due to imperfect posterior updates are also isolated. We note that under this setting
%      \item In Section\ref{} we demonstrate why our methodology performs better than other baselines - by devising various synthetic experiments ()
%     \item  \textbf{UQ Benchmarking }(Section \ref{}): Before diving into the experiments using $\ensembleplus$ and ENNs,  we showcase our benchmarking experiments in Section \ref{}. We use real datasets We observe that ENNs perform better
%      \item \textbf{Ensemble $+$}, objective: recall, accuracy
%     \item \textbf{ENN}, objective: recall, accuracy
% \end{itemize}




% In Section {}, we test 
% \subsection{Experimental details}

% \begin{itemize}
%     \item UQ methodologies - GPs, ENNs
%     \item Objectives - Recall,  ATE
%     \item Datasets - ATE-synthetic datasets, Recall-synthetic, real datasets
%     \item Baselines - 
%     \begin{itemize}
%         \item Random sampling
%         \item Active learning - Uncertainty based sampling - In regression setting almost all of the 
%         \item Myopic greedy - Greedy Batch based sampling
%         \item Policy Gradient
%     \end{itemize}
    
% \end{itemize}

% \subsection{Experiments}
%     \begin{itemize}
%     \item GPs with square loss
%     \item Benchmarking ENN
%         \item ENNs with ATE
%         \item ENNs with Recall
%     \end{itemize}

% \subsection{Benefits over other algorithms - intuition and experiments}

%Active learning - Myopic greedy / Don't rely on the objective rather some entropy version.


%%% Local Variables:
%%% mode: latex
%%% TeX-master: "main"
%%% End:


%%%%%%%%%%%%%%%%%%%%%%%%%%%%%%%%%%%%%%%%%%%%%%%%%%%%%%%%%%%%%%%%%%%%%%%%

\section{Conclusion Remarks}
This work proposes a RBG graph model for disease spreading via hubs. We study the joint effect of the agent density, hub density, and connection function. The existence of a critical hub density depends only on the boundedness of the support of the connection function, which relates to curbing the traveling distance of individuals. When it comes to dispersion, both the degree distribution and the percolation threshold suggest that increasing dispersion helps spread the disease. The percolation properties of RBG graphs relate to unipartite graphs with modified connection functions. 
An interesting question in this direction is if and when the properties of the RBG graphs can be well represented by unipartite graphs with some modified connection functions. Our conjecture is that for independent connections between different pairs of agents, such representation is unlikely due to the oblivion of the local dependence (present in the RBG models). 
 Another direction is to consider hybrid models where agents may get infected either through common hubs or direct interactions between agents. The former infection mechanism is more centralized than the latter. 

%%%%%%%%%%%%%%%%%%%%%%%%%%%%%%%%%%%%%%%%%%%%%%%%%%%%%%%%%%%%%%%%%%%%%%%%

%\section*{Acknowledgments}
%The work presented in this paper has been carried out in the context of the SORTEDMOBILITY project. This project is supported by the European Commission and funded under the Horizon 2020 ERA-NET Cofund scheme under grant agreement N 875022. Vito Trianni and Leo D'Amato acknowledge partial support by TAILOR, a project funded by EU Horizon 2020 research and innovation program under GA No 952215.

%%%%%%%%%%%%%%%%%%%%%%%%%%%%%%%%%%%%%%%%%%%%%%%%%%%%%%%%%%%%%%%%%%%%%%%%

\bibliographystyle{ijcai-template/named}
\bibliography{src/references}

%%%%%%%%%%%%%%%%%%%%%%%%%%%%%%%%%%%%%%%%%%%%%%%%%%%%%%%%%%%%%%%%%%%%%%%%

\newpage
\appendix
%reset the counter
\setcounter{figure}{0} 
\renewcommand{\thefigure}{S\arabic{figure}}
\renewcommand{\theHfigure}{S\arabic{figure}}

\setcounter{section}{0} 
\renewcommand{\thesection}{S\arabic{section}}
% Custom appendix autoref labels
\preto\appendix{%
   \renewcommand{\sectionautorefname}{Appendix}%
   \renewcommand{\subsectionautorefname}{Appendix Subsection}%
}

\begin{center}
    \makebox[1\width]{
    \begin{tabular}{c}
        \huge \textbf{Supplementary Materials}
    \end{tabular}
    }
\end{center}

\section{Introduction}

The structure of supplementary material is as follows:
Section \ref{sec:data_gen} illustrates the generation process of a problem instance.
Section~\ref{sec:centralised_appr} describes an integer linear-programming formulation of the problem, useful to compute optimal solutions in a centralised way, as a benchmark for decentralised approaches.
Section \ref{sec:suppl_dsa} shows the effect of the activation probability $\alpha$ by running experiments with $\alpha=1$, $0.9$ and $0.7$.

Recall from the main text that an instance of dec-rtRTMP can be compactly represented as a pair of graphs $(\mathcal{G}_I, \mathcal{G}_C)$ where $\mathcal{G}_I = (\mathcal{V}_I, \mathcal{E}_I)$ is the \emph{interaction graph}, i.e. a graph whose nodes $\mathcal{V}_I$ are agents and links $\mathcal{E}_I$ represents neighbouring agents, and $\mathcal{G}_C = (\mathcal{V}_C, \mathcal{E}_C)$ is the constraint graph, i.e. a $n$-partite graph whose nodes $\mathcal{V}_C$ are all the possible paths $d \in \bigcup_{D \in \mathfrak{D}} D$ generated by the agents and the links $\mathcal{E}_C$ denote compatible paths. An example of problem instance is reported in Figure \ref{fig:pi_example}.

\begin{figure}[ht!]
    \centering
    \includegraphics[width=1\linewidth]{assets/imgs/pi_formulation.pdf}
    \caption{An example of problem instance in our DCOP-based formulation of the dec-rtRTMP. 
        \textbf{(A)} Interaction graph $\mathcal{G}_I$.
        \textbf{(B)} Constraint graph $\mathcal{G}_C$.
        \textbf{(C)} Two possible solutions of the problem instance.
        }
    \label{fig:pi_example}
\end{figure}

%Since a path is an hypothesis formulated by the train about future evolution of railway traffic, from here on we refer to it simply as \emph{hypothesis}.


\begin{tikzpicture}[>=stealth,yscale=0.7,xscale=0.9]

\node[draw, ellipse, align=center] (source) at (0,0)
{Set of all \\ Markov kernels};

% P_1
\draw[->]
  (source.30)
  .. controls ++(0.5,0.5) and (2,1.5) .. node[midway, above=0.3cm] {i.i.d.\ $\mathrm{Dir}(\bbeta)$} (3,1.5)
  node[right] (p1) {$P_{1}$};
% P_2
\draw[->]
  (source.5)
  .. controls ++(0.3,0.13) and (2.7,0.4) .. 
  % node[midway, above] {$\mathrm{Dir}(\beta)$} 
  (3,0.4)
  node[right] (p2) {$P_{2}$};
% P_B
\draw[->]
  (source.-30)
  .. controls ++(0.5,-0.5) and (2,-1.5) .. 
  % node[midway, below] {$\mathrm{Dir}(\beta)$} 
  (2.9,-1.5)
  node[right] (pb) {$P_{B}$};
% ellipsis
 \draw[draw=none] (p2) -- (pb) node[midway,align=center, anchor=south, yshift=-7pt] {\vdots};

\draw[->] (p1.east) -- ++(0.7,0) coordinate(arr_end);
\draw[->] (p2.east) -- ++(0.7,0) node[right, anchor=west] () {$\qquad\qquad\dots$};
\draw[->] (pb.east) -- ++(0.7,0) node[right, anchor=west] () {$\qquad\qquad\dots$};
    \node[anchor=base west] at (arr_end) {
      \begin{tikzpicture}[yshift=-1cm,baseline=(left.base)]
        \node[anchor=base] (left) {$x_1, x_2, \dots,$};
        \node[
          % draw=red,  % if you want border and not shading
          fill=red!30,
          dashed,
          rectangle,
          inner sep=4pt,
          right=-1mm of left
        ] (box_k_prev) {$x_{t-k+1}, \dots, x_t$};
        \node[right=-1mm of box_k_prev] (x_tplus1) {$,x_{t+1}$};
        \draw[-latex]
          (box_k_prev.south) to[out=-60, in=-120]
          node[midway, below] {order $k$} 
          (x_tplus1.south);
      \end{tikzpicture}
    };

\end{tikzpicture}


\section{Centralised solution to the coordination problem}\label{sec:centralised_appr}

In this section, we aim at determining if a given problem instance $(\mathcal{G}_I, \mathcal{G}_C)$ admits a solution and, in case it does, we want to compute all possible solutions while identifying the optimal ones.
%\textcolor{blue}{forse conviene specificare che non è possibile farlo in maniera decentralizzata ? }
To accomplish this, we can reformulate the problem as an integer linear programming (ILP) task and utilize widely available software such as CPLEX \footnote{https://www.ibm.com/it-it/products/ilog-cplex-optimization-studio} as solvers.
%
%Given a problem instance $(\mathcal{G}_I, \mathcal{G}_C)$, we would like to know whether it admits a solution or not and, in case it does, what are all the possible solutions and which ones are optimal. A method to achieve this goal is by reformulating it as an integer linear programming (ILP) problem and solve it by means of standard software like CPLEX. 
%
%The output of our decentralised algorithm at convergence is a solution to the problem instance being solved. A way to assess the performance of the our algorithm is to compare the value of the solution found with the value of the optimal solution to the problem being solved. A method to compute the optimal solution of a problem instance is by reformulating it as an integer linear programming (ILP) problem and solve it by means of tools like CPLEX. 
%
Such a centralised approach provides a reference for the evaluation of the decentralised solutions in our experiments, as discussed in the main text. 

%The output of our decentralised algorithm at convergence is a solution to the problem instance being solved. In order to assess the performance of the consensus procedure we must access all the solutions of the problem instance. One way to compute the entire set of solutions of a problem instance is by reformulating the rtRTMP as an integer linear programming (ILP) problem and solve it by means of a solver like CPLEX. Such an approach is no longer decentralised and serves as a baseline for our experiments in section \ref{sec:exps}. 
To define the ILP forumulation, we first introduce the mapping $\alpha : \mathcal{V}_C \mapsto \{1, \ldots, n\}$ associating to each path $d$ the index of the agent it belongs to. Then, we define the binary decision variables $y_d$ as follows:
%
\begin{equation}
\label{eq:binary}
y_d = \left\{
                \begin{array}{ll}
                  1 \quad \mbox{if $v_{\alpha(d)}=d$}     \\
                  0	\quad \mbox{otherwise}\\
                \end{array}
                 \right. \quad \forall d \in \mathcal{V}_C
\end{equation}
where the notation $v_{\alpha(d)}=d$ means that the path $d$ has been assigned to the variable $v_{\alpha(d)}$ controlled by the agent $a_{\alpha(d)}$. 
Recall also that each node $d \in \mathcal{V}_C$ has a value $u(d)$. 
Then, the centralized optimization problem consists in finding the appropriate complete assignment of paths to maximise the following objective function: 
%
\begin{equation}\label{eq:objective}
\max\sum_{d \in \mathcal{V}_C} u(d) \ y_d 
\end{equation}
provided that the following constraints are satisfied:
\begin{align}
\sum_{d \in D_i} y_d = 1,\quad      &\forall i = 1 \ldots n  \label{eq:eq3} \\[10pt] 
\sum_{d^{\prime} \in D_j : (d, d^{\prime}) \in \mathcal{E}_C} y_{d^{\prime}}  \geq y_d \quad &\forall d \in \mathcal{V}_C, \nonumber \\[-15pt] 
                                                                                       &\forall j=1 \ldots n : A_j  \in \mathcal{N}_{\alpha(d)}, j \neq \alpha(d) \label{eq:eq4}
\end{align}
Constraints~(\ref{eq:eq3}) ensure that exactly one path per train is selected (each agent can only assign one value to the variable it controls). Constraints~(\ref{eq:eq4}) state that, if path $d$ is selected and it is associated to agent $\alpha(d)$, then each neighbouring agent in $\mathcal{N}_{\alpha(d)}$ must select a path that is compatible with $d$.

By exploiting CPLEX as solver for this ILP formulation of the agent coordination in the dec-rtRTMP, we are able to find all possible solutions to a given problem instance, and among them, to identify the optimal ones as well.
%and among these to also identify the optimal solution.

\section{DSA}
\label{sec:suppl_dsa}

We run the DSA algorithm with $\alpha=1, 0.9, 0.7$ and evaluate it performance as in the main text. See Figures \ref{fig:suppl_dsa_ranking}, \ref{fig:suppl_dsa_regret} and \ref{fig:suppl_dsa_convTime}.

\begin{figure*}[!t]
    \centering
    \includegraphics[width=\textwidth]{assets/imgs/suppl_dsa_ranking_h.pdf}
    \caption{Ranking of the solutions given by DSA algorithms. Data are grouped by the type of agent, by number of agents $n$ and by minimum number of solutions $n_{sol}$ we require the problem instance to have. Bars with label ``$1$'' represent the fraction of executions that converged to an optimal solution. Bars with label ``$\geq 10$'' represent the fraction of executions that converged to a solution in position greater than 10 in the ranking. Bars with label ``Fail'' represent the fraction of executions that exceeded the upper bound of $10^5$ iterations.}
    \label{fig:suppl_dsa_ranking}
\end{figure*}

\begin{figure*}[!t]
    \centering
    \includegraphics[width=\textwidth]{assets/imgs/suppl_dsa_regret_top3.pdf}
    \caption{Top: Regret (percentage loss from the optimal solution value) of the solutions given by DSA algorithm in position 2 or 3 of the ranking. Data are grouped by type of agent, by number of agents $n$ and by minimum number of solutions $n_{sol}$ we require the problem instance to have.
    Bottom: Average convergence rate to a top-3 solution.}
    \label{fig:suppl_dsa_regret}
\end{figure*}

\begin{figure*}[!t]
    \centering
    \includegraphics[width=\textwidth]{assets/imgs/suppl_dsa_convTime_h.pdf}
    \caption{Distribution of convergence times (in terms of number of iterations) per type of agent on problem instances grouped by number of agents $n$ and by minimum number of solutions $n_{sol}$ we require the problem instance to have. For each of the 100 problem instances characterised by $n$ and $n_{sol}$, we performed $100$ executions of DSA algorithm for each agent type. Each execution has an upper bound of $10^5$ iterations, beyond which it fails.}
    \label{fig:suppl_dsa_convTime}
\end{figure*}



%%%%%%%%%%%%%%%%%%%%%%%%%%%%%%%%%%%%%%%%%%%%%%%%%%%%%%%%%%%%%%%%%%%%%%%%

\end{document}