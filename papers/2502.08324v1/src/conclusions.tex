\section{Conclusions}
\label{sec:conclusions}

In this paper, we presented a decentralised multi-agent coordination algorithm to solve the dec-rtRTMP, a challenging optimisation problem in railway transportation. It involves the efficient management of train movements on a railway network while minimising delay propagation caused by unexpected events such as temporary speed limitations or signal failures. 
%
Our contributions include a formal DCOP formulation of the dec-rtRTMP, a benchmark dataset for evaluating decentralised solvers, and a novel stochastic algorithm for decentralised multi-agent coordination. Through extensive experimentation, we tested three variants of our approach against a classical DSA from DCOP literature and we evaluated the results in terms of solution quality and convergence speed. We found that our algorithm, when using adaptive agents, achieves high-quality solutions, typically ranking within the top-3 solutions, significantly outperforming the classical DSA. Additionally, the algorithm demonstrates robustness in convergence, particularly in scenarios with varying numbers of agents and solution complexity.

Our work builds upon recent research trends that advocate for decentralisation in railway traffic management. Our DCOP-based reformulation of the dec-rtRTMP represents a proposal for an abstract mathematical framework to deal with such problem. However, translating this abstract framework into concrete case studies can be challenging. Depending on the application, one should carefully define some key aspects like the concept of interaction between trains, the concept of compatibility between paths of distinct trains and other operational aspects. Nonetheless, recent work on railway traffic management has demonstrated that similar approaches to decentralised railway traffic management are feasible \cite{mohanty2020flatlandrl,DAMATO2024100427}, paving the way for new developments in this field.

Besides addressing the dec-rtRTMP, the proposed algorithm for DCOP could be applied to other application domains, related to traffic management or to other decentralised coordination problems \cite{9078053,10.5555/3535850.3535969,10.1613/jair.1.16997,picard:hal-03181968,YANG20214505}. The key feature of target problems are the decentralised choice among a set of alternatives, respecting the compatibility of choices among neighbouring agents. A key aspect of the proposed approach is a policy that prioritises binary constraints (e.g., compatibility of value assignments) over unary constraints (e.g., quality of value assignment), as the former determines the ranking of value assignments while the latter is exploited only for choosing among equally-ranked assignments. On the contrary, DSA merges all constraints in a single utility.
%, and we conjecture that this could be a penalising factor resulting in slower convergence, because high utility of respecting the unary constraints may mask the low utility of the binary ones. 
In future work, we will deepen our analyses to understand to what extent the prioritisation of binary constraints over unary ones is beneficial to convergence or detrimental to quality. Additionally, we will formally address the existence of deadlock conditions, to find ways of avoiding them while maximising the coordination ability within a neighbourhood of agents. Finally, we aim at deploying adaptive algorithms that learn the parameters from the outcome of previous coordination rounds.    