\section{Introduction}

In the realm of railway transportation, the \emph{real-time Railway Traffic Management Problem} (rtRTMP) \cite{10.1007/978-3-319-24264-4_45,PelMarPesRod15:ieeetits} is the problem of efficiently coordinating train movements across a railway network to counteract possible knock-on delays caused by traffic perturbations such as train malfunctions, signal failures, or temporary speed limitations. Knock-on delays are due to conflicts, that require external intervention on train paths through rerouting or rescheduling. Traditionally, such interventions have relied on centralised decision-making by human dispatchers, often with limited computational support, and primarily guided by personal experience.

In the academic literature, centralised approaches dominate, employing methods such as Integer Linear Programming (ILP) \cite{Caimi2012,Meng2014,TOLETTI2020100173}, Mixed Integer Linear Programming (MILP) \cite{Luan2020,Fischetti2017,PelMarPesRod15:ieeetits,Tornquist2007,LU2022102622,REYNOLDS2022105719,LeuBonCor:2023}, and graph-based formulations \cite{Corman2010,LamMan15,Mascis02,SamDarCorPar17,Bettinelli2017,Rod07}. However, centralised approaches struggle to scale with increasing network sizes due to computational constraints. These constraints may be overcome by developing decentralised approaches, where decision-making is distributed across individual trains acting as autonomous agents \cite{MarPel20:IEEE}. 

%The decentralised approach, termed the decentralised Railway Traffic Management Problem (dec-rtRTMP), eliminates the need for a central controller by allowing trains to adapt their schedules dynamically based on local interactions. This method scales well for large networks but sacrifices guaranteed optimality in favour of practical, effective solutions. Among existing approaches to the dec-rtRTMP, algorithms like Distributed Stochastic Algorithm (DSA) \cite{fitzpatrick2003distributed,Vanthielen2019}, multi-agent reinforcement learning \cite{Khadilkar,mohanty2020flatlandrl,flatland}, and swarm intelligence \cite{cui2017swarm} have shown promise but face challenges such as slow convergence or difficulty in handling complex scenarios.

Decentralisation potentially scales better to large-scale area networks, possibly at the cost of accepting non-optimal but still effective solutions.
One possible approach to the decentralised rtRTMP (dec-rtRTMP), presented by \cite{Vanthielen2019}, focuses on resolving conflicts individually by adjusting train schedules or routes. In contrast, \cite{shang2018distributed} suggest empowering trains to make individual decisions, optimizing their movements based on observations of preceding trains. Another proposal by \cite{cui2017swarm} introduces swarm intelligence, organizing trains into groups to address common conflicts collectively.
%
In addition, several methods from Artificial Intelligence (AI) have gained attention in this domain, especially following the recent developments in deep neural architectures \cite{JusupTrivellaCorman}. For instance, \cite{Khadilkar} advocates for reinforcement learning (RL), where agents learn from past experiences to make decisions. The Flatland challenges, initiated by European railway managers, have spurred research in this direction \cite{mohanty2020flatlandrl}, offering a simplified railway simulator for testing different machine learning (ML) approaches. However, deploying learning algorithms in such complex environment can be challenging and the lack of  guarantees on the feasibility of the obtained solution together with the black-box nature of these approaches, makes them difficult to accept by stakeholders.

%However, while AI offers promise, challenges remain. ML models for individual decision-making may be difficult to deploy without sufficient data for training.  Multi-agent RL models lack guarantees of always finding feasible solutions and can struggle in complex situations. Moreover, the black-box nature of deep neural approaches makes them difficult to understand and accept by stakeholders. 

An alternative approach consists in merging optimisation-based planning with self-organisation \cite{DAMATO2024100427}. In this approach, individual trains need to agree on possible schedules resulting from local optimisation by interacting with neighbours. 
\cite{DAMATO2024100427} proved the viability of the deployment to real world scenarios of such an hybrid approach to the dec-rtRTMP. More specifically, they studied a small portion of the line connecting Paris and Le Havre, in France, showing that, in case of traffic perturbations, not only the proposed approach is better than following the original timetable, but also it achieves performance comparable to the centralised state of the art. One of the key components is the self-organised process that enables reaching coordination among the agents on a feasible solution. To this end, decentralised consensus protocols \cite{Paxos:Lamport,amirkhani2022consensus} offer a framework for achieving such an agreement among agents without central control. Simple stochastic models, like the Voter Model, are often sufficient for a population to converge on shared opinions \cite{Holley:1975we}. These have been adapted to the dec-rtRTMP as a proof of concept \cite{DAMATO2024100427}, but without providing a clear problem formulation or a characterisation of the expected performance. In this study, we move a crucial step in this direction. 

We reframe the dec-rtRTMP as a Distributed Constraint Optimisation Problem (DCOP) \cite{fioretto2018distributed}, a mathematical framework used to model problems where multiple agents coordinate to find an optimal solution while respecting constraints imposed by their interactions. This perspective provides a principled foundation for designing and analysing decentralised coordination algorithms. Inspired by the literature on DCOP, we propose a novel decentralised multi-agent coordination algorithm that extends the classical Decetralised Stochastic Algorithm (DSA) \cite{fitzpatrick2003distributed}, tailored for solving the dec-rtRTMP. Our method leverages asynchronous local interactions among agents and employs adaptive strategies to efficiently resolve conflicts and optimise train routes and schedules.

Our contributions are summarised as follows:
\begin{itemize}\parsep0pt
    \item We provide a formal DCOP formulation of the dec-rtRTMP.
    \item We build a dataset for benchmarking decentralised solvers of the dec-rtRTMP.
    \item We introduce a novel stochastic algorithm for the solution of the dec-rtRTMP inspired by the DCOP solvers existing in the literature.
\end{itemize}

Despite being framed in the railway traffic management domain, the proposed approach is versatile and can be extended to other multi-agent systems requiring decentralised coordination, such as autonomous vehicle routing \cite{9078053} or inter-satellite coordination and scheduling \cite{picard:hal-03181968,YANG20214505}.

The rest of the paper is organised as follows: in Section \ref{sec:pf}, we describe our DCOP formulation of the dec-rtRTMP. In Section \ref{sec:consensus}, we present our decentralised multi-agent coordination algorithm adopted to solve a given instance of the dec-rtRTMP. In Section \ref{sec:exps}, we present the experimental settings we designed to study the main properties and the results obtained with our approach and compare them to those of a classical DSA algorithm. Finally, Section~\ref{sec:conclusions} concludes the paper with discussions about future research directions.

