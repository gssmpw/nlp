\section{Data Generation}\label{sec:data_gen}

% The dataset contains synthetically generated problem instances. 
To generate a problem instance, we specify (i) the number of agents $n$, (ii) the interaction rate between agents $p_{\operatorname{int}}$, (iii) the maximum number of paths each agent can generate $n_d$, (iv) the minimum number of solutions $n_{sol}$ we want the problem instance to have, and (v) a random seed $\sigma$ since the generation process is not deterministic.

The parameters $n$ and $p_{\operatorname{int}}$ determine the stochastic generation of the interaction graph $\mathcal{G}_I$: $n$ corresponds to the number of nodes of $\mathcal{V}_I$ while $p_{\operatorname{int}}$ corresponds to the probability of an edge between two nodes, and therefore it varies between 0 and 1. $\mathcal{G}_I$ is generated as a connected graph because, when it is not connected, our algorithm can be applied to each connected component independently. To generate $\mathcal{G}_I$ as random connected graph, we operate the following steps: we first create a tree with nodes $\mathcal{V}_I$ and then we add random links between nodes with probability $p_{\operatorname{int}}$.
%is the same as the one for generating a Erd\H{o}s-R\'enyi graph \cite{Erdos:1959:pmd}, except that we enforce the resulting graph to be connected by adding one random edge for each node before adding any other edge.

The parameters $n_d$ and $n_{sol}$ are instead used to generate the constraint graph $\mathcal{G}_C$. Each agent in $\mathcal{G}_I$ generates a random number of paths between $1$ and $n_d$. Each path $d$ is associated with a utility value $u(d)$. We assume that each agent has one preferred path and other less desirable ones. The former has a utility value equal to 1, while the latter have a small utility value equal to 0.1. This choice follows a standard practice for experimentation in multi-alternative decisions in which the less-preferred options are considered distractors with an equally low value~\cite{Reina:2017jl}.
%Possible values for $u_h$ are either $0.1$ or $1$, with the constraint that exactly one path among the generated ones has a value of $1$. The rationale behind this design choice...  } 
The set of paths of all the agents correspond to the nodes $\mathcal{V}_C$ of $\mathcal{G}_C$.
The links in $\mathcal{G}_C$ are inserted according to the following procedure: we first construct $n_{sol}$ different solutions (see below), and the links constituting such solutions are then added to $\mathcal{G}_C$. At this point not all the nodes in $\mathcal{G}_C$ have a link and thus we randomly add one link for each node with degree $0$ to avoid having paths that are not compatible with any other path. When adding random links, we must take into account the interactions in $\mathcal{G}_I$. Indeed, if two agents do not interact, there is no reason to check the compatibility of their paths and thus we only randomly add links between paths of neighbouring agents.

The construction of a solution to the problem is straightforward: it is enough to randomly select one path per agent and then add all the possible links between paths of neighbouring agents in $\mathcal{G}_C$. Note that even if we add $n_{sol}$ solutions to a problem instance, the total number of solutions can be greater than $n_{sol}$ because the union of the links constituting two solutions can generate several additional solutions, especially if the number of nodes in $\mathcal{V}_C$ is small. This means that the total number of solutions in a problem instance cannot be controlled at generation time but must be computed after the generation, in a centralised fashion, by means of the CPLEX solver, as described in Section~\ref{sec:centralised_appr}. Figure 1 in the main text shows the distribution of the number of solutions in the problem instances of our dataset.  Note that 
%$n_{sol}$ determines the minimum number of solutions, but not the total number, as shown in Figure~\ref{fig:sol_distr}. Indeed, 
the highest number of solutions is found in instances in which $n$ is not much larger than $n_{sol}$, with instances in the group determined by $n=20$ and $n_{sol}=10$ having up to $10^4$ possible solutions. Instead, when $n$ is much larger than $n_{sol}$ (e.g., $n=100$ and $n_{sol}=10$), only the minimum number of solutions $n_{sol}$ is present.

We generated problem instances by fixing the interaction rate $p_{\operatorname{int}}=0.3$ and the maximum number of paths per train $n_d=8$. These values have been selected empirically to obtain a sufficiently rich topology for $\mathcal{G}_I$ and $\mathcal{G}_C$. Then, we generate problem instances by varying $n \in \{10, 20, 50, 100\}$ and $n_{sol}\in\{3, 5, 10\}$. For each combination, we generate 100 instances by varying the random seed $\sigma\in\{0,\ldots,99\}$. The seed $\sigma$ is  used to make the generation process reproducible. Overall, the total number of problem instances in our dataset is $1200$.\footnote{The dataset will be publicly available with the aim of stimulating research on dec-rtRTMP.}
