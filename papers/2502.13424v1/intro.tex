\remove{%%%%%%%%%% START  REMOVE  %%%%%%
\section*{TODO:}

- re-state lemmas 1 to 5 in the details section using the tool of thms 1 to 4.

- pictures to illustrate techniques

%- table of results (or other visual form to improve comparison)




- what if $\Delta$ is unknown - could we still simulate our algorithm? we could interleave protocols for $\Delta$ being a power of $2$, but first, we need to assure consistency between them, and second, could we determine termination?

- how to compute short schedules? maybe there is a trade-off between short schedule implementable (even in off-line fashion) on beeping network and the number of rounds to compute such a schedule in a distributed way? check radio networks literature

- repeated local broadcast?

- Check carefully best beeping solutions to other problems, so that we could really make a long justified list of problems we improve (not only MIS)

- Prove lower bounds for specific problems, eg. MIS; even $\Delta$ could be good, if it distinguishes from randomized solutions (in MIS, randomized is only $\log^3 n$, without any $\Delta$); I thought I have one, but I found a bug and I need to re-think it; we may also try developing lower bounds for other problems

- Can we find an important problem that does not require $\Delta^2$? In other words, it does not need local broadcast in beeping in order to be solved deterministically? This would nicely contrast with our lower bound $\Delta^2/\log^2 n$ for local broadcast

- Could we improve our lower bound? E.g., eliminate assumption on content-oblivious, and if not entirely possible, we could consider bounded local memory (number of states)? Or extend to other problems, e.g., maximal matching (since we use matching as a tool in realization)?

\dk{OPEN PROBLEMS:

- solving same problems but with adversarially delayed wake-up times -- especially if the diameter is much larger than our best time to solve problems in case all are awaken simultaneously

- how to synchronize the network?

- energy aspects, awake model, etc.

}

- counting problem -- determine the exact number of nodes $n$.

- k-hop neighborhood -- It can be solved in $O(\Delta^{k+1} polylog(n))$ rounds: Solve $(k-1)$-neighborhood in $O(\Delta^{k} polylog(n))$ rounds; now each node has $\Delta^{k-1} \log n$ bits of information to transmit ($\Delta^{k-1}$ IDs, where each ID takes $\log n$ bits) -- let's transmit them using our routine that takes $O(\Delta^2 \polylog n)$ rounds per bit, for a total time complexity $O(\Delta^{k+1} polylog(n))$.

- lower bounds for k-neighborhood

- generalize our local broadcast lower bound to other problems, such as finding neighborhood

- deterministic convergecast on a tree in beeping model -- $O(n\cdot \Delta)$ rounds? Could we do just $O(n+\Delta+D)$ rounds instead?

- coloring, 2-hop MIS, other specific problems

- even weak lower bounds such as $\Omega(\Delta)$ on some problems, such as MIS

- lower bound of $\Omega(\Delta)$ for Learning neighborhood based on Davies lower bound (Lemma 14 in Davies' paper); write it down!

- multihop version of Davies lower bound (Lemma 14 in Davies' paper)? Write it down! Both for Local Broadcast and for sending different messages to different k-hop neighbors.

- algorithm for multihop Learning neighborhood via flooding (repeated use of Local Broadcast); we can also compute the shortest paths to each node within k hops.

- computing local aggregation functions, e.g., computing the OR of my neighbors inputs can be done in 1 round. Computing AND is also done in 1 round. XOR may require $\Delta$ rounds. Prove that XOR requires $\Delta$ rounds? Can there be problems that require more than 1 round but less than $\Delta$ rounds?

- learning triangles and cycles of length $4$ and so on

- CONGESTed Clique model and MPC model (Massive Parallel Computation)

- sleeping model - implementing such algorithm, in general or specific ones, could allow us to use more sparse communication or pipeline, because such algorithms use much smaller number of communication rounds per node

% - different wake-up times -- can we use Darek's asynchronous selectors?


}%%%%%%%%%%%%%%  END  REMOVE  %%%%%%%%%%


\section{Introduction}

The study of the Beeping Networks (BN) model is simultaneously interesting, challenging, and useful in several respects. 
Even 
%less restrictive ad-hoc multiple (e.g., wireless) networks 
multiple less-restrictive ad-hoc networks (e.g., wireless) 
are frequently the algorithmist delight: a seemingly simple computational problem that becomes challenging under harsh yet realistic conditions.  In terms of communication, Beeping Networks~\cite{cornejo2010deploying} may be the harshest model: network nodes can only \emph{beep} (emit a signal) or \emph{listen} (detect if a signal is emitted in its vicinity). A listening node may distinguish between \emph{silence} (no beeps) and \emph{noise} (at least one beep), but it cannot distinguish between a single beep and multiple. Theoretically, the Beeping model is important since it enables one to study whether distributed tasks can be performed efficiently under minimal conditions,
 \mam{and it is a fundamental model to study communication complexity on channels where signals are superimposed (by applying the OR operator), which makes it more challenging than the basic model in which transmissions on links are independent. Moreover, in a one-hop network,\footnote{\mam{For any given path of links connecting two nodes, the number of \emph{hops} is the number of links in such path.}} \dk{non-adaptive communication schedules in the Beeping model are} equivalent to superimposed codes, which have been widely applied not only to communication problems, but also to text alignments, bio-computing, data pooling, dimensionality reduction, and other~areas.}
%\sk{Beside the wireless networks motivation, Beeping Networks are also motivated by Biologically Inspired Distributed Algorithms; for example, a node may model a body cell, and a beep may be some molecule secreted by neighboring cells \cite{afek2013beeping,navlakhadistributed}.}

%Algorithms designed for Beeping Networks are useful for massive networks because they can be implemented on non-expensive devices with low energy consumption. For example, on Internet of Things (IoT) applications such as Sensor Networks where a massive number of nodes are deployed for monitoring purposes. 
%The model is also useful for studying natural communication in biological networks~\cite{navlakha2014distributed,afek2011biological} (e.g., fireflies or cells). \mm{Indeed, better understanding the power of biological communication processes may lead to better nature-inspired algorithms~\cite{MooreNeuron24}.} 
%The model 

 

In addition to its theoretical importance, 
this model is recognized as useful for studying natural communication in biological networks~\cite{navlakha2014distributed,afek2011biological} (e.g., 
%fireflies or 
cells). A better understanding of the power of biological communication processes may lead to better nature-inspired algorithms~\cite{MooreNeuron24}.  

It has been argued that the model is also a practical tool because algorithms designed for Beeping Networks can be implemented on non-expensive devices with low energy consumption. For example, nodes are deployed (in ad-hoc topologies) for monitoring in Internet of Things (IoT) applications such as Sensor Networks. It should be noted that a mechanism similar in properties to beeping, called ``busy tone,'' has been in wide use in wireless networks but for very limited purposes in channel access algorithms (as opposed to the Beeping model that is intended for general purposes distributed algorithms). See, e.g., \cite{tobagi1975packet,haas2002dual}.  

 



%\mm{A restricted communication model such as Beeping Networks yields}
%However, such a restricted communication model makes 
%the task of designing algorithms very challenging. 
%\sk{I moved this comment of very challenging to the first paragraph. I think this is the main theme there.}

A wealth of successful research on computational problems in Beeping Networks has appeared in the literature (see, for instance \cite{beauquier2018fast,dufoulon2022beeping} and the references therein). Nevertheless, fundamental distributed computing questions 
%in distributed computing 
remain open in the context of Beeping Networks, especially for deterministic algorithms. 
On the other hand, the \congest ~\cite{peleg2000distributed} model has been profusely studied.
%
%The \congest model is a synchronous, message-passing model of communication where, in each round, nodes are limited to send a (possibly different) message of logarithmic size to each neighboring node.
%
A natural question that follows is how to efficiently transform \congest Network algorithms into Beepping Network algorithms. 

 

%and Maximal Independent Set. 
%In the following sub-section, we detail our positive and negative results. 

\vspace*{-1ex}
\subsection{Our Contributions}

%\todo{We refer to "beeping rounds" and "\congest rounds", but these models were barely mentioned before. Is this fine? SOLUTION: Write a short paragraph about models in the introduction before Our Contributions.}

%\todo{Should we present the results in a more concise form? Such as a table? Or a summary of results in bullet points and the end of the section or at the beginning?}

%\todo{Redact this and add before the first paragraph: "We develop tools: Learning neighborhood, .... Then, we use those tools to obtain one of our main results, beeping simulation of \congest round."}

% \todo{Should we switch the order between Our Contributions and Related Work?}
\vspace*{-0.2ex}
The results are detailed and compared to previous work in Table \ref{table:relwork}.
Our main contributions are the near optimal\footnote{Optimal up to a polylogarithmic factor.} deterministic implementations of two simulators.
The main simulator efficiently translates any \congest algorithms \mam{(deterministic or randomized)} to the Beeping model.
Each \congest round is simulated using $O(\Delta ^2 \polylog n \log \Delta )$ rounds, improving the previous result of $O(\Delta^4 \log n)$ of \cite{beauquier2018fast}.
The other simulator translates (more efficiently -
$O(B\Delta^2 \log n)$ for sending $B$ bits)
the more specialized
\emph{local broadcast} operation 
(called ``broadcast \congest'' in \cite{davies2023optimal}) 
%\mm{[[MM: should we call it instead Broadcast \congest simulation as in Davies paper? Either way, we have to point out that simulating Broadcast CONGEST is the same as \emph{concurrent} local broadcast, which is what our algorithm does, or otherwise the comparison with Davies upper bound may be confusing]]} 
where, whenever a node sends a message to its neighbors, it sends all of them the same message. It seems interesting that for the specialized simulator, we managed to employ non-adaptive ``Beeping codes'', 
i.e.,
%that is, 
not needing to know the topology nor to adapt an action based on beeps heard from the channel; \dk{as such, its performance almost matches a lower bound $\Omega(\Delta^2\log_\Delta n)$ for {\em non-adaptive} algorithms~\cite{CLEMENTI2003337}.} 
%(It, in \mam{[[something's wrong here, anyway, do we need this parenthesis? if so, we should refer to sections for the other results as well]]} Sections \ref{sec:primitives} with the details in Section  \ref{sec:prim_details}.)
In contrast, for the more general simulator, we had to develop a technique that may be of interest in itself. We started with more complex codes and added a hierarchical structure. On top of that, we developed an adaptive algorithm that employs a novel three-stage handshake to distinguish between a genuine message and a message the adversary may have composed by colliding~several~transmissions.
  
 % we had to construct an adaptive algorithm using a techniq more complex codes
%  (puting a hifor the general simulator.
  %
% The construction is built hierarchically using the known family of so-called ``avoiding selectors''. On top of the above (intuitively, saying ``when to beep'',) we also add a 
%taken for different parameters. 
 %3-stage adaptive %acknowledgement 
%handshake
%procedure (announcing, responding and confirming) on the top of the code. 
%which is an adaptive part. 
%This procedure is tightly correlated with the hierarchical structure of the code -- higher level code triggers announcing, while the lower level codes trigger responding and confirming.}
We demonstrate the usefulness of the general simulator by improving the known results for the problem most heavily investigated under the Beeping model, namely, the Maximal Independent Set (MIS). We also demonstrate the use of the more specialized simulator (plus some additional machinery) to construct efficient algorithms for other common building blocks of algorithms, such as a node learning its neighborhood (in $O(\Delta^2 \log^2 n)$ beeping rounds), 
learning the whole cluster  ($O(\Delta^2 \log^4 n)$ for clusters of depth $O(\log ^2 n)$) for all the clusters in parallel, 
and Network Decomposition ($O(\Delta^2 \log^8 n)$). 

 

Another set of results concerns whether a gap exists between the complexity of randomized and deterministic algorithms. For the task of a general simulation of \congest algorithms, the time taken by our \emph{deterministic} simulator 
%is 
%
($O(\Delta ^2 \polylog n \log \Delta )$),  
matches the known lower bound ($\Omega(\Delta^2 \log n)$~\cite{davies2023optimal}) up to a poly-logarithmic factor. 
%
The proof of the $\Omega(\Delta^2 \log n)$ lower bound in~\cite{davies2023optimal} is based on requiring each node to transmit a different string of bits to each of its neighbors.\footnote{This problem is 
%in fact 
named local broadcast 
%for the lower bound 
in~\cite{davies2023optimal}.
However, 
if a different message is sent to each neighbor, the reason for calling it broadcast is unclear. 
We reserve the name local broadcast for the classic problem of sending the {\em same message to~all~neighbors}.} If the string of bits could be the same, that is, the much simpler problem known as local broadcast, that lower bound would collapse to $\Omega(\Delta \log n)$.
\dk{Both lower bounds also hold for randomized solutions, which implies no (substantial) gap between randomness and determinism in the case of general \congest simulation algorithms.}
%
\remove{??? On the other hand, for local broadcast, we show a lower bound of $\Omega\left(\min\left\{n,\Delta^2/\log^2 n\right\}\right)$ for deterministic \mam{non-adaptive content-oblivious} beeping algorithms, which to the best of our knowledge is the first (nearly) quadratic \mam{local broadcast (same message for all neighbors)} lower bound in the Beeping model.
%
The complexity of randomized local broadcast is much better - $O(\Delta \log n)$~\cite{davies2023optimal}, establishing a gap for algorithms that use only local broadcasts. (Note that local broadcasts are a common primitive in various models). ???}



% Please add the following required packages to your document preamble:
% \usepackage[table,xcdraw]{xcolor}
% If you use beamer only pass "xcolor=table" option, i.e. \documentclass[xcolor=table]{beamer}
\begin{table}[]
\begin{tabular}{lllllll}
                                                                                                                     & Storyline Development Stage &                       & Character Design Stage &                        & Character Drawing Stage &                       \\
\multicolumn{7}{c}{\cellcolor[HTML]{D9D9D9}Needs \& Challenges}                                                                                                                                                                                                                \\
                                                                                                                     & Needs                       & Challenges            & Needs                  & Challenges             & Needs                   & Challenges            \\
gauge reader reactions                                                                                               & \multicolumn{1}{r}{3}       &                       &                        & \multicolumn{1}{r}{3}  &                         &                       \\
organize the story flow                                                                                              & \multicolumn{1}{r}{3}       & \multicolumn{1}{r}{4} &                        &                        &                         &                       \\
find interesting story sources that could captivate readers’ interest                                                & \multicolumn{1}{r}{2}       & \multicolumn{1}{r}{4} &                        &                        &                         &                       \\
struggled with creating a story                                                                                      &                             & \multicolumn{1}{r}{8} &                        &                        &                         &                       \\
maintaining a consistent storyline regardless of their condition                                                     &                             & \multicolumn{1}{r}{1} &                        &                        &                         &                       \\
get feedback from readers                                                                                            &                             &                       & \multicolumn{1}{r}{2}  &                        &                         &                       \\
get specific design ideas                                                                                            &                             &                       & \multicolumn{1}{r}{1}  &                        &                         &                       \\
uniqueness of their characters                                                                                       &                             &                       & \multicolumn{1}{r}{4}  & \multicolumn{1}{r}{10} &                         &                       \\
\begin{tabular}[c]{@{}l@{}}enhance the efficiency in their\\ work process by improving repetitive tasks\end{tabular} &                             &                       &                        &                        & \multicolumn{1}{r}{7}   &                       \\
receive materials related to composition                                                                             &                             &                       &                        &                        & \multicolumn{1}{r}{4}   &                       \\
provided to research reference materials                                                                             &                             &                       &                        &                        & \multicolumn{1}{r}{1}   &                       \\
struggled with drawing characters in a variety of compositions                                                       &                             &                       &                        &                        &                         & \multicolumn{1}{r}{6} \\
limited drawinig skills                                                                                              &                             &                       &                        &                        &                         & \multicolumn{1}{r}{6} \\
get information that could be used as reference material                                                             & \multicolumn{1}{r}{2}       &                       & \multicolumn{1}{r}{6}  & \multicolumn{1}{r}{2}  & \multicolumn{1}{r}{1}   &                       \\
\multicolumn{7}{c}{\cellcolor[HTML]{D9D9D9}Expectations}                                                                                                                                                                                                                       \\
Expectations for Generating Ideas                                                                                    & \multicolumn{1}{r}{9}       &                       & \multicolumn{1}{r}{11} &                        & \multicolumn{1}{r}{1}   &                       \\
Expectations for Receiving References from AI                                                                        & \multicolumn{1}{r}{4}       &                       & \multicolumn{1}{r}{0}  &                        & \multicolumn{1}{r}{5}   &                       \\
Expectations of rapid visualization using AI                                                                         & \multicolumn{1}{r}{0}       &                       & \multicolumn{1}{r}{2}  &                        & \multicolumn{1}{r}{2}   &                       \\
\multicolumn{7}{c}{\cellcolor[HTML]{D9D9D9}Considerations}                                                                                                                                                                                                                     \\
Skepticism Regarding AI Capabilities                                                                                 & \multicolumn{1}{r}{2}       &                       & \multicolumn{1}{r}{3}  &                        & \multicolumn{1}{r}{0}   &                       \\
Refusal to Use AI Due to Copyright Infringement and Concerns About the Author's Identity                             & \multicolumn{1}{r}{3}       &                       & \multicolumn{1}{r}{1}  &                        & \multicolumn{1}{r}{2}   &                       \\
Low Utility of AI Outputs                                                                                            & \multicolumn{1}{r}{3}       &                       & \multicolumn{1}{r}{1}  &                        & \multicolumn{1}{r}{1}   &                      
\end{tabular}
\end{table}



\mam{
An interesting generalization is to extend the notion of neighborhood to $h\geq 1$ hops. For the case where each node has a possibly different $B$-bit message to deliver to all nodes in its $h$-hop neighborhood, a problem that we call \emph{$B$-bit $h$-hop simulation}, we show bounds $O(h\cdot B\Delta^{h+2} \polylog n)$ and $\Omega(B\Delta^{h+1})$, 
%even using randomization. 
where the lower bound holds also for randomized solutions. Our algorithm, on the other hand, efficiently ``pipelines'' point-to-point messages, and achieves substantially better complexity (for $h>3$) than a straightforward application of 1-hop simulator $h$ times (which would need $O(\Delta^{2h}\polylog n)$ time).
For the simpler problem when each node has to deliver the same message to all nodes in its $h$-hop neighborhood, which we call \emph{$B$-bit $h$-hop Local Broadcast}, we show a lower bound of $\Omega(B\Delta^h)$ even with randomization. 
}

\remove{

\subsection{Main theorems proven in this paper}


%\mm{The main result is
%a protocol that simulates 
%any round of communication of a \congest Network algorithm in the Beeping Network. 
%Our protocol is optimal, modulo some poly-logarithmic factor. 



%To use them as building blocks of our simulator,
%we also study several fundamental problems in Beeping Networks, which are of independent interest. Namely, Local Broadcast, Neighborhood learning, 
%Cluster Gathering, 
%and Network Decomposition.

%As an example of the myriad of graph problems where our simulation can be applied, we show how to apply our simulation to the Maximal Independent Set (MIS) problem. Formal definitions of all the problems studied are included in Section~\ref{sec:model}.}

\sk{COMMENT TO BE REMOVED: We talked today about shortening 1.1 and moving some text from it to other places. I wrote a much shorter 1.1, moved the formal theorem STATEMENT out of 1.1 to 1.2, and shortened it by putting a lot of text here in a "remove" macro. I think some of the text should go to other places, and some are said twice here and elsewhere.


Please feel free to edit if we can afford it or if necessary. 


Another issue: stating the theorems and corollaries both here and elsewhere seems an unjustifiable overuse of space real estate. I would either remove here altogether (delete section 1.2) or, e.g. in section 4, write "proof of theorem 6", but NOT repeat theorem 6.} 
\mm{[[MM: I agree that section 1.2 is not needed, except for the last paragraph that could be added to section 1.1, and the MIS section that needs to be moved somewhere else.]]}

%\remove{

%\paragraph{\mm{Simulation of \congest Networks Algorithms in Beeping Networks:}}
%\parhead{Main distributed beeping simulation of any \congest round}
%Our %second and 
%main result, described in Section~\ref{sec:main-simulation}, is a {\em deterministic distributed
%beeping 
%simulation of any round} of any \congest algorithm using $O(\Delta^2 \polylog n)$ beeping rounds (each round accommodates at most one beep per node). \mm{We call such algorithm \alg.
%With respect to the lower bound proved in~\cite{davies2023optimal}, our \alg algorithm is optimal modulo a poly-logarithmic factor.}

%The main challenge is that in such a general \congest round, each node can possibly send different messages to different neighbors, and there does not exist a centralized coordination mechanism that could tell a node what its neighbors or their neighbors intend to send. 
%More precisely, 
%\mm{In this context,} we prove the following:
%}  %of remove

%\mm{[[MM:  rather than calling the following "main algorithmic theorem", I would use the restatable command and let the theorem get its number here, it will appear later in the corresponding section with the same number. If you agree, I'll do it.]]}


%\begin{thmA*}[Theorem \ref{thm:congest-sim}, Section~\ref{sec:main-simulation}]
%The main deterministic distributed 
%\mm{Algorithm \alg  deterministically and distributedly} 
%simulates any round of any algorithm designed for the \congest Networks model in $O(\Delta^2 \polylog n \log\Delta)$ beeping rounds. 
%where the $\polylog n$ is the square of the (poly-)logarithm in the construction of avoiding-selectors in Theorem~\ref{thm:avoiding-selectors} multiplied by $\log n$.
%\end{thmA*}

\remove{

%Our beeping schedules combine specific codes, called avoiding selectors, with adaptive mechanisms to encode and decode logarithmic pieces of information at positions where neighboring codes have exclusive 1. 
%\sk{NEIGHBORING CODES? EXCLUSIVE 1?}

\SK{THE FOLLOWING MAY BELONG UNDER A SEPARATE HEADING OF "CHALLENGES", NOT UNDER CONTRIBUTIONS, NOR THEOREM.}
The main challenge is that even if the nodes already learned their neighbors' unique identities (IDs) in the network (using our previously described auxiliary techniques), they still do not know when they have a unique 1 in their code for each neighbor. Intuitively, learning this would require learning 2-hop neighborhoods, but this is inefficient even using our auxiliary methods (which were efficient for learning 1-hop neighborhoods in parallel) and would have resulted in asymptotically $\Delta^4 \polylog n$ beeping rounds.  
Instead, we use general avoiding-selector codes that allow us to ``announce'' some fraction of ``non-realized'' neighbor connections (not yet used) and then realize them using faster codes, i.e., ``avoiding-selectors'' for specific parameters of roughly linear length. Interestingly, such a combination reduces the number of non-realized connections to neighbors by half and requires only $O(\Delta^2 \polylog n)$ beeping rounds. We repeat the above $\log\Delta$ times to get the final result.
} % REMOVE

\paragraph{\mm{Maximal Independent Set (MIS):}}
Applying our \congest round simulation to~\cite{ghaffari2021improved}% 
% \dk{??? and others such as ????}
, we get the following result, which improves polynomially (with respect to $\Delta$) the best-known solutions 
for MIS (c.f. \cite{beauquier2018fast})  but also yield efficient results for many other problems.

%\dk{(c.f.,~\cite{???})}:


\todo{More uses for Network Decomposition! Add corollaries here and citations in Related Work.}

\begin{corollary}
\label{cor:mis}
% Graph problems such as MIS, \dk{????????????} 
MIS can be solved deterministically on any network of maximum node-degree $\Delta$ in $O(\Delta^2 \polylog n)$ beeping rounds.
\end{corollary}

\sk{WE SHOULD HAVE A PLACE IN THE PAPER THAT STATES THAT... I COULD NOT FIND IT}
\sk{I moved this to the end of the general simulations section.}

%\parhead{Lower bound on Local Broadcast}
\paragraph{\mm{Lower Bounds:}} (See Section~\ref{sec:lower-bound}.)
\sk{wE SAID WE MAY MOVE THE LOWER BOUND DISCUSSION TO THE LOWER BOUND SECTION. \\
i SUGGEST TO CLARIFY WHAT DOES IT MEAN THE "REALIZATION OF A LINK" THAT APPEARS IN THE DISCUSSION (THOUGH I CAN GUESS WHAT IT IS)}
\remove {
We also prove a lower bound for Local Broadcast in the Beeping model in Section~\ref{sec:lower-bound}. To do that, we focus on the Radio Network model with collision detection, which is stronger than the Beeping Network model in the sense that in the beeping model, a listening node is not able to distinguish between a single beep and multiple beeps, whereas in the former, nodes can detect collisions and recognize a uniquely transmitting neighbor (see more details in Section~\ref{sec:model}). Thus, our lower bound for Radio Networks with collision detection also applies to the Beeping model. Our lower bound applies to deterministic adaptive Local Broadcast protocols for Radio Networks with collision detection that are \emph{content-oblivious}, that is, protocols where the content of the messages received is not used to decide future transmissions. \mm{Given that in Beeping Networks the information conveyed by beeps is the same as channel feedback (silence or noise), the lower bound proved applies to general protocols for Beeping Networks.}
%
Specifically, we prove the following. 
} %remove
\remove{
\begin{thmL*}[Theorem \ref{thm:LBlower}] 
Consider any deterministic content-oblivious adaptive protocol $\cP$ that solves Local Broadcast in a Radio Network with collision detection. Let $\tau$ be the number of rounds needed by $\cP$ in the worst case. Then, for each $\cP$, there exists an adversarial input network with maximum degree $\Delta$ such that  
$$\tau\in\Omega\left(\min\left\{n,\frac{\Delta^2}{\log^2 n}\right\}\right).$$
\end{thmL*}
}

\begin{restatable}[]{corollary}{LBlowerbound} 
\label{cor:LBlower}
Consider any deterministic adaptive protocol $\cP$ that solves Local Broadcast in a Beeping Network. Let $\tau$ be the number of beeping rounds needed by $\cP$ in the worst case. Then, for each $\cP$, there exists an adversarial input network with maximum degree $\Delta$ such that  
$$\tau\in\Omega\left(\min\left\{n,\frac{\Delta^2}{\log^2 n}\right\}\right).$$
\end{restatable}

\remove{
It follows that any Beeping Networks algorithm that uses, a local broadcast as a subroutine (e.g., simulating algorithms from other message-passing models, including \congest) requires 
%$\Omega\left(\min\left\{n,\Delta^2/\log^2 n\right\}\right)$ rounds.
the~same~time.
}
\remove{
Our proof of the lower bound relies on the novel techniques of choosing an arbitrary matching and designing a network that postpones the realization of some edges of the matching by a given local broadcast algorithm as much as possible. Network construction starts with a particular random graph and simulates consecutive rounds of the local broadcast algorithm. During the simulation, additional edges can be added to the network to delay the realization of some links in the matching. Still, at the same time, they do not change the beeping feedback (or, more generally, radio network feedback) received by nodes at all previously considered simulation rounds. 
}

%\mm{
%\parhead{Learning neighborhood, Local Broadcast, and other auxiliary primitives}
%\paragraph{Local Broadcast, Learning Neighborhood, Cluster Gathering, and Network Decomposition:}
%Our first auxiliary results show how to implement \congest rounds in the beeping model and can be found in 
%=\paragraph {Implementing frequently used  and well known building blocks}
%(See Section~\ref{sec:primitives} and Section~\ref{sec:decomposition}.)

%We also show in Section~\ref{sec:primitives} how to implement some auxiliary primitives in the Beeping Networks setting frequently used as building blocks of more complex tasks. The studied primitives are fundamental and well-known in the context of distributed computing. %Hence, our solutions to these problems are of independent interest. 
%}

%Specifically, we develop a Local Broadcast routine that allows all nodes to broadcast a message of length at most $k$ \mm{bits} to every neighbor concurrently \mm{(even though the problem definition does not require concurrency), taking} $O(k\Delta^2 \log n)$ beeping rounds, for any $k$. 
%We also show how all the nodes can learn their neighborhoods in $O(\Delta^2 \log^2 n)$ beeping rounds.
%Next, we present an algorithm for gathering information from all nodes within a cluster; this is done for all clusters in the graph simultaneously. The procedure takes $O(\Delta^2 \log^4 n)$ beeping rounds.
%\parhead{Network Decomposition}
%We use the above primitives to adapt a network decomposition algorithm from the \congest model to the beeping model in $O(\Delta^2 \log^8 n)$ beeping rounds, see Section~\ref{sec:decomposition}.

%%%%%%%%%%%%%%%%%%%%%%%%%%%%%%%%%%%%%%%%%
}
\remove {
\subsection{Some challenges and technical approaches for the general simulator}

\mm{[[MM:If we keep this section we may want to move back here the description of lower bound technique]]}

In a general \congest round, each node can send different messages to different neighbors. A centralized coordination mechanism does not exist that could tell a node what its neighbors or their neighbors intend to send. 
 
\sk{UNCLEAR: Our beeping schedules combine specific codes, called avoiding selectors, with adaptive mechanisms to encode and decode logarithmic pieces of information at positions where neighboring codes have exclusive 1. } 
\sk{NEIGHBORING CODES? EXCLUSIVE 1?}


\sk{NOT CLEAR TO ME: The main challenge is that even if the nodes already learned the unique identities (ids) of their neighbors in the network (using our previously described auxiliary techniques), they still do not know when they have a unique 1 in their code for each neighbor.} \sk{It is not clear what is a unique 1 and why is it useful. I guess that the idea is (1) that there are rounds as many as the bits in my ID and that when I have 1 in the id bit corresponding to the number of the current slot, I transmit.  this is the meaning of 1 . I think that this meaning is not at all clear to the reader at this point, and (2) a unique 1 means that I am the unique neighbor with 1 at that position of the id. This too , I think is not known to the reader at this point.}


Suppose the nodes use their different identities (IDs) to transmit at different times, distinguishing the messages. For example, suppose that in the $i$th time slot, a node $v$ transmits if it has 1 in the $i$th position of its id. For a neighbor $u$ to notice that a single node ($v$) is transmitted at that time slot, at least the following two conditions must hold. First, $u$ must not be transmitted, that is, have no 1 in that id position. Second, $v$ has to be the only neighbor of $u$ transmitting, that is, the unique neighbor of $u$ who has 1 in that id position. 

Addressing the first condition is relatively easier since we can use our building block (also presented in this paper) of a node to learn its neighbors efficiently. Unfortunately, addressing the second condition seems to require learning 2-hop neighborhoods. This would be inefficient, resulting in asymptotically $\Delta^4 \polylog n$ beeping rounds.  

Instead, we use general avoiding-selector codes that allow us to ``announce'' some fraction of non-realized neighbor connections and then realize them using faster codes, i.e., avoiding-selectors for specific parameters of roughly linear length. Interestingly, such a combination reduces the number of non-realized connections to neighbors by half and requires only $O(\Delta^2 \polylog n)$ beeping rounds. We repeat the above $\log\Delta$ times to get the final result.

\sk{WHAT IS THE MEANING OF (NON) REALIZED CONNECTION AND HOW DOES ONE REALIZE THEM }

}