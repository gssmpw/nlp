\section{Related Work}
\label{sec:relwork}

%\pga{[NOTES: Compare our results to deterministic results and another comparison with randomized results (Davies' paper). DONE
%Mention that Theorem 8 (lower bound) only applies to content-oblivious protocols, but in the beeping model all protocols are content-oblivious. DONE ]}

%A variety of computational problems that are fundamental for distributed computing in communication networks has been studied for Beeping Networks. A summary of the closest related results in comparison with our main results is shown in Table~\ref{table:relwork}.

The BN model was defined by Cornejo and Kuhn in~\cite{cornejo2010deploying} in 2010, inspired by 
continuous beeping studied by Degesys et al.~\cite{degesys2007desync} and Motskin et al.~\cite{motskin2009lightweight}, and by the implementation of coordination by carrier sensing given by Flury and Wattenhofer in~\cite{flury2010slotted}.
%
Since then, the literature has included studies on 
MIS and Coloring~\cite{afek2011biological,afek2013beeping,jeavons2016feedback,holzer2016brief,beauquier2018fast,casteigts2019design}, 
Naming~\cite{chlebus2017naming}, 
Leader Election~\cite{ghaffari2013near,forster2014deterministic,dufoulon2018beeping}, 
Broadcast~\cite{ghaffari2013near,hounkanli2015deterministic,hounkanli2016asynchronous,czumaj2019communicating,beauquier2019optimal},
and Shortest Paths~\cite{dufoulon2022beeping}.

Techniques to implement \congest algorithms in Beeping Networks were studied. In~\cite{beauquier2018fast},
the approach is to schedule transmissions according to a 2-hop $c$-coloring to avoid collisions. The multiplicative overhead introduced by the simulation is in $O(c^2 \log n)$. A constant $c$ is enough for the simulation, but the only coloring algorithm provided in the same paper takes time $O(a^2\Delta^2\log^2 n +a^3 \Delta^3 \log n)$ for a $(\Delta^2+1)$-coloring. Thus, the multiplicative overhead is in $O(\Delta^4 \log n)$. 
%In comparison with our $O(\Delta^2 \polylog n)$ overhead, we improve by a factor of $\Delta^2/\polylog n$.
Thus, with respect to such work, our simulation improves by a factor of $\Delta^2/\polylog n$.

On the side of randomized protocols, in a recent work by Davies~\cite{davies2023optimal}, a protocol that simulates a \congest round in a Beeping Network with $O(\Delta^2\log n)$ overhead is presented. The protocol works even in the presence of random noise in the communication channel. Still, it is correct only with high probability (whp),\footnote{An event $E$ occurs \emph{with high probability} if $Prob(E)\geq 1-1/n^c$ for some $c>0$.} and requires a polynomial number of \congest rounds in the simulated algorithm.   
More relevant for comparison with our work, in the same paper, a lower bound of $\Omega(\Delta^2 \log n)$ on the overhead to simulate a \congest round is shown. The lower bound applies even in a noiseless environment and regardless of randomization. 
Thus, our simulation is optimal modulo some poly-logarithmic factor. 
%\mam{The proof of the lower bound in~\cite{davies2023optimal} is based on requiring each node to transmit a different string of bits to each neighbor. If the string of bits could be the same (a much simpler problem known as local broadcast), that lower bound would collapse to $\Omega(\Delta \log n)$. In this work, we prove a $\Omega\left(\min\left\{n,\Delta^2/\log^2 n\right\}\right)$ lower bound for local broadcast.}
%
Another randomized simulation was previously presented in~\cite{ashkenazi2020brief} with overhead of $O(\Delta\min(n,\Delta^2)\log n)$ whp.


%Another simulation of \congest algorithms in Beeping Networks is presented in~\cite{ashkenazi2020brief}. The multiplicative overhead is $O(\Delta\min(n,\Delta^2)\log n)$, but the simulation succeeds only whp.

With respect to our study on MIS, the closest work is the deterministic protocol presented in~\cite{beauquier2018fast}, which runs in $O(\Delta^2 \log n + \Delta^3)$. Thus, our results improve by a factor of $\Delta/\polylog n$ for any $\Delta\in \omega(\log n)$, and match the running time (modulo poly-logarithmic factor) for $\Delta\in O(\log n)$. 

On the side of randomized MIS protocols, an upper bound of $O(\log^3 n)$ has been shown in~\cite{afek2013beeping} for the same beeping model, and faster with some additional assumptions.

Our network decomposition algorithm for Beeping Networks is heavily based on the protocol for the \congest model~\cite{ghaffari2021improved} that shows a $(\log n, \log^2 n)$ network decomposition in $O(\log^5 n)$ \congest rounds. For comparison, we complete the same network decomposition in $O(\Delta^2 \log^7 n)$ beeping rounds.

Most works in the Beeping Networks literature assume that all nodes start execution simultaneously, called global synchronization. For settings where that is not the case, it is possible to simulate global synchronization in Beeping Networks where nodes start at different times, as shown in~\cite{afek2013beeping,forster2014deterministic,dufoulon2018beeping,hounkanli2020global}.