\vspace*{-1ex}
\section{Initial Results}
\label{sec:primitives}

In this section, we present beeping protocols for four fundamental network problems, usually used as building blocks of more complex tasks. Namely, Local Broadcast, Cluster Gathering, Learning Neighborhood, and Network Decomposition. 
%These protocols will be used as building blocks in our simulation of \congest rounds in beeping networks (Section~\ref{sec:main-simulation}).}
%algorithms that will be used in the network decomposition algorithm, namely, algorithms solving efficiently the problems of learning neighborhood, local broadcast and cluster gathering.
%
The following theorems establish the performance of our protocols. The details of the algorithms as well as the proofs of the theorems are left to Section~\ref{sec:prim_details}.

Recall that IDs of nodes come from the range $[1,n^c]$, for some constant $c \geq 1$.

\begin{restatable}[]{theorem}{localbroadcastthm} %\begin{theorem}
\label{th:local_broadcast}
    Let $\cN$ be a Beeping Network with 
    %set $V$ of 
    $n$ nodes, where each node 
    $v$
    %$v\in V$ 
    knows $n$, parameter~$c$, the maximum degree $\Delta$, and its neighborhood $N(v)$, and holds a message $m_v$ of length at most $B>0$.
    %Then, there is a deterministic distributed local broadcast algorithm that works in $O(k\Delta^2 \log^2 n)$ beeping rounds.
    There is a deterministic distributed algorithm that solves local broadcast on $\cN$ in $O(B\Delta^2 \log n)$ beeping~rounds.
    %\dk{???? SHOULDN'T IT BE $O(\Delta^2 (B+\log n)\log n)$ ????}
%\end{theorem}
\end{restatable}

\vspace*{-1.5ex}
\begin{restatable}[]{theorem}{learningneighthm} %\begin{theorem}
\label{th:learning_neighbourhood}
    Let $\cN$ be a Beeping Network with 
    %set $V$ of 
    $n$ nodes, where each node 
    %$v\in V$ 
    $v$
    knows $n$ and parameter~$c$.
    %There is a deterministic distributed learning-neighborhood algorithm that works in $O(\Delta^2 \log^2 n)$ beeping rounds.
    There is a deterministic distributed algorithm that solves learning neighborhood  on $\cN$ in $O(\Delta^2 \log^2 n)$~beeping~rounds.
%\end{theorem}
\end{restatable}

\vspace*{-1.5ex}
\begin{restatable}[]{theorem}{clustergatherthm} %\begin{theorem}
\label{th:cluster_gathering}
    Let $\cN$ be a Beeping Network with %set $V$ of 
    $n$ nodes, where each node 
    $v$
    %$v\in V$ 
    knows $n$, parameter $c$, the maximum degree $\Delta$, and its neighborhood $N(v)$.
    %There is a deterministic distributed cluster gathering algorithm that works in $O(\Delta^2 \log^4 n)$ beeping rounds.
    There is a deterministic distributed algorithm that solves cluster gathering on $\cN$ in $O(\Delta^2 \log^4 n)$ beeping rounds.
%\end{theorem}
\end{restatable}

% \pga{In the next theorem, we assume that node IDs come from range $[1,n]$.}

\vspace*{-1.5ex}
\begin{restatable}[]{theorem}{networkdecompthm} %\begin{theorem}
\label{thm:local-decomposition}
    Let $\cN$ be a Beeping Network with set $V$ of $n$ nodes, where each node $v\in V$ knows $n$, parameter $c$ and the maximum degree $\Delta$.
    There is a deterministic distributed algorithm that computes a $(\log n, \log^2 n)$-network decomposition of $\cN$ in $O(\Delta^2 \log^8 n)$ beeping rounds.
%\end{theorem}
\end{restatable}

% \pga{[TODO: Check if GGR algorithm can really be adapted to $n^c$ IDs trivially.]}