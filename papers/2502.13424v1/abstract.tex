\remove{
The Beeping Network (BN) model captures important properties of 
biological processes,
for instance when the beeping entity, called \emph{node}, models a cell
 (see Navlakha and Bar-Josef, CACM 2014, and Afek et al., Science 2011).  Perhaps paradoxically, even the fact that the communication capabilities of such nodes are extremely limited has helped 
BN become one of the fundamental models for networks where nodes' transmissions interfere with each other. Since, in each round, a node may transmit at most one bit, 
it is useful to treat the communications in the network as distributed coding and design it to overcome the interference. We study both non-adaptive and adaptive codes. Some communication and graph problems already studied in the Beeping Networks admit fast (i.e., polylogarithmic in the network size $n$) \emph{randomized} algorithms.
On the other hand, all known \emph{deterministic} algorithms for non-trivial problems have time complexity (i.e., the number of beeping rounds, corresponding to the length of the used codes) at least polynomial in the maximum node-degree $\Delta$. 

We improve known results for deterministic algorithms by first showing that this polynomial can be as low as $\tilde{O}(\Delta^2)$. More precisely, we prove that beeping out a single round of any \congest algorithm in any network of maximum node-degree $\Delta$ can be done in $O(\Delta^2 \polylog n)$ beeping rounds, each accommodating at most one beep per node, even if the nodes intend to send different messages to different neighbors. This upper bound reduces the time for a \emph{deterministic} simulation of \congest in a Beeping network to (up to a poly-logarithmic factor) the time obtained recently using \emph{randomization} (see Davies, ACM PODC 2023).
This simulator allows us to implement any efficient algorithm designed for the \congest networks in the Beeping Networks, with $O(\Delta^2 \polylog n)$ overhead. This $O(\Delta^2 \polylog n)$ implementation results in a polynomial improvement upon the best-to-date $\Theta(\Delta^3)$-round beeping MIS algorithm (and of related tasks). Using a more specialized (and thus, more efficient) transformer and some additional machinery,  we constructed various other efficient deterministic Beeping algorithms for various other commonly used building blocks, such as 
 Network Decomposition (seminal in the field of \congest graph algorithms).
\dk{For $h$-hop simulations, we prove a lower bound $\Omega(\Delta^{h+1})$, and we design nearly matching algorithm that is able to ``pipeline'' the information in a faster way than layer-to-layer.}
We also prove that non-adaptive content-oblivious deterministic algorithms that use at least a single local broadcast require $\Omega(\min\{n,\Delta^2/\log^2 n\})$ beeping rounds in some networks of maximum node-degree $\Delta$. This lower bound establishes a gap between randomized and non-adaptive content-oblivious deterministic algorithms for many such tasks (in contrast to the general case where we have shown no gap exists, up to a polylog factor). 
\\
}

The Beeping Network (BN) model captures important properties of 
biological processes, for instance, when the beeping entity, called \emph{node}, models a cell.
 %(see Navlakha and Bar-Josef, CACM 2014, and Afek et al., Science 2011).
 Perhaps paradoxically, even the fact that the communication capabilities of such nodes are extremely limited has helped 
BN become one of the fundamental models for networks where nodes' transmissions interfere with each other. Since in each round, a node may transmit at most one bit, 
it is useful to treat the communications in the network as distributed coding and design it to overcome the interference. We study both non-adaptive and adaptive codes. Some communication and graph problems already studied in the Beeping Networks admit fast (i.e., polylogarithmic in the network size $n$) \emph{randomized} algorithms.
On the other hand, all known \emph{deterministic} algorithms for non-trivial problems have time complexity (i.e., the number of beeping rounds, corresponding to the length of the used codes) at least polynomial in the maximum node-degree $\Delta$. 

We improve known results for deterministic algorithms by first showing that this polynomial can be as low as $\tilde{O}(\Delta^2)$. More precisely, we prove that beeping out a single round of any \congest algorithm in any network of maximum node-degree $\Delta$ can be done in $O(\Delta^2 \polylog n)$ beeping rounds, each accommodating at most one beep per node, even if the nodes intend to send different messages to different neighbors. This upper bound reduces polynomially the time for a \emph{deterministic} simulation of \congest in a Beeping network, comparing to the best known algorithms, and nearly matches the time obtained recently using \emph{randomization} (up to a poly-logarithmic factor). %(see Davies, ACM PODC 2023).
Our simulator allows us to implement any efficient algorithm designed for the \congest networks in the Beeping Networks, with $O(\Delta^2 \polylog n)$ overhead. This $O(\Delta^2 \polylog n)$ implementation results in a polynomial improvement upon the best-to-date $\Theta(\Delta^3)$-round beeping MIS algorithm (and of related tasks). Using a more specialized (and thus, more efficient) transformer and some additional machinery,  we constructed various other efficient deterministic Beeping algorithms for various other commonly used building blocks, such as 
 Network Decomposition (seminal in the field of \congest graph algorithms).
For $h$-hop simulations, we prove a lower bound $\Omega(\Delta^{h+1})$, and we design a nearly matching algorithm that is able to ``pipeline'' the information in a faster way than
 working layer by layer.
%We also prove that non-adaptive content-oblivious deterministic algorithms that use at least a single local broadcast require $\Omega(\min\{n,\Delta^2/\log^2 n\})$ beeping rounds in some networks of maximum node-degree $\Delta$. This lower bound establishes a gap between randomized and non-adaptive content-oblivious deterministic algorithms for many such tasks (in contrast to the general case where we have shown no gap exists, up to a polylog factor). 

\

\noindent
{\bf Keywords:} Beeping Networks, \congest Networks, deterministic simulations, 
%Local Broadcast, 
graph algorithms.

%Single-hop: leader election (Gilber and Newport, DISC 2012?) and weaker synchronization principles

%Multi-hop: MIS and distance $2$ MIS principles 

%Efficient translations of message-passing or population protocols or stone age model~\cite{???}
