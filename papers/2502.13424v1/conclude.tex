\section{Conclusions}

We provided deterministic distributed algorithms to efficiently simulate a round of algorithms designed for the CONGEST model on the Beeping Networks. This allowed us to improve polynomially the time complexity of several (also graph) problems on Beeping  Networks. The first simulation by the Local Broadcast algorithm is shorter by a polylogarithmic factor than the other, more general one -- yet still powerful enough to implement some algorithms, including the prominent solution to Network Decomposition~\cite{ghaffari2021improved}.
The more general one could be used for solving problems such as MIS.
We also considered efficient pipelining of messages via several layers of BN.
%We also proved that our solutions could not be substantially improved if the considered problems require content-oblivious local broadcast, by proving an almost-tight lower bound.

Two important lines of research arise from our work.
First, whether some (graph) problems do not need local broadcast to be solved deterministically, and whether their time complexity could be asymptotically below $\Delta^2$.
Second, could a lower bound on any deterministic local broadcast algorithm, better than $\Omega(\Delta\log n)$, be proved?
%our lower bound be tightened and extended to any, not necessarily content-oblivious \mam{and non-adaptive}, solutions to the Local Broadcast problem?

% \todo{Propose to develop algorithms that work in time depending on the diameter of the network}

% \todo{Discussion of noisy beeping channel.}