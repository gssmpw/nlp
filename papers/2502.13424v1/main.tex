\documentclass{article}

% Language setting
% Replace `english' with e.g. `spanish' to change the document language
\usepackage[english]{babel}
\usepackage{tikz}
\usetikzlibrary{automata, positioning, arrows}

% Set page size and margins
% Replace `letterpaper' with `a4paper' for UK/EU standard size
%\usepackage[letterpaper,top=2cm,bottom=2cm,left=3cm,right=3cm,marginparwidth=1.75cm]{geometry}
\usepackage[letterpaper,top=1in,bottom=1in,left=1in,right=1in]{geometry}

% Useful packages
\usepackage{amsmath}
\usepackage{graphicx}
\usepackage{amsthm}
\usepackage[colorlinks=true, allcolors=blue]{hyperref}
\usepackage{color}
\usepackage[boxed,vlined,ruled,linesnumbered]{algorithm2e}
%\usepackage{algorithm}
%\usepackage{algorithmic}
\usepackage{subcaption}
\usepackage{multirow}
\usepackage{colortbl}

%\newtheorem{theorem}{Theorem}
\newtheorem{hypothesis}{Hypothesis}
\newtheorem{lemma}{Lemma}
\newtheorem{claim}{Claim}
\newtheorem{definition}{Definition}
\newtheorem{corollary}{Corollary}
\newtheorem{fact}{Fact}

\newtheorem*{thmA*}{Main Algorithmic Theorem}
%\newtheorem*{thmL*}{Main Lower Bound Theorem}
\newtheorem*{thmL*}{Lower Bound Theorem}

% re-statable theorems
\usepackage{thmtools, thm-restate}
\declaretheorem{theorem}

% colors
% \newcommand{\lm}[1]{{\color{blue} #1}}
\newcommand{\dk}[1]{{\color{purple} #1}}
% \newcommand{\sk}[1]{{\color{green} #1}}
\newcommand{\pga}[1]{{\color{red} #1}}
\newcommand{\mm}[1]{{\color{cyan}{#1}}}
\newcommand{\mam}[1]{{\color{cyan}{#1}}}
\newcommand{\mamr}[1]{{\color{cyan}{#1}}}
% \newcommand{\pg}[1]{{\color{blue}{#1}}}
\newcommand{\todo}[1]{{\color{red} [TODO: #1]}}
% uncomment to remove colors
\renewcommand{\dk}[1]{#1}
% \renewcommand{\sk}[1]{#1}
\renewcommand{\mm}[1]{#1}
\renewcommand{\mam}[1]{#1}
\renewcommand{\mamr}[1]{#1}
% \renewcommand{\pg}[1]{#1}
\renewcommand{\pga}[1]{#1}
% \renewcommand{\lm}[1]{#1}

\newcommand{\mF}{\mathcal{F}}
\newcommand{\cP}{\mathcal{P}}
\newcommand{\cG}{\mathcal{G}}
\newcommand{\cM}{\mathcal{M}}
\newcommand{\cN}{\mathcal{N}}
\newcommand{\remove}[1]{}
\newcommand{\polylog}{\text{\em\ polylog }}%{\mbox{ polylog } }
%\newcommand{\polylog}{\mathrm{polylog}~}

\usepackage{xspace}
\newcommand{\parhead}[1]{\noindent{\textbf{#1.}\xspace}}
\newcommand{\congest}{{\fontfamily{cmss}\selectfont CONGEST}\xspace}
\newcommand{\alg}{{\sc c2b}\xspace}

%\renewcommand{\paragraph}[1]{\vspace*{-1ex}\textbf{#1}}

\title{Beeping Deterministic \congest Algorithms in Graphs
%Efficiently 
}
%Towards Efficient Deterministic Graph Algorithms\\ in Beeping Networks\\
%{Beeping out \congest Network Algorithms Deterministically}
%{Efficient Algorithms in Beeping Networks\\ via Almost Tight Simulations of \congest Algorithms}
%{Transforming Algorithms from \congest to Beeping Networks\\
%\mm{Simulating \congest Algorithms on Beeping Networks?}}
%{Deterministic Beeping Networks}

%\author{}
%\author{Pawel Garncarek \and Dariusz R. Kowalski \and Shay Kutten \and Miguel A. Mosteiro}
\author{
Pawel Garncarek\thanks{University of Wroclaw, Institute of Computer Science, Wroclaw, Poland; supported by the National Science Center, Poland (NCN), grant 2020/39/B/ST6/03288.} 
\and Dariusz R. Kowalski\thanks{Augusta University, Department of Computer \& Cyber Sciences, Augusta, GA, USA} 
\and Shay Kutten\thanks{Technion, Israel Institute of Technology, Haifa, Israel; a large part of this author's research was performed while he was on a sabbatical at Fraunhofer SIT in Darmstadt. Research supported in part by the Israeli Science Foundation and by The Bernard M. Gordon Center for Systems Engineering at the Technion.} 
\and Miguel A. Mosteiro\thanks{Pace University, Computer Science Department, New York, NY, USA; partially supported by Pace SRC grant and Kenan fund.}
}

\date{}

\begin{document}
\maketitle


%\vspace*{-5.1ex}

\begin{abstract}
\begin{abstract}
Retrieval-Augmented Generation (RAG) is often used with Large Language Models (LLMs) to infuse domain knowledge or user-specific information. In RAG, given a user query, a retriever extracts chunks of relevant text from a knowledge base. These chunks are sent to an LLM as part of the input prompt. Typically, any given chunk is repeatedly retrieved across user questions. However, currently, for every question, attention-layers in LLMs fully compute the key values (KVs) repeatedly for the input chunks, as state-of-the-art methods cannot reuse KV-caches when chunks appear at arbitrary locations with arbitrary contexts. Naive reuse leads to output quality degradation.  This leads to potentially redundant computations on expensive GPUs and increases latency. In this work, we propose \sys, a system for managing and reusing precomputed KVs corresponding to the text chunks (we call \textit{chunk-caches}) in RAG-based systems. We present how to identify \hl{\textit{chunk-caches} that are reusable}, how to efficiently perform a small fraction of recomputation to \textit{fix} the cache to maintain output quality, and how to efficiently store and evict \textit{chunk-caches} in the hardware for maximizing reuse while masking any overheads. With real production workloads as well as synthetic datasets, we show that \sys reduces redundant computation by \textbf{51\%} over SOTA prefix-caching and \textbf{75\%} over full recomputation.
\hl{Additionally, with continuous batching on a real production workload, we get a \textbf{1.6$\times$} speedup in throughput and a \textbf{2$\times$} reduction in end-to-end response latency over prefix-caching while maintaining quality, for both the \llama-3-8B and \llama-3-70B models. 
}
\end{abstract}





\end{abstract}

% to be removed
%% propose BF GDA/SGDA
% introduce adaptive regret notion and non-degenerative populations
%%% theoretical results
% convergence of GDA/SGDA in our setting for exponential policies
% hypotethis that it should work still by clipping the policy side ( then experiments), then argue that it can help for convergence of NN policies (talk about L0-L1 smoothness and past results on that matter, theoretical results left for future work on GDA/SGDA)
% 
%%% experiments
% repeated prisonner's dilemma as a first example to show results on GDA, (with specific or full set of deterministic policies??)
% Show clipping effect (works on higher lr)
% boxplots, learned distributions, learning curve on worst-case regret ?

% more advanced experiments on random POMDPS (provide figure for the env ?)
% show empirically that it still works. clipping as well ?

% final set of experiments on complex environments with NN policies.
% > leduc poker, melting pot
% > cooperation on mujoco tasks (say the "human" controls a part of the robot, and the agent assists)

\thispagestyle{empty}

\newpage

 \setcounter{page}{1}


\section{Introduction}
\label{sec:intro}

\begin{figure*}[tb]
    \centering
    \includegraphics[width=0.848\linewidth]{figs/circuitnn.pdf} 
    \caption{Illustration of differentiable CircuitNN. CircuitNN is designed based on differentiable NAND gates. After DAS is guided by PI and PO pairs of the truth table, CircuitNN can get the precise circuit architecture logic equivalent to the truth table.}
    \label{fig:circuitnn}
\end{figure*}

% 1. Describe the importance of logic synthesis
% 2. Existing Problems
% (a) Neural Architecture Search: Unstable, Predefined Setting, etc.
% (b) Circuit Generation: Probabilistic Model, Logic Equivalence

With the rapid advancement of technology, the scale of integrated circuits (ICs) has expanded exponentially. 
This expansion has introduced significant challenges in chip manufacturing, particularly concerning power and area metrics.
A primary objective in IC design is achieving the same circuit function with fewer transistors, thereby reducing power usage and area occupancy.

Logic synthesis~\cite{hachtel2005logicsynth}, a critical step in electronic design automation (EDA), transforms behavioral-level circuit designs into optimized gate-level circuits, ultimately yielding the final IC layout. 
The primary goal of logic synthesis is to identify the physical implementation with the fewest gates for a given circuit function. 
This task constitutes a challenging NP-hard combinatorial optimization problem. 
Current logic synthesis tools~\cite{brayton2010abc, wolf2013yosys} rely on human-designed heuristics, often leading to sub-optimal outcomes.

Differentiable architecture search (DAS) techniques~\cite{liu2018darts, chu2020darts} offer novel perspectives on addressing challenges in this problem.
Circuit functions can be represented through truth tables, which map binary inputs to their corresponding outputs. 
Truth tables provide a precise representation of input-output relationships, ensuring the design of functionally equivalent circuits.
Inspired by this, researchers~\cite{deepmind2024ai4sys, wang2024tnet} have begun exploring the application of DAS to synthesize circuits directly from truth tables.
Specifically, \citet{deepmind2024ai4sys} proposed CircuitNN, a framework that learns differentiable connection structures with logic gates, enabling the automatic generation of logic circuits from truth tables.
This approach significantly reduces the complexity of traditional circuit generation. 
Building on this, \citet{wang2024tnet} introduced T-Net, a triangle-shaped variant of CircuitNN, incorporating regularization techniques to enhance the efficiency of DAS.

Despite these advancements, several challenges remain. 
The computational complexity of DAS grows quadratically with the number of gates, posing scalability issues.
Although triangle-shaped architecture~\cite{wang2024tnet} partially mitigates this problem, redundancy persists. 
%Additionally, DAS is susceptible to converging to local optima, limiting the ability to search architectures that satisfy the given truth tables~\cite{liu2018darts}. 
%Furthermore, hyperparameters (network depth and layer width) require extensive searches, introducing complexity and prolonging the synthesis process. 
Additionally, DAS is susceptible to converging to local optima~\cite{liu2018darts} and hyperparameters (network depth and layer width) require extensive searches. 
The challenges arise from the vast search space in DAS. 
% Even with predefined settings for CircuitNN, finding a configuration that meets the truth table requires extensive trial and error during the DAS process. 
Intuitively, limiting the search space through predefined parameters (network depth, gates per layer, and connection probabilities) can significantly reduce the complexity.

Recent advances~\cite{openai2023gpt4, abramson2024alphafold3, esser2024sd3, li2024mar} in conditional generative models have demonstrated remarkable performance across language, vision, and graph generation tasks. 
Motivated by these developments, we propose a novel approach to circuit generation that generates preliminary circuit structures to guide DAS in generating refined circuits matching specified truth tables. 
Firstly, we introduce CircuitVQ, a tokenizer with a discrete codebook for circuit tokenization. 
Built upon our Circuit AutoEncoder framework~\cite{hou2022graphmae,li2023maskgae,wu2025mgvga}, CircuitVQ is trained through a circuit reconstruction task. 
Specifically, the CircuitVQ encoder encodes input circuits into discrete tokens using a learnable codebook, while the decoder reconstructs the circuit adjacency matrix based on these tokens.
Subsequently, the CircuitVQ encoder serves as a circuit tokenizer for CircuitAR pretraining, which employs a masked autoregressive modeling paradigm~\cite{chang2022maskgit, li2023mage}. 
In this process, the discrete codes function as supervision signals. 
After training, CircuitAR can generate discrete tokens progressively, which can be decoded into initial circuit structures by the decoder of the CircuitVQ. 
These prior insights can guide DAS in producing refined circuits that match the target truth tables precisely.

Our key contributions can be summarized as follows:
\begin{itemize}
\item We introduce CircuitVQ, a circuit tokenizer that facilitates graph autoregressive modeling for circuit generation, based on our Circuit AutoEncoder framework;
\item Develop CircuitAR, a model trained using masked autoregressive modeling, which generates initial circuit structures conditioned on given truth tables;
\item Propose a refinement framework that integrates differentiable architecture search to produce functionally equivalent circuits guided by target truth tables;
\item Comprehensive experiments demonstrating the scalability and capability emergence of our CircuitAR and the superior performance of the proposed circuit generation approach.
\end{itemize}

% Motivation
% (a) Diffusion (Vision, Graph), Autoregressive (Language, Vision)
% (b) Circuit Generation for Predefined Setting
% (c) Neural Architecture Search for Strict Logic Equivalence

% Contribution
% (a) Circuit Tokenizer (new transformer arch, training strategy)
% (b) CircuitAR (train and gen strategies, post-ar strategy)
% (c) Extensive Evaluation including BitD (Bit Distance) for Scalability

%\begin{table*}[ht]
    \centering
    \caption{Comparison of different algorithms across four datasets based on Test MSE and MAE.\@ 
            $\dagger$ indicates that results are reported from \citeauthor{Zhang.Irregular}~\cite{Zhang.Irregular}.
            Only tPatchGNN, GraFITi and \model{} use tuned hyperparameters. 
            For tPatchGNN we report both our own results as well the (untuned) results from \citeauthor{Zhang.Irregular}
            For the comparison, we focus on our own results, which is why the previously reported performance of tPatchGNN is in small font and parentheses.
            The best model is highlighted in \large \textbf{bold} \normalsize and the second best is \underline{underlined}.  
            }\label{tab:main} 
    \begin{tabular}{l cc cc cc cc}
        \toprule
        Algorithm & \multicolumn{2}{c}{PhysioNet} & \multicolumn{2}{c}{MIMIC} & \multicolumn{2}{c}{Human Activity} & \multicolumn{2}{c}{USHCN} \\ 
        \midrule
        & MSE$\times 10^{-3}$ & MAE$\times 10^{-2}$ & MSE$\times 10^{-2}$ & MAE$\times 10^{-2}$ & MSE$\times 10^{-3}$ & MAE$\times 10^{-2}$ & MSE$\times 10^{-1}$ & MAE$\times 10^{-1}$ \\ 
        \midrule
        DLinear$^\dagger$ & 41.86 ± 0.05 & 15.52 ± 0.03 & 4.90 ± 0.00 & 16.29 ± 0.05 & 4.03 ± 0.01 & 4.21 ± 0.01 & 6.21 ± 0.00 & 3.88 ± 0.02 \\ 
        TimesNet$^\dagger$ & 16.48 ± 0.11 & 6.14 ± 0.03 & 5.88 ± 0.08 & 13.62 ± 0.07 & 3.12 ± 0.01 & 3.56 ± 0.02 & 5.58 ± 0.05 & 3.60 ± 0.04 \\ 
        PatchTST$^\dagger$ & 12.00 ± 0.23 & 6.02 ± 0.14 & 3.78 ± 0.03 & 12.43 ± 0.10 & 4.29 ± 0.14 & 4.80 ± 0.09 & 5.75 ± 0.01 & 3.57 ± 0.02 \\ 
        Crossformer$^\dagger$ & 6.66 ± 0.11 & 4.81 ± 0.11 & 2.65 ± 0.10 & 9.56 ± 0.29 & 4.29 ± 0.20 & 4.89 ± 0.17 & 5.25 ± 0.04 & 3.27 ± 0.09 \\
        Graph Wavenet$^\dagger$  & 6.04 ± 0.28 & 4.41 ± 0.11 & 2.93 ± 0.09 & 10.50 ± 0.15 & 2.89 ± 0.03 & 3.40 ± 0.05 & 5.29 ± 0.04 & 3.16 ± 0.09 \\ 
        MTGNN$^\dagger$ & 6.26 ± 0.18 & 4.46 ± 0.07 & 2.71 ± 0.23 & 9.55 ± 0.65 & 3.03 ± 0.03 & 3.53 ± 0.03 & 5.39 ± 0.05 & 3.34 ± 0.02 \\ 
        StemGNN$^\dagger$ & 6.86 ± 0.28 & 4.76 ± 0.19 & 1.73 ± 0.02 & 7.71 ± 0.11 & 8.81 ± 0.37 & 6.90 ± 0.02 & 5.75 ± 0.09 & 3.40 ± 0.09 \\ 
        CrossGNN$^\dagger$ & 7.22 ± 0.36 & 4.96 ± 0.12 & 2.95 ± 0.16 & 10.82 ± 0.21 & 3.03 ± 0.10 & 3.48 ± 0.08 & 5.66 ± 0.04 & 3.53 ± 0.05 \\ 
        FourierGNN$^\dagger$ & 6.84 ± 0.35 & 4.65 ± 0.12 & 2.55 ± 0.03 & 10.22 ± 0.08 & 2.99 ± 0.02 & 3.42 ± 0.02 & 5.82 ± 0.06 & 3.62 ± 0.07 \\ 
        GRU-D$^\dagger$ & {5.59 ± 0.09} & 4.08 ± 0.05 & 1.76 ± 0.03 & 7.53 ± 0.09 & 2.94 ± 0.05 & 3.53 ± 0.06 & 5.54 ± 0.38 & 3.40 ± 0.28 \\ 
        SeFT$^\dagger$ & 9.22 ± 0.18 & 5.40 ± 0.08 & 1.87 ± 0.01 & 7.84 ± 0.08 & 12.20 ± 0.17 & 8.43 ± 0.07 & 5.80 ± 0.19 & 3.70 ± 0.11 \\ 
        RainDrop$^\dagger$ & 9.82 ± 0.08 & 5.57 ± 0.06 & 1.99 ± 0.03 & 8.27 ± 0.07 & 14.92 ± 0.14 & 9.45 ± 0.05 & 5.78 ± 0.22 & 3.67 ± 0.17 \\ 
        Warpformer$^\dagger$ & 5.94 ± 0.35 & 4.21 ± 0.12 & 1.73 ± 0.04 & 7.58 ± 0.13 & 2.79 ± 0.04 & 3.39 ± 0.03 & 5.25 ± 0.05 & 3.23 ± 0.05 \\ 
        mTAND$^\dagger$ & 6.23 ± 0.24 & 4.51 ± 0.17 & 1.85 ± 0.06 & 7.73 ± 0.13 & 3.22 ± 0.07 & 3.81 ± 0.07 & 5.33 ± 0.05 & 3.26 ± 0.10 \\ 
        Latent-ODE$^\dagger$ & 6.05 ± 0.57 & 4.23 ± 0.26 & 1.89 ± 0.19 & 8.11 ± 0.52 & 3.34 ± 0.11 & 3.94 ± 0.12 & 5.62 ± 0.03 & 3.60 ± 0.12 \\ 
        CRU$^\dagger$ & 8.56 ± 0.26 & 5.16 ± 0.09 & 1.97 ± 0.02 & 7.93 ± 0.19 & 6.97 ± 0.78 & 6.30 ± 0.47 & 6.09 ± 0.17 & 3.54 ± 0.18 \\ 
        Neural Flow$^\dagger$ & 7.20 ± 0.07 & 4.67 ± 0.04 & 1.87 ± 0.05 & 8.03 ± 0.19 & 4.05 ± 0.13 & 4.46 ± 0.09 & 5.35 ± 0.05 & 3.25 ± 0.05 \\ 
        \footnotesize (tPatchGNN$^\dagger$)  & \footnotesize({4.98 ± 0.08}) & \footnotesize(3.72 ± 0.03) & \footnotesize(1.69 ± 0.03) & \footnotesize(7.22 ± 0.09) & \footnotesize({2.66 ± 0.03}) & \footnotesize({3.15 ± 0.02}) & \footnotesize(5.00 ± 0.04) & \footnotesize({3.08 ± 0.04}) \\
        \midrule       
        tPatchGNN & 5.44 ± 0.14 & 3.85 ± 0.24   &  1.33 ± 0.02 & 6.58 ± 0.11 &  2.70  ± 0.06 & 3.18 ± 0.06 & \UL{5.06 ± 0.02} &  \UL{3.11 ± 0.05} \\ 
        GraFITi & \UL{4.91 ± 0.05} & \UL{3.57 ± 0.03} & \BF{1.21 ± 0.01} & \BF{6.19 ± 0.07}& \UL{2.64 ± 0.05} & \UL{3.08 ± 0.01} & {5.17 ± 0.07} & {3.19 ± 0.19}\\ 
        \model{} (ours) & \BF{4.88 ± 0.03} & \BF{3.47 ± 0.01} & \UL{1.25 ± 0.02} & \UL{6.20 ± 0.05} & \BF{2.49 ± 0.01} & \BF{3.06  ± 0.01} & \BF{5.01  ± 0.08} & \BF{3.05 ± 0.03} \\ 
        \bottomrule
    \end{tabular}
\end{table*}

\section{Related Work} \label{sec:related}

% \textbf{Adversarial Attack}
\textbf{Attacks on SLAM.} 
%With the rise of machine learning, 
The robustness of computer vision systems is being actively investigated. With the emergence of adversarial images in the digital domain by adding optimized noise directly to images~\cite{szegedy2013intriguing,carlini2017towards}, researchers find that such attacks also exist physically in the real world \cite{eykholt2018robust,song2018physical,zhao2019seeing}. To fill the gap between attacks in the digital and physical worlds, recent studies have demonstrated that attacks on real-world computer vision systems are practical \cite{eykholt2018robust,li2019adversarial,man2020ghostimage,sharif2016accessorize,zhao2019seeing,zhou2018invisible}. However, attacks on traditional computer vision methods such as SLAM are relatively less explored. \cite{yoshida2022adversarial} proposes an attack against the scan matching algorithm in LiDAR-based SLAM, while most SLAMs in AR/VR devices rely on different sensors like RGB/depth cameras and IMUs. \cite{ikram2022perceptual} and \cite{chen2024adversary} mislead visual SLAM by poisoning the images with special patterns, and \cite{wang2021can} causes the camera to fail using infrared light. In our work, we demonstrate attacks on Visual-Inertial SLAM (VI-SLAM) by perturbing the IMU readings, rather than cameras, and showing its impact on XR user experience. 

\textbf{Acoustic Injection Attacks.} Among various physical attacks, acoustic injection attacks are attractive due to their low cost. Son~\etal~\cite{son2015rocking} were the first to introduce acoustic attacks on MEMS gyroscopes, demonstrating how these attacks could lead to sensor denial-of-service and result in drone crashes. WALNUT~\cite{trippel2017walnut} expanded on this by developing output biasing and control attacks that enable precise manipulation of MEMS accelerometer outputs using modulated sound waves. Wang et al.~\cite{wang2017sonic} demonstrated a sonic gun, showcasing the vulnerability of various smart devices (\eg drones and self-balancing vehicles) to acoustic attacks. Tu et al. \cite{tu2018injected} designed side-swing and switching attacks to alter the outputs of MEMS gyroscopes and accelerometers. Furthermore, Ji et al. \cite{ji2021poltergeist} fool the object detectors by applying acoustic attack to the image stabilizers commonly used in modern cameras. However, none of the existing works study the relationship between the acoustic injections and SLAM outputs on recent XR devices. 

% \zijian{Do we need one session about security in AR/VR?}
% \yicheng{TODO}
%\jiasi{cite the AIVR paper (UMass Amherst?) paper is we have not already. They add IMU perturbation but w/o SLAM, iirc} \yicheng{Cited}

\textbf{XR Security and Privacy.} 
%Security and privacy concerns in XR systems have gained significant attention. 
For single-user XR systems, researchers have demonstrated various side-channel attacks to extract sensitive information (\eg keystrokes) through video feeds~\cite{ling2019know}, head movements~\cite{nair2023unique, slocum2023going}, architectural hints~\cite{zhang2023its,shang2020arspy}, power usage~\cite{li2024dangers}, and EM side-channel leakages~\cite{al2021vr}. In multi-user XR systems, Su et al.~\cite{su2024remote} use avatar motion data to infer keystrokes in shared VR environments. Slocum et al.~\cite{slocum2024doesn} reveal vulnerabilities in the shared state frameworks of multi-user AR. Similarly, Lebeck et al.~\cite{lebeck2017securing} highlight risks like deceptive virtual objects and emphasize access control for managing shared physical and virtual spaces. Ruth et al.~\cite{ruth2019secure} further propose a secure multi-user AR framework focusing on content sharing and permissions.
Chandio et al.~\cite{chandio2024stealthy} %introduced a multi-modal spatiotemporal attack that 
simultaneously manipulated visual and inertial sensors to disrupt XR pose estimation. However, their study evaluated the attack using offline datasets and assumed the attacker's capability to manipulate IMU data streams through acoustic means, without real experiments. Ours is the first to demonstrate acoustic injection attacks on recent XR devices, like the Hololens 2, in the real world.
 


To illustrate equilibria and dynamics of performative prediction games, we focus on a scenario in which a \emph{duopoly} of mortgage companies, i.e. banks, compete to sell loans to customers.

\paragraph{Customer Model:} In our game, each bank is trying to attract customers from a given population $\mathcal{P}$. We model this population as comprised of individuals with a single-dimensional type: we denote individual $j$'s type as $y_j \in [0,1]$. For simplicity, we assume that \(y\) represents the customer’s probability of repaying the loan\footnote{In practice, a customer's (normalized) credit score can be interpreted as a noisy observation of $y_j$. This also corresponds to credit scores being \emph{calibrated}.}, i.e., $y_j := \P[Y_j = 1]$, where $Y_j$ is a random variable such that $Y_j = 0$ means that $j$ defaults on their loan, and $Y_j = 1$ means they repay their loan. Customer types in the population are drawn from a known distribution $D_y$ supported on $[0,1]$. 

\paragraph{Game between Banks:} Each Bank \(i \in \{1, 2\}\) selects two parameters \( (\tau_i, \gamma_i) := \theta_i\), where:
\begin{itemize}
    \item \(\tau_i \in \{\tau_l,\tau_h\}\) is the credit score threshold for approving a customer\footnote{We restrict the bank to only pick between two thresholds, $\tau_l$ and $\tau_h$. However, we highlight how our results are affected when we expand the strategy space to $n > 2$ actions in our experiments of Appendix \ref{app:3gamma}.}. Specifically, a customer $j$ with credit score \(y_j\) is approved by Bank $i$ if and only if \(y_j \geq \tau_i\);
    \item \(\gamma_i \in \{\gamma_l, \gamma_h\}\) is the interest rate offered to approved customers.
\end{itemize}
We denote as shorthand the space of allowable thresholds by $\Gamma := [0,1]$ and allowable interests rates by $\Lambda := [0,1]$. %The latter is set without loss of generality---we simply normalize the rates to be at most $1$. 
% {\color{red} Vidya: just thinking about this but is it natural to restrict interest rate to $1$? I don't think it would affect the equilibrium structure of the game but theoretically I think the interest rate could be anything in $[0,\infty)$.} {\color{green} Guanghui: Could we say something like this is without loss of generality} \gua{changed.}\juba{I think we repeated this twice, the next sentence already had this}
The loan amount is normalized to $1$ in the entire paper, without loss of generality; in this case, if a customer chooses Bank $i$, and the customer is approved by the bank at an interest rate of $\gamma_i$, the expected utility for the bank is equal to
\[
(1+\gamma_i)\cdot \P[Y_i = 1]-\P[Y_i = 0] = (1+\gamma_i)y_i-(1-y_i).
\]


%In practice, the credit score \(y\) serves as a noisy observation of the true likelihood of the customer's repayment. 

\paragraph{Banks' Utilities:} For given parameter choices \(\theta_1 = (\tau_1, \gamma_1)\) by Bank 1 and \(\theta_2 = (\tau_2, \gamma_2)\) by Bank 2, a \emph{rational} customer with credit score $y$ acts as follows:

\begin{enumerate}
    \item \textbf{Qualified for a single bank}: 
        \begin{itemize}
        \item If \(\tau_1 \leq y < \tau_2\), the customer goes to Bank 1, as the score qualifies for Bank 1 but not Bank 2. Conversely, if \(\tau_2 \leq y < \tau_1\), the customer chooses Bank 2.
    \end{itemize}
    \item \textbf{Qualified for both banks}:
     \begin{itemize}
        \item If \(\tau_1, \tau_2 \leq y\) and \(\gamma_1 < \gamma_2\), the customer selects Bank 1 for its lower interest rate. Conversely, if \(\gamma_1 > \gamma_2\), the customer chooses Bank 2.
        \item If \(\gamma_1 = \gamma_2\), the customer picks each bank with probability $1/2$. 
    \end{itemize}
    \item \textbf{Unqualified for both banks}:
    \begin{itemize}
        \item If \(y < \tau_1\) and \(y < \tau_2\), the customer is rejected by both banks.
    \end{itemize}
\end{enumerate}

The expected reward for Bank 1, denoted as \(u_1(\theta_1, \theta_2)\), can then be expressed as:
\begin{align}\label{eq:utility}
    u_1(\theta_1, \theta_2) 
    &=  \mathbb{E}_{y \sim D_y} \left[ \mathbb{I}\{\underbrace{\tau_1 \leq y < \tau_2 \ \cup \ (\tau_1, \tau_2 \leq y \ \cap \ \gamma_1 < \gamma_2)}_{\text{accepted by Bank 1}}\} \cdot \big((1+\gamma_1)y - (1-y)\big) \right] \nonumber\\
    & + \frac{1}{2} \mathbb{E}_{y \sim D_y} \left[ \mathbb{I}\{\underbrace{\tau_1, \tau_2 \leq y \ \cap \ \gamma_1 = \gamma_2}_{\text{accepted by both Banks}}\} \cdot \big((1+\gamma_1)y - (1-y)\big) \right].
\end{align}
Note that the problem is \emph{symmetric}, i.e., the utility function for Bank 2 can be derived by swapping the roles of \(\theta_1\) and \(\theta_2\). I.e., $u_2(\theta_1, \theta_2) = u_1(\theta_2, \theta_1)$. 

% If a bank only attracts customers between thresholds $\tau_a$ and $\tau_b$, for $\tau_a<\tau_b$, we call $[\tau_a,\tau_b]$ the \emph{threshold} range for that bank. For example, if Bank $1$ sets a threshold of $\tau_1$, Bank $2$ a threshold of $\tau_2 > \tau_1$, and $\gamma_1 > \gamma_2$, then Bank 1 has a threshold range of $[\tau_1,\tau_2]$, while bank $2$ has a threshold range of $[\tau_2,1]$.
% Note that the parameters set by \emph{both} banks, i.e. $(\theta_1,\theta_2)$ both influence the threshold range for each of Bank 1 and 2.  If $\tau_1>\tau_2$, $\gamma_1>\gamma_2$, then $\tau_a>\tau_b$, and the bank does not attract any customers. 
% {\color{red} is it possible for $\tau_a > \tau_b$, leading to the bank never attracting customers?} \gua{if $\gamma_1>\gamma_2$, $\tau_1>\tau_2$, then it gets no customer. I think it also makes sense.}\juba{I think we said we wanted to delete the discussion of the threshold range, no?}

% \noindent \textbf{Discrete Model}   
% We now present the discrete version of our model, where the interest rates and thresholds are selected from finite sets \(\Gamma\) and \(\Lambda\), respectively, with $\tau\in[0,1], \gamma\in[0,1]$,  for all $\tau\in\Lambda$ and $\gamma\in\Gamma$, \(|\Gamma| = n\) and \(|\Lambda| = m\). Let \(p_1, p_2 \in \Delta(\Gamma \times \Lambda)\) represent the mixed strategies of the two banks, where \(\Delta(\Gamma \times \Lambda)\) denotes the set of probability distributions over the discrete decision space \(\Gamma \times \Lambda\).


% \begin{Remark}
%    Note that our proposed problem can be reformulated as a standard multi-player performative prediction problem \citep{narang2023multiplayer}. However, in our problem, the data distribution faced by each learner breaks the Lipschitzness assumption of previous work~\citep{hardt2023performative,narang2023multiplayer}. A small modification in one of the learner's thresholds can completely change how demand is allocated across both learners, as is often the case in Bertrand-style games. 
% \end{Remark} 

% \gua{I made some changes to Remark 1, please have a look}
\begin{Remark}
   Previous works in multi-learner performative prediction~\citep{narang2023multiplayer} resort to an insensitivity assumption, i.e., the data distribution faced by each player can only changes slightly when the parameters also change slightly; formally, the data distribution faced by each player is Lipschitz in their decisions. This is immediately not true in our setting: the bank slightly changing its parameters can completely changes the demand distribution of customers it faces. Intuitively, this is because of Bertrand-competition-style effects, where if two banks have similar rates, one bank that lowers their rate by a small amount suddenly captures the entire customer demand that is eligible for that rate.%\juba{made further light edits adding intuition}
   
   In Appendix \ref{Appendix:refumulation}, we discuss this problem more carefully by reformulating our problem in the standard multi-learner performative prediction form given by~\citep{narang2023multiplayer}. We show the distribution is not Lipschitz with respect to the parameters, and thus does not satisfy the insensitivity assumption. 
%Prior work~\citep{hardt2023performative,narang2023multiplayer} showed that, for a general multi-agent performative prediction framework to work, insensitivity assumptions are needed: in the \textbf{worst case}, they can construct settings where the insensitivity assumption does not hold and simple dynamics do not converge anymore. We add nuance to this picture. We will show that our dynamics often converge, even absent insensitivity assumptions, highlighting that while the impossibility results of previous work hold in the worst case, they may not hold in the ``average case'' and especially not in problems motivated by applications. In particular, we will show convergence to a variety of equilibria of our game, and often to symmetric Nash equilibria where insensitivity is immediately violated.
     
\end{Remark}



% \paragraph{Relationship to Performative Prediction} A central point of our work is to highlight that \textcolor{red}{needs writing from intro}. We highlight how our work specifically ties to ``Performative Prediction'' below:


%\textcolor{red}{needs a definition environment}



%Here, \(\E_{\theta_1, \theta_2}\) represents the expected utility of the banks over their respective strategies \((\theta_1, \theta_2)\). These inequalities ensure that neither bank can unilaterally improve its expected utility by deviating from its mixed strategy in the equilibrium.



%and  for all $\tau\in\Gamma$, we have $\tau\in\Lambda$, $(\tau,\gamma)\in[0,1]^2$. Let $\Gamma\times\Lambda$
%In this paper, we focus on the most fundamental case, where there are two choices for each parameter: $0\leq\tau_{\ell}<\tau_{h}\leq 1$, and $0\leq \gamma_{\ell}< \gamma_{h}\leq 1$. In this case, the utility for each pair of decisions forms a $4\times4$ matrix (given in Table \ref{tab:my-table}). We consider the canonical case where $\tau_{\ell}=\frac{1}{2+\gamma_{h}}$, and $\tau_{h}=\frac{1}{2+\gamma_{\ell}}.$ Note that these are natural choices for the thresholds, in the sense that, if there is only one bank and the interest rate is set to be $\gamma$, then $\frac{1}{2+\gamma}$ is the optimal threshold corresponding to the fixed $\gamma$.


%and the thresholds are chosen in $\Lambda=\{\tau^{(1)},\dots,\tau^{(m)}\}$. Here, we only assume that, for each $\gamma\in\Gamma$, there at least exist one $\tau\in\Lambda$ such that $f(\gamma,\tau,1)>0$. Note that this is a very minor assumption, in the sense that, if for a $\gamma$ such that $f(\gamma,\tau,1)<0$ for all $\tau\in\Lambda$, then adopting this decision will lead to negative utility regardless of the opponent's decision, and thus is not an interesting case. 

%\textcolor{red}{The model section is missing the dynamic version of the game. We should clearly define the one-shot and the dynamic game}
% we only considered one-shot case in our paper



\vspace*{-1ex}
\section{Initial Results}
\label{sec:primitives}

In this section, we present beeping protocols for four fundamental network problems, usually used as building blocks of more complex tasks. Namely, Local Broadcast, Cluster Gathering, Learning Neighborhood, and Network Decomposition. 
%These protocols will be used as building blocks in our simulation of \congest rounds in beeping networks (Section~\ref{sec:main-simulation}).}
%algorithms that will be used in the network decomposition algorithm, namely, algorithms solving efficiently the problems of learning neighborhood, local broadcast and cluster gathering.
%
The following theorems establish the performance of our protocols. The details of the algorithms as well as the proofs of the theorems are left to Section~\ref{sec:prim_details}.

Recall that IDs of nodes come from the range $[1,n^c]$, for some constant $c \geq 1$.

\begin{restatable}[]{theorem}{localbroadcastthm} %\begin{theorem}
\label{th:local_broadcast}
    Let $\cN$ be a Beeping Network with 
    %set $V$ of 
    $n$ nodes, where each node 
    $v$
    %$v\in V$ 
    knows $n$, parameter~$c$, the maximum degree $\Delta$, and its neighborhood $N(v)$, and holds a message $m_v$ of length at most $B>0$.
    %Then, there is a deterministic distributed local broadcast algorithm that works in $O(k\Delta^2 \log^2 n)$ beeping rounds.
    There is a deterministic distributed algorithm that solves local broadcast on $\cN$ in $O(B\Delta^2 \log n)$ beeping~rounds.
    %\dk{???? SHOULDN'T IT BE $O(\Delta^2 (B+\log n)\log n)$ ????}
%\end{theorem}
\end{restatable}

\vspace*{-1.5ex}
\begin{restatable}[]{theorem}{learningneighthm} %\begin{theorem}
\label{th:learning_neighbourhood}
    Let $\cN$ be a Beeping Network with 
    %set $V$ of 
    $n$ nodes, where each node 
    %$v\in V$ 
    $v$
    knows $n$ and parameter~$c$.
    %There is a deterministic distributed learning-neighborhood algorithm that works in $O(\Delta^2 \log^2 n)$ beeping rounds.
    There is a deterministic distributed algorithm that solves learning neighborhood  on $\cN$ in $O(\Delta^2 \log^2 n)$~beeping~rounds.
%\end{theorem}
\end{restatable}

\vspace*{-1.5ex}
\begin{restatable}[]{theorem}{clustergatherthm} %\begin{theorem}
\label{th:cluster_gathering}
    Let $\cN$ be a Beeping Network with %set $V$ of 
    $n$ nodes, where each node 
    $v$
    %$v\in V$ 
    knows $n$, parameter $c$, the maximum degree $\Delta$, and its neighborhood $N(v)$.
    %There is a deterministic distributed cluster gathering algorithm that works in $O(\Delta^2 \log^4 n)$ beeping rounds.
    There is a deterministic distributed algorithm that solves cluster gathering on $\cN$ in $O(\Delta^2 \log^4 n)$ beeping rounds.
%\end{theorem}
\end{restatable}

% \pga{In the next theorem, we assume that node IDs come from range $[1,n]$.}

\vspace*{-1.5ex}
\begin{restatable}[]{theorem}{networkdecompthm} %\begin{theorem}
\label{thm:local-decomposition}
    Let $\cN$ be a Beeping Network with set $V$ of $n$ nodes, where each node $v\in V$ knows $n$, parameter $c$ and the maximum degree $\Delta$.
    There is a deterministic distributed algorithm that computes a $(\log n, \log^2 n)$-network decomposition of $\cN$ in $O(\Delta^2 \log^8 n)$ beeping rounds.
%\end{theorem}
\end{restatable}

% \pga{[TODO: Check if GGR algorithm can really be adapted to $n^c$ IDs trivially.]}

%%%%%%%%%%%%%%%%%%%%%%%%%%%%%%%%%%%%%%%%%%%%%%%%%%%%%%%%%%%

\vspace*{-3ex}
\section{Simulation of a \congest Round in Beeping Networks}
\label{sec:main-simulation}

%We already showed how to efficiently simulate a single round of a \congest model in a general beeping network, provided each node wants to send the same message to all of its neighbors. It already allows us to simulate many algorithms designed for the \congest model on a beeping network, in a deterministic and distributed way. 
Unfortunately, not all efficient graph algorithms in the \congest networks have the property of always broadcasting the same (short) message to every neighbor, which we exploit in Section~\ref{sec:local-broadcast}.\footnote{%
Note that in the LOCAL model, where the sizes of messages are of second importance (as long as they are polynomial), nodes can combine individual messages into one joint message and send it to all neighbors.}
%
In this section, we present a deterministic distributed algorithm that simulates a round of {\em any algorithm in the \congest model}, even if the algorithm sends different messages to neighbors.
It is only somewhat (polylogarithmically) slower than the more restricted one
(local broadcast, which required a node to send the same message to all its neighbors),
given in Theorem~\ref{th:local_broadcast} and Section~\ref{sec:local-broadcast}, but it is adaptive and uses heavier machinery.
%-- we show later that any non-adaptive solutions (beeping codes) are substantially less efficient and require $\Omega(\Delta^3)$ rounds. \mm{[[MM: do we prove this??]]}
 % we had to construct an adaptive algorithm using a technique for more complex codes
%  (puting a hifor the general simulator.
  %
 The novel construction is built hierarchically using the known family of code called ``avoiding selectors''. This, intuitively, already says ``when to beep.'' However, it is still possible that, for example, when two neighbors of   some node $v$ 
 %each 
 send a (different) message, of multiple bits per message, node $v$ will receive a ``message'' that is a logical OR of the two.
This is efficiently resolved by the new adaptive algorithm by employing a 3-stage handshake procedure, which sends pieces of the code that now serve in identifying what the IDs and messages are; it allows to spot overlapping transmissions from more than one neighbor and successfully decode those that do not overlap. Intuitively, each stage is ``triggered'' by a different level of the code.


%taken for different parameters. 
% 3-stage adaptive  
%handshake
%procedure (announcing, responding and confirming). on the top of the code. 
%which is an adaptive part. 
%This procedure is tightly correlated with the hierarchical structure of the code -- higher level code triggers announcing, while the lower level codes trigger responding and confirming.}

\vspace*{-2ex}
\paragraph{Preliminaries and Challenges.}
Suppose every node has a possibly different message to deliver to each of its neighbors. We could use the algorithm from Section~\ref{sub:neighbourhood} to learn neighbors' IDs first in $O(\Delta^2\log^2 n)$ beeping rounds.
W.l.o.g., assume that each message from a node $v$ to a node $w$ has $O(\log n)$ bits
%; otherwise, we could easily split it into chunks of size $O(\log n)$ and apply our new algorithm to each of them, sequentially 
(otherwise, 
%asymptotic formula on 
the bound on the time complexity is increased by a factor of $\cM/\log n$, where $\cM$ is the maximum size of a single message).
Simulation of a \congest round faces the following challenges.

\noindent
{\em Challenge 1.}
A node could try to compute its beeping schedule to avoid overlapping with other neighbors of the receiving node. However, it requires knowing at least $2$-hop neighborhood, which is costly. 
%(even $1$-hop requires $\Omega(\Delta^2\log n)$ rounds).
%The first main challenge to overcome is that a node $v$ does not know its $2$-hop neighborhood graph. Learning it could have potentially helped $v$ to use the specific differences between the identities to decide on a schedule when a successful transmission to/from each specific neighbor could take place (especially if transmission schedules are fixed, and nodes only decide what message to beep); however, it would require retrieving up to $\Theta(\Delta^2)$ node IDs. 
A node could try to learn first the IDs of its $1$-hop neighbors, and then broadcast them, one after another, using the local broadcast algorithm, 
%from Section~\ref{sec:local-broadcast}, 
but since there could be $\Theta(\Delta)$ such IDs (each represented by $O(\log n)$ bits), the overall time complexity would be $O(\Delta^3 \log^3 n)$, by Theorem~\ref{th:local_broadcast}.
Instead, our algorithm uses specific codes, called avoiding selectors (see Definition~\ref{def:avoid-selector}), to assure partial progress in information exchange in periods that sum up to $\Theta(\Delta^2 \polylog n \log \Delta)$.

\noindent
{\em Challenge 2.}
%Even knowing its $1$-hop neighbors, 
A node has to choose which of its input messages to beep at a time or find a more complex beeping code to encode many of its input messages. 
If it chooses ``wrongly,'' the message could be ``jammed'' by other beeping neighbors of the potential receiver.
To overcome this, avoiding selectors ensures that many nodes ``announce'' themselves successfully (i.e., without interference) to many of their neighbors, and these ``responders'' use avoiding selectors to respond. Once an announcer hears the ID of its responder, the handshaking procedure allows them to fix rounds for their point-to-point, non-interrupted communication.
%Such a chosen message could be arbitrarily ``jammed'' by some $2$-hop neighbors, which are not initially known to the node (see Challenge 1 above). 
%We prove later in Section~\ref{} \textcolor{red}{---------PENDING} that this is indeed a challenge, and any {\em non-adaptive beeping code} requires $\Omega(\Delta^3 \log\Delta n)$ \mm{[[MM: this might be confusing, is it n times log ?? ]]} rounds.
%To overcome it, the avoiding selectors (mentioned above) could be used in an adaptive way with properly chosen parameters to guarantee initiations of beeping communication in many links that are ``isolated'' in the network. We exploit them by designing a system of hand-shaking procedures, organized in three types of longer messages to beep: announcing (a node, called an announcer, beeps that it wants to communicate), responding (a node retrieving a message from an announcer, beeps a message destined to the announcer) and confirming (the announcer beeps a confirmation message destined to the responder). If indeed such links are isolated enough in the graph, we prove that this process guarantees successful message exchange between the announcer and the responder.


%the ad hoc topology and, consequently, 

The abovementioned avoiding selectors for $n$ nodes are parameterized by two numbers, $k,\ell$, corresponding to the number of competing neighbors/responders versus the other (potentially interrupting) neighbors:
%their definition~follows:

\begin{definition}[Avoiding selectors]
\label{def:avoid-selector}
    A family $\mathcal{F}$ of subsets of $[n]$ 
    %of size at most $k$ each 
    is called an \emph{$(n,k,\ell)$-avoiding selector}, where $1\le \ell < k\le n$, if for every non-empty subset $S \in [n]$ such that $|S| \leq k$ and for any subset $R\subseteq S$ of size at most $\ell$, there is an element $a \in S\setminus R$ for which there exists a set $F \in \mathcal{F}$ such that $|F \cap S| = \{a\}$.
\end{definition}


The following fact follows directly from Definition~\ref{def:avoid-selector}, see also \cite{BonisGV05,ChlebusK05}.

\begin{fact}
\label{fact:avoiding-selectors}
Suppose we are given an $(n,k,\ell)$-avoiding selector $\mathcal{F}$ and a set $S$ of size at most $k$.
Then, the number of elements in $S$ not ``selected'' by selector $\mF$ (i.e., for which there is no set in the selector that intersects $S$ on such singleton element) is smaller than $k-\ell$.
\end{fact}


\begin{theorem}[\cite{BonisGV05,ChlebusK05}]
\label{thm:avoiding-selectors}
There exists an $(n,k,\ell)$-avoiding selector of length $O\left(\frac{k^2}{k-\ell}\log n\right)$, and moreover, an $(n,k,\ell)$-avoiding selector of length $O\left(\frac{k^2}{k-\ell}\text{\em\ polylog } n\right)$ can be efficiently deterministically constructed (in polynomial time of $n$) for some polylogarithmic function $\text{\em\ polylog } n$, locally by each~node.
\end{theorem}




%\subsection{Main deterministic distributed algorithm simulating any \congest~round}
\subsection{The \alg Algorithm}
\label{sec:main-general-algorithm}

%The main 
Our simulator algorithm proceeds in epochs $i=1,\ldots\log\Delta $. 
A pseudo-code for an epoch $i$ is provided at the end of this subsection.
In the beginning, each node has all its links not successfully realized -- here by a link $\{v,w\}$ being realized we understand that up to the current round, an input message/ID sent by $v$ (using a sequence of beeps) has been successfully encoded by $w$ and vice versa (note that these are two different messages and were sent/encoded each in a different round); the formal definition of link realization will be given later.
The goal of the algorithm is to preserve the following invariant for epoch $i\ge 1$: 
\begin{quote}
\hspace*{-1em}
At the end of epoch 
$i=1,\ldots,\log \Delta$, 
%At the beginning of epoch 
%$i=1,\ldots,\log_{3/2} \Delta$, 
each vertex has less than 
$\kappa_i= \Delta / 2^{i}$ 
%$\kappa_i= \Delta \cdot (2/3)^{i-1}$ 
incident links not realized. 
\end{quote}
We also set an auxiliary value $\kappa_0=\Delta$, which corresponds to the maximum number of adjacent links per node at the beginning of the computation. For ease of presentation, we assume that node IDs come from the range $[1,n]$. Note that in all the formulas, the number of possible IDs appears only under logarithms, so the algorithm and proof for range $[1,n^c]$ are the same.
%By definition, $\kappa_1\le \Delta$.


\vspace*{-1.5ex}
%\paragraph{Main algorithm for epoch $i$.}
\paragraph{Algorithm for epoch $i$: Preliminaries and main concepts.} 

Epoch $i$ proceeds in subsequent batches of $2\log n$ rounds, each batch is called a \defn{super-round}. In a single super-round, a node can constantly listen or keep beeping according to some 0-1 sequence of length $2\log n$, where 1 corresponds to beeping in the related round and 0 means staying silent. 
The sequences that the nodes use during the algorithm are \defn{extended-IDs}, defined as follows: the first $\log n$ positions contain an ID of some node in $\{1,\ldots,n\}$, while the next $\log n$ positions contain the same ID 
%but 
\mamr{with the bits flipped, that is,}
with ones swapped to zeros and vice versa. 
Note that extended-IDs are pairwise different, and each of them contains exactly $\log n$ ones and $\log n$ zeros.
We say that a node $v$ {\em beeps an extended-ID of node $w$ in a super-round} 
%if $v$ keeps beeping exactly in rounds corresponding to the extended-ID of $w$ within this super-round 
\mamr{$s$ if, within super-round~$s$, node $v$ beeps only in rounds corresponding to positions with $1$'s in the extended ID of $w$ } 
($w$ could be a different node id than $v$).
We say that a node $w$ {\em receives an extended-ID of a node $v$ in a super-round} \mamr{$s$} if:
\vspace*{-0.8ex}
\begin{itemize}
\item 
$w$ does not beep in super-round \mamr{$s$},
\vspace*{-0.8ex}
\item 
the sequence of 
%beeps received 
\mamr{noise/silence heard by $w$}
in super-round \mamr{$s$} form an extended-ID of $v$.
\end{itemize}


\vspace*{-0.8ex}
\noindent
From the perspective of receiving information in a super-round, all other cases not falling under the above definition of receiving an extended-ID, i.e., when a node is not silent in the super-round or receives a sequence of beeps that does not form any extended-ID, are ignored by the algorithm, in the sense that it could be treated as meaningless information noise. 

%Analogously to extended-ID, each node $w$ creates an {\em extended-message addressed to a neighbor $v$}. Node $w$ does it 
\mamr{Analogously to extended-ID's, nodes create an \defn{extended-message}} by taking the binary representation of the message of logarithmic length and transforming it to a $2\log n$ binary sequence in the same way as an extended-ID is created from the binary ID of a node.
An extended-message, as well as an extended-ID, is easily decodable after being received without interruptions from other neighbors.

%A crucial definition specifies what does it mean to one-to-one communications, as given for free in the \congest model, in beeping networks. 
\mamr{A specification of the conditions to achieve one-to-one communication, which is given ``for free'' in the \congest model, is crucial. An illustration of the following handshake communication procedure is shown in Figure~\ref{fig:alg}.}
We say that our algorithm \defn{realizes link $\{v,w\}$} if the following 
%conditions
are~satisfied:
\vspace*{-0.8ex}
\begin{itemize}
\item[(a)] 
there are three consecutive super-rounds (called ``responding'') in which $v$ beeps an extended-ID of itself followed by an extended-ID of $w$ and then by extended-message of $v$ addressed to $w$, and $w$ receives them in these super-rounds; intuitively, it corresponds to the situation when $v$ ``tells'' $w$ that it dedicates these three super-rounds for communication from itself to $w$, and $w$ receives this information;
\vspace*{-3.5ex}
\item[(b)] 
there are three consecutive super-rounds (called ``confirming'') in which $w$ beeps an extended-ID of itself followed by an extended-ID of $v$ and by its extended-message addressed to $v$, and $v$ receives them in these super-rounds; intuitively, it corresponds to the situation when $w$ ``tells'' $v$ that it dedicates these three super-rounds for communication from itself to $v$, and $v$ receives this information;
\vspace*{-0.8ex}
\item[(c)]
there is a super-round, not earlier than the one specified in point (a), at the end of which node $w$ locally marks link $\{v,w\}$ as realized, 
and analogously, 
there is a super-round, not earlier than the one specified in point (b), at the end of which node $v$ locally marks link $\{v,w\}$ as realized.
\end{itemize}

\vspace*{-0.8ex}
\noindent
It is straightforward to see that in super-rounds specified in points (a) and (b), a multi-directional communication between $v$ and $w$ takes place -- by sending and receiving both ``directed pairs'' of extended-IDs of these two nodes, each of them commits that the super-rounds specified in points (a) and (b) are dedicated for sending a message dedicated to the other node, and vice versa. Additionally, in some super-round(s) both nodes commit that it has happened (c.f., point (c) above).

\vspace*{-2ex}
\paragraph{Algorithm for epoch $i$: Structure.} 

\section{The general case: Proof of \texorpdfstring{\Cref{thm:main-decomp}}{Theorem 1.6}}\label{sec:algo}

First, we show that data structure of \Cref{l:max_min_query} can be used to compute distances witnessed by shortest paths that pass through a constant-size separator.

\begin{lemma}\label{l:single_adhesion}
Fix a constant $k \in \mathbb{N}$. There exists an algorithm which as the input receives an edge-weighted graph $G$ on $n$ vertices and $m$ edges together with a partition of its vertices into three sets $A, B, C$ such that $|B| \leq k$ and there are no edges between $A$ and $C$, and as the output computes $\max_{c \in C} \dist(a, c)$ for every $a \in A$. The running time is $\Oh(m \log n + n \log^{k - 1} n)$.
\end{lemma}

\begin{proof}
Let $B = \{b_1, \ldots, b_k\}$. For any $a \in A, c \in C$, we have $\dist(a, c) = \min_{i \in [k]} \dist(a, b_i) + \dist(c, b_i)$. First, we run Dijkstra's algorithm from every vertex in $B$ to find $\dist(v, b_i)$ for every $v \in V(G)$ and $i \in [k]$. Next, we use \Cref{l:max_min_query} to construct a data structure $\mathbb{D}$ for the point set $\{(\dist(c, b_1), \dots, \dist(c, b_k))\colon c\in C\}\subseteq \mathbb{R}^k$. Now, the value $\max_{c \in C} \dist(a, c)$ for any given $a$ is equal to the answer of $\mathbb{D}$ to the query with argument $(\dist(a, b_1), \dots, \dist(a, b_k))$.
\end{proof}

After computing the distances over a constant-size separator, we will use the following observation to simplify one of the sides of the separation.

\begin{lemma}\label{l:inserting_paths}
Let $G$ be a edge-weighted connected graph and let $A, B, C$ be a partition of its vertices such that there are no edges between $A$ and $C$. For every pair of vertices $u, v \in B$, let $P_{u, v}$ be any shortest path from $u$ to $v$ with all internal vertices in $C$ (assuming such a path exists).

Let $G'$ denote a graph obtained from $G[A \cup B]$ by adding an edge from $u$ to $v$ of weight equal to the length of $P_{u, v}$, for all $u, v \in B$ for which $P_{u, v}$ exists. Then,  $$\dist_G(s, t) = \dist_{G'}(s, t)\qquad\textrm{for all }s,t\in A\cup B.$$
\end{lemma}
\begin{proof}
Let $G''$ be the graph obtained by adding new edges of $G'$ to $G$.
Fix any $s, t \in A \cup B$ and let $P$ denote the shortest path from $s$ to $t$ in $G''$ which minimizes the number of vertices from $C$ visited. Naturally, the weight of $P$ is equal $\dist_G(s, t)$. Assume that such path visits at least one vertex of $C$. Then, the path $P$ is of the form $s \xrightarrow{P_1} x \xrightarrow{P_2} y \xrightarrow{P_3} t$, where $x, y \in B$ and all the internal vertices of $P_2$ are in $C$. By the construction of $G'$, $P_2$ can be replaced with a direct edge from $x$ to $y$ of the same weight. We obtain a same weight path with a smaller number of vertices of $C$ visited, which is a contradiction. Therefore, $P$ is entirely contained in $A \cup B$, hence it exists in $G'$. This shows that $\dist_G(s, t) = \dist_{G'}(s, t)$.
\end{proof}


The next lemma encapsulates the main algorithmic content of the proof of \Cref{thm:main-decomp}. The algorithm will split the tree decomposition provided on input into smaller parts for which the eccentricities are easier to calculate. We use the following lemma to handle a single such part.
\begin{lemma}\label{l:star}
Fix constants $k, g \in \mathbb{N}, 0 < \delta < \frac{1}{54}$. Assume we are given $n \in \mathbb{N}$, an edge-weighted graph $G$ on at most $n$ vertices with a weight function $w \colon E(G) \to \mathbb{N}$, a vertex subset $A$ and a collection of non-empty vertex subsets $V_0, V_1, \dots, V_\ell$ satisfying the following conditions:
\begin{itemize}[nosep]
	\item The sum of weights of all the edges in $G$ is bounded by $\Oh(n)$.
	\item $V(G) \setminus A = V_0 \cup V_1 \cup \dots \cup V_\ell$.
	\item $|A| \leq k$.
	\item For every $i \in [\ell]$, $G[V_i \setminus V_0]$ is connected, $N_G(V_i \setminus V_0) = V_i \cap V_0$, $|V_i| = \Oh(n^\delta)$, and $|V_0 \cap V_i| \leq 4$.
	\item For all $i, j \in [\ell], i \neq j$, $V_i \setminus V_0$ and $V_j \setminus V_0$ are disjoint and non-adjacent in $G$.
	\item Every edge $uv \in E(G)$ with $u, v \not\in A$ is contained in $G[V_i]$ for some $i\in \{0,1,\ldots,\ell\}$.
	\item The graph obtained by taking $G[V_0]$ and adding a clique on $V_0 \cap V_i$ for every $i \in [\ell]$ has Euler genus bounded by $g$.
\end{itemize}
Then, we can compute the eccentricity of every vertex of $G$ in time $\Oh \left( n^{1 + \frac{150 + 54 \delta}{151}} \log^k n \right)$.
\end{lemma}

\begin{proof}
Fix $\delta' = \frac{1 + 97 \delta}{151}$; we have $\delta' - \delta = \frac{1 - 54\delta}{151} > 0$.
Let $E_i$ denote the set of edges with one endpoint in $V_i$ and the other endpoint in $V_i \setminus V_0$. For $i \in [\ell]$, we shall say that $V_i$ is {\em{heavy}} if the sum of weights of $E_i$ is larger than $n^{\delta'}$. Since the sets $E_i$ are pairwise disjoint and the total sum of weights of all the edges is bounded by $\Oh(n)$, the number of heavy subsets is bounded by $\Oh(n^{1 - \delta'})$. Without loss of generality, we may assume that $V_{\ell' + 1}, \dots, V_\ell$ are heavy and $V_1, \dots, V_{\ell'}$ are not, for some $\ell'\in \{0,\ldots,\ell\}$.


For any source vertex $s$, we can calculate distances from $s$ to every vertex of $G$  using breadth first search in time $\Oh(\sum_{e \in E(G)} w(e)) = \Oh(n)$.
In particular, for every $\ell' < i \leq \ell$, we can compute the distances from every vertex of $V_i$ to every vertex of $G$ in total time $\Oh(n^{2 - \delta' + \delta})$, because $$|V_{\ell'+1}\cup \ldots\cup V_{\ell}|\leq n^{1-\delta'}\cdot \Oh(n^\delta)=\Oh(n^{1-\delta'+
\delta}).$$
Additionally, we calculate distances $\dist_G(a, v)$ for every $a \in A, v \in V(G)$ in time $O(n)$.

For every $i \in [\ell]$ and $u,v \in V_0 \cap V_i$, there exists a shortest path $P_{i,u,v}$ from $u$ to $v$ with all internal vertices belonging to $V_i - V_0$ due to the assumption that $G[V_i - V_0]$ is connected and $N_G(V_i - V_0) = V_i \cap V_0$. Therefore, the distance from $u$ to $v$ is bounded by the sum of weights of edges in $E_i$. In particular, for $i \in [\ell']$, $\dist_G(u, v) \leq n^{\delta'}$.

We define $\widetilde{G}$ to be the graph obtained by taking $G[A \cup V_0 \cup \dots \cup V_{\ell'}]$ and applying the following operation for every $i \in \{\ell' + 1, \dots, \ell\}$:
for each pair of vertices $u, v \in A \cup (V_0 \cap V_i)$, add an edge in $\widetilde{G}$ between $u$ and $v$ with weight equal to the total weight of $P_{i,u,v}$. For a fixed $i, u$, we can find $P_{i, u, v}$ for all $v$ using breadth first search in time $\Oh(n)$. Taking a sum over all $i, u$, we get that $\tilde{G}$ can be computed in total time $\Oh(n^{2 - \delta'})$.


\begin{claim}\label{cl:wG}
The sum of the edge weights in $\widetilde{G}$ is $\Oh(n)$. Moreover, for all $u, v \in V(\widetilde{G})$, we have $\dist_{\widetilde{G}}(u, v) = \dist_{G}(u, v)$.
\end{claim}

\begin{proof}
Consider $i \in \{\ell' + 1, \dots, \ell\}$ and any $u, v \in A \cup (V_0 \cap V_i)$ for which we added an edge. Its weight is bounded by the sum of weights of edges in $E_i$. Therefore, the total weight of all edges added is at most
$$
\sum_{i \in \{\ell' + 1, \dots, \ell\}} \left( |A \cup (V_0 \cap V_i)|^2 \sum_{e \in E_i} w(e) \right) \leq (4 + k)^2 \sum_{e \in E(G)} w(e) = \Oh(n).
$$
This proves the first part of the claim.

For the second part of the claim, consider any $i \in \{\ell' + 1, \dots, \ell \}$ and observe that by our assumptions, $A \cup (V_0 \cap V_i)$ separates $(V_0 \cup \dots \cup V_{\ell'} \cup V_{i + 1} \cup \dots \cup V_\ell) \setminus V_i$ from $V_i \setminus V_0$. Hence it suffices to repeatedly apply \Cref{l:inserting_paths}.
\end{proof}

For every $u \in V(\widetilde{G})$, we have $\ecc_G(u) = \max(\ecc_{\widetilde{G}}(v), \max_{v \in V(G) \setminus V(\widetilde{G})} \dist_G(u, v))$. Note, that we already know all the distances $\dist_G(u, v)$ for $v \in V(G) \setminus V(\widetilde{G})$. Similarly, we can already compute $\ecc_G(u)$ for every $u \in V(G) \setminus V(\widetilde{G})$. Therefore, it remains to compute $\ecc_{\widetilde{G}}(v)$ for each $v \in V(\widetilde{G})$. Our goal is to show that this can be done efficiently using \Cref{l:main_ecc}.

Now, let $G'$ be the graph obtained from $\tilde{G}$ by replacing every edge $e$ non-indicent to $A$ with $w(e)\geq 2$ with a path of length $w(e)$ consisting of unit-weight edges. This operation again preserves the distances. Since the sum of edge weights in $\tilde{G}$ is of $\Oh(n)$, the total number of vertices in $G'$ is of $\Oh(n)$. For $0 \leq i \leq \ell'$, we write $V'_i$ to denote the set $V_i$ together with all the vertices added as a part of a path between two endpoints in $V_i$.
As $V_i$ is not heavy for each $i\in [\ell']$, we have
$$
|V'_i \setminus V'_0| \leq |V_i| + \sum_{e \in E_i} w(e) = \Oh(n^{\delta'})\qquad \textrm{for all }i\in [\ell'].
$$

Let $G_0$ denote the graph $G'[V'_0]$ and let $G_0^*$ denote the graph $G'- A$ with $V'_i - V'_0$ contracted to a single vertex $v_i^*$, for each $i \in [\ell']$; note that, all edges of $G_0$ and $G_0^*$ have unit weight.

\begin{claim}
	The graph $G_0^*$ is does not contain $K_{t}$ as a minor, where $t = \Oh(\sqrt{g})$.
\end{claim}

\begin{proof}
Let $\bar{G}_0$ denote the graph obtained by taking $G_0$ and adding a clique on $V_0 \cap V_i$ for every $i \in [\ell']$.
By lemma assumptions and the fact that subdividing edges does not increase the Euler genus, $\bar{G}_0$ has Euler genus at most $g$. In particular, $\bar{G}_0$ is $K_{t'}$-minor-free for some $t' = \Oh(\sqrt{g})$, because the Euler genus of $K_{t'}$ is $\Omega({t'}^2)$.

Similarly, let $\bar{G}_0^*$ be the graph obtained by taking $G_0^*$ and adding a clique on each $V_0 \cap V_i$.
Note, that $\bar{G}_0^* - \{v_1^*, \dots, v_{\ell'}^*\}$ is precisely $\bar{G}_0$. Let $t = \max(t', 6)$.
Recall that a minor model of a clique $K_t$ consists of $t$ pairwise vertex-disjoint connected subgraphs, called
branch sets, such that there is at least one edge between each pair of the branch sets.
Consider a minor model $\varphi$ of $K_{t}$ in $\bar{G}^*_0$.
Note that $\varphi$ cannot contain any singleton branch set of the form $\{v^*_i\}$, for the degree of $v^*_i$ in $\bar{G}^*_0$ is at most $4 < t - 1$. Furthermore, since $N_{\bar{G}^*_0}(v^*_i) = V_0 \cap V_i$, any branch set containing $v^*_i$ and at least one other vertex contains some $u \in V_0 \cap V_i$, and $N_{\bar{G}^*_0}(v^*_i)\subseteq N_{\bar{G}^*_0}(u)$, hence removing $v^*_i$ from this branch set preserves the model. Therefore, we can assume without loss of generality that all branch sets of $\varphi$ are disjoint from $\{v^*_1, \dots, v^*_{\ell'}\}$, hence $\varphi$ is a minor model of $K_{t}$ in $\bar{G}_0$. This is a contradiction, as $t \geq t'$ and $\bar{G}_0$ is $K_{t'}$-minor-free. Therefore, $\bar{G}_0^*$ is $K_t$-minor-free, hence $G_0^*$ also.
\end{proof}

Let $\rho' = \frac{2 - 108 \delta}{151} > 0$. The graph $G^*_0$ is a unit-weight graph and is $K_{t}$-minor-free.
Hence, by applying \Cref{t:r_division} to $G^*_0$ (with $\varepsilon = \rho'/2$)
we obtain an $n^{\rho'}$-division $\mathcal{R}_0$ in time $\Oh(n^{1 + \rho'})$.
We extend it to $G' - A$ by mapping every contracted vertex $v^*_i$ to $N_{G' - A}[V'_i - V'_0] = (V'_i - V'_0) \cup (V_0 \cap V_i)$. Formally, we put $V''_i \coloneqq N_{G' - A}[V'_i - V'_0]$ and 
$$
\mathcal{R} \coloneqq \left\{ (R_0 \cap V'_0) \cup \bigcup_{i \colon v^*_i \in R_0} V''_i \colon R_0 \in \mathcal{R}_0 \right\}.
$$

Now, we argue that $\mathcal{R}$ is a reasonable division of $G' - A$. Clearly, all sets $R \in \mathcal{R}$ are connected in $G' - A$. Pick any $R \in \mathcal{R}$ and let $R_0$ be its corresponding set in $\mathcal{R}_0$.
Every vertex $v^*_i$ is mapped to a set of size $\Oh(n^{\delta'})$, therefore
$$|R| \leq |R_0| \cdot \Oh(n^{\delta'}) = \Oh(n^{\rho' + \delta'}).$$

By our construction, for every $i \in [\ell']$, $R$ is either disjoint from $V'_i - V'_0$ or contains whole $N_{G' - A}[V'_i - V'_0]$. This means that no vertex belonging to any $V'_i - V'_0$ can be in $\partial R$, hence $\partial R \subseteq V'_0$.

Pick any $u \in \partial R \cap R_0$. Assume that $u \not\in \partial R_0$. Then every vertex of $N_{G_0^*}(u)$ must be in $R_0$, hence $N_{G - A'}(u) \subseteq R$, which is a contradiction. This means that $\partial R \cap R_0 \subseteq \partial R_0$.

Pick any $u \in \partial R - R_0$. Then, $u \in V_0 \cap V_i$ for some $i \in [\ell']$ such that $v_i^* \in R_0$. Moreover, $v_i^* \in \partial R_0$ and is adjacent to $u$ in $G_0^*$. The number of such $u$ is bounded by $4 |\partial R_0 \cap \{ v_1^*, \dots, v_{\ell'}^* \}|$.

Putting two cases together, we obtain:
$$
\sum_{R \in \mathcal{R}} |\partial R| = \sum_{R \in \mathcal{R}} \left(|\partial R \cap R_0| + |\partial R - R_0|\right) \leq \sum_{R_0 \in \mathcal{R}_0} \left(|\partial R_0| + 4 |\partial R_0 \cap \{ v_1^*, \dots, v_{\ell'}^* \}|\right) = \Oh(n^{1 - \frac{1}{2}\rho'}).
$$

It remains to show the following claim.

\begin{claim}
Pick any $R \in \mathcal{R}, s_R \in R$. The number of different distance profiles on $R$ relative to $s_R$ in $G' - A$ is of $\Oh(n^{48\rho' + 54\delta'})$.
\end{claim}
\begin{proof}
We look at every vertex $v \in V(G') \setminus A$ and consider three cases: $v \in R$, $v \in V'_0$, and $v \in V'_i \setminus (V'_0 \cup R)$ for some $i \in [\ell']$. By our construction, $R \cap V'_0$ is non-empty, hence w.l.o.g. we can assume that $s_R \in V'_0$ as whether two vertices have the same profile on $R$ is independent of the choice of the pivot vertex.

In the first case, there are at most $|R| = \Oh(n^{\rho' + \delta'})$ such vertices, hence they realise at most that many profiles.

In the second case, we want to observe that profile of any vertex $v \in V'_0$ on $R$ depends only on its profile on $R \cap V'_0$ (relative to $s_R$). Pick any $t \in R - V'_0$. Then $t \in V'_i - V'_0$ for some $i \in [\ell']$, $V_i \cap V_0 \subseteq R \cap V'_0$, and every path from $v$ to $t$ intersects $V_i \cap V_0$. In particular, distances from $v$ to vertices of $V_i \cap V_0$ determine its distance to $t$, which proves the observation.

Let $\tilde{G}_0$ denote the graph obtained by taking $G'[V'_0]$ and for every $i \in [\ell'], u, v \in V_0 \cap V_i$ adding a disjoint path from $u$ to $v$ of length $\dist(u, v)$. Let $P_i$ denote the vertex set of paths added between $V_0 \cap V_i$. For every $t \in V'_0$ we have $\dist_{G' - A}(v, t) = \dist_{\tilde{G}_0}(v, t)$, so it suffices to bound the number of profiles on $R \cap V'_0$ in $\tilde{G}_0$. By our assumptions, $\tilde{G}_0$ has Euler genus bounded by $g$ and all $P_i$ are of size $\Oh(n^{\delta'})$.

Let $R_0$ be the set of $\mathcal{R}_0$ corresponding to $R$. Let $\tilde{R}_0$ denote the set $(R \cap V'_0) \cup \bigcup_{i : v^*_i \in R_0} P_i$. Such set is connected in $\tilde{G}_0$. Moreover, similarly to $R$, its size is $\Oh(n^{\rho' + \delta'})$. Applying \Cref{thm:distprofiles}, we get that the number of distance profiles on $\tilde{R}_0$ in $\tilde{G}_0$ is $\Oh(n^{12(\rho' + \delta')})$, which also bounds the number of profiles on $R$ in $G' - A$ realised by $V'_0$.

For the third case, assume $v \in V'_i \setminus (V'_0 \cup R)$ for some $i\in [\ell']$. Every path from $v$ to any vertex of $R$ in $G' - A$ intersects $V_i \cap V_0$. Let $v_1, \dots v_p$ be the vertices of $V_i \cap V_0$, where $p \leq 4$. The profile of $v$ on $R$ is then determined by the following:
\begin{itemize}[nosep]
 \item[(a)] the profile of each $v_j$ on $R$,
 \item[(b)] $\dist_{G' - A}(v, v_j) - \dist_{G' - A}(v, v_1)$ for each $2 \leq j \leq p$, and
 \item[(c)] $\dist_{G' - A}(s_R, v_j) - \dist_{G' - A}(s_R, v_1)$ for each $2 \leq j \leq p$ where $s_R$ is some pivot vertex of $R$.
\end{itemize}
By the previous case, the number of distance profiles of each $v_j$ is $\Oh(n^{12(\rho' + \delta')})$. The distances between $v$ and $v_j$ are bounded by $|V'_i|$, hence each quantity described in (b) can take $\Oh(n^{\delta'})$ different possible values. Similarly, since $v_1$ and $v_j$ are connected via $V'_i$, $|\dist_{G' - A}(s_R, v_j) - \dist_{G' - A}(s_R, v_1)| \leq \Oh(n^{\delta'})$. The number of different possible profiles of such $v$ is therefore bounded by $\Oh(n^{48(\rho' + \delta') + 6\delta'}) = \Oh(n^{48\rho' + 54\delta'})$. This finishes the proof of the claim.
\end{proof}

Now we can apply \Cref{l:main_ecc} to graph $G'$ with apex set $A$, $X = V(\widetilde{G})$, and the following constants: $$\rho = \rho' + \delta',\qquad \gamma = 1 - \frac{1}{2}\rho',\quad \textrm{and}\quad \alpha = 48\rho' + 54 \delta'.$$ This allows us to calculate all $V(\widetilde{G})$-eccentricities in $G'$ in time
$$
\Oh \left( \left(
	n^{ 2 - \frac{1}{2} \rho' } +
	n^{ 1 + 48\rho' + 54 \delta' }
\right) \log^k n \right) =
\Oh \left( n^{1 + \frac{150 + 54 \delta}{151}} \log^k n \right).
$$
Since for each $v\in V(\widetilde{G})$ we have $\ecc_{\widetilde{G}}(v) = \max_{u \in V(\widetilde{G})} \dist_{\widetilde{G}}(v, u) = \max_{u \in V(\widetilde{G})} \dist_{G'}(v, u)$, this means that we have successfully computed all the eccentricities in $\widetilde{G}$; and as we argued, this is enough to compute all the eccentricities in $G$ as well.

Finally, the total running time of the algorithm is
$$
\Oh \left( n^{1 + \frac{150 + 54 \delta}{151}} \log^k n + n^{2 - \delta' + \delta} \right) =
\Oh \left( n^{1 + \frac{150 + 54 \delta}{151}} \log^k n \right).
$$\qedhere
\end{proof}


\begin{lemma}\label{l:star2}
Fix constants $k, g \in \mathbb{N}, 0 < \delta < \frac{1}{54}$. Assume we are given $n \in \mathbb{N}$, an edge-weighted graph $G$ on at most $n$ vertices with a weight function $w \colon E(G) \to \mathbb{N}$, a vertex subset $A$ and a collection of non-empty vertex subsets $V_0, V_1, \dots, V_\ell$ satisfying the same conditions as in \Cref{l:star} with the following differences:
\begin{itemize}
	\item we don't require $G[V_i - V_0]$ to be connected and $V_i - V_0$ to be adjacent to whole $V_i \cap V_0$;
	\item instead of $|V_0 \cap V_i| \leq 4$, we require $|V_0 \cap V_i| \leq k$.
\end{itemize}
Then, we can compute the eccentricity of every vertex of $G$ in time $\Oh \left( n^{1 + \frac{150 + 54 \delta}{151}} \log^{k + 5g} n \right)$.
\end{lemma}

\begin{proof}
We will reduce our input to one which will satisfy the conditions of \Cref{l:star}. We start by addressing the adhesions $V_0 \cap V_i$ containing too many vertices.

Let $G_0$ denote the graph $G[V_0]$ with cliques placed at $V_0 \cap V_i$ for every $i \in [\ell]$.
For every $i \in [\ell]$ we repeat the following procedure: while $|V_0 \cap V_i| > 4$,
remove arbitrary $5$ vertices from $V_0 \cap V_i$. Since $|V_0 \cap V_i| \leq k$ for each $i\in [\ell]$,
this procedure can be implemented in total time $\Oh(n)$. As a result, at the end we have $|V_0 \cap V_i| \leq 4$ for all $i \in [\ell]$. Let $M$ be the set of all the removed vertices. By our assumptions, $G_0$ has Euler genus bounded by $g$, hence it cannot contain $g + 1$ pairwise disjoint copies of $K_5$
(as the Euler genus of a graph is the sum of the Euler genera of its 2-connected components~\cite{StahlB77} and $K_5$ is not planar). Each removed quintiple of vertices induces a $K_5$ in $G_0$, hence we have $|M| \leq 5g$. We set $A' = A \cup M$ and may thus assume that $V_i$ is disjoint from $A'$ for all $0 \leq i \leq \ell$.

Now, fix $i \in [\ell]$. Let $C^i_1, \dots, C^i_{r_i}$ denote the connected components of $V_i - V_0$ in $G - A'$. We define $W^i_j := N_{G - A'}[C^i_j]$ for every $j \in [r_i]$. Clearly, all $W^i_j$ induce a connected subgraph of $G$ and satisfy $N_{G - A'}(W^i_j - V_0) = W^i_j \cap V_0$. We put $V'_0 := V_0$ and enumerate
$$
\{V'_1, V'_2, \dots V'_{\ell'}\} := \{ W^i_j \colon i \in [\ell], j \in [r_i] \}.
$$
It is easy to verify that the sets $A'$ and $V'_0, V'_1, \dots, V'_{\ell'}$ satisfy the conditions of \Cref{l:star}. We apply said lemma to calculate the eccentricity of every vertex of $G$ in the desired time.
\end{proof}



The next statement is a reformulation of \Cref{thm:main-decomp}.

\begin{theorem}
Fix constants $k, g \in \mathbb{N}$. Assume we are given a graph $G$ on $n$ vertices together with its tree decomposition $(T, \beta)$ and a set of private apices $A_t \subseteq \beta(t)$ for each node $t\in V(T)$ such that the following conditions hold:
\begin{itemize}[nosep]
 \item For every node $t \in V(T)$, we have $|A_t| \leq k$.
 \item For every edge $st \in E(T)$,  we have $|\beta(v) \cap \beta(u)|\leq k$.
 \item For every node $t \in V(T)$, graph obtained by taking $G[\beta(t)] - A_t$ and turning  $(\beta(t) \cap \beta(s))\setminus A_t$ into a clique for every edge $st \in E(T)$ has Euler genus bounded by $g$.
\end{itemize}
Then, we can compute the eccentricity of every vertex of $G$ in time $\Oh \left( n^{1 + \frac{355}{356}} \log^{k + 5g} n \right)$.
\end{theorem}

\begin{proof}
We may assume that $|V(T)|\leq n$, for every tree decomposition with no two bags comparable by inclusion has this property; and adjacent comparable bags can be merged by contracting the edge between them.

For a node $t\in V(T)$, by the {\em{weight}} of $t$ we mean the size of the corresponding bag, that is, $|\beta(t)|$. For any subset of nodes $S \subseteq V(T)$, we define $\beta(S) \coloneqq \bigcup_{t \in S} \beta(t)$ By the {\em{weight}} of $S$, we mean the total weight of the elements of $S$, that is, $\sum_{t\in S} |\beta(t)|$. 

\begin{claim}\label{cl:weight-T}
The weight of $V(T)$ is of $\Oh(n)$.
\end{claim}

\begin{proof}
The sets $\beta'(t) := \beta(t) - \bigcup_{s \in N_T(t)} \beta(s)$ are pairwise disjoint. We have
$$
\sum_{t \in V(T)} |\beta(t)| =
\sum_{t \in V(T)} |\beta'(t)| + 2 \cdot \sum_{st \in E(T)} |\beta(s) \cap \beta(t)| \leq
|V(T)| + 2k|E(T)| = \Oh(n).
$$
\end{proof}

Since every bag induces a graph of bounded Euler genus, the number of edges contained in a bag is linear in its size. In particular, this implies that the total number of edges of $G$ is also bounded by $\Oh(n)$.

We set $$\delta \coloneqq \frac{1}{356}\qquad\textrm{and}\qquad \Delta \coloneqq \frac{355}{356}.$$ Root the tree $T$ in an arbitrarily chosen node; this naturally imposes an ancestor-descendant relation in $T$ (for convenience, every node is considered its own ancestor and descendant).

We start by partitioning $T$ into connected subtrees using the following procedure.
We proceed bottom-up over $T$, processing nodes in any order so that a node is processed after all its strict descendants have been processed. Along the way, we mark some nodes and split the edges of $T$ into heavy and light. Let $t \in V(T)$ be the currently processed non-root node of $T$ and let $e \in E(T)$ be the edge connecting $t$ with its parent. If the total weight of all the unmarked nodes that are descendants of $t$ is at least $n^\delta$ (recall that this includes $t$ itself as well), then we declare $e$ heavy and mark all the descendants of $t$ that were unmarked so far. Otherwise, the edge $e$ is declared light and the procedure proceeds to further nodes of $T$.

Observe that
removing all heavy edges splits $T$ into connected subtrees, say $T'_1, \cdots T'_m$. All of the subtrees, except for possibly the subtree containing the root node, are of weight at least $n^\delta$. In particular, the number of subtrees $m$, and therefore the number of heavy edges, is  bounded by $\Oh(n^{1 - \delta})$. Moreover, in every subtree $T'_i$, removing the node closest to the root splits $T'_i$ into smaller components, each of weight less than $n^\delta$.

Fix a heavy edge $e$ and let $T^e_1$ and $T^e_2$ be the two subtrees into which $T$ splits after removing~$e$. Let $X^e_i = \beta(T^e_i)$ for $i \in \{1, 2\}$. Put $A_e = X^e_1 \setminus X^e_2$, $C_e = X^e_2 \setminus X^e_1$, and $B_e = X^e_1 \cap X^e_2$. By the properties of tree decompositions, such choice of $A_e, B_e, C_e$ satisfies the conditions of \Cref{l:single_adhesion}, hence in time $\Oh(n \log^{k - 1} n)$ we can compute $\max_{v \in X^e_2} \dist_G(u,v)$ for every $u \in X^e_1$, and $\max_{u \in X^e_1} \dist_G(u,v)$ for every $v \in X^e_2$. Computing this for every heavy edge $e$ takes total time $\Oh(n^{2 - \delta} \log^{k - 1} n)$.

Fix any subtree $T'=T'_j$. Let $e_1 = t^{e_1}_1t^{e_1}_2, e_2 = t^{e_2}_1 t^{e_2}_2, \dots, e_\ell = t^{e_\ell}_1 t^{e_\ell}_2$ denote the heavy edges incident to $T'$, where $t^{e_i}_1 \in V(T')$ and $V(T') \subseteq V(T_1^{e_i})$ for every $i \in [\ell]$.
For a vertex $v \in \beta(T')$, let
$$d_0(v) = \max_{u \in \beta(T')} \dist_G(v, u)\qquad\textrm{and}\qquad d_i(v) = \max_{u \in X_2^{e_i}}\dist_G(v,u),\quad\textrm{for } i \in [\ell].$$ We have $\ecc(v) = \max \{ d_i(v)\colon i\in \{0,1,\ldots,\ell\}\}$.The values of $d_i(v)$ are already calculated for all $i\in [\ell]$, hence it remains to compute $d_0(v)$.

For every $i \in [\ell]$ and every pair of vertices $u, v \in \beta(t^{e_i}_1) \cap \beta(t^{e_i}_2)$ we find a shortest path between $u$ and $v$ with all internal vertices inside $X^{e_i}_2$ (or determine that it doesn't exist). For a fixed $u, v$ this can be done in time $\Oh(n)$. Since in total we perform this step at most $2k^2$ times per heavy edge, it takes $\Oh(n^{2 - \delta})$ time in total. Let $P_{i, u, v}$ denote such path, assuming it exists.

Let $G'$ denote the graph obtained from $G[\beta(T')]$ by taking every $i, u, v$ for which $P_{i, u, v}$ exists and adding an edge between $u$ and $v$ of weight equal to the total weight of $P_{i, u, v}$.
The weight of every edge inserted in $\beta(t^{e_i}_1) \cap \beta(t^{e_i}_2)$ is bounded by $|X^{e_i}_2|+1$. The total weight of all edges inserted is therefore at most
$$
\sum_{i \in [\ell]} |\beta(t^{e_i}_1) \cap \beta(t^{e_i}_2)|^2 \cdot (|X^{e_i}_2|+1) \leq
k^2 \sum_{i \in [\ell]} (|X^{e_i}_2|+1) = \Oh(n),
$$
where the last equality follows from the fact that all the trees $T^{e_i}_2$ are pairwise disjoint.
By \Cref{l:inserting_paths}, we have $\dist_{G'}(u, v) = \dist_G(u, v)$ for each $u, v \in \beta(T')$. Hence, computing $d_0(v)$ for every $v \in \beta(T')$ is equivalent to computing the eccentricity of every vertex in $G'$.

If the size of $\beta(T')$ is smaller than $n^\Delta$, we compute the eccentricities naively in time $\Oh(|\beta(T')|^2)$, 
noting that $G'$ has $\Oh(|\beta(T')|)$ edges (thanks to Claim~\ref{cl:weight-T} and bounded genus assumption 
of the last bullet of the theorem statement). Otherwise, we argue that we can use the algorithm in \Cref{l:star} as follows.

Let $t$ be the node of $T'$ closest to the root. Let $s_1, \dots, s_p$ be the children of $t$ in $T$ and let $T''_i$ denote the connected component of $T' - \{t\}$ containing $s_i$. Set $V_0 = \beta(t)$ and $V_i = \beta(T''_i)$ for $i \in [p]$.

It is now easy to verify that $G'$ and sets $A, \{V_i\colon 0\leq i\leq p\}$ selected this way satisfy the assumptions of \Cref{l:star2}. This allows us to use it to compute the eccentricities in $G'$ in time
$$
\Oh \left( n^{1 + \frac{150 + 54\delta}{151}} \log^{k + 5g} n \right) =
\Oh \left( n^{1 + \frac{354}{356}} \log^{k + 5g} n \right).
$$
As we argued, from these eccentricities, we may easily compute all the eccentricities in $G$.

Now, let us analyse the total running time of the whole algorithm. We invoke \Cref{l:star} $\Oh(n^{1 - \Delta})$ times, since we apply it only to subtrees $T'_i$ of size at least $n^\Delta$. The total running time of those applications is hence
$$
\Oh \left( n^{2 + \frac{354}{356} - \Delta} \log^{k + 5g} n \right) =
\Oh \left( n^{1 + \frac{355}{356}} \log^{k + 5g} n \right).
$$
We compute the eccentricities naively for subtrees smaller than $n^\Delta$, hence the total running time of this computation is
$$
\sum_{i \in [m] \colon |\beta(T'_i)| \leq n^\Delta} |\beta(T'_i)|^2 \leq
n^\Delta \cdot \sum_{i \in m} |\beta(T'_i)| = \Oh(n^{1 + \Delta})=\Oh\left(n^{1+\frac{355}{356}}\right).
$$
The rest of computation can be done in $\Oh(n^{2 - \delta} \log^k n)$. Therefore, the whole algorithm runs in time $\Oh \left( n^{1 + \frac{355}{356}} \log^{k + 5g} n \right)$.
\end{proof}


An epoch $i$ is split into $|\mF_{\Delta,k_i}|$ {\em phases}, for a given $(n,\Delta,\Delta - k_i)$-avoiding selector $\mF_{\Delta,k_i}$ and parameter $k_i=\Delta/2^i$, parameterized by a variable $j$. Each phase starts with one {\em announcing super-round}, in which nodes in set $\mF_{\Delta,k_i}(j)$ beep in pursuit to be received by some of their neighbors. This super-round is followed by $\log k_i$ {\em sub-phases}, parameterized by $a=1,\ldots, \log k_i$. A sub-phase $a$ uses sets from an $(n,k_i/2^{a-2},k_i/2^{a-1})$-avoiding selector $\mF_{k_i/2^{a-2},k_i/2^{a-1}}$ to determine who beeps in which super-round (together with additional rules to decide what extended-ID and extended-message to beep and how to confirm receiving them), and consists of 
%$\sum_{a=1}^{\log k_i} 
$|\mF_{k_i/2^{a-2},k_i/2^{a-1}}|$ %quadruples 
$6$-tuples
of super-rounds ($3$ responding super-rounds and $3$ confirming super-rounds). 
The goal of a phase is to realize links that were successfully received (``announced'') in the first (announcing) super-round of this phase. This is particularly challenging in a distributed setting since many neighbors could receive such an announcement, but the links between them and the announcing node must be confirmed so that one-to-one communication between the announcer and responders could take place in different super-rounds (in one super-round, a node can receive only logarithmic-size information).

\begin{algorithm}[t!]
\DontPrintSemicolon
\SetKwFunction{announcer}{announcer}
\SetKwProg{myalg}{Procedure}{}{}
\let\oldnl\nl
\newcommand{\nlnonumber}{\renewcommand{\nl}{\let\nl\oldnl}}
\nlnonumber
\myalg{\announcer{$v,k_i$}}{
    \tcp{announcing super-round}
    \For{each round $r=1,2,\dots,2\log n$}{
        \leIf{$\langle v\rangle(r)=1$}{beep}{listen}
    }
        \For{each sub-phase $a=1,2,\dots,\log k_i$}{\label{line:subphaseloopA}
            \For{$b=1,2,\dots,|\mF_{k_i/2^{a-2},k_i/2^{a-1}}|$}{
                \tcp{responding 3 super-rounds}
                \lFor(\tcp*[h]{announcer only listens}){$6\log n$ rounds}{listen}
                \eIf{some $\langle w\rangle\langle v\rangle\langle m_{w,v}\rangle$ was heard {\bf and} 
                $\{w,v\}\in E(v)$}{
                    \tcp{confirming 3 super-rounds}
                    \For{each round $r=1,2,\dots,2\log n$}{
                        \leIf{
                        $\langle v\rangle(r)=1$}
                        {beep}{listen}
                    }
                    \For{each round $r=1,2,\dots,2\log n$}{
                        \leIf{
                        $\langle w\rangle(r)=1$}
                        {beep}{listen}
                    }
                    \For{each round $r=1,2,\dots,2\log n$}{
                        \leIf{
                        $\langle m_{v,w}\rangle(r)=1$}
                        {beep}{listen}
                    }
                    $E(v)\leftarrow E(v)\setminus \{w,v\}$ \tcp{link realized}
                    \lIf{$E(v)=\emptyset$}{$v$ stops executing}
                }{
                    \lFor(\tcp*[h]{wait to synchronize}){$6\log n$ rounds}{listen}
                }
            }
        }
    }
\caption{\alg algorithm for \underline{announcer} node $v$.} 
\label{algC2Bv2A}
\end{algorithm}

\mamr{
%\paragraph{Pseudo-code for epoch $i$.} 
%Below is a detailed description of the algorithm for {\bf\em Epoch $i$}.
\vspace*{-2ex}
\paragraph{Algorithm for epoch $i$: Definitions and notation.} 
The pseudocode of the \alg algorithm can be seen in Algorithm~\ref{algC2Bv2}, and its subroutines in Algorithms~\ref{algC2Bv2A} and~\ref{algC2Bv2L}.
$\mF_{\Delta,k_i}$ is a locally computed $(n,\Delta,\Delta-k_i)$-avoiding selector, and for any $a=1,\ldots,\log k_i$,
%selector 
$\mF_{k_i/2^{a-2},k_i/2^{a-1}}$ is a (locally computed) $(n,k_i/2^{a-2},k_i/2^{a-1})$-avoiding selector, as in Theorem~\ref{thm:avoiding-selectors}.
We denote the extended-ID of node $x$ as $\langle x\rangle$, and the extended-message of node $x$ for node $y$ as $\langle m_{x,y}\rangle$, both given as a sequence of bits. For any sequence of bits $s$, $s(i)$ is the $i^{th}$ bit of $s$.
}


\remove{
%\begin{enumerate}
%\item for $i=0,1,\ldots,\log \Delta$
\begin{enumerate}
\item all nodes become active, 
$k_i\gets \Delta / 2^{i}$
%$k_i\gets \Delta \cdot (2/3)^{i-1}$
    \item for each phase $j=1,2,\ldots,|\mF_{\Delta,k_i}|$
    \begin{enumerate}
        \item {\bf\em Announcing super-round:} each active node $v$ in set $\mF_{\Delta,k_i}(j)$ beeps its extended-ID in a super-round (recall that a super-round contains $2\log n$ subsequent rounds); \\
        a node $w$ that receives an extended-ID of some node $v$ and has not realized the link $\{v,w\}$ yet, becomes {\em $(j,v)$-responsive}
        \label{l:first-beep}
\item for each sub-phase $a=1,\ldots,\log k_i$
\label{alg:sub-phase}
       \begin{enumerate}
            \item for $b=1,2,\ldots,|\mF_{k_i/2^{a-2},k_i/2^{a-1}}|$ 
            \begin{enumerate}
            \item {\bf\em Responding $3$ super-rounds:}  
            if $w$ is $(j,v)$-responsive, for some $v$, and $w$ is in set $\mF_{k_i/2^{a-2},k_i/2^{a-1}}(b)$, node $w$ beeps its extended-ID in one super-round, followed by the extended-ID of $v$ in the next super-round, followed by the extended-message of $w$ addressed to $v$;
%            beeping rounds, nodes that heard a beep in preceding line~\ref{l:first-beep}, keep transmitting according to their corresponding row in $\mF_{k_i,k_i/3}$
        \item {\bf\em Confirming $3$ super-rounds:} if $v$ is in set $\mF_{\Delta,k_i}(j)$ (i.e., it beeped its extended-ID in a super-round in line~\ref{l:first-beep}) received an extended-IDs of $w$ and of itself and an extended-message in the preceding $3$ responding super-rounds, for some $w$, it beeps an extended-ID of itself in the one super-round, followed by the extended-ID of $w$ in the next super-round, followed by its extended-message addresses to $w$;\\
        Then, at the end of the third confirming super-round, the beeping node $v$ (locally) marks the link $\{v,w\}$ as realized; \\
        If a $(j,v')$-responsive node $w'$ receives an extended-ID of $v'$ followed by its extended-ID and an extended-message in the current confirming $3$ super-rounds, it (locally) marks link $\{v',w'\}$ as realized and $w'$ abandons its $(j,v')$-responsive status (as the corresponding link has been already marked as realized)
            \end{enumerate}
        \end{enumerate}
    \end{enumerate}
\end{enumerate}
%\end{enumerate}
}




%\subsection{Analysis of the algorithm from Section~\ref{sec:main-general-algorithm}}
\vspace*{-2ex}
\subsection{Analysis of the \alg Algorithm}

Recall that the algorithm proceeds in synchronized super-rounds, each containing a subsequent $2\log n$ rounds. Therefore, our analysis assumes that the computation is partitioned into consecutive super-rounds and, unless stated otherwise, it focuses on correctness and progress in super-rounds. 
Recall also that each node either stays silent (no beeping at all) or beeps an extended ID of some node or an extended message of one node addressed to one of its neighbors in a super-round.
%
The missing proofs 
%from this section 
are deferred to Section~\ref{sec:proofs-main-simulation}.

In the next two technical results, we state and prove the facts that receiving an extended-ID by a node $w$ in a super-round can happen if and only if there is {\em exactly one neighbor} of $w$ has been beeping {\em the same extended-ID} during the considered super-round. 

\begin{fact}[Single beeping]
\label{fact:single-beeping}
If during a super-round, exactly one neighbor of a node $w$ beeps an extended-ID of some $z$, then $w$ receives this extended-ID in this super-round.
\end{fact}

\begin{proof}
Directly from the definition of receiving an extended-ID. 
\end{proof}


\begin{lemma}[Correct receiving]
\label{lem:correct-receiving}
During the algorithm, if a node $w$ 
%stays silent and 
receives some extended-ID of $z$ in a super-round, then some unique neighbor $v$ of $w$ has been beeping an extended-ID of $z$ in this super-round while all other neighbors of $w$ have been silent. 
%in this super-round. 
The above holds except, possibly, some second responding super-rounds, in which a node can receive an extended-ID of $z$ that has been beeped by more than one neighbor.
\end{lemma}

%\begin{minipage}{1\linewidth}
%\begin{algorithm}[t!]
\DontPrintSemicolon
\SetKwFunction{announcer}{announcer}
\SetKwProg{myalg}{Procedure}{}{}
\let\oldnl\nl
\newcommand{\nlnonumber}{\renewcommand{\nl}{\let\nl\oldnl}}
\nlnonumber
\myalg{\announcer{$v,k_i$}}{
    \tcp{announcing super-round}
    \For{each round $r=1,2,\dots,2\log n$}{
        \leIf{$\langle v\rangle(r)=1$}{beep}{listen}
    }
        \For{each sub-phase $a=1,2,\dots,\log k_i$}{\label{line:subphaseloopA}
            \For{$b=1,2,\dots,|\mF_{k_i/2^{a-2},k_i/2^{a-1}}|$}{
                \tcp{responding 3 super-rounds}
                \lFor(\tcp*[h]{announcer only listens}){$6\log n$ rounds}{listen}
                \eIf{some $\langle w\rangle\langle v\rangle\langle m_{w,v}\rangle$ was heard {\bf and} 
                $\{w,v\}\in E(v)$}{
                    \tcp{confirming 3 super-rounds}
                    \For{each round $r=1,2,\dots,2\log n$}{
                        \leIf{
                        $\langle v\rangle(r)=1$}
                        {beep}{listen}
                    }
                    \For{each round $r=1,2,\dots,2\log n$}{
                        \leIf{
                        $\langle w\rangle(r)=1$}
                        {beep}{listen}
                    }
                    \For{each round $r=1,2,\dots,2\log n$}{
                        \leIf{
                        $\langle m_{v,w}\rangle(r)=1$}
                        {beep}{listen}
                    }
                    $E(v)\leftarrow E(v)\setminus \{w,v\}$ \tcp{link realized}
                    \lIf{$E(v)=\emptyset$}{$v$ stops executing}
                }{
                    \lFor(\tcp*[h]{wait to synchronize}){$6\log n$ rounds}{listen}
                }
            }
        }
    }
\caption{\alg algorithm for \underline{announcer} node $v$.} 
\label{algC2Bv2A}
\end{algorithm}
\begin{algorithm}[t!]
\DontPrintSemicolon
\SetKwFunction{listener}{listener}
\SetKwProg{myalg}{Procedure}{}{}
\let\oldnl\nl
\newcommand{\nlnonumber}{\renewcommand{\nl}{\let\nl\oldnl}}
\nlnonumber
\myalg{\listener{$w,k_i$}}{
    $status(u)\leftarrow$ \texttt{nil}\;
    \tcp{announcing super-round}
    \lFor(\tcp*[h]{listener only listens}){$2\log n$ rounds}{listen}
    \If{some $\langle v\rangle$ was heard {\bf and} $\{v,w\}\in E(w)$}{
        $status(w) \leftarrow v$-$responsive$\;
    }
    \For{each sub-phase $a=1,2,\dots,\log k_i$}{\label{line:subphaseloopL}
        \For{$b=1,2,\dots,|\mF_{k_i/2^{a-2},k_i/2^{a-1}}|$}{
            \eIf{$status(w)=v$-$responsive$ {\bf and} 
                $w\in\mF_{k_i/2^{a-2},k_i/2^{a-1}}(b)$}{
                \tcp{responding 3 super-rounds}
                \For{each round $r=1,2,\dots,2\log n$}{
                    \leIf{$\langle w\rangle(r)=1$}{beep}{listen}
                }
                \For{each round $r=1,2,\dots,2\log n$}{
                    \leIf{$\langle v\rangle(r)=1$}{beep}{listen}
                }
                \For{each round $r=1,2,\dots,2\log n$}{
                    \leIf{$\langle m_{w,v}\rangle(r)=1$}{beep}{listen}
                }
                \tcp{confirming 3 super-rounds}
                \lFor(\tcp*[h]{listener only listens}){$6\log n$ rounds}{listen}
                \If{some $\langle v\rangle\langle w\rangle\langle m_{v,w}\rangle$ was heard}{
                    $status(w)\leftarrow$ \texttt{nil}\;
                    $E(w)\leftarrow E(w)\setminus \{v,w\}$ \tcp{link realized}
                    \lIf{$E(w)=\emptyset$}{$w$ stops executing}
                }            
            }{
                \lFor(\tcp*[h]{wait to synchronize}){$12\log n$ rounds}{listen}
            }
        }
    }
}
\caption{\alg algorithm for \underline{listener} node $w$.} 
\label{algC2Bv2L}
\end{algorithm}



%\end{minipage}


We now prove that link realization implemented by our algorithm is consistent with the definition -- it allocates in a distributed way super-rounds for bi-directional communication of distinct messages.

\begin{lemma}[Correct realization]
\label{lem:correct-realization}
If a node $v$ (locally) marks some link $\{v,w\}$ as realized, which may happen only at the end of a second confirming super-round, the link has been realized by then. 
\end{lemma}



As mentioned earlier in the description of the phase, the goal of a phase $j$ (of epoch $i$) is to assure that any node $v$ that was received by some other nodes $w$ in the announcing super-round, gets all such links $\{v,w\}$ realized by the end of the phase (and vice versa, because the condition on the realization by this algorithm is symmetric).
The next step is conditional progress in a sub-phase $a$ of a phase $j$.

\begin{lemma}[Sub-phase progress]
\label{lem:subphase-progress}
Consider any node $v$ and suppose that in the beginning of sub-phase $a$ of phase $j$, there are at most $\Delta/2^{i+a-2}$ nodes $w$ such that $w$ is $(j,v)$-responsive and it does not mark link $\{v,w\}$ as realized. Then, by the end of the sub-phase, the number of such nodes is reduced to less~than~$\Delta/2^{i+a-1}$.
\end{lemma}



\begin{lemma}[Phase progress]
\label{lem:phase-progress}
Consider a phase $j$ of epoch $i$ and assume that in the beginning, there are at most $2k_i$ non-realized incident links to any node. Every node $w$ that becomes $(j,v)$-responsive in the first (announcing) super-round of the phase, for some $v$, mark locally the link $\{v,w\}$ as realized during this phase. And vice versa, also node $v$ marks locally that link as realized. 
\end{lemma}

The next lemma proves the invariant for epoch $i$, assuming that it holds in the previous epochs. 

\begin{lemma}[Epoch invariant]
\label{lem:epoch-invariant}
The invariant for epoch $i\ge 1$ holds. 
\end{lemma}




\begin{theorem}
\label{thm:congest-sim}
%The main deterministic distributed algorithm 
The \alg algorithm deterministically and distributedly
simulates any round of any algorithm designed for the \congest networks in $O(\Delta^2 \polylog n \log\Delta)$ beeping rounds, where the $\polylog n$ is the square of the (poly-)logarithm in the construction of avoiding-selectors in Theorem~\ref{thm:avoiding-selectors} multiplied by $\log n$.
\end{theorem}

%\sk{REPEATING THE THEOREM IN 2 PLACES  SEEMS UNJUSTIFIABLY COSTLY IN TERMS OF SPACE REAL ESTATE. I WOULD START DIRECTLY WITH "PROOF OF THEOREM 6". OR REMOVE IN THE INTRO. tHE SAME GOES WITH THE COROLLARIES GIVEN IN THE INTRO. ALREADY HALF A PAGE SAVING.  }

\begin{proof}
By Lemma~\ref{lem:epoch-invariant}, each epoch $i$ reduces by at least half the number of non-realized incident links. 
We next count the number of rounds in each epoch by counting the number of super-rounds and multiplying the result by the $O(\log n)$ length of each super-round.
Recall that link realization means that some triples of responding and confirming super rounds were not interrupted by other neighbors of both end nodes of that link; therefore, the attached extended messages (in the third super-rounds in a row) were correctly received. Thus, the local exchange of messages addressed to specific neighbors took place successfully.

Each sub-phase $a$ has $O(\Delta^2 \polylog n)$ super-rounds, because for each set in of the $(n,k_i/2^{a-1},k_i/2^a)$-avoiding selector $\mF_{k_i/2^{a-1},k_i/2^a}$, there are four super-rounds and the selector itself has $O((k_i/2^a) \polylog n)$ set, by Theorem~\ref{thm:avoiding-selectors}.

Therefore, the total number of super-rounds in all sub-phases executed 
within 
%point~\ref{alg:sub-phase} of the algorithm~
\mamr{the loops in Line~\ref{line:subphaseloopA} of Algorithm~\ref{algC2Bv2A} and Line~\ref{line:subphaseloopL} of Algorithm~\ref{algC2Bv2L}}
is 
\vspace*{-1.3ex}
\[
O(\sum_{a=1}^{\log k_i} (k_i/2^a) \polylog n) \le
O( k_i \polylog n)
\ .
\]

\vspace*{-0.7ex}
\noindent
Within one phase, they are executed as many times as the number of announcing super-rounds. 
The number of announcing super-rounds in a phase is
$|\mF_{\Delta,k_i}|$, which is $O((\Delta^2/k_i)\cdot \polylog n)$ by Theorem~\ref{thm:avoiding-selectors}.
Hence, the total number of super-rounds in a phase is 
%\[
$O( (\Delta^2/k_i)\cdot \polylog n \cdot k_i \polylog n)
\le 
O(\Delta^2 \polylog n)$,
%\ ,
%\]
where the final $\polylog n$ is a square of the (poly-)logarithms from Theorem~\ref{thm:avoiding-selectors}.

Since there are $\log\Delta$ epochs, the total number of super-rounds is $O(\Delta^2 \polylog n \log\Delta)$, which is additionally multiplied by $O(\log n)$ -- the length of each super-round -- if we want to refer the total number of beeping rounds.
%
%super-rounds of $O(\log n)$ beeping rounds each
\end{proof}

\vspace*{-2.5ex}
\paragraph{Maximal Independent Set (MIS):}
To demonstrate that the above efficient simulator can yield efficient results for many graph problems, we apply it to the 
algorithm of~\cite{ghaffari2021improved}% 
% \dk{??? and others such as ????}
 to improve polynomially (with respect to $\Delta$) the best-known solutions 
for MIS (c.f. \cite{beauquier2018fast}):

%\dk{(c.f.,~\cite{???})}:

%\todo{More uses for Network Decomposition! Add corollaries here and citations in Related Work.}

\begin{corollary}
\label{cor:mis}
% Graph problems such as MIS, \dk{????????????} 
MIS can be solved deterministically on any network of maximum node-degree $\Delta$ in $O(\Delta^2 \polylog n)$ beeping rounds.
\end{corollary}

\remove{%%%%%%%%%%%%%%%%%

\subsection{Cubic Lower Bound for Non-adaptive Beeping Codes}
\label{sec:lower-non-adaptive}

Consider the following simplification of the simulation problem. Each node $v$ is given, as an input, parameters $n,\Delta$ and a vector of numbers in $[n]$ of length $x_v\le \Delta$. The goal is: for any graph $G$ such that $x_v=|N(v)|$, for any node $v$, every $i$th neighbor of $v$ (according to the order of IDs) learns the $i$th number in the vector of $v$, for any $1\le i \le x_v$.
We call this problem {\em local ports' learning}, as we could think about the numbers in the input vectors as (arbitrary) labels of ports from the node to its corresponding neighbor.


\begin{theorem}
Any beeping code solving the local ports' learning problem has length $\Omega(\Delta^3\log n)$.
\end{theorem}

\begin{proof}


\end{proof}

}%%%%%%%%%%%  END  REMOVE  %%%%%%%%%%

%%%%%%%%%%%%%%%%%%%%%%%%%%%%%%%%%%%%%%%%%%%%%%%%%%%%%%%%%%%
%\section{Multi-hop Bounds}
%\vspace*{-3.1ex}
\section{Multi-hop \mam{Simulation}}

% - lower bound of $\Omega(\Delta)$ for Learning neighborhood based on Davies lower bound (Lemma 14 in Davies' paper); write it down!

% - algorithm for multihop Learning neighborhood via flooding (repeated use of Local Broadcast); we can also compute the shortest paths to each node within k hops.


% \dk{!!!! POTENTIAL NOTATION CLASH - $k$ denoted size of messages in previous section(s) while here denotes $k$-hop distance !!!!}

\dk{We generalize the 
%local communication 
\mam{simulation}
at distance $1$ to the following {\em $B$-bit $h$-hop simulation} problem: each node has messages, potentially different, of size at most $B$ \mam{bits} addressed to any other node, and it needs to deliver them to all destination nodes \mam{within} distance at most $h$ \mam{hops}. 
If each node has only a single message of size at most $B$ \mam{bits} to be delivered to all nodes \mam{within} distance at most $h$ \mam{hops}, then we call this restricted version {\em $B$-bit $h$-hop Local Broadcast}.
Note also that we do not require messages addressed to nodes of distance larger than $h$ to be delivered.}
%
Below we generalize the lower bound for single-hop simulation %by Davies
in~\cite{davies2023optimal} 
to multi-hop simulation and multi-hop local broadcast.

\begin{theorem}\label{thm:multihoplb}
    There 
    %exists 
    is
    an adversarial network of size $\Theta(\Delta^h)$ such that any $B$-bit $h$-hop simulation algorithm requires $\Omega(B\Delta^{h+1})$ beeping rounds to succeed with probability more than $2^{-\frac{1}{2}\cdot B(\Delta-1)^{h-2}(\Delta/2)^3} = 2^{-\Theta\left(B\Delta^{h+1}\right)}$.

    There exists an adversarial network of size $\Theta(\Delta^h)$ such that any $B$-bit $h$-hop Local Broadcast requires $\Omega(B\Delta^{h})$ beeping rounds to succeed with probability more than $2^{-\frac{1}{2}\cdot B(\Delta-1)^{h-2}(\Delta/2)^2} = 2^{-\Theta\left(B\Delta^{h}\right)}$.
\end{theorem}
\begin{proof}
    \noindent\textbf{Problem instance.}
    \mamr{We describe the construction of the adversarial network and input set of messages used to prove our lower bound as follows (refer to Figures~\ref{fig:multihop_graph} and~\ref{fig:multihop_graph2}).}
    Consider a full bipartite graph $K_{\Delta/2,\Delta/2}$, with one part called $T$ and the other $R$. We 
    %will 
    focus on transmissions going towards nodes in $R$, hence nodes in $R$ will be called receivers, while nodes in $T = T_1$ will be called the first \mamr{of $h$ layers}~of~transmitters. 
    
    % Let us ignore for a moment $R$ and its links. 
    Each node in $T_1$ will be a root of an $(h-1)$-depth tree of transmitters. We create a second layer of transmitters $T_2$ composed of $(\Delta/2)^2$ nodes. Each node in $T_1$ \mamr{(already connected to each node in $R$)} will also be connected to different $\Delta/2$ nodes in $T_2$.
    % , thus each node in $T$ has degree $\Delta/2 + \Delta/2 = \Delta$. 
    %Every
    \mamr{For subsequent layers of transmitters, that is each} layer $T_i$, for $3 \leq i \leq h$, will be composed of $(\Delta/2)^2 (\Delta-1)^{i-2}$ nodes. Each node in layer $T_{i-1}$, for $3 \leq i \leq h$, will be connected to different $\Delta-1$ nodes in layer~$T_i$. 
    %\pga{See Figure~\ref{fig:multihop_graph}.}
    % Thus, nodes in layer $T_{i-1}$ for $3 \leq i \leq k-2$ have degree $1 + \Delta-1 = \Delta$. Finally, nodes in layer $T_{k-1}$ have degree $1$.
%
    \begin{figure}[t]
        \centering
        \includegraphics[width=0.5\linewidth]{multihop_graph2.pdf}
        \caption{An illustration of the structure of the graph. The graph is partitioned into vertical layers $T_h$, \dots, $T_1$, $R$. \pga{The graph is branching out heavily, so we show only a path} from an arbitrary node in $T_1$ layer to an arbitrary node in $T_h$ layer, with all the edges incident to the path. The numbers between the layers denote the number of edges between the layers that are incident \pga{to the path or on} the path. Recall that layers $T_1$ and $R$ have $\Delta/2$ nodes each, while the other layers have significantly more nodes, but we only show nodes that are adjacent to the considered path.}
        \label{fig:multihop_graph}
    \end{figure}
%    
    Note that each node \mamr{in the defined network} has at most~$\Delta$~neighbors.

    %We have defined the graph we will be working on. Now we will describe the transmissions. 
    \mamr{We define now the input set of messages as follows.}
    Let each node $v \in T_{h}$ have a $B$-bit message $m_{v \rightarrow u}$ to each node $u \in R$. We choose those messages uniformly at random. We will show that just these messages cannot be relayed efficiently and we do not need any other messages in our problem instance\footnote{Alternatively we can make all the other messages known to the optimal algorithm, e.g., by setting them to be $0^B$.}.

%    \noindent\textbf{Gossiping.} Here we analyze the Gossiping algorithms.
    \noindent\textbf{\mam{Multihop Simulation.}} Here we analyze the \mam{multihop simulation} algorithms.

    There are $(\Delta/2)^2(\Delta-1)^{h-2}$ nodes in $T_h$, and each of them has $\Delta/2$ (possibly different) messages, one for each node in $R$. Therefore, there are $(\Delta/2)^3 (\Delta-1)^{h-2} \pga{=} \Theta(\Delta^{h+1})$ messages to nodes in $R$ that are passing through nodes in $T_1$.

    % Note that each node in $T_1$ will have to relay $(\Delta-1)^{k-1}\Delta/2 = \Theta(\Delta^k)$ messages to each node in $R$, i.e., $(\Delta-1)^{k-1}(\Delta/2)^2 = \Theta(\Delta^{k+1})$ messages total.
    
    % Note that each of the $\Delta/2$ nodes in $T_1$ will have to relay messages from $\frac{(\Delta/2)^2(\Delta-1)^{k-2}}{\Delta/2}$ nodes in $T_{k}$. Each node in $T_{k}$ has $\Delta/2$ (possibly different) messages, one for each node in $R$. Therefore, there are $(\Delta-1)^{k-2}(\Delta/2)^2 = \Theta(\Delta^{k})$ messages to nodes in $R$.

    Let $\mathcal{R}$ be the concatenated string of local randomness in all the nodes in $R$.
    %in $T_i$, for $1 \leq i \leq h-1$. 
    The output of any receiver $u \in R$ must depend only on $\mathcal{R}$, node IDs and the pattern of beeps and silences of nodes in $T_1$.

    There are $2^t$ possible patterns of beeps and silences in $t$ rounds. Therefore, the output of nodes in $R$ must be one of the $2^t$ possible distributions, where a distribution is over the randomness of $\mathcal{R}$. The correct output of nodes in $R$ is a string $\{0,1\}^{B(\Delta-1)^{h-2}(\Delta/2)^3}=\{0,1\}^{\Theta(B\Delta^{h+1})}$ chosen uniformly at random (since the input messages of nodes in $T_{h}$ were chosen uniformly at random). Therefore, the probability of picking the correct result is at most $2^{t-B(\Delta-1)^{h-2}(\Delta/2)^3}$,
    %. Any 
    \mamr{and any} algorithm that finishes within $t \leq \frac{1}{2}\cdot B(\Delta-1)^{h-2}(\Delta/2)^3$ rounds has at most $2^{-\frac{1}{2}\cdot B(\Delta-1)^{h-2}(\Delta/2)^3}$ probability of outputting the correct answer.
    
    \noindent\textbf{Local Broadcast.} The analysis of Local Broadcast is analogous to the analysis of \mam{multihop simulation}, except that there are $\Delta/2$ times fewer messages to transmit \mamr{(because the same message is transmitted to all nodes located within distance $h$ hops of the transmitter)}. The full analysis of Local Broadcast~is~below.

    There are $(\Delta/2)^2(\Delta-1)^{h-2}$ nodes in $T_h$ and each of them has $1$ message to nodes in $R$. Therefore, there are $(\Delta-1)^{h-2}(\Delta/2)^2 \pga{=} \Theta(\Delta^{h})$ messages to nodes in $R$ that are passing through nodes in $T_1$.

    Let $\mathcal{R}$ be the concatenated string of local randomness in all the nodes in $R$. The output of any receiver $u \in R$ must depend only on $\mathcal{R}$, node IDs, and the pattern of beeps and silences of nodes in $T_1$.

    There are $2^t$ possible patterns of beeps and silences in $t$ rounds. Therefore, the output of nodes in $R$ must be one of the $2^t$ possible distributions, where a distribution is over the randomness of $\mathcal{R}$. The correct output of nodes in $R$ is a string $\{0,1\}^{B(\Delta-1)^{h-2}(\Delta/2)^2}=\{0,1\}^{\Theta(B\Delta^{h})}$ chosen uniformly at random (since the input messages of nodes in $T_{h}$ were chosen uniformly at random). Therefore, the probability of picking the correct result is at most $2^{t-B(\Delta-1)^{h-2}(\Delta/2)^2}$,
    %. Any 
    \mamr{and any} algorithm that finishes within $t \leq \frac{1}{2}\cdot B(\Delta-1)^{h-2}(\Delta/2)^2$ rounds has at most $2^{-\frac{1}{2}\cdot B(\Delta-1)^{h-2}(\Delta/2)^2}$ probability of outputting the correct answer.
\end{proof}

    \begin{figure}[t]
        \centering
        \includegraphics[width=1\linewidth]{ctree.jpeg}
        \caption{Illustration for Theorem~\ref{thm:multihoplb}. Example of adversarial graph for $\Delta=4$ and $h=5$.}
        \label{fig:multihop_graph2}
    \end{figure}
    
\vspace*{-1.5ex}
\paragraph{Algorithm.}
%
\pga{A simple algorithm would repeatedly use a 1-hop Local Broadcast routine to flood the network with the messages until nodes at a distance $h$ received the messages. This, however, can take $\Omega(\Delta^{2h})$ rounds. Instead, we limit the flooding by only sending messages along the shortest paths to their destinations using a 1-hop simulation algorithm. The details of the algorithm as well as its analysis are presented next.}

In the beginning, 
%the
nodes use a standard protocol to disseminate their IDs up to distance $h$. They do it in $h$ subsequent epochs, each epoch $i$ of $t_i$ rounds sufficient to run our  Local Broadcast from Section~\ref{sec:primitives} (see Theorem~\ref{th:local_broadcast}) for messages of size $\Delta^i\log n$.
These messages contain different IDs learned by the node at the beginning of the current epoch.
A direct inductive argument, also using the property that there are at most $\Delta^i$ nodes at a distance at most $i$, shows that at the end of epoch $i$, each node knows the IDs of all nodes at a distance at most $i$ from it.
Additionally, each node records in which epoch $i$ it learned each known ID $v$ for the first time and from which of its neighbors $w$ -- and stores this information as a triple $(v,w,i)$.
%
The invariant for $i=h$ proves that at the end of epoch $h$, each node knows IDs of all nodes of distance at most $h$ from it. The round complexity is %clearly 
% $O(\Delta^{k}\log n \cdot \Delta^2\log^2 n)\le O(\Delta^{k+2}\log^3 n)$
$\sum_{i=1}^h \Delta^{i}\log n \cdot \Delta^2 \log n \pga{=} O(\Delta^{h+2} \log^2 n)$, and as will be seen later, it is subsumed by the round complexity of the second part of our algorithm (as the $\polylog n$ function in Theorem~\ref{thm:congest-sim} is asymptotically bigger than $\log^2 n$).

% NOT SURE IF THIS PARAGRAPH IS NEEDED:
Note that a sequence of triples $(v,w_1=v,1),\ldots,(v,w_{\ell},\ell)$,
stored at nodes $w_2,w_3,\dots,w_\ell,w_{\ell+1}=u$ respectively, represents a shortest path to node $v$ starting from the node $u$; the length of that path~is~$\ell$.

In the second part, nodes also proceed in epochs, but this time each epoch $i$ takes $t^*_i$ rounds sufficient to execute $1$-hop simulation algorithm from Section~\ref{sec:main-simulation} (see Theorem~\ref{thm:congest-sim}) for point-to-point messages of size $(B+\log n)\Delta^h$. 
Here $B$ denotes the known upper bound on the size of any input message.
In epoch $i$, every node $u$ transmits a (possibly different) message of size $(B+\log n)\Delta^h$ to each neighbor $w$. Such a message contains all the input messages of nodes within $i-1$ distance and the recipients of these messages such that $w$ is the next node on the saved shortest path to the recipient. The messages have already traveled $i-1$ hops, so their destination is at most $h-(i-1)$ hops away. More specifically, the message from node $u$
addressed to a neighbor $w$ in epoch $i$ 
contains pairs $(v,m_{z \rightarrow v})$, where $v$ is such that $(v,w,i')$ is stored at the node for some $i'\le h-(i-1)$ and $m_{z \rightarrow v}$ is a message received by the node $u$ in epoch $i-1$ (in case of $i-1=0$, it is the original message of the node addressed to $v$).
A direct inductive argument shows that at the end of each epoch $i$ a node knows at most $\Delta^i$ messages addressed to any node $v$ of distance $\ell \le h-i$ from the node. This invariant is based on the following arguments: 
\begin{itemize}
\item 
Because there is a unique neighbor $w$ of the node $u$ such that a triple $(v,w,\ell)$ is stored at the node, 
% \item 
the number of such nodes $v$ of distance at most $h-i+1$ from the node $u$ is at most $\Delta^{h-i+1}$, 
\item \pga{by the end of epoch $i-1$, node $u$ could receive messages to be relayed to $v$ from $\Delta^{i-1}$ different nodes at distance $i-1$,}
\item \pga{each message contains up to $B$-bit long original message and an ID of length $\log n$,}
\item 
hence, messages of size at most $(B+\log n)\Delta^{i-1} \cdot \Delta^{h-i+1} = (B+\log n) \Delta^h$ are being sent to each neighbor in epoch $i$, and by definition -- epoch $i$ has sufficient number of rounds to deliver them. 
\end{itemize}
The invariant for $i=h$ proves the desired property of $B$-bit $h$-hop simulation. The total number of rounds~is 
\[
O(h\cdot (B+\log n)\Delta^h \cdot \Delta^2\polylog n) 
%\le
\subseteq
O(h\cdot B\Delta^{h+2} \polylog n) 
\ ,
\]
where factor $h$ comes from the number of epochs, each sending at most $\Delta$ point-to-point messages of size at most $(B+\log n)\Delta^h$ to neighbors (by the invariant) using the $1$-hop simulation protocol with overhead $O(\Delta^2 \polylog n)$ (by Theorem~\ref{thm:congest-sim}).
Hence we proved the following.


\begin{theorem}\label{thm:multihopub}
There is a distributed deterministic algorithm solving the $B$-bit $h$-hop simulation problem \mam{in a beeping network} in $O(h\cdot B\Delta^{h+2}\polylog n)$ rounds.
\end{theorem}


%\input{localbroadcastlowerbound}

%\section{Missing proofs from Section~\ref{sec:main-simulation}}
%\section{Proofs from Section~\ref{sec:main-simulation} -- Analysis of Main Algorithm}
\section{Details of Section~\ref{sec:main-simulation} -- Analysis of the \alg Algorithm}
\label{sec:proofs-main-simulation}



\begin{figure}[thbp]
\centering
\begin{subfigure}[htbp]{0.30\textwidth}
\centering
\vspace*{-10ex}
\includegraphics[width=\linewidth]{beep001.jpeg}
\caption{Some part of a beeping network.}
\label{subfig:bn}
\end{subfigure}
\hspace{0.1in}
\begin{subfigure}[htbp]{0.30\textwidth}
\centering
\includegraphics[width=\linewidth]{beep002.jpeg}
\caption{Announcing super-round of some phase $j$: $\{c,g\}$ announce, $\{a,b,d,e,f\}$ hear noise, but only $\{b,d,e,f\}$ receive an extended-ID $\langle c\rangle$ and become $c$-$responsive$.}
\label{subfig:announce}
\end{subfigure}
\hspace{0.1in}
\begin{subfigure}[htbp]{0.30\textwidth}
\centering
\vspace*{-3ex}
\includegraphics[width=\linewidth]{beep004.jpeg}
\caption{Responding $3$ super-rounds within some sub-phase $a'$: $\{b,e\}$ respond and $c$ receives $\langle b\rangle\langle c\rangle\langle m_{b,c}\rangle$ and $\langle e\rangle\langle c\rangle\langle m_{e,c}\rangle$.}
\label{subfig:resp1}
\end{subfigure}
\\
\begin{subfigure}[htbp]{0.30\textwidth}
\centering
\vspace*{3ex}
\includegraphics[width=\linewidth]{beep005.jpeg}
\caption{Confirming $3$ super-rounds during sub-phase $a'$: $c$ confirms, $\{b,e\}$ receive $\langle c\rangle\langle b\rangle\langle m_{c,b}\rangle$ and $\langle c\rangle\langle e\rangle\langle m_{c,e}\rangle$ respectively. After this $\{b,e\}$ abandon the $c$-$responsive$ status and $\{\{b,c\},\{e,c\}\}$ are marked as realized.}
\label{subfig:conf1}
\end{subfigure}
\hspace{0.1in}
\begin{subfigure}[htbp]{0.30\textwidth}
\centering
\vspace*{-8ex}
\includegraphics[width=\linewidth]{beep006.jpeg}
\caption{Responding $3$ super-rounds within some sub-phase $a''$: $d$ responds and $c$ receives $\langle d\rangle\langle c\rangle\langle m_{d,c}\rangle$.}
\label{subfig:resp2}
\end{subfigure}
\hspace{0.1in}
\begin{subfigure}[htbp]{0.30\textwidth}
\centering
\vspace*{-3ex}
\includegraphics[width=\linewidth]{beep007.jpeg}
\caption{Confirming $3$ super-rounds during sub-phase $a''$: $c$ confirms, $d$ receives $\langle c\rangle\langle d\rangle\langle m_{c,d}\rangle$. After this $d$ abandons the $c$-$responsive$ status and $\{d,c\}$ is marked as realized.}
\label{subfig:conf2}
\end{subfigure}
\\
\begin{subfigure}[htbp]{0.30\textwidth}
\centering
\includegraphics[width=\linewidth]{beep008.jpeg}
\caption{Responding $3$ super-rounds within some sub-phase $a'''$: $f$ responds and $c$ receives $\langle f\rangle\langle c\rangle\langle m_{f,c}\rangle$.}
\label{subfig:resp3}
\end{subfigure}
\hspace{0.1in}
\begin{subfigure}[htbp]{0.30\textwidth}
\centering
\vspace*{5ex}
\includegraphics[width=\linewidth]{beep009.jpeg}
\caption{Confirming $3$ super-rounds during sub-phase $a'''$: $c$ confirms, $f$ receives $\langle c\rangle\langle f\rangle\langle m_{c,f}\rangle$. After this $f$ abandons the $c$-$responsive$ status and $\{f,c\}$ is marked as realized.}
\label{subfig:conf3}
\end{subfigure}
\hspace{0.1in}
\begin{subfigure}[htbp]{0.30\textwidth}
\centering
\includegraphics[width=\linewidth]{beep010.jpeg}
\caption{By the end of phase $j$ links $\{\{b,c\},\{d,c\},\{e,c\},\{f,c\}\}$ have been realized.}
\label{subfig:end}
\end{subfigure}
\caption{Illustration of \alg algorithm -- consecutive handshakes between the announcer $c$ and its responders during a phase.}
\label{fig:alg}
\end{figure}


\begin{proof}[Proof of Lemma~\ref{lem:correct-receiving}]
The proof is by contradiction -- suppose that in some super-round a node $w$ 
%stays silent and 
receives an extended-ID $z$ but the claim of the lemma does not hold. 
Without lost of generality, we may assume that this is the first such super-round.

Recall that the definition of receiving an extended-ID requires that node $w$ has been silent in this super-round. Note that if exactly one neighbor of node $w$ has been beeping during the super-round, it must have been an extended-ID of some node (by specification of the algorithm), and therefore node $w$ receives this extended-ID (as other neighbors do not beep at all).
Similarly, we argue that at least one neighbor of node $w$ must have been beeping (some extended-ID) in the super-round, as otherwise node $w$ would not have received any beep (and so, also no extended-ID) in the considered super-round.

In the remainder we focus on the complementary case that at least two neighbors of node $w$ have been beeping in the super-round, each of them some extended-ID (again, by specification of the algorithm, a node beeps only some extended-ID or stay silent during any super-round).

First, suppose that some two neighbors, $v_1,v_2$, beeped different extended-IDs, say $z_1\ne z_2$, respectively.
It means that node $w$ received more than $\log n$ beeps during the super-round: $\log n$ beeps coming from one of the extended-IDs and at least one more because the extended IDs of different nodes differ by at least one position. Hence, the received sequence of beeps does not form any extended-ID, as it must always have $\log n$ bits $1$ corresponding to the beeps. This contradicts the fact that $w$ receives an extended-ID in the considered super-round.

Second, suppose that all extended-IDs beeped by (at least two) neighbors on node $w$ are the same. If this happens in an announcing, or a first responding, or a first confirming super-round, it is a contradiction because all nodes that beep in such super-rounds beep their own extended-IDs, which are pairwise different.
%
If this happens in a second confirming super-round, it means that these two neighbors $v_1,v_2$ belong to the same set $\mF_{\Delta,k_i}(j)$, for some phase number $j$, and both of them received an extended-ID of $z$ in the preceding responding super-round. By the fact that the considered super-round is the first when the lemma's claim does not hold, we get that in this preceding responding super-round, both $v_1,v_2$ received the extended-ID of $z$ when $z$ was their unique beeping neighbor (beeping its own extended-ID, by the specification of the responding super-rounds).
This, however, implies that in the beginning of the current phase $j$, i.e., during its announcing super-round, both $v_1,v_2$ beeped their extended-IDs and, again by the choice of the current contradictory super-round, their neighbor $w$ could not have received any extended-ID -- this is a contradiction with the fact that $z$ was transmitting in the responding super-round preceding the considered (contradictory) super-round. More precisely, only $(j,\cdot)$-responsive nodes can transmit in responding super-rounds, but $z$ is not $(j,\cdot)$-responsive because it had not received any extended-IDs in the first (announcing) super-round of the current phase.

The last sub-case of the above scenario, when all extended-IDs beeped by (at least two) neighbors on node $w$ are the same, is as follows. 
If this situation happens in a second responding super-round, it means 
that these two neighbors $v_1,v_2$ are both $(j,z)$-responsive and beep extended-ID of $z$. This is, however, acceptable due to the exception in the lemma's statement.

This completes the proof of the lemma.
\end{proof}


\begin{proof}[Proof of Lemma~\ref{lem:correct-realization}]
It is enough to show that points (a) and (b) of the definition of link realization occurred in the last four super-rounds (two responding and two confirming) and also that the other node, $w$, (locally) marks link $\{v,w\}$ as realized at the same time when $v$ does.

If node $v$ marked the link $\{v,w\}$ as realized, it could be because of one of two reasons. 

First, it is in the set $\mF_{\Delta,k_i}(j)$, where $i$ is the number of the current epoch and $j$ is the number of the current phase and received an extended-ID of $w$ followed by its own extended-ID in the preceding two responding super-rounds. This satisfies point (a) of the definition of link realization, as both were beeped by node $w$, by the algorithm specification, and by Lemma~\ref{lem:correct-receiving}. Note that the exception in that lemma does not really apply here because if there were two or more neighbors beeping the same extended-ID (of $v$) in the second responding round, they would also be beeping their own extended-IDs in the first responding round, which could contradict the fact that $v$ received a single extended-ID at that super round.

This also means that $v$ has beeped its own extended-ID followed by extended-ID of $w$ in the last two confirming super-rounds, which must have been received by $w$ because $w$ is $(j,v)$-responsive (because only such nodes could have beeped in the preceding responding super-rounds) and thus its only neighbor in set $\mF_{\Delta,k_i}(j)$ (only such nodes are allowed to beep in confirming super-rounds) is $v$; here we use Fact~\ref{fact:single-beeping}. Hence, $w$ also marks link 
$\{v,w\}$ as realized at the end of the two confirming super-rounds; by the algorithm's specification, point (c) of the definition also holds in this case.

Second, it is $(j,z)$-responsive and received an extended-ID of $z$ followed by its own extended-ID in the current two confirming super-rounds. This satisfies point (b) of the definition, as both were beeped by node $z$, by the specification of the algorithm and Lemma~\ref{lem:correct-receiving} (exception in that lemma does not apply here because we now consider only confirming super-rounds).

This also means that $v$ beeped its own extended-ID followed by extended-ID of $z$ in the preceding two responding super-rounds (because $v$ is $(j,z)$-responsive and only such nodes could beep in the preceding responding super-rounds), which must have been received by $z$ (otherwise, by the specification of the algorithm, $z$ would not beep its extended-ID followed by the extended-ID of $v$ in the last two confirming super-rounds). Hence, $z$ also marks link $\{v,z\}$ as realized at the end of the two confirming super-rounds by the algorithm's specification, and point (c) of the definition also holds in this case.
\end{proof}



\begin{proof}[Proof of Lemma~\ref{lem:subphase-progress}]
The lemma follows from the definition of the $(n,k_i/2^{a-2},k_i/2^{a-1})$-avoiding selector $\mF_{k_i/2^{a-2},k_i/2^{a-1}}$ used throughout sub-phase $a$ of phase $j$ of epoch $i$. 
By specification of the sub-phase, only nodes $w$ such that $w$ is $(j,v)$-responsive and it does not marked link $\{v,w\}$ as realized take active part in sub-phase $a$ (in the sense that only those nodes can beep extended-IDs of itself followed by $v$ in pairs of responding super-rounds), while other neighbors of $v$ do not beep at all. The latter statement needs more justification -- in the beginning of the current phase, in the announcing super-round, $v$ must have beeped because some nodes have become $(j,v)$-responsive in this phase (w.l.o.g. we may assume that at least one node has become $(j,v)$-responsive, because otherwise the lemma trivially holds), therefore, by Lemma~\ref{lem:correct-receiving}, other neighbors of $v$ could not receive another announcement and become $(j,v')$-responsive, for some $v'\ne v$, and thus by the description of the algorithm -- they stay silent throughout the whole phase. 

By lemma assumption, there are at most $\Delta/2^{i+a-2}=k_i/2^{a-2}$ $(j,v)$-responsive nodes $w$ that have not marked link $\{v,w\}$ as realized by the beginning of the sub-phase. Hence, at least half of them will be in a singleton intersection with some set $\mF_{k_i/2^{a-2},k_i/2^{a-1}}(b)$, by Definition~\ref{def:avoid-selector} and Fact~\ref{fact:avoiding-selectors}, in which case $v$ receives their beeping in the corresponding pair of the responding super-rounds. Consequently, $v$ beeps back its own extended-ID and the extended-ID of $w$ in the following two confirming super-rounds. 

Node $w$ receives those beepings, as there is no other neighbor of $w$ who is allowed to beep in these two rounds -- indeed if there was, it would belong to set $\mF_{\Delta,k_i}(j)$ and thus it would have been beeping in the announcing super-round of this phase, preventing (together with neighbor $v$ of $w$) node $w$ from receiving anything in that super-round (by Lemma~\ref{lem:correct-receiving}), which contradicts the fact that $w$ must have received an extended-ID of $v$ in that round to become $(j,v)$-responsive (as assumed). Therefore, by the description of the algorithm, $w$ marks link $\{v,w\}$ as realized. This completes the proof that the number of $(j,v)$-responsive neighbors $w$ of $v$ who remain without realizing link $\{v,w\}$ becomes less than $\Delta/2^{i+a-1}$ at the end of the considered sub-phase. 
\end{proof}


\begin{proof}[Proof of Lemma~\ref{lem:phase-progress}]
It follows directly from the fact that a phase, after its announcing super-round, iterates sub-phases $a=1,\ldots,\log k_i$. Each subsequent sub-phase halves the number of not-realized links $\{v,w\}$, for $(j,v)$-responsive nodes $w$ and each announcing node $v$, c.f., Lemma~\ref{lem:subphase-progress}, starting from the assumed $2k_i$ maximum number of $(j,v)$-responding nodes (recall that $(j,v)$-responding nodes form a subset of those to whom links are not realized, hence there are at most $2k_i$ of them in the beginning).
\end{proof}


\begin{proof}[Proof of Lemma~\ref{lem:epoch-invariant}]
The proof is by induction on epoch number $i$.
Obviously, the invariant holds at the beginning of the first epoch, i.e., $\kappa_1\le k_1$, where $\kappa_i$ was defined as the sharp upper bound on the maximum number of not realized links at a node at the end of epoch $i$ and $k_i$ is the parameter used in the algorithm for epoch $i$. 

Consider epoch $i\ge 1$.
We have to prove that:
assuming that $\kappa_{i'}\le k_{i'}$, for any $1\le i' < i$, we also have $\kappa_i \le k_i$.
Technically we can assume that $\kappa_0=k_0=\Delta$.

Consider a node $w$. 
By the inductive assumption, it has at most $k_{i-1}$ neighbors $v$ such that link $\{v,w\}$ has not been marked by $w$ as realized.
By Definition~\ref{def:avoid-selector} and Fact~\ref{fact:avoiding-selectors} applied to $(n,\Delta,\Delta-k_i)$-avoiding selector $\mF_{\Delta,k_i}$, 
which sets are used for announcing super-rounds (and later for confirming super-rounds), the number of neighbors $v$ of node $w$ from whom node $w$ has not received their extended-ID during the announcing super-round is smaller than $k_i$. By Lemma~\ref{lem:phase-progress}, all such nodes $w$ realize their links, and by Lemma~\ref{lem:correct-realization}, also node $v$ realizes these links during the considered phase. Hence, the number of non-realized links incident to any node $w$ drops below $k_i$ by the end of epoch $k_i$.
\end{proof}



\section{Details of Section~\ref{sec:primitives} -- Algorithms and Analysis of Building Blocks}
\label{sec:prim_details}

In this section, we include the remaining details of our building blocks (Section~\ref{sec:primitives}). Theorems are restated for easy reference. First, let us introduce the following combinatorial object, to be used later.

\begin{definition}[Strong selector]
\label{def:strong-selector}
    A family $\mathcal{F}$ of subsets of $[n]$ of size at most $k$ each is called an \emph{$(n,k)$ strong selector} if for every non-empty subset $S \in [n]$ such that $|S| \leq k$, for every element $a \in S$, there exists a set $F \in \mathcal{F}$ such that $|F \cap S| = \{a\}$.
\end{definition}

Note that there are known constructions of $(n,\Delta)$ strong selectors of length at most $O(\Delta^2 \log n)$~\cite{5967914}. Next, we show how to use an $(n,\Delta)$ strong selector to perform a local broadcast.

\subsection{Local broadcast}
\label{sec:local-broadcast}

%One can perform local broadcast routine based on strong selectors. 
Our local broadcast routine is non-adaptive. That is, each node 
%will have a prepared-in-advance 
has a predefined schedule 
%determining 
specifying in which rounds beeps and in which rounds listens. 

\parhead{Assumptions} % \pga{The nodes IDs come from range $[1,n^c]$.} 
The nodes know the total number of nodes $n$, parameter $c$,
 %\mm{[[the range has been defined above, do we need to recall it?]]} 
the maximum degree $\Delta$ of the graph and have access to a global clock. Additionally, we assume that each node $v$ knows its neighborhood $N(v)$. (In Subsection~\ref{sub:neighbourhood} we will show how all nodes can learn their neighborhoods in $O(\Delta^2 \log^2 n)$ beeping rounds.)

\localbroadcastthm*
\remove{
\begin{theorem}
\label{th:local_broadcast}
\mm{Consider a Beeping Network where each node $v$ knows $n$, $\Delta$, and $N(v)$.}
    Assume that every message $m_v$ of each node $v$ has length at most $k$ for some $k>0$. Then, there is a deterministic distributed local broadcast algorithm that works in $O(k\Delta^2 \log^2 n)$ beeping rounds.
\end{theorem}
}

\begin{proof}
Consider an $(n^c,\Delta)$ strong selector $\mathcal{F}=\{S_1,S_2,\dots, S_L\}$ of length $L=O(\Delta^2 \log n)$, known to all nodes. Our local broadcast schedule will take $L$ rounds. At any round $i$, nodes $v \in S_i$ that have bit 1 (indicating to transmit) send a beep while all the other nodes are silent.

Consider any receiver $r$. Consider the set $N(r)$ of neighbors of $r$. Note that $|N(r)| \leq \Delta$. From the definition of a $(n^c,\Delta)$ strong selector $\mathcal{F}$, for every $v \in N(r)$ there exists an index $i$ such that $S_i \cap N(r) = \{v\}$. Therefore, for every pair of transmitters $v$ and receiver $r$ that are adjacent to each other, there exists a round $i$ such that $v$ is the only transmitting neighbor of $r$.

%Here we will assume that each node knows its neighborhood. In Subsection~\ref{sub:neighbourhood} we will show how all the nodes can learn their neighborhoods in $O(\Delta^2 \log^2 n)$ beeping rounds.

Since every node $v$ knows its neighborhood $N(v)$ and the sets $S_i$ for all $i$, node $v$ also knows for each neighbor $u \in N(v)$ at what round $t$ neighbor $u$ is the only neighbor transmitting. If at round $t$ node $v$ hears a beep, it means that $u$ transmitted bit 1. If at round $t$ node $v$ hears silence, it means that $u$ transmitted bit 0.

Therefore, after $L$ rounds the algorithm will go through the entire strong selector $\mathcal{F}$ and each node will learn a bit of information from each of its neighbors.

The procedure can be repeated $B$ times to broadcast messages of at most $B$ bits.
Hence, the claim follows.
%
\end{proof}


\subsection{Learning neighborhood}
\label{sub:neighbourhood}


Now we show how all the nodes can learn their neighborhoods in $O(\Delta^2 \log^2 n)$ beeping rounds. The following procedure will be non-adaptive. 

\parhead{Assumptions} %\pga{The nodes IDs come from range $[1,n^c]$.} 
The nodes know the total number of nodes $n$ and parameter $c$.
% \pga{the range of possible IDs $[1,n^c]$ for some constant $c \geq 1$}
% and have access to a global clock.

\learningneighthm*
\remove{
\begin{theorem}
\label{th:learning_neighbourhood}
\mm{Consider a Beeping Network where each node $v$ knows $n$.}
    There is a deterministic distributed learning neighborhood algorithm that works in $O(\Delta^2 \log^2 n)$ beeping rounds.
\end{theorem}
}

\begin{proof}
The nodes will beep according to a strong selector $\mF$ as in the previous subsection. This time, however, for each set $S_i \in \mF$ there will be $2\log n$ beeping rounds instead of $1$ round. First, nodes $v \in S_1$ will transmit for $2\log n$ rounds, then nodes $v \in S_2$ will transmit for $2\log n$ rounds and so on.

In each block of $2\log n$ rounds corresponding to a set $S_i$ for some $i$, each node $v \in S_i$ will encode its ID in the following way. For each bit in its ID, if the bit is 1, then the node listens for 1 round and then beeps. If the bit is 0, then the node beeps once and then listens for 1 round. The process is repeated for each bit in the ID.

After all $2\log n$ beeping rounds corresponding to a set $S_i$ for some $i$ pass, each node $v$ can look at the string of beeps and silences that it heard during the block. If there were beeps in both rounds $2k$ and $2k+1$ for some $k$, then there were multiple neighbors transmitting in the block and $v$ will ignore this block. Otherwise, the string of beeps encodes the ID of the only transmitting neighbor $u$. 
% node $v$ will assume that this string of bits is an ID of a node $u$. If the ID of $u$ is such that $u \in S_i$, then $u$ was the only neighbor of $v$ transmitting in the block. 
% \pga{[We may need to transmit 01 for each bit 1 and 10 for each bit 0 to really make this unambiguous.]} 
In particular, $u$ can be added to the list of neighbors of $v$. 
% On the other hand, if $u \notin S_i$, then that means that multiple neighbors of $v$ were overlapping their beeps during the block corresponding to $S_i$ and $u$ should not be added to the list of neighbors of $v$.

% \pga{According to the definition of strong selector $\mF$,} 
After all $L$ blocks of transmissions pass, each node $v$ heard from each $u \in N(v)$ in a block such that $u$ was the only transmitting neighbor. Therefore, each $u \in N(v)$ was added to the list of neighbors of $v$. No other nodes were added to the list of neighbors of $v$. Thus, $v$ knows its neighborhood from now on and the claim follows.
\end{proof}

\subsection{Cluster gathering}
\label{sec:gathering}

% \noindent \textbf{Aggregating information via overlapping Steiner trees.} 

\noindent \textbf{Assumptions.} %\pga{The nodes IDs come from range $[1,n^c]$.} 
Thanks to running cluster gathering inside the network decomposition algorithm, we will have access to additional structures. During the working of the network decomposition algorithm, each cluster $C$ will have a Steiner tree $S$ associated with it. All nodes $v \in C$ will be regular nodes in the Steiner tree $S$, while there may be some additional nodes $u \notin C$ that are Steiner nodes in $S$. All Steiner trees will have depth at most $O(\log^2 n)$, i.e., the diameter of the Steiner tree $S$ will be the same as the weak-diameter of the cluster $C$ that corresponds to $S$. Each node and each edge will be in at most $O(\log n)$ Steiner trees. Each Steiner tree $S$ will have a fixed root node $r$.
Given that our cluster gathering algorithm uses the local broadcast algorithm defined above, the previous assumptions also~apply.

% We can develop a cluster gathering algorithm where all nodes in all clusters work in parallel, resulting in only $O(local\_broadcast\cdot \log^4 n)$ beeping rounds.

\noindent \textbf{Effect and efficiency.} Given the above assumptions, we can develop an algorithm that gathers and aggregates limited information (each node passes at most $O(\log n)$ bits for each Steiner tree it is in) from all nodes in a cluster $C$ to the root $r$ of the corresponding Steiner tree $S$, with all clusters working in parallel. Additionally, the root $r$ can broadcast $O(\log n)$ bits to all nodes in $C$. The algorithm will work in $O(\Delta^2 \cdot \log^6 n)$ beeping rounds.

\noindent \textbf{Utilization.} The gathering algorithm will be used throughout the network decomposition algorithm but may be of independent use, especially since the network decomposition algorithm may output the Steiner trees it was using as well as the decomposition. Therefore, any algorithm using our network decomposition algorithm will have access to the appropriate Steiner trees to use for %gathering 
collecting information using our cluster gathering algorithm.

\clustergatherthm*
\remove{
\begin{theorem}
\label{th:cluster_gathering}
\mm{Consider a Beeping Network where each node $v$ knows $n$, $\Delta$, and $N(v)$.}
    There is a deterministic distributed cluster gathering algorithm that works in $O(\Delta^2 \log^4 n)$ beeping rounds.
\end{theorem}
}

%\noindent \textbf{Algorithm description.} 
\begin{proof}
The algorithm will utilize the local broadcast subroutine at each step. When we say "transmit", we mean to use local broadcast subroutine, unless specified otherwise. Each node $v \in C$ broadcasts its $O(\log n)$ bits (e.g., a number less than $n$) in parallel using the local broadcast subroutine $O(\log n)$ times. At each step, the parent $p$ of $v$ in the corresponding Steiner tree will listen for the message from $v$ as well as messages from its other children. Whenever $p$ receive messages from some of its children, $p$ prepares its own message (e.g., sum of numbers provided by its children in the current step, assuming that the sum is smaller than $n$) of $O(\log n)$ bits. The process will repeat until the root $r$ receives all the messages, which will last the number of steps equal to the depth of the Steiner tree, $O(\log^2 n)$. Note that node $p$ may receive messages from its children in multiple steps; in that case in each step $t$ node $p$ transmits the aggregate of messages it received at $t$, thus transmitting multiple times.

Similarly, one can broadcast a message from $r$ to all the nodes $v\in C$ in $O(\log^2 n)$ steps. The entire algorithm takes $O(\log^2 n)$ steps, which is $O(t_{local\_broadcast} \cdot \log^2 n)$ beeping rounds and the claim follows.
\end{proof}

%%%%%%%%%%%%%%%%%%%%%%%%%%%%%%%%%%%%%%%%%%%%%%%%%%%%

\subsection{Network Decomposition}
\label{sec:decomposition}


% \subsection{Network decomposition from~\cite{ghaffari2021improved}}


In this section, we present how to adapt the network decomposition algorithm of Ghaffari et al.~\cite{ghaffari2021improved} to the beeping model. The only changes are in the way that nodes communicate. The original algorithm was %made
designed for the \congest model, where communication was straightforward. 
Instead, for the beeping model, we have to carefully implement all the concurrent communication, so that the algorithm remains efficient.
%
First, let us recall the result from~\cite{ghaffari2021improved}. 

% \noindent\textbf{Notations:} $b$ is the length of identifiers, $n$ is the number of nodes in graph $G$.

\begin{theorem}\cite{ghaffari2021improved}
    There is a deterministic distributed algorithm that computes a $(\log n, \log^2 n)$ network decomposition in $O(\log^5 n)$ \congest rounds.
\end{theorem}

We adapt the above result to the beeping model and obtain the next theorem.

\networkdecompthm*
\remove{
\begin{theorem}
\label{thm:local-decomposition}
    There is a deterministic distributed algorithm that computes a $(\log n, \log^2 n)$ network decomposition in $O(\Delta^2 \log^8 n)$ beeping rounds.
\end{theorem}
}

The algorithm works as follows. In preprocessing, each node learns its neighborhood, so that we will be able to use local broadcast routine.
%freely. 
This part can take up to $O(\Delta^2 \log^2 n)$ beeping rounds (see Theorem~\ref{th:learning_neighbourhood}). Next, we perform the network decomposition algorithm from~\cite{ghaffari2021improved} with all the communication carefully replaced, as shown 
%in Section~\ref{sec:rundown} 
below. This completes our network decomposition algorithm.

% \dk{In order to combine our Local Broadcast with the tools in~\cite{ghaffari2021improved}, detailed check of these tools need to be done to assure, among others, that they rely on Local Broadcast (not the general \congest model rules).
% We provide the relevant parts in Section~\ref{sec:Ghaffari-verbatim}.}



%\subsubsection{Summary of messages}
\label{sec:rundown}

% Let $local\_broadcast$ denote the number of beeping rounds to perform a local broadcast in the beeping model, i.e., make sure that for every node $v$ all neighbors of $v$ receive a message from $v$ \pga{of length at most $\log n$. According to Theorem~\ref{th:local_broadcast}, such local broadcast lasts $O(\Delta^2 \log^2 n$ beeping rounds.}

\paragraph{Summary of messages:}
We need to carefully replace all the communication from~\cite{ghaffari2021improved} with routines that work in the beeping model. We will use our local broadcast and cluster gathering routines from Section~\ref{sec:primitives}.

% Below we present a list of messages transmitted in \pga{the original algorithm for the \congest model~\cite{ghaffari2021improved} and how to implement this information transmission in the beeping model. The original algorithm can be viewed in Subsection~\ref{sec:ghaffari}. In the list below we count the messages done per \emph{step}. There will be at most $O(\log^2 n)$ steps in the algorithm.}

Below, we present a list of messages transmitted in %Subsection~\ref{sec:Ghaffari-verbatim} in 
a step and how to implement this information %transmission 
propagation in the beeping model (for convenience, we attach the network decomposition algorithm from~\cite{ghaffari2021improved} in Section~\ref{sec:Ghaffari-verbatim}):
\begin{enumerate}
    \item \textbf{Before proposals.} A node $v$ needs to check which clusters are adjacent to it and what are the parameters of these clusters (id, level). Every node $u$ can broadcast the parameters of the cluster $u$ is in, which is at most $O(\log n)$ bits. This is done once per step, %. This used to cost
    at a cost of $O(1)$ \congest rounds per step. In the beeping model, it can be done via a local broadcast with messages of length at most $O(\log n)$ bits. According to Theorem~\ref{th:local_broadcast}, it will cost $O(\Delta^2 \log^2 n)$ beeping rounds per step.
    \item \textbf{Proposals.} Proposals to join a cluster can be transmitted to all neighbors if the target cluster (or the target node) is specified in the message. Other receivers may ignore the message. This part used to cost $O(1)$ \congest rounds per step. 
    % In beeping model, this can be done in $O(local\_broadcast)$ rounds per step.
    In the beeping model, it can be done via a local broadcast, where the message specifies the IDs of both, sender and receiver, meaning messages of length at most $O(\log n)$ bits. According to Theorem~\ref{th:local_broadcast}, it will cost $O(\Delta^2 \log^2 n)$ beeping rounds per step.
    \item \textbf{Gathering proposals.} The leader of each cluster should learn the total number of proposals and the total number of tokens in the cluster. This part was done in $O(\log^3 n)$ \congest rounds. %of \congest model. 
    Note that the algorithm keeps the Steiner tree of $O(\log^2 n)$ diameter for each cluster, such that each node (and therefore each edge) is in at most $O(\log n)$ Steiner trees. Additionally, each node that participates in this cluster gathering  can add the numbers of proposals/tokens it receives from other nodes and then transmit the sums instead of relaying each message separately; these sums can never exceed the total number of nodes $n$, so this guarantees that each node only transmits $O(\log n)$ bits per Steiner tree it is in. Therefore, we can use the cluster gathering algorithm. According to Theorem~\ref{th:cluster_gathering}, this takes at most $O(\Delta^2 \log^6 n)$ beeping rounds.
    % given a $O(\log^2 n)$ diameter Steiner tree such that each node (and therefore each edge) is in at most $O(\log n)$ Steiner trees. 
    % This can be done in $O(\log^3 n)$ rounds of \congest model, since a message from a leaf to the root can collide with at most $O(\log^3 n)$ different messages. In beeping model, we can use the 
    % this can be done in $O(local\_broadcast \cdot \log^3 n)$ rounds per step. \pga{There should be an additional factor of $O(\log n)$ due to the length of the messages passed. Also, how do we determine, which message (from which Steiner tree) to transmit first?}
    \item \textbf{Responding to proposals.} Each node informs all its neighbors either that all the proposals were accepted or that all the proposing nodes should be killed. This part used to cost $O(1)$ \congest rounds per step. In the beeping model, the same can be done in a local broadcast, with $O(1)$ bit messages. According to Theorem~\ref{th:local_broadcast}, it will cost $O(\Delta^2 \log n)$ beeping rounds.
    \item \textbf{Stalling.} If a cluster decides to stall, all nodes neighboring the cluster should be informed about it. This \mm{part} used to cost $O(1)$ \congest rounds per step. In the beeping model, \mm{the same} can be done in a local broadcast with $O(1)$ bit messages. According to Theorem~\ref{th:local_broadcast}, it will cost $O(\Delta^2 \log n)$ beeping rounds.
\end{enumerate}

In the procedure described above, there is a total of $O(\Delta^2 \cdot \log^6 n)$ beeping rounds required per step. There are up to $O(\log n)$ steps per phase and up to $O(\log n)$ phases in the algorithm, which results in $O(\Delta^2 \cdot \log^8 n)$ beeping rounds for the entire algorithm. Note that only the means of communication changed. Therefore, the correctness of the algorithm is unaffected. This completes the proof of Theorem~\ref{thm:local-decomposition}.

% \pga{[TODO: We don't know whether we hear a single signal or multiple signals!]}

% \pga{[Solution 2: Have a preprocessing phase where all nodes learn their neighborhood. This can be done in $O(\Delta^2 \log^2 n)$ beeping rounds.]}

% \pga{Solution 1: Let $n$ be the number of nodes in the network. Consider $(n^2,\Delta)$ strong selector $\mathcal{F}$. Consider a node $v$ with id equal $id(v)$ that attempts to transmits a message $m$ with the first $\log n$ bits equal to $\hat{m}$. Let $id(v,m)$ be the concatenation of $id(v)$ and $\hat{m}$. Every $id(v,m)$ corresponds to exactly one element of the universum $U=[n^2]$. A node $v$ with a message $m$ will "transmit" (bit 0 is encoded as silence, while bit 1 is encoded as a beep) a bit in round $t$ if $t$-th set of $\mathcal{F}$ contains $id(v,m)$. 
% % [TODO: We need a single bit, i.e., 1 or 0. We can encode a single bit in two rounds: code 10 (i.e., transmission and then silence) corresponds to bit 0, while code 01 (silence followed by transmission) corresponds to bit 1.]

% The strong selector ensures that each node-message pair successfully transmits a bit to any neighbor.

% Intuitively, we copied each node $n$ times (id of a node is concatenated with one of $n$ different beginnings of a message on $\log n$ bits), but only at most one copy can transmit at any round. A transmission from node $u$ to node $v$ can overlap with at most $\Delta - 1$ other neighbors of $v$. The strong selector $\mathcal{F}$ guarantees that for every listener $v$ that has at most $\Delta$ neighbors $N(v)$, for each $u\in N(v)$ there is at least one set $S \in \mathcal{F}$ such that $S \cap N(v) = {u}$. }

% \pga{[COMMENT: Do we assume that a node knows its neighbors? If yes, the following paragraph is trivial. Otherwise, we may need a linear program. Problem with a linear program: Nodes that transmit 0 are silent. We can run a single local broadcast routine to learn our neighbors.]

% Furthermore, after every set of $\mathcal{F}$ was used, every listener $v$ can determine which round was the round with exactly one neighbor transmitting and which rounds had multiple neighbors transmitting.

% Let $f_v^i$ denote feedback that node $v$ receives at round $i$, i.e., $f_v^i=0$ if node $v$ heard silence in round $i$ and $f_v^i=1$ if node $v$ heard at least one beep in round $i$. After all $L$ rounds, each node $v$ determines the rounds with a unique transmitter based on the feedback $f_v^i$ it received at round $i$ for all $1 \leq i \leq L$ and the knowledge of what every set $S_i \in \mathcal{F}$ contains. Let $x_u=1$ if $u$ is beeping and $x_u=0$ otherwise. The linear program consists of $nL$ equations, one for each pair node-round $(v,i)$:

% \[ \sum_{u \in N(v)} x_u = 0 \text{ iff } f_v^i=0\]
% \[ \sum_{u \in N(v)} x_u \geq 1 \text{ iff } f_v^i=1\]

% }


% \subsection{Possible outputs}
% \pga{
% Besides the $(C,D)$ network decomposition, the algorithm above computes a few other structures that can be %helpful.
% \mm{useful and are of independent interest.}
% One such structure is a Steiner tree of depth $O(D)$ for each cluster. Additionally, a root of each Steiner tree can serve as a leader for the corresponding cluster\footnote{Note, the leader may be outside of the cluster it is serving.}. During the construction of the Steiner trees, any aggregate information (i.e., information that can be aggregated without increasing the size of messages beyond $O(\log n)$ bits, e.g., the number of descendants) can be gathered in each node in the Steiner tree without changing the asymptotic cost of the operation. During the gathering proposals stage the leader can broadcast $O(1)$ bits to its entire clusters without changing the asymptotic cost of the operation.
% }

% \pga{TODO: A separate algorithm for gathering information in multiple Steiner trees simultaneously. We use Lemmas/Theorems from Ghaffari's paper, so that we know that the Steiner trees have the right properties and there are only $\log n$ overlaps. Then, we can use THIS algorithm to describe gathering proposals part, as well as provide this algorithm as a separate primitive. This algorithm should be described before gathering proposals, perhaps before network decomposition section.}

%\section{Missing details from Section~\ref{}}
%\section{Details from Section~\ref{sec:decomposition}: Efficient Network Decomposition in Beeping Networks}


% \textcolor{green}{---------------------------------???????????}

% \dk{In order to prove Theorem~\ref{thm:local-decomposition}, we need to make sure that the network decomposition algorithm in~\cite{ghaffari2021improved} can indeed be implemented using only beeps.}
% indeed uses Local Broadcast as communication tool.}
\subsection{GGR network decomposition algorithm}
\label{sec:Ghaffari-verbatim}

\noindent\textbf{Notation:} $b$ is the length of identifiers, $n$ is the number of nodes in graph $G$.

The remainder of this subsection, which is important from perspective of assurance that our Local Broadcast could be combined with the tools in~\cite{ghaffari2021improved}, is cited from~\cite{ghaffari2021improved} verbatim.

\noindent\textbf{Construction  outline:}   The  construction  has  $2(b+\log n) =O(\log n)$ phases. Each phase has $28(b+\log n) =O(\log n)$ steps.  Initially, all nodes of $G$ are \emph{living}, during the construction some living nodes \emph{die}. Each living node is  part  of  exactly  one  cluster.   Initially,  there  is  one cluster $C_v$ for each vertex $v\in V(G)$ and we define the identifier $id(C)$ of $C$ as the unique identifier of $v$ and use $id_i(C)$  to  denote  the $i$-th  least  significant  bit  of  $id(C)$. From now on, we talk only about identifiers of clusters and do not think of vertices as having identifiers, though they will still use them for simple symmetry breaking tasks.   Also,  at  the  beginning,  the  Steiner  tree $T_{C_v}$ of a cluster $C_v$ contains just one node, namely $v$ itself, as a  terminal  node.   Clusters  will  grow  or  shrink  during the iterations, while their Steiner trees collecting their vertices can only grow.  When a cluster does not contain any nodes, it does not participate in the algorithm anymore.

\noindent\textbf{Parameters of each cluster:} Each cluster $C$ keeps two other parameters besides its identifier $id(C)$ to make its decisions:  its number of tokens $t(C)$ and its level $lev(C)$.The number of tokens can change in each step -- more precisely it is incremented by one whenever a new vertex joins $C$, while it does not decrease when a vertex leaves $C$.  The number of tokens only decreases when $C$ actively deletes nodes.  We define $t_i(C)$ as the number of tokens of $C$ at the beginning of the $i$-th phase and set $t_1(C) = 1$. Each  cluster  starts  in  level  $0$.   The  level  of  each cluster does not change within a phase $i$ and can only increment by one between two phases; it is bounded by $b$.  We denote with $lev_i(C)$ the level of $C$ during phase $i$.   Moreover,  for  the  purpose  of  the  analysis,  we  keep track  of  the  potential  $\Phi(C)$  of  a  cluster $C$ defined  as $\Phi_i(C) = 3i - 2lev_i(C) + id_{lev_i(C)+1}(C)$.  The potential of each cluster stays the same within a phase.

\noindent\textbf{Description  of  a  step:} In each step, first, each node $v$ of each cluster $C$ checks whether it is adjacent to a  cluster $C'$ such that  $lev(C')<lev(C)$. If  so, then $v$ proposes  to  an arbitrary  neighboring  cluster $C'$ among the neighbors with the smallest level $lev(C')$ and if there is a choice, it prefers to join clusters with $id_{lev(C')+1}(C') = 1$.  Otherwise, if there is a neighboring cluster $C'$ with $lev(C') = lev(C)$ and $id_{lev(C')+1}(C') = 1$, while  $id_{lev(C)+1}(C)  =  0$,  then $v$ proposes  to  arbitrary such cluster.

Second, each cluster $C$ collects the number of proposals  it  received.   Once  the  cluster  has  collected  the number  of  proposals,  it  does  the  following.   If  there are $p$ proposing nodes,  then they join $C$ if and only if $p \geq t(C)/(28(b+ \log n))$.  The denominator is equal to the number of steps. If $C$ accepts these proposals, then $C$ receives $p$ new tokens, one from each newly joined node. On the other hand, if $C$ does not accept the proposals as their number is not sufficiently large, then $C$ decides to kill all those proposing nodes.  These nodes are then removed from $G$.  Cluster $C$ pays $p \cdot 14(b+ \log n)$ tokens for this, i.e., it pays $14(b+ \log n)$ tokens for every vertex that it deletes.  These tokens are forever gone.  Then the cluster does not participate in growing anymore,  until the end of the phase and throughout that time we call that cluster \emph{stalling}.  The cluster tells that it is stalling to neighboring nodes so that they do not propose to it. At the end of the phase, each stalling cluster increments its level by one.

If the cluster is in level $b-1$ and goes to the last level $b$, it will not grow anymore during the whole algorithm, and  we  say  that  it  has finished.    Other  neighboring clusters can still eat its vertices (by this we mean that vertices of the finished clusters may still propose to join other clusters). 

Whenever  a  node $u$ joins  a  cluster $C$ via  a  vertex $v\in C$, we add $u$ to the Steiner tree $T_C$ as a new terminal node and connect it via an edge $uv$.  Whenever a node $u\in C$ is deleted or eaten by a different cluster, it stays in the Steiner tree $T_C$ but is changed to a non-terminal node.



%\input{boolean}
\section{Concluding Discussion}
\label{sec:conclusion}
In this paper, we examine the  trade-offs among privacy, utility, and efficiency while fine-tuning an LLM. The traditional wisdom of achieving privacy comes at the cost of computational inefficiency using dedicated methods like DP. In contrast, we demonstrate that parameter efficient fine-tuning methods like LoRA, initially designed for efficiency, achieves privacy of sensitive data without any computational overhead. Simultaneously, LoRA retains the utility of general language understanding compared to DP, or even full-fine-tuning, realizing the superiority of LoRA in optimizing all three aspects. Towards our investigation, we  establish the significance of redefining privacy and utility using a careful distinction between sensitive and non-sensitive counterparts of the fine-tuned data. Through case studies, we demonstrate how existing measures exaggerate privacy threats and undermine the utility of an LLM. Our paper calls for a joint venture of   privacy and systems communities in achieving privacy-aware efficient fine-tuning of LLMs while retaining utility.

%\bibliographystyle{alpha}
\bibliographystyle{plain}
\bibliography{bibliography}

\newpage

\appendix

\end{document}