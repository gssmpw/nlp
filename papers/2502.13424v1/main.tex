\documentclass{article}

% Language setting
% Replace `english' with e.g. `spanish' to change the document language
\usepackage[english]{babel}
\usepackage{tikz}
\usetikzlibrary{automata, positioning, arrows}

% Set page size and margins
% Replace `letterpaper' with `a4paper' for UK/EU standard size
%\usepackage[letterpaper,top=2cm,bottom=2cm,left=3cm,right=3cm,marginparwidth=1.75cm]{geometry}
\usepackage[letterpaper,top=1in,bottom=1in,left=1in,right=1in]{geometry}

% Useful packages
\usepackage{amsmath}
\usepackage{graphicx}
\usepackage{amsthm}
\usepackage[colorlinks=true, allcolors=blue]{hyperref}
\usepackage{color}
\usepackage[boxed,vlined,ruled,linesnumbered]{algorithm2e}
%\usepackage{algorithm}
%\usepackage{algorithmic}
\usepackage{subcaption}
\usepackage{multirow}
\usepackage{colortbl}

%\newtheorem{theorem}{Theorem}
\newtheorem{hypothesis}{Hypothesis}
\newtheorem{lemma}{Lemma}
\newtheorem{claim}{Claim}
\newtheorem{definition}{Definition}
\newtheorem{corollary}{Corollary}
\newtheorem{fact}{Fact}

\newtheorem*{thmA*}{Main Algorithmic Theorem}
%\newtheorem*{thmL*}{Main Lower Bound Theorem}
\newtheorem*{thmL*}{Lower Bound Theorem}

% re-statable theorems
\usepackage{thmtools, thm-restate}
\declaretheorem{theorem}

% colors
% \newcommand{\lm}[1]{{\color{blue} #1}}
\newcommand{\dk}[1]{{\color{purple} #1}}
% \newcommand{\sk}[1]{{\color{green} #1}}
\newcommand{\pga}[1]{{\color{red} #1}}
\newcommand{\mm}[1]{{\color{cyan}{#1}}}
\newcommand{\mam}[1]{{\color{cyan}{#1}}}
\newcommand{\mamr}[1]{{\color{cyan}{#1}}}
% \newcommand{\pg}[1]{{\color{blue}{#1}}}
\newcommand{\todo}[1]{{\color{red} [TODO: #1]}}
% uncomment to remove colors
\renewcommand{\dk}[1]{#1}
% \renewcommand{\sk}[1]{#1}
\renewcommand{\mm}[1]{#1}
\renewcommand{\mam}[1]{#1}
\renewcommand{\mamr}[1]{#1}
% \renewcommand{\pg}[1]{#1}
\renewcommand{\pga}[1]{#1}
% \renewcommand{\lm}[1]{#1}

\newcommand{\mF}{\mathcal{F}}
\newcommand{\cP}{\mathcal{P}}
\newcommand{\cG}{\mathcal{G}}
\newcommand{\cM}{\mathcal{M}}
\newcommand{\cN}{\mathcal{N}}
\newcommand{\remove}[1]{}
\newcommand{\polylog}{\text{\em\ polylog }}%{\mbox{ polylog } }
%\newcommand{\polylog}{\mathrm{polylog}~}

\usepackage{xspace}
\newcommand{\parhead}[1]{\noindent{\textbf{#1.}\xspace}}
\newcommand{\congest}{{\fontfamily{cmss}\selectfont CONGEST}\xspace}
\newcommand{\alg}{{\sc c2b}\xspace}

%\renewcommand{\paragraph}[1]{\vspace*{-1ex}\textbf{#1}}

\title{Beeping Deterministic \congest Algorithms in Graphs
%Efficiently 
}
%Towards Efficient Deterministic Graph Algorithms\\ in Beeping Networks\\
%{Beeping out \congest Network Algorithms Deterministically}
%{Efficient Algorithms in Beeping Networks\\ via Almost Tight Simulations of \congest Algorithms}
%{Transforming Algorithms from \congest to Beeping Networks\\
%\mm{Simulating \congest Algorithms on Beeping Networks?}}
%{Deterministic Beeping Networks}

%\author{}
%\author{Pawel Garncarek \and Dariusz R. Kowalski \and Shay Kutten \and Miguel A. Mosteiro}
\author{
Pawel Garncarek\thanks{University of Wroclaw, Institute of Computer Science, Wroclaw, Poland; supported by the National Science Center, Poland (NCN), grant 2020/39/B/ST6/03288.} 
\and Dariusz R. Kowalski\thanks{Augusta University, Department of Computer \& Cyber Sciences, Augusta, GA, USA} 
\and Shay Kutten\thanks{Technion, Israel Institute of Technology, Haifa, Israel; a large part of this author's research was performed while he was on a sabbatical at Fraunhofer SIT in Darmstadt. Research supported in part by the Israeli Science Foundation and by The Bernard M. Gordon Center for Systems Engineering at the Technion.} 
\and Miguel A. Mosteiro\thanks{Pace University, Computer Science Department, New York, NY, USA; partially supported by Pace SRC grant and Kenan fund.}
}

\date{}

\begin{document}
\maketitle


%\vspace*{-5.1ex}

\begin{abstract}
\begin{abstract}  
Test time scaling is currently one of the most active research areas that shows promise after training time scaling has reached its limits.
Deep-thinking (DT) models are a class of recurrent models that can perform easy-to-hard generalization by assigning more compute to harder test samples.
However, due to their inability to determine the complexity of a test sample, DT models have to use a large amount of computation for both easy and hard test samples.
Excessive test time computation is wasteful and can cause the ``overthinking'' problem where more test time computation leads to worse results.
In this paper, we introduce a test time training method for determining the optimal amount of computation needed for each sample during test time.
We also propose Conv-LiGRU, a novel recurrent architecture for efficient and robust visual reasoning. 
Extensive experiments demonstrate that Conv-LiGRU is more stable than DT, effectively mitigates the ``overthinking'' phenomenon, and achieves superior accuracy.
\end{abstract}  
\end{abstract}

% to be removed
%Logging:


UserID
ScenarioID
controlMode
requestID (Nummer der Request)
elapsed Time
distanceTravelledSinceLastLog
distanceToEndOfInstructedPath (Luftlinie zum Ende des gefolgten Pfades)
lengthOfCurrentInstructedPath (Gesamtlänge noch zu folgendem gezeichneter Pfad + generierte Verbindungsstrecke Auto <-> Pfad)
lengthOfCurrentInstructedInputPath (Gesamtlänge noch zu folgendem gezeichneter Pfad)
distanceToEnd (Auto <-> Ende bis wo hin operiert werden muss)
vehiclePosition (Verlauf der unity positions vektoren -> nun auch mit offset, also vergleichbar, wenn man damit was anfangen will)
vehicleSpeed (in kmh)
constructionSiteEntered (ob erstmalig unmittelbar vor der ConstructionSite angekommen)
endReached ("Request geschafft")
closest Lane (nummer lane 0-indiziert von links bis rechts)
currentLaneDeviation (Spurabweichung zur closest Lane Mitte  auch in metern)
timeOfCollisionAvoidanceTraffic (Zeit summe wenn Verkehr in der CollisionAvoidance Range ist (egal ob hinten oder vorne, noch ob es das auto stört))
timeOfCollisionAvoidanceObstacle (wie oben für Obstacles <- Baustellenfahrzeuge, Baustellenmarker, Metallgrenzen links und rechts von Straße)
timeOfCollisionAvoidancePedestrian (wie oben für Fußgänger)
amountOfAdditionInput (Summe wie oft der Pfad erweitert wurde +1 pro Aktion)
amountOfAdditionMarkers (Summe um wie viele "Stellen" der Pfad erweitert wurde +x pro Aktion <- relevant für Anteil an Snap to Mid in Trajectory, für PathPlanning und Waypoint immer +1)
amountOfSnapToMiddleInput (+1 wenn Addition Input und mindestens einmal Snap To Mid verwendet wurde, für path planning immer 1, weil im Grunde immer Mittig)
amountOfSnapToMiddleMarkers (wie amount Of AdditionMarkers, nur dass nur hochgezählt wird wenn die "Stelle" gesnappt wurde)
amountOfReadjustmentInput (wie obiges nur statt Addition: Readjustment, heißt: anfang und ende sind in alter Trajektorie oder nah parallel dazu, Waypoint wird verschoben, PathPlanning backwards, immer +1)
timeSinceLastInput (Zeit seit letztem input)
currentlyNeglectedTime (aktuelle Zeit wie lang vehikel schon still steht (colision avoidance) oder wenn Ende vom angegebenen Pfad)
blindTimeSum (wenn vehikel nicht neglected und schon eingaben gemacht wurden -> aufsummieren Zeit wenn nicht in Main oder secondary view)
isMainRequest
isSecondaryRequest
totalRequestAmount (insgesamte Anzahl an Requests in dem Szenario, hat nicht direkt was mit dieser request instanz zu tun, aber generelle info)
sideOfConstructionSite (ob construction site links oder rechts bei der Straße




Für ein Szenario/ case jeweils::
OverviewLog:
UserID
ScenarioID
controlMode
elapsedTime
activeRequests (wie viele grade angezeigt werden)
succeededRequests (wie viele bisher geschafft sind)
mousePositionX
mousePositionY
usingMultiView (also wenn zwei Requests gleichzeitig angezeigt = true sonst = false)
usingSingleViewMain (auch nur = true wenn nur ein vehikel in der Hauptanzeige ist)
EyeGazeArea (das was du noch meintest -> was aktuell angeschaut wird: string wert aus: RequestList, MainView, SecondaryView, Instructions, None)
mouseClickLeft (summe)
mouseClickRight (summe)
pressLeftCtrl (summe)
pressShift (summe)
TimestampLog:
UserID
ScenarioID
controlMode
elapsed time
timeStampEvent (String wert aus RequestStarted, RequestFinished, RequestOpenedMain, RequestOpenedSecondary, RequestRemovedMain, RequestRemovedSecondary)
additionalInfo (eigentlich immer nur die Request Nummer des vehikels)
-> Es kommen immer Logs rein sobald eines der genannten Events auftritt. Bei den Request Opened/removed events muss man aufpassen, da jede variablen änderung geloggt wird. Demnach sind für den selben elapsedTimeTimestamp nur die Anfangs und Endzustände der jeweiligen Slots relevant.



UnitEyeLog (-> Standard Implementation) + 
 4 areas of interest definierst: links das panel mit den requests, das video mit der aktuellen Szene (ego view),  die szene top down view, und noch das Panel mit den Gründen unten
ist im OverviewLog

\thispagestyle{empty}

\newpage

 \setcounter{page}{1}


\section{Introduction}


\begin{figure}[t]
\centering
\includegraphics[width=0.6\columnwidth]{figures/evaluation_desiderata_V5.pdf}
\vspace{-0.5cm}
\caption{\systemName is a platform for conducting realistic evaluations of code LLMs, collecting human preferences of coding models with real users, real tasks, and in realistic environments, aimed at addressing the limitations of existing evaluations.
}
\label{fig:motivation}
\end{figure}

\begin{figure*}[t]
\centering
\includegraphics[width=\textwidth]{figures/system_design_v2.png}
\caption{We introduce \systemName, a VSCode extension to collect human preferences of code directly in a developer's IDE. \systemName enables developers to use code completions from various models. The system comprises a) the interface in the user's IDE which presents paired completions to users (left), b) a sampling strategy that picks model pairs to reduce latency (right, top), and c) a prompting scheme that allows diverse LLMs to perform code completions with high fidelity.
Users can select between the top completion (green box) using \texttt{tab} or the bottom completion (blue box) using \texttt{shift+tab}.}
\label{fig:overview}
\end{figure*}

As model capabilities improve, large language models (LLMs) are increasingly integrated into user environments and workflows.
For example, software developers code with AI in integrated developer environments (IDEs)~\citep{peng2023impact}, doctors rely on notes generated through ambient listening~\citep{oberst2024science}, and lawyers consider case evidence identified by electronic discovery systems~\citep{yang2024beyond}.
Increasing deployment of models in productivity tools demands evaluation that more closely reflects real-world circumstances~\citep{hutchinson2022evaluation, saxon2024benchmarks, kapoor2024ai}.
While newer benchmarks and live platforms incorporate human feedback to capture real-world usage, they almost exclusively focus on evaluating LLMs in chat conversations~\citep{zheng2023judging,dubois2023alpacafarm,chiang2024chatbot, kirk2024the}.
Model evaluation must move beyond chat-based interactions and into specialized user environments.



 

In this work, we focus on evaluating LLM-based coding assistants. 
Despite the popularity of these tools---millions of developers use Github Copilot~\citep{Copilot}---existing
evaluations of the coding capabilities of new models exhibit multiple limitations (Figure~\ref{fig:motivation}, bottom).
Traditional ML benchmarks evaluate LLM capabilities by measuring how well a model can complete static, interview-style coding tasks~\citep{chen2021evaluating,austin2021program,jain2024livecodebench, white2024livebench} and lack \emph{real users}. 
User studies recruit real users to evaluate the effectiveness of LLMs as coding assistants, but are often limited to simple programming tasks as opposed to \emph{real tasks}~\citep{vaithilingam2022expectation,ross2023programmer, mozannar2024realhumaneval}.
Recent efforts to collect human feedback such as Chatbot Arena~\citep{chiang2024chatbot} are still removed from a \emph{realistic environment}, resulting in users and data that deviate from typical software development processes.
We introduce \systemName to address these limitations (Figure~\ref{fig:motivation}, top), and we describe our three main contributions below.


\textbf{We deploy \systemName in-the-wild to collect human preferences on code.} 
\systemName is a Visual Studio Code extension, collecting preferences directly in a developer's IDE within their actual workflow (Figure~\ref{fig:overview}).
\systemName provides developers with code completions, akin to the type of support provided by Github Copilot~\citep{Copilot}. 
Over the past 3 months, \systemName has served over~\completions suggestions from 10 state-of-the-art LLMs, 
gathering \sampleCount~votes from \userCount~users.
To collect user preferences,
\systemName presents a novel interface that shows users paired code completions from two different LLMs, which are determined based on a sampling strategy that aims to 
mitigate latency while preserving coverage across model comparisons.
Additionally, we devise a prompting scheme that allows a diverse set of models to perform code completions with high fidelity.
See Section~\ref{sec:system} and Section~\ref{sec:deployment} for details about system design and deployment respectively.



\textbf{We construct a leaderboard of user preferences and find notable differences from existing static benchmarks and human preference leaderboards.}
In general, we observe that smaller models seem to overperform in static benchmarks compared to our leaderboard, while performance among larger models is mixed (Section~\ref{sec:leaderboard_calculation}).
We attribute these differences to the fact that \systemName is exposed to users and tasks that differ drastically from code evaluations in the past. 
Our data spans 103 programming languages and 24 natural languages as well as a variety of real-world applications and code structures, while static benchmarks tend to focus on a specific programming and natural language and task (e.g. coding competition problems).
Additionally, while all of \systemName interactions contain code contexts and the majority involve infilling tasks, a much smaller fraction of Chatbot Arena's coding tasks contain code context, with infilling tasks appearing even more rarely. 
We analyze our data in depth in Section~\ref{subsec:comparison}.



\textbf{We derive new insights into user preferences of code by analyzing \systemName's diverse and distinct data distribution.}
We compare user preferences across different stratifications of input data (e.g., common versus rare languages) and observe which affect observed preferences most (Section~\ref{sec:analysis}).
For example, while user preferences stay relatively consistent across various programming languages, they differ drastically between different task categories (e.g. frontend/backend versus algorithm design).
We also observe variations in user preference due to different features related to code structure 
(e.g., context length and completion patterns).
We open-source \systemName and release a curated subset of code contexts.
Altogether, our results highlight the necessity of model evaluation in realistic and domain-specific settings.





%% Please add the following required packages to your document preamble:
% \usepackage[table,xcdraw]{xcolor}
% If you use beamer only pass "xcolor=table" option, i.e. \documentclass[xcolor=table]{beamer}
\begin{table}[]
\begin{tabular}{lllllll}
                                                                                                                     & Storyline Development Stage &                       & Character Design Stage &                        & Character Drawing Stage &                       \\
\multicolumn{7}{c}{\cellcolor[HTML]{D9D9D9}Needs \& Challenges}                                                                                                                                                                                                                \\
                                                                                                                     & Needs                       & Challenges            & Needs                  & Challenges             & Needs                   & Challenges            \\
gauge reader reactions                                                                                               & \multicolumn{1}{r}{3}       &                       &                        & \multicolumn{1}{r}{3}  &                         &                       \\
organize the story flow                                                                                              & \multicolumn{1}{r}{3}       & \multicolumn{1}{r}{4} &                        &                        &                         &                       \\
find interesting story sources that could captivate readers’ interest                                                & \multicolumn{1}{r}{2}       & \multicolumn{1}{r}{4} &                        &                        &                         &                       \\
struggled with creating a story                                                                                      &                             & \multicolumn{1}{r}{8} &                        &                        &                         &                       \\
maintaining a consistent storyline regardless of their condition                                                     &                             & \multicolumn{1}{r}{1} &                        &                        &                         &                       \\
get feedback from readers                                                                                            &                             &                       & \multicolumn{1}{r}{2}  &                        &                         &                       \\
get specific design ideas                                                                                            &                             &                       & \multicolumn{1}{r}{1}  &                        &                         &                       \\
uniqueness of their characters                                                                                       &                             &                       & \multicolumn{1}{r}{4}  & \multicolumn{1}{r}{10} &                         &                       \\
\begin{tabular}[c]{@{}l@{}}enhance the efficiency in their\\ work process by improving repetitive tasks\end{tabular} &                             &                       &                        &                        & \multicolumn{1}{r}{7}   &                       \\
receive materials related to composition                                                                             &                             &                       &                        &                        & \multicolumn{1}{r}{4}   &                       \\
provided to research reference materials                                                                             &                             &                       &                        &                        & \multicolumn{1}{r}{1}   &                       \\
struggled with drawing characters in a variety of compositions                                                       &                             &                       &                        &                        &                         & \multicolumn{1}{r}{6} \\
limited drawinig skills                                                                                              &                             &                       &                        &                        &                         & \multicolumn{1}{r}{6} \\
get information that could be used as reference material                                                             & \multicolumn{1}{r}{2}       &                       & \multicolumn{1}{r}{6}  & \multicolumn{1}{r}{2}  & \multicolumn{1}{r}{1}   &                       \\
\multicolumn{7}{c}{\cellcolor[HTML]{D9D9D9}Expectations}                                                                                                                                                                                                                       \\
Expectations for Generating Ideas                                                                                    & \multicolumn{1}{r}{9}       &                       & \multicolumn{1}{r}{11} &                        & \multicolumn{1}{r}{1}   &                       \\
Expectations for Receiving References from AI                                                                        & \multicolumn{1}{r}{4}       &                       & \multicolumn{1}{r}{0}  &                        & \multicolumn{1}{r}{5}   &                       \\
Expectations of rapid visualization using AI                                                                         & \multicolumn{1}{r}{0}       &                       & \multicolumn{1}{r}{2}  &                        & \multicolumn{1}{r}{2}   &                       \\
\multicolumn{7}{c}{\cellcolor[HTML]{D9D9D9}Considerations}                                                                                                                                                                                                                     \\
Skepticism Regarding AI Capabilities                                                                                 & \multicolumn{1}{r}{2}       &                       & \multicolumn{1}{r}{3}  &                        & \multicolumn{1}{r}{0}   &                       \\
Refusal to Use AI Due to Copyright Infringement and Concerns About the Author's Identity                             & \multicolumn{1}{r}{3}       &                       & \multicolumn{1}{r}{1}  &                        & \multicolumn{1}{r}{2}   &                       \\
Low Utility of AI Outputs                                                                                            & \multicolumn{1}{r}{3}       &                       & \multicolumn{1}{r}{1}  &                        & \multicolumn{1}{r}{1}   &                      
\end{tabular}
\end{table}
\putsec{related}{Related Work}

\noindent \textbf{Efficient Radiance Field Rendering.}
%
The introduction of Neural Radiance Fields (NeRF)~\cite{mil:sri20} has
generated significant interest in efficient 3D scene representation and
rendering for radiance fields.
%
Over the past years, there has been a large amount of research aimed at
accelerating NeRFs through algorithmic or software
optimizations~\cite{mul:eva22,fri:yu22,che:fun23,sun:sun22}, and the
development of hardware
accelerators~\cite{lee:cho23,li:li23,son:wen23,mub:kan23,fen:liu24}.
%
The state-of-the-art method, 3D Gaussian splatting~\cite{ker:kop23}, has
further fueled interest in accelerating radiance field
rendering~\cite{rad:ste24,lee:lee24,nie:stu24,lee:rho24,ham:mel24} as it
employs rasterization primitives that can be rendered much faster than NeRFs.
%
However, previous research focused on software graphics rendering on
programmable cores or building dedicated hardware accelerators. In contrast,
\name{} investigates the potential of efficient radiance field rendering while
utilizing fixed-function units in graphics hardware.
%
To our knowledge, this is the first work that assesses the performance
implications of rendering Gaussian-based radiance fields on the hardware
graphics pipeline with software and hardware optimizations.

%%%%%%%%%%%%%%%%%%%%%%%%%%%%%%%%%%%%%%%%%%%%%%%%%%%%%%%%%%%%%%%%%%%%%%%%%%
\myparagraph{Enhancing Graphics Rendering Hardware.}
%
The performance advantage of executing graphics rendering on either
programmable shader cores or fixed-function units varies depending on the
rendering methods and hardware designs.
%
Previous studies have explored the performance implication of graphics hardware
design by developing simulation infrastructures for graphics
workloads~\cite{bar:gon06,gub:aam19,tin:sax23,arn:par13}.
%
Additionally, several studies have aimed to improve the performance of
special-purpose hardware such as ray tracing units in graphics
hardware~\cite{cho:now23,liu:cha21} and proposed hardware accelerators for
graphics applications~\cite{lu:hua17,ram:gri09}.
%
In contrast to these works, which primarily evaluate traditional graphics
workloads, our work focuses on improving the performance of volume rendering
workloads, such as Gaussian splatting, which require blending a huge number of
fragments per pixel.

%%%%%%%%%%%%%%%%%%%%%%%%%%%%%%%%%%%%%%%%%%%%%%%%%%%%%%%%%%%%%%%%%%%%%%%%%%
%
In the context of multi-sample anti-aliasing, prior work proposed reducing the
amount of redundant shading by merging fragments from adjacent triangles in a
mesh at the quad granularity~\cite{fat:bou10}.
%
While both our work and quad-fragment merging (QFM)~\cite{fat:bou10} aim to
reduce operations by merging quads, our proposed technique differs from QFM in
many aspects.
%
Our method aims to blend \emph{overlapping primitives} along the depth
direction and applies to quads from any primitive. In contrast, QFM merges quad
fragments from small (e.g., pixel-sized) triangles that \emph{share} an edge
(i.e., \emph{connected}, \emph{non-overlapping} triangles).
%
As such, QFM is not applicable to the scenes consisting of a number of
unconnected transparent triangles, such as those in 3D Gaussian splatting.
%
In addition, our method computes the \emph{exact} color for each pixel by
offloading blending operations from ROPs to shader units, whereas QFM
\emph{approximates} pixel colors by using the color from one triangle when
multiple triangles are merged into a single quad.


\section{Model}
\label{sec:model}
Let $[N] = \{1, 2, \dots, N \}$ be a set of $N$ agents.
We examine an environment in which a system interacts with the agents over $T$ rounds.
Every round $t\leq T$ comprises $N$ \emph{sessions}, each session represents an encounter of the system with exactly one agent, and each agent interacts exactly once with the system every round.
I.e., in each round $t$ the agents arrive sequentially. 


\paragraph{Arrival order} The \emph{arrival order} of round $t$, denoted as $\ordv_t=(\ord_t(1),\dots, \ord_t(N))$, is an element from set of all permutations of $[N]$. Each entry $q$ in $\ordv_t$ is the index of the agent that arrives in the $q^{\text{th}}$ session of round $t$.
For example, if $\ord_t(1) = 2$, then agent $2$ arrives in the first session of round $t$.
Correspondingly, $\ord_t^{-1}(i)=q$ implies that agent $i$ arrives in the $q^{\text{th}}$ session of round $t$. 

As we demonstrate later, the arrival order has an immediate impact on agent rewards. We call the mechanism by which the arrival order is set \emph{arrival function} and denote it by $\ordname$. Throughout the paper, we consider several arrival functions such as the \emph{uniform arrival} function, denoted by $\uniord$, and the \emph{nudged arrival} $\sugord$; we introduce those formally in Sections~\ref{sec:uniform} and~\ref{sec:nudge}, respectively.

%We elaborate more on this concept in Section~\ref{sec: arrival}.


\paragraph{Arms} We consider a set of $K \geq 2$ arms, $A = \{a_1, \ldots, a_K\}$. The reward of arm $a_i$ in round $t$ is a random variable $X_i^t \sim \mathcal{D}^t_i$, where the rewards $(X_i^t)_{i,t}$ are mutually independent and bounded within the interval $[0,1]$. The reward distribution $\mathcal{D}^t_i$ of arm $a_i$, $i\in [K]$ at round $t\in T$ is assumed to be non-stationary but independent across arms and rounds. We denote the realized reward of arm $a_i$ in round $t$ by $x_i^t$. We assume \emph{reward consistency}, meaning that rewards may vary between rounds but remain constant within the sessions of a single round. Specifically, if an arm $a_i$ is selected multiple times during round~$t$, each selection yields the same reward $x_i^t$, where the superscript $t$ indicates its dependence on the round rather than the session. This consistency enables the system to leverage information obtained from earlier sessions to make more informed decisions in later sessions within the same round. We provide further details on this principle in Subsection~\ref{subsec:information}.


\paragraph{Algorithms} An algorithm is a mapping from histories to actions. We typically expect algorithms to maximize some aggregated agent metric like social welfare. Let $\mathcal H^{t,q}$ denote the information observed during all sessions of rounds $1$ to $t-1$ and sessions $1$ to $q-1$ in round $t$.  The history $\mathcal H^{t,q}$ is an element from $(A \times [0,1])^{(t-1) \cdot N +q-1}$, consisting of pairs of the form (pulled arm, realized reward). Notice that we restrict our attention to \emph{anonymous} algorithms, i.e., algorithms that do not distinguish between agents based on their identities. Instead, they only respond to the history of arms pulled and rewards observed, without conditioning on which specific agent performed each action.
%In the most general case, algorithms make decisions at session $q$ of round $t$  based on the entire history $\mathcal H^{t,q}$ and the index of the arriving agent $\ord_t(q)$. %Furthermore, we sometimes assume that algorithms have Bayesian information, i.e., algorithms are aware of the distributions $(\mathcal D_i)^K_{i=1}$. 
Furthermore, we sometimes assume that algorithms have Bayesian information, meaning they are aware of the reward distributions $(\mathcal{D}^t_i)_{i,t}$. If such an assumption is required to derive a result, we make it explicit. %Otherwise, we do not assume any additional knowledge about the algorithm’s information. %This distinction allows us to analyze both general algorithms without prior distributional knowledge and specialized algorithms that leverage Bayesian information.


\paragraph{Rewards} Let $\rt{i}$ denote the reward received by agent $i \in [N]$ at round $t$, and let $\Rt{i}$ denote her cumulative reward at the end of round $t$, i.e., $\Rt{i} = \sum_{\tau=1}^{t}{r^{\tau}_{i}}$. We further denote the \emph{social welfare} as the sum of the rewards all agents receive after $T$ rounds. Formally, $\sw=\sum^{N}_{i=1}{R^T_i}$. We emphasize that social welfare is independent of the arrival order. 


\paragraph{Envy}
We denote by $\adift{i}{j}$ the reward discrepancy of agents $i$ and $j$ in round $t$; namely, $\adift{i}{j}= \rt{i} - \rt{j}$. %We call this term \omer{name??} reward discrepancy in round $t$. 
The (cumulative) \emph{envy} between two agents at round $t$ is the difference in their cumulative rewards. Formally, $\env_{i,j}^t= \Rt{i} - \Rt{j}$ is the envy after $t$ rounds between agent $i$ and $j$. We can also formulate envy as the sum of reward discrepancies, $\env_{i,j}^t= \sum^{t}_{\tau=1}{\adif{i}{j}^\tau}$. Notice that envy is a signed quantity and can be either positive or negative. Specifically, if $\env_{i,j}^t < 0$, we say that agent $i$ envies agent $j$, and if $\env_{i,j}^t > 0$, agent $j$ envies agent $i$. The main goal of this paper is to investigate the behavior of the \emph{maximal envy}, defined as
\[
\env^t = \max_{i,j \in [N]} \env^t_{i,j}.
\]
For clarity, the term \emph{envy} will refer to the maximal envy.\footnote{ We address alternative definitions of envy in Section~\ref{sec:discussion}.} % Envy can also be defined in alternative ways, such as by averaging pairwise envy across all agents. We address average envy in Section~\ref{sec:avg_envy}.}
Note that $\env_{i,j}^t$ are random variables that depend on the decision-making algorithm, realized rewards, and the arrival order, and therefore, so is $\env^t$. If a result we obtain regarding envy depends on the arrival order $\ordname$, we write $\env^t(\ordname)$. Similarly, to ease notation, if $\ordname$ can be understood from the context, it is omitted.



\paragraph{Further Notation} We use the subscript $(q)$ to address elements of the $q^{\text{th}}$ session, for $q\in [N]$.
That is, we use the notation $\rt{(q)}$ to denote the reward granted to the agent that arrives in the $q^{\text{th}}$ session of round $t$ and $\Rt{(q)}$ to denote her cumulative reward. %Additionally, we introduce the notation $\at{(q)}$ to denote the arm pulled in that session.
Correspondingly, $\sdift{q}{w} = \rt{(q)} - \rt{(w)}$ is the reward discrepancy of the agents arriving in the $q^{\text{th}}$ and $w^{\text{th}}$ sessions of round $t$, respectively. 
To distinguish agents, arms, sessions and rounds, we use the letters $i,j$ to mark agents and arms, $q,w$ for sessions, and $t,\tau$ for rounds.


\subsection{Example}
\label{sec: example}
To illustrate the proposed setting and notation, we present the following example, which serves as a running example throughout the paper.

\begin{table}[t]
\centering
\begin{tabular}{|c|c|c|c|}
\hline
$t$ (round) & $\ordv_t$ (arrival order) & $x_1^t$ & $x_2^t$ \\ \hline
1           & 2, 1                     & 0.6     & 0.92    \\ \hline
2           & 1, 2                     & 0.48    & 0.1     \\ \hline
3           & 2, 1                     & 0.15    & 0.8     \\ \hline
\end{tabular}
\caption{
    Data for Example~\ref{example 1}.
}
\label{tbl: example}
\end{table}

\begin{algorithm}[t]
\caption{Algorithm for Example~\ref{example 1}}
\label{alguni}
\DontPrintSemicolon 
\For{round $t = 1$ to $T$}{
    pull $a_{1}$ in the first session\label{alguniexample: first}\\
    \lIf{$x^t_1 \geq \frac{1}{2}$}{pull $a_{1}$ again in second session \label{alguniexample: pulling a again}}
    \lElse{pull $a_{2}$ in second session \label{alguniexample: sopt else}}
}
\end{algorithm}


\begin{example}\label{example 1}
We consider $K=2$ uniform arms, $X_1,X_2 \sim \uni{0,1}$, and $N=2$ for some $T\geq 3$. We shall assume arm decision are made by Algorithm~\ref{alguni}: In the first session, the algorithm pulls $a_1$; if it yields a reward greater than $\nicefrac{1}{2}$, the algorithm pulls it again in the second session (the ``if'' clause). Otherwise, it pulls $a_2$.



We further assume that the arrival orders and rewards are as specified in Table~\ref{tbl: example}. Specifically, agent 2 arrives in the first session of round $t=1$, and pulling arm $a_2$ in this round would yield a reward of $x^1_2 = 0.92$. Importantly, \emph{this information is not available to the decision-making algorithm in advance} and is only revealed when or if the corresponding arms are pulled.




In the first round, $\boldsymbol{\eta}^1 = \left(2,1\right)$; thus, agent 2 arrives in the first session.
The algorithm pulls arm $a_1$, which means, $a^1_{(1)} = a_1$, and the agent receives $r_{2}^1=r_{(1)}^1=x_1^1=0.6$.
Later that round, in the second session, agent 1 arrives, and the algorithm pulls the same arm again since $x^1_1 = 0.6 \geq \nicefrac{1}{2}$ due to the ``if'' clause.
I.e., $a^1_{(2)} = a_1$ and $r_{1}^1 = r_{(2)}^1 = x_1^1 = 0.6$.
Even though the realized reward of arm $a_2$ in that round is higher ($0.92$), the algorithm is not aware of that value.
At the end of the first round, $R^1_1 = R^1_{(2)} = R^1_2 = R^1_{(1)} = 0.6$. The reward discrepancy is thus $\adif{1}{2}^1 = \adif{2}{1}^1= \sdif{2}{1}^1 = 0.6 - 0.6 =0$. 



In the second round, agent 1 arrives first, followed by agent 2.
Firstly, the algorithm pulls arm $a_1$ and agent 1 receives a reward of $r_{1}^2 = r_{(1)}^2 = x_1^2 = 0.48$.
Because the reward is lower than $\nicefrac{1}{2}$, in the second session the algorithm pulls the other arm ($a^2_{(2)} = a_2$), granting agent 2 a reward of $r_{2}^2 = r_{(2)}^2 = x_2^2 = 0.1$.
At the end of the second round, $R^2_1 = R^2_{(1)} = 0.6 + 0.48 = 1.08$ and $R^2_2 = R^2_{(2)} = 0.6 + 0.1 = 0.7$. Furthermore, $\sdif{2}{1}^2 = \adif{2}{1}^2 = r^2_{2} - r^2_{1} = 0.1 - 0.48 = -0.38$.

In the third and final round, agent 2 arrives first again, and receives a reward  of $0.15$ from $a_1$. When agent 1 arrives in the second session, the algorithm pulls arm $a_2$, and she receives a reward of $0.8$. As for the reward discrepancy, $\sdif{2}{1}^3 = \adif{2}{1}^3 = r^3_{2} - r^3_{1} = 0.15 - 0.8 = -0.75$. 

Finally, agent 1 has a cumulative reward of $R^3_1 = R^3_{(2)} = 0.6 + 0.48 + 0.8 = 1.88$, whereas agent~2 has a cumulative reward of $R^3_2 = R^3_{(1)} = 0.6 + 0.1 + 0.15 = 0.85$. In terms of envy, $\env^1_{1,2}= \adif{1}{2}^1 =0$, $\env^2_{1,2}=\adif{1}{2}^1+\adif{1}{2}^2= 0.38$, and $\env^3_{1,2} = -\env^3_{2,1} = R^3_1-R^3_2 = 1.88-0.85 = 1.03$, and consequently the envy in round 3 is $\env^3 = 1.03$.
\end{example}


\subsection{Information Exploitation}
\label{subsec:information}

In this subsection, we explain how algorithms can exploit intra-round information.
Since rewards are consistent in the sessions of each round, acquiring information in each session can be used to increase the reward of the following sessions.
In other words, the earlier sessions can be used for exploration, and we generally expect agents arriving in later sessions to receive higher rewards.
Taken to the extreme, an agent that arrives after all arms have been pulled could potentially obtain the highest reward of that round, depending on how the algorithm operates.

To further demonstrate the advantage of late arrival, we reconsider Example~\ref{example 1} and Algorithm~\ref{alguni}. 
The expected reward for the agent in the first session of round $t$ is $\E{\rt{(1)}}=\mu_1=\frac{1}{2}$, yet the expected reward of the agent in the second session is
\begin{align*}
\E{\rt{(2)}}=\E{\rt{(2)}\mid X^t_1 \geq \frac{1}{2} }\prb{X^t_1 \geq \frac{1}{2}} + \E{\rt{(2)}\mid X^t_1 < \frac{1}{2} }\prb{X^t_1 < \frac{1}{2}};
\end{align*}
thus, $\E{\rt{(2)}} =\E{X^t_1\mid X^t_1 \geq \frac{1}{2} }\cdot \frac{1}{2} + \mu_2\cdot\frac{1}{2} = \frac{5}{8}$.
Consequently, the expected welfare per round is $\E{\rt{(1)}+\rt{(2)}}=1+\frac{1}{8}$, and the benefit of arriving in the second session of any round $t$ is $\E{\rt{(2)} - \rt{(1)}} = \frac{1}{8}$. This gap creates envy over time, which we aim to measure and understand.
%This discrepancy generates envy over time, and our paper aims to better understand it.
\subsection{Socially Optimal Algorithms}
\label{sec: sw}
Since our model is novel, particularly in its focus on the reward consistency element, studying social welfare maximizing algorithms represents an important extension of our work. While the primary focus of this paper is to analyze envy under minimal assumptions about algorithmic operations, we also make progress in the direction of social welfare optimization. See more details in Section~\ref{sec:discussion}.%Due to space limitations, we defer the discussion on socially optimal algorithms to  \ifnum\Includeappendix=0{the appendix}\else{Section~\ref{appendix:sociallyopt}}\fi.




% Since our model is novel and specifically the reward consistency element, it might be interesting to study social welfare optimization. While the main focus of our paper is to study envy under minimal assumptions on how the algorithm operates, we take steps toward this direction as well. Due to space limitations, we defer the discussion on socially optimal algorithms to  \ifnum\Includeappendix=0{the appendix}\else{Section~\ref{appendix:sociallyopt}}\fi.  We devise a socially optimal algorithm for the two-agent case, offer efficient and optimal algorithms for special cases of $N>2$ agents, and an inefficient and approximately optimal algorithm for any instance with $N>2$. Moreover, we address the welfare-envy tradeoff in Section~\ref{sec:extensions}.


% Social welfare, unlike envy, is entirely independent of the arrival order. While the main focus of our paper is to study envy under minimal assumptions on how the algorithm operates, socially optimal algorithms might also be of interest. Due to space limitations, we defer the discussion on socially optimal algorithms to  \ifnum\Includeappendix=0{the appendix}\else{Section~\ref{appendix:sociallyopt}}\fi. We devise a socially optimal algorithm for the two-agent case, offer efficient and optimal algorithms for special cases of $N>2$ agents, and an inefficient and approximately optimal algorithm for any instance with $N>2$. %Furthermore, we treat the welfare-envy tradeoff of the special case of Example~\ref{example 1}.



\vspace*{-1ex}
\section{Initial Results}
\label{sec:primitives}

In this section, we present beeping protocols for four fundamental network problems, usually used as building blocks of more complex tasks. Namely, Local Broadcast, Cluster Gathering, Learning Neighborhood, and Network Decomposition. 
%These protocols will be used as building blocks in our simulation of \congest rounds in beeping networks (Section~\ref{sec:main-simulation}).}
%algorithms that will be used in the network decomposition algorithm, namely, algorithms solving efficiently the problems of learning neighborhood, local broadcast and cluster gathering.
%
The following theorems establish the performance of our protocols. The details of the algorithms as well as the proofs of the theorems are left to Section~\ref{sec:prim_details}.

Recall that IDs of nodes come from the range $[1,n^c]$, for some constant $c \geq 1$.

\begin{restatable}[]{theorem}{localbroadcastthm} %\begin{theorem}
\label{th:local_broadcast}
    Let $\cN$ be a Beeping Network with 
    %set $V$ of 
    $n$ nodes, where each node 
    $v$
    %$v\in V$ 
    knows $n$, parameter~$c$, the maximum degree $\Delta$, and its neighborhood $N(v)$, and holds a message $m_v$ of length at most $B>0$.
    %Then, there is a deterministic distributed local broadcast algorithm that works in $O(k\Delta^2 \log^2 n)$ beeping rounds.
    There is a deterministic distributed algorithm that solves local broadcast on $\cN$ in $O(B\Delta^2 \log n)$ beeping~rounds.
    %\dk{???? SHOULDN'T IT BE $O(\Delta^2 (B+\log n)\log n)$ ????}
%\end{theorem}
\end{restatable}

\vspace*{-1.5ex}
\begin{restatable}[]{theorem}{learningneighthm} %\begin{theorem}
\label{th:learning_neighbourhood}
    Let $\cN$ be a Beeping Network with 
    %set $V$ of 
    $n$ nodes, where each node 
    %$v\in V$ 
    $v$
    knows $n$ and parameter~$c$.
    %There is a deterministic distributed learning-neighborhood algorithm that works in $O(\Delta^2 \log^2 n)$ beeping rounds.
    There is a deterministic distributed algorithm that solves learning neighborhood  on $\cN$ in $O(\Delta^2 \log^2 n)$~beeping~rounds.
%\end{theorem}
\end{restatable}

\vspace*{-1.5ex}
\begin{restatable}[]{theorem}{clustergatherthm} %\begin{theorem}
\label{th:cluster_gathering}
    Let $\cN$ be a Beeping Network with %set $V$ of 
    $n$ nodes, where each node 
    $v$
    %$v\in V$ 
    knows $n$, parameter $c$, the maximum degree $\Delta$, and its neighborhood $N(v)$.
    %There is a deterministic distributed cluster gathering algorithm that works in $O(\Delta^2 \log^4 n)$ beeping rounds.
    There is a deterministic distributed algorithm that solves cluster gathering on $\cN$ in $O(\Delta^2 \log^4 n)$ beeping rounds.
%\end{theorem}
\end{restatable}

% \pga{In the next theorem, we assume that node IDs come from range $[1,n]$.}

\vspace*{-1.5ex}
\begin{restatable}[]{theorem}{networkdecompthm} %\begin{theorem}
\label{thm:local-decomposition}
    Let $\cN$ be a Beeping Network with set $V$ of $n$ nodes, where each node $v\in V$ knows $n$, parameter $c$ and the maximum degree $\Delta$.
    There is a deterministic distributed algorithm that computes a $(\log n, \log^2 n)$-network decomposition of $\cN$ in $O(\Delta^2 \log^8 n)$ beeping rounds.
%\end{theorem}
\end{restatable}

% \pga{[TODO: Check if GGR algorithm can really be adapted to $n^c$ IDs trivially.]}

%%%%%%%%%%%%%%%%%%%%%%%%%%%%%%%%%%%%%%%%%%%%%%%%%%%%%%%%%%%

\vspace*{-3ex}
\section{Simulation of a \congest Round in Beeping Networks}
\label{sec:main-simulation}

%We already showed how to efficiently simulate a single round of a \congest model in a general beeping network, provided each node wants to send the same message to all of its neighbors. It already allows us to simulate many algorithms designed for the \congest model on a beeping network, in a deterministic and distributed way. 
Unfortunately, not all efficient graph algorithms in the \congest networks have the property of always broadcasting the same (short) message to every neighbor, which we exploit in Section~\ref{sec:local-broadcast}.\footnote{%
Note that in the LOCAL model, where the sizes of messages are of second importance (as long as they are polynomial), nodes can combine individual messages into one joint message and send it to all neighbors.}
%
In this section, we present a deterministic distributed algorithm that simulates a round of {\em any algorithm in the \congest model}, even if the algorithm sends different messages to neighbors.
It is only somewhat (polylogarithmically) slower than the more restricted one
(local broadcast, which required a node to send the same message to all its neighbors),
given in Theorem~\ref{th:local_broadcast} and Section~\ref{sec:local-broadcast}, but it is adaptive and uses heavier machinery.
%-- we show later that any non-adaptive solutions (beeping codes) are substantially less efficient and require $\Omega(\Delta^3)$ rounds. \mm{[[MM: do we prove this??]]}
 % we had to construct an adaptive algorithm using a technique for more complex codes
%  (puting a hifor the general simulator.
  %
 The novel construction is built hierarchically using the known family of code called ``avoiding selectors''. This, intuitively, already says ``when to beep.'' However, it is still possible that, for example, when two neighbors of   some node $v$ 
 %each 
 send a (different) message, of multiple bits per message, node $v$ will receive a ``message'' that is a logical OR of the two.
This is efficiently resolved by the new adaptive algorithm by employing a 3-stage handshake procedure, which sends pieces of the code that now serve in identifying what the IDs and messages are; it allows to spot overlapping transmissions from more than one neighbor and successfully decode those that do not overlap. Intuitively, each stage is ``triggered'' by a different level of the code.


%taken for different parameters. 
% 3-stage adaptive  
%handshake
%procedure (announcing, responding and confirming). on the top of the code. 
%which is an adaptive part. 
%This procedure is tightly correlated with the hierarchical structure of the code -- higher level code triggers announcing, while the lower level codes trigger responding and confirming.}

\vspace*{-2ex}
\paragraph{Preliminaries and Challenges.}
Suppose every node has a possibly different message to deliver to each of its neighbors. We could use the algorithm from Section~\ref{sub:neighbourhood} to learn neighbors' IDs first in $O(\Delta^2\log^2 n)$ beeping rounds.
W.l.o.g., assume that each message from a node $v$ to a node $w$ has $O(\log n)$ bits
%; otherwise, we could easily split it into chunks of size $O(\log n)$ and apply our new algorithm to each of them, sequentially 
(otherwise, 
%asymptotic formula on 
the bound on the time complexity is increased by a factor of $\cM/\log n$, where $\cM$ is the maximum size of a single message).
Simulation of a \congest round faces the following challenges.

\noindent
{\em Challenge 1.}
A node could try to compute its beeping schedule to avoid overlapping with other neighbors of the receiving node. However, it requires knowing at least $2$-hop neighborhood, which is costly. 
%(even $1$-hop requires $\Omega(\Delta^2\log n)$ rounds).
%The first main challenge to overcome is that a node $v$ does not know its $2$-hop neighborhood graph. Learning it could have potentially helped $v$ to use the specific differences between the identities to decide on a schedule when a successful transmission to/from each specific neighbor could take place (especially if transmission schedules are fixed, and nodes only decide what message to beep); however, it would require retrieving up to $\Theta(\Delta^2)$ node IDs. 
A node could try to learn first the IDs of its $1$-hop neighbors, and then broadcast them, one after another, using the local broadcast algorithm, 
%from Section~\ref{sec:local-broadcast}, 
but since there could be $\Theta(\Delta)$ such IDs (each represented by $O(\log n)$ bits), the overall time complexity would be $O(\Delta^3 \log^3 n)$, by Theorem~\ref{th:local_broadcast}.
Instead, our algorithm uses specific codes, called avoiding selectors (see Definition~\ref{def:avoid-selector}), to assure partial progress in information exchange in periods that sum up to $\Theta(\Delta^2 \polylog n \log \Delta)$.

\noindent
{\em Challenge 2.}
%Even knowing its $1$-hop neighbors, 
A node has to choose which of its input messages to beep at a time or find a more complex beeping code to encode many of its input messages. 
If it chooses ``wrongly,'' the message could be ``jammed'' by other beeping neighbors of the potential receiver.
To overcome this, avoiding selectors ensures that many nodes ``announce'' themselves successfully (i.e., without interference) to many of their neighbors, and these ``responders'' use avoiding selectors to respond. Once an announcer hears the ID of its responder, the handshaking procedure allows them to fix rounds for their point-to-point, non-interrupted communication.
%Such a chosen message could be arbitrarily ``jammed'' by some $2$-hop neighbors, which are not initially known to the node (see Challenge 1 above). 
%We prove later in Section~\ref{} \textcolor{red}{---------PENDING} that this is indeed a challenge, and any {\em non-adaptive beeping code} requires $\Omega(\Delta^3 \log\Delta n)$ \mm{[[MM: this might be confusing, is it n times log ?? ]]} rounds.
%To overcome it, the avoiding selectors (mentioned above) could be used in an adaptive way with properly chosen parameters to guarantee initiations of beeping communication in many links that are ``isolated'' in the network. We exploit them by designing a system of hand-shaking procedures, organized in three types of longer messages to beep: announcing (a node, called an announcer, beeps that it wants to communicate), responding (a node retrieving a message from an announcer, beeps a message destined to the announcer) and confirming (the announcer beeps a confirmation message destined to the responder). If indeed such links are isolated enough in the graph, we prove that this process guarantees successful message exchange between the announcer and the responder.


%the ad hoc topology and, consequently, 

The abovementioned avoiding selectors for $n$ nodes are parameterized by two numbers, $k,\ell$, corresponding to the number of competing neighbors/responders versus the other (potentially interrupting) neighbors:
%their definition~follows:

\begin{definition}[Avoiding selectors]
\label{def:avoid-selector}
    A family $\mathcal{F}$ of subsets of $[n]$ 
    %of size at most $k$ each 
    is called an \emph{$(n,k,\ell)$-avoiding selector}, where $1\le \ell < k\le n$, if for every non-empty subset $S \in [n]$ such that $|S| \leq k$ and for any subset $R\subseteq S$ of size at most $\ell$, there is an element $a \in S\setminus R$ for which there exists a set $F \in \mathcal{F}$ such that $|F \cap S| = \{a\}$.
\end{definition}


The following fact follows directly from Definition~\ref{def:avoid-selector}, see also \cite{BonisGV05,ChlebusK05}.

\begin{fact}
\label{fact:avoiding-selectors}
Suppose we are given an $(n,k,\ell)$-avoiding selector $\mathcal{F}$ and a set $S$ of size at most $k$.
Then, the number of elements in $S$ not ``selected'' by selector $\mF$ (i.e., for which there is no set in the selector that intersects $S$ on such singleton element) is smaller than $k-\ell$.
\end{fact}


\begin{theorem}[\cite{BonisGV05,ChlebusK05}]
\label{thm:avoiding-selectors}
There exists an $(n,k,\ell)$-avoiding selector of length $O\left(\frac{k^2}{k-\ell}\log n\right)$, and moreover, an $(n,k,\ell)$-avoiding selector of length $O\left(\frac{k^2}{k-\ell}\text{\em\ polylog } n\right)$ can be efficiently deterministically constructed (in polynomial time of $n$) for some polylogarithmic function $\text{\em\ polylog } n$, locally by each~node.
\end{theorem}




%\subsection{Main deterministic distributed algorithm simulating any \congest~round}
\subsection{The \alg Algorithm}
\label{sec:main-general-algorithm}

%The main 
Our simulator algorithm proceeds in epochs $i=1,\ldots\log\Delta $. 
A pseudo-code for an epoch $i$ is provided at the end of this subsection.
In the beginning, each node has all its links not successfully realized -- here by a link $\{v,w\}$ being realized we understand that up to the current round, an input message/ID sent by $v$ (using a sequence of beeps) has been successfully encoded by $w$ and vice versa (note that these are two different messages and were sent/encoded each in a different round); the formal definition of link realization will be given later.
The goal of the algorithm is to preserve the following invariant for epoch $i\ge 1$: 
\begin{quote}
\hspace*{-1em}
At the end of epoch 
$i=1,\ldots,\log \Delta$, 
%At the beginning of epoch 
%$i=1,\ldots,\log_{3/2} \Delta$, 
each vertex has less than 
$\kappa_i= \Delta / 2^{i}$ 
%$\kappa_i= \Delta \cdot (2/3)^{i-1}$ 
incident links not realized. 
\end{quote}
We also set an auxiliary value $\kappa_0=\Delta$, which corresponds to the maximum number of adjacent links per node at the beginning of the computation. For ease of presentation, we assume that node IDs come from the range $[1,n]$. Note that in all the formulas, the number of possible IDs appears only under logarithms, so the algorithm and proof for range $[1,n^c]$ are the same.
%By definition, $\kappa_1\le \Delta$.


\vspace*{-1.5ex}
%\paragraph{Main algorithm for epoch $i$.}
\paragraph{Algorithm for epoch $i$: Preliminaries and main concepts.} 

Epoch $i$ proceeds in subsequent batches of $2\log n$ rounds, each batch is called a \defn{super-round}. In a single super-round, a node can constantly listen or keep beeping according to some 0-1 sequence of length $2\log n$, where 1 corresponds to beeping in the related round and 0 means staying silent. 
The sequences that the nodes use during the algorithm are \defn{extended-IDs}, defined as follows: the first $\log n$ positions contain an ID of some node in $\{1,\ldots,n\}$, while the next $\log n$ positions contain the same ID 
%but 
\mamr{with the bits flipped, that is,}
with ones swapped to zeros and vice versa. 
Note that extended-IDs are pairwise different, and each of them contains exactly $\log n$ ones and $\log n$ zeros.
We say that a node $v$ {\em beeps an extended-ID of node $w$ in a super-round} 
%if $v$ keeps beeping exactly in rounds corresponding to the extended-ID of $w$ within this super-round 
\mamr{$s$ if, within super-round~$s$, node $v$ beeps only in rounds corresponding to positions with $1$'s in the extended ID of $w$ } 
($w$ could be a different node id than $v$).
We say that a node $w$ {\em receives an extended-ID of a node $v$ in a super-round} \mamr{$s$} if:
\vspace*{-0.8ex}
\begin{itemize}
\item 
$w$ does not beep in super-round \mamr{$s$},
\vspace*{-0.8ex}
\item 
the sequence of 
%beeps received 
\mamr{noise/silence heard by $w$}
in super-round \mamr{$s$} form an extended-ID of $v$.
\end{itemize}


\vspace*{-0.8ex}
\noindent
From the perspective of receiving information in a super-round, all other cases not falling under the above definition of receiving an extended-ID, i.e., when a node is not silent in the super-round or receives a sequence of beeps that does not form any extended-ID, are ignored by the algorithm, in the sense that it could be treated as meaningless information noise. 

%Analogously to extended-ID, each node $w$ creates an {\em extended-message addressed to a neighbor $v$}. Node $w$ does it 
\mamr{Analogously to extended-ID's, nodes create an \defn{extended-message}} by taking the binary representation of the message of logarithmic length and transforming it to a $2\log n$ binary sequence in the same way as an extended-ID is created from the binary ID of a node.
An extended-message, as well as an extended-ID, is easily decodable after being received without interruptions from other neighbors.

%A crucial definition specifies what does it mean to one-to-one communications, as given for free in the \congest model, in beeping networks. 
\mamr{A specification of the conditions to achieve one-to-one communication, which is given ``for free'' in the \congest model, is crucial. An illustration of the following handshake communication procedure is shown in Figure~\ref{fig:alg}.}
We say that our algorithm \defn{realizes link $\{v,w\}$} if the following 
%conditions
are~satisfied:
\vspace*{-0.8ex}
\begin{itemize}
\item[(a)] 
there are three consecutive super-rounds (called ``responding'') in which $v$ beeps an extended-ID of itself followed by an extended-ID of $w$ and then by extended-message of $v$ addressed to $w$, and $w$ receives them in these super-rounds; intuitively, it corresponds to the situation when $v$ ``tells'' $w$ that it dedicates these three super-rounds for communication from itself to $w$, and $w$ receives this information;
\vspace*{-3.5ex}
\item[(b)] 
there are three consecutive super-rounds (called ``confirming'') in which $w$ beeps an extended-ID of itself followed by an extended-ID of $v$ and by its extended-message addressed to $v$, and $v$ receives them in these super-rounds; intuitively, it corresponds to the situation when $w$ ``tells'' $v$ that it dedicates these three super-rounds for communication from itself to $v$, and $v$ receives this information;
\vspace*{-0.8ex}
\item[(c)]
there is a super-round, not earlier than the one specified in point (a), at the end of which node $w$ locally marks link $\{v,w\}$ as realized, 
and analogously, 
there is a super-round, not earlier than the one specified in point (b), at the end of which node $v$ locally marks link $\{v,w\}$ as realized.
\end{itemize}

\vspace*{-0.8ex}
\noindent
It is straightforward to see that in super-rounds specified in points (a) and (b), a multi-directional communication between $v$ and $w$ takes place -- by sending and receiving both ``directed pairs'' of extended-IDs of these two nodes, each of them commits that the super-rounds specified in points (a) and (b) are dedicated for sending a message dedicated to the other node, and vice versa. Additionally, in some super-round(s) both nodes commit that it has happened (c.f., point (c) above).

\vspace*{-2ex}
\paragraph{Algorithm for epoch $i$: Structure.} 

\begin{algorithm}[ht!]
\caption{\textit{NovelSelect}}
\label{alg:novelselect}
\begin{algorithmic}[1]
\State \textbf{Input:} Data pool $\mathcal{X}^{all}$, data budget $n$
\State Initialize an empty dataset, $\mathcal{X} \gets \emptyset$
\While{$|\mathcal{X}| < n$}
    \State $x^{new} \gets \arg\max_{x \in \mathcal{X}^{all}} v(x)$
    \State $\mathcal{X} \gets \mathcal{X} \cup \{x^{new}\}$
    \State $\mathcal{X}^{all} \gets \mathcal{X}^{all} \setminus \{x^{new}\}$
\EndWhile
\State \textbf{return} $\mathcal{X}$
\end{algorithmic}
\end{algorithm}


An epoch $i$ is split into $|\mF_{\Delta,k_i}|$ {\em phases}, for a given $(n,\Delta,\Delta - k_i)$-avoiding selector $\mF_{\Delta,k_i}$ and parameter $k_i=\Delta/2^i$, parameterized by a variable $j$. Each phase starts with one {\em announcing super-round}, in which nodes in set $\mF_{\Delta,k_i}(j)$ beep in pursuit to be received by some of their neighbors. This super-round is followed by $\log k_i$ {\em sub-phases}, parameterized by $a=1,\ldots, \log k_i$. A sub-phase $a$ uses sets from an $(n,k_i/2^{a-2},k_i/2^{a-1})$-avoiding selector $\mF_{k_i/2^{a-2},k_i/2^{a-1}}$ to determine who beeps in which super-round (together with additional rules to decide what extended-ID and extended-message to beep and how to confirm receiving them), and consists of 
%$\sum_{a=1}^{\log k_i} 
$|\mF_{k_i/2^{a-2},k_i/2^{a-1}}|$ %quadruples 
$6$-tuples
of super-rounds ($3$ responding super-rounds and $3$ confirming super-rounds). 
The goal of a phase is to realize links that were successfully received (``announced'') in the first (announcing) super-round of this phase. This is particularly challenging in a distributed setting since many neighbors could receive such an announcement, but the links between them and the announcing node must be confirmed so that one-to-one communication between the announcer and responders could take place in different super-rounds (in one super-round, a node can receive only logarithmic-size information).

\begin{algorithm}[t!]
\DontPrintSemicolon
\SetKwFunction{announcer}{announcer}
\SetKwProg{myalg}{Procedure}{}{}
\let\oldnl\nl
\newcommand{\nlnonumber}{\renewcommand{\nl}{\let\nl\oldnl}}
\nlnonumber
\myalg{\announcer{$v,k_i$}}{
    \tcp{announcing super-round}
    \For{each round $r=1,2,\dots,2\log n$}{
        \leIf{$\langle v\rangle(r)=1$}{beep}{listen}
    }
        \For{each sub-phase $a=1,2,\dots,\log k_i$}{\label{line:subphaseloopA}
            \For{$b=1,2,\dots,|\mF_{k_i/2^{a-2},k_i/2^{a-1}}|$}{
                \tcp{responding 3 super-rounds}
                \lFor(\tcp*[h]{announcer only listens}){$6\log n$ rounds}{listen}
                \eIf{some $\langle w\rangle\langle v\rangle\langle m_{w,v}\rangle$ was heard {\bf and} 
                $\{w,v\}\in E(v)$}{
                    \tcp{confirming 3 super-rounds}
                    \For{each round $r=1,2,\dots,2\log n$}{
                        \leIf{
                        $\langle v\rangle(r)=1$}
                        {beep}{listen}
                    }
                    \For{each round $r=1,2,\dots,2\log n$}{
                        \leIf{
                        $\langle w\rangle(r)=1$}
                        {beep}{listen}
                    }
                    \For{each round $r=1,2,\dots,2\log n$}{
                        \leIf{
                        $\langle m_{v,w}\rangle(r)=1$}
                        {beep}{listen}
                    }
                    $E(v)\leftarrow E(v)\setminus \{w,v\}$ \tcp{link realized}
                    \lIf{$E(v)=\emptyset$}{$v$ stops executing}
                }{
                    \lFor(\tcp*[h]{wait to synchronize}){$6\log n$ rounds}{listen}
                }
            }
        }
    }
\caption{\alg algorithm for \underline{announcer} node $v$.} 
\label{algC2Bv2A}
\end{algorithm}

\mamr{
%\paragraph{Pseudo-code for epoch $i$.} 
%Below is a detailed description of the algorithm for {\bf\em Epoch $i$}.
\vspace*{-2ex}
\paragraph{Algorithm for epoch $i$: Definitions and notation.} 
The pseudocode of the \alg algorithm can be seen in Algorithm~\ref{algC2Bv2}, and its subroutines in Algorithms~\ref{algC2Bv2A} and~\ref{algC2Bv2L}.
$\mF_{\Delta,k_i}$ is a locally computed $(n,\Delta,\Delta-k_i)$-avoiding selector, and for any $a=1,\ldots,\log k_i$,
%selector 
$\mF_{k_i/2^{a-2},k_i/2^{a-1}}$ is a (locally computed) $(n,k_i/2^{a-2},k_i/2^{a-1})$-avoiding selector, as in Theorem~\ref{thm:avoiding-selectors}.
We denote the extended-ID of node $x$ as $\langle x\rangle$, and the extended-message of node $x$ for node $y$ as $\langle m_{x,y}\rangle$, both given as a sequence of bits. For any sequence of bits $s$, $s(i)$ is the $i^{th}$ bit of $s$.
}


\remove{
%\begin{enumerate}
%\item for $i=0,1,\ldots,\log \Delta$
\begin{enumerate}
\item all nodes become active, 
$k_i\gets \Delta / 2^{i}$
%$k_i\gets \Delta \cdot (2/3)^{i-1}$
    \item for each phase $j=1,2,\ldots,|\mF_{\Delta,k_i}|$
    \begin{enumerate}
        \item {\bf\em Announcing super-round:} each active node $v$ in set $\mF_{\Delta,k_i}(j)$ beeps its extended-ID in a super-round (recall that a super-round contains $2\log n$ subsequent rounds); \\
        a node $w$ that receives an extended-ID of some node $v$ and has not realized the link $\{v,w\}$ yet, becomes {\em $(j,v)$-responsive}
        \label{l:first-beep}
\item for each sub-phase $a=1,\ldots,\log k_i$
\label{alg:sub-phase}
       \begin{enumerate}
            \item for $b=1,2,\ldots,|\mF_{k_i/2^{a-2},k_i/2^{a-1}}|$ 
            \begin{enumerate}
            \item {\bf\em Responding $3$ super-rounds:}  
            if $w$ is $(j,v)$-responsive, for some $v$, and $w$ is in set $\mF_{k_i/2^{a-2},k_i/2^{a-1}}(b)$, node $w$ beeps its extended-ID in one super-round, followed by the extended-ID of $v$ in the next super-round, followed by the extended-message of $w$ addressed to $v$;
%            beeping rounds, nodes that heard a beep in preceding line~\ref{l:first-beep}, keep transmitting according to their corresponding row in $\mF_{k_i,k_i/3}$
        \item {\bf\em Confirming $3$ super-rounds:} if $v$ is in set $\mF_{\Delta,k_i}(j)$ (i.e., it beeped its extended-ID in a super-round in line~\ref{l:first-beep}) received an extended-IDs of $w$ and of itself and an extended-message in the preceding $3$ responding super-rounds, for some $w$, it beeps an extended-ID of itself in the one super-round, followed by the extended-ID of $w$ in the next super-round, followed by its extended-message addresses to $w$;\\
        Then, at the end of the third confirming super-round, the beeping node $v$ (locally) marks the link $\{v,w\}$ as realized; \\
        If a $(j,v')$-responsive node $w'$ receives an extended-ID of $v'$ followed by its extended-ID and an extended-message in the current confirming $3$ super-rounds, it (locally) marks link $\{v',w'\}$ as realized and $w'$ abandons its $(j,v')$-responsive status (as the corresponding link has been already marked as realized)
            \end{enumerate}
        \end{enumerate}
    \end{enumerate}
\end{enumerate}
%\end{enumerate}
}




%\subsection{Analysis of the algorithm from Section~\ref{sec:main-general-algorithm}}
\vspace*{-2ex}
\subsection{Analysis of the \alg Algorithm}

Recall that the algorithm proceeds in synchronized super-rounds, each containing a subsequent $2\log n$ rounds. Therefore, our analysis assumes that the computation is partitioned into consecutive super-rounds and, unless stated otherwise, it focuses on correctness and progress in super-rounds. 
Recall also that each node either stays silent (no beeping at all) or beeps an extended ID of some node or an extended message of one node addressed to one of its neighbors in a super-round.
%
The missing proofs 
%from this section 
are deferred to Section~\ref{sec:proofs-main-simulation}.

In the next two technical results, we state and prove the facts that receiving an extended-ID by a node $w$ in a super-round can happen if and only if there is {\em exactly one neighbor} of $w$ has been beeping {\em the same extended-ID} during the considered super-round. 

\begin{fact}[Single beeping]
\label{fact:single-beeping}
If during a super-round, exactly one neighbor of a node $w$ beeps an extended-ID of some $z$, then $w$ receives this extended-ID in this super-round.
\end{fact}

\begin{proof}
Directly from the definition of receiving an extended-ID. 
\end{proof}


\begin{lemma}[Correct receiving]
\label{lem:correct-receiving}
During the algorithm, if a node $w$ 
%stays silent and 
receives some extended-ID of $z$ in a super-round, then some unique neighbor $v$ of $w$ has been beeping an extended-ID of $z$ in this super-round while all other neighbors of $w$ have been silent. 
%in this super-round. 
The above holds except, possibly, some second responding super-rounds, in which a node can receive an extended-ID of $z$ that has been beeped by more than one neighbor.
\end{lemma}

%\begin{minipage}{1\linewidth}
%\begin{algorithm}[t!]
\DontPrintSemicolon
\SetKwFunction{announcer}{announcer}
\SetKwProg{myalg}{Procedure}{}{}
\let\oldnl\nl
\newcommand{\nlnonumber}{\renewcommand{\nl}{\let\nl\oldnl}}
\nlnonumber
\myalg{\announcer{$v,k_i$}}{
    \tcp{announcing super-round}
    \For{each round $r=1,2,\dots,2\log n$}{
        \leIf{$\langle v\rangle(r)=1$}{beep}{listen}
    }
        \For{each sub-phase $a=1,2,\dots,\log k_i$}{\label{line:subphaseloopA}
            \For{$b=1,2,\dots,|\mF_{k_i/2^{a-2},k_i/2^{a-1}}|$}{
                \tcp{responding 3 super-rounds}
                \lFor(\tcp*[h]{announcer only listens}){$6\log n$ rounds}{listen}
                \eIf{some $\langle w\rangle\langle v\rangle\langle m_{w,v}\rangle$ was heard {\bf and} 
                $\{w,v\}\in E(v)$}{
                    \tcp{confirming 3 super-rounds}
                    \For{each round $r=1,2,\dots,2\log n$}{
                        \leIf{
                        $\langle v\rangle(r)=1$}
                        {beep}{listen}
                    }
                    \For{each round $r=1,2,\dots,2\log n$}{
                        \leIf{
                        $\langle w\rangle(r)=1$}
                        {beep}{listen}
                    }
                    \For{each round $r=1,2,\dots,2\log n$}{
                        \leIf{
                        $\langle m_{v,w}\rangle(r)=1$}
                        {beep}{listen}
                    }
                    $E(v)\leftarrow E(v)\setminus \{w,v\}$ \tcp{link realized}
                    \lIf{$E(v)=\emptyset$}{$v$ stops executing}
                }{
                    \lFor(\tcp*[h]{wait to synchronize}){$6\log n$ rounds}{listen}
                }
            }
        }
    }
\caption{\alg algorithm for \underline{announcer} node $v$.} 
\label{algC2Bv2A}
\end{algorithm}
\begin{algorithm}[t!]
\DontPrintSemicolon
\SetKwFunction{listener}{listener}
\SetKwProg{myalg}{Procedure}{}{}
\let\oldnl\nl
\newcommand{\nlnonumber}{\renewcommand{\nl}{\let\nl\oldnl}}
\nlnonumber
\myalg{\listener{$w,k_i$}}{
    $status(u)\leftarrow$ \texttt{nil}\;
    \tcp{announcing super-round}
    \lFor(\tcp*[h]{listener only listens}){$2\log n$ rounds}{listen}
    \If{some $\langle v\rangle$ was heard {\bf and} $\{v,w\}\in E(w)$}{
        $status(w) \leftarrow v$-$responsive$\;
    }
    \For{each sub-phase $a=1,2,\dots,\log k_i$}{\label{line:subphaseloopL}
        \For{$b=1,2,\dots,|\mF_{k_i/2^{a-2},k_i/2^{a-1}}|$}{
            \eIf{$status(w)=v$-$responsive$ {\bf and} 
                $w\in\mF_{k_i/2^{a-2},k_i/2^{a-1}}(b)$}{
                \tcp{responding 3 super-rounds}
                \For{each round $r=1,2,\dots,2\log n$}{
                    \leIf{$\langle w\rangle(r)=1$}{beep}{listen}
                }
                \For{each round $r=1,2,\dots,2\log n$}{
                    \leIf{$\langle v\rangle(r)=1$}{beep}{listen}
                }
                \For{each round $r=1,2,\dots,2\log n$}{
                    \leIf{$\langle m_{w,v}\rangle(r)=1$}{beep}{listen}
                }
                \tcp{confirming 3 super-rounds}
                \lFor(\tcp*[h]{listener only listens}){$6\log n$ rounds}{listen}
                \If{some $\langle v\rangle\langle w\rangle\langle m_{v,w}\rangle$ was heard}{
                    $status(w)\leftarrow$ \texttt{nil}\;
                    $E(w)\leftarrow E(w)\setminus \{v,w\}$ \tcp{link realized}
                    \lIf{$E(w)=\emptyset$}{$w$ stops executing}
                }            
            }{
                \lFor(\tcp*[h]{wait to synchronize}){$12\log n$ rounds}{listen}
            }
        }
    }
}
\caption{\alg algorithm for \underline{listener} node $w$.} 
\label{algC2Bv2L}
\end{algorithm}



%\end{minipage}


We now prove that link realization implemented by our algorithm is consistent with the definition -- it allocates in a distributed way super-rounds for bi-directional communication of distinct messages.

\begin{lemma}[Correct realization]
\label{lem:correct-realization}
If a node $v$ (locally) marks some link $\{v,w\}$ as realized, which may happen only at the end of a second confirming super-round, the link has been realized by then. 
\end{lemma}



As mentioned earlier in the description of the phase, the goal of a phase $j$ (of epoch $i$) is to assure that any node $v$ that was received by some other nodes $w$ in the announcing super-round, gets all such links $\{v,w\}$ realized by the end of the phase (and vice versa, because the condition on the realization by this algorithm is symmetric).
The next step is conditional progress in a sub-phase $a$ of a phase $j$.

\begin{lemma}[Sub-phase progress]
\label{lem:subphase-progress}
Consider any node $v$ and suppose that in the beginning of sub-phase $a$ of phase $j$, there are at most $\Delta/2^{i+a-2}$ nodes $w$ such that $w$ is $(j,v)$-responsive and it does not mark link $\{v,w\}$ as realized. Then, by the end of the sub-phase, the number of such nodes is reduced to less~than~$\Delta/2^{i+a-1}$.
\end{lemma}



\begin{lemma}[Phase progress]
\label{lem:phase-progress}
Consider a phase $j$ of epoch $i$ and assume that in the beginning, there are at most $2k_i$ non-realized incident links to any node. Every node $w$ that becomes $(j,v)$-responsive in the first (announcing) super-round of the phase, for some $v$, mark locally the link $\{v,w\}$ as realized during this phase. And vice versa, also node $v$ marks locally that link as realized. 
\end{lemma}

The next lemma proves the invariant for epoch $i$, assuming that it holds in the previous epochs. 

\begin{lemma}[Epoch invariant]
\label{lem:epoch-invariant}
The invariant for epoch $i\ge 1$ holds. 
\end{lemma}




\begin{theorem}
\label{thm:congest-sim}
%The main deterministic distributed algorithm 
The \alg algorithm deterministically and distributedly
simulates any round of any algorithm designed for the \congest networks in $O(\Delta^2 \polylog n \log\Delta)$ beeping rounds, where the $\polylog n$ is the square of the (poly-)logarithm in the construction of avoiding-selectors in Theorem~\ref{thm:avoiding-selectors} multiplied by $\log n$.
\end{theorem}

%\sk{REPEATING THE THEOREM IN 2 PLACES  SEEMS UNJUSTIFIABLY COSTLY IN TERMS OF SPACE REAL ESTATE. I WOULD START DIRECTLY WITH "PROOF OF THEOREM 6". OR REMOVE IN THE INTRO. tHE SAME GOES WITH THE COROLLARIES GIVEN IN THE INTRO. ALREADY HALF A PAGE SAVING.  }

\begin{proof}
By Lemma~\ref{lem:epoch-invariant}, each epoch $i$ reduces by at least half the number of non-realized incident links. 
We next count the number of rounds in each epoch by counting the number of super-rounds and multiplying the result by the $O(\log n)$ length of each super-round.
Recall that link realization means that some triples of responding and confirming super rounds were not interrupted by other neighbors of both end nodes of that link; therefore, the attached extended messages (in the third super-rounds in a row) were correctly received. Thus, the local exchange of messages addressed to specific neighbors took place successfully.

Each sub-phase $a$ has $O(\Delta^2 \polylog n)$ super-rounds, because for each set in of the $(n,k_i/2^{a-1},k_i/2^a)$-avoiding selector $\mF_{k_i/2^{a-1},k_i/2^a}$, there are four super-rounds and the selector itself has $O((k_i/2^a) \polylog n)$ set, by Theorem~\ref{thm:avoiding-selectors}.

Therefore, the total number of super-rounds in all sub-phases executed 
within 
%point~\ref{alg:sub-phase} of the algorithm~
\mamr{the loops in Line~\ref{line:subphaseloopA} of Algorithm~\ref{algC2Bv2A} and Line~\ref{line:subphaseloopL} of Algorithm~\ref{algC2Bv2L}}
is 
\vspace*{-1.3ex}
\[
O(\sum_{a=1}^{\log k_i} (k_i/2^a) \polylog n) \le
O( k_i \polylog n)
\ .
\]

\vspace*{-0.7ex}
\noindent
Within one phase, they are executed as many times as the number of announcing super-rounds. 
The number of announcing super-rounds in a phase is
$|\mF_{\Delta,k_i}|$, which is $O((\Delta^2/k_i)\cdot \polylog n)$ by Theorem~\ref{thm:avoiding-selectors}.
Hence, the total number of super-rounds in a phase is 
%\[
$O( (\Delta^2/k_i)\cdot \polylog n \cdot k_i \polylog n)
\le 
O(\Delta^2 \polylog n)$,
%\ ,
%\]
where the final $\polylog n$ is a square of the (poly-)logarithms from Theorem~\ref{thm:avoiding-selectors}.

Since there are $\log\Delta$ epochs, the total number of super-rounds is $O(\Delta^2 \polylog n \log\Delta)$, which is additionally multiplied by $O(\log n)$ -- the length of each super-round -- if we want to refer the total number of beeping rounds.
%
%super-rounds of $O(\log n)$ beeping rounds each
\end{proof}

\vspace*{-2.5ex}
\paragraph{Maximal Independent Set (MIS):}
To demonstrate that the above efficient simulator can yield efficient results for many graph problems, we apply it to the 
algorithm of~\cite{ghaffari2021improved}% 
% \dk{??? and others such as ????}
 to improve polynomially (with respect to $\Delta$) the best-known solutions 
for MIS (c.f. \cite{beauquier2018fast}):

%\dk{(c.f.,~\cite{???})}:

%\todo{More uses for Network Decomposition! Add corollaries here and citations in Related Work.}

\begin{corollary}
\label{cor:mis}
% Graph problems such as MIS, \dk{????????????} 
MIS can be solved deterministically on any network of maximum node-degree $\Delta$ in $O(\Delta^2 \polylog n)$ beeping rounds.
\end{corollary}

\remove{%%%%%%%%%%%%%%%%%

\subsection{Cubic Lower Bound for Non-adaptive Beeping Codes}
\label{sec:lower-non-adaptive}

Consider the following simplification of the simulation problem. Each node $v$ is given, as an input, parameters $n,\Delta$ and a vector of numbers in $[n]$ of length $x_v\le \Delta$. The goal is: for any graph $G$ such that $x_v=|N(v)|$, for any node $v$, every $i$th neighbor of $v$ (according to the order of IDs) learns the $i$th number in the vector of $v$, for any $1\le i \le x_v$.
We call this problem {\em local ports' learning}, as we could think about the numbers in the input vectors as (arbitrary) labels of ports from the node to its corresponding neighbor.


\begin{theorem}
Any beeping code solving the local ports' learning problem has length $\Omega(\Delta^3\log n)$.
\end{theorem}

\begin{proof}


\end{proof}

}%%%%%%%%%%%  END  REMOVE  %%%%%%%%%%

%%%%%%%%%%%%%%%%%%%%%%%%%%%%%%%%%%%%%%%%%%%%%%%%%%%%%%%%%%%
\paragraph{Task} This task extends diagnostic evaluation to complex multi-step reasoning, and multiple facts. Like Fact Check, it ensures identical answers across counterfactual settings but requires multi-hop chains to reach conclusions. Unlike FactCheck, this task requires explanations grounded in multi-step reasoning chains, where even divergent explanation pairs share overlapping intermediate steps. \\ 
\noindent \textbf{Dataset} We construct this dataset using StrategyQA \citep{Geva2021DidAU}, a multi-hop QA benchmark that provides gold-standard fact decompositions for each example. We generate two counterfactual variants for one fact per question, preserving the original answer while altering the reasoning. When facts are interdependent, we propagate modifications to ensure consistency. Next, we generate explanations for each counterfactual set using the original decompositions. We use \texttt{gpt-4o} for generating counterfactuals and explanations, which we manually verify for logical coherence. The final data consists of 200 high-quality examples, balancing complexity with computational feasibility. 
%\input{localbroadcastlowerbound}

%\section{Missing proofs from Section~\ref{sec:main-simulation}}
%\section{Proofs from Section~\ref{sec:main-simulation} -- Analysis of Main Algorithm}
\section{Details of Section~\ref{sec:main-simulation} -- Analysis of the \alg Algorithm}
\label{sec:proofs-main-simulation}



\begin{figure}[thbp]
\centering
\begin{subfigure}[htbp]{0.30\textwidth}
\centering
\vspace*{-10ex}
\includegraphics[width=\linewidth]{beep001.jpeg}
\caption{Some part of a beeping network.}
\label{subfig:bn}
\end{subfigure}
\hspace{0.1in}
\begin{subfigure}[htbp]{0.30\textwidth}
\centering
\includegraphics[width=\linewidth]{beep002.jpeg}
\caption{Announcing super-round of some phase $j$: $\{c,g\}$ announce, $\{a,b,d,e,f\}$ hear noise, but only $\{b,d,e,f\}$ receive an extended-ID $\langle c\rangle$ and become $c$-$responsive$.}
\label{subfig:announce}
\end{subfigure}
\hspace{0.1in}
\begin{subfigure}[htbp]{0.30\textwidth}
\centering
\vspace*{-3ex}
\includegraphics[width=\linewidth]{beep004.jpeg}
\caption{Responding $3$ super-rounds within some sub-phase $a'$: $\{b,e\}$ respond and $c$ receives $\langle b\rangle\langle c\rangle\langle m_{b,c}\rangle$ and $\langle e\rangle\langle c\rangle\langle m_{e,c}\rangle$.}
\label{subfig:resp1}
\end{subfigure}
\\
\begin{subfigure}[htbp]{0.30\textwidth}
\centering
\vspace*{3ex}
\includegraphics[width=\linewidth]{beep005.jpeg}
\caption{Confirming $3$ super-rounds during sub-phase $a'$: $c$ confirms, $\{b,e\}$ receive $\langle c\rangle\langle b\rangle\langle m_{c,b}\rangle$ and $\langle c\rangle\langle e\rangle\langle m_{c,e}\rangle$ respectively. After this $\{b,e\}$ abandon the $c$-$responsive$ status and $\{\{b,c\},\{e,c\}\}$ are marked as realized.}
\label{subfig:conf1}
\end{subfigure}
\hspace{0.1in}
\begin{subfigure}[htbp]{0.30\textwidth}
\centering
\vspace*{-8ex}
\includegraphics[width=\linewidth]{beep006.jpeg}
\caption{Responding $3$ super-rounds within some sub-phase $a''$: $d$ responds and $c$ receives $\langle d\rangle\langle c\rangle\langle m_{d,c}\rangle$.}
\label{subfig:resp2}
\end{subfigure}
\hspace{0.1in}
\begin{subfigure}[htbp]{0.30\textwidth}
\centering
\vspace*{-3ex}
\includegraphics[width=\linewidth]{beep007.jpeg}
\caption{Confirming $3$ super-rounds during sub-phase $a''$: $c$ confirms, $d$ receives $\langle c\rangle\langle d\rangle\langle m_{c,d}\rangle$. After this $d$ abandons the $c$-$responsive$ status and $\{d,c\}$ is marked as realized.}
\label{subfig:conf2}
\end{subfigure}
\\
\begin{subfigure}[htbp]{0.30\textwidth}
\centering
\includegraphics[width=\linewidth]{beep008.jpeg}
\caption{Responding $3$ super-rounds within some sub-phase $a'''$: $f$ responds and $c$ receives $\langle f\rangle\langle c\rangle\langle m_{f,c}\rangle$.}
\label{subfig:resp3}
\end{subfigure}
\hspace{0.1in}
\begin{subfigure}[htbp]{0.30\textwidth}
\centering
\vspace*{5ex}
\includegraphics[width=\linewidth]{beep009.jpeg}
\caption{Confirming $3$ super-rounds during sub-phase $a'''$: $c$ confirms, $f$ receives $\langle c\rangle\langle f\rangle\langle m_{c,f}\rangle$. After this $f$ abandons the $c$-$responsive$ status and $\{f,c\}$ is marked as realized.}
\label{subfig:conf3}
\end{subfigure}
\hspace{0.1in}
\begin{subfigure}[htbp]{0.30\textwidth}
\centering
\includegraphics[width=\linewidth]{beep010.jpeg}
\caption{By the end of phase $j$ links $\{\{b,c\},\{d,c\},\{e,c\},\{f,c\}\}$ have been realized.}
\label{subfig:end}
\end{subfigure}
\caption{Illustration of \alg algorithm -- consecutive handshakes between the announcer $c$ and its responders during a phase.}
\label{fig:alg}
\end{figure}


\begin{proof}[Proof of Lemma~\ref{lem:correct-receiving}]
The proof is by contradiction -- suppose that in some super-round a node $w$ 
%stays silent and 
receives an extended-ID $z$ but the claim of the lemma does not hold. 
Without lost of generality, we may assume that this is the first such super-round.

Recall that the definition of receiving an extended-ID requires that node $w$ has been silent in this super-round. Note that if exactly one neighbor of node $w$ has been beeping during the super-round, it must have been an extended-ID of some node (by specification of the algorithm), and therefore node $w$ receives this extended-ID (as other neighbors do not beep at all).
Similarly, we argue that at least one neighbor of node $w$ must have been beeping (some extended-ID) in the super-round, as otherwise node $w$ would not have received any beep (and so, also no extended-ID) in the considered super-round.

In the remainder we focus on the complementary case that at least two neighbors of node $w$ have been beeping in the super-round, each of them some extended-ID (again, by specification of the algorithm, a node beeps only some extended-ID or stay silent during any super-round).

First, suppose that some two neighbors, $v_1,v_2$, beeped different extended-IDs, say $z_1\ne z_2$, respectively.
It means that node $w$ received more than $\log n$ beeps during the super-round: $\log n$ beeps coming from one of the extended-IDs and at least one more because the extended IDs of different nodes differ by at least one position. Hence, the received sequence of beeps does not form any extended-ID, as it must always have $\log n$ bits $1$ corresponding to the beeps. This contradicts the fact that $w$ receives an extended-ID in the considered super-round.

Second, suppose that all extended-IDs beeped by (at least two) neighbors on node $w$ are the same. If this happens in an announcing, or a first responding, or a first confirming super-round, it is a contradiction because all nodes that beep in such super-rounds beep their own extended-IDs, which are pairwise different.
%
If this happens in a second confirming super-round, it means that these two neighbors $v_1,v_2$ belong to the same set $\mF_{\Delta,k_i}(j)$, for some phase number $j$, and both of them received an extended-ID of $z$ in the preceding responding super-round. By the fact that the considered super-round is the first when the lemma's claim does not hold, we get that in this preceding responding super-round, both $v_1,v_2$ received the extended-ID of $z$ when $z$ was their unique beeping neighbor (beeping its own extended-ID, by the specification of the responding super-rounds).
This, however, implies that in the beginning of the current phase $j$, i.e., during its announcing super-round, both $v_1,v_2$ beeped their extended-IDs and, again by the choice of the current contradictory super-round, their neighbor $w$ could not have received any extended-ID -- this is a contradiction with the fact that $z$ was transmitting in the responding super-round preceding the considered (contradictory) super-round. More precisely, only $(j,\cdot)$-responsive nodes can transmit in responding super-rounds, but $z$ is not $(j,\cdot)$-responsive because it had not received any extended-IDs in the first (announcing) super-round of the current phase.

The last sub-case of the above scenario, when all extended-IDs beeped by (at least two) neighbors on node $w$ are the same, is as follows. 
If this situation happens in a second responding super-round, it means 
that these two neighbors $v_1,v_2$ are both $(j,z)$-responsive and beep extended-ID of $z$. This is, however, acceptable due to the exception in the lemma's statement.

This completes the proof of the lemma.
\end{proof}


\begin{proof}[Proof of Lemma~\ref{lem:correct-realization}]
It is enough to show that points (a) and (b) of the definition of link realization occurred in the last four super-rounds (two responding and two confirming) and also that the other node, $w$, (locally) marks link $\{v,w\}$ as realized at the same time when $v$ does.

If node $v$ marked the link $\{v,w\}$ as realized, it could be because of one of two reasons. 

First, it is in the set $\mF_{\Delta,k_i}(j)$, where $i$ is the number of the current epoch and $j$ is the number of the current phase and received an extended-ID of $w$ followed by its own extended-ID in the preceding two responding super-rounds. This satisfies point (a) of the definition of link realization, as both were beeped by node $w$, by the algorithm specification, and by Lemma~\ref{lem:correct-receiving}. Note that the exception in that lemma does not really apply here because if there were two or more neighbors beeping the same extended-ID (of $v$) in the second responding round, they would also be beeping their own extended-IDs in the first responding round, which could contradict the fact that $v$ received a single extended-ID at that super round.

This also means that $v$ has beeped its own extended-ID followed by extended-ID of $w$ in the last two confirming super-rounds, which must have been received by $w$ because $w$ is $(j,v)$-responsive (because only such nodes could have beeped in the preceding responding super-rounds) and thus its only neighbor in set $\mF_{\Delta,k_i}(j)$ (only such nodes are allowed to beep in confirming super-rounds) is $v$; here we use Fact~\ref{fact:single-beeping}. Hence, $w$ also marks link 
$\{v,w\}$ as realized at the end of the two confirming super-rounds; by the algorithm's specification, point (c) of the definition also holds in this case.

Second, it is $(j,z)$-responsive and received an extended-ID of $z$ followed by its own extended-ID in the current two confirming super-rounds. This satisfies point (b) of the definition, as both were beeped by node $z$, by the specification of the algorithm and Lemma~\ref{lem:correct-receiving} (exception in that lemma does not apply here because we now consider only confirming super-rounds).

This also means that $v$ beeped its own extended-ID followed by extended-ID of $z$ in the preceding two responding super-rounds (because $v$ is $(j,z)$-responsive and only such nodes could beep in the preceding responding super-rounds), which must have been received by $z$ (otherwise, by the specification of the algorithm, $z$ would not beep its extended-ID followed by the extended-ID of $v$ in the last two confirming super-rounds). Hence, $z$ also marks link $\{v,z\}$ as realized at the end of the two confirming super-rounds by the algorithm's specification, and point (c) of the definition also holds in this case.
\end{proof}



\begin{proof}[Proof of Lemma~\ref{lem:subphase-progress}]
The lemma follows from the definition of the $(n,k_i/2^{a-2},k_i/2^{a-1})$-avoiding selector $\mF_{k_i/2^{a-2},k_i/2^{a-1}}$ used throughout sub-phase $a$ of phase $j$ of epoch $i$. 
By specification of the sub-phase, only nodes $w$ such that $w$ is $(j,v)$-responsive and it does not marked link $\{v,w\}$ as realized take active part in sub-phase $a$ (in the sense that only those nodes can beep extended-IDs of itself followed by $v$ in pairs of responding super-rounds), while other neighbors of $v$ do not beep at all. The latter statement needs more justification -- in the beginning of the current phase, in the announcing super-round, $v$ must have beeped because some nodes have become $(j,v)$-responsive in this phase (w.l.o.g. we may assume that at least one node has become $(j,v)$-responsive, because otherwise the lemma trivially holds), therefore, by Lemma~\ref{lem:correct-receiving}, other neighbors of $v$ could not receive another announcement and become $(j,v')$-responsive, for some $v'\ne v$, and thus by the description of the algorithm -- they stay silent throughout the whole phase. 

By lemma assumption, there are at most $\Delta/2^{i+a-2}=k_i/2^{a-2}$ $(j,v)$-responsive nodes $w$ that have not marked link $\{v,w\}$ as realized by the beginning of the sub-phase. Hence, at least half of them will be in a singleton intersection with some set $\mF_{k_i/2^{a-2},k_i/2^{a-1}}(b)$, by Definition~\ref{def:avoid-selector} and Fact~\ref{fact:avoiding-selectors}, in which case $v$ receives their beeping in the corresponding pair of the responding super-rounds. Consequently, $v$ beeps back its own extended-ID and the extended-ID of $w$ in the following two confirming super-rounds. 

Node $w$ receives those beepings, as there is no other neighbor of $w$ who is allowed to beep in these two rounds -- indeed if there was, it would belong to set $\mF_{\Delta,k_i}(j)$ and thus it would have been beeping in the announcing super-round of this phase, preventing (together with neighbor $v$ of $w$) node $w$ from receiving anything in that super-round (by Lemma~\ref{lem:correct-receiving}), which contradicts the fact that $w$ must have received an extended-ID of $v$ in that round to become $(j,v)$-responsive (as assumed). Therefore, by the description of the algorithm, $w$ marks link $\{v,w\}$ as realized. This completes the proof that the number of $(j,v)$-responsive neighbors $w$ of $v$ who remain without realizing link $\{v,w\}$ becomes less than $\Delta/2^{i+a-1}$ at the end of the considered sub-phase. 
\end{proof}


\begin{proof}[Proof of Lemma~\ref{lem:phase-progress}]
It follows directly from the fact that a phase, after its announcing super-round, iterates sub-phases $a=1,\ldots,\log k_i$. Each subsequent sub-phase halves the number of not-realized links $\{v,w\}$, for $(j,v)$-responsive nodes $w$ and each announcing node $v$, c.f., Lemma~\ref{lem:subphase-progress}, starting from the assumed $2k_i$ maximum number of $(j,v)$-responding nodes (recall that $(j,v)$-responding nodes form a subset of those to whom links are not realized, hence there are at most $2k_i$ of them in the beginning).
\end{proof}


\begin{proof}[Proof of Lemma~\ref{lem:epoch-invariant}]
The proof is by induction on epoch number $i$.
Obviously, the invariant holds at the beginning of the first epoch, i.e., $\kappa_1\le k_1$, where $\kappa_i$ was defined as the sharp upper bound on the maximum number of not realized links at a node at the end of epoch $i$ and $k_i$ is the parameter used in the algorithm for epoch $i$. 

Consider epoch $i\ge 1$.
We have to prove that:
assuming that $\kappa_{i'}\le k_{i'}$, for any $1\le i' < i$, we also have $\kappa_i \le k_i$.
Technically we can assume that $\kappa_0=k_0=\Delta$.

Consider a node $w$. 
By the inductive assumption, it has at most $k_{i-1}$ neighbors $v$ such that link $\{v,w\}$ has not been marked by $w$ as realized.
By Definition~\ref{def:avoid-selector} and Fact~\ref{fact:avoiding-selectors} applied to $(n,\Delta,\Delta-k_i)$-avoiding selector $\mF_{\Delta,k_i}$, 
which sets are used for announcing super-rounds (and later for confirming super-rounds), the number of neighbors $v$ of node $w$ from whom node $w$ has not received their extended-ID during the announcing super-round is smaller than $k_i$. By Lemma~\ref{lem:phase-progress}, all such nodes $w$ realize their links, and by Lemma~\ref{lem:correct-realization}, also node $v$ realizes these links during the considered phase. Hence, the number of non-realized links incident to any node $w$ drops below $k_i$ by the end of epoch $k_i$.
\end{proof}



\section{Details of Section~\ref{sec:primitives} -- Algorithms and Analysis of Building Blocks}
\label{sec:prim_details}

In this section, we include the remaining details of our building blocks (Section~\ref{sec:primitives}). Theorems are restated for easy reference. First, let us introduce the following combinatorial object, to be used later.

\begin{definition}[Strong selector]
\label{def:strong-selector}
    A family $\mathcal{F}$ of subsets of $[n]$ of size at most $k$ each is called an \emph{$(n,k)$ strong selector} if for every non-empty subset $S \in [n]$ such that $|S| \leq k$, for every element $a \in S$, there exists a set $F \in \mathcal{F}$ such that $|F \cap S| = \{a\}$.
\end{definition}

Note that there are known constructions of $(n,\Delta)$ strong selectors of length at most $O(\Delta^2 \log n)$~\cite{5967914}. Next, we show how to use an $(n,\Delta)$ strong selector to perform a local broadcast.

\subsection{Local broadcast}
\label{sec:local-broadcast}

%One can perform local broadcast routine based on strong selectors. 
Our local broadcast routine is non-adaptive. That is, each node 
%will have a prepared-in-advance 
has a predefined schedule 
%determining 
specifying in which rounds beeps and in which rounds listens. 

\parhead{Assumptions} % \pga{The nodes IDs come from range $[1,n^c]$.} 
The nodes know the total number of nodes $n$, parameter $c$,
 %\mm{[[the range has been defined above, do we need to recall it?]]} 
the maximum degree $\Delta$ of the graph and have access to a global clock. Additionally, we assume that each node $v$ knows its neighborhood $N(v)$. (In Subsection~\ref{sub:neighbourhood} we will show how all nodes can learn their neighborhoods in $O(\Delta^2 \log^2 n)$ beeping rounds.)

\localbroadcastthm*
\remove{
\begin{theorem}
\label{th:local_broadcast}
\mm{Consider a Beeping Network where each node $v$ knows $n$, $\Delta$, and $N(v)$.}
    Assume that every message $m_v$ of each node $v$ has length at most $k$ for some $k>0$. Then, there is a deterministic distributed local broadcast algorithm that works in $O(k\Delta^2 \log^2 n)$ beeping rounds.
\end{theorem}
}

\begin{proof}
Consider an $(n^c,\Delta)$ strong selector $\mathcal{F}=\{S_1,S_2,\dots, S_L\}$ of length $L=O(\Delta^2 \log n)$, known to all nodes. Our local broadcast schedule will take $L$ rounds. At any round $i$, nodes $v \in S_i$ that have bit 1 (indicating to transmit) send a beep while all the other nodes are silent.

Consider any receiver $r$. Consider the set $N(r)$ of neighbors of $r$. Note that $|N(r)| \leq \Delta$. From the definition of a $(n^c,\Delta)$ strong selector $\mathcal{F}$, for every $v \in N(r)$ there exists an index $i$ such that $S_i \cap N(r) = \{v\}$. Therefore, for every pair of transmitters $v$ and receiver $r$ that are adjacent to each other, there exists a round $i$ such that $v$ is the only transmitting neighbor of $r$.

%Here we will assume that each node knows its neighborhood. In Subsection~\ref{sub:neighbourhood} we will show how all the nodes can learn their neighborhoods in $O(\Delta^2 \log^2 n)$ beeping rounds.

Since every node $v$ knows its neighborhood $N(v)$ and the sets $S_i$ for all $i$, node $v$ also knows for each neighbor $u \in N(v)$ at what round $t$ neighbor $u$ is the only neighbor transmitting. If at round $t$ node $v$ hears a beep, it means that $u$ transmitted bit 1. If at round $t$ node $v$ hears silence, it means that $u$ transmitted bit 0.

Therefore, after $L$ rounds the algorithm will go through the entire strong selector $\mathcal{F}$ and each node will learn a bit of information from each of its neighbors.

The procedure can be repeated $B$ times to broadcast messages of at most $B$ bits.
Hence, the claim follows.
%
\end{proof}


\subsection{Learning neighborhood}
\label{sub:neighbourhood}


Now we show how all the nodes can learn their neighborhoods in $O(\Delta^2 \log^2 n)$ beeping rounds. The following procedure will be non-adaptive. 

\parhead{Assumptions} %\pga{The nodes IDs come from range $[1,n^c]$.} 
The nodes know the total number of nodes $n$ and parameter $c$.
% \pga{the range of possible IDs $[1,n^c]$ for some constant $c \geq 1$}
% and have access to a global clock.

\learningneighthm*
\remove{
\begin{theorem}
\label{th:learning_neighbourhood}
\mm{Consider a Beeping Network where each node $v$ knows $n$.}
    There is a deterministic distributed learning neighborhood algorithm that works in $O(\Delta^2 \log^2 n)$ beeping rounds.
\end{theorem}
}

\begin{proof}
The nodes will beep according to a strong selector $\mF$ as in the previous subsection. This time, however, for each set $S_i \in \mF$ there will be $2\log n$ beeping rounds instead of $1$ round. First, nodes $v \in S_1$ will transmit for $2\log n$ rounds, then nodes $v \in S_2$ will transmit for $2\log n$ rounds and so on.

In each block of $2\log n$ rounds corresponding to a set $S_i$ for some $i$, each node $v \in S_i$ will encode its ID in the following way. For each bit in its ID, if the bit is 1, then the node listens for 1 round and then beeps. If the bit is 0, then the node beeps once and then listens for 1 round. The process is repeated for each bit in the ID.

After all $2\log n$ beeping rounds corresponding to a set $S_i$ for some $i$ pass, each node $v$ can look at the string of beeps and silences that it heard during the block. If there were beeps in both rounds $2k$ and $2k+1$ for some $k$, then there were multiple neighbors transmitting in the block and $v$ will ignore this block. Otherwise, the string of beeps encodes the ID of the only transmitting neighbor $u$. 
% node $v$ will assume that this string of bits is an ID of a node $u$. If the ID of $u$ is such that $u \in S_i$, then $u$ was the only neighbor of $v$ transmitting in the block. 
% \pga{[We may need to transmit 01 for each bit 1 and 10 for each bit 0 to really make this unambiguous.]} 
In particular, $u$ can be added to the list of neighbors of $v$. 
% On the other hand, if $u \notin S_i$, then that means that multiple neighbors of $v$ were overlapping their beeps during the block corresponding to $S_i$ and $u$ should not be added to the list of neighbors of $v$.

% \pga{According to the definition of strong selector $\mF$,} 
After all $L$ blocks of transmissions pass, each node $v$ heard from each $u \in N(v)$ in a block such that $u$ was the only transmitting neighbor. Therefore, each $u \in N(v)$ was added to the list of neighbors of $v$. No other nodes were added to the list of neighbors of $v$. Thus, $v$ knows its neighborhood from now on and the claim follows.
\end{proof}

\subsection{Cluster gathering}
\label{sec:gathering}

% \noindent \textbf{Aggregating information via overlapping Steiner trees.} 

\noindent \textbf{Assumptions.} %\pga{The nodes IDs come from range $[1,n^c]$.} 
Thanks to running cluster gathering inside the network decomposition algorithm, we will have access to additional structures. During the working of the network decomposition algorithm, each cluster $C$ will have a Steiner tree $S$ associated with it. All nodes $v \in C$ will be regular nodes in the Steiner tree $S$, while there may be some additional nodes $u \notin C$ that are Steiner nodes in $S$. All Steiner trees will have depth at most $O(\log^2 n)$, i.e., the diameter of the Steiner tree $S$ will be the same as the weak-diameter of the cluster $C$ that corresponds to $S$. Each node and each edge will be in at most $O(\log n)$ Steiner trees. Each Steiner tree $S$ will have a fixed root node $r$.
Given that our cluster gathering algorithm uses the local broadcast algorithm defined above, the previous assumptions also~apply.

% We can develop a cluster gathering algorithm where all nodes in all clusters work in parallel, resulting in only $O(local\_broadcast\cdot \log^4 n)$ beeping rounds.

\noindent \textbf{Effect and efficiency.} Given the above assumptions, we can develop an algorithm that gathers and aggregates limited information (each node passes at most $O(\log n)$ bits for each Steiner tree it is in) from all nodes in a cluster $C$ to the root $r$ of the corresponding Steiner tree $S$, with all clusters working in parallel. Additionally, the root $r$ can broadcast $O(\log n)$ bits to all nodes in $C$. The algorithm will work in $O(\Delta^2 \cdot \log^6 n)$ beeping rounds.

\noindent \textbf{Utilization.} The gathering algorithm will be used throughout the network decomposition algorithm but may be of independent use, especially since the network decomposition algorithm may output the Steiner trees it was using as well as the decomposition. Therefore, any algorithm using our network decomposition algorithm will have access to the appropriate Steiner trees to use for %gathering 
collecting information using our cluster gathering algorithm.

\clustergatherthm*
\remove{
\begin{theorem}
\label{th:cluster_gathering}
\mm{Consider a Beeping Network where each node $v$ knows $n$, $\Delta$, and $N(v)$.}
    There is a deterministic distributed cluster gathering algorithm that works in $O(\Delta^2 \log^4 n)$ beeping rounds.
\end{theorem}
}

%\noindent \textbf{Algorithm description.} 
\begin{proof}
The algorithm will utilize the local broadcast subroutine at each step. When we say "transmit", we mean to use local broadcast subroutine, unless specified otherwise. Each node $v \in C$ broadcasts its $O(\log n)$ bits (e.g., a number less than $n$) in parallel using the local broadcast subroutine $O(\log n)$ times. At each step, the parent $p$ of $v$ in the corresponding Steiner tree will listen for the message from $v$ as well as messages from its other children. Whenever $p$ receive messages from some of its children, $p$ prepares its own message (e.g., sum of numbers provided by its children in the current step, assuming that the sum is smaller than $n$) of $O(\log n)$ bits. The process will repeat until the root $r$ receives all the messages, which will last the number of steps equal to the depth of the Steiner tree, $O(\log^2 n)$. Note that node $p$ may receive messages from its children in multiple steps; in that case in each step $t$ node $p$ transmits the aggregate of messages it received at $t$, thus transmitting multiple times.

Similarly, one can broadcast a message from $r$ to all the nodes $v\in C$ in $O(\log^2 n)$ steps. The entire algorithm takes $O(\log^2 n)$ steps, which is $O(t_{local\_broadcast} \cdot \log^2 n)$ beeping rounds and the claim follows.
\end{proof}

%%%%%%%%%%%%%%%%%%%%%%%%%%%%%%%%%%%%%%%%%%%%%%%%%%%%

\subsection{Network Decomposition}
\label{sec:decomposition}


% \subsection{Network decomposition from~\cite{ghaffari2021improved}}


In this section, we present how to adapt the network decomposition algorithm of Ghaffari et al.~\cite{ghaffari2021improved} to the beeping model. The only changes are in the way that nodes communicate. The original algorithm was %made
designed for the \congest model, where communication was straightforward. 
Instead, for the beeping model, we have to carefully implement all the concurrent communication, so that the algorithm remains efficient.
%
First, let us recall the result from~\cite{ghaffari2021improved}. 

% \noindent\textbf{Notations:} $b$ is the length of identifiers, $n$ is the number of nodes in graph $G$.

\begin{theorem}\cite{ghaffari2021improved}
    There is a deterministic distributed algorithm that computes a $(\log n, \log^2 n)$ network decomposition in $O(\log^5 n)$ \congest rounds.
\end{theorem}

We adapt the above result to the beeping model and obtain the next theorem.

\networkdecompthm*
\remove{
\begin{theorem}
\label{thm:local-decomposition}
    There is a deterministic distributed algorithm that computes a $(\log n, \log^2 n)$ network decomposition in $O(\Delta^2 \log^8 n)$ beeping rounds.
\end{theorem}
}

The algorithm works as follows. In preprocessing, each node learns its neighborhood, so that we will be able to use local broadcast routine.
%freely. 
This part can take up to $O(\Delta^2 \log^2 n)$ beeping rounds (see Theorem~\ref{th:learning_neighbourhood}). Next, we perform the network decomposition algorithm from~\cite{ghaffari2021improved} with all the communication carefully replaced, as shown 
%in Section~\ref{sec:rundown} 
below. This completes our network decomposition algorithm.

% \dk{In order to combine our Local Broadcast with the tools in~\cite{ghaffari2021improved}, detailed check of these tools need to be done to assure, among others, that they rely on Local Broadcast (not the general \congest model rules).
% We provide the relevant parts in Section~\ref{sec:Ghaffari-verbatim}.}



%\subsubsection{Summary of messages}
\label{sec:rundown}

% Let $local\_broadcast$ denote the number of beeping rounds to perform a local broadcast in the beeping model, i.e., make sure that for every node $v$ all neighbors of $v$ receive a message from $v$ \pga{of length at most $\log n$. According to Theorem~\ref{th:local_broadcast}, such local broadcast lasts $O(\Delta^2 \log^2 n$ beeping rounds.}

\paragraph{Summary of messages:}
We need to carefully replace all the communication from~\cite{ghaffari2021improved} with routines that work in the beeping model. We will use our local broadcast and cluster gathering routines from Section~\ref{sec:primitives}.

% Below we present a list of messages transmitted in \pga{the original algorithm for the \congest model~\cite{ghaffari2021improved} and how to implement this information transmission in the beeping model. The original algorithm can be viewed in Subsection~\ref{sec:ghaffari}. In the list below we count the messages done per \emph{step}. There will be at most $O(\log^2 n)$ steps in the algorithm.}

Below, we present a list of messages transmitted in %Subsection~\ref{sec:Ghaffari-verbatim} in 
a step and how to implement this information %transmission 
propagation in the beeping model (for convenience, we attach the network decomposition algorithm from~\cite{ghaffari2021improved} in Section~\ref{sec:Ghaffari-verbatim}):
\begin{enumerate}
    \item \textbf{Before proposals.} A node $v$ needs to check which clusters are adjacent to it and what are the parameters of these clusters (id, level). Every node $u$ can broadcast the parameters of the cluster $u$ is in, which is at most $O(\log n)$ bits. This is done once per step, %. This used to cost
    at a cost of $O(1)$ \congest rounds per step. In the beeping model, it can be done via a local broadcast with messages of length at most $O(\log n)$ bits. According to Theorem~\ref{th:local_broadcast}, it will cost $O(\Delta^2 \log^2 n)$ beeping rounds per step.
    \item \textbf{Proposals.} Proposals to join a cluster can be transmitted to all neighbors if the target cluster (or the target node) is specified in the message. Other receivers may ignore the message. This part used to cost $O(1)$ \congest rounds per step. 
    % In beeping model, this can be done in $O(local\_broadcast)$ rounds per step.
    In the beeping model, it can be done via a local broadcast, where the message specifies the IDs of both, sender and receiver, meaning messages of length at most $O(\log n)$ bits. According to Theorem~\ref{th:local_broadcast}, it will cost $O(\Delta^2 \log^2 n)$ beeping rounds per step.
    \item \textbf{Gathering proposals.} The leader of each cluster should learn the total number of proposals and the total number of tokens in the cluster. This part was done in $O(\log^3 n)$ \congest rounds. %of \congest model. 
    Note that the algorithm keeps the Steiner tree of $O(\log^2 n)$ diameter for each cluster, such that each node (and therefore each edge) is in at most $O(\log n)$ Steiner trees. Additionally, each node that participates in this cluster gathering  can add the numbers of proposals/tokens it receives from other nodes and then transmit the sums instead of relaying each message separately; these sums can never exceed the total number of nodes $n$, so this guarantees that each node only transmits $O(\log n)$ bits per Steiner tree it is in. Therefore, we can use the cluster gathering algorithm. According to Theorem~\ref{th:cluster_gathering}, this takes at most $O(\Delta^2 \log^6 n)$ beeping rounds.
    % given a $O(\log^2 n)$ diameter Steiner tree such that each node (and therefore each edge) is in at most $O(\log n)$ Steiner trees. 
    % This can be done in $O(\log^3 n)$ rounds of \congest model, since a message from a leaf to the root can collide with at most $O(\log^3 n)$ different messages. In beeping model, we can use the 
    % this can be done in $O(local\_broadcast \cdot \log^3 n)$ rounds per step. \pga{There should be an additional factor of $O(\log n)$ due to the length of the messages passed. Also, how do we determine, which message (from which Steiner tree) to transmit first?}
    \item \textbf{Responding to proposals.} Each node informs all its neighbors either that all the proposals were accepted or that all the proposing nodes should be killed. This part used to cost $O(1)$ \congest rounds per step. In the beeping model, the same can be done in a local broadcast, with $O(1)$ bit messages. According to Theorem~\ref{th:local_broadcast}, it will cost $O(\Delta^2 \log n)$ beeping rounds.
    \item \textbf{Stalling.} If a cluster decides to stall, all nodes neighboring the cluster should be informed about it. This \mm{part} used to cost $O(1)$ \congest rounds per step. In the beeping model, \mm{the same} can be done in a local broadcast with $O(1)$ bit messages. According to Theorem~\ref{th:local_broadcast}, it will cost $O(\Delta^2 \log n)$ beeping rounds.
\end{enumerate}

In the procedure described above, there is a total of $O(\Delta^2 \cdot \log^6 n)$ beeping rounds required per step. There are up to $O(\log n)$ steps per phase and up to $O(\log n)$ phases in the algorithm, which results in $O(\Delta^2 \cdot \log^8 n)$ beeping rounds for the entire algorithm. Note that only the means of communication changed. Therefore, the correctness of the algorithm is unaffected. This completes the proof of Theorem~\ref{thm:local-decomposition}.

% \pga{[TODO: We don't know whether we hear a single signal or multiple signals!]}

% \pga{[Solution 2: Have a preprocessing phase where all nodes learn their neighborhood. This can be done in $O(\Delta^2 \log^2 n)$ beeping rounds.]}

% \pga{Solution 1: Let $n$ be the number of nodes in the network. Consider $(n^2,\Delta)$ strong selector $\mathcal{F}$. Consider a node $v$ with id equal $id(v)$ that attempts to transmits a message $m$ with the first $\log n$ bits equal to $\hat{m}$. Let $id(v,m)$ be the concatenation of $id(v)$ and $\hat{m}$. Every $id(v,m)$ corresponds to exactly one element of the universum $U=[n^2]$. A node $v$ with a message $m$ will "transmit" (bit 0 is encoded as silence, while bit 1 is encoded as a beep) a bit in round $t$ if $t$-th set of $\mathcal{F}$ contains $id(v,m)$. 
% % [TODO: We need a single bit, i.e., 1 or 0. We can encode a single bit in two rounds: code 10 (i.e., transmission and then silence) corresponds to bit 0, while code 01 (silence followed by transmission) corresponds to bit 1.]

% The strong selector ensures that each node-message pair successfully transmits a bit to any neighbor.

% Intuitively, we copied each node $n$ times (id of a node is concatenated with one of $n$ different beginnings of a message on $\log n$ bits), but only at most one copy can transmit at any round. A transmission from node $u$ to node $v$ can overlap with at most $\Delta - 1$ other neighbors of $v$. The strong selector $\mathcal{F}$ guarantees that for every listener $v$ that has at most $\Delta$ neighbors $N(v)$, for each $u\in N(v)$ there is at least one set $S \in \mathcal{F}$ such that $S \cap N(v) = {u}$. }

% \pga{[COMMENT: Do we assume that a node knows its neighbors? If yes, the following paragraph is trivial. Otherwise, we may need a linear program. Problem with a linear program: Nodes that transmit 0 are silent. We can run a single local broadcast routine to learn our neighbors.]

% Furthermore, after every set of $\mathcal{F}$ was used, every listener $v$ can determine which round was the round with exactly one neighbor transmitting and which rounds had multiple neighbors transmitting.

% Let $f_v^i$ denote feedback that node $v$ receives at round $i$, i.e., $f_v^i=0$ if node $v$ heard silence in round $i$ and $f_v^i=1$ if node $v$ heard at least one beep in round $i$. After all $L$ rounds, each node $v$ determines the rounds with a unique transmitter based on the feedback $f_v^i$ it received at round $i$ for all $1 \leq i \leq L$ and the knowledge of what every set $S_i \in \mathcal{F}$ contains. Let $x_u=1$ if $u$ is beeping and $x_u=0$ otherwise. The linear program consists of $nL$ equations, one for each pair node-round $(v,i)$:

% \[ \sum_{u \in N(v)} x_u = 0 \text{ iff } f_v^i=0\]
% \[ \sum_{u \in N(v)} x_u \geq 1 \text{ iff } f_v^i=1\]

% }


% \subsection{Possible outputs}
% \pga{
% Besides the $(C,D)$ network decomposition, the algorithm above computes a few other structures that can be %helpful.
% \mm{useful and are of independent interest.}
% One such structure is a Steiner tree of depth $O(D)$ for each cluster. Additionally, a root of each Steiner tree can serve as a leader for the corresponding cluster\footnote{Note, the leader may be outside of the cluster it is serving.}. During the construction of the Steiner trees, any aggregate information (i.e., information that can be aggregated without increasing the size of messages beyond $O(\log n)$ bits, e.g., the number of descendants) can be gathered in each node in the Steiner tree without changing the asymptotic cost of the operation. During the gathering proposals stage the leader can broadcast $O(1)$ bits to its entire clusters without changing the asymptotic cost of the operation.
% }

% \pga{TODO: A separate algorithm for gathering information in multiple Steiner trees simultaneously. We use Lemmas/Theorems from Ghaffari's paper, so that we know that the Steiner trees have the right properties and there are only $\log n$ overlaps. Then, we can use THIS algorithm to describe gathering proposals part, as well as provide this algorithm as a separate primitive. This algorithm should be described before gathering proposals, perhaps before network decomposition section.}

%\section{Missing details from Section~\ref{}}
%\section{Details from Section~\ref{sec:decomposition}: Efficient Network Decomposition in Beeping Networks}


% \textcolor{green}{---------------------------------???????????}

% \dk{In order to prove Theorem~\ref{thm:local-decomposition}, we need to make sure that the network decomposition algorithm in~\cite{ghaffari2021improved} can indeed be implemented using only beeps.}
% indeed uses Local Broadcast as communication tool.}
\subsection{GGR network decomposition algorithm}
\label{sec:Ghaffari-verbatim}

\noindent\textbf{Notation:} $b$ is the length of identifiers, $n$ is the number of nodes in graph $G$.

The remainder of this subsection, which is important from perspective of assurance that our Local Broadcast could be combined with the tools in~\cite{ghaffari2021improved}, is cited from~\cite{ghaffari2021improved} verbatim.

\noindent\textbf{Construction  outline:}   The  construction  has  $2(b+\log n) =O(\log n)$ phases. Each phase has $28(b+\log n) =O(\log n)$ steps.  Initially, all nodes of $G$ are \emph{living}, during the construction some living nodes \emph{die}. Each living node is  part  of  exactly  one  cluster.   Initially,  there  is  one cluster $C_v$ for each vertex $v\in V(G)$ and we define the identifier $id(C)$ of $C$ as the unique identifier of $v$ and use $id_i(C)$  to  denote  the $i$-th  least  significant  bit  of  $id(C)$. From now on, we talk only about identifiers of clusters and do not think of vertices as having identifiers, though they will still use them for simple symmetry breaking tasks.   Also,  at  the  beginning,  the  Steiner  tree $T_{C_v}$ of a cluster $C_v$ contains just one node, namely $v$ itself, as a  terminal  node.   Clusters  will  grow  or  shrink  during the iterations, while their Steiner trees collecting their vertices can only grow.  When a cluster does not contain any nodes, it does not participate in the algorithm anymore.

\noindent\textbf{Parameters of each cluster:} Each cluster $C$ keeps two other parameters besides its identifier $id(C)$ to make its decisions:  its number of tokens $t(C)$ and its level $lev(C)$.The number of tokens can change in each step -- more precisely it is incremented by one whenever a new vertex joins $C$, while it does not decrease when a vertex leaves $C$.  The number of tokens only decreases when $C$ actively deletes nodes.  We define $t_i(C)$ as the number of tokens of $C$ at the beginning of the $i$-th phase and set $t_1(C) = 1$. Each  cluster  starts  in  level  $0$.   The  level  of  each cluster does not change within a phase $i$ and can only increment by one between two phases; it is bounded by $b$.  We denote with $lev_i(C)$ the level of $C$ during phase $i$.   Moreover,  for  the  purpose  of  the  analysis,  we  keep track  of  the  potential  $\Phi(C)$  of  a  cluster $C$ defined  as $\Phi_i(C) = 3i - 2lev_i(C) + id_{lev_i(C)+1}(C)$.  The potential of each cluster stays the same within a phase.

\noindent\textbf{Description  of  a  step:} In each step, first, each node $v$ of each cluster $C$ checks whether it is adjacent to a  cluster $C'$ such that  $lev(C')<lev(C)$. If  so, then $v$ proposes  to  an arbitrary  neighboring  cluster $C'$ among the neighbors with the smallest level $lev(C')$ and if there is a choice, it prefers to join clusters with $id_{lev(C')+1}(C') = 1$.  Otherwise, if there is a neighboring cluster $C'$ with $lev(C') = lev(C)$ and $id_{lev(C')+1}(C') = 1$, while  $id_{lev(C)+1}(C)  =  0$,  then $v$ proposes  to  arbitrary such cluster.

Second, each cluster $C$ collects the number of proposals  it  received.   Once  the  cluster  has  collected  the number  of  proposals,  it  does  the  following.   If  there are $p$ proposing nodes,  then they join $C$ if and only if $p \geq t(C)/(28(b+ \log n))$.  The denominator is equal to the number of steps. If $C$ accepts these proposals, then $C$ receives $p$ new tokens, one from each newly joined node. On the other hand, if $C$ does not accept the proposals as their number is not sufficiently large, then $C$ decides to kill all those proposing nodes.  These nodes are then removed from $G$.  Cluster $C$ pays $p \cdot 14(b+ \log n)$ tokens for this, i.e., it pays $14(b+ \log n)$ tokens for every vertex that it deletes.  These tokens are forever gone.  Then the cluster does not participate in growing anymore,  until the end of the phase and throughout that time we call that cluster \emph{stalling}.  The cluster tells that it is stalling to neighboring nodes so that they do not propose to it. At the end of the phase, each stalling cluster increments its level by one.

If the cluster is in level $b-1$ and goes to the last level $b$, it will not grow anymore during the whole algorithm, and  we  say  that  it  has finished.    Other  neighboring clusters can still eat its vertices (by this we mean that vertices of the finished clusters may still propose to join other clusters). 

Whenever  a  node $u$ joins  a  cluster $C$ via  a  vertex $v\in C$, we add $u$ to the Steiner tree $T_C$ as a new terminal node and connect it via an edge $uv$.  Whenever a node $u\in C$ is deleted or eaten by a different cluster, it stays in the Steiner tree $T_C$ but is changed to a non-terminal node.



%\input{boolean}
\section{Conclusions}

We provided deterministic distributed algorithms to efficiently simulate a round of algorithms designed for the CONGEST model on the Beeping Networks. This allowed us to improve polynomially the time complexity of several (also graph) problems on Beeping  Networks. The first simulation by the Local Broadcast algorithm is shorter by a polylogarithmic factor than the other, more general one -- yet still powerful enough to implement some algorithms, including the prominent solution to Network Decomposition~\cite{ghaffari2021improved}.
The more general one could be used for solving problems such as MIS.
We also considered efficient pipelining of messages via several layers of BN.
%We also proved that our solutions could not be substantially improved if the considered problems require content-oblivious local broadcast, by proving an almost-tight lower bound.

Two important lines of research arise from our work.
First, whether some (graph) problems do not need local broadcast to be solved deterministically, and whether their time complexity could be asymptotically below $\Delta^2$.
Second, could a lower bound on any deterministic local broadcast algorithm, better than $\Omega(\Delta\log n)$, be proved?
%our lower bound be tightened and extended to any, not necessarily content-oblivious \mam{and non-adaptive}, solutions to the Local Broadcast problem?

% \todo{Propose to develop algorithms that work in time depending on the diameter of the network}

% \todo{Discussion of noisy beeping channel.}

%\bibliographystyle{alpha}
\bibliographystyle{plain}
\bibliography{bibliography}

\newpage

\appendix

\end{document}